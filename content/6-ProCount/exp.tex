





\section{Additional Experimental Evaluation}
\label{appendix:exp}
% Our \procount{} source code and experimental data are available in a public repository (\url{https://github.com/vardigroup/DPMC/tree/v2.0.0/experiments}).

%%%%%%%%%%%%%%%%%%%%%%%%%%%%%%%%%%%%%%%%%%%%%%%%%%%%%%%%%%%%%%%%%%%%%%%%%%%%%%%%

\subsection{Experiment 1: Comparing Planners}

Figure \ref{figPlanningA} illustrates how the planner \Lg{} compares to the planner \htb{} across several settings.
\Lg{} is an \emph{anytime} tool, which outputs better and better project-join trees the longer it runs.
In Experiment 1, within 100 seconds, each \Lg{} setting may produce several project-join trees (of decreasing widths) for a single benchmark.
Figure \ref{figPlanningA} plots the time of the first such project-join tree of width at most \maxWidth.
\htb{} is a \emph{one-shot} tool, which outputs only one project-join tree for each benchmark.
\begin{figure}[H]
    \centering
    %% Creator: Matplotlib, PGF backend
%%
%% To include the figure in your LaTeX document, write
%%   \input{<filename>.pgf}
%%
%% Make sure the required packages are loaded in your preamble
%%   \usepackage{pgf}
%%
%% and, on pdftex
%%   \usepackage[utf8]{inputenc}\DeclareUnicodeCharacter{2212}{-}
%%
%% or, on luatex and xetex
%%   \usepackage{unicode-math}
%%
%% Figures using additional raster images can only be included by \input if
%% they are in the same directory as the main LaTeX file. For loading figures
%% from other directories you can use the `import` package
%%   \usepackage{import}
%%
%% and then include the figures with
%%   \import{<path to file>}{<filename>.pgf}
%%
%% Matplotlib used the following preamble
%%   \usepackage[utf8x]{inputenc}
%%   \usepackage[T1]{fontenc}
%%
\begingroup%
\makeatletter%
\begin{pgfpicture}%
\pgfpathrectangle{\pgfpointorigin}{\pgfqpoint{6.000000in}{2.800000in}}%
\pgfusepath{use as bounding box, clip}%
\begin{pgfscope}%
\pgfsetbuttcap%
\pgfsetmiterjoin%
\definecolor{currentfill}{rgb}{1.000000,1.000000,1.000000}%
\pgfsetfillcolor{currentfill}%
\pgfsetlinewidth{0.000000pt}%
\definecolor{currentstroke}{rgb}{1.000000,1.000000,1.000000}%
\pgfsetstrokecolor{currentstroke}%
\pgfsetdash{}{0pt}%
\pgfpathmoveto{\pgfqpoint{0.000000in}{0.000000in}}%
\pgfpathlineto{\pgfqpoint{6.000000in}{0.000000in}}%
\pgfpathlineto{\pgfqpoint{6.000000in}{2.800000in}}%
\pgfpathlineto{\pgfqpoint{0.000000in}{2.800000in}}%
\pgfpathclose%
\pgfusepath{fill}%
\end{pgfscope}%
\begin{pgfscope}%
\pgfsetbuttcap%
\pgfsetmiterjoin%
\definecolor{currentfill}{rgb}{1.000000,1.000000,1.000000}%
\pgfsetfillcolor{currentfill}%
\pgfsetlinewidth{0.000000pt}%
\definecolor{currentstroke}{rgb}{0.000000,0.000000,0.000000}%
\pgfsetstrokecolor{currentstroke}%
\pgfsetstrokeopacity{0.000000}%
\pgfsetdash{}{0pt}%
\pgfpathmoveto{\pgfqpoint{0.708220in}{0.535823in}}%
\pgfpathlineto{\pgfqpoint{5.753646in}{0.535823in}}%
\pgfpathlineto{\pgfqpoint{5.753646in}{2.605275in}}%
\pgfpathlineto{\pgfqpoint{0.708220in}{2.605275in}}%
\pgfpathclose%
\pgfusepath{fill}%
\end{pgfscope}%
\begin{pgfscope}%
\pgfsetbuttcap%
\pgfsetroundjoin%
\definecolor{currentfill}{rgb}{0.000000,0.000000,0.000000}%
\pgfsetfillcolor{currentfill}%
\pgfsetlinewidth{0.803000pt}%
\definecolor{currentstroke}{rgb}{0.000000,0.000000,0.000000}%
\pgfsetstrokecolor{currentstroke}%
\pgfsetdash{}{0pt}%
\pgfsys@defobject{currentmarker}{\pgfqpoint{0.000000in}{-0.048611in}}{\pgfqpoint{0.000000in}{0.000000in}}{%
\pgfpathmoveto{\pgfqpoint{0.000000in}{0.000000in}}%
\pgfpathlineto{\pgfqpoint{0.000000in}{-0.048611in}}%
\pgfusepath{stroke,fill}%
}%
\begin{pgfscope}%
\pgfsys@transformshift{0.708220in}{0.535823in}%
\pgfsys@useobject{currentmarker}{}%
\end{pgfscope}%
\end{pgfscope}%
\begin{pgfscope}%
\definecolor{textcolor}{rgb}{0.000000,0.000000,0.000000}%
\pgfsetstrokecolor{textcolor}%
\pgfsetfillcolor{textcolor}%
\pgftext[x=0.708220in,y=0.438600in,,top]{\color{textcolor}\rmfamily\fontsize{9.000000}{10.800000}\selectfont \(\displaystyle {0}\)}%
\end{pgfscope}%
\begin{pgfscope}%
\pgfsetbuttcap%
\pgfsetroundjoin%
\definecolor{currentfill}{rgb}{0.000000,0.000000,0.000000}%
\pgfsetfillcolor{currentfill}%
\pgfsetlinewidth{0.803000pt}%
\definecolor{currentstroke}{rgb}{0.000000,0.000000,0.000000}%
\pgfsetstrokecolor{currentstroke}%
\pgfsetdash{}{0pt}%
\pgfsys@defobject{currentmarker}{\pgfqpoint{0.000000in}{-0.048611in}}{\pgfqpoint{0.000000in}{0.000000in}}{%
\pgfpathmoveto{\pgfqpoint{0.000000in}{0.000000in}}%
\pgfpathlineto{\pgfqpoint{0.000000in}{-0.048611in}}%
\pgfusepath{stroke,fill}%
}%
\begin{pgfscope}%
\pgfsys@transformshift{1.338898in}{0.535823in}%
\pgfsys@useobject{currentmarker}{}%
\end{pgfscope}%
\end{pgfscope}%
\begin{pgfscope}%
\definecolor{textcolor}{rgb}{0.000000,0.000000,0.000000}%
\pgfsetstrokecolor{textcolor}%
\pgfsetfillcolor{textcolor}%
\pgftext[x=1.338898in,y=0.438600in,,top]{\color{textcolor}\rmfamily\fontsize{9.000000}{10.800000}\selectfont \(\displaystyle {25}\)}%
\end{pgfscope}%
\begin{pgfscope}%
\pgfsetbuttcap%
\pgfsetroundjoin%
\definecolor{currentfill}{rgb}{0.000000,0.000000,0.000000}%
\pgfsetfillcolor{currentfill}%
\pgfsetlinewidth{0.803000pt}%
\definecolor{currentstroke}{rgb}{0.000000,0.000000,0.000000}%
\pgfsetstrokecolor{currentstroke}%
\pgfsetdash{}{0pt}%
\pgfsys@defobject{currentmarker}{\pgfqpoint{0.000000in}{-0.048611in}}{\pgfqpoint{0.000000in}{0.000000in}}{%
\pgfpathmoveto{\pgfqpoint{0.000000in}{0.000000in}}%
\pgfpathlineto{\pgfqpoint{0.000000in}{-0.048611in}}%
\pgfusepath{stroke,fill}%
}%
\begin{pgfscope}%
\pgfsys@transformshift{1.969577in}{0.535823in}%
\pgfsys@useobject{currentmarker}{}%
\end{pgfscope}%
\end{pgfscope}%
\begin{pgfscope}%
\definecolor{textcolor}{rgb}{0.000000,0.000000,0.000000}%
\pgfsetstrokecolor{textcolor}%
\pgfsetfillcolor{textcolor}%
\pgftext[x=1.969577in,y=0.438600in,,top]{\color{textcolor}\rmfamily\fontsize{9.000000}{10.800000}\selectfont \(\displaystyle {50}\)}%
\end{pgfscope}%
\begin{pgfscope}%
\pgfsetbuttcap%
\pgfsetroundjoin%
\definecolor{currentfill}{rgb}{0.000000,0.000000,0.000000}%
\pgfsetfillcolor{currentfill}%
\pgfsetlinewidth{0.803000pt}%
\definecolor{currentstroke}{rgb}{0.000000,0.000000,0.000000}%
\pgfsetstrokecolor{currentstroke}%
\pgfsetdash{}{0pt}%
\pgfsys@defobject{currentmarker}{\pgfqpoint{0.000000in}{-0.048611in}}{\pgfqpoint{0.000000in}{0.000000in}}{%
\pgfpathmoveto{\pgfqpoint{0.000000in}{0.000000in}}%
\pgfpathlineto{\pgfqpoint{0.000000in}{-0.048611in}}%
\pgfusepath{stroke,fill}%
}%
\begin{pgfscope}%
\pgfsys@transformshift{2.600255in}{0.535823in}%
\pgfsys@useobject{currentmarker}{}%
\end{pgfscope}%
\end{pgfscope}%
\begin{pgfscope}%
\definecolor{textcolor}{rgb}{0.000000,0.000000,0.000000}%
\pgfsetstrokecolor{textcolor}%
\pgfsetfillcolor{textcolor}%
\pgftext[x=2.600255in,y=0.438600in,,top]{\color{textcolor}\rmfamily\fontsize{9.000000}{10.800000}\selectfont \(\displaystyle {75}\)}%
\end{pgfscope}%
\begin{pgfscope}%
\pgfsetbuttcap%
\pgfsetroundjoin%
\definecolor{currentfill}{rgb}{0.000000,0.000000,0.000000}%
\pgfsetfillcolor{currentfill}%
\pgfsetlinewidth{0.803000pt}%
\definecolor{currentstroke}{rgb}{0.000000,0.000000,0.000000}%
\pgfsetstrokecolor{currentstroke}%
\pgfsetdash{}{0pt}%
\pgfsys@defobject{currentmarker}{\pgfqpoint{0.000000in}{-0.048611in}}{\pgfqpoint{0.000000in}{0.000000in}}{%
\pgfpathmoveto{\pgfqpoint{0.000000in}{0.000000in}}%
\pgfpathlineto{\pgfqpoint{0.000000in}{-0.048611in}}%
\pgfusepath{stroke,fill}%
}%
\begin{pgfscope}%
\pgfsys@transformshift{3.230933in}{0.535823in}%
\pgfsys@useobject{currentmarker}{}%
\end{pgfscope}%
\end{pgfscope}%
\begin{pgfscope}%
\definecolor{textcolor}{rgb}{0.000000,0.000000,0.000000}%
\pgfsetstrokecolor{textcolor}%
\pgfsetfillcolor{textcolor}%
\pgftext[x=3.230933in,y=0.438600in,,top]{\color{textcolor}\rmfamily\fontsize{9.000000}{10.800000}\selectfont \(\displaystyle {100}\)}%
\end{pgfscope}%
\begin{pgfscope}%
\pgfsetbuttcap%
\pgfsetroundjoin%
\definecolor{currentfill}{rgb}{0.000000,0.000000,0.000000}%
\pgfsetfillcolor{currentfill}%
\pgfsetlinewidth{0.803000pt}%
\definecolor{currentstroke}{rgb}{0.000000,0.000000,0.000000}%
\pgfsetstrokecolor{currentstroke}%
\pgfsetdash{}{0pt}%
\pgfsys@defobject{currentmarker}{\pgfqpoint{0.000000in}{-0.048611in}}{\pgfqpoint{0.000000in}{0.000000in}}{%
\pgfpathmoveto{\pgfqpoint{0.000000in}{0.000000in}}%
\pgfpathlineto{\pgfqpoint{0.000000in}{-0.048611in}}%
\pgfusepath{stroke,fill}%
}%
\begin{pgfscope}%
\pgfsys@transformshift{3.861611in}{0.535823in}%
\pgfsys@useobject{currentmarker}{}%
\end{pgfscope}%
\end{pgfscope}%
\begin{pgfscope}%
\definecolor{textcolor}{rgb}{0.000000,0.000000,0.000000}%
\pgfsetstrokecolor{textcolor}%
\pgfsetfillcolor{textcolor}%
\pgftext[x=3.861611in,y=0.438600in,,top]{\color{textcolor}\rmfamily\fontsize{9.000000}{10.800000}\selectfont \(\displaystyle {125}\)}%
\end{pgfscope}%
\begin{pgfscope}%
\pgfsetbuttcap%
\pgfsetroundjoin%
\definecolor{currentfill}{rgb}{0.000000,0.000000,0.000000}%
\pgfsetfillcolor{currentfill}%
\pgfsetlinewidth{0.803000pt}%
\definecolor{currentstroke}{rgb}{0.000000,0.000000,0.000000}%
\pgfsetstrokecolor{currentstroke}%
\pgfsetdash{}{0pt}%
\pgfsys@defobject{currentmarker}{\pgfqpoint{0.000000in}{-0.048611in}}{\pgfqpoint{0.000000in}{0.000000in}}{%
\pgfpathmoveto{\pgfqpoint{0.000000in}{0.000000in}}%
\pgfpathlineto{\pgfqpoint{0.000000in}{-0.048611in}}%
\pgfusepath{stroke,fill}%
}%
\begin{pgfscope}%
\pgfsys@transformshift{4.492290in}{0.535823in}%
\pgfsys@useobject{currentmarker}{}%
\end{pgfscope}%
\end{pgfscope}%
\begin{pgfscope}%
\definecolor{textcolor}{rgb}{0.000000,0.000000,0.000000}%
\pgfsetstrokecolor{textcolor}%
\pgfsetfillcolor{textcolor}%
\pgftext[x=4.492290in,y=0.438600in,,top]{\color{textcolor}\rmfamily\fontsize{9.000000}{10.800000}\selectfont \(\displaystyle {150}\)}%
\end{pgfscope}%
\begin{pgfscope}%
\pgfsetbuttcap%
\pgfsetroundjoin%
\definecolor{currentfill}{rgb}{0.000000,0.000000,0.000000}%
\pgfsetfillcolor{currentfill}%
\pgfsetlinewidth{0.803000pt}%
\definecolor{currentstroke}{rgb}{0.000000,0.000000,0.000000}%
\pgfsetstrokecolor{currentstroke}%
\pgfsetdash{}{0pt}%
\pgfsys@defobject{currentmarker}{\pgfqpoint{0.000000in}{-0.048611in}}{\pgfqpoint{0.000000in}{0.000000in}}{%
\pgfpathmoveto{\pgfqpoint{0.000000in}{0.000000in}}%
\pgfpathlineto{\pgfqpoint{0.000000in}{-0.048611in}}%
\pgfusepath{stroke,fill}%
}%
\begin{pgfscope}%
\pgfsys@transformshift{5.122968in}{0.535823in}%
\pgfsys@useobject{currentmarker}{}%
\end{pgfscope}%
\end{pgfscope}%
\begin{pgfscope}%
\definecolor{textcolor}{rgb}{0.000000,0.000000,0.000000}%
\pgfsetstrokecolor{textcolor}%
\pgfsetfillcolor{textcolor}%
\pgftext[x=5.122968in,y=0.438600in,,top]{\color{textcolor}\rmfamily\fontsize{9.000000}{10.800000}\selectfont \(\displaystyle {175}\)}%
\end{pgfscope}%
\begin{pgfscope}%
\pgfsetbuttcap%
\pgfsetroundjoin%
\definecolor{currentfill}{rgb}{0.000000,0.000000,0.000000}%
\pgfsetfillcolor{currentfill}%
\pgfsetlinewidth{0.803000pt}%
\definecolor{currentstroke}{rgb}{0.000000,0.000000,0.000000}%
\pgfsetstrokecolor{currentstroke}%
\pgfsetdash{}{0pt}%
\pgfsys@defobject{currentmarker}{\pgfqpoint{0.000000in}{-0.048611in}}{\pgfqpoint{0.000000in}{0.000000in}}{%
\pgfpathmoveto{\pgfqpoint{0.000000in}{0.000000in}}%
\pgfpathlineto{\pgfqpoint{0.000000in}{-0.048611in}}%
\pgfusepath{stroke,fill}%
}%
\begin{pgfscope}%
\pgfsys@transformshift{5.753646in}{0.535823in}%
\pgfsys@useobject{currentmarker}{}%
\end{pgfscope}%
\end{pgfscope}%
\begin{pgfscope}%
\definecolor{textcolor}{rgb}{0.000000,0.000000,0.000000}%
\pgfsetstrokecolor{textcolor}%
\pgfsetfillcolor{textcolor}%
\pgftext[x=5.753646in,y=0.438600in,,top]{\color{textcolor}\rmfamily\fontsize{9.000000}{10.800000}\selectfont \(\displaystyle {200}\)}%
\end{pgfscope}%
\begin{pgfscope}%
\definecolor{textcolor}{rgb}{0.000000,0.000000,0.000000}%
\pgfsetstrokecolor{textcolor}%
\pgfsetfillcolor{textcolor}%
\pgftext[x=3.230933in,y=0.272655in,,top]{\color{textcolor}\rmfamily\fontsize{10.000000}{12.000000}\selectfont Number of benchmarks solved}%
\end{pgfscope}%
\begin{pgfscope}%
\pgfsetbuttcap%
\pgfsetroundjoin%
\definecolor{currentfill}{rgb}{0.000000,0.000000,0.000000}%
\pgfsetfillcolor{currentfill}%
\pgfsetlinewidth{0.803000pt}%
\definecolor{currentstroke}{rgb}{0.000000,0.000000,0.000000}%
\pgfsetstrokecolor{currentstroke}%
\pgfsetdash{}{0pt}%
\pgfsys@defobject{currentmarker}{\pgfqpoint{-0.048611in}{0.000000in}}{\pgfqpoint{-0.000000in}{0.000000in}}{%
\pgfpathmoveto{\pgfqpoint{-0.000000in}{0.000000in}}%
\pgfpathlineto{\pgfqpoint{-0.048611in}{0.000000in}}%
\pgfusepath{stroke,fill}%
}%
\begin{pgfscope}%
\pgfsys@transformshift{0.708220in}{0.680664in}%
\pgfsys@useobject{currentmarker}{}%
\end{pgfscope}%
\end{pgfscope}%
\begin{pgfscope}%
\definecolor{textcolor}{rgb}{0.000000,0.000000,0.000000}%
\pgfsetstrokecolor{textcolor}%
\pgfsetfillcolor{textcolor}%
\pgftext[x=0.344411in, y=0.635939in, left, base]{\color{textcolor}\rmfamily\fontsize{9.000000}{10.800000}\selectfont \(\displaystyle {10^{-2}}\)}%
\end{pgfscope}%
\begin{pgfscope}%
\pgfsetbuttcap%
\pgfsetroundjoin%
\definecolor{currentfill}{rgb}{0.000000,0.000000,0.000000}%
\pgfsetfillcolor{currentfill}%
\pgfsetlinewidth{0.803000pt}%
\definecolor{currentstroke}{rgb}{0.000000,0.000000,0.000000}%
\pgfsetstrokecolor{currentstroke}%
\pgfsetdash{}{0pt}%
\pgfsys@defobject{currentmarker}{\pgfqpoint{-0.048611in}{0.000000in}}{\pgfqpoint{-0.000000in}{0.000000in}}{%
\pgfpathmoveto{\pgfqpoint{-0.000000in}{0.000000in}}%
\pgfpathlineto{\pgfqpoint{-0.048611in}{0.000000in}}%
\pgfusepath{stroke,fill}%
}%
\begin{pgfscope}%
\pgfsys@transformshift{0.708220in}{1.161817in}%
\pgfsys@useobject{currentmarker}{}%
\end{pgfscope}%
\end{pgfscope}%
\begin{pgfscope}%
\definecolor{textcolor}{rgb}{0.000000,0.000000,0.000000}%
\pgfsetstrokecolor{textcolor}%
\pgfsetfillcolor{textcolor}%
\pgftext[x=0.344411in, y=1.117092in, left, base]{\color{textcolor}\rmfamily\fontsize{9.000000}{10.800000}\selectfont \(\displaystyle {10^{-1}}\)}%
\end{pgfscope}%
\begin{pgfscope}%
\pgfsetbuttcap%
\pgfsetroundjoin%
\definecolor{currentfill}{rgb}{0.000000,0.000000,0.000000}%
\pgfsetfillcolor{currentfill}%
\pgfsetlinewidth{0.803000pt}%
\definecolor{currentstroke}{rgb}{0.000000,0.000000,0.000000}%
\pgfsetstrokecolor{currentstroke}%
\pgfsetdash{}{0pt}%
\pgfsys@defobject{currentmarker}{\pgfqpoint{-0.048611in}{0.000000in}}{\pgfqpoint{-0.000000in}{0.000000in}}{%
\pgfpathmoveto{\pgfqpoint{-0.000000in}{0.000000in}}%
\pgfpathlineto{\pgfqpoint{-0.048611in}{0.000000in}}%
\pgfusepath{stroke,fill}%
}%
\begin{pgfscope}%
\pgfsys@transformshift{0.708220in}{1.642970in}%
\pgfsys@useobject{currentmarker}{}%
\end{pgfscope}%
\end{pgfscope}%
\begin{pgfscope}%
\definecolor{textcolor}{rgb}{0.000000,0.000000,0.000000}%
\pgfsetstrokecolor{textcolor}%
\pgfsetfillcolor{textcolor}%
\pgftext[x=0.424657in, y=1.598245in, left, base]{\color{textcolor}\rmfamily\fontsize{9.000000}{10.800000}\selectfont \(\displaystyle {10^{0}}\)}%
\end{pgfscope}%
\begin{pgfscope}%
\pgfsetbuttcap%
\pgfsetroundjoin%
\definecolor{currentfill}{rgb}{0.000000,0.000000,0.000000}%
\pgfsetfillcolor{currentfill}%
\pgfsetlinewidth{0.803000pt}%
\definecolor{currentstroke}{rgb}{0.000000,0.000000,0.000000}%
\pgfsetstrokecolor{currentstroke}%
\pgfsetdash{}{0pt}%
\pgfsys@defobject{currentmarker}{\pgfqpoint{-0.048611in}{0.000000in}}{\pgfqpoint{-0.000000in}{0.000000in}}{%
\pgfpathmoveto{\pgfqpoint{-0.000000in}{0.000000in}}%
\pgfpathlineto{\pgfqpoint{-0.048611in}{0.000000in}}%
\pgfusepath{stroke,fill}%
}%
\begin{pgfscope}%
\pgfsys@transformshift{0.708220in}{2.124122in}%
\pgfsys@useobject{currentmarker}{}%
\end{pgfscope}%
\end{pgfscope}%
\begin{pgfscope}%
\definecolor{textcolor}{rgb}{0.000000,0.000000,0.000000}%
\pgfsetstrokecolor{textcolor}%
\pgfsetfillcolor{textcolor}%
\pgftext[x=0.424657in, y=2.079398in, left, base]{\color{textcolor}\rmfamily\fontsize{9.000000}{10.800000}\selectfont \(\displaystyle {10^{1}}\)}%
\end{pgfscope}%
\begin{pgfscope}%
\pgfsetbuttcap%
\pgfsetroundjoin%
\definecolor{currentfill}{rgb}{0.000000,0.000000,0.000000}%
\pgfsetfillcolor{currentfill}%
\pgfsetlinewidth{0.803000pt}%
\definecolor{currentstroke}{rgb}{0.000000,0.000000,0.000000}%
\pgfsetstrokecolor{currentstroke}%
\pgfsetdash{}{0pt}%
\pgfsys@defobject{currentmarker}{\pgfqpoint{-0.048611in}{0.000000in}}{\pgfqpoint{-0.000000in}{0.000000in}}{%
\pgfpathmoveto{\pgfqpoint{-0.000000in}{0.000000in}}%
\pgfpathlineto{\pgfqpoint{-0.048611in}{0.000000in}}%
\pgfusepath{stroke,fill}%
}%
\begin{pgfscope}%
\pgfsys@transformshift{0.708220in}{2.605275in}%
\pgfsys@useobject{currentmarker}{}%
\end{pgfscope}%
\end{pgfscope}%
\begin{pgfscope}%
\definecolor{textcolor}{rgb}{0.000000,0.000000,0.000000}%
\pgfsetstrokecolor{textcolor}%
\pgfsetfillcolor{textcolor}%
\pgftext[x=0.424657in, y=2.560550in, left, base]{\color{textcolor}\rmfamily\fontsize{9.000000}{10.800000}\selectfont \(\displaystyle {10^{2}}\)}%
\end{pgfscope}%
\begin{pgfscope}%
\pgfsetbuttcap%
\pgfsetroundjoin%
\definecolor{currentfill}{rgb}{0.000000,0.000000,0.000000}%
\pgfsetfillcolor{currentfill}%
\pgfsetlinewidth{0.602250pt}%
\definecolor{currentstroke}{rgb}{0.000000,0.000000,0.000000}%
\pgfsetstrokecolor{currentstroke}%
\pgfsetdash{}{0pt}%
\pgfsys@defobject{currentmarker}{\pgfqpoint{-0.027778in}{0.000000in}}{\pgfqpoint{-0.000000in}{0.000000in}}{%
\pgfpathmoveto{\pgfqpoint{-0.000000in}{0.000000in}}%
\pgfpathlineto{\pgfqpoint{-0.027778in}{0.000000in}}%
\pgfusepath{stroke,fill}%
}%
\begin{pgfscope}%
\pgfsys@transformshift{0.708220in}{0.535823in}%
\pgfsys@useobject{currentmarker}{}%
\end{pgfscope}%
\end{pgfscope}%
\begin{pgfscope}%
\pgfsetbuttcap%
\pgfsetroundjoin%
\definecolor{currentfill}{rgb}{0.000000,0.000000,0.000000}%
\pgfsetfillcolor{currentfill}%
\pgfsetlinewidth{0.602250pt}%
\definecolor{currentstroke}{rgb}{0.000000,0.000000,0.000000}%
\pgfsetstrokecolor{currentstroke}%
\pgfsetdash{}{0pt}%
\pgfsys@defobject{currentmarker}{\pgfqpoint{-0.027778in}{0.000000in}}{\pgfqpoint{-0.000000in}{0.000000in}}{%
\pgfpathmoveto{\pgfqpoint{-0.000000in}{0.000000in}}%
\pgfpathlineto{\pgfqpoint{-0.027778in}{0.000000in}}%
\pgfusepath{stroke,fill}%
}%
\begin{pgfscope}%
\pgfsys@transformshift{0.708220in}{0.573921in}%
\pgfsys@useobject{currentmarker}{}%
\end{pgfscope}%
\end{pgfscope}%
\begin{pgfscope}%
\pgfsetbuttcap%
\pgfsetroundjoin%
\definecolor{currentfill}{rgb}{0.000000,0.000000,0.000000}%
\pgfsetfillcolor{currentfill}%
\pgfsetlinewidth{0.602250pt}%
\definecolor{currentstroke}{rgb}{0.000000,0.000000,0.000000}%
\pgfsetstrokecolor{currentstroke}%
\pgfsetdash{}{0pt}%
\pgfsys@defobject{currentmarker}{\pgfqpoint{-0.027778in}{0.000000in}}{\pgfqpoint{-0.000000in}{0.000000in}}{%
\pgfpathmoveto{\pgfqpoint{-0.000000in}{0.000000in}}%
\pgfpathlineto{\pgfqpoint{-0.027778in}{0.000000in}}%
\pgfusepath{stroke,fill}%
}%
\begin{pgfscope}%
\pgfsys@transformshift{0.708220in}{0.606133in}%
\pgfsys@useobject{currentmarker}{}%
\end{pgfscope}%
\end{pgfscope}%
\begin{pgfscope}%
\pgfsetbuttcap%
\pgfsetroundjoin%
\definecolor{currentfill}{rgb}{0.000000,0.000000,0.000000}%
\pgfsetfillcolor{currentfill}%
\pgfsetlinewidth{0.602250pt}%
\definecolor{currentstroke}{rgb}{0.000000,0.000000,0.000000}%
\pgfsetstrokecolor{currentstroke}%
\pgfsetdash{}{0pt}%
\pgfsys@defobject{currentmarker}{\pgfqpoint{-0.027778in}{0.000000in}}{\pgfqpoint{-0.000000in}{0.000000in}}{%
\pgfpathmoveto{\pgfqpoint{-0.000000in}{0.000000in}}%
\pgfpathlineto{\pgfqpoint{-0.027778in}{0.000000in}}%
\pgfusepath{stroke,fill}%
}%
\begin{pgfscope}%
\pgfsys@transformshift{0.708220in}{0.634036in}%
\pgfsys@useobject{currentmarker}{}%
\end{pgfscope}%
\end{pgfscope}%
\begin{pgfscope}%
\pgfsetbuttcap%
\pgfsetroundjoin%
\definecolor{currentfill}{rgb}{0.000000,0.000000,0.000000}%
\pgfsetfillcolor{currentfill}%
\pgfsetlinewidth{0.602250pt}%
\definecolor{currentstroke}{rgb}{0.000000,0.000000,0.000000}%
\pgfsetstrokecolor{currentstroke}%
\pgfsetdash{}{0pt}%
\pgfsys@defobject{currentmarker}{\pgfqpoint{-0.027778in}{0.000000in}}{\pgfqpoint{-0.000000in}{0.000000in}}{%
\pgfpathmoveto{\pgfqpoint{-0.000000in}{0.000000in}}%
\pgfpathlineto{\pgfqpoint{-0.027778in}{0.000000in}}%
\pgfusepath{stroke,fill}%
}%
\begin{pgfscope}%
\pgfsys@transformshift{0.708220in}{0.658648in}%
\pgfsys@useobject{currentmarker}{}%
\end{pgfscope}%
\end{pgfscope}%
\begin{pgfscope}%
\pgfsetbuttcap%
\pgfsetroundjoin%
\definecolor{currentfill}{rgb}{0.000000,0.000000,0.000000}%
\pgfsetfillcolor{currentfill}%
\pgfsetlinewidth{0.602250pt}%
\definecolor{currentstroke}{rgb}{0.000000,0.000000,0.000000}%
\pgfsetstrokecolor{currentstroke}%
\pgfsetdash{}{0pt}%
\pgfsys@defobject{currentmarker}{\pgfqpoint{-0.027778in}{0.000000in}}{\pgfqpoint{-0.000000in}{0.000000in}}{%
\pgfpathmoveto{\pgfqpoint{-0.000000in}{0.000000in}}%
\pgfpathlineto{\pgfqpoint{-0.027778in}{0.000000in}}%
\pgfusepath{stroke,fill}%
}%
\begin{pgfscope}%
\pgfsys@transformshift{0.708220in}{0.825505in}%
\pgfsys@useobject{currentmarker}{}%
\end{pgfscope}%
\end{pgfscope}%
\begin{pgfscope}%
\pgfsetbuttcap%
\pgfsetroundjoin%
\definecolor{currentfill}{rgb}{0.000000,0.000000,0.000000}%
\pgfsetfillcolor{currentfill}%
\pgfsetlinewidth{0.602250pt}%
\definecolor{currentstroke}{rgb}{0.000000,0.000000,0.000000}%
\pgfsetstrokecolor{currentstroke}%
\pgfsetdash{}{0pt}%
\pgfsys@defobject{currentmarker}{\pgfqpoint{-0.027778in}{0.000000in}}{\pgfqpoint{-0.000000in}{0.000000in}}{%
\pgfpathmoveto{\pgfqpoint{-0.000000in}{0.000000in}}%
\pgfpathlineto{\pgfqpoint{-0.027778in}{0.000000in}}%
\pgfusepath{stroke,fill}%
}%
\begin{pgfscope}%
\pgfsys@transformshift{0.708220in}{0.910232in}%
\pgfsys@useobject{currentmarker}{}%
\end{pgfscope}%
\end{pgfscope}%
\begin{pgfscope}%
\pgfsetbuttcap%
\pgfsetroundjoin%
\definecolor{currentfill}{rgb}{0.000000,0.000000,0.000000}%
\pgfsetfillcolor{currentfill}%
\pgfsetlinewidth{0.602250pt}%
\definecolor{currentstroke}{rgb}{0.000000,0.000000,0.000000}%
\pgfsetstrokecolor{currentstroke}%
\pgfsetdash{}{0pt}%
\pgfsys@defobject{currentmarker}{\pgfqpoint{-0.027778in}{0.000000in}}{\pgfqpoint{-0.000000in}{0.000000in}}{%
\pgfpathmoveto{\pgfqpoint{-0.000000in}{0.000000in}}%
\pgfpathlineto{\pgfqpoint{-0.027778in}{0.000000in}}%
\pgfusepath{stroke,fill}%
}%
\begin{pgfscope}%
\pgfsys@transformshift{0.708220in}{0.970347in}%
\pgfsys@useobject{currentmarker}{}%
\end{pgfscope}%
\end{pgfscope}%
\begin{pgfscope}%
\pgfsetbuttcap%
\pgfsetroundjoin%
\definecolor{currentfill}{rgb}{0.000000,0.000000,0.000000}%
\pgfsetfillcolor{currentfill}%
\pgfsetlinewidth{0.602250pt}%
\definecolor{currentstroke}{rgb}{0.000000,0.000000,0.000000}%
\pgfsetstrokecolor{currentstroke}%
\pgfsetdash{}{0pt}%
\pgfsys@defobject{currentmarker}{\pgfqpoint{-0.027778in}{0.000000in}}{\pgfqpoint{-0.000000in}{0.000000in}}{%
\pgfpathmoveto{\pgfqpoint{-0.000000in}{0.000000in}}%
\pgfpathlineto{\pgfqpoint{-0.027778in}{0.000000in}}%
\pgfusepath{stroke,fill}%
}%
\begin{pgfscope}%
\pgfsys@transformshift{0.708220in}{1.016975in}%
\pgfsys@useobject{currentmarker}{}%
\end{pgfscope}%
\end{pgfscope}%
\begin{pgfscope}%
\pgfsetbuttcap%
\pgfsetroundjoin%
\definecolor{currentfill}{rgb}{0.000000,0.000000,0.000000}%
\pgfsetfillcolor{currentfill}%
\pgfsetlinewidth{0.602250pt}%
\definecolor{currentstroke}{rgb}{0.000000,0.000000,0.000000}%
\pgfsetstrokecolor{currentstroke}%
\pgfsetdash{}{0pt}%
\pgfsys@defobject{currentmarker}{\pgfqpoint{-0.027778in}{0.000000in}}{\pgfqpoint{-0.000000in}{0.000000in}}{%
\pgfpathmoveto{\pgfqpoint{-0.000000in}{0.000000in}}%
\pgfpathlineto{\pgfqpoint{-0.027778in}{0.000000in}}%
\pgfusepath{stroke,fill}%
}%
\begin{pgfscope}%
\pgfsys@transformshift{0.708220in}{1.055074in}%
\pgfsys@useobject{currentmarker}{}%
\end{pgfscope}%
\end{pgfscope}%
\begin{pgfscope}%
\pgfsetbuttcap%
\pgfsetroundjoin%
\definecolor{currentfill}{rgb}{0.000000,0.000000,0.000000}%
\pgfsetfillcolor{currentfill}%
\pgfsetlinewidth{0.602250pt}%
\definecolor{currentstroke}{rgb}{0.000000,0.000000,0.000000}%
\pgfsetstrokecolor{currentstroke}%
\pgfsetdash{}{0pt}%
\pgfsys@defobject{currentmarker}{\pgfqpoint{-0.027778in}{0.000000in}}{\pgfqpoint{-0.000000in}{0.000000in}}{%
\pgfpathmoveto{\pgfqpoint{-0.000000in}{0.000000in}}%
\pgfpathlineto{\pgfqpoint{-0.027778in}{0.000000in}}%
\pgfusepath{stroke,fill}%
}%
\begin{pgfscope}%
\pgfsys@transformshift{0.708220in}{1.087285in}%
\pgfsys@useobject{currentmarker}{}%
\end{pgfscope}%
\end{pgfscope}%
\begin{pgfscope}%
\pgfsetbuttcap%
\pgfsetroundjoin%
\definecolor{currentfill}{rgb}{0.000000,0.000000,0.000000}%
\pgfsetfillcolor{currentfill}%
\pgfsetlinewidth{0.602250pt}%
\definecolor{currentstroke}{rgb}{0.000000,0.000000,0.000000}%
\pgfsetstrokecolor{currentstroke}%
\pgfsetdash{}{0pt}%
\pgfsys@defobject{currentmarker}{\pgfqpoint{-0.027778in}{0.000000in}}{\pgfqpoint{-0.000000in}{0.000000in}}{%
\pgfpathmoveto{\pgfqpoint{-0.000000in}{0.000000in}}%
\pgfpathlineto{\pgfqpoint{-0.027778in}{0.000000in}}%
\pgfusepath{stroke,fill}%
}%
\begin{pgfscope}%
\pgfsys@transformshift{0.708220in}{1.115188in}%
\pgfsys@useobject{currentmarker}{}%
\end{pgfscope}%
\end{pgfscope}%
\begin{pgfscope}%
\pgfsetbuttcap%
\pgfsetroundjoin%
\definecolor{currentfill}{rgb}{0.000000,0.000000,0.000000}%
\pgfsetfillcolor{currentfill}%
\pgfsetlinewidth{0.602250pt}%
\definecolor{currentstroke}{rgb}{0.000000,0.000000,0.000000}%
\pgfsetstrokecolor{currentstroke}%
\pgfsetdash{}{0pt}%
\pgfsys@defobject{currentmarker}{\pgfqpoint{-0.027778in}{0.000000in}}{\pgfqpoint{-0.000000in}{0.000000in}}{%
\pgfpathmoveto{\pgfqpoint{-0.000000in}{0.000000in}}%
\pgfpathlineto{\pgfqpoint{-0.027778in}{0.000000in}}%
\pgfusepath{stroke,fill}%
}%
\begin{pgfscope}%
\pgfsys@transformshift{0.708220in}{1.139801in}%
\pgfsys@useobject{currentmarker}{}%
\end{pgfscope}%
\end{pgfscope}%
\begin{pgfscope}%
\pgfsetbuttcap%
\pgfsetroundjoin%
\definecolor{currentfill}{rgb}{0.000000,0.000000,0.000000}%
\pgfsetfillcolor{currentfill}%
\pgfsetlinewidth{0.602250pt}%
\definecolor{currentstroke}{rgb}{0.000000,0.000000,0.000000}%
\pgfsetstrokecolor{currentstroke}%
\pgfsetdash{}{0pt}%
\pgfsys@defobject{currentmarker}{\pgfqpoint{-0.027778in}{0.000000in}}{\pgfqpoint{-0.000000in}{0.000000in}}{%
\pgfpathmoveto{\pgfqpoint{-0.000000in}{0.000000in}}%
\pgfpathlineto{\pgfqpoint{-0.027778in}{0.000000in}}%
\pgfusepath{stroke,fill}%
}%
\begin{pgfscope}%
\pgfsys@transformshift{0.708220in}{1.306658in}%
\pgfsys@useobject{currentmarker}{}%
\end{pgfscope}%
\end{pgfscope}%
\begin{pgfscope}%
\pgfsetbuttcap%
\pgfsetroundjoin%
\definecolor{currentfill}{rgb}{0.000000,0.000000,0.000000}%
\pgfsetfillcolor{currentfill}%
\pgfsetlinewidth{0.602250pt}%
\definecolor{currentstroke}{rgb}{0.000000,0.000000,0.000000}%
\pgfsetstrokecolor{currentstroke}%
\pgfsetdash{}{0pt}%
\pgfsys@defobject{currentmarker}{\pgfqpoint{-0.027778in}{0.000000in}}{\pgfqpoint{-0.000000in}{0.000000in}}{%
\pgfpathmoveto{\pgfqpoint{-0.000000in}{0.000000in}}%
\pgfpathlineto{\pgfqpoint{-0.027778in}{0.000000in}}%
\pgfusepath{stroke,fill}%
}%
\begin{pgfscope}%
\pgfsys@transformshift{0.708220in}{1.391385in}%
\pgfsys@useobject{currentmarker}{}%
\end{pgfscope}%
\end{pgfscope}%
\begin{pgfscope}%
\pgfsetbuttcap%
\pgfsetroundjoin%
\definecolor{currentfill}{rgb}{0.000000,0.000000,0.000000}%
\pgfsetfillcolor{currentfill}%
\pgfsetlinewidth{0.602250pt}%
\definecolor{currentstroke}{rgb}{0.000000,0.000000,0.000000}%
\pgfsetstrokecolor{currentstroke}%
\pgfsetdash{}{0pt}%
\pgfsys@defobject{currentmarker}{\pgfqpoint{-0.027778in}{0.000000in}}{\pgfqpoint{-0.000000in}{0.000000in}}{%
\pgfpathmoveto{\pgfqpoint{-0.000000in}{0.000000in}}%
\pgfpathlineto{\pgfqpoint{-0.027778in}{0.000000in}}%
\pgfusepath{stroke,fill}%
}%
\begin{pgfscope}%
\pgfsys@transformshift{0.708220in}{1.451500in}%
\pgfsys@useobject{currentmarker}{}%
\end{pgfscope}%
\end{pgfscope}%
\begin{pgfscope}%
\pgfsetbuttcap%
\pgfsetroundjoin%
\definecolor{currentfill}{rgb}{0.000000,0.000000,0.000000}%
\pgfsetfillcolor{currentfill}%
\pgfsetlinewidth{0.602250pt}%
\definecolor{currentstroke}{rgb}{0.000000,0.000000,0.000000}%
\pgfsetstrokecolor{currentstroke}%
\pgfsetdash{}{0pt}%
\pgfsys@defobject{currentmarker}{\pgfqpoint{-0.027778in}{0.000000in}}{\pgfqpoint{-0.000000in}{0.000000in}}{%
\pgfpathmoveto{\pgfqpoint{-0.000000in}{0.000000in}}%
\pgfpathlineto{\pgfqpoint{-0.027778in}{0.000000in}}%
\pgfusepath{stroke,fill}%
}%
\begin{pgfscope}%
\pgfsys@transformshift{0.708220in}{1.498128in}%
\pgfsys@useobject{currentmarker}{}%
\end{pgfscope}%
\end{pgfscope}%
\begin{pgfscope}%
\pgfsetbuttcap%
\pgfsetroundjoin%
\definecolor{currentfill}{rgb}{0.000000,0.000000,0.000000}%
\pgfsetfillcolor{currentfill}%
\pgfsetlinewidth{0.602250pt}%
\definecolor{currentstroke}{rgb}{0.000000,0.000000,0.000000}%
\pgfsetstrokecolor{currentstroke}%
\pgfsetdash{}{0pt}%
\pgfsys@defobject{currentmarker}{\pgfqpoint{-0.027778in}{0.000000in}}{\pgfqpoint{-0.000000in}{0.000000in}}{%
\pgfpathmoveto{\pgfqpoint{-0.000000in}{0.000000in}}%
\pgfpathlineto{\pgfqpoint{-0.027778in}{0.000000in}}%
\pgfusepath{stroke,fill}%
}%
\begin{pgfscope}%
\pgfsys@transformshift{0.708220in}{1.536226in}%
\pgfsys@useobject{currentmarker}{}%
\end{pgfscope}%
\end{pgfscope}%
\begin{pgfscope}%
\pgfsetbuttcap%
\pgfsetroundjoin%
\definecolor{currentfill}{rgb}{0.000000,0.000000,0.000000}%
\pgfsetfillcolor{currentfill}%
\pgfsetlinewidth{0.602250pt}%
\definecolor{currentstroke}{rgb}{0.000000,0.000000,0.000000}%
\pgfsetstrokecolor{currentstroke}%
\pgfsetdash{}{0pt}%
\pgfsys@defobject{currentmarker}{\pgfqpoint{-0.027778in}{0.000000in}}{\pgfqpoint{-0.000000in}{0.000000in}}{%
\pgfpathmoveto{\pgfqpoint{-0.000000in}{0.000000in}}%
\pgfpathlineto{\pgfqpoint{-0.027778in}{0.000000in}}%
\pgfusepath{stroke,fill}%
}%
\begin{pgfscope}%
\pgfsys@transformshift{0.708220in}{1.568438in}%
\pgfsys@useobject{currentmarker}{}%
\end{pgfscope}%
\end{pgfscope}%
\begin{pgfscope}%
\pgfsetbuttcap%
\pgfsetroundjoin%
\definecolor{currentfill}{rgb}{0.000000,0.000000,0.000000}%
\pgfsetfillcolor{currentfill}%
\pgfsetlinewidth{0.602250pt}%
\definecolor{currentstroke}{rgb}{0.000000,0.000000,0.000000}%
\pgfsetstrokecolor{currentstroke}%
\pgfsetdash{}{0pt}%
\pgfsys@defobject{currentmarker}{\pgfqpoint{-0.027778in}{0.000000in}}{\pgfqpoint{-0.000000in}{0.000000in}}{%
\pgfpathmoveto{\pgfqpoint{-0.000000in}{0.000000in}}%
\pgfpathlineto{\pgfqpoint{-0.027778in}{0.000000in}}%
\pgfusepath{stroke,fill}%
}%
\begin{pgfscope}%
\pgfsys@transformshift{0.708220in}{1.596341in}%
\pgfsys@useobject{currentmarker}{}%
\end{pgfscope}%
\end{pgfscope}%
\begin{pgfscope}%
\pgfsetbuttcap%
\pgfsetroundjoin%
\definecolor{currentfill}{rgb}{0.000000,0.000000,0.000000}%
\pgfsetfillcolor{currentfill}%
\pgfsetlinewidth{0.602250pt}%
\definecolor{currentstroke}{rgb}{0.000000,0.000000,0.000000}%
\pgfsetstrokecolor{currentstroke}%
\pgfsetdash{}{0pt}%
\pgfsys@defobject{currentmarker}{\pgfqpoint{-0.027778in}{0.000000in}}{\pgfqpoint{-0.000000in}{0.000000in}}{%
\pgfpathmoveto{\pgfqpoint{-0.000000in}{0.000000in}}%
\pgfpathlineto{\pgfqpoint{-0.027778in}{0.000000in}}%
\pgfusepath{stroke,fill}%
}%
\begin{pgfscope}%
\pgfsys@transformshift{0.708220in}{1.620953in}%
\pgfsys@useobject{currentmarker}{}%
\end{pgfscope}%
\end{pgfscope}%
\begin{pgfscope}%
\pgfsetbuttcap%
\pgfsetroundjoin%
\definecolor{currentfill}{rgb}{0.000000,0.000000,0.000000}%
\pgfsetfillcolor{currentfill}%
\pgfsetlinewidth{0.602250pt}%
\definecolor{currentstroke}{rgb}{0.000000,0.000000,0.000000}%
\pgfsetstrokecolor{currentstroke}%
\pgfsetdash{}{0pt}%
\pgfsys@defobject{currentmarker}{\pgfqpoint{-0.027778in}{0.000000in}}{\pgfqpoint{-0.000000in}{0.000000in}}{%
\pgfpathmoveto{\pgfqpoint{-0.000000in}{0.000000in}}%
\pgfpathlineto{\pgfqpoint{-0.027778in}{0.000000in}}%
\pgfusepath{stroke,fill}%
}%
\begin{pgfscope}%
\pgfsys@transformshift{0.708220in}{1.787811in}%
\pgfsys@useobject{currentmarker}{}%
\end{pgfscope}%
\end{pgfscope}%
\begin{pgfscope}%
\pgfsetbuttcap%
\pgfsetroundjoin%
\definecolor{currentfill}{rgb}{0.000000,0.000000,0.000000}%
\pgfsetfillcolor{currentfill}%
\pgfsetlinewidth{0.602250pt}%
\definecolor{currentstroke}{rgb}{0.000000,0.000000,0.000000}%
\pgfsetstrokecolor{currentstroke}%
\pgfsetdash{}{0pt}%
\pgfsys@defobject{currentmarker}{\pgfqpoint{-0.027778in}{0.000000in}}{\pgfqpoint{-0.000000in}{0.000000in}}{%
\pgfpathmoveto{\pgfqpoint{-0.000000in}{0.000000in}}%
\pgfpathlineto{\pgfqpoint{-0.027778in}{0.000000in}}%
\pgfusepath{stroke,fill}%
}%
\begin{pgfscope}%
\pgfsys@transformshift{0.708220in}{1.872538in}%
\pgfsys@useobject{currentmarker}{}%
\end{pgfscope}%
\end{pgfscope}%
\begin{pgfscope}%
\pgfsetbuttcap%
\pgfsetroundjoin%
\definecolor{currentfill}{rgb}{0.000000,0.000000,0.000000}%
\pgfsetfillcolor{currentfill}%
\pgfsetlinewidth{0.602250pt}%
\definecolor{currentstroke}{rgb}{0.000000,0.000000,0.000000}%
\pgfsetstrokecolor{currentstroke}%
\pgfsetdash{}{0pt}%
\pgfsys@defobject{currentmarker}{\pgfqpoint{-0.027778in}{0.000000in}}{\pgfqpoint{-0.000000in}{0.000000in}}{%
\pgfpathmoveto{\pgfqpoint{-0.000000in}{0.000000in}}%
\pgfpathlineto{\pgfqpoint{-0.027778in}{0.000000in}}%
\pgfusepath{stroke,fill}%
}%
\begin{pgfscope}%
\pgfsys@transformshift{0.708220in}{1.932652in}%
\pgfsys@useobject{currentmarker}{}%
\end{pgfscope}%
\end{pgfscope}%
\begin{pgfscope}%
\pgfsetbuttcap%
\pgfsetroundjoin%
\definecolor{currentfill}{rgb}{0.000000,0.000000,0.000000}%
\pgfsetfillcolor{currentfill}%
\pgfsetlinewidth{0.602250pt}%
\definecolor{currentstroke}{rgb}{0.000000,0.000000,0.000000}%
\pgfsetstrokecolor{currentstroke}%
\pgfsetdash{}{0pt}%
\pgfsys@defobject{currentmarker}{\pgfqpoint{-0.027778in}{0.000000in}}{\pgfqpoint{-0.000000in}{0.000000in}}{%
\pgfpathmoveto{\pgfqpoint{-0.000000in}{0.000000in}}%
\pgfpathlineto{\pgfqpoint{-0.027778in}{0.000000in}}%
\pgfusepath{stroke,fill}%
}%
\begin{pgfscope}%
\pgfsys@transformshift{0.708220in}{1.979281in}%
\pgfsys@useobject{currentmarker}{}%
\end{pgfscope}%
\end{pgfscope}%
\begin{pgfscope}%
\pgfsetbuttcap%
\pgfsetroundjoin%
\definecolor{currentfill}{rgb}{0.000000,0.000000,0.000000}%
\pgfsetfillcolor{currentfill}%
\pgfsetlinewidth{0.602250pt}%
\definecolor{currentstroke}{rgb}{0.000000,0.000000,0.000000}%
\pgfsetstrokecolor{currentstroke}%
\pgfsetdash{}{0pt}%
\pgfsys@defobject{currentmarker}{\pgfqpoint{-0.027778in}{0.000000in}}{\pgfqpoint{-0.000000in}{0.000000in}}{%
\pgfpathmoveto{\pgfqpoint{-0.000000in}{0.000000in}}%
\pgfpathlineto{\pgfqpoint{-0.027778in}{0.000000in}}%
\pgfusepath{stroke,fill}%
}%
\begin{pgfscope}%
\pgfsys@transformshift{0.708220in}{2.017379in}%
\pgfsys@useobject{currentmarker}{}%
\end{pgfscope}%
\end{pgfscope}%
\begin{pgfscope}%
\pgfsetbuttcap%
\pgfsetroundjoin%
\definecolor{currentfill}{rgb}{0.000000,0.000000,0.000000}%
\pgfsetfillcolor{currentfill}%
\pgfsetlinewidth{0.602250pt}%
\definecolor{currentstroke}{rgb}{0.000000,0.000000,0.000000}%
\pgfsetstrokecolor{currentstroke}%
\pgfsetdash{}{0pt}%
\pgfsys@defobject{currentmarker}{\pgfqpoint{-0.027778in}{0.000000in}}{\pgfqpoint{-0.000000in}{0.000000in}}{%
\pgfpathmoveto{\pgfqpoint{-0.000000in}{0.000000in}}%
\pgfpathlineto{\pgfqpoint{-0.027778in}{0.000000in}}%
\pgfusepath{stroke,fill}%
}%
\begin{pgfscope}%
\pgfsys@transformshift{0.708220in}{2.049591in}%
\pgfsys@useobject{currentmarker}{}%
\end{pgfscope}%
\end{pgfscope}%
\begin{pgfscope}%
\pgfsetbuttcap%
\pgfsetroundjoin%
\definecolor{currentfill}{rgb}{0.000000,0.000000,0.000000}%
\pgfsetfillcolor{currentfill}%
\pgfsetlinewidth{0.602250pt}%
\definecolor{currentstroke}{rgb}{0.000000,0.000000,0.000000}%
\pgfsetstrokecolor{currentstroke}%
\pgfsetdash{}{0pt}%
\pgfsys@defobject{currentmarker}{\pgfqpoint{-0.027778in}{0.000000in}}{\pgfqpoint{-0.000000in}{0.000000in}}{%
\pgfpathmoveto{\pgfqpoint{-0.000000in}{0.000000in}}%
\pgfpathlineto{\pgfqpoint{-0.027778in}{0.000000in}}%
\pgfusepath{stroke,fill}%
}%
\begin{pgfscope}%
\pgfsys@transformshift{0.708220in}{2.077494in}%
\pgfsys@useobject{currentmarker}{}%
\end{pgfscope}%
\end{pgfscope}%
\begin{pgfscope}%
\pgfsetbuttcap%
\pgfsetroundjoin%
\definecolor{currentfill}{rgb}{0.000000,0.000000,0.000000}%
\pgfsetfillcolor{currentfill}%
\pgfsetlinewidth{0.602250pt}%
\definecolor{currentstroke}{rgb}{0.000000,0.000000,0.000000}%
\pgfsetstrokecolor{currentstroke}%
\pgfsetdash{}{0pt}%
\pgfsys@defobject{currentmarker}{\pgfqpoint{-0.027778in}{0.000000in}}{\pgfqpoint{-0.000000in}{0.000000in}}{%
\pgfpathmoveto{\pgfqpoint{-0.000000in}{0.000000in}}%
\pgfpathlineto{\pgfqpoint{-0.027778in}{0.000000in}}%
\pgfusepath{stroke,fill}%
}%
\begin{pgfscope}%
\pgfsys@transformshift{0.708220in}{2.102106in}%
\pgfsys@useobject{currentmarker}{}%
\end{pgfscope}%
\end{pgfscope}%
\begin{pgfscope}%
\pgfsetbuttcap%
\pgfsetroundjoin%
\definecolor{currentfill}{rgb}{0.000000,0.000000,0.000000}%
\pgfsetfillcolor{currentfill}%
\pgfsetlinewidth{0.602250pt}%
\definecolor{currentstroke}{rgb}{0.000000,0.000000,0.000000}%
\pgfsetstrokecolor{currentstroke}%
\pgfsetdash{}{0pt}%
\pgfsys@defobject{currentmarker}{\pgfqpoint{-0.027778in}{0.000000in}}{\pgfqpoint{-0.000000in}{0.000000in}}{%
\pgfpathmoveto{\pgfqpoint{-0.000000in}{0.000000in}}%
\pgfpathlineto{\pgfqpoint{-0.027778in}{0.000000in}}%
\pgfusepath{stroke,fill}%
}%
\begin{pgfscope}%
\pgfsys@transformshift{0.708220in}{2.268964in}%
\pgfsys@useobject{currentmarker}{}%
\end{pgfscope}%
\end{pgfscope}%
\begin{pgfscope}%
\pgfsetbuttcap%
\pgfsetroundjoin%
\definecolor{currentfill}{rgb}{0.000000,0.000000,0.000000}%
\pgfsetfillcolor{currentfill}%
\pgfsetlinewidth{0.602250pt}%
\definecolor{currentstroke}{rgb}{0.000000,0.000000,0.000000}%
\pgfsetstrokecolor{currentstroke}%
\pgfsetdash{}{0pt}%
\pgfsys@defobject{currentmarker}{\pgfqpoint{-0.027778in}{0.000000in}}{\pgfqpoint{-0.000000in}{0.000000in}}{%
\pgfpathmoveto{\pgfqpoint{-0.000000in}{0.000000in}}%
\pgfpathlineto{\pgfqpoint{-0.027778in}{0.000000in}}%
\pgfusepath{stroke,fill}%
}%
\begin{pgfscope}%
\pgfsys@transformshift{0.708220in}{2.353691in}%
\pgfsys@useobject{currentmarker}{}%
\end{pgfscope}%
\end{pgfscope}%
\begin{pgfscope}%
\pgfsetbuttcap%
\pgfsetroundjoin%
\definecolor{currentfill}{rgb}{0.000000,0.000000,0.000000}%
\pgfsetfillcolor{currentfill}%
\pgfsetlinewidth{0.602250pt}%
\definecolor{currentstroke}{rgb}{0.000000,0.000000,0.000000}%
\pgfsetstrokecolor{currentstroke}%
\pgfsetdash{}{0pt}%
\pgfsys@defobject{currentmarker}{\pgfqpoint{-0.027778in}{0.000000in}}{\pgfqpoint{-0.000000in}{0.000000in}}{%
\pgfpathmoveto{\pgfqpoint{-0.000000in}{0.000000in}}%
\pgfpathlineto{\pgfqpoint{-0.027778in}{0.000000in}}%
\pgfusepath{stroke,fill}%
}%
\begin{pgfscope}%
\pgfsys@transformshift{0.708220in}{2.413805in}%
\pgfsys@useobject{currentmarker}{}%
\end{pgfscope}%
\end{pgfscope}%
\begin{pgfscope}%
\pgfsetbuttcap%
\pgfsetroundjoin%
\definecolor{currentfill}{rgb}{0.000000,0.000000,0.000000}%
\pgfsetfillcolor{currentfill}%
\pgfsetlinewidth{0.602250pt}%
\definecolor{currentstroke}{rgb}{0.000000,0.000000,0.000000}%
\pgfsetstrokecolor{currentstroke}%
\pgfsetdash{}{0pt}%
\pgfsys@defobject{currentmarker}{\pgfqpoint{-0.027778in}{0.000000in}}{\pgfqpoint{-0.000000in}{0.000000in}}{%
\pgfpathmoveto{\pgfqpoint{-0.000000in}{0.000000in}}%
\pgfpathlineto{\pgfqpoint{-0.027778in}{0.000000in}}%
\pgfusepath{stroke,fill}%
}%
\begin{pgfscope}%
\pgfsys@transformshift{0.708220in}{2.460434in}%
\pgfsys@useobject{currentmarker}{}%
\end{pgfscope}%
\end{pgfscope}%
\begin{pgfscope}%
\pgfsetbuttcap%
\pgfsetroundjoin%
\definecolor{currentfill}{rgb}{0.000000,0.000000,0.000000}%
\pgfsetfillcolor{currentfill}%
\pgfsetlinewidth{0.602250pt}%
\definecolor{currentstroke}{rgb}{0.000000,0.000000,0.000000}%
\pgfsetstrokecolor{currentstroke}%
\pgfsetdash{}{0pt}%
\pgfsys@defobject{currentmarker}{\pgfqpoint{-0.027778in}{0.000000in}}{\pgfqpoint{-0.000000in}{0.000000in}}{%
\pgfpathmoveto{\pgfqpoint{-0.000000in}{0.000000in}}%
\pgfpathlineto{\pgfqpoint{-0.027778in}{0.000000in}}%
\pgfusepath{stroke,fill}%
}%
\begin{pgfscope}%
\pgfsys@transformshift{0.708220in}{2.498532in}%
\pgfsys@useobject{currentmarker}{}%
\end{pgfscope}%
\end{pgfscope}%
\begin{pgfscope}%
\pgfsetbuttcap%
\pgfsetroundjoin%
\definecolor{currentfill}{rgb}{0.000000,0.000000,0.000000}%
\pgfsetfillcolor{currentfill}%
\pgfsetlinewidth{0.602250pt}%
\definecolor{currentstroke}{rgb}{0.000000,0.000000,0.000000}%
\pgfsetstrokecolor{currentstroke}%
\pgfsetdash{}{0pt}%
\pgfsys@defobject{currentmarker}{\pgfqpoint{-0.027778in}{0.000000in}}{\pgfqpoint{-0.000000in}{0.000000in}}{%
\pgfpathmoveto{\pgfqpoint{-0.000000in}{0.000000in}}%
\pgfpathlineto{\pgfqpoint{-0.027778in}{0.000000in}}%
\pgfusepath{stroke,fill}%
}%
\begin{pgfscope}%
\pgfsys@transformshift{0.708220in}{2.530744in}%
\pgfsys@useobject{currentmarker}{}%
\end{pgfscope}%
\end{pgfscope}%
\begin{pgfscope}%
\pgfsetbuttcap%
\pgfsetroundjoin%
\definecolor{currentfill}{rgb}{0.000000,0.000000,0.000000}%
\pgfsetfillcolor{currentfill}%
\pgfsetlinewidth{0.602250pt}%
\definecolor{currentstroke}{rgb}{0.000000,0.000000,0.000000}%
\pgfsetstrokecolor{currentstroke}%
\pgfsetdash{}{0pt}%
\pgfsys@defobject{currentmarker}{\pgfqpoint{-0.027778in}{0.000000in}}{\pgfqpoint{-0.000000in}{0.000000in}}{%
\pgfpathmoveto{\pgfqpoint{-0.000000in}{0.000000in}}%
\pgfpathlineto{\pgfqpoint{-0.027778in}{0.000000in}}%
\pgfusepath{stroke,fill}%
}%
\begin{pgfscope}%
\pgfsys@transformshift{0.708220in}{2.558647in}%
\pgfsys@useobject{currentmarker}{}%
\end{pgfscope}%
\end{pgfscope}%
\begin{pgfscope}%
\pgfsetbuttcap%
\pgfsetroundjoin%
\definecolor{currentfill}{rgb}{0.000000,0.000000,0.000000}%
\pgfsetfillcolor{currentfill}%
\pgfsetlinewidth{0.602250pt}%
\definecolor{currentstroke}{rgb}{0.000000,0.000000,0.000000}%
\pgfsetstrokecolor{currentstroke}%
\pgfsetdash{}{0pt}%
\pgfsys@defobject{currentmarker}{\pgfqpoint{-0.027778in}{0.000000in}}{\pgfqpoint{-0.000000in}{0.000000in}}{%
\pgfpathmoveto{\pgfqpoint{-0.000000in}{0.000000in}}%
\pgfpathlineto{\pgfqpoint{-0.027778in}{0.000000in}}%
\pgfusepath{stroke,fill}%
}%
\begin{pgfscope}%
\pgfsys@transformshift{0.708220in}{2.583259in}%
\pgfsys@useobject{currentmarker}{}%
\end{pgfscope}%
\end{pgfscope}%
\begin{pgfscope}%
\definecolor{textcolor}{rgb}{0.000000,0.000000,0.000000}%
\pgfsetstrokecolor{textcolor}%
\pgfsetfillcolor{textcolor}%
\pgftext[x=0.288855in,y=1.570549in,,bottom,rotate=90.000000]{\color{textcolor}\rmfamily\fontsize{10.000000}{12.000000}\selectfont Longest solving time (s)}%
\end{pgfscope}%
\begin{pgfscope}%
\pgfpathrectangle{\pgfqpoint{0.708220in}{0.535823in}}{\pgfqpoint{5.045427in}{2.069453in}}%
\pgfusepath{clip}%
\pgfsetrectcap%
\pgfsetroundjoin%
\pgfsetlinewidth{1.003750pt}%
\definecolor{currentstroke}{rgb}{0.121569,0.466667,0.705882}%
\pgfsetstrokecolor{currentstroke}%
\pgfsetdash{}{0pt}%
\pgfpathmoveto{\pgfqpoint{0.906891in}{0.525823in}}%
\pgfpathlineto{\pgfqpoint{0.910037in}{0.562690in}}%
\pgfpathlineto{\pgfqpoint{0.935264in}{0.566435in}}%
\pgfpathlineto{\pgfqpoint{0.960491in}{0.579726in}}%
\pgfpathlineto{\pgfqpoint{0.985718in}{0.584916in}}%
\pgfpathlineto{\pgfqpoint{1.010945in}{0.602671in}}%
\pgfpathlineto{\pgfqpoint{1.036173in}{0.622093in}}%
\pgfpathlineto{\pgfqpoint{1.061400in}{0.636609in}}%
\pgfpathlineto{\pgfqpoint{1.086627in}{0.643266in}}%
\pgfpathlineto{\pgfqpoint{1.111854in}{0.645054in}}%
\pgfpathlineto{\pgfqpoint{1.137081in}{0.648157in}}%
\pgfpathlineto{\pgfqpoint{1.162308in}{0.649635in}}%
\pgfpathlineto{\pgfqpoint{1.187535in}{0.659759in}}%
\pgfpathlineto{\pgfqpoint{1.212763in}{0.666407in}}%
\pgfpathlineto{\pgfqpoint{1.237990in}{0.672612in}}%
\pgfpathlineto{\pgfqpoint{1.263217in}{0.674164in}}%
\pgfpathlineto{\pgfqpoint{1.288444in}{0.680411in}}%
\pgfpathlineto{\pgfqpoint{1.313671in}{0.687133in}}%
\pgfpathlineto{\pgfqpoint{1.338898in}{0.687701in}}%
\pgfpathlineto{\pgfqpoint{1.364125in}{0.693549in}}%
\pgfpathlineto{\pgfqpoint{1.389352in}{0.698998in}}%
\pgfpathlineto{\pgfqpoint{1.414580in}{0.700985in}}%
\pgfpathlineto{\pgfqpoint{1.439807in}{0.704830in}}%
\pgfpathlineto{\pgfqpoint{1.465034in}{0.705790in}}%
\pgfpathlineto{\pgfqpoint{1.490261in}{0.710877in}}%
\pgfpathlineto{\pgfqpoint{1.515488in}{0.714518in}}%
\pgfpathlineto{\pgfqpoint{1.540715in}{0.719213in}}%
\pgfpathlineto{\pgfqpoint{1.565942in}{0.721575in}}%
\pgfpathlineto{\pgfqpoint{1.591170in}{0.724190in}}%
\pgfpathlineto{\pgfqpoint{1.616397in}{0.725614in}}%
\pgfpathlineto{\pgfqpoint{1.641624in}{0.727441in}}%
\pgfpathlineto{\pgfqpoint{1.666851in}{0.727732in}}%
\pgfpathlineto{\pgfqpoint{1.692078in}{0.727922in}}%
\pgfpathlineto{\pgfqpoint{1.717305in}{0.730978in}}%
\pgfpathlineto{\pgfqpoint{1.742532in}{0.731678in}}%
\pgfpathlineto{\pgfqpoint{1.767759in}{0.733029in}}%
\pgfpathlineto{\pgfqpoint{1.792987in}{0.733753in}}%
\pgfpathlineto{\pgfqpoint{1.818214in}{0.734019in}}%
\pgfpathlineto{\pgfqpoint{1.843441in}{0.736246in}}%
\pgfpathlineto{\pgfqpoint{1.868668in}{0.736599in}}%
\pgfpathlineto{\pgfqpoint{1.893895in}{0.737727in}}%
\pgfpathlineto{\pgfqpoint{1.919122in}{0.739486in}}%
\pgfpathlineto{\pgfqpoint{1.944349in}{0.739599in}}%
\pgfpathlineto{\pgfqpoint{1.969577in}{0.740609in}}%
\pgfpathlineto{\pgfqpoint{1.994804in}{0.740835in}}%
\pgfpathlineto{\pgfqpoint{2.020031in}{0.747113in}}%
\pgfpathlineto{\pgfqpoint{2.045258in}{0.748543in}}%
\pgfpathlineto{\pgfqpoint{2.070485in}{0.750198in}}%
\pgfpathlineto{\pgfqpoint{2.095712in}{0.750310in}}%
\pgfpathlineto{\pgfqpoint{2.120939in}{0.751475in}}%
\pgfpathlineto{\pgfqpoint{2.146166in}{0.752306in}}%
\pgfpathlineto{\pgfqpoint{2.171394in}{0.753747in}}%
\pgfpathlineto{\pgfqpoint{2.196621in}{0.755170in}}%
\pgfpathlineto{\pgfqpoint{2.221848in}{0.755640in}}%
\pgfpathlineto{\pgfqpoint{2.247075in}{0.758585in}}%
\pgfpathlineto{\pgfqpoint{2.272302in}{0.761807in}}%
\pgfpathlineto{\pgfqpoint{2.297529in}{0.764556in}}%
\pgfpathlineto{\pgfqpoint{2.322756in}{0.770631in}}%
\pgfpathlineto{\pgfqpoint{2.347984in}{0.775334in}}%
\pgfpathlineto{\pgfqpoint{2.373211in}{0.777569in}}%
\pgfpathlineto{\pgfqpoint{2.398438in}{0.780933in}}%
\pgfpathlineto{\pgfqpoint{2.423665in}{0.787094in}}%
\pgfpathlineto{\pgfqpoint{2.448892in}{0.791158in}}%
\pgfpathlineto{\pgfqpoint{2.474119in}{0.793672in}}%
\pgfpathlineto{\pgfqpoint{2.499346in}{0.793980in}}%
\pgfpathlineto{\pgfqpoint{2.524573in}{0.794504in}}%
\pgfpathlineto{\pgfqpoint{2.549801in}{0.794908in}}%
\pgfpathlineto{\pgfqpoint{2.575028in}{0.802177in}}%
\pgfpathlineto{\pgfqpoint{2.600255in}{0.806037in}}%
\pgfpathlineto{\pgfqpoint{2.625482in}{0.809710in}}%
\pgfpathlineto{\pgfqpoint{2.650709in}{0.812694in}}%
\pgfpathlineto{\pgfqpoint{2.675936in}{0.812916in}}%
\pgfpathlineto{\pgfqpoint{2.701163in}{0.815033in}}%
\pgfpathlineto{\pgfqpoint{2.726390in}{0.817351in}}%
\pgfpathlineto{\pgfqpoint{2.751618in}{0.818113in}}%
\pgfpathlineto{\pgfqpoint{2.776845in}{0.826743in}}%
\pgfpathlineto{\pgfqpoint{2.802072in}{0.828526in}}%
\pgfpathlineto{\pgfqpoint{2.827299in}{0.828538in}}%
\pgfpathlineto{\pgfqpoint{2.852526in}{0.829459in}}%
\pgfpathlineto{\pgfqpoint{2.877753in}{0.831941in}}%
\pgfpathlineto{\pgfqpoint{2.902980in}{0.845884in}}%
\pgfpathlineto{\pgfqpoint{2.928208in}{0.849747in}}%
\pgfpathlineto{\pgfqpoint{2.953435in}{0.852436in}}%
\pgfpathlineto{\pgfqpoint{2.978662in}{0.862136in}}%
\pgfpathlineto{\pgfqpoint{3.003889in}{0.870585in}}%
\pgfpathlineto{\pgfqpoint{3.029116in}{0.871902in}}%
\pgfpathlineto{\pgfqpoint{3.054343in}{0.877405in}}%
\pgfpathlineto{\pgfqpoint{3.079570in}{0.877711in}}%
\pgfpathlineto{\pgfqpoint{3.104797in}{0.896825in}}%
\pgfpathlineto{\pgfqpoint{3.130025in}{0.901026in}}%
\pgfpathlineto{\pgfqpoint{3.155252in}{0.902329in}}%
\pgfpathlineto{\pgfqpoint{3.180479in}{0.930788in}}%
\pgfpathlineto{\pgfqpoint{3.205706in}{0.937714in}}%
\pgfpathlineto{\pgfqpoint{3.230933in}{0.958120in}}%
\pgfpathlineto{\pgfqpoint{3.256160in}{0.961204in}}%
\pgfpathlineto{\pgfqpoint{3.281387in}{0.964988in}}%
\pgfpathlineto{\pgfqpoint{3.306615in}{0.968228in}}%
\pgfpathlineto{\pgfqpoint{3.331842in}{0.979901in}}%
\pgfpathlineto{\pgfqpoint{3.357069in}{1.004788in}}%
\pgfpathlineto{\pgfqpoint{3.382296in}{1.008068in}}%
\pgfpathlineto{\pgfqpoint{3.407523in}{1.264779in}}%
\pgfpathlineto{\pgfqpoint{3.432750in}{1.282527in}}%
\pgfpathlineto{\pgfqpoint{3.457977in}{2.605275in}}%
\pgfusepath{stroke}%
\end{pgfscope}%
\begin{pgfscope}%
\pgfpathrectangle{\pgfqpoint{0.708220in}{0.535823in}}{\pgfqpoint{5.045427in}{2.069453in}}%
\pgfusepath{clip}%
\pgfsetrectcap%
\pgfsetroundjoin%
\pgfsetlinewidth{1.003750pt}%
\definecolor{currentstroke}{rgb}{1.000000,0.498039,0.054902}%
\pgfsetstrokecolor{currentstroke}%
\pgfsetdash{}{0pt}%
\pgfpathmoveto{\pgfqpoint{0.708220in}{0.816591in}}%
\pgfpathlineto{\pgfqpoint{0.733447in}{0.827443in}}%
\pgfpathlineto{\pgfqpoint{0.758674in}{0.830069in}}%
\pgfpathlineto{\pgfqpoint{0.783901in}{0.830165in}}%
\pgfpathlineto{\pgfqpoint{0.809128in}{0.832490in}}%
\pgfpathlineto{\pgfqpoint{0.834356in}{0.832580in}}%
\pgfpathlineto{\pgfqpoint{0.859583in}{0.834741in}}%
\pgfpathlineto{\pgfqpoint{0.884810in}{0.835661in}}%
\pgfpathlineto{\pgfqpoint{0.910037in}{0.838448in}}%
\pgfpathlineto{\pgfqpoint{0.935264in}{0.839207in}}%
\pgfpathlineto{\pgfqpoint{0.960491in}{0.842585in}}%
\pgfpathlineto{\pgfqpoint{0.985718in}{0.843440in}}%
\pgfpathlineto{\pgfqpoint{1.010945in}{0.843442in}}%
\pgfpathlineto{\pgfqpoint{1.036173in}{0.844201in}}%
\pgfpathlineto{\pgfqpoint{1.061400in}{0.844615in}}%
\pgfpathlineto{\pgfqpoint{1.086627in}{0.847653in}}%
\pgfpathlineto{\pgfqpoint{1.111854in}{0.849185in}}%
\pgfpathlineto{\pgfqpoint{1.137081in}{0.849312in}}%
\pgfpathlineto{\pgfqpoint{1.162308in}{0.849711in}}%
\pgfpathlineto{\pgfqpoint{1.187535in}{0.850350in}}%
\pgfpathlineto{\pgfqpoint{1.212763in}{0.851311in}}%
\pgfpathlineto{\pgfqpoint{1.237990in}{0.853361in}}%
\pgfpathlineto{\pgfqpoint{1.263217in}{0.855106in}}%
\pgfpathlineto{\pgfqpoint{1.288444in}{0.855356in}}%
\pgfpathlineto{\pgfqpoint{1.313671in}{0.856152in}}%
\pgfpathlineto{\pgfqpoint{1.338898in}{0.856704in}}%
\pgfpathlineto{\pgfqpoint{1.364125in}{0.856858in}}%
\pgfpathlineto{\pgfqpoint{1.389352in}{0.856880in}}%
\pgfpathlineto{\pgfqpoint{1.414580in}{0.857143in}}%
\pgfpathlineto{\pgfqpoint{1.439807in}{0.858035in}}%
\pgfpathlineto{\pgfqpoint{1.465034in}{0.859392in}}%
\pgfpathlineto{\pgfqpoint{1.490261in}{0.859491in}}%
\pgfpathlineto{\pgfqpoint{1.515488in}{0.859507in}}%
\pgfpathlineto{\pgfqpoint{1.540715in}{0.860151in}}%
\pgfpathlineto{\pgfqpoint{1.565942in}{0.860457in}}%
\pgfpathlineto{\pgfqpoint{1.591170in}{0.860942in}}%
\pgfpathlineto{\pgfqpoint{1.616397in}{0.861197in}}%
\pgfpathlineto{\pgfqpoint{1.641624in}{0.861604in}}%
\pgfpathlineto{\pgfqpoint{1.666851in}{0.861740in}}%
\pgfpathlineto{\pgfqpoint{1.692078in}{0.861908in}}%
\pgfpathlineto{\pgfqpoint{1.717305in}{0.862940in}}%
\pgfpathlineto{\pgfqpoint{1.742532in}{0.863189in}}%
\pgfpathlineto{\pgfqpoint{1.767759in}{0.863508in}}%
\pgfpathlineto{\pgfqpoint{1.792987in}{0.863624in}}%
\pgfpathlineto{\pgfqpoint{1.818214in}{0.864178in}}%
\pgfpathlineto{\pgfqpoint{1.843441in}{0.864416in}}%
\pgfpathlineto{\pgfqpoint{1.868668in}{0.864983in}}%
\pgfpathlineto{\pgfqpoint{1.893895in}{0.866437in}}%
\pgfpathlineto{\pgfqpoint{1.919122in}{0.866887in}}%
\pgfpathlineto{\pgfqpoint{1.944349in}{0.866909in}}%
\pgfpathlineto{\pgfqpoint{1.969577in}{0.866951in}}%
\pgfpathlineto{\pgfqpoint{1.994804in}{0.867574in}}%
\pgfpathlineto{\pgfqpoint{2.020031in}{0.867912in}}%
\pgfpathlineto{\pgfqpoint{2.045258in}{0.868899in}}%
\pgfpathlineto{\pgfqpoint{2.070485in}{0.869000in}}%
\pgfpathlineto{\pgfqpoint{2.095712in}{0.869656in}}%
\pgfpathlineto{\pgfqpoint{2.120939in}{0.871495in}}%
\pgfpathlineto{\pgfqpoint{2.146166in}{0.873151in}}%
\pgfpathlineto{\pgfqpoint{2.171394in}{0.873679in}}%
\pgfpathlineto{\pgfqpoint{2.196621in}{0.874467in}}%
\pgfpathlineto{\pgfqpoint{2.221848in}{0.875147in}}%
\pgfpathlineto{\pgfqpoint{2.247075in}{0.875964in}}%
\pgfpathlineto{\pgfqpoint{2.272302in}{0.876496in}}%
\pgfpathlineto{\pgfqpoint{2.297529in}{0.877440in}}%
\pgfpathlineto{\pgfqpoint{2.322756in}{0.878314in}}%
\pgfpathlineto{\pgfqpoint{2.347984in}{0.878494in}}%
\pgfpathlineto{\pgfqpoint{2.373211in}{0.879819in}}%
\pgfpathlineto{\pgfqpoint{2.398438in}{0.879985in}}%
\pgfpathlineto{\pgfqpoint{2.423665in}{0.880190in}}%
\pgfpathlineto{\pgfqpoint{2.448892in}{0.880689in}}%
\pgfpathlineto{\pgfqpoint{2.474119in}{0.882661in}}%
\pgfpathlineto{\pgfqpoint{2.499346in}{0.883302in}}%
\pgfpathlineto{\pgfqpoint{2.524573in}{0.883496in}}%
\pgfpathlineto{\pgfqpoint{2.549801in}{0.884387in}}%
\pgfpathlineto{\pgfqpoint{2.575028in}{0.885695in}}%
\pgfpathlineto{\pgfqpoint{2.600255in}{0.887694in}}%
\pgfpathlineto{\pgfqpoint{2.625482in}{0.888059in}}%
\pgfpathlineto{\pgfqpoint{2.650709in}{0.888598in}}%
\pgfpathlineto{\pgfqpoint{2.675936in}{0.888708in}}%
\pgfpathlineto{\pgfqpoint{2.701163in}{0.890338in}}%
\pgfpathlineto{\pgfqpoint{2.726390in}{0.891945in}}%
\pgfpathlineto{\pgfqpoint{2.751618in}{0.892090in}}%
\pgfpathlineto{\pgfqpoint{2.776845in}{0.894773in}}%
\pgfpathlineto{\pgfqpoint{2.802072in}{0.896126in}}%
\pgfpathlineto{\pgfqpoint{2.827299in}{0.896649in}}%
\pgfpathlineto{\pgfqpoint{2.852526in}{0.898236in}}%
\pgfpathlineto{\pgfqpoint{2.877753in}{0.899261in}}%
\pgfpathlineto{\pgfqpoint{2.902980in}{0.901245in}}%
\pgfpathlineto{\pgfqpoint{2.928208in}{0.901534in}}%
\pgfpathlineto{\pgfqpoint{2.953435in}{0.901610in}}%
\pgfpathlineto{\pgfqpoint{2.978662in}{0.901980in}}%
\pgfpathlineto{\pgfqpoint{3.003889in}{0.907032in}}%
\pgfpathlineto{\pgfqpoint{3.029116in}{0.909789in}}%
\pgfpathlineto{\pgfqpoint{3.054343in}{0.912230in}}%
\pgfpathlineto{\pgfqpoint{3.079570in}{0.913005in}}%
\pgfpathlineto{\pgfqpoint{3.104797in}{0.913908in}}%
\pgfpathlineto{\pgfqpoint{3.130025in}{0.916740in}}%
\pgfpathlineto{\pgfqpoint{3.155252in}{0.920438in}}%
\pgfpathlineto{\pgfqpoint{3.180479in}{0.922582in}}%
\pgfpathlineto{\pgfqpoint{3.205706in}{0.927532in}}%
\pgfpathlineto{\pgfqpoint{3.230933in}{0.931644in}}%
\pgfpathlineto{\pgfqpoint{3.256160in}{0.932287in}}%
\pgfpathlineto{\pgfqpoint{3.281387in}{0.932691in}}%
\pgfpathlineto{\pgfqpoint{3.306615in}{0.936050in}}%
\pgfpathlineto{\pgfqpoint{3.331842in}{0.940833in}}%
\pgfpathlineto{\pgfqpoint{3.357069in}{0.945486in}}%
\pgfpathlineto{\pgfqpoint{3.382296in}{0.946178in}}%
\pgfpathlineto{\pgfqpoint{3.407523in}{0.967349in}}%
\pgfpathlineto{\pgfqpoint{3.432750in}{0.978249in}}%
\pgfpathlineto{\pgfqpoint{3.457977in}{1.039994in}}%
\pgfpathlineto{\pgfqpoint{3.483204in}{1.043415in}}%
\pgfpathlineto{\pgfqpoint{3.508432in}{1.165832in}}%
\pgfpathlineto{\pgfqpoint{3.533659in}{1.211400in}}%
\pgfpathlineto{\pgfqpoint{3.558886in}{1.535793in}}%
\pgfpathlineto{\pgfqpoint{3.584113in}{1.661414in}}%
\pgfpathlineto{\pgfqpoint{3.609340in}{1.897724in}}%
\pgfpathlineto{\pgfqpoint{3.634567in}{2.605275in}}%
\pgfusepath{stroke}%
\end{pgfscope}%
\begin{pgfscope}%
\pgfpathrectangle{\pgfqpoint{0.708220in}{0.535823in}}{\pgfqpoint{5.045427in}{2.069453in}}%
\pgfusepath{clip}%
\pgfsetrectcap%
\pgfsetroundjoin%
\pgfsetlinewidth{1.003750pt}%
\definecolor{currentstroke}{rgb}{0.172549,0.627451,0.172549}%
\pgfsetstrokecolor{currentstroke}%
\pgfsetdash{}{0pt}%
\pgfpathmoveto{\pgfqpoint{0.708220in}{1.428597in}}%
\pgfpathlineto{\pgfqpoint{0.733447in}{1.429444in}}%
\pgfpathlineto{\pgfqpoint{0.758674in}{1.430597in}}%
\pgfpathlineto{\pgfqpoint{0.783901in}{1.431586in}}%
\pgfpathlineto{\pgfqpoint{0.809128in}{1.431794in}}%
\pgfpathlineto{\pgfqpoint{0.834356in}{1.433949in}}%
\pgfpathlineto{\pgfqpoint{0.859583in}{1.434549in}}%
\pgfpathlineto{\pgfqpoint{0.884810in}{1.435122in}}%
\pgfpathlineto{\pgfqpoint{0.910037in}{1.437826in}}%
\pgfpathlineto{\pgfqpoint{0.935264in}{1.438612in}}%
\pgfpathlineto{\pgfqpoint{0.960491in}{1.439903in}}%
\pgfpathlineto{\pgfqpoint{0.985718in}{1.440736in}}%
\pgfpathlineto{\pgfqpoint{1.010945in}{1.441036in}}%
\pgfpathlineto{\pgfqpoint{1.036173in}{1.441715in}}%
\pgfpathlineto{\pgfqpoint{1.061400in}{1.441947in}}%
\pgfpathlineto{\pgfqpoint{1.086627in}{1.442599in}}%
\pgfpathlineto{\pgfqpoint{1.111854in}{1.446115in}}%
\pgfpathlineto{\pgfqpoint{1.137081in}{1.452040in}}%
\pgfpathlineto{\pgfqpoint{1.162308in}{1.453679in}}%
\pgfpathlineto{\pgfqpoint{1.187535in}{1.457609in}}%
\pgfpathlineto{\pgfqpoint{1.212763in}{1.460099in}}%
\pgfpathlineto{\pgfqpoint{1.237990in}{1.460820in}}%
\pgfpathlineto{\pgfqpoint{1.263217in}{1.462005in}}%
\pgfpathlineto{\pgfqpoint{1.288444in}{1.463661in}}%
\pgfpathlineto{\pgfqpoint{1.313671in}{1.463802in}}%
\pgfpathlineto{\pgfqpoint{1.338898in}{1.463983in}}%
\pgfpathlineto{\pgfqpoint{1.364125in}{1.465757in}}%
\pgfpathlineto{\pgfqpoint{1.389352in}{1.465821in}}%
\pgfpathlineto{\pgfqpoint{1.414580in}{1.468568in}}%
\pgfpathlineto{\pgfqpoint{1.439807in}{1.469585in}}%
\pgfpathlineto{\pgfqpoint{1.465034in}{1.470932in}}%
\pgfpathlineto{\pgfqpoint{1.490261in}{1.472122in}}%
\pgfpathlineto{\pgfqpoint{1.515488in}{1.472347in}}%
\pgfpathlineto{\pgfqpoint{1.540715in}{1.473756in}}%
\pgfpathlineto{\pgfqpoint{1.565942in}{1.476565in}}%
\pgfpathlineto{\pgfqpoint{1.591170in}{1.478698in}}%
\pgfpathlineto{\pgfqpoint{1.616397in}{1.486781in}}%
\pgfpathlineto{\pgfqpoint{1.641624in}{1.488971in}}%
\pgfpathlineto{\pgfqpoint{1.666851in}{1.490038in}}%
\pgfpathlineto{\pgfqpoint{1.692078in}{1.490458in}}%
\pgfpathlineto{\pgfqpoint{1.717305in}{1.491044in}}%
\pgfpathlineto{\pgfqpoint{1.742532in}{1.492488in}}%
\pgfpathlineto{\pgfqpoint{1.767759in}{1.494991in}}%
\pgfpathlineto{\pgfqpoint{1.792987in}{1.495479in}}%
\pgfpathlineto{\pgfqpoint{1.818214in}{1.510258in}}%
\pgfpathlineto{\pgfqpoint{1.843441in}{1.510768in}}%
\pgfpathlineto{\pgfqpoint{1.868668in}{1.512325in}}%
\pgfpathlineto{\pgfqpoint{1.893895in}{1.513913in}}%
\pgfpathlineto{\pgfqpoint{1.919122in}{1.515650in}}%
\pgfpathlineto{\pgfqpoint{1.944349in}{1.517129in}}%
\pgfpathlineto{\pgfqpoint{1.969577in}{1.517422in}}%
\pgfpathlineto{\pgfqpoint{1.994804in}{1.518489in}}%
\pgfpathlineto{\pgfqpoint{2.020031in}{1.523804in}}%
\pgfpathlineto{\pgfqpoint{2.045258in}{1.524863in}}%
\pgfpathlineto{\pgfqpoint{2.070485in}{1.527141in}}%
\pgfpathlineto{\pgfqpoint{2.095712in}{1.529360in}}%
\pgfpathlineto{\pgfqpoint{2.120939in}{1.532793in}}%
\pgfpathlineto{\pgfqpoint{2.146166in}{1.533909in}}%
\pgfpathlineto{\pgfqpoint{2.171394in}{1.534058in}}%
\pgfpathlineto{\pgfqpoint{2.196621in}{1.539152in}}%
\pgfpathlineto{\pgfqpoint{2.221848in}{1.539403in}}%
\pgfpathlineto{\pgfqpoint{2.247075in}{1.541185in}}%
\pgfpathlineto{\pgfqpoint{2.272302in}{1.541766in}}%
\pgfpathlineto{\pgfqpoint{2.297529in}{1.544343in}}%
\pgfpathlineto{\pgfqpoint{2.322756in}{1.544837in}}%
\pgfpathlineto{\pgfqpoint{2.347984in}{1.548686in}}%
\pgfpathlineto{\pgfqpoint{2.373211in}{1.548797in}}%
\pgfpathlineto{\pgfqpoint{2.398438in}{1.550312in}}%
\pgfpathlineto{\pgfqpoint{2.423665in}{1.553524in}}%
\pgfpathlineto{\pgfqpoint{2.448892in}{1.556640in}}%
\pgfpathlineto{\pgfqpoint{2.474119in}{1.557765in}}%
\pgfpathlineto{\pgfqpoint{2.499346in}{1.558304in}}%
\pgfpathlineto{\pgfqpoint{2.524573in}{1.558462in}}%
\pgfpathlineto{\pgfqpoint{2.549801in}{1.561829in}}%
\pgfpathlineto{\pgfqpoint{2.575028in}{1.564725in}}%
\pgfpathlineto{\pgfqpoint{2.600255in}{1.566601in}}%
\pgfpathlineto{\pgfqpoint{2.625482in}{1.570424in}}%
\pgfpathlineto{\pgfqpoint{2.650709in}{1.570498in}}%
\pgfpathlineto{\pgfqpoint{2.675936in}{1.571638in}}%
\pgfpathlineto{\pgfqpoint{2.701163in}{1.572040in}}%
\pgfpathlineto{\pgfqpoint{2.726390in}{1.578488in}}%
\pgfpathlineto{\pgfqpoint{2.751618in}{1.582131in}}%
\pgfpathlineto{\pgfqpoint{2.776845in}{1.585468in}}%
\pgfpathlineto{\pgfqpoint{2.802072in}{1.587781in}}%
\pgfpathlineto{\pgfqpoint{2.827299in}{1.587918in}}%
\pgfpathlineto{\pgfqpoint{2.852526in}{1.589028in}}%
\pgfpathlineto{\pgfqpoint{2.877753in}{1.590196in}}%
\pgfpathlineto{\pgfqpoint{2.902980in}{1.595058in}}%
\pgfpathlineto{\pgfqpoint{2.928208in}{1.615180in}}%
\pgfpathlineto{\pgfqpoint{2.953435in}{1.619351in}}%
\pgfpathlineto{\pgfqpoint{2.978662in}{1.620152in}}%
\pgfpathlineto{\pgfqpoint{3.003889in}{1.625763in}}%
\pgfpathlineto{\pgfqpoint{3.029116in}{1.628275in}}%
\pgfpathlineto{\pgfqpoint{3.054343in}{1.633619in}}%
\pgfpathlineto{\pgfqpoint{3.079570in}{1.633629in}}%
\pgfpathlineto{\pgfqpoint{3.104797in}{1.635095in}}%
\pgfpathlineto{\pgfqpoint{3.130025in}{1.637866in}}%
\pgfpathlineto{\pgfqpoint{3.155252in}{1.649623in}}%
\pgfpathlineto{\pgfqpoint{3.180479in}{1.652085in}}%
\pgfpathlineto{\pgfqpoint{3.205706in}{1.654027in}}%
\pgfpathlineto{\pgfqpoint{3.230933in}{1.658769in}}%
\pgfpathlineto{\pgfqpoint{3.256160in}{1.670117in}}%
\pgfpathlineto{\pgfqpoint{3.281387in}{1.714170in}}%
\pgfpathlineto{\pgfqpoint{3.306615in}{1.730816in}}%
\pgfpathlineto{\pgfqpoint{3.331842in}{1.732546in}}%
\pgfpathlineto{\pgfqpoint{3.357069in}{1.751590in}}%
\pgfpathlineto{\pgfqpoint{3.382296in}{1.760919in}}%
\pgfpathlineto{\pgfqpoint{3.407523in}{1.762586in}}%
\pgfpathlineto{\pgfqpoint{3.432750in}{1.763391in}}%
\pgfpathlineto{\pgfqpoint{3.457977in}{1.769225in}}%
\pgfpathlineto{\pgfqpoint{3.483204in}{1.771998in}}%
\pgfpathlineto{\pgfqpoint{3.508432in}{1.788797in}}%
\pgfpathlineto{\pgfqpoint{3.533659in}{1.883063in}}%
\pgfpathlineto{\pgfqpoint{3.558886in}{1.889347in}}%
\pgfpathlineto{\pgfqpoint{3.584113in}{1.921358in}}%
\pgfpathlineto{\pgfqpoint{3.609340in}{1.937144in}}%
\pgfpathlineto{\pgfqpoint{3.634567in}{2.079462in}}%
\pgfpathlineto{\pgfqpoint{3.659794in}{2.088092in}}%
\pgfpathlineto{\pgfqpoint{3.685022in}{2.253367in}}%
\pgfpathlineto{\pgfqpoint{3.710249in}{2.605275in}}%
\pgfusepath{stroke}%
\end{pgfscope}%
\begin{pgfscope}%
\pgfpathrectangle{\pgfqpoint{0.708220in}{0.535823in}}{\pgfqpoint{5.045427in}{2.069453in}}%
\pgfusepath{clip}%
\pgfsetbuttcap%
\pgfsetroundjoin%
\pgfsetlinewidth{1.003750pt}%
\definecolor{currentstroke}{rgb}{0.839216,0.152941,0.156863}%
\pgfsetstrokecolor{currentstroke}%
\pgfsetdash{{3.700000pt}{1.600000pt}}{0.000000pt}%
\pgfpathmoveto{\pgfqpoint{0.708220in}{0.535823in}}%
\pgfpathlineto{\pgfqpoint{0.733447in}{0.535823in}}%
\pgfpathlineto{\pgfqpoint{0.758674in}{0.535823in}}%
\pgfpathlineto{\pgfqpoint{0.783901in}{0.535823in}}%
\pgfpathlineto{\pgfqpoint{0.809128in}{0.535823in}}%
\pgfpathlineto{\pgfqpoint{0.834356in}{0.573921in}}%
\pgfpathlineto{\pgfqpoint{0.859583in}{0.573921in}}%
\pgfpathlineto{\pgfqpoint{0.884810in}{0.606133in}}%
\pgfpathlineto{\pgfqpoint{0.910037in}{0.606133in}}%
\pgfpathlineto{\pgfqpoint{0.935264in}{0.658648in}}%
\pgfpathlineto{\pgfqpoint{0.960491in}{0.658648in}}%
\pgfpathlineto{\pgfqpoint{0.985718in}{0.658648in}}%
\pgfpathlineto{\pgfqpoint{1.010945in}{0.680664in}}%
\pgfpathlineto{\pgfqpoint{1.036173in}{0.700580in}}%
\pgfpathlineto{\pgfqpoint{1.061400in}{0.718762in}}%
\pgfpathlineto{\pgfqpoint{1.086627in}{0.750974in}}%
\pgfpathlineto{\pgfqpoint{1.111854in}{0.765391in}}%
\pgfpathlineto{\pgfqpoint{1.137081in}{0.765391in}}%
\pgfpathlineto{\pgfqpoint{1.162308in}{0.803489in}}%
\pgfpathlineto{\pgfqpoint{1.187535in}{0.814787in}}%
\pgfpathlineto{\pgfqpoint{1.212763in}{0.845422in}}%
\pgfpathlineto{\pgfqpoint{1.237990in}{0.845422in}}%
\pgfpathlineto{\pgfqpoint{1.263217in}{0.845422in}}%
\pgfpathlineto{\pgfqpoint{1.288444in}{0.854710in}}%
\pgfpathlineto{\pgfqpoint{1.313671in}{0.854710in}}%
\pgfpathlineto{\pgfqpoint{1.338898in}{0.863604in}}%
\pgfpathlineto{\pgfqpoint{1.364125in}{0.880330in}}%
\pgfpathlineto{\pgfqpoint{1.389352in}{0.888216in}}%
\pgfpathlineto{\pgfqpoint{1.414580in}{0.888216in}}%
\pgfpathlineto{\pgfqpoint{1.439807in}{0.895815in}}%
\pgfpathlineto{\pgfqpoint{1.465034in}{0.895815in}}%
\pgfpathlineto{\pgfqpoint{1.490261in}{0.970347in}}%
\pgfpathlineto{\pgfqpoint{1.490261in}{2.605275in}}%
\pgfusepath{stroke}%
\end{pgfscope}%
\begin{pgfscope}%
\pgfpathrectangle{\pgfqpoint{0.708220in}{0.535823in}}{\pgfqpoint{5.045427in}{2.069453in}}%
\pgfusepath{clip}%
\pgfsetbuttcap%
\pgfsetroundjoin%
\pgfsetlinewidth{1.003750pt}%
\definecolor{currentstroke}{rgb}{0.580392,0.403922,0.741176}%
\pgfsetstrokecolor{currentstroke}%
\pgfsetdash{{3.700000pt}{1.600000pt}}{0.000000pt}%
\pgfpathmoveto{\pgfqpoint{0.708220in}{0.535823in}}%
\pgfpathlineto{\pgfqpoint{0.733447in}{0.535823in}}%
\pgfpathlineto{\pgfqpoint{0.758674in}{0.535823in}}%
\pgfpathlineto{\pgfqpoint{0.783901in}{0.535823in}}%
\pgfpathlineto{\pgfqpoint{0.809128in}{0.573921in}}%
\pgfpathlineto{\pgfqpoint{0.834356in}{0.573921in}}%
\pgfpathlineto{\pgfqpoint{0.859583in}{0.634036in}}%
\pgfpathlineto{\pgfqpoint{0.884810in}{0.658648in}}%
\pgfpathlineto{\pgfqpoint{0.910037in}{0.700580in}}%
\pgfpathlineto{\pgfqpoint{0.935264in}{0.718762in}}%
\pgfpathlineto{\pgfqpoint{0.960491in}{0.735488in}}%
\pgfpathlineto{\pgfqpoint{0.985718in}{0.735488in}}%
\pgfpathlineto{\pgfqpoint{1.010945in}{0.750974in}}%
\pgfpathlineto{\pgfqpoint{1.036173in}{0.750974in}}%
\pgfpathlineto{\pgfqpoint{1.061400in}{0.750974in}}%
\pgfpathlineto{\pgfqpoint{1.086627in}{0.765391in}}%
\pgfpathlineto{\pgfqpoint{1.111854in}{0.765391in}}%
\pgfpathlineto{\pgfqpoint{1.137081in}{0.778877in}}%
\pgfpathlineto{\pgfqpoint{1.162308in}{0.778877in}}%
\pgfpathlineto{\pgfqpoint{1.187535in}{0.778877in}}%
\pgfpathlineto{\pgfqpoint{1.212763in}{0.791545in}}%
\pgfpathlineto{\pgfqpoint{1.237990in}{0.803489in}}%
\pgfpathlineto{\pgfqpoint{1.263217in}{0.803489in}}%
\pgfpathlineto{\pgfqpoint{1.288444in}{0.803489in}}%
\pgfpathlineto{\pgfqpoint{1.313671in}{0.803489in}}%
\pgfpathlineto{\pgfqpoint{1.338898in}{0.814787in}}%
\pgfpathlineto{\pgfqpoint{1.364125in}{0.814787in}}%
\pgfpathlineto{\pgfqpoint{1.389352in}{0.835701in}}%
\pgfpathlineto{\pgfqpoint{1.414580in}{0.835701in}}%
\pgfpathlineto{\pgfqpoint{1.439807in}{0.854710in}}%
\pgfpathlineto{\pgfqpoint{1.465034in}{0.854710in}}%
\pgfpathlineto{\pgfqpoint{1.490261in}{0.888216in}}%
\pgfpathlineto{\pgfqpoint{1.515488in}{0.888216in}}%
\pgfpathlineto{\pgfqpoint{1.540715in}{0.910232in}}%
\pgfpathlineto{\pgfqpoint{1.565942in}{0.923718in}}%
\pgfpathlineto{\pgfqpoint{1.591170in}{0.948331in}}%
\pgfpathlineto{\pgfqpoint{1.616397in}{0.959629in}}%
\pgfpathlineto{\pgfqpoint{1.641624in}{0.965056in}}%
\pgfpathlineto{\pgfqpoint{1.666851in}{0.980542in}}%
\pgfpathlineto{\pgfqpoint{1.692078in}{0.985459in}}%
\pgfpathlineto{\pgfqpoint{1.717305in}{0.999552in}}%
\pgfpathlineto{\pgfqpoint{1.742532in}{1.008445in}}%
\pgfpathlineto{\pgfqpoint{1.767759in}{1.012754in}}%
\pgfpathlineto{\pgfqpoint{1.792987in}{1.051562in}}%
\pgfpathlineto{\pgfqpoint{1.818214in}{1.055074in}}%
\pgfpathlineto{\pgfqpoint{1.843441in}{1.068560in}}%
\pgfpathlineto{\pgfqpoint{1.868668in}{1.081228in}}%
\pgfpathlineto{\pgfqpoint{1.893895in}{1.084279in}}%
\pgfpathlineto{\pgfqpoint{1.919122in}{1.115188in}}%
\pgfpathlineto{\pgfqpoint{1.944349in}{1.130301in}}%
\pgfpathlineto{\pgfqpoint{1.969577in}{1.155452in}}%
\pgfpathlineto{\pgfqpoint{1.969577in}{2.605275in}}%
\pgfusepath{stroke}%
\end{pgfscope}%
\begin{pgfscope}%
\pgfpathrectangle{\pgfqpoint{0.708220in}{0.535823in}}{\pgfqpoint{5.045427in}{2.069453in}}%
\pgfusepath{clip}%
\pgfsetbuttcap%
\pgfsetroundjoin%
\pgfsetlinewidth{1.003750pt}%
\definecolor{currentstroke}{rgb}{0.549020,0.337255,0.294118}%
\pgfsetstrokecolor{currentstroke}%
\pgfsetdash{{1.000000pt}{1.650000pt}}{0.000000pt}%
\pgfpathmoveto{\pgfqpoint{0.753264in}{0.525823in}}%
\pgfpathlineto{\pgfqpoint{0.758674in}{0.535823in}}%
\pgfpathlineto{\pgfqpoint{0.783901in}{0.535823in}}%
\pgfpathlineto{\pgfqpoint{0.809128in}{0.535823in}}%
\pgfpathlineto{\pgfqpoint{0.834356in}{0.535823in}}%
\pgfpathlineto{\pgfqpoint{0.859583in}{0.573921in}}%
\pgfpathlineto{\pgfqpoint{0.884810in}{0.573921in}}%
\pgfpathlineto{\pgfqpoint{0.910037in}{0.573921in}}%
\pgfpathlineto{\pgfqpoint{0.935264in}{0.573921in}}%
\pgfpathlineto{\pgfqpoint{0.960491in}{0.573921in}}%
\pgfpathlineto{\pgfqpoint{0.985718in}{0.573921in}}%
\pgfpathlineto{\pgfqpoint{1.010945in}{0.634036in}}%
\pgfpathlineto{\pgfqpoint{1.036173in}{0.680664in}}%
\pgfpathlineto{\pgfqpoint{1.061400in}{0.680664in}}%
\pgfpathlineto{\pgfqpoint{1.086627in}{0.735488in}}%
\pgfpathlineto{\pgfqpoint{1.111854in}{0.750974in}}%
\pgfpathlineto{\pgfqpoint{1.137081in}{0.750974in}}%
\pgfpathlineto{\pgfqpoint{1.162308in}{0.750974in}}%
\pgfpathlineto{\pgfqpoint{1.187535in}{0.765391in}}%
\pgfpathlineto{\pgfqpoint{1.212763in}{0.778877in}}%
\pgfpathlineto{\pgfqpoint{1.237990in}{0.791545in}}%
\pgfpathlineto{\pgfqpoint{1.263217in}{0.803489in}}%
\pgfpathlineto{\pgfqpoint{1.288444in}{0.803489in}}%
\pgfpathlineto{\pgfqpoint{1.313671in}{0.803489in}}%
\pgfpathlineto{\pgfqpoint{1.338898in}{0.814787in}}%
\pgfpathlineto{\pgfqpoint{1.364125in}{0.835701in}}%
\pgfpathlineto{\pgfqpoint{1.389352in}{0.845422in}}%
\pgfpathlineto{\pgfqpoint{1.414580in}{0.845422in}}%
\pgfpathlineto{\pgfqpoint{1.439807in}{0.863604in}}%
\pgfpathlineto{\pgfqpoint{1.465034in}{0.863604in}}%
\pgfpathlineto{\pgfqpoint{1.490261in}{0.880330in}}%
\pgfpathlineto{\pgfqpoint{1.515488in}{0.888216in}}%
\pgfpathlineto{\pgfqpoint{1.540715in}{0.895815in}}%
\pgfpathlineto{\pgfqpoint{1.565942in}{0.917084in}}%
\pgfpathlineto{\pgfqpoint{1.565942in}{2.605275in}}%
\pgfusepath{stroke}%
\end{pgfscope}%
\begin{pgfscope}%
\pgfpathrectangle{\pgfqpoint{0.708220in}{0.535823in}}{\pgfqpoint{5.045427in}{2.069453in}}%
\pgfusepath{clip}%
\pgfsetbuttcap%
\pgfsetroundjoin%
\pgfsetlinewidth{1.003750pt}%
\definecolor{currentstroke}{rgb}{0.890196,0.466667,0.760784}%
\pgfsetstrokecolor{currentstroke}%
\pgfsetdash{{1.000000pt}{1.650000pt}}{0.000000pt}%
\pgfpathmoveto{\pgfqpoint{0.728037in}{0.525823in}}%
\pgfpathlineto{\pgfqpoint{0.733447in}{0.535823in}}%
\pgfpathlineto{\pgfqpoint{0.758674in}{0.535823in}}%
\pgfpathlineto{\pgfqpoint{0.783901in}{0.535823in}}%
\pgfpathlineto{\pgfqpoint{0.809128in}{0.535823in}}%
\pgfpathlineto{\pgfqpoint{0.834356in}{0.606133in}}%
\pgfpathlineto{\pgfqpoint{0.859583in}{0.658648in}}%
\pgfpathlineto{\pgfqpoint{0.884810in}{0.658648in}}%
\pgfpathlineto{\pgfqpoint{0.910037in}{0.680664in}}%
\pgfpathlineto{\pgfqpoint{0.935264in}{0.700580in}}%
\pgfpathlineto{\pgfqpoint{0.960491in}{0.700580in}}%
\pgfpathlineto{\pgfqpoint{0.985718in}{0.718762in}}%
\pgfpathlineto{\pgfqpoint{1.010945in}{0.718762in}}%
\pgfpathlineto{\pgfqpoint{1.036173in}{0.718762in}}%
\pgfpathlineto{\pgfqpoint{1.061400in}{0.750974in}}%
\pgfpathlineto{\pgfqpoint{1.086627in}{0.765391in}}%
\pgfpathlineto{\pgfqpoint{1.111854in}{0.765391in}}%
\pgfpathlineto{\pgfqpoint{1.137081in}{0.778877in}}%
\pgfpathlineto{\pgfqpoint{1.162308in}{0.791545in}}%
\pgfpathlineto{\pgfqpoint{1.187535in}{0.791545in}}%
\pgfpathlineto{\pgfqpoint{1.212763in}{0.791545in}}%
\pgfpathlineto{\pgfqpoint{1.237990in}{0.791545in}}%
\pgfpathlineto{\pgfqpoint{1.263217in}{0.803489in}}%
\pgfpathlineto{\pgfqpoint{1.288444in}{0.825505in}}%
\pgfpathlineto{\pgfqpoint{1.313671in}{0.825505in}}%
\pgfpathlineto{\pgfqpoint{1.338898in}{0.854710in}}%
\pgfpathlineto{\pgfqpoint{1.364125in}{0.872134in}}%
\pgfpathlineto{\pgfqpoint{1.389352in}{0.888216in}}%
\pgfpathlineto{\pgfqpoint{1.414580in}{0.888216in}}%
\pgfpathlineto{\pgfqpoint{1.439807in}{0.903148in}}%
\pgfpathlineto{\pgfqpoint{1.465034in}{0.910232in}}%
\pgfpathlineto{\pgfqpoint{1.490261in}{0.917084in}}%
\pgfpathlineto{\pgfqpoint{1.515488in}{0.936387in}}%
\pgfpathlineto{\pgfqpoint{1.540715in}{0.954056in}}%
\pgfpathlineto{\pgfqpoint{1.565942in}{0.980542in}}%
\pgfpathlineto{\pgfqpoint{1.591170in}{1.093172in}}%
\pgfpathlineto{\pgfqpoint{1.616397in}{1.146652in}}%
\pgfpathlineto{\pgfqpoint{1.641624in}{1.155452in}}%
\pgfpathlineto{\pgfqpoint{1.666851in}{1.954789in}}%
\pgfpathlineto{\pgfqpoint{1.666851in}{2.605275in}}%
\pgfusepath{stroke}%
\end{pgfscope}%
\begin{pgfscope}%
\pgfpathrectangle{\pgfqpoint{0.708220in}{0.535823in}}{\pgfqpoint{5.045427in}{2.069453in}}%
\pgfusepath{clip}%
\pgfsetbuttcap%
\pgfsetroundjoin%
\pgfsetlinewidth{1.003750pt}%
\definecolor{currentstroke}{rgb}{0.498039,0.498039,0.498039}%
\pgfsetstrokecolor{currentstroke}%
\pgfsetdash{{6.400000pt}{1.600000pt}{1.000000pt}{1.600000pt}}{0.000000pt}%
\pgfpathmoveto{\pgfqpoint{0.708220in}{0.535823in}}%
\pgfpathlineto{\pgfqpoint{0.733447in}{0.573921in}}%
\pgfpathlineto{\pgfqpoint{0.758674in}{0.573921in}}%
\pgfpathlineto{\pgfqpoint{0.783901in}{0.573921in}}%
\pgfpathlineto{\pgfqpoint{0.809128in}{0.573921in}}%
\pgfpathlineto{\pgfqpoint{0.834356in}{0.573921in}}%
\pgfpathlineto{\pgfqpoint{0.859583in}{0.573921in}}%
\pgfpathlineto{\pgfqpoint{0.884810in}{0.658648in}}%
\pgfpathlineto{\pgfqpoint{0.910037in}{0.700580in}}%
\pgfpathlineto{\pgfqpoint{0.935264in}{0.718762in}}%
\pgfpathlineto{\pgfqpoint{0.960491in}{0.778877in}}%
\pgfpathlineto{\pgfqpoint{0.985718in}{0.845422in}}%
\pgfpathlineto{\pgfqpoint{1.010945in}{0.917084in}}%
\pgfpathlineto{\pgfqpoint{1.036173in}{0.923718in}}%
\pgfpathlineto{\pgfqpoint{1.061400in}{0.959629in}}%
\pgfpathlineto{\pgfqpoint{1.086627in}{0.970347in}}%
\pgfpathlineto{\pgfqpoint{1.111854in}{0.985459in}}%
\pgfpathlineto{\pgfqpoint{1.137081in}{0.990263in}}%
\pgfpathlineto{\pgfqpoint{1.162308in}{1.008445in}}%
\pgfpathlineto{\pgfqpoint{1.187535in}{1.016975in}}%
\pgfpathlineto{\pgfqpoint{1.212763in}{1.068560in}}%
\pgfpathlineto{\pgfqpoint{1.237990in}{1.071800in}}%
\pgfpathlineto{\pgfqpoint{1.263217in}{1.104470in}}%
\pgfpathlineto{\pgfqpoint{1.288444in}{1.107202in}}%
\pgfpathlineto{\pgfqpoint{1.313671in}{1.185498in}}%
\pgfpathlineto{\pgfqpoint{1.338898in}{1.239460in}}%
\pgfpathlineto{\pgfqpoint{1.364125in}{1.242322in}}%
\pgfpathlineto{\pgfqpoint{1.389352in}{1.256075in}}%
\pgfpathlineto{\pgfqpoint{1.414580in}{1.265190in}}%
\pgfpathlineto{\pgfqpoint{1.439807in}{1.268977in}}%
\pgfpathlineto{\pgfqpoint{1.465034in}{1.373203in}}%
\pgfpathlineto{\pgfqpoint{1.490261in}{1.505519in}}%
\pgfpathlineto{\pgfqpoint{1.490261in}{2.605275in}}%
\pgfusepath{stroke}%
\end{pgfscope}%
\begin{pgfscope}%
\pgfpathrectangle{\pgfqpoint{0.708220in}{0.535823in}}{\pgfqpoint{5.045427in}{2.069453in}}%
\pgfusepath{clip}%
\pgfsetbuttcap%
\pgfsetroundjoin%
\pgfsetlinewidth{1.003750pt}%
\definecolor{currentstroke}{rgb}{0.737255,0.741176,0.133333}%
\pgfsetstrokecolor{currentstroke}%
\pgfsetdash{{6.400000pt}{1.600000pt}{1.000000pt}{1.600000pt}}{0.000000pt}%
\pgfpathmoveto{\pgfqpoint{0.708220in}{0.535823in}}%
\pgfpathlineto{\pgfqpoint{0.733447in}{0.535823in}}%
\pgfpathlineto{\pgfqpoint{0.758674in}{0.535823in}}%
\pgfpathlineto{\pgfqpoint{0.783901in}{0.573921in}}%
\pgfpathlineto{\pgfqpoint{0.809128in}{0.573921in}}%
\pgfpathlineto{\pgfqpoint{0.834356in}{0.573921in}}%
\pgfpathlineto{\pgfqpoint{0.859583in}{0.606133in}}%
\pgfpathlineto{\pgfqpoint{0.884810in}{0.658648in}}%
\pgfpathlineto{\pgfqpoint{0.910037in}{0.735488in}}%
\pgfpathlineto{\pgfqpoint{0.935264in}{0.735488in}}%
\pgfpathlineto{\pgfqpoint{0.960491in}{0.778877in}}%
\pgfpathlineto{\pgfqpoint{0.985718in}{0.791545in}}%
\pgfpathlineto{\pgfqpoint{1.010945in}{0.825505in}}%
\pgfpathlineto{\pgfqpoint{1.036173in}{0.835701in}}%
\pgfpathlineto{\pgfqpoint{1.061400in}{0.965056in}}%
\pgfpathlineto{\pgfqpoint{1.086627in}{0.975507in}}%
\pgfpathlineto{\pgfqpoint{1.111854in}{0.985459in}}%
\pgfpathlineto{\pgfqpoint{1.137081in}{1.008445in}}%
\pgfpathlineto{\pgfqpoint{1.162308in}{1.016975in}}%
\pgfpathlineto{\pgfqpoint{1.187535in}{1.068560in}}%
\pgfpathlineto{\pgfqpoint{1.212763in}{1.078132in}}%
\pgfpathlineto{\pgfqpoint{1.237990in}{1.078132in}}%
\pgfpathlineto{\pgfqpoint{1.263217in}{1.078132in}}%
\pgfpathlineto{\pgfqpoint{1.288444in}{1.084279in}}%
\pgfpathlineto{\pgfqpoint{1.313671in}{1.084279in}}%
\pgfpathlineto{\pgfqpoint{1.338898in}{1.096054in}}%
\pgfpathlineto{\pgfqpoint{1.364125in}{1.101702in}}%
\pgfpathlineto{\pgfqpoint{1.389352in}{1.148887in}}%
\pgfpathlineto{\pgfqpoint{1.414580in}{1.187356in}}%
\pgfpathlineto{\pgfqpoint{1.439807in}{1.239460in}}%
\pgfpathlineto{\pgfqpoint{1.465034in}{1.253395in}}%
\pgfpathlineto{\pgfqpoint{1.490261in}{1.258720in}}%
\pgfpathlineto{\pgfqpoint{1.515488in}{1.503492in}}%
\pgfpathlineto{\pgfqpoint{1.515488in}{2.605275in}}%
\pgfusepath{stroke}%
\end{pgfscope}%
\begin{pgfscope}%
\pgfpathrectangle{\pgfqpoint{0.708220in}{0.535823in}}{\pgfqpoint{5.045427in}{2.069453in}}%
\pgfusepath{clip}%
\pgfsetrectcap%
\pgfsetroundjoin%
\pgfsetlinewidth{1.003750pt}%
\definecolor{currentstroke}{rgb}{0.090196,0.745098,0.811765}%
\pgfsetstrokecolor{currentstroke}%
\pgfsetdash{}{0pt}%
\pgfpathmoveto{\pgfqpoint{0.708220in}{0.535823in}}%
\pgfpathlineto{\pgfqpoint{0.733447in}{0.535823in}}%
\pgfpathlineto{\pgfqpoint{0.758674in}{0.573921in}}%
\pgfpathlineto{\pgfqpoint{0.783901in}{0.573921in}}%
\pgfpathlineto{\pgfqpoint{0.809128in}{0.573921in}}%
\pgfpathlineto{\pgfqpoint{0.834356in}{0.606133in}}%
\pgfpathlineto{\pgfqpoint{0.859583in}{0.658648in}}%
\pgfpathlineto{\pgfqpoint{0.884810in}{0.680664in}}%
\pgfpathlineto{\pgfqpoint{0.910037in}{0.680664in}}%
\pgfpathlineto{\pgfqpoint{0.935264in}{0.700580in}}%
\pgfpathlineto{\pgfqpoint{0.960491in}{0.735488in}}%
\pgfpathlineto{\pgfqpoint{0.985718in}{0.750974in}}%
\pgfpathlineto{\pgfqpoint{1.010945in}{0.765391in}}%
\pgfpathlineto{\pgfqpoint{1.036173in}{0.778877in}}%
\pgfpathlineto{\pgfqpoint{1.061400in}{0.778877in}}%
\pgfpathlineto{\pgfqpoint{1.086627in}{0.791545in}}%
\pgfpathlineto{\pgfqpoint{1.111854in}{0.825505in}}%
\pgfpathlineto{\pgfqpoint{1.137081in}{0.854710in}}%
\pgfpathlineto{\pgfqpoint{1.162308in}{0.863604in}}%
\pgfpathlineto{\pgfqpoint{1.187535in}{0.895815in}}%
\pgfpathlineto{\pgfqpoint{1.212763in}{0.903148in}}%
\pgfpathlineto{\pgfqpoint{1.237990in}{0.903148in}}%
\pgfpathlineto{\pgfqpoint{1.263217in}{0.917084in}}%
\pgfpathlineto{\pgfqpoint{1.288444in}{0.930148in}}%
\pgfpathlineto{\pgfqpoint{1.313671in}{0.942444in}}%
\pgfpathlineto{\pgfqpoint{1.338898in}{0.965056in}}%
\pgfpathlineto{\pgfqpoint{1.364125in}{0.970347in}}%
\pgfpathlineto{\pgfqpoint{1.389352in}{0.970347in}}%
\pgfpathlineto{\pgfqpoint{1.414580in}{0.970347in}}%
\pgfpathlineto{\pgfqpoint{1.439807in}{0.985459in}}%
\pgfpathlineto{\pgfqpoint{1.465034in}{0.990263in}}%
\pgfpathlineto{\pgfqpoint{1.490261in}{1.004046in}}%
\pgfpathlineto{\pgfqpoint{1.515488in}{1.016975in}}%
\pgfpathlineto{\pgfqpoint{1.540715in}{1.058528in}}%
\pgfpathlineto{\pgfqpoint{1.565942in}{1.071800in}}%
\pgfpathlineto{\pgfqpoint{1.591170in}{1.090249in}}%
\pgfpathlineto{\pgfqpoint{1.616397in}{1.101702in}}%
\pgfpathlineto{\pgfqpoint{1.641624in}{1.163896in}}%
\pgfpathlineto{\pgfqpoint{1.666851in}{1.170012in}}%
\pgfpathlineto{\pgfqpoint{1.692078in}{1.240896in}}%
\pgfpathlineto{\pgfqpoint{1.717305in}{1.268977in}}%
\pgfpathlineto{\pgfqpoint{1.742532in}{1.271465in}}%
\pgfpathlineto{\pgfqpoint{1.767759in}{1.299214in}}%
\pgfpathlineto{\pgfqpoint{1.792987in}{1.306658in}}%
\pgfpathlineto{\pgfqpoint{1.818214in}{1.337672in}}%
\pgfpathlineto{\pgfqpoint{1.843441in}{1.346491in}}%
\pgfpathlineto{\pgfqpoint{1.868668in}{1.350764in}}%
\pgfpathlineto{\pgfqpoint{1.893895in}{1.363084in}}%
\pgfpathlineto{\pgfqpoint{1.919122in}{1.363880in}}%
\pgfpathlineto{\pgfqpoint{1.944349in}{1.364673in}}%
\pgfpathlineto{\pgfqpoint{1.969577in}{1.373203in}}%
\pgfpathlineto{\pgfqpoint{1.969577in}{2.605275in}}%
\pgfusepath{stroke}%
\end{pgfscope}%
\begin{pgfscope}%
\pgfpathrectangle{\pgfqpoint{0.708220in}{0.535823in}}{\pgfqpoint{5.045427in}{2.069453in}}%
\pgfusepath{clip}%
\pgfsetrectcap%
\pgfsetroundjoin%
\pgfsetlinewidth{1.003750pt}%
\definecolor{currentstroke}{rgb}{0.121569,0.466667,0.705882}%
\pgfsetstrokecolor{currentstroke}%
\pgfsetdash{}{0pt}%
\pgfpathmoveto{\pgfqpoint{0.708220in}{0.535823in}}%
\pgfpathlineto{\pgfqpoint{0.733447in}{0.535823in}}%
\pgfpathlineto{\pgfqpoint{0.758674in}{0.573921in}}%
\pgfpathlineto{\pgfqpoint{0.783901in}{0.573921in}}%
\pgfpathlineto{\pgfqpoint{0.809128in}{0.573921in}}%
\pgfpathlineto{\pgfqpoint{0.834356in}{0.606133in}}%
\pgfpathlineto{\pgfqpoint{0.859583in}{0.658648in}}%
\pgfpathlineto{\pgfqpoint{0.884810in}{0.680664in}}%
\pgfpathlineto{\pgfqpoint{0.910037in}{0.718762in}}%
\pgfpathlineto{\pgfqpoint{0.935264in}{0.765391in}}%
\pgfpathlineto{\pgfqpoint{0.960491in}{0.791545in}}%
\pgfpathlineto{\pgfqpoint{0.985718in}{0.791545in}}%
\pgfpathlineto{\pgfqpoint{1.010945in}{0.791545in}}%
\pgfpathlineto{\pgfqpoint{1.036173in}{0.803489in}}%
\pgfpathlineto{\pgfqpoint{1.061400in}{0.803489in}}%
\pgfpathlineto{\pgfqpoint{1.086627in}{0.803489in}}%
\pgfpathlineto{\pgfqpoint{1.111854in}{0.854710in}}%
\pgfpathlineto{\pgfqpoint{1.137081in}{0.854710in}}%
\pgfpathlineto{\pgfqpoint{1.162308in}{0.863604in}}%
\pgfpathlineto{\pgfqpoint{1.187535in}{0.863604in}}%
\pgfpathlineto{\pgfqpoint{1.212763in}{0.872134in}}%
\pgfpathlineto{\pgfqpoint{1.237990in}{0.872134in}}%
\pgfpathlineto{\pgfqpoint{1.263217in}{0.880330in}}%
\pgfpathlineto{\pgfqpoint{1.288444in}{0.903148in}}%
\pgfpathlineto{\pgfqpoint{1.313671in}{0.923718in}}%
\pgfpathlineto{\pgfqpoint{1.338898in}{0.948331in}}%
\pgfpathlineto{\pgfqpoint{1.364125in}{0.970347in}}%
\pgfpathlineto{\pgfqpoint{1.389352in}{0.975507in}}%
\pgfpathlineto{\pgfqpoint{1.414580in}{1.036892in}}%
\pgfpathlineto{\pgfqpoint{1.439807in}{2.214091in}}%
\pgfpathlineto{\pgfqpoint{1.439807in}{2.605275in}}%
\pgfusepath{stroke}%
\end{pgfscope}%
\begin{pgfscope}%
\pgfsetrectcap%
\pgfsetmiterjoin%
\pgfsetlinewidth{0.803000pt}%
\definecolor{currentstroke}{rgb}{0.000000,0.000000,0.000000}%
\pgfsetstrokecolor{currentstroke}%
\pgfsetdash{}{0pt}%
\pgfpathmoveto{\pgfqpoint{0.708220in}{0.535823in}}%
\pgfpathlineto{\pgfqpoint{0.708220in}{2.605275in}}%
\pgfusepath{stroke}%
\end{pgfscope}%
\begin{pgfscope}%
\pgfsetrectcap%
\pgfsetmiterjoin%
\pgfsetlinewidth{0.803000pt}%
\definecolor{currentstroke}{rgb}{0.000000,0.000000,0.000000}%
\pgfsetstrokecolor{currentstroke}%
\pgfsetdash{}{0pt}%
\pgfpathmoveto{\pgfqpoint{5.753646in}{0.535823in}}%
\pgfpathlineto{\pgfqpoint{5.753646in}{2.605275in}}%
\pgfusepath{stroke}%
\end{pgfscope}%
\begin{pgfscope}%
\pgfsetrectcap%
\pgfsetmiterjoin%
\pgfsetlinewidth{0.803000pt}%
\definecolor{currentstroke}{rgb}{0.000000,0.000000,0.000000}%
\pgfsetstrokecolor{currentstroke}%
\pgfsetdash{}{0pt}%
\pgfpathmoveto{\pgfqpoint{0.708220in}{0.535823in}}%
\pgfpathlineto{\pgfqpoint{5.753646in}{0.535823in}}%
\pgfusepath{stroke}%
\end{pgfscope}%
\begin{pgfscope}%
\pgfsetrectcap%
\pgfsetmiterjoin%
\pgfsetlinewidth{0.803000pt}%
\definecolor{currentstroke}{rgb}{0.000000,0.000000,0.000000}%
\pgfsetstrokecolor{currentstroke}%
\pgfsetdash{}{0pt}%
\pgfpathmoveto{\pgfqpoint{0.708220in}{2.605275in}}%
\pgfpathlineto{\pgfqpoint{5.753646in}{2.605275in}}%
\pgfusepath{stroke}%
\end{pgfscope}%
\begin{pgfscope}%
\pgfsetrectcap%
\pgfsetroundjoin%
\pgfsetlinewidth{1.003750pt}%
\definecolor{currentstroke}{rgb}{0.121569,0.466667,0.705882}%
\pgfsetstrokecolor{currentstroke}%
\pgfsetdash{}{0pt}%
\pgfpathmoveto{\pgfqpoint{4.479690in}{2.410509in}}%
\pgfpathlineto{\pgfqpoint{4.729690in}{2.410509in}}%
\pgfusepath{stroke}%
\end{pgfscope}%
\begin{pgfscope}%
\definecolor{textcolor}{rgb}{0.000000,0.000000,0.000000}%
\pgfsetstrokecolor{textcolor}%
\pgfsetfillcolor{textcolor}%
\pgftext[x=4.754690in,y=2.366759in,left,base]{\color{textcolor}\rmfamily\fontsize{9.000000}{10.800000}\selectfont LG(FlowCutter)}%
\end{pgfscope}%
\begin{pgfscope}%
\pgfsetrectcap%
\pgfsetroundjoin%
\pgfsetlinewidth{1.003750pt}%
\definecolor{currentstroke}{rgb}{1.000000,0.498039,0.054902}%
\pgfsetstrokecolor{currentstroke}%
\pgfsetdash{}{0pt}%
\pgfpathmoveto{\pgfqpoint{4.479690in}{2.235540in}}%
\pgfpathlineto{\pgfqpoint{4.729690in}{2.235540in}}%
\pgfusepath{stroke}%
\end{pgfscope}%
\begin{pgfscope}%
\definecolor{textcolor}{rgb}{0.000000,0.000000,0.000000}%
\pgfsetstrokecolor{textcolor}%
\pgfsetfillcolor{textcolor}%
\pgftext[x=4.754690in,y=2.191790in,left,base]{\color{textcolor}\rmfamily\fontsize{9.000000}{10.800000}\selectfont LG(htd)}%
\end{pgfscope}%
\begin{pgfscope}%
\pgfsetrectcap%
\pgfsetroundjoin%
\pgfsetlinewidth{1.003750pt}%
\definecolor{currentstroke}{rgb}{0.172549,0.627451,0.172549}%
\pgfsetstrokecolor{currentstroke}%
\pgfsetdash{}{0pt}%
\pgfpathmoveto{\pgfqpoint{4.479690in}{2.060571in}}%
\pgfpathlineto{\pgfqpoint{4.729690in}{2.060571in}}%
\pgfusepath{stroke}%
\end{pgfscope}%
\begin{pgfscope}%
\definecolor{textcolor}{rgb}{0.000000,0.000000,0.000000}%
\pgfsetstrokecolor{textcolor}%
\pgfsetfillcolor{textcolor}%
\pgftext[x=4.754690in,y=2.016821in,left,base]{\color{textcolor}\rmfamily\fontsize{9.000000}{10.800000}\selectfont LG(Tamaki)}%
\end{pgfscope}%
\begin{pgfscope}%
\pgfsetbuttcap%
\pgfsetroundjoin%
\pgfsetlinewidth{1.003750pt}%
\definecolor{currentstroke}{rgb}{0.839216,0.152941,0.156863}%
\pgfsetstrokecolor{currentstroke}%
\pgfsetdash{{3.700000pt}{1.600000pt}}{0.000000pt}%
\pgfpathmoveto{\pgfqpoint{4.479690in}{1.885601in}}%
\pgfpathlineto{\pgfqpoint{4.729690in}{1.885601in}}%
\pgfusepath{stroke}%
\end{pgfscope}%
\begin{pgfscope}%
\definecolor{textcolor}{rgb}{0.000000,0.000000,0.000000}%
\pgfsetstrokecolor{textcolor}%
\pgfsetfillcolor{textcolor}%
\pgftext[x=4.754690in,y=1.841851in,left,base]{\color{textcolor}\rmfamily\fontsize{9.000000}{10.800000}\selectfont HTB(MCS, BE)}%
\end{pgfscope}%
\begin{pgfscope}%
\pgfsetbuttcap%
\pgfsetroundjoin%
\pgfsetlinewidth{1.003750pt}%
\definecolor{currentstroke}{rgb}{0.580392,0.403922,0.741176}%
\pgfsetstrokecolor{currentstroke}%
\pgfsetdash{{3.700000pt}{1.600000pt}}{0.000000pt}%
\pgfpathmoveto{\pgfqpoint{4.479690in}{1.710632in}}%
\pgfpathlineto{\pgfqpoint{4.729690in}{1.710632in}}%
\pgfusepath{stroke}%
\end{pgfscope}%
\begin{pgfscope}%
\definecolor{textcolor}{rgb}{0.000000,0.000000,0.000000}%
\pgfsetstrokecolor{textcolor}%
\pgfsetfillcolor{textcolor}%
\pgftext[x=4.754690in,y=1.666882in,left,base]{\color{textcolor}\rmfamily\fontsize{9.000000}{10.800000}\selectfont HTB(MCS, BM)}%
\end{pgfscope}%
\begin{pgfscope}%
\pgfsetbuttcap%
\pgfsetroundjoin%
\pgfsetlinewidth{1.003750pt}%
\definecolor{currentstroke}{rgb}{0.549020,0.337255,0.294118}%
\pgfsetstrokecolor{currentstroke}%
\pgfsetdash{{1.000000pt}{1.650000pt}}{0.000000pt}%
\pgfpathmoveto{\pgfqpoint{4.479690in}{1.535662in}}%
\pgfpathlineto{\pgfqpoint{4.729690in}{1.535662in}}%
\pgfusepath{stroke}%
\end{pgfscope}%
\begin{pgfscope}%
\definecolor{textcolor}{rgb}{0.000000,0.000000,0.000000}%
\pgfsetstrokecolor{textcolor}%
\pgfsetfillcolor{textcolor}%
\pgftext[x=4.754690in,y=1.491912in,left,base]{\color{textcolor}\rmfamily\fontsize{9.000000}{10.800000}\selectfont HTB(LP, BE)}%
\end{pgfscope}%
\begin{pgfscope}%
\pgfsetbuttcap%
\pgfsetroundjoin%
\pgfsetlinewidth{1.003750pt}%
\definecolor{currentstroke}{rgb}{0.890196,0.466667,0.760784}%
\pgfsetstrokecolor{currentstroke}%
\pgfsetdash{{1.000000pt}{1.650000pt}}{0.000000pt}%
\pgfpathmoveto{\pgfqpoint{4.479690in}{1.360693in}}%
\pgfpathlineto{\pgfqpoint{4.729690in}{1.360693in}}%
\pgfusepath{stroke}%
\end{pgfscope}%
\begin{pgfscope}%
\definecolor{textcolor}{rgb}{0.000000,0.000000,0.000000}%
\pgfsetstrokecolor{textcolor}%
\pgfsetfillcolor{textcolor}%
\pgftext[x=4.754690in,y=1.316943in,left,base]{\color{textcolor}\rmfamily\fontsize{9.000000}{10.800000}\selectfont HTB(LP, BM)}%
\end{pgfscope}%
\begin{pgfscope}%
\pgfsetbuttcap%
\pgfsetroundjoin%
\pgfsetlinewidth{1.003750pt}%
\definecolor{currentstroke}{rgb}{0.498039,0.498039,0.498039}%
\pgfsetstrokecolor{currentstroke}%
\pgfsetdash{{6.400000pt}{1.600000pt}{1.000000pt}{1.600000pt}}{0.000000pt}%
\pgfpathmoveto{\pgfqpoint{4.479690in}{1.185723in}}%
\pgfpathlineto{\pgfqpoint{4.729690in}{1.185723in}}%
\pgfusepath{stroke}%
\end{pgfscope}%
\begin{pgfscope}%
\definecolor{textcolor}{rgb}{0.000000,0.000000,0.000000}%
\pgfsetstrokecolor{textcolor}%
\pgfsetfillcolor{textcolor}%
\pgftext[x=4.754690in,y=1.141973in,left,base]{\color{textcolor}\rmfamily\fontsize{9.000000}{10.800000}\selectfont HTB(LM, BE)}%
\end{pgfscope}%
\begin{pgfscope}%
\pgfsetbuttcap%
\pgfsetroundjoin%
\pgfsetlinewidth{1.003750pt}%
\definecolor{currentstroke}{rgb}{0.737255,0.741176,0.133333}%
\pgfsetstrokecolor{currentstroke}%
\pgfsetdash{{6.400000pt}{1.600000pt}{1.000000pt}{1.600000pt}}{0.000000pt}%
\pgfpathmoveto{\pgfqpoint{4.479690in}{1.010754in}}%
\pgfpathlineto{\pgfqpoint{4.729690in}{1.010754in}}%
\pgfusepath{stroke}%
\end{pgfscope}%
\begin{pgfscope}%
\definecolor{textcolor}{rgb}{0.000000,0.000000,0.000000}%
\pgfsetstrokecolor{textcolor}%
\pgfsetfillcolor{textcolor}%
\pgftext[x=4.754690in,y=0.967004in,left,base]{\color{textcolor}\rmfamily\fontsize{9.000000}{10.800000}\selectfont HTB(LM, BM)}%
\end{pgfscope}%
\begin{pgfscope}%
\pgfsetrectcap%
\pgfsetroundjoin%
\pgfsetlinewidth{1.003750pt}%
\definecolor{currentstroke}{rgb}{0.090196,0.745098,0.811765}%
\pgfsetstrokecolor{currentstroke}%
\pgfsetdash{}{0pt}%
\pgfpathmoveto{\pgfqpoint{4.479690in}{0.835784in}}%
\pgfpathlineto{\pgfqpoint{4.729690in}{0.835784in}}%
\pgfusepath{stroke}%
\end{pgfscope}%
\begin{pgfscope}%
\definecolor{textcolor}{rgb}{0.000000,0.000000,0.000000}%
\pgfsetstrokecolor{textcolor}%
\pgfsetfillcolor{textcolor}%
\pgftext[x=4.754690in,y=0.792034in,left,base]{\color{textcolor}\rmfamily\fontsize{9.000000}{10.800000}\selectfont HTB(MF, BE)}%
\end{pgfscope}%
\begin{pgfscope}%
\pgfsetrectcap%
\pgfsetroundjoin%
\pgfsetlinewidth{1.003750pt}%
\definecolor{currentstroke}{rgb}{0.121569,0.466667,0.705882}%
\pgfsetstrokecolor{currentstroke}%
\pgfsetdash{}{0pt}%
\pgfpathmoveto{\pgfqpoint{4.479690in}{0.660815in}}%
\pgfpathlineto{\pgfqpoint{4.729690in}{0.660815in}}%
\pgfusepath{stroke}%
\end{pgfscope}%
\begin{pgfscope}%
\definecolor{textcolor}{rgb}{0.000000,0.000000,0.000000}%
\pgfsetstrokecolor{textcolor}%
\pgfsetfillcolor{textcolor}%
\pgftext[x=4.754690in,y=0.617065in,left,base]{\color{textcolor}\rmfamily\fontsize{9.000000}{10.800000}\selectfont HTB(MF, BM)}%
\end{pgfscope}%
\end{pgfpicture}%
\makeatother%
\endgroup%

    \caption{
        Experiment 1 compares the tree-decomposition-based planner \Lg{} to the constraint-satisfaction-based planner \htb{}. 
        \Lg{} can be used with a tree decomposer (\flowcutter{} \cite{strasser2017computing}, \htd{} \cite{abseher2017htd}, or \tamaki{} \cite{tamaki2019positive}).
        \htb{} requires a variable-ordering heuristic (\mcs{} \cite{tarjan1984simple}, \lexp/\lexm{} \cite{koster2001treewidth}, or \minfill{} \cite{dechter2003constraint}) and a clause-ordering heuristic (\be{} \cite{dechter1999bucket} or \bm{} \cite{bouquet1999gestion}).
        A planner ``solves'' a benchmark when it eventually finds a project-join tree of width \maxWidth{} or lower.
        \Lg{} is an anytime tool that can output several trees (of decreasing widths) for each benchmark.
        On this plot, for each \Lg{} benchmark, we use the time of the first tree whose width is at most \maxWidth.
        In contrast, in Figure \ref{figPlanning}, we discard an \Lg{} benchmark when the first tree has width over \maxWidth, even if a later tree has width at most \maxWidth.
    }
    \label{figPlanningA}
\end{figure}

%%%%%%%%%%%%%%%%%%%%%%%%%%%%%%%%%%%%%%%%%%%%%%%%%%%%%%%%%%%%%%%%%%%%%%%%%%%%%%%%

\subsection{Experiment 2: Comparing Execution Heuristics}

Figure \ref{figExecutionA} shows the performance of four ADD variable-ordering heuristics with the executor \dmc{} in 100 seconds (execution time only, excluding planning time).
The graded project-join trees here are taken from Experiment 1.
Recall that \Lg{} is an anytime tool that may produce several project-join trees (of decreasing widths) for each benchmark.
We measure the execution time using the first tree and the last tree produced within 100 seconds for each benchmark.
\begin{figure}[H]
    \centering
    %% Creator: Matplotlib, PGF backend
%%
%% To include the figure in your LaTeX document, write
%%   \input{<filename>.pgf}
%%
%% Make sure the required packages are loaded in your preamble
%%   \usepackage{pgf}
%%
%% and, on pdftex
%%   \usepackage[utf8]{inputenc}\DeclareUnicodeCharacter{2212}{-}
%%
%% or, on luatex and xetex
%%   \usepackage{unicode-math}
%%
%% Figures using additional raster images can only be included by \input if
%% they are in the same directory as the main LaTeX file. For loading figures
%% from other directories you can use the `import` package
%%   \usepackage{import}
%%
%% and then include the figures with
%%   \import{<path to file>}{<filename>.pgf}
%%
%% Matplotlib used the following preamble
%%   \usepackage[utf8x]{inputenc}
%%   \usepackage[T1]{fontenc}
%%
\begingroup%
\makeatletter%
\begin{pgfpicture}%
\pgfpathrectangle{\pgfpointorigin}{\pgfqpoint{6.000000in}{2.500000in}}%
\pgfusepath{use as bounding box, clip}%
\begin{pgfscope}%
\pgfsetbuttcap%
\pgfsetmiterjoin%
\definecolor{currentfill}{rgb}{1.000000,1.000000,1.000000}%
\pgfsetfillcolor{currentfill}%
\pgfsetlinewidth{0.000000pt}%
\definecolor{currentstroke}{rgb}{1.000000,1.000000,1.000000}%
\pgfsetstrokecolor{currentstroke}%
\pgfsetdash{}{0pt}%
\pgfpathmoveto{\pgfqpoint{0.000000in}{0.000000in}}%
\pgfpathlineto{\pgfqpoint{6.000000in}{0.000000in}}%
\pgfpathlineto{\pgfqpoint{6.000000in}{2.500000in}}%
\pgfpathlineto{\pgfqpoint{0.000000in}{2.500000in}}%
\pgfpathclose%
\pgfusepath{fill}%
\end{pgfscope}%
\begin{pgfscope}%
\pgfsetbuttcap%
\pgfsetmiterjoin%
\definecolor{currentfill}{rgb}{1.000000,1.000000,1.000000}%
\pgfsetfillcolor{currentfill}%
\pgfsetlinewidth{0.000000pt}%
\definecolor{currentstroke}{rgb}{0.000000,0.000000,0.000000}%
\pgfsetstrokecolor{currentstroke}%
\pgfsetstrokeopacity{0.000000}%
\pgfsetdash{}{0pt}%
\pgfpathmoveto{\pgfqpoint{0.708220in}{0.535823in}}%
\pgfpathlineto{\pgfqpoint{5.753646in}{0.535823in}}%
\pgfpathlineto{\pgfqpoint{5.753646in}{2.305275in}}%
\pgfpathlineto{\pgfqpoint{0.708220in}{2.305275in}}%
\pgfpathclose%
\pgfusepath{fill}%
\end{pgfscope}%
\begin{pgfscope}%
\pgfsetbuttcap%
\pgfsetroundjoin%
\definecolor{currentfill}{rgb}{0.000000,0.000000,0.000000}%
\pgfsetfillcolor{currentfill}%
\pgfsetlinewidth{0.803000pt}%
\definecolor{currentstroke}{rgb}{0.000000,0.000000,0.000000}%
\pgfsetstrokecolor{currentstroke}%
\pgfsetdash{}{0pt}%
\pgfsys@defobject{currentmarker}{\pgfqpoint{0.000000in}{-0.048611in}}{\pgfqpoint{0.000000in}{0.000000in}}{%
\pgfpathmoveto{\pgfqpoint{0.000000in}{0.000000in}}%
\pgfpathlineto{\pgfqpoint{0.000000in}{-0.048611in}}%
\pgfusepath{stroke,fill}%
}%
\begin{pgfscope}%
\pgfsys@transformshift{0.708220in}{0.535823in}%
\pgfsys@useobject{currentmarker}{}%
\end{pgfscope}%
\end{pgfscope}%
\begin{pgfscope}%
\definecolor{textcolor}{rgb}{0.000000,0.000000,0.000000}%
\pgfsetstrokecolor{textcolor}%
\pgfsetfillcolor{textcolor}%
\pgftext[x=0.708220in,y=0.438600in,,top]{\color{textcolor}\rmfamily\fontsize{9.000000}{10.800000}\selectfont \(\displaystyle {0}\)}%
\end{pgfscope}%
\begin{pgfscope}%
\pgfsetbuttcap%
\pgfsetroundjoin%
\definecolor{currentfill}{rgb}{0.000000,0.000000,0.000000}%
\pgfsetfillcolor{currentfill}%
\pgfsetlinewidth{0.803000pt}%
\definecolor{currentstroke}{rgb}{0.000000,0.000000,0.000000}%
\pgfsetstrokecolor{currentstroke}%
\pgfsetdash{}{0pt}%
\pgfsys@defobject{currentmarker}{\pgfqpoint{0.000000in}{-0.048611in}}{\pgfqpoint{0.000000in}{0.000000in}}{%
\pgfpathmoveto{\pgfqpoint{0.000000in}{0.000000in}}%
\pgfpathlineto{\pgfqpoint{0.000000in}{-0.048611in}}%
\pgfusepath{stroke,fill}%
}%
\begin{pgfscope}%
\pgfsys@transformshift{1.338898in}{0.535823in}%
\pgfsys@useobject{currentmarker}{}%
\end{pgfscope}%
\end{pgfscope}%
\begin{pgfscope}%
\definecolor{textcolor}{rgb}{0.000000,0.000000,0.000000}%
\pgfsetstrokecolor{textcolor}%
\pgfsetfillcolor{textcolor}%
\pgftext[x=1.338898in,y=0.438600in,,top]{\color{textcolor}\rmfamily\fontsize{9.000000}{10.800000}\selectfont \(\displaystyle {50}\)}%
\end{pgfscope}%
\begin{pgfscope}%
\pgfsetbuttcap%
\pgfsetroundjoin%
\definecolor{currentfill}{rgb}{0.000000,0.000000,0.000000}%
\pgfsetfillcolor{currentfill}%
\pgfsetlinewidth{0.803000pt}%
\definecolor{currentstroke}{rgb}{0.000000,0.000000,0.000000}%
\pgfsetstrokecolor{currentstroke}%
\pgfsetdash{}{0pt}%
\pgfsys@defobject{currentmarker}{\pgfqpoint{0.000000in}{-0.048611in}}{\pgfqpoint{0.000000in}{0.000000in}}{%
\pgfpathmoveto{\pgfqpoint{0.000000in}{0.000000in}}%
\pgfpathlineto{\pgfqpoint{0.000000in}{-0.048611in}}%
\pgfusepath{stroke,fill}%
}%
\begin{pgfscope}%
\pgfsys@transformshift{1.969577in}{0.535823in}%
\pgfsys@useobject{currentmarker}{}%
\end{pgfscope}%
\end{pgfscope}%
\begin{pgfscope}%
\definecolor{textcolor}{rgb}{0.000000,0.000000,0.000000}%
\pgfsetstrokecolor{textcolor}%
\pgfsetfillcolor{textcolor}%
\pgftext[x=1.969577in,y=0.438600in,,top]{\color{textcolor}\rmfamily\fontsize{9.000000}{10.800000}\selectfont \(\displaystyle {100}\)}%
\end{pgfscope}%
\begin{pgfscope}%
\pgfsetbuttcap%
\pgfsetroundjoin%
\definecolor{currentfill}{rgb}{0.000000,0.000000,0.000000}%
\pgfsetfillcolor{currentfill}%
\pgfsetlinewidth{0.803000pt}%
\definecolor{currentstroke}{rgb}{0.000000,0.000000,0.000000}%
\pgfsetstrokecolor{currentstroke}%
\pgfsetdash{}{0pt}%
\pgfsys@defobject{currentmarker}{\pgfqpoint{0.000000in}{-0.048611in}}{\pgfqpoint{0.000000in}{0.000000in}}{%
\pgfpathmoveto{\pgfqpoint{0.000000in}{0.000000in}}%
\pgfpathlineto{\pgfqpoint{0.000000in}{-0.048611in}}%
\pgfusepath{stroke,fill}%
}%
\begin{pgfscope}%
\pgfsys@transformshift{2.600255in}{0.535823in}%
\pgfsys@useobject{currentmarker}{}%
\end{pgfscope}%
\end{pgfscope}%
\begin{pgfscope}%
\definecolor{textcolor}{rgb}{0.000000,0.000000,0.000000}%
\pgfsetstrokecolor{textcolor}%
\pgfsetfillcolor{textcolor}%
\pgftext[x=2.600255in,y=0.438600in,,top]{\color{textcolor}\rmfamily\fontsize{9.000000}{10.800000}\selectfont \(\displaystyle {150}\)}%
\end{pgfscope}%
\begin{pgfscope}%
\pgfsetbuttcap%
\pgfsetroundjoin%
\definecolor{currentfill}{rgb}{0.000000,0.000000,0.000000}%
\pgfsetfillcolor{currentfill}%
\pgfsetlinewidth{0.803000pt}%
\definecolor{currentstroke}{rgb}{0.000000,0.000000,0.000000}%
\pgfsetstrokecolor{currentstroke}%
\pgfsetdash{}{0pt}%
\pgfsys@defobject{currentmarker}{\pgfqpoint{0.000000in}{-0.048611in}}{\pgfqpoint{0.000000in}{0.000000in}}{%
\pgfpathmoveto{\pgfqpoint{0.000000in}{0.000000in}}%
\pgfpathlineto{\pgfqpoint{0.000000in}{-0.048611in}}%
\pgfusepath{stroke,fill}%
}%
\begin{pgfscope}%
\pgfsys@transformshift{3.230933in}{0.535823in}%
\pgfsys@useobject{currentmarker}{}%
\end{pgfscope}%
\end{pgfscope}%
\begin{pgfscope}%
\definecolor{textcolor}{rgb}{0.000000,0.000000,0.000000}%
\pgfsetstrokecolor{textcolor}%
\pgfsetfillcolor{textcolor}%
\pgftext[x=3.230933in,y=0.438600in,,top]{\color{textcolor}\rmfamily\fontsize{9.000000}{10.800000}\selectfont \(\displaystyle {200}\)}%
\end{pgfscope}%
\begin{pgfscope}%
\pgfsetbuttcap%
\pgfsetroundjoin%
\definecolor{currentfill}{rgb}{0.000000,0.000000,0.000000}%
\pgfsetfillcolor{currentfill}%
\pgfsetlinewidth{0.803000pt}%
\definecolor{currentstroke}{rgb}{0.000000,0.000000,0.000000}%
\pgfsetstrokecolor{currentstroke}%
\pgfsetdash{}{0pt}%
\pgfsys@defobject{currentmarker}{\pgfqpoint{0.000000in}{-0.048611in}}{\pgfqpoint{0.000000in}{0.000000in}}{%
\pgfpathmoveto{\pgfqpoint{0.000000in}{0.000000in}}%
\pgfpathlineto{\pgfqpoint{0.000000in}{-0.048611in}}%
\pgfusepath{stroke,fill}%
}%
\begin{pgfscope}%
\pgfsys@transformshift{3.861611in}{0.535823in}%
\pgfsys@useobject{currentmarker}{}%
\end{pgfscope}%
\end{pgfscope}%
\begin{pgfscope}%
\definecolor{textcolor}{rgb}{0.000000,0.000000,0.000000}%
\pgfsetstrokecolor{textcolor}%
\pgfsetfillcolor{textcolor}%
\pgftext[x=3.861611in,y=0.438600in,,top]{\color{textcolor}\rmfamily\fontsize{9.000000}{10.800000}\selectfont \(\displaystyle {250}\)}%
\end{pgfscope}%
\begin{pgfscope}%
\pgfsetbuttcap%
\pgfsetroundjoin%
\definecolor{currentfill}{rgb}{0.000000,0.000000,0.000000}%
\pgfsetfillcolor{currentfill}%
\pgfsetlinewidth{0.803000pt}%
\definecolor{currentstroke}{rgb}{0.000000,0.000000,0.000000}%
\pgfsetstrokecolor{currentstroke}%
\pgfsetdash{}{0pt}%
\pgfsys@defobject{currentmarker}{\pgfqpoint{0.000000in}{-0.048611in}}{\pgfqpoint{0.000000in}{0.000000in}}{%
\pgfpathmoveto{\pgfqpoint{0.000000in}{0.000000in}}%
\pgfpathlineto{\pgfqpoint{0.000000in}{-0.048611in}}%
\pgfusepath{stroke,fill}%
}%
\begin{pgfscope}%
\pgfsys@transformshift{4.492290in}{0.535823in}%
\pgfsys@useobject{currentmarker}{}%
\end{pgfscope}%
\end{pgfscope}%
\begin{pgfscope}%
\definecolor{textcolor}{rgb}{0.000000,0.000000,0.000000}%
\pgfsetstrokecolor{textcolor}%
\pgfsetfillcolor{textcolor}%
\pgftext[x=4.492290in,y=0.438600in,,top]{\color{textcolor}\rmfamily\fontsize{9.000000}{10.800000}\selectfont \(\displaystyle {300}\)}%
\end{pgfscope}%
\begin{pgfscope}%
\pgfsetbuttcap%
\pgfsetroundjoin%
\definecolor{currentfill}{rgb}{0.000000,0.000000,0.000000}%
\pgfsetfillcolor{currentfill}%
\pgfsetlinewidth{0.803000pt}%
\definecolor{currentstroke}{rgb}{0.000000,0.000000,0.000000}%
\pgfsetstrokecolor{currentstroke}%
\pgfsetdash{}{0pt}%
\pgfsys@defobject{currentmarker}{\pgfqpoint{0.000000in}{-0.048611in}}{\pgfqpoint{0.000000in}{0.000000in}}{%
\pgfpathmoveto{\pgfqpoint{0.000000in}{0.000000in}}%
\pgfpathlineto{\pgfqpoint{0.000000in}{-0.048611in}}%
\pgfusepath{stroke,fill}%
}%
\begin{pgfscope}%
\pgfsys@transformshift{5.122968in}{0.535823in}%
\pgfsys@useobject{currentmarker}{}%
\end{pgfscope}%
\end{pgfscope}%
\begin{pgfscope}%
\definecolor{textcolor}{rgb}{0.000000,0.000000,0.000000}%
\pgfsetstrokecolor{textcolor}%
\pgfsetfillcolor{textcolor}%
\pgftext[x=5.122968in,y=0.438600in,,top]{\color{textcolor}\rmfamily\fontsize{9.000000}{10.800000}\selectfont \(\displaystyle {350}\)}%
\end{pgfscope}%
\begin{pgfscope}%
\pgfsetbuttcap%
\pgfsetroundjoin%
\definecolor{currentfill}{rgb}{0.000000,0.000000,0.000000}%
\pgfsetfillcolor{currentfill}%
\pgfsetlinewidth{0.803000pt}%
\definecolor{currentstroke}{rgb}{0.000000,0.000000,0.000000}%
\pgfsetstrokecolor{currentstroke}%
\pgfsetdash{}{0pt}%
\pgfsys@defobject{currentmarker}{\pgfqpoint{0.000000in}{-0.048611in}}{\pgfqpoint{0.000000in}{0.000000in}}{%
\pgfpathmoveto{\pgfqpoint{0.000000in}{0.000000in}}%
\pgfpathlineto{\pgfqpoint{0.000000in}{-0.048611in}}%
\pgfusepath{stroke,fill}%
}%
\begin{pgfscope}%
\pgfsys@transformshift{5.753646in}{0.535823in}%
\pgfsys@useobject{currentmarker}{}%
\end{pgfscope}%
\end{pgfscope}%
\begin{pgfscope}%
\definecolor{textcolor}{rgb}{0.000000,0.000000,0.000000}%
\pgfsetstrokecolor{textcolor}%
\pgfsetfillcolor{textcolor}%
\pgftext[x=5.753646in,y=0.438600in,,top]{\color{textcolor}\rmfamily\fontsize{9.000000}{10.800000}\selectfont \(\displaystyle {400}\)}%
\end{pgfscope}%
\begin{pgfscope}%
\definecolor{textcolor}{rgb}{0.000000,0.000000,0.000000}%
\pgfsetstrokecolor{textcolor}%
\pgfsetfillcolor{textcolor}%
\pgftext[x=3.230933in,y=0.272655in,,top]{\color{textcolor}\rmfamily\fontsize{10.000000}{12.000000}\selectfont Number of benchmarks solved}%
\end{pgfscope}%
\begin{pgfscope}%
\pgfsetbuttcap%
\pgfsetroundjoin%
\definecolor{currentfill}{rgb}{0.000000,0.000000,0.000000}%
\pgfsetfillcolor{currentfill}%
\pgfsetlinewidth{0.803000pt}%
\definecolor{currentstroke}{rgb}{0.000000,0.000000,0.000000}%
\pgfsetstrokecolor{currentstroke}%
\pgfsetdash{}{0pt}%
\pgfsys@defobject{currentmarker}{\pgfqpoint{-0.048611in}{0.000000in}}{\pgfqpoint{-0.000000in}{0.000000in}}{%
\pgfpathmoveto{\pgfqpoint{-0.000000in}{0.000000in}}%
\pgfpathlineto{\pgfqpoint{-0.048611in}{0.000000in}}%
\pgfusepath{stroke,fill}%
}%
\begin{pgfscope}%
\pgfsys@transformshift{0.708220in}{0.659667in}%
\pgfsys@useobject{currentmarker}{}%
\end{pgfscope}%
\end{pgfscope}%
\begin{pgfscope}%
\definecolor{textcolor}{rgb}{0.000000,0.000000,0.000000}%
\pgfsetstrokecolor{textcolor}%
\pgfsetfillcolor{textcolor}%
\pgftext[x=0.344411in, y=0.614942in, left, base]{\color{textcolor}\rmfamily\fontsize{9.000000}{10.800000}\selectfont \(\displaystyle {10^{-2}}\)}%
\end{pgfscope}%
\begin{pgfscope}%
\pgfsetbuttcap%
\pgfsetroundjoin%
\definecolor{currentfill}{rgb}{0.000000,0.000000,0.000000}%
\pgfsetfillcolor{currentfill}%
\pgfsetlinewidth{0.803000pt}%
\definecolor{currentstroke}{rgb}{0.000000,0.000000,0.000000}%
\pgfsetstrokecolor{currentstroke}%
\pgfsetdash{}{0pt}%
\pgfsys@defobject{currentmarker}{\pgfqpoint{-0.048611in}{0.000000in}}{\pgfqpoint{-0.000000in}{0.000000in}}{%
\pgfpathmoveto{\pgfqpoint{-0.000000in}{0.000000in}}%
\pgfpathlineto{\pgfqpoint{-0.048611in}{0.000000in}}%
\pgfusepath{stroke,fill}%
}%
\begin{pgfscope}%
\pgfsys@transformshift{0.708220in}{1.071069in}%
\pgfsys@useobject{currentmarker}{}%
\end{pgfscope}%
\end{pgfscope}%
\begin{pgfscope}%
\definecolor{textcolor}{rgb}{0.000000,0.000000,0.000000}%
\pgfsetstrokecolor{textcolor}%
\pgfsetfillcolor{textcolor}%
\pgftext[x=0.344411in, y=1.026344in, left, base]{\color{textcolor}\rmfamily\fontsize{9.000000}{10.800000}\selectfont \(\displaystyle {10^{-1}}\)}%
\end{pgfscope}%
\begin{pgfscope}%
\pgfsetbuttcap%
\pgfsetroundjoin%
\definecolor{currentfill}{rgb}{0.000000,0.000000,0.000000}%
\pgfsetfillcolor{currentfill}%
\pgfsetlinewidth{0.803000pt}%
\definecolor{currentstroke}{rgb}{0.000000,0.000000,0.000000}%
\pgfsetstrokecolor{currentstroke}%
\pgfsetdash{}{0pt}%
\pgfsys@defobject{currentmarker}{\pgfqpoint{-0.048611in}{0.000000in}}{\pgfqpoint{-0.000000in}{0.000000in}}{%
\pgfpathmoveto{\pgfqpoint{-0.000000in}{0.000000in}}%
\pgfpathlineto{\pgfqpoint{-0.048611in}{0.000000in}}%
\pgfusepath{stroke,fill}%
}%
\begin{pgfscope}%
\pgfsys@transformshift{0.708220in}{1.482471in}%
\pgfsys@useobject{currentmarker}{}%
\end{pgfscope}%
\end{pgfscope}%
\begin{pgfscope}%
\definecolor{textcolor}{rgb}{0.000000,0.000000,0.000000}%
\pgfsetstrokecolor{textcolor}%
\pgfsetfillcolor{textcolor}%
\pgftext[x=0.424657in, y=1.437746in, left, base]{\color{textcolor}\rmfamily\fontsize{9.000000}{10.800000}\selectfont \(\displaystyle {10^{0}}\)}%
\end{pgfscope}%
\begin{pgfscope}%
\pgfsetbuttcap%
\pgfsetroundjoin%
\definecolor{currentfill}{rgb}{0.000000,0.000000,0.000000}%
\pgfsetfillcolor{currentfill}%
\pgfsetlinewidth{0.803000pt}%
\definecolor{currentstroke}{rgb}{0.000000,0.000000,0.000000}%
\pgfsetstrokecolor{currentstroke}%
\pgfsetdash{}{0pt}%
\pgfsys@defobject{currentmarker}{\pgfqpoint{-0.048611in}{0.000000in}}{\pgfqpoint{-0.000000in}{0.000000in}}{%
\pgfpathmoveto{\pgfqpoint{-0.000000in}{0.000000in}}%
\pgfpathlineto{\pgfqpoint{-0.048611in}{0.000000in}}%
\pgfusepath{stroke,fill}%
}%
\begin{pgfscope}%
\pgfsys@transformshift{0.708220in}{1.893873in}%
\pgfsys@useobject{currentmarker}{}%
\end{pgfscope}%
\end{pgfscope}%
\begin{pgfscope}%
\definecolor{textcolor}{rgb}{0.000000,0.000000,0.000000}%
\pgfsetstrokecolor{textcolor}%
\pgfsetfillcolor{textcolor}%
\pgftext[x=0.424657in, y=1.849148in, left, base]{\color{textcolor}\rmfamily\fontsize{9.000000}{10.800000}\selectfont \(\displaystyle {10^{1}}\)}%
\end{pgfscope}%
\begin{pgfscope}%
\pgfsetbuttcap%
\pgfsetroundjoin%
\definecolor{currentfill}{rgb}{0.000000,0.000000,0.000000}%
\pgfsetfillcolor{currentfill}%
\pgfsetlinewidth{0.803000pt}%
\definecolor{currentstroke}{rgb}{0.000000,0.000000,0.000000}%
\pgfsetstrokecolor{currentstroke}%
\pgfsetdash{}{0pt}%
\pgfsys@defobject{currentmarker}{\pgfqpoint{-0.048611in}{0.000000in}}{\pgfqpoint{-0.000000in}{0.000000in}}{%
\pgfpathmoveto{\pgfqpoint{-0.000000in}{0.000000in}}%
\pgfpathlineto{\pgfqpoint{-0.048611in}{0.000000in}}%
\pgfusepath{stroke,fill}%
}%
\begin{pgfscope}%
\pgfsys@transformshift{0.708220in}{2.305275in}%
\pgfsys@useobject{currentmarker}{}%
\end{pgfscope}%
\end{pgfscope}%
\begin{pgfscope}%
\definecolor{textcolor}{rgb}{0.000000,0.000000,0.000000}%
\pgfsetstrokecolor{textcolor}%
\pgfsetfillcolor{textcolor}%
\pgftext[x=0.424657in, y=2.260550in, left, base]{\color{textcolor}\rmfamily\fontsize{9.000000}{10.800000}\selectfont \(\displaystyle {10^{2}}\)}%
\end{pgfscope}%
\begin{pgfscope}%
\pgfsetbuttcap%
\pgfsetroundjoin%
\definecolor{currentfill}{rgb}{0.000000,0.000000,0.000000}%
\pgfsetfillcolor{currentfill}%
\pgfsetlinewidth{0.602250pt}%
\definecolor{currentstroke}{rgb}{0.000000,0.000000,0.000000}%
\pgfsetstrokecolor{currentstroke}%
\pgfsetdash{}{0pt}%
\pgfsys@defobject{currentmarker}{\pgfqpoint{-0.027778in}{0.000000in}}{\pgfqpoint{-0.000000in}{0.000000in}}{%
\pgfpathmoveto{\pgfqpoint{-0.000000in}{0.000000in}}%
\pgfpathlineto{\pgfqpoint{-0.027778in}{0.000000in}}%
\pgfusepath{stroke,fill}%
}%
\begin{pgfscope}%
\pgfsys@transformshift{0.708220in}{0.535823in}%
\pgfsys@useobject{currentmarker}{}%
\end{pgfscope}%
\end{pgfscope}%
\begin{pgfscope}%
\pgfsetbuttcap%
\pgfsetroundjoin%
\definecolor{currentfill}{rgb}{0.000000,0.000000,0.000000}%
\pgfsetfillcolor{currentfill}%
\pgfsetlinewidth{0.602250pt}%
\definecolor{currentstroke}{rgb}{0.000000,0.000000,0.000000}%
\pgfsetstrokecolor{currentstroke}%
\pgfsetdash{}{0pt}%
\pgfsys@defobject{currentmarker}{\pgfqpoint{-0.027778in}{0.000000in}}{\pgfqpoint{-0.000000in}{0.000000in}}{%
\pgfpathmoveto{\pgfqpoint{-0.000000in}{0.000000in}}%
\pgfpathlineto{\pgfqpoint{-0.027778in}{0.000000in}}%
\pgfusepath{stroke,fill}%
}%
\begin{pgfscope}%
\pgfsys@transformshift{0.708220in}{0.568398in}%
\pgfsys@useobject{currentmarker}{}%
\end{pgfscope}%
\end{pgfscope}%
\begin{pgfscope}%
\pgfsetbuttcap%
\pgfsetroundjoin%
\definecolor{currentfill}{rgb}{0.000000,0.000000,0.000000}%
\pgfsetfillcolor{currentfill}%
\pgfsetlinewidth{0.602250pt}%
\definecolor{currentstroke}{rgb}{0.000000,0.000000,0.000000}%
\pgfsetstrokecolor{currentstroke}%
\pgfsetdash{}{0pt}%
\pgfsys@defobject{currentmarker}{\pgfqpoint{-0.027778in}{0.000000in}}{\pgfqpoint{-0.000000in}{0.000000in}}{%
\pgfpathmoveto{\pgfqpoint{-0.000000in}{0.000000in}}%
\pgfpathlineto{\pgfqpoint{-0.027778in}{0.000000in}}%
\pgfusepath{stroke,fill}%
}%
\begin{pgfscope}%
\pgfsys@transformshift{0.708220in}{0.595940in}%
\pgfsys@useobject{currentmarker}{}%
\end{pgfscope}%
\end{pgfscope}%
\begin{pgfscope}%
\pgfsetbuttcap%
\pgfsetroundjoin%
\definecolor{currentfill}{rgb}{0.000000,0.000000,0.000000}%
\pgfsetfillcolor{currentfill}%
\pgfsetlinewidth{0.602250pt}%
\definecolor{currentstroke}{rgb}{0.000000,0.000000,0.000000}%
\pgfsetstrokecolor{currentstroke}%
\pgfsetdash{}{0pt}%
\pgfsys@defobject{currentmarker}{\pgfqpoint{-0.027778in}{0.000000in}}{\pgfqpoint{-0.000000in}{0.000000in}}{%
\pgfpathmoveto{\pgfqpoint{-0.000000in}{0.000000in}}%
\pgfpathlineto{\pgfqpoint{-0.027778in}{0.000000in}}%
\pgfusepath{stroke,fill}%
}%
\begin{pgfscope}%
\pgfsys@transformshift{0.708220in}{0.619798in}%
\pgfsys@useobject{currentmarker}{}%
\end{pgfscope}%
\end{pgfscope}%
\begin{pgfscope}%
\pgfsetbuttcap%
\pgfsetroundjoin%
\definecolor{currentfill}{rgb}{0.000000,0.000000,0.000000}%
\pgfsetfillcolor{currentfill}%
\pgfsetlinewidth{0.602250pt}%
\definecolor{currentstroke}{rgb}{0.000000,0.000000,0.000000}%
\pgfsetstrokecolor{currentstroke}%
\pgfsetdash{}{0pt}%
\pgfsys@defobject{currentmarker}{\pgfqpoint{-0.027778in}{0.000000in}}{\pgfqpoint{-0.000000in}{0.000000in}}{%
\pgfpathmoveto{\pgfqpoint{-0.000000in}{0.000000in}}%
\pgfpathlineto{\pgfqpoint{-0.027778in}{0.000000in}}%
\pgfusepath{stroke,fill}%
}%
\begin{pgfscope}%
\pgfsys@transformshift{0.708220in}{0.640842in}%
\pgfsys@useobject{currentmarker}{}%
\end{pgfscope}%
\end{pgfscope}%
\begin{pgfscope}%
\pgfsetbuttcap%
\pgfsetroundjoin%
\definecolor{currentfill}{rgb}{0.000000,0.000000,0.000000}%
\pgfsetfillcolor{currentfill}%
\pgfsetlinewidth{0.602250pt}%
\definecolor{currentstroke}{rgb}{0.000000,0.000000,0.000000}%
\pgfsetstrokecolor{currentstroke}%
\pgfsetdash{}{0pt}%
\pgfsys@defobject{currentmarker}{\pgfqpoint{-0.027778in}{0.000000in}}{\pgfqpoint{-0.000000in}{0.000000in}}{%
\pgfpathmoveto{\pgfqpoint{-0.000000in}{0.000000in}}%
\pgfpathlineto{\pgfqpoint{-0.027778in}{0.000000in}}%
\pgfusepath{stroke,fill}%
}%
\begin{pgfscope}%
\pgfsys@transformshift{0.708220in}{0.783511in}%
\pgfsys@useobject{currentmarker}{}%
\end{pgfscope}%
\end{pgfscope}%
\begin{pgfscope}%
\pgfsetbuttcap%
\pgfsetroundjoin%
\definecolor{currentfill}{rgb}{0.000000,0.000000,0.000000}%
\pgfsetfillcolor{currentfill}%
\pgfsetlinewidth{0.602250pt}%
\definecolor{currentstroke}{rgb}{0.000000,0.000000,0.000000}%
\pgfsetstrokecolor{currentstroke}%
\pgfsetdash{}{0pt}%
\pgfsys@defobject{currentmarker}{\pgfqpoint{-0.027778in}{0.000000in}}{\pgfqpoint{-0.000000in}{0.000000in}}{%
\pgfpathmoveto{\pgfqpoint{-0.000000in}{0.000000in}}%
\pgfpathlineto{\pgfqpoint{-0.027778in}{0.000000in}}%
\pgfusepath{stroke,fill}%
}%
\begin{pgfscope}%
\pgfsys@transformshift{0.708220in}{0.855956in}%
\pgfsys@useobject{currentmarker}{}%
\end{pgfscope}%
\end{pgfscope}%
\begin{pgfscope}%
\pgfsetbuttcap%
\pgfsetroundjoin%
\definecolor{currentfill}{rgb}{0.000000,0.000000,0.000000}%
\pgfsetfillcolor{currentfill}%
\pgfsetlinewidth{0.602250pt}%
\definecolor{currentstroke}{rgb}{0.000000,0.000000,0.000000}%
\pgfsetstrokecolor{currentstroke}%
\pgfsetdash{}{0pt}%
\pgfsys@defobject{currentmarker}{\pgfqpoint{-0.027778in}{0.000000in}}{\pgfqpoint{-0.000000in}{0.000000in}}{%
\pgfpathmoveto{\pgfqpoint{-0.000000in}{0.000000in}}%
\pgfpathlineto{\pgfqpoint{-0.027778in}{0.000000in}}%
\pgfusepath{stroke,fill}%
}%
\begin{pgfscope}%
\pgfsys@transformshift{0.708220in}{0.907356in}%
\pgfsys@useobject{currentmarker}{}%
\end{pgfscope}%
\end{pgfscope}%
\begin{pgfscope}%
\pgfsetbuttcap%
\pgfsetroundjoin%
\definecolor{currentfill}{rgb}{0.000000,0.000000,0.000000}%
\pgfsetfillcolor{currentfill}%
\pgfsetlinewidth{0.602250pt}%
\definecolor{currentstroke}{rgb}{0.000000,0.000000,0.000000}%
\pgfsetstrokecolor{currentstroke}%
\pgfsetdash{}{0pt}%
\pgfsys@defobject{currentmarker}{\pgfqpoint{-0.027778in}{0.000000in}}{\pgfqpoint{-0.000000in}{0.000000in}}{%
\pgfpathmoveto{\pgfqpoint{-0.000000in}{0.000000in}}%
\pgfpathlineto{\pgfqpoint{-0.027778in}{0.000000in}}%
\pgfusepath{stroke,fill}%
}%
\begin{pgfscope}%
\pgfsys@transformshift{0.708220in}{0.947225in}%
\pgfsys@useobject{currentmarker}{}%
\end{pgfscope}%
\end{pgfscope}%
\begin{pgfscope}%
\pgfsetbuttcap%
\pgfsetroundjoin%
\definecolor{currentfill}{rgb}{0.000000,0.000000,0.000000}%
\pgfsetfillcolor{currentfill}%
\pgfsetlinewidth{0.602250pt}%
\definecolor{currentstroke}{rgb}{0.000000,0.000000,0.000000}%
\pgfsetstrokecolor{currentstroke}%
\pgfsetdash{}{0pt}%
\pgfsys@defobject{currentmarker}{\pgfqpoint{-0.027778in}{0.000000in}}{\pgfqpoint{-0.000000in}{0.000000in}}{%
\pgfpathmoveto{\pgfqpoint{-0.000000in}{0.000000in}}%
\pgfpathlineto{\pgfqpoint{-0.027778in}{0.000000in}}%
\pgfusepath{stroke,fill}%
}%
\begin{pgfscope}%
\pgfsys@transformshift{0.708220in}{0.979800in}%
\pgfsys@useobject{currentmarker}{}%
\end{pgfscope}%
\end{pgfscope}%
\begin{pgfscope}%
\pgfsetbuttcap%
\pgfsetroundjoin%
\definecolor{currentfill}{rgb}{0.000000,0.000000,0.000000}%
\pgfsetfillcolor{currentfill}%
\pgfsetlinewidth{0.602250pt}%
\definecolor{currentstroke}{rgb}{0.000000,0.000000,0.000000}%
\pgfsetstrokecolor{currentstroke}%
\pgfsetdash{}{0pt}%
\pgfsys@defobject{currentmarker}{\pgfqpoint{-0.027778in}{0.000000in}}{\pgfqpoint{-0.000000in}{0.000000in}}{%
\pgfpathmoveto{\pgfqpoint{-0.000000in}{0.000000in}}%
\pgfpathlineto{\pgfqpoint{-0.027778in}{0.000000in}}%
\pgfusepath{stroke,fill}%
}%
\begin{pgfscope}%
\pgfsys@transformshift{0.708220in}{1.007342in}%
\pgfsys@useobject{currentmarker}{}%
\end{pgfscope}%
\end{pgfscope}%
\begin{pgfscope}%
\pgfsetbuttcap%
\pgfsetroundjoin%
\definecolor{currentfill}{rgb}{0.000000,0.000000,0.000000}%
\pgfsetfillcolor{currentfill}%
\pgfsetlinewidth{0.602250pt}%
\definecolor{currentstroke}{rgb}{0.000000,0.000000,0.000000}%
\pgfsetstrokecolor{currentstroke}%
\pgfsetdash{}{0pt}%
\pgfsys@defobject{currentmarker}{\pgfqpoint{-0.027778in}{0.000000in}}{\pgfqpoint{-0.000000in}{0.000000in}}{%
\pgfpathmoveto{\pgfqpoint{-0.000000in}{0.000000in}}%
\pgfpathlineto{\pgfqpoint{-0.027778in}{0.000000in}}%
\pgfusepath{stroke,fill}%
}%
\begin{pgfscope}%
\pgfsys@transformshift{0.708220in}{1.031200in}%
\pgfsys@useobject{currentmarker}{}%
\end{pgfscope}%
\end{pgfscope}%
\begin{pgfscope}%
\pgfsetbuttcap%
\pgfsetroundjoin%
\definecolor{currentfill}{rgb}{0.000000,0.000000,0.000000}%
\pgfsetfillcolor{currentfill}%
\pgfsetlinewidth{0.602250pt}%
\definecolor{currentstroke}{rgb}{0.000000,0.000000,0.000000}%
\pgfsetstrokecolor{currentstroke}%
\pgfsetdash{}{0pt}%
\pgfsys@defobject{currentmarker}{\pgfqpoint{-0.027778in}{0.000000in}}{\pgfqpoint{-0.000000in}{0.000000in}}{%
\pgfpathmoveto{\pgfqpoint{-0.000000in}{0.000000in}}%
\pgfpathlineto{\pgfqpoint{-0.027778in}{0.000000in}}%
\pgfusepath{stroke,fill}%
}%
\begin{pgfscope}%
\pgfsys@transformshift{0.708220in}{1.052244in}%
\pgfsys@useobject{currentmarker}{}%
\end{pgfscope}%
\end{pgfscope}%
\begin{pgfscope}%
\pgfsetbuttcap%
\pgfsetroundjoin%
\definecolor{currentfill}{rgb}{0.000000,0.000000,0.000000}%
\pgfsetfillcolor{currentfill}%
\pgfsetlinewidth{0.602250pt}%
\definecolor{currentstroke}{rgb}{0.000000,0.000000,0.000000}%
\pgfsetstrokecolor{currentstroke}%
\pgfsetdash{}{0pt}%
\pgfsys@defobject{currentmarker}{\pgfqpoint{-0.027778in}{0.000000in}}{\pgfqpoint{-0.000000in}{0.000000in}}{%
\pgfpathmoveto{\pgfqpoint{-0.000000in}{0.000000in}}%
\pgfpathlineto{\pgfqpoint{-0.027778in}{0.000000in}}%
\pgfusepath{stroke,fill}%
}%
\begin{pgfscope}%
\pgfsys@transformshift{0.708220in}{1.194913in}%
\pgfsys@useobject{currentmarker}{}%
\end{pgfscope}%
\end{pgfscope}%
\begin{pgfscope}%
\pgfsetbuttcap%
\pgfsetroundjoin%
\definecolor{currentfill}{rgb}{0.000000,0.000000,0.000000}%
\pgfsetfillcolor{currentfill}%
\pgfsetlinewidth{0.602250pt}%
\definecolor{currentstroke}{rgb}{0.000000,0.000000,0.000000}%
\pgfsetstrokecolor{currentstroke}%
\pgfsetdash{}{0pt}%
\pgfsys@defobject{currentmarker}{\pgfqpoint{-0.027778in}{0.000000in}}{\pgfqpoint{-0.000000in}{0.000000in}}{%
\pgfpathmoveto{\pgfqpoint{-0.000000in}{0.000000in}}%
\pgfpathlineto{\pgfqpoint{-0.027778in}{0.000000in}}%
\pgfusepath{stroke,fill}%
}%
\begin{pgfscope}%
\pgfsys@transformshift{0.708220in}{1.267358in}%
\pgfsys@useobject{currentmarker}{}%
\end{pgfscope}%
\end{pgfscope}%
\begin{pgfscope}%
\pgfsetbuttcap%
\pgfsetroundjoin%
\definecolor{currentfill}{rgb}{0.000000,0.000000,0.000000}%
\pgfsetfillcolor{currentfill}%
\pgfsetlinewidth{0.602250pt}%
\definecolor{currentstroke}{rgb}{0.000000,0.000000,0.000000}%
\pgfsetstrokecolor{currentstroke}%
\pgfsetdash{}{0pt}%
\pgfsys@defobject{currentmarker}{\pgfqpoint{-0.027778in}{0.000000in}}{\pgfqpoint{-0.000000in}{0.000000in}}{%
\pgfpathmoveto{\pgfqpoint{-0.000000in}{0.000000in}}%
\pgfpathlineto{\pgfqpoint{-0.027778in}{0.000000in}}%
\pgfusepath{stroke,fill}%
}%
\begin{pgfscope}%
\pgfsys@transformshift{0.708220in}{1.318758in}%
\pgfsys@useobject{currentmarker}{}%
\end{pgfscope}%
\end{pgfscope}%
\begin{pgfscope}%
\pgfsetbuttcap%
\pgfsetroundjoin%
\definecolor{currentfill}{rgb}{0.000000,0.000000,0.000000}%
\pgfsetfillcolor{currentfill}%
\pgfsetlinewidth{0.602250pt}%
\definecolor{currentstroke}{rgb}{0.000000,0.000000,0.000000}%
\pgfsetstrokecolor{currentstroke}%
\pgfsetdash{}{0pt}%
\pgfsys@defobject{currentmarker}{\pgfqpoint{-0.027778in}{0.000000in}}{\pgfqpoint{-0.000000in}{0.000000in}}{%
\pgfpathmoveto{\pgfqpoint{-0.000000in}{0.000000in}}%
\pgfpathlineto{\pgfqpoint{-0.027778in}{0.000000in}}%
\pgfusepath{stroke,fill}%
}%
\begin{pgfscope}%
\pgfsys@transformshift{0.708220in}{1.358627in}%
\pgfsys@useobject{currentmarker}{}%
\end{pgfscope}%
\end{pgfscope}%
\begin{pgfscope}%
\pgfsetbuttcap%
\pgfsetroundjoin%
\definecolor{currentfill}{rgb}{0.000000,0.000000,0.000000}%
\pgfsetfillcolor{currentfill}%
\pgfsetlinewidth{0.602250pt}%
\definecolor{currentstroke}{rgb}{0.000000,0.000000,0.000000}%
\pgfsetstrokecolor{currentstroke}%
\pgfsetdash{}{0pt}%
\pgfsys@defobject{currentmarker}{\pgfqpoint{-0.027778in}{0.000000in}}{\pgfqpoint{-0.000000in}{0.000000in}}{%
\pgfpathmoveto{\pgfqpoint{-0.000000in}{0.000000in}}%
\pgfpathlineto{\pgfqpoint{-0.027778in}{0.000000in}}%
\pgfusepath{stroke,fill}%
}%
\begin{pgfscope}%
\pgfsys@transformshift{0.708220in}{1.391202in}%
\pgfsys@useobject{currentmarker}{}%
\end{pgfscope}%
\end{pgfscope}%
\begin{pgfscope}%
\pgfsetbuttcap%
\pgfsetroundjoin%
\definecolor{currentfill}{rgb}{0.000000,0.000000,0.000000}%
\pgfsetfillcolor{currentfill}%
\pgfsetlinewidth{0.602250pt}%
\definecolor{currentstroke}{rgb}{0.000000,0.000000,0.000000}%
\pgfsetstrokecolor{currentstroke}%
\pgfsetdash{}{0pt}%
\pgfsys@defobject{currentmarker}{\pgfqpoint{-0.027778in}{0.000000in}}{\pgfqpoint{-0.000000in}{0.000000in}}{%
\pgfpathmoveto{\pgfqpoint{-0.000000in}{0.000000in}}%
\pgfpathlineto{\pgfqpoint{-0.027778in}{0.000000in}}%
\pgfusepath{stroke,fill}%
}%
\begin{pgfscope}%
\pgfsys@transformshift{0.708220in}{1.418744in}%
\pgfsys@useobject{currentmarker}{}%
\end{pgfscope}%
\end{pgfscope}%
\begin{pgfscope}%
\pgfsetbuttcap%
\pgfsetroundjoin%
\definecolor{currentfill}{rgb}{0.000000,0.000000,0.000000}%
\pgfsetfillcolor{currentfill}%
\pgfsetlinewidth{0.602250pt}%
\definecolor{currentstroke}{rgb}{0.000000,0.000000,0.000000}%
\pgfsetstrokecolor{currentstroke}%
\pgfsetdash{}{0pt}%
\pgfsys@defobject{currentmarker}{\pgfqpoint{-0.027778in}{0.000000in}}{\pgfqpoint{-0.000000in}{0.000000in}}{%
\pgfpathmoveto{\pgfqpoint{-0.000000in}{0.000000in}}%
\pgfpathlineto{\pgfqpoint{-0.027778in}{0.000000in}}%
\pgfusepath{stroke,fill}%
}%
\begin{pgfscope}%
\pgfsys@transformshift{0.708220in}{1.442602in}%
\pgfsys@useobject{currentmarker}{}%
\end{pgfscope}%
\end{pgfscope}%
\begin{pgfscope}%
\pgfsetbuttcap%
\pgfsetroundjoin%
\definecolor{currentfill}{rgb}{0.000000,0.000000,0.000000}%
\pgfsetfillcolor{currentfill}%
\pgfsetlinewidth{0.602250pt}%
\definecolor{currentstroke}{rgb}{0.000000,0.000000,0.000000}%
\pgfsetstrokecolor{currentstroke}%
\pgfsetdash{}{0pt}%
\pgfsys@defobject{currentmarker}{\pgfqpoint{-0.027778in}{0.000000in}}{\pgfqpoint{-0.000000in}{0.000000in}}{%
\pgfpathmoveto{\pgfqpoint{-0.000000in}{0.000000in}}%
\pgfpathlineto{\pgfqpoint{-0.027778in}{0.000000in}}%
\pgfusepath{stroke,fill}%
}%
\begin{pgfscope}%
\pgfsys@transformshift{0.708220in}{1.463646in}%
\pgfsys@useobject{currentmarker}{}%
\end{pgfscope}%
\end{pgfscope}%
\begin{pgfscope}%
\pgfsetbuttcap%
\pgfsetroundjoin%
\definecolor{currentfill}{rgb}{0.000000,0.000000,0.000000}%
\pgfsetfillcolor{currentfill}%
\pgfsetlinewidth{0.602250pt}%
\definecolor{currentstroke}{rgb}{0.000000,0.000000,0.000000}%
\pgfsetstrokecolor{currentstroke}%
\pgfsetdash{}{0pt}%
\pgfsys@defobject{currentmarker}{\pgfqpoint{-0.027778in}{0.000000in}}{\pgfqpoint{-0.000000in}{0.000000in}}{%
\pgfpathmoveto{\pgfqpoint{-0.000000in}{0.000000in}}%
\pgfpathlineto{\pgfqpoint{-0.027778in}{0.000000in}}%
\pgfusepath{stroke,fill}%
}%
\begin{pgfscope}%
\pgfsys@transformshift{0.708220in}{1.606315in}%
\pgfsys@useobject{currentmarker}{}%
\end{pgfscope}%
\end{pgfscope}%
\begin{pgfscope}%
\pgfsetbuttcap%
\pgfsetroundjoin%
\definecolor{currentfill}{rgb}{0.000000,0.000000,0.000000}%
\pgfsetfillcolor{currentfill}%
\pgfsetlinewidth{0.602250pt}%
\definecolor{currentstroke}{rgb}{0.000000,0.000000,0.000000}%
\pgfsetstrokecolor{currentstroke}%
\pgfsetdash{}{0pt}%
\pgfsys@defobject{currentmarker}{\pgfqpoint{-0.027778in}{0.000000in}}{\pgfqpoint{-0.000000in}{0.000000in}}{%
\pgfpathmoveto{\pgfqpoint{-0.000000in}{0.000000in}}%
\pgfpathlineto{\pgfqpoint{-0.027778in}{0.000000in}}%
\pgfusepath{stroke,fill}%
}%
\begin{pgfscope}%
\pgfsys@transformshift{0.708220in}{1.678760in}%
\pgfsys@useobject{currentmarker}{}%
\end{pgfscope}%
\end{pgfscope}%
\begin{pgfscope}%
\pgfsetbuttcap%
\pgfsetroundjoin%
\definecolor{currentfill}{rgb}{0.000000,0.000000,0.000000}%
\pgfsetfillcolor{currentfill}%
\pgfsetlinewidth{0.602250pt}%
\definecolor{currentstroke}{rgb}{0.000000,0.000000,0.000000}%
\pgfsetstrokecolor{currentstroke}%
\pgfsetdash{}{0pt}%
\pgfsys@defobject{currentmarker}{\pgfqpoint{-0.027778in}{0.000000in}}{\pgfqpoint{-0.000000in}{0.000000in}}{%
\pgfpathmoveto{\pgfqpoint{-0.000000in}{0.000000in}}%
\pgfpathlineto{\pgfqpoint{-0.027778in}{0.000000in}}%
\pgfusepath{stroke,fill}%
}%
\begin{pgfscope}%
\pgfsys@transformshift{0.708220in}{1.730160in}%
\pgfsys@useobject{currentmarker}{}%
\end{pgfscope}%
\end{pgfscope}%
\begin{pgfscope}%
\pgfsetbuttcap%
\pgfsetroundjoin%
\definecolor{currentfill}{rgb}{0.000000,0.000000,0.000000}%
\pgfsetfillcolor{currentfill}%
\pgfsetlinewidth{0.602250pt}%
\definecolor{currentstroke}{rgb}{0.000000,0.000000,0.000000}%
\pgfsetstrokecolor{currentstroke}%
\pgfsetdash{}{0pt}%
\pgfsys@defobject{currentmarker}{\pgfqpoint{-0.027778in}{0.000000in}}{\pgfqpoint{-0.000000in}{0.000000in}}{%
\pgfpathmoveto{\pgfqpoint{-0.000000in}{0.000000in}}%
\pgfpathlineto{\pgfqpoint{-0.027778in}{0.000000in}}%
\pgfusepath{stroke,fill}%
}%
\begin{pgfscope}%
\pgfsys@transformshift{0.708220in}{1.770029in}%
\pgfsys@useobject{currentmarker}{}%
\end{pgfscope}%
\end{pgfscope}%
\begin{pgfscope}%
\pgfsetbuttcap%
\pgfsetroundjoin%
\definecolor{currentfill}{rgb}{0.000000,0.000000,0.000000}%
\pgfsetfillcolor{currentfill}%
\pgfsetlinewidth{0.602250pt}%
\definecolor{currentstroke}{rgb}{0.000000,0.000000,0.000000}%
\pgfsetstrokecolor{currentstroke}%
\pgfsetdash{}{0pt}%
\pgfsys@defobject{currentmarker}{\pgfqpoint{-0.027778in}{0.000000in}}{\pgfqpoint{-0.000000in}{0.000000in}}{%
\pgfpathmoveto{\pgfqpoint{-0.000000in}{0.000000in}}%
\pgfpathlineto{\pgfqpoint{-0.027778in}{0.000000in}}%
\pgfusepath{stroke,fill}%
}%
\begin{pgfscope}%
\pgfsys@transformshift{0.708220in}{1.802604in}%
\pgfsys@useobject{currentmarker}{}%
\end{pgfscope}%
\end{pgfscope}%
\begin{pgfscope}%
\pgfsetbuttcap%
\pgfsetroundjoin%
\definecolor{currentfill}{rgb}{0.000000,0.000000,0.000000}%
\pgfsetfillcolor{currentfill}%
\pgfsetlinewidth{0.602250pt}%
\definecolor{currentstroke}{rgb}{0.000000,0.000000,0.000000}%
\pgfsetstrokecolor{currentstroke}%
\pgfsetdash{}{0pt}%
\pgfsys@defobject{currentmarker}{\pgfqpoint{-0.027778in}{0.000000in}}{\pgfqpoint{-0.000000in}{0.000000in}}{%
\pgfpathmoveto{\pgfqpoint{-0.000000in}{0.000000in}}%
\pgfpathlineto{\pgfqpoint{-0.027778in}{0.000000in}}%
\pgfusepath{stroke,fill}%
}%
\begin{pgfscope}%
\pgfsys@transformshift{0.708220in}{1.830146in}%
\pgfsys@useobject{currentmarker}{}%
\end{pgfscope}%
\end{pgfscope}%
\begin{pgfscope}%
\pgfsetbuttcap%
\pgfsetroundjoin%
\definecolor{currentfill}{rgb}{0.000000,0.000000,0.000000}%
\pgfsetfillcolor{currentfill}%
\pgfsetlinewidth{0.602250pt}%
\definecolor{currentstroke}{rgb}{0.000000,0.000000,0.000000}%
\pgfsetstrokecolor{currentstroke}%
\pgfsetdash{}{0pt}%
\pgfsys@defobject{currentmarker}{\pgfqpoint{-0.027778in}{0.000000in}}{\pgfqpoint{-0.000000in}{0.000000in}}{%
\pgfpathmoveto{\pgfqpoint{-0.000000in}{0.000000in}}%
\pgfpathlineto{\pgfqpoint{-0.027778in}{0.000000in}}%
\pgfusepath{stroke,fill}%
}%
\begin{pgfscope}%
\pgfsys@transformshift{0.708220in}{1.854004in}%
\pgfsys@useobject{currentmarker}{}%
\end{pgfscope}%
\end{pgfscope}%
\begin{pgfscope}%
\pgfsetbuttcap%
\pgfsetroundjoin%
\definecolor{currentfill}{rgb}{0.000000,0.000000,0.000000}%
\pgfsetfillcolor{currentfill}%
\pgfsetlinewidth{0.602250pt}%
\definecolor{currentstroke}{rgb}{0.000000,0.000000,0.000000}%
\pgfsetstrokecolor{currentstroke}%
\pgfsetdash{}{0pt}%
\pgfsys@defobject{currentmarker}{\pgfqpoint{-0.027778in}{0.000000in}}{\pgfqpoint{-0.000000in}{0.000000in}}{%
\pgfpathmoveto{\pgfqpoint{-0.000000in}{0.000000in}}%
\pgfpathlineto{\pgfqpoint{-0.027778in}{0.000000in}}%
\pgfusepath{stroke,fill}%
}%
\begin{pgfscope}%
\pgfsys@transformshift{0.708220in}{1.875048in}%
\pgfsys@useobject{currentmarker}{}%
\end{pgfscope}%
\end{pgfscope}%
\begin{pgfscope}%
\pgfsetbuttcap%
\pgfsetroundjoin%
\definecolor{currentfill}{rgb}{0.000000,0.000000,0.000000}%
\pgfsetfillcolor{currentfill}%
\pgfsetlinewidth{0.602250pt}%
\definecolor{currentstroke}{rgb}{0.000000,0.000000,0.000000}%
\pgfsetstrokecolor{currentstroke}%
\pgfsetdash{}{0pt}%
\pgfsys@defobject{currentmarker}{\pgfqpoint{-0.027778in}{0.000000in}}{\pgfqpoint{-0.000000in}{0.000000in}}{%
\pgfpathmoveto{\pgfqpoint{-0.000000in}{0.000000in}}%
\pgfpathlineto{\pgfqpoint{-0.027778in}{0.000000in}}%
\pgfusepath{stroke,fill}%
}%
\begin{pgfscope}%
\pgfsys@transformshift{0.708220in}{2.017718in}%
\pgfsys@useobject{currentmarker}{}%
\end{pgfscope}%
\end{pgfscope}%
\begin{pgfscope}%
\pgfsetbuttcap%
\pgfsetroundjoin%
\definecolor{currentfill}{rgb}{0.000000,0.000000,0.000000}%
\pgfsetfillcolor{currentfill}%
\pgfsetlinewidth{0.602250pt}%
\definecolor{currentstroke}{rgb}{0.000000,0.000000,0.000000}%
\pgfsetstrokecolor{currentstroke}%
\pgfsetdash{}{0pt}%
\pgfsys@defobject{currentmarker}{\pgfqpoint{-0.027778in}{0.000000in}}{\pgfqpoint{-0.000000in}{0.000000in}}{%
\pgfpathmoveto{\pgfqpoint{-0.000000in}{0.000000in}}%
\pgfpathlineto{\pgfqpoint{-0.027778in}{0.000000in}}%
\pgfusepath{stroke,fill}%
}%
\begin{pgfscope}%
\pgfsys@transformshift{0.708220in}{2.090162in}%
\pgfsys@useobject{currentmarker}{}%
\end{pgfscope}%
\end{pgfscope}%
\begin{pgfscope}%
\pgfsetbuttcap%
\pgfsetroundjoin%
\definecolor{currentfill}{rgb}{0.000000,0.000000,0.000000}%
\pgfsetfillcolor{currentfill}%
\pgfsetlinewidth{0.602250pt}%
\definecolor{currentstroke}{rgb}{0.000000,0.000000,0.000000}%
\pgfsetstrokecolor{currentstroke}%
\pgfsetdash{}{0pt}%
\pgfsys@defobject{currentmarker}{\pgfqpoint{-0.027778in}{0.000000in}}{\pgfqpoint{-0.000000in}{0.000000in}}{%
\pgfpathmoveto{\pgfqpoint{-0.000000in}{0.000000in}}%
\pgfpathlineto{\pgfqpoint{-0.027778in}{0.000000in}}%
\pgfusepath{stroke,fill}%
}%
\begin{pgfscope}%
\pgfsys@transformshift{0.708220in}{2.141562in}%
\pgfsys@useobject{currentmarker}{}%
\end{pgfscope}%
\end{pgfscope}%
\begin{pgfscope}%
\pgfsetbuttcap%
\pgfsetroundjoin%
\definecolor{currentfill}{rgb}{0.000000,0.000000,0.000000}%
\pgfsetfillcolor{currentfill}%
\pgfsetlinewidth{0.602250pt}%
\definecolor{currentstroke}{rgb}{0.000000,0.000000,0.000000}%
\pgfsetstrokecolor{currentstroke}%
\pgfsetdash{}{0pt}%
\pgfsys@defobject{currentmarker}{\pgfqpoint{-0.027778in}{0.000000in}}{\pgfqpoint{-0.000000in}{0.000000in}}{%
\pgfpathmoveto{\pgfqpoint{-0.000000in}{0.000000in}}%
\pgfpathlineto{\pgfqpoint{-0.027778in}{0.000000in}}%
\pgfusepath{stroke,fill}%
}%
\begin{pgfscope}%
\pgfsys@transformshift{0.708220in}{2.181431in}%
\pgfsys@useobject{currentmarker}{}%
\end{pgfscope}%
\end{pgfscope}%
\begin{pgfscope}%
\pgfsetbuttcap%
\pgfsetroundjoin%
\definecolor{currentfill}{rgb}{0.000000,0.000000,0.000000}%
\pgfsetfillcolor{currentfill}%
\pgfsetlinewidth{0.602250pt}%
\definecolor{currentstroke}{rgb}{0.000000,0.000000,0.000000}%
\pgfsetstrokecolor{currentstroke}%
\pgfsetdash{}{0pt}%
\pgfsys@defobject{currentmarker}{\pgfqpoint{-0.027778in}{0.000000in}}{\pgfqpoint{-0.000000in}{0.000000in}}{%
\pgfpathmoveto{\pgfqpoint{-0.000000in}{0.000000in}}%
\pgfpathlineto{\pgfqpoint{-0.027778in}{0.000000in}}%
\pgfusepath{stroke,fill}%
}%
\begin{pgfscope}%
\pgfsys@transformshift{0.708220in}{2.214006in}%
\pgfsys@useobject{currentmarker}{}%
\end{pgfscope}%
\end{pgfscope}%
\begin{pgfscope}%
\pgfsetbuttcap%
\pgfsetroundjoin%
\definecolor{currentfill}{rgb}{0.000000,0.000000,0.000000}%
\pgfsetfillcolor{currentfill}%
\pgfsetlinewidth{0.602250pt}%
\definecolor{currentstroke}{rgb}{0.000000,0.000000,0.000000}%
\pgfsetstrokecolor{currentstroke}%
\pgfsetdash{}{0pt}%
\pgfsys@defobject{currentmarker}{\pgfqpoint{-0.027778in}{0.000000in}}{\pgfqpoint{-0.000000in}{0.000000in}}{%
\pgfpathmoveto{\pgfqpoint{-0.000000in}{0.000000in}}%
\pgfpathlineto{\pgfqpoint{-0.027778in}{0.000000in}}%
\pgfusepath{stroke,fill}%
}%
\begin{pgfscope}%
\pgfsys@transformshift{0.708220in}{2.241548in}%
\pgfsys@useobject{currentmarker}{}%
\end{pgfscope}%
\end{pgfscope}%
\begin{pgfscope}%
\pgfsetbuttcap%
\pgfsetroundjoin%
\definecolor{currentfill}{rgb}{0.000000,0.000000,0.000000}%
\pgfsetfillcolor{currentfill}%
\pgfsetlinewidth{0.602250pt}%
\definecolor{currentstroke}{rgb}{0.000000,0.000000,0.000000}%
\pgfsetstrokecolor{currentstroke}%
\pgfsetdash{}{0pt}%
\pgfsys@defobject{currentmarker}{\pgfqpoint{-0.027778in}{0.000000in}}{\pgfqpoint{-0.000000in}{0.000000in}}{%
\pgfpathmoveto{\pgfqpoint{-0.000000in}{0.000000in}}%
\pgfpathlineto{\pgfqpoint{-0.027778in}{0.000000in}}%
\pgfusepath{stroke,fill}%
}%
\begin{pgfscope}%
\pgfsys@transformshift{0.708220in}{2.265406in}%
\pgfsys@useobject{currentmarker}{}%
\end{pgfscope}%
\end{pgfscope}%
\begin{pgfscope}%
\pgfsetbuttcap%
\pgfsetroundjoin%
\definecolor{currentfill}{rgb}{0.000000,0.000000,0.000000}%
\pgfsetfillcolor{currentfill}%
\pgfsetlinewidth{0.602250pt}%
\definecolor{currentstroke}{rgb}{0.000000,0.000000,0.000000}%
\pgfsetstrokecolor{currentstroke}%
\pgfsetdash{}{0pt}%
\pgfsys@defobject{currentmarker}{\pgfqpoint{-0.027778in}{0.000000in}}{\pgfqpoint{-0.000000in}{0.000000in}}{%
\pgfpathmoveto{\pgfqpoint{-0.000000in}{0.000000in}}%
\pgfpathlineto{\pgfqpoint{-0.027778in}{0.000000in}}%
\pgfusepath{stroke,fill}%
}%
\begin{pgfscope}%
\pgfsys@transformshift{0.708220in}{2.286450in}%
\pgfsys@useobject{currentmarker}{}%
\end{pgfscope}%
\end{pgfscope}%
\begin{pgfscope}%
\definecolor{textcolor}{rgb}{0.000000,0.000000,0.000000}%
\pgfsetstrokecolor{textcolor}%
\pgfsetfillcolor{textcolor}%
\pgftext[x=0.288855in,y=1.420549in,,bottom,rotate=90.000000]{\color{textcolor}\rmfamily\fontsize{10.000000}{12.000000}\selectfont Longest solving time (s)}%
\end{pgfscope}%
\begin{pgfscope}%
\pgfpathrectangle{\pgfqpoint{0.708220in}{0.535823in}}{\pgfqpoint{5.045427in}{1.769453in}}%
\pgfusepath{clip}%
\pgfsetrectcap%
\pgfsetroundjoin%
\pgfsetlinewidth{1.003750pt}%
\definecolor{currentstroke}{rgb}{0.121569,0.466667,0.705882}%
\pgfsetstrokecolor{currentstroke}%
\pgfsetdash{}{0pt}%
\pgfpathmoveto{\pgfqpoint{0.708220in}{0.595940in}}%
\pgfpathlineto{\pgfqpoint{0.733447in}{0.595940in}}%
\pgfpathlineto{\pgfqpoint{0.746061in}{0.640842in}}%
\pgfpathlineto{\pgfqpoint{0.758674in}{0.659667in}}%
\pgfpathlineto{\pgfqpoint{0.783901in}{0.659667in}}%
\pgfpathlineto{\pgfqpoint{0.796515in}{0.676696in}}%
\pgfpathlineto{\pgfqpoint{0.846969in}{0.676696in}}%
\pgfpathlineto{\pgfqpoint{0.859583in}{0.692242in}}%
\pgfpathlineto{\pgfqpoint{0.884810in}{0.692242in}}%
\pgfpathlineto{\pgfqpoint{0.897423in}{0.706544in}}%
\pgfpathlineto{\pgfqpoint{0.910037in}{0.706544in}}%
\pgfpathlineto{\pgfqpoint{0.922650in}{0.719784in}}%
\pgfpathlineto{\pgfqpoint{0.985718in}{0.719784in}}%
\pgfpathlineto{\pgfqpoint{0.998332in}{0.732111in}}%
\pgfpathlineto{\pgfqpoint{1.010945in}{0.732111in}}%
\pgfpathlineto{\pgfqpoint{1.023559in}{0.743642in}}%
\pgfpathlineto{\pgfqpoint{1.061400in}{0.743642in}}%
\pgfpathlineto{\pgfqpoint{1.074013in}{0.764687in}}%
\pgfpathlineto{\pgfqpoint{1.111854in}{0.764687in}}%
\pgfpathlineto{\pgfqpoint{1.124468in}{0.774347in}}%
\pgfpathlineto{\pgfqpoint{1.137081in}{0.774347in}}%
\pgfpathlineto{\pgfqpoint{1.149695in}{0.783511in}}%
\pgfpathlineto{\pgfqpoint{1.200149in}{0.783511in}}%
\pgfpathlineto{\pgfqpoint{1.212763in}{0.792229in}}%
\pgfpathlineto{\pgfqpoint{1.225376in}{0.792229in}}%
\pgfpathlineto{\pgfqpoint{1.237990in}{0.800540in}}%
\pgfpathlineto{\pgfqpoint{1.301057in}{0.800540in}}%
\pgfpathlineto{\pgfqpoint{1.313671in}{0.808483in}}%
\pgfpathlineto{\pgfqpoint{1.338898in}{0.808483in}}%
\pgfpathlineto{\pgfqpoint{1.351512in}{0.816087in}}%
\pgfpathlineto{\pgfqpoint{1.376739in}{0.816087in}}%
\pgfpathlineto{\pgfqpoint{1.389352in}{0.823380in}}%
\pgfpathlineto{\pgfqpoint{1.439807in}{0.823380in}}%
\pgfpathlineto{\pgfqpoint{1.452420in}{0.830388in}}%
\pgfpathlineto{\pgfqpoint{1.490261in}{0.830388in}}%
\pgfpathlineto{\pgfqpoint{1.502875in}{0.837131in}}%
\pgfpathlineto{\pgfqpoint{1.565942in}{0.837131in}}%
\pgfpathlineto{\pgfqpoint{1.578556in}{0.843629in}}%
\pgfpathlineto{\pgfqpoint{1.629010in}{0.843629in}}%
\pgfpathlineto{\pgfqpoint{1.641624in}{0.849898in}}%
\pgfpathlineto{\pgfqpoint{1.692078in}{0.849898in}}%
\pgfpathlineto{\pgfqpoint{1.704692in}{0.855956in}}%
\pgfpathlineto{\pgfqpoint{1.729919in}{0.855956in}}%
\pgfpathlineto{\pgfqpoint{1.742532in}{0.861814in}}%
\pgfpathlineto{\pgfqpoint{1.767759in}{0.861814in}}%
\pgfpathlineto{\pgfqpoint{1.780373in}{0.867487in}}%
\pgfpathlineto{\pgfqpoint{1.830827in}{0.867487in}}%
\pgfpathlineto{\pgfqpoint{1.843441in}{0.872985in}}%
\pgfpathlineto{\pgfqpoint{1.856054in}{0.872985in}}%
\pgfpathlineto{\pgfqpoint{1.868668in}{0.878319in}}%
\pgfpathlineto{\pgfqpoint{1.893895in}{0.878319in}}%
\pgfpathlineto{\pgfqpoint{1.906509in}{0.883498in}}%
\pgfpathlineto{\pgfqpoint{1.982190in}{0.883498in}}%
\pgfpathlineto{\pgfqpoint{1.994804in}{0.888531in}}%
\pgfpathlineto{\pgfqpoint{2.045258in}{0.888531in}}%
\pgfpathlineto{\pgfqpoint{2.057871in}{0.893426in}}%
\pgfpathlineto{\pgfqpoint{2.120939in}{0.893426in}}%
\pgfpathlineto{\pgfqpoint{2.133553in}{0.902832in}}%
\pgfpathlineto{\pgfqpoint{2.146166in}{0.902832in}}%
\pgfpathlineto{\pgfqpoint{2.158780in}{0.907356in}}%
\pgfpathlineto{\pgfqpoint{2.171394in}{0.907356in}}%
\pgfpathlineto{\pgfqpoint{2.184007in}{0.911768in}}%
\pgfpathlineto{\pgfqpoint{2.209234in}{0.911768in}}%
\pgfpathlineto{\pgfqpoint{2.221848in}{0.916073in}}%
\pgfpathlineto{\pgfqpoint{2.247075in}{0.916073in}}%
\pgfpathlineto{\pgfqpoint{2.259689in}{0.920277in}}%
\pgfpathlineto{\pgfqpoint{2.297529in}{0.920277in}}%
\pgfpathlineto{\pgfqpoint{2.310143in}{0.924385in}}%
\pgfpathlineto{\pgfqpoint{2.335370in}{0.924385in}}%
\pgfpathlineto{\pgfqpoint{2.347984in}{0.932327in}}%
\pgfpathlineto{\pgfqpoint{2.360597in}{0.932327in}}%
\pgfpathlineto{\pgfqpoint{2.373211in}{0.936169in}}%
\pgfpathlineto{\pgfqpoint{2.398438in}{0.936169in}}%
\pgfpathlineto{\pgfqpoint{2.423665in}{0.943615in}}%
\pgfpathlineto{\pgfqpoint{2.448892in}{0.943615in}}%
\pgfpathlineto{\pgfqpoint{2.474119in}{0.950763in}}%
\pgfpathlineto{\pgfqpoint{2.486733in}{0.957636in}}%
\pgfpathlineto{\pgfqpoint{2.499346in}{0.957636in}}%
\pgfpathlineto{\pgfqpoint{2.511960in}{0.964254in}}%
\pgfpathlineto{\pgfqpoint{2.524573in}{0.964254in}}%
\pgfpathlineto{\pgfqpoint{2.537187in}{0.967473in}}%
\pgfpathlineto{\pgfqpoint{2.549801in}{0.967473in}}%
\pgfpathlineto{\pgfqpoint{2.562414in}{0.973743in}}%
\pgfpathlineto{\pgfqpoint{2.575028in}{0.976797in}}%
\pgfpathlineto{\pgfqpoint{2.587641in}{0.976797in}}%
\pgfpathlineto{\pgfqpoint{2.600255in}{0.979800in}}%
\pgfpathlineto{\pgfqpoint{2.612868in}{0.985659in}}%
\pgfpathlineto{\pgfqpoint{2.625482in}{0.988517in}}%
\pgfpathlineto{\pgfqpoint{2.638096in}{0.988517in}}%
\pgfpathlineto{\pgfqpoint{2.650709in}{0.991331in}}%
\pgfpathlineto{\pgfqpoint{2.663323in}{0.991331in}}%
\pgfpathlineto{\pgfqpoint{2.688550in}{0.996829in}}%
\pgfpathlineto{\pgfqpoint{2.701163in}{0.996829in}}%
\pgfpathlineto{\pgfqpoint{2.726390in}{1.002163in}}%
\pgfpathlineto{\pgfqpoint{2.739004in}{1.002163in}}%
\pgfpathlineto{\pgfqpoint{2.751618in}{1.004771in}}%
\pgfpathlineto{\pgfqpoint{2.764231in}{1.004771in}}%
\pgfpathlineto{\pgfqpoint{2.776845in}{1.012375in}}%
\pgfpathlineto{\pgfqpoint{2.789458in}{1.017271in}}%
\pgfpathlineto{\pgfqpoint{2.814685in}{1.017271in}}%
\pgfpathlineto{\pgfqpoint{2.827299in}{1.019669in}}%
\pgfpathlineto{\pgfqpoint{2.839913in}{1.028953in}}%
\pgfpathlineto{\pgfqpoint{2.852526in}{1.028953in}}%
\pgfpathlineto{\pgfqpoint{2.865140in}{1.033420in}}%
\pgfpathlineto{\pgfqpoint{2.915594in}{1.042032in}}%
\pgfpathlineto{\pgfqpoint{2.928208in}{1.042032in}}%
\pgfpathlineto{\pgfqpoint{2.940821in}{1.044122in}}%
\pgfpathlineto{\pgfqpoint{2.953435in}{1.048229in}}%
\pgfpathlineto{\pgfqpoint{2.966048in}{1.050248in}}%
\pgfpathlineto{\pgfqpoint{2.978662in}{1.050248in}}%
\pgfpathlineto{\pgfqpoint{2.991275in}{1.054219in}}%
\pgfpathlineto{\pgfqpoint{3.003889in}{1.054219in}}%
\pgfpathlineto{\pgfqpoint{3.016503in}{1.056171in}}%
\pgfpathlineto{\pgfqpoint{3.029116in}{1.071069in}}%
\pgfpathlineto{\pgfqpoint{3.041730in}{1.074607in}}%
\pgfpathlineto{\pgfqpoint{3.066957in}{1.078077in}}%
\pgfpathlineto{\pgfqpoint{3.079570in}{1.081480in}}%
\pgfpathlineto{\pgfqpoint{3.092184in}{1.088098in}}%
\pgfpathlineto{\pgfqpoint{3.104797in}{1.089715in}}%
\pgfpathlineto{\pgfqpoint{3.117411in}{1.094480in}}%
\pgfpathlineto{\pgfqpoint{3.130025in}{1.096040in}}%
\pgfpathlineto{\pgfqpoint{3.142638in}{1.099121in}}%
\pgfpathlineto{\pgfqpoint{3.155252in}{1.099121in}}%
\pgfpathlineto{\pgfqpoint{3.167865in}{1.100641in}}%
\pgfpathlineto{\pgfqpoint{3.180479in}{1.103644in}}%
\pgfpathlineto{\pgfqpoint{3.193092in}{1.110938in}}%
\pgfpathlineto{\pgfqpoint{3.205706in}{1.110938in}}%
\pgfpathlineto{\pgfqpoint{3.218320in}{1.115175in}}%
\pgfpathlineto{\pgfqpoint{3.230933in}{1.120673in}}%
\pgfpathlineto{\pgfqpoint{3.243547in}{1.128616in}}%
\pgfpathlineto{\pgfqpoint{3.256160in}{1.129906in}}%
\pgfpathlineto{\pgfqpoint{3.268774in}{1.150521in}}%
\pgfpathlineto{\pgfqpoint{3.294001in}{1.155044in}}%
\pgfpathlineto{\pgfqpoint{3.331842in}{1.158363in}}%
\pgfpathlineto{\pgfqpoint{3.357069in}{1.163762in}}%
\pgfpathlineto{\pgfqpoint{3.369682in}{1.164822in}}%
\pgfpathlineto{\pgfqpoint{3.382296in}{1.170032in}}%
\pgfpathlineto{\pgfqpoint{3.394910in}{1.182905in}}%
\pgfpathlineto{\pgfqpoint{3.407523in}{1.191304in}}%
\pgfpathlineto{\pgfqpoint{3.420137in}{1.194913in}}%
\pgfpathlineto{\pgfqpoint{3.432750in}{1.204480in}}%
\pgfpathlineto{\pgfqpoint{3.445364in}{1.210311in}}%
\pgfpathlineto{\pgfqpoint{3.457977in}{1.219106in}}%
\pgfpathlineto{\pgfqpoint{3.470591in}{1.233347in}}%
\pgfpathlineto{\pgfqpoint{3.483204in}{1.245866in}}%
\pgfpathlineto{\pgfqpoint{3.495818in}{1.255031in}}%
\pgfpathlineto{\pgfqpoint{3.533659in}{1.270896in}}%
\pgfpathlineto{\pgfqpoint{3.546272in}{1.290245in}}%
\pgfpathlineto{\pgfqpoint{3.558886in}{1.292329in}}%
\pgfpathlineto{\pgfqpoint{3.571499in}{1.313316in}}%
\pgfpathlineto{\pgfqpoint{3.584113in}{1.322296in}}%
\pgfpathlineto{\pgfqpoint{3.596727in}{1.351333in}}%
\pgfpathlineto{\pgfqpoint{3.609340in}{1.364255in}}%
\pgfpathlineto{\pgfqpoint{3.621954in}{1.367344in}}%
\pgfpathlineto{\pgfqpoint{3.634567in}{1.379830in}}%
\pgfpathlineto{\pgfqpoint{3.647181in}{1.386679in}}%
\pgfpathlineto{\pgfqpoint{3.659794in}{1.406326in}}%
\pgfpathlineto{\pgfqpoint{3.672408in}{1.411981in}}%
\pgfpathlineto{\pgfqpoint{3.685022in}{1.476660in}}%
\pgfpathlineto{\pgfqpoint{3.697635in}{1.480856in}}%
\pgfpathlineto{\pgfqpoint{3.710249in}{1.481396in}}%
\pgfpathlineto{\pgfqpoint{3.722862in}{1.538588in}}%
\pgfpathlineto{\pgfqpoint{3.735476in}{1.568445in}}%
\pgfpathlineto{\pgfqpoint{3.760703in}{1.576752in}}%
\pgfpathlineto{\pgfqpoint{3.773317in}{1.578431in}}%
\pgfpathlineto{\pgfqpoint{3.785930in}{1.660925in}}%
\pgfpathlineto{\pgfqpoint{3.798544in}{1.726914in}}%
\pgfpathlineto{\pgfqpoint{3.811157in}{1.740149in}}%
\pgfpathlineto{\pgfqpoint{3.823771in}{1.764660in}}%
\pgfpathlineto{\pgfqpoint{3.836384in}{1.765762in}}%
\pgfpathlineto{\pgfqpoint{3.848998in}{1.812706in}}%
\pgfpathlineto{\pgfqpoint{3.861611in}{1.821866in}}%
\pgfpathlineto{\pgfqpoint{3.874225in}{1.832605in}}%
\pgfpathlineto{\pgfqpoint{3.886839in}{1.835055in}}%
\pgfpathlineto{\pgfqpoint{3.899452in}{1.839567in}}%
\pgfpathlineto{\pgfqpoint{3.912066in}{1.903879in}}%
\pgfpathlineto{\pgfqpoint{3.924679in}{1.949493in}}%
\pgfpathlineto{\pgfqpoint{3.937293in}{2.000228in}}%
\pgfpathlineto{\pgfqpoint{3.949906in}{2.075865in}}%
\pgfpathlineto{\pgfqpoint{3.962520in}{2.144363in}}%
\pgfpathlineto{\pgfqpoint{3.975134in}{2.162860in}}%
\pgfpathlineto{\pgfqpoint{3.987747in}{2.184418in}}%
\pgfpathlineto{\pgfqpoint{4.000361in}{2.232674in}}%
\pgfpathlineto{\pgfqpoint{4.012974in}{2.260702in}}%
\pgfpathlineto{\pgfqpoint{4.025588in}{2.265533in}}%
\pgfpathlineto{\pgfqpoint{4.025588in}{2.305275in}}%
\pgfpathlineto{\pgfqpoint{4.025588in}{2.305275in}}%
\pgfusepath{stroke}%
\end{pgfscope}%
\begin{pgfscope}%
\pgfpathrectangle{\pgfqpoint{0.708220in}{0.535823in}}{\pgfqpoint{5.045427in}{1.769453in}}%
\pgfusepath{clip}%
\pgfsetrectcap%
\pgfsetroundjoin%
\pgfsetlinewidth{1.003750pt}%
\definecolor{currentstroke}{rgb}{1.000000,0.498039,0.054902}%
\pgfsetstrokecolor{currentstroke}%
\pgfsetdash{}{0pt}%
\pgfpathmoveto{\pgfqpoint{0.793351in}{0.525823in}}%
\pgfpathlineto{\pgfqpoint{0.796515in}{0.535823in}}%
\pgfpathlineto{\pgfqpoint{0.809128in}{0.535823in}}%
\pgfpathlineto{\pgfqpoint{0.821742in}{0.595940in}}%
\pgfpathlineto{\pgfqpoint{0.834356in}{0.619798in}}%
\pgfpathlineto{\pgfqpoint{0.859583in}{0.619798in}}%
\pgfpathlineto{\pgfqpoint{0.872196in}{0.640842in}}%
\pgfpathlineto{\pgfqpoint{0.884810in}{0.659667in}}%
\pgfpathlineto{\pgfqpoint{0.897423in}{0.659667in}}%
\pgfpathlineto{\pgfqpoint{0.910037in}{0.676696in}}%
\pgfpathlineto{\pgfqpoint{0.973105in}{0.676696in}}%
\pgfpathlineto{\pgfqpoint{0.985718in}{0.692242in}}%
\pgfpathlineto{\pgfqpoint{0.998332in}{0.692242in}}%
\pgfpathlineto{\pgfqpoint{1.010945in}{0.706544in}}%
\pgfpathlineto{\pgfqpoint{1.023559in}{0.706544in}}%
\pgfpathlineto{\pgfqpoint{1.036173in}{0.719784in}}%
\pgfpathlineto{\pgfqpoint{1.111854in}{0.719784in}}%
\pgfpathlineto{\pgfqpoint{1.124468in}{0.732111in}}%
\pgfpathlineto{\pgfqpoint{1.149695in}{0.732111in}}%
\pgfpathlineto{\pgfqpoint{1.162308in}{0.743642in}}%
\pgfpathlineto{\pgfqpoint{1.225376in}{0.743642in}}%
\pgfpathlineto{\pgfqpoint{1.237990in}{0.754474in}}%
\pgfpathlineto{\pgfqpoint{1.288444in}{0.754474in}}%
\pgfpathlineto{\pgfqpoint{1.301057in}{0.764687in}}%
\pgfpathlineto{\pgfqpoint{1.326285in}{0.764687in}}%
\pgfpathlineto{\pgfqpoint{1.338898in}{0.774347in}}%
\pgfpathlineto{\pgfqpoint{1.401966in}{0.774347in}}%
\pgfpathlineto{\pgfqpoint{1.427193in}{0.792229in}}%
\pgfpathlineto{\pgfqpoint{1.528102in}{0.792229in}}%
\pgfpathlineto{\pgfqpoint{1.540715in}{0.800540in}}%
\pgfpathlineto{\pgfqpoint{1.553329in}{0.800540in}}%
\pgfpathlineto{\pgfqpoint{1.565942in}{0.808483in}}%
\pgfpathlineto{\pgfqpoint{1.578556in}{0.808483in}}%
\pgfpathlineto{\pgfqpoint{1.591170in}{0.816087in}}%
\pgfpathlineto{\pgfqpoint{1.654237in}{0.816087in}}%
\pgfpathlineto{\pgfqpoint{1.666851in}{0.823380in}}%
\pgfpathlineto{\pgfqpoint{1.679464in}{0.823380in}}%
\pgfpathlineto{\pgfqpoint{1.692078in}{0.830388in}}%
\pgfpathlineto{\pgfqpoint{1.704692in}{0.830388in}}%
\pgfpathlineto{\pgfqpoint{1.717305in}{0.837131in}}%
\pgfpathlineto{\pgfqpoint{1.755146in}{0.837131in}}%
\pgfpathlineto{\pgfqpoint{1.767759in}{0.843629in}}%
\pgfpathlineto{\pgfqpoint{1.818214in}{0.843629in}}%
\pgfpathlineto{\pgfqpoint{1.830827in}{0.849898in}}%
\pgfpathlineto{\pgfqpoint{1.868668in}{0.849898in}}%
\pgfpathlineto{\pgfqpoint{1.881282in}{0.855956in}}%
\pgfpathlineto{\pgfqpoint{1.906509in}{0.855956in}}%
\pgfpathlineto{\pgfqpoint{1.919122in}{0.861814in}}%
\pgfpathlineto{\pgfqpoint{1.956963in}{0.861814in}}%
\pgfpathlineto{\pgfqpoint{1.969577in}{0.867487in}}%
\pgfpathlineto{\pgfqpoint{2.020031in}{0.867487in}}%
\pgfpathlineto{\pgfqpoint{2.032644in}{0.872985in}}%
\pgfpathlineto{\pgfqpoint{2.070485in}{0.872985in}}%
\pgfpathlineto{\pgfqpoint{2.083099in}{0.878319in}}%
\pgfpathlineto{\pgfqpoint{2.095712in}{0.878319in}}%
\pgfpathlineto{\pgfqpoint{2.108326in}{0.893426in}}%
\pgfpathlineto{\pgfqpoint{2.146166in}{0.893426in}}%
\pgfpathlineto{\pgfqpoint{2.158780in}{0.902832in}}%
\pgfpathlineto{\pgfqpoint{2.171394in}{0.902832in}}%
\pgfpathlineto{\pgfqpoint{2.184007in}{0.907356in}}%
\pgfpathlineto{\pgfqpoint{2.209234in}{0.907356in}}%
\pgfpathlineto{\pgfqpoint{2.221848in}{0.911768in}}%
\pgfpathlineto{\pgfqpoint{2.247075in}{0.911768in}}%
\pgfpathlineto{\pgfqpoint{2.259689in}{0.916073in}}%
\pgfpathlineto{\pgfqpoint{2.272302in}{0.916073in}}%
\pgfpathlineto{\pgfqpoint{2.335370in}{0.936169in}}%
\pgfpathlineto{\pgfqpoint{2.347984in}{0.936169in}}%
\pgfpathlineto{\pgfqpoint{2.360597in}{0.943615in}}%
\pgfpathlineto{\pgfqpoint{2.436278in}{0.964254in}}%
\pgfpathlineto{\pgfqpoint{2.448892in}{0.967473in}}%
\pgfpathlineto{\pgfqpoint{2.461506in}{0.973743in}}%
\pgfpathlineto{\pgfqpoint{2.474119in}{0.985659in}}%
\pgfpathlineto{\pgfqpoint{2.499346in}{0.991331in}}%
\pgfpathlineto{\pgfqpoint{2.511960in}{0.991331in}}%
\pgfpathlineto{\pgfqpoint{2.537187in}{0.996829in}}%
\pgfpathlineto{\pgfqpoint{2.562414in}{0.996829in}}%
\pgfpathlineto{\pgfqpoint{2.612868in}{1.007342in}}%
\pgfpathlineto{\pgfqpoint{2.625482in}{1.012375in}}%
\pgfpathlineto{\pgfqpoint{2.650709in}{1.012375in}}%
\pgfpathlineto{\pgfqpoint{2.675936in}{1.022036in}}%
\pgfpathlineto{\pgfqpoint{2.688550in}{1.022036in}}%
\pgfpathlineto{\pgfqpoint{2.701163in}{1.024371in}}%
\pgfpathlineto{\pgfqpoint{2.713777in}{1.028953in}}%
\pgfpathlineto{\pgfqpoint{2.726390in}{1.031200in}}%
\pgfpathlineto{\pgfqpoint{2.739004in}{1.037778in}}%
\pgfpathlineto{\pgfqpoint{2.764231in}{1.037778in}}%
\pgfpathlineto{\pgfqpoint{2.776845in}{1.039917in}}%
\pgfpathlineto{\pgfqpoint{2.789458in}{1.044122in}}%
\pgfpathlineto{\pgfqpoint{2.802072in}{1.046187in}}%
\pgfpathlineto{\pgfqpoint{2.814685in}{1.046187in}}%
\pgfpathlineto{\pgfqpoint{2.827299in}{1.052244in}}%
\pgfpathlineto{\pgfqpoint{2.852526in}{1.056171in}}%
\pgfpathlineto{\pgfqpoint{2.865140in}{1.056171in}}%
\pgfpathlineto{\pgfqpoint{2.877753in}{1.058103in}}%
\pgfpathlineto{\pgfqpoint{2.902980in}{1.058103in}}%
\pgfpathlineto{\pgfqpoint{2.928208in}{1.061904in}}%
\pgfpathlineto{\pgfqpoint{2.940821in}{1.069273in}}%
\pgfpathlineto{\pgfqpoint{2.966048in}{1.072847in}}%
\pgfpathlineto{\pgfqpoint{2.991275in}{1.072847in}}%
\pgfpathlineto{\pgfqpoint{3.003889in}{1.074607in}}%
\pgfpathlineto{\pgfqpoint{3.016503in}{1.074607in}}%
\pgfpathlineto{\pgfqpoint{3.029116in}{1.079786in}}%
\pgfpathlineto{\pgfqpoint{3.041730in}{1.088098in}}%
\pgfpathlineto{\pgfqpoint{3.054343in}{1.089715in}}%
\pgfpathlineto{\pgfqpoint{3.066957in}{1.103644in}}%
\pgfpathlineto{\pgfqpoint{3.079570in}{1.108056in}}%
\pgfpathlineto{\pgfqpoint{3.092184in}{1.110938in}}%
\pgfpathlineto{\pgfqpoint{3.104797in}{1.122022in}}%
\pgfpathlineto{\pgfqpoint{3.117411in}{1.123360in}}%
\pgfpathlineto{\pgfqpoint{3.130025in}{1.143513in}}%
\pgfpathlineto{\pgfqpoint{3.142638in}{1.143513in}}%
\pgfpathlineto{\pgfqpoint{3.155252in}{1.144701in}}%
\pgfpathlineto{\pgfqpoint{3.167865in}{1.152797in}}%
\pgfpathlineto{\pgfqpoint{3.180479in}{1.155044in}}%
\pgfpathlineto{\pgfqpoint{3.193092in}{1.156158in}}%
\pgfpathlineto{\pgfqpoint{3.205706in}{1.177079in}}%
\pgfpathlineto{\pgfqpoint{3.230933in}{1.179042in}}%
\pgfpathlineto{\pgfqpoint{3.243547in}{1.185749in}}%
\pgfpathlineto{\pgfqpoint{3.268774in}{1.194018in}}%
\pgfpathlineto{\pgfqpoint{3.281387in}{1.201921in}}%
\pgfpathlineto{\pgfqpoint{3.294001in}{1.215162in}}%
\pgfpathlineto{\pgfqpoint{3.306615in}{1.215958in}}%
\pgfpathlineto{\pgfqpoint{3.319228in}{1.219106in}}%
\pgfpathlineto{\pgfqpoint{3.331842in}{1.220660in}}%
\pgfpathlineto{\pgfqpoint{3.344455in}{1.225241in}}%
\pgfpathlineto{\pgfqpoint{3.357069in}{1.236206in}}%
\pgfpathlineto{\pgfqpoint{3.369682in}{1.236206in}}%
\pgfpathlineto{\pgfqpoint{3.382296in}{1.239716in}}%
\pgfpathlineto{\pgfqpoint{3.420137in}{1.241790in}}%
\pgfpathlineto{\pgfqpoint{3.432750in}{1.247205in}}%
\pgfpathlineto{\pgfqpoint{3.445364in}{1.249852in}}%
\pgfpathlineto{\pgfqpoint{3.457977in}{1.255668in}}%
\pgfpathlineto{\pgfqpoint{3.470591in}{1.258193in}}%
\pgfpathlineto{\pgfqpoint{3.483204in}{1.261916in}}%
\pgfpathlineto{\pgfqpoint{3.495818in}{1.291290in}}%
\pgfpathlineto{\pgfqpoint{3.508432in}{1.292846in}}%
\pgfpathlineto{\pgfqpoint{3.521045in}{1.313316in}}%
\pgfpathlineto{\pgfqpoint{3.533659in}{1.319204in}}%
\pgfpathlineto{\pgfqpoint{3.546272in}{1.321418in}}%
\pgfpathlineto{\pgfqpoint{3.558886in}{1.330428in}}%
\pgfpathlineto{\pgfqpoint{3.571499in}{1.342560in}}%
\pgfpathlineto{\pgfqpoint{3.584113in}{1.351333in}}%
\pgfpathlineto{\pgfqpoint{3.596727in}{1.393569in}}%
\pgfpathlineto{\pgfqpoint{3.609340in}{1.449180in}}%
\pgfpathlineto{\pgfqpoint{3.634567in}{1.460643in}}%
\pgfpathlineto{\pgfqpoint{3.647181in}{1.515344in}}%
\pgfpathlineto{\pgfqpoint{3.659794in}{1.526018in}}%
\pgfpathlineto{\pgfqpoint{3.672408in}{1.611423in}}%
\pgfpathlineto{\pgfqpoint{3.685022in}{1.615966in}}%
\pgfpathlineto{\pgfqpoint{3.697635in}{1.616811in}}%
\pgfpathlineto{\pgfqpoint{3.710249in}{1.671652in}}%
\pgfpathlineto{\pgfqpoint{3.722862in}{1.696491in}}%
\pgfpathlineto{\pgfqpoint{3.735476in}{1.701017in}}%
\pgfpathlineto{\pgfqpoint{3.748089in}{1.710689in}}%
\pgfpathlineto{\pgfqpoint{3.760703in}{1.742332in}}%
\pgfpathlineto{\pgfqpoint{3.773317in}{1.785786in}}%
\pgfpathlineto{\pgfqpoint{3.785930in}{1.789958in}}%
\pgfpathlineto{\pgfqpoint{3.798544in}{1.859350in}}%
\pgfpathlineto{\pgfqpoint{3.823771in}{1.859762in}}%
\pgfpathlineto{\pgfqpoint{3.836384in}{2.012551in}}%
\pgfpathlineto{\pgfqpoint{3.848998in}{2.033737in}}%
\pgfpathlineto{\pgfqpoint{3.861611in}{2.048181in}}%
\pgfpathlineto{\pgfqpoint{3.874225in}{2.058286in}}%
\pgfpathlineto{\pgfqpoint{3.886839in}{2.060760in}}%
\pgfpathlineto{\pgfqpoint{3.899452in}{2.067532in}}%
\pgfpathlineto{\pgfqpoint{3.912066in}{2.078548in}}%
\pgfpathlineto{\pgfqpoint{3.924679in}{2.106350in}}%
\pgfpathlineto{\pgfqpoint{3.937293in}{2.126280in}}%
\pgfpathlineto{\pgfqpoint{3.949906in}{2.134180in}}%
\pgfpathlineto{\pgfqpoint{3.962520in}{2.144182in}}%
\pgfpathlineto{\pgfqpoint{3.975134in}{2.149815in}}%
\pgfpathlineto{\pgfqpoint{3.987747in}{2.191298in}}%
\pgfpathlineto{\pgfqpoint{4.000361in}{2.196539in}}%
\pgfpathlineto{\pgfqpoint{4.012974in}{2.198431in}}%
\pgfpathlineto{\pgfqpoint{4.012974in}{2.305275in}}%
\pgfpathlineto{\pgfqpoint{4.012974in}{2.305275in}}%
\pgfusepath{stroke}%
\end{pgfscope}%
\begin{pgfscope}%
\pgfpathrectangle{\pgfqpoint{0.708220in}{0.535823in}}{\pgfqpoint{5.045427in}{1.769453in}}%
\pgfusepath{clip}%
\pgfsetbuttcap%
\pgfsetroundjoin%
\pgfsetlinewidth{1.003750pt}%
\definecolor{currentstroke}{rgb}{0.172549,0.627451,0.172549}%
\pgfsetstrokecolor{currentstroke}%
\pgfsetdash{{3.700000pt}{1.600000pt}}{0.000000pt}%
\pgfpathmoveto{\pgfqpoint{0.843805in}{0.525823in}}%
\pgfpathlineto{\pgfqpoint{0.846969in}{0.535823in}}%
\pgfpathlineto{\pgfqpoint{0.859583in}{0.568398in}}%
\pgfpathlineto{\pgfqpoint{0.872196in}{0.568398in}}%
\pgfpathlineto{\pgfqpoint{0.884810in}{0.595940in}}%
\pgfpathlineto{\pgfqpoint{0.897423in}{0.619798in}}%
\pgfpathlineto{\pgfqpoint{0.910037in}{0.676696in}}%
\pgfpathlineto{\pgfqpoint{0.935264in}{0.676696in}}%
\pgfpathlineto{\pgfqpoint{0.960491in}{0.706544in}}%
\pgfpathlineto{\pgfqpoint{0.973105in}{0.706544in}}%
\pgfpathlineto{\pgfqpoint{0.998332in}{0.732111in}}%
\pgfpathlineto{\pgfqpoint{1.010945in}{0.774347in}}%
\pgfpathlineto{\pgfqpoint{1.074013in}{0.774347in}}%
\pgfpathlineto{\pgfqpoint{1.099240in}{0.792229in}}%
\pgfpathlineto{\pgfqpoint{1.111854in}{0.816087in}}%
\pgfpathlineto{\pgfqpoint{1.124468in}{0.823380in}}%
\pgfpathlineto{\pgfqpoint{1.137081in}{0.849898in}}%
\pgfpathlineto{\pgfqpoint{1.149695in}{0.883498in}}%
\pgfpathlineto{\pgfqpoint{1.162308in}{0.902832in}}%
\pgfpathlineto{\pgfqpoint{1.174922in}{0.902832in}}%
\pgfpathlineto{\pgfqpoint{1.187535in}{0.911768in}}%
\pgfpathlineto{\pgfqpoint{1.200149in}{0.916073in}}%
\pgfpathlineto{\pgfqpoint{1.212763in}{0.947225in}}%
\pgfpathlineto{\pgfqpoint{1.225376in}{0.954232in}}%
\pgfpathlineto{\pgfqpoint{1.237990in}{0.973743in}}%
\pgfpathlineto{\pgfqpoint{1.250603in}{1.014840in}}%
\pgfpathlineto{\pgfqpoint{1.263217in}{1.017271in}}%
\pgfpathlineto{\pgfqpoint{1.275830in}{1.017271in}}%
\pgfpathlineto{\pgfqpoint{1.288444in}{1.046187in}}%
\pgfpathlineto{\pgfqpoint{1.301057in}{1.048229in}}%
\pgfpathlineto{\pgfqpoint{1.313671in}{1.060014in}}%
\pgfpathlineto{\pgfqpoint{1.326285in}{1.061904in}}%
\pgfpathlineto{\pgfqpoint{1.338898in}{1.067459in}}%
\pgfpathlineto{\pgfqpoint{1.351512in}{1.067459in}}%
\pgfpathlineto{\pgfqpoint{1.364125in}{1.086466in}}%
\pgfpathlineto{\pgfqpoint{1.376739in}{1.113774in}}%
\pgfpathlineto{\pgfqpoint{1.389352in}{1.122022in}}%
\pgfpathlineto{\pgfqpoint{1.401966in}{1.148215in}}%
\pgfpathlineto{\pgfqpoint{1.414580in}{1.157264in}}%
\pgfpathlineto{\pgfqpoint{1.427193in}{1.211128in}}%
\pgfpathlineto{\pgfqpoint{1.439807in}{1.218324in}}%
\pgfpathlineto{\pgfqpoint{1.452420in}{1.220660in}}%
\pgfpathlineto{\pgfqpoint{1.465034in}{1.235496in}}%
\pgfpathlineto{\pgfqpoint{1.490261in}{1.236206in}}%
\pgfpathlineto{\pgfqpoint{1.502875in}{1.245866in}}%
\pgfpathlineto{\pgfqpoint{1.528102in}{1.255031in}}%
\pgfpathlineto{\pgfqpoint{1.540715in}{1.258193in}}%
\pgfpathlineto{\pgfqpoint{1.553329in}{1.259443in}}%
\pgfpathlineto{\pgfqpoint{1.565942in}{1.277769in}}%
\pgfpathlineto{\pgfqpoint{1.591170in}{1.280002in}}%
\pgfpathlineto{\pgfqpoint{1.629010in}{1.282755in}}%
\pgfpathlineto{\pgfqpoint{1.654237in}{1.283845in}}%
\pgfpathlineto{\pgfqpoint{1.666851in}{1.293362in}}%
\pgfpathlineto{\pgfqpoint{1.679464in}{1.295918in}}%
\pgfpathlineto{\pgfqpoint{1.692078in}{1.295918in}}%
\pgfpathlineto{\pgfqpoint{1.704692in}{1.299436in}}%
\pgfpathlineto{\pgfqpoint{1.717305in}{1.299436in}}%
\pgfpathlineto{\pgfqpoint{1.729919in}{1.304345in}}%
\pgfpathlineto{\pgfqpoint{1.742532in}{1.327049in}}%
\pgfpathlineto{\pgfqpoint{1.755146in}{1.359340in}}%
\pgfpathlineto{\pgfqpoint{1.767759in}{1.359696in}}%
\pgfpathlineto{\pgfqpoint{1.780373in}{1.364946in}}%
\pgfpathlineto{\pgfqpoint{1.792987in}{1.385453in}}%
\pgfpathlineto{\pgfqpoint{1.805600in}{1.389105in}}%
\pgfpathlineto{\pgfqpoint{1.818214in}{1.396194in}}%
\pgfpathlineto{\pgfqpoint{1.830827in}{1.410384in}}%
\pgfpathlineto{\pgfqpoint{1.843441in}{1.419254in}}%
\pgfpathlineto{\pgfqpoint{1.856054in}{1.421530in}}%
\pgfpathlineto{\pgfqpoint{1.868668in}{1.426242in}}%
\pgfpathlineto{\pgfqpoint{1.881282in}{1.426486in}}%
\pgfpathlineto{\pgfqpoint{1.906509in}{1.432495in}}%
\pgfpathlineto{\pgfqpoint{1.919122in}{1.436699in}}%
\pgfpathlineto{\pgfqpoint{1.944349in}{1.440355in}}%
\pgfpathlineto{\pgfqpoint{1.994804in}{1.441482in}}%
\pgfpathlineto{\pgfqpoint{2.007417in}{1.447883in}}%
\pgfpathlineto{\pgfqpoint{2.045258in}{1.450039in}}%
\pgfpathlineto{\pgfqpoint{2.057871in}{1.452802in}}%
\pgfpathlineto{\pgfqpoint{2.083099in}{1.454691in}}%
\pgfpathlineto{\pgfqpoint{2.095712in}{1.458204in}}%
\pgfpathlineto{\pgfqpoint{2.108326in}{1.463050in}}%
\pgfpathlineto{\pgfqpoint{2.146166in}{1.466404in}}%
\pgfpathlineto{\pgfqpoint{2.158780in}{1.470272in}}%
\pgfpathlineto{\pgfqpoint{2.171394in}{1.471985in}}%
\pgfpathlineto{\pgfqpoint{2.234461in}{1.474431in}}%
\pgfpathlineto{\pgfqpoint{2.247075in}{1.477397in}}%
\pgfpathlineto{\pgfqpoint{2.272302in}{1.479408in}}%
\pgfpathlineto{\pgfqpoint{2.284916in}{1.488272in}}%
\pgfpathlineto{\pgfqpoint{2.360597in}{1.491018in}}%
\pgfpathlineto{\pgfqpoint{2.373211in}{1.493219in}}%
\pgfpathlineto{\pgfqpoint{2.398438in}{1.494393in}}%
\pgfpathlineto{\pgfqpoint{2.411051in}{1.512950in}}%
\pgfpathlineto{\pgfqpoint{2.423665in}{1.524047in}}%
\pgfpathlineto{\pgfqpoint{2.436278in}{1.525176in}}%
\pgfpathlineto{\pgfqpoint{2.448892in}{1.536223in}}%
\pgfpathlineto{\pgfqpoint{2.461506in}{1.536355in}}%
\pgfpathlineto{\pgfqpoint{2.486733in}{1.540018in}}%
\pgfpathlineto{\pgfqpoint{2.499346in}{1.553960in}}%
\pgfpathlineto{\pgfqpoint{2.511960in}{1.554080in}}%
\pgfpathlineto{\pgfqpoint{2.524573in}{1.555629in}}%
\pgfpathlineto{\pgfqpoint{2.587641in}{1.556929in}}%
\pgfpathlineto{\pgfqpoint{2.600255in}{1.572269in}}%
\pgfpathlineto{\pgfqpoint{2.612868in}{1.578117in}}%
\pgfpathlineto{\pgfqpoint{2.625482in}{1.580404in}}%
\pgfpathlineto{\pgfqpoint{2.650709in}{1.581331in}}%
\pgfpathlineto{\pgfqpoint{2.663323in}{1.582763in}}%
\pgfpathlineto{\pgfqpoint{2.675936in}{1.604790in}}%
\pgfpathlineto{\pgfqpoint{2.701163in}{1.626723in}}%
\pgfpathlineto{\pgfqpoint{2.713777in}{1.647395in}}%
\pgfpathlineto{\pgfqpoint{2.726390in}{1.651466in}}%
\pgfpathlineto{\pgfqpoint{2.764231in}{1.652710in}}%
\pgfpathlineto{\pgfqpoint{2.776845in}{1.661844in}}%
\pgfpathlineto{\pgfqpoint{2.814685in}{1.662563in}}%
\pgfpathlineto{\pgfqpoint{2.827299in}{1.662628in}}%
\pgfpathlineto{\pgfqpoint{2.839913in}{1.664895in}}%
\pgfpathlineto{\pgfqpoint{2.852526in}{1.677505in}}%
\pgfpathlineto{\pgfqpoint{2.865140in}{1.678462in}}%
\pgfpathlineto{\pgfqpoint{2.877753in}{1.684157in}}%
\pgfpathlineto{\pgfqpoint{2.890367in}{1.696976in}}%
\pgfpathlineto{\pgfqpoint{2.902980in}{1.697674in}}%
\pgfpathlineto{\pgfqpoint{2.915594in}{1.715019in}}%
\pgfpathlineto{\pgfqpoint{2.928208in}{1.715795in}}%
\pgfpathlineto{\pgfqpoint{2.940821in}{1.718962in}}%
\pgfpathlineto{\pgfqpoint{2.953435in}{1.719200in}}%
\pgfpathlineto{\pgfqpoint{2.966048in}{1.721886in}}%
\pgfpathlineto{\pgfqpoint{2.978662in}{1.722214in}}%
\pgfpathlineto{\pgfqpoint{2.991275in}{1.737081in}}%
\pgfpathlineto{\pgfqpoint{3.003889in}{1.741076in}}%
\pgfpathlineto{\pgfqpoint{3.029116in}{1.742957in}}%
\pgfpathlineto{\pgfqpoint{3.054343in}{1.744653in}}%
\pgfpathlineto{\pgfqpoint{3.066957in}{1.791549in}}%
\pgfpathlineto{\pgfqpoint{3.079570in}{1.795217in}}%
\pgfpathlineto{\pgfqpoint{3.092184in}{1.807335in}}%
\pgfpathlineto{\pgfqpoint{3.155252in}{1.810440in}}%
\pgfpathlineto{\pgfqpoint{3.167865in}{1.814581in}}%
\pgfpathlineto{\pgfqpoint{3.193092in}{1.815498in}}%
\pgfpathlineto{\pgfqpoint{3.218320in}{1.816658in}}%
\pgfpathlineto{\pgfqpoint{3.230933in}{1.823171in}}%
\pgfpathlineto{\pgfqpoint{3.243547in}{1.824283in}}%
\pgfpathlineto{\pgfqpoint{3.256160in}{1.828763in}}%
\pgfpathlineto{\pgfqpoint{3.268774in}{1.828865in}}%
\pgfpathlineto{\pgfqpoint{3.294001in}{1.832781in}}%
\pgfpathlineto{\pgfqpoint{3.306615in}{1.833409in}}%
\pgfpathlineto{\pgfqpoint{3.319228in}{1.840605in}}%
\pgfpathlineto{\pgfqpoint{3.331842in}{1.856752in}}%
\pgfpathlineto{\pgfqpoint{3.344455in}{1.857411in}}%
\pgfpathlineto{\pgfqpoint{3.357069in}{1.881118in}}%
\pgfpathlineto{\pgfqpoint{3.369682in}{1.881655in}}%
\pgfpathlineto{\pgfqpoint{3.382296in}{1.883974in}}%
\pgfpathlineto{\pgfqpoint{3.394910in}{1.884539in}}%
\pgfpathlineto{\pgfqpoint{3.407523in}{1.887118in}}%
\pgfpathlineto{\pgfqpoint{3.420137in}{1.887711in}}%
\pgfpathlineto{\pgfqpoint{3.432750in}{1.892510in}}%
\pgfpathlineto{\pgfqpoint{3.445364in}{1.893229in}}%
\pgfpathlineto{\pgfqpoint{3.457977in}{1.903541in}}%
\pgfpathlineto{\pgfqpoint{3.470591in}{1.905443in}}%
\pgfpathlineto{\pgfqpoint{3.483204in}{1.917284in}}%
\pgfpathlineto{\pgfqpoint{3.495818in}{1.950055in}}%
\pgfpathlineto{\pgfqpoint{3.508432in}{1.950537in}}%
\pgfpathlineto{\pgfqpoint{3.521045in}{1.952388in}}%
\pgfpathlineto{\pgfqpoint{3.533659in}{1.952491in}}%
\pgfpathlineto{\pgfqpoint{3.546272in}{1.978172in}}%
\pgfpathlineto{\pgfqpoint{3.558886in}{1.978751in}}%
\pgfpathlineto{\pgfqpoint{3.571499in}{1.994023in}}%
\pgfpathlineto{\pgfqpoint{3.596727in}{1.994705in}}%
\pgfpathlineto{\pgfqpoint{3.634567in}{1.998047in}}%
\pgfpathlineto{\pgfqpoint{3.647181in}{2.000405in}}%
\pgfpathlineto{\pgfqpoint{3.659794in}{2.001014in}}%
\pgfpathlineto{\pgfqpoint{3.672408in}{2.010694in}}%
\pgfpathlineto{\pgfqpoint{3.697635in}{2.012487in}}%
\pgfpathlineto{\pgfqpoint{3.722862in}{2.013460in}}%
\pgfpathlineto{\pgfqpoint{3.748089in}{2.019672in}}%
\pgfpathlineto{\pgfqpoint{3.760703in}{2.032664in}}%
\pgfpathlineto{\pgfqpoint{3.773317in}{2.047872in}}%
\pgfpathlineto{\pgfqpoint{3.785930in}{2.053356in}}%
\pgfpathlineto{\pgfqpoint{3.798544in}{2.054908in}}%
\pgfpathlineto{\pgfqpoint{3.811157in}{2.080665in}}%
\pgfpathlineto{\pgfqpoint{3.823771in}{2.102111in}}%
\pgfpathlineto{\pgfqpoint{3.836384in}{2.105788in}}%
\pgfpathlineto{\pgfqpoint{3.848998in}{2.105996in}}%
\pgfpathlineto{\pgfqpoint{3.861611in}{2.107980in}}%
\pgfpathlineto{\pgfqpoint{3.874225in}{2.112414in}}%
\pgfpathlineto{\pgfqpoint{3.886839in}{2.127430in}}%
\pgfpathlineto{\pgfqpoint{3.899452in}{2.128965in}}%
\pgfpathlineto{\pgfqpoint{3.912066in}{2.160558in}}%
\pgfpathlineto{\pgfqpoint{3.924679in}{2.167444in}}%
\pgfpathlineto{\pgfqpoint{3.937293in}{2.179888in}}%
\pgfpathlineto{\pgfqpoint{3.949906in}{2.180071in}}%
\pgfpathlineto{\pgfqpoint{3.962520in}{2.181827in}}%
\pgfpathlineto{\pgfqpoint{3.975134in}{2.185312in}}%
\pgfpathlineto{\pgfqpoint{3.987747in}{2.186132in}}%
\pgfpathlineto{\pgfqpoint{4.000361in}{2.191427in}}%
\pgfpathlineto{\pgfqpoint{4.012974in}{2.223730in}}%
\pgfpathlineto{\pgfqpoint{4.025588in}{2.227387in}}%
\pgfpathlineto{\pgfqpoint{4.038201in}{2.227519in}}%
\pgfpathlineto{\pgfqpoint{4.050815in}{2.248561in}}%
\pgfpathlineto{\pgfqpoint{4.050815in}{2.305275in}}%
\pgfpathlineto{\pgfqpoint{4.050815in}{2.305275in}}%
\pgfusepath{stroke}%
\end{pgfscope}%
\begin{pgfscope}%
\pgfpathrectangle{\pgfqpoint{0.708220in}{0.535823in}}{\pgfqpoint{5.045427in}{1.769453in}}%
\pgfusepath{clip}%
\pgfsetbuttcap%
\pgfsetroundjoin%
\pgfsetlinewidth{1.003750pt}%
\definecolor{currentstroke}{rgb}{0.839216,0.152941,0.156863}%
\pgfsetstrokecolor{currentstroke}%
\pgfsetdash{{3.700000pt}{1.600000pt}}{0.000000pt}%
\pgfpathmoveto{\pgfqpoint{0.932100in}{0.525823in}}%
\pgfpathlineto{\pgfqpoint{0.935264in}{0.535823in}}%
\pgfpathlineto{\pgfqpoint{0.985718in}{0.535823in}}%
\pgfpathlineto{\pgfqpoint{0.998332in}{0.568398in}}%
\pgfpathlineto{\pgfqpoint{1.023559in}{0.568398in}}%
\pgfpathlineto{\pgfqpoint{1.036173in}{0.595940in}}%
\pgfpathlineto{\pgfqpoint{1.086627in}{0.595940in}}%
\pgfpathlineto{\pgfqpoint{1.099240in}{0.619798in}}%
\pgfpathlineto{\pgfqpoint{1.111854in}{0.640842in}}%
\pgfpathlineto{\pgfqpoint{1.149695in}{0.640842in}}%
\pgfpathlineto{\pgfqpoint{1.162308in}{0.692242in}}%
\pgfpathlineto{\pgfqpoint{1.187535in}{0.692242in}}%
\pgfpathlineto{\pgfqpoint{1.200149in}{0.719784in}}%
\pgfpathlineto{\pgfqpoint{1.212763in}{0.719784in}}%
\pgfpathlineto{\pgfqpoint{1.237990in}{0.743642in}}%
\pgfpathlineto{\pgfqpoint{1.250603in}{0.764687in}}%
\pgfpathlineto{\pgfqpoint{1.263217in}{0.764687in}}%
\pgfpathlineto{\pgfqpoint{1.275830in}{0.774347in}}%
\pgfpathlineto{\pgfqpoint{1.301057in}{0.774347in}}%
\pgfpathlineto{\pgfqpoint{1.326285in}{0.808483in}}%
\pgfpathlineto{\pgfqpoint{1.364125in}{0.808483in}}%
\pgfpathlineto{\pgfqpoint{1.376739in}{0.816087in}}%
\pgfpathlineto{\pgfqpoint{1.401966in}{0.816087in}}%
\pgfpathlineto{\pgfqpoint{1.414580in}{0.823380in}}%
\pgfpathlineto{\pgfqpoint{1.427193in}{0.823380in}}%
\pgfpathlineto{\pgfqpoint{1.439807in}{0.837131in}}%
\pgfpathlineto{\pgfqpoint{1.452420in}{0.843629in}}%
\pgfpathlineto{\pgfqpoint{1.465034in}{0.861814in}}%
\pgfpathlineto{\pgfqpoint{1.477647in}{0.861814in}}%
\pgfpathlineto{\pgfqpoint{1.490261in}{0.878319in}}%
\pgfpathlineto{\pgfqpoint{1.502875in}{0.883498in}}%
\pgfpathlineto{\pgfqpoint{1.515488in}{0.883498in}}%
\pgfpathlineto{\pgfqpoint{1.528102in}{0.888531in}}%
\pgfpathlineto{\pgfqpoint{1.540715in}{0.888531in}}%
\pgfpathlineto{\pgfqpoint{1.565942in}{0.898191in}}%
\pgfpathlineto{\pgfqpoint{1.591170in}{0.898191in}}%
\pgfpathlineto{\pgfqpoint{1.616397in}{0.907356in}}%
\pgfpathlineto{\pgfqpoint{1.629010in}{0.907356in}}%
\pgfpathlineto{\pgfqpoint{1.641624in}{0.911768in}}%
\pgfpathlineto{\pgfqpoint{1.679464in}{0.911768in}}%
\pgfpathlineto{\pgfqpoint{1.717305in}{0.924385in}}%
\pgfpathlineto{\pgfqpoint{1.755146in}{0.924385in}}%
\pgfpathlineto{\pgfqpoint{1.780373in}{0.932327in}}%
\pgfpathlineto{\pgfqpoint{1.805600in}{0.947225in}}%
\pgfpathlineto{\pgfqpoint{1.818214in}{0.947225in}}%
\pgfpathlineto{\pgfqpoint{1.843441in}{0.954232in}}%
\pgfpathlineto{\pgfqpoint{1.856054in}{0.954232in}}%
\pgfpathlineto{\pgfqpoint{1.868668in}{0.957636in}}%
\pgfpathlineto{\pgfqpoint{1.881282in}{0.973743in}}%
\pgfpathlineto{\pgfqpoint{1.893895in}{0.973743in}}%
\pgfpathlineto{\pgfqpoint{1.906509in}{0.976797in}}%
\pgfpathlineto{\pgfqpoint{1.919122in}{0.982753in}}%
\pgfpathlineto{\pgfqpoint{1.931736in}{0.985659in}}%
\pgfpathlineto{\pgfqpoint{1.956963in}{0.985659in}}%
\pgfpathlineto{\pgfqpoint{1.982190in}{1.012375in}}%
\pgfpathlineto{\pgfqpoint{1.994804in}{1.019669in}}%
\pgfpathlineto{\pgfqpoint{2.007417in}{1.019669in}}%
\pgfpathlineto{\pgfqpoint{2.020031in}{1.028953in}}%
\pgfpathlineto{\pgfqpoint{2.032644in}{1.028953in}}%
\pgfpathlineto{\pgfqpoint{2.045258in}{1.031200in}}%
\pgfpathlineto{\pgfqpoint{2.057871in}{1.035612in}}%
\pgfpathlineto{\pgfqpoint{2.070485in}{1.037778in}}%
\pgfpathlineto{\pgfqpoint{2.083099in}{1.037778in}}%
\pgfpathlineto{\pgfqpoint{2.095712in}{1.044122in}}%
\pgfpathlineto{\pgfqpoint{2.120939in}{1.044122in}}%
\pgfpathlineto{\pgfqpoint{2.133553in}{1.052244in}}%
\pgfpathlineto{\pgfqpoint{2.146166in}{1.054219in}}%
\pgfpathlineto{\pgfqpoint{2.158780in}{1.054219in}}%
\pgfpathlineto{\pgfqpoint{2.171394in}{1.056171in}}%
\pgfpathlineto{\pgfqpoint{2.184007in}{1.056171in}}%
\pgfpathlineto{\pgfqpoint{2.196621in}{1.065627in}}%
\pgfpathlineto{\pgfqpoint{2.209234in}{1.065627in}}%
\pgfpathlineto{\pgfqpoint{2.247075in}{1.076350in}}%
\pgfpathlineto{\pgfqpoint{2.272302in}{1.076350in}}%
\pgfpathlineto{\pgfqpoint{2.284916in}{1.078077in}}%
\pgfpathlineto{\pgfqpoint{2.297529in}{1.078077in}}%
\pgfpathlineto{\pgfqpoint{2.310143in}{1.106598in}}%
\pgfpathlineto{\pgfqpoint{2.322756in}{1.108056in}}%
\pgfpathlineto{\pgfqpoint{2.335370in}{1.141115in}}%
\pgfpathlineto{\pgfqpoint{2.347984in}{1.145880in}}%
\pgfpathlineto{\pgfqpoint{2.360597in}{1.149372in}}%
\pgfpathlineto{\pgfqpoint{2.373211in}{1.150521in}}%
\pgfpathlineto{\pgfqpoint{2.385824in}{1.152797in}}%
\pgfpathlineto{\pgfqpoint{2.398438in}{1.152797in}}%
\pgfpathlineto{\pgfqpoint{2.411051in}{1.155044in}}%
\pgfpathlineto{\pgfqpoint{2.423665in}{1.159456in}}%
\pgfpathlineto{\pgfqpoint{2.436278in}{1.182905in}}%
\pgfpathlineto{\pgfqpoint{2.448892in}{1.191304in}}%
\pgfpathlineto{\pgfqpoint{2.474119in}{1.195805in}}%
\pgfpathlineto{\pgfqpoint{2.486733in}{1.201921in}}%
\pgfpathlineto{\pgfqpoint{2.499346in}{1.202778in}}%
\pgfpathlineto{\pgfqpoint{2.511960in}{1.215162in}}%
\pgfpathlineto{\pgfqpoint{2.524573in}{1.221432in}}%
\pgfpathlineto{\pgfqpoint{2.537187in}{1.225994in}}%
\pgfpathlineto{\pgfqpoint{2.549801in}{1.243159in}}%
\pgfpathlineto{\pgfqpoint{2.562414in}{1.244518in}}%
\pgfpathlineto{\pgfqpoint{2.575028in}{1.244518in}}%
\pgfpathlineto{\pgfqpoint{2.587641in}{1.257565in}}%
\pgfpathlineto{\pgfqpoint{2.600255in}{1.260683in}}%
\pgfpathlineto{\pgfqpoint{2.612868in}{1.265562in}}%
\pgfpathlineto{\pgfqpoint{2.638096in}{1.267358in}}%
\pgfpathlineto{\pgfqpoint{2.650709in}{1.273216in}}%
\pgfpathlineto{\pgfqpoint{2.663323in}{1.276641in}}%
\pgfpathlineto{\pgfqpoint{2.675936in}{1.277206in}}%
\pgfpathlineto{\pgfqpoint{2.701163in}{1.288137in}}%
\pgfpathlineto{\pgfqpoint{2.713777in}{1.290245in}}%
\pgfpathlineto{\pgfqpoint{2.751618in}{1.292329in}}%
\pgfpathlineto{\pgfqpoint{2.764231in}{1.305792in}}%
\pgfpathlineto{\pgfqpoint{2.776845in}{1.309593in}}%
\pgfpathlineto{\pgfqpoint{2.789458in}{1.310063in}}%
\pgfpathlineto{\pgfqpoint{2.802072in}{1.311929in}}%
\pgfpathlineto{\pgfqpoint{2.814685in}{1.319649in}}%
\pgfpathlineto{\pgfqpoint{2.827299in}{1.330428in}}%
\pgfpathlineto{\pgfqpoint{2.839913in}{1.336192in}}%
\pgfpathlineto{\pgfqpoint{2.852526in}{1.348330in}}%
\pgfpathlineto{\pgfqpoint{2.865140in}{1.362165in}}%
\pgfpathlineto{\pgfqpoint{2.902980in}{1.374679in}}%
\pgfpathlineto{\pgfqpoint{2.915594in}{1.383908in}}%
\pgfpathlineto{\pgfqpoint{2.928208in}{1.386679in}}%
\pgfpathlineto{\pgfqpoint{2.940821in}{1.416173in}}%
\pgfpathlineto{\pgfqpoint{2.953435in}{1.426731in}}%
\pgfpathlineto{\pgfqpoint{2.991275in}{1.428431in}}%
\pgfpathlineto{\pgfqpoint{3.003889in}{1.430833in}}%
\pgfpathlineto{\pgfqpoint{3.029116in}{1.431071in}}%
\pgfpathlineto{\pgfqpoint{3.054343in}{1.435773in}}%
\pgfpathlineto{\pgfqpoint{3.104797in}{1.439675in}}%
\pgfpathlineto{\pgfqpoint{3.117411in}{1.443715in}}%
\pgfpathlineto{\pgfqpoint{3.142638in}{1.449180in}}%
\pgfpathlineto{\pgfqpoint{3.155252in}{1.450039in}}%
\pgfpathlineto{\pgfqpoint{3.167865in}{1.458409in}}%
\pgfpathlineto{\pgfqpoint{3.180479in}{1.474805in}}%
\pgfpathlineto{\pgfqpoint{3.193092in}{1.481216in}}%
\pgfpathlineto{\pgfqpoint{3.205706in}{1.485483in}}%
\pgfpathlineto{\pgfqpoint{3.218320in}{1.494225in}}%
\pgfpathlineto{\pgfqpoint{3.230933in}{1.513551in}}%
\pgfpathlineto{\pgfqpoint{3.243547in}{1.528106in}}%
\pgfpathlineto{\pgfqpoint{3.256160in}{1.557869in}}%
\pgfpathlineto{\pgfqpoint{3.268774in}{1.557986in}}%
\pgfpathlineto{\pgfqpoint{3.281387in}{1.567338in}}%
\pgfpathlineto{\pgfqpoint{3.294001in}{1.579679in}}%
\pgfpathlineto{\pgfqpoint{3.306615in}{1.646755in}}%
\pgfpathlineto{\pgfqpoint{3.319228in}{1.650073in}}%
\pgfpathlineto{\pgfqpoint{3.331842in}{1.651118in}}%
\pgfpathlineto{\pgfqpoint{3.344455in}{1.655987in}}%
\pgfpathlineto{\pgfqpoint{3.357069in}{1.663991in}}%
\pgfpathlineto{\pgfqpoint{3.369682in}{1.680538in}}%
\pgfpathlineto{\pgfqpoint{3.382296in}{1.693884in}}%
\pgfpathlineto{\pgfqpoint{3.394910in}{1.698795in}}%
\pgfpathlineto{\pgfqpoint{3.407523in}{1.705893in}}%
\pgfpathlineto{\pgfqpoint{3.420137in}{1.706608in}}%
\pgfpathlineto{\pgfqpoint{3.432750in}{1.724626in}}%
\pgfpathlineto{\pgfqpoint{3.445364in}{1.725361in}}%
\pgfpathlineto{\pgfqpoint{3.457977in}{1.727459in}}%
\pgfpathlineto{\pgfqpoint{3.470591in}{1.733084in}}%
\pgfpathlineto{\pgfqpoint{3.483204in}{1.762661in}}%
\pgfpathlineto{\pgfqpoint{3.508432in}{1.776003in}}%
\pgfpathlineto{\pgfqpoint{3.521045in}{1.803791in}}%
\pgfpathlineto{\pgfqpoint{3.533659in}{1.810782in}}%
\pgfpathlineto{\pgfqpoint{3.546272in}{1.812874in}}%
\pgfpathlineto{\pgfqpoint{3.558886in}{1.816658in}}%
\pgfpathlineto{\pgfqpoint{3.571499in}{1.825125in}}%
\pgfpathlineto{\pgfqpoint{3.584113in}{1.858808in}}%
\pgfpathlineto{\pgfqpoint{3.596727in}{1.926419in}}%
\pgfpathlineto{\pgfqpoint{3.621954in}{1.929534in}}%
\pgfpathlineto{\pgfqpoint{3.634567in}{1.982293in}}%
\pgfpathlineto{\pgfqpoint{3.647181in}{1.987985in}}%
\pgfpathlineto{\pgfqpoint{3.659794in}{1.992363in}}%
\pgfpathlineto{\pgfqpoint{3.672408in}{2.074030in}}%
\pgfpathlineto{\pgfqpoint{3.685022in}{2.084012in}}%
\pgfpathlineto{\pgfqpoint{3.697635in}{2.213613in}}%
\pgfpathlineto{\pgfqpoint{3.710249in}{2.216599in}}%
\pgfpathlineto{\pgfqpoint{3.722862in}{2.226919in}}%
\pgfpathlineto{\pgfqpoint{3.735476in}{2.279161in}}%
\pgfpathlineto{\pgfqpoint{3.748089in}{2.279165in}}%
\pgfpathlineto{\pgfqpoint{3.760703in}{2.284203in}}%
\pgfpathlineto{\pgfqpoint{3.785930in}{2.303116in}}%
\pgfpathlineto{\pgfqpoint{3.785930in}{2.305275in}}%
\pgfpathlineto{\pgfqpoint{3.785930in}{2.305275in}}%
\pgfusepath{stroke}%
\end{pgfscope}%
\begin{pgfscope}%
\pgfpathrectangle{\pgfqpoint{0.708220in}{0.535823in}}{\pgfqpoint{5.045427in}{1.769453in}}%
\pgfusepath{clip}%
\pgfsetbuttcap%
\pgfsetroundjoin%
\pgfsetlinewidth{1.003750pt}%
\definecolor{currentstroke}{rgb}{0.580392,0.403922,0.741176}%
\pgfsetstrokecolor{currentstroke}%
\pgfsetdash{{1.000000pt}{1.650000pt}}{0.000000pt}%
\pgfpathmoveto{\pgfqpoint{0.718098in}{0.525823in}}%
\pgfpathlineto{\pgfqpoint{0.720833in}{0.568398in}}%
\pgfpathlineto{\pgfqpoint{0.733447in}{0.568398in}}%
\pgfpathlineto{\pgfqpoint{0.746061in}{0.619798in}}%
\pgfpathlineto{\pgfqpoint{0.758674in}{0.640842in}}%
\pgfpathlineto{\pgfqpoint{0.771288in}{0.659667in}}%
\pgfpathlineto{\pgfqpoint{0.783901in}{0.659667in}}%
\pgfpathlineto{\pgfqpoint{0.809128in}{0.692242in}}%
\pgfpathlineto{\pgfqpoint{0.821742in}{0.692242in}}%
\pgfpathlineto{\pgfqpoint{0.834356in}{0.706544in}}%
\pgfpathlineto{\pgfqpoint{0.846969in}{0.706544in}}%
\pgfpathlineto{\pgfqpoint{0.859583in}{0.719784in}}%
\pgfpathlineto{\pgfqpoint{0.872196in}{0.719784in}}%
\pgfpathlineto{\pgfqpoint{0.884810in}{0.732111in}}%
\pgfpathlineto{\pgfqpoint{0.910037in}{0.732111in}}%
\pgfpathlineto{\pgfqpoint{0.935264in}{0.754474in}}%
\pgfpathlineto{\pgfqpoint{0.947878in}{0.754474in}}%
\pgfpathlineto{\pgfqpoint{0.960491in}{0.764687in}}%
\pgfpathlineto{\pgfqpoint{1.010945in}{0.764687in}}%
\pgfpathlineto{\pgfqpoint{1.023559in}{0.774347in}}%
\pgfpathlineto{\pgfqpoint{1.061400in}{0.774347in}}%
\pgfpathlineto{\pgfqpoint{1.074013in}{0.783511in}}%
\pgfpathlineto{\pgfqpoint{1.086627in}{0.783511in}}%
\pgfpathlineto{\pgfqpoint{1.111854in}{0.800540in}}%
\pgfpathlineto{\pgfqpoint{1.137081in}{0.800540in}}%
\pgfpathlineto{\pgfqpoint{1.149695in}{0.808483in}}%
\pgfpathlineto{\pgfqpoint{1.187535in}{0.808483in}}%
\pgfpathlineto{\pgfqpoint{1.200149in}{0.816087in}}%
\pgfpathlineto{\pgfqpoint{1.250603in}{0.816087in}}%
\pgfpathlineto{\pgfqpoint{1.263217in}{0.823380in}}%
\pgfpathlineto{\pgfqpoint{1.338898in}{0.823380in}}%
\pgfpathlineto{\pgfqpoint{1.351512in}{0.830388in}}%
\pgfpathlineto{\pgfqpoint{1.414580in}{0.830388in}}%
\pgfpathlineto{\pgfqpoint{1.427193in}{0.837131in}}%
\pgfpathlineto{\pgfqpoint{1.452420in}{0.837131in}}%
\pgfpathlineto{\pgfqpoint{1.465034in}{0.843629in}}%
\pgfpathlineto{\pgfqpoint{1.540715in}{0.843629in}}%
\pgfpathlineto{\pgfqpoint{1.553329in}{0.849898in}}%
\pgfpathlineto{\pgfqpoint{1.654237in}{0.849898in}}%
\pgfpathlineto{\pgfqpoint{1.666851in}{0.855956in}}%
\pgfpathlineto{\pgfqpoint{1.692078in}{0.855956in}}%
\pgfpathlineto{\pgfqpoint{1.704692in}{0.861814in}}%
\pgfpathlineto{\pgfqpoint{1.755146in}{0.861814in}}%
\pgfpathlineto{\pgfqpoint{1.767759in}{0.867487in}}%
\pgfpathlineto{\pgfqpoint{1.780373in}{0.867487in}}%
\pgfpathlineto{\pgfqpoint{1.792987in}{0.872985in}}%
\pgfpathlineto{\pgfqpoint{1.843441in}{0.872985in}}%
\pgfpathlineto{\pgfqpoint{1.856054in}{0.878319in}}%
\pgfpathlineto{\pgfqpoint{1.893895in}{0.878319in}}%
\pgfpathlineto{\pgfqpoint{1.906509in}{0.883498in}}%
\pgfpathlineto{\pgfqpoint{1.944349in}{0.883498in}}%
\pgfpathlineto{\pgfqpoint{1.956963in}{0.888531in}}%
\pgfpathlineto{\pgfqpoint{1.982190in}{0.888531in}}%
\pgfpathlineto{\pgfqpoint{1.994804in}{0.893426in}}%
\pgfpathlineto{\pgfqpoint{2.020031in}{0.893426in}}%
\pgfpathlineto{\pgfqpoint{2.032644in}{0.898191in}}%
\pgfpathlineto{\pgfqpoint{2.070485in}{0.898191in}}%
\pgfpathlineto{\pgfqpoint{2.083099in}{0.902832in}}%
\pgfpathlineto{\pgfqpoint{2.095712in}{0.902832in}}%
\pgfpathlineto{\pgfqpoint{2.120939in}{0.911768in}}%
\pgfpathlineto{\pgfqpoint{2.158780in}{0.911768in}}%
\pgfpathlineto{\pgfqpoint{2.184007in}{0.920277in}}%
\pgfpathlineto{\pgfqpoint{2.221848in}{0.920277in}}%
\pgfpathlineto{\pgfqpoint{2.234461in}{0.924385in}}%
\pgfpathlineto{\pgfqpoint{2.272302in}{0.924385in}}%
\pgfpathlineto{\pgfqpoint{2.284916in}{0.928400in}}%
\pgfpathlineto{\pgfqpoint{2.297529in}{0.928400in}}%
\pgfpathlineto{\pgfqpoint{2.310143in}{0.932327in}}%
\pgfpathlineto{\pgfqpoint{2.322756in}{0.932327in}}%
\pgfpathlineto{\pgfqpoint{2.335370in}{0.943615in}}%
\pgfpathlineto{\pgfqpoint{2.347984in}{0.950763in}}%
\pgfpathlineto{\pgfqpoint{2.360597in}{0.954232in}}%
\pgfpathlineto{\pgfqpoint{2.411051in}{0.954232in}}%
\pgfpathlineto{\pgfqpoint{2.423665in}{0.957636in}}%
\pgfpathlineto{\pgfqpoint{2.448892in}{0.957636in}}%
\pgfpathlineto{\pgfqpoint{2.474119in}{0.964254in}}%
\pgfpathlineto{\pgfqpoint{2.486733in}{0.964254in}}%
\pgfpathlineto{\pgfqpoint{2.499346in}{0.967473in}}%
\pgfpathlineto{\pgfqpoint{2.511960in}{0.967473in}}%
\pgfpathlineto{\pgfqpoint{2.524573in}{0.970635in}}%
\pgfpathlineto{\pgfqpoint{2.537187in}{0.970635in}}%
\pgfpathlineto{\pgfqpoint{2.562414in}{0.982753in}}%
\pgfpathlineto{\pgfqpoint{2.587641in}{0.982753in}}%
\pgfpathlineto{\pgfqpoint{2.600255in}{0.991331in}}%
\pgfpathlineto{\pgfqpoint{2.612868in}{0.994101in}}%
\pgfpathlineto{\pgfqpoint{2.625482in}{0.994101in}}%
\pgfpathlineto{\pgfqpoint{2.638096in}{0.996829in}}%
\pgfpathlineto{\pgfqpoint{2.650709in}{0.996829in}}%
\pgfpathlineto{\pgfqpoint{2.675936in}{1.002163in}}%
\pgfpathlineto{\pgfqpoint{2.701163in}{1.012375in}}%
\pgfpathlineto{\pgfqpoint{2.713777in}{1.014840in}}%
\pgfpathlineto{\pgfqpoint{2.726390in}{1.014840in}}%
\pgfpathlineto{\pgfqpoint{2.751618in}{1.019669in}}%
\pgfpathlineto{\pgfqpoint{2.776845in}{1.019669in}}%
\pgfpathlineto{\pgfqpoint{2.789458in}{1.022036in}}%
\pgfpathlineto{\pgfqpoint{2.814685in}{1.022036in}}%
\pgfpathlineto{\pgfqpoint{2.827299in}{1.024371in}}%
\pgfpathlineto{\pgfqpoint{2.839913in}{1.028953in}}%
\pgfpathlineto{\pgfqpoint{2.852526in}{1.031200in}}%
\pgfpathlineto{\pgfqpoint{2.877753in}{1.031200in}}%
\pgfpathlineto{\pgfqpoint{2.890367in}{1.035612in}}%
\pgfpathlineto{\pgfqpoint{2.902980in}{1.037778in}}%
\pgfpathlineto{\pgfqpoint{2.915594in}{1.037778in}}%
\pgfpathlineto{\pgfqpoint{2.940821in}{1.042032in}}%
\pgfpathlineto{\pgfqpoint{2.953435in}{1.048229in}}%
\pgfpathlineto{\pgfqpoint{2.991275in}{1.060014in}}%
\pgfpathlineto{\pgfqpoint{3.003889in}{1.061904in}}%
\pgfpathlineto{\pgfqpoint{3.016503in}{1.069273in}}%
\pgfpathlineto{\pgfqpoint{3.041730in}{1.076350in}}%
\pgfpathlineto{\pgfqpoint{3.054343in}{1.078077in}}%
\pgfpathlineto{\pgfqpoint{3.066957in}{1.084820in}}%
\pgfpathlineto{\pgfqpoint{3.079570in}{1.086466in}}%
\pgfpathlineto{\pgfqpoint{3.092184in}{1.091317in}}%
\pgfpathlineto{\pgfqpoint{3.104797in}{1.099121in}}%
\pgfpathlineto{\pgfqpoint{3.117411in}{1.102149in}}%
\pgfpathlineto{\pgfqpoint{3.130025in}{1.102149in}}%
\pgfpathlineto{\pgfqpoint{3.142638in}{1.103644in}}%
\pgfpathlineto{\pgfqpoint{3.167865in}{1.109503in}}%
\pgfpathlineto{\pgfqpoint{3.180479in}{1.110938in}}%
\pgfpathlineto{\pgfqpoint{3.193092in}{1.122022in}}%
\pgfpathlineto{\pgfqpoint{3.205706in}{1.131186in}}%
\pgfpathlineto{\pgfqpoint{3.218320in}{1.143513in}}%
\pgfpathlineto{\pgfqpoint{3.243547in}{1.148215in}}%
\pgfpathlineto{\pgfqpoint{3.256160in}{1.153924in}}%
\pgfpathlineto{\pgfqpoint{3.281387in}{1.157264in}}%
\pgfpathlineto{\pgfqpoint{3.294001in}{1.173086in}}%
\pgfpathlineto{\pgfqpoint{3.306615in}{1.178063in}}%
\pgfpathlineto{\pgfqpoint{3.319228in}{1.180984in}}%
\pgfpathlineto{\pgfqpoint{3.331842in}{1.186687in}}%
\pgfpathlineto{\pgfqpoint{3.357069in}{1.189471in}}%
\pgfpathlineto{\pgfqpoint{3.369682in}{1.193118in}}%
\pgfpathlineto{\pgfqpoint{3.394910in}{1.209489in}}%
\pgfpathlineto{\pgfqpoint{3.407523in}{1.219885in}}%
\pgfpathlineto{\pgfqpoint{3.420137in}{1.220660in}}%
\pgfpathlineto{\pgfqpoint{3.432750in}{1.227489in}}%
\pgfpathlineto{\pgfqpoint{3.445364in}{1.236914in}}%
\pgfpathlineto{\pgfqpoint{3.470591in}{1.244518in}}%
\pgfpathlineto{\pgfqpoint{3.495818in}{1.251811in}}%
\pgfpathlineto{\pgfqpoint{3.508432in}{1.267358in}}%
\pgfpathlineto{\pgfqpoint{3.521045in}{1.281108in}}%
\pgfpathlineto{\pgfqpoint{3.533659in}{1.297434in}}%
\pgfpathlineto{\pgfqpoint{3.546272in}{1.308650in}}%
\pgfpathlineto{\pgfqpoint{3.558886in}{1.315148in}}%
\pgfpathlineto{\pgfqpoint{3.571499in}{1.326194in}}%
\pgfpathlineto{\pgfqpoint{3.584113in}{1.362515in}}%
\pgfpathlineto{\pgfqpoint{3.609340in}{1.383598in}}%
\pgfpathlineto{\pgfqpoint{3.621954in}{1.383598in}}%
\pgfpathlineto{\pgfqpoint{3.634567in}{1.420522in}}%
\pgfpathlineto{\pgfqpoint{3.647181in}{1.476476in}}%
\pgfpathlineto{\pgfqpoint{3.659794in}{1.537803in}}%
\pgfpathlineto{\pgfqpoint{3.672408in}{1.550575in}}%
\pgfpathlineto{\pgfqpoint{3.685022in}{1.554796in}}%
\pgfpathlineto{\pgfqpoint{3.697635in}{1.572269in}}%
\pgfpathlineto{\pgfqpoint{3.735476in}{1.575907in}}%
\pgfpathlineto{\pgfqpoint{3.748089in}{1.654902in}}%
\pgfpathlineto{\pgfqpoint{3.760703in}{1.677565in}}%
\pgfpathlineto{\pgfqpoint{3.773317in}{1.697137in}}%
\pgfpathlineto{\pgfqpoint{3.785930in}{1.764033in}}%
\pgfpathlineto{\pgfqpoint{3.798544in}{1.773917in}}%
\pgfpathlineto{\pgfqpoint{3.811157in}{1.777893in}}%
\pgfpathlineto{\pgfqpoint{3.823771in}{1.790086in}}%
\pgfpathlineto{\pgfqpoint{3.836384in}{1.790309in}}%
\pgfpathlineto{\pgfqpoint{3.861611in}{1.795806in}}%
\pgfpathlineto{\pgfqpoint{3.874225in}{1.803317in}}%
\pgfpathlineto{\pgfqpoint{3.886839in}{1.805118in}}%
\pgfpathlineto{\pgfqpoint{3.899452in}{1.845426in}}%
\pgfpathlineto{\pgfqpoint{3.912066in}{1.858721in}}%
\pgfpathlineto{\pgfqpoint{3.924679in}{1.943762in}}%
\pgfpathlineto{\pgfqpoint{3.937293in}{1.956323in}}%
\pgfpathlineto{\pgfqpoint{3.949906in}{2.020940in}}%
\pgfpathlineto{\pgfqpoint{3.975134in}{2.134491in}}%
\pgfpathlineto{\pgfqpoint{3.987747in}{2.140055in}}%
\pgfpathlineto{\pgfqpoint{4.000361in}{2.182709in}}%
\pgfpathlineto{\pgfqpoint{4.012974in}{2.229957in}}%
\pgfpathlineto{\pgfqpoint{4.025588in}{2.254180in}}%
\pgfpathlineto{\pgfqpoint{4.038201in}{2.270577in}}%
\pgfpathlineto{\pgfqpoint{4.038201in}{2.305275in}}%
\pgfpathlineto{\pgfqpoint{4.038201in}{2.305275in}}%
\pgfusepath{stroke}%
\end{pgfscope}%
\begin{pgfscope}%
\pgfpathrectangle{\pgfqpoint{0.708220in}{0.535823in}}{\pgfqpoint{5.045427in}{1.769453in}}%
\pgfusepath{clip}%
\pgfsetbuttcap%
\pgfsetroundjoin%
\pgfsetlinewidth{1.003750pt}%
\definecolor{currentstroke}{rgb}{0.549020,0.337255,0.294118}%
\pgfsetstrokecolor{currentstroke}%
\pgfsetdash{{1.000000pt}{1.650000pt}}{0.000000pt}%
\pgfpathmoveto{\pgfqpoint{0.730283in}{0.525823in}}%
\pgfpathlineto{\pgfqpoint{0.733447in}{0.535823in}}%
\pgfpathlineto{\pgfqpoint{0.758674in}{0.535823in}}%
\pgfpathlineto{\pgfqpoint{0.771288in}{0.568398in}}%
\pgfpathlineto{\pgfqpoint{0.783901in}{0.568398in}}%
\pgfpathlineto{\pgfqpoint{0.796515in}{0.595940in}}%
\pgfpathlineto{\pgfqpoint{0.872196in}{0.595940in}}%
\pgfpathlineto{\pgfqpoint{0.884810in}{0.619798in}}%
\pgfpathlineto{\pgfqpoint{0.960491in}{0.619798in}}%
\pgfpathlineto{\pgfqpoint{0.973105in}{0.640842in}}%
\pgfpathlineto{\pgfqpoint{1.023559in}{0.640842in}}%
\pgfpathlineto{\pgfqpoint{1.048786in}{0.676696in}}%
\pgfpathlineto{\pgfqpoint{1.086627in}{0.676696in}}%
\pgfpathlineto{\pgfqpoint{1.099240in}{0.692242in}}%
\pgfpathlineto{\pgfqpoint{1.124468in}{0.692242in}}%
\pgfpathlineto{\pgfqpoint{1.137081in}{0.706544in}}%
\pgfpathlineto{\pgfqpoint{1.162308in}{0.706544in}}%
\pgfpathlineto{\pgfqpoint{1.174922in}{0.719784in}}%
\pgfpathlineto{\pgfqpoint{1.212763in}{0.719784in}}%
\pgfpathlineto{\pgfqpoint{1.225376in}{0.732111in}}%
\pgfpathlineto{\pgfqpoint{1.237990in}{0.732111in}}%
\pgfpathlineto{\pgfqpoint{1.250603in}{0.743642in}}%
\pgfpathlineto{\pgfqpoint{1.288444in}{0.743642in}}%
\pgfpathlineto{\pgfqpoint{1.301057in}{0.754474in}}%
\pgfpathlineto{\pgfqpoint{1.351512in}{0.754474in}}%
\pgfpathlineto{\pgfqpoint{1.376739in}{0.774347in}}%
\pgfpathlineto{\pgfqpoint{1.389352in}{0.774347in}}%
\pgfpathlineto{\pgfqpoint{1.401966in}{0.783511in}}%
\pgfpathlineto{\pgfqpoint{1.439807in}{0.783511in}}%
\pgfpathlineto{\pgfqpoint{1.452420in}{0.792229in}}%
\pgfpathlineto{\pgfqpoint{1.477647in}{0.792229in}}%
\pgfpathlineto{\pgfqpoint{1.490261in}{0.800540in}}%
\pgfpathlineto{\pgfqpoint{1.528102in}{0.800540in}}%
\pgfpathlineto{\pgfqpoint{1.540715in}{0.808483in}}%
\pgfpathlineto{\pgfqpoint{1.591170in}{0.808483in}}%
\pgfpathlineto{\pgfqpoint{1.603783in}{0.816087in}}%
\pgfpathlineto{\pgfqpoint{1.679464in}{0.816087in}}%
\pgfpathlineto{\pgfqpoint{1.692078in}{0.823380in}}%
\pgfpathlineto{\pgfqpoint{1.717305in}{0.823380in}}%
\pgfpathlineto{\pgfqpoint{1.729919in}{0.830388in}}%
\pgfpathlineto{\pgfqpoint{1.780373in}{0.830388in}}%
\pgfpathlineto{\pgfqpoint{1.792987in}{0.837131in}}%
\pgfpathlineto{\pgfqpoint{1.830827in}{0.837131in}}%
\pgfpathlineto{\pgfqpoint{1.843441in}{0.843629in}}%
\pgfpathlineto{\pgfqpoint{1.881282in}{0.843629in}}%
\pgfpathlineto{\pgfqpoint{1.893895in}{0.849898in}}%
\pgfpathlineto{\pgfqpoint{1.931736in}{0.849898in}}%
\pgfpathlineto{\pgfqpoint{1.944349in}{0.855956in}}%
\pgfpathlineto{\pgfqpoint{1.969577in}{0.855956in}}%
\pgfpathlineto{\pgfqpoint{2.007417in}{0.872985in}}%
\pgfpathlineto{\pgfqpoint{2.045258in}{0.872985in}}%
\pgfpathlineto{\pgfqpoint{2.057871in}{0.878319in}}%
\pgfpathlineto{\pgfqpoint{2.083099in}{0.878319in}}%
\pgfpathlineto{\pgfqpoint{2.095712in}{0.883498in}}%
\pgfpathlineto{\pgfqpoint{2.133553in}{0.883498in}}%
\pgfpathlineto{\pgfqpoint{2.146166in}{0.888531in}}%
\pgfpathlineto{\pgfqpoint{2.171394in}{0.888531in}}%
\pgfpathlineto{\pgfqpoint{2.184007in}{0.893426in}}%
\pgfpathlineto{\pgfqpoint{2.221848in}{0.893426in}}%
\pgfpathlineto{\pgfqpoint{2.234461in}{0.902832in}}%
\pgfpathlineto{\pgfqpoint{2.247075in}{0.902832in}}%
\pgfpathlineto{\pgfqpoint{2.259689in}{0.911768in}}%
\pgfpathlineto{\pgfqpoint{2.284916in}{0.920277in}}%
\pgfpathlineto{\pgfqpoint{2.310143in}{0.920277in}}%
\pgfpathlineto{\pgfqpoint{2.322756in}{0.924385in}}%
\pgfpathlineto{\pgfqpoint{2.347984in}{0.939931in}}%
\pgfpathlineto{\pgfqpoint{2.360597in}{0.943615in}}%
\pgfpathlineto{\pgfqpoint{2.373211in}{0.943615in}}%
\pgfpathlineto{\pgfqpoint{2.385824in}{0.950763in}}%
\pgfpathlineto{\pgfqpoint{2.398438in}{0.950763in}}%
\pgfpathlineto{\pgfqpoint{2.436278in}{0.960975in}}%
\pgfpathlineto{\pgfqpoint{2.448892in}{0.960975in}}%
\pgfpathlineto{\pgfqpoint{2.461506in}{0.967473in}}%
\pgfpathlineto{\pgfqpoint{2.474119in}{0.967473in}}%
\pgfpathlineto{\pgfqpoint{2.486733in}{0.970635in}}%
\pgfpathlineto{\pgfqpoint{2.499346in}{0.970635in}}%
\pgfpathlineto{\pgfqpoint{2.511960in}{0.973743in}}%
\pgfpathlineto{\pgfqpoint{2.524573in}{0.973743in}}%
\pgfpathlineto{\pgfqpoint{2.537187in}{0.976797in}}%
\pgfpathlineto{\pgfqpoint{2.549801in}{0.982753in}}%
\pgfpathlineto{\pgfqpoint{2.562414in}{0.982753in}}%
\pgfpathlineto{\pgfqpoint{2.575028in}{0.985659in}}%
\pgfpathlineto{\pgfqpoint{2.587641in}{0.991331in}}%
\pgfpathlineto{\pgfqpoint{2.625482in}{0.999516in}}%
\pgfpathlineto{\pgfqpoint{2.638096in}{0.999516in}}%
\pgfpathlineto{\pgfqpoint{2.650709in}{1.004771in}}%
\pgfpathlineto{\pgfqpoint{2.663323in}{1.004771in}}%
\pgfpathlineto{\pgfqpoint{2.675936in}{1.007342in}}%
\pgfpathlineto{\pgfqpoint{2.688550in}{1.014840in}}%
\pgfpathlineto{\pgfqpoint{2.713777in}{1.014840in}}%
\pgfpathlineto{\pgfqpoint{2.726390in}{1.017271in}}%
\pgfpathlineto{\pgfqpoint{2.751618in}{1.017271in}}%
\pgfpathlineto{\pgfqpoint{2.764231in}{1.019669in}}%
\pgfpathlineto{\pgfqpoint{2.776845in}{1.024371in}}%
\pgfpathlineto{\pgfqpoint{2.789458in}{1.024371in}}%
\pgfpathlineto{\pgfqpoint{2.827299in}{1.031200in}}%
\pgfpathlineto{\pgfqpoint{2.839913in}{1.031200in}}%
\pgfpathlineto{\pgfqpoint{2.852526in}{1.035612in}}%
\pgfpathlineto{\pgfqpoint{2.865140in}{1.037778in}}%
\pgfpathlineto{\pgfqpoint{2.877753in}{1.037778in}}%
\pgfpathlineto{\pgfqpoint{2.902980in}{1.050248in}}%
\pgfpathlineto{\pgfqpoint{2.915594in}{1.054219in}}%
\pgfpathlineto{\pgfqpoint{2.928208in}{1.063775in}}%
\pgfpathlineto{\pgfqpoint{2.940821in}{1.069273in}}%
\pgfpathlineto{\pgfqpoint{2.953435in}{1.071069in}}%
\pgfpathlineto{\pgfqpoint{2.966048in}{1.071069in}}%
\pgfpathlineto{\pgfqpoint{2.978662in}{1.074607in}}%
\pgfpathlineto{\pgfqpoint{2.991275in}{1.074607in}}%
\pgfpathlineto{\pgfqpoint{3.016503in}{1.081480in}}%
\pgfpathlineto{\pgfqpoint{3.029116in}{1.081480in}}%
\pgfpathlineto{\pgfqpoint{3.041730in}{1.088098in}}%
\pgfpathlineto{\pgfqpoint{3.054343in}{1.088098in}}%
\pgfpathlineto{\pgfqpoint{3.066957in}{1.099121in}}%
\pgfpathlineto{\pgfqpoint{3.079570in}{1.106598in}}%
\pgfpathlineto{\pgfqpoint{3.092184in}{1.109503in}}%
\pgfpathlineto{\pgfqpoint{3.104797in}{1.113774in}}%
\pgfpathlineto{\pgfqpoint{3.117411in}{1.124689in}}%
\pgfpathlineto{\pgfqpoint{3.130025in}{1.133721in}}%
\pgfpathlineto{\pgfqpoint{3.142638in}{1.133721in}}%
\pgfpathlineto{\pgfqpoint{3.155252in}{1.134975in}}%
\pgfpathlineto{\pgfqpoint{3.167865in}{1.143513in}}%
\pgfpathlineto{\pgfqpoint{3.180479in}{1.143513in}}%
\pgfpathlineto{\pgfqpoint{3.205706in}{1.145880in}}%
\pgfpathlineto{\pgfqpoint{3.218320in}{1.160542in}}%
\pgfpathlineto{\pgfqpoint{3.230933in}{1.161622in}}%
\pgfpathlineto{\pgfqpoint{3.243547in}{1.175093in}}%
\pgfpathlineto{\pgfqpoint{3.281387in}{1.178063in}}%
\pgfpathlineto{\pgfqpoint{3.294001in}{1.180984in}}%
\pgfpathlineto{\pgfqpoint{3.306615in}{1.192213in}}%
\pgfpathlineto{\pgfqpoint{3.319228in}{1.194913in}}%
\pgfpathlineto{\pgfqpoint{3.331842in}{1.207835in}}%
\pgfpathlineto{\pgfqpoint{3.344455in}{1.215958in}}%
\pgfpathlineto{\pgfqpoint{3.369682in}{1.216750in}}%
\pgfpathlineto{\pgfqpoint{3.382296in}{1.218324in}}%
\pgfpathlineto{\pgfqpoint{3.394910in}{1.227489in}}%
\pgfpathlineto{\pgfqpoint{3.407523in}{1.228232in}}%
\pgfpathlineto{\pgfqpoint{3.420137in}{1.234066in}}%
\pgfpathlineto{\pgfqpoint{3.445364in}{1.258193in}}%
\pgfpathlineto{\pgfqpoint{3.483204in}{1.260683in}}%
\pgfpathlineto{\pgfqpoint{3.495818in}{1.292329in}}%
\pgfpathlineto{\pgfqpoint{3.508432in}{1.293362in}}%
\pgfpathlineto{\pgfqpoint{3.533659in}{1.297434in}}%
\pgfpathlineto{\pgfqpoint{3.546272in}{1.298938in}}%
\pgfpathlineto{\pgfqpoint{3.558886in}{1.324039in}}%
\pgfpathlineto{\pgfqpoint{3.571499in}{1.492545in}}%
\pgfpathlineto{\pgfqpoint{3.584113in}{1.493555in}}%
\pgfpathlineto{\pgfqpoint{3.596727in}{1.603252in}}%
\pgfpathlineto{\pgfqpoint{3.621954in}{1.612116in}}%
\pgfpathlineto{\pgfqpoint{3.634567in}{1.613237in}}%
\pgfpathlineto{\pgfqpoint{3.647181in}{1.616220in}}%
\pgfpathlineto{\pgfqpoint{3.659794in}{1.677864in}}%
\pgfpathlineto{\pgfqpoint{3.685022in}{1.694430in}}%
\pgfpathlineto{\pgfqpoint{3.697635in}{1.700227in}}%
\pgfpathlineto{\pgfqpoint{3.710249in}{1.717242in}}%
\pgfpathlineto{\pgfqpoint{3.722862in}{1.737382in}}%
\pgfpathlineto{\pgfqpoint{3.735476in}{1.786602in}}%
\pgfpathlineto{\pgfqpoint{3.748089in}{1.842521in}}%
\pgfpathlineto{\pgfqpoint{3.760703in}{1.843186in}}%
\pgfpathlineto{\pgfqpoint{3.773317in}{1.850121in}}%
\pgfpathlineto{\pgfqpoint{3.785930in}{1.869586in}}%
\pgfpathlineto{\pgfqpoint{3.798544in}{1.870342in}}%
\pgfpathlineto{\pgfqpoint{3.823771in}{1.879441in}}%
\pgfpathlineto{\pgfqpoint{3.836384in}{1.995009in}}%
\pgfpathlineto{\pgfqpoint{3.848998in}{2.005480in}}%
\pgfpathlineto{\pgfqpoint{3.861611in}{2.022573in}}%
\pgfpathlineto{\pgfqpoint{3.874225in}{2.023215in}}%
\pgfpathlineto{\pgfqpoint{3.886839in}{2.027157in}}%
\pgfpathlineto{\pgfqpoint{3.899452in}{2.029212in}}%
\pgfpathlineto{\pgfqpoint{3.912066in}{2.029798in}}%
\pgfpathlineto{\pgfqpoint{3.924679in}{2.039246in}}%
\pgfpathlineto{\pgfqpoint{3.937293in}{2.128677in}}%
\pgfpathlineto{\pgfqpoint{3.949906in}{2.131374in}}%
\pgfpathlineto{\pgfqpoint{3.962520in}{2.132016in}}%
\pgfpathlineto{\pgfqpoint{3.975134in}{2.150806in}}%
\pgfpathlineto{\pgfqpoint{3.987747in}{2.161379in}}%
\pgfpathlineto{\pgfqpoint{4.000361in}{2.212505in}}%
\pgfpathlineto{\pgfqpoint{4.000361in}{2.305275in}}%
\pgfpathlineto{\pgfqpoint{4.000361in}{2.305275in}}%
\pgfusepath{stroke}%
\end{pgfscope}%
\begin{pgfscope}%
\pgfpathrectangle{\pgfqpoint{0.708220in}{0.535823in}}{\pgfqpoint{5.045427in}{1.769453in}}%
\pgfusepath{clip}%
\pgfsetbuttcap%
\pgfsetroundjoin%
\pgfsetlinewidth{1.003750pt}%
\definecolor{currentstroke}{rgb}{0.890196,0.466667,0.760784}%
\pgfsetstrokecolor{currentstroke}%
\pgfsetdash{{6.400000pt}{1.600000pt}{1.000000pt}{1.600000pt}}{0.000000pt}%
\pgfpathmoveto{\pgfqpoint{0.711988in}{0.525823in}}%
\pgfpathlineto{\pgfqpoint{0.720833in}{0.595940in}}%
\pgfpathlineto{\pgfqpoint{0.733447in}{0.719784in}}%
\pgfpathlineto{\pgfqpoint{0.746061in}{0.732111in}}%
\pgfpathlineto{\pgfqpoint{0.758674in}{0.764687in}}%
\pgfpathlineto{\pgfqpoint{0.783901in}{0.783511in}}%
\pgfpathlineto{\pgfqpoint{0.809128in}{0.800540in}}%
\pgfpathlineto{\pgfqpoint{0.821742in}{0.800540in}}%
\pgfpathlineto{\pgfqpoint{0.846969in}{0.816087in}}%
\pgfpathlineto{\pgfqpoint{0.859583in}{0.816087in}}%
\pgfpathlineto{\pgfqpoint{0.872196in}{0.830388in}}%
\pgfpathlineto{\pgfqpoint{0.897423in}{0.843629in}}%
\pgfpathlineto{\pgfqpoint{0.910037in}{0.843629in}}%
\pgfpathlineto{\pgfqpoint{0.922650in}{0.855956in}}%
\pgfpathlineto{\pgfqpoint{0.935264in}{0.861814in}}%
\pgfpathlineto{\pgfqpoint{0.960491in}{0.883498in}}%
\pgfpathlineto{\pgfqpoint{0.973105in}{0.883498in}}%
\pgfpathlineto{\pgfqpoint{0.985718in}{0.893426in}}%
\pgfpathlineto{\pgfqpoint{0.998332in}{0.898191in}}%
\pgfpathlineto{\pgfqpoint{1.023559in}{0.916073in}}%
\pgfpathlineto{\pgfqpoint{1.048786in}{0.916073in}}%
\pgfpathlineto{\pgfqpoint{1.061400in}{0.920277in}}%
\pgfpathlineto{\pgfqpoint{1.074013in}{0.920277in}}%
\pgfpathlineto{\pgfqpoint{1.086627in}{0.960975in}}%
\pgfpathlineto{\pgfqpoint{1.099240in}{0.970635in}}%
\pgfpathlineto{\pgfqpoint{1.111854in}{0.985659in}}%
\pgfpathlineto{\pgfqpoint{1.124468in}{0.991331in}}%
\pgfpathlineto{\pgfqpoint{1.137081in}{0.999516in}}%
\pgfpathlineto{\pgfqpoint{1.162308in}{1.009876in}}%
\pgfpathlineto{\pgfqpoint{1.174922in}{1.024371in}}%
\pgfpathlineto{\pgfqpoint{1.187535in}{1.028953in}}%
\pgfpathlineto{\pgfqpoint{1.200149in}{1.039917in}}%
\pgfpathlineto{\pgfqpoint{1.225376in}{1.039917in}}%
\pgfpathlineto{\pgfqpoint{1.237990in}{1.044122in}}%
\pgfpathlineto{\pgfqpoint{1.250603in}{1.058103in}}%
\pgfpathlineto{\pgfqpoint{1.263217in}{1.061904in}}%
\pgfpathlineto{\pgfqpoint{1.275830in}{1.063775in}}%
\pgfpathlineto{\pgfqpoint{1.288444in}{1.071069in}}%
\pgfpathlineto{\pgfqpoint{1.301057in}{1.074607in}}%
\pgfpathlineto{\pgfqpoint{1.313671in}{1.091317in}}%
\pgfpathlineto{\pgfqpoint{1.326285in}{1.091317in}}%
\pgfpathlineto{\pgfqpoint{1.338898in}{1.116566in}}%
\pgfpathlineto{\pgfqpoint{1.351512in}{1.137456in}}%
\pgfpathlineto{\pgfqpoint{1.364125in}{1.137456in}}%
\pgfpathlineto{\pgfqpoint{1.376739in}{1.145880in}}%
\pgfpathlineto{\pgfqpoint{1.389352in}{1.152797in}}%
\pgfpathlineto{\pgfqpoint{1.401966in}{1.166924in}}%
\pgfpathlineto{\pgfqpoint{1.414580in}{1.171055in}}%
\pgfpathlineto{\pgfqpoint{1.427193in}{1.219885in}}%
\pgfpathlineto{\pgfqpoint{1.439807in}{1.235496in}}%
\pgfpathlineto{\pgfqpoint{1.452420in}{1.243159in}}%
\pgfpathlineto{\pgfqpoint{1.465034in}{1.244518in}}%
\pgfpathlineto{\pgfqpoint{1.477647in}{1.247205in}}%
\pgfpathlineto{\pgfqpoint{1.490261in}{1.253750in}}%
\pgfpathlineto{\pgfqpoint{1.502875in}{1.268545in}}%
\pgfpathlineto{\pgfqpoint{1.515488in}{1.269136in}}%
\pgfpathlineto{\pgfqpoint{1.528102in}{1.273216in}}%
\pgfpathlineto{\pgfqpoint{1.540715in}{1.284387in}}%
\pgfpathlineto{\pgfqpoint{1.553329in}{1.289194in}}%
\pgfpathlineto{\pgfqpoint{1.578556in}{1.290245in}}%
\pgfpathlineto{\pgfqpoint{1.616397in}{1.292329in}}%
\pgfpathlineto{\pgfqpoint{1.629010in}{1.297434in}}%
\pgfpathlineto{\pgfqpoint{1.641624in}{1.304345in}}%
\pgfpathlineto{\pgfqpoint{1.679464in}{1.308177in}}%
\pgfpathlineto{\pgfqpoint{1.692078in}{1.316962in}}%
\pgfpathlineto{\pgfqpoint{1.704692in}{1.321418in}}%
\pgfpathlineto{\pgfqpoint{1.717305in}{1.321857in}}%
\pgfpathlineto{\pgfqpoint{1.729919in}{1.333745in}}%
\pgfpathlineto{\pgfqpoint{1.742532in}{1.347571in}}%
\pgfpathlineto{\pgfqpoint{1.755146in}{1.357192in}}%
\pgfpathlineto{\pgfqpoint{1.767759in}{1.363213in}}%
\pgfpathlineto{\pgfqpoint{1.792987in}{1.386067in}}%
\pgfpathlineto{\pgfqpoint{1.805600in}{1.392980in}}%
\pgfpathlineto{\pgfqpoint{1.818214in}{1.420016in}}%
\pgfpathlineto{\pgfqpoint{1.843441in}{1.422532in}}%
\pgfpathlineto{\pgfqpoint{1.856054in}{1.428914in}}%
\pgfpathlineto{\pgfqpoint{1.881282in}{1.430355in}}%
\pgfpathlineto{\pgfqpoint{1.893895in}{1.432495in}}%
\pgfpathlineto{\pgfqpoint{1.906509in}{1.436005in}}%
\pgfpathlineto{\pgfqpoint{1.919122in}{1.436930in}}%
\pgfpathlineto{\pgfqpoint{1.931736in}{1.440128in}}%
\pgfpathlineto{\pgfqpoint{1.969577in}{1.442379in}}%
\pgfpathlineto{\pgfqpoint{2.108326in}{1.459428in}}%
\pgfpathlineto{\pgfqpoint{2.133553in}{1.470272in}}%
\pgfpathlineto{\pgfqpoint{2.184007in}{1.475549in}}%
\pgfpathlineto{\pgfqpoint{2.196621in}{1.475920in}}%
\pgfpathlineto{\pgfqpoint{2.209234in}{1.479408in}}%
\pgfpathlineto{\pgfqpoint{2.221848in}{1.479771in}}%
\pgfpathlineto{\pgfqpoint{2.234461in}{1.482113in}}%
\pgfpathlineto{\pgfqpoint{2.284916in}{1.486009in}}%
\pgfpathlineto{\pgfqpoint{2.297529in}{1.491528in}}%
\pgfpathlineto{\pgfqpoint{2.335370in}{1.494560in}}%
\pgfpathlineto{\pgfqpoint{2.385824in}{1.497540in}}%
\pgfpathlineto{\pgfqpoint{2.398438in}{1.503991in}}%
\pgfpathlineto{\pgfqpoint{2.411051in}{1.506195in}}%
\pgfpathlineto{\pgfqpoint{2.423665in}{1.529210in}}%
\pgfpathlineto{\pgfqpoint{2.436278in}{1.529622in}}%
\pgfpathlineto{\pgfqpoint{2.448892in}{1.534094in}}%
\pgfpathlineto{\pgfqpoint{2.474119in}{1.538327in}}%
\pgfpathlineto{\pgfqpoint{2.486733in}{1.549105in}}%
\pgfpathlineto{\pgfqpoint{2.499346in}{1.555272in}}%
\pgfpathlineto{\pgfqpoint{2.524573in}{1.556811in}}%
\pgfpathlineto{\pgfqpoint{2.537187in}{1.557282in}}%
\pgfpathlineto{\pgfqpoint{2.549801in}{1.559153in}}%
\pgfpathlineto{\pgfqpoint{2.575028in}{1.560081in}}%
\pgfpathlineto{\pgfqpoint{2.587641in}{1.563178in}}%
\pgfpathlineto{\pgfqpoint{2.600255in}{1.573239in}}%
\pgfpathlineto{\pgfqpoint{2.612868in}{1.574844in}}%
\pgfpathlineto{\pgfqpoint{2.625482in}{1.583374in}}%
\pgfpathlineto{\pgfqpoint{2.663323in}{1.585193in}}%
\pgfpathlineto{\pgfqpoint{2.675936in}{1.607295in}}%
\pgfpathlineto{\pgfqpoint{2.701163in}{1.627439in}}%
\pgfpathlineto{\pgfqpoint{2.713777in}{1.651466in}}%
\pgfpathlineto{\pgfqpoint{2.751618in}{1.655242in}}%
\pgfpathlineto{\pgfqpoint{2.764231in}{1.655445in}}%
\pgfpathlineto{\pgfqpoint{2.776845in}{1.658072in}}%
\pgfpathlineto{\pgfqpoint{2.789458in}{1.662563in}}%
\pgfpathlineto{\pgfqpoint{2.802072in}{1.663473in}}%
\pgfpathlineto{\pgfqpoint{2.814685in}{1.667387in}}%
\pgfpathlineto{\pgfqpoint{2.827299in}{1.667958in}}%
\pgfpathlineto{\pgfqpoint{2.852526in}{1.676059in}}%
\pgfpathlineto{\pgfqpoint{2.865140in}{1.683578in}}%
\pgfpathlineto{\pgfqpoint{2.877753in}{1.684964in}}%
\pgfpathlineto{\pgfqpoint{2.890367in}{1.697137in}}%
\pgfpathlineto{\pgfqpoint{2.902980in}{1.699168in}}%
\pgfpathlineto{\pgfqpoint{2.915594in}{1.715698in}}%
\pgfpathlineto{\pgfqpoint{2.928208in}{1.719911in}}%
\pgfpathlineto{\pgfqpoint{2.940821in}{1.720336in}}%
\pgfpathlineto{\pgfqpoint{2.953435in}{1.722400in}}%
\pgfpathlineto{\pgfqpoint{2.966048in}{1.723052in}}%
\pgfpathlineto{\pgfqpoint{2.978662in}{1.725132in}}%
\pgfpathlineto{\pgfqpoint{2.991275in}{1.741160in}}%
\pgfpathlineto{\pgfqpoint{3.003889in}{1.744365in}}%
\pgfpathlineto{\pgfqpoint{3.029116in}{1.744695in}}%
\pgfpathlineto{\pgfqpoint{3.054343in}{1.747108in}}%
\pgfpathlineto{\pgfqpoint{3.066957in}{1.791485in}}%
\pgfpathlineto{\pgfqpoint{3.079570in}{1.793346in}}%
\pgfpathlineto{\pgfqpoint{3.092184in}{1.806171in}}%
\pgfpathlineto{\pgfqpoint{3.117411in}{1.808808in}}%
\pgfpathlineto{\pgfqpoint{3.167865in}{1.815387in}}%
\pgfpathlineto{\pgfqpoint{3.180479in}{1.816437in}}%
\pgfpathlineto{\pgfqpoint{3.193092in}{1.824203in}}%
\pgfpathlineto{\pgfqpoint{3.218320in}{1.826745in}}%
\pgfpathlineto{\pgfqpoint{3.243547in}{1.827886in}}%
\pgfpathlineto{\pgfqpoint{3.256160in}{1.831696in}}%
\pgfpathlineto{\pgfqpoint{3.281387in}{1.833984in}}%
\pgfpathlineto{\pgfqpoint{3.294001in}{1.842306in}}%
\pgfpathlineto{\pgfqpoint{3.306615in}{1.860108in}}%
\pgfpathlineto{\pgfqpoint{3.331842in}{1.881329in}}%
\pgfpathlineto{\pgfqpoint{3.344455in}{1.884915in}}%
\pgfpathlineto{\pgfqpoint{3.394910in}{1.887582in}}%
\pgfpathlineto{\pgfqpoint{3.407523in}{1.891951in}}%
\pgfpathlineto{\pgfqpoint{3.420137in}{1.894480in}}%
\pgfpathlineto{\pgfqpoint{3.432750in}{1.899449in}}%
\pgfpathlineto{\pgfqpoint{3.445364in}{1.899501in}}%
\pgfpathlineto{\pgfqpoint{3.457977in}{1.903490in}}%
\pgfpathlineto{\pgfqpoint{3.470591in}{1.903727in}}%
\pgfpathlineto{\pgfqpoint{3.483204in}{1.913738in}}%
\pgfpathlineto{\pgfqpoint{3.495818in}{1.913802in}}%
\pgfpathlineto{\pgfqpoint{3.508432in}{1.923264in}}%
\pgfpathlineto{\pgfqpoint{3.521045in}{1.924126in}}%
\pgfpathlineto{\pgfqpoint{3.533659in}{1.928609in}}%
\pgfpathlineto{\pgfqpoint{3.546272in}{1.951782in}}%
\pgfpathlineto{\pgfqpoint{3.558886in}{1.951834in}}%
\pgfpathlineto{\pgfqpoint{3.571499in}{1.968178in}}%
\pgfpathlineto{\pgfqpoint{3.584113in}{1.968272in}}%
\pgfpathlineto{\pgfqpoint{3.596727in}{1.997798in}}%
\pgfpathlineto{\pgfqpoint{3.647181in}{1.999468in}}%
\pgfpathlineto{\pgfqpoint{3.697635in}{2.006310in}}%
\pgfpathlineto{\pgfqpoint{3.710249in}{2.010740in}}%
\pgfpathlineto{\pgfqpoint{3.735476in}{2.014290in}}%
\pgfpathlineto{\pgfqpoint{3.773317in}{2.017288in}}%
\pgfpathlineto{\pgfqpoint{3.785930in}{2.032844in}}%
\pgfpathlineto{\pgfqpoint{3.798544in}{2.053488in}}%
\pgfpathlineto{\pgfqpoint{3.811157in}{2.054101in}}%
\pgfpathlineto{\pgfqpoint{3.823771in}{2.096906in}}%
\pgfpathlineto{\pgfqpoint{3.836384in}{2.107120in}}%
\pgfpathlineto{\pgfqpoint{3.848998in}{2.111010in}}%
\pgfpathlineto{\pgfqpoint{3.861611in}{2.128062in}}%
\pgfpathlineto{\pgfqpoint{3.886839in}{2.131743in}}%
\pgfpathlineto{\pgfqpoint{3.899452in}{2.137637in}}%
\pgfpathlineto{\pgfqpoint{3.912066in}{2.139509in}}%
\pgfpathlineto{\pgfqpoint{3.924679in}{2.150164in}}%
\pgfpathlineto{\pgfqpoint{3.937293in}{2.150292in}}%
\pgfpathlineto{\pgfqpoint{3.949906in}{2.161387in}}%
\pgfpathlineto{\pgfqpoint{3.962520in}{2.164356in}}%
\pgfpathlineto{\pgfqpoint{3.975134in}{2.179375in}}%
\pgfpathlineto{\pgfqpoint{4.000361in}{2.180413in}}%
\pgfpathlineto{\pgfqpoint{4.038201in}{2.186278in}}%
\pgfpathlineto{\pgfqpoint{4.050815in}{2.187581in}}%
\pgfpathlineto{\pgfqpoint{4.063429in}{2.190580in}}%
\pgfpathlineto{\pgfqpoint{4.076042in}{2.195013in}}%
\pgfpathlineto{\pgfqpoint{4.076042in}{2.305275in}}%
\pgfpathlineto{\pgfqpoint{4.076042in}{2.305275in}}%
\pgfusepath{stroke}%
\end{pgfscope}%
\begin{pgfscope}%
\pgfpathrectangle{\pgfqpoint{0.708220in}{0.535823in}}{\pgfqpoint{5.045427in}{1.769453in}}%
\pgfusepath{clip}%
\pgfsetbuttcap%
\pgfsetroundjoin%
\pgfsetlinewidth{1.003750pt}%
\definecolor{currentstroke}{rgb}{0.498039,0.498039,0.498039}%
\pgfsetstrokecolor{currentstroke}%
\pgfsetdash{{6.400000pt}{1.600000pt}{1.000000pt}{1.600000pt}}{0.000000pt}%
\pgfpathmoveto{\pgfqpoint{0.881646in}{0.525823in}}%
\pgfpathlineto{\pgfqpoint{0.884810in}{0.535823in}}%
\pgfpathlineto{\pgfqpoint{0.897423in}{0.568398in}}%
\pgfpathlineto{\pgfqpoint{0.935264in}{0.568398in}}%
\pgfpathlineto{\pgfqpoint{0.947878in}{0.595940in}}%
\pgfpathlineto{\pgfqpoint{0.960491in}{0.595940in}}%
\pgfpathlineto{\pgfqpoint{0.973105in}{0.619798in}}%
\pgfpathlineto{\pgfqpoint{0.985718in}{0.619798in}}%
\pgfpathlineto{\pgfqpoint{0.998332in}{0.640842in}}%
\pgfpathlineto{\pgfqpoint{1.036173in}{0.640842in}}%
\pgfpathlineto{\pgfqpoint{1.061400in}{0.676696in}}%
\pgfpathlineto{\pgfqpoint{1.074013in}{0.676696in}}%
\pgfpathlineto{\pgfqpoint{1.086627in}{0.692242in}}%
\pgfpathlineto{\pgfqpoint{1.099240in}{0.692242in}}%
\pgfpathlineto{\pgfqpoint{1.124468in}{0.719784in}}%
\pgfpathlineto{\pgfqpoint{1.137081in}{0.754474in}}%
\pgfpathlineto{\pgfqpoint{1.149695in}{0.774347in}}%
\pgfpathlineto{\pgfqpoint{1.174922in}{0.774347in}}%
\pgfpathlineto{\pgfqpoint{1.187535in}{0.783511in}}%
\pgfpathlineto{\pgfqpoint{1.200149in}{0.783511in}}%
\pgfpathlineto{\pgfqpoint{1.212763in}{0.792229in}}%
\pgfpathlineto{\pgfqpoint{1.237990in}{0.792229in}}%
\pgfpathlineto{\pgfqpoint{1.275830in}{0.816087in}}%
\pgfpathlineto{\pgfqpoint{1.288444in}{0.816087in}}%
\pgfpathlineto{\pgfqpoint{1.301057in}{0.823380in}}%
\pgfpathlineto{\pgfqpoint{1.326285in}{0.823380in}}%
\pgfpathlineto{\pgfqpoint{1.338898in}{0.830388in}}%
\pgfpathlineto{\pgfqpoint{1.351512in}{0.830388in}}%
\pgfpathlineto{\pgfqpoint{1.364125in}{0.837131in}}%
\pgfpathlineto{\pgfqpoint{1.389352in}{0.837131in}}%
\pgfpathlineto{\pgfqpoint{1.401966in}{0.855956in}}%
\pgfpathlineto{\pgfqpoint{1.452420in}{0.855956in}}%
\pgfpathlineto{\pgfqpoint{1.465034in}{0.872985in}}%
\pgfpathlineto{\pgfqpoint{1.477647in}{0.878319in}}%
\pgfpathlineto{\pgfqpoint{1.490261in}{0.878319in}}%
\pgfpathlineto{\pgfqpoint{1.502875in}{0.883498in}}%
\pgfpathlineto{\pgfqpoint{1.528102in}{0.883498in}}%
\pgfpathlineto{\pgfqpoint{1.540715in}{0.898191in}}%
\pgfpathlineto{\pgfqpoint{1.553329in}{0.898191in}}%
\pgfpathlineto{\pgfqpoint{1.578556in}{0.907356in}}%
\pgfpathlineto{\pgfqpoint{1.591170in}{0.907356in}}%
\pgfpathlineto{\pgfqpoint{1.603783in}{0.911768in}}%
\pgfpathlineto{\pgfqpoint{1.629010in}{0.911768in}}%
\pgfpathlineto{\pgfqpoint{1.641624in}{0.920277in}}%
\pgfpathlineto{\pgfqpoint{1.654237in}{0.924385in}}%
\pgfpathlineto{\pgfqpoint{1.666851in}{0.924385in}}%
\pgfpathlineto{\pgfqpoint{1.704692in}{0.936169in}}%
\pgfpathlineto{\pgfqpoint{1.717305in}{0.936169in}}%
\pgfpathlineto{\pgfqpoint{1.729919in}{0.943615in}}%
\pgfpathlineto{\pgfqpoint{1.742532in}{0.943615in}}%
\pgfpathlineto{\pgfqpoint{1.755146in}{0.954232in}}%
\pgfpathlineto{\pgfqpoint{1.767759in}{0.954232in}}%
\pgfpathlineto{\pgfqpoint{1.780373in}{0.957636in}}%
\pgfpathlineto{\pgfqpoint{1.792987in}{0.964254in}}%
\pgfpathlineto{\pgfqpoint{1.818214in}{0.964254in}}%
\pgfpathlineto{\pgfqpoint{1.830827in}{0.970635in}}%
\pgfpathlineto{\pgfqpoint{1.856054in}{0.976797in}}%
\pgfpathlineto{\pgfqpoint{1.868668in}{0.988517in}}%
\pgfpathlineto{\pgfqpoint{1.881282in}{0.988517in}}%
\pgfpathlineto{\pgfqpoint{1.893895in}{0.991331in}}%
\pgfpathlineto{\pgfqpoint{1.906509in}{0.996829in}}%
\pgfpathlineto{\pgfqpoint{1.919122in}{0.999516in}}%
\pgfpathlineto{\pgfqpoint{1.931736in}{0.999516in}}%
\pgfpathlineto{\pgfqpoint{1.944349in}{1.002163in}}%
\pgfpathlineto{\pgfqpoint{1.956963in}{1.002163in}}%
\pgfpathlineto{\pgfqpoint{1.969577in}{1.007342in}}%
\pgfpathlineto{\pgfqpoint{1.982190in}{1.007342in}}%
\pgfpathlineto{\pgfqpoint{1.994804in}{1.009876in}}%
\pgfpathlineto{\pgfqpoint{2.020031in}{1.009876in}}%
\pgfpathlineto{\pgfqpoint{2.032644in}{1.012375in}}%
\pgfpathlineto{\pgfqpoint{2.045258in}{1.012375in}}%
\pgfpathlineto{\pgfqpoint{2.057871in}{1.014840in}}%
\pgfpathlineto{\pgfqpoint{2.070485in}{1.014840in}}%
\pgfpathlineto{\pgfqpoint{2.083099in}{1.019669in}}%
\pgfpathlineto{\pgfqpoint{2.095712in}{1.022036in}}%
\pgfpathlineto{\pgfqpoint{2.108326in}{1.022036in}}%
\pgfpathlineto{\pgfqpoint{2.120939in}{1.035612in}}%
\pgfpathlineto{\pgfqpoint{2.133553in}{1.037778in}}%
\pgfpathlineto{\pgfqpoint{2.146166in}{1.048229in}}%
\pgfpathlineto{\pgfqpoint{2.184007in}{1.054219in}}%
\pgfpathlineto{\pgfqpoint{2.209234in}{1.054219in}}%
\pgfpathlineto{\pgfqpoint{2.221848in}{1.056171in}}%
\pgfpathlineto{\pgfqpoint{2.234461in}{1.060014in}}%
\pgfpathlineto{\pgfqpoint{2.247075in}{1.061904in}}%
\pgfpathlineto{\pgfqpoint{2.259689in}{1.061904in}}%
\pgfpathlineto{\pgfqpoint{2.272302in}{1.072847in}}%
\pgfpathlineto{\pgfqpoint{2.284916in}{1.072847in}}%
\pgfpathlineto{\pgfqpoint{2.297529in}{1.079786in}}%
\pgfpathlineto{\pgfqpoint{2.310143in}{1.079786in}}%
\pgfpathlineto{\pgfqpoint{2.335370in}{1.083158in}}%
\pgfpathlineto{\pgfqpoint{2.347984in}{1.091317in}}%
\pgfpathlineto{\pgfqpoint{2.360597in}{1.094480in}}%
\pgfpathlineto{\pgfqpoint{2.373211in}{1.106598in}}%
\pgfpathlineto{\pgfqpoint{2.411051in}{1.115175in}}%
\pgfpathlineto{\pgfqpoint{2.423665in}{1.123360in}}%
\pgfpathlineto{\pgfqpoint{2.436278in}{1.152797in}}%
\pgfpathlineto{\pgfqpoint{2.461506in}{1.156158in}}%
\pgfpathlineto{\pgfqpoint{2.486733in}{1.169002in}}%
\pgfpathlineto{\pgfqpoint{2.499346in}{1.173086in}}%
\pgfpathlineto{\pgfqpoint{2.511960in}{1.175093in}}%
\pgfpathlineto{\pgfqpoint{2.524573in}{1.191304in}}%
\pgfpathlineto{\pgfqpoint{2.537187in}{1.203631in}}%
\pgfpathlineto{\pgfqpoint{2.575028in}{1.206165in}}%
\pgfpathlineto{\pgfqpoint{2.587641in}{1.224486in}}%
\pgfpathlineto{\pgfqpoint{2.600255in}{1.234066in}}%
\pgfpathlineto{\pgfqpoint{2.612868in}{1.245193in}}%
\pgfpathlineto{\pgfqpoint{2.625482in}{1.247870in}}%
\pgfpathlineto{\pgfqpoint{2.650709in}{1.258819in}}%
\pgfpathlineto{\pgfqpoint{2.663323in}{1.259443in}}%
\pgfpathlineto{\pgfqpoint{2.675936in}{1.261301in}}%
\pgfpathlineto{\pgfqpoint{2.688550in}{1.291290in}}%
\pgfpathlineto{\pgfqpoint{2.701163in}{1.291290in}}%
\pgfpathlineto{\pgfqpoint{2.713777in}{1.301416in}}%
\pgfpathlineto{\pgfqpoint{2.726390in}{1.304345in}}%
\pgfpathlineto{\pgfqpoint{2.776845in}{1.310063in}}%
\pgfpathlineto{\pgfqpoint{2.802072in}{1.335380in}}%
\pgfpathlineto{\pgfqpoint{2.814685in}{1.335380in}}%
\pgfpathlineto{\pgfqpoint{2.839913in}{1.347191in}}%
\pgfpathlineto{\pgfqpoint{2.852526in}{1.347191in}}%
\pgfpathlineto{\pgfqpoint{2.890367in}{1.371382in}}%
\pgfpathlineto{\pgfqpoint{2.902980in}{1.385453in}}%
\pgfpathlineto{\pgfqpoint{2.915594in}{1.386373in}}%
\pgfpathlineto{\pgfqpoint{2.928208in}{1.408231in}}%
\pgfpathlineto{\pgfqpoint{2.940821in}{1.424767in}}%
\pgfpathlineto{\pgfqpoint{2.953435in}{1.430594in}}%
\pgfpathlineto{\pgfqpoint{2.978662in}{1.431547in}}%
\pgfpathlineto{\pgfqpoint{3.003889in}{1.433907in}}%
\pgfpathlineto{\pgfqpoint{3.066957in}{1.436930in}}%
\pgfpathlineto{\pgfqpoint{3.079570in}{1.439448in}}%
\pgfpathlineto{\pgfqpoint{3.092184in}{1.439902in}}%
\pgfpathlineto{\pgfqpoint{3.104797in}{1.443271in}}%
\pgfpathlineto{\pgfqpoint{3.155252in}{1.445702in}}%
\pgfpathlineto{\pgfqpoint{3.167865in}{1.481396in}}%
\pgfpathlineto{\pgfqpoint{3.180479in}{1.481755in}}%
\pgfpathlineto{\pgfqpoint{3.193092in}{1.490165in}}%
\pgfpathlineto{\pgfqpoint{3.205706in}{1.491018in}}%
\pgfpathlineto{\pgfqpoint{3.218320in}{1.494727in}}%
\pgfpathlineto{\pgfqpoint{3.230933in}{1.495060in}}%
\pgfpathlineto{\pgfqpoint{3.243547in}{1.531533in}}%
\pgfpathlineto{\pgfqpoint{3.256160in}{1.550453in}}%
\pgfpathlineto{\pgfqpoint{3.268774in}{1.552638in}}%
\pgfpathlineto{\pgfqpoint{3.281387in}{1.557751in}}%
\pgfpathlineto{\pgfqpoint{3.294001in}{1.629255in}}%
\pgfpathlineto{\pgfqpoint{3.306615in}{1.630508in}}%
\pgfpathlineto{\pgfqpoint{3.319228in}{1.646470in}}%
\pgfpathlineto{\pgfqpoint{3.331842in}{1.660596in}}%
\pgfpathlineto{\pgfqpoint{3.344455in}{1.663668in}}%
\pgfpathlineto{\pgfqpoint{3.357069in}{1.670034in}}%
\pgfpathlineto{\pgfqpoint{3.369682in}{1.689058in}}%
\pgfpathlineto{\pgfqpoint{3.382296in}{1.696815in}}%
\pgfpathlineto{\pgfqpoint{3.394910in}{1.697298in}}%
\pgfpathlineto{\pgfqpoint{3.407523in}{1.703368in}}%
\pgfpathlineto{\pgfqpoint{3.420137in}{1.726277in}}%
\pgfpathlineto{\pgfqpoint{3.432750in}{1.727958in}}%
\pgfpathlineto{\pgfqpoint{3.457977in}{1.734223in}}%
\pgfpathlineto{\pgfqpoint{3.470591in}{1.740655in}}%
\pgfpathlineto{\pgfqpoint{3.483204in}{1.740697in}}%
\pgfpathlineto{\pgfqpoint{3.495818in}{1.742624in}}%
\pgfpathlineto{\pgfqpoint{3.508432in}{1.755286in}}%
\pgfpathlineto{\pgfqpoint{3.521045in}{1.757178in}}%
\pgfpathlineto{\pgfqpoint{3.533659in}{1.767074in}}%
\pgfpathlineto{\pgfqpoint{3.546272in}{1.769599in}}%
\pgfpathlineto{\pgfqpoint{3.558886in}{1.777893in}}%
\pgfpathlineto{\pgfqpoint{3.571499in}{1.781750in}}%
\pgfpathlineto{\pgfqpoint{3.584113in}{1.800206in}}%
\pgfpathlineto{\pgfqpoint{3.596727in}{1.803258in}}%
\pgfpathlineto{\pgfqpoint{3.609340in}{1.815082in}}%
\pgfpathlineto{\pgfqpoint{3.621954in}{1.947863in}}%
\pgfpathlineto{\pgfqpoint{3.634567in}{1.967942in}}%
\pgfpathlineto{\pgfqpoint{3.647181in}{1.968684in}}%
\pgfpathlineto{\pgfqpoint{3.659794in}{2.012082in}}%
\pgfpathlineto{\pgfqpoint{3.672408in}{2.015515in}}%
\pgfpathlineto{\pgfqpoint{3.685022in}{2.107185in}}%
\pgfpathlineto{\pgfqpoint{3.697635in}{2.110575in}}%
\pgfpathlineto{\pgfqpoint{3.710249in}{2.128399in}}%
\pgfpathlineto{\pgfqpoint{3.722862in}{2.128466in}}%
\pgfpathlineto{\pgfqpoint{3.735476in}{2.164898in}}%
\pgfpathlineto{\pgfqpoint{3.748089in}{2.193686in}}%
\pgfpathlineto{\pgfqpoint{3.760703in}{2.278546in}}%
\pgfpathlineto{\pgfqpoint{3.773317in}{2.283122in}}%
\pgfpathlineto{\pgfqpoint{3.785930in}{2.285561in}}%
\pgfpathlineto{\pgfqpoint{3.785930in}{2.305275in}}%
\pgfpathlineto{\pgfqpoint{3.785930in}{2.305275in}}%
\pgfusepath{stroke}%
\end{pgfscope}%
\begin{pgfscope}%
\pgfsetrectcap%
\pgfsetmiterjoin%
\pgfsetlinewidth{0.803000pt}%
\definecolor{currentstroke}{rgb}{0.000000,0.000000,0.000000}%
\pgfsetstrokecolor{currentstroke}%
\pgfsetdash{}{0pt}%
\pgfpathmoveto{\pgfqpoint{0.708220in}{0.535823in}}%
\pgfpathlineto{\pgfqpoint{0.708220in}{2.305275in}}%
\pgfusepath{stroke}%
\end{pgfscope}%
\begin{pgfscope}%
\pgfsetrectcap%
\pgfsetmiterjoin%
\pgfsetlinewidth{0.803000pt}%
\definecolor{currentstroke}{rgb}{0.000000,0.000000,0.000000}%
\pgfsetstrokecolor{currentstroke}%
\pgfsetdash{}{0pt}%
\pgfpathmoveto{\pgfqpoint{5.753646in}{0.535823in}}%
\pgfpathlineto{\pgfqpoint{5.753646in}{2.305275in}}%
\pgfusepath{stroke}%
\end{pgfscope}%
\begin{pgfscope}%
\pgfsetrectcap%
\pgfsetmiterjoin%
\pgfsetlinewidth{0.803000pt}%
\definecolor{currentstroke}{rgb}{0.000000,0.000000,0.000000}%
\pgfsetstrokecolor{currentstroke}%
\pgfsetdash{}{0pt}%
\pgfpathmoveto{\pgfqpoint{0.708220in}{0.535823in}}%
\pgfpathlineto{\pgfqpoint{5.753646in}{0.535823in}}%
\pgfusepath{stroke}%
\end{pgfscope}%
\begin{pgfscope}%
\pgfsetrectcap%
\pgfsetmiterjoin%
\pgfsetlinewidth{0.803000pt}%
\definecolor{currentstroke}{rgb}{0.000000,0.000000,0.000000}%
\pgfsetstrokecolor{currentstroke}%
\pgfsetdash{}{0pt}%
\pgfpathmoveto{\pgfqpoint{0.708220in}{2.305275in}}%
\pgfpathlineto{\pgfqpoint{5.753646in}{2.305275in}}%
\pgfusepath{stroke}%
\end{pgfscope}%
\begin{pgfscope}%
\pgfsetrectcap%
\pgfsetroundjoin%
\pgfsetlinewidth{1.003750pt}%
\definecolor{currentstroke}{rgb}{0.121569,0.466667,0.705882}%
\pgfsetstrokecolor{currentstroke}%
\pgfsetdash{}{0pt}%
\pgfpathmoveto{\pgfqpoint{4.846112in}{1.786469in}}%
\pgfpathlineto{\pgfqpoint{5.096112in}{1.786469in}}%
\pgfusepath{stroke}%
\end{pgfscope}%
\begin{pgfscope}%
\definecolor{textcolor}{rgb}{0.000000,0.000000,0.000000}%
\pgfsetstrokecolor{textcolor}%
\pgfsetfillcolor{textcolor}%
\pgftext[x=5.121112in,y=1.742719in,left,base]{\color{textcolor}\rmfamily\fontsize{9.000000}{10.800000}\selectfont MCS, first}%
\end{pgfscope}%
\begin{pgfscope}%
\pgfsetrectcap%
\pgfsetroundjoin%
\pgfsetlinewidth{1.003750pt}%
\definecolor{currentstroke}{rgb}{1.000000,0.498039,0.054902}%
\pgfsetstrokecolor{currentstroke}%
\pgfsetdash{}{0pt}%
\pgfpathmoveto{\pgfqpoint{4.846112in}{1.624670in}}%
\pgfpathlineto{\pgfqpoint{5.096112in}{1.624670in}}%
\pgfusepath{stroke}%
\end{pgfscope}%
\begin{pgfscope}%
\definecolor{textcolor}{rgb}{0.000000,0.000000,0.000000}%
\pgfsetstrokecolor{textcolor}%
\pgfsetfillcolor{textcolor}%
\pgftext[x=5.121112in,y=1.580920in,left,base]{\color{textcolor}\rmfamily\fontsize{9.000000}{10.800000}\selectfont LP, first}%
\end{pgfscope}%
\begin{pgfscope}%
\pgfsetbuttcap%
\pgfsetroundjoin%
\pgfsetlinewidth{1.003750pt}%
\definecolor{currentstroke}{rgb}{0.172549,0.627451,0.172549}%
\pgfsetstrokecolor{currentstroke}%
\pgfsetdash{{3.700000pt}{1.600000pt}}{0.000000pt}%
\pgfpathmoveto{\pgfqpoint{4.846112in}{1.462870in}}%
\pgfpathlineto{\pgfqpoint{5.096112in}{1.462870in}}%
\pgfusepath{stroke}%
\end{pgfscope}%
\begin{pgfscope}%
\definecolor{textcolor}{rgb}{0.000000,0.000000,0.000000}%
\pgfsetstrokecolor{textcolor}%
\pgfsetfillcolor{textcolor}%
\pgftext[x=5.121112in,y=1.419120in,left,base]{\color{textcolor}\rmfamily\fontsize{9.000000}{10.800000}\selectfont LM, first}%
\end{pgfscope}%
\begin{pgfscope}%
\pgfsetbuttcap%
\pgfsetroundjoin%
\pgfsetlinewidth{1.003750pt}%
\definecolor{currentstroke}{rgb}{0.839216,0.152941,0.156863}%
\pgfsetstrokecolor{currentstroke}%
\pgfsetdash{{3.700000pt}{1.600000pt}}{0.000000pt}%
\pgfpathmoveto{\pgfqpoint{4.846112in}{1.301071in}}%
\pgfpathlineto{\pgfqpoint{5.096112in}{1.301071in}}%
\pgfusepath{stroke}%
\end{pgfscope}%
\begin{pgfscope}%
\definecolor{textcolor}{rgb}{0.000000,0.000000,0.000000}%
\pgfsetstrokecolor{textcolor}%
\pgfsetfillcolor{textcolor}%
\pgftext[x=5.121112in,y=1.257321in,left,base]{\color{textcolor}\rmfamily\fontsize{9.000000}{10.800000}\selectfont MF, first}%
\end{pgfscope}%
\begin{pgfscope}%
\pgfsetbuttcap%
\pgfsetroundjoin%
\pgfsetlinewidth{1.003750pt}%
\definecolor{currentstroke}{rgb}{0.580392,0.403922,0.741176}%
\pgfsetstrokecolor{currentstroke}%
\pgfsetdash{{1.000000pt}{1.650000pt}}{0.000000pt}%
\pgfpathmoveto{\pgfqpoint{4.846112in}{1.139271in}}%
\pgfpathlineto{\pgfqpoint{5.096112in}{1.139271in}}%
\pgfusepath{stroke}%
\end{pgfscope}%
\begin{pgfscope}%
\definecolor{textcolor}{rgb}{0.000000,0.000000,0.000000}%
\pgfsetstrokecolor{textcolor}%
\pgfsetfillcolor{textcolor}%
\pgftext[x=5.121112in,y=1.095521in,left,base]{\color{textcolor}\rmfamily\fontsize{9.000000}{10.800000}\selectfont MCS, last}%
\end{pgfscope}%
\begin{pgfscope}%
\pgfsetbuttcap%
\pgfsetroundjoin%
\pgfsetlinewidth{1.003750pt}%
\definecolor{currentstroke}{rgb}{0.549020,0.337255,0.294118}%
\pgfsetstrokecolor{currentstroke}%
\pgfsetdash{{1.000000pt}{1.650000pt}}{0.000000pt}%
\pgfpathmoveto{\pgfqpoint{4.846112in}{0.977471in}}%
\pgfpathlineto{\pgfqpoint{5.096112in}{0.977471in}}%
\pgfusepath{stroke}%
\end{pgfscope}%
\begin{pgfscope}%
\definecolor{textcolor}{rgb}{0.000000,0.000000,0.000000}%
\pgfsetstrokecolor{textcolor}%
\pgfsetfillcolor{textcolor}%
\pgftext[x=5.121112in,y=0.933721in,left,base]{\color{textcolor}\rmfamily\fontsize{9.000000}{10.800000}\selectfont LP, last}%
\end{pgfscope}%
\begin{pgfscope}%
\pgfsetbuttcap%
\pgfsetroundjoin%
\pgfsetlinewidth{1.003750pt}%
\definecolor{currentstroke}{rgb}{0.890196,0.466667,0.760784}%
\pgfsetstrokecolor{currentstroke}%
\pgfsetdash{{6.400000pt}{1.600000pt}{1.000000pt}{1.600000pt}}{0.000000pt}%
\pgfpathmoveto{\pgfqpoint{4.846112in}{0.815672in}}%
\pgfpathlineto{\pgfqpoint{5.096112in}{0.815672in}}%
\pgfusepath{stroke}%
\end{pgfscope}%
\begin{pgfscope}%
\definecolor{textcolor}{rgb}{0.000000,0.000000,0.000000}%
\pgfsetstrokecolor{textcolor}%
\pgfsetfillcolor{textcolor}%
\pgftext[x=5.121112in,y=0.771922in,left,base]{\color{textcolor}\rmfamily\fontsize{9.000000}{10.800000}\selectfont LM, last}%
\end{pgfscope}%
\begin{pgfscope}%
\pgfsetbuttcap%
\pgfsetroundjoin%
\pgfsetlinewidth{1.003750pt}%
\definecolor{currentstroke}{rgb}{0.498039,0.498039,0.498039}%
\pgfsetstrokecolor{currentstroke}%
\pgfsetdash{{6.400000pt}{1.600000pt}{1.000000pt}{1.600000pt}}{0.000000pt}%
\pgfpathmoveto{\pgfqpoint{4.846112in}{0.653872in}}%
\pgfpathlineto{\pgfqpoint{5.096112in}{0.653872in}}%
\pgfusepath{stroke}%
\end{pgfscope}%
\begin{pgfscope}%
\definecolor{textcolor}{rgb}{0.000000,0.000000,0.000000}%
\pgfsetstrokecolor{textcolor}%
\pgfsetfillcolor{textcolor}%
\pgftext[x=5.121112in,y=0.610122in,left,base]{\color{textcolor}\rmfamily\fontsize{9.000000}{10.800000}\selectfont MF, last}%
\end{pgfscope}%
\end{pgfpicture}%
\makeatother%
\endgroup%

    \caption{
        Experiment 2 compares various variable-ordering heuristics (\mcs{}, \lexp, \lexm, and \minfill{}) for the ADD-based executor \dmc.
        The graded project-join trees here were produced by the planner \Lg{} with the tree decomposer \flowcutter{} from Experiment 1.
        \Lg{} is an anytime tool that may produce several trees of decreasing widths per benchmark.
        Nevertheless, across all four variable-ordering heuristics, there is little difference in execution time between using the first tree and using the last tree (planning time is excluded).
        % This behavior for \dmc{} has been observed in prior work \cite{dudek2020dpmc}.
    }
    \label{figExecutionA}
\end{figure}

%%%%%%%%%%%%%%%%%%%%%%%%%%%%%%%%%%%%%%%%%%%%%%%%%%%%%%%%%%%%%%%%%%%%%%%%%%%%%%%%

\subsection{Experiment 3: Comparing Weighted Projected Model Counters}

Figure \ref{figSolvingA} illustrates how six combinations of three \Lg{} tree-decomposition tools (\flowcutter{}, \htd, and \tamaki) and two \dmc{} variable-ordering heuristics (\mcs{} and \lexp) compare in 1000 seconds.
We exclude the planner \htb{} because it is slower than \Lg{} in Experiment 1.
We also exclude the variable-ordering heuristics \lexm{} and \minfill{} because they are slower than \mcs{} and \lexp{} in Experiment 2.
\begin{figure}[H]
    \centering
    %% Creator: Matplotlib, PGF backend
%%
%% To include the figure in your LaTeX document, write
%%   \input{<filename>.pgf}
%%
%% Make sure the required packages are loaded in your preamble
%%   \usepackage{pgf}
%%
%% and, on pdftex
%%   \usepackage[utf8]{inputenc}\DeclareUnicodeCharacter{2212}{-}
%%
%% or, on luatex and xetex
%%   \usepackage{unicode-math}
%%
%% Figures using additional raster images can only be included by \input if
%% they are in the same directory as the main LaTeX file. For loading figures
%% from other directories you can use the `import` package
%%   \usepackage{import}
%%
%% and then include the figures with
%%   \import{<path to file>}{<filename>.pgf}
%%
%% Matplotlib used the following preamble
%%   \usepackage[utf8x]{inputenc}
%%   \usepackage[T1]{fontenc}
%%
\begingroup%
\makeatletter%
\begin{pgfpicture}%
\pgfpathrectangle{\pgfpointorigin}{\pgfqpoint{6.000000in}{2.500000in}}%
\pgfusepath{use as bounding box, clip}%
\begin{pgfscope}%
\pgfsetbuttcap%
\pgfsetmiterjoin%
\definecolor{currentfill}{rgb}{1.000000,1.000000,1.000000}%
\pgfsetfillcolor{currentfill}%
\pgfsetlinewidth{0.000000pt}%
\definecolor{currentstroke}{rgb}{1.000000,1.000000,1.000000}%
\pgfsetstrokecolor{currentstroke}%
\pgfsetdash{}{0pt}%
\pgfpathmoveto{\pgfqpoint{0.000000in}{0.000000in}}%
\pgfpathlineto{\pgfqpoint{6.000000in}{0.000000in}}%
\pgfpathlineto{\pgfqpoint{6.000000in}{2.500000in}}%
\pgfpathlineto{\pgfqpoint{0.000000in}{2.500000in}}%
\pgfpathclose%
\pgfusepath{fill}%
\end{pgfscope}%
\begin{pgfscope}%
\pgfsetbuttcap%
\pgfsetmiterjoin%
\definecolor{currentfill}{rgb}{1.000000,1.000000,1.000000}%
\pgfsetfillcolor{currentfill}%
\pgfsetlinewidth{0.000000pt}%
\definecolor{currentstroke}{rgb}{0.000000,0.000000,0.000000}%
\pgfsetstrokecolor{currentstroke}%
\pgfsetstrokeopacity{0.000000}%
\pgfsetdash{}{0pt}%
\pgfpathmoveto{\pgfqpoint{0.708220in}{0.535823in}}%
\pgfpathlineto{\pgfqpoint{5.753646in}{0.535823in}}%
\pgfpathlineto{\pgfqpoint{5.753646in}{2.305275in}}%
\pgfpathlineto{\pgfqpoint{0.708220in}{2.305275in}}%
\pgfpathclose%
\pgfusepath{fill}%
\end{pgfscope}%
\begin{pgfscope}%
\pgfsetbuttcap%
\pgfsetroundjoin%
\definecolor{currentfill}{rgb}{0.000000,0.000000,0.000000}%
\pgfsetfillcolor{currentfill}%
\pgfsetlinewidth{0.803000pt}%
\definecolor{currentstroke}{rgb}{0.000000,0.000000,0.000000}%
\pgfsetstrokecolor{currentstroke}%
\pgfsetdash{}{0pt}%
\pgfsys@defobject{currentmarker}{\pgfqpoint{0.000000in}{-0.048611in}}{\pgfqpoint{0.000000in}{0.000000in}}{%
\pgfpathmoveto{\pgfqpoint{0.000000in}{0.000000in}}%
\pgfpathlineto{\pgfqpoint{0.000000in}{-0.048611in}}%
\pgfusepath{stroke,fill}%
}%
\begin{pgfscope}%
\pgfsys@transformshift{0.708220in}{0.535823in}%
\pgfsys@useobject{currentmarker}{}%
\end{pgfscope}%
\end{pgfscope}%
\begin{pgfscope}%
\definecolor{textcolor}{rgb}{0.000000,0.000000,0.000000}%
\pgfsetstrokecolor{textcolor}%
\pgfsetfillcolor{textcolor}%
\pgftext[x=0.708220in,y=0.438600in,,top]{\color{textcolor}\rmfamily\fontsize{9.000000}{10.800000}\selectfont \(\displaystyle {0}\)}%
\end{pgfscope}%
\begin{pgfscope}%
\pgfsetbuttcap%
\pgfsetroundjoin%
\definecolor{currentfill}{rgb}{0.000000,0.000000,0.000000}%
\pgfsetfillcolor{currentfill}%
\pgfsetlinewidth{0.803000pt}%
\definecolor{currentstroke}{rgb}{0.000000,0.000000,0.000000}%
\pgfsetstrokecolor{currentstroke}%
\pgfsetdash{}{0pt}%
\pgfsys@defobject{currentmarker}{\pgfqpoint{0.000000in}{-0.048611in}}{\pgfqpoint{0.000000in}{0.000000in}}{%
\pgfpathmoveto{\pgfqpoint{0.000000in}{0.000000in}}%
\pgfpathlineto{\pgfqpoint{0.000000in}{-0.048611in}}%
\pgfusepath{stroke,fill}%
}%
\begin{pgfscope}%
\pgfsys@transformshift{1.338898in}{0.535823in}%
\pgfsys@useobject{currentmarker}{}%
\end{pgfscope}%
\end{pgfscope}%
\begin{pgfscope}%
\definecolor{textcolor}{rgb}{0.000000,0.000000,0.000000}%
\pgfsetstrokecolor{textcolor}%
\pgfsetfillcolor{textcolor}%
\pgftext[x=1.338898in,y=0.438600in,,top]{\color{textcolor}\rmfamily\fontsize{9.000000}{10.800000}\selectfont \(\displaystyle {50}\)}%
\end{pgfscope}%
\begin{pgfscope}%
\pgfsetbuttcap%
\pgfsetroundjoin%
\definecolor{currentfill}{rgb}{0.000000,0.000000,0.000000}%
\pgfsetfillcolor{currentfill}%
\pgfsetlinewidth{0.803000pt}%
\definecolor{currentstroke}{rgb}{0.000000,0.000000,0.000000}%
\pgfsetstrokecolor{currentstroke}%
\pgfsetdash{}{0pt}%
\pgfsys@defobject{currentmarker}{\pgfqpoint{0.000000in}{-0.048611in}}{\pgfqpoint{0.000000in}{0.000000in}}{%
\pgfpathmoveto{\pgfqpoint{0.000000in}{0.000000in}}%
\pgfpathlineto{\pgfqpoint{0.000000in}{-0.048611in}}%
\pgfusepath{stroke,fill}%
}%
\begin{pgfscope}%
\pgfsys@transformshift{1.969577in}{0.535823in}%
\pgfsys@useobject{currentmarker}{}%
\end{pgfscope}%
\end{pgfscope}%
\begin{pgfscope}%
\definecolor{textcolor}{rgb}{0.000000,0.000000,0.000000}%
\pgfsetstrokecolor{textcolor}%
\pgfsetfillcolor{textcolor}%
\pgftext[x=1.969577in,y=0.438600in,,top]{\color{textcolor}\rmfamily\fontsize{9.000000}{10.800000}\selectfont \(\displaystyle {100}\)}%
\end{pgfscope}%
\begin{pgfscope}%
\pgfsetbuttcap%
\pgfsetroundjoin%
\definecolor{currentfill}{rgb}{0.000000,0.000000,0.000000}%
\pgfsetfillcolor{currentfill}%
\pgfsetlinewidth{0.803000pt}%
\definecolor{currentstroke}{rgb}{0.000000,0.000000,0.000000}%
\pgfsetstrokecolor{currentstroke}%
\pgfsetdash{}{0pt}%
\pgfsys@defobject{currentmarker}{\pgfqpoint{0.000000in}{-0.048611in}}{\pgfqpoint{0.000000in}{0.000000in}}{%
\pgfpathmoveto{\pgfqpoint{0.000000in}{0.000000in}}%
\pgfpathlineto{\pgfqpoint{0.000000in}{-0.048611in}}%
\pgfusepath{stroke,fill}%
}%
\begin{pgfscope}%
\pgfsys@transformshift{2.600255in}{0.535823in}%
\pgfsys@useobject{currentmarker}{}%
\end{pgfscope}%
\end{pgfscope}%
\begin{pgfscope}%
\definecolor{textcolor}{rgb}{0.000000,0.000000,0.000000}%
\pgfsetstrokecolor{textcolor}%
\pgfsetfillcolor{textcolor}%
\pgftext[x=2.600255in,y=0.438600in,,top]{\color{textcolor}\rmfamily\fontsize{9.000000}{10.800000}\selectfont \(\displaystyle {150}\)}%
\end{pgfscope}%
\begin{pgfscope}%
\pgfsetbuttcap%
\pgfsetroundjoin%
\definecolor{currentfill}{rgb}{0.000000,0.000000,0.000000}%
\pgfsetfillcolor{currentfill}%
\pgfsetlinewidth{0.803000pt}%
\definecolor{currentstroke}{rgb}{0.000000,0.000000,0.000000}%
\pgfsetstrokecolor{currentstroke}%
\pgfsetdash{}{0pt}%
\pgfsys@defobject{currentmarker}{\pgfqpoint{0.000000in}{-0.048611in}}{\pgfqpoint{0.000000in}{0.000000in}}{%
\pgfpathmoveto{\pgfqpoint{0.000000in}{0.000000in}}%
\pgfpathlineto{\pgfqpoint{0.000000in}{-0.048611in}}%
\pgfusepath{stroke,fill}%
}%
\begin{pgfscope}%
\pgfsys@transformshift{3.230933in}{0.535823in}%
\pgfsys@useobject{currentmarker}{}%
\end{pgfscope}%
\end{pgfscope}%
\begin{pgfscope}%
\definecolor{textcolor}{rgb}{0.000000,0.000000,0.000000}%
\pgfsetstrokecolor{textcolor}%
\pgfsetfillcolor{textcolor}%
\pgftext[x=3.230933in,y=0.438600in,,top]{\color{textcolor}\rmfamily\fontsize{9.000000}{10.800000}\selectfont \(\displaystyle {200}\)}%
\end{pgfscope}%
\begin{pgfscope}%
\pgfsetbuttcap%
\pgfsetroundjoin%
\definecolor{currentfill}{rgb}{0.000000,0.000000,0.000000}%
\pgfsetfillcolor{currentfill}%
\pgfsetlinewidth{0.803000pt}%
\definecolor{currentstroke}{rgb}{0.000000,0.000000,0.000000}%
\pgfsetstrokecolor{currentstroke}%
\pgfsetdash{}{0pt}%
\pgfsys@defobject{currentmarker}{\pgfqpoint{0.000000in}{-0.048611in}}{\pgfqpoint{0.000000in}{0.000000in}}{%
\pgfpathmoveto{\pgfqpoint{0.000000in}{0.000000in}}%
\pgfpathlineto{\pgfqpoint{0.000000in}{-0.048611in}}%
\pgfusepath{stroke,fill}%
}%
\begin{pgfscope}%
\pgfsys@transformshift{3.861611in}{0.535823in}%
\pgfsys@useobject{currentmarker}{}%
\end{pgfscope}%
\end{pgfscope}%
\begin{pgfscope}%
\definecolor{textcolor}{rgb}{0.000000,0.000000,0.000000}%
\pgfsetstrokecolor{textcolor}%
\pgfsetfillcolor{textcolor}%
\pgftext[x=3.861611in,y=0.438600in,,top]{\color{textcolor}\rmfamily\fontsize{9.000000}{10.800000}\selectfont \(\displaystyle {250}\)}%
\end{pgfscope}%
\begin{pgfscope}%
\pgfsetbuttcap%
\pgfsetroundjoin%
\definecolor{currentfill}{rgb}{0.000000,0.000000,0.000000}%
\pgfsetfillcolor{currentfill}%
\pgfsetlinewidth{0.803000pt}%
\definecolor{currentstroke}{rgb}{0.000000,0.000000,0.000000}%
\pgfsetstrokecolor{currentstroke}%
\pgfsetdash{}{0pt}%
\pgfsys@defobject{currentmarker}{\pgfqpoint{0.000000in}{-0.048611in}}{\pgfqpoint{0.000000in}{0.000000in}}{%
\pgfpathmoveto{\pgfqpoint{0.000000in}{0.000000in}}%
\pgfpathlineto{\pgfqpoint{0.000000in}{-0.048611in}}%
\pgfusepath{stroke,fill}%
}%
\begin{pgfscope}%
\pgfsys@transformshift{4.492290in}{0.535823in}%
\pgfsys@useobject{currentmarker}{}%
\end{pgfscope}%
\end{pgfscope}%
\begin{pgfscope}%
\definecolor{textcolor}{rgb}{0.000000,0.000000,0.000000}%
\pgfsetstrokecolor{textcolor}%
\pgfsetfillcolor{textcolor}%
\pgftext[x=4.492290in,y=0.438600in,,top]{\color{textcolor}\rmfamily\fontsize{9.000000}{10.800000}\selectfont \(\displaystyle {300}\)}%
\end{pgfscope}%
\begin{pgfscope}%
\pgfsetbuttcap%
\pgfsetroundjoin%
\definecolor{currentfill}{rgb}{0.000000,0.000000,0.000000}%
\pgfsetfillcolor{currentfill}%
\pgfsetlinewidth{0.803000pt}%
\definecolor{currentstroke}{rgb}{0.000000,0.000000,0.000000}%
\pgfsetstrokecolor{currentstroke}%
\pgfsetdash{}{0pt}%
\pgfsys@defobject{currentmarker}{\pgfqpoint{0.000000in}{-0.048611in}}{\pgfqpoint{0.000000in}{0.000000in}}{%
\pgfpathmoveto{\pgfqpoint{0.000000in}{0.000000in}}%
\pgfpathlineto{\pgfqpoint{0.000000in}{-0.048611in}}%
\pgfusepath{stroke,fill}%
}%
\begin{pgfscope}%
\pgfsys@transformshift{5.122968in}{0.535823in}%
\pgfsys@useobject{currentmarker}{}%
\end{pgfscope}%
\end{pgfscope}%
\begin{pgfscope}%
\definecolor{textcolor}{rgb}{0.000000,0.000000,0.000000}%
\pgfsetstrokecolor{textcolor}%
\pgfsetfillcolor{textcolor}%
\pgftext[x=5.122968in,y=0.438600in,,top]{\color{textcolor}\rmfamily\fontsize{9.000000}{10.800000}\selectfont \(\displaystyle {350}\)}%
\end{pgfscope}%
\begin{pgfscope}%
\pgfsetbuttcap%
\pgfsetroundjoin%
\definecolor{currentfill}{rgb}{0.000000,0.000000,0.000000}%
\pgfsetfillcolor{currentfill}%
\pgfsetlinewidth{0.803000pt}%
\definecolor{currentstroke}{rgb}{0.000000,0.000000,0.000000}%
\pgfsetstrokecolor{currentstroke}%
\pgfsetdash{}{0pt}%
\pgfsys@defobject{currentmarker}{\pgfqpoint{0.000000in}{-0.048611in}}{\pgfqpoint{0.000000in}{0.000000in}}{%
\pgfpathmoveto{\pgfqpoint{0.000000in}{0.000000in}}%
\pgfpathlineto{\pgfqpoint{0.000000in}{-0.048611in}}%
\pgfusepath{stroke,fill}%
}%
\begin{pgfscope}%
\pgfsys@transformshift{5.753646in}{0.535823in}%
\pgfsys@useobject{currentmarker}{}%
\end{pgfscope}%
\end{pgfscope}%
\begin{pgfscope}%
\definecolor{textcolor}{rgb}{0.000000,0.000000,0.000000}%
\pgfsetstrokecolor{textcolor}%
\pgfsetfillcolor{textcolor}%
\pgftext[x=5.753646in,y=0.438600in,,top]{\color{textcolor}\rmfamily\fontsize{9.000000}{10.800000}\selectfont \(\displaystyle {400}\)}%
\end{pgfscope}%
\begin{pgfscope}%
\definecolor{textcolor}{rgb}{0.000000,0.000000,0.000000}%
\pgfsetstrokecolor{textcolor}%
\pgfsetfillcolor{textcolor}%
\pgftext[x=3.230933in,y=0.272655in,,top]{\color{textcolor}\rmfamily\fontsize{10.000000}{12.000000}\selectfont Number of benchmarks solved}%
\end{pgfscope}%
\begin{pgfscope}%
\pgfsetbuttcap%
\pgfsetroundjoin%
\definecolor{currentfill}{rgb}{0.000000,0.000000,0.000000}%
\pgfsetfillcolor{currentfill}%
\pgfsetlinewidth{0.803000pt}%
\definecolor{currentstroke}{rgb}{0.000000,0.000000,0.000000}%
\pgfsetstrokecolor{currentstroke}%
\pgfsetdash{}{0pt}%
\pgfsys@defobject{currentmarker}{\pgfqpoint{-0.048611in}{0.000000in}}{\pgfqpoint{-0.000000in}{0.000000in}}{%
\pgfpathmoveto{\pgfqpoint{-0.000000in}{0.000000in}}%
\pgfpathlineto{\pgfqpoint{-0.048611in}{0.000000in}}%
\pgfusepath{stroke,fill}%
}%
\begin{pgfscope}%
\pgfsys@transformshift{0.708220in}{0.620358in}%
\pgfsys@useobject{currentmarker}{}%
\end{pgfscope}%
\end{pgfscope}%
\begin{pgfscope}%
\definecolor{textcolor}{rgb}{0.000000,0.000000,0.000000}%
\pgfsetstrokecolor{textcolor}%
\pgfsetfillcolor{textcolor}%
\pgftext[x=0.344411in, y=0.575633in, left, base]{\color{textcolor}\rmfamily\fontsize{9.000000}{10.800000}\selectfont \(\displaystyle {10^{-3}}\)}%
\end{pgfscope}%
\begin{pgfscope}%
\pgfsetbuttcap%
\pgfsetroundjoin%
\definecolor{currentfill}{rgb}{0.000000,0.000000,0.000000}%
\pgfsetfillcolor{currentfill}%
\pgfsetlinewidth{0.803000pt}%
\definecolor{currentstroke}{rgb}{0.000000,0.000000,0.000000}%
\pgfsetstrokecolor{currentstroke}%
\pgfsetdash{}{0pt}%
\pgfsys@defobject{currentmarker}{\pgfqpoint{-0.048611in}{0.000000in}}{\pgfqpoint{-0.000000in}{0.000000in}}{%
\pgfpathmoveto{\pgfqpoint{-0.000000in}{0.000000in}}%
\pgfpathlineto{\pgfqpoint{-0.048611in}{0.000000in}}%
\pgfusepath{stroke,fill}%
}%
\begin{pgfscope}%
\pgfsys@transformshift{0.708220in}{0.901177in}%
\pgfsys@useobject{currentmarker}{}%
\end{pgfscope}%
\end{pgfscope}%
\begin{pgfscope}%
\definecolor{textcolor}{rgb}{0.000000,0.000000,0.000000}%
\pgfsetstrokecolor{textcolor}%
\pgfsetfillcolor{textcolor}%
\pgftext[x=0.344411in, y=0.856453in, left, base]{\color{textcolor}\rmfamily\fontsize{9.000000}{10.800000}\selectfont \(\displaystyle {10^{-2}}\)}%
\end{pgfscope}%
\begin{pgfscope}%
\pgfsetbuttcap%
\pgfsetroundjoin%
\definecolor{currentfill}{rgb}{0.000000,0.000000,0.000000}%
\pgfsetfillcolor{currentfill}%
\pgfsetlinewidth{0.803000pt}%
\definecolor{currentstroke}{rgb}{0.000000,0.000000,0.000000}%
\pgfsetstrokecolor{currentstroke}%
\pgfsetdash{}{0pt}%
\pgfsys@defobject{currentmarker}{\pgfqpoint{-0.048611in}{0.000000in}}{\pgfqpoint{-0.000000in}{0.000000in}}{%
\pgfpathmoveto{\pgfqpoint{-0.000000in}{0.000000in}}%
\pgfpathlineto{\pgfqpoint{-0.048611in}{0.000000in}}%
\pgfusepath{stroke,fill}%
}%
\begin{pgfscope}%
\pgfsys@transformshift{0.708220in}{1.181997in}%
\pgfsys@useobject{currentmarker}{}%
\end{pgfscope}%
\end{pgfscope}%
\begin{pgfscope}%
\definecolor{textcolor}{rgb}{0.000000,0.000000,0.000000}%
\pgfsetstrokecolor{textcolor}%
\pgfsetfillcolor{textcolor}%
\pgftext[x=0.344411in, y=1.137272in, left, base]{\color{textcolor}\rmfamily\fontsize{9.000000}{10.800000}\selectfont \(\displaystyle {10^{-1}}\)}%
\end{pgfscope}%
\begin{pgfscope}%
\pgfsetbuttcap%
\pgfsetroundjoin%
\definecolor{currentfill}{rgb}{0.000000,0.000000,0.000000}%
\pgfsetfillcolor{currentfill}%
\pgfsetlinewidth{0.803000pt}%
\definecolor{currentstroke}{rgb}{0.000000,0.000000,0.000000}%
\pgfsetstrokecolor{currentstroke}%
\pgfsetdash{}{0pt}%
\pgfsys@defobject{currentmarker}{\pgfqpoint{-0.048611in}{0.000000in}}{\pgfqpoint{-0.000000in}{0.000000in}}{%
\pgfpathmoveto{\pgfqpoint{-0.000000in}{0.000000in}}%
\pgfpathlineto{\pgfqpoint{-0.048611in}{0.000000in}}%
\pgfusepath{stroke,fill}%
}%
\begin{pgfscope}%
\pgfsys@transformshift{0.708220in}{1.462816in}%
\pgfsys@useobject{currentmarker}{}%
\end{pgfscope}%
\end{pgfscope}%
\begin{pgfscope}%
\definecolor{textcolor}{rgb}{0.000000,0.000000,0.000000}%
\pgfsetstrokecolor{textcolor}%
\pgfsetfillcolor{textcolor}%
\pgftext[x=0.424657in, y=1.418092in, left, base]{\color{textcolor}\rmfamily\fontsize{9.000000}{10.800000}\selectfont \(\displaystyle {10^{0}}\)}%
\end{pgfscope}%
\begin{pgfscope}%
\pgfsetbuttcap%
\pgfsetroundjoin%
\definecolor{currentfill}{rgb}{0.000000,0.000000,0.000000}%
\pgfsetfillcolor{currentfill}%
\pgfsetlinewidth{0.803000pt}%
\definecolor{currentstroke}{rgb}{0.000000,0.000000,0.000000}%
\pgfsetstrokecolor{currentstroke}%
\pgfsetdash{}{0pt}%
\pgfsys@defobject{currentmarker}{\pgfqpoint{-0.048611in}{0.000000in}}{\pgfqpoint{-0.000000in}{0.000000in}}{%
\pgfpathmoveto{\pgfqpoint{-0.000000in}{0.000000in}}%
\pgfpathlineto{\pgfqpoint{-0.048611in}{0.000000in}}%
\pgfusepath{stroke,fill}%
}%
\begin{pgfscope}%
\pgfsys@transformshift{0.708220in}{1.743636in}%
\pgfsys@useobject{currentmarker}{}%
\end{pgfscope}%
\end{pgfscope}%
\begin{pgfscope}%
\definecolor{textcolor}{rgb}{0.000000,0.000000,0.000000}%
\pgfsetstrokecolor{textcolor}%
\pgfsetfillcolor{textcolor}%
\pgftext[x=0.424657in, y=1.698911in, left, base]{\color{textcolor}\rmfamily\fontsize{9.000000}{10.800000}\selectfont \(\displaystyle {10^{1}}\)}%
\end{pgfscope}%
\begin{pgfscope}%
\pgfsetbuttcap%
\pgfsetroundjoin%
\definecolor{currentfill}{rgb}{0.000000,0.000000,0.000000}%
\pgfsetfillcolor{currentfill}%
\pgfsetlinewidth{0.803000pt}%
\definecolor{currentstroke}{rgb}{0.000000,0.000000,0.000000}%
\pgfsetstrokecolor{currentstroke}%
\pgfsetdash{}{0pt}%
\pgfsys@defobject{currentmarker}{\pgfqpoint{-0.048611in}{0.000000in}}{\pgfqpoint{-0.000000in}{0.000000in}}{%
\pgfpathmoveto{\pgfqpoint{-0.000000in}{0.000000in}}%
\pgfpathlineto{\pgfqpoint{-0.048611in}{0.000000in}}%
\pgfusepath{stroke,fill}%
}%
\begin{pgfscope}%
\pgfsys@transformshift{0.708220in}{2.024456in}%
\pgfsys@useobject{currentmarker}{}%
\end{pgfscope}%
\end{pgfscope}%
\begin{pgfscope}%
\definecolor{textcolor}{rgb}{0.000000,0.000000,0.000000}%
\pgfsetstrokecolor{textcolor}%
\pgfsetfillcolor{textcolor}%
\pgftext[x=0.424657in, y=1.979731in, left, base]{\color{textcolor}\rmfamily\fontsize{9.000000}{10.800000}\selectfont \(\displaystyle {10^{2}}\)}%
\end{pgfscope}%
\begin{pgfscope}%
\pgfsetbuttcap%
\pgfsetroundjoin%
\definecolor{currentfill}{rgb}{0.000000,0.000000,0.000000}%
\pgfsetfillcolor{currentfill}%
\pgfsetlinewidth{0.803000pt}%
\definecolor{currentstroke}{rgb}{0.000000,0.000000,0.000000}%
\pgfsetstrokecolor{currentstroke}%
\pgfsetdash{}{0pt}%
\pgfsys@defobject{currentmarker}{\pgfqpoint{-0.048611in}{0.000000in}}{\pgfqpoint{-0.000000in}{0.000000in}}{%
\pgfpathmoveto{\pgfqpoint{-0.000000in}{0.000000in}}%
\pgfpathlineto{\pgfqpoint{-0.048611in}{0.000000in}}%
\pgfusepath{stroke,fill}%
}%
\begin{pgfscope}%
\pgfsys@transformshift{0.708220in}{2.305275in}%
\pgfsys@useobject{currentmarker}{}%
\end{pgfscope}%
\end{pgfscope}%
\begin{pgfscope}%
\definecolor{textcolor}{rgb}{0.000000,0.000000,0.000000}%
\pgfsetstrokecolor{textcolor}%
\pgfsetfillcolor{textcolor}%
\pgftext[x=0.424657in, y=2.260550in, left, base]{\color{textcolor}\rmfamily\fontsize{9.000000}{10.800000}\selectfont \(\displaystyle {10^{3}}\)}%
\end{pgfscope}%
\begin{pgfscope}%
\pgfsetbuttcap%
\pgfsetroundjoin%
\definecolor{currentfill}{rgb}{0.000000,0.000000,0.000000}%
\pgfsetfillcolor{currentfill}%
\pgfsetlinewidth{0.602250pt}%
\definecolor{currentstroke}{rgb}{0.000000,0.000000,0.000000}%
\pgfsetstrokecolor{currentstroke}%
\pgfsetdash{}{0pt}%
\pgfsys@defobject{currentmarker}{\pgfqpoint{-0.027778in}{0.000000in}}{\pgfqpoint{-0.000000in}{0.000000in}}{%
\pgfpathmoveto{\pgfqpoint{-0.000000in}{0.000000in}}%
\pgfpathlineto{\pgfqpoint{-0.027778in}{0.000000in}}%
\pgfusepath{stroke,fill}%
}%
\begin{pgfscope}%
\pgfsys@transformshift{0.708220in}{0.535823in}%
\pgfsys@useobject{currentmarker}{}%
\end{pgfscope}%
\end{pgfscope}%
\begin{pgfscope}%
\pgfsetbuttcap%
\pgfsetroundjoin%
\definecolor{currentfill}{rgb}{0.000000,0.000000,0.000000}%
\pgfsetfillcolor{currentfill}%
\pgfsetlinewidth{0.602250pt}%
\definecolor{currentstroke}{rgb}{0.000000,0.000000,0.000000}%
\pgfsetstrokecolor{currentstroke}%
\pgfsetdash{}{0pt}%
\pgfsys@defobject{currentmarker}{\pgfqpoint{-0.027778in}{0.000000in}}{\pgfqpoint{-0.000000in}{0.000000in}}{%
\pgfpathmoveto{\pgfqpoint{-0.000000in}{0.000000in}}%
\pgfpathlineto{\pgfqpoint{-0.027778in}{0.000000in}}%
\pgfusepath{stroke,fill}%
}%
\begin{pgfscope}%
\pgfsys@transformshift{0.708220in}{0.558058in}%
\pgfsys@useobject{currentmarker}{}%
\end{pgfscope}%
\end{pgfscope}%
\begin{pgfscope}%
\pgfsetbuttcap%
\pgfsetroundjoin%
\definecolor{currentfill}{rgb}{0.000000,0.000000,0.000000}%
\pgfsetfillcolor{currentfill}%
\pgfsetlinewidth{0.602250pt}%
\definecolor{currentstroke}{rgb}{0.000000,0.000000,0.000000}%
\pgfsetstrokecolor{currentstroke}%
\pgfsetdash{}{0pt}%
\pgfsys@defobject{currentmarker}{\pgfqpoint{-0.027778in}{0.000000in}}{\pgfqpoint{-0.000000in}{0.000000in}}{%
\pgfpathmoveto{\pgfqpoint{-0.000000in}{0.000000in}}%
\pgfpathlineto{\pgfqpoint{-0.027778in}{0.000000in}}%
\pgfusepath{stroke,fill}%
}%
\begin{pgfscope}%
\pgfsys@transformshift{0.708220in}{0.576858in}%
\pgfsys@useobject{currentmarker}{}%
\end{pgfscope}%
\end{pgfscope}%
\begin{pgfscope}%
\pgfsetbuttcap%
\pgfsetroundjoin%
\definecolor{currentfill}{rgb}{0.000000,0.000000,0.000000}%
\pgfsetfillcolor{currentfill}%
\pgfsetlinewidth{0.602250pt}%
\definecolor{currentstroke}{rgb}{0.000000,0.000000,0.000000}%
\pgfsetstrokecolor{currentstroke}%
\pgfsetdash{}{0pt}%
\pgfsys@defobject{currentmarker}{\pgfqpoint{-0.027778in}{0.000000in}}{\pgfqpoint{-0.000000in}{0.000000in}}{%
\pgfpathmoveto{\pgfqpoint{-0.000000in}{0.000000in}}%
\pgfpathlineto{\pgfqpoint{-0.027778in}{0.000000in}}%
\pgfusepath{stroke,fill}%
}%
\begin{pgfscope}%
\pgfsys@transformshift{0.708220in}{0.593144in}%
\pgfsys@useobject{currentmarker}{}%
\end{pgfscope}%
\end{pgfscope}%
\begin{pgfscope}%
\pgfsetbuttcap%
\pgfsetroundjoin%
\definecolor{currentfill}{rgb}{0.000000,0.000000,0.000000}%
\pgfsetfillcolor{currentfill}%
\pgfsetlinewidth{0.602250pt}%
\definecolor{currentstroke}{rgb}{0.000000,0.000000,0.000000}%
\pgfsetstrokecolor{currentstroke}%
\pgfsetdash{}{0pt}%
\pgfsys@defobject{currentmarker}{\pgfqpoint{-0.027778in}{0.000000in}}{\pgfqpoint{-0.000000in}{0.000000in}}{%
\pgfpathmoveto{\pgfqpoint{-0.000000in}{0.000000in}}%
\pgfpathlineto{\pgfqpoint{-0.027778in}{0.000000in}}%
\pgfusepath{stroke,fill}%
}%
\begin{pgfscope}%
\pgfsys@transformshift{0.708220in}{0.607508in}%
\pgfsys@useobject{currentmarker}{}%
\end{pgfscope}%
\end{pgfscope}%
\begin{pgfscope}%
\pgfsetbuttcap%
\pgfsetroundjoin%
\definecolor{currentfill}{rgb}{0.000000,0.000000,0.000000}%
\pgfsetfillcolor{currentfill}%
\pgfsetlinewidth{0.602250pt}%
\definecolor{currentstroke}{rgb}{0.000000,0.000000,0.000000}%
\pgfsetstrokecolor{currentstroke}%
\pgfsetdash{}{0pt}%
\pgfsys@defobject{currentmarker}{\pgfqpoint{-0.027778in}{0.000000in}}{\pgfqpoint{-0.000000in}{0.000000in}}{%
\pgfpathmoveto{\pgfqpoint{-0.000000in}{0.000000in}}%
\pgfpathlineto{\pgfqpoint{-0.027778in}{0.000000in}}%
\pgfusepath{stroke,fill}%
}%
\begin{pgfscope}%
\pgfsys@transformshift{0.708220in}{0.704893in}%
\pgfsys@useobject{currentmarker}{}%
\end{pgfscope}%
\end{pgfscope}%
\begin{pgfscope}%
\pgfsetbuttcap%
\pgfsetroundjoin%
\definecolor{currentfill}{rgb}{0.000000,0.000000,0.000000}%
\pgfsetfillcolor{currentfill}%
\pgfsetlinewidth{0.602250pt}%
\definecolor{currentstroke}{rgb}{0.000000,0.000000,0.000000}%
\pgfsetstrokecolor{currentstroke}%
\pgfsetdash{}{0pt}%
\pgfsys@defobject{currentmarker}{\pgfqpoint{-0.027778in}{0.000000in}}{\pgfqpoint{-0.000000in}{0.000000in}}{%
\pgfpathmoveto{\pgfqpoint{-0.000000in}{0.000000in}}%
\pgfpathlineto{\pgfqpoint{-0.027778in}{0.000000in}}%
\pgfusepath{stroke,fill}%
}%
\begin{pgfscope}%
\pgfsys@transformshift{0.708220in}{0.754343in}%
\pgfsys@useobject{currentmarker}{}%
\end{pgfscope}%
\end{pgfscope}%
\begin{pgfscope}%
\pgfsetbuttcap%
\pgfsetroundjoin%
\definecolor{currentfill}{rgb}{0.000000,0.000000,0.000000}%
\pgfsetfillcolor{currentfill}%
\pgfsetlinewidth{0.602250pt}%
\definecolor{currentstroke}{rgb}{0.000000,0.000000,0.000000}%
\pgfsetstrokecolor{currentstroke}%
\pgfsetdash{}{0pt}%
\pgfsys@defobject{currentmarker}{\pgfqpoint{-0.027778in}{0.000000in}}{\pgfqpoint{-0.000000in}{0.000000in}}{%
\pgfpathmoveto{\pgfqpoint{-0.000000in}{0.000000in}}%
\pgfpathlineto{\pgfqpoint{-0.027778in}{0.000000in}}%
\pgfusepath{stroke,fill}%
}%
\begin{pgfscope}%
\pgfsys@transformshift{0.708220in}{0.789428in}%
\pgfsys@useobject{currentmarker}{}%
\end{pgfscope}%
\end{pgfscope}%
\begin{pgfscope}%
\pgfsetbuttcap%
\pgfsetroundjoin%
\definecolor{currentfill}{rgb}{0.000000,0.000000,0.000000}%
\pgfsetfillcolor{currentfill}%
\pgfsetlinewidth{0.602250pt}%
\definecolor{currentstroke}{rgb}{0.000000,0.000000,0.000000}%
\pgfsetstrokecolor{currentstroke}%
\pgfsetdash{}{0pt}%
\pgfsys@defobject{currentmarker}{\pgfqpoint{-0.027778in}{0.000000in}}{\pgfqpoint{-0.000000in}{0.000000in}}{%
\pgfpathmoveto{\pgfqpoint{-0.000000in}{0.000000in}}%
\pgfpathlineto{\pgfqpoint{-0.027778in}{0.000000in}}%
\pgfusepath{stroke,fill}%
}%
\begin{pgfscope}%
\pgfsys@transformshift{0.708220in}{0.816642in}%
\pgfsys@useobject{currentmarker}{}%
\end{pgfscope}%
\end{pgfscope}%
\begin{pgfscope}%
\pgfsetbuttcap%
\pgfsetroundjoin%
\definecolor{currentfill}{rgb}{0.000000,0.000000,0.000000}%
\pgfsetfillcolor{currentfill}%
\pgfsetlinewidth{0.602250pt}%
\definecolor{currentstroke}{rgb}{0.000000,0.000000,0.000000}%
\pgfsetstrokecolor{currentstroke}%
\pgfsetdash{}{0pt}%
\pgfsys@defobject{currentmarker}{\pgfqpoint{-0.027778in}{0.000000in}}{\pgfqpoint{-0.000000in}{0.000000in}}{%
\pgfpathmoveto{\pgfqpoint{-0.000000in}{0.000000in}}%
\pgfpathlineto{\pgfqpoint{-0.027778in}{0.000000in}}%
\pgfusepath{stroke,fill}%
}%
\begin{pgfscope}%
\pgfsys@transformshift{0.708220in}{0.838878in}%
\pgfsys@useobject{currentmarker}{}%
\end{pgfscope}%
\end{pgfscope}%
\begin{pgfscope}%
\pgfsetbuttcap%
\pgfsetroundjoin%
\definecolor{currentfill}{rgb}{0.000000,0.000000,0.000000}%
\pgfsetfillcolor{currentfill}%
\pgfsetlinewidth{0.602250pt}%
\definecolor{currentstroke}{rgb}{0.000000,0.000000,0.000000}%
\pgfsetstrokecolor{currentstroke}%
\pgfsetdash{}{0pt}%
\pgfsys@defobject{currentmarker}{\pgfqpoint{-0.027778in}{0.000000in}}{\pgfqpoint{-0.000000in}{0.000000in}}{%
\pgfpathmoveto{\pgfqpoint{-0.000000in}{0.000000in}}%
\pgfpathlineto{\pgfqpoint{-0.027778in}{0.000000in}}%
\pgfusepath{stroke,fill}%
}%
\begin{pgfscope}%
\pgfsys@transformshift{0.708220in}{0.857678in}%
\pgfsys@useobject{currentmarker}{}%
\end{pgfscope}%
\end{pgfscope}%
\begin{pgfscope}%
\pgfsetbuttcap%
\pgfsetroundjoin%
\definecolor{currentfill}{rgb}{0.000000,0.000000,0.000000}%
\pgfsetfillcolor{currentfill}%
\pgfsetlinewidth{0.602250pt}%
\definecolor{currentstroke}{rgb}{0.000000,0.000000,0.000000}%
\pgfsetstrokecolor{currentstroke}%
\pgfsetdash{}{0pt}%
\pgfsys@defobject{currentmarker}{\pgfqpoint{-0.027778in}{0.000000in}}{\pgfqpoint{-0.000000in}{0.000000in}}{%
\pgfpathmoveto{\pgfqpoint{-0.000000in}{0.000000in}}%
\pgfpathlineto{\pgfqpoint{-0.027778in}{0.000000in}}%
\pgfusepath{stroke,fill}%
}%
\begin{pgfscope}%
\pgfsys@transformshift{0.708220in}{0.873963in}%
\pgfsys@useobject{currentmarker}{}%
\end{pgfscope}%
\end{pgfscope}%
\begin{pgfscope}%
\pgfsetbuttcap%
\pgfsetroundjoin%
\definecolor{currentfill}{rgb}{0.000000,0.000000,0.000000}%
\pgfsetfillcolor{currentfill}%
\pgfsetlinewidth{0.602250pt}%
\definecolor{currentstroke}{rgb}{0.000000,0.000000,0.000000}%
\pgfsetstrokecolor{currentstroke}%
\pgfsetdash{}{0pt}%
\pgfsys@defobject{currentmarker}{\pgfqpoint{-0.027778in}{0.000000in}}{\pgfqpoint{-0.000000in}{0.000000in}}{%
\pgfpathmoveto{\pgfqpoint{-0.000000in}{0.000000in}}%
\pgfpathlineto{\pgfqpoint{-0.027778in}{0.000000in}}%
\pgfusepath{stroke,fill}%
}%
\begin{pgfscope}%
\pgfsys@transformshift{0.708220in}{0.888328in}%
\pgfsys@useobject{currentmarker}{}%
\end{pgfscope}%
\end{pgfscope}%
\begin{pgfscope}%
\pgfsetbuttcap%
\pgfsetroundjoin%
\definecolor{currentfill}{rgb}{0.000000,0.000000,0.000000}%
\pgfsetfillcolor{currentfill}%
\pgfsetlinewidth{0.602250pt}%
\definecolor{currentstroke}{rgb}{0.000000,0.000000,0.000000}%
\pgfsetstrokecolor{currentstroke}%
\pgfsetdash{}{0pt}%
\pgfsys@defobject{currentmarker}{\pgfqpoint{-0.027778in}{0.000000in}}{\pgfqpoint{-0.000000in}{0.000000in}}{%
\pgfpathmoveto{\pgfqpoint{-0.000000in}{0.000000in}}%
\pgfpathlineto{\pgfqpoint{-0.027778in}{0.000000in}}%
\pgfusepath{stroke,fill}%
}%
\begin{pgfscope}%
\pgfsys@transformshift{0.708220in}{0.985712in}%
\pgfsys@useobject{currentmarker}{}%
\end{pgfscope}%
\end{pgfscope}%
\begin{pgfscope}%
\pgfsetbuttcap%
\pgfsetroundjoin%
\definecolor{currentfill}{rgb}{0.000000,0.000000,0.000000}%
\pgfsetfillcolor{currentfill}%
\pgfsetlinewidth{0.602250pt}%
\definecolor{currentstroke}{rgb}{0.000000,0.000000,0.000000}%
\pgfsetstrokecolor{currentstroke}%
\pgfsetdash{}{0pt}%
\pgfsys@defobject{currentmarker}{\pgfqpoint{-0.027778in}{0.000000in}}{\pgfqpoint{-0.000000in}{0.000000in}}{%
\pgfpathmoveto{\pgfqpoint{-0.000000in}{0.000000in}}%
\pgfpathlineto{\pgfqpoint{-0.027778in}{0.000000in}}%
\pgfusepath{stroke,fill}%
}%
\begin{pgfscope}%
\pgfsys@transformshift{0.708220in}{1.035162in}%
\pgfsys@useobject{currentmarker}{}%
\end{pgfscope}%
\end{pgfscope}%
\begin{pgfscope}%
\pgfsetbuttcap%
\pgfsetroundjoin%
\definecolor{currentfill}{rgb}{0.000000,0.000000,0.000000}%
\pgfsetfillcolor{currentfill}%
\pgfsetlinewidth{0.602250pt}%
\definecolor{currentstroke}{rgb}{0.000000,0.000000,0.000000}%
\pgfsetstrokecolor{currentstroke}%
\pgfsetdash{}{0pt}%
\pgfsys@defobject{currentmarker}{\pgfqpoint{-0.027778in}{0.000000in}}{\pgfqpoint{-0.000000in}{0.000000in}}{%
\pgfpathmoveto{\pgfqpoint{-0.000000in}{0.000000in}}%
\pgfpathlineto{\pgfqpoint{-0.027778in}{0.000000in}}%
\pgfusepath{stroke,fill}%
}%
\begin{pgfscope}%
\pgfsys@transformshift{0.708220in}{1.070248in}%
\pgfsys@useobject{currentmarker}{}%
\end{pgfscope}%
\end{pgfscope}%
\begin{pgfscope}%
\pgfsetbuttcap%
\pgfsetroundjoin%
\definecolor{currentfill}{rgb}{0.000000,0.000000,0.000000}%
\pgfsetfillcolor{currentfill}%
\pgfsetlinewidth{0.602250pt}%
\definecolor{currentstroke}{rgb}{0.000000,0.000000,0.000000}%
\pgfsetstrokecolor{currentstroke}%
\pgfsetdash{}{0pt}%
\pgfsys@defobject{currentmarker}{\pgfqpoint{-0.027778in}{0.000000in}}{\pgfqpoint{-0.000000in}{0.000000in}}{%
\pgfpathmoveto{\pgfqpoint{-0.000000in}{0.000000in}}%
\pgfpathlineto{\pgfqpoint{-0.027778in}{0.000000in}}%
\pgfusepath{stroke,fill}%
}%
\begin{pgfscope}%
\pgfsys@transformshift{0.708220in}{1.097462in}%
\pgfsys@useobject{currentmarker}{}%
\end{pgfscope}%
\end{pgfscope}%
\begin{pgfscope}%
\pgfsetbuttcap%
\pgfsetroundjoin%
\definecolor{currentfill}{rgb}{0.000000,0.000000,0.000000}%
\pgfsetfillcolor{currentfill}%
\pgfsetlinewidth{0.602250pt}%
\definecolor{currentstroke}{rgb}{0.000000,0.000000,0.000000}%
\pgfsetstrokecolor{currentstroke}%
\pgfsetdash{}{0pt}%
\pgfsys@defobject{currentmarker}{\pgfqpoint{-0.027778in}{0.000000in}}{\pgfqpoint{-0.000000in}{0.000000in}}{%
\pgfpathmoveto{\pgfqpoint{-0.000000in}{0.000000in}}%
\pgfpathlineto{\pgfqpoint{-0.027778in}{0.000000in}}%
\pgfusepath{stroke,fill}%
}%
\begin{pgfscope}%
\pgfsys@transformshift{0.708220in}{1.119697in}%
\pgfsys@useobject{currentmarker}{}%
\end{pgfscope}%
\end{pgfscope}%
\begin{pgfscope}%
\pgfsetbuttcap%
\pgfsetroundjoin%
\definecolor{currentfill}{rgb}{0.000000,0.000000,0.000000}%
\pgfsetfillcolor{currentfill}%
\pgfsetlinewidth{0.602250pt}%
\definecolor{currentstroke}{rgb}{0.000000,0.000000,0.000000}%
\pgfsetstrokecolor{currentstroke}%
\pgfsetdash{}{0pt}%
\pgfsys@defobject{currentmarker}{\pgfqpoint{-0.027778in}{0.000000in}}{\pgfqpoint{-0.000000in}{0.000000in}}{%
\pgfpathmoveto{\pgfqpoint{-0.000000in}{0.000000in}}%
\pgfpathlineto{\pgfqpoint{-0.027778in}{0.000000in}}%
\pgfusepath{stroke,fill}%
}%
\begin{pgfscope}%
\pgfsys@transformshift{0.708220in}{1.138497in}%
\pgfsys@useobject{currentmarker}{}%
\end{pgfscope}%
\end{pgfscope}%
\begin{pgfscope}%
\pgfsetbuttcap%
\pgfsetroundjoin%
\definecolor{currentfill}{rgb}{0.000000,0.000000,0.000000}%
\pgfsetfillcolor{currentfill}%
\pgfsetlinewidth{0.602250pt}%
\definecolor{currentstroke}{rgb}{0.000000,0.000000,0.000000}%
\pgfsetstrokecolor{currentstroke}%
\pgfsetdash{}{0pt}%
\pgfsys@defobject{currentmarker}{\pgfqpoint{-0.027778in}{0.000000in}}{\pgfqpoint{-0.000000in}{0.000000in}}{%
\pgfpathmoveto{\pgfqpoint{-0.000000in}{0.000000in}}%
\pgfpathlineto{\pgfqpoint{-0.027778in}{0.000000in}}%
\pgfusepath{stroke,fill}%
}%
\begin{pgfscope}%
\pgfsys@transformshift{0.708220in}{1.154783in}%
\pgfsys@useobject{currentmarker}{}%
\end{pgfscope}%
\end{pgfscope}%
\begin{pgfscope}%
\pgfsetbuttcap%
\pgfsetroundjoin%
\definecolor{currentfill}{rgb}{0.000000,0.000000,0.000000}%
\pgfsetfillcolor{currentfill}%
\pgfsetlinewidth{0.602250pt}%
\definecolor{currentstroke}{rgb}{0.000000,0.000000,0.000000}%
\pgfsetstrokecolor{currentstroke}%
\pgfsetdash{}{0pt}%
\pgfsys@defobject{currentmarker}{\pgfqpoint{-0.027778in}{0.000000in}}{\pgfqpoint{-0.000000in}{0.000000in}}{%
\pgfpathmoveto{\pgfqpoint{-0.000000in}{0.000000in}}%
\pgfpathlineto{\pgfqpoint{-0.027778in}{0.000000in}}%
\pgfusepath{stroke,fill}%
}%
\begin{pgfscope}%
\pgfsys@transformshift{0.708220in}{1.169147in}%
\pgfsys@useobject{currentmarker}{}%
\end{pgfscope}%
\end{pgfscope}%
\begin{pgfscope}%
\pgfsetbuttcap%
\pgfsetroundjoin%
\definecolor{currentfill}{rgb}{0.000000,0.000000,0.000000}%
\pgfsetfillcolor{currentfill}%
\pgfsetlinewidth{0.602250pt}%
\definecolor{currentstroke}{rgb}{0.000000,0.000000,0.000000}%
\pgfsetstrokecolor{currentstroke}%
\pgfsetdash{}{0pt}%
\pgfsys@defobject{currentmarker}{\pgfqpoint{-0.027778in}{0.000000in}}{\pgfqpoint{-0.000000in}{0.000000in}}{%
\pgfpathmoveto{\pgfqpoint{-0.000000in}{0.000000in}}%
\pgfpathlineto{\pgfqpoint{-0.027778in}{0.000000in}}%
\pgfusepath{stroke,fill}%
}%
\begin{pgfscope}%
\pgfsys@transformshift{0.708220in}{1.266532in}%
\pgfsys@useobject{currentmarker}{}%
\end{pgfscope}%
\end{pgfscope}%
\begin{pgfscope}%
\pgfsetbuttcap%
\pgfsetroundjoin%
\definecolor{currentfill}{rgb}{0.000000,0.000000,0.000000}%
\pgfsetfillcolor{currentfill}%
\pgfsetlinewidth{0.602250pt}%
\definecolor{currentstroke}{rgb}{0.000000,0.000000,0.000000}%
\pgfsetstrokecolor{currentstroke}%
\pgfsetdash{}{0pt}%
\pgfsys@defobject{currentmarker}{\pgfqpoint{-0.027778in}{0.000000in}}{\pgfqpoint{-0.000000in}{0.000000in}}{%
\pgfpathmoveto{\pgfqpoint{-0.000000in}{0.000000in}}%
\pgfpathlineto{\pgfqpoint{-0.027778in}{0.000000in}}%
\pgfusepath{stroke,fill}%
}%
\begin{pgfscope}%
\pgfsys@transformshift{0.708220in}{1.315982in}%
\pgfsys@useobject{currentmarker}{}%
\end{pgfscope}%
\end{pgfscope}%
\begin{pgfscope}%
\pgfsetbuttcap%
\pgfsetroundjoin%
\definecolor{currentfill}{rgb}{0.000000,0.000000,0.000000}%
\pgfsetfillcolor{currentfill}%
\pgfsetlinewidth{0.602250pt}%
\definecolor{currentstroke}{rgb}{0.000000,0.000000,0.000000}%
\pgfsetstrokecolor{currentstroke}%
\pgfsetdash{}{0pt}%
\pgfsys@defobject{currentmarker}{\pgfqpoint{-0.027778in}{0.000000in}}{\pgfqpoint{-0.000000in}{0.000000in}}{%
\pgfpathmoveto{\pgfqpoint{-0.000000in}{0.000000in}}%
\pgfpathlineto{\pgfqpoint{-0.027778in}{0.000000in}}%
\pgfusepath{stroke,fill}%
}%
\begin{pgfscope}%
\pgfsys@transformshift{0.708220in}{1.351067in}%
\pgfsys@useobject{currentmarker}{}%
\end{pgfscope}%
\end{pgfscope}%
\begin{pgfscope}%
\pgfsetbuttcap%
\pgfsetroundjoin%
\definecolor{currentfill}{rgb}{0.000000,0.000000,0.000000}%
\pgfsetfillcolor{currentfill}%
\pgfsetlinewidth{0.602250pt}%
\definecolor{currentstroke}{rgb}{0.000000,0.000000,0.000000}%
\pgfsetstrokecolor{currentstroke}%
\pgfsetdash{}{0pt}%
\pgfsys@defobject{currentmarker}{\pgfqpoint{-0.027778in}{0.000000in}}{\pgfqpoint{-0.000000in}{0.000000in}}{%
\pgfpathmoveto{\pgfqpoint{-0.000000in}{0.000000in}}%
\pgfpathlineto{\pgfqpoint{-0.027778in}{0.000000in}}%
\pgfusepath{stroke,fill}%
}%
\begin{pgfscope}%
\pgfsys@transformshift{0.708220in}{1.378281in}%
\pgfsys@useobject{currentmarker}{}%
\end{pgfscope}%
\end{pgfscope}%
\begin{pgfscope}%
\pgfsetbuttcap%
\pgfsetroundjoin%
\definecolor{currentfill}{rgb}{0.000000,0.000000,0.000000}%
\pgfsetfillcolor{currentfill}%
\pgfsetlinewidth{0.602250pt}%
\definecolor{currentstroke}{rgb}{0.000000,0.000000,0.000000}%
\pgfsetstrokecolor{currentstroke}%
\pgfsetdash{}{0pt}%
\pgfsys@defobject{currentmarker}{\pgfqpoint{-0.027778in}{0.000000in}}{\pgfqpoint{-0.000000in}{0.000000in}}{%
\pgfpathmoveto{\pgfqpoint{-0.000000in}{0.000000in}}%
\pgfpathlineto{\pgfqpoint{-0.027778in}{0.000000in}}%
\pgfusepath{stroke,fill}%
}%
\begin{pgfscope}%
\pgfsys@transformshift{0.708220in}{1.400517in}%
\pgfsys@useobject{currentmarker}{}%
\end{pgfscope}%
\end{pgfscope}%
\begin{pgfscope}%
\pgfsetbuttcap%
\pgfsetroundjoin%
\definecolor{currentfill}{rgb}{0.000000,0.000000,0.000000}%
\pgfsetfillcolor{currentfill}%
\pgfsetlinewidth{0.602250pt}%
\definecolor{currentstroke}{rgb}{0.000000,0.000000,0.000000}%
\pgfsetstrokecolor{currentstroke}%
\pgfsetdash{}{0pt}%
\pgfsys@defobject{currentmarker}{\pgfqpoint{-0.027778in}{0.000000in}}{\pgfqpoint{-0.000000in}{0.000000in}}{%
\pgfpathmoveto{\pgfqpoint{-0.000000in}{0.000000in}}%
\pgfpathlineto{\pgfqpoint{-0.027778in}{0.000000in}}%
\pgfusepath{stroke,fill}%
}%
\begin{pgfscope}%
\pgfsys@transformshift{0.708220in}{1.419317in}%
\pgfsys@useobject{currentmarker}{}%
\end{pgfscope}%
\end{pgfscope}%
\begin{pgfscope}%
\pgfsetbuttcap%
\pgfsetroundjoin%
\definecolor{currentfill}{rgb}{0.000000,0.000000,0.000000}%
\pgfsetfillcolor{currentfill}%
\pgfsetlinewidth{0.602250pt}%
\definecolor{currentstroke}{rgb}{0.000000,0.000000,0.000000}%
\pgfsetstrokecolor{currentstroke}%
\pgfsetdash{}{0pt}%
\pgfsys@defobject{currentmarker}{\pgfqpoint{-0.027778in}{0.000000in}}{\pgfqpoint{-0.000000in}{0.000000in}}{%
\pgfpathmoveto{\pgfqpoint{-0.000000in}{0.000000in}}%
\pgfpathlineto{\pgfqpoint{-0.027778in}{0.000000in}}%
\pgfusepath{stroke,fill}%
}%
\begin{pgfscope}%
\pgfsys@transformshift{0.708220in}{1.435602in}%
\pgfsys@useobject{currentmarker}{}%
\end{pgfscope}%
\end{pgfscope}%
\begin{pgfscope}%
\pgfsetbuttcap%
\pgfsetroundjoin%
\definecolor{currentfill}{rgb}{0.000000,0.000000,0.000000}%
\pgfsetfillcolor{currentfill}%
\pgfsetlinewidth{0.602250pt}%
\definecolor{currentstroke}{rgb}{0.000000,0.000000,0.000000}%
\pgfsetstrokecolor{currentstroke}%
\pgfsetdash{}{0pt}%
\pgfsys@defobject{currentmarker}{\pgfqpoint{-0.027778in}{0.000000in}}{\pgfqpoint{-0.000000in}{0.000000in}}{%
\pgfpathmoveto{\pgfqpoint{-0.000000in}{0.000000in}}%
\pgfpathlineto{\pgfqpoint{-0.027778in}{0.000000in}}%
\pgfusepath{stroke,fill}%
}%
\begin{pgfscope}%
\pgfsys@transformshift{0.708220in}{1.449967in}%
\pgfsys@useobject{currentmarker}{}%
\end{pgfscope}%
\end{pgfscope}%
\begin{pgfscope}%
\pgfsetbuttcap%
\pgfsetroundjoin%
\definecolor{currentfill}{rgb}{0.000000,0.000000,0.000000}%
\pgfsetfillcolor{currentfill}%
\pgfsetlinewidth{0.602250pt}%
\definecolor{currentstroke}{rgb}{0.000000,0.000000,0.000000}%
\pgfsetstrokecolor{currentstroke}%
\pgfsetdash{}{0pt}%
\pgfsys@defobject{currentmarker}{\pgfqpoint{-0.027778in}{0.000000in}}{\pgfqpoint{-0.000000in}{0.000000in}}{%
\pgfpathmoveto{\pgfqpoint{-0.000000in}{0.000000in}}%
\pgfpathlineto{\pgfqpoint{-0.027778in}{0.000000in}}%
\pgfusepath{stroke,fill}%
}%
\begin{pgfscope}%
\pgfsys@transformshift{0.708220in}{1.547352in}%
\pgfsys@useobject{currentmarker}{}%
\end{pgfscope}%
\end{pgfscope}%
\begin{pgfscope}%
\pgfsetbuttcap%
\pgfsetroundjoin%
\definecolor{currentfill}{rgb}{0.000000,0.000000,0.000000}%
\pgfsetfillcolor{currentfill}%
\pgfsetlinewidth{0.602250pt}%
\definecolor{currentstroke}{rgb}{0.000000,0.000000,0.000000}%
\pgfsetstrokecolor{currentstroke}%
\pgfsetdash{}{0pt}%
\pgfsys@defobject{currentmarker}{\pgfqpoint{-0.027778in}{0.000000in}}{\pgfqpoint{-0.000000in}{0.000000in}}{%
\pgfpathmoveto{\pgfqpoint{-0.000000in}{0.000000in}}%
\pgfpathlineto{\pgfqpoint{-0.027778in}{0.000000in}}%
\pgfusepath{stroke,fill}%
}%
\begin{pgfscope}%
\pgfsys@transformshift{0.708220in}{1.596801in}%
\pgfsys@useobject{currentmarker}{}%
\end{pgfscope}%
\end{pgfscope}%
\begin{pgfscope}%
\pgfsetbuttcap%
\pgfsetroundjoin%
\definecolor{currentfill}{rgb}{0.000000,0.000000,0.000000}%
\pgfsetfillcolor{currentfill}%
\pgfsetlinewidth{0.602250pt}%
\definecolor{currentstroke}{rgb}{0.000000,0.000000,0.000000}%
\pgfsetstrokecolor{currentstroke}%
\pgfsetdash{}{0pt}%
\pgfsys@defobject{currentmarker}{\pgfqpoint{-0.027778in}{0.000000in}}{\pgfqpoint{-0.000000in}{0.000000in}}{%
\pgfpathmoveto{\pgfqpoint{-0.000000in}{0.000000in}}%
\pgfpathlineto{\pgfqpoint{-0.027778in}{0.000000in}}%
\pgfusepath{stroke,fill}%
}%
\begin{pgfscope}%
\pgfsys@transformshift{0.708220in}{1.631887in}%
\pgfsys@useobject{currentmarker}{}%
\end{pgfscope}%
\end{pgfscope}%
\begin{pgfscope}%
\pgfsetbuttcap%
\pgfsetroundjoin%
\definecolor{currentfill}{rgb}{0.000000,0.000000,0.000000}%
\pgfsetfillcolor{currentfill}%
\pgfsetlinewidth{0.602250pt}%
\definecolor{currentstroke}{rgb}{0.000000,0.000000,0.000000}%
\pgfsetstrokecolor{currentstroke}%
\pgfsetdash{}{0pt}%
\pgfsys@defobject{currentmarker}{\pgfqpoint{-0.027778in}{0.000000in}}{\pgfqpoint{-0.000000in}{0.000000in}}{%
\pgfpathmoveto{\pgfqpoint{-0.000000in}{0.000000in}}%
\pgfpathlineto{\pgfqpoint{-0.027778in}{0.000000in}}%
\pgfusepath{stroke,fill}%
}%
\begin{pgfscope}%
\pgfsys@transformshift{0.708220in}{1.659101in}%
\pgfsys@useobject{currentmarker}{}%
\end{pgfscope}%
\end{pgfscope}%
\begin{pgfscope}%
\pgfsetbuttcap%
\pgfsetroundjoin%
\definecolor{currentfill}{rgb}{0.000000,0.000000,0.000000}%
\pgfsetfillcolor{currentfill}%
\pgfsetlinewidth{0.602250pt}%
\definecolor{currentstroke}{rgb}{0.000000,0.000000,0.000000}%
\pgfsetstrokecolor{currentstroke}%
\pgfsetdash{}{0pt}%
\pgfsys@defobject{currentmarker}{\pgfqpoint{-0.027778in}{0.000000in}}{\pgfqpoint{-0.000000in}{0.000000in}}{%
\pgfpathmoveto{\pgfqpoint{-0.000000in}{0.000000in}}%
\pgfpathlineto{\pgfqpoint{-0.027778in}{0.000000in}}%
\pgfusepath{stroke,fill}%
}%
\begin{pgfscope}%
\pgfsys@transformshift{0.708220in}{1.681337in}%
\pgfsys@useobject{currentmarker}{}%
\end{pgfscope}%
\end{pgfscope}%
\begin{pgfscope}%
\pgfsetbuttcap%
\pgfsetroundjoin%
\definecolor{currentfill}{rgb}{0.000000,0.000000,0.000000}%
\pgfsetfillcolor{currentfill}%
\pgfsetlinewidth{0.602250pt}%
\definecolor{currentstroke}{rgb}{0.000000,0.000000,0.000000}%
\pgfsetstrokecolor{currentstroke}%
\pgfsetdash{}{0pt}%
\pgfsys@defobject{currentmarker}{\pgfqpoint{-0.027778in}{0.000000in}}{\pgfqpoint{-0.000000in}{0.000000in}}{%
\pgfpathmoveto{\pgfqpoint{-0.000000in}{0.000000in}}%
\pgfpathlineto{\pgfqpoint{-0.027778in}{0.000000in}}%
\pgfusepath{stroke,fill}%
}%
\begin{pgfscope}%
\pgfsys@transformshift{0.708220in}{1.700137in}%
\pgfsys@useobject{currentmarker}{}%
\end{pgfscope}%
\end{pgfscope}%
\begin{pgfscope}%
\pgfsetbuttcap%
\pgfsetroundjoin%
\definecolor{currentfill}{rgb}{0.000000,0.000000,0.000000}%
\pgfsetfillcolor{currentfill}%
\pgfsetlinewidth{0.602250pt}%
\definecolor{currentstroke}{rgb}{0.000000,0.000000,0.000000}%
\pgfsetstrokecolor{currentstroke}%
\pgfsetdash{}{0pt}%
\pgfsys@defobject{currentmarker}{\pgfqpoint{-0.027778in}{0.000000in}}{\pgfqpoint{-0.000000in}{0.000000in}}{%
\pgfpathmoveto{\pgfqpoint{-0.000000in}{0.000000in}}%
\pgfpathlineto{\pgfqpoint{-0.027778in}{0.000000in}}%
\pgfusepath{stroke,fill}%
}%
\begin{pgfscope}%
\pgfsys@transformshift{0.708220in}{1.716422in}%
\pgfsys@useobject{currentmarker}{}%
\end{pgfscope}%
\end{pgfscope}%
\begin{pgfscope}%
\pgfsetbuttcap%
\pgfsetroundjoin%
\definecolor{currentfill}{rgb}{0.000000,0.000000,0.000000}%
\pgfsetfillcolor{currentfill}%
\pgfsetlinewidth{0.602250pt}%
\definecolor{currentstroke}{rgb}{0.000000,0.000000,0.000000}%
\pgfsetstrokecolor{currentstroke}%
\pgfsetdash{}{0pt}%
\pgfsys@defobject{currentmarker}{\pgfqpoint{-0.027778in}{0.000000in}}{\pgfqpoint{-0.000000in}{0.000000in}}{%
\pgfpathmoveto{\pgfqpoint{-0.000000in}{0.000000in}}%
\pgfpathlineto{\pgfqpoint{-0.027778in}{0.000000in}}%
\pgfusepath{stroke,fill}%
}%
\begin{pgfscope}%
\pgfsys@transformshift{0.708220in}{1.730786in}%
\pgfsys@useobject{currentmarker}{}%
\end{pgfscope}%
\end{pgfscope}%
\begin{pgfscope}%
\pgfsetbuttcap%
\pgfsetroundjoin%
\definecolor{currentfill}{rgb}{0.000000,0.000000,0.000000}%
\pgfsetfillcolor{currentfill}%
\pgfsetlinewidth{0.602250pt}%
\definecolor{currentstroke}{rgb}{0.000000,0.000000,0.000000}%
\pgfsetstrokecolor{currentstroke}%
\pgfsetdash{}{0pt}%
\pgfsys@defobject{currentmarker}{\pgfqpoint{-0.027778in}{0.000000in}}{\pgfqpoint{-0.000000in}{0.000000in}}{%
\pgfpathmoveto{\pgfqpoint{-0.000000in}{0.000000in}}%
\pgfpathlineto{\pgfqpoint{-0.027778in}{0.000000in}}%
\pgfusepath{stroke,fill}%
}%
\begin{pgfscope}%
\pgfsys@transformshift{0.708220in}{1.828171in}%
\pgfsys@useobject{currentmarker}{}%
\end{pgfscope}%
\end{pgfscope}%
\begin{pgfscope}%
\pgfsetbuttcap%
\pgfsetroundjoin%
\definecolor{currentfill}{rgb}{0.000000,0.000000,0.000000}%
\pgfsetfillcolor{currentfill}%
\pgfsetlinewidth{0.602250pt}%
\definecolor{currentstroke}{rgb}{0.000000,0.000000,0.000000}%
\pgfsetstrokecolor{currentstroke}%
\pgfsetdash{}{0pt}%
\pgfsys@defobject{currentmarker}{\pgfqpoint{-0.027778in}{0.000000in}}{\pgfqpoint{-0.000000in}{0.000000in}}{%
\pgfpathmoveto{\pgfqpoint{-0.000000in}{0.000000in}}%
\pgfpathlineto{\pgfqpoint{-0.027778in}{0.000000in}}%
\pgfusepath{stroke,fill}%
}%
\begin{pgfscope}%
\pgfsys@transformshift{0.708220in}{1.877621in}%
\pgfsys@useobject{currentmarker}{}%
\end{pgfscope}%
\end{pgfscope}%
\begin{pgfscope}%
\pgfsetbuttcap%
\pgfsetroundjoin%
\definecolor{currentfill}{rgb}{0.000000,0.000000,0.000000}%
\pgfsetfillcolor{currentfill}%
\pgfsetlinewidth{0.602250pt}%
\definecolor{currentstroke}{rgb}{0.000000,0.000000,0.000000}%
\pgfsetstrokecolor{currentstroke}%
\pgfsetdash{}{0pt}%
\pgfsys@defobject{currentmarker}{\pgfqpoint{-0.027778in}{0.000000in}}{\pgfqpoint{-0.000000in}{0.000000in}}{%
\pgfpathmoveto{\pgfqpoint{-0.000000in}{0.000000in}}%
\pgfpathlineto{\pgfqpoint{-0.027778in}{0.000000in}}%
\pgfusepath{stroke,fill}%
}%
\begin{pgfscope}%
\pgfsys@transformshift{0.708220in}{1.912706in}%
\pgfsys@useobject{currentmarker}{}%
\end{pgfscope}%
\end{pgfscope}%
\begin{pgfscope}%
\pgfsetbuttcap%
\pgfsetroundjoin%
\definecolor{currentfill}{rgb}{0.000000,0.000000,0.000000}%
\pgfsetfillcolor{currentfill}%
\pgfsetlinewidth{0.602250pt}%
\definecolor{currentstroke}{rgb}{0.000000,0.000000,0.000000}%
\pgfsetstrokecolor{currentstroke}%
\pgfsetdash{}{0pt}%
\pgfsys@defobject{currentmarker}{\pgfqpoint{-0.027778in}{0.000000in}}{\pgfqpoint{-0.000000in}{0.000000in}}{%
\pgfpathmoveto{\pgfqpoint{-0.000000in}{0.000000in}}%
\pgfpathlineto{\pgfqpoint{-0.027778in}{0.000000in}}%
\pgfusepath{stroke,fill}%
}%
\begin{pgfscope}%
\pgfsys@transformshift{0.708220in}{1.939921in}%
\pgfsys@useobject{currentmarker}{}%
\end{pgfscope}%
\end{pgfscope}%
\begin{pgfscope}%
\pgfsetbuttcap%
\pgfsetroundjoin%
\definecolor{currentfill}{rgb}{0.000000,0.000000,0.000000}%
\pgfsetfillcolor{currentfill}%
\pgfsetlinewidth{0.602250pt}%
\definecolor{currentstroke}{rgb}{0.000000,0.000000,0.000000}%
\pgfsetstrokecolor{currentstroke}%
\pgfsetdash{}{0pt}%
\pgfsys@defobject{currentmarker}{\pgfqpoint{-0.027778in}{0.000000in}}{\pgfqpoint{-0.000000in}{0.000000in}}{%
\pgfpathmoveto{\pgfqpoint{-0.000000in}{0.000000in}}%
\pgfpathlineto{\pgfqpoint{-0.027778in}{0.000000in}}%
\pgfusepath{stroke,fill}%
}%
\begin{pgfscope}%
\pgfsys@transformshift{0.708220in}{1.962156in}%
\pgfsys@useobject{currentmarker}{}%
\end{pgfscope}%
\end{pgfscope}%
\begin{pgfscope}%
\pgfsetbuttcap%
\pgfsetroundjoin%
\definecolor{currentfill}{rgb}{0.000000,0.000000,0.000000}%
\pgfsetfillcolor{currentfill}%
\pgfsetlinewidth{0.602250pt}%
\definecolor{currentstroke}{rgb}{0.000000,0.000000,0.000000}%
\pgfsetstrokecolor{currentstroke}%
\pgfsetdash{}{0pt}%
\pgfsys@defobject{currentmarker}{\pgfqpoint{-0.027778in}{0.000000in}}{\pgfqpoint{-0.000000in}{0.000000in}}{%
\pgfpathmoveto{\pgfqpoint{-0.000000in}{0.000000in}}%
\pgfpathlineto{\pgfqpoint{-0.027778in}{0.000000in}}%
\pgfusepath{stroke,fill}%
}%
\begin{pgfscope}%
\pgfsys@transformshift{0.708220in}{1.980956in}%
\pgfsys@useobject{currentmarker}{}%
\end{pgfscope}%
\end{pgfscope}%
\begin{pgfscope}%
\pgfsetbuttcap%
\pgfsetroundjoin%
\definecolor{currentfill}{rgb}{0.000000,0.000000,0.000000}%
\pgfsetfillcolor{currentfill}%
\pgfsetlinewidth{0.602250pt}%
\definecolor{currentstroke}{rgb}{0.000000,0.000000,0.000000}%
\pgfsetstrokecolor{currentstroke}%
\pgfsetdash{}{0pt}%
\pgfsys@defobject{currentmarker}{\pgfqpoint{-0.027778in}{0.000000in}}{\pgfqpoint{-0.000000in}{0.000000in}}{%
\pgfpathmoveto{\pgfqpoint{-0.000000in}{0.000000in}}%
\pgfpathlineto{\pgfqpoint{-0.027778in}{0.000000in}}%
\pgfusepath{stroke,fill}%
}%
\begin{pgfscope}%
\pgfsys@transformshift{0.708220in}{1.997241in}%
\pgfsys@useobject{currentmarker}{}%
\end{pgfscope}%
\end{pgfscope}%
\begin{pgfscope}%
\pgfsetbuttcap%
\pgfsetroundjoin%
\definecolor{currentfill}{rgb}{0.000000,0.000000,0.000000}%
\pgfsetfillcolor{currentfill}%
\pgfsetlinewidth{0.602250pt}%
\definecolor{currentstroke}{rgb}{0.000000,0.000000,0.000000}%
\pgfsetstrokecolor{currentstroke}%
\pgfsetdash{}{0pt}%
\pgfsys@defobject{currentmarker}{\pgfqpoint{-0.027778in}{0.000000in}}{\pgfqpoint{-0.000000in}{0.000000in}}{%
\pgfpathmoveto{\pgfqpoint{-0.000000in}{0.000000in}}%
\pgfpathlineto{\pgfqpoint{-0.027778in}{0.000000in}}%
\pgfusepath{stroke,fill}%
}%
\begin{pgfscope}%
\pgfsys@transformshift{0.708220in}{2.011606in}%
\pgfsys@useobject{currentmarker}{}%
\end{pgfscope}%
\end{pgfscope}%
\begin{pgfscope}%
\pgfsetbuttcap%
\pgfsetroundjoin%
\definecolor{currentfill}{rgb}{0.000000,0.000000,0.000000}%
\pgfsetfillcolor{currentfill}%
\pgfsetlinewidth{0.602250pt}%
\definecolor{currentstroke}{rgb}{0.000000,0.000000,0.000000}%
\pgfsetstrokecolor{currentstroke}%
\pgfsetdash{}{0pt}%
\pgfsys@defobject{currentmarker}{\pgfqpoint{-0.027778in}{0.000000in}}{\pgfqpoint{-0.000000in}{0.000000in}}{%
\pgfpathmoveto{\pgfqpoint{-0.000000in}{0.000000in}}%
\pgfpathlineto{\pgfqpoint{-0.027778in}{0.000000in}}%
\pgfusepath{stroke,fill}%
}%
\begin{pgfscope}%
\pgfsys@transformshift{0.708220in}{2.108991in}%
\pgfsys@useobject{currentmarker}{}%
\end{pgfscope}%
\end{pgfscope}%
\begin{pgfscope}%
\pgfsetbuttcap%
\pgfsetroundjoin%
\definecolor{currentfill}{rgb}{0.000000,0.000000,0.000000}%
\pgfsetfillcolor{currentfill}%
\pgfsetlinewidth{0.602250pt}%
\definecolor{currentstroke}{rgb}{0.000000,0.000000,0.000000}%
\pgfsetstrokecolor{currentstroke}%
\pgfsetdash{}{0pt}%
\pgfsys@defobject{currentmarker}{\pgfqpoint{-0.027778in}{0.000000in}}{\pgfqpoint{-0.000000in}{0.000000in}}{%
\pgfpathmoveto{\pgfqpoint{-0.000000in}{0.000000in}}%
\pgfpathlineto{\pgfqpoint{-0.027778in}{0.000000in}}%
\pgfusepath{stroke,fill}%
}%
\begin{pgfscope}%
\pgfsys@transformshift{0.708220in}{2.158441in}%
\pgfsys@useobject{currentmarker}{}%
\end{pgfscope}%
\end{pgfscope}%
\begin{pgfscope}%
\pgfsetbuttcap%
\pgfsetroundjoin%
\definecolor{currentfill}{rgb}{0.000000,0.000000,0.000000}%
\pgfsetfillcolor{currentfill}%
\pgfsetlinewidth{0.602250pt}%
\definecolor{currentstroke}{rgb}{0.000000,0.000000,0.000000}%
\pgfsetstrokecolor{currentstroke}%
\pgfsetdash{}{0pt}%
\pgfsys@defobject{currentmarker}{\pgfqpoint{-0.027778in}{0.000000in}}{\pgfqpoint{-0.000000in}{0.000000in}}{%
\pgfpathmoveto{\pgfqpoint{-0.000000in}{0.000000in}}%
\pgfpathlineto{\pgfqpoint{-0.027778in}{0.000000in}}%
\pgfusepath{stroke,fill}%
}%
\begin{pgfscope}%
\pgfsys@transformshift{0.708220in}{2.193526in}%
\pgfsys@useobject{currentmarker}{}%
\end{pgfscope}%
\end{pgfscope}%
\begin{pgfscope}%
\pgfsetbuttcap%
\pgfsetroundjoin%
\definecolor{currentfill}{rgb}{0.000000,0.000000,0.000000}%
\pgfsetfillcolor{currentfill}%
\pgfsetlinewidth{0.602250pt}%
\definecolor{currentstroke}{rgb}{0.000000,0.000000,0.000000}%
\pgfsetstrokecolor{currentstroke}%
\pgfsetdash{}{0pt}%
\pgfsys@defobject{currentmarker}{\pgfqpoint{-0.027778in}{0.000000in}}{\pgfqpoint{-0.000000in}{0.000000in}}{%
\pgfpathmoveto{\pgfqpoint{-0.000000in}{0.000000in}}%
\pgfpathlineto{\pgfqpoint{-0.027778in}{0.000000in}}%
\pgfusepath{stroke,fill}%
}%
\begin{pgfscope}%
\pgfsys@transformshift{0.708220in}{2.220740in}%
\pgfsys@useobject{currentmarker}{}%
\end{pgfscope}%
\end{pgfscope}%
\begin{pgfscope}%
\pgfsetbuttcap%
\pgfsetroundjoin%
\definecolor{currentfill}{rgb}{0.000000,0.000000,0.000000}%
\pgfsetfillcolor{currentfill}%
\pgfsetlinewidth{0.602250pt}%
\definecolor{currentstroke}{rgb}{0.000000,0.000000,0.000000}%
\pgfsetstrokecolor{currentstroke}%
\pgfsetdash{}{0pt}%
\pgfsys@defobject{currentmarker}{\pgfqpoint{-0.027778in}{0.000000in}}{\pgfqpoint{-0.000000in}{0.000000in}}{%
\pgfpathmoveto{\pgfqpoint{-0.000000in}{0.000000in}}%
\pgfpathlineto{\pgfqpoint{-0.027778in}{0.000000in}}%
\pgfusepath{stroke,fill}%
}%
\begin{pgfscope}%
\pgfsys@transformshift{0.708220in}{2.242976in}%
\pgfsys@useobject{currentmarker}{}%
\end{pgfscope}%
\end{pgfscope}%
\begin{pgfscope}%
\pgfsetbuttcap%
\pgfsetroundjoin%
\definecolor{currentfill}{rgb}{0.000000,0.000000,0.000000}%
\pgfsetfillcolor{currentfill}%
\pgfsetlinewidth{0.602250pt}%
\definecolor{currentstroke}{rgb}{0.000000,0.000000,0.000000}%
\pgfsetstrokecolor{currentstroke}%
\pgfsetdash{}{0pt}%
\pgfsys@defobject{currentmarker}{\pgfqpoint{-0.027778in}{0.000000in}}{\pgfqpoint{-0.000000in}{0.000000in}}{%
\pgfpathmoveto{\pgfqpoint{-0.000000in}{0.000000in}}%
\pgfpathlineto{\pgfqpoint{-0.027778in}{0.000000in}}%
\pgfusepath{stroke,fill}%
}%
\begin{pgfscope}%
\pgfsys@transformshift{0.708220in}{2.261776in}%
\pgfsys@useobject{currentmarker}{}%
\end{pgfscope}%
\end{pgfscope}%
\begin{pgfscope}%
\pgfsetbuttcap%
\pgfsetroundjoin%
\definecolor{currentfill}{rgb}{0.000000,0.000000,0.000000}%
\pgfsetfillcolor{currentfill}%
\pgfsetlinewidth{0.602250pt}%
\definecolor{currentstroke}{rgb}{0.000000,0.000000,0.000000}%
\pgfsetstrokecolor{currentstroke}%
\pgfsetdash{}{0pt}%
\pgfsys@defobject{currentmarker}{\pgfqpoint{-0.027778in}{0.000000in}}{\pgfqpoint{-0.000000in}{0.000000in}}{%
\pgfpathmoveto{\pgfqpoint{-0.000000in}{0.000000in}}%
\pgfpathlineto{\pgfqpoint{-0.027778in}{0.000000in}}%
\pgfusepath{stroke,fill}%
}%
\begin{pgfscope}%
\pgfsys@transformshift{0.708220in}{2.278061in}%
\pgfsys@useobject{currentmarker}{}%
\end{pgfscope}%
\end{pgfscope}%
\begin{pgfscope}%
\pgfsetbuttcap%
\pgfsetroundjoin%
\definecolor{currentfill}{rgb}{0.000000,0.000000,0.000000}%
\pgfsetfillcolor{currentfill}%
\pgfsetlinewidth{0.602250pt}%
\definecolor{currentstroke}{rgb}{0.000000,0.000000,0.000000}%
\pgfsetstrokecolor{currentstroke}%
\pgfsetdash{}{0pt}%
\pgfsys@defobject{currentmarker}{\pgfqpoint{-0.027778in}{0.000000in}}{\pgfqpoint{-0.000000in}{0.000000in}}{%
\pgfpathmoveto{\pgfqpoint{-0.000000in}{0.000000in}}%
\pgfpathlineto{\pgfqpoint{-0.027778in}{0.000000in}}%
\pgfusepath{stroke,fill}%
}%
\begin{pgfscope}%
\pgfsys@transformshift{0.708220in}{2.292426in}%
\pgfsys@useobject{currentmarker}{}%
\end{pgfscope}%
\end{pgfscope}%
\begin{pgfscope}%
\definecolor{textcolor}{rgb}{0.000000,0.000000,0.000000}%
\pgfsetstrokecolor{textcolor}%
\pgfsetfillcolor{textcolor}%
\pgftext[x=0.288855in,y=1.420549in,,bottom,rotate=90.000000]{\color{textcolor}\rmfamily\fontsize{10.000000}{12.000000}\selectfont Longest solving time (s)}%
\end{pgfscope}%
\begin{pgfscope}%
\pgfpathrectangle{\pgfqpoint{0.708220in}{0.535823in}}{\pgfqpoint{5.045427in}{1.769453in}}%
\pgfusepath{clip}%
\pgfsetrectcap%
\pgfsetroundjoin%
\pgfsetlinewidth{1.003750pt}%
\definecolor{currentstroke}{rgb}{0.121569,0.466667,0.705882}%
\pgfsetstrokecolor{currentstroke}%
\pgfsetdash{}{0pt}%
\pgfpathmoveto{\pgfqpoint{0.708220in}{1.081871in}}%
\pgfpathlineto{\pgfqpoint{0.720833in}{1.119697in}}%
\pgfpathlineto{\pgfqpoint{0.733447in}{1.177018in}}%
\pgfpathlineto{\pgfqpoint{0.771288in}{1.180771in}}%
\pgfpathlineto{\pgfqpoint{0.796515in}{1.180771in}}%
\pgfpathlineto{\pgfqpoint{0.809128in}{1.181997in}}%
\pgfpathlineto{\pgfqpoint{0.846969in}{1.181997in}}%
\pgfpathlineto{\pgfqpoint{0.859583in}{1.183210in}}%
\pgfpathlineto{\pgfqpoint{0.897423in}{1.183210in}}%
\pgfpathlineto{\pgfqpoint{0.910037in}{1.184412in}}%
\pgfpathlineto{\pgfqpoint{0.947878in}{1.184412in}}%
\pgfpathlineto{\pgfqpoint{0.960491in}{1.185602in}}%
\pgfpathlineto{\pgfqpoint{0.985718in}{1.185602in}}%
\pgfpathlineto{\pgfqpoint{0.998332in}{1.186780in}}%
\pgfpathlineto{\pgfqpoint{1.010945in}{1.186780in}}%
\pgfpathlineto{\pgfqpoint{1.023559in}{1.187947in}}%
\pgfpathlineto{\pgfqpoint{1.036173in}{1.187947in}}%
\pgfpathlineto{\pgfqpoint{1.048786in}{1.189103in}}%
\pgfpathlineto{\pgfqpoint{1.074013in}{1.189103in}}%
\pgfpathlineto{\pgfqpoint{1.099240in}{1.191383in}}%
\pgfpathlineto{\pgfqpoint{1.124468in}{1.191383in}}%
\pgfpathlineto{\pgfqpoint{1.137081in}{1.192507in}}%
\pgfpathlineto{\pgfqpoint{1.162308in}{1.192507in}}%
\pgfpathlineto{\pgfqpoint{1.187535in}{1.193621in}}%
\pgfpathlineto{\pgfqpoint{1.263217in}{1.194724in}}%
\pgfpathlineto{\pgfqpoint{1.301057in}{1.196902in}}%
\pgfpathlineto{\pgfqpoint{1.439807in}{1.197977in}}%
\pgfpathlineto{\pgfqpoint{1.465034in}{1.199042in}}%
\pgfpathlineto{\pgfqpoint{1.528102in}{1.200098in}}%
\pgfpathlineto{\pgfqpoint{1.553329in}{1.201145in}}%
\pgfpathlineto{\pgfqpoint{1.578556in}{1.201145in}}%
\pgfpathlineto{\pgfqpoint{1.591170in}{1.203212in}}%
\pgfpathlineto{\pgfqpoint{1.629010in}{1.204233in}}%
\pgfpathlineto{\pgfqpoint{1.654237in}{1.205245in}}%
\pgfpathlineto{\pgfqpoint{1.729919in}{1.206248in}}%
\pgfpathlineto{\pgfqpoint{1.755146in}{1.207244in}}%
\pgfpathlineto{\pgfqpoint{1.805600in}{1.208232in}}%
\pgfpathlineto{\pgfqpoint{1.843441in}{1.210183in}}%
\pgfpathlineto{\pgfqpoint{1.906509in}{1.210183in}}%
\pgfpathlineto{\pgfqpoint{1.919122in}{1.212104in}}%
\pgfpathlineto{\pgfqpoint{1.944349in}{1.213053in}}%
\pgfpathlineto{\pgfqpoint{1.969577in}{1.213994in}}%
\pgfpathlineto{\pgfqpoint{2.045258in}{1.214929in}}%
\pgfpathlineto{\pgfqpoint{2.070485in}{1.215856in}}%
\pgfpathlineto{\pgfqpoint{2.158780in}{1.216777in}}%
\pgfpathlineto{\pgfqpoint{2.184007in}{1.217690in}}%
\pgfpathlineto{\pgfqpoint{2.234461in}{1.218597in}}%
\pgfpathlineto{\pgfqpoint{2.272302in}{1.220391in}}%
\pgfpathlineto{\pgfqpoint{2.297529in}{1.221278in}}%
\pgfpathlineto{\pgfqpoint{2.310143in}{1.224762in}}%
\pgfpathlineto{\pgfqpoint{2.322756in}{1.225618in}}%
\pgfpathlineto{\pgfqpoint{2.335370in}{1.228150in}}%
\pgfpathlineto{\pgfqpoint{2.360597in}{1.228983in}}%
\pgfpathlineto{\pgfqpoint{2.373211in}{1.231447in}}%
\pgfpathlineto{\pgfqpoint{2.385824in}{1.231447in}}%
\pgfpathlineto{\pgfqpoint{2.398438in}{1.233062in}}%
\pgfpathlineto{\pgfqpoint{2.461506in}{1.233862in}}%
\pgfpathlineto{\pgfqpoint{2.549801in}{1.238553in}}%
\pgfpathlineto{\pgfqpoint{2.562414in}{1.238553in}}%
\pgfpathlineto{\pgfqpoint{2.612868in}{1.244540in}}%
\pgfpathlineto{\pgfqpoint{2.650709in}{1.245268in}}%
\pgfpathlineto{\pgfqpoint{2.663323in}{1.246711in}}%
\pgfpathlineto{\pgfqpoint{2.675936in}{1.249548in}}%
\pgfpathlineto{\pgfqpoint{2.688550in}{1.250942in}}%
\pgfpathlineto{\pgfqpoint{2.713777in}{1.251633in}}%
\pgfpathlineto{\pgfqpoint{2.751618in}{1.255030in}}%
\pgfpathlineto{\pgfqpoint{2.991275in}{1.268348in}}%
\pgfpathlineto{\pgfqpoint{3.016503in}{1.270137in}}%
\pgfpathlineto{\pgfqpoint{3.041730in}{1.270728in}}%
\pgfpathlineto{\pgfqpoint{3.054343in}{1.271900in}}%
\pgfpathlineto{\pgfqpoint{3.079570in}{1.275918in}}%
\pgfpathlineto{\pgfqpoint{3.142638in}{1.279260in}}%
\pgfpathlineto{\pgfqpoint{3.167865in}{1.283046in}}%
\pgfpathlineto{\pgfqpoint{3.193092in}{1.283577in}}%
\pgfpathlineto{\pgfqpoint{3.205706in}{1.289780in}}%
\pgfpathlineto{\pgfqpoint{3.218320in}{1.293257in}}%
\pgfpathlineto{\pgfqpoint{3.268774in}{1.298060in}}%
\pgfpathlineto{\pgfqpoint{3.319228in}{1.304926in}}%
\pgfpathlineto{\pgfqpoint{3.331842in}{1.305370in}}%
\pgfpathlineto{\pgfqpoint{3.357069in}{1.319587in}}%
\pgfpathlineto{\pgfqpoint{3.369682in}{1.319981in}}%
\pgfpathlineto{\pgfqpoint{3.394910in}{1.323088in}}%
\pgfpathlineto{\pgfqpoint{3.407523in}{1.326118in}}%
\pgfpathlineto{\pgfqpoint{3.420137in}{1.336512in}}%
\pgfpathlineto{\pgfqpoint{3.457977in}{1.340566in}}%
\pgfpathlineto{\pgfqpoint{3.470591in}{1.344168in}}%
\pgfpathlineto{\pgfqpoint{3.495818in}{1.346089in}}%
\pgfpathlineto{\pgfqpoint{3.508432in}{1.353781in}}%
\pgfpathlineto{\pgfqpoint{3.546272in}{1.355850in}}%
\pgfpathlineto{\pgfqpoint{3.571499in}{1.365702in}}%
\pgfpathlineto{\pgfqpoint{3.584113in}{1.375319in}}%
\pgfpathlineto{\pgfqpoint{3.596727in}{1.375568in}}%
\pgfpathlineto{\pgfqpoint{3.609340in}{1.393187in}}%
\pgfpathlineto{\pgfqpoint{3.621954in}{1.403330in}}%
\pgfpathlineto{\pgfqpoint{3.634567in}{1.410466in}}%
\pgfpathlineto{\pgfqpoint{3.647181in}{1.415422in}}%
\pgfpathlineto{\pgfqpoint{3.659794in}{1.430624in}}%
\pgfpathlineto{\pgfqpoint{3.672408in}{1.438465in}}%
\pgfpathlineto{\pgfqpoint{3.685022in}{1.452780in}}%
\pgfpathlineto{\pgfqpoint{3.697635in}{1.471976in}}%
\pgfpathlineto{\pgfqpoint{3.710249in}{1.475324in}}%
\pgfpathlineto{\pgfqpoint{3.722862in}{1.508552in}}%
\pgfpathlineto{\pgfqpoint{3.735476in}{1.533550in}}%
\pgfpathlineto{\pgfqpoint{3.748089in}{1.535110in}}%
\pgfpathlineto{\pgfqpoint{3.760703in}{1.546311in}}%
\pgfpathlineto{\pgfqpoint{3.773317in}{1.585433in}}%
\pgfpathlineto{\pgfqpoint{3.785930in}{1.632130in}}%
\pgfpathlineto{\pgfqpoint{3.798544in}{1.640820in}}%
\pgfpathlineto{\pgfqpoint{3.811157in}{1.664352in}}%
\pgfpathlineto{\pgfqpoint{3.823771in}{1.675124in}}%
\pgfpathlineto{\pgfqpoint{3.836384in}{1.689341in}}%
\pgfpathlineto{\pgfqpoint{3.848998in}{1.691679in}}%
\pgfpathlineto{\pgfqpoint{3.874225in}{1.708535in}}%
\pgfpathlineto{\pgfqpoint{3.886839in}{1.716772in}}%
\pgfpathlineto{\pgfqpoint{3.912066in}{1.786847in}}%
\pgfpathlineto{\pgfqpoint{3.924679in}{1.788251in}}%
\pgfpathlineto{\pgfqpoint{3.937293in}{1.842763in}}%
\pgfpathlineto{\pgfqpoint{3.949906in}{1.848558in}}%
\pgfpathlineto{\pgfqpoint{3.962520in}{1.869329in}}%
\pgfpathlineto{\pgfqpoint{3.975134in}{1.910590in}}%
\pgfpathlineto{\pgfqpoint{3.987747in}{1.946298in}}%
\pgfpathlineto{\pgfqpoint{4.012974in}{1.993094in}}%
\pgfpathlineto{\pgfqpoint{4.025588in}{2.002672in}}%
\pgfpathlineto{\pgfqpoint{4.038201in}{2.006356in}}%
\pgfpathlineto{\pgfqpoint{4.050815in}{2.021025in}}%
\pgfpathlineto{\pgfqpoint{4.063429in}{2.098041in}}%
\pgfpathlineto{\pgfqpoint{4.076042in}{2.106049in}}%
\pgfpathlineto{\pgfqpoint{4.088656in}{2.142626in}}%
\pgfpathlineto{\pgfqpoint{4.101269in}{2.152286in}}%
\pgfpathlineto{\pgfqpoint{4.113883in}{2.154419in}}%
\pgfpathlineto{\pgfqpoint{4.139110in}{2.167796in}}%
\pgfpathlineto{\pgfqpoint{4.151724in}{2.168054in}}%
\pgfpathlineto{\pgfqpoint{4.164337in}{2.174326in}}%
\pgfpathlineto{\pgfqpoint{4.176951in}{2.175430in}}%
\pgfpathlineto{\pgfqpoint{4.189564in}{2.184958in}}%
\pgfpathlineto{\pgfqpoint{4.202178in}{2.206324in}}%
\pgfpathlineto{\pgfqpoint{4.214791in}{2.206670in}}%
\pgfpathlineto{\pgfqpoint{4.227405in}{2.210631in}}%
\pgfpathlineto{\pgfqpoint{4.240018in}{2.212381in}}%
\pgfpathlineto{\pgfqpoint{4.265246in}{2.281125in}}%
\pgfpathlineto{\pgfqpoint{4.277859in}{2.305275in}}%
\pgfpathlineto{\pgfqpoint{4.277859in}{2.305275in}}%
\pgfusepath{stroke}%
\end{pgfscope}%
\begin{pgfscope}%
\pgfpathrectangle{\pgfqpoint{0.708220in}{0.535823in}}{\pgfqpoint{5.045427in}{1.769453in}}%
\pgfusepath{clip}%
\pgfsetbuttcap%
\pgfsetroundjoin%
\pgfsetlinewidth{1.003750pt}%
\definecolor{currentstroke}{rgb}{1.000000,0.498039,0.054902}%
\pgfsetstrokecolor{currentstroke}%
\pgfsetdash{{6.400000pt}{1.600000pt}{1.000000pt}{1.600000pt}}{0.000000pt}%
\pgfpathmoveto{\pgfqpoint{0.708220in}{1.179533in}}%
\pgfpathlineto{\pgfqpoint{0.733447in}{1.179533in}}%
\pgfpathlineto{\pgfqpoint{0.746061in}{1.180771in}}%
\pgfpathlineto{\pgfqpoint{0.771288in}{1.180771in}}%
\pgfpathlineto{\pgfqpoint{0.796515in}{1.183210in}}%
\pgfpathlineto{\pgfqpoint{0.834356in}{1.183210in}}%
\pgfpathlineto{\pgfqpoint{0.846969in}{1.184412in}}%
\pgfpathlineto{\pgfqpoint{0.859583in}{1.184412in}}%
\pgfpathlineto{\pgfqpoint{0.897423in}{1.187947in}}%
\pgfpathlineto{\pgfqpoint{0.910037in}{1.187947in}}%
\pgfpathlineto{\pgfqpoint{0.922650in}{1.189103in}}%
\pgfpathlineto{\pgfqpoint{0.973105in}{1.189103in}}%
\pgfpathlineto{\pgfqpoint{0.985718in}{1.190248in}}%
\pgfpathlineto{\pgfqpoint{1.010945in}{1.190248in}}%
\pgfpathlineto{\pgfqpoint{1.023559in}{1.191383in}}%
\pgfpathlineto{\pgfqpoint{1.048786in}{1.191383in}}%
\pgfpathlineto{\pgfqpoint{1.061400in}{1.192507in}}%
\pgfpathlineto{\pgfqpoint{1.099240in}{1.192507in}}%
\pgfpathlineto{\pgfqpoint{1.124468in}{1.193621in}}%
\pgfpathlineto{\pgfqpoint{1.200149in}{1.194724in}}%
\pgfpathlineto{\pgfqpoint{1.237990in}{1.196902in}}%
\pgfpathlineto{\pgfqpoint{1.351512in}{1.197977in}}%
\pgfpathlineto{\pgfqpoint{1.376739in}{1.199042in}}%
\pgfpathlineto{\pgfqpoint{1.439807in}{1.200098in}}%
\pgfpathlineto{\pgfqpoint{1.490261in}{1.203212in}}%
\pgfpathlineto{\pgfqpoint{1.528102in}{1.204233in}}%
\pgfpathlineto{\pgfqpoint{1.553329in}{1.205245in}}%
\pgfpathlineto{\pgfqpoint{1.603783in}{1.206248in}}%
\pgfpathlineto{\pgfqpoint{1.629010in}{1.207244in}}%
\pgfpathlineto{\pgfqpoint{1.692078in}{1.208232in}}%
\pgfpathlineto{\pgfqpoint{1.717305in}{1.209211in}}%
\pgfpathlineto{\pgfqpoint{1.767759in}{1.210183in}}%
\pgfpathlineto{\pgfqpoint{1.792987in}{1.211147in}}%
\pgfpathlineto{\pgfqpoint{1.856054in}{1.212104in}}%
\pgfpathlineto{\pgfqpoint{1.881282in}{1.213053in}}%
\pgfpathlineto{\pgfqpoint{1.919122in}{1.213994in}}%
\pgfpathlineto{\pgfqpoint{1.931736in}{1.215856in}}%
\pgfpathlineto{\pgfqpoint{1.982190in}{1.216777in}}%
\pgfpathlineto{\pgfqpoint{2.020031in}{1.218597in}}%
\pgfpathlineto{\pgfqpoint{2.032644in}{1.218597in}}%
\pgfpathlineto{\pgfqpoint{2.045258in}{1.220391in}}%
\pgfpathlineto{\pgfqpoint{2.133553in}{1.221278in}}%
\pgfpathlineto{\pgfqpoint{2.158780in}{1.222158in}}%
\pgfpathlineto{\pgfqpoint{2.209234in}{1.223033in}}%
\pgfpathlineto{\pgfqpoint{2.234461in}{1.223901in}}%
\pgfpathlineto{\pgfqpoint{2.259689in}{1.224762in}}%
\pgfpathlineto{\pgfqpoint{2.347984in}{1.231447in}}%
\pgfpathlineto{\pgfqpoint{2.373211in}{1.233862in}}%
\pgfpathlineto{\pgfqpoint{2.537187in}{1.245268in}}%
\pgfpathlineto{\pgfqpoint{2.549801in}{1.250942in}}%
\pgfpathlineto{\pgfqpoint{2.575028in}{1.251633in}}%
\pgfpathlineto{\pgfqpoint{2.600255in}{1.253682in}}%
\pgfpathlineto{\pgfqpoint{2.625482in}{1.254358in}}%
\pgfpathlineto{\pgfqpoint{2.663323in}{1.257024in}}%
\pgfpathlineto{\pgfqpoint{2.675936in}{1.258335in}}%
\pgfpathlineto{\pgfqpoint{2.688550in}{1.258335in}}%
\pgfpathlineto{\pgfqpoint{2.701163in}{1.260276in}}%
\pgfpathlineto{\pgfqpoint{2.726390in}{1.260917in}}%
\pgfpathlineto{\pgfqpoint{2.739004in}{1.263444in}}%
\pgfpathlineto{\pgfqpoint{2.877753in}{1.269543in}}%
\pgfpathlineto{\pgfqpoint{2.915594in}{1.270137in}}%
\pgfpathlineto{\pgfqpoint{2.966048in}{1.273062in}}%
\pgfpathlineto{\pgfqpoint{2.978662in}{1.277600in}}%
\pgfpathlineto{\pgfqpoint{2.991275in}{1.278156in}}%
\pgfpathlineto{\pgfqpoint{3.003889in}{1.280353in}}%
\pgfpathlineto{\pgfqpoint{3.029116in}{1.280897in}}%
\pgfpathlineto{\pgfqpoint{3.066957in}{1.283577in}}%
\pgfpathlineto{\pgfqpoint{3.079570in}{1.287234in}}%
\pgfpathlineto{\pgfqpoint{3.104797in}{1.288258in}}%
\pgfpathlineto{\pgfqpoint{3.117411in}{1.291779in}}%
\pgfpathlineto{\pgfqpoint{3.130025in}{1.298530in}}%
\pgfpathlineto{\pgfqpoint{3.142638in}{1.299464in}}%
\pgfpathlineto{\pgfqpoint{3.167865in}{1.303132in}}%
\pgfpathlineto{\pgfqpoint{3.180479in}{1.303132in}}%
\pgfpathlineto{\pgfqpoint{3.205706in}{1.308867in}}%
\pgfpathlineto{\pgfqpoint{3.218320in}{1.318795in}}%
\pgfpathlineto{\pgfqpoint{3.256160in}{1.320374in}}%
\pgfpathlineto{\pgfqpoint{3.268774in}{1.323471in}}%
\pgfpathlineto{\pgfqpoint{3.281387in}{1.329803in}}%
\pgfpathlineto{\pgfqpoint{3.306615in}{1.330887in}}%
\pgfpathlineto{\pgfqpoint{3.319228in}{1.332318in}}%
\pgfpathlineto{\pgfqpoint{3.331842in}{1.335477in}}%
\pgfpathlineto{\pgfqpoint{3.369682in}{1.337878in}}%
\pgfpathlineto{\pgfqpoint{3.382296in}{1.340898in}}%
\pgfpathlineto{\pgfqpoint{3.394910in}{1.341888in}}%
\pgfpathlineto{\pgfqpoint{3.407523in}{1.345771in}}%
\pgfpathlineto{\pgfqpoint{3.445364in}{1.348603in}}%
\pgfpathlineto{\pgfqpoint{3.457977in}{1.363520in}}%
\pgfpathlineto{\pgfqpoint{3.470591in}{1.368377in}}%
\pgfpathlineto{\pgfqpoint{3.483204in}{1.377547in}}%
\pgfpathlineto{\pgfqpoint{3.495818in}{1.378768in}}%
\pgfpathlineto{\pgfqpoint{3.508432in}{1.383299in}}%
\pgfpathlineto{\pgfqpoint{3.521045in}{1.386305in}}%
\pgfpathlineto{\pgfqpoint{3.533659in}{1.391228in}}%
\pgfpathlineto{\pgfqpoint{3.546272in}{1.392538in}}%
\pgfpathlineto{\pgfqpoint{3.558886in}{1.407431in}}%
\pgfpathlineto{\pgfqpoint{3.571499in}{1.460849in}}%
\pgfpathlineto{\pgfqpoint{3.584113in}{1.462938in}}%
\pgfpathlineto{\pgfqpoint{3.609340in}{1.464151in}}%
\pgfpathlineto{\pgfqpoint{3.621954in}{1.495282in}}%
\pgfpathlineto{\pgfqpoint{3.634567in}{1.506182in}}%
\pgfpathlineto{\pgfqpoint{3.647181in}{1.556058in}}%
\pgfpathlineto{\pgfqpoint{3.672408in}{1.562580in}}%
\pgfpathlineto{\pgfqpoint{3.685022in}{1.591823in}}%
\pgfpathlineto{\pgfqpoint{3.697635in}{1.614446in}}%
\pgfpathlineto{\pgfqpoint{3.710249in}{1.615183in}}%
\pgfpathlineto{\pgfqpoint{3.722862in}{1.638705in}}%
\pgfpathlineto{\pgfqpoint{3.735476in}{1.640877in}}%
\pgfpathlineto{\pgfqpoint{3.748089in}{1.640962in}}%
\pgfpathlineto{\pgfqpoint{3.760703in}{1.645108in}}%
\pgfpathlineto{\pgfqpoint{3.773317in}{1.670924in}}%
\pgfpathlineto{\pgfqpoint{3.785930in}{1.676316in}}%
\pgfpathlineto{\pgfqpoint{3.798544in}{1.685237in}}%
\pgfpathlineto{\pgfqpoint{3.811157in}{1.717892in}}%
\pgfpathlineto{\pgfqpoint{3.823771in}{1.720322in}}%
\pgfpathlineto{\pgfqpoint{3.836384in}{1.721571in}}%
\pgfpathlineto{\pgfqpoint{3.848998in}{1.812664in}}%
\pgfpathlineto{\pgfqpoint{3.861611in}{1.849606in}}%
\pgfpathlineto{\pgfqpoint{3.874225in}{1.850447in}}%
\pgfpathlineto{\pgfqpoint{3.886839in}{1.852617in}}%
\pgfpathlineto{\pgfqpoint{3.899452in}{1.863208in}}%
\pgfpathlineto{\pgfqpoint{3.912066in}{1.871656in}}%
\pgfpathlineto{\pgfqpoint{3.924679in}{1.874721in}}%
\pgfpathlineto{\pgfqpoint{3.937293in}{1.888700in}}%
\pgfpathlineto{\pgfqpoint{3.949906in}{1.898099in}}%
\pgfpathlineto{\pgfqpoint{3.962520in}{1.905720in}}%
\pgfpathlineto{\pgfqpoint{3.975134in}{1.918784in}}%
\pgfpathlineto{\pgfqpoint{3.987747in}{1.919455in}}%
\pgfpathlineto{\pgfqpoint{4.000361in}{1.945102in}}%
\pgfpathlineto{\pgfqpoint{4.012974in}{1.949119in}}%
\pgfpathlineto{\pgfqpoint{4.025588in}{1.954828in}}%
\pgfpathlineto{\pgfqpoint{4.038201in}{2.065928in}}%
\pgfpathlineto{\pgfqpoint{4.050815in}{2.069409in}}%
\pgfpathlineto{\pgfqpoint{4.063429in}{2.108985in}}%
\pgfpathlineto{\pgfqpoint{4.076042in}{2.134900in}}%
\pgfpathlineto{\pgfqpoint{4.088656in}{2.139175in}}%
\pgfpathlineto{\pgfqpoint{4.101269in}{2.182566in}}%
\pgfpathlineto{\pgfqpoint{4.113883in}{2.185487in}}%
\pgfpathlineto{\pgfqpoint{4.126496in}{2.185693in}}%
\pgfpathlineto{\pgfqpoint{4.151724in}{2.187505in}}%
\pgfpathlineto{\pgfqpoint{4.164337in}{2.212091in}}%
\pgfpathlineto{\pgfqpoint{4.176951in}{2.216907in}}%
\pgfpathlineto{\pgfqpoint{4.189564in}{2.237440in}}%
\pgfpathlineto{\pgfqpoint{4.202178in}{2.247054in}}%
\pgfpathlineto{\pgfqpoint{4.214791in}{2.247219in}}%
\pgfpathlineto{\pgfqpoint{4.227405in}{2.249812in}}%
\pgfpathlineto{\pgfqpoint{4.240018in}{2.269432in}}%
\pgfpathlineto{\pgfqpoint{4.252632in}{2.272099in}}%
\pgfpathlineto{\pgfqpoint{4.265246in}{2.283600in}}%
\pgfpathlineto{\pgfqpoint{4.277859in}{2.293428in}}%
\pgfpathlineto{\pgfqpoint{4.290473in}{2.305275in}}%
\pgfpathlineto{\pgfqpoint{4.290473in}{2.305275in}}%
\pgfusepath{stroke}%
\end{pgfscope}%
\begin{pgfscope}%
\pgfpathrectangle{\pgfqpoint{0.708220in}{0.535823in}}{\pgfqpoint{5.045427in}{1.769453in}}%
\pgfusepath{clip}%
\pgfsetbuttcap%
\pgfsetroundjoin%
\pgfsetlinewidth{1.003750pt}%
\definecolor{currentstroke}{rgb}{0.172549,0.627451,0.172549}%
\pgfsetstrokecolor{currentstroke}%
\pgfsetdash{{6.400000pt}{1.600000pt}{1.000000pt}{1.600000pt}}{0.000000pt}%
\pgfpathmoveto{\pgfqpoint{0.708220in}{1.210183in}}%
\pgfpathlineto{\pgfqpoint{0.758674in}{1.211147in}}%
\pgfpathlineto{\pgfqpoint{0.771288in}{1.213053in}}%
\pgfpathlineto{\pgfqpoint{0.809128in}{1.213994in}}%
\pgfpathlineto{\pgfqpoint{0.859583in}{1.216777in}}%
\pgfpathlineto{\pgfqpoint{0.960491in}{1.217690in}}%
\pgfpathlineto{\pgfqpoint{0.985718in}{1.220391in}}%
\pgfpathlineto{\pgfqpoint{1.010945in}{1.221278in}}%
\pgfpathlineto{\pgfqpoint{1.061400in}{1.222158in}}%
\pgfpathlineto{\pgfqpoint{1.086627in}{1.223033in}}%
\pgfpathlineto{\pgfqpoint{1.162308in}{1.223901in}}%
\pgfpathlineto{\pgfqpoint{1.200149in}{1.225618in}}%
\pgfpathlineto{\pgfqpoint{1.263217in}{1.226468in}}%
\pgfpathlineto{\pgfqpoint{1.313671in}{1.228983in}}%
\pgfpathlineto{\pgfqpoint{1.364125in}{1.229810in}}%
\pgfpathlineto{\pgfqpoint{1.389352in}{1.230631in}}%
\pgfpathlineto{\pgfqpoint{1.490261in}{1.231447in}}%
\pgfpathlineto{\pgfqpoint{1.502875in}{1.233062in}}%
\pgfpathlineto{\pgfqpoint{1.603783in}{1.233862in}}%
\pgfpathlineto{\pgfqpoint{1.629010in}{1.234656in}}%
\pgfpathlineto{\pgfqpoint{1.692078in}{1.235446in}}%
\pgfpathlineto{\pgfqpoint{1.704692in}{1.237009in}}%
\pgfpathlineto{\pgfqpoint{1.767759in}{1.237784in}}%
\pgfpathlineto{\pgfqpoint{1.792987in}{1.238553in}}%
\pgfpathlineto{\pgfqpoint{1.868668in}{1.239318in}}%
\pgfpathlineto{\pgfqpoint{1.893895in}{1.240078in}}%
\pgfpathlineto{\pgfqpoint{1.994804in}{1.240833in}}%
\pgfpathlineto{\pgfqpoint{2.095712in}{1.245992in}}%
\pgfpathlineto{\pgfqpoint{2.146166in}{1.246711in}}%
\pgfpathlineto{\pgfqpoint{2.221848in}{1.250247in}}%
\pgfpathlineto{\pgfqpoint{2.234461in}{1.250247in}}%
\pgfpathlineto{\pgfqpoint{2.272302in}{1.254358in}}%
\pgfpathlineto{\pgfqpoint{2.297529in}{1.255030in}}%
\pgfpathlineto{\pgfqpoint{2.347984in}{1.257024in}}%
\pgfpathlineto{\pgfqpoint{2.360597in}{1.257024in}}%
\pgfpathlineto{\pgfqpoint{2.373211in}{1.258986in}}%
\pgfpathlineto{\pgfqpoint{2.398438in}{1.258986in}}%
\pgfpathlineto{\pgfqpoint{2.411051in}{1.260276in}}%
\pgfpathlineto{\pgfqpoint{2.461506in}{1.260917in}}%
\pgfpathlineto{\pgfqpoint{2.486733in}{1.261553in}}%
\pgfpathlineto{\pgfqpoint{2.524573in}{1.262187in}}%
\pgfpathlineto{\pgfqpoint{2.562414in}{1.263444in}}%
\pgfpathlineto{\pgfqpoint{2.587641in}{1.264068in}}%
\pgfpathlineto{\pgfqpoint{2.638096in}{1.265921in}}%
\pgfpathlineto{\pgfqpoint{2.650709in}{1.265921in}}%
\pgfpathlineto{\pgfqpoint{2.675936in}{1.267746in}}%
\pgfpathlineto{\pgfqpoint{2.713777in}{1.268348in}}%
\pgfpathlineto{\pgfqpoint{2.726390in}{1.269543in}}%
\pgfpathlineto{\pgfqpoint{2.739004in}{1.269543in}}%
\pgfpathlineto{\pgfqpoint{2.776845in}{1.273062in}}%
\pgfpathlineto{\pgfqpoint{2.789458in}{1.273062in}}%
\pgfpathlineto{\pgfqpoint{2.802072in}{1.277042in}}%
\pgfpathlineto{\pgfqpoint{2.827299in}{1.277600in}}%
\pgfpathlineto{\pgfqpoint{2.865140in}{1.278709in}}%
\pgfpathlineto{\pgfqpoint{2.877753in}{1.281437in}}%
\pgfpathlineto{\pgfqpoint{2.915594in}{1.282512in}}%
\pgfpathlineto{\pgfqpoint{2.978662in}{1.287234in}}%
\pgfpathlineto{\pgfqpoint{3.003889in}{1.288258in}}%
\pgfpathlineto{\pgfqpoint{3.029116in}{1.291779in}}%
\pgfpathlineto{\pgfqpoint{3.054343in}{1.293257in}}%
\pgfpathlineto{\pgfqpoint{3.066957in}{1.295201in}}%
\pgfpathlineto{\pgfqpoint{3.092184in}{1.295682in}}%
\pgfpathlineto{\pgfqpoint{3.104797in}{1.303583in}}%
\pgfpathlineto{\pgfqpoint{3.117411in}{1.309298in}}%
\pgfpathlineto{\pgfqpoint{3.130025in}{1.313518in}}%
\pgfpathlineto{\pgfqpoint{3.142638in}{1.313518in}}%
\pgfpathlineto{\pgfqpoint{3.193092in}{1.319587in}}%
\pgfpathlineto{\pgfqpoint{3.205706in}{1.320374in}}%
\pgfpathlineto{\pgfqpoint{3.218320in}{1.326118in}}%
\pgfpathlineto{\pgfqpoint{3.281387in}{1.331962in}}%
\pgfpathlineto{\pgfqpoint{3.294001in}{1.334433in}}%
\pgfpathlineto{\pgfqpoint{3.306615in}{1.340898in}}%
\pgfpathlineto{\pgfqpoint{3.344455in}{1.343196in}}%
\pgfpathlineto{\pgfqpoint{3.357069in}{1.352281in}}%
\pgfpathlineto{\pgfqpoint{3.382296in}{1.354376in}}%
\pgfpathlineto{\pgfqpoint{3.407523in}{1.355850in}}%
\pgfpathlineto{\pgfqpoint{3.420137in}{1.363520in}}%
\pgfpathlineto{\pgfqpoint{3.445364in}{1.365160in}}%
\pgfpathlineto{\pgfqpoint{3.470591in}{1.368112in}}%
\pgfpathlineto{\pgfqpoint{3.495818in}{1.373048in}}%
\pgfpathlineto{\pgfqpoint{3.508432in}{1.385618in}}%
\pgfpathlineto{\pgfqpoint{3.521045in}{1.387441in}}%
\pgfpathlineto{\pgfqpoint{3.533659in}{1.388118in}}%
\pgfpathlineto{\pgfqpoint{3.546272in}{1.396593in}}%
\pgfpathlineto{\pgfqpoint{3.558886in}{1.397846in}}%
\pgfpathlineto{\pgfqpoint{3.571499in}{1.408578in}}%
\pgfpathlineto{\pgfqpoint{3.584113in}{1.417562in}}%
\pgfpathlineto{\pgfqpoint{3.596727in}{1.417739in}}%
\pgfpathlineto{\pgfqpoint{3.609340in}{1.432045in}}%
\pgfpathlineto{\pgfqpoint{3.621954in}{1.434222in}}%
\pgfpathlineto{\pgfqpoint{3.647181in}{1.443139in}}%
\pgfpathlineto{\pgfqpoint{3.672408in}{1.443996in}}%
\pgfpathlineto{\pgfqpoint{3.685022in}{1.449696in}}%
\pgfpathlineto{\pgfqpoint{3.697635in}{1.461344in}}%
\pgfpathlineto{\pgfqpoint{3.710249in}{1.468185in}}%
\pgfpathlineto{\pgfqpoint{3.722862in}{1.515238in}}%
\pgfpathlineto{\pgfqpoint{3.735476in}{1.530437in}}%
\pgfpathlineto{\pgfqpoint{3.748089in}{1.532934in}}%
\pgfpathlineto{\pgfqpoint{3.760703in}{1.579302in}}%
\pgfpathlineto{\pgfqpoint{3.773317in}{1.584672in}}%
\pgfpathlineto{\pgfqpoint{3.785930in}{1.619645in}}%
\pgfpathlineto{\pgfqpoint{3.798544in}{1.624664in}}%
\pgfpathlineto{\pgfqpoint{3.811157in}{1.632525in}}%
\pgfpathlineto{\pgfqpoint{3.823771in}{1.638330in}}%
\pgfpathlineto{\pgfqpoint{3.836384in}{1.638474in}}%
\pgfpathlineto{\pgfqpoint{3.848998in}{1.658441in}}%
\pgfpathlineto{\pgfqpoint{3.861611in}{1.663013in}}%
\pgfpathlineto{\pgfqpoint{3.874225in}{1.673292in}}%
\pgfpathlineto{\pgfqpoint{3.886839in}{1.675870in}}%
\pgfpathlineto{\pgfqpoint{3.899452in}{1.686569in}}%
\pgfpathlineto{\pgfqpoint{3.912066in}{1.686957in}}%
\pgfpathlineto{\pgfqpoint{3.924679in}{1.697744in}}%
\pgfpathlineto{\pgfqpoint{3.937293in}{1.706153in}}%
\pgfpathlineto{\pgfqpoint{3.949906in}{1.779684in}}%
\pgfpathlineto{\pgfqpoint{3.962520in}{1.785592in}}%
\pgfpathlineto{\pgfqpoint{3.975134in}{1.865537in}}%
\pgfpathlineto{\pgfqpoint{3.987747in}{1.901472in}}%
\pgfpathlineto{\pgfqpoint{4.000361in}{1.913208in}}%
\pgfpathlineto{\pgfqpoint{4.012974in}{1.922872in}}%
\pgfpathlineto{\pgfqpoint{4.025588in}{1.952685in}}%
\pgfpathlineto{\pgfqpoint{4.038201in}{1.969594in}}%
\pgfpathlineto{\pgfqpoint{4.050815in}{1.971014in}}%
\pgfpathlineto{\pgfqpoint{4.063429in}{1.995957in}}%
\pgfpathlineto{\pgfqpoint{4.076042in}{2.067370in}}%
\pgfpathlineto{\pgfqpoint{4.088656in}{2.068699in}}%
\pgfpathlineto{\pgfqpoint{4.101269in}{2.099816in}}%
\pgfpathlineto{\pgfqpoint{4.113883in}{2.104935in}}%
\pgfpathlineto{\pgfqpoint{4.126496in}{2.116029in}}%
\pgfpathlineto{\pgfqpoint{4.139110in}{2.156021in}}%
\pgfpathlineto{\pgfqpoint{4.151724in}{2.168623in}}%
\pgfpathlineto{\pgfqpoint{4.176951in}{2.175082in}}%
\pgfpathlineto{\pgfqpoint{4.189564in}{2.197260in}}%
\pgfpathlineto{\pgfqpoint{4.202178in}{2.208581in}}%
\pgfpathlineto{\pgfqpoint{4.214791in}{2.269455in}}%
\pgfpathlineto{\pgfqpoint{4.227405in}{2.274202in}}%
\pgfpathlineto{\pgfqpoint{4.240018in}{2.284535in}}%
\pgfpathlineto{\pgfqpoint{4.252632in}{2.287747in}}%
\pgfpathlineto{\pgfqpoint{4.265246in}{2.303369in}}%
\pgfpathlineto{\pgfqpoint{4.277859in}{2.305275in}}%
\pgfpathlineto{\pgfqpoint{4.277859in}{2.305275in}}%
\pgfusepath{stroke}%
\end{pgfscope}%
\begin{pgfscope}%
\pgfpathrectangle{\pgfqpoint{0.708220in}{0.535823in}}{\pgfqpoint{5.045427in}{1.769453in}}%
\pgfusepath{clip}%
\pgfsetbuttcap%
\pgfsetroundjoin%
\pgfsetlinewidth{1.003750pt}%
\definecolor{currentstroke}{rgb}{0.839216,0.152941,0.156863}%
\pgfsetstrokecolor{currentstroke}%
\pgfsetdash{{6.400000pt}{1.600000pt}{1.000000pt}{1.600000pt}}{0.000000pt}%
\pgfpathmoveto{\pgfqpoint{0.708220in}{1.208232in}}%
\pgfpathlineto{\pgfqpoint{0.758674in}{1.211147in}}%
\pgfpathlineto{\pgfqpoint{0.809128in}{1.212104in}}%
\pgfpathlineto{\pgfqpoint{0.821742in}{1.214929in}}%
\pgfpathlineto{\pgfqpoint{0.846969in}{1.215856in}}%
\pgfpathlineto{\pgfqpoint{0.872196in}{1.219497in}}%
\pgfpathlineto{\pgfqpoint{0.910037in}{1.220391in}}%
\pgfpathlineto{\pgfqpoint{0.960491in}{1.223033in}}%
\pgfpathlineto{\pgfqpoint{1.036173in}{1.223901in}}%
\pgfpathlineto{\pgfqpoint{1.074013in}{1.225618in}}%
\pgfpathlineto{\pgfqpoint{1.162308in}{1.226468in}}%
\pgfpathlineto{\pgfqpoint{1.187535in}{1.227312in}}%
\pgfpathlineto{\pgfqpoint{1.212763in}{1.228150in}}%
\pgfpathlineto{\pgfqpoint{1.237990in}{1.228983in}}%
\pgfpathlineto{\pgfqpoint{1.275830in}{1.229810in}}%
\pgfpathlineto{\pgfqpoint{1.301057in}{1.230631in}}%
\pgfpathlineto{\pgfqpoint{1.326285in}{1.231447in}}%
\pgfpathlineto{\pgfqpoint{1.351512in}{1.232257in}}%
\pgfpathlineto{\pgfqpoint{1.427193in}{1.233062in}}%
\pgfpathlineto{\pgfqpoint{1.452420in}{1.233862in}}%
\pgfpathlineto{\pgfqpoint{1.490261in}{1.234656in}}%
\pgfpathlineto{\pgfqpoint{1.515488in}{1.235446in}}%
\pgfpathlineto{\pgfqpoint{1.578556in}{1.236230in}}%
\pgfpathlineto{\pgfqpoint{1.616397in}{1.237784in}}%
\pgfpathlineto{\pgfqpoint{1.704692in}{1.238553in}}%
\pgfpathlineto{\pgfqpoint{1.742532in}{1.240078in}}%
\pgfpathlineto{\pgfqpoint{1.755146in}{1.240078in}}%
\pgfpathlineto{\pgfqpoint{1.767759in}{1.241583in}}%
\pgfpathlineto{\pgfqpoint{1.805600in}{1.242329in}}%
\pgfpathlineto{\pgfqpoint{1.818214in}{1.244540in}}%
\pgfpathlineto{\pgfqpoint{1.830827in}{1.245268in}}%
\pgfpathlineto{\pgfqpoint{1.843441in}{1.247427in}}%
\pgfpathlineto{\pgfqpoint{1.931736in}{1.251633in}}%
\pgfpathlineto{\pgfqpoint{1.956963in}{1.252320in}}%
\pgfpathlineto{\pgfqpoint{1.982190in}{1.253003in}}%
\pgfpathlineto{\pgfqpoint{2.007417in}{1.253682in}}%
\pgfpathlineto{\pgfqpoint{2.209234in}{1.263444in}}%
\pgfpathlineto{\pgfqpoint{2.247075in}{1.264068in}}%
\pgfpathlineto{\pgfqpoint{2.259689in}{1.265921in}}%
\pgfpathlineto{\pgfqpoint{2.297529in}{1.266532in}}%
\pgfpathlineto{\pgfqpoint{2.360597in}{1.268947in}}%
\pgfpathlineto{\pgfqpoint{2.411051in}{1.269543in}}%
\pgfpathlineto{\pgfqpoint{2.486733in}{1.272482in}}%
\pgfpathlineto{\pgfqpoint{2.575028in}{1.273062in}}%
\pgfpathlineto{\pgfqpoint{2.600255in}{1.274212in}}%
\pgfpathlineto{\pgfqpoint{2.612868in}{1.275918in}}%
\pgfpathlineto{\pgfqpoint{2.663323in}{1.279260in}}%
\pgfpathlineto{\pgfqpoint{2.739004in}{1.285158in}}%
\pgfpathlineto{\pgfqpoint{2.814685in}{1.288258in}}%
\pgfpathlineto{\pgfqpoint{2.852526in}{1.293257in}}%
\pgfpathlineto{\pgfqpoint{2.890367in}{1.298060in}}%
\pgfpathlineto{\pgfqpoint{2.915594in}{1.298998in}}%
\pgfpathlineto{\pgfqpoint{2.928208in}{1.304480in}}%
\pgfpathlineto{\pgfqpoint{2.940821in}{1.306254in}}%
\pgfpathlineto{\pgfqpoint{2.953435in}{1.310579in}}%
\pgfpathlineto{\pgfqpoint{3.003889in}{1.312686in}}%
\pgfpathlineto{\pgfqpoint{3.016503in}{1.314345in}}%
\pgfpathlineto{\pgfqpoint{3.079570in}{1.317195in}}%
\pgfpathlineto{\pgfqpoint{3.092184in}{1.318795in}}%
\pgfpathlineto{\pgfqpoint{3.104797in}{1.318795in}}%
\pgfpathlineto{\pgfqpoint{3.117411in}{1.320765in}}%
\pgfpathlineto{\pgfqpoint{3.167865in}{1.324991in}}%
\pgfpathlineto{\pgfqpoint{3.180479in}{1.326492in}}%
\pgfpathlineto{\pgfqpoint{3.193092in}{1.332318in}}%
\pgfpathlineto{\pgfqpoint{3.205706in}{1.348914in}}%
\pgfpathlineto{\pgfqpoint{3.243547in}{1.354376in}}%
\pgfpathlineto{\pgfqpoint{3.256160in}{1.361017in}}%
\pgfpathlineto{\pgfqpoint{3.281387in}{1.361577in}}%
\pgfpathlineto{\pgfqpoint{3.294001in}{1.368112in}}%
\pgfpathlineto{\pgfqpoint{3.306615in}{1.377302in}}%
\pgfpathlineto{\pgfqpoint{3.319228in}{1.378281in}}%
\pgfpathlineto{\pgfqpoint{3.344455in}{1.388567in}}%
\pgfpathlineto{\pgfqpoint{3.357069in}{1.388791in}}%
\pgfpathlineto{\pgfqpoint{3.369682in}{1.391009in}}%
\pgfpathlineto{\pgfqpoint{3.382296in}{1.416140in}}%
\pgfpathlineto{\pgfqpoint{3.394910in}{1.425599in}}%
\pgfpathlineto{\pgfqpoint{3.407523in}{1.441843in}}%
\pgfpathlineto{\pgfqpoint{3.420137in}{1.447917in}}%
\pgfpathlineto{\pgfqpoint{3.445364in}{1.468068in}}%
\pgfpathlineto{\pgfqpoint{3.457977in}{1.497505in}}%
\pgfpathlineto{\pgfqpoint{3.470591in}{1.503678in}}%
\pgfpathlineto{\pgfqpoint{3.483204in}{1.524921in}}%
\pgfpathlineto{\pgfqpoint{3.495818in}{1.525797in}}%
\pgfpathlineto{\pgfqpoint{3.521045in}{1.530647in}}%
\pgfpathlineto{\pgfqpoint{3.546272in}{1.548565in}}%
\pgfpathlineto{\pgfqpoint{3.571499in}{1.551075in}}%
\pgfpathlineto{\pgfqpoint{3.584113in}{1.555660in}}%
\pgfpathlineto{\pgfqpoint{3.609340in}{1.557526in}}%
\pgfpathlineto{\pgfqpoint{3.621954in}{1.571052in}}%
\pgfpathlineto{\pgfqpoint{3.634567in}{1.574175in}}%
\pgfpathlineto{\pgfqpoint{3.647181in}{1.604291in}}%
\pgfpathlineto{\pgfqpoint{3.659794in}{1.610004in}}%
\pgfpathlineto{\pgfqpoint{3.672408in}{1.619173in}}%
\pgfpathlineto{\pgfqpoint{3.685022in}{1.623690in}}%
\pgfpathlineto{\pgfqpoint{3.710249in}{1.627099in}}%
\pgfpathlineto{\pgfqpoint{3.722862in}{1.735635in}}%
\pgfpathlineto{\pgfqpoint{3.735476in}{1.759209in}}%
\pgfpathlineto{\pgfqpoint{3.748089in}{1.767337in}}%
\pgfpathlineto{\pgfqpoint{3.773317in}{1.798415in}}%
\pgfpathlineto{\pgfqpoint{3.785930in}{1.801398in}}%
\pgfpathlineto{\pgfqpoint{3.798544in}{1.816240in}}%
\pgfpathlineto{\pgfqpoint{3.823771in}{1.854632in}}%
\pgfpathlineto{\pgfqpoint{3.836384in}{1.858387in}}%
\pgfpathlineto{\pgfqpoint{3.848998in}{1.879597in}}%
\pgfpathlineto{\pgfqpoint{3.861611in}{1.883222in}}%
\pgfpathlineto{\pgfqpoint{3.874225in}{1.916059in}}%
\pgfpathlineto{\pgfqpoint{3.886839in}{1.916766in}}%
\pgfpathlineto{\pgfqpoint{3.912066in}{1.932424in}}%
\pgfpathlineto{\pgfqpoint{3.924679in}{1.933837in}}%
\pgfpathlineto{\pgfqpoint{3.937293in}{2.009996in}}%
\pgfpathlineto{\pgfqpoint{3.949906in}{2.012147in}}%
\pgfpathlineto{\pgfqpoint{3.975134in}{2.013790in}}%
\pgfpathlineto{\pgfqpoint{3.987747in}{2.026688in}}%
\pgfpathlineto{\pgfqpoint{4.000361in}{2.055701in}}%
\pgfpathlineto{\pgfqpoint{4.025588in}{2.058238in}}%
\pgfpathlineto{\pgfqpoint{4.038201in}{2.059673in}}%
\pgfpathlineto{\pgfqpoint{4.063429in}{2.065637in}}%
\pgfpathlineto{\pgfqpoint{4.076042in}{2.086302in}}%
\pgfpathlineto{\pgfqpoint{4.088656in}{2.099460in}}%
\pgfpathlineto{\pgfqpoint{4.139110in}{2.102731in}}%
\pgfpathlineto{\pgfqpoint{4.151724in}{2.107888in}}%
\pgfpathlineto{\pgfqpoint{4.164337in}{2.116061in}}%
\pgfpathlineto{\pgfqpoint{4.202178in}{2.180818in}}%
\pgfpathlineto{\pgfqpoint{4.214791in}{2.186078in}}%
\pgfpathlineto{\pgfqpoint{4.227405in}{2.208213in}}%
\pgfpathlineto{\pgfqpoint{4.240018in}{2.223616in}}%
\pgfpathlineto{\pgfqpoint{4.252632in}{2.253959in}}%
\pgfpathlineto{\pgfqpoint{4.265246in}{2.258077in}}%
\pgfpathlineto{\pgfqpoint{4.277859in}{2.268515in}}%
\pgfpathlineto{\pgfqpoint{4.290473in}{2.305275in}}%
\pgfpathlineto{\pgfqpoint{4.290473in}{2.305275in}}%
\pgfusepath{stroke}%
\end{pgfscope}%
\begin{pgfscope}%
\pgfpathrectangle{\pgfqpoint{0.708220in}{0.535823in}}{\pgfqpoint{5.045427in}{1.769453in}}%
\pgfusepath{clip}%
\pgfsetbuttcap%
\pgfsetroundjoin%
\pgfsetlinewidth{1.003750pt}%
\definecolor{currentstroke}{rgb}{0.580392,0.403922,0.741176}%
\pgfsetstrokecolor{currentstroke}%
\pgfsetdash{{6.400000pt}{1.600000pt}{1.000000pt}{1.600000pt}}{0.000000pt}%
\pgfpathmoveto{\pgfqpoint{0.708220in}{1.370215in}}%
\pgfpathlineto{\pgfqpoint{0.783901in}{1.373810in}}%
\pgfpathlineto{\pgfqpoint{0.809128in}{1.375817in}}%
\pgfpathlineto{\pgfqpoint{0.859583in}{1.379736in}}%
\pgfpathlineto{\pgfqpoint{0.897423in}{1.381174in}}%
\pgfpathlineto{\pgfqpoint{0.922650in}{1.381412in}}%
\pgfpathlineto{\pgfqpoint{0.947878in}{1.385157in}}%
\pgfpathlineto{\pgfqpoint{0.960491in}{1.385388in}}%
\pgfpathlineto{\pgfqpoint{0.973105in}{1.387667in}}%
\pgfpathlineto{\pgfqpoint{0.998332in}{1.388118in}}%
\pgfpathlineto{\pgfqpoint{1.036173in}{1.392103in}}%
\pgfpathlineto{\pgfqpoint{1.061400in}{1.409148in}}%
\pgfpathlineto{\pgfqpoint{1.086627in}{1.409715in}}%
\pgfpathlineto{\pgfqpoint{1.237990in}{1.422243in}}%
\pgfpathlineto{\pgfqpoint{1.275830in}{1.424100in}}%
\pgfpathlineto{\pgfqpoint{1.288444in}{1.427569in}}%
\pgfpathlineto{\pgfqpoint{1.364125in}{1.432358in}}%
\pgfpathlineto{\pgfqpoint{1.414580in}{1.433604in}}%
\pgfpathlineto{\pgfqpoint{1.427193in}{1.435602in}}%
\pgfpathlineto{\pgfqpoint{1.515488in}{1.440532in}}%
\pgfpathlineto{\pgfqpoint{1.528102in}{1.442276in}}%
\pgfpathlineto{\pgfqpoint{1.565942in}{1.443426in}}%
\pgfpathlineto{\pgfqpoint{1.578556in}{1.445972in}}%
\pgfpathlineto{\pgfqpoint{1.591170in}{1.446531in}}%
\pgfpathlineto{\pgfqpoint{1.603783in}{1.448878in}}%
\pgfpathlineto{\pgfqpoint{1.641624in}{1.450777in}}%
\pgfpathlineto{\pgfqpoint{1.666851in}{1.453176in}}%
\pgfpathlineto{\pgfqpoint{1.679464in}{1.456046in}}%
\pgfpathlineto{\pgfqpoint{1.717305in}{1.458471in}}%
\pgfpathlineto{\pgfqpoint{1.729919in}{1.461221in}}%
\pgfpathlineto{\pgfqpoint{1.742532in}{1.462816in}}%
\pgfpathlineto{\pgfqpoint{1.780373in}{1.464392in}}%
\pgfpathlineto{\pgfqpoint{1.792987in}{1.465828in}}%
\pgfpathlineto{\pgfqpoint{1.881282in}{1.468418in}}%
\pgfpathlineto{\pgfqpoint{1.906509in}{1.470954in}}%
\pgfpathlineto{\pgfqpoint{1.969577in}{1.471863in}}%
\pgfpathlineto{\pgfqpoint{1.994804in}{1.473215in}}%
\pgfpathlineto{\pgfqpoint{2.057871in}{1.476747in}}%
\pgfpathlineto{\pgfqpoint{2.070485in}{1.480496in}}%
\pgfpathlineto{\pgfqpoint{2.120939in}{1.483002in}}%
\pgfpathlineto{\pgfqpoint{2.184007in}{1.489346in}}%
\pgfpathlineto{\pgfqpoint{2.360597in}{1.495469in}}%
\pgfpathlineto{\pgfqpoint{2.385824in}{1.502626in}}%
\pgfpathlineto{\pgfqpoint{2.411051in}{1.504979in}}%
\pgfpathlineto{\pgfqpoint{2.423665in}{1.510134in}}%
\pgfpathlineto{\pgfqpoint{2.511960in}{1.514841in}}%
\pgfpathlineto{\pgfqpoint{2.537187in}{1.519066in}}%
\pgfpathlineto{\pgfqpoint{2.549801in}{1.520746in}}%
\pgfpathlineto{\pgfqpoint{2.562414in}{1.520821in}}%
\pgfpathlineto{\pgfqpoint{2.612868in}{1.525870in}}%
\pgfpathlineto{\pgfqpoint{2.625482in}{1.528957in}}%
\pgfpathlineto{\pgfqpoint{2.638096in}{1.530157in}}%
\pgfpathlineto{\pgfqpoint{2.650709in}{1.530227in}}%
\pgfpathlineto{\pgfqpoint{2.663323in}{1.532176in}}%
\pgfpathlineto{\pgfqpoint{2.688550in}{1.532728in}}%
\pgfpathlineto{\pgfqpoint{2.701163in}{1.534908in}}%
\pgfpathlineto{\pgfqpoint{2.739004in}{1.535581in}}%
\pgfpathlineto{\pgfqpoint{2.751618in}{1.541800in}}%
\pgfpathlineto{\pgfqpoint{2.764231in}{1.552194in}}%
\pgfpathlineto{\pgfqpoint{2.776845in}{1.553302in}}%
\pgfpathlineto{\pgfqpoint{2.802072in}{1.556907in}}%
\pgfpathlineto{\pgfqpoint{2.852526in}{1.562634in}}%
\pgfpathlineto{\pgfqpoint{2.865140in}{1.566656in}}%
\pgfpathlineto{\pgfqpoint{2.902980in}{1.568156in}}%
\pgfpathlineto{\pgfqpoint{2.915594in}{1.572201in}}%
\pgfpathlineto{\pgfqpoint{2.928208in}{1.572747in}}%
\pgfpathlineto{\pgfqpoint{2.940821in}{1.574566in}}%
\pgfpathlineto{\pgfqpoint{2.978662in}{1.576742in}}%
\pgfpathlineto{\pgfqpoint{3.029116in}{1.578218in}}%
\pgfpathlineto{\pgfqpoint{3.054343in}{1.579957in}}%
\pgfpathlineto{\pgfqpoint{3.066957in}{1.580702in}}%
\pgfpathlineto{\pgfqpoint{3.079570in}{1.584178in}}%
\pgfpathlineto{\pgfqpoint{3.130025in}{1.588431in}}%
\pgfpathlineto{\pgfqpoint{3.142638in}{1.591441in}}%
\pgfpathlineto{\pgfqpoint{3.167865in}{1.592161in}}%
\pgfpathlineto{\pgfqpoint{3.193092in}{1.595576in}}%
\pgfpathlineto{\pgfqpoint{3.205706in}{1.600840in}}%
\pgfpathlineto{\pgfqpoint{3.230933in}{1.601428in}}%
\pgfpathlineto{\pgfqpoint{3.243547in}{1.603908in}}%
\pgfpathlineto{\pgfqpoint{3.268774in}{1.610659in}}%
\pgfpathlineto{\pgfqpoint{3.306615in}{1.613705in}}%
\pgfpathlineto{\pgfqpoint{3.319228in}{1.623657in}}%
\pgfpathlineto{\pgfqpoint{3.331842in}{1.624761in}}%
\pgfpathlineto{\pgfqpoint{3.369682in}{1.642871in}}%
\pgfpathlineto{\pgfqpoint{3.382296in}{1.643870in}}%
\pgfpathlineto{\pgfqpoint{3.394910in}{1.649408in}}%
\pgfpathlineto{\pgfqpoint{3.407523in}{1.656363in}}%
\pgfpathlineto{\pgfqpoint{3.457977in}{1.660580in}}%
\pgfpathlineto{\pgfqpoint{3.470591in}{1.660748in}}%
\pgfpathlineto{\pgfqpoint{3.495818in}{1.663908in}}%
\pgfpathlineto{\pgfqpoint{3.508432in}{1.666345in}}%
\pgfpathlineto{\pgfqpoint{3.521045in}{1.667056in}}%
\pgfpathlineto{\pgfqpoint{3.533659in}{1.670102in}}%
\pgfpathlineto{\pgfqpoint{3.558886in}{1.671587in}}%
\pgfpathlineto{\pgfqpoint{3.571499in}{1.674760in}}%
\pgfpathlineto{\pgfqpoint{3.584113in}{1.676527in}}%
\pgfpathlineto{\pgfqpoint{3.596727in}{1.680848in}}%
\pgfpathlineto{\pgfqpoint{3.609340in}{1.702893in}}%
\pgfpathlineto{\pgfqpoint{3.621954in}{1.716742in}}%
\pgfpathlineto{\pgfqpoint{3.634567in}{1.716772in}}%
\pgfpathlineto{\pgfqpoint{3.659794in}{1.722314in}}%
\pgfpathlineto{\pgfqpoint{3.672408in}{1.727615in}}%
\pgfpathlineto{\pgfqpoint{3.685022in}{1.751796in}}%
\pgfpathlineto{\pgfqpoint{3.697635in}{1.773026in}}%
\pgfpathlineto{\pgfqpoint{3.710249in}{1.775887in}}%
\pgfpathlineto{\pgfqpoint{3.722862in}{1.775990in}}%
\pgfpathlineto{\pgfqpoint{3.735476in}{1.778389in}}%
\pgfpathlineto{\pgfqpoint{3.748089in}{1.788378in}}%
\pgfpathlineto{\pgfqpoint{3.760703in}{1.817444in}}%
\pgfpathlineto{\pgfqpoint{3.773317in}{1.821369in}}%
\pgfpathlineto{\pgfqpoint{3.785930in}{1.832042in}}%
\pgfpathlineto{\pgfqpoint{3.798544in}{1.876494in}}%
\pgfpathlineto{\pgfqpoint{3.823771in}{1.897012in}}%
\pgfpathlineto{\pgfqpoint{3.836384in}{1.902265in}}%
\pgfpathlineto{\pgfqpoint{3.861611in}{1.925456in}}%
\pgfpathlineto{\pgfqpoint{3.874225in}{1.959041in}}%
\pgfpathlineto{\pgfqpoint{3.886839in}{1.965096in}}%
\pgfpathlineto{\pgfqpoint{3.899452in}{1.997922in}}%
\pgfpathlineto{\pgfqpoint{3.912066in}{2.079696in}}%
\pgfpathlineto{\pgfqpoint{3.924679in}{2.089409in}}%
\pgfpathlineto{\pgfqpoint{3.937293in}{2.097534in}}%
\pgfpathlineto{\pgfqpoint{3.949906in}{2.112552in}}%
\pgfpathlineto{\pgfqpoint{3.962520in}{2.124502in}}%
\pgfpathlineto{\pgfqpoint{3.975134in}{2.128606in}}%
\pgfpathlineto{\pgfqpoint{3.987747in}{2.161637in}}%
\pgfpathlineto{\pgfqpoint{4.000361in}{2.166166in}}%
\pgfpathlineto{\pgfqpoint{4.025588in}{2.170200in}}%
\pgfpathlineto{\pgfqpoint{4.038201in}{2.173836in}}%
\pgfpathlineto{\pgfqpoint{4.050815in}{2.191333in}}%
\pgfpathlineto{\pgfqpoint{4.076042in}{2.278083in}}%
\pgfpathlineto{\pgfqpoint{4.088656in}{2.278171in}}%
\pgfpathlineto{\pgfqpoint{4.101269in}{2.285507in}}%
\pgfpathlineto{\pgfqpoint{4.113883in}{2.305275in}}%
\pgfpathlineto{\pgfqpoint{4.113883in}{2.305275in}}%
\pgfusepath{stroke}%
\end{pgfscope}%
\begin{pgfscope}%
\pgfpathrectangle{\pgfqpoint{0.708220in}{0.535823in}}{\pgfqpoint{5.045427in}{1.769453in}}%
\pgfusepath{clip}%
\pgfsetbuttcap%
\pgfsetroundjoin%
\pgfsetlinewidth{1.003750pt}%
\definecolor{currentstroke}{rgb}{0.549020,0.337255,0.294118}%
\pgfsetstrokecolor{currentstroke}%
\pgfsetdash{{6.400000pt}{1.600000pt}{1.000000pt}{1.600000pt}}{0.000000pt}%
\pgfpathmoveto{\pgfqpoint{0.708220in}{1.372282in}}%
\pgfpathlineto{\pgfqpoint{0.771288in}{1.373303in}}%
\pgfpathlineto{\pgfqpoint{0.834356in}{1.379011in}}%
\pgfpathlineto{\pgfqpoint{0.897423in}{1.381886in}}%
\pgfpathlineto{\pgfqpoint{0.910037in}{1.383065in}}%
\pgfpathlineto{\pgfqpoint{0.935264in}{1.383299in}}%
\pgfpathlineto{\pgfqpoint{0.947878in}{1.385388in}}%
\pgfpathlineto{\pgfqpoint{0.985718in}{1.389015in}}%
\pgfpathlineto{\pgfqpoint{1.023559in}{1.394902in}}%
\pgfpathlineto{\pgfqpoint{1.036173in}{1.394902in}}%
\pgfpathlineto{\pgfqpoint{1.048786in}{1.398880in}}%
\pgfpathlineto{\pgfqpoint{1.086627in}{1.402932in}}%
\pgfpathlineto{\pgfqpoint{1.099240in}{1.411956in}}%
\pgfpathlineto{\pgfqpoint{1.162308in}{1.413061in}}%
\pgfpathlineto{\pgfqpoint{1.263217in}{1.422243in}}%
\pgfpathlineto{\pgfqpoint{1.439807in}{1.429987in}}%
\pgfpathlineto{\pgfqpoint{1.452420in}{1.431257in}}%
\pgfpathlineto{\pgfqpoint{1.465034in}{1.434530in}}%
\pgfpathlineto{\pgfqpoint{1.490261in}{1.435450in}}%
\pgfpathlineto{\pgfqpoint{1.528102in}{1.435907in}}%
\pgfpathlineto{\pgfqpoint{1.540715in}{1.437418in}}%
\pgfpathlineto{\pgfqpoint{1.578556in}{1.438167in}}%
\pgfpathlineto{\pgfqpoint{1.629010in}{1.439355in}}%
\pgfpathlineto{\pgfqpoint{1.641624in}{1.442565in}}%
\pgfpathlineto{\pgfqpoint{1.679464in}{1.444138in}}%
\pgfpathlineto{\pgfqpoint{1.692078in}{1.449015in}}%
\pgfpathlineto{\pgfqpoint{1.704692in}{1.449015in}}%
\pgfpathlineto{\pgfqpoint{1.717305in}{1.452116in}}%
\pgfpathlineto{\pgfqpoint{1.742532in}{1.454097in}}%
\pgfpathlineto{\pgfqpoint{1.755146in}{1.455788in}}%
\pgfpathlineto{\pgfqpoint{1.767759in}{1.459227in}}%
\pgfpathlineto{\pgfqpoint{1.780373in}{1.460849in}}%
\pgfpathlineto{\pgfqpoint{1.792987in}{1.465828in}}%
\pgfpathlineto{\pgfqpoint{1.982190in}{1.468418in}}%
\pgfpathlineto{\pgfqpoint{1.994804in}{1.470038in}}%
\pgfpathlineto{\pgfqpoint{2.032644in}{1.470840in}}%
\pgfpathlineto{\pgfqpoint{2.057871in}{1.473327in}}%
\pgfpathlineto{\pgfqpoint{2.095712in}{1.474662in}}%
\pgfpathlineto{\pgfqpoint{2.120939in}{1.477722in}}%
\pgfpathlineto{\pgfqpoint{2.133553in}{1.482173in}}%
\pgfpathlineto{\pgfqpoint{2.184007in}{1.483518in}}%
\pgfpathlineto{\pgfqpoint{2.221848in}{1.485963in}}%
\pgfpathlineto{\pgfqpoint{2.310143in}{1.491389in}}%
\pgfpathlineto{\pgfqpoint{2.373211in}{1.493114in}}%
\pgfpathlineto{\pgfqpoint{2.461506in}{1.502802in}}%
\pgfpathlineto{\pgfqpoint{2.486733in}{1.503852in}}%
\pgfpathlineto{\pgfqpoint{2.499346in}{1.504807in}}%
\pgfpathlineto{\pgfqpoint{2.511960in}{1.510217in}}%
\pgfpathlineto{\pgfqpoint{2.575028in}{1.515476in}}%
\pgfpathlineto{\pgfqpoint{2.587641in}{1.515950in}}%
\pgfpathlineto{\pgfqpoint{2.650709in}{1.525360in}}%
\pgfpathlineto{\pgfqpoint{2.688550in}{1.526667in}}%
\pgfpathlineto{\pgfqpoint{2.713777in}{1.530087in}}%
\pgfpathlineto{\pgfqpoint{2.739004in}{1.530717in}}%
\pgfpathlineto{\pgfqpoint{2.751618in}{1.533481in}}%
\pgfpathlineto{\pgfqpoint{2.802072in}{1.534975in}}%
\pgfpathlineto{\pgfqpoint{2.814685in}{1.536917in}}%
\pgfpathlineto{\pgfqpoint{2.839913in}{1.539220in}}%
\pgfpathlineto{\pgfqpoint{2.852526in}{1.550957in}}%
\pgfpathlineto{\pgfqpoint{2.865140in}{1.557357in}}%
\pgfpathlineto{\pgfqpoint{2.877753in}{1.559694in}}%
\pgfpathlineto{\pgfqpoint{2.890367in}{1.559969in}}%
\pgfpathlineto{\pgfqpoint{2.915594in}{1.562149in}}%
\pgfpathlineto{\pgfqpoint{2.928208in}{1.566082in}}%
\pgfpathlineto{\pgfqpoint{2.966048in}{1.566812in}}%
\pgfpathlineto{\pgfqpoint{2.978662in}{1.567641in}}%
\pgfpathlineto{\pgfqpoint{2.991275in}{1.570246in}}%
\pgfpathlineto{\pgfqpoint{3.003889in}{1.570246in}}%
\pgfpathlineto{\pgfqpoint{3.029116in}{1.574566in}}%
\pgfpathlineto{\pgfqpoint{3.079570in}{1.578076in}}%
\pgfpathlineto{\pgfqpoint{3.092184in}{1.579817in}}%
\pgfpathlineto{\pgfqpoint{3.130025in}{1.581350in}}%
\pgfpathlineto{\pgfqpoint{3.142638in}{1.585031in}}%
\pgfpathlineto{\pgfqpoint{3.180479in}{1.588561in}}%
\pgfpathlineto{\pgfqpoint{3.218320in}{1.600879in}}%
\pgfpathlineto{\pgfqpoint{3.230933in}{1.601428in}}%
\pgfpathlineto{\pgfqpoint{3.256160in}{1.605659in}}%
\pgfpathlineto{\pgfqpoint{3.268774in}{1.609346in}}%
\pgfpathlineto{\pgfqpoint{3.281387in}{1.609822in}}%
\pgfpathlineto{\pgfqpoint{3.306615in}{1.617056in}}%
\pgfpathlineto{\pgfqpoint{3.331842in}{1.620016in}}%
\pgfpathlineto{\pgfqpoint{3.344455in}{1.623494in}}%
\pgfpathlineto{\pgfqpoint{3.382296in}{1.626239in}}%
\pgfpathlineto{\pgfqpoint{3.394910in}{1.629485in}}%
\pgfpathlineto{\pgfqpoint{3.407523in}{1.629578in}}%
\pgfpathlineto{\pgfqpoint{3.420137in}{1.630907in}}%
\pgfpathlineto{\pgfqpoint{3.432750in}{1.638532in}}%
\pgfpathlineto{\pgfqpoint{3.445364in}{1.655034in}}%
\pgfpathlineto{\pgfqpoint{3.457977in}{1.655210in}}%
\pgfpathlineto{\pgfqpoint{3.470591in}{1.657555in}}%
\pgfpathlineto{\pgfqpoint{3.495818in}{1.658072in}}%
\pgfpathlineto{\pgfqpoint{3.584113in}{1.664772in}}%
\pgfpathlineto{\pgfqpoint{3.596727in}{1.675295in}}%
\pgfpathlineto{\pgfqpoint{3.609340in}{1.675636in}}%
\pgfpathlineto{\pgfqpoint{3.647181in}{1.683851in}}%
\pgfpathlineto{\pgfqpoint{3.659794in}{1.694466in}}%
\pgfpathlineto{\pgfqpoint{3.672408in}{1.696637in}}%
\pgfpathlineto{\pgfqpoint{3.697635in}{1.697513in}}%
\pgfpathlineto{\pgfqpoint{3.710249in}{1.697566in}}%
\pgfpathlineto{\pgfqpoint{3.722862in}{1.704568in}}%
\pgfpathlineto{\pgfqpoint{3.748089in}{1.710439in}}%
\pgfpathlineto{\pgfqpoint{3.760703in}{1.732054in}}%
\pgfpathlineto{\pgfqpoint{3.773317in}{1.743746in}}%
\pgfpathlineto{\pgfqpoint{3.811157in}{1.752320in}}%
\pgfpathlineto{\pgfqpoint{3.823771in}{1.805167in}}%
\pgfpathlineto{\pgfqpoint{3.836384in}{1.812629in}}%
\pgfpathlineto{\pgfqpoint{3.848998in}{1.864210in}}%
\pgfpathlineto{\pgfqpoint{3.861611in}{1.890356in}}%
\pgfpathlineto{\pgfqpoint{3.899452in}{1.894613in}}%
\pgfpathlineto{\pgfqpoint{3.912066in}{1.894694in}}%
\pgfpathlineto{\pgfqpoint{3.924679in}{1.899010in}}%
\pgfpathlineto{\pgfqpoint{3.937293in}{1.906399in}}%
\pgfpathlineto{\pgfqpoint{3.949906in}{1.931520in}}%
\pgfpathlineto{\pgfqpoint{3.962520in}{1.967074in}}%
\pgfpathlineto{\pgfqpoint{3.975134in}{2.010297in}}%
\pgfpathlineto{\pgfqpoint{4.012974in}{2.012598in}}%
\pgfpathlineto{\pgfqpoint{4.025588in}{2.017048in}}%
\pgfpathlineto{\pgfqpoint{4.038201in}{2.042673in}}%
\pgfpathlineto{\pgfqpoint{4.050815in}{2.061181in}}%
\pgfpathlineto{\pgfqpoint{4.063429in}{2.066738in}}%
\pgfpathlineto{\pgfqpoint{4.088656in}{2.068735in}}%
\pgfpathlineto{\pgfqpoint{4.101269in}{2.072767in}}%
\pgfpathlineto{\pgfqpoint{4.113883in}{2.073537in}}%
\pgfpathlineto{\pgfqpoint{4.126496in}{2.080522in}}%
\pgfpathlineto{\pgfqpoint{4.139110in}{2.090198in}}%
\pgfpathlineto{\pgfqpoint{4.151724in}{2.093140in}}%
\pgfpathlineto{\pgfqpoint{4.164337in}{2.093666in}}%
\pgfpathlineto{\pgfqpoint{4.176951in}{2.115130in}}%
\pgfpathlineto{\pgfqpoint{4.189564in}{2.156864in}}%
\pgfpathlineto{\pgfqpoint{4.202178in}{2.164009in}}%
\pgfpathlineto{\pgfqpoint{4.214791in}{2.169487in}}%
\pgfpathlineto{\pgfqpoint{4.227405in}{2.186298in}}%
\pgfpathlineto{\pgfqpoint{4.240018in}{2.196415in}}%
\pgfpathlineto{\pgfqpoint{4.252632in}{2.220210in}}%
\pgfpathlineto{\pgfqpoint{4.265246in}{2.224771in}}%
\pgfpathlineto{\pgfqpoint{4.277859in}{2.230741in}}%
\pgfpathlineto{\pgfqpoint{4.290473in}{2.245595in}}%
\pgfpathlineto{\pgfqpoint{4.303086in}{2.246548in}}%
\pgfpathlineto{\pgfqpoint{4.315700in}{2.252430in}}%
\pgfpathlineto{\pgfqpoint{4.328313in}{2.288875in}}%
\pgfpathlineto{\pgfqpoint{4.353541in}{2.305275in}}%
\pgfpathlineto{\pgfqpoint{4.353541in}{2.305275in}}%
\pgfusepath{stroke}%
\end{pgfscope}%
\begin{pgfscope}%
\pgfsetrectcap%
\pgfsetmiterjoin%
\pgfsetlinewidth{0.803000pt}%
\definecolor{currentstroke}{rgb}{0.000000,0.000000,0.000000}%
\pgfsetstrokecolor{currentstroke}%
\pgfsetdash{}{0pt}%
\pgfpathmoveto{\pgfqpoint{0.708220in}{0.535823in}}%
\pgfpathlineto{\pgfqpoint{0.708220in}{2.305275in}}%
\pgfusepath{stroke}%
\end{pgfscope}%
\begin{pgfscope}%
\pgfsetrectcap%
\pgfsetmiterjoin%
\pgfsetlinewidth{0.803000pt}%
\definecolor{currentstroke}{rgb}{0.000000,0.000000,0.000000}%
\pgfsetstrokecolor{currentstroke}%
\pgfsetdash{}{0pt}%
\pgfpathmoveto{\pgfqpoint{5.753646in}{0.535823in}}%
\pgfpathlineto{\pgfqpoint{5.753646in}{2.305275in}}%
\pgfusepath{stroke}%
\end{pgfscope}%
\begin{pgfscope}%
\pgfsetrectcap%
\pgfsetmiterjoin%
\pgfsetlinewidth{0.803000pt}%
\definecolor{currentstroke}{rgb}{0.000000,0.000000,0.000000}%
\pgfsetstrokecolor{currentstroke}%
\pgfsetdash{}{0pt}%
\pgfpathmoveto{\pgfqpoint{0.708220in}{0.535823in}}%
\pgfpathlineto{\pgfqpoint{5.753646in}{0.535823in}}%
\pgfusepath{stroke}%
\end{pgfscope}%
\begin{pgfscope}%
\pgfsetrectcap%
\pgfsetmiterjoin%
\pgfsetlinewidth{0.803000pt}%
\definecolor{currentstroke}{rgb}{0.000000,0.000000,0.000000}%
\pgfsetstrokecolor{currentstroke}%
\pgfsetdash{}{0pt}%
\pgfpathmoveto{\pgfqpoint{0.708220in}{2.305275in}}%
\pgfpathlineto{\pgfqpoint{5.753646in}{2.305275in}}%
\pgfusepath{stroke}%
\end{pgfscope}%
\begin{pgfscope}%
\pgfsetrectcap%
\pgfsetroundjoin%
\pgfsetlinewidth{1.003750pt}%
\definecolor{currentstroke}{rgb}{0.121569,0.466667,0.705882}%
\pgfsetstrokecolor{currentstroke}%
\pgfsetdash{}{0pt}%
\pgfpathmoveto{\pgfqpoint{3.776169in}{1.535662in}}%
\pgfpathlineto{\pgfqpoint{4.026169in}{1.535662in}}%
\pgfusepath{stroke}%
\end{pgfscope}%
\begin{pgfscope}%
\definecolor{textcolor}{rgb}{0.000000,0.000000,0.000000}%
\pgfsetstrokecolor{textcolor}%
\pgfsetfillcolor{textcolor}%
\pgftext[x=4.051169in,y=1.491912in,left,base]{\color{textcolor}\rmfamily\fontsize{9.000000}{10.800000}\selectfont ProCount(FlowCutter, MCS)}%
\end{pgfscope}%
\begin{pgfscope}%
\pgfsetbuttcap%
\pgfsetroundjoin%
\pgfsetlinewidth{1.003750pt}%
\definecolor{currentstroke}{rgb}{1.000000,0.498039,0.054902}%
\pgfsetstrokecolor{currentstroke}%
\pgfsetdash{{6.400000pt}{1.600000pt}{1.000000pt}{1.600000pt}}{0.000000pt}%
\pgfpathmoveto{\pgfqpoint{3.776169in}{1.360693in}}%
\pgfpathlineto{\pgfqpoint{4.026169in}{1.360693in}}%
\pgfusepath{stroke}%
\end{pgfscope}%
\begin{pgfscope}%
\definecolor{textcolor}{rgb}{0.000000,0.000000,0.000000}%
\pgfsetstrokecolor{textcolor}%
\pgfsetfillcolor{textcolor}%
\pgftext[x=4.051169in,y=1.316943in,left,base]{\color{textcolor}\rmfamily\fontsize{9.000000}{10.800000}\selectfont ProCount(FlowCutter, LP)}%
\end{pgfscope}%
\begin{pgfscope}%
\pgfsetbuttcap%
\pgfsetroundjoin%
\pgfsetlinewidth{1.003750pt}%
\definecolor{currentstroke}{rgb}{0.172549,0.627451,0.172549}%
\pgfsetstrokecolor{currentstroke}%
\pgfsetdash{{6.400000pt}{1.600000pt}{1.000000pt}{1.600000pt}}{0.000000pt}%
\pgfpathmoveto{\pgfqpoint{3.776169in}{1.185723in}}%
\pgfpathlineto{\pgfqpoint{4.026169in}{1.185723in}}%
\pgfusepath{stroke}%
\end{pgfscope}%
\begin{pgfscope}%
\definecolor{textcolor}{rgb}{0.000000,0.000000,0.000000}%
\pgfsetstrokecolor{textcolor}%
\pgfsetfillcolor{textcolor}%
\pgftext[x=4.051169in,y=1.141973in,left,base]{\color{textcolor}\rmfamily\fontsize{9.000000}{10.800000}\selectfont ProCount(htd, MCS)}%
\end{pgfscope}%
\begin{pgfscope}%
\pgfsetbuttcap%
\pgfsetroundjoin%
\pgfsetlinewidth{1.003750pt}%
\definecolor{currentstroke}{rgb}{0.839216,0.152941,0.156863}%
\pgfsetstrokecolor{currentstroke}%
\pgfsetdash{{6.400000pt}{1.600000pt}{1.000000pt}{1.600000pt}}{0.000000pt}%
\pgfpathmoveto{\pgfqpoint{3.776169in}{1.010754in}}%
\pgfpathlineto{\pgfqpoint{4.026169in}{1.010754in}}%
\pgfusepath{stroke}%
\end{pgfscope}%
\begin{pgfscope}%
\definecolor{textcolor}{rgb}{0.000000,0.000000,0.000000}%
\pgfsetstrokecolor{textcolor}%
\pgfsetfillcolor{textcolor}%
\pgftext[x=4.051169in,y=0.967004in,left,base]{\color{textcolor}\rmfamily\fontsize{9.000000}{10.800000}\selectfont ProCount(htd, LP)}%
\end{pgfscope}%
\begin{pgfscope}%
\pgfsetbuttcap%
\pgfsetroundjoin%
\pgfsetlinewidth{1.003750pt}%
\definecolor{currentstroke}{rgb}{0.580392,0.403922,0.741176}%
\pgfsetstrokecolor{currentstroke}%
\pgfsetdash{{6.400000pt}{1.600000pt}{1.000000pt}{1.600000pt}}{0.000000pt}%
\pgfpathmoveto{\pgfqpoint{3.776169in}{0.835784in}}%
\pgfpathlineto{\pgfqpoint{4.026169in}{0.835784in}}%
\pgfusepath{stroke}%
\end{pgfscope}%
\begin{pgfscope}%
\definecolor{textcolor}{rgb}{0.000000,0.000000,0.000000}%
\pgfsetstrokecolor{textcolor}%
\pgfsetfillcolor{textcolor}%
\pgftext[x=4.051169in,y=0.792034in,left,base]{\color{textcolor}\rmfamily\fontsize{9.000000}{10.800000}\selectfont ProCount(Tamaki, MCS)}%
\end{pgfscope}%
\begin{pgfscope}%
\pgfsetbuttcap%
\pgfsetroundjoin%
\pgfsetlinewidth{1.003750pt}%
\definecolor{currentstroke}{rgb}{0.549020,0.337255,0.294118}%
\pgfsetstrokecolor{currentstroke}%
\pgfsetdash{{6.400000pt}{1.600000pt}{1.000000pt}{1.600000pt}}{0.000000pt}%
\pgfpathmoveto{\pgfqpoint{3.776169in}{0.660815in}}%
\pgfpathlineto{\pgfqpoint{4.026169in}{0.660815in}}%
\pgfusepath{stroke}%
\end{pgfscope}%
\begin{pgfscope}%
\definecolor{textcolor}{rgb}{0.000000,0.000000,0.000000}%
\pgfsetstrokecolor{textcolor}%
\pgfsetfillcolor{textcolor}%
\pgftext[x=4.051169in,y=0.617065in,left,base]{\color{textcolor}\rmfamily\fontsize{9.000000}{10.800000}\selectfont ProCount(Tamaki, LP)}%
\end{pgfscope}%
\end{pgfpicture}%
\makeatother%
\endgroup%

    \caption{
        This plot compares different combinations of an \Lg{} tree decomposer (\flowcutter{}, \htd{}, or \tamaki{}) and a \dmc{} variable-ordering heuristic (\mcs{} or \lexp) for our framework \procount.
        We choose \Lg{} with \flowcutter{} and \dmc{} width \mcs{} as the representative setting of \procount{} to compete with existing projected model counters.
    }
    \label{figSolvingA}
\end{figure}
