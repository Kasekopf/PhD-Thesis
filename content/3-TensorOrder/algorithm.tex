\section{Separating Planning and Execution}
\label{sec:tensors:algorithm}
In this section, we discuss the algorithm for literal-weighted model counting using tensor networks. This algorithm is presented as Algorithm \ref{alg:wmc} and has three phases.

% TODO: References in TensorOrder2 assume that \alpha is incorporated
\begin{algorithm*}[t]
    \label{alg:wmc}
    \caption{Computing the weighted model count with a TN}
    \DontPrintSemicolon
    \KwIn{$\varphi$: a CNF formula}
    \KwIn{$W$: a literal-weight function}
    \KwOut{$W(\varphi)$, the weighted model count of $\varphi$ w.r.t. $W$}
    $N \gets \func{Reduce}(\varphi, W)$\;
    \Repeat{$\alpha \cdot \func{TimeCost}(M, T) < \text{elapsed time in seconds}$}{
      $M, T \gets \func{Plan}(N)$ \tcc*{e.g., \textbf{LG} or \textbf{FG}}
    }
    \Return{$\func{Execute}(M,T)$}
\end{algorithm*}

First, in the \emph{reduction} phase the input formula $\varphi$ and weight function $W$ is transformed into a tensor network $N$. The goal of the reduction phase is that the contraction of $N$, $\tntensor{N}$, should be equal to the $W$-weighted model count of $\varphi$. We discuss in more detail in Section \ref{sec:tensors:wmc}.

Second, in the \emph{planning} phase an order to contract tensors in the network is determined.
This order takes the form of a \emph{contraction tree} \cite{EP14}:
\begin{definition}[Contraction Tree] \label{def:contraction-tree}
	Let $N$ be a tensor network. A \emph{contraction tree} for $N$ is a rooted binary tree $T$ whose leaves are the tensors of $N$. %, i.e. $\Lv{T} = N$.
\end{definition}

In our database analogy from Section \ref{sec:tensors:tensors}, a contraction tree for a tensor network representing a project-join query is a join tree of that query (with projections done as early as possible). 
In our factor-graph analogy from Section \ref{sec:tensors:tensors}, a contraction tree corresponds to a \emph{dtree} \cite{darwiche01}, to an elimination order \cite{darwiche01b}, and to a binary join tree \cite{shenoy97} where factors are assigned to leaf nodes.

There are two subtleties in the planning process. 
Firstly, the planning phase is allowed to modify the input tensor network $N$ as long as the new tensor network $M$ contracts to an identical tensor. 
This allows the planning phase to tweak the tensor network to enable better contraction trees.
Secondly, planning in Algorithm \ref{alg:wmc} is an anytime process: we heuristically generate better contraction trees until one is ``good enough'' to use (governed by a parameter $\alpha \in \mathbb{R}$). 
This allows us to incorporate heuristic, anytime high-level reasoning tools into a planning algorithm. 
% For this chapter, we choose $\alpha$ so that we expect to have spent about half of the running time in planning (see Section \ref{sec:tensors:experiments:implementation}), but note that we determine $\alpha$ empirically in Chapter \ref{ch:parallel}.
In this work we discuss two planning techniques: the \textbf{Line-Graph} planner (see Section \ref{sec:tensors:contraction-theory}) and the \textbf{Factor-Tree} planner (see Section \ref{sec:tensors:preprocessing}).

Third, in the \emph{execution} phase the chosen contraction tree is used to contract the tensor network.
The goal of the execution phase is to compute the contraction of $M$ and thus compute the $W$-weighted model count of $\varphi$. The computation cost of the execution phase is heavily determined by the chosen contraction tree. We discuss this phase in more detail in Section \ref{sec:tensors:execution}.

We assert the correctness of Algorithm \ref{alg:wmc} in the following theorem.
\begin{theorem}
\label{thm:alg-correctness}
Let $\varphi$ be a CNF formula and let $W: 2^{\vars{\varphi}} \rightarrow \mathbb{R}$ be a literal-weight function.
    Assume:
    \begin{enumerate}
        \item $\func{Reduce}(\varphi, W)$ returns a tensor network $N$ s.t. $\tntensor{N}(\emptyset) = W(\varphi)$,
        \item $\func{Plan}(N)$ returns a tensor network $M$ and a contraction tree $T$ for $M$ s.t. $\tntensor{M} = \tntensor{N}$, and
        \item $\func{Execute}(M, T)$ returns $\tntensor{M}$ for all tensor networks $M$ and contraction trees $T$ for $M$.
    \end{enumerate}
Then Algorithm \ref{alg:wmc} returns $W(\varphi)$.
\end{theorem}
\begin{proof}
By Assumption 3, Algorithm \ref{alg:wmc} returns $\tntensor{M}(\emptyset)$. 
By Assumption 2, this is equal to $\tntensor{N}(\emptyset)$, which by Assumption 1 is exactly $W(\varphi)$.
\end{proof}

Organizationally, note that we discuss the execution phase in Section \ref{sec:tensors:execution} before we discuss the planning phase in Section \ref{sec:tensors:planning}. 
This is because we must understand how plans are used before we can evaluate various planning algorithms.

% TODO: Work in?
% Theorem \ref{thm:wmc-reduction} suggests that the weighted model count of $\varphi$ can be computed by constructing and contracting $N_\varphi$. We present this framework as Algorithm \ref{alg:wmc}. Algorithm \ref{alg:wmc} is a fixed-parameter algorithm for model counting, parameterized by carving-width of the incidence graph. The existence of such algorithms is easily implied by fixed-parameter algorithms for model counting parameterized by treewidth \cite{FMR08,SS10} since treewidth is bounded by thrice the carving width \cite{sasak10}. A variety of planners can be used in Step 2 to find a contraction tree to contract $N_\varphi$, including the planners \textbf{LG} and \textbf{FT} that we discuss in the following sections.