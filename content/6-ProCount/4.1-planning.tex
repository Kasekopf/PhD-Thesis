\section{Planning Phase: Building Graded Project-Join Trees}
\label{sec:procount:planning}

In the planning phase, we are given a CNF formula $\phi$ over Boolean variables $X \cup Y$.
The goal is to construct an $\{X, Y\}$-graded project-join tree of $\phi$.

We show how building graded project-join trees can be reduced to building ungraded project-join trees. 
This allows us our techniques for ungraded project-join trees (see Section \ref{sec_planning}) to directly compute graded project-join trees.
As a building block, we first show how constructing project-join trees with free variables can be reduced to constructing ungraded project-join trees. This both illustrates the key ideas of our approach and appears as a subroutine in the larger graded reduction.

\subsection{From Free Project-Join Trees to Ungraded Project-Join Trees}
\label{sec:planning:free}
Project-join trees project out every variable in the set of corresponding clauses. This is desirable for applications where all variables are processed in the same way, e.g., model counting. In many other applications, however, it is desirable to process a set of clauses while leaving specified \emph{free variables} untouched. This is analogous to the set of free indices of a tensor network (see Definition \ref{def:contraction}).
 
We model free variables by ensuring that they are projected in the project-join tree as late as possible, at the root node. Thus free variables must be ``kept alive'' throughout the entire tree.
\begin{definition}
\label{def:free}
Let $F$ be a set of variables, and let $\T = (T, r, \gamma, \pi)$ be a project-join tree. We say $\T$ is \emph{$F$-free} if $F = \pi(r)$.
\end{definition}

Note that Definition \ref{def:free} is a much stronger restriction than Definition \ref{def:graded}. 
In particular, if a project-join tree $\T$ of a CNF formula $\phi$ is $F$-free, then $\T$ is also $(F, \vars(\phi) \setminus F)$-graded.

We now reduce the problem of building $F$-free project-join trees to building ungraded project-join trees. One approach is to build a project-join tree while ignoring all variables in $F$, then insert the variables in $F$ as projections at the root. 
However, building minimal-width project-join trees while ignoring variables may not produce minimal-width $F$-free project-join trees for the full formula.

Instead, we adapt a similar reduction in the context of tensor networks (see Theorem \ref{thm:contraction-equiv-carving}, which includes tensor networks with free indices) for the context of project-join trees. 
The key idea is to add to $\phi$ a \emph{virtual clause} that contains all variables in $F$. 
For a set $Z$ of variables, let $\func{virtual}(Z)$ denote a fresh clause with variables $Z$. 
Project-join trees of $\phi \cup \{\func{virtual}(F)\}$ can then be used to find $F$-free project-join trees of $\phi$. 
This virtual clause can be viewed as a goal atom in \tool{DataLog} \cite{lloyd2012foundations}, and is analogous to the free vertex in Definition \ref{def:structure}.

We present this reduction as Algorithm \ref{alg_planning_free}. The input is a project-join tree $\T$ of $\phi \cup \{C_F\}$, where $C_F$ is a fresh clause with variables $\vars(C_F) = F$. 
On lines \ref{line_pi_prime_init}-\ref{line_pi_prime_union}, we rotate $\T$ (without increasing the width) so that the leaf $s$ corresponding to $C_F$ becomes the root. 
We then remove $s$ and obtain a project-join tree of $\phi$. Projecting $F$ at the new root still does not increase the width.
\begin{algorithm}[t]
\caption{Building an $F$-free project-join tree of $\phi$}
\label{alg_planning_free}
    \DontPrintSemicolon
    \KwIn{$\phi$: a CNF formula}
    \KwIn{$F$: a subset of $\vars(\phi)$}
    \KwIn{$\T = (T, r, \gamma, \pi)$: a project-join tree of $\phi \cup \{C_F\}$, where $C_F = \func{virtual}(F)$ is a virtual clause with variables $F$}
    \KwOut{an $F$-free project-join tree of $\phi$}
    $s \gets \gamma^{-1}(C_F)$ \tcp{$s$ will be the root node of the returned project-join tree}
    $\pi' \gets$ a mapping where $\pi'(n) = \emptyset$ for all $n \in \V{T} \setminus \Lv{T}$ \tcp{$\pi'$ will be the labeling function of the returned project-join tree} \label{line_pi_prime_init}
    \For{\upshape $y \in \vars(\phi) \setminus F$}{
        $\phi_y = \{ C \in \phi : y \in \vars(C)\}$\;
        $i \gets$ lowest common ancestor of $\set{ \gamma^{-1}(C) : C \in \phi_y }$ in the rooted tree $(T, s)$\;
        $\pi'(i) \gets \pi'(i) \cup \{y\}$ \tcp{project out $y$ at the lowest allowable node} \label{line_pi_prime_union}
    }
    $\pi'(s) \gets F$ \label{line_pi_F} \tcp{project out variables in $F$ at the new root $s$}
    $\gamma' \gets \gamma \setminus \set{s \mapsto C_F}$ \tcp{$\gamma'$ is the bijection $\gamma$ without the pair $(s, C_F)$}
    \Return{$(T, s, \gamma', \pi')$}
\end{algorithm}

We state the correctness of Algorithm \ref{alg_planning_free} in the following theorem. In particular, the width of the output $F$-free project join tree is no worse than the width of the unrestricted input tree.
\begin{theorem}\label{thm:planning_free_correctness}
Let $\phi$ be a CNF formula and let $F \subseteq \vars(\phi)$. 
If $\T$ is a project-join tree of $\phi \cup \{\func{virtual}(F)\}$, then Algorithm \ref{alg_planning_free} returns an $F$-free project-join tree of $\phi$ of width at most $\func{width}(\T)$.
\end{theorem}
\begin{proof}
Let $C_F = \func{virtual}(F)$.
Let $\T = (T, r, \gamma, \pi)$ be the input project-join tree for $\phi \cup \{ C_F \}$, and let $\T' = (T, s, \gamma', \pi')$ be the output of Algorithm \ref{alg_planning_free}. 
Moreover, let $\vars_{\T}(n)$ and $\vars_{\T'}(n)$ denote the sets of variables at a node $n$ of $\T$ and $\T'$ respectively.


First, we prove that $\T'$ is a project-join tree. Since $\gamma$ is a bijection onto $\phi \cup \{C_F\}$ and we have removed both $C_F$ and the leaf corresponding to $C_F$, $\gamma'$ is indeed a bijection onto $\phi$. 
Moreover, since every variable appears in exactly one set in the image of $\pi'$, the first condition of Definition \ref{def_jointree} is satisfied. Finally, each variable $y$ is projected out at the lowest common ancestor of all leaves corresponding to clauses that contain $y$; thus the second condition of Definition \ref{def_jointree} is satisfied. It follows that $\T'$ is a project-join tree.

Second, we prove that $\T'$ is $F$-free. 
Since $s$ has degree 1, $s$ is never the lowest common ancestor of a set of leaves of $\T'$. Thus $\pi'(s) \setminus F = \emptyset$. By line \ref{line_pi_F}, it follows that $\pi'(s) = F$, so $\T'$ is $F$-free.

%To do this, for a set of leaves $L \subseteq \Lv{T}$ and a root $o$, define $A_{T, o}(L) \in \V{T}$ to be the lowest common ancestor of $L$ in $(T, o)$. By Algorithm \ref{alg_planning_free}, for each $y \in \vars(\phi) \setminus F$, $y \in \pi'(A_{T,s}(\{ C : C \in \phi~\text{and}~y \in \vars(C)\}))$. We can then define $B: \V{T} \rightarrow \vars(\phi)$ by $A(n) = \{ x \in \vars(\phi) : \exists \ell, \ell' \in \Lv{T}~\text{s.t.}~n~\text{is on the shortest path between}~\ell~\text{and}~\ell',~\text{and}~x \in \vars(\gamma(\ell)) \cap \vars(\gamma(\ell'))\}$, i.e., $B(n)$ is the set of all variables $x$ such that there are two leaves, $\ell$ and $\ell'$, on different sides of $n$ whose clauses contain $x$.


Finally, we prove that the width of $\T'$ is at most $\func{width}(\T)$. 
To do this, if $S$ is a project-join tree and $n$ is a node of $S$, define $\func{rel}_S(n) \equiv \vars_S(n)$ for leaf nodes and $\func{rel}_S(n) \equiv \vars_S(n) \cup \pi(n)$ for internal nodes. Notice the size of $n$ in $S$ is exactly $|\func{rel}_S(n)|$, so the width of $S$ is exactly the maximum size of $\func{rel}_S(n)$ across all nodes $n$. 

Consider an arbitrary node $n \in \V{T} \setminus \{s\}$. Define:
\begin{align*}
    A(n) &= \{ y : \exists \ell \in \Lv{T} \text{ s.t. $\ell$ is a descendant of $n$ in the rooted tree $(T, s)$ and } \\&y \in \func{rel}_{\T'}(\gamma'(\ell)) \} \\
    B(n) &= \{ x \in \vars(\phi) : \exists \ell, \ell' \in \Lv{T} \text{ s.t. $n$ is between $\ell, \ell'$ in $(T, r)$ and } \\&x \in \func{rel}_{\T}(\gamma(\ell)) \cap \func{rel}_{\T}(\gamma(\ell')) \} \\
    B'(n) &= \{ x \in \vars(\phi) : \exists \ell, \ell' \in \Lv{T} \text{ s.t. $n$ is between $\ell, \ell'$ in $(T, s)$ and } \\&x \in \func{rel}_{\T'}(\gamma'(\ell)) \cap \func{rel}_{\T'}(\gamma'(\ell')) \}
\end{align*}
Note that a node $n$ is \emph{between} $\ell, \ell' \in \V{T}$ if $n$ is on the unique shortest path between $\ell$ and $\ell'$. There are several key relationships among $A(n)$, $B(n)$, $B'(n)$, $\func{rel}_{\T}(n)$, and $\func{rel}_{\T'}(n)$: 
\begin{enumerate}
    \item By the construction in Algorithm \ref{alg_planning_free}, we know $\func{rel}_{\T'}(n) \setminus F = B'(n) \setminus F$ and $\func{rel}_{\T'}(n) \cap F = A(n) \cap F$.
    \item Since $\gamma$ and $\gamma'$ agree on all nodes of $T$ except for $s$, we know $B(n) \setminus F = B'(n) \setminus F$.
    \item Since $\vars(C_F) = F$, we know $B(n) \cap F = A(n) \cap F$.
    \item By Property \ref{prop2} of Definition \ref{def_jointree}, $B(n) \subseteq \func{rel}_\T(n)$. 
\end{enumerate}
Putting these relationships together, we observe that:
\begin{align*}
    \func{rel}_{\T'}(n) &= (\func{rel}_{\T'}(n) \setminus F) \cup (\func{rel}_{\T'}(n) \cap F) \\
    &= (B'(n) \setminus F) \cup (A(n) \cap F) \\
    &= (B(n) \setminus F) \cup (B(n) \cap F) \\
    &= B(n) \\
    &\subseteq \func{rel}_{\T}(n)
\end{align*}
Finally, observe that $\func{rel}_{\T'}(s) = \func{rel}_{\T}(s) = F$.
Hence the width of $\T'$ is indeed no larger than the width of $\T$, as desired.
\end{proof}

We also prove that Algorithm \ref{alg_planning_free} is optimal. That is, a minimal-width project-join tree for $\phi \cup \{C_F\}$ produces a minimal-width $F$-free project-join tree for $\phi$. 
\begin{theorem}
\label{thm:planning_free_optimal}
Let $\phi$ be a CNF formula, let $F \subseteq \vars(\phi)$, and let $w$ be a positive integer. % Define $C_F$ to be a fresh clause with $\vars(C_F) = \vars(N) \cap X$.
If there is an $F$-free project-join tree of $\phi$ of width $w$, then there is a project-join tree of $\phi \cup \{\func{virtual}(F)\}$ of width $w$.
\end{theorem}
\begin{proof}
Let $\T$ be an $F$-free project-join tree for $\phi$ of width at most $w$. 
Produce $\T'$ by attaching to the root of $\T$ a new leaf node corresponding to $C_F$. 
Then $\T'$ is a project-join tree for $\phi \cup \{C_F\}$, and its width is identical to $\T$.
\end{proof}

\subsection{From Graded Project-Join Trees to Ungraded Project-Join Trees}
%\cite{dudek2020dpmc} contained two algorithms for constructing ungraded project-join trees: an algorithm based on heuristics from constraint programming, and an algorithm based on tree decompositions. In this section, we adapt these algorithms to construct $(X,Y)$-graded project-join trees from a given instance of weighted projected model counting $(X, Y, \phi, W)$.

In this section, we use free project-join trees as a building block to construct graded project-join trees. We present this framework as Algorithm \ref{alg_planning_graded}. The key idea is to create a graded project-join tree by combining many free project-join trees for subformulas. We first combine clauses to remove $Y$ variables, then we combine project-join-tree components to remove $X$ variables.
\begin{algorithm}[t]
\caption{Framework for building a graded project-join tree}
\label{alg_planning_graded}
    \DontPrintSemicolon
    \KwIn{$X$: a set of $\Sigma$-variables}
    \KwIn{$Y$: a set of $\exists$-variables ($X \cap Y = \varnothing$)}
    \KwIn{$\phi$: a CNF formula where $\vars(\phi) = X \cup Y$}
    \KwOut{$\T$: an $(X, Y)$-graded project-join tree of $\phi$}
    $partition \gets \func{GroupBy}(\phi, Y)$ \label{alg_line_graded_group} \tcp{group clauses that share $Y$ variables}
    \For{$N \in partition$}{ \label{alg_line_graded_loop}
        $\T_N \gets \func{BuildComponent}(N, \vars(N) \cap X)$ \label{alg_line_graded_project_y} \tcp{build a $(\vars(N) \cap X)$-free project-join tree of $N$}
        $\T_N \gets \T_N$ with all projections at the root of $\T_N$ removed \label{alg_line_projections_removed}\;
        $C_N \gets \func{virtual}(\vars(N) \cap X)$ \label{alg_line_graded_virtual}\;    
    }
    $\T \gets \func{BuildComponent}(\{C_N : N \in partition\}, \emptyset)$ \label{alg_line_graded_project_x} \tcp{build a project-join tree from virtual clauses $C_N$}
    \For{$N \in partition$ \label{alg_line_graded_hookup_start}} {
        $\ell_N \gets$ leaf of $\T$ corresponding to $C_N$\;
        $\T \gets \T$ with $\ell_N$ replaced by $\T_N$ \label{alg_line_graded_hookup_end}\;
    }
    \Return{$\T$}\;
\end{algorithm}

In detail, on line \ref{alg_line_graded_group}, we partition the clauses of $\phi$ into blocks that share $Y$ variables.
On line \ref{alg_line_graded_project_y}, we find a project-join tree $\T_N$ for each block $N$. This tree must keep all $X$ variables free, i.e., must be $(\vars(N)\cap X)$-free. The trees $\{\T_N\}$ collectively project out all $Y$ variables. On line \ref{alg_line_graded_project_x}, we construct a project-join tree $\T$ that will guide the combination of all trees in $\{\T_N\}$ while projecting out all $X$ variables, where each $\T_N$ is represented by the virtual clause $C_N$. 
On lines \ref{alg_line_graded_hookup_start}-\ref{alg_line_graded_hookup_end}, we hook the trees in $\{\T_N\}$ together as indicated by $\T$.

The function $\func{GroupBy}(\phi, Y)$ in Algorithm \ref{alg_planning_graded} partitions the clauses of $\phi$ so that every pair of clauses that share a variable from $Y$ appear together in the same block of the partition.
Formally:
\begin{definition}
\label{def:groupby}
Let $\phi$ be a set of clauses, and let $Y$ be a subset of $\vars(\phi)$.
Define $\sim_Y \subseteq \phi \times \phi$ to be the relation such that, for clauses $c, c' \in \phi$, we have $c \sim_Y c'$ if and only if $(\vars(c) \cap \vars(c') \cap Y) \ne \emptyset$. Then $\func{GroupBy}(\phi, Y)$ is the set of equivalence classes of the reflexive transitive closure of $\sim_Y$.
\end{definition}

The intuition is that two clauses in the same block in $\func{GroupBy}(\phi, Y)$ must be combined to project out all variables in $Y$.
Conversely, clauses that appear in separate blocks need not be combined in order to project out all variables in $Y$. 

In Algorithm \ref{alg_planning_graded}, each call to $\func{BuildComponent}(\alpha, F)$ returns an $F$-free project-join tree of $\alpha$, where $\alpha$ is a set of clauses and $F \subseteq \vars(\alpha)$.
$\func{BuildComponent}$ can be implemented by implementing Algorithm \ref{alg_planning_free} on top of an algorithm for building ungraded project-join trees. 
For example, in Section \ref{sec:experiments}, we consider two implementations of Algorithm \ref{alg_planning_graded} built on top of the two algorithms to construct standard project-join trees \cite{dudek2020dpmc} discussed in Chapter \ref{ch:dpmc}.

We next state the correctness of Algorithm \ref{alg_planning_graded} and show that the width of the output graded project join tree is no worse than the widths of the trees used for the components. Formally:
\begin{theorem}
\label{thm:planning_graded_correctness}
Let $\phi$ be a CNF formula, let $\{X, Y\}$ be a partition of $\vars(\phi)$, and let $w$ be a positive integer. 
Assume each call to $\func{BuildComponent}(\alpha, F)$ returns an $F$-free project-join tree for $\alpha$ of width at most $w$. 
Then Algorithm \ref{alg_planning_graded} returns an $(X,Y)$-graded project-join tree for $\phi$ of width at most $w$.
\end{theorem}
\begin{proof}
Let $\T$ be the project-join tree produced on line \ref{alg_line_graded_project_x}, and let $\T' = (T, r, \gamma, \pi)$ be the project-join tree returned by Algorithm \ref{alg_planning_graded}. 
By Definition \ref{def:groupby}, every $y \in Y$ is a variable of exactly one $N_y \in \func{GroupBy}(\phi, Y)$. It follows that every $y \in Y$ is projected out at exactly one node of $\T'$, namely the node at which $y$ is projected out in $\T_{N_y}$. Similarly, after the loop on line \ref{alg_line_graded_loop} completes, no variable from $X$ is projected out across all of $\{\T_N : N \in \func{GroupBy}(\phi, Y)\}$, since all $X$ projections are removed on line \ref{alg_line_projections_removed}. Thus every $x \in X$ is also projected out at exactly one node of $\T'$, namely the node at which $x$ is projected out in $\T$. Thus $\T'$ satisfies the first property of Definition \ref{def_jointree}.

We prove the second property of Definition \ref{def_jointree} by contrapositive. That is, assume that there is some $n \in \V{T} \setminus \Lv{T}$, variable $z \in \pi(n)$, and $c \in \phi$ s.t. $z \in \vars(c)$ but $\gamma^{-1}(c)$ is not a descendant of $n$ in $\T'$. Let $N$ be the block of $\func{GroupBy}(\phi, Y)$ that contains $c$. We split into two cases:
\begin{itemize}
    \item \textit{Case: $z \in Y$.} Let $n'$ be the node in $\T_N$ that corresponds to $n$, where $z$ is projected. Then $\gamma^{-1}(c)$ is not a descendant of $n'$ in $\T_N$. It follows that $\T_N$ is not a project-join tree.
    \item \textit{Case: $z \in X$.} Let $n'$ be the node in $\T$ that corresponds to $n$, where $z$ is projected. Since  $\gamma^{-1}(c)$ is not a descendant of $n$ in $\T$, the leaf corresponding to $C_N$ is not a descendant of $n'$ in $\T$. But $z \in \vars(N) \cap X$, so $z \in \vars(C_N)$. It follows that $\T$ is not a project-join tree. 
\end{itemize} 
We conclude that $\T'$ satisfies the second property of Definition \ref{def_jointree} provided that $\func{BuildComponent}$ always returns project-join trees.

Finally, we prove that the width of $\T'$ is at most $w$. To see this, we observe that the set of variables at each node of $\T'$ is exactly the set of variables appearing at the node of the corresponding component project-join tree. The width of $\T'$ is thus the maximum size that appears across all component project-join trees returned by $\func{BuildComponent}$. 
\end{proof}

Although Algorithm \ref{alg_planning_graded} constructs a sequence of small ungraded project-join trees, it is sufficient to compute a single ungraded project-join tree from which all smaller trees can be extracted. This is demonstrated by the following theorem.
\begin{theorem}
\label{thm:graded_from_virtual}
Let $\phi$ be a CNF formula, let $\{X, Y\}$ be a partition of $\vars(\phi)$, and let $w$ be a positive integer. Let $\psi = \phi \cup \{\func{virtual}(\vars(N)\cap X) : N \in \func{GroupBy}(\phi, Y)\}$.
% For each $N \in \func{GroupBy}(\phi, Y)$, define $C_N$ to be a new clause with $\vars(C_N) = \vars(N) \cap X$.
If there is a project-join tree $\T'$ for $\psi$ of width $w$, then there is an $(X,Y)$-graded project-join tree for $\phi$ of width at most $w$.
\end{theorem}
\begin{proof}
The key idea of the proof is to answer every $\func{BuildComponent}$ call in Algorithm \ref{alg_planning_graded} by extracting a subtree of $\T'$ and applying Theorem \ref{thm:planning_free_correctness}.

We first show that, for each call to $\func{BuildComponent}(\alpha, F)$ in Algorithm \ref{alg_planning_graded}, there is an $F$-free project-join tree for $\alpha$ of width at most $w$. There are two cases to consider: the calls on line \ref{alg_line_graded_project_y} and the call on line \ref{alg_line_graded_project_x}.
\begin{itemize}
    \item \textit{Case: line \ref{alg_line_graded_project_y}.} Consider some $N \in \func{GroupBy}(\phi, Y)$. Our goal is to find a $(\vars(N) \cap X)$-free project-join tree for $N$. Observe that $N \cup \{C_N\}$ is a subset of $\psi$. Thus let $S_N$ be the smallest subtree of $\T'$ containing all leaves labeled by some element of $N \cup \{C_N\}$. $S_N$ is a project-join tree for $N \cup \{C_N\}$ whose width is no more than $w$. By Theorem  \ref{thm:planning_free_correctness}, there is a $(\vars(N) \cap X)$-free project-join tree for $N$ of width no more than $w$.
    
    \item \textit{Case: line \ref{alg_line_graded_project_x}.} Similarly, let $N' = \{C_N : N \in \func{GroupBy}(\phi, Y)\}$ and observe that $N'$ is a subset of $\psi$. Let $S$ be the smallest subtree of $\T'$ containing all leaves labeled by some element of $N'$. Then $S$ is a project-join tree for $N'$ whose width is no more than $w$. 
\end{itemize}

It then follows from Theorem \ref{thm:planning_graded_correctness} that there is an $(X,Y)$-graded project-join tree for $\phi$ of width at most $w$. 
\end{proof}

We show in the following theorem that this approach is optimal. Thus $(X,Y)$-graded project-join trees of $\phi$ are equivalent to project-join trees of $\psi$.
\begin{theorem}
\label{thm:planning_graded_optimal}
Let $\phi$ be a CNF formula, let $\{X, Y\}$ be a partition of $\vars(\phi)$, and let $w$ be a positive integer. Let $\psi = \phi \cup \{\func{virtual}(\vars(N)\cap X) : N \in \func{GroupBy}(\phi, Y)\}$. % For each $N \in \func{GroupBy}(\phi, Y)$, define $C_N$ to be a new clause with $\vars(C_N) = \vars(N) \cap X$.
If there is an $(X,Y)$-graded project-join tree for $\phi$ of width $w$, then there is a project-join tree for $\psi$ of width $w$.
\end{theorem}
\begin{proof}
Let $\T$ be an $(X,Y)$-graded project-join tree for $\phi$ of width $w$, and let $\mathcal{I}_X, \mathcal{I}_Y$ be the grades of $\T$. 
We first aim to show that, for every $N \in \func{GroupBy}(\phi, Y)$, there is some node $n_N$ of $\T$ with $N \cap X \subseteq \vars(n_N)$. 

Consider an arbitrary $N \in \func{GroupBy}(\phi, Y)$. If $|N| = 1$, define $n_N$ to be the node of $\T$ corresponding to the only element of $N$; thus $\vars(n_N) = \vars(N)$, so indeed $N \cap X \subseteq \vars(n_N)$. 
Otherwise, define $n_N \in \V{T}$ to be the lowest common ancestor of $N$ in $\T$. 
Then there exist distinct clauses $A, B \in N$ s.t. $n_N$ is also the lowest common ancestor of the two leaves labeled by $A$ and $B$. 
By Definition \ref{def:groupby}, since $A, B \in N$, there must be a sequence $C_1, C_2, \cdots, C_k \in N$ s.t. $C_1 = A$, $C_k = B$, and for each $1 \leq i \leq k$, we have $\vars(C_i) \cap \vars(C_{i+1}) \cap Y \neq \emptyset$. 
Since $n_N$ is the lowest common ancestor of $A$ and $B$, and because $\T$ is a tree, there must be some $1 \leq i \leq k$ s.t. $n_N$ is also the lowest common ancestor of $C_i$ and $C_{i+1}$. Thus $\vars(C_i) \cap \vars(C_{i+1}) \subseteq \vars(n_N)$. 
Since $\vars(C_i) \cap \vars(C_{i+1}) \cap Y \neq \emptyset$, it follows that $Y \cap \vars(n_N) \neq \emptyset$ as well. Thus $n_N \in \mathcal{I}_Y$. By Definition \ref{def:graded}, this means that for all descendants $o$ of $n_N$, $\pi(o) \cap X = \emptyset$. It follows that $\vars(n_N)$ must still contain all variables in $N$ from $X$; that is, $\vars(N) \cap X \subseteq \vars(n_N)$.

Construct $\T'$ from $\T$ by, for each $N \in \func{GroupBy}(\phi, Y)$, attaching a new leaf labeled by $\func{virtual}(\vars(N) \cap X)$ as a child of $n_N$. 
Since $N \cap X \subseteq \vars(n_N)$ in the initial tree, the width of $\T'$ is equal to the width of $\T$. Moreover, $\T'$ is now a project-join tree for $\psi$.
\end{proof}

We emphasize that our theoretical results do not imply that gradedness can be obtained for a project-join free with no increase in width.
Requiring the project-join tree to be graded may significantly increase the width of available project-join trees.
Rather, Theorems \ref{thm:planning_graded_correctness} and \ref{thm:planning_graded_optimal} indicate that our algorithm for constructing a graded project-join tree pays no additional cost in width beyond what is required by gradedness.

%To conclude, we compare this theoretical result to a different algorithm for (unweighted) projected counting \cite{fichte2018exploiting}. This algorithm runs on a formula $\varphi$ in time $2^{2^{O(k)}}} \cdot p(\varphi)$, where $p$ scales polynomially in the size of $\varphi$ and $k$ is the \emph{primal treewidth} \cite{SS10} of $\varphi$. Moreover, assuming the exponential time hypothesis \cite{impagliazzo2001problems} the double exponential is unavoidable in general. On the other hand, by Theorem 5 of Dudek et. al \cite{dudek2020dpmc} and Theorem 7 our algorithm runs in time $2^{O(k')}$, where $k'$ is the primal treewidth of $\psi$. While $k'$ is larger than $k$ (and, in the worst case, $k' = \max(k, |X|)$), on many problems $k'$ is much smaller than $2^k$. 