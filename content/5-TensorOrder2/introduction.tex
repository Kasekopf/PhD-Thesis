Most tools for discrete integration focus on single-core performance. 
There have been only a few parallel counters, notably the multi-core unweighted model counter \tool{countAntom} \cite{BSB15} and the GPU-based weighted model counter \tool{gpuSAT2} \cite{FHWZ18,FHZ19}.
Developing tools for discrete integration that can leverage available parallelism is increasingly important. %TODO: This is weak

% In \emph{weighted model counting}, the task is to count the total weight, subject to a given weight function, of the set of solutions of input constraints. This fundamental task has applications in probabilistic reasoning, planning, inexact computing, engineering reliability, and statistical physics \cite{Bacchus2003,DH07,GSS08}. The development of model counters that can successfully compute the total weight on large industrial formulas is an area of active research \cite{OD15,Thurley2006}. Although most model counters focus on single-core performance, there have been several parallel model counters, notably the multi-core (unweighted) model counter \tool{countAntom} \cite{BSB15} and the GPU-based weighted model counter \tool{gpuSAT2} \cite{FHWZ18,FHZ19}. 

Neural networks are a recent success story of parallelization. The parallelization of neural network training and inference has seen massive research across the machine learning and high-performance computing communities \cite{ABCCDDDGII16,JYPPABBBBB17,PGMLJGKLGA19}. Consequently, GPUs give orders of magnitude of speedup over a single core for neural-network algorithms \cite{KSTKPPRS19,NRBHHJN15}. 
Another line of recent work has focused on specialized hardware (e.g., using TPUs--Tensor Processing Units \cite{JYPPABBBBB17}) to provide further speedup.
In this work, we aim to directly leverage advances in parallelization designed for neural-network algorithms in the service of weighted model counting. % We leave as future work leveraging advances in specialized hardware (e.g., using TPUs--Tensor Processing Units \cite{JYPPABBBBB17}).
% Contraction is a key operation in neural network training and inference \cite{BK07,Hirata03,KKCLA17,VZTGDMVAC18}, and, as such, many of the optimizations for neural networks also apply to tensor-network contraction \cite{KSTKPPRS19,NRBHHJN15,RMGZFZHVL19}.

Chapter \ref{ch:tensors} demonstrated that \emph{tensor networks} are a natural bridge between high-performance computing and weighted model counting.
In this chapter, we further exploit this bridge to bring parallelization techniques from high-performance computing to weighted model counting. Specifically, we explore the impact of multiple-core, GPU, and TPU use on tensor network contraction for weighted model counting.

In detail, Chapter \ref{ch:tensors} presented a 3-phase algorithm for weighted model counting. First, in the \emph{reduction} phase, a counting instance is reduced to a tensor-network problem. Second, in the \emph{planning} phase, an order to contract tensors in the network is determined. Finally, in the \emph{execution} phase, tensors in the network are contracted using the determined order. The algorithmic and experimental results of Chapter \ref{ch:tensors} focused on a single-core performance.  Since the reduction phase was not a significant source of runtime, we focus on opportunities for parallelism in the planning and execution phases.

The planning phase in Chapter \ref{ch:tensors} was done using a choice of several single-core heuristic tree-decomposition solvers \cite{AMW17,HS18,Tamaki17}. There is little recent work on parallelizing heuristic tree-decomposition solvers. Instead, we implement a parallel portfolio of single-core tree-decomposition solvers and find that this portfolio significantly improves planning on multiple cores. Similar portfolio approaches have been well-studied and shown to be beneficial in the context of SAT solvers \cite{BSS15,XHHL08}. As a theoretical contribution, we prove that branch-decomposition solvers can also be included in this portfolio. Unfortunately, we find that a current branch-decomposition solver only marginally improves the portfolio.

The execution phase in Chapter \ref{ch:tensors} was done using \pkg{numpy} \cite{numpy} and evaluated on a single core. We add an implementation of the execution phase that uses \pkg{TensorFlow} \cite{ABCCDDDGII16} to leverage a GPU for large contractions. Since GPU memory is significantly more limited than CPU memory, we add an implementation of \emph{index slicing}. Index slicing is a recent technique from the tensor-network community \cite{CZHNS18,GK20,VBNHRBM19}, analogous to the classic technique of conditioning in Bayesian Network reasoning \cite{darwiche01,dechter99,pearl86,SAS94}, that allows memory to be significantly reduced at the cost of additional time. We find that, while multiple cores do not significantly improve the contraction phase, a GPU provides significant speedup, especially when augmented with index slicing. 

We also add an implementation of the execution phase that uses index slicing with \pkg{jax} \cite{jax2018github} in order to efficiently leverage a TPU. 
Unfortunately, we observe that current tensor libraries fail to compile tensor computations with high-dimensional tensors to a TPU.
Thus our approach does not scale well on a TPU for practical model counting benchmarks.

We implement our techniques in \tool{TensorOrder2}, a new parallel weighted model counter. We compare \tool{TensorOrder2} to a variety of state-of-the-art counters. We show that the improved \tool{TensorOrder2} is the fastest counter on 11\% of benchmarks after preprocessing \cite{LM14}, outperforming the GPU-based counter \tool{gpuSAT2} \cite{FHZ19}. Thus \tool{TensorOrder2} is useful as part of a portfolio of counters. All code and data are available at  \url{https://github.com/vardigroup/TensorOrder}.

The rest of the chapter is organized as follows. 
% In Section~\ref{sec:prelim}, we give background information on weighted model counting, graph decompositions, and tensor networks. 
% In Section~\ref{sec:algorithm}, we discuss the algorithm for performing weighted model counting using tensor networks as outlined by \cite{DDV19}.
In Section~\ref{sec:parallel:planning}, we describe parallelization of the planning phase with an algorithmic portfolio and planning with branch decompositions.
In Section~\ref{sec:parallel:execution}, we describe parallelization of the execution phase with index slicing on a GPU and on a TPU. 
In Section~\ref{sec:experiments}, we implement this parallelization in \tool{TensorOrder2} and analyze its performance experimentally.

%both the machine learning and the high-performance computing communities.
% Many problems in artificial intelligence have significantly benefited from parallelization. 

%In this work, we aim to leverage the massive practical work in machine learning and high-performance computing on the parallelization of neural network inference 


%for weighted model counting, leveraging both multiple CPUs \cite{BSB15} and GPUs \cite{FHWZ18}.
% Most counting tools focus on single-core performance. Moreover, parallelizing such tools directly is challenging. For example, CDCL is known to be difficult to parallelize well. Instead, recent work designed a weighted model counting solver for a GPU from the ground up. 
%Even when the weight function is a constant function, constrained counting is \#P-Complete \cite{Valiant79}. Nevertheless, the development of tools that can successfully compute the total weight on large industrial formulas is an area of active research \cite{OD15,Thurley2006}. 

% Paragraph about work in TensorFlow

% A natural algorithmic framework to use is tensor networks \cite{google TN paper}. Tensor networks are great. Prior work showed they are promising for model counting.

% Constrained counting can be reduced to the problem of tensor-network contraction \cite{BMT15}. \emph{Tensor networks} are a tool used across quantum physics and computer science for describing and reasoning about quantum systems, big-data processing, and more \cite{BB17,Cichocki14,Orus14}. A tensor network describes a complex tensor as a computation on many simpler tensors, and the problem of tensor-network \emph{contraction} is to perform this computation. Although tensor networks can be seen as a variant of factor graphs \cite{KFL01}, working directly with tensor networks allows us to leverage massive practical work in machine learning and high-performance computing on tensor contraction \cite{BK07,Hirata03,KKCLA17,VZTGDMVAC18} (which also includes GPU support \cite{KSTKPPRS19,NRBHHJN15}) to perform constrained counting. Since tensor networks are relatively unknown in the artificial intelligence community, we give an introduction of relevant material on tensors and tensor networks in Section \ref{sec:tensors}.

% Contracting a tensor network requires determining an order to contract the tensors inside the network, and so efficient contraction requires finding a contraction order that minimizes computational cost. Since the number of possible contraction orders grows exponentially in the number of tensors, exhaustive algorithms, e.g. \cite{PHV14}, cannot scale to handle the large tensor networks required for the reduction from constrained counting. Instead, recent work \cite{KCMR18} gave heuristics that can sometimes find a ``good-enough'' contraction order. Finding efficient contraction orders for tensor networks remains an area of active research \cite{RTPCTSL19}.

% and discuss prior work on the optimization of tensor-network contraction in Section~\ref{sec:tensors}. We introduce a framework for solving the problem of weighted model counting with tensor networks in Section~\ref{sec:wmc}. We use branch decompositions to factor tensor networks in Section~\ref{sec:branch-preprocessing}. We present an experimental evaluation of tensor-based approaches to model counting in Section~\ref{sec:experiments}. Finally, we discuss future work and conclude in Section~\ref{sec:conclusion}.