\section{Execution Phase: Contracting the Tensor Network}
\label{sec:tensors:execution}
For formulas $\varphi$ with hundreds of clauses, the corresponding tensor network produced by Theorem \ref{thm:wmc-reduction} has hundreds of bond indices. Directly following Equation \ref{eqn:contraction} in this case is infeasible, since it sums over an exponential number of terms (one for each assignment in $\domain{\tnbound{N}}$).

\begin{algorithm*}[t]
    \label{alg:network-contraction}
    \caption{Recursively contracting a tensor network}
    
    \DontPrintSemicolon
    \KwIn{A tensor network $N$}
    \KwIn{A contraction tree $T$}
    \KwOut{$\tntensor{N}$, the contraction of $N$ as given in Definition \ref{def:contraction}}
    \Function{\upshape $\func{Execute}(N, T)$}{
        \If{$\left|N\right| = 1$} {
            \Return{$\textup{the tensor contained in}~N$} \label{line:network-contraction:lookup}
        }\Else{
            $T_1, T_2 \gets \text{immediate subtrees of}~T$\;
            $A_1 \gets \func{Execute}(\Lv{T_1}, T_1)$\;
            $A_2 \gets \func{Execute}(\Lv{T_2}, T_2)$\;
            \Return{$A_1 \cdot A_2$} \label{line:network-contraction:subcontraction}
        }
    }
\end{algorithm*}

Instead, Algorithm \ref{alg:network-contraction} shows how to compute $\tntensor{N}$ for a tensor network $N$ using a contraction tree $T$ as a guide \cite{EP14}. The key idea is to repeatedly choose two tensors $A_1, A_2 \in N$ (according to the structure of $T$) and contract them. One can prove inductively that Algorithm \ref{alg:network-contraction} satisfies Assumption 3 of Theorem \ref{thm:alg-correctness}.

Each contraction in Algorithm \ref{alg:network-contraction} contains exactly two tensors and so can be implemented as a matrix multiplication with a variety of tensor libraries (e.g. \tool{numpy} \cite{numpy} or \tool{TensorFlow} \cite{ABCCDDDGII16})
Although the choice of contraction tree does not affect the correctness of Algorithm \ref{alg:network-contraction}, it may have a dramatic impact on the running-time and memory usage. We explore this further in the following section.

