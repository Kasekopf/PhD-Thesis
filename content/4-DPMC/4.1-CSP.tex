\subsection{Planning with One-Shot Constraint-Satisfaction Heuristics}
\label{sec_csp}

A variety of constraint-satisfaction heuristics for model counting were presented in a single algorithmic framework by \cite{DPV20}.
These heuristics have a long history in constraint programming \cite{dechter03}, database-query optimization \cite{MPPV04}, and propositional reasoning \cite{pan2005symbolic}.
In this section, we adapt the framework of \cite{DPV20} to produce project-join trees.
This adaption was done by Vu H. N. Phan and is not a contribution of this thesis. We outline it briefly here and refer the reader to \cite{dudek2020dpmc} for the complete details.

This algorithm is presented as Algorithm \ref{alg_csp_jt}, which constructs a project-join tree of a CNF formula using constraint-satisfaction heuristics.
Each of the functions $\clusterVarOrder$, $\clauseRank$, and $\chosenCluster$ represent heuristics for fine-tuning the specifics of the algorithm.
Before discussing the various heuristics, we assert the correctness of Algorithm \ref{alg_csp_jt} in the following theorem.
\begin{theorem}
\label{thm_csp_jt}
    Let $X$ be a set of variables and $\phi$ be a CNF formula over $X$.
    Assume that $\clusterVarOrder$ returns an injection $X \to \N$.
    Furthermore, assume that all $\clauseRank$ and $\chosenCluster$ calls satisfy the following conditions:
    \begin{enumerate}[ref=\arabic*]
        \item $1 \le \clauseRank(c, \rho) \le m$, \label{cond1}
        \item $i < \chosenCluster(n_i) \le m$, and \label{cond2}
        \item $X_s \cap \vars(n_i) = \emptyset$ for all integers $s$ where $i < s < \chosenCluster(n_i)$. \label{cond3}
    \end{enumerate}
    Then Algorithm \ref{alg_csp_jt} returns a project-join tree of $\phi$.
\end{theorem}
% \begin{proof}
% Let $\mathcal{T} = (T, n_m, \gamma, \pi)$ be the object returned by Algorithm \ref{alg_csp_jt}. To prove that $\mathcal{T}$ is a project-join tree we must verify that $T$ is a tree, that $\gamma$ is a bijection and that both properties from Definition \ref{def_jointree} hold.

% \paragraph{Part 1: $(T, n_m)$ is a rooted tree containing every node created by Algorithm \ref{alg_csp_jt}.}
% After the loop at Line \ref{line_creation_loop}, every leaf node created by Algorithm \ref{alg_csp_jt} in contained in some element of $\{ \kappa_j : 1 \leq j \leq m\}$. One can prove by induction that, after iteration $i < m$ of the loop on Line \ref{line_construction_loop}, because of Condition \ref{cond2} every node created by Algorithm \ref{alg_csp_jt} so far is the descendent of exactly one node of one element of $\{ \kappa_j : i < j \leq m\}$. Thus before the final iteration (with $i = m$) every node created by Algorithm \ref{alg_csp_jt} so far is the descendent of exactly one node in $\kappa_m$. It follows that $(T, n_m)$ is indeed a rooted tree containing every node created by Algorithm \ref{alg_csp_jt}.

% \paragraph{Part 2: $\gamma$ is a bijection.}
% By condition \ref{cond1}, $\{ \Gamma_i : 1 \leq i \leq m\}$ is a partition of the clauses of $\varphi$. Moreover, observe that after Line \ref{line_kappa} $\{ \kappa_i : 1 \leq i \leq m\}$ is a partition of $\Lv{T}$. Finally, by construction $\gamma$ is a bijection between $\kappa_i$ and $\Gamma_i$ for each $1 \leq i \leq m$. Thus $\gamma$ is a bijection between $\Lv{T}$ and the clauses of $\varphi$.

% \paragraph{Part 3: Property 1 of Definition \ref{def_jointree}.} 
% That is, we must verify that $P = \{\pi(a) : a \in \V{T} \setminus \Lv{T} \}$ is a partition of $X$.
% For each $i = 1, 2, \ldots, m$, Algorithm \ref{alg_csp_jt} constructs an internal node $n_i$ with $\pi(n_i) = X_i$.
% Since $\set{X_i}_{i = 1}^m$ is a partition of $X$, so is $P = \set{\pi(n_i)}_{i = 1}^m$.

% \paragraph{Part 4: Property 2 of Definition \ref{def_jointree}.}
% That is, consider an an internal node $n_i \in \V{T} \setminus \Lv{T}$ (created on Line \ref{line_internal_node} at iteration $i$), variable $x \in \pi(n_i)$, and clause $c \in \phi$ \st{} $x$ appears in $c$. We must verify that $\ell = \gamma^{-1}(c)$ is a descendant of $n_i$ in $T$.

% Define a sequence $p_0, p_1, \cdots, p_L$ of integers by $p_0 = p$, $p_{k+1} = \func{ChosenCluster}(n_{p_k})$ if $p_k < m$, and $p_L = m$.
% Condition \ref{cond2} implies that this sequence is indeed finite and strictly increasing.
% The loop on Line \ref{line_construction_loop} ensures that $\ell$ is a child of $n_{p_0}$ and that, for each $0 \leq k < L$, the internal node $n_{p_k}$ is a child of $n_{p_{k+1}}$.
% Eventually we reach $n_{p_L} = n_m$, the root of $\mathcal{T}$.
% So $n_{p_0}, n_{p_1}, \cdots, n_{p_L}$ are exactly the ancestors of $\ell$ in $T$.

% Assume for the sake of a contradiction that $i$ is not an ancestor of $\ell$.
% Thus $i$ does not appear in the sequence $p_0, p_1, \cdots, p_L$. 
% Since $x \in \pi(n_i) = X_i$, we know by Line \ref{line_varsets} of Algorithm \ref{alg_csp_jt} that $p \leq i$.
% Therefore $p_0, p_1, \cdots, p_L$ is an increasing sequence of integers from $p_0 = p \leq i$ to $p_L = m \geq i$.
% Hence there exists some $k'$ s.t. $p_{k'} < i < p_{k'+1}$. 
% Condition \ref{cond3} then implies that $X_i \cap \vars(n_{p_{k'}}) = \emptyset$.
% In particular, since $x \in X_i$ this means that $x \not\in \vars(n_{p_{k'}})$.

% On the other hand, since $\set{X_i}_{i = 1}^m$ is a partition of $X$ and $x \in \pi(n_i) = X_i$, we know that $x \not\in X_{i_k}$ for all $0 \leq k \leq L$.
% Because $\ell$ is a child of $n_{p_0}$ and $x \in \vars(\ell)$, this implies that $x \in \vars(n_k)$ for all $0 \leq k \leq L$.
% In particular, $x \in \vars(n_{p_{k'}})$.
% This is a contradiction.
% \end{proof}

\begin{algorithm*}[t]
\label{alg_csp_jt}
\caption{Using combined constraint-satisfaction heuristics to build a project-join tree}
    \DontPrintSemicolon
    \KwIn{$X$: set of $m \ge 1$ Boolean variables}
    \KwIn{$\phi$: CNF formula over $X$}
    \KwOut{$(T, r, \gamma, \pi)$: project-join tree of $\phi$}
    $(T, \nil, \gamma, \pi) \gets \text{empty project-join tree}$\;
    $\rho \gets \clusterVarOrder(\phi)$
        \tcc*{injection $\rho : X \to \N$}
    \For{$i = m, m - 1, \ldots, 1$}{\label{line_creation_loop}
        $\Gamma_i \gets \set{c \in \phi : \clauseRank(c, \rho) = i}$ \label{line_kappa}
            \tcc*{$1 \le \clauseRank(c, \rho) \le m$}
        $\kappa_i \gets \set{\leaf(T, c) : c \in \Gamma_i}$\;
            \tcc*{for each $c$, a leaf $l$ with $\gamma(l) = c$ is constructed and put in cluster $\kappa_i$}
        $X_i \gets \vars(\Gamma_i) \setminus \bigcup_{j = i + 1}^m \vars(\Gamma_j)$ \label{line_varsets}
            \tcc*{$\set{X_i}_{i = 1}^m$ is a partition of $X$}
    }
    \For{$i = 1, 2, \ldots, m$}{\label{line_construction_loop}
        \If{$\kappa_i \ne \emptyset$}{
            $n_i \gets \internal(T, \kappa_i, X_i)$
                \tcc*{$\C{T}{r}{n_i} = \kappa_i$ and $\pi(n_i) = X_i$}
                \label{line_internal_node}
            \If{$i < m$}{
                $j \gets \chosenCluster(n_i)$ \label{line_chosen_cluster}
                    \tcc*{$i < j \le m$}
                $\kappa_j \gets \kappa_j \cup \set{n_i}$ \label{line_cluster_union}
            }
        }
    }
    \Return{$(T, n_m, \gamma, \pi)$}
\end{algorithm*}

By Condition \ref{cond1}, we know that $\set{\Gamma_i}_{i = 1}^m$ is a partition of the clauses of $\phi$.
Condition \ref{cond2} ensures that Lines \ref{line_chosen_cluster}-\ref{line_cluster_union} place a new internal node $n_i$ in a cluster that has not yet been processed.
Also on Lines \ref{line_chosen_cluster}-\ref{line_cluster_union}, Condition \ref{cond3} prevents the node $n_i$ from skipping a cluster $\kappa_s$ if there exists some $x \in X_s \cap \vars(n_i)$, since $x$ is projected in iteration $s$, \ie, $x$ is added to $\pi(n_s)$.
These invariants are sufficient to prove that Algorithm \ref{alg_csp_jt} indeed returns a project-join tree of $\phi$.
All heuristics we use in this work satisfy the conditions of Theorem \ref{thm_csp_jt}.

There are a variety of heuristics to fine-tune Algorithm \ref{alg_csp_jt}.
For the function $\clusterVarOrder$, we consider the heuristics \Random, \Mcs{} (\textdef{maximum-cardinality search} \cite{tarjan1984simple}), \Lexp/\Lexm{} (\textdef{lexicographic search for perfect/minimal orders} \cite{koster2001treewidth}), and \Minfill{} (\textdef{minimal fill-in} \cite{dechter03}) as well as their inverses (\Invmcs, \Invlexp, \Invlexm, and \Invminfill).
Heuristics for $\clauseRank$ include \Be{} (\textdef{bucket elimination} \cite{dechter99}) and \Bm{} (\textdef{Bouquet's Method} \cite{bouquet1999gestion}).
For $\chosenCluster$, the heuristics we use are \ListH{} and \TreeH{} \cite{DPV20}.
We combine $\clauseRank$ and $\chosenCluster$ as \textdef{clustering heuristics}: $\Be-\ListH$, $\Be-\TreeH$, $\Bm-\ListH$, and $\Bm-\TreeH$. See \cite{dudek2020dpmc} for a full description of all heuristics.

% The only heuristic we consider that does not appear in \cite{DPV20} is the \Minfill{} heuristic for variable order \cite{dechter03}. 
% This heuristic operates on the Gaifman graph of $\varphi$ and chooses variables one-by-one.
% Whenever a variable $v$ is chosen, we add \textdef{fill-in} edges to connect all of $v$'s neighbors in the Gaifman graph.
% At each step of the \Minfill{} heuristic, the next variable chosen is the variable that minimizes the number of fill-in edges.
% All other heuristics are described in \cite{DPV20}.
