\section{Graphs}
\label{sec:tensors:prelim}
A \emph{graph} $G$ has a nonempty set of vertices $\V{G}$, a set of (undirected) edges $\E{G}$, a function $\delta_G: \V{G} \rightarrow 2^{\E{G}}$ that gives the set of edges incident to each vertex, and a function $\epsilon_G: \E{G} \rightarrow 2^{\V{G}}$ that gives the set of vertices incident to each edge. Each edge must be incident to exactly two vertices, but multiple edges can exist between two vertices. If $E \subset \E{G}$, let $\eincs{G}{E} = \bigcup_{e \in E} \einc{G}{e}$. Similarly, if $V \subset \V{G}$, let $\vincs{G}{V} = \bigcup_{v \in V} \vinc{G}{v}$.
An \emph{edge clique cover} of a graph $G$ is a set $A \subseteq 2^{\V{G}}$ such that (1) every vertex $v \in \V{G}$ is an element of some set in $A$, and (2) every element of $A$ is a clique in $G$ (that is, for every $C \in A$ and every pair of distinct $v, w \in C$ there is an edge between $v$ and $w$ in $G$).

\subsection{Trees}
A \emph{tree} is a simple, connected, and acyclic graph. A \emph{leaf} of a tree $T$ is a vertex of degree one, and we use $\Lv{T}$ to denote the set of leaves of $T$. For every edge $a$ of $T$, deleting $a$ from $T$ yields exactly two trees, whose leaves define a partition of $\Lv{T}$. Let $C_a \subseteq \Lv{T}$ denote an arbitrary element of this partition. Throughout this work, we often refer to a vertex of a tree as a \emph{node} and an edge as an \emph{arc}, since our proofs will frequently work simultaneously with a graph and an associated tree.

A \emph{rooted tree} is a tree $T$ together with a distinguished node $r \in \V{T}$ called the \emph{root}. 
In a rooted tree $(T, r)$, each node $n \in \V{T}$ has a (possibly empty) set of \emph{children}, denoted $\C{T}{r}{n}$, which contains all nodes $n'$ adjacent to $n$ \st{} all paths from $n'$ to $r$ contain $n$.

A \emph{rooted binary tree} is a rooted tree where either $|\V{T}| = 1$ or the root has degree two and every non-root node has degree three or one. If $|\V{T}| > 1$, the \emph{immediate subtrees of $T$} are the two rooted binary trees that are the connected components of $T$ after the root is removed. 

\subsection{Graph Decompositions}
In this work, we use three decompositions of a graph as a tree: carving decompositions \cite{ST94}, branch decompositions \cite{RS91}, and tree decompositions \cite{RS91}. All decompose the graph into an \emph{unrooted binary tree}, which is a tree where every vertex has degree one or three. First, we describe carving decompositions \cite{ST94}:
\begin{definition}[Carving Decomposition]
\label{def:carving}
	Let $G$ be a graph. A \emph{carving decomposition} for $G$ is an unrooted binary tree $T$ whose leaves are the vertices of $G$, i.e. $\Lv{T} = \V{G}$. 
	
	The \emph{width} of $T$, denoted $width_c(T)$, is the maximum number of edges in $G$ between $C_a$ and $\V{G} \setminus C_a$ for all $a \in \E{T}$, i.e.,
    
	$$width_c(T) = \max_{a \in \E{T}} \left| \left( \bigcup_{v \in C_a} \vinc{G}{v} \right) \cap \left( \bigcup_{v \in \V{G} \setminus C_a} \vinc{G}{v} \right) \right|.$$
	
	% $$width_c(T) = \max_{a \in \E{T}} \left| \{ e \in \E{G}~\text{s.t.}~\einc{G}{e} \cap C_a \neq \emptyset~\text{and}~\epsilon_G(e) \cap (\V{G} \setminus C_a) \neq \emptyset \}\right|.$$
	
	
	% $$width_c(T) = \max_{a \in \E{T}} \left| \vinc{G}{C_a} \cap \vinc{G}{\V{G} \setminus C_a} \right|.$$
	
    The width of a carving decomposition $T$ with no edges is 0.
\end{definition}

%In Definition \ref{def:carving}, $\vinc{G}{V}$ refers to the set of edges of $G$ incident to some vertex in $V \subseteq \V{G}$.
The \emph{carving width} of a graph $G$, denoted $width_c(G)$, is the minimum width across all carving decompositions for $G$. Note that an equivalent definition of carving decompositions allows for degree two vertices within the tree as well. 
Branch decompositions are the dual of carving decompositions and hence can be defined by swapping the role of $\V{G}$ and $\E{G}$ in Definition \ref{def:carving}.

Finally, we define tree decompositions \cite{RS91}:
\begin{definition}[Tree Decomposition]
\label{def:treedecomposition}
	A \emph{tree decomposition} for a graph $G$ is an unrooted binary tree $T$ together with a labeling function $\chi : \V{T} \rightarrow 2^{\V{G}}$ that satisfies the following three properties:
	\begin{enumerate}
		\item Every vertex of $G$ is contained in the label of some node of $T$. That is, $\V{G} = \bigcup_{n \in \V{T}} \chi(n)$.
		\item For every edge $e \in \E{G}$, there is a node $n \in \V{T}$ whose label is a superset of $\einc{G}{e}$, i.e. $\einc{G}{e} \subseteq \chi(n)$.
		\item If $n$ and $o$ are nodes in $T$ and $p$ is a node on the path from $n$ to $o$, then $\chi(n) \cap \chi(o) \subseteq \chi(p)$.
	\end{enumerate}
	The \emph{width} of a tree decomposition, denoted $width_t(T, \chi)$, is the maximum size (minus 1) of the label of every node, i.e.,
	$$width_t(T, \chi) = \max_{n \in \V{T}} | \chi(n) | - 1.$$
\end{definition}

The \emph{treewidth} of a graph $G$, denoted $width_t(G)$, is the minimum width across all tree decompositions for $G$. The treewidth of a tree is $1$. Treewidth is bounded by thrice the carving width \cite{sasak10}. The treewidth (plus 1) of graph is no smaller than the branchwidth and is bounded from above by $3/2$ times the branchwidth \cite{RS91}. 

Given a CNF formula $\varphi$, a variety of associated graphs have been considered. The \emph{incidence graph} of $\varphi$ is the bipartite graph where both variables and clauses are vertices and edges indicate that the variable appears in the connected clause. The \emph{primal graph} of $\varphi$ is the graph where variables are vertices and edges indicate that two variables appear together in a clause. There are fixed-parameter tractable model counting algorithms with respect to the treewidth of the incidence graph and the primal graph \cite{SS10}. If the treewidth of the primal graph of a formula $\varphi$ is $k$, the treewidth of the incidence graph of $\varphi$ is at most $k+1$ \cite{KV00}.
