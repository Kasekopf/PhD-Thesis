% \emph{Weighted projected model counting} is a fundamental problem in artificial intelligence, with applications in planning \cite{aziz2015projected}, formal verification \cite{klebanov2013sat}, and reliability estimation \cite{duenas2017counting}. Counting is also closely connected to sampling
% \cite{jerrum1986random}, a problem of major interest in probabilistic reasoning \cite{kelly2019adaptive}.
% The input is a set of constraints, whose variables are divided into \emph{relevant variables} $X$ and \emph{irrelevant variables} $Y$.
% The goal is to compute the weighted number of assignments to $X$ that, together with some assignment to $Y$, satisfy the constraints.
% This problem is complete for the complexity class \#P\textsuperscript{NP[1]} 
% (solvable by a polynomially bounded counting Turing machine with one query to an NP-complete oracle)
% \cite{zawadzki2013generalization}, and
% there are recent tools for exact weighted projected model counting \cite{lagniez2019recursive,lee2017solving}.

% unweighted tools: \cite{zawadzki2013generalization,aziz2015projected,mohle2018dualizing,hecher2020taming}

% Dynamic programming is a powerful technique that has been applied across computer science \cite{bellman1966dynamic}.
% The key idea is to solve a large problem by solving many smaller subproblems, and then combining partial solutions into the final result.
% Dynamic programming is a natural framework to solve a variety of problems defined on sets of constraints, as subproblems can be formed by partitioning the constraints.
% This framework has been instantiated into algorithms for database-query optimization \cite{MPPV04}, SAT solving \cite{uribe1994ordered,aguirre2001random,pan2005symbolic}, QBF evaluation \cite{charwat2016bdd}, model counting \cite{BDP09,SS10,jegou2016improving,dudek2020dpmc,fichte2020exploiting}, and projected model counting \cite{fichte2020counting}.

Chapter \ref{ch:dpmc} proposed a unifying framework for dynamic-programming algorithms based on \emph{project-join trees}. 
The key idea is to consider project-join trees as \emph{execution plans} and decompose dynamic-programming algorithms into two phases: a \emph{planning phase}, where a project-join tree is constructed from an input problem instance, and an \emph{execution phase}, where the project-join tree is used to compute the result. 
% The resulting project-join-tree-based model counter \dpmc{} was found to be competitive with state-of-the-art exact weighted model counters \cite{
%     % DPV20, % ADDMC
%     % DDV19, % TensorOrder
%     sang2004combining, % Cachet
%     darwiche2004new, % C2D
%     LM17, % d4
%     OD15% miniC2D
% }.
% %MYV: Say something about competition?
% Notably, \dpmc{} subsumes \addmc{} \cite{DPV20}, which tied with \df{} \cite{LM17} for first place of the weighted track of the 2020 Model Counting Competition \cite{fichte2020model}.

We adapt this framework for weighted projected model counting. The central challenge is that for projected model counting there are two kinds of variables: relevant and irrelevant. This contrasts with model counting, where all variables are relevant and so can be treated similarly. This challenge also occurs for other problems.
For example, in Boolean functional synthesis \cite{tabajara2017factored}, some variables are \emph{free} and must not be projected out. 
Our solution is to model multiple types of variables by requiring the project-join tree to be \emph{graded}, meaning that irrelevant variables must be projected before relevant variables. Our main theoretical contribution is a novel algorithm to construct graded project-join trees from standard project-join trees. This has two primary advantages. 

The first advantage is that the construction of graded project-join trees can be done by using existing tools for standard project-join trees \cite{dudek2020dpmc} in a black-box way. Tools exist to construct standard project-join trees with tree decompositions \cite{RS91} or with constraint-satisfaction heuristics \cite{tarjan1984simple,koster2001treewidth,dechter03,dechter99,bouquet1999gestion}.
We can thus easily leverage all current and future work in tree-decomposition solvers \cite{strasser2017computing,AMW17,Tamaki17} and constraint-satisfaction heuristics to produce graded project-join trees. This is crucial for the practical success of our tool.

The second advantage to our approach is in the simplicity of the algorithm. Given a project-join tree, its gradedness can be easily verified before starting the execution stage. 
Moreover, the execution algorithm that computes the projected model count from a graded project-join tree is straightforward. This gives us confidence in the correctness of our implementation.
During our experimental evaluation, we found correctness errors in \dfp{} \cite{lagniez2019recursive}, \projmc{} \cite{lagniez2019recursive}, and \ssat{} \cite{lee2017solving} that we reported to the authors, who then fixed the tools. 
We believe that this work is a step towards a certificate for the verification of projected model counters, similar to SAT solver certificates \cite{wetzler2014drat}.

The primary contribution of this work is a dynamic-programming framework for weighted projected model counting based on project-join trees. In particular:
\begin{enumerate}
    \item We show that project-join trees can be used to compute weighted projected model counts, provided that the project-join trees are graded.
    \item We prove that building graded project-join trees and project-join trees with free variables can be reduced to building standard project-join trees.
    \item We find that project-join-tree-based algorithms make a significant contribution to the portfolio of exact weighted projected model counters (\dfp{}, \projmc{}, and \ssat{}).
    Our tool, \procount{}, achieves the shortest solving time on \dpmcFastestBenchmarks{} benchmarks of \solvedBenchmarks{} benchmarks solved by at least one tool, from \benchmarks{} benchmarks in total.
\end{enumerate}