\section{Chapter Summary}
\label{sec_procount_discussion}

We adapted an existing dynamic-programming framework \cite{dudek2020dpmc} to perform projected model counting by requiring project-join trees to be graded.
This framework decomposes projected model counting into two phases.
First, the planning phase produces a graded project-join tree from a CNF formula.
Second, the execution phase uses the this tree to guide the computation of the projected model count of the formula \wrt{} a literal-weight function. 
We proved that algorithms for building project-join trees can be used in a black-box way to build graded project-join trees.

Our framework \procount{} is competitive with the exact weighted projected model counters \dfp{} \cite{lagniez2019recursive}, \projmc{} \cite{lagniez2019recursive}, and \ssat{} \cite{lee2017solving}. 
\procount{} considerably improves the virtual best solver and thus is a valuable addition to the portfolio.
