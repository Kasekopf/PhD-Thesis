





\subsection{Algebraic Decision Diagrams}

An \emph{algebraic decision diagram (ADD)} is a compact representation of a pseudo-Boolean function as a directed acyclic graph \cite{bahar1997algebraic}.
For functions with logical structure, an ADD representation can be exponentially smaller than the explicit representation.
Originally designed for matrix multiplication and shortest path algorithms, ADDs have also been used for Bayesian inference \cite{chavira2007compiling,gogate2011approximation}, stochastic planning \cite{hoey1999spudd}, model checking \cite{kwiatkowska2007stochastic}, and model counting \cite{fargier2014knowledge,dudek2020addmc}.

Formally, an ADD is a tuple $(X, S, \sigma, G)$, where $X$ is a set of Boolean variables, $S$ is an arbitrary set (called the \textdef{carrier set}), $\sigma: X \to \N$ is an injection (called the \textdef{diagram variable order}), and $G$ is a rooted directed acyclic graph satisfying the following three properties.
First, every leaf node of $G$ is labeled with an element of $S$.
Second, every internal node of $G$ is labeled with an element of $X$ and has two outgoing edges, labeled 0 and 1.
Finally, for every path in $G$, the labels of internal nodes must occur in increasing order under $\sigma$.
In this work, we only need to consider ADDs with the carrier set $S = \mathbb{R}$.

An ADD $(X, S, \sigma, G)$ is a compact representation of a function $f: 2^X \to S$.
Although there are many ADDs representing $f$, for each injection $\sigma: X \to \N$, there is a unique minimal ADD that represents $f$ with $\sigma$ as the diagram variable order, called the \textdef{canonical ADD}.
ADDs can be minimized in polynomial time, so it is typical to only work with canonical ADDs.

Several packages exist for efficiently manipulating ADDs.
For example, \cudd{} \cite{somenzi2015cudd} implements both product and projection on ADDs in polynomial time (in the size of the ADD representation).
\cudd{} was used as the primary data structure for weighted model counting in \cite{dudek2020addmc}.
In this work, we also use ADDs with \cudd{} to compute $W$-valuations.

\Mcs{} was the best diagram variable order on a set of standard weighted model counting benchmarks in \cite{dudek2020addmc}.
So we use \Mcs{} as the diagram variable order in this work.
Note that all other heuristics discussed in Section \ref{sec_csp} for cluster variable order could also be used as heuristics for diagram variable order.
% JD: We only consider MCS in Section 6, so this paragraph is not really needed
% One challenge in using ADDs is choosing the diagram variable order.
% The choice of diagram variable order can have a dramatic impact on the size of the ADD; some variable orders may produce ADDs that are exponentially smaller than others for the same real-valued function.
% A variety of techniques exist in prior work to heuristically find diagram variable orders \cite{tarjan1984simple,koster2001treewidth,dechter2003constraint}.
% Moreover, since binary decision diagrams (BDDs) \cite{bryant1986graph} can be seen as ADDs with carrier set $S = \set{0, 1}$, there is significant overlap with the techniques to find variable orders for BDDs.
% All cluster-variable-order heuristics discussed in Section \ref{sec_csp} (\Random, \Mcs, \Invmcs, \Lexp, \Invlexp, \Lexm, \Invlexm, \Minfill, and \Invminfill) can also be used as diagram-variable-order heuristics.
% In \cite{dudek2020addmc}, \Mcs{} was the best diagram variable order on a set of standard weighted model counting benchmarks.

% ADDs were explored for weighted model counting in a dynamic-programming framework in \cite{dudek2020addmc}.
% In detail, their model-counting algorithm essentially combines Algorithm \ref{alg_csp_jt} and Algorithm \ref{alg_jt_wmc}, using ADDs as the primary data structure.
% Their two best heuristic configurations have \Mcs{} as the diagram variable order.
% The other diagram variable orders they investigated were \Invmcs, \Lexp, \Invlexp, \Lexm, \Invlexm, and \Random.
