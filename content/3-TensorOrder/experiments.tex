\section{Implementation and Evaluation} \label{sec:tensors:experiments}
We aim to answer the following experimental research questions:
\begin{enumerate}
    \item[(RQ1)] Are tensor-network-based approaches competitive with existing state-of-the-art unweighted model counters?
    \item[(RQ2)] Are tensor-network-based approaches competitive with existing state-of-the-art weighted model counters?
    \item[(RQ3)] What are the structural properties of benchmarks for which the tensor-network-based approaches perform well?
\end{enumerate}

To answer these questions, we implement Algorithm 2 in \tool{TensorOrder}, a new tool for weighted model counting using tensor networks. \tool{TensorOrder} can be configured to find contraction trees using existing planning methods -- \textbf{greedy} (using a greedy algorithm), \textbf{metis} (using graph partitioning), and \textbf{GN} (using community structure detection) \cite{KCMR18}-- or planning methods presented in this paper-- \textbf{LG} (Section \ref{sec:tensors:contraction-theory}) and \textbf{FT} (Section \ref{sec:tensors:preprocessing}). Implementation details appear in Section \ref{sec:tensors:experiments:implementation}.

To answer RQ1, in Section \ref{sec:tensors:experiments:cubic} we compare \tool{TensorOrder} with existing state-of-the-art tools for unweighted model counting (\tool{dynQBF} \cite{CW16}, \tool{dynasp} \cite{FHMW17}, \tool{SharpSAT} \cite{Thurley2006}, \tool{cachet} \cite{SBK05}, \tool{miniC2D} \cite{OD15} and \tool{d4} \cite{LM17}) on formulas that count the number of vertex covers of randomly-generated cubic graphs \cite{KCMR18}. Note \tool{dynQBF} and \tool{dynasp} are solvers from related domains (that can be used as tools for unweighted model counting) that also use tree decompositions.

To answer RQ2, in Section \ref{sec:tensors:experiments:cachet} we compare \tool{TensorOrder} with existing state-of-the-art tools for weighted model counting (\tool{cachet} \cite{SBK05}, \tool{miniC2D} \cite{OD15} and \tool{d4} \cite{LM17}) on formulas whose weighted count corresponds to exact inference on Bayesian networks \cite{SBK05}. Note that the other tools (\tool{dynQBF}, \tool{dynasp}, and \tool{SharpSAT}) cannot perform weighted model counting.

% We use \tool{TensorOrder} to compare tensor-based methods with existing state-of-the-art tools for weighted model counting: \tool{cachet} \cite{SBK05}, \tool{miniC2D} \cite{OD15} and \tool{d4} \cite{LM17}. %Note that \tool{d4} requires a d-DNNF reasoner to perform weighted model counting; we use \cite{CDLL18}. 
% We also compare with \tool{dynQBF} \cite{CW16}, \tool{dynasp} \cite{FHMW17} and \tool{SharpSAT} \cite{Thurley2006} when the benchmarks are unweighted. Note \tool{dynQBF} and \tool{dynasp} are solvers from related domains (that can be used as model counters) that also use tree decompositions.

% We then compare our tensor-based algorithms for weighted model counting against state-of-the-art weighted model counters and against existing tensor-based algorithms. We demonstrate that our tensor-based algorithms are useful as part of a portfolio of weighted model counters. 

% We compare \tool{TensorOrder} on two sets of existing benchmarks. First, in Section \ref{sec:tensors:experiments:cubic} we compare on formulas that count the number of vertex covers of randomly-generated cubic graphs \cite{KCMR18}. Second, in Section \ref{sec:tensors:experiments:cachet} we compare 
To answer RQ3, we compute upper bounds on the treewidth and carving width of these benchmarks. We run each of three heuristic tree decomposition solvers (\pkg{Tamaki} \cite{Tamaki17}, \pkg{FlowCutter} \cite{HS18}, and \pkg{htd} \cite{AMW17}) on each benchmark with a timeout of 1000 seconds: once on the structure graph $G$ corresponding to the benchmark, and once on $\Line{G}$. The minimal width of all produced tree decompositions for $G$ (resp. $\Line{G}$) is an upper bound for the treewidth of $G$ (resp. $\Line{G}$). The minimal max-rank of the contraction trees produced by running \textbf{LG} (resp. \textbf{FT}) on each tree decomposition of $\Line{G}$ (resp. $G$) is an upper bound for the carving width of $G$ (resp. $G$ after \text{FT}-preprocessing).

Each experiment was run in a high-performance cluster (Linux kernel 2.6.32) using a single 2.80 GHz core of an Intel Xeon X5660 CPU and 48 GB RAM. We provide all code, benchmarks, and detailed data of benchmark runs at \url{https://github.com/vardigroup/TensorOrder}.

% \paragraph{Experimental Methodology}
% To evaluate runtime performance, we run each tool once on each benchmark with a timeout of 1000 seconds and record the wall-clock time taken. For \tool{TensorOrder}, recorded times include all of Algorithm \ref{alg:wmc} (and, specifically, include the time of the underlying tree-decomposition solver).



% We then use these tree decompositions to compute upper bounds on the treewidth of $G$ (the minimal width of all found tree decompositions for $G$), the treewidth of $\Line{G}$ (the minimal width of all found tree decompositions for $\Line{G}$), the carving width of $G$ (the minimal max-rank of all contraction trees produced by running \textbf{LG} on tree decompositions of $\Line{G}$), and the carving width of $G$ after \textbf{FT}-preprocessing (the minimal max-rank of all contraction trees produced by running \textbf{FT} on tree decompositions of $\Line{G}$).

% \begin{enumerate}
%     \item the treewidth of $G$ by taking the minimum width amongst all discovered tree decompositions for $G$,
%     \item the treewidth of $\Line{G}$ by taking the minimum width amongst all discovered tree decompositions for $\Line{G}$,
%     \item the carving width of $G$ by taking the minimum width amongst all contraction trees produced by running \textbf{LG} on each tree decomposition of $\Line{G}$, and
%     \item the carving width of $G$ after \textbf{FT}-preprocessing by taking the minimum width amongst all contraction trees produced by running \textbf{FT} on each tree decomposition of $G$.
% \end{enumerate}

% compute an upper bound for the treewidth of $G$ (resp. $\Line{G}$) by taking the minimum width amongst all discovered tree decompositions for $G$ (resp. $\Line{G}$). Similarly, we compute an upper bound for the carving width of $G$ by taking the minimum width amongst all contraction trees produced by running \textbf{LG} on each tree decomposition of $\Line{G}$.


% We then record the width of the best tree decomposition found amongst all tree-decomposition solvers. We also 

% We also evaluate the structural properties of these benchmarks by performing an experimental comparison of treewidth and carving width of the incidence. For each benchmark with corresponding incidence graph $G$, we run each of the three heuristic tree decomposition solvers \tool{Tamaki}, \tool{FlowCutter}, and \tool{htd} for 1000 seconds on $G$ and $\Line{G}$. We then record the width of the best tree decomposition found amongst all tree-decomposition solvers. On each tree decomposition for $\Line{G}$ found by the solvers, we use \textbf{LG} to compute the corresponding carving decomposition of $G$ and recorded the smallest width found amongst all decompositions. Similarly, on each tree decomposition for $G$ found by the solvers, we use \textbf{FT} to compute the corresponding carving decomposition of the preprocessed graph and recorded the smallest width found amongst all decompositions. 

\begin{figure}
	\centering
	%% Creator: Matplotlib, PGF backend
%%
%% To include the figure in your LaTeX document, write
%%   \input{<filename>.pgf}
%%
%% Make sure the required packages are loaded in your preamble
%%   \usepackage{pgf}
%%
%% and, on pdftex
%%   \usepackage[utf8]{inputenc}\DeclareUnicodeCharacter{2212}{-}
%%
%% or, on luatex and xetex
%%   \usepackage{unicode-math}
%%
%% Figures using additional raster images can only be included by \input if
%% they are in the same directory as the main LaTeX file. For loading figures
%% from other directories you can use the `import` package
%%   \usepackage{import}
%%
%% and then include the figures with
%%   \import{<path to file>}{<filename>.pgf}
%%
%% Matplotlib used the following preamble
%%   \usepackage[utf8x]{inputenc}
%%   \usepackage[T1]{fontenc}
%%
\begingroup%
\makeatletter%
\begin{pgfpicture}%
\pgfpathrectangle{\pgfpointorigin}{\pgfqpoint{6.000000in}{3.400000in}}%
\pgfusepath{use as bounding box, clip}%
\begin{pgfscope}%
\pgfsetbuttcap%
\pgfsetmiterjoin%
\definecolor{currentfill}{rgb}{1.000000,1.000000,1.000000}%
\pgfsetfillcolor{currentfill}%
\pgfsetlinewidth{0.000000pt}%
\definecolor{currentstroke}{rgb}{1.000000,1.000000,1.000000}%
\pgfsetstrokecolor{currentstroke}%
\pgfsetdash{}{0pt}%
\pgfpathmoveto{\pgfqpoint{0.000000in}{0.000000in}}%
\pgfpathlineto{\pgfqpoint{6.000000in}{0.000000in}}%
\pgfpathlineto{\pgfqpoint{6.000000in}{3.400000in}}%
\pgfpathlineto{\pgfqpoint{0.000000in}{3.400000in}}%
\pgfpathclose%
\pgfusepath{fill}%
\end{pgfscope}%
\begin{pgfscope}%
\pgfsetbuttcap%
\pgfsetmiterjoin%
\definecolor{currentfill}{rgb}{1.000000,1.000000,1.000000}%
\pgfsetfillcolor{currentfill}%
\pgfsetlinewidth{0.000000pt}%
\definecolor{currentstroke}{rgb}{0.000000,0.000000,0.000000}%
\pgfsetstrokecolor{currentstroke}%
\pgfsetstrokeopacity{0.000000}%
\pgfsetdash{}{0pt}%
\pgfpathmoveto{\pgfqpoint{0.708220in}{0.535823in}}%
\pgfpathlineto{\pgfqpoint{5.850000in}{0.535823in}}%
\pgfpathlineto{\pgfqpoint{5.850000in}{3.205275in}}%
\pgfpathlineto{\pgfqpoint{0.708220in}{3.205275in}}%
\pgfpathclose%
\pgfusepath{fill}%
\end{pgfscope}%
\begin{pgfscope}%
\pgfsetbuttcap%
\pgfsetroundjoin%
\definecolor{currentfill}{rgb}{0.000000,0.000000,0.000000}%
\pgfsetfillcolor{currentfill}%
\pgfsetlinewidth{0.803000pt}%
\definecolor{currentstroke}{rgb}{0.000000,0.000000,0.000000}%
\pgfsetstrokecolor{currentstroke}%
\pgfsetdash{}{0pt}%
\pgfsys@defobject{currentmarker}{\pgfqpoint{0.000000in}{-0.048611in}}{\pgfqpoint{0.000000in}{0.000000in}}{%
\pgfpathmoveto{\pgfqpoint{0.000000in}{0.000000in}}%
\pgfpathlineto{\pgfqpoint{0.000000in}{-0.048611in}}%
\pgfusepath{stroke,fill}%
}%
\begin{pgfscope}%
\pgfsys@transformshift{0.708220in}{0.535823in}%
\pgfsys@useobject{currentmarker}{}%
\end{pgfscope}%
\end{pgfscope}%
\begin{pgfscope}%
\definecolor{textcolor}{rgb}{0.000000,0.000000,0.000000}%
\pgfsetstrokecolor{textcolor}%
\pgfsetfillcolor{textcolor}%
\pgftext[x=0.708220in,y=0.438600in,,top]{\color{textcolor}\rmfamily\fontsize{9.000000}{10.800000}\selectfont \(\displaystyle {0}\)}%
\end{pgfscope}%
\begin{pgfscope}%
\pgfsetbuttcap%
\pgfsetroundjoin%
\definecolor{currentfill}{rgb}{0.000000,0.000000,0.000000}%
\pgfsetfillcolor{currentfill}%
\pgfsetlinewidth{0.803000pt}%
\definecolor{currentstroke}{rgb}{0.000000,0.000000,0.000000}%
\pgfsetstrokecolor{currentstroke}%
\pgfsetdash{}{0pt}%
\pgfsys@defobject{currentmarker}{\pgfqpoint{0.000000in}{-0.048611in}}{\pgfqpoint{0.000000in}{0.000000in}}{%
\pgfpathmoveto{\pgfqpoint{0.000000in}{0.000000in}}%
\pgfpathlineto{\pgfqpoint{0.000000in}{-0.048611in}}%
\pgfusepath{stroke,fill}%
}%
\begin{pgfscope}%
\pgfsys@transformshift{1.833336in}{0.535823in}%
\pgfsys@useobject{currentmarker}{}%
\end{pgfscope}%
\end{pgfscope}%
\begin{pgfscope}%
\definecolor{textcolor}{rgb}{0.000000,0.000000,0.000000}%
\pgfsetstrokecolor{textcolor}%
\pgfsetfillcolor{textcolor}%
\pgftext[x=1.833336in,y=0.438600in,,top]{\color{textcolor}\rmfamily\fontsize{9.000000}{10.800000}\selectfont \(\displaystyle {50}\)}%
\end{pgfscope}%
\begin{pgfscope}%
\pgfsetbuttcap%
\pgfsetroundjoin%
\definecolor{currentfill}{rgb}{0.000000,0.000000,0.000000}%
\pgfsetfillcolor{currentfill}%
\pgfsetlinewidth{0.803000pt}%
\definecolor{currentstroke}{rgb}{0.000000,0.000000,0.000000}%
\pgfsetstrokecolor{currentstroke}%
\pgfsetdash{}{0pt}%
\pgfsys@defobject{currentmarker}{\pgfqpoint{0.000000in}{-0.048611in}}{\pgfqpoint{0.000000in}{0.000000in}}{%
\pgfpathmoveto{\pgfqpoint{0.000000in}{0.000000in}}%
\pgfpathlineto{\pgfqpoint{0.000000in}{-0.048611in}}%
\pgfusepath{stroke,fill}%
}%
\begin{pgfscope}%
\pgfsys@transformshift{2.958452in}{0.535823in}%
\pgfsys@useobject{currentmarker}{}%
\end{pgfscope}%
\end{pgfscope}%
\begin{pgfscope}%
\definecolor{textcolor}{rgb}{0.000000,0.000000,0.000000}%
\pgfsetstrokecolor{textcolor}%
\pgfsetfillcolor{textcolor}%
\pgftext[x=2.958452in,y=0.438600in,,top]{\color{textcolor}\rmfamily\fontsize{9.000000}{10.800000}\selectfont \(\displaystyle {100}\)}%
\end{pgfscope}%
\begin{pgfscope}%
\pgfsetbuttcap%
\pgfsetroundjoin%
\definecolor{currentfill}{rgb}{0.000000,0.000000,0.000000}%
\pgfsetfillcolor{currentfill}%
\pgfsetlinewidth{0.803000pt}%
\definecolor{currentstroke}{rgb}{0.000000,0.000000,0.000000}%
\pgfsetstrokecolor{currentstroke}%
\pgfsetdash{}{0pt}%
\pgfsys@defobject{currentmarker}{\pgfqpoint{0.000000in}{-0.048611in}}{\pgfqpoint{0.000000in}{0.000000in}}{%
\pgfpathmoveto{\pgfqpoint{0.000000in}{0.000000in}}%
\pgfpathlineto{\pgfqpoint{0.000000in}{-0.048611in}}%
\pgfusepath{stroke,fill}%
}%
\begin{pgfscope}%
\pgfsys@transformshift{4.083568in}{0.535823in}%
\pgfsys@useobject{currentmarker}{}%
\end{pgfscope}%
\end{pgfscope}%
\begin{pgfscope}%
\definecolor{textcolor}{rgb}{0.000000,0.000000,0.000000}%
\pgfsetstrokecolor{textcolor}%
\pgfsetfillcolor{textcolor}%
\pgftext[x=4.083568in,y=0.438600in,,top]{\color{textcolor}\rmfamily\fontsize{9.000000}{10.800000}\selectfont \(\displaystyle {150}\)}%
\end{pgfscope}%
\begin{pgfscope}%
\pgfsetbuttcap%
\pgfsetroundjoin%
\definecolor{currentfill}{rgb}{0.000000,0.000000,0.000000}%
\pgfsetfillcolor{currentfill}%
\pgfsetlinewidth{0.803000pt}%
\definecolor{currentstroke}{rgb}{0.000000,0.000000,0.000000}%
\pgfsetstrokecolor{currentstroke}%
\pgfsetdash{}{0pt}%
\pgfsys@defobject{currentmarker}{\pgfqpoint{0.000000in}{-0.048611in}}{\pgfqpoint{0.000000in}{0.000000in}}{%
\pgfpathmoveto{\pgfqpoint{0.000000in}{0.000000in}}%
\pgfpathlineto{\pgfqpoint{0.000000in}{-0.048611in}}%
\pgfusepath{stroke,fill}%
}%
\begin{pgfscope}%
\pgfsys@transformshift{5.208684in}{0.535823in}%
\pgfsys@useobject{currentmarker}{}%
\end{pgfscope}%
\end{pgfscope}%
\begin{pgfscope}%
\definecolor{textcolor}{rgb}{0.000000,0.000000,0.000000}%
\pgfsetstrokecolor{textcolor}%
\pgfsetfillcolor{textcolor}%
\pgftext[x=5.208684in,y=0.438600in,,top]{\color{textcolor}\rmfamily\fontsize{9.000000}{10.800000}\selectfont \(\displaystyle {200}\)}%
\end{pgfscope}%
\begin{pgfscope}%
\definecolor{textcolor}{rgb}{0.000000,0.000000,0.000000}%
\pgfsetstrokecolor{textcolor}%
\pgfsetfillcolor{textcolor}%
\pgftext[x=3.279110in,y=0.272655in,,top]{\color{textcolor}\rmfamily\fontsize{10.000000}{12.000000}\selectfont \(\displaystyle n\): Number of vertices}%
\end{pgfscope}%
\begin{pgfscope}%
\pgfsetbuttcap%
\pgfsetroundjoin%
\definecolor{currentfill}{rgb}{0.000000,0.000000,0.000000}%
\pgfsetfillcolor{currentfill}%
\pgfsetlinewidth{0.803000pt}%
\definecolor{currentstroke}{rgb}{0.000000,0.000000,0.000000}%
\pgfsetstrokecolor{currentstroke}%
\pgfsetdash{}{0pt}%
\pgfsys@defobject{currentmarker}{\pgfqpoint{-0.048611in}{0.000000in}}{\pgfqpoint{-0.000000in}{0.000000in}}{%
\pgfpathmoveto{\pgfqpoint{-0.000000in}{0.000000in}}%
\pgfpathlineto{\pgfqpoint{-0.048611in}{0.000000in}}%
\pgfusepath{stroke,fill}%
}%
\begin{pgfscope}%
\pgfsys@transformshift{0.708220in}{0.535823in}%
\pgfsys@useobject{currentmarker}{}%
\end{pgfscope}%
\end{pgfscope}%
\begin{pgfscope}%
\definecolor{textcolor}{rgb}{0.000000,0.000000,0.000000}%
\pgfsetstrokecolor{textcolor}%
\pgfsetfillcolor{textcolor}%
\pgftext[x=0.344411in, y=0.491098in, left, base]{\color{textcolor}\rmfamily\fontsize{9.000000}{10.800000}\selectfont \(\displaystyle {10^{-1}}\)}%
\end{pgfscope}%
\begin{pgfscope}%
\pgfsetbuttcap%
\pgfsetroundjoin%
\definecolor{currentfill}{rgb}{0.000000,0.000000,0.000000}%
\pgfsetfillcolor{currentfill}%
\pgfsetlinewidth{0.803000pt}%
\definecolor{currentstroke}{rgb}{0.000000,0.000000,0.000000}%
\pgfsetstrokecolor{currentstroke}%
\pgfsetdash{}{0pt}%
\pgfsys@defobject{currentmarker}{\pgfqpoint{-0.048611in}{0.000000in}}{\pgfqpoint{-0.000000in}{0.000000in}}{%
\pgfpathmoveto{\pgfqpoint{-0.000000in}{0.000000in}}%
\pgfpathlineto{\pgfqpoint{-0.048611in}{0.000000in}}%
\pgfusepath{stroke,fill}%
}%
\begin{pgfscope}%
\pgfsys@transformshift{0.708220in}{1.203186in}%
\pgfsys@useobject{currentmarker}{}%
\end{pgfscope}%
\end{pgfscope}%
\begin{pgfscope}%
\definecolor{textcolor}{rgb}{0.000000,0.000000,0.000000}%
\pgfsetstrokecolor{textcolor}%
\pgfsetfillcolor{textcolor}%
\pgftext[x=0.424657in, y=1.158461in, left, base]{\color{textcolor}\rmfamily\fontsize{9.000000}{10.800000}\selectfont \(\displaystyle {10^{0}}\)}%
\end{pgfscope}%
\begin{pgfscope}%
\pgfsetbuttcap%
\pgfsetroundjoin%
\definecolor{currentfill}{rgb}{0.000000,0.000000,0.000000}%
\pgfsetfillcolor{currentfill}%
\pgfsetlinewidth{0.803000pt}%
\definecolor{currentstroke}{rgb}{0.000000,0.000000,0.000000}%
\pgfsetstrokecolor{currentstroke}%
\pgfsetdash{}{0pt}%
\pgfsys@defobject{currentmarker}{\pgfqpoint{-0.048611in}{0.000000in}}{\pgfqpoint{-0.000000in}{0.000000in}}{%
\pgfpathmoveto{\pgfqpoint{-0.000000in}{0.000000in}}%
\pgfpathlineto{\pgfqpoint{-0.048611in}{0.000000in}}%
\pgfusepath{stroke,fill}%
}%
\begin{pgfscope}%
\pgfsys@transformshift{0.708220in}{1.870549in}%
\pgfsys@useobject{currentmarker}{}%
\end{pgfscope}%
\end{pgfscope}%
\begin{pgfscope}%
\definecolor{textcolor}{rgb}{0.000000,0.000000,0.000000}%
\pgfsetstrokecolor{textcolor}%
\pgfsetfillcolor{textcolor}%
\pgftext[x=0.424657in, y=1.825824in, left, base]{\color{textcolor}\rmfamily\fontsize{9.000000}{10.800000}\selectfont \(\displaystyle {10^{1}}\)}%
\end{pgfscope}%
\begin{pgfscope}%
\pgfsetbuttcap%
\pgfsetroundjoin%
\definecolor{currentfill}{rgb}{0.000000,0.000000,0.000000}%
\pgfsetfillcolor{currentfill}%
\pgfsetlinewidth{0.803000pt}%
\definecolor{currentstroke}{rgb}{0.000000,0.000000,0.000000}%
\pgfsetstrokecolor{currentstroke}%
\pgfsetdash{}{0pt}%
\pgfsys@defobject{currentmarker}{\pgfqpoint{-0.048611in}{0.000000in}}{\pgfqpoint{-0.000000in}{0.000000in}}{%
\pgfpathmoveto{\pgfqpoint{-0.000000in}{0.000000in}}%
\pgfpathlineto{\pgfqpoint{-0.048611in}{0.000000in}}%
\pgfusepath{stroke,fill}%
}%
\begin{pgfscope}%
\pgfsys@transformshift{0.708220in}{2.537912in}%
\pgfsys@useobject{currentmarker}{}%
\end{pgfscope}%
\end{pgfscope}%
\begin{pgfscope}%
\definecolor{textcolor}{rgb}{0.000000,0.000000,0.000000}%
\pgfsetstrokecolor{textcolor}%
\pgfsetfillcolor{textcolor}%
\pgftext[x=0.424657in, y=2.493187in, left, base]{\color{textcolor}\rmfamily\fontsize{9.000000}{10.800000}\selectfont \(\displaystyle {10^{2}}\)}%
\end{pgfscope}%
\begin{pgfscope}%
\pgfsetbuttcap%
\pgfsetroundjoin%
\definecolor{currentfill}{rgb}{0.000000,0.000000,0.000000}%
\pgfsetfillcolor{currentfill}%
\pgfsetlinewidth{0.803000pt}%
\definecolor{currentstroke}{rgb}{0.000000,0.000000,0.000000}%
\pgfsetstrokecolor{currentstroke}%
\pgfsetdash{}{0pt}%
\pgfsys@defobject{currentmarker}{\pgfqpoint{-0.048611in}{0.000000in}}{\pgfqpoint{-0.000000in}{0.000000in}}{%
\pgfpathmoveto{\pgfqpoint{-0.000000in}{0.000000in}}%
\pgfpathlineto{\pgfqpoint{-0.048611in}{0.000000in}}%
\pgfusepath{stroke,fill}%
}%
\begin{pgfscope}%
\pgfsys@transformshift{0.708220in}{3.205275in}%
\pgfsys@useobject{currentmarker}{}%
\end{pgfscope}%
\end{pgfscope}%
\begin{pgfscope}%
\definecolor{textcolor}{rgb}{0.000000,0.000000,0.000000}%
\pgfsetstrokecolor{textcolor}%
\pgfsetfillcolor{textcolor}%
\pgftext[x=0.424657in, y=3.160550in, left, base]{\color{textcolor}\rmfamily\fontsize{9.000000}{10.800000}\selectfont \(\displaystyle {10^{3}}\)}%
\end{pgfscope}%
\begin{pgfscope}%
\pgfsetbuttcap%
\pgfsetroundjoin%
\definecolor{currentfill}{rgb}{0.000000,0.000000,0.000000}%
\pgfsetfillcolor{currentfill}%
\pgfsetlinewidth{0.602250pt}%
\definecolor{currentstroke}{rgb}{0.000000,0.000000,0.000000}%
\pgfsetstrokecolor{currentstroke}%
\pgfsetdash{}{0pt}%
\pgfsys@defobject{currentmarker}{\pgfqpoint{-0.027778in}{0.000000in}}{\pgfqpoint{-0.000000in}{0.000000in}}{%
\pgfpathmoveto{\pgfqpoint{-0.000000in}{0.000000in}}%
\pgfpathlineto{\pgfqpoint{-0.027778in}{0.000000in}}%
\pgfusepath{stroke,fill}%
}%
\begin{pgfscope}%
\pgfsys@transformshift{0.708220in}{0.736719in}%
\pgfsys@useobject{currentmarker}{}%
\end{pgfscope}%
\end{pgfscope}%
\begin{pgfscope}%
\pgfsetbuttcap%
\pgfsetroundjoin%
\definecolor{currentfill}{rgb}{0.000000,0.000000,0.000000}%
\pgfsetfillcolor{currentfill}%
\pgfsetlinewidth{0.602250pt}%
\definecolor{currentstroke}{rgb}{0.000000,0.000000,0.000000}%
\pgfsetstrokecolor{currentstroke}%
\pgfsetdash{}{0pt}%
\pgfsys@defobject{currentmarker}{\pgfqpoint{-0.027778in}{0.000000in}}{\pgfqpoint{-0.000000in}{0.000000in}}{%
\pgfpathmoveto{\pgfqpoint{-0.000000in}{0.000000in}}%
\pgfpathlineto{\pgfqpoint{-0.027778in}{0.000000in}}%
\pgfusepath{stroke,fill}%
}%
\begin{pgfscope}%
\pgfsys@transformshift{0.708220in}{0.854236in}%
\pgfsys@useobject{currentmarker}{}%
\end{pgfscope}%
\end{pgfscope}%
\begin{pgfscope}%
\pgfsetbuttcap%
\pgfsetroundjoin%
\definecolor{currentfill}{rgb}{0.000000,0.000000,0.000000}%
\pgfsetfillcolor{currentfill}%
\pgfsetlinewidth{0.602250pt}%
\definecolor{currentstroke}{rgb}{0.000000,0.000000,0.000000}%
\pgfsetstrokecolor{currentstroke}%
\pgfsetdash{}{0pt}%
\pgfsys@defobject{currentmarker}{\pgfqpoint{-0.027778in}{0.000000in}}{\pgfqpoint{-0.000000in}{0.000000in}}{%
\pgfpathmoveto{\pgfqpoint{-0.000000in}{0.000000in}}%
\pgfpathlineto{\pgfqpoint{-0.027778in}{0.000000in}}%
\pgfusepath{stroke,fill}%
}%
\begin{pgfscope}%
\pgfsys@transformshift{0.708220in}{0.937615in}%
\pgfsys@useobject{currentmarker}{}%
\end{pgfscope}%
\end{pgfscope}%
\begin{pgfscope}%
\pgfsetbuttcap%
\pgfsetroundjoin%
\definecolor{currentfill}{rgb}{0.000000,0.000000,0.000000}%
\pgfsetfillcolor{currentfill}%
\pgfsetlinewidth{0.602250pt}%
\definecolor{currentstroke}{rgb}{0.000000,0.000000,0.000000}%
\pgfsetstrokecolor{currentstroke}%
\pgfsetdash{}{0pt}%
\pgfsys@defobject{currentmarker}{\pgfqpoint{-0.027778in}{0.000000in}}{\pgfqpoint{-0.000000in}{0.000000in}}{%
\pgfpathmoveto{\pgfqpoint{-0.000000in}{0.000000in}}%
\pgfpathlineto{\pgfqpoint{-0.027778in}{0.000000in}}%
\pgfusepath{stroke,fill}%
}%
\begin{pgfscope}%
\pgfsys@transformshift{0.708220in}{1.002289in}%
\pgfsys@useobject{currentmarker}{}%
\end{pgfscope}%
\end{pgfscope}%
\begin{pgfscope}%
\pgfsetbuttcap%
\pgfsetroundjoin%
\definecolor{currentfill}{rgb}{0.000000,0.000000,0.000000}%
\pgfsetfillcolor{currentfill}%
\pgfsetlinewidth{0.602250pt}%
\definecolor{currentstroke}{rgb}{0.000000,0.000000,0.000000}%
\pgfsetstrokecolor{currentstroke}%
\pgfsetdash{}{0pt}%
\pgfsys@defobject{currentmarker}{\pgfqpoint{-0.027778in}{0.000000in}}{\pgfqpoint{-0.000000in}{0.000000in}}{%
\pgfpathmoveto{\pgfqpoint{-0.000000in}{0.000000in}}%
\pgfpathlineto{\pgfqpoint{-0.027778in}{0.000000in}}%
\pgfusepath{stroke,fill}%
}%
\begin{pgfscope}%
\pgfsys@transformshift{0.708220in}{1.055132in}%
\pgfsys@useobject{currentmarker}{}%
\end{pgfscope}%
\end{pgfscope}%
\begin{pgfscope}%
\pgfsetbuttcap%
\pgfsetroundjoin%
\definecolor{currentfill}{rgb}{0.000000,0.000000,0.000000}%
\pgfsetfillcolor{currentfill}%
\pgfsetlinewidth{0.602250pt}%
\definecolor{currentstroke}{rgb}{0.000000,0.000000,0.000000}%
\pgfsetstrokecolor{currentstroke}%
\pgfsetdash{}{0pt}%
\pgfsys@defobject{currentmarker}{\pgfqpoint{-0.027778in}{0.000000in}}{\pgfqpoint{-0.000000in}{0.000000in}}{%
\pgfpathmoveto{\pgfqpoint{-0.000000in}{0.000000in}}%
\pgfpathlineto{\pgfqpoint{-0.027778in}{0.000000in}}%
\pgfusepath{stroke,fill}%
}%
\begin{pgfscope}%
\pgfsys@transformshift{0.708220in}{1.099810in}%
\pgfsys@useobject{currentmarker}{}%
\end{pgfscope}%
\end{pgfscope}%
\begin{pgfscope}%
\pgfsetbuttcap%
\pgfsetroundjoin%
\definecolor{currentfill}{rgb}{0.000000,0.000000,0.000000}%
\pgfsetfillcolor{currentfill}%
\pgfsetlinewidth{0.602250pt}%
\definecolor{currentstroke}{rgb}{0.000000,0.000000,0.000000}%
\pgfsetstrokecolor{currentstroke}%
\pgfsetdash{}{0pt}%
\pgfsys@defobject{currentmarker}{\pgfqpoint{-0.027778in}{0.000000in}}{\pgfqpoint{-0.000000in}{0.000000in}}{%
\pgfpathmoveto{\pgfqpoint{-0.000000in}{0.000000in}}%
\pgfpathlineto{\pgfqpoint{-0.027778in}{0.000000in}}%
\pgfusepath{stroke,fill}%
}%
\begin{pgfscope}%
\pgfsys@transformshift{0.708220in}{1.138512in}%
\pgfsys@useobject{currentmarker}{}%
\end{pgfscope}%
\end{pgfscope}%
\begin{pgfscope}%
\pgfsetbuttcap%
\pgfsetroundjoin%
\definecolor{currentfill}{rgb}{0.000000,0.000000,0.000000}%
\pgfsetfillcolor{currentfill}%
\pgfsetlinewidth{0.602250pt}%
\definecolor{currentstroke}{rgb}{0.000000,0.000000,0.000000}%
\pgfsetstrokecolor{currentstroke}%
\pgfsetdash{}{0pt}%
\pgfsys@defobject{currentmarker}{\pgfqpoint{-0.027778in}{0.000000in}}{\pgfqpoint{-0.000000in}{0.000000in}}{%
\pgfpathmoveto{\pgfqpoint{-0.000000in}{0.000000in}}%
\pgfpathlineto{\pgfqpoint{-0.027778in}{0.000000in}}%
\pgfusepath{stroke,fill}%
}%
\begin{pgfscope}%
\pgfsys@transformshift{0.708220in}{1.172649in}%
\pgfsys@useobject{currentmarker}{}%
\end{pgfscope}%
\end{pgfscope}%
\begin{pgfscope}%
\pgfsetbuttcap%
\pgfsetroundjoin%
\definecolor{currentfill}{rgb}{0.000000,0.000000,0.000000}%
\pgfsetfillcolor{currentfill}%
\pgfsetlinewidth{0.602250pt}%
\definecolor{currentstroke}{rgb}{0.000000,0.000000,0.000000}%
\pgfsetstrokecolor{currentstroke}%
\pgfsetdash{}{0pt}%
\pgfsys@defobject{currentmarker}{\pgfqpoint{-0.027778in}{0.000000in}}{\pgfqpoint{-0.000000in}{0.000000in}}{%
\pgfpathmoveto{\pgfqpoint{-0.000000in}{0.000000in}}%
\pgfpathlineto{\pgfqpoint{-0.027778in}{0.000000in}}%
\pgfusepath{stroke,fill}%
}%
\begin{pgfscope}%
\pgfsys@transformshift{0.708220in}{1.404082in}%
\pgfsys@useobject{currentmarker}{}%
\end{pgfscope}%
\end{pgfscope}%
\begin{pgfscope}%
\pgfsetbuttcap%
\pgfsetroundjoin%
\definecolor{currentfill}{rgb}{0.000000,0.000000,0.000000}%
\pgfsetfillcolor{currentfill}%
\pgfsetlinewidth{0.602250pt}%
\definecolor{currentstroke}{rgb}{0.000000,0.000000,0.000000}%
\pgfsetstrokecolor{currentstroke}%
\pgfsetdash{}{0pt}%
\pgfsys@defobject{currentmarker}{\pgfqpoint{-0.027778in}{0.000000in}}{\pgfqpoint{-0.000000in}{0.000000in}}{%
\pgfpathmoveto{\pgfqpoint{-0.000000in}{0.000000in}}%
\pgfpathlineto{\pgfqpoint{-0.027778in}{0.000000in}}%
\pgfusepath{stroke,fill}%
}%
\begin{pgfscope}%
\pgfsys@transformshift{0.708220in}{1.521599in}%
\pgfsys@useobject{currentmarker}{}%
\end{pgfscope}%
\end{pgfscope}%
\begin{pgfscope}%
\pgfsetbuttcap%
\pgfsetroundjoin%
\definecolor{currentfill}{rgb}{0.000000,0.000000,0.000000}%
\pgfsetfillcolor{currentfill}%
\pgfsetlinewidth{0.602250pt}%
\definecolor{currentstroke}{rgb}{0.000000,0.000000,0.000000}%
\pgfsetstrokecolor{currentstroke}%
\pgfsetdash{}{0pt}%
\pgfsys@defobject{currentmarker}{\pgfqpoint{-0.027778in}{0.000000in}}{\pgfqpoint{-0.000000in}{0.000000in}}{%
\pgfpathmoveto{\pgfqpoint{-0.000000in}{0.000000in}}%
\pgfpathlineto{\pgfqpoint{-0.027778in}{0.000000in}}%
\pgfusepath{stroke,fill}%
}%
\begin{pgfscope}%
\pgfsys@transformshift{0.708220in}{1.604978in}%
\pgfsys@useobject{currentmarker}{}%
\end{pgfscope}%
\end{pgfscope}%
\begin{pgfscope}%
\pgfsetbuttcap%
\pgfsetroundjoin%
\definecolor{currentfill}{rgb}{0.000000,0.000000,0.000000}%
\pgfsetfillcolor{currentfill}%
\pgfsetlinewidth{0.602250pt}%
\definecolor{currentstroke}{rgb}{0.000000,0.000000,0.000000}%
\pgfsetstrokecolor{currentstroke}%
\pgfsetdash{}{0pt}%
\pgfsys@defobject{currentmarker}{\pgfqpoint{-0.027778in}{0.000000in}}{\pgfqpoint{-0.000000in}{0.000000in}}{%
\pgfpathmoveto{\pgfqpoint{-0.000000in}{0.000000in}}%
\pgfpathlineto{\pgfqpoint{-0.027778in}{0.000000in}}%
\pgfusepath{stroke,fill}%
}%
\begin{pgfscope}%
\pgfsys@transformshift{0.708220in}{1.669653in}%
\pgfsys@useobject{currentmarker}{}%
\end{pgfscope}%
\end{pgfscope}%
\begin{pgfscope}%
\pgfsetbuttcap%
\pgfsetroundjoin%
\definecolor{currentfill}{rgb}{0.000000,0.000000,0.000000}%
\pgfsetfillcolor{currentfill}%
\pgfsetlinewidth{0.602250pt}%
\definecolor{currentstroke}{rgb}{0.000000,0.000000,0.000000}%
\pgfsetstrokecolor{currentstroke}%
\pgfsetdash{}{0pt}%
\pgfsys@defobject{currentmarker}{\pgfqpoint{-0.027778in}{0.000000in}}{\pgfqpoint{-0.000000in}{0.000000in}}{%
\pgfpathmoveto{\pgfqpoint{-0.000000in}{0.000000in}}%
\pgfpathlineto{\pgfqpoint{-0.027778in}{0.000000in}}%
\pgfusepath{stroke,fill}%
}%
\begin{pgfscope}%
\pgfsys@transformshift{0.708220in}{1.722495in}%
\pgfsys@useobject{currentmarker}{}%
\end{pgfscope}%
\end{pgfscope}%
\begin{pgfscope}%
\pgfsetbuttcap%
\pgfsetroundjoin%
\definecolor{currentfill}{rgb}{0.000000,0.000000,0.000000}%
\pgfsetfillcolor{currentfill}%
\pgfsetlinewidth{0.602250pt}%
\definecolor{currentstroke}{rgb}{0.000000,0.000000,0.000000}%
\pgfsetstrokecolor{currentstroke}%
\pgfsetdash{}{0pt}%
\pgfsys@defobject{currentmarker}{\pgfqpoint{-0.027778in}{0.000000in}}{\pgfqpoint{-0.000000in}{0.000000in}}{%
\pgfpathmoveto{\pgfqpoint{-0.000000in}{0.000000in}}%
\pgfpathlineto{\pgfqpoint{-0.027778in}{0.000000in}}%
\pgfusepath{stroke,fill}%
}%
\begin{pgfscope}%
\pgfsys@transformshift{0.708220in}{1.767173in}%
\pgfsys@useobject{currentmarker}{}%
\end{pgfscope}%
\end{pgfscope}%
\begin{pgfscope}%
\pgfsetbuttcap%
\pgfsetroundjoin%
\definecolor{currentfill}{rgb}{0.000000,0.000000,0.000000}%
\pgfsetfillcolor{currentfill}%
\pgfsetlinewidth{0.602250pt}%
\definecolor{currentstroke}{rgb}{0.000000,0.000000,0.000000}%
\pgfsetstrokecolor{currentstroke}%
\pgfsetdash{}{0pt}%
\pgfsys@defobject{currentmarker}{\pgfqpoint{-0.027778in}{0.000000in}}{\pgfqpoint{-0.000000in}{0.000000in}}{%
\pgfpathmoveto{\pgfqpoint{-0.000000in}{0.000000in}}%
\pgfpathlineto{\pgfqpoint{-0.027778in}{0.000000in}}%
\pgfusepath{stroke,fill}%
}%
\begin{pgfscope}%
\pgfsys@transformshift{0.708220in}{1.805875in}%
\pgfsys@useobject{currentmarker}{}%
\end{pgfscope}%
\end{pgfscope}%
\begin{pgfscope}%
\pgfsetbuttcap%
\pgfsetroundjoin%
\definecolor{currentfill}{rgb}{0.000000,0.000000,0.000000}%
\pgfsetfillcolor{currentfill}%
\pgfsetlinewidth{0.602250pt}%
\definecolor{currentstroke}{rgb}{0.000000,0.000000,0.000000}%
\pgfsetstrokecolor{currentstroke}%
\pgfsetdash{}{0pt}%
\pgfsys@defobject{currentmarker}{\pgfqpoint{-0.027778in}{0.000000in}}{\pgfqpoint{-0.000000in}{0.000000in}}{%
\pgfpathmoveto{\pgfqpoint{-0.000000in}{0.000000in}}%
\pgfpathlineto{\pgfqpoint{-0.027778in}{0.000000in}}%
\pgfusepath{stroke,fill}%
}%
\begin{pgfscope}%
\pgfsys@transformshift{0.708220in}{1.840012in}%
\pgfsys@useobject{currentmarker}{}%
\end{pgfscope}%
\end{pgfscope}%
\begin{pgfscope}%
\pgfsetbuttcap%
\pgfsetroundjoin%
\definecolor{currentfill}{rgb}{0.000000,0.000000,0.000000}%
\pgfsetfillcolor{currentfill}%
\pgfsetlinewidth{0.602250pt}%
\definecolor{currentstroke}{rgb}{0.000000,0.000000,0.000000}%
\pgfsetstrokecolor{currentstroke}%
\pgfsetdash{}{0pt}%
\pgfsys@defobject{currentmarker}{\pgfqpoint{-0.027778in}{0.000000in}}{\pgfqpoint{-0.000000in}{0.000000in}}{%
\pgfpathmoveto{\pgfqpoint{-0.000000in}{0.000000in}}%
\pgfpathlineto{\pgfqpoint{-0.027778in}{0.000000in}}%
\pgfusepath{stroke,fill}%
}%
\begin{pgfscope}%
\pgfsys@transformshift{0.708220in}{2.071445in}%
\pgfsys@useobject{currentmarker}{}%
\end{pgfscope}%
\end{pgfscope}%
\begin{pgfscope}%
\pgfsetbuttcap%
\pgfsetroundjoin%
\definecolor{currentfill}{rgb}{0.000000,0.000000,0.000000}%
\pgfsetfillcolor{currentfill}%
\pgfsetlinewidth{0.602250pt}%
\definecolor{currentstroke}{rgb}{0.000000,0.000000,0.000000}%
\pgfsetstrokecolor{currentstroke}%
\pgfsetdash{}{0pt}%
\pgfsys@defobject{currentmarker}{\pgfqpoint{-0.027778in}{0.000000in}}{\pgfqpoint{-0.000000in}{0.000000in}}{%
\pgfpathmoveto{\pgfqpoint{-0.000000in}{0.000000in}}%
\pgfpathlineto{\pgfqpoint{-0.027778in}{0.000000in}}%
\pgfusepath{stroke,fill}%
}%
\begin{pgfscope}%
\pgfsys@transformshift{0.708220in}{2.188962in}%
\pgfsys@useobject{currentmarker}{}%
\end{pgfscope}%
\end{pgfscope}%
\begin{pgfscope}%
\pgfsetbuttcap%
\pgfsetroundjoin%
\definecolor{currentfill}{rgb}{0.000000,0.000000,0.000000}%
\pgfsetfillcolor{currentfill}%
\pgfsetlinewidth{0.602250pt}%
\definecolor{currentstroke}{rgb}{0.000000,0.000000,0.000000}%
\pgfsetstrokecolor{currentstroke}%
\pgfsetdash{}{0pt}%
\pgfsys@defobject{currentmarker}{\pgfqpoint{-0.027778in}{0.000000in}}{\pgfqpoint{-0.000000in}{0.000000in}}{%
\pgfpathmoveto{\pgfqpoint{-0.000000in}{0.000000in}}%
\pgfpathlineto{\pgfqpoint{-0.027778in}{0.000000in}}%
\pgfusepath{stroke,fill}%
}%
\begin{pgfscope}%
\pgfsys@transformshift{0.708220in}{2.272342in}%
\pgfsys@useobject{currentmarker}{}%
\end{pgfscope}%
\end{pgfscope}%
\begin{pgfscope}%
\pgfsetbuttcap%
\pgfsetroundjoin%
\definecolor{currentfill}{rgb}{0.000000,0.000000,0.000000}%
\pgfsetfillcolor{currentfill}%
\pgfsetlinewidth{0.602250pt}%
\definecolor{currentstroke}{rgb}{0.000000,0.000000,0.000000}%
\pgfsetstrokecolor{currentstroke}%
\pgfsetdash{}{0pt}%
\pgfsys@defobject{currentmarker}{\pgfqpoint{-0.027778in}{0.000000in}}{\pgfqpoint{-0.000000in}{0.000000in}}{%
\pgfpathmoveto{\pgfqpoint{-0.000000in}{0.000000in}}%
\pgfpathlineto{\pgfqpoint{-0.027778in}{0.000000in}}%
\pgfusepath{stroke,fill}%
}%
\begin{pgfscope}%
\pgfsys@transformshift{0.708220in}{2.337016in}%
\pgfsys@useobject{currentmarker}{}%
\end{pgfscope}%
\end{pgfscope}%
\begin{pgfscope}%
\pgfsetbuttcap%
\pgfsetroundjoin%
\definecolor{currentfill}{rgb}{0.000000,0.000000,0.000000}%
\pgfsetfillcolor{currentfill}%
\pgfsetlinewidth{0.602250pt}%
\definecolor{currentstroke}{rgb}{0.000000,0.000000,0.000000}%
\pgfsetstrokecolor{currentstroke}%
\pgfsetdash{}{0pt}%
\pgfsys@defobject{currentmarker}{\pgfqpoint{-0.027778in}{0.000000in}}{\pgfqpoint{-0.000000in}{0.000000in}}{%
\pgfpathmoveto{\pgfqpoint{-0.000000in}{0.000000in}}%
\pgfpathlineto{\pgfqpoint{-0.027778in}{0.000000in}}%
\pgfusepath{stroke,fill}%
}%
\begin{pgfscope}%
\pgfsys@transformshift{0.708220in}{2.389858in}%
\pgfsys@useobject{currentmarker}{}%
\end{pgfscope}%
\end{pgfscope}%
\begin{pgfscope}%
\pgfsetbuttcap%
\pgfsetroundjoin%
\definecolor{currentfill}{rgb}{0.000000,0.000000,0.000000}%
\pgfsetfillcolor{currentfill}%
\pgfsetlinewidth{0.602250pt}%
\definecolor{currentstroke}{rgb}{0.000000,0.000000,0.000000}%
\pgfsetstrokecolor{currentstroke}%
\pgfsetdash{}{0pt}%
\pgfsys@defobject{currentmarker}{\pgfqpoint{-0.027778in}{0.000000in}}{\pgfqpoint{-0.000000in}{0.000000in}}{%
\pgfpathmoveto{\pgfqpoint{-0.000000in}{0.000000in}}%
\pgfpathlineto{\pgfqpoint{-0.027778in}{0.000000in}}%
\pgfusepath{stroke,fill}%
}%
\begin{pgfscope}%
\pgfsys@transformshift{0.708220in}{2.434536in}%
\pgfsys@useobject{currentmarker}{}%
\end{pgfscope}%
\end{pgfscope}%
\begin{pgfscope}%
\pgfsetbuttcap%
\pgfsetroundjoin%
\definecolor{currentfill}{rgb}{0.000000,0.000000,0.000000}%
\pgfsetfillcolor{currentfill}%
\pgfsetlinewidth{0.602250pt}%
\definecolor{currentstroke}{rgb}{0.000000,0.000000,0.000000}%
\pgfsetstrokecolor{currentstroke}%
\pgfsetdash{}{0pt}%
\pgfsys@defobject{currentmarker}{\pgfqpoint{-0.027778in}{0.000000in}}{\pgfqpoint{-0.000000in}{0.000000in}}{%
\pgfpathmoveto{\pgfqpoint{-0.000000in}{0.000000in}}%
\pgfpathlineto{\pgfqpoint{-0.027778in}{0.000000in}}%
\pgfusepath{stroke,fill}%
}%
\begin{pgfscope}%
\pgfsys@transformshift{0.708220in}{2.473238in}%
\pgfsys@useobject{currentmarker}{}%
\end{pgfscope}%
\end{pgfscope}%
\begin{pgfscope}%
\pgfsetbuttcap%
\pgfsetroundjoin%
\definecolor{currentfill}{rgb}{0.000000,0.000000,0.000000}%
\pgfsetfillcolor{currentfill}%
\pgfsetlinewidth{0.602250pt}%
\definecolor{currentstroke}{rgb}{0.000000,0.000000,0.000000}%
\pgfsetstrokecolor{currentstroke}%
\pgfsetdash{}{0pt}%
\pgfsys@defobject{currentmarker}{\pgfqpoint{-0.027778in}{0.000000in}}{\pgfqpoint{-0.000000in}{0.000000in}}{%
\pgfpathmoveto{\pgfqpoint{-0.000000in}{0.000000in}}%
\pgfpathlineto{\pgfqpoint{-0.027778in}{0.000000in}}%
\pgfusepath{stroke,fill}%
}%
\begin{pgfscope}%
\pgfsys@transformshift{0.708220in}{2.507375in}%
\pgfsys@useobject{currentmarker}{}%
\end{pgfscope}%
\end{pgfscope}%
\begin{pgfscope}%
\pgfsetbuttcap%
\pgfsetroundjoin%
\definecolor{currentfill}{rgb}{0.000000,0.000000,0.000000}%
\pgfsetfillcolor{currentfill}%
\pgfsetlinewidth{0.602250pt}%
\definecolor{currentstroke}{rgb}{0.000000,0.000000,0.000000}%
\pgfsetstrokecolor{currentstroke}%
\pgfsetdash{}{0pt}%
\pgfsys@defobject{currentmarker}{\pgfqpoint{-0.027778in}{0.000000in}}{\pgfqpoint{-0.000000in}{0.000000in}}{%
\pgfpathmoveto{\pgfqpoint{-0.000000in}{0.000000in}}%
\pgfpathlineto{\pgfqpoint{-0.027778in}{0.000000in}}%
\pgfusepath{stroke,fill}%
}%
\begin{pgfscope}%
\pgfsys@transformshift{0.708220in}{2.738808in}%
\pgfsys@useobject{currentmarker}{}%
\end{pgfscope}%
\end{pgfscope}%
\begin{pgfscope}%
\pgfsetbuttcap%
\pgfsetroundjoin%
\definecolor{currentfill}{rgb}{0.000000,0.000000,0.000000}%
\pgfsetfillcolor{currentfill}%
\pgfsetlinewidth{0.602250pt}%
\definecolor{currentstroke}{rgb}{0.000000,0.000000,0.000000}%
\pgfsetstrokecolor{currentstroke}%
\pgfsetdash{}{0pt}%
\pgfsys@defobject{currentmarker}{\pgfqpoint{-0.027778in}{0.000000in}}{\pgfqpoint{-0.000000in}{0.000000in}}{%
\pgfpathmoveto{\pgfqpoint{-0.000000in}{0.000000in}}%
\pgfpathlineto{\pgfqpoint{-0.027778in}{0.000000in}}%
\pgfusepath{stroke,fill}%
}%
\begin{pgfscope}%
\pgfsys@transformshift{0.708220in}{2.856325in}%
\pgfsys@useobject{currentmarker}{}%
\end{pgfscope}%
\end{pgfscope}%
\begin{pgfscope}%
\pgfsetbuttcap%
\pgfsetroundjoin%
\definecolor{currentfill}{rgb}{0.000000,0.000000,0.000000}%
\pgfsetfillcolor{currentfill}%
\pgfsetlinewidth{0.602250pt}%
\definecolor{currentstroke}{rgb}{0.000000,0.000000,0.000000}%
\pgfsetstrokecolor{currentstroke}%
\pgfsetdash{}{0pt}%
\pgfsys@defobject{currentmarker}{\pgfqpoint{-0.027778in}{0.000000in}}{\pgfqpoint{-0.000000in}{0.000000in}}{%
\pgfpathmoveto{\pgfqpoint{-0.000000in}{0.000000in}}%
\pgfpathlineto{\pgfqpoint{-0.027778in}{0.000000in}}%
\pgfusepath{stroke,fill}%
}%
\begin{pgfscope}%
\pgfsys@transformshift{0.708220in}{2.939705in}%
\pgfsys@useobject{currentmarker}{}%
\end{pgfscope}%
\end{pgfscope}%
\begin{pgfscope}%
\pgfsetbuttcap%
\pgfsetroundjoin%
\definecolor{currentfill}{rgb}{0.000000,0.000000,0.000000}%
\pgfsetfillcolor{currentfill}%
\pgfsetlinewidth{0.602250pt}%
\definecolor{currentstroke}{rgb}{0.000000,0.000000,0.000000}%
\pgfsetstrokecolor{currentstroke}%
\pgfsetdash{}{0pt}%
\pgfsys@defobject{currentmarker}{\pgfqpoint{-0.027778in}{0.000000in}}{\pgfqpoint{-0.000000in}{0.000000in}}{%
\pgfpathmoveto{\pgfqpoint{-0.000000in}{0.000000in}}%
\pgfpathlineto{\pgfqpoint{-0.027778in}{0.000000in}}%
\pgfusepath{stroke,fill}%
}%
\begin{pgfscope}%
\pgfsys@transformshift{0.708220in}{3.004379in}%
\pgfsys@useobject{currentmarker}{}%
\end{pgfscope}%
\end{pgfscope}%
\begin{pgfscope}%
\pgfsetbuttcap%
\pgfsetroundjoin%
\definecolor{currentfill}{rgb}{0.000000,0.000000,0.000000}%
\pgfsetfillcolor{currentfill}%
\pgfsetlinewidth{0.602250pt}%
\definecolor{currentstroke}{rgb}{0.000000,0.000000,0.000000}%
\pgfsetstrokecolor{currentstroke}%
\pgfsetdash{}{0pt}%
\pgfsys@defobject{currentmarker}{\pgfqpoint{-0.027778in}{0.000000in}}{\pgfqpoint{-0.000000in}{0.000000in}}{%
\pgfpathmoveto{\pgfqpoint{-0.000000in}{0.000000in}}%
\pgfpathlineto{\pgfqpoint{-0.027778in}{0.000000in}}%
\pgfusepath{stroke,fill}%
}%
\begin{pgfscope}%
\pgfsys@transformshift{0.708220in}{3.057222in}%
\pgfsys@useobject{currentmarker}{}%
\end{pgfscope}%
\end{pgfscope}%
\begin{pgfscope}%
\pgfsetbuttcap%
\pgfsetroundjoin%
\definecolor{currentfill}{rgb}{0.000000,0.000000,0.000000}%
\pgfsetfillcolor{currentfill}%
\pgfsetlinewidth{0.602250pt}%
\definecolor{currentstroke}{rgb}{0.000000,0.000000,0.000000}%
\pgfsetstrokecolor{currentstroke}%
\pgfsetdash{}{0pt}%
\pgfsys@defobject{currentmarker}{\pgfqpoint{-0.027778in}{0.000000in}}{\pgfqpoint{-0.000000in}{0.000000in}}{%
\pgfpathmoveto{\pgfqpoint{-0.000000in}{0.000000in}}%
\pgfpathlineto{\pgfqpoint{-0.027778in}{0.000000in}}%
\pgfusepath{stroke,fill}%
}%
\begin{pgfscope}%
\pgfsys@transformshift{0.708220in}{3.101899in}%
\pgfsys@useobject{currentmarker}{}%
\end{pgfscope}%
\end{pgfscope}%
\begin{pgfscope}%
\pgfsetbuttcap%
\pgfsetroundjoin%
\definecolor{currentfill}{rgb}{0.000000,0.000000,0.000000}%
\pgfsetfillcolor{currentfill}%
\pgfsetlinewidth{0.602250pt}%
\definecolor{currentstroke}{rgb}{0.000000,0.000000,0.000000}%
\pgfsetstrokecolor{currentstroke}%
\pgfsetdash{}{0pt}%
\pgfsys@defobject{currentmarker}{\pgfqpoint{-0.027778in}{0.000000in}}{\pgfqpoint{-0.000000in}{0.000000in}}{%
\pgfpathmoveto{\pgfqpoint{-0.000000in}{0.000000in}}%
\pgfpathlineto{\pgfqpoint{-0.027778in}{0.000000in}}%
\pgfusepath{stroke,fill}%
}%
\begin{pgfscope}%
\pgfsys@transformshift{0.708220in}{3.140601in}%
\pgfsys@useobject{currentmarker}{}%
\end{pgfscope}%
\end{pgfscope}%
\begin{pgfscope}%
\pgfsetbuttcap%
\pgfsetroundjoin%
\definecolor{currentfill}{rgb}{0.000000,0.000000,0.000000}%
\pgfsetfillcolor{currentfill}%
\pgfsetlinewidth{0.602250pt}%
\definecolor{currentstroke}{rgb}{0.000000,0.000000,0.000000}%
\pgfsetstrokecolor{currentstroke}%
\pgfsetdash{}{0pt}%
\pgfsys@defobject{currentmarker}{\pgfqpoint{-0.027778in}{0.000000in}}{\pgfqpoint{-0.000000in}{0.000000in}}{%
\pgfpathmoveto{\pgfqpoint{-0.000000in}{0.000000in}}%
\pgfpathlineto{\pgfqpoint{-0.027778in}{0.000000in}}%
\pgfusepath{stroke,fill}%
}%
\begin{pgfscope}%
\pgfsys@transformshift{0.708220in}{3.174738in}%
\pgfsys@useobject{currentmarker}{}%
\end{pgfscope}%
\end{pgfscope}%
\begin{pgfscope}%
\definecolor{textcolor}{rgb}{0.000000,0.000000,0.000000}%
\pgfsetstrokecolor{textcolor}%
\pgfsetfillcolor{textcolor}%
\pgftext[x=0.288855in,y=1.870549in,,bottom,rotate=90.000000]{\color{textcolor}\rmfamily\fontsize{10.000000}{12.000000}\selectfont Median solving time (s)}%
\end{pgfscope}%
\begin{pgfscope}%
\pgfpathrectangle{\pgfqpoint{0.708220in}{0.535823in}}{\pgfqpoint{5.141780in}{2.669453in}}%
\pgfusepath{clip}%
\pgfsetrectcap%
\pgfsetroundjoin%
\pgfsetlinewidth{1.003750pt}%
\definecolor{currentstroke}{rgb}{0.866667,0.058824,0.058824}%
\pgfsetstrokecolor{currentstroke}%
\pgfsetdash{}{0pt}%
\pgfpathmoveto{\pgfqpoint{1.928026in}{0.525823in}}%
\pgfpathlineto{\pgfqpoint{2.058359in}{0.761651in}}%
\pgfpathlineto{\pgfqpoint{2.283382in}{1.194612in}}%
\pgfpathlineto{\pgfqpoint{2.508405in}{1.631965in}}%
\pgfpathlineto{\pgfqpoint{2.733429in}{2.079842in}}%
\pgfpathlineto{\pgfqpoint{2.958452in}{2.526097in}}%
\pgfpathlineto{\pgfqpoint{3.183475in}{3.000213in}}%
\pgfusepath{stroke}%
\end{pgfscope}%
\begin{pgfscope}%
\pgfpathrectangle{\pgfqpoint{0.708220in}{0.535823in}}{\pgfqpoint{5.141780in}{2.669453in}}%
\pgfusepath{clip}%
\pgfsetbuttcap%
\pgfsetmiterjoin%
\definecolor{currentfill}{rgb}{0.866667,0.058824,0.058824}%
\pgfsetfillcolor{currentfill}%
\pgfsetlinewidth{0.501875pt}%
\definecolor{currentstroke}{rgb}{0.000000,0.000000,0.000000}%
\pgfsetstrokecolor{currentstroke}%
\pgfsetdash{}{0pt}%
\pgfsys@defobject{currentmarker}{\pgfqpoint{-0.033023in}{-0.028091in}}{\pgfqpoint{0.033023in}{0.034722in}}{%
\pgfpathmoveto{\pgfqpoint{0.000000in}{0.034722in}}%
\pgfpathlineto{\pgfqpoint{-0.033023in}{0.010730in}}%
\pgfpathlineto{\pgfqpoint{-0.020409in}{-0.028091in}}%
\pgfpathlineto{\pgfqpoint{0.020409in}{-0.028091in}}%
\pgfpathlineto{\pgfqpoint{0.033023in}{0.010730in}}%
\pgfpathclose%
\pgfusepath{stroke,fill}%
}%
\begin{pgfscope}%
\pgfsys@transformshift{1.833336in}{0.354487in}%
\pgfsys@useobject{currentmarker}{}%
\end{pgfscope}%
\begin{pgfscope}%
\pgfsys@transformshift{2.058359in}{0.761651in}%
\pgfsys@useobject{currentmarker}{}%
\end{pgfscope}%
\begin{pgfscope}%
\pgfsys@transformshift{2.283382in}{1.194612in}%
\pgfsys@useobject{currentmarker}{}%
\end{pgfscope}%
\begin{pgfscope}%
\pgfsys@transformshift{2.508405in}{1.631965in}%
\pgfsys@useobject{currentmarker}{}%
\end{pgfscope}%
\begin{pgfscope}%
\pgfsys@transformshift{2.733429in}{2.079842in}%
\pgfsys@useobject{currentmarker}{}%
\end{pgfscope}%
\begin{pgfscope}%
\pgfsys@transformshift{2.958452in}{2.526097in}%
\pgfsys@useobject{currentmarker}{}%
\end{pgfscope}%
\begin{pgfscope}%
\pgfsys@transformshift{3.183475in}{3.000213in}%
\pgfsys@useobject{currentmarker}{}%
\end{pgfscope}%
\end{pgfscope}%
\begin{pgfscope}%
\pgfpathrectangle{\pgfqpoint{0.708220in}{0.535823in}}{\pgfqpoint{5.141780in}{2.669453in}}%
\pgfusepath{clip}%
\pgfsetrectcap%
\pgfsetroundjoin%
\pgfsetlinewidth{1.003750pt}%
\definecolor{currentstroke}{rgb}{0.000000,0.000000,0.200000}%
\pgfsetstrokecolor{currentstroke}%
\pgfsetdash{}{0pt}%
\pgfpathmoveto{\pgfqpoint{1.833336in}{0.676718in}}%
\pgfpathlineto{\pgfqpoint{2.058359in}{0.906415in}}%
\pgfpathlineto{\pgfqpoint{2.283382in}{1.295336in}}%
\pgfpathlineto{\pgfqpoint{2.508405in}{1.701898in}}%
\pgfpathlineto{\pgfqpoint{2.733429in}{1.970334in}}%
\pgfpathlineto{\pgfqpoint{2.958452in}{2.394586in}}%
\pgfpathlineto{\pgfqpoint{3.183475in}{2.793972in}}%
\pgfusepath{stroke}%
\end{pgfscope}%
\begin{pgfscope}%
\pgfpathrectangle{\pgfqpoint{0.708220in}{0.535823in}}{\pgfqpoint{5.141780in}{2.669453in}}%
\pgfusepath{clip}%
\pgfsetbuttcap%
\pgfsetmiterjoin%
\definecolor{currentfill}{rgb}{0.000000,0.000000,0.200000}%
\pgfsetfillcolor{currentfill}%
\pgfsetlinewidth{0.501875pt}%
\definecolor{currentstroke}{rgb}{0.000000,0.000000,0.000000}%
\pgfsetstrokecolor{currentstroke}%
\pgfsetdash{}{0pt}%
\pgfsys@defobject{currentmarker}{\pgfqpoint{-0.034722in}{-0.034722in}}{\pgfqpoint{0.034722in}{0.034722in}}{%
\pgfpathmoveto{\pgfqpoint{-0.011574in}{-0.034722in}}%
\pgfpathlineto{\pgfqpoint{0.011574in}{-0.034722in}}%
\pgfpathlineto{\pgfqpoint{0.011574in}{-0.011574in}}%
\pgfpathlineto{\pgfqpoint{0.034722in}{-0.011574in}}%
\pgfpathlineto{\pgfqpoint{0.034722in}{0.011574in}}%
\pgfpathlineto{\pgfqpoint{0.011574in}{0.011574in}}%
\pgfpathlineto{\pgfqpoint{0.011574in}{0.034722in}}%
\pgfpathlineto{\pgfqpoint{-0.011574in}{0.034722in}}%
\pgfpathlineto{\pgfqpoint{-0.011574in}{0.011574in}}%
\pgfpathlineto{\pgfqpoint{-0.034722in}{0.011574in}}%
\pgfpathlineto{\pgfqpoint{-0.034722in}{-0.011574in}}%
\pgfpathlineto{\pgfqpoint{-0.011574in}{-0.011574in}}%
\pgfpathclose%
\pgfusepath{stroke,fill}%
}%
\begin{pgfscope}%
\pgfsys@transformshift{1.833336in}{0.676718in}%
\pgfsys@useobject{currentmarker}{}%
\end{pgfscope}%
\begin{pgfscope}%
\pgfsys@transformshift{2.058359in}{0.906415in}%
\pgfsys@useobject{currentmarker}{}%
\end{pgfscope}%
\begin{pgfscope}%
\pgfsys@transformshift{2.283382in}{1.295336in}%
\pgfsys@useobject{currentmarker}{}%
\end{pgfscope}%
\begin{pgfscope}%
\pgfsys@transformshift{2.508405in}{1.701898in}%
\pgfsys@useobject{currentmarker}{}%
\end{pgfscope}%
\begin{pgfscope}%
\pgfsys@transformshift{2.733429in}{1.970334in}%
\pgfsys@useobject{currentmarker}{}%
\end{pgfscope}%
\begin{pgfscope}%
\pgfsys@transformshift{2.958452in}{2.394586in}%
\pgfsys@useobject{currentmarker}{}%
\end{pgfscope}%
\begin{pgfscope}%
\pgfsys@transformshift{3.183475in}{2.793972in}%
\pgfsys@useobject{currentmarker}{}%
\end{pgfscope}%
\end{pgfscope}%
\begin{pgfscope}%
\pgfpathrectangle{\pgfqpoint{0.708220in}{0.535823in}}{\pgfqpoint{5.141780in}{2.669453in}}%
\pgfusepath{clip}%
\pgfsetrectcap%
\pgfsetroundjoin%
\pgfsetlinewidth{1.003750pt}%
\definecolor{currentstroke}{rgb}{0.000000,0.000000,0.866667}%
\pgfsetstrokecolor{currentstroke}%
\pgfsetdash{}{0pt}%
\pgfpathmoveto{\pgfqpoint{1.833336in}{0.610115in}}%
\pgfpathlineto{\pgfqpoint{2.058359in}{0.767859in}}%
\pgfpathlineto{\pgfqpoint{2.283382in}{1.115023in}}%
\pgfpathlineto{\pgfqpoint{2.508405in}{1.546157in}}%
\pgfpathlineto{\pgfqpoint{2.733429in}{1.940281in}}%
\pgfpathlineto{\pgfqpoint{2.958452in}{2.294916in}}%
\pgfpathlineto{\pgfqpoint{3.183475in}{2.632491in}}%
\pgfpathlineto{\pgfqpoint{3.408498in}{3.043909in}}%
\pgfusepath{stroke}%
\end{pgfscope}%
\begin{pgfscope}%
\pgfpathrectangle{\pgfqpoint{0.708220in}{0.535823in}}{\pgfqpoint{5.141780in}{2.669453in}}%
\pgfusepath{clip}%
\pgfsetbuttcap%
\pgfsetmiterjoin%
\definecolor{currentfill}{rgb}{0.000000,0.000000,0.866667}%
\pgfsetfillcolor{currentfill}%
\pgfsetlinewidth{0.501875pt}%
\definecolor{currentstroke}{rgb}{0.000000,0.000000,0.000000}%
\pgfsetstrokecolor{currentstroke}%
\pgfsetdash{}{0pt}%
\pgfsys@defobject{currentmarker}{\pgfqpoint{-0.029463in}{-0.049105in}}{\pgfqpoint{0.029463in}{0.049105in}}{%
\pgfpathmoveto{\pgfqpoint{0.000000in}{-0.049105in}}%
\pgfpathlineto{\pgfqpoint{0.029463in}{0.000000in}}%
\pgfpathlineto{\pgfqpoint{0.000000in}{0.049105in}}%
\pgfpathlineto{\pgfqpoint{-0.029463in}{0.000000in}}%
\pgfpathclose%
\pgfusepath{stroke,fill}%
}%
\begin{pgfscope}%
\pgfsys@transformshift{1.833336in}{0.610115in}%
\pgfsys@useobject{currentmarker}{}%
\end{pgfscope}%
\begin{pgfscope}%
\pgfsys@transformshift{2.058359in}{0.767859in}%
\pgfsys@useobject{currentmarker}{}%
\end{pgfscope}%
\begin{pgfscope}%
\pgfsys@transformshift{2.283382in}{1.115023in}%
\pgfsys@useobject{currentmarker}{}%
\end{pgfscope}%
\begin{pgfscope}%
\pgfsys@transformshift{2.508405in}{1.546157in}%
\pgfsys@useobject{currentmarker}{}%
\end{pgfscope}%
\begin{pgfscope}%
\pgfsys@transformshift{2.733429in}{1.940281in}%
\pgfsys@useobject{currentmarker}{}%
\end{pgfscope}%
\begin{pgfscope}%
\pgfsys@transformshift{2.958452in}{2.294916in}%
\pgfsys@useobject{currentmarker}{}%
\end{pgfscope}%
\begin{pgfscope}%
\pgfsys@transformshift{3.183475in}{2.632491in}%
\pgfsys@useobject{currentmarker}{}%
\end{pgfscope}%
\begin{pgfscope}%
\pgfsys@transformshift{3.408498in}{3.043909in}%
\pgfsys@useobject{currentmarker}{}%
\end{pgfscope}%
\end{pgfscope}%
\begin{pgfscope}%
\pgfpathrectangle{\pgfqpoint{0.708220in}{0.535823in}}{\pgfqpoint{5.141780in}{2.669453in}}%
\pgfusepath{clip}%
\pgfsetrectcap%
\pgfsetroundjoin%
\pgfsetlinewidth{1.003750pt}%
\definecolor{currentstroke}{rgb}{0.250980,0.231373,0.796078}%
\pgfsetstrokecolor{currentstroke}%
\pgfsetdash{}{0pt}%
\pgfpathmoveto{\pgfqpoint{2.095649in}{0.525823in}}%
\pgfpathlineto{\pgfqpoint{2.283382in}{0.846558in}}%
\pgfpathlineto{\pgfqpoint{2.508405in}{1.257717in}}%
\pgfpathlineto{\pgfqpoint{2.733429in}{1.657480in}}%
\pgfpathlineto{\pgfqpoint{2.958452in}{2.073652in}}%
\pgfpathlineto{\pgfqpoint{3.183475in}{2.469956in}}%
\pgfpathlineto{\pgfqpoint{3.408498in}{2.885012in}}%
\pgfusepath{stroke}%
\end{pgfscope}%
\begin{pgfscope}%
\pgfpathrectangle{\pgfqpoint{0.708220in}{0.535823in}}{\pgfqpoint{5.141780in}{2.669453in}}%
\pgfusepath{clip}%
\pgfsetbuttcap%
\pgfsetmiterjoin%
\definecolor{currentfill}{rgb}{0.250980,0.231373,0.796078}%
\pgfsetfillcolor{currentfill}%
\pgfsetlinewidth{0.501875pt}%
\definecolor{currentstroke}{rgb}{0.000000,0.000000,0.000000}%
\pgfsetstrokecolor{currentstroke}%
\pgfsetdash{}{0pt}%
\pgfsys@defobject{currentmarker}{\pgfqpoint{-0.034722in}{-0.034722in}}{\pgfqpoint{0.034722in}{0.034722in}}{%
\pgfpathmoveto{\pgfqpoint{-0.034722in}{-0.034722in}}%
\pgfpathlineto{\pgfqpoint{0.034722in}{-0.034722in}}%
\pgfpathlineto{\pgfqpoint{0.034722in}{0.034722in}}%
\pgfpathlineto{\pgfqpoint{-0.034722in}{0.034722in}}%
\pgfpathclose%
\pgfusepath{stroke,fill}%
}%
\begin{pgfscope}%
\pgfsys@transformshift{1.833336in}{0.132041in}%
\pgfsys@useobject{currentmarker}{}%
\end{pgfscope}%
\begin{pgfscope}%
\pgfsys@transformshift{2.058359in}{0.462115in}%
\pgfsys@useobject{currentmarker}{}%
\end{pgfscope}%
\begin{pgfscope}%
\pgfsys@transformshift{2.283382in}{0.846558in}%
\pgfsys@useobject{currentmarker}{}%
\end{pgfscope}%
\begin{pgfscope}%
\pgfsys@transformshift{2.508405in}{1.257717in}%
\pgfsys@useobject{currentmarker}{}%
\end{pgfscope}%
\begin{pgfscope}%
\pgfsys@transformshift{2.733429in}{1.657480in}%
\pgfsys@useobject{currentmarker}{}%
\end{pgfscope}%
\begin{pgfscope}%
\pgfsys@transformshift{2.958452in}{2.073652in}%
\pgfsys@useobject{currentmarker}{}%
\end{pgfscope}%
\begin{pgfscope}%
\pgfsys@transformshift{3.183475in}{2.469956in}%
\pgfsys@useobject{currentmarker}{}%
\end{pgfscope}%
\begin{pgfscope}%
\pgfsys@transformshift{3.408498in}{2.885012in}%
\pgfsys@useobject{currentmarker}{}%
\end{pgfscope}%
\end{pgfscope}%
\begin{pgfscope}%
\pgfpathrectangle{\pgfqpoint{0.708220in}{0.535823in}}{\pgfqpoint{5.141780in}{2.669453in}}%
\pgfusepath{clip}%
\pgfsetrectcap%
\pgfsetroundjoin%
\pgfsetlinewidth{1.003750pt}%
\definecolor{currentstroke}{rgb}{0.615686,0.007843,0.843137}%
\pgfsetstrokecolor{currentstroke}%
\pgfsetdash{}{0pt}%
\pgfpathmoveto{\pgfqpoint{2.027184in}{0.525823in}}%
\pgfpathlineto{\pgfqpoint{2.058359in}{0.571199in}}%
\pgfpathlineto{\pgfqpoint{2.283382in}{0.866486in}}%
\pgfpathlineto{\pgfqpoint{2.508405in}{1.219071in}}%
\pgfpathlineto{\pgfqpoint{2.733429in}{1.535650in}}%
\pgfpathlineto{\pgfqpoint{2.958452in}{1.852695in}}%
\pgfpathlineto{\pgfqpoint{3.183475in}{2.103243in}}%
\pgfpathlineto{\pgfqpoint{3.408498in}{2.465875in}}%
\pgfpathlineto{\pgfqpoint{3.633521in}{2.755438in}}%
\pgfpathlineto{\pgfqpoint{3.858545in}{3.018596in}}%
\pgfusepath{stroke}%
\end{pgfscope}%
\begin{pgfscope}%
\pgfpathrectangle{\pgfqpoint{0.708220in}{0.535823in}}{\pgfqpoint{5.141780in}{2.669453in}}%
\pgfusepath{clip}%
\pgfsetbuttcap%
\pgfsetroundjoin%
\definecolor{currentfill}{rgb}{0.615686,0.007843,0.843137}%
\pgfsetfillcolor{currentfill}%
\pgfsetlinewidth{0.501875pt}%
\definecolor{currentstroke}{rgb}{0.000000,0.000000,0.000000}%
\pgfsetstrokecolor{currentstroke}%
\pgfsetdash{}{0pt}%
\pgfsys@defobject{currentmarker}{\pgfqpoint{-0.034722in}{-0.034722in}}{\pgfqpoint{0.034722in}{0.034722in}}{%
\pgfpathmoveto{\pgfqpoint{0.000000in}{-0.034722in}}%
\pgfpathcurveto{\pgfqpoint{0.009208in}{-0.034722in}}{\pgfqpoint{0.018041in}{-0.031064in}}{\pgfqpoint{0.024552in}{-0.024552in}}%
\pgfpathcurveto{\pgfqpoint{0.031064in}{-0.018041in}}{\pgfqpoint{0.034722in}{-0.009208in}}{\pgfqpoint{0.034722in}{0.000000in}}%
\pgfpathcurveto{\pgfqpoint{0.034722in}{0.009208in}}{\pgfqpoint{0.031064in}{0.018041in}}{\pgfqpoint{0.024552in}{0.024552in}}%
\pgfpathcurveto{\pgfqpoint{0.018041in}{0.031064in}}{\pgfqpoint{0.009208in}{0.034722in}}{\pgfqpoint{0.000000in}{0.034722in}}%
\pgfpathcurveto{\pgfqpoint{-0.009208in}{0.034722in}}{\pgfqpoint{-0.018041in}{0.031064in}}{\pgfqpoint{-0.024552in}{0.024552in}}%
\pgfpathcurveto{\pgfqpoint{-0.031064in}{0.018041in}}{\pgfqpoint{-0.034722in}{0.009208in}}{\pgfqpoint{-0.034722in}{0.000000in}}%
\pgfpathcurveto{\pgfqpoint{-0.034722in}{-0.009208in}}{\pgfqpoint{-0.031064in}{-0.018041in}}{\pgfqpoint{-0.024552in}{-0.024552in}}%
\pgfpathcurveto{\pgfqpoint{-0.018041in}{-0.031064in}}{\pgfqpoint{-0.009208in}{-0.034722in}}{\pgfqpoint{0.000000in}{-0.034722in}}%
\pgfpathclose%
\pgfusepath{stroke,fill}%
}%
\begin{pgfscope}%
\pgfsys@transformshift{1.833336in}{0.243665in}%
\pgfsys@useobject{currentmarker}{}%
\end{pgfscope}%
\begin{pgfscope}%
\pgfsys@transformshift{2.058359in}{0.571199in}%
\pgfsys@useobject{currentmarker}{}%
\end{pgfscope}%
\begin{pgfscope}%
\pgfsys@transformshift{2.283382in}{0.866486in}%
\pgfsys@useobject{currentmarker}{}%
\end{pgfscope}%
\begin{pgfscope}%
\pgfsys@transformshift{2.508405in}{1.219071in}%
\pgfsys@useobject{currentmarker}{}%
\end{pgfscope}%
\begin{pgfscope}%
\pgfsys@transformshift{2.733429in}{1.535650in}%
\pgfsys@useobject{currentmarker}{}%
\end{pgfscope}%
\begin{pgfscope}%
\pgfsys@transformshift{2.958452in}{1.852695in}%
\pgfsys@useobject{currentmarker}{}%
\end{pgfscope}%
\begin{pgfscope}%
\pgfsys@transformshift{3.183475in}{2.103243in}%
\pgfsys@useobject{currentmarker}{}%
\end{pgfscope}%
\begin{pgfscope}%
\pgfsys@transformshift{3.408498in}{2.465875in}%
\pgfsys@useobject{currentmarker}{}%
\end{pgfscope}%
\begin{pgfscope}%
\pgfsys@transformshift{3.633521in}{2.755438in}%
\pgfsys@useobject{currentmarker}{}%
\end{pgfscope}%
\begin{pgfscope}%
\pgfsys@transformshift{3.858545in}{3.018596in}%
\pgfsys@useobject{currentmarker}{}%
\end{pgfscope}%
\end{pgfscope}%
\begin{pgfscope}%
\pgfpathrectangle{\pgfqpoint{0.708220in}{0.535823in}}{\pgfqpoint{5.141780in}{2.669453in}}%
\pgfusepath{clip}%
\pgfsetrectcap%
\pgfsetroundjoin%
\pgfsetlinewidth{1.003750pt}%
\definecolor{currentstroke}{rgb}{0.917647,0.372549,0.580392}%
\pgfsetstrokecolor{currentstroke}%
\pgfsetdash{}{0pt}%
\pgfpathmoveto{\pgfqpoint{2.685237in}{0.525823in}}%
\pgfpathlineto{\pgfqpoint{2.733429in}{0.563447in}}%
\pgfpathlineto{\pgfqpoint{2.958452in}{0.736719in}}%
\pgfpathlineto{\pgfqpoint{3.183475in}{0.978123in}}%
\pgfpathlineto{\pgfqpoint{3.408498in}{1.305836in}}%
\pgfpathlineto{\pgfqpoint{3.633521in}{1.559575in}}%
\pgfpathlineto{\pgfqpoint{3.858545in}{1.866610in}}%
\pgfpathlineto{\pgfqpoint{4.083568in}{2.126813in}}%
\pgfpathlineto{\pgfqpoint{4.308591in}{2.390895in}}%
\pgfpathlineto{\pgfqpoint{4.533614in}{2.643174in}}%
\pgfpathlineto{\pgfqpoint{4.758637in}{3.041851in}}%
\pgfusepath{stroke}%
\end{pgfscope}%
\begin{pgfscope}%
\pgfpathrectangle{\pgfqpoint{0.708220in}{0.535823in}}{\pgfqpoint{5.141780in}{2.669453in}}%
\pgfusepath{clip}%
\pgfsetbuttcap%
\pgfsetmiterjoin%
\definecolor{currentfill}{rgb}{0.917647,0.372549,0.580392}%
\pgfsetfillcolor{currentfill}%
\pgfsetlinewidth{0.501875pt}%
\definecolor{currentstroke}{rgb}{0.000000,0.000000,0.000000}%
\pgfsetstrokecolor{currentstroke}%
\pgfsetdash{}{0pt}%
\pgfsys@defobject{currentmarker}{\pgfqpoint{-0.049105in}{-0.049105in}}{\pgfqpoint{0.049105in}{0.049105in}}{%
\pgfpathmoveto{\pgfqpoint{0.000000in}{-0.049105in}}%
\pgfpathlineto{\pgfqpoint{0.049105in}{0.000000in}}%
\pgfpathlineto{\pgfqpoint{0.000000in}{0.049105in}}%
\pgfpathlineto{\pgfqpoint{-0.049105in}{0.000000in}}%
\pgfpathclose%
\pgfusepath{stroke,fill}%
}%
\begin{pgfscope}%
\pgfsys@transformshift{1.833336in}{0.270252in}%
\pgfsys@useobject{currentmarker}{}%
\end{pgfscope}%
\begin{pgfscope}%
\pgfsys@transformshift{2.058359in}{0.270252in}%
\pgfsys@useobject{currentmarker}{}%
\end{pgfscope}%
\begin{pgfscope}%
\pgfsys@transformshift{2.283382in}{0.270252in}%
\pgfsys@useobject{currentmarker}{}%
\end{pgfscope}%
\begin{pgfscope}%
\pgfsys@transformshift{2.508405in}{0.387769in}%
\pgfsys@useobject{currentmarker}{}%
\end{pgfscope}%
\begin{pgfscope}%
\pgfsys@transformshift{2.733429in}{0.563447in}%
\pgfsys@useobject{currentmarker}{}%
\end{pgfscope}%
\begin{pgfscope}%
\pgfsys@transformshift{2.958452in}{0.736719in}%
\pgfsys@useobject{currentmarker}{}%
\end{pgfscope}%
\begin{pgfscope}%
\pgfsys@transformshift{3.183475in}{0.978123in}%
\pgfsys@useobject{currentmarker}{}%
\end{pgfscope}%
\begin{pgfscope}%
\pgfsys@transformshift{3.408498in}{1.305836in}%
\pgfsys@useobject{currentmarker}{}%
\end{pgfscope}%
\begin{pgfscope}%
\pgfsys@transformshift{3.633521in}{1.559575in}%
\pgfsys@useobject{currentmarker}{}%
\end{pgfscope}%
\begin{pgfscope}%
\pgfsys@transformshift{3.858545in}{1.866610in}%
\pgfsys@useobject{currentmarker}{}%
\end{pgfscope}%
\begin{pgfscope}%
\pgfsys@transformshift{4.083568in}{2.126813in}%
\pgfsys@useobject{currentmarker}{}%
\end{pgfscope}%
\begin{pgfscope}%
\pgfsys@transformshift{4.308591in}{2.390895in}%
\pgfsys@useobject{currentmarker}{}%
\end{pgfscope}%
\begin{pgfscope}%
\pgfsys@transformshift{4.533614in}{2.643174in}%
\pgfsys@useobject{currentmarker}{}%
\end{pgfscope}%
\begin{pgfscope}%
\pgfsys@transformshift{4.758637in}{3.041851in}%
\pgfsys@useobject{currentmarker}{}%
\end{pgfscope}%
\end{pgfscope}%
\begin{pgfscope}%
\pgfpathrectangle{\pgfqpoint{0.708220in}{0.535823in}}{\pgfqpoint{5.141780in}{2.669453in}}%
\pgfusepath{clip}%
\pgfsetrectcap%
\pgfsetroundjoin%
\pgfsetlinewidth{1.003750pt}%
\definecolor{currentstroke}{rgb}{0.529412,0.462745,0.384314}%
\pgfsetstrokecolor{currentstroke}%
\pgfsetdash{}{0pt}%
\pgfpathmoveto{\pgfqpoint{1.920185in}{0.525823in}}%
\pgfpathlineto{\pgfqpoint{2.058359in}{0.573473in}}%
\pgfpathlineto{\pgfqpoint{2.283382in}{0.647919in}}%
\pgfpathlineto{\pgfqpoint{2.508405in}{0.725868in}}%
\pgfpathlineto{\pgfqpoint{2.733429in}{0.798513in}}%
\pgfpathlineto{\pgfqpoint{2.958452in}{0.899271in}}%
\pgfpathlineto{\pgfqpoint{3.183475in}{1.039475in}}%
\pgfpathlineto{\pgfqpoint{3.408498in}{1.399329in}}%
\pgfpathlineto{\pgfqpoint{3.633521in}{1.819793in}}%
\pgfpathlineto{\pgfqpoint{3.858545in}{2.283030in}}%
\pgfpathlineto{\pgfqpoint{4.083568in}{2.802781in}}%
\pgfusepath{stroke}%
\end{pgfscope}%
\begin{pgfscope}%
\pgfpathrectangle{\pgfqpoint{0.708220in}{0.535823in}}{\pgfqpoint{5.141780in}{2.669453in}}%
\pgfusepath{clip}%
\pgfsetbuttcap%
\pgfsetmiterjoin%
\definecolor{currentfill}{rgb}{0.529412,0.462745,0.384314}%
\pgfsetfillcolor{currentfill}%
\pgfsetlinewidth{0.501875pt}%
\definecolor{currentstroke}{rgb}{0.000000,0.000000,0.000000}%
\pgfsetstrokecolor{currentstroke}%
\pgfsetdash{}{0pt}%
\pgfsys@defobject{currentmarker}{\pgfqpoint{-0.034722in}{-0.034722in}}{\pgfqpoint{0.034722in}{0.034722in}}{%
\pgfpathmoveto{\pgfqpoint{-0.000000in}{-0.034722in}}%
\pgfpathlineto{\pgfqpoint{0.034722in}{0.034722in}}%
\pgfpathlineto{\pgfqpoint{-0.034722in}{0.034722in}}%
\pgfpathclose%
\pgfusepath{stroke,fill}%
}%
\begin{pgfscope}%
\pgfsys@transformshift{1.833336in}{0.495872in}%
\pgfsys@useobject{currentmarker}{}%
\end{pgfscope}%
\begin{pgfscope}%
\pgfsys@transformshift{2.058359in}{0.573473in}%
\pgfsys@useobject{currentmarker}{}%
\end{pgfscope}%
\begin{pgfscope}%
\pgfsys@transformshift{2.283382in}{0.647919in}%
\pgfsys@useobject{currentmarker}{}%
\end{pgfscope}%
\begin{pgfscope}%
\pgfsys@transformshift{2.508405in}{0.725868in}%
\pgfsys@useobject{currentmarker}{}%
\end{pgfscope}%
\begin{pgfscope}%
\pgfsys@transformshift{2.733429in}{0.798513in}%
\pgfsys@useobject{currentmarker}{}%
\end{pgfscope}%
\begin{pgfscope}%
\pgfsys@transformshift{2.958452in}{0.899271in}%
\pgfsys@useobject{currentmarker}{}%
\end{pgfscope}%
\begin{pgfscope}%
\pgfsys@transformshift{3.183475in}{1.039475in}%
\pgfsys@useobject{currentmarker}{}%
\end{pgfscope}%
\begin{pgfscope}%
\pgfsys@transformshift{3.408498in}{1.399329in}%
\pgfsys@useobject{currentmarker}{}%
\end{pgfscope}%
\begin{pgfscope}%
\pgfsys@transformshift{3.633521in}{1.819793in}%
\pgfsys@useobject{currentmarker}{}%
\end{pgfscope}%
\begin{pgfscope}%
\pgfsys@transformshift{3.858545in}{2.283030in}%
\pgfsys@useobject{currentmarker}{}%
\end{pgfscope}%
\begin{pgfscope}%
\pgfsys@transformshift{4.083568in}{2.802781in}%
\pgfsys@useobject{currentmarker}{}%
\end{pgfscope}%
\end{pgfscope}%
\begin{pgfscope}%
\pgfpathrectangle{\pgfqpoint{0.708220in}{0.535823in}}{\pgfqpoint{5.141780in}{2.669453in}}%
\pgfusepath{clip}%
\pgfsetrectcap%
\pgfsetroundjoin%
\pgfsetlinewidth{1.003750pt}%
\definecolor{currentstroke}{rgb}{0.611765,0.568627,0.274510}%
\pgfsetstrokecolor{currentstroke}%
\pgfsetdash{}{0pt}%
\pgfpathmoveto{\pgfqpoint{1.833336in}{0.533110in}}%
\pgfpathlineto{\pgfqpoint{2.058359in}{0.600165in}}%
\pgfpathlineto{\pgfqpoint{2.283382in}{0.656378in}}%
\pgfpathlineto{\pgfqpoint{2.508405in}{0.708555in}}%
\pgfpathlineto{\pgfqpoint{2.733429in}{0.755478in}}%
\pgfpathlineto{\pgfqpoint{2.958452in}{0.800865in}}%
\pgfpathlineto{\pgfqpoint{3.183475in}{0.845169in}}%
\pgfpathlineto{\pgfqpoint{3.408498in}{0.901495in}}%
\pgfpathlineto{\pgfqpoint{3.633521in}{0.983805in}}%
\pgfpathlineto{\pgfqpoint{3.858545in}{1.134783in}}%
\pgfpathlineto{\pgfqpoint{4.083568in}{1.319608in}}%
\pgfpathlineto{\pgfqpoint{4.308591in}{1.587480in}}%
\pgfpathlineto{\pgfqpoint{4.533614in}{1.922967in}}%
\pgfpathlineto{\pgfqpoint{4.758637in}{2.377229in}}%
\pgfpathlineto{\pgfqpoint{4.983661in}{2.606679in}}%
\pgfpathlineto{\pgfqpoint{5.208684in}{3.081033in}}%
\pgfusepath{stroke}%
\end{pgfscope}%
\begin{pgfscope}%
\pgfpathrectangle{\pgfqpoint{0.708220in}{0.535823in}}{\pgfqpoint{5.141780in}{2.669453in}}%
\pgfusepath{clip}%
\pgfsetbuttcap%
\pgfsetmiterjoin%
\definecolor{currentfill}{rgb}{0.611765,0.568627,0.274510}%
\pgfsetfillcolor{currentfill}%
\pgfsetlinewidth{0.501875pt}%
\definecolor{currentstroke}{rgb}{0.000000,0.000000,0.000000}%
\pgfsetstrokecolor{currentstroke}%
\pgfsetdash{}{0pt}%
\pgfsys@defobject{currentmarker}{\pgfqpoint{-0.034722in}{-0.034722in}}{\pgfqpoint{0.034722in}{0.034722in}}{%
\pgfpathmoveto{\pgfqpoint{-0.034722in}{0.000000in}}%
\pgfpathlineto{\pgfqpoint{0.034722in}{-0.034722in}}%
\pgfpathlineto{\pgfqpoint{0.034722in}{0.034722in}}%
\pgfpathclose%
\pgfusepath{stroke,fill}%
}%
\begin{pgfscope}%
\pgfsys@transformshift{1.833336in}{0.533110in}%
\pgfsys@useobject{currentmarker}{}%
\end{pgfscope}%
\begin{pgfscope}%
\pgfsys@transformshift{2.058359in}{0.600165in}%
\pgfsys@useobject{currentmarker}{}%
\end{pgfscope}%
\begin{pgfscope}%
\pgfsys@transformshift{2.283382in}{0.656378in}%
\pgfsys@useobject{currentmarker}{}%
\end{pgfscope}%
\begin{pgfscope}%
\pgfsys@transformshift{2.508405in}{0.708555in}%
\pgfsys@useobject{currentmarker}{}%
\end{pgfscope}%
\begin{pgfscope}%
\pgfsys@transformshift{2.733429in}{0.755478in}%
\pgfsys@useobject{currentmarker}{}%
\end{pgfscope}%
\begin{pgfscope}%
\pgfsys@transformshift{2.958452in}{0.800865in}%
\pgfsys@useobject{currentmarker}{}%
\end{pgfscope}%
\begin{pgfscope}%
\pgfsys@transformshift{3.183475in}{0.845169in}%
\pgfsys@useobject{currentmarker}{}%
\end{pgfscope}%
\begin{pgfscope}%
\pgfsys@transformshift{3.408498in}{0.901495in}%
\pgfsys@useobject{currentmarker}{}%
\end{pgfscope}%
\begin{pgfscope}%
\pgfsys@transformshift{3.633521in}{0.983805in}%
\pgfsys@useobject{currentmarker}{}%
\end{pgfscope}%
\begin{pgfscope}%
\pgfsys@transformshift{3.858545in}{1.134783in}%
\pgfsys@useobject{currentmarker}{}%
\end{pgfscope}%
\begin{pgfscope}%
\pgfsys@transformshift{4.083568in}{1.319608in}%
\pgfsys@useobject{currentmarker}{}%
\end{pgfscope}%
\begin{pgfscope}%
\pgfsys@transformshift{4.308591in}{1.587480in}%
\pgfsys@useobject{currentmarker}{}%
\end{pgfscope}%
\begin{pgfscope}%
\pgfsys@transformshift{4.533614in}{1.922967in}%
\pgfsys@useobject{currentmarker}{}%
\end{pgfscope}%
\begin{pgfscope}%
\pgfsys@transformshift{4.758637in}{2.377229in}%
\pgfsys@useobject{currentmarker}{}%
\end{pgfscope}%
\begin{pgfscope}%
\pgfsys@transformshift{4.983661in}{2.606679in}%
\pgfsys@useobject{currentmarker}{}%
\end{pgfscope}%
\begin{pgfscope}%
\pgfsys@transformshift{5.208684in}{3.081033in}%
\pgfsys@useobject{currentmarker}{}%
\end{pgfscope}%
\end{pgfscope}%
\begin{pgfscope}%
\pgfpathrectangle{\pgfqpoint{0.708220in}{0.535823in}}{\pgfqpoint{5.141780in}{2.669453in}}%
\pgfusepath{clip}%
\pgfsetrectcap%
\pgfsetroundjoin%
\pgfsetlinewidth{1.003750pt}%
\definecolor{currentstroke}{rgb}{0.780392,0.643137,0.254902}%
\pgfsetstrokecolor{currentstroke}%
\pgfsetdash{}{0pt}%
\pgfpathmoveto{\pgfqpoint{1.833336in}{0.607768in}}%
\pgfpathlineto{\pgfqpoint{2.058359in}{0.714594in}}%
\pgfpathlineto{\pgfqpoint{2.283382in}{0.810268in}}%
\pgfpathlineto{\pgfqpoint{2.508405in}{0.899028in}}%
\pgfpathlineto{\pgfqpoint{2.733429in}{0.976962in}}%
\pgfpathlineto{\pgfqpoint{2.958452in}{1.048705in}}%
\pgfpathlineto{\pgfqpoint{3.183475in}{1.117476in}}%
\pgfpathlineto{\pgfqpoint{3.408498in}{1.184622in}}%
\pgfpathlineto{\pgfqpoint{3.633521in}{1.247489in}}%
\pgfpathlineto{\pgfqpoint{3.858545in}{1.333602in}}%
\pgfpathlineto{\pgfqpoint{4.083568in}{1.407839in}}%
\pgfpathlineto{\pgfqpoint{4.308591in}{1.554250in}}%
\pgfpathlineto{\pgfqpoint{4.533614in}{1.754038in}}%
\pgfpathlineto{\pgfqpoint{4.758637in}{2.105422in}}%
\pgfpathlineto{\pgfqpoint{4.983661in}{2.186292in}}%
\pgfpathlineto{\pgfqpoint{5.208684in}{2.855359in}}%
\pgfusepath{stroke}%
\end{pgfscope}%
\begin{pgfscope}%
\pgfpathrectangle{\pgfqpoint{0.708220in}{0.535823in}}{\pgfqpoint{5.141780in}{2.669453in}}%
\pgfusepath{clip}%
\pgfsetbuttcap%
\pgfsetmiterjoin%
\definecolor{currentfill}{rgb}{0.780392,0.643137,0.254902}%
\pgfsetfillcolor{currentfill}%
\pgfsetlinewidth{0.501875pt}%
\definecolor{currentstroke}{rgb}{0.000000,0.000000,0.000000}%
\pgfsetstrokecolor{currentstroke}%
\pgfsetdash{}{0pt}%
\pgfsys@defobject{currentmarker}{\pgfqpoint{-0.034722in}{-0.034722in}}{\pgfqpoint{0.034722in}{0.034722in}}{%
\pgfpathmoveto{\pgfqpoint{0.034722in}{-0.000000in}}%
\pgfpathlineto{\pgfqpoint{-0.034722in}{0.034722in}}%
\pgfpathlineto{\pgfqpoint{-0.034722in}{-0.034722in}}%
\pgfpathclose%
\pgfusepath{stroke,fill}%
}%
\begin{pgfscope}%
\pgfsys@transformshift{1.833336in}{0.607768in}%
\pgfsys@useobject{currentmarker}{}%
\end{pgfscope}%
\begin{pgfscope}%
\pgfsys@transformshift{2.058359in}{0.714594in}%
\pgfsys@useobject{currentmarker}{}%
\end{pgfscope}%
\begin{pgfscope}%
\pgfsys@transformshift{2.283382in}{0.810268in}%
\pgfsys@useobject{currentmarker}{}%
\end{pgfscope}%
\begin{pgfscope}%
\pgfsys@transformshift{2.508405in}{0.899028in}%
\pgfsys@useobject{currentmarker}{}%
\end{pgfscope}%
\begin{pgfscope}%
\pgfsys@transformshift{2.733429in}{0.976962in}%
\pgfsys@useobject{currentmarker}{}%
\end{pgfscope}%
\begin{pgfscope}%
\pgfsys@transformshift{2.958452in}{1.048705in}%
\pgfsys@useobject{currentmarker}{}%
\end{pgfscope}%
\begin{pgfscope}%
\pgfsys@transformshift{3.183475in}{1.117476in}%
\pgfsys@useobject{currentmarker}{}%
\end{pgfscope}%
\begin{pgfscope}%
\pgfsys@transformshift{3.408498in}{1.184622in}%
\pgfsys@useobject{currentmarker}{}%
\end{pgfscope}%
\begin{pgfscope}%
\pgfsys@transformshift{3.633521in}{1.247489in}%
\pgfsys@useobject{currentmarker}{}%
\end{pgfscope}%
\begin{pgfscope}%
\pgfsys@transformshift{3.858545in}{1.333602in}%
\pgfsys@useobject{currentmarker}{}%
\end{pgfscope}%
\begin{pgfscope}%
\pgfsys@transformshift{4.083568in}{1.407839in}%
\pgfsys@useobject{currentmarker}{}%
\end{pgfscope}%
\begin{pgfscope}%
\pgfsys@transformshift{4.308591in}{1.554250in}%
\pgfsys@useobject{currentmarker}{}%
\end{pgfscope}%
\begin{pgfscope}%
\pgfsys@transformshift{4.533614in}{1.754038in}%
\pgfsys@useobject{currentmarker}{}%
\end{pgfscope}%
\begin{pgfscope}%
\pgfsys@transformshift{4.758637in}{2.105422in}%
\pgfsys@useobject{currentmarker}{}%
\end{pgfscope}%
\begin{pgfscope}%
\pgfsys@transformshift{4.983661in}{2.186292in}%
\pgfsys@useobject{currentmarker}{}%
\end{pgfscope}%
\begin{pgfscope}%
\pgfsys@transformshift{5.208684in}{2.855359in}%
\pgfsys@useobject{currentmarker}{}%
\end{pgfscope}%
\end{pgfscope}%
\begin{pgfscope}%
\pgfpathrectangle{\pgfqpoint{0.708220in}{0.535823in}}{\pgfqpoint{5.141780in}{2.669453in}}%
\pgfusepath{clip}%
\pgfsetrectcap%
\pgfsetroundjoin%
\pgfsetlinewidth{1.003750pt}%
\definecolor{currentstroke}{rgb}{1.000000,0.694118,0.305882}%
\pgfsetstrokecolor{currentstroke}%
\pgfsetdash{}{0pt}%
\pgfpathmoveto{\pgfqpoint{2.423238in}{0.525823in}}%
\pgfpathlineto{\pgfqpoint{2.508405in}{0.555088in}}%
\pgfpathlineto{\pgfqpoint{2.733429in}{0.713199in}}%
\pgfpathlineto{\pgfqpoint{2.958452in}{0.872472in}}%
\pgfpathlineto{\pgfqpoint{3.183475in}{1.125431in}}%
\pgfpathlineto{\pgfqpoint{3.408498in}{1.272615in}}%
\pgfpathlineto{\pgfqpoint{3.633521in}{1.392305in}}%
\pgfpathlineto{\pgfqpoint{3.858545in}{1.478523in}}%
\pgfpathlineto{\pgfqpoint{4.083568in}{1.557487in}}%
\pgfpathlineto{\pgfqpoint{4.308591in}{1.662572in}}%
\pgfpathlineto{\pgfqpoint{4.533614in}{1.734776in}}%
\pgfpathlineto{\pgfqpoint{4.758637in}{1.924610in}}%
\pgfpathlineto{\pgfqpoint{4.983661in}{2.079869in}}%
\pgfpathlineto{\pgfqpoint{5.208684in}{2.525204in}}%
\pgfpathlineto{\pgfqpoint{5.433707in}{2.672837in}}%
\pgfpathlineto{\pgfqpoint{5.658730in}{3.144706in}}%
\pgfusepath{stroke}%
\end{pgfscope}%
\begin{pgfscope}%
\pgfpathrectangle{\pgfqpoint{0.708220in}{0.535823in}}{\pgfqpoint{5.141780in}{2.669453in}}%
\pgfusepath{clip}%
\pgfsetbuttcap%
\pgfsetbeveljoin%
\definecolor{currentfill}{rgb}{1.000000,0.694118,0.305882}%
\pgfsetfillcolor{currentfill}%
\pgfsetlinewidth{0.501875pt}%
\definecolor{currentstroke}{rgb}{0.000000,0.000000,0.000000}%
\pgfsetstrokecolor{currentstroke}%
\pgfsetdash{}{0pt}%
\pgfsys@defobject{currentmarker}{\pgfqpoint{-0.033023in}{-0.028091in}}{\pgfqpoint{0.033023in}{0.034722in}}{%
\pgfpathmoveto{\pgfqpoint{0.000000in}{0.034722in}}%
\pgfpathlineto{\pgfqpoint{-0.007796in}{0.010730in}}%
\pgfpathlineto{\pgfqpoint{-0.033023in}{0.010730in}}%
\pgfpathlineto{\pgfqpoint{-0.012614in}{-0.004098in}}%
\pgfpathlineto{\pgfqpoint{-0.020409in}{-0.028091in}}%
\pgfpathlineto{\pgfqpoint{-0.000000in}{-0.013263in}}%
\pgfpathlineto{\pgfqpoint{0.020409in}{-0.028091in}}%
\pgfpathlineto{\pgfqpoint{0.012614in}{-0.004098in}}%
\pgfpathlineto{\pgfqpoint{0.033023in}{0.010730in}}%
\pgfpathlineto{\pgfqpoint{0.007796in}{0.010730in}}%
\pgfpathclose%
\pgfusepath{stroke,fill}%
}%
\begin{pgfscope}%
\pgfsys@transformshift{1.833336in}{0.359278in}%
\pgfsys@useobject{currentmarker}{}%
\end{pgfscope}%
\begin{pgfscope}%
\pgfsys@transformshift{2.058359in}{0.431153in}%
\pgfsys@useobject{currentmarker}{}%
\end{pgfscope}%
\begin{pgfscope}%
\pgfsys@transformshift{2.283382in}{0.477766in}%
\pgfsys@useobject{currentmarker}{}%
\end{pgfscope}%
\begin{pgfscope}%
\pgfsys@transformshift{2.508405in}{0.555088in}%
\pgfsys@useobject{currentmarker}{}%
\end{pgfscope}%
\begin{pgfscope}%
\pgfsys@transformshift{2.733429in}{0.713199in}%
\pgfsys@useobject{currentmarker}{}%
\end{pgfscope}%
\begin{pgfscope}%
\pgfsys@transformshift{2.958452in}{0.872472in}%
\pgfsys@useobject{currentmarker}{}%
\end{pgfscope}%
\begin{pgfscope}%
\pgfsys@transformshift{3.183475in}{1.125431in}%
\pgfsys@useobject{currentmarker}{}%
\end{pgfscope}%
\begin{pgfscope}%
\pgfsys@transformshift{3.408498in}{1.272615in}%
\pgfsys@useobject{currentmarker}{}%
\end{pgfscope}%
\begin{pgfscope}%
\pgfsys@transformshift{3.633521in}{1.392305in}%
\pgfsys@useobject{currentmarker}{}%
\end{pgfscope}%
\begin{pgfscope}%
\pgfsys@transformshift{3.858545in}{1.478523in}%
\pgfsys@useobject{currentmarker}{}%
\end{pgfscope}%
\begin{pgfscope}%
\pgfsys@transformshift{4.083568in}{1.557487in}%
\pgfsys@useobject{currentmarker}{}%
\end{pgfscope}%
\begin{pgfscope}%
\pgfsys@transformshift{4.308591in}{1.662572in}%
\pgfsys@useobject{currentmarker}{}%
\end{pgfscope}%
\begin{pgfscope}%
\pgfsys@transformshift{4.533614in}{1.734776in}%
\pgfsys@useobject{currentmarker}{}%
\end{pgfscope}%
\begin{pgfscope}%
\pgfsys@transformshift{4.758637in}{1.924610in}%
\pgfsys@useobject{currentmarker}{}%
\end{pgfscope}%
\begin{pgfscope}%
\pgfsys@transformshift{4.983661in}{2.079869in}%
\pgfsys@useobject{currentmarker}{}%
\end{pgfscope}%
\begin{pgfscope}%
\pgfsys@transformshift{5.208684in}{2.525204in}%
\pgfsys@useobject{currentmarker}{}%
\end{pgfscope}%
\begin{pgfscope}%
\pgfsys@transformshift{5.433707in}{2.672837in}%
\pgfsys@useobject{currentmarker}{}%
\end{pgfscope}%
\begin{pgfscope}%
\pgfsys@transformshift{5.658730in}{3.144706in}%
\pgfsys@useobject{currentmarker}{}%
\end{pgfscope}%
\end{pgfscope}%
\begin{pgfscope}%
\pgfsetrectcap%
\pgfsetmiterjoin%
\pgfsetlinewidth{0.803000pt}%
\definecolor{currentstroke}{rgb}{0.000000,0.000000,0.000000}%
\pgfsetstrokecolor{currentstroke}%
\pgfsetdash{}{0pt}%
\pgfpathmoveto{\pgfqpoint{0.708220in}{0.535823in}}%
\pgfpathlineto{\pgfqpoint{0.708220in}{3.205275in}}%
\pgfusepath{stroke}%
\end{pgfscope}%
\begin{pgfscope}%
\pgfsetrectcap%
\pgfsetmiterjoin%
\pgfsetlinewidth{0.803000pt}%
\definecolor{currentstroke}{rgb}{0.000000,0.000000,0.000000}%
\pgfsetstrokecolor{currentstroke}%
\pgfsetdash{}{0pt}%
\pgfpathmoveto{\pgfqpoint{5.850000in}{0.535823in}}%
\pgfpathlineto{\pgfqpoint{5.850000in}{3.205275in}}%
\pgfusepath{stroke}%
\end{pgfscope}%
\begin{pgfscope}%
\pgfsetrectcap%
\pgfsetmiterjoin%
\pgfsetlinewidth{0.803000pt}%
\definecolor{currentstroke}{rgb}{0.000000,0.000000,0.000000}%
\pgfsetstrokecolor{currentstroke}%
\pgfsetdash{}{0pt}%
\pgfpathmoveto{\pgfqpoint{0.708220in}{0.535823in}}%
\pgfpathlineto{\pgfqpoint{5.850000in}{0.535823in}}%
\pgfusepath{stroke}%
\end{pgfscope}%
\begin{pgfscope}%
\pgfsetrectcap%
\pgfsetmiterjoin%
\pgfsetlinewidth{0.803000pt}%
\definecolor{currentstroke}{rgb}{0.000000,0.000000,0.000000}%
\pgfsetstrokecolor{currentstroke}%
\pgfsetdash{}{0pt}%
\pgfpathmoveto{\pgfqpoint{0.708220in}{3.205275in}}%
\pgfpathlineto{\pgfqpoint{5.850000in}{3.205275in}}%
\pgfusepath{stroke}%
\end{pgfscope}%
\begin{pgfscope}%
\pgfsetrectcap%
\pgfsetroundjoin%
\pgfsetlinewidth{1.003750pt}%
\definecolor{currentstroke}{rgb}{0.866667,0.058824,0.058824}%
\pgfsetstrokecolor{currentstroke}%
\pgfsetdash{}{0pt}%
\pgfpathmoveto{\pgfqpoint{0.758220in}{3.111525in}}%
\pgfpathlineto{\pgfqpoint{1.008220in}{3.111525in}}%
\pgfusepath{stroke}%
\end{pgfscope}%
\begin{pgfscope}%
\pgfsetbuttcap%
\pgfsetmiterjoin%
\definecolor{currentfill}{rgb}{0.866667,0.058824,0.058824}%
\pgfsetfillcolor{currentfill}%
\pgfsetlinewidth{0.501875pt}%
\definecolor{currentstroke}{rgb}{0.000000,0.000000,0.000000}%
\pgfsetstrokecolor{currentstroke}%
\pgfsetdash{}{0pt}%
\pgfsys@defobject{currentmarker}{\pgfqpoint{-0.033023in}{-0.028091in}}{\pgfqpoint{0.033023in}{0.034722in}}{%
\pgfpathmoveto{\pgfqpoint{0.000000in}{0.034722in}}%
\pgfpathlineto{\pgfqpoint{-0.033023in}{0.010730in}}%
\pgfpathlineto{\pgfqpoint{-0.020409in}{-0.028091in}}%
\pgfpathlineto{\pgfqpoint{0.020409in}{-0.028091in}}%
\pgfpathlineto{\pgfqpoint{0.033023in}{0.010730in}}%
\pgfpathclose%
\pgfusepath{stroke,fill}%
}%
\begin{pgfscope}%
\pgfsys@transformshift{0.883220in}{3.111525in}%
\pgfsys@useobject{currentmarker}{}%
\end{pgfscope}%
\end{pgfscope}%
\begin{pgfscope}%
\definecolor{textcolor}{rgb}{0.000000,0.000000,0.000000}%
\pgfsetstrokecolor{textcolor}%
\pgfsetfillcolor{textcolor}%
\pgftext[x=1.033220in,y=3.067775in,left,base]{\color{textcolor}\rmfamily\fontsize{9.000000}{10.800000}\selectfont cachet}%
\end{pgfscope}%
\begin{pgfscope}%
\pgfsetrectcap%
\pgfsetroundjoin%
\pgfsetlinewidth{1.003750pt}%
\definecolor{currentstroke}{rgb}{0.000000,0.000000,0.200000}%
\pgfsetstrokecolor{currentstroke}%
\pgfsetdash{}{0pt}%
\pgfpathmoveto{\pgfqpoint{0.758220in}{2.949726in}}%
\pgfpathlineto{\pgfqpoint{1.008220in}{2.949726in}}%
\pgfusepath{stroke}%
\end{pgfscope}%
\begin{pgfscope}%
\pgfsetbuttcap%
\pgfsetmiterjoin%
\definecolor{currentfill}{rgb}{0.000000,0.000000,0.200000}%
\pgfsetfillcolor{currentfill}%
\pgfsetlinewidth{0.501875pt}%
\definecolor{currentstroke}{rgb}{0.000000,0.000000,0.000000}%
\pgfsetstrokecolor{currentstroke}%
\pgfsetdash{}{0pt}%
\pgfsys@defobject{currentmarker}{\pgfqpoint{-0.034722in}{-0.034722in}}{\pgfqpoint{0.034722in}{0.034722in}}{%
\pgfpathmoveto{\pgfqpoint{-0.011574in}{-0.034722in}}%
\pgfpathlineto{\pgfqpoint{0.011574in}{-0.034722in}}%
\pgfpathlineto{\pgfqpoint{0.011574in}{-0.011574in}}%
\pgfpathlineto{\pgfqpoint{0.034722in}{-0.011574in}}%
\pgfpathlineto{\pgfqpoint{0.034722in}{0.011574in}}%
\pgfpathlineto{\pgfqpoint{0.011574in}{0.011574in}}%
\pgfpathlineto{\pgfqpoint{0.011574in}{0.034722in}}%
\pgfpathlineto{\pgfqpoint{-0.011574in}{0.034722in}}%
\pgfpathlineto{\pgfqpoint{-0.011574in}{0.011574in}}%
\pgfpathlineto{\pgfqpoint{-0.034722in}{0.011574in}}%
\pgfpathlineto{\pgfqpoint{-0.034722in}{-0.011574in}}%
\pgfpathlineto{\pgfqpoint{-0.011574in}{-0.011574in}}%
\pgfpathclose%
\pgfusepath{stroke,fill}%
}%
\begin{pgfscope}%
\pgfsys@transformshift{0.883220in}{2.949726in}%
\pgfsys@useobject{currentmarker}{}%
\end{pgfscope}%
\end{pgfscope}%
\begin{pgfscope}%
\definecolor{textcolor}{rgb}{0.000000,0.000000,0.000000}%
\pgfsetstrokecolor{textcolor}%
\pgfsetfillcolor{textcolor}%
\pgftext[x=1.033220in,y=2.905976in,left,base]{\color{textcolor}\rmfamily\fontsize{9.000000}{10.800000}\selectfont dynQBF}%
\end{pgfscope}%
\begin{pgfscope}%
\pgfsetrectcap%
\pgfsetroundjoin%
\pgfsetlinewidth{1.003750pt}%
\definecolor{currentstroke}{rgb}{0.000000,0.000000,0.866667}%
\pgfsetstrokecolor{currentstroke}%
\pgfsetdash{}{0pt}%
\pgfpathmoveto{\pgfqpoint{0.758220in}{2.787926in}}%
\pgfpathlineto{\pgfqpoint{1.008220in}{2.787926in}}%
\pgfusepath{stroke}%
\end{pgfscope}%
\begin{pgfscope}%
\pgfsetbuttcap%
\pgfsetmiterjoin%
\definecolor{currentfill}{rgb}{0.000000,0.000000,0.866667}%
\pgfsetfillcolor{currentfill}%
\pgfsetlinewidth{0.501875pt}%
\definecolor{currentstroke}{rgb}{0.000000,0.000000,0.000000}%
\pgfsetstrokecolor{currentstroke}%
\pgfsetdash{}{0pt}%
\pgfsys@defobject{currentmarker}{\pgfqpoint{-0.029463in}{-0.049105in}}{\pgfqpoint{0.029463in}{0.049105in}}{%
\pgfpathmoveto{\pgfqpoint{0.000000in}{-0.049105in}}%
\pgfpathlineto{\pgfqpoint{0.029463in}{0.000000in}}%
\pgfpathlineto{\pgfqpoint{0.000000in}{0.049105in}}%
\pgfpathlineto{\pgfqpoint{-0.029463in}{0.000000in}}%
\pgfpathclose%
\pgfusepath{stroke,fill}%
}%
\begin{pgfscope}%
\pgfsys@transformshift{0.883220in}{2.787926in}%
\pgfsys@useobject{currentmarker}{}%
\end{pgfscope}%
\end{pgfscope}%
\begin{pgfscope}%
\definecolor{textcolor}{rgb}{0.000000,0.000000,0.000000}%
\pgfsetstrokecolor{textcolor}%
\pgfsetfillcolor{textcolor}%
\pgftext[x=1.033220in,y=2.744176in,left,base]{\color{textcolor}\rmfamily\fontsize{9.000000}{10.800000}\selectfont dynasp}%
\end{pgfscope}%
\begin{pgfscope}%
\pgfsetrectcap%
\pgfsetroundjoin%
\pgfsetlinewidth{1.003750pt}%
\definecolor{currentstroke}{rgb}{0.250980,0.231373,0.796078}%
\pgfsetstrokecolor{currentstroke}%
\pgfsetdash{}{0pt}%
\pgfpathmoveto{\pgfqpoint{0.758220in}{2.626126in}}%
\pgfpathlineto{\pgfqpoint{1.008220in}{2.626126in}}%
\pgfusepath{stroke}%
\end{pgfscope}%
\begin{pgfscope}%
\pgfsetbuttcap%
\pgfsetmiterjoin%
\definecolor{currentfill}{rgb}{0.250980,0.231373,0.796078}%
\pgfsetfillcolor{currentfill}%
\pgfsetlinewidth{0.501875pt}%
\definecolor{currentstroke}{rgb}{0.000000,0.000000,0.000000}%
\pgfsetstrokecolor{currentstroke}%
\pgfsetdash{}{0pt}%
\pgfsys@defobject{currentmarker}{\pgfqpoint{-0.034722in}{-0.034722in}}{\pgfqpoint{0.034722in}{0.034722in}}{%
\pgfpathmoveto{\pgfqpoint{-0.034722in}{-0.034722in}}%
\pgfpathlineto{\pgfqpoint{0.034722in}{-0.034722in}}%
\pgfpathlineto{\pgfqpoint{0.034722in}{0.034722in}}%
\pgfpathlineto{\pgfqpoint{-0.034722in}{0.034722in}}%
\pgfpathclose%
\pgfusepath{stroke,fill}%
}%
\begin{pgfscope}%
\pgfsys@transformshift{0.883220in}{2.626126in}%
\pgfsys@useobject{currentmarker}{}%
\end{pgfscope}%
\end{pgfscope}%
\begin{pgfscope}%
\definecolor{textcolor}{rgb}{0.000000,0.000000,0.000000}%
\pgfsetstrokecolor{textcolor}%
\pgfsetfillcolor{textcolor}%
\pgftext[x=1.033220in,y=2.582376in,left,base]{\color{textcolor}\rmfamily\fontsize{9.000000}{10.800000}\selectfont sharpSAT}%
\end{pgfscope}%
\begin{pgfscope}%
\pgfsetrectcap%
\pgfsetroundjoin%
\pgfsetlinewidth{1.003750pt}%
\definecolor{currentstroke}{rgb}{0.615686,0.007843,0.843137}%
\pgfsetstrokecolor{currentstroke}%
\pgfsetdash{}{0pt}%
\pgfpathmoveto{\pgfqpoint{0.758220in}{2.464327in}}%
\pgfpathlineto{\pgfqpoint{1.008220in}{2.464327in}}%
\pgfusepath{stroke}%
\end{pgfscope}%
\begin{pgfscope}%
\pgfsetbuttcap%
\pgfsetroundjoin%
\definecolor{currentfill}{rgb}{0.615686,0.007843,0.843137}%
\pgfsetfillcolor{currentfill}%
\pgfsetlinewidth{0.501875pt}%
\definecolor{currentstroke}{rgb}{0.000000,0.000000,0.000000}%
\pgfsetstrokecolor{currentstroke}%
\pgfsetdash{}{0pt}%
\pgfsys@defobject{currentmarker}{\pgfqpoint{-0.034722in}{-0.034722in}}{\pgfqpoint{0.034722in}{0.034722in}}{%
\pgfpathmoveto{\pgfqpoint{0.000000in}{-0.034722in}}%
\pgfpathcurveto{\pgfqpoint{0.009208in}{-0.034722in}}{\pgfqpoint{0.018041in}{-0.031064in}}{\pgfqpoint{0.024552in}{-0.024552in}}%
\pgfpathcurveto{\pgfqpoint{0.031064in}{-0.018041in}}{\pgfqpoint{0.034722in}{-0.009208in}}{\pgfqpoint{0.034722in}{0.000000in}}%
\pgfpathcurveto{\pgfqpoint{0.034722in}{0.009208in}}{\pgfqpoint{0.031064in}{0.018041in}}{\pgfqpoint{0.024552in}{0.024552in}}%
\pgfpathcurveto{\pgfqpoint{0.018041in}{0.031064in}}{\pgfqpoint{0.009208in}{0.034722in}}{\pgfqpoint{0.000000in}{0.034722in}}%
\pgfpathcurveto{\pgfqpoint{-0.009208in}{0.034722in}}{\pgfqpoint{-0.018041in}{0.031064in}}{\pgfqpoint{-0.024552in}{0.024552in}}%
\pgfpathcurveto{\pgfqpoint{-0.031064in}{0.018041in}}{\pgfqpoint{-0.034722in}{0.009208in}}{\pgfqpoint{-0.034722in}{0.000000in}}%
\pgfpathcurveto{\pgfqpoint{-0.034722in}{-0.009208in}}{\pgfqpoint{-0.031064in}{-0.018041in}}{\pgfqpoint{-0.024552in}{-0.024552in}}%
\pgfpathcurveto{\pgfqpoint{-0.018041in}{-0.031064in}}{\pgfqpoint{-0.009208in}{-0.034722in}}{\pgfqpoint{0.000000in}{-0.034722in}}%
\pgfpathclose%
\pgfusepath{stroke,fill}%
}%
\begin{pgfscope}%
\pgfsys@transformshift{0.883220in}{2.464327in}%
\pgfsys@useobject{currentmarker}{}%
\end{pgfscope}%
\end{pgfscope}%
\begin{pgfscope}%
\definecolor{textcolor}{rgb}{0.000000,0.000000,0.000000}%
\pgfsetstrokecolor{textcolor}%
\pgfsetfillcolor{textcolor}%
\pgftext[x=1.033220in,y=2.420577in,left,base]{\color{textcolor}\rmfamily\fontsize{9.000000}{10.800000}\selectfont d4}%
\end{pgfscope}%
\begin{pgfscope}%
\pgfsetrectcap%
\pgfsetroundjoin%
\pgfsetlinewidth{1.003750pt}%
\definecolor{currentstroke}{rgb}{0.917647,0.372549,0.580392}%
\pgfsetstrokecolor{currentstroke}%
\pgfsetdash{}{0pt}%
\pgfpathmoveto{\pgfqpoint{0.758220in}{2.302527in}}%
\pgfpathlineto{\pgfqpoint{1.008220in}{2.302527in}}%
\pgfusepath{stroke}%
\end{pgfscope}%
\begin{pgfscope}%
\pgfsetbuttcap%
\pgfsetmiterjoin%
\definecolor{currentfill}{rgb}{0.917647,0.372549,0.580392}%
\pgfsetfillcolor{currentfill}%
\pgfsetlinewidth{0.501875pt}%
\definecolor{currentstroke}{rgb}{0.000000,0.000000,0.000000}%
\pgfsetstrokecolor{currentstroke}%
\pgfsetdash{}{0pt}%
\pgfsys@defobject{currentmarker}{\pgfqpoint{-0.049105in}{-0.049105in}}{\pgfqpoint{0.049105in}{0.049105in}}{%
\pgfpathmoveto{\pgfqpoint{0.000000in}{-0.049105in}}%
\pgfpathlineto{\pgfqpoint{0.049105in}{0.000000in}}%
\pgfpathlineto{\pgfqpoint{0.000000in}{0.049105in}}%
\pgfpathlineto{\pgfqpoint{-0.049105in}{0.000000in}}%
\pgfpathclose%
\pgfusepath{stroke,fill}%
}%
\begin{pgfscope}%
\pgfsys@transformshift{0.883220in}{2.302527in}%
\pgfsys@useobject{currentmarker}{}%
\end{pgfscope}%
\end{pgfscope}%
\begin{pgfscope}%
\definecolor{textcolor}{rgb}{0.000000,0.000000,0.000000}%
\pgfsetstrokecolor{textcolor}%
\pgfsetfillcolor{textcolor}%
\pgftext[x=1.033220in,y=2.258777in,left,base]{\color{textcolor}\rmfamily\fontsize{9.000000}{10.800000}\selectfont miniC2D}%
\end{pgfscope}%
\begin{pgfscope}%
\pgfsetrectcap%
\pgfsetroundjoin%
\pgfsetlinewidth{1.003750pt}%
\definecolor{currentstroke}{rgb}{0.529412,0.462745,0.384314}%
\pgfsetstrokecolor{currentstroke}%
\pgfsetdash{}{0pt}%
\pgfpathmoveto{\pgfqpoint{0.758220in}{2.140728in}}%
\pgfpathlineto{\pgfqpoint{1.008220in}{2.140728in}}%
\pgfusepath{stroke}%
\end{pgfscope}%
\begin{pgfscope}%
\pgfsetbuttcap%
\pgfsetmiterjoin%
\definecolor{currentfill}{rgb}{0.529412,0.462745,0.384314}%
\pgfsetfillcolor{currentfill}%
\pgfsetlinewidth{0.501875pt}%
\definecolor{currentstroke}{rgb}{0.000000,0.000000,0.000000}%
\pgfsetstrokecolor{currentstroke}%
\pgfsetdash{}{0pt}%
\pgfsys@defobject{currentmarker}{\pgfqpoint{-0.034722in}{-0.034722in}}{\pgfqpoint{0.034722in}{0.034722in}}{%
\pgfpathmoveto{\pgfqpoint{-0.000000in}{-0.034722in}}%
\pgfpathlineto{\pgfqpoint{0.034722in}{0.034722in}}%
\pgfpathlineto{\pgfqpoint{-0.034722in}{0.034722in}}%
\pgfpathclose%
\pgfusepath{stroke,fill}%
}%
\begin{pgfscope}%
\pgfsys@transformshift{0.883220in}{2.140728in}%
\pgfsys@useobject{currentmarker}{}%
\end{pgfscope}%
\end{pgfscope}%
\begin{pgfscope}%
\definecolor{textcolor}{rgb}{0.000000,0.000000,0.000000}%
\pgfsetstrokecolor{textcolor}%
\pgfsetfillcolor{textcolor}%
\pgftext[x=1.033220in,y=2.096978in,left,base]{\color{textcolor}\rmfamily\fontsize{9.000000}{10.800000}\selectfont greedy}%
\end{pgfscope}%
\begin{pgfscope}%
\pgfsetrectcap%
\pgfsetroundjoin%
\pgfsetlinewidth{1.003750pt}%
\definecolor{currentstroke}{rgb}{0.611765,0.568627,0.274510}%
\pgfsetstrokecolor{currentstroke}%
\pgfsetdash{}{0pt}%
\pgfpathmoveto{\pgfqpoint{0.758220in}{1.978928in}}%
\pgfpathlineto{\pgfqpoint{1.008220in}{1.978928in}}%
\pgfusepath{stroke}%
\end{pgfscope}%
\begin{pgfscope}%
\pgfsetbuttcap%
\pgfsetmiterjoin%
\definecolor{currentfill}{rgb}{0.611765,0.568627,0.274510}%
\pgfsetfillcolor{currentfill}%
\pgfsetlinewidth{0.501875pt}%
\definecolor{currentstroke}{rgb}{0.000000,0.000000,0.000000}%
\pgfsetstrokecolor{currentstroke}%
\pgfsetdash{}{0pt}%
\pgfsys@defobject{currentmarker}{\pgfqpoint{-0.034722in}{-0.034722in}}{\pgfqpoint{0.034722in}{0.034722in}}{%
\pgfpathmoveto{\pgfqpoint{-0.034722in}{0.000000in}}%
\pgfpathlineto{\pgfqpoint{0.034722in}{-0.034722in}}%
\pgfpathlineto{\pgfqpoint{0.034722in}{0.034722in}}%
\pgfpathclose%
\pgfusepath{stroke,fill}%
}%
\begin{pgfscope}%
\pgfsys@transformshift{0.883220in}{1.978928in}%
\pgfsys@useobject{currentmarker}{}%
\end{pgfscope}%
\end{pgfscope}%
\begin{pgfscope}%
\definecolor{textcolor}{rgb}{0.000000,0.000000,0.000000}%
\pgfsetstrokecolor{textcolor}%
\pgfsetfillcolor{textcolor}%
\pgftext[x=1.033220in,y=1.935178in,left,base]{\color{textcolor}\rmfamily\fontsize{9.000000}{10.800000}\selectfont metis}%
\end{pgfscope}%
\begin{pgfscope}%
\pgfsetrectcap%
\pgfsetroundjoin%
\pgfsetlinewidth{1.003750pt}%
\definecolor{currentstroke}{rgb}{0.780392,0.643137,0.254902}%
\pgfsetstrokecolor{currentstroke}%
\pgfsetdash{}{0pt}%
\pgfpathmoveto{\pgfqpoint{0.758220in}{1.817129in}}%
\pgfpathlineto{\pgfqpoint{1.008220in}{1.817129in}}%
\pgfusepath{stroke}%
\end{pgfscope}%
\begin{pgfscope}%
\pgfsetbuttcap%
\pgfsetmiterjoin%
\definecolor{currentfill}{rgb}{0.780392,0.643137,0.254902}%
\pgfsetfillcolor{currentfill}%
\pgfsetlinewidth{0.501875pt}%
\definecolor{currentstroke}{rgb}{0.000000,0.000000,0.000000}%
\pgfsetstrokecolor{currentstroke}%
\pgfsetdash{}{0pt}%
\pgfsys@defobject{currentmarker}{\pgfqpoint{-0.034722in}{-0.034722in}}{\pgfqpoint{0.034722in}{0.034722in}}{%
\pgfpathmoveto{\pgfqpoint{0.034722in}{-0.000000in}}%
\pgfpathlineto{\pgfqpoint{-0.034722in}{0.034722in}}%
\pgfpathlineto{\pgfqpoint{-0.034722in}{-0.034722in}}%
\pgfpathclose%
\pgfusepath{stroke,fill}%
}%
\begin{pgfscope}%
\pgfsys@transformshift{0.883220in}{1.817129in}%
\pgfsys@useobject{currentmarker}{}%
\end{pgfscope}%
\end{pgfscope}%
\begin{pgfscope}%
\definecolor{textcolor}{rgb}{0.000000,0.000000,0.000000}%
\pgfsetstrokecolor{textcolor}%
\pgfsetfillcolor{textcolor}%
\pgftext[x=1.033220in,y=1.773379in,left,base]{\color{textcolor}\rmfamily\fontsize{9.000000}{10.800000}\selectfont GN}%
\end{pgfscope}%
\begin{pgfscope}%
\pgfsetrectcap%
\pgfsetroundjoin%
\pgfsetlinewidth{1.003750pt}%
\definecolor{currentstroke}{rgb}{1.000000,0.694118,0.305882}%
\pgfsetstrokecolor{currentstroke}%
\pgfsetdash{}{0pt}%
\pgfpathmoveto{\pgfqpoint{0.758220in}{1.655329in}}%
\pgfpathlineto{\pgfqpoint{1.008220in}{1.655329in}}%
\pgfusepath{stroke}%
\end{pgfscope}%
\begin{pgfscope}%
\pgfsetbuttcap%
\pgfsetbeveljoin%
\definecolor{currentfill}{rgb}{1.000000,0.694118,0.305882}%
\pgfsetfillcolor{currentfill}%
\pgfsetlinewidth{0.501875pt}%
\definecolor{currentstroke}{rgb}{0.000000,0.000000,0.000000}%
\pgfsetstrokecolor{currentstroke}%
\pgfsetdash{}{0pt}%
\pgfsys@defobject{currentmarker}{\pgfqpoint{-0.033023in}{-0.028091in}}{\pgfqpoint{0.033023in}{0.034722in}}{%
\pgfpathmoveto{\pgfqpoint{0.000000in}{0.034722in}}%
\pgfpathlineto{\pgfqpoint{-0.007796in}{0.010730in}}%
\pgfpathlineto{\pgfqpoint{-0.033023in}{0.010730in}}%
\pgfpathlineto{\pgfqpoint{-0.012614in}{-0.004098in}}%
\pgfpathlineto{\pgfqpoint{-0.020409in}{-0.028091in}}%
\pgfpathlineto{\pgfqpoint{-0.000000in}{-0.013263in}}%
\pgfpathlineto{\pgfqpoint{0.020409in}{-0.028091in}}%
\pgfpathlineto{\pgfqpoint{0.012614in}{-0.004098in}}%
\pgfpathlineto{\pgfqpoint{0.033023in}{0.010730in}}%
\pgfpathlineto{\pgfqpoint{0.007796in}{0.010730in}}%
\pgfpathclose%
\pgfusepath{stroke,fill}%
}%
\begin{pgfscope}%
\pgfsys@transformshift{0.883220in}{1.655329in}%
\pgfsys@useobject{currentmarker}{}%
\end{pgfscope}%
\end{pgfscope}%
\begin{pgfscope}%
\definecolor{textcolor}{rgb}{0.000000,0.000000,0.000000}%
\pgfsetstrokecolor{textcolor}%
\pgfsetfillcolor{textcolor}%
\pgftext[x=1.033220in,y=1.611579in,left,base]{\color{textcolor}\rmfamily\fontsize{9.000000}{10.800000}\selectfont LG+Flow}%
\end{pgfscope}%
\end{pgfpicture}%
\makeatother%
\endgroup%

	%% Creator: Matplotlib, PGF backend
%%
%% To include the figure in your LaTeX document, write
%%   \input{<filename>.pgf}
%%
%% Make sure the required packages are loaded in your preamble
%%   \usepackage{pgf}
%%
%% and, on pdftex
%%   \usepackage[utf8]{inputenc}\DeclareUnicodeCharacter{2212}{-}
%%
%% or, on luatex and xetex
%%   \usepackage{unicode-math}
%%
%% Figures using additional raster images can only be included by \input if
%% they are in the same directory as the main LaTeX file. For loading figures
%% from other directories you can use the `import` package
%%   \usepackage{import}
%%
%% and then include the figures with
%%   \import{<path to file>}{<filename>.pgf}
%%
%% Matplotlib used the following preamble
%%   \usepackage[utf8x]{inputenc}
%%   \usepackage[T1]{fontenc}
%%
\begingroup%
\makeatletter%
\begin{pgfpicture}%
\pgfpathrectangle{\pgfpointorigin}{\pgfqpoint{6.000000in}{3.400000in}}%
\pgfusepath{use as bounding box, clip}%
\begin{pgfscope}%
\pgfsetbuttcap%
\pgfsetmiterjoin%
\definecolor{currentfill}{rgb}{1.000000,1.000000,1.000000}%
\pgfsetfillcolor{currentfill}%
\pgfsetlinewidth{0.000000pt}%
\definecolor{currentstroke}{rgb}{1.000000,1.000000,1.000000}%
\pgfsetstrokecolor{currentstroke}%
\pgfsetdash{}{0pt}%
\pgfpathmoveto{\pgfqpoint{0.000000in}{0.000000in}}%
\pgfpathlineto{\pgfqpoint{6.000000in}{0.000000in}}%
\pgfpathlineto{\pgfqpoint{6.000000in}{3.400000in}}%
\pgfpathlineto{\pgfqpoint{0.000000in}{3.400000in}}%
\pgfpathclose%
\pgfusepath{fill}%
\end{pgfscope}%
\begin{pgfscope}%
\pgfsetbuttcap%
\pgfsetmiterjoin%
\definecolor{currentfill}{rgb}{1.000000,1.000000,1.000000}%
\pgfsetfillcolor{currentfill}%
\pgfsetlinewidth{0.000000pt}%
\definecolor{currentstroke}{rgb}{0.000000,0.000000,0.000000}%
\pgfsetstrokecolor{currentstroke}%
\pgfsetstrokeopacity{0.000000}%
\pgfsetdash{}{0pt}%
\pgfpathmoveto{\pgfqpoint{0.553904in}{0.535823in}}%
\pgfpathlineto{\pgfqpoint{5.850000in}{0.535823in}}%
\pgfpathlineto{\pgfqpoint{5.850000in}{3.250000in}}%
\pgfpathlineto{\pgfqpoint{0.553904in}{3.250000in}}%
\pgfpathclose%
\pgfusepath{fill}%
\end{pgfscope}%
\begin{pgfscope}%
\pgfsetbuttcap%
\pgfsetroundjoin%
\definecolor{currentfill}{rgb}{0.000000,0.000000,0.000000}%
\pgfsetfillcolor{currentfill}%
\pgfsetlinewidth{0.803000pt}%
\definecolor{currentstroke}{rgb}{0.000000,0.000000,0.000000}%
\pgfsetstrokecolor{currentstroke}%
\pgfsetdash{}{0pt}%
\pgfsys@defobject{currentmarker}{\pgfqpoint{0.000000in}{-0.048611in}}{\pgfqpoint{0.000000in}{0.000000in}}{%
\pgfpathmoveto{\pgfqpoint{0.000000in}{0.000000in}}%
\pgfpathlineto{\pgfqpoint{0.000000in}{-0.048611in}}%
\pgfusepath{stroke,fill}%
}%
\begin{pgfscope}%
\pgfsys@transformshift{0.794636in}{0.535823in}%
\pgfsys@useobject{currentmarker}{}%
\end{pgfscope}%
\end{pgfscope}%
\begin{pgfscope}%
\definecolor{textcolor}{rgb}{0.000000,0.000000,0.000000}%
\pgfsetstrokecolor{textcolor}%
\pgfsetfillcolor{textcolor}%
\pgftext[x=0.794636in,y=0.438600in,,top]{\color{textcolor}\rmfamily\fontsize{9.000000}{10.800000}\selectfont \(\displaystyle {50}\)}%
\end{pgfscope}%
\begin{pgfscope}%
\pgfsetbuttcap%
\pgfsetroundjoin%
\definecolor{currentfill}{rgb}{0.000000,0.000000,0.000000}%
\pgfsetfillcolor{currentfill}%
\pgfsetlinewidth{0.803000pt}%
\definecolor{currentstroke}{rgb}{0.000000,0.000000,0.000000}%
\pgfsetstrokecolor{currentstroke}%
\pgfsetdash{}{0pt}%
\pgfsys@defobject{currentmarker}{\pgfqpoint{0.000000in}{-0.048611in}}{\pgfqpoint{0.000000in}{0.000000in}}{%
\pgfpathmoveto{\pgfqpoint{0.000000in}{0.000000in}}%
\pgfpathlineto{\pgfqpoint{0.000000in}{-0.048611in}}%
\pgfusepath{stroke,fill}%
}%
\begin{pgfscope}%
\pgfsys@transformshift{1.998294in}{0.535823in}%
\pgfsys@useobject{currentmarker}{}%
\end{pgfscope}%
\end{pgfscope}%
\begin{pgfscope}%
\definecolor{textcolor}{rgb}{0.000000,0.000000,0.000000}%
\pgfsetstrokecolor{textcolor}%
\pgfsetfillcolor{textcolor}%
\pgftext[x=1.998294in,y=0.438600in,,top]{\color{textcolor}\rmfamily\fontsize{9.000000}{10.800000}\selectfont \(\displaystyle {100}\)}%
\end{pgfscope}%
\begin{pgfscope}%
\pgfsetbuttcap%
\pgfsetroundjoin%
\definecolor{currentfill}{rgb}{0.000000,0.000000,0.000000}%
\pgfsetfillcolor{currentfill}%
\pgfsetlinewidth{0.803000pt}%
\definecolor{currentstroke}{rgb}{0.000000,0.000000,0.000000}%
\pgfsetstrokecolor{currentstroke}%
\pgfsetdash{}{0pt}%
\pgfsys@defobject{currentmarker}{\pgfqpoint{0.000000in}{-0.048611in}}{\pgfqpoint{0.000000in}{0.000000in}}{%
\pgfpathmoveto{\pgfqpoint{0.000000in}{0.000000in}}%
\pgfpathlineto{\pgfqpoint{0.000000in}{-0.048611in}}%
\pgfusepath{stroke,fill}%
}%
\begin{pgfscope}%
\pgfsys@transformshift{3.201952in}{0.535823in}%
\pgfsys@useobject{currentmarker}{}%
\end{pgfscope}%
\end{pgfscope}%
\begin{pgfscope}%
\definecolor{textcolor}{rgb}{0.000000,0.000000,0.000000}%
\pgfsetstrokecolor{textcolor}%
\pgfsetfillcolor{textcolor}%
\pgftext[x=3.201952in,y=0.438600in,,top]{\color{textcolor}\rmfamily\fontsize{9.000000}{10.800000}\selectfont \(\displaystyle {150}\)}%
\end{pgfscope}%
\begin{pgfscope}%
\pgfsetbuttcap%
\pgfsetroundjoin%
\definecolor{currentfill}{rgb}{0.000000,0.000000,0.000000}%
\pgfsetfillcolor{currentfill}%
\pgfsetlinewidth{0.803000pt}%
\definecolor{currentstroke}{rgb}{0.000000,0.000000,0.000000}%
\pgfsetstrokecolor{currentstroke}%
\pgfsetdash{}{0pt}%
\pgfsys@defobject{currentmarker}{\pgfqpoint{0.000000in}{-0.048611in}}{\pgfqpoint{0.000000in}{0.000000in}}{%
\pgfpathmoveto{\pgfqpoint{0.000000in}{0.000000in}}%
\pgfpathlineto{\pgfqpoint{0.000000in}{-0.048611in}}%
\pgfusepath{stroke,fill}%
}%
\begin{pgfscope}%
\pgfsys@transformshift{4.405610in}{0.535823in}%
\pgfsys@useobject{currentmarker}{}%
\end{pgfscope}%
\end{pgfscope}%
\begin{pgfscope}%
\definecolor{textcolor}{rgb}{0.000000,0.000000,0.000000}%
\pgfsetstrokecolor{textcolor}%
\pgfsetfillcolor{textcolor}%
\pgftext[x=4.405610in,y=0.438600in,,top]{\color{textcolor}\rmfamily\fontsize{9.000000}{10.800000}\selectfont \(\displaystyle {200}\)}%
\end{pgfscope}%
\begin{pgfscope}%
\pgfsetbuttcap%
\pgfsetroundjoin%
\definecolor{currentfill}{rgb}{0.000000,0.000000,0.000000}%
\pgfsetfillcolor{currentfill}%
\pgfsetlinewidth{0.803000pt}%
\definecolor{currentstroke}{rgb}{0.000000,0.000000,0.000000}%
\pgfsetstrokecolor{currentstroke}%
\pgfsetdash{}{0pt}%
\pgfsys@defobject{currentmarker}{\pgfqpoint{0.000000in}{-0.048611in}}{\pgfqpoint{0.000000in}{0.000000in}}{%
\pgfpathmoveto{\pgfqpoint{0.000000in}{0.000000in}}%
\pgfpathlineto{\pgfqpoint{0.000000in}{-0.048611in}}%
\pgfusepath{stroke,fill}%
}%
\begin{pgfscope}%
\pgfsys@transformshift{5.609268in}{0.535823in}%
\pgfsys@useobject{currentmarker}{}%
\end{pgfscope}%
\end{pgfscope}%
\begin{pgfscope}%
\definecolor{textcolor}{rgb}{0.000000,0.000000,0.000000}%
\pgfsetstrokecolor{textcolor}%
\pgfsetfillcolor{textcolor}%
\pgftext[x=5.609268in,y=0.438600in,,top]{\color{textcolor}\rmfamily\fontsize{9.000000}{10.800000}\selectfont \(\displaystyle {250}\)}%
\end{pgfscope}%
\begin{pgfscope}%
\definecolor{textcolor}{rgb}{0.000000,0.000000,0.000000}%
\pgfsetstrokecolor{textcolor}%
\pgfsetfillcolor{textcolor}%
\pgftext[x=3.201952in,y=0.272655in,,top]{\color{textcolor}\rmfamily\fontsize{10.000000}{12.000000}\selectfont \(\displaystyle n\): Number of vertices}%
\end{pgfscope}%
\begin{pgfscope}%
\pgfsetbuttcap%
\pgfsetroundjoin%
\definecolor{currentfill}{rgb}{0.000000,0.000000,0.000000}%
\pgfsetfillcolor{currentfill}%
\pgfsetlinewidth{0.803000pt}%
\definecolor{currentstroke}{rgb}{0.000000,0.000000,0.000000}%
\pgfsetstrokecolor{currentstroke}%
\pgfsetdash{}{0pt}%
\pgfsys@defobject{currentmarker}{\pgfqpoint{-0.048611in}{0.000000in}}{\pgfqpoint{-0.000000in}{0.000000in}}{%
\pgfpathmoveto{\pgfqpoint{-0.000000in}{0.000000in}}%
\pgfpathlineto{\pgfqpoint{-0.048611in}{0.000000in}}%
\pgfusepath{stroke,fill}%
}%
\begin{pgfscope}%
\pgfsys@transformshift{0.553904in}{0.719376in}%
\pgfsys@useobject{currentmarker}{}%
\end{pgfscope}%
\end{pgfscope}%
\begin{pgfscope}%
\definecolor{textcolor}{rgb}{0.000000,0.000000,0.000000}%
\pgfsetstrokecolor{textcolor}%
\pgfsetfillcolor{textcolor}%
\pgftext[x=0.328211in, y=0.676331in, left, base]{\color{textcolor}\rmfamily\fontsize{9.000000}{10.800000}\selectfont \(\displaystyle {10}\)}%
\end{pgfscope}%
\begin{pgfscope}%
\pgfsetbuttcap%
\pgfsetroundjoin%
\definecolor{currentfill}{rgb}{0.000000,0.000000,0.000000}%
\pgfsetfillcolor{currentfill}%
\pgfsetlinewidth{0.803000pt}%
\definecolor{currentstroke}{rgb}{0.000000,0.000000,0.000000}%
\pgfsetstrokecolor{currentstroke}%
\pgfsetdash{}{0pt}%
\pgfsys@defobject{currentmarker}{\pgfqpoint{-0.048611in}{0.000000in}}{\pgfqpoint{-0.000000in}{0.000000in}}{%
\pgfpathmoveto{\pgfqpoint{-0.000000in}{0.000000in}}%
\pgfpathlineto{\pgfqpoint{-0.048611in}{0.000000in}}%
\pgfusepath{stroke,fill}%
}%
\begin{pgfscope}%
\pgfsys@transformshift{0.553904in}{1.321189in}%
\pgfsys@useobject{currentmarker}{}%
\end{pgfscope}%
\end{pgfscope}%
\begin{pgfscope}%
\definecolor{textcolor}{rgb}{0.000000,0.000000,0.000000}%
\pgfsetstrokecolor{textcolor}%
\pgfsetfillcolor{textcolor}%
\pgftext[x=0.328211in, y=1.278144in, left, base]{\color{textcolor}\rmfamily\fontsize{9.000000}{10.800000}\selectfont \(\displaystyle {20}\)}%
\end{pgfscope}%
\begin{pgfscope}%
\pgfsetbuttcap%
\pgfsetroundjoin%
\definecolor{currentfill}{rgb}{0.000000,0.000000,0.000000}%
\pgfsetfillcolor{currentfill}%
\pgfsetlinewidth{0.803000pt}%
\definecolor{currentstroke}{rgb}{0.000000,0.000000,0.000000}%
\pgfsetstrokecolor{currentstroke}%
\pgfsetdash{}{0pt}%
\pgfsys@defobject{currentmarker}{\pgfqpoint{-0.048611in}{0.000000in}}{\pgfqpoint{-0.000000in}{0.000000in}}{%
\pgfpathmoveto{\pgfqpoint{-0.000000in}{0.000000in}}%
\pgfpathlineto{\pgfqpoint{-0.048611in}{0.000000in}}%
\pgfusepath{stroke,fill}%
}%
\begin{pgfscope}%
\pgfsys@transformshift{0.553904in}{1.923002in}%
\pgfsys@useobject{currentmarker}{}%
\end{pgfscope}%
\end{pgfscope}%
\begin{pgfscope}%
\definecolor{textcolor}{rgb}{0.000000,0.000000,0.000000}%
\pgfsetstrokecolor{textcolor}%
\pgfsetfillcolor{textcolor}%
\pgftext[x=0.328211in, y=1.879957in, left, base]{\color{textcolor}\rmfamily\fontsize{9.000000}{10.800000}\selectfont \(\displaystyle {30}\)}%
\end{pgfscope}%
\begin{pgfscope}%
\pgfsetbuttcap%
\pgfsetroundjoin%
\definecolor{currentfill}{rgb}{0.000000,0.000000,0.000000}%
\pgfsetfillcolor{currentfill}%
\pgfsetlinewidth{0.803000pt}%
\definecolor{currentstroke}{rgb}{0.000000,0.000000,0.000000}%
\pgfsetstrokecolor{currentstroke}%
\pgfsetdash{}{0pt}%
\pgfsys@defobject{currentmarker}{\pgfqpoint{-0.048611in}{0.000000in}}{\pgfqpoint{-0.000000in}{0.000000in}}{%
\pgfpathmoveto{\pgfqpoint{-0.000000in}{0.000000in}}%
\pgfpathlineto{\pgfqpoint{-0.048611in}{0.000000in}}%
\pgfusepath{stroke,fill}%
}%
\begin{pgfscope}%
\pgfsys@transformshift{0.553904in}{2.524815in}%
\pgfsys@useobject{currentmarker}{}%
\end{pgfscope}%
\end{pgfscope}%
\begin{pgfscope}%
\definecolor{textcolor}{rgb}{0.000000,0.000000,0.000000}%
\pgfsetstrokecolor{textcolor}%
\pgfsetfillcolor{textcolor}%
\pgftext[x=0.328211in, y=2.481770in, left, base]{\color{textcolor}\rmfamily\fontsize{9.000000}{10.800000}\selectfont \(\displaystyle {40}\)}%
\end{pgfscope}%
\begin{pgfscope}%
\pgfsetbuttcap%
\pgfsetroundjoin%
\definecolor{currentfill}{rgb}{0.000000,0.000000,0.000000}%
\pgfsetfillcolor{currentfill}%
\pgfsetlinewidth{0.803000pt}%
\definecolor{currentstroke}{rgb}{0.000000,0.000000,0.000000}%
\pgfsetstrokecolor{currentstroke}%
\pgfsetdash{}{0pt}%
\pgfsys@defobject{currentmarker}{\pgfqpoint{-0.048611in}{0.000000in}}{\pgfqpoint{-0.000000in}{0.000000in}}{%
\pgfpathmoveto{\pgfqpoint{-0.000000in}{0.000000in}}%
\pgfpathlineto{\pgfqpoint{-0.048611in}{0.000000in}}%
\pgfusepath{stroke,fill}%
}%
\begin{pgfscope}%
\pgfsys@transformshift{0.553904in}{3.126628in}%
\pgfsys@useobject{currentmarker}{}%
\end{pgfscope}%
\end{pgfscope}%
\begin{pgfscope}%
\definecolor{textcolor}{rgb}{0.000000,0.000000,0.000000}%
\pgfsetstrokecolor{textcolor}%
\pgfsetfillcolor{textcolor}%
\pgftext[x=0.328211in, y=3.083583in, left, base]{\color{textcolor}\rmfamily\fontsize{9.000000}{10.800000}\selectfont \(\displaystyle {50}\)}%
\end{pgfscope}%
\begin{pgfscope}%
\definecolor{textcolor}{rgb}{0.000000,0.000000,0.000000}%
\pgfsetstrokecolor{textcolor}%
\pgfsetfillcolor{textcolor}%
\pgftext[x=0.272655in,y=1.892911in,,bottom,rotate=90.000000]{\color{textcolor}\rmfamily\fontsize{10.000000}{12.000000}\selectfont Median max rank}%
\end{pgfscope}%
\begin{pgfscope}%
\pgfpathrectangle{\pgfqpoint{0.553904in}{0.535823in}}{\pgfqpoint{5.296096in}{2.714177in}}%
\pgfusepath{clip}%
\pgfsetrectcap%
\pgfsetroundjoin%
\pgfsetlinewidth{1.003750pt}%
\definecolor{currentstroke}{rgb}{0.529412,0.462745,0.384314}%
\pgfsetstrokecolor{currentstroke}%
\pgfsetdash{}{0pt}%
\pgfpathmoveto{\pgfqpoint{0.794636in}{0.779557in}}%
\pgfpathlineto{\pgfqpoint{1.035368in}{0.899920in}}%
\pgfpathlineto{\pgfqpoint{1.276099in}{0.960101in}}%
\pgfpathlineto{\pgfqpoint{1.516831in}{1.080464in}}%
\pgfpathlineto{\pgfqpoint{1.757562in}{1.200826in}}%
\pgfpathlineto{\pgfqpoint{1.998294in}{1.321189in}}%
\pgfpathlineto{\pgfqpoint{2.239026in}{1.441551in}}%
\pgfpathlineto{\pgfqpoint{2.479757in}{1.561914in}}%
\pgfpathlineto{\pgfqpoint{2.720489in}{1.682277in}}%
\pgfpathlineto{\pgfqpoint{2.961221in}{1.772549in}}%
\pgfpathlineto{\pgfqpoint{3.201952in}{1.923002in}}%
\pgfpathlineto{\pgfqpoint{3.442684in}{2.013274in}}%
\pgfpathlineto{\pgfqpoint{3.683415in}{2.103546in}}%
\pgfpathlineto{\pgfqpoint{3.924147in}{2.223909in}}%
\pgfpathlineto{\pgfqpoint{4.164879in}{2.344271in}}%
\pgfpathlineto{\pgfqpoint{4.405610in}{2.464634in}}%
\pgfpathlineto{\pgfqpoint{4.646342in}{2.584996in}}%
\pgfpathlineto{\pgfqpoint{4.887074in}{2.705359in}}%
\pgfpathlineto{\pgfqpoint{5.127805in}{2.885903in}}%
\pgfpathlineto{\pgfqpoint{5.368537in}{2.946084in}}%
\pgfpathlineto{\pgfqpoint{5.609268in}{3.126628in}}%
\pgfusepath{stroke}%
\end{pgfscope}%
\begin{pgfscope}%
\pgfpathrectangle{\pgfqpoint{0.553904in}{0.535823in}}{\pgfqpoint{5.296096in}{2.714177in}}%
\pgfusepath{clip}%
\pgfsetbuttcap%
\pgfsetmiterjoin%
\definecolor{currentfill}{rgb}{0.529412,0.462745,0.384314}%
\pgfsetfillcolor{currentfill}%
\pgfsetlinewidth{0.501875pt}%
\definecolor{currentstroke}{rgb}{0.000000,0.000000,0.000000}%
\pgfsetstrokecolor{currentstroke}%
\pgfsetdash{}{0pt}%
\pgfsys@defobject{currentmarker}{\pgfqpoint{-0.034722in}{-0.034722in}}{\pgfqpoint{0.034722in}{0.034722in}}{%
\pgfpathmoveto{\pgfqpoint{-0.000000in}{-0.034722in}}%
\pgfpathlineto{\pgfqpoint{0.034722in}{0.034722in}}%
\pgfpathlineto{\pgfqpoint{-0.034722in}{0.034722in}}%
\pgfpathclose%
\pgfusepath{stroke,fill}%
}%
\begin{pgfscope}%
\pgfsys@transformshift{0.794636in}{0.779557in}%
\pgfsys@useobject{currentmarker}{}%
\end{pgfscope}%
\begin{pgfscope}%
\pgfsys@transformshift{1.035368in}{0.899920in}%
\pgfsys@useobject{currentmarker}{}%
\end{pgfscope}%
\begin{pgfscope}%
\pgfsys@transformshift{1.276099in}{0.960101in}%
\pgfsys@useobject{currentmarker}{}%
\end{pgfscope}%
\begin{pgfscope}%
\pgfsys@transformshift{1.516831in}{1.080464in}%
\pgfsys@useobject{currentmarker}{}%
\end{pgfscope}%
\begin{pgfscope}%
\pgfsys@transformshift{1.757562in}{1.200826in}%
\pgfsys@useobject{currentmarker}{}%
\end{pgfscope}%
\begin{pgfscope}%
\pgfsys@transformshift{1.998294in}{1.321189in}%
\pgfsys@useobject{currentmarker}{}%
\end{pgfscope}%
\begin{pgfscope}%
\pgfsys@transformshift{2.239026in}{1.441551in}%
\pgfsys@useobject{currentmarker}{}%
\end{pgfscope}%
\begin{pgfscope}%
\pgfsys@transformshift{2.479757in}{1.561914in}%
\pgfsys@useobject{currentmarker}{}%
\end{pgfscope}%
\begin{pgfscope}%
\pgfsys@transformshift{2.720489in}{1.682277in}%
\pgfsys@useobject{currentmarker}{}%
\end{pgfscope}%
\begin{pgfscope}%
\pgfsys@transformshift{2.961221in}{1.772549in}%
\pgfsys@useobject{currentmarker}{}%
\end{pgfscope}%
\begin{pgfscope}%
\pgfsys@transformshift{3.201952in}{1.923002in}%
\pgfsys@useobject{currentmarker}{}%
\end{pgfscope}%
\begin{pgfscope}%
\pgfsys@transformshift{3.442684in}{2.013274in}%
\pgfsys@useobject{currentmarker}{}%
\end{pgfscope}%
\begin{pgfscope}%
\pgfsys@transformshift{3.683415in}{2.103546in}%
\pgfsys@useobject{currentmarker}{}%
\end{pgfscope}%
\begin{pgfscope}%
\pgfsys@transformshift{3.924147in}{2.223909in}%
\pgfsys@useobject{currentmarker}{}%
\end{pgfscope}%
\begin{pgfscope}%
\pgfsys@transformshift{4.164879in}{2.344271in}%
\pgfsys@useobject{currentmarker}{}%
\end{pgfscope}%
\begin{pgfscope}%
\pgfsys@transformshift{4.405610in}{2.464634in}%
\pgfsys@useobject{currentmarker}{}%
\end{pgfscope}%
\begin{pgfscope}%
\pgfsys@transformshift{4.646342in}{2.584996in}%
\pgfsys@useobject{currentmarker}{}%
\end{pgfscope}%
\begin{pgfscope}%
\pgfsys@transformshift{4.887074in}{2.705359in}%
\pgfsys@useobject{currentmarker}{}%
\end{pgfscope}%
\begin{pgfscope}%
\pgfsys@transformshift{5.127805in}{2.885903in}%
\pgfsys@useobject{currentmarker}{}%
\end{pgfscope}%
\begin{pgfscope}%
\pgfsys@transformshift{5.368537in}{2.946084in}%
\pgfsys@useobject{currentmarker}{}%
\end{pgfscope}%
\begin{pgfscope}%
\pgfsys@transformshift{5.609268in}{3.126628in}%
\pgfsys@useobject{currentmarker}{}%
\end{pgfscope}%
\end{pgfscope}%
\begin{pgfscope}%
\pgfpathrectangle{\pgfqpoint{0.553904in}{0.535823in}}{\pgfqpoint{5.296096in}{2.714177in}}%
\pgfusepath{clip}%
\pgfsetrectcap%
\pgfsetroundjoin%
\pgfsetlinewidth{1.003750pt}%
\definecolor{currentstroke}{rgb}{0.611765,0.568627,0.274510}%
\pgfsetstrokecolor{currentstroke}%
\pgfsetdash{}{0pt}%
\pgfpathmoveto{\pgfqpoint{0.794636in}{0.719376in}}%
\pgfpathlineto{\pgfqpoint{1.035368in}{0.779557in}}%
\pgfpathlineto{\pgfqpoint{1.276099in}{0.839738in}}%
\pgfpathlineto{\pgfqpoint{1.516831in}{0.899920in}}%
\pgfpathlineto{\pgfqpoint{1.757562in}{1.020282in}}%
\pgfpathlineto{\pgfqpoint{1.998294in}{1.080464in}}%
\pgfpathlineto{\pgfqpoint{2.239026in}{1.140645in}}%
\pgfpathlineto{\pgfqpoint{2.479757in}{1.200826in}}%
\pgfpathlineto{\pgfqpoint{2.720489in}{1.321189in}}%
\pgfpathlineto{\pgfqpoint{2.961221in}{1.411461in}}%
\pgfpathlineto{\pgfqpoint{3.201952in}{1.471642in}}%
\pgfpathlineto{\pgfqpoint{3.442684in}{1.561914in}}%
\pgfpathlineto{\pgfqpoint{3.683415in}{1.622095in}}%
\pgfpathlineto{\pgfqpoint{3.924147in}{1.742458in}}%
\pgfpathlineto{\pgfqpoint{4.164879in}{1.802639in}}%
\pgfpathlineto{\pgfqpoint{4.405610in}{1.862821in}}%
\pgfpathlineto{\pgfqpoint{4.646342in}{1.923002in}}%
\pgfpathlineto{\pgfqpoint{4.887074in}{2.043365in}}%
\pgfpathlineto{\pgfqpoint{5.127805in}{2.163727in}}%
\pgfpathlineto{\pgfqpoint{5.368537in}{2.223909in}}%
\pgfpathlineto{\pgfqpoint{5.609268in}{2.284090in}}%
\pgfusepath{stroke}%
\end{pgfscope}%
\begin{pgfscope}%
\pgfpathrectangle{\pgfqpoint{0.553904in}{0.535823in}}{\pgfqpoint{5.296096in}{2.714177in}}%
\pgfusepath{clip}%
\pgfsetbuttcap%
\pgfsetmiterjoin%
\definecolor{currentfill}{rgb}{0.611765,0.568627,0.274510}%
\pgfsetfillcolor{currentfill}%
\pgfsetlinewidth{0.501875pt}%
\definecolor{currentstroke}{rgb}{0.000000,0.000000,0.000000}%
\pgfsetstrokecolor{currentstroke}%
\pgfsetdash{}{0pt}%
\pgfsys@defobject{currentmarker}{\pgfqpoint{-0.034722in}{-0.034722in}}{\pgfqpoint{0.034722in}{0.034722in}}{%
\pgfpathmoveto{\pgfqpoint{-0.034722in}{0.000000in}}%
\pgfpathlineto{\pgfqpoint{0.034722in}{-0.034722in}}%
\pgfpathlineto{\pgfqpoint{0.034722in}{0.034722in}}%
\pgfpathclose%
\pgfusepath{stroke,fill}%
}%
\begin{pgfscope}%
\pgfsys@transformshift{0.794636in}{0.719376in}%
\pgfsys@useobject{currentmarker}{}%
\end{pgfscope}%
\begin{pgfscope}%
\pgfsys@transformshift{1.035368in}{0.779557in}%
\pgfsys@useobject{currentmarker}{}%
\end{pgfscope}%
\begin{pgfscope}%
\pgfsys@transformshift{1.276099in}{0.839738in}%
\pgfsys@useobject{currentmarker}{}%
\end{pgfscope}%
\begin{pgfscope}%
\pgfsys@transformshift{1.516831in}{0.899920in}%
\pgfsys@useobject{currentmarker}{}%
\end{pgfscope}%
\begin{pgfscope}%
\pgfsys@transformshift{1.757562in}{1.020282in}%
\pgfsys@useobject{currentmarker}{}%
\end{pgfscope}%
\begin{pgfscope}%
\pgfsys@transformshift{1.998294in}{1.080464in}%
\pgfsys@useobject{currentmarker}{}%
\end{pgfscope}%
\begin{pgfscope}%
\pgfsys@transformshift{2.239026in}{1.140645in}%
\pgfsys@useobject{currentmarker}{}%
\end{pgfscope}%
\begin{pgfscope}%
\pgfsys@transformshift{2.479757in}{1.200826in}%
\pgfsys@useobject{currentmarker}{}%
\end{pgfscope}%
\begin{pgfscope}%
\pgfsys@transformshift{2.720489in}{1.321189in}%
\pgfsys@useobject{currentmarker}{}%
\end{pgfscope}%
\begin{pgfscope}%
\pgfsys@transformshift{2.961221in}{1.411461in}%
\pgfsys@useobject{currentmarker}{}%
\end{pgfscope}%
\begin{pgfscope}%
\pgfsys@transformshift{3.201952in}{1.471642in}%
\pgfsys@useobject{currentmarker}{}%
\end{pgfscope}%
\begin{pgfscope}%
\pgfsys@transformshift{3.442684in}{1.561914in}%
\pgfsys@useobject{currentmarker}{}%
\end{pgfscope}%
\begin{pgfscope}%
\pgfsys@transformshift{3.683415in}{1.622095in}%
\pgfsys@useobject{currentmarker}{}%
\end{pgfscope}%
\begin{pgfscope}%
\pgfsys@transformshift{3.924147in}{1.742458in}%
\pgfsys@useobject{currentmarker}{}%
\end{pgfscope}%
\begin{pgfscope}%
\pgfsys@transformshift{4.164879in}{1.802639in}%
\pgfsys@useobject{currentmarker}{}%
\end{pgfscope}%
\begin{pgfscope}%
\pgfsys@transformshift{4.405610in}{1.862821in}%
\pgfsys@useobject{currentmarker}{}%
\end{pgfscope}%
\begin{pgfscope}%
\pgfsys@transformshift{4.646342in}{1.923002in}%
\pgfsys@useobject{currentmarker}{}%
\end{pgfscope}%
\begin{pgfscope}%
\pgfsys@transformshift{4.887074in}{2.043365in}%
\pgfsys@useobject{currentmarker}{}%
\end{pgfscope}%
\begin{pgfscope}%
\pgfsys@transformshift{5.127805in}{2.163727in}%
\pgfsys@useobject{currentmarker}{}%
\end{pgfscope}%
\begin{pgfscope}%
\pgfsys@transformshift{5.368537in}{2.223909in}%
\pgfsys@useobject{currentmarker}{}%
\end{pgfscope}%
\begin{pgfscope}%
\pgfsys@transformshift{5.609268in}{2.284090in}%
\pgfsys@useobject{currentmarker}{}%
\end{pgfscope}%
\end{pgfscope}%
\begin{pgfscope}%
\pgfpathrectangle{\pgfqpoint{0.553904in}{0.535823in}}{\pgfqpoint{5.296096in}{2.714177in}}%
\pgfusepath{clip}%
\pgfsetrectcap%
\pgfsetroundjoin%
\pgfsetlinewidth{1.003750pt}%
\definecolor{currentstroke}{rgb}{0.780392,0.643137,0.254902}%
\pgfsetstrokecolor{currentstroke}%
\pgfsetdash{}{0pt}%
\pgfpathmoveto{\pgfqpoint{0.794636in}{0.659194in}}%
\pgfpathlineto{\pgfqpoint{1.035368in}{0.779557in}}%
\pgfpathlineto{\pgfqpoint{1.276099in}{0.839738in}}%
\pgfpathlineto{\pgfqpoint{1.516831in}{0.899920in}}%
\pgfpathlineto{\pgfqpoint{1.757562in}{1.020282in}}%
\pgfpathlineto{\pgfqpoint{1.998294in}{1.080464in}}%
\pgfpathlineto{\pgfqpoint{2.239026in}{1.140645in}}%
\pgfpathlineto{\pgfqpoint{2.479757in}{1.261007in}}%
\pgfpathlineto{\pgfqpoint{2.720489in}{1.321189in}}%
\pgfpathlineto{\pgfqpoint{2.961221in}{1.441551in}}%
\pgfpathlineto{\pgfqpoint{3.201952in}{1.501733in}}%
\pgfpathlineto{\pgfqpoint{3.442684in}{1.561914in}}%
\pgfpathlineto{\pgfqpoint{3.683415in}{1.622095in}}%
\pgfpathlineto{\pgfqpoint{3.924147in}{1.742458in}}%
\pgfpathlineto{\pgfqpoint{4.164879in}{1.802639in}}%
\pgfpathlineto{\pgfqpoint{4.405610in}{1.923002in}}%
\pgfpathlineto{\pgfqpoint{4.646342in}{1.923002in}}%
\pgfpathlineto{\pgfqpoint{4.887074in}{2.043365in}}%
\pgfpathlineto{\pgfqpoint{5.127805in}{2.103546in}}%
\pgfpathlineto{\pgfqpoint{5.368537in}{2.223909in}}%
\pgfpathlineto{\pgfqpoint{5.609268in}{2.344271in}}%
\pgfusepath{stroke}%
\end{pgfscope}%
\begin{pgfscope}%
\pgfpathrectangle{\pgfqpoint{0.553904in}{0.535823in}}{\pgfqpoint{5.296096in}{2.714177in}}%
\pgfusepath{clip}%
\pgfsetbuttcap%
\pgfsetmiterjoin%
\definecolor{currentfill}{rgb}{0.780392,0.643137,0.254902}%
\pgfsetfillcolor{currentfill}%
\pgfsetlinewidth{0.501875pt}%
\definecolor{currentstroke}{rgb}{0.000000,0.000000,0.000000}%
\pgfsetstrokecolor{currentstroke}%
\pgfsetdash{}{0pt}%
\pgfsys@defobject{currentmarker}{\pgfqpoint{-0.034722in}{-0.034722in}}{\pgfqpoint{0.034722in}{0.034722in}}{%
\pgfpathmoveto{\pgfqpoint{0.034722in}{-0.000000in}}%
\pgfpathlineto{\pgfqpoint{-0.034722in}{0.034722in}}%
\pgfpathlineto{\pgfqpoint{-0.034722in}{-0.034722in}}%
\pgfpathclose%
\pgfusepath{stroke,fill}%
}%
\begin{pgfscope}%
\pgfsys@transformshift{0.794636in}{0.659194in}%
\pgfsys@useobject{currentmarker}{}%
\end{pgfscope}%
\begin{pgfscope}%
\pgfsys@transformshift{1.035368in}{0.779557in}%
\pgfsys@useobject{currentmarker}{}%
\end{pgfscope}%
\begin{pgfscope}%
\pgfsys@transformshift{1.276099in}{0.839738in}%
\pgfsys@useobject{currentmarker}{}%
\end{pgfscope}%
\begin{pgfscope}%
\pgfsys@transformshift{1.516831in}{0.899920in}%
\pgfsys@useobject{currentmarker}{}%
\end{pgfscope}%
\begin{pgfscope}%
\pgfsys@transformshift{1.757562in}{1.020282in}%
\pgfsys@useobject{currentmarker}{}%
\end{pgfscope}%
\begin{pgfscope}%
\pgfsys@transformshift{1.998294in}{1.080464in}%
\pgfsys@useobject{currentmarker}{}%
\end{pgfscope}%
\begin{pgfscope}%
\pgfsys@transformshift{2.239026in}{1.140645in}%
\pgfsys@useobject{currentmarker}{}%
\end{pgfscope}%
\begin{pgfscope}%
\pgfsys@transformshift{2.479757in}{1.261007in}%
\pgfsys@useobject{currentmarker}{}%
\end{pgfscope}%
\begin{pgfscope}%
\pgfsys@transformshift{2.720489in}{1.321189in}%
\pgfsys@useobject{currentmarker}{}%
\end{pgfscope}%
\begin{pgfscope}%
\pgfsys@transformshift{2.961221in}{1.441551in}%
\pgfsys@useobject{currentmarker}{}%
\end{pgfscope}%
\begin{pgfscope}%
\pgfsys@transformshift{3.201952in}{1.501733in}%
\pgfsys@useobject{currentmarker}{}%
\end{pgfscope}%
\begin{pgfscope}%
\pgfsys@transformshift{3.442684in}{1.561914in}%
\pgfsys@useobject{currentmarker}{}%
\end{pgfscope}%
\begin{pgfscope}%
\pgfsys@transformshift{3.683415in}{1.622095in}%
\pgfsys@useobject{currentmarker}{}%
\end{pgfscope}%
\begin{pgfscope}%
\pgfsys@transformshift{3.924147in}{1.742458in}%
\pgfsys@useobject{currentmarker}{}%
\end{pgfscope}%
\begin{pgfscope}%
\pgfsys@transformshift{4.164879in}{1.802639in}%
\pgfsys@useobject{currentmarker}{}%
\end{pgfscope}%
\begin{pgfscope}%
\pgfsys@transformshift{4.405610in}{1.923002in}%
\pgfsys@useobject{currentmarker}{}%
\end{pgfscope}%
\begin{pgfscope}%
\pgfsys@transformshift{4.646342in}{1.923002in}%
\pgfsys@useobject{currentmarker}{}%
\end{pgfscope}%
\begin{pgfscope}%
\pgfsys@transformshift{4.887074in}{2.043365in}%
\pgfsys@useobject{currentmarker}{}%
\end{pgfscope}%
\begin{pgfscope}%
\pgfsys@transformshift{5.127805in}{2.103546in}%
\pgfsys@useobject{currentmarker}{}%
\end{pgfscope}%
\begin{pgfscope}%
\pgfsys@transformshift{5.368537in}{2.223909in}%
\pgfsys@useobject{currentmarker}{}%
\end{pgfscope}%
\begin{pgfscope}%
\pgfsys@transformshift{5.609268in}{2.344271in}%
\pgfsys@useobject{currentmarker}{}%
\end{pgfscope}%
\end{pgfscope}%
\begin{pgfscope}%
\pgfpathrectangle{\pgfqpoint{0.553904in}{0.535823in}}{\pgfqpoint{5.296096in}{2.714177in}}%
\pgfusepath{clip}%
\pgfsetrectcap%
\pgfsetroundjoin%
\pgfsetlinewidth{1.003750pt}%
\definecolor{currentstroke}{rgb}{1.000000,0.694118,0.305882}%
\pgfsetstrokecolor{currentstroke}%
\pgfsetdash{}{0pt}%
\pgfpathmoveto{\pgfqpoint{0.794636in}{0.839738in}}%
\pgfpathlineto{\pgfqpoint{1.035368in}{0.960101in}}%
\pgfpathlineto{\pgfqpoint{1.276099in}{1.080464in}}%
\pgfpathlineto{\pgfqpoint{1.516831in}{1.200826in}}%
\pgfpathlineto{\pgfqpoint{1.757562in}{1.321189in}}%
\pgfpathlineto{\pgfqpoint{1.998294in}{1.381370in}}%
\pgfpathlineto{\pgfqpoint{2.239026in}{1.471642in}}%
\pgfpathlineto{\pgfqpoint{2.479757in}{1.501733in}}%
\pgfpathlineto{\pgfqpoint{2.720489in}{1.561914in}}%
\pgfpathlineto{\pgfqpoint{2.961221in}{1.561914in}}%
\pgfpathlineto{\pgfqpoint{3.201952in}{1.561914in}}%
\pgfpathlineto{\pgfqpoint{3.442684in}{1.561914in}}%
\pgfpathlineto{\pgfqpoint{3.683415in}{1.561914in}}%
\pgfpathlineto{\pgfqpoint{3.924147in}{1.622095in}}%
\pgfpathlineto{\pgfqpoint{4.164879in}{1.622095in}}%
\pgfpathlineto{\pgfqpoint{4.405610in}{1.682277in}}%
\pgfpathlineto{\pgfqpoint{4.646342in}{1.742458in}}%
\pgfpathlineto{\pgfqpoint{4.887074in}{1.802639in}}%
\pgfpathlineto{\pgfqpoint{5.127805in}{1.923002in}}%
\pgfpathlineto{\pgfqpoint{5.368537in}{1.953093in}}%
\pgfpathlineto{\pgfqpoint{5.609268in}{2.043365in}}%
\pgfusepath{stroke}%
\end{pgfscope}%
\begin{pgfscope}%
\pgfpathrectangle{\pgfqpoint{0.553904in}{0.535823in}}{\pgfqpoint{5.296096in}{2.714177in}}%
\pgfusepath{clip}%
\pgfsetbuttcap%
\pgfsetbeveljoin%
\definecolor{currentfill}{rgb}{1.000000,0.694118,0.305882}%
\pgfsetfillcolor{currentfill}%
\pgfsetlinewidth{0.501875pt}%
\definecolor{currentstroke}{rgb}{0.000000,0.000000,0.000000}%
\pgfsetstrokecolor{currentstroke}%
\pgfsetdash{}{0pt}%
\pgfsys@defobject{currentmarker}{\pgfqpoint{-0.033023in}{-0.028091in}}{\pgfqpoint{0.033023in}{0.034722in}}{%
\pgfpathmoveto{\pgfqpoint{0.000000in}{0.034722in}}%
\pgfpathlineto{\pgfqpoint{-0.007796in}{0.010730in}}%
\pgfpathlineto{\pgfqpoint{-0.033023in}{0.010730in}}%
\pgfpathlineto{\pgfqpoint{-0.012614in}{-0.004098in}}%
\pgfpathlineto{\pgfqpoint{-0.020409in}{-0.028091in}}%
\pgfpathlineto{\pgfqpoint{-0.000000in}{-0.013263in}}%
\pgfpathlineto{\pgfqpoint{0.020409in}{-0.028091in}}%
\pgfpathlineto{\pgfqpoint{0.012614in}{-0.004098in}}%
\pgfpathlineto{\pgfqpoint{0.033023in}{0.010730in}}%
\pgfpathlineto{\pgfqpoint{0.007796in}{0.010730in}}%
\pgfpathclose%
\pgfusepath{stroke,fill}%
}%
\begin{pgfscope}%
\pgfsys@transformshift{0.794636in}{0.839738in}%
\pgfsys@useobject{currentmarker}{}%
\end{pgfscope}%
\begin{pgfscope}%
\pgfsys@transformshift{1.035368in}{0.960101in}%
\pgfsys@useobject{currentmarker}{}%
\end{pgfscope}%
\begin{pgfscope}%
\pgfsys@transformshift{1.276099in}{1.080464in}%
\pgfsys@useobject{currentmarker}{}%
\end{pgfscope}%
\begin{pgfscope}%
\pgfsys@transformshift{1.516831in}{1.200826in}%
\pgfsys@useobject{currentmarker}{}%
\end{pgfscope}%
\begin{pgfscope}%
\pgfsys@transformshift{1.757562in}{1.321189in}%
\pgfsys@useobject{currentmarker}{}%
\end{pgfscope}%
\begin{pgfscope}%
\pgfsys@transformshift{1.998294in}{1.381370in}%
\pgfsys@useobject{currentmarker}{}%
\end{pgfscope}%
\begin{pgfscope}%
\pgfsys@transformshift{2.239026in}{1.471642in}%
\pgfsys@useobject{currentmarker}{}%
\end{pgfscope}%
\begin{pgfscope}%
\pgfsys@transformshift{2.479757in}{1.501733in}%
\pgfsys@useobject{currentmarker}{}%
\end{pgfscope}%
\begin{pgfscope}%
\pgfsys@transformshift{2.720489in}{1.561914in}%
\pgfsys@useobject{currentmarker}{}%
\end{pgfscope}%
\begin{pgfscope}%
\pgfsys@transformshift{2.961221in}{1.561914in}%
\pgfsys@useobject{currentmarker}{}%
\end{pgfscope}%
\begin{pgfscope}%
\pgfsys@transformshift{3.201952in}{1.561914in}%
\pgfsys@useobject{currentmarker}{}%
\end{pgfscope}%
\begin{pgfscope}%
\pgfsys@transformshift{3.442684in}{1.561914in}%
\pgfsys@useobject{currentmarker}{}%
\end{pgfscope}%
\begin{pgfscope}%
\pgfsys@transformshift{3.683415in}{1.561914in}%
\pgfsys@useobject{currentmarker}{}%
\end{pgfscope}%
\begin{pgfscope}%
\pgfsys@transformshift{3.924147in}{1.622095in}%
\pgfsys@useobject{currentmarker}{}%
\end{pgfscope}%
\begin{pgfscope}%
\pgfsys@transformshift{4.164879in}{1.622095in}%
\pgfsys@useobject{currentmarker}{}%
\end{pgfscope}%
\begin{pgfscope}%
\pgfsys@transformshift{4.405610in}{1.682277in}%
\pgfsys@useobject{currentmarker}{}%
\end{pgfscope}%
\begin{pgfscope}%
\pgfsys@transformshift{4.646342in}{1.742458in}%
\pgfsys@useobject{currentmarker}{}%
\end{pgfscope}%
\begin{pgfscope}%
\pgfsys@transformshift{4.887074in}{1.802639in}%
\pgfsys@useobject{currentmarker}{}%
\end{pgfscope}%
\begin{pgfscope}%
\pgfsys@transformshift{5.127805in}{1.923002in}%
\pgfsys@useobject{currentmarker}{}%
\end{pgfscope}%
\begin{pgfscope}%
\pgfsys@transformshift{5.368537in}{1.953093in}%
\pgfsys@useobject{currentmarker}{}%
\end{pgfscope}%
\begin{pgfscope}%
\pgfsys@transformshift{5.609268in}{2.043365in}%
\pgfsys@useobject{currentmarker}{}%
\end{pgfscope}%
\end{pgfscope}%
\begin{pgfscope}%
\pgfsetrectcap%
\pgfsetmiterjoin%
\pgfsetlinewidth{0.803000pt}%
\definecolor{currentstroke}{rgb}{0.000000,0.000000,0.000000}%
\pgfsetstrokecolor{currentstroke}%
\pgfsetdash{}{0pt}%
\pgfpathmoveto{\pgfqpoint{0.553904in}{0.535823in}}%
\pgfpathlineto{\pgfqpoint{0.553904in}{3.250000in}}%
\pgfusepath{stroke}%
\end{pgfscope}%
\begin{pgfscope}%
\pgfsetrectcap%
\pgfsetmiterjoin%
\pgfsetlinewidth{0.803000pt}%
\definecolor{currentstroke}{rgb}{0.000000,0.000000,0.000000}%
\pgfsetstrokecolor{currentstroke}%
\pgfsetdash{}{0pt}%
\pgfpathmoveto{\pgfqpoint{5.850000in}{0.535823in}}%
\pgfpathlineto{\pgfqpoint{5.850000in}{3.250000in}}%
\pgfusepath{stroke}%
\end{pgfscope}%
\begin{pgfscope}%
\pgfsetrectcap%
\pgfsetmiterjoin%
\pgfsetlinewidth{0.803000pt}%
\definecolor{currentstroke}{rgb}{0.000000,0.000000,0.000000}%
\pgfsetstrokecolor{currentstroke}%
\pgfsetdash{}{0pt}%
\pgfpathmoveto{\pgfqpoint{0.553904in}{0.535823in}}%
\pgfpathlineto{\pgfqpoint{5.850000in}{0.535823in}}%
\pgfusepath{stroke}%
\end{pgfscope}%
\begin{pgfscope}%
\pgfsetrectcap%
\pgfsetmiterjoin%
\pgfsetlinewidth{0.803000pt}%
\definecolor{currentstroke}{rgb}{0.000000,0.000000,0.000000}%
\pgfsetstrokecolor{currentstroke}%
\pgfsetdash{}{0pt}%
\pgfpathmoveto{\pgfqpoint{0.553904in}{3.250000in}}%
\pgfpathlineto{\pgfqpoint{5.850000in}{3.250000in}}%
\pgfusepath{stroke}%
\end{pgfscope}%
\begin{pgfscope}%
\pgfsetrectcap%
\pgfsetroundjoin%
\pgfsetlinewidth{1.003750pt}%
\definecolor{currentstroke}{rgb}{0.529412,0.462745,0.384314}%
\pgfsetstrokecolor{currentstroke}%
\pgfsetdash{}{0pt}%
\pgfpathmoveto{\pgfqpoint{0.603904in}{3.156250in}}%
\pgfpathlineto{\pgfqpoint{0.853904in}{3.156250in}}%
\pgfusepath{stroke}%
\end{pgfscope}%
\begin{pgfscope}%
\pgfsetbuttcap%
\pgfsetmiterjoin%
\definecolor{currentfill}{rgb}{0.529412,0.462745,0.384314}%
\pgfsetfillcolor{currentfill}%
\pgfsetlinewidth{0.501875pt}%
\definecolor{currentstroke}{rgb}{0.000000,0.000000,0.000000}%
\pgfsetstrokecolor{currentstroke}%
\pgfsetdash{}{0pt}%
\pgfsys@defobject{currentmarker}{\pgfqpoint{-0.034722in}{-0.034722in}}{\pgfqpoint{0.034722in}{0.034722in}}{%
\pgfpathmoveto{\pgfqpoint{-0.000000in}{-0.034722in}}%
\pgfpathlineto{\pgfqpoint{0.034722in}{0.034722in}}%
\pgfpathlineto{\pgfqpoint{-0.034722in}{0.034722in}}%
\pgfpathclose%
\pgfusepath{stroke,fill}%
}%
\begin{pgfscope}%
\pgfsys@transformshift{0.728904in}{3.156250in}%
\pgfsys@useobject{currentmarker}{}%
\end{pgfscope}%
\end{pgfscope}%
\begin{pgfscope}%
\definecolor{textcolor}{rgb}{0.000000,0.000000,0.000000}%
\pgfsetstrokecolor{textcolor}%
\pgfsetfillcolor{textcolor}%
\pgftext[x=0.878904in,y=3.112500in,left,base]{\color{textcolor}\rmfamily\fontsize{9.000000}{10.800000}\selectfont greedy}%
\end{pgfscope}%
\begin{pgfscope}%
\pgfsetrectcap%
\pgfsetroundjoin%
\pgfsetlinewidth{1.003750pt}%
\definecolor{currentstroke}{rgb}{0.611765,0.568627,0.274510}%
\pgfsetstrokecolor{currentstroke}%
\pgfsetdash{}{0pt}%
\pgfpathmoveto{\pgfqpoint{0.603904in}{2.994450in}}%
\pgfpathlineto{\pgfqpoint{0.853904in}{2.994450in}}%
\pgfusepath{stroke}%
\end{pgfscope}%
\begin{pgfscope}%
\pgfsetbuttcap%
\pgfsetmiterjoin%
\definecolor{currentfill}{rgb}{0.611765,0.568627,0.274510}%
\pgfsetfillcolor{currentfill}%
\pgfsetlinewidth{0.501875pt}%
\definecolor{currentstroke}{rgb}{0.000000,0.000000,0.000000}%
\pgfsetstrokecolor{currentstroke}%
\pgfsetdash{}{0pt}%
\pgfsys@defobject{currentmarker}{\pgfqpoint{-0.034722in}{-0.034722in}}{\pgfqpoint{0.034722in}{0.034722in}}{%
\pgfpathmoveto{\pgfqpoint{-0.034722in}{0.000000in}}%
\pgfpathlineto{\pgfqpoint{0.034722in}{-0.034722in}}%
\pgfpathlineto{\pgfqpoint{0.034722in}{0.034722in}}%
\pgfpathclose%
\pgfusepath{stroke,fill}%
}%
\begin{pgfscope}%
\pgfsys@transformshift{0.728904in}{2.994450in}%
\pgfsys@useobject{currentmarker}{}%
\end{pgfscope}%
\end{pgfscope}%
\begin{pgfscope}%
\definecolor{textcolor}{rgb}{0.000000,0.000000,0.000000}%
\pgfsetstrokecolor{textcolor}%
\pgfsetfillcolor{textcolor}%
\pgftext[x=0.878904in,y=2.950700in,left,base]{\color{textcolor}\rmfamily\fontsize{9.000000}{10.800000}\selectfont metis}%
\end{pgfscope}%
\begin{pgfscope}%
\pgfsetrectcap%
\pgfsetroundjoin%
\pgfsetlinewidth{1.003750pt}%
\definecolor{currentstroke}{rgb}{0.780392,0.643137,0.254902}%
\pgfsetstrokecolor{currentstroke}%
\pgfsetdash{}{0pt}%
\pgfpathmoveto{\pgfqpoint{0.603904in}{2.832651in}}%
\pgfpathlineto{\pgfqpoint{0.853904in}{2.832651in}}%
\pgfusepath{stroke}%
\end{pgfscope}%
\begin{pgfscope}%
\pgfsetbuttcap%
\pgfsetmiterjoin%
\definecolor{currentfill}{rgb}{0.780392,0.643137,0.254902}%
\pgfsetfillcolor{currentfill}%
\pgfsetlinewidth{0.501875pt}%
\definecolor{currentstroke}{rgb}{0.000000,0.000000,0.000000}%
\pgfsetstrokecolor{currentstroke}%
\pgfsetdash{}{0pt}%
\pgfsys@defobject{currentmarker}{\pgfqpoint{-0.034722in}{-0.034722in}}{\pgfqpoint{0.034722in}{0.034722in}}{%
\pgfpathmoveto{\pgfqpoint{0.034722in}{-0.000000in}}%
\pgfpathlineto{\pgfqpoint{-0.034722in}{0.034722in}}%
\pgfpathlineto{\pgfqpoint{-0.034722in}{-0.034722in}}%
\pgfpathclose%
\pgfusepath{stroke,fill}%
}%
\begin{pgfscope}%
\pgfsys@transformshift{0.728904in}{2.832651in}%
\pgfsys@useobject{currentmarker}{}%
\end{pgfscope}%
\end{pgfscope}%
\begin{pgfscope}%
\definecolor{textcolor}{rgb}{0.000000,0.000000,0.000000}%
\pgfsetstrokecolor{textcolor}%
\pgfsetfillcolor{textcolor}%
\pgftext[x=0.878904in,y=2.788901in,left,base]{\color{textcolor}\rmfamily\fontsize{9.000000}{10.800000}\selectfont GN}%
\end{pgfscope}%
\begin{pgfscope}%
\pgfsetrectcap%
\pgfsetroundjoin%
\pgfsetlinewidth{1.003750pt}%
\definecolor{currentstroke}{rgb}{1.000000,0.694118,0.305882}%
\pgfsetstrokecolor{currentstroke}%
\pgfsetdash{}{0pt}%
\pgfpathmoveto{\pgfqpoint{0.603904in}{2.670851in}}%
\pgfpathlineto{\pgfqpoint{0.853904in}{2.670851in}}%
\pgfusepath{stroke}%
\end{pgfscope}%
\begin{pgfscope}%
\pgfsetbuttcap%
\pgfsetbeveljoin%
\definecolor{currentfill}{rgb}{1.000000,0.694118,0.305882}%
\pgfsetfillcolor{currentfill}%
\pgfsetlinewidth{0.501875pt}%
\definecolor{currentstroke}{rgb}{0.000000,0.000000,0.000000}%
\pgfsetstrokecolor{currentstroke}%
\pgfsetdash{}{0pt}%
\pgfsys@defobject{currentmarker}{\pgfqpoint{-0.033023in}{-0.028091in}}{\pgfqpoint{0.033023in}{0.034722in}}{%
\pgfpathmoveto{\pgfqpoint{0.000000in}{0.034722in}}%
\pgfpathlineto{\pgfqpoint{-0.007796in}{0.010730in}}%
\pgfpathlineto{\pgfqpoint{-0.033023in}{0.010730in}}%
\pgfpathlineto{\pgfqpoint{-0.012614in}{-0.004098in}}%
\pgfpathlineto{\pgfqpoint{-0.020409in}{-0.028091in}}%
\pgfpathlineto{\pgfqpoint{-0.000000in}{-0.013263in}}%
\pgfpathlineto{\pgfqpoint{0.020409in}{-0.028091in}}%
\pgfpathlineto{\pgfqpoint{0.012614in}{-0.004098in}}%
\pgfpathlineto{\pgfqpoint{0.033023in}{0.010730in}}%
\pgfpathlineto{\pgfqpoint{0.007796in}{0.010730in}}%
\pgfpathclose%
\pgfusepath{stroke,fill}%
}%
\begin{pgfscope}%
\pgfsys@transformshift{0.728904in}{2.670851in}%
\pgfsys@useobject{currentmarker}{}%
\end{pgfscope}%
\end{pgfscope}%
\begin{pgfscope}%
\definecolor{textcolor}{rgb}{0.000000,0.000000,0.000000}%
\pgfsetstrokecolor{textcolor}%
\pgfsetfillcolor{textcolor}%
\pgftext[x=0.878904in,y=2.627101in,left,base]{\color{textcolor}\rmfamily\fontsize{9.000000}{10.800000}\selectfont LG+Flow}%
\end{pgfscope}%
\end{pgfpicture}%
\makeatother%
\endgroup%

	\caption{\label{fig:cubic-time} Median solving time (top) and max-rank of the computed contraction tree (bottom) of various counters and tensor-based methods, run on benchmarks counting the number of vertex covers of 100 cubic graphs with $n$ vertices. Solving time of datapoints that ran out of time ($1000$ seconds) or memory (48 GB) are not shown. When $n \geq 170$, our contribution \textbf{LG+Flow} is faster than all other methods and finds contraction trees of lower max-rank than all other tensor-based methods.}
\end{figure}

\subsection{Implementation Details of \tool{TensorOrder}}
\label{sec:tensors:experiments:implementation}
\tool{TensorOrder} is implemented in Python 3.6. All tensor contractions are performed using \pkg{numpy} 1.15 and 64-bit double precision floats. \tool{TensorOrder} also supports infinite-precision integer arithmetic, but the performance is significantly degraded by limited \pkg{numpy} support. Note that \pkg{numpy} is able to leverage SIMD parallelism for tensor contraction.

Both \textbf{LG} and \textbf{FT} require first finding a tree decomposition. To do this, we leverage three heuristic tree-decomposition solvers: \pkg{Tamaki} \cite{Tamaki17}, \pkg{FlowCutter} \cite{HS18}, and \pkg{htd} \cite{AMW17}. \tool{TensorOrder} therefore has three implementations of \textbf{LG} (using \textbf{LG+Tamaki}, \textbf{LG+Flow}, and \textbf{LG+htd}) and three implementations of \textbf{FT} (using \textbf{FT+Tamaki}, \textbf{FT+Flow}, and \textbf{FT+htd}) for different choices of solver.

All the tree-decomposition solvers we consider are anytime solvers and so each implementation must decide how long to run the solver (this time is included in the measured running time). 
In Algorithm \ref{alg:wmc}, this is governed by the parameter $\alpha$.
\tool{TensorOrder} estimates the time to contract each potential contraction tree (using techniques from the \pkg{einsum} package of \pkg{numpy}) and configures $\alpha$ so that it continues to look for better tree decompositions until it expects to have spent more than half of the running time on finding a tree decomposition.  This strikes a balance between improving and using the contraction trees.
% Note that we determine $\alpha$ empirically in Chapter \ref{ch:parallel}.

\begin{figure}
	\centering
	%% Creator: Matplotlib, PGF backend
%%
%% To include the figure in your LaTeX document, write
%%   \input{<filename>.pgf}
%%
%% Make sure the required packages are loaded in your preamble
%%   \usepackage{pgf}
%%
%% and, on pdftex
%%   \usepackage[utf8]{inputenc}\DeclareUnicodeCharacter{2212}{-}
%%
%% or, on luatex and xetex
%%   \usepackage{unicode-math}
%%
%% Figures using additional raster images can only be included by \input if
%% they are in the same directory as the main LaTeX file. For loading figures
%% from other directories you can use the `import` package
%%   \usepackage{import}
%%
%% and then include the figures with
%%   \import{<path to file>}{<filename>.pgf}
%%
%% Matplotlib used the following preamble
%%   \usepackage[utf8x]{inputenc}
%%   \usepackage[T1]{fontenc}
%%
\begingroup%
\makeatletter%
\begin{pgfpicture}%
\pgfpathrectangle{\pgfpointorigin}{\pgfqpoint{6.000000in}{3.700000in}}%
\pgfusepath{use as bounding box, clip}%
\begin{pgfscope}%
\pgfsetbuttcap%
\pgfsetmiterjoin%
\definecolor{currentfill}{rgb}{1.000000,1.000000,1.000000}%
\pgfsetfillcolor{currentfill}%
\pgfsetlinewidth{0.000000pt}%
\definecolor{currentstroke}{rgb}{1.000000,1.000000,1.000000}%
\pgfsetstrokecolor{currentstroke}%
\pgfsetdash{}{0pt}%
\pgfpathmoveto{\pgfqpoint{0.000000in}{0.000000in}}%
\pgfpathlineto{\pgfqpoint{6.000000in}{0.000000in}}%
\pgfpathlineto{\pgfqpoint{6.000000in}{3.700000in}}%
\pgfpathlineto{\pgfqpoint{0.000000in}{3.700000in}}%
\pgfpathclose%
\pgfusepath{fill}%
\end{pgfscope}%
\begin{pgfscope}%
\pgfsetbuttcap%
\pgfsetmiterjoin%
\definecolor{currentfill}{rgb}{1.000000,1.000000,1.000000}%
\pgfsetfillcolor{currentfill}%
\pgfsetlinewidth{0.000000pt}%
\definecolor{currentstroke}{rgb}{0.000000,0.000000,0.000000}%
\pgfsetstrokecolor{currentstroke}%
\pgfsetstrokeopacity{0.000000}%
\pgfsetdash{}{0pt}%
\pgfpathmoveto{\pgfqpoint{0.553904in}{0.535823in}}%
\pgfpathlineto{\pgfqpoint{5.850000in}{0.535823in}}%
\pgfpathlineto{\pgfqpoint{5.850000in}{3.550000in}}%
\pgfpathlineto{\pgfqpoint{0.553904in}{3.550000in}}%
\pgfpathclose%
\pgfusepath{fill}%
\end{pgfscope}%
\begin{pgfscope}%
\pgfsetbuttcap%
\pgfsetroundjoin%
\definecolor{currentfill}{rgb}{0.000000,0.000000,0.000000}%
\pgfsetfillcolor{currentfill}%
\pgfsetlinewidth{0.803000pt}%
\definecolor{currentstroke}{rgb}{0.000000,0.000000,0.000000}%
\pgfsetstrokecolor{currentstroke}%
\pgfsetdash{}{0pt}%
\pgfsys@defobject{currentmarker}{\pgfqpoint{0.000000in}{-0.048611in}}{\pgfqpoint{0.000000in}{0.000000in}}{%
\pgfpathmoveto{\pgfqpoint{0.000000in}{0.000000in}}%
\pgfpathlineto{\pgfqpoint{0.000000in}{-0.048611in}}%
\pgfusepath{stroke,fill}%
}%
\begin{pgfscope}%
\pgfsys@transformshift{0.794636in}{0.535823in}%
\pgfsys@useobject{currentmarker}{}%
\end{pgfscope}%
\end{pgfscope}%
\begin{pgfscope}%
\definecolor{textcolor}{rgb}{0.000000,0.000000,0.000000}%
\pgfsetstrokecolor{textcolor}%
\pgfsetfillcolor{textcolor}%
\pgftext[x=0.794636in,y=0.438600in,,top]{\color{textcolor}\rmfamily\fontsize{9.000000}{10.800000}\selectfont \(\displaystyle {50}\)}%
\end{pgfscope}%
\begin{pgfscope}%
\pgfsetbuttcap%
\pgfsetroundjoin%
\definecolor{currentfill}{rgb}{0.000000,0.000000,0.000000}%
\pgfsetfillcolor{currentfill}%
\pgfsetlinewidth{0.803000pt}%
\definecolor{currentstroke}{rgb}{0.000000,0.000000,0.000000}%
\pgfsetstrokecolor{currentstroke}%
\pgfsetdash{}{0pt}%
\pgfsys@defobject{currentmarker}{\pgfqpoint{0.000000in}{-0.048611in}}{\pgfqpoint{0.000000in}{0.000000in}}{%
\pgfpathmoveto{\pgfqpoint{0.000000in}{0.000000in}}%
\pgfpathlineto{\pgfqpoint{0.000000in}{-0.048611in}}%
\pgfusepath{stroke,fill}%
}%
\begin{pgfscope}%
\pgfsys@transformshift{1.998294in}{0.535823in}%
\pgfsys@useobject{currentmarker}{}%
\end{pgfscope}%
\end{pgfscope}%
\begin{pgfscope}%
\definecolor{textcolor}{rgb}{0.000000,0.000000,0.000000}%
\pgfsetstrokecolor{textcolor}%
\pgfsetfillcolor{textcolor}%
\pgftext[x=1.998294in,y=0.438600in,,top]{\color{textcolor}\rmfamily\fontsize{9.000000}{10.800000}\selectfont \(\displaystyle {100}\)}%
\end{pgfscope}%
\begin{pgfscope}%
\pgfsetbuttcap%
\pgfsetroundjoin%
\definecolor{currentfill}{rgb}{0.000000,0.000000,0.000000}%
\pgfsetfillcolor{currentfill}%
\pgfsetlinewidth{0.803000pt}%
\definecolor{currentstroke}{rgb}{0.000000,0.000000,0.000000}%
\pgfsetstrokecolor{currentstroke}%
\pgfsetdash{}{0pt}%
\pgfsys@defobject{currentmarker}{\pgfqpoint{0.000000in}{-0.048611in}}{\pgfqpoint{0.000000in}{0.000000in}}{%
\pgfpathmoveto{\pgfqpoint{0.000000in}{0.000000in}}%
\pgfpathlineto{\pgfqpoint{0.000000in}{-0.048611in}}%
\pgfusepath{stroke,fill}%
}%
\begin{pgfscope}%
\pgfsys@transformshift{3.201952in}{0.535823in}%
\pgfsys@useobject{currentmarker}{}%
\end{pgfscope}%
\end{pgfscope}%
\begin{pgfscope}%
\definecolor{textcolor}{rgb}{0.000000,0.000000,0.000000}%
\pgfsetstrokecolor{textcolor}%
\pgfsetfillcolor{textcolor}%
\pgftext[x=3.201952in,y=0.438600in,,top]{\color{textcolor}\rmfamily\fontsize{9.000000}{10.800000}\selectfont \(\displaystyle {150}\)}%
\end{pgfscope}%
\begin{pgfscope}%
\pgfsetbuttcap%
\pgfsetroundjoin%
\definecolor{currentfill}{rgb}{0.000000,0.000000,0.000000}%
\pgfsetfillcolor{currentfill}%
\pgfsetlinewidth{0.803000pt}%
\definecolor{currentstroke}{rgb}{0.000000,0.000000,0.000000}%
\pgfsetstrokecolor{currentstroke}%
\pgfsetdash{}{0pt}%
\pgfsys@defobject{currentmarker}{\pgfqpoint{0.000000in}{-0.048611in}}{\pgfqpoint{0.000000in}{0.000000in}}{%
\pgfpathmoveto{\pgfqpoint{0.000000in}{0.000000in}}%
\pgfpathlineto{\pgfqpoint{0.000000in}{-0.048611in}}%
\pgfusepath{stroke,fill}%
}%
\begin{pgfscope}%
\pgfsys@transformshift{4.405610in}{0.535823in}%
\pgfsys@useobject{currentmarker}{}%
\end{pgfscope}%
\end{pgfscope}%
\begin{pgfscope}%
\definecolor{textcolor}{rgb}{0.000000,0.000000,0.000000}%
\pgfsetstrokecolor{textcolor}%
\pgfsetfillcolor{textcolor}%
\pgftext[x=4.405610in,y=0.438600in,,top]{\color{textcolor}\rmfamily\fontsize{9.000000}{10.800000}\selectfont \(\displaystyle {200}\)}%
\end{pgfscope}%
\begin{pgfscope}%
\pgfsetbuttcap%
\pgfsetroundjoin%
\definecolor{currentfill}{rgb}{0.000000,0.000000,0.000000}%
\pgfsetfillcolor{currentfill}%
\pgfsetlinewidth{0.803000pt}%
\definecolor{currentstroke}{rgb}{0.000000,0.000000,0.000000}%
\pgfsetstrokecolor{currentstroke}%
\pgfsetdash{}{0pt}%
\pgfsys@defobject{currentmarker}{\pgfqpoint{0.000000in}{-0.048611in}}{\pgfqpoint{0.000000in}{0.000000in}}{%
\pgfpathmoveto{\pgfqpoint{0.000000in}{0.000000in}}%
\pgfpathlineto{\pgfqpoint{0.000000in}{-0.048611in}}%
\pgfusepath{stroke,fill}%
}%
\begin{pgfscope}%
\pgfsys@transformshift{5.609268in}{0.535823in}%
\pgfsys@useobject{currentmarker}{}%
\end{pgfscope}%
\end{pgfscope}%
\begin{pgfscope}%
\definecolor{textcolor}{rgb}{0.000000,0.000000,0.000000}%
\pgfsetstrokecolor{textcolor}%
\pgfsetfillcolor{textcolor}%
\pgftext[x=5.609268in,y=0.438600in,,top]{\color{textcolor}\rmfamily\fontsize{9.000000}{10.800000}\selectfont \(\displaystyle {250}\)}%
\end{pgfscope}%
\begin{pgfscope}%
\definecolor{textcolor}{rgb}{0.000000,0.000000,0.000000}%
\pgfsetstrokecolor{textcolor}%
\pgfsetfillcolor{textcolor}%
\pgftext[x=3.201952in,y=0.272655in,,top]{\color{textcolor}\rmfamily\fontsize{10.000000}{12.000000}\selectfont \(\displaystyle n\): Number of vertices}%
\end{pgfscope}%
\begin{pgfscope}%
\pgfsetbuttcap%
\pgfsetroundjoin%
\definecolor{currentfill}{rgb}{0.000000,0.000000,0.000000}%
\pgfsetfillcolor{currentfill}%
\pgfsetlinewidth{0.803000pt}%
\definecolor{currentstroke}{rgb}{0.000000,0.000000,0.000000}%
\pgfsetstrokecolor{currentstroke}%
\pgfsetdash{}{0pt}%
\pgfsys@defobject{currentmarker}{\pgfqpoint{-0.048611in}{0.000000in}}{\pgfqpoint{-0.000000in}{0.000000in}}{%
\pgfpathmoveto{\pgfqpoint{-0.000000in}{0.000000in}}%
\pgfpathlineto{\pgfqpoint{-0.048611in}{0.000000in}}%
\pgfusepath{stroke,fill}%
}%
\begin{pgfscope}%
\pgfsys@transformshift{0.553904in}{0.865123in}%
\pgfsys@useobject{currentmarker}{}%
\end{pgfscope}%
\end{pgfscope}%
\begin{pgfscope}%
\definecolor{textcolor}{rgb}{0.000000,0.000000,0.000000}%
\pgfsetstrokecolor{textcolor}%
\pgfsetfillcolor{textcolor}%
\pgftext[x=0.328211in, y=0.822078in, left, base]{\color{textcolor}\rmfamily\fontsize{9.000000}{10.800000}\selectfont \(\displaystyle {10}\)}%
\end{pgfscope}%
\begin{pgfscope}%
\pgfsetbuttcap%
\pgfsetroundjoin%
\definecolor{currentfill}{rgb}{0.000000,0.000000,0.000000}%
\pgfsetfillcolor{currentfill}%
\pgfsetlinewidth{0.803000pt}%
\definecolor{currentstroke}{rgb}{0.000000,0.000000,0.000000}%
\pgfsetstrokecolor{currentstroke}%
\pgfsetdash{}{0pt}%
\pgfsys@defobject{currentmarker}{\pgfqpoint{-0.048611in}{0.000000in}}{\pgfqpoint{-0.000000in}{0.000000in}}{%
\pgfpathmoveto{\pgfqpoint{-0.000000in}{0.000000in}}%
\pgfpathlineto{\pgfqpoint{-0.048611in}{0.000000in}}%
\pgfusepath{stroke,fill}%
}%
\begin{pgfscope}%
\pgfsys@transformshift{0.553904in}{1.345853in}%
\pgfsys@useobject{currentmarker}{}%
\end{pgfscope}%
\end{pgfscope}%
\begin{pgfscope}%
\definecolor{textcolor}{rgb}{0.000000,0.000000,0.000000}%
\pgfsetstrokecolor{textcolor}%
\pgfsetfillcolor{textcolor}%
\pgftext[x=0.328211in, y=1.302808in, left, base]{\color{textcolor}\rmfamily\fontsize{9.000000}{10.800000}\selectfont \(\displaystyle {15}\)}%
\end{pgfscope}%
\begin{pgfscope}%
\pgfsetbuttcap%
\pgfsetroundjoin%
\definecolor{currentfill}{rgb}{0.000000,0.000000,0.000000}%
\pgfsetfillcolor{currentfill}%
\pgfsetlinewidth{0.803000pt}%
\definecolor{currentstroke}{rgb}{0.000000,0.000000,0.000000}%
\pgfsetstrokecolor{currentstroke}%
\pgfsetdash{}{0pt}%
\pgfsys@defobject{currentmarker}{\pgfqpoint{-0.048611in}{0.000000in}}{\pgfqpoint{-0.000000in}{0.000000in}}{%
\pgfpathmoveto{\pgfqpoint{-0.000000in}{0.000000in}}%
\pgfpathlineto{\pgfqpoint{-0.048611in}{0.000000in}}%
\pgfusepath{stroke,fill}%
}%
\begin{pgfscope}%
\pgfsys@transformshift{0.553904in}{1.826583in}%
\pgfsys@useobject{currentmarker}{}%
\end{pgfscope}%
\end{pgfscope}%
\begin{pgfscope}%
\definecolor{textcolor}{rgb}{0.000000,0.000000,0.000000}%
\pgfsetstrokecolor{textcolor}%
\pgfsetfillcolor{textcolor}%
\pgftext[x=0.328211in, y=1.783538in, left, base]{\color{textcolor}\rmfamily\fontsize{9.000000}{10.800000}\selectfont \(\displaystyle {20}\)}%
\end{pgfscope}%
\begin{pgfscope}%
\pgfsetbuttcap%
\pgfsetroundjoin%
\definecolor{currentfill}{rgb}{0.000000,0.000000,0.000000}%
\pgfsetfillcolor{currentfill}%
\pgfsetlinewidth{0.803000pt}%
\definecolor{currentstroke}{rgb}{0.000000,0.000000,0.000000}%
\pgfsetstrokecolor{currentstroke}%
\pgfsetdash{}{0pt}%
\pgfsys@defobject{currentmarker}{\pgfqpoint{-0.048611in}{0.000000in}}{\pgfqpoint{-0.000000in}{0.000000in}}{%
\pgfpathmoveto{\pgfqpoint{-0.000000in}{0.000000in}}%
\pgfpathlineto{\pgfqpoint{-0.048611in}{0.000000in}}%
\pgfusepath{stroke,fill}%
}%
\begin{pgfscope}%
\pgfsys@transformshift{0.553904in}{2.307313in}%
\pgfsys@useobject{currentmarker}{}%
\end{pgfscope}%
\end{pgfscope}%
\begin{pgfscope}%
\definecolor{textcolor}{rgb}{0.000000,0.000000,0.000000}%
\pgfsetstrokecolor{textcolor}%
\pgfsetfillcolor{textcolor}%
\pgftext[x=0.328211in, y=2.264268in, left, base]{\color{textcolor}\rmfamily\fontsize{9.000000}{10.800000}\selectfont \(\displaystyle {25}\)}%
\end{pgfscope}%
\begin{pgfscope}%
\pgfsetbuttcap%
\pgfsetroundjoin%
\definecolor{currentfill}{rgb}{0.000000,0.000000,0.000000}%
\pgfsetfillcolor{currentfill}%
\pgfsetlinewidth{0.803000pt}%
\definecolor{currentstroke}{rgb}{0.000000,0.000000,0.000000}%
\pgfsetstrokecolor{currentstroke}%
\pgfsetdash{}{0pt}%
\pgfsys@defobject{currentmarker}{\pgfqpoint{-0.048611in}{0.000000in}}{\pgfqpoint{-0.000000in}{0.000000in}}{%
\pgfpathmoveto{\pgfqpoint{-0.000000in}{0.000000in}}%
\pgfpathlineto{\pgfqpoint{-0.048611in}{0.000000in}}%
\pgfusepath{stroke,fill}%
}%
\begin{pgfscope}%
\pgfsys@transformshift{0.553904in}{2.788043in}%
\pgfsys@useobject{currentmarker}{}%
\end{pgfscope}%
\end{pgfscope}%
\begin{pgfscope}%
\definecolor{textcolor}{rgb}{0.000000,0.000000,0.000000}%
\pgfsetstrokecolor{textcolor}%
\pgfsetfillcolor{textcolor}%
\pgftext[x=0.328211in, y=2.744998in, left, base]{\color{textcolor}\rmfamily\fontsize{9.000000}{10.800000}\selectfont \(\displaystyle {30}\)}%
\end{pgfscope}%
\begin{pgfscope}%
\pgfsetbuttcap%
\pgfsetroundjoin%
\definecolor{currentfill}{rgb}{0.000000,0.000000,0.000000}%
\pgfsetfillcolor{currentfill}%
\pgfsetlinewidth{0.803000pt}%
\definecolor{currentstroke}{rgb}{0.000000,0.000000,0.000000}%
\pgfsetstrokecolor{currentstroke}%
\pgfsetdash{}{0pt}%
\pgfsys@defobject{currentmarker}{\pgfqpoint{-0.048611in}{0.000000in}}{\pgfqpoint{-0.000000in}{0.000000in}}{%
\pgfpathmoveto{\pgfqpoint{-0.000000in}{0.000000in}}%
\pgfpathlineto{\pgfqpoint{-0.048611in}{0.000000in}}%
\pgfusepath{stroke,fill}%
}%
\begin{pgfscope}%
\pgfsys@transformshift{0.553904in}{3.268773in}%
\pgfsys@useobject{currentmarker}{}%
\end{pgfscope}%
\end{pgfscope}%
\begin{pgfscope}%
\definecolor{textcolor}{rgb}{0.000000,0.000000,0.000000}%
\pgfsetstrokecolor{textcolor}%
\pgfsetfillcolor{textcolor}%
\pgftext[x=0.328211in, y=3.225728in, left, base]{\color{textcolor}\rmfamily\fontsize{9.000000}{10.800000}\selectfont \(\displaystyle {35}\)}%
\end{pgfscope}%
\begin{pgfscope}%
\definecolor{textcolor}{rgb}{0.000000,0.000000,0.000000}%
\pgfsetstrokecolor{textcolor}%
\pgfsetfillcolor{textcolor}%
\pgftext[x=0.272655in,y=2.042911in,,bottom,rotate=90.000000]{\color{textcolor}\rmfamily\fontsize{10.000000}{12.000000}\selectfont Width of decomposition}%
\end{pgfscope}%
\begin{pgfscope}%
\pgfpathrectangle{\pgfqpoint{0.553904in}{0.535823in}}{\pgfqpoint{5.296096in}{3.014177in}}%
\pgfusepath{clip}%
\pgfsetrectcap%
\pgfsetroundjoin%
\pgfsetlinewidth{1.003750pt}%
\definecolor{currentstroke}{rgb}{0.756863,0.117647,0.588235}%
\pgfsetstrokecolor{currentstroke}%
\pgfsetdash{}{0pt}%
\pgfpathmoveto{\pgfqpoint{0.794636in}{0.865123in}}%
\pgfpathlineto{\pgfqpoint{1.035368in}{0.961269in}}%
\pgfpathlineto{\pgfqpoint{1.276099in}{1.057415in}}%
\pgfpathlineto{\pgfqpoint{1.516831in}{1.153561in}}%
\pgfpathlineto{\pgfqpoint{1.757562in}{1.345853in}}%
\pgfpathlineto{\pgfqpoint{1.998294in}{1.441999in}}%
\pgfpathlineto{\pgfqpoint{2.239026in}{1.538145in}}%
\pgfpathlineto{\pgfqpoint{2.479757in}{1.730437in}}%
\pgfpathlineto{\pgfqpoint{2.720489in}{1.826583in}}%
\pgfpathlineto{\pgfqpoint{2.961221in}{2.018875in}}%
\pgfpathlineto{\pgfqpoint{3.201952in}{2.115021in}}%
\pgfpathlineto{\pgfqpoint{3.442684in}{2.211167in}}%
\pgfpathlineto{\pgfqpoint{3.683415in}{2.307313in}}%
\pgfpathlineto{\pgfqpoint{3.924147in}{2.499605in}}%
\pgfpathlineto{\pgfqpoint{4.164879in}{2.595751in}}%
\pgfpathlineto{\pgfqpoint{4.405610in}{2.788043in}}%
\pgfpathlineto{\pgfqpoint{4.646342in}{2.884189in}}%
\pgfpathlineto{\pgfqpoint{4.887074in}{2.980335in}}%
\pgfpathlineto{\pgfqpoint{5.127805in}{3.172627in}}%
\pgfpathlineto{\pgfqpoint{5.368537in}{3.268773in}}%
\pgfpathlineto{\pgfqpoint{5.609268in}{3.412992in}}%
\pgfusepath{stroke}%
\end{pgfscope}%
\begin{pgfscope}%
\pgfpathrectangle{\pgfqpoint{0.553904in}{0.535823in}}{\pgfqpoint{5.296096in}{3.014177in}}%
\pgfusepath{clip}%
\pgfsetbuttcap%
\pgfsetroundjoin%
\definecolor{currentfill}{rgb}{0.756863,0.117647,0.588235}%
\pgfsetfillcolor{currentfill}%
\pgfsetlinewidth{0.501875pt}%
\definecolor{currentstroke}{rgb}{0.000000,0.000000,0.000000}%
\pgfsetstrokecolor{currentstroke}%
\pgfsetdash{}{0pt}%
\pgfsys@defobject{currentmarker}{\pgfqpoint{-0.034722in}{-0.034722in}}{\pgfqpoint{0.034722in}{0.034722in}}{%
\pgfpathmoveto{\pgfqpoint{0.000000in}{-0.034722in}}%
\pgfpathcurveto{\pgfqpoint{0.009208in}{-0.034722in}}{\pgfqpoint{0.018041in}{-0.031064in}}{\pgfqpoint{0.024552in}{-0.024552in}}%
\pgfpathcurveto{\pgfqpoint{0.031064in}{-0.018041in}}{\pgfqpoint{0.034722in}{-0.009208in}}{\pgfqpoint{0.034722in}{0.000000in}}%
\pgfpathcurveto{\pgfqpoint{0.034722in}{0.009208in}}{\pgfqpoint{0.031064in}{0.018041in}}{\pgfqpoint{0.024552in}{0.024552in}}%
\pgfpathcurveto{\pgfqpoint{0.018041in}{0.031064in}}{\pgfqpoint{0.009208in}{0.034722in}}{\pgfqpoint{0.000000in}{0.034722in}}%
\pgfpathcurveto{\pgfqpoint{-0.009208in}{0.034722in}}{\pgfqpoint{-0.018041in}{0.031064in}}{\pgfqpoint{-0.024552in}{0.024552in}}%
\pgfpathcurveto{\pgfqpoint{-0.031064in}{0.018041in}}{\pgfqpoint{-0.034722in}{0.009208in}}{\pgfqpoint{-0.034722in}{0.000000in}}%
\pgfpathcurveto{\pgfqpoint{-0.034722in}{-0.009208in}}{\pgfqpoint{-0.031064in}{-0.018041in}}{\pgfqpoint{-0.024552in}{-0.024552in}}%
\pgfpathcurveto{\pgfqpoint{-0.018041in}{-0.031064in}}{\pgfqpoint{-0.009208in}{-0.034722in}}{\pgfqpoint{0.000000in}{-0.034722in}}%
\pgfpathclose%
\pgfusepath{stroke,fill}%
}%
\begin{pgfscope}%
\pgfsys@transformshift{0.794636in}{0.865123in}%
\pgfsys@useobject{currentmarker}{}%
\end{pgfscope}%
\begin{pgfscope}%
\pgfsys@transformshift{1.035368in}{0.961269in}%
\pgfsys@useobject{currentmarker}{}%
\end{pgfscope}%
\begin{pgfscope}%
\pgfsys@transformshift{1.276099in}{1.057415in}%
\pgfsys@useobject{currentmarker}{}%
\end{pgfscope}%
\begin{pgfscope}%
\pgfsys@transformshift{1.516831in}{1.153561in}%
\pgfsys@useobject{currentmarker}{}%
\end{pgfscope}%
\begin{pgfscope}%
\pgfsys@transformshift{1.757562in}{1.345853in}%
\pgfsys@useobject{currentmarker}{}%
\end{pgfscope}%
\begin{pgfscope}%
\pgfsys@transformshift{1.998294in}{1.441999in}%
\pgfsys@useobject{currentmarker}{}%
\end{pgfscope}%
\begin{pgfscope}%
\pgfsys@transformshift{2.239026in}{1.538145in}%
\pgfsys@useobject{currentmarker}{}%
\end{pgfscope}%
\begin{pgfscope}%
\pgfsys@transformshift{2.479757in}{1.730437in}%
\pgfsys@useobject{currentmarker}{}%
\end{pgfscope}%
\begin{pgfscope}%
\pgfsys@transformshift{2.720489in}{1.826583in}%
\pgfsys@useobject{currentmarker}{}%
\end{pgfscope}%
\begin{pgfscope}%
\pgfsys@transformshift{2.961221in}{2.018875in}%
\pgfsys@useobject{currentmarker}{}%
\end{pgfscope}%
\begin{pgfscope}%
\pgfsys@transformshift{3.201952in}{2.115021in}%
\pgfsys@useobject{currentmarker}{}%
\end{pgfscope}%
\begin{pgfscope}%
\pgfsys@transformshift{3.442684in}{2.211167in}%
\pgfsys@useobject{currentmarker}{}%
\end{pgfscope}%
\begin{pgfscope}%
\pgfsys@transformshift{3.683415in}{2.307313in}%
\pgfsys@useobject{currentmarker}{}%
\end{pgfscope}%
\begin{pgfscope}%
\pgfsys@transformshift{3.924147in}{2.499605in}%
\pgfsys@useobject{currentmarker}{}%
\end{pgfscope}%
\begin{pgfscope}%
\pgfsys@transformshift{4.164879in}{2.595751in}%
\pgfsys@useobject{currentmarker}{}%
\end{pgfscope}%
\begin{pgfscope}%
\pgfsys@transformshift{4.405610in}{2.788043in}%
\pgfsys@useobject{currentmarker}{}%
\end{pgfscope}%
\begin{pgfscope}%
\pgfsys@transformshift{4.646342in}{2.884189in}%
\pgfsys@useobject{currentmarker}{}%
\end{pgfscope}%
\begin{pgfscope}%
\pgfsys@transformshift{4.887074in}{2.980335in}%
\pgfsys@useobject{currentmarker}{}%
\end{pgfscope}%
\begin{pgfscope}%
\pgfsys@transformshift{5.127805in}{3.172627in}%
\pgfsys@useobject{currentmarker}{}%
\end{pgfscope}%
\begin{pgfscope}%
\pgfsys@transformshift{5.368537in}{3.268773in}%
\pgfsys@useobject{currentmarker}{}%
\end{pgfscope}%
\begin{pgfscope}%
\pgfsys@transformshift{5.609268in}{3.412992in}%
\pgfsys@useobject{currentmarker}{}%
\end{pgfscope}%
\end{pgfscope}%
\begin{pgfscope}%
\pgfpathrectangle{\pgfqpoint{0.553904in}{0.535823in}}{\pgfqpoint{5.296096in}{3.014177in}}%
\pgfusepath{clip}%
\pgfsetrectcap%
\pgfsetroundjoin%
\pgfsetlinewidth{1.003750pt}%
\definecolor{currentstroke}{rgb}{0.007843,0.219608,0.501961}%
\pgfsetstrokecolor{currentstroke}%
\pgfsetdash{}{0pt}%
\pgfpathmoveto{\pgfqpoint{0.794636in}{0.672831in}}%
\pgfpathlineto{\pgfqpoint{1.035368in}{0.768977in}}%
\pgfpathlineto{\pgfqpoint{1.276099in}{0.961269in}}%
\pgfpathlineto{\pgfqpoint{1.516831in}{1.057415in}}%
\pgfpathlineto{\pgfqpoint{1.757562in}{1.153561in}}%
\pgfpathlineto{\pgfqpoint{1.998294in}{1.249707in}}%
\pgfpathlineto{\pgfqpoint{2.239026in}{1.345853in}}%
\pgfpathlineto{\pgfqpoint{2.479757in}{1.538145in}}%
\pgfpathlineto{\pgfqpoint{2.720489in}{1.634291in}}%
\pgfpathlineto{\pgfqpoint{2.961221in}{1.826583in}}%
\pgfpathlineto{\pgfqpoint{3.201952in}{1.922729in}}%
\pgfpathlineto{\pgfqpoint{3.442684in}{2.018875in}}%
\pgfpathlineto{\pgfqpoint{3.683415in}{2.115021in}}%
\pgfpathlineto{\pgfqpoint{3.924147in}{2.307313in}}%
\pgfpathlineto{\pgfqpoint{4.164879in}{2.403459in}}%
\pgfpathlineto{\pgfqpoint{4.405610in}{2.595751in}}%
\pgfpathlineto{\pgfqpoint{4.646342in}{2.595751in}}%
\pgfpathlineto{\pgfqpoint{4.887074in}{2.788043in}}%
\pgfpathlineto{\pgfqpoint{5.127805in}{2.884189in}}%
\pgfpathlineto{\pgfqpoint{5.368537in}{2.980335in}}%
\pgfpathlineto{\pgfqpoint{5.609268in}{3.172627in}}%
\pgfusepath{stroke}%
\end{pgfscope}%
\begin{pgfscope}%
\pgfpathrectangle{\pgfqpoint{0.553904in}{0.535823in}}{\pgfqpoint{5.296096in}{3.014177in}}%
\pgfusepath{clip}%
\pgfsetbuttcap%
\pgfsetmiterjoin%
\definecolor{currentfill}{rgb}{0.007843,0.219608,0.501961}%
\pgfsetfillcolor{currentfill}%
\pgfsetlinewidth{0.501875pt}%
\definecolor{currentstroke}{rgb}{0.000000,0.000000,0.000000}%
\pgfsetstrokecolor{currentstroke}%
\pgfsetdash{}{0pt}%
\pgfsys@defobject{currentmarker}{\pgfqpoint{-0.034722in}{-0.034722in}}{\pgfqpoint{0.034722in}{0.034722in}}{%
\pgfpathmoveto{\pgfqpoint{-0.000000in}{-0.034722in}}%
\pgfpathlineto{\pgfqpoint{0.034722in}{0.034722in}}%
\pgfpathlineto{\pgfqpoint{-0.034722in}{0.034722in}}%
\pgfpathclose%
\pgfusepath{stroke,fill}%
}%
\begin{pgfscope}%
\pgfsys@transformshift{0.794636in}{0.672831in}%
\pgfsys@useobject{currentmarker}{}%
\end{pgfscope}%
\begin{pgfscope}%
\pgfsys@transformshift{1.035368in}{0.768977in}%
\pgfsys@useobject{currentmarker}{}%
\end{pgfscope}%
\begin{pgfscope}%
\pgfsys@transformshift{1.276099in}{0.961269in}%
\pgfsys@useobject{currentmarker}{}%
\end{pgfscope}%
\begin{pgfscope}%
\pgfsys@transformshift{1.516831in}{1.057415in}%
\pgfsys@useobject{currentmarker}{}%
\end{pgfscope}%
\begin{pgfscope}%
\pgfsys@transformshift{1.757562in}{1.153561in}%
\pgfsys@useobject{currentmarker}{}%
\end{pgfscope}%
\begin{pgfscope}%
\pgfsys@transformshift{1.998294in}{1.249707in}%
\pgfsys@useobject{currentmarker}{}%
\end{pgfscope}%
\begin{pgfscope}%
\pgfsys@transformshift{2.239026in}{1.345853in}%
\pgfsys@useobject{currentmarker}{}%
\end{pgfscope}%
\begin{pgfscope}%
\pgfsys@transformshift{2.479757in}{1.538145in}%
\pgfsys@useobject{currentmarker}{}%
\end{pgfscope}%
\begin{pgfscope}%
\pgfsys@transformshift{2.720489in}{1.634291in}%
\pgfsys@useobject{currentmarker}{}%
\end{pgfscope}%
\begin{pgfscope}%
\pgfsys@transformshift{2.961221in}{1.826583in}%
\pgfsys@useobject{currentmarker}{}%
\end{pgfscope}%
\begin{pgfscope}%
\pgfsys@transformshift{3.201952in}{1.922729in}%
\pgfsys@useobject{currentmarker}{}%
\end{pgfscope}%
\begin{pgfscope}%
\pgfsys@transformshift{3.442684in}{2.018875in}%
\pgfsys@useobject{currentmarker}{}%
\end{pgfscope}%
\begin{pgfscope}%
\pgfsys@transformshift{3.683415in}{2.115021in}%
\pgfsys@useobject{currentmarker}{}%
\end{pgfscope}%
\begin{pgfscope}%
\pgfsys@transformshift{3.924147in}{2.307313in}%
\pgfsys@useobject{currentmarker}{}%
\end{pgfscope}%
\begin{pgfscope}%
\pgfsys@transformshift{4.164879in}{2.403459in}%
\pgfsys@useobject{currentmarker}{}%
\end{pgfscope}%
\begin{pgfscope}%
\pgfsys@transformshift{4.405610in}{2.595751in}%
\pgfsys@useobject{currentmarker}{}%
\end{pgfscope}%
\begin{pgfscope}%
\pgfsys@transformshift{4.646342in}{2.595751in}%
\pgfsys@useobject{currentmarker}{}%
\end{pgfscope}%
\begin{pgfscope}%
\pgfsys@transformshift{4.887074in}{2.788043in}%
\pgfsys@useobject{currentmarker}{}%
\end{pgfscope}%
\begin{pgfscope}%
\pgfsys@transformshift{5.127805in}{2.884189in}%
\pgfsys@useobject{currentmarker}{}%
\end{pgfscope}%
\begin{pgfscope}%
\pgfsys@transformshift{5.368537in}{2.980335in}%
\pgfsys@useobject{currentmarker}{}%
\end{pgfscope}%
\begin{pgfscope}%
\pgfsys@transformshift{5.609268in}{3.172627in}%
\pgfsys@useobject{currentmarker}{}%
\end{pgfscope}%
\end{pgfscope}%
\begin{pgfscope}%
\pgfpathrectangle{\pgfqpoint{0.553904in}{0.535823in}}{\pgfqpoint{5.296096in}{3.014177in}}%
\pgfusepath{clip}%
\pgfsetrectcap%
\pgfsetroundjoin%
\pgfsetlinewidth{1.003750pt}%
\definecolor{currentstroke}{rgb}{0.654902,0.909804,0.192157}%
\pgfsetstrokecolor{currentstroke}%
\pgfsetdash{}{0pt}%
\pgfpathmoveto{\pgfqpoint{0.794636in}{0.672831in}}%
\pgfpathlineto{\pgfqpoint{1.035368in}{0.865123in}}%
\pgfpathlineto{\pgfqpoint{1.276099in}{0.961269in}}%
\pgfpathlineto{\pgfqpoint{1.516831in}{1.057415in}}%
\pgfpathlineto{\pgfqpoint{1.757562in}{1.153561in}}%
\pgfpathlineto{\pgfqpoint{1.998294in}{1.249707in}}%
\pgfpathlineto{\pgfqpoint{2.239026in}{1.441999in}}%
\pgfpathlineto{\pgfqpoint{2.479757in}{1.538145in}}%
\pgfpathlineto{\pgfqpoint{2.720489in}{1.634291in}}%
\pgfpathlineto{\pgfqpoint{2.961221in}{1.826583in}}%
\pgfpathlineto{\pgfqpoint{3.201952in}{1.922729in}}%
\pgfpathlineto{\pgfqpoint{3.442684in}{2.018875in}}%
\pgfpathlineto{\pgfqpoint{3.683415in}{2.115021in}}%
\pgfpathlineto{\pgfqpoint{3.924147in}{2.259240in}}%
\pgfpathlineto{\pgfqpoint{4.164879in}{2.403459in}}%
\pgfpathlineto{\pgfqpoint{4.405610in}{2.499605in}}%
\pgfpathlineto{\pgfqpoint{4.646342in}{2.595751in}}%
\pgfpathlineto{\pgfqpoint{4.887074in}{2.691897in}}%
\pgfpathlineto{\pgfqpoint{5.127805in}{2.884189in}}%
\pgfpathlineto{\pgfqpoint{5.368537in}{2.980335in}}%
\pgfpathlineto{\pgfqpoint{5.609268in}{3.124554in}}%
\pgfusepath{stroke}%
\end{pgfscope}%
\begin{pgfscope}%
\pgfpathrectangle{\pgfqpoint{0.553904in}{0.535823in}}{\pgfqpoint{5.296096in}{3.014177in}}%
\pgfusepath{clip}%
\pgfsetbuttcap%
\pgfsetmiterjoin%
\definecolor{currentfill}{rgb}{0.654902,0.909804,0.192157}%
\pgfsetfillcolor{currentfill}%
\pgfsetlinewidth{0.501875pt}%
\definecolor{currentstroke}{rgb}{0.000000,0.000000,0.000000}%
\pgfsetstrokecolor{currentstroke}%
\pgfsetdash{}{0pt}%
\pgfsys@defobject{currentmarker}{\pgfqpoint{-0.034722in}{-0.034722in}}{\pgfqpoint{0.034722in}{0.034722in}}{%
\pgfpathmoveto{\pgfqpoint{-0.034722in}{-0.034722in}}%
\pgfpathlineto{\pgfqpoint{0.034722in}{-0.034722in}}%
\pgfpathlineto{\pgfqpoint{0.034722in}{0.034722in}}%
\pgfpathlineto{\pgfqpoint{-0.034722in}{0.034722in}}%
\pgfpathclose%
\pgfusepath{stroke,fill}%
}%
\begin{pgfscope}%
\pgfsys@transformshift{0.794636in}{0.672831in}%
\pgfsys@useobject{currentmarker}{}%
\end{pgfscope}%
\begin{pgfscope}%
\pgfsys@transformshift{1.035368in}{0.865123in}%
\pgfsys@useobject{currentmarker}{}%
\end{pgfscope}%
\begin{pgfscope}%
\pgfsys@transformshift{1.276099in}{0.961269in}%
\pgfsys@useobject{currentmarker}{}%
\end{pgfscope}%
\begin{pgfscope}%
\pgfsys@transformshift{1.516831in}{1.057415in}%
\pgfsys@useobject{currentmarker}{}%
\end{pgfscope}%
\begin{pgfscope}%
\pgfsys@transformshift{1.757562in}{1.153561in}%
\pgfsys@useobject{currentmarker}{}%
\end{pgfscope}%
\begin{pgfscope}%
\pgfsys@transformshift{1.998294in}{1.249707in}%
\pgfsys@useobject{currentmarker}{}%
\end{pgfscope}%
\begin{pgfscope}%
\pgfsys@transformshift{2.239026in}{1.441999in}%
\pgfsys@useobject{currentmarker}{}%
\end{pgfscope}%
\begin{pgfscope}%
\pgfsys@transformshift{2.479757in}{1.538145in}%
\pgfsys@useobject{currentmarker}{}%
\end{pgfscope}%
\begin{pgfscope}%
\pgfsys@transformshift{2.720489in}{1.634291in}%
\pgfsys@useobject{currentmarker}{}%
\end{pgfscope}%
\begin{pgfscope}%
\pgfsys@transformshift{2.961221in}{1.826583in}%
\pgfsys@useobject{currentmarker}{}%
\end{pgfscope}%
\begin{pgfscope}%
\pgfsys@transformshift{3.201952in}{1.922729in}%
\pgfsys@useobject{currentmarker}{}%
\end{pgfscope}%
\begin{pgfscope}%
\pgfsys@transformshift{3.442684in}{2.018875in}%
\pgfsys@useobject{currentmarker}{}%
\end{pgfscope}%
\begin{pgfscope}%
\pgfsys@transformshift{3.683415in}{2.115021in}%
\pgfsys@useobject{currentmarker}{}%
\end{pgfscope}%
\begin{pgfscope}%
\pgfsys@transformshift{3.924147in}{2.259240in}%
\pgfsys@useobject{currentmarker}{}%
\end{pgfscope}%
\begin{pgfscope}%
\pgfsys@transformshift{4.164879in}{2.403459in}%
\pgfsys@useobject{currentmarker}{}%
\end{pgfscope}%
\begin{pgfscope}%
\pgfsys@transformshift{4.405610in}{2.499605in}%
\pgfsys@useobject{currentmarker}{}%
\end{pgfscope}%
\begin{pgfscope}%
\pgfsys@transformshift{4.646342in}{2.595751in}%
\pgfsys@useobject{currentmarker}{}%
\end{pgfscope}%
\begin{pgfscope}%
\pgfsys@transformshift{4.887074in}{2.691897in}%
\pgfsys@useobject{currentmarker}{}%
\end{pgfscope}%
\begin{pgfscope}%
\pgfsys@transformshift{5.127805in}{2.884189in}%
\pgfsys@useobject{currentmarker}{}%
\end{pgfscope}%
\begin{pgfscope}%
\pgfsys@transformshift{5.368537in}{2.980335in}%
\pgfsys@useobject{currentmarker}{}%
\end{pgfscope}%
\begin{pgfscope}%
\pgfsys@transformshift{5.609268in}{3.124554in}%
\pgfsys@useobject{currentmarker}{}%
\end{pgfscope}%
\end{pgfscope}%
\begin{pgfscope}%
\pgfpathrectangle{\pgfqpoint{0.553904in}{0.535823in}}{\pgfqpoint{5.296096in}{3.014177in}}%
\pgfusepath{clip}%
\pgfsetrectcap%
\pgfsetroundjoin%
\pgfsetlinewidth{1.003750pt}%
\definecolor{currentstroke}{rgb}{0.525490,0.843137,0.662745}%
\pgfsetstrokecolor{currentstroke}%
\pgfsetdash{}{0pt}%
\pgfpathmoveto{\pgfqpoint{0.794636in}{0.768977in}}%
\pgfpathlineto{\pgfqpoint{1.035368in}{0.865123in}}%
\pgfpathlineto{\pgfqpoint{1.276099in}{0.961269in}}%
\pgfpathlineto{\pgfqpoint{1.516831in}{1.057415in}}%
\pgfpathlineto{\pgfqpoint{1.757562in}{1.249707in}}%
\pgfpathlineto{\pgfqpoint{1.998294in}{1.345853in}}%
\pgfpathlineto{\pgfqpoint{2.239026in}{1.441999in}}%
\pgfpathlineto{\pgfqpoint{2.479757in}{1.538145in}}%
\pgfpathlineto{\pgfqpoint{2.720489in}{1.634291in}}%
\pgfpathlineto{\pgfqpoint{2.961221in}{1.730437in}}%
\pgfpathlineto{\pgfqpoint{3.201952in}{1.826583in}}%
\pgfpathlineto{\pgfqpoint{3.442684in}{1.970802in}}%
\pgfpathlineto{\pgfqpoint{3.683415in}{2.115021in}}%
\pgfpathlineto{\pgfqpoint{3.924147in}{2.211167in}}%
\pgfpathlineto{\pgfqpoint{4.164879in}{2.307313in}}%
\pgfpathlineto{\pgfqpoint{4.405610in}{2.403459in}}%
\pgfpathlineto{\pgfqpoint{4.646342in}{2.499605in}}%
\pgfpathlineto{\pgfqpoint{4.887074in}{2.595751in}}%
\pgfpathlineto{\pgfqpoint{5.127805in}{2.788043in}}%
\pgfpathlineto{\pgfqpoint{5.368537in}{2.836116in}}%
\pgfpathlineto{\pgfqpoint{5.609268in}{2.980335in}}%
\pgfusepath{stroke}%
\end{pgfscope}%
\begin{pgfscope}%
\pgfpathrectangle{\pgfqpoint{0.553904in}{0.535823in}}{\pgfqpoint{5.296096in}{3.014177in}}%
\pgfusepath{clip}%
\pgfsetbuttcap%
\pgfsetbeveljoin%
\definecolor{currentfill}{rgb}{0.525490,0.843137,0.662745}%
\pgfsetfillcolor{currentfill}%
\pgfsetlinewidth{0.501875pt}%
\definecolor{currentstroke}{rgb}{0.000000,0.000000,0.000000}%
\pgfsetstrokecolor{currentstroke}%
\pgfsetdash{}{0pt}%
\pgfsys@defobject{currentmarker}{\pgfqpoint{-0.033023in}{-0.028091in}}{\pgfqpoint{0.033023in}{0.034722in}}{%
\pgfpathmoveto{\pgfqpoint{0.000000in}{0.034722in}}%
\pgfpathlineto{\pgfqpoint{-0.007796in}{0.010730in}}%
\pgfpathlineto{\pgfqpoint{-0.033023in}{0.010730in}}%
\pgfpathlineto{\pgfqpoint{-0.012614in}{-0.004098in}}%
\pgfpathlineto{\pgfqpoint{-0.020409in}{-0.028091in}}%
\pgfpathlineto{\pgfqpoint{-0.000000in}{-0.013263in}}%
\pgfpathlineto{\pgfqpoint{0.020409in}{-0.028091in}}%
\pgfpathlineto{\pgfqpoint{0.012614in}{-0.004098in}}%
\pgfpathlineto{\pgfqpoint{0.033023in}{0.010730in}}%
\pgfpathlineto{\pgfqpoint{0.007796in}{0.010730in}}%
\pgfpathclose%
\pgfusepath{stroke,fill}%
}%
\begin{pgfscope}%
\pgfsys@transformshift{0.794636in}{0.768977in}%
\pgfsys@useobject{currentmarker}{}%
\end{pgfscope}%
\begin{pgfscope}%
\pgfsys@transformshift{1.035368in}{0.865123in}%
\pgfsys@useobject{currentmarker}{}%
\end{pgfscope}%
\begin{pgfscope}%
\pgfsys@transformshift{1.276099in}{0.961269in}%
\pgfsys@useobject{currentmarker}{}%
\end{pgfscope}%
\begin{pgfscope}%
\pgfsys@transformshift{1.516831in}{1.057415in}%
\pgfsys@useobject{currentmarker}{}%
\end{pgfscope}%
\begin{pgfscope}%
\pgfsys@transformshift{1.757562in}{1.249707in}%
\pgfsys@useobject{currentmarker}{}%
\end{pgfscope}%
\begin{pgfscope}%
\pgfsys@transformshift{1.998294in}{1.345853in}%
\pgfsys@useobject{currentmarker}{}%
\end{pgfscope}%
\begin{pgfscope}%
\pgfsys@transformshift{2.239026in}{1.441999in}%
\pgfsys@useobject{currentmarker}{}%
\end{pgfscope}%
\begin{pgfscope}%
\pgfsys@transformshift{2.479757in}{1.538145in}%
\pgfsys@useobject{currentmarker}{}%
\end{pgfscope}%
\begin{pgfscope}%
\pgfsys@transformshift{2.720489in}{1.634291in}%
\pgfsys@useobject{currentmarker}{}%
\end{pgfscope}%
\begin{pgfscope}%
\pgfsys@transformshift{2.961221in}{1.730437in}%
\pgfsys@useobject{currentmarker}{}%
\end{pgfscope}%
\begin{pgfscope}%
\pgfsys@transformshift{3.201952in}{1.826583in}%
\pgfsys@useobject{currentmarker}{}%
\end{pgfscope}%
\begin{pgfscope}%
\pgfsys@transformshift{3.442684in}{1.970802in}%
\pgfsys@useobject{currentmarker}{}%
\end{pgfscope}%
\begin{pgfscope}%
\pgfsys@transformshift{3.683415in}{2.115021in}%
\pgfsys@useobject{currentmarker}{}%
\end{pgfscope}%
\begin{pgfscope}%
\pgfsys@transformshift{3.924147in}{2.211167in}%
\pgfsys@useobject{currentmarker}{}%
\end{pgfscope}%
\begin{pgfscope}%
\pgfsys@transformshift{4.164879in}{2.307313in}%
\pgfsys@useobject{currentmarker}{}%
\end{pgfscope}%
\begin{pgfscope}%
\pgfsys@transformshift{4.405610in}{2.403459in}%
\pgfsys@useobject{currentmarker}{}%
\end{pgfscope}%
\begin{pgfscope}%
\pgfsys@transformshift{4.646342in}{2.499605in}%
\pgfsys@useobject{currentmarker}{}%
\end{pgfscope}%
\begin{pgfscope}%
\pgfsys@transformshift{4.887074in}{2.595751in}%
\pgfsys@useobject{currentmarker}{}%
\end{pgfscope}%
\begin{pgfscope}%
\pgfsys@transformshift{5.127805in}{2.788043in}%
\pgfsys@useobject{currentmarker}{}%
\end{pgfscope}%
\begin{pgfscope}%
\pgfsys@transformshift{5.368537in}{2.836116in}%
\pgfsys@useobject{currentmarker}{}%
\end{pgfscope}%
\begin{pgfscope}%
\pgfsys@transformshift{5.609268in}{2.980335in}%
\pgfsys@useobject{currentmarker}{}%
\end{pgfscope}%
\end{pgfscope}%
\begin{pgfscope}%
\pgfsetrectcap%
\pgfsetmiterjoin%
\pgfsetlinewidth{0.803000pt}%
\definecolor{currentstroke}{rgb}{0.000000,0.000000,0.000000}%
\pgfsetstrokecolor{currentstroke}%
\pgfsetdash{}{0pt}%
\pgfpathmoveto{\pgfqpoint{0.553904in}{0.535823in}}%
\pgfpathlineto{\pgfqpoint{0.553904in}{3.550000in}}%
\pgfusepath{stroke}%
\end{pgfscope}%
\begin{pgfscope}%
\pgfsetrectcap%
\pgfsetmiterjoin%
\pgfsetlinewidth{0.803000pt}%
\definecolor{currentstroke}{rgb}{0.000000,0.000000,0.000000}%
\pgfsetstrokecolor{currentstroke}%
\pgfsetdash{}{0pt}%
\pgfpathmoveto{\pgfqpoint{5.850000in}{0.535823in}}%
\pgfpathlineto{\pgfqpoint{5.850000in}{3.550000in}}%
\pgfusepath{stroke}%
\end{pgfscope}%
\begin{pgfscope}%
\pgfsetrectcap%
\pgfsetmiterjoin%
\pgfsetlinewidth{0.803000pt}%
\definecolor{currentstroke}{rgb}{0.000000,0.000000,0.000000}%
\pgfsetstrokecolor{currentstroke}%
\pgfsetdash{}{0pt}%
\pgfpathmoveto{\pgfqpoint{0.553904in}{0.535823in}}%
\pgfpathlineto{\pgfqpoint{5.850000in}{0.535823in}}%
\pgfusepath{stroke}%
\end{pgfscope}%
\begin{pgfscope}%
\pgfsetrectcap%
\pgfsetmiterjoin%
\pgfsetlinewidth{0.803000pt}%
\definecolor{currentstroke}{rgb}{0.000000,0.000000,0.000000}%
\pgfsetstrokecolor{currentstroke}%
\pgfsetdash{}{0pt}%
\pgfpathmoveto{\pgfqpoint{0.553904in}{3.550000in}}%
\pgfpathlineto{\pgfqpoint{5.850000in}{3.550000in}}%
\pgfusepath{stroke}%
\end{pgfscope}%
\begin{pgfscope}%
\pgfsetrectcap%
\pgfsetroundjoin%
\pgfsetlinewidth{1.003750pt}%
\definecolor{currentstroke}{rgb}{0.756863,0.117647,0.588235}%
\pgfsetstrokecolor{currentstroke}%
\pgfsetdash{}{0pt}%
\pgfpathmoveto{\pgfqpoint{0.603904in}{3.450000in}}%
\pgfpathlineto{\pgfqpoint{0.853904in}{3.450000in}}%
\pgfusepath{stroke}%
\end{pgfscope}%
\begin{pgfscope}%
\pgfsetbuttcap%
\pgfsetroundjoin%
\definecolor{currentfill}{rgb}{0.756863,0.117647,0.588235}%
\pgfsetfillcolor{currentfill}%
\pgfsetlinewidth{0.501875pt}%
\definecolor{currentstroke}{rgb}{0.000000,0.000000,0.000000}%
\pgfsetstrokecolor{currentstroke}%
\pgfsetdash{}{0pt}%
\pgfsys@defobject{currentmarker}{\pgfqpoint{-0.034722in}{-0.034722in}}{\pgfqpoint{0.034722in}{0.034722in}}{%
\pgfpathmoveto{\pgfqpoint{0.000000in}{-0.034722in}}%
\pgfpathcurveto{\pgfqpoint{0.009208in}{-0.034722in}}{\pgfqpoint{0.018041in}{-0.031064in}}{\pgfqpoint{0.024552in}{-0.024552in}}%
\pgfpathcurveto{\pgfqpoint{0.031064in}{-0.018041in}}{\pgfqpoint{0.034722in}{-0.009208in}}{\pgfqpoint{0.034722in}{0.000000in}}%
\pgfpathcurveto{\pgfqpoint{0.034722in}{0.009208in}}{\pgfqpoint{0.031064in}{0.018041in}}{\pgfqpoint{0.024552in}{0.024552in}}%
\pgfpathcurveto{\pgfqpoint{0.018041in}{0.031064in}}{\pgfqpoint{0.009208in}{0.034722in}}{\pgfqpoint{0.000000in}{0.034722in}}%
\pgfpathcurveto{\pgfqpoint{-0.009208in}{0.034722in}}{\pgfqpoint{-0.018041in}{0.031064in}}{\pgfqpoint{-0.024552in}{0.024552in}}%
\pgfpathcurveto{\pgfqpoint{-0.031064in}{0.018041in}}{\pgfqpoint{-0.034722in}{0.009208in}}{\pgfqpoint{-0.034722in}{0.000000in}}%
\pgfpathcurveto{\pgfqpoint{-0.034722in}{-0.009208in}}{\pgfqpoint{-0.031064in}{-0.018041in}}{\pgfqpoint{-0.024552in}{-0.024552in}}%
\pgfpathcurveto{\pgfqpoint{-0.018041in}{-0.031064in}}{\pgfqpoint{-0.009208in}{-0.034722in}}{\pgfqpoint{0.000000in}{-0.034722in}}%
\pgfpathclose%
\pgfusepath{stroke,fill}%
}%
\begin{pgfscope}%
\pgfsys@transformshift{0.728904in}{3.450000in}%
\pgfsys@useobject{currentmarker}{}%
\end{pgfscope}%
\end{pgfscope}%
\begin{pgfscope}%
\definecolor{textcolor}{rgb}{0.000000,0.000000,0.000000}%
\pgfsetstrokecolor{textcolor}%
\pgfsetfillcolor{textcolor}%
\pgftext[x=0.878904in,y=3.406250in,left,base]{\color{textcolor}\rmfamily\fontsize{9.000000}{10.800000}\selectfont Treewidth of \(\displaystyle Line(G)\)}%
\end{pgfscope}%
\begin{pgfscope}%
\pgfsetrectcap%
\pgfsetroundjoin%
\pgfsetlinewidth{1.003750pt}%
\definecolor{currentstroke}{rgb}{0.007843,0.219608,0.501961}%
\pgfsetstrokecolor{currentstroke}%
\pgfsetdash{}{0pt}%
\pgfpathmoveto{\pgfqpoint{0.603904in}{3.281250in}}%
\pgfpathlineto{\pgfqpoint{0.853904in}{3.281250in}}%
\pgfusepath{stroke}%
\end{pgfscope}%
\begin{pgfscope}%
\pgfsetbuttcap%
\pgfsetmiterjoin%
\definecolor{currentfill}{rgb}{0.007843,0.219608,0.501961}%
\pgfsetfillcolor{currentfill}%
\pgfsetlinewidth{0.501875pt}%
\definecolor{currentstroke}{rgb}{0.000000,0.000000,0.000000}%
\pgfsetstrokecolor{currentstroke}%
\pgfsetdash{}{0pt}%
\pgfsys@defobject{currentmarker}{\pgfqpoint{-0.034722in}{-0.034722in}}{\pgfqpoint{0.034722in}{0.034722in}}{%
\pgfpathmoveto{\pgfqpoint{-0.000000in}{-0.034722in}}%
\pgfpathlineto{\pgfqpoint{0.034722in}{0.034722in}}%
\pgfpathlineto{\pgfqpoint{-0.034722in}{0.034722in}}%
\pgfpathclose%
\pgfusepath{stroke,fill}%
}%
\begin{pgfscope}%
\pgfsys@transformshift{0.728904in}{3.281250in}%
\pgfsys@useobject{currentmarker}{}%
\end{pgfscope}%
\end{pgfscope}%
\begin{pgfscope}%
\definecolor{textcolor}{rgb}{0.000000,0.000000,0.000000}%
\pgfsetstrokecolor{textcolor}%
\pgfsetfillcolor{textcolor}%
\pgftext[x=0.878904in,y=3.237500in,left,base]{\color{textcolor}\rmfamily\fontsize{9.000000}{10.800000}\selectfont Treewidth of \(\displaystyle G\)}%
\end{pgfscope}%
\begin{pgfscope}%
\pgfsetrectcap%
\pgfsetroundjoin%
\pgfsetlinewidth{1.003750pt}%
\definecolor{currentstroke}{rgb}{0.654902,0.909804,0.192157}%
\pgfsetstrokecolor{currentstroke}%
\pgfsetdash{}{0pt}%
\pgfpathmoveto{\pgfqpoint{0.603904in}{3.119450in}}%
\pgfpathlineto{\pgfqpoint{0.853904in}{3.119450in}}%
\pgfusepath{stroke}%
\end{pgfscope}%
\begin{pgfscope}%
\pgfsetbuttcap%
\pgfsetmiterjoin%
\definecolor{currentfill}{rgb}{0.654902,0.909804,0.192157}%
\pgfsetfillcolor{currentfill}%
\pgfsetlinewidth{0.501875pt}%
\definecolor{currentstroke}{rgb}{0.000000,0.000000,0.000000}%
\pgfsetstrokecolor{currentstroke}%
\pgfsetdash{}{0pt}%
\pgfsys@defobject{currentmarker}{\pgfqpoint{-0.034722in}{-0.034722in}}{\pgfqpoint{0.034722in}{0.034722in}}{%
\pgfpathmoveto{\pgfqpoint{-0.034722in}{-0.034722in}}%
\pgfpathlineto{\pgfqpoint{0.034722in}{-0.034722in}}%
\pgfpathlineto{\pgfqpoint{0.034722in}{0.034722in}}%
\pgfpathlineto{\pgfqpoint{-0.034722in}{0.034722in}}%
\pgfpathclose%
\pgfusepath{stroke,fill}%
}%
\begin{pgfscope}%
\pgfsys@transformshift{0.728904in}{3.119450in}%
\pgfsys@useobject{currentmarker}{}%
\end{pgfscope}%
\end{pgfscope}%
\begin{pgfscope}%
\definecolor{textcolor}{rgb}{0.000000,0.000000,0.000000}%
\pgfsetstrokecolor{textcolor}%
\pgfsetfillcolor{textcolor}%
\pgftext[x=0.878904in,y=3.075700in,left,base]{\color{textcolor}\rmfamily\fontsize{9.000000}{10.800000}\selectfont Carving width of \(\displaystyle G\) using \textbf{FT}}%
\end{pgfscope}%
\begin{pgfscope}%
\pgfsetrectcap%
\pgfsetroundjoin%
\pgfsetlinewidth{1.003750pt}%
\definecolor{currentstroke}{rgb}{0.525490,0.843137,0.662745}%
\pgfsetstrokecolor{currentstroke}%
\pgfsetdash{}{0pt}%
\pgfpathmoveto{\pgfqpoint{0.603904in}{2.957651in}}%
\pgfpathlineto{\pgfqpoint{0.853904in}{2.957651in}}%
\pgfusepath{stroke}%
\end{pgfscope}%
\begin{pgfscope}%
\pgfsetbuttcap%
\pgfsetbeveljoin%
\definecolor{currentfill}{rgb}{0.525490,0.843137,0.662745}%
\pgfsetfillcolor{currentfill}%
\pgfsetlinewidth{0.501875pt}%
\definecolor{currentstroke}{rgb}{0.000000,0.000000,0.000000}%
\pgfsetstrokecolor{currentstroke}%
\pgfsetdash{}{0pt}%
\pgfsys@defobject{currentmarker}{\pgfqpoint{-0.033023in}{-0.028091in}}{\pgfqpoint{0.033023in}{0.034722in}}{%
\pgfpathmoveto{\pgfqpoint{0.000000in}{0.034722in}}%
\pgfpathlineto{\pgfqpoint{-0.007796in}{0.010730in}}%
\pgfpathlineto{\pgfqpoint{-0.033023in}{0.010730in}}%
\pgfpathlineto{\pgfqpoint{-0.012614in}{-0.004098in}}%
\pgfpathlineto{\pgfqpoint{-0.020409in}{-0.028091in}}%
\pgfpathlineto{\pgfqpoint{-0.000000in}{-0.013263in}}%
\pgfpathlineto{\pgfqpoint{0.020409in}{-0.028091in}}%
\pgfpathlineto{\pgfqpoint{0.012614in}{-0.004098in}}%
\pgfpathlineto{\pgfqpoint{0.033023in}{0.010730in}}%
\pgfpathlineto{\pgfqpoint{0.007796in}{0.010730in}}%
\pgfpathclose%
\pgfusepath{stroke,fill}%
}%
\begin{pgfscope}%
\pgfsys@transformshift{0.728904in}{2.957651in}%
\pgfsys@useobject{currentmarker}{}%
\end{pgfscope}%
\end{pgfscope}%
\begin{pgfscope}%
\definecolor{textcolor}{rgb}{0.000000,0.000000,0.000000}%
\pgfsetstrokecolor{textcolor}%
\pgfsetfillcolor{textcolor}%
\pgftext[x=0.878904in,y=2.913901in,left,base]{\color{textcolor}\rmfamily\fontsize{9.000000}{10.800000}\selectfont Carving width of \(\displaystyle G\) using \textbf{LG}}%
\end{pgfscope}%
\end{pgfpicture}%
\makeatother%
\endgroup%

	\caption{\label{fig:vertex-cover-width} Median of the best upper bound found for treewidth and carving width of 100 cubic graphs with $n$ vertices. For most large graphs, the carving width of $G$ is smaller than the treewidth of $G$, which is smaller that the treewidth of $\Line{G}$.}
\end{figure}

%\begin{figure}
%	\centering
%	%% Creator: Matplotlib, PGF backend
%%
%% To include the figure in your LaTeX document, write
%%   \input{<filename>.pgf}
%%
%% Make sure the required packages are loaded in your preamble
%%   \usepackage{pgf}
%%
%% and, on pdftex
%%   \usepackage[utf8]{inputenc}\DeclareUnicodeCharacter{2212}{-}
%%
%% or, on luatex and xetex
%%   \usepackage{unicode-math}
%%
%% Figures using additional raster images can only be included by \input if
%% they are in the same directory as the main LaTeX file. For loading figures
%% from other directories you can use the `import` package
%%   \usepackage{import}
%%
%% and then include the figures with
%%   \import{<path to file>}{<filename>.pgf}
%%
%% Matplotlib used the following preamble
%%   \usepackage[utf8x]{inputenc}
%%   \usepackage[T1]{fontenc}
%%
\begingroup%
\makeatletter%
\begin{pgfpicture}%
\pgfpathrectangle{\pgfpointorigin}{\pgfqpoint{6.000000in}{3.400000in}}%
\pgfusepath{use as bounding box, clip}%
\begin{pgfscope}%
\pgfsetbuttcap%
\pgfsetmiterjoin%
\definecolor{currentfill}{rgb}{1.000000,1.000000,1.000000}%
\pgfsetfillcolor{currentfill}%
\pgfsetlinewidth{0.000000pt}%
\definecolor{currentstroke}{rgb}{1.000000,1.000000,1.000000}%
\pgfsetstrokecolor{currentstroke}%
\pgfsetdash{}{0pt}%
\pgfpathmoveto{\pgfqpoint{0.000000in}{0.000000in}}%
\pgfpathlineto{\pgfqpoint{6.000000in}{0.000000in}}%
\pgfpathlineto{\pgfqpoint{6.000000in}{3.400000in}}%
\pgfpathlineto{\pgfqpoint{0.000000in}{3.400000in}}%
\pgfpathclose%
\pgfusepath{fill}%
\end{pgfscope}%
\begin{pgfscope}%
\pgfsetbuttcap%
\pgfsetmiterjoin%
\definecolor{currentfill}{rgb}{1.000000,1.000000,1.000000}%
\pgfsetfillcolor{currentfill}%
\pgfsetlinewidth{0.000000pt}%
\definecolor{currentstroke}{rgb}{0.000000,0.000000,0.000000}%
\pgfsetstrokecolor{currentstroke}%
\pgfsetstrokeopacity{0.000000}%
\pgfsetdash{}{0pt}%
\pgfpathmoveto{\pgfqpoint{0.708220in}{0.535823in}}%
\pgfpathlineto{\pgfqpoint{5.850000in}{0.535823in}}%
\pgfpathlineto{\pgfqpoint{5.850000in}{3.205275in}}%
\pgfpathlineto{\pgfqpoint{0.708220in}{3.205275in}}%
\pgfpathclose%
\pgfusepath{fill}%
\end{pgfscope}%
\begin{pgfscope}%
\pgfsetbuttcap%
\pgfsetroundjoin%
\definecolor{currentfill}{rgb}{0.000000,0.000000,0.000000}%
\pgfsetfillcolor{currentfill}%
\pgfsetlinewidth{0.803000pt}%
\definecolor{currentstroke}{rgb}{0.000000,0.000000,0.000000}%
\pgfsetstrokecolor{currentstroke}%
\pgfsetdash{}{0pt}%
\pgfsys@defobject{currentmarker}{\pgfqpoint{0.000000in}{-0.048611in}}{\pgfqpoint{0.000000in}{0.000000in}}{%
\pgfpathmoveto{\pgfqpoint{0.000000in}{0.000000in}}%
\pgfpathlineto{\pgfqpoint{0.000000in}{-0.048611in}}%
\pgfusepath{stroke,fill}%
}%
\begin{pgfscope}%
\pgfsys@transformshift{0.708220in}{0.535823in}%
\pgfsys@useobject{currentmarker}{}%
\end{pgfscope}%
\end{pgfscope}%
\begin{pgfscope}%
\definecolor{textcolor}{rgb}{0.000000,0.000000,0.000000}%
\pgfsetstrokecolor{textcolor}%
\pgfsetfillcolor{textcolor}%
\pgftext[x=0.708220in,y=0.438600in,,top]{\color{textcolor}\rmfamily\fontsize{9.000000}{10.800000}\selectfont \(\displaystyle {0}\)}%
\end{pgfscope}%
\begin{pgfscope}%
\pgfsetbuttcap%
\pgfsetroundjoin%
\definecolor{currentfill}{rgb}{0.000000,0.000000,0.000000}%
\pgfsetfillcolor{currentfill}%
\pgfsetlinewidth{0.803000pt}%
\definecolor{currentstroke}{rgb}{0.000000,0.000000,0.000000}%
\pgfsetstrokecolor{currentstroke}%
\pgfsetdash{}{0pt}%
\pgfsys@defobject{currentmarker}{\pgfqpoint{0.000000in}{-0.048611in}}{\pgfqpoint{0.000000in}{0.000000in}}{%
\pgfpathmoveto{\pgfqpoint{0.000000in}{0.000000in}}%
\pgfpathlineto{\pgfqpoint{0.000000in}{-0.048611in}}%
\pgfusepath{stroke,fill}%
}%
\begin{pgfscope}%
\pgfsys@transformshift{1.833336in}{0.535823in}%
\pgfsys@useobject{currentmarker}{}%
\end{pgfscope}%
\end{pgfscope}%
\begin{pgfscope}%
\definecolor{textcolor}{rgb}{0.000000,0.000000,0.000000}%
\pgfsetstrokecolor{textcolor}%
\pgfsetfillcolor{textcolor}%
\pgftext[x=1.833336in,y=0.438600in,,top]{\color{textcolor}\rmfamily\fontsize{9.000000}{10.800000}\selectfont \(\displaystyle {50}\)}%
\end{pgfscope}%
\begin{pgfscope}%
\pgfsetbuttcap%
\pgfsetroundjoin%
\definecolor{currentfill}{rgb}{0.000000,0.000000,0.000000}%
\pgfsetfillcolor{currentfill}%
\pgfsetlinewidth{0.803000pt}%
\definecolor{currentstroke}{rgb}{0.000000,0.000000,0.000000}%
\pgfsetstrokecolor{currentstroke}%
\pgfsetdash{}{0pt}%
\pgfsys@defobject{currentmarker}{\pgfqpoint{0.000000in}{-0.048611in}}{\pgfqpoint{0.000000in}{0.000000in}}{%
\pgfpathmoveto{\pgfqpoint{0.000000in}{0.000000in}}%
\pgfpathlineto{\pgfqpoint{0.000000in}{-0.048611in}}%
\pgfusepath{stroke,fill}%
}%
\begin{pgfscope}%
\pgfsys@transformshift{2.958452in}{0.535823in}%
\pgfsys@useobject{currentmarker}{}%
\end{pgfscope}%
\end{pgfscope}%
\begin{pgfscope}%
\definecolor{textcolor}{rgb}{0.000000,0.000000,0.000000}%
\pgfsetstrokecolor{textcolor}%
\pgfsetfillcolor{textcolor}%
\pgftext[x=2.958452in,y=0.438600in,,top]{\color{textcolor}\rmfamily\fontsize{9.000000}{10.800000}\selectfont \(\displaystyle {100}\)}%
\end{pgfscope}%
\begin{pgfscope}%
\pgfsetbuttcap%
\pgfsetroundjoin%
\definecolor{currentfill}{rgb}{0.000000,0.000000,0.000000}%
\pgfsetfillcolor{currentfill}%
\pgfsetlinewidth{0.803000pt}%
\definecolor{currentstroke}{rgb}{0.000000,0.000000,0.000000}%
\pgfsetstrokecolor{currentstroke}%
\pgfsetdash{}{0pt}%
\pgfsys@defobject{currentmarker}{\pgfqpoint{0.000000in}{-0.048611in}}{\pgfqpoint{0.000000in}{0.000000in}}{%
\pgfpathmoveto{\pgfqpoint{0.000000in}{0.000000in}}%
\pgfpathlineto{\pgfqpoint{0.000000in}{-0.048611in}}%
\pgfusepath{stroke,fill}%
}%
\begin{pgfscope}%
\pgfsys@transformshift{4.083568in}{0.535823in}%
\pgfsys@useobject{currentmarker}{}%
\end{pgfscope}%
\end{pgfscope}%
\begin{pgfscope}%
\definecolor{textcolor}{rgb}{0.000000,0.000000,0.000000}%
\pgfsetstrokecolor{textcolor}%
\pgfsetfillcolor{textcolor}%
\pgftext[x=4.083568in,y=0.438600in,,top]{\color{textcolor}\rmfamily\fontsize{9.000000}{10.800000}\selectfont \(\displaystyle {150}\)}%
\end{pgfscope}%
\begin{pgfscope}%
\pgfsetbuttcap%
\pgfsetroundjoin%
\definecolor{currentfill}{rgb}{0.000000,0.000000,0.000000}%
\pgfsetfillcolor{currentfill}%
\pgfsetlinewidth{0.803000pt}%
\definecolor{currentstroke}{rgb}{0.000000,0.000000,0.000000}%
\pgfsetstrokecolor{currentstroke}%
\pgfsetdash{}{0pt}%
\pgfsys@defobject{currentmarker}{\pgfqpoint{0.000000in}{-0.048611in}}{\pgfqpoint{0.000000in}{0.000000in}}{%
\pgfpathmoveto{\pgfqpoint{0.000000in}{0.000000in}}%
\pgfpathlineto{\pgfqpoint{0.000000in}{-0.048611in}}%
\pgfusepath{stroke,fill}%
}%
\begin{pgfscope}%
\pgfsys@transformshift{5.208684in}{0.535823in}%
\pgfsys@useobject{currentmarker}{}%
\end{pgfscope}%
\end{pgfscope}%
\begin{pgfscope}%
\definecolor{textcolor}{rgb}{0.000000,0.000000,0.000000}%
\pgfsetstrokecolor{textcolor}%
\pgfsetfillcolor{textcolor}%
\pgftext[x=5.208684in,y=0.438600in,,top]{\color{textcolor}\rmfamily\fontsize{9.000000}{10.800000}\selectfont \(\displaystyle {200}\)}%
\end{pgfscope}%
\begin{pgfscope}%
\definecolor{textcolor}{rgb}{0.000000,0.000000,0.000000}%
\pgfsetstrokecolor{textcolor}%
\pgfsetfillcolor{textcolor}%
\pgftext[x=3.279110in,y=0.272655in,,top]{\color{textcolor}\rmfamily\fontsize{10.000000}{12.000000}\selectfont \(\displaystyle n\): Number of vertices}%
\end{pgfscope}%
\begin{pgfscope}%
\pgfsetbuttcap%
\pgfsetroundjoin%
\definecolor{currentfill}{rgb}{0.000000,0.000000,0.000000}%
\pgfsetfillcolor{currentfill}%
\pgfsetlinewidth{0.803000pt}%
\definecolor{currentstroke}{rgb}{0.000000,0.000000,0.000000}%
\pgfsetstrokecolor{currentstroke}%
\pgfsetdash{}{0pt}%
\pgfsys@defobject{currentmarker}{\pgfqpoint{-0.048611in}{0.000000in}}{\pgfqpoint{-0.000000in}{0.000000in}}{%
\pgfpathmoveto{\pgfqpoint{-0.000000in}{0.000000in}}%
\pgfpathlineto{\pgfqpoint{-0.048611in}{0.000000in}}%
\pgfusepath{stroke,fill}%
}%
\begin{pgfscope}%
\pgfsys@transformshift{0.708220in}{0.535823in}%
\pgfsys@useobject{currentmarker}{}%
\end{pgfscope}%
\end{pgfscope}%
\begin{pgfscope}%
\definecolor{textcolor}{rgb}{0.000000,0.000000,0.000000}%
\pgfsetstrokecolor{textcolor}%
\pgfsetfillcolor{textcolor}%
\pgftext[x=0.344411in, y=0.491098in, left, base]{\color{textcolor}\rmfamily\fontsize{9.000000}{10.800000}\selectfont \(\displaystyle {10^{-1}}\)}%
\end{pgfscope}%
\begin{pgfscope}%
\pgfsetbuttcap%
\pgfsetroundjoin%
\definecolor{currentfill}{rgb}{0.000000,0.000000,0.000000}%
\pgfsetfillcolor{currentfill}%
\pgfsetlinewidth{0.803000pt}%
\definecolor{currentstroke}{rgb}{0.000000,0.000000,0.000000}%
\pgfsetstrokecolor{currentstroke}%
\pgfsetdash{}{0pt}%
\pgfsys@defobject{currentmarker}{\pgfqpoint{-0.048611in}{0.000000in}}{\pgfqpoint{-0.000000in}{0.000000in}}{%
\pgfpathmoveto{\pgfqpoint{-0.000000in}{0.000000in}}%
\pgfpathlineto{\pgfqpoint{-0.048611in}{0.000000in}}%
\pgfusepath{stroke,fill}%
}%
\begin{pgfscope}%
\pgfsys@transformshift{0.708220in}{1.203186in}%
\pgfsys@useobject{currentmarker}{}%
\end{pgfscope}%
\end{pgfscope}%
\begin{pgfscope}%
\definecolor{textcolor}{rgb}{0.000000,0.000000,0.000000}%
\pgfsetstrokecolor{textcolor}%
\pgfsetfillcolor{textcolor}%
\pgftext[x=0.424657in, y=1.158461in, left, base]{\color{textcolor}\rmfamily\fontsize{9.000000}{10.800000}\selectfont \(\displaystyle {10^{0}}\)}%
\end{pgfscope}%
\begin{pgfscope}%
\pgfsetbuttcap%
\pgfsetroundjoin%
\definecolor{currentfill}{rgb}{0.000000,0.000000,0.000000}%
\pgfsetfillcolor{currentfill}%
\pgfsetlinewidth{0.803000pt}%
\definecolor{currentstroke}{rgb}{0.000000,0.000000,0.000000}%
\pgfsetstrokecolor{currentstroke}%
\pgfsetdash{}{0pt}%
\pgfsys@defobject{currentmarker}{\pgfqpoint{-0.048611in}{0.000000in}}{\pgfqpoint{-0.000000in}{0.000000in}}{%
\pgfpathmoveto{\pgfqpoint{-0.000000in}{0.000000in}}%
\pgfpathlineto{\pgfqpoint{-0.048611in}{0.000000in}}%
\pgfusepath{stroke,fill}%
}%
\begin{pgfscope}%
\pgfsys@transformshift{0.708220in}{1.870549in}%
\pgfsys@useobject{currentmarker}{}%
\end{pgfscope}%
\end{pgfscope}%
\begin{pgfscope}%
\definecolor{textcolor}{rgb}{0.000000,0.000000,0.000000}%
\pgfsetstrokecolor{textcolor}%
\pgfsetfillcolor{textcolor}%
\pgftext[x=0.424657in, y=1.825824in, left, base]{\color{textcolor}\rmfamily\fontsize{9.000000}{10.800000}\selectfont \(\displaystyle {10^{1}}\)}%
\end{pgfscope}%
\begin{pgfscope}%
\pgfsetbuttcap%
\pgfsetroundjoin%
\definecolor{currentfill}{rgb}{0.000000,0.000000,0.000000}%
\pgfsetfillcolor{currentfill}%
\pgfsetlinewidth{0.803000pt}%
\definecolor{currentstroke}{rgb}{0.000000,0.000000,0.000000}%
\pgfsetstrokecolor{currentstroke}%
\pgfsetdash{}{0pt}%
\pgfsys@defobject{currentmarker}{\pgfqpoint{-0.048611in}{0.000000in}}{\pgfqpoint{-0.000000in}{0.000000in}}{%
\pgfpathmoveto{\pgfqpoint{-0.000000in}{0.000000in}}%
\pgfpathlineto{\pgfqpoint{-0.048611in}{0.000000in}}%
\pgfusepath{stroke,fill}%
}%
\begin{pgfscope}%
\pgfsys@transformshift{0.708220in}{2.537912in}%
\pgfsys@useobject{currentmarker}{}%
\end{pgfscope}%
\end{pgfscope}%
\begin{pgfscope}%
\definecolor{textcolor}{rgb}{0.000000,0.000000,0.000000}%
\pgfsetstrokecolor{textcolor}%
\pgfsetfillcolor{textcolor}%
\pgftext[x=0.424657in, y=2.493187in, left, base]{\color{textcolor}\rmfamily\fontsize{9.000000}{10.800000}\selectfont \(\displaystyle {10^{2}}\)}%
\end{pgfscope}%
\begin{pgfscope}%
\pgfsetbuttcap%
\pgfsetroundjoin%
\definecolor{currentfill}{rgb}{0.000000,0.000000,0.000000}%
\pgfsetfillcolor{currentfill}%
\pgfsetlinewidth{0.803000pt}%
\definecolor{currentstroke}{rgb}{0.000000,0.000000,0.000000}%
\pgfsetstrokecolor{currentstroke}%
\pgfsetdash{}{0pt}%
\pgfsys@defobject{currentmarker}{\pgfqpoint{-0.048611in}{0.000000in}}{\pgfqpoint{-0.000000in}{0.000000in}}{%
\pgfpathmoveto{\pgfqpoint{-0.000000in}{0.000000in}}%
\pgfpathlineto{\pgfqpoint{-0.048611in}{0.000000in}}%
\pgfusepath{stroke,fill}%
}%
\begin{pgfscope}%
\pgfsys@transformshift{0.708220in}{3.205275in}%
\pgfsys@useobject{currentmarker}{}%
\end{pgfscope}%
\end{pgfscope}%
\begin{pgfscope}%
\definecolor{textcolor}{rgb}{0.000000,0.000000,0.000000}%
\pgfsetstrokecolor{textcolor}%
\pgfsetfillcolor{textcolor}%
\pgftext[x=0.424657in, y=3.160550in, left, base]{\color{textcolor}\rmfamily\fontsize{9.000000}{10.800000}\selectfont \(\displaystyle {10^{3}}\)}%
\end{pgfscope}%
\begin{pgfscope}%
\pgfsetbuttcap%
\pgfsetroundjoin%
\definecolor{currentfill}{rgb}{0.000000,0.000000,0.000000}%
\pgfsetfillcolor{currentfill}%
\pgfsetlinewidth{0.602250pt}%
\definecolor{currentstroke}{rgb}{0.000000,0.000000,0.000000}%
\pgfsetstrokecolor{currentstroke}%
\pgfsetdash{}{0pt}%
\pgfsys@defobject{currentmarker}{\pgfqpoint{-0.027778in}{0.000000in}}{\pgfqpoint{-0.000000in}{0.000000in}}{%
\pgfpathmoveto{\pgfqpoint{-0.000000in}{0.000000in}}%
\pgfpathlineto{\pgfqpoint{-0.027778in}{0.000000in}}%
\pgfusepath{stroke,fill}%
}%
\begin{pgfscope}%
\pgfsys@transformshift{0.708220in}{0.736719in}%
\pgfsys@useobject{currentmarker}{}%
\end{pgfscope}%
\end{pgfscope}%
\begin{pgfscope}%
\pgfsetbuttcap%
\pgfsetroundjoin%
\definecolor{currentfill}{rgb}{0.000000,0.000000,0.000000}%
\pgfsetfillcolor{currentfill}%
\pgfsetlinewidth{0.602250pt}%
\definecolor{currentstroke}{rgb}{0.000000,0.000000,0.000000}%
\pgfsetstrokecolor{currentstroke}%
\pgfsetdash{}{0pt}%
\pgfsys@defobject{currentmarker}{\pgfqpoint{-0.027778in}{0.000000in}}{\pgfqpoint{-0.000000in}{0.000000in}}{%
\pgfpathmoveto{\pgfqpoint{-0.000000in}{0.000000in}}%
\pgfpathlineto{\pgfqpoint{-0.027778in}{0.000000in}}%
\pgfusepath{stroke,fill}%
}%
\begin{pgfscope}%
\pgfsys@transformshift{0.708220in}{0.854236in}%
\pgfsys@useobject{currentmarker}{}%
\end{pgfscope}%
\end{pgfscope}%
\begin{pgfscope}%
\pgfsetbuttcap%
\pgfsetroundjoin%
\definecolor{currentfill}{rgb}{0.000000,0.000000,0.000000}%
\pgfsetfillcolor{currentfill}%
\pgfsetlinewidth{0.602250pt}%
\definecolor{currentstroke}{rgb}{0.000000,0.000000,0.000000}%
\pgfsetstrokecolor{currentstroke}%
\pgfsetdash{}{0pt}%
\pgfsys@defobject{currentmarker}{\pgfqpoint{-0.027778in}{0.000000in}}{\pgfqpoint{-0.000000in}{0.000000in}}{%
\pgfpathmoveto{\pgfqpoint{-0.000000in}{0.000000in}}%
\pgfpathlineto{\pgfqpoint{-0.027778in}{0.000000in}}%
\pgfusepath{stroke,fill}%
}%
\begin{pgfscope}%
\pgfsys@transformshift{0.708220in}{0.937615in}%
\pgfsys@useobject{currentmarker}{}%
\end{pgfscope}%
\end{pgfscope}%
\begin{pgfscope}%
\pgfsetbuttcap%
\pgfsetroundjoin%
\definecolor{currentfill}{rgb}{0.000000,0.000000,0.000000}%
\pgfsetfillcolor{currentfill}%
\pgfsetlinewidth{0.602250pt}%
\definecolor{currentstroke}{rgb}{0.000000,0.000000,0.000000}%
\pgfsetstrokecolor{currentstroke}%
\pgfsetdash{}{0pt}%
\pgfsys@defobject{currentmarker}{\pgfqpoint{-0.027778in}{0.000000in}}{\pgfqpoint{-0.000000in}{0.000000in}}{%
\pgfpathmoveto{\pgfqpoint{-0.000000in}{0.000000in}}%
\pgfpathlineto{\pgfqpoint{-0.027778in}{0.000000in}}%
\pgfusepath{stroke,fill}%
}%
\begin{pgfscope}%
\pgfsys@transformshift{0.708220in}{1.002289in}%
\pgfsys@useobject{currentmarker}{}%
\end{pgfscope}%
\end{pgfscope}%
\begin{pgfscope}%
\pgfsetbuttcap%
\pgfsetroundjoin%
\definecolor{currentfill}{rgb}{0.000000,0.000000,0.000000}%
\pgfsetfillcolor{currentfill}%
\pgfsetlinewidth{0.602250pt}%
\definecolor{currentstroke}{rgb}{0.000000,0.000000,0.000000}%
\pgfsetstrokecolor{currentstroke}%
\pgfsetdash{}{0pt}%
\pgfsys@defobject{currentmarker}{\pgfqpoint{-0.027778in}{0.000000in}}{\pgfqpoint{-0.000000in}{0.000000in}}{%
\pgfpathmoveto{\pgfqpoint{-0.000000in}{0.000000in}}%
\pgfpathlineto{\pgfqpoint{-0.027778in}{0.000000in}}%
\pgfusepath{stroke,fill}%
}%
\begin{pgfscope}%
\pgfsys@transformshift{0.708220in}{1.055132in}%
\pgfsys@useobject{currentmarker}{}%
\end{pgfscope}%
\end{pgfscope}%
\begin{pgfscope}%
\pgfsetbuttcap%
\pgfsetroundjoin%
\definecolor{currentfill}{rgb}{0.000000,0.000000,0.000000}%
\pgfsetfillcolor{currentfill}%
\pgfsetlinewidth{0.602250pt}%
\definecolor{currentstroke}{rgb}{0.000000,0.000000,0.000000}%
\pgfsetstrokecolor{currentstroke}%
\pgfsetdash{}{0pt}%
\pgfsys@defobject{currentmarker}{\pgfqpoint{-0.027778in}{0.000000in}}{\pgfqpoint{-0.000000in}{0.000000in}}{%
\pgfpathmoveto{\pgfqpoint{-0.000000in}{0.000000in}}%
\pgfpathlineto{\pgfqpoint{-0.027778in}{0.000000in}}%
\pgfusepath{stroke,fill}%
}%
\begin{pgfscope}%
\pgfsys@transformshift{0.708220in}{1.099810in}%
\pgfsys@useobject{currentmarker}{}%
\end{pgfscope}%
\end{pgfscope}%
\begin{pgfscope}%
\pgfsetbuttcap%
\pgfsetroundjoin%
\definecolor{currentfill}{rgb}{0.000000,0.000000,0.000000}%
\pgfsetfillcolor{currentfill}%
\pgfsetlinewidth{0.602250pt}%
\definecolor{currentstroke}{rgb}{0.000000,0.000000,0.000000}%
\pgfsetstrokecolor{currentstroke}%
\pgfsetdash{}{0pt}%
\pgfsys@defobject{currentmarker}{\pgfqpoint{-0.027778in}{0.000000in}}{\pgfqpoint{-0.000000in}{0.000000in}}{%
\pgfpathmoveto{\pgfqpoint{-0.000000in}{0.000000in}}%
\pgfpathlineto{\pgfqpoint{-0.027778in}{0.000000in}}%
\pgfusepath{stroke,fill}%
}%
\begin{pgfscope}%
\pgfsys@transformshift{0.708220in}{1.138512in}%
\pgfsys@useobject{currentmarker}{}%
\end{pgfscope}%
\end{pgfscope}%
\begin{pgfscope}%
\pgfsetbuttcap%
\pgfsetroundjoin%
\definecolor{currentfill}{rgb}{0.000000,0.000000,0.000000}%
\pgfsetfillcolor{currentfill}%
\pgfsetlinewidth{0.602250pt}%
\definecolor{currentstroke}{rgb}{0.000000,0.000000,0.000000}%
\pgfsetstrokecolor{currentstroke}%
\pgfsetdash{}{0pt}%
\pgfsys@defobject{currentmarker}{\pgfqpoint{-0.027778in}{0.000000in}}{\pgfqpoint{-0.000000in}{0.000000in}}{%
\pgfpathmoveto{\pgfqpoint{-0.000000in}{0.000000in}}%
\pgfpathlineto{\pgfqpoint{-0.027778in}{0.000000in}}%
\pgfusepath{stroke,fill}%
}%
\begin{pgfscope}%
\pgfsys@transformshift{0.708220in}{1.172649in}%
\pgfsys@useobject{currentmarker}{}%
\end{pgfscope}%
\end{pgfscope}%
\begin{pgfscope}%
\pgfsetbuttcap%
\pgfsetroundjoin%
\definecolor{currentfill}{rgb}{0.000000,0.000000,0.000000}%
\pgfsetfillcolor{currentfill}%
\pgfsetlinewidth{0.602250pt}%
\definecolor{currentstroke}{rgb}{0.000000,0.000000,0.000000}%
\pgfsetstrokecolor{currentstroke}%
\pgfsetdash{}{0pt}%
\pgfsys@defobject{currentmarker}{\pgfqpoint{-0.027778in}{0.000000in}}{\pgfqpoint{-0.000000in}{0.000000in}}{%
\pgfpathmoveto{\pgfqpoint{-0.000000in}{0.000000in}}%
\pgfpathlineto{\pgfqpoint{-0.027778in}{0.000000in}}%
\pgfusepath{stroke,fill}%
}%
\begin{pgfscope}%
\pgfsys@transformshift{0.708220in}{1.404082in}%
\pgfsys@useobject{currentmarker}{}%
\end{pgfscope}%
\end{pgfscope}%
\begin{pgfscope}%
\pgfsetbuttcap%
\pgfsetroundjoin%
\definecolor{currentfill}{rgb}{0.000000,0.000000,0.000000}%
\pgfsetfillcolor{currentfill}%
\pgfsetlinewidth{0.602250pt}%
\definecolor{currentstroke}{rgb}{0.000000,0.000000,0.000000}%
\pgfsetstrokecolor{currentstroke}%
\pgfsetdash{}{0pt}%
\pgfsys@defobject{currentmarker}{\pgfqpoint{-0.027778in}{0.000000in}}{\pgfqpoint{-0.000000in}{0.000000in}}{%
\pgfpathmoveto{\pgfqpoint{-0.000000in}{0.000000in}}%
\pgfpathlineto{\pgfqpoint{-0.027778in}{0.000000in}}%
\pgfusepath{stroke,fill}%
}%
\begin{pgfscope}%
\pgfsys@transformshift{0.708220in}{1.521599in}%
\pgfsys@useobject{currentmarker}{}%
\end{pgfscope}%
\end{pgfscope}%
\begin{pgfscope}%
\pgfsetbuttcap%
\pgfsetroundjoin%
\definecolor{currentfill}{rgb}{0.000000,0.000000,0.000000}%
\pgfsetfillcolor{currentfill}%
\pgfsetlinewidth{0.602250pt}%
\definecolor{currentstroke}{rgb}{0.000000,0.000000,0.000000}%
\pgfsetstrokecolor{currentstroke}%
\pgfsetdash{}{0pt}%
\pgfsys@defobject{currentmarker}{\pgfqpoint{-0.027778in}{0.000000in}}{\pgfqpoint{-0.000000in}{0.000000in}}{%
\pgfpathmoveto{\pgfqpoint{-0.000000in}{0.000000in}}%
\pgfpathlineto{\pgfqpoint{-0.027778in}{0.000000in}}%
\pgfusepath{stroke,fill}%
}%
\begin{pgfscope}%
\pgfsys@transformshift{0.708220in}{1.604978in}%
\pgfsys@useobject{currentmarker}{}%
\end{pgfscope}%
\end{pgfscope}%
\begin{pgfscope}%
\pgfsetbuttcap%
\pgfsetroundjoin%
\definecolor{currentfill}{rgb}{0.000000,0.000000,0.000000}%
\pgfsetfillcolor{currentfill}%
\pgfsetlinewidth{0.602250pt}%
\definecolor{currentstroke}{rgb}{0.000000,0.000000,0.000000}%
\pgfsetstrokecolor{currentstroke}%
\pgfsetdash{}{0pt}%
\pgfsys@defobject{currentmarker}{\pgfqpoint{-0.027778in}{0.000000in}}{\pgfqpoint{-0.000000in}{0.000000in}}{%
\pgfpathmoveto{\pgfqpoint{-0.000000in}{0.000000in}}%
\pgfpathlineto{\pgfqpoint{-0.027778in}{0.000000in}}%
\pgfusepath{stroke,fill}%
}%
\begin{pgfscope}%
\pgfsys@transformshift{0.708220in}{1.669653in}%
\pgfsys@useobject{currentmarker}{}%
\end{pgfscope}%
\end{pgfscope}%
\begin{pgfscope}%
\pgfsetbuttcap%
\pgfsetroundjoin%
\definecolor{currentfill}{rgb}{0.000000,0.000000,0.000000}%
\pgfsetfillcolor{currentfill}%
\pgfsetlinewidth{0.602250pt}%
\definecolor{currentstroke}{rgb}{0.000000,0.000000,0.000000}%
\pgfsetstrokecolor{currentstroke}%
\pgfsetdash{}{0pt}%
\pgfsys@defobject{currentmarker}{\pgfqpoint{-0.027778in}{0.000000in}}{\pgfqpoint{-0.000000in}{0.000000in}}{%
\pgfpathmoveto{\pgfqpoint{-0.000000in}{0.000000in}}%
\pgfpathlineto{\pgfqpoint{-0.027778in}{0.000000in}}%
\pgfusepath{stroke,fill}%
}%
\begin{pgfscope}%
\pgfsys@transformshift{0.708220in}{1.722495in}%
\pgfsys@useobject{currentmarker}{}%
\end{pgfscope}%
\end{pgfscope}%
\begin{pgfscope}%
\pgfsetbuttcap%
\pgfsetroundjoin%
\definecolor{currentfill}{rgb}{0.000000,0.000000,0.000000}%
\pgfsetfillcolor{currentfill}%
\pgfsetlinewidth{0.602250pt}%
\definecolor{currentstroke}{rgb}{0.000000,0.000000,0.000000}%
\pgfsetstrokecolor{currentstroke}%
\pgfsetdash{}{0pt}%
\pgfsys@defobject{currentmarker}{\pgfqpoint{-0.027778in}{0.000000in}}{\pgfqpoint{-0.000000in}{0.000000in}}{%
\pgfpathmoveto{\pgfqpoint{-0.000000in}{0.000000in}}%
\pgfpathlineto{\pgfqpoint{-0.027778in}{0.000000in}}%
\pgfusepath{stroke,fill}%
}%
\begin{pgfscope}%
\pgfsys@transformshift{0.708220in}{1.767173in}%
\pgfsys@useobject{currentmarker}{}%
\end{pgfscope}%
\end{pgfscope}%
\begin{pgfscope}%
\pgfsetbuttcap%
\pgfsetroundjoin%
\definecolor{currentfill}{rgb}{0.000000,0.000000,0.000000}%
\pgfsetfillcolor{currentfill}%
\pgfsetlinewidth{0.602250pt}%
\definecolor{currentstroke}{rgb}{0.000000,0.000000,0.000000}%
\pgfsetstrokecolor{currentstroke}%
\pgfsetdash{}{0pt}%
\pgfsys@defobject{currentmarker}{\pgfqpoint{-0.027778in}{0.000000in}}{\pgfqpoint{-0.000000in}{0.000000in}}{%
\pgfpathmoveto{\pgfqpoint{-0.000000in}{0.000000in}}%
\pgfpathlineto{\pgfqpoint{-0.027778in}{0.000000in}}%
\pgfusepath{stroke,fill}%
}%
\begin{pgfscope}%
\pgfsys@transformshift{0.708220in}{1.805875in}%
\pgfsys@useobject{currentmarker}{}%
\end{pgfscope}%
\end{pgfscope}%
\begin{pgfscope}%
\pgfsetbuttcap%
\pgfsetroundjoin%
\definecolor{currentfill}{rgb}{0.000000,0.000000,0.000000}%
\pgfsetfillcolor{currentfill}%
\pgfsetlinewidth{0.602250pt}%
\definecolor{currentstroke}{rgb}{0.000000,0.000000,0.000000}%
\pgfsetstrokecolor{currentstroke}%
\pgfsetdash{}{0pt}%
\pgfsys@defobject{currentmarker}{\pgfqpoint{-0.027778in}{0.000000in}}{\pgfqpoint{-0.000000in}{0.000000in}}{%
\pgfpathmoveto{\pgfqpoint{-0.000000in}{0.000000in}}%
\pgfpathlineto{\pgfqpoint{-0.027778in}{0.000000in}}%
\pgfusepath{stroke,fill}%
}%
\begin{pgfscope}%
\pgfsys@transformshift{0.708220in}{1.840012in}%
\pgfsys@useobject{currentmarker}{}%
\end{pgfscope}%
\end{pgfscope}%
\begin{pgfscope}%
\pgfsetbuttcap%
\pgfsetroundjoin%
\definecolor{currentfill}{rgb}{0.000000,0.000000,0.000000}%
\pgfsetfillcolor{currentfill}%
\pgfsetlinewidth{0.602250pt}%
\definecolor{currentstroke}{rgb}{0.000000,0.000000,0.000000}%
\pgfsetstrokecolor{currentstroke}%
\pgfsetdash{}{0pt}%
\pgfsys@defobject{currentmarker}{\pgfqpoint{-0.027778in}{0.000000in}}{\pgfqpoint{-0.000000in}{0.000000in}}{%
\pgfpathmoveto{\pgfqpoint{-0.000000in}{0.000000in}}%
\pgfpathlineto{\pgfqpoint{-0.027778in}{0.000000in}}%
\pgfusepath{stroke,fill}%
}%
\begin{pgfscope}%
\pgfsys@transformshift{0.708220in}{2.071445in}%
\pgfsys@useobject{currentmarker}{}%
\end{pgfscope}%
\end{pgfscope}%
\begin{pgfscope}%
\pgfsetbuttcap%
\pgfsetroundjoin%
\definecolor{currentfill}{rgb}{0.000000,0.000000,0.000000}%
\pgfsetfillcolor{currentfill}%
\pgfsetlinewidth{0.602250pt}%
\definecolor{currentstroke}{rgb}{0.000000,0.000000,0.000000}%
\pgfsetstrokecolor{currentstroke}%
\pgfsetdash{}{0pt}%
\pgfsys@defobject{currentmarker}{\pgfqpoint{-0.027778in}{0.000000in}}{\pgfqpoint{-0.000000in}{0.000000in}}{%
\pgfpathmoveto{\pgfqpoint{-0.000000in}{0.000000in}}%
\pgfpathlineto{\pgfqpoint{-0.027778in}{0.000000in}}%
\pgfusepath{stroke,fill}%
}%
\begin{pgfscope}%
\pgfsys@transformshift{0.708220in}{2.188962in}%
\pgfsys@useobject{currentmarker}{}%
\end{pgfscope}%
\end{pgfscope}%
\begin{pgfscope}%
\pgfsetbuttcap%
\pgfsetroundjoin%
\definecolor{currentfill}{rgb}{0.000000,0.000000,0.000000}%
\pgfsetfillcolor{currentfill}%
\pgfsetlinewidth{0.602250pt}%
\definecolor{currentstroke}{rgb}{0.000000,0.000000,0.000000}%
\pgfsetstrokecolor{currentstroke}%
\pgfsetdash{}{0pt}%
\pgfsys@defobject{currentmarker}{\pgfqpoint{-0.027778in}{0.000000in}}{\pgfqpoint{-0.000000in}{0.000000in}}{%
\pgfpathmoveto{\pgfqpoint{-0.000000in}{0.000000in}}%
\pgfpathlineto{\pgfqpoint{-0.027778in}{0.000000in}}%
\pgfusepath{stroke,fill}%
}%
\begin{pgfscope}%
\pgfsys@transformshift{0.708220in}{2.272342in}%
\pgfsys@useobject{currentmarker}{}%
\end{pgfscope}%
\end{pgfscope}%
\begin{pgfscope}%
\pgfsetbuttcap%
\pgfsetroundjoin%
\definecolor{currentfill}{rgb}{0.000000,0.000000,0.000000}%
\pgfsetfillcolor{currentfill}%
\pgfsetlinewidth{0.602250pt}%
\definecolor{currentstroke}{rgb}{0.000000,0.000000,0.000000}%
\pgfsetstrokecolor{currentstroke}%
\pgfsetdash{}{0pt}%
\pgfsys@defobject{currentmarker}{\pgfqpoint{-0.027778in}{0.000000in}}{\pgfqpoint{-0.000000in}{0.000000in}}{%
\pgfpathmoveto{\pgfqpoint{-0.000000in}{0.000000in}}%
\pgfpathlineto{\pgfqpoint{-0.027778in}{0.000000in}}%
\pgfusepath{stroke,fill}%
}%
\begin{pgfscope}%
\pgfsys@transformshift{0.708220in}{2.337016in}%
\pgfsys@useobject{currentmarker}{}%
\end{pgfscope}%
\end{pgfscope}%
\begin{pgfscope}%
\pgfsetbuttcap%
\pgfsetroundjoin%
\definecolor{currentfill}{rgb}{0.000000,0.000000,0.000000}%
\pgfsetfillcolor{currentfill}%
\pgfsetlinewidth{0.602250pt}%
\definecolor{currentstroke}{rgb}{0.000000,0.000000,0.000000}%
\pgfsetstrokecolor{currentstroke}%
\pgfsetdash{}{0pt}%
\pgfsys@defobject{currentmarker}{\pgfqpoint{-0.027778in}{0.000000in}}{\pgfqpoint{-0.000000in}{0.000000in}}{%
\pgfpathmoveto{\pgfqpoint{-0.000000in}{0.000000in}}%
\pgfpathlineto{\pgfqpoint{-0.027778in}{0.000000in}}%
\pgfusepath{stroke,fill}%
}%
\begin{pgfscope}%
\pgfsys@transformshift{0.708220in}{2.389858in}%
\pgfsys@useobject{currentmarker}{}%
\end{pgfscope}%
\end{pgfscope}%
\begin{pgfscope}%
\pgfsetbuttcap%
\pgfsetroundjoin%
\definecolor{currentfill}{rgb}{0.000000,0.000000,0.000000}%
\pgfsetfillcolor{currentfill}%
\pgfsetlinewidth{0.602250pt}%
\definecolor{currentstroke}{rgb}{0.000000,0.000000,0.000000}%
\pgfsetstrokecolor{currentstroke}%
\pgfsetdash{}{0pt}%
\pgfsys@defobject{currentmarker}{\pgfqpoint{-0.027778in}{0.000000in}}{\pgfqpoint{-0.000000in}{0.000000in}}{%
\pgfpathmoveto{\pgfqpoint{-0.000000in}{0.000000in}}%
\pgfpathlineto{\pgfqpoint{-0.027778in}{0.000000in}}%
\pgfusepath{stroke,fill}%
}%
\begin{pgfscope}%
\pgfsys@transformshift{0.708220in}{2.434536in}%
\pgfsys@useobject{currentmarker}{}%
\end{pgfscope}%
\end{pgfscope}%
\begin{pgfscope}%
\pgfsetbuttcap%
\pgfsetroundjoin%
\definecolor{currentfill}{rgb}{0.000000,0.000000,0.000000}%
\pgfsetfillcolor{currentfill}%
\pgfsetlinewidth{0.602250pt}%
\definecolor{currentstroke}{rgb}{0.000000,0.000000,0.000000}%
\pgfsetstrokecolor{currentstroke}%
\pgfsetdash{}{0pt}%
\pgfsys@defobject{currentmarker}{\pgfqpoint{-0.027778in}{0.000000in}}{\pgfqpoint{-0.000000in}{0.000000in}}{%
\pgfpathmoveto{\pgfqpoint{-0.000000in}{0.000000in}}%
\pgfpathlineto{\pgfqpoint{-0.027778in}{0.000000in}}%
\pgfusepath{stroke,fill}%
}%
\begin{pgfscope}%
\pgfsys@transformshift{0.708220in}{2.473238in}%
\pgfsys@useobject{currentmarker}{}%
\end{pgfscope}%
\end{pgfscope}%
\begin{pgfscope}%
\pgfsetbuttcap%
\pgfsetroundjoin%
\definecolor{currentfill}{rgb}{0.000000,0.000000,0.000000}%
\pgfsetfillcolor{currentfill}%
\pgfsetlinewidth{0.602250pt}%
\definecolor{currentstroke}{rgb}{0.000000,0.000000,0.000000}%
\pgfsetstrokecolor{currentstroke}%
\pgfsetdash{}{0pt}%
\pgfsys@defobject{currentmarker}{\pgfqpoint{-0.027778in}{0.000000in}}{\pgfqpoint{-0.000000in}{0.000000in}}{%
\pgfpathmoveto{\pgfqpoint{-0.000000in}{0.000000in}}%
\pgfpathlineto{\pgfqpoint{-0.027778in}{0.000000in}}%
\pgfusepath{stroke,fill}%
}%
\begin{pgfscope}%
\pgfsys@transformshift{0.708220in}{2.507375in}%
\pgfsys@useobject{currentmarker}{}%
\end{pgfscope}%
\end{pgfscope}%
\begin{pgfscope}%
\pgfsetbuttcap%
\pgfsetroundjoin%
\definecolor{currentfill}{rgb}{0.000000,0.000000,0.000000}%
\pgfsetfillcolor{currentfill}%
\pgfsetlinewidth{0.602250pt}%
\definecolor{currentstroke}{rgb}{0.000000,0.000000,0.000000}%
\pgfsetstrokecolor{currentstroke}%
\pgfsetdash{}{0pt}%
\pgfsys@defobject{currentmarker}{\pgfqpoint{-0.027778in}{0.000000in}}{\pgfqpoint{-0.000000in}{0.000000in}}{%
\pgfpathmoveto{\pgfqpoint{-0.000000in}{0.000000in}}%
\pgfpathlineto{\pgfqpoint{-0.027778in}{0.000000in}}%
\pgfusepath{stroke,fill}%
}%
\begin{pgfscope}%
\pgfsys@transformshift{0.708220in}{2.738808in}%
\pgfsys@useobject{currentmarker}{}%
\end{pgfscope}%
\end{pgfscope}%
\begin{pgfscope}%
\pgfsetbuttcap%
\pgfsetroundjoin%
\definecolor{currentfill}{rgb}{0.000000,0.000000,0.000000}%
\pgfsetfillcolor{currentfill}%
\pgfsetlinewidth{0.602250pt}%
\definecolor{currentstroke}{rgb}{0.000000,0.000000,0.000000}%
\pgfsetstrokecolor{currentstroke}%
\pgfsetdash{}{0pt}%
\pgfsys@defobject{currentmarker}{\pgfqpoint{-0.027778in}{0.000000in}}{\pgfqpoint{-0.000000in}{0.000000in}}{%
\pgfpathmoveto{\pgfqpoint{-0.000000in}{0.000000in}}%
\pgfpathlineto{\pgfqpoint{-0.027778in}{0.000000in}}%
\pgfusepath{stroke,fill}%
}%
\begin{pgfscope}%
\pgfsys@transformshift{0.708220in}{2.856325in}%
\pgfsys@useobject{currentmarker}{}%
\end{pgfscope}%
\end{pgfscope}%
\begin{pgfscope}%
\pgfsetbuttcap%
\pgfsetroundjoin%
\definecolor{currentfill}{rgb}{0.000000,0.000000,0.000000}%
\pgfsetfillcolor{currentfill}%
\pgfsetlinewidth{0.602250pt}%
\definecolor{currentstroke}{rgb}{0.000000,0.000000,0.000000}%
\pgfsetstrokecolor{currentstroke}%
\pgfsetdash{}{0pt}%
\pgfsys@defobject{currentmarker}{\pgfqpoint{-0.027778in}{0.000000in}}{\pgfqpoint{-0.000000in}{0.000000in}}{%
\pgfpathmoveto{\pgfqpoint{-0.000000in}{0.000000in}}%
\pgfpathlineto{\pgfqpoint{-0.027778in}{0.000000in}}%
\pgfusepath{stroke,fill}%
}%
\begin{pgfscope}%
\pgfsys@transformshift{0.708220in}{2.939705in}%
\pgfsys@useobject{currentmarker}{}%
\end{pgfscope}%
\end{pgfscope}%
\begin{pgfscope}%
\pgfsetbuttcap%
\pgfsetroundjoin%
\definecolor{currentfill}{rgb}{0.000000,0.000000,0.000000}%
\pgfsetfillcolor{currentfill}%
\pgfsetlinewidth{0.602250pt}%
\definecolor{currentstroke}{rgb}{0.000000,0.000000,0.000000}%
\pgfsetstrokecolor{currentstroke}%
\pgfsetdash{}{0pt}%
\pgfsys@defobject{currentmarker}{\pgfqpoint{-0.027778in}{0.000000in}}{\pgfqpoint{-0.000000in}{0.000000in}}{%
\pgfpathmoveto{\pgfqpoint{-0.000000in}{0.000000in}}%
\pgfpathlineto{\pgfqpoint{-0.027778in}{0.000000in}}%
\pgfusepath{stroke,fill}%
}%
\begin{pgfscope}%
\pgfsys@transformshift{0.708220in}{3.004379in}%
\pgfsys@useobject{currentmarker}{}%
\end{pgfscope}%
\end{pgfscope}%
\begin{pgfscope}%
\pgfsetbuttcap%
\pgfsetroundjoin%
\definecolor{currentfill}{rgb}{0.000000,0.000000,0.000000}%
\pgfsetfillcolor{currentfill}%
\pgfsetlinewidth{0.602250pt}%
\definecolor{currentstroke}{rgb}{0.000000,0.000000,0.000000}%
\pgfsetstrokecolor{currentstroke}%
\pgfsetdash{}{0pt}%
\pgfsys@defobject{currentmarker}{\pgfqpoint{-0.027778in}{0.000000in}}{\pgfqpoint{-0.000000in}{0.000000in}}{%
\pgfpathmoveto{\pgfqpoint{-0.000000in}{0.000000in}}%
\pgfpathlineto{\pgfqpoint{-0.027778in}{0.000000in}}%
\pgfusepath{stroke,fill}%
}%
\begin{pgfscope}%
\pgfsys@transformshift{0.708220in}{3.057222in}%
\pgfsys@useobject{currentmarker}{}%
\end{pgfscope}%
\end{pgfscope}%
\begin{pgfscope}%
\pgfsetbuttcap%
\pgfsetroundjoin%
\definecolor{currentfill}{rgb}{0.000000,0.000000,0.000000}%
\pgfsetfillcolor{currentfill}%
\pgfsetlinewidth{0.602250pt}%
\definecolor{currentstroke}{rgb}{0.000000,0.000000,0.000000}%
\pgfsetstrokecolor{currentstroke}%
\pgfsetdash{}{0pt}%
\pgfsys@defobject{currentmarker}{\pgfqpoint{-0.027778in}{0.000000in}}{\pgfqpoint{-0.000000in}{0.000000in}}{%
\pgfpathmoveto{\pgfqpoint{-0.000000in}{0.000000in}}%
\pgfpathlineto{\pgfqpoint{-0.027778in}{0.000000in}}%
\pgfusepath{stroke,fill}%
}%
\begin{pgfscope}%
\pgfsys@transformshift{0.708220in}{3.101899in}%
\pgfsys@useobject{currentmarker}{}%
\end{pgfscope}%
\end{pgfscope}%
\begin{pgfscope}%
\pgfsetbuttcap%
\pgfsetroundjoin%
\definecolor{currentfill}{rgb}{0.000000,0.000000,0.000000}%
\pgfsetfillcolor{currentfill}%
\pgfsetlinewidth{0.602250pt}%
\definecolor{currentstroke}{rgb}{0.000000,0.000000,0.000000}%
\pgfsetstrokecolor{currentstroke}%
\pgfsetdash{}{0pt}%
\pgfsys@defobject{currentmarker}{\pgfqpoint{-0.027778in}{0.000000in}}{\pgfqpoint{-0.000000in}{0.000000in}}{%
\pgfpathmoveto{\pgfqpoint{-0.000000in}{0.000000in}}%
\pgfpathlineto{\pgfqpoint{-0.027778in}{0.000000in}}%
\pgfusepath{stroke,fill}%
}%
\begin{pgfscope}%
\pgfsys@transformshift{0.708220in}{3.140601in}%
\pgfsys@useobject{currentmarker}{}%
\end{pgfscope}%
\end{pgfscope}%
\begin{pgfscope}%
\pgfsetbuttcap%
\pgfsetroundjoin%
\definecolor{currentfill}{rgb}{0.000000,0.000000,0.000000}%
\pgfsetfillcolor{currentfill}%
\pgfsetlinewidth{0.602250pt}%
\definecolor{currentstroke}{rgb}{0.000000,0.000000,0.000000}%
\pgfsetstrokecolor{currentstroke}%
\pgfsetdash{}{0pt}%
\pgfsys@defobject{currentmarker}{\pgfqpoint{-0.027778in}{0.000000in}}{\pgfqpoint{-0.000000in}{0.000000in}}{%
\pgfpathmoveto{\pgfqpoint{-0.000000in}{0.000000in}}%
\pgfpathlineto{\pgfqpoint{-0.027778in}{0.000000in}}%
\pgfusepath{stroke,fill}%
}%
\begin{pgfscope}%
\pgfsys@transformshift{0.708220in}{3.174738in}%
\pgfsys@useobject{currentmarker}{}%
\end{pgfscope}%
\end{pgfscope}%
\begin{pgfscope}%
\definecolor{textcolor}{rgb}{0.000000,0.000000,0.000000}%
\pgfsetstrokecolor{textcolor}%
\pgfsetfillcolor{textcolor}%
\pgftext[x=0.288855in,y=1.870549in,,bottom,rotate=90.000000]{\color{textcolor}\rmfamily\fontsize{10.000000}{12.000000}\selectfont Median solving time (s)}%
\end{pgfscope}%
\begin{pgfscope}%
\pgfpathrectangle{\pgfqpoint{0.708220in}{0.535823in}}{\pgfqpoint{5.141780in}{2.669453in}}%
\pgfusepath{clip}%
\pgfsetrectcap%
\pgfsetroundjoin%
\pgfsetlinewidth{1.003750pt}%
\definecolor{currentstroke}{rgb}{0.866667,0.058824,0.058824}%
\pgfsetstrokecolor{currentstroke}%
\pgfsetdash{}{0pt}%
\pgfpathmoveto{\pgfqpoint{1.928026in}{0.525823in}}%
\pgfpathlineto{\pgfqpoint{2.058359in}{0.761651in}}%
\pgfpathlineto{\pgfqpoint{2.283382in}{1.194612in}}%
\pgfpathlineto{\pgfqpoint{2.508405in}{1.631965in}}%
\pgfpathlineto{\pgfqpoint{2.733429in}{2.079842in}}%
\pgfpathlineto{\pgfqpoint{2.958452in}{2.526097in}}%
\pgfpathlineto{\pgfqpoint{3.183475in}{3.000213in}}%
\pgfusepath{stroke}%
\end{pgfscope}%
\begin{pgfscope}%
\pgfpathrectangle{\pgfqpoint{0.708220in}{0.535823in}}{\pgfqpoint{5.141780in}{2.669453in}}%
\pgfusepath{clip}%
\pgfsetbuttcap%
\pgfsetmiterjoin%
\definecolor{currentfill}{rgb}{0.866667,0.058824,0.058824}%
\pgfsetfillcolor{currentfill}%
\pgfsetlinewidth{0.501875pt}%
\definecolor{currentstroke}{rgb}{0.000000,0.000000,0.000000}%
\pgfsetstrokecolor{currentstroke}%
\pgfsetdash{}{0pt}%
\pgfsys@defobject{currentmarker}{\pgfqpoint{-0.033023in}{-0.028091in}}{\pgfqpoint{0.033023in}{0.034722in}}{%
\pgfpathmoveto{\pgfqpoint{0.000000in}{0.034722in}}%
\pgfpathlineto{\pgfqpoint{-0.033023in}{0.010730in}}%
\pgfpathlineto{\pgfqpoint{-0.020409in}{-0.028091in}}%
\pgfpathlineto{\pgfqpoint{0.020409in}{-0.028091in}}%
\pgfpathlineto{\pgfqpoint{0.033023in}{0.010730in}}%
\pgfpathclose%
\pgfusepath{stroke,fill}%
}%
\begin{pgfscope}%
\pgfsys@transformshift{1.833336in}{0.354487in}%
\pgfsys@useobject{currentmarker}{}%
\end{pgfscope}%
\begin{pgfscope}%
\pgfsys@transformshift{2.058359in}{0.761651in}%
\pgfsys@useobject{currentmarker}{}%
\end{pgfscope}%
\begin{pgfscope}%
\pgfsys@transformshift{2.283382in}{1.194612in}%
\pgfsys@useobject{currentmarker}{}%
\end{pgfscope}%
\begin{pgfscope}%
\pgfsys@transformshift{2.508405in}{1.631965in}%
\pgfsys@useobject{currentmarker}{}%
\end{pgfscope}%
\begin{pgfscope}%
\pgfsys@transformshift{2.733429in}{2.079842in}%
\pgfsys@useobject{currentmarker}{}%
\end{pgfscope}%
\begin{pgfscope}%
\pgfsys@transformshift{2.958452in}{2.526097in}%
\pgfsys@useobject{currentmarker}{}%
\end{pgfscope}%
\begin{pgfscope}%
\pgfsys@transformshift{3.183475in}{3.000213in}%
\pgfsys@useobject{currentmarker}{}%
\end{pgfscope}%
\end{pgfscope}%
\begin{pgfscope}%
\pgfpathrectangle{\pgfqpoint{0.708220in}{0.535823in}}{\pgfqpoint{5.141780in}{2.669453in}}%
\pgfusepath{clip}%
\pgfsetrectcap%
\pgfsetroundjoin%
\pgfsetlinewidth{1.003750pt}%
\definecolor{currentstroke}{rgb}{0.000000,0.000000,0.200000}%
\pgfsetstrokecolor{currentstroke}%
\pgfsetdash{}{0pt}%
\pgfpathmoveto{\pgfqpoint{1.833336in}{0.676718in}}%
\pgfpathlineto{\pgfqpoint{2.058359in}{0.906415in}}%
\pgfpathlineto{\pgfqpoint{2.283382in}{1.295336in}}%
\pgfpathlineto{\pgfqpoint{2.508405in}{1.701898in}}%
\pgfpathlineto{\pgfqpoint{2.733429in}{1.970334in}}%
\pgfpathlineto{\pgfqpoint{2.958452in}{2.394586in}}%
\pgfpathlineto{\pgfqpoint{3.183475in}{2.793972in}}%
\pgfusepath{stroke}%
\end{pgfscope}%
\begin{pgfscope}%
\pgfpathrectangle{\pgfqpoint{0.708220in}{0.535823in}}{\pgfqpoint{5.141780in}{2.669453in}}%
\pgfusepath{clip}%
\pgfsetbuttcap%
\pgfsetmiterjoin%
\definecolor{currentfill}{rgb}{0.000000,0.000000,0.200000}%
\pgfsetfillcolor{currentfill}%
\pgfsetlinewidth{0.501875pt}%
\definecolor{currentstroke}{rgb}{0.000000,0.000000,0.000000}%
\pgfsetstrokecolor{currentstroke}%
\pgfsetdash{}{0pt}%
\pgfsys@defobject{currentmarker}{\pgfqpoint{-0.034722in}{-0.034722in}}{\pgfqpoint{0.034722in}{0.034722in}}{%
\pgfpathmoveto{\pgfqpoint{-0.011574in}{-0.034722in}}%
\pgfpathlineto{\pgfqpoint{0.011574in}{-0.034722in}}%
\pgfpathlineto{\pgfqpoint{0.011574in}{-0.011574in}}%
\pgfpathlineto{\pgfqpoint{0.034722in}{-0.011574in}}%
\pgfpathlineto{\pgfqpoint{0.034722in}{0.011574in}}%
\pgfpathlineto{\pgfqpoint{0.011574in}{0.011574in}}%
\pgfpathlineto{\pgfqpoint{0.011574in}{0.034722in}}%
\pgfpathlineto{\pgfqpoint{-0.011574in}{0.034722in}}%
\pgfpathlineto{\pgfqpoint{-0.011574in}{0.011574in}}%
\pgfpathlineto{\pgfqpoint{-0.034722in}{0.011574in}}%
\pgfpathlineto{\pgfqpoint{-0.034722in}{-0.011574in}}%
\pgfpathlineto{\pgfqpoint{-0.011574in}{-0.011574in}}%
\pgfpathclose%
\pgfusepath{stroke,fill}%
}%
\begin{pgfscope}%
\pgfsys@transformshift{1.833336in}{0.676718in}%
\pgfsys@useobject{currentmarker}{}%
\end{pgfscope}%
\begin{pgfscope}%
\pgfsys@transformshift{2.058359in}{0.906415in}%
\pgfsys@useobject{currentmarker}{}%
\end{pgfscope}%
\begin{pgfscope}%
\pgfsys@transformshift{2.283382in}{1.295336in}%
\pgfsys@useobject{currentmarker}{}%
\end{pgfscope}%
\begin{pgfscope}%
\pgfsys@transformshift{2.508405in}{1.701898in}%
\pgfsys@useobject{currentmarker}{}%
\end{pgfscope}%
\begin{pgfscope}%
\pgfsys@transformshift{2.733429in}{1.970334in}%
\pgfsys@useobject{currentmarker}{}%
\end{pgfscope}%
\begin{pgfscope}%
\pgfsys@transformshift{2.958452in}{2.394586in}%
\pgfsys@useobject{currentmarker}{}%
\end{pgfscope}%
\begin{pgfscope}%
\pgfsys@transformshift{3.183475in}{2.793972in}%
\pgfsys@useobject{currentmarker}{}%
\end{pgfscope}%
\end{pgfscope}%
\begin{pgfscope}%
\pgfpathrectangle{\pgfqpoint{0.708220in}{0.535823in}}{\pgfqpoint{5.141780in}{2.669453in}}%
\pgfusepath{clip}%
\pgfsetrectcap%
\pgfsetroundjoin%
\pgfsetlinewidth{1.003750pt}%
\definecolor{currentstroke}{rgb}{0.000000,0.000000,0.866667}%
\pgfsetstrokecolor{currentstroke}%
\pgfsetdash{}{0pt}%
\pgfpathmoveto{\pgfqpoint{1.833336in}{0.610115in}}%
\pgfpathlineto{\pgfqpoint{2.058359in}{0.767859in}}%
\pgfpathlineto{\pgfqpoint{2.283382in}{1.115023in}}%
\pgfpathlineto{\pgfqpoint{2.508405in}{1.546157in}}%
\pgfpathlineto{\pgfqpoint{2.733429in}{1.940281in}}%
\pgfpathlineto{\pgfqpoint{2.958452in}{2.294916in}}%
\pgfpathlineto{\pgfqpoint{3.183475in}{2.632491in}}%
\pgfpathlineto{\pgfqpoint{3.408498in}{3.043909in}}%
\pgfusepath{stroke}%
\end{pgfscope}%
\begin{pgfscope}%
\pgfpathrectangle{\pgfqpoint{0.708220in}{0.535823in}}{\pgfqpoint{5.141780in}{2.669453in}}%
\pgfusepath{clip}%
\pgfsetbuttcap%
\pgfsetmiterjoin%
\definecolor{currentfill}{rgb}{0.000000,0.000000,0.866667}%
\pgfsetfillcolor{currentfill}%
\pgfsetlinewidth{0.501875pt}%
\definecolor{currentstroke}{rgb}{0.000000,0.000000,0.000000}%
\pgfsetstrokecolor{currentstroke}%
\pgfsetdash{}{0pt}%
\pgfsys@defobject{currentmarker}{\pgfqpoint{-0.029463in}{-0.049105in}}{\pgfqpoint{0.029463in}{0.049105in}}{%
\pgfpathmoveto{\pgfqpoint{0.000000in}{-0.049105in}}%
\pgfpathlineto{\pgfqpoint{0.029463in}{0.000000in}}%
\pgfpathlineto{\pgfqpoint{0.000000in}{0.049105in}}%
\pgfpathlineto{\pgfqpoint{-0.029463in}{0.000000in}}%
\pgfpathclose%
\pgfusepath{stroke,fill}%
}%
\begin{pgfscope}%
\pgfsys@transformshift{1.833336in}{0.610115in}%
\pgfsys@useobject{currentmarker}{}%
\end{pgfscope}%
\begin{pgfscope}%
\pgfsys@transformshift{2.058359in}{0.767859in}%
\pgfsys@useobject{currentmarker}{}%
\end{pgfscope}%
\begin{pgfscope}%
\pgfsys@transformshift{2.283382in}{1.115023in}%
\pgfsys@useobject{currentmarker}{}%
\end{pgfscope}%
\begin{pgfscope}%
\pgfsys@transformshift{2.508405in}{1.546157in}%
\pgfsys@useobject{currentmarker}{}%
\end{pgfscope}%
\begin{pgfscope}%
\pgfsys@transformshift{2.733429in}{1.940281in}%
\pgfsys@useobject{currentmarker}{}%
\end{pgfscope}%
\begin{pgfscope}%
\pgfsys@transformshift{2.958452in}{2.294916in}%
\pgfsys@useobject{currentmarker}{}%
\end{pgfscope}%
\begin{pgfscope}%
\pgfsys@transformshift{3.183475in}{2.632491in}%
\pgfsys@useobject{currentmarker}{}%
\end{pgfscope}%
\begin{pgfscope}%
\pgfsys@transformshift{3.408498in}{3.043909in}%
\pgfsys@useobject{currentmarker}{}%
\end{pgfscope}%
\end{pgfscope}%
\begin{pgfscope}%
\pgfpathrectangle{\pgfqpoint{0.708220in}{0.535823in}}{\pgfqpoint{5.141780in}{2.669453in}}%
\pgfusepath{clip}%
\pgfsetrectcap%
\pgfsetroundjoin%
\pgfsetlinewidth{1.003750pt}%
\definecolor{currentstroke}{rgb}{0.250980,0.231373,0.796078}%
\pgfsetstrokecolor{currentstroke}%
\pgfsetdash{}{0pt}%
\pgfpathmoveto{\pgfqpoint{2.095649in}{0.525823in}}%
\pgfpathlineto{\pgfqpoint{2.283382in}{0.846558in}}%
\pgfpathlineto{\pgfqpoint{2.508405in}{1.257717in}}%
\pgfpathlineto{\pgfqpoint{2.733429in}{1.657480in}}%
\pgfpathlineto{\pgfqpoint{2.958452in}{2.073652in}}%
\pgfpathlineto{\pgfqpoint{3.183475in}{2.469956in}}%
\pgfpathlineto{\pgfqpoint{3.408498in}{2.885012in}}%
\pgfusepath{stroke}%
\end{pgfscope}%
\begin{pgfscope}%
\pgfpathrectangle{\pgfqpoint{0.708220in}{0.535823in}}{\pgfqpoint{5.141780in}{2.669453in}}%
\pgfusepath{clip}%
\pgfsetbuttcap%
\pgfsetmiterjoin%
\definecolor{currentfill}{rgb}{0.250980,0.231373,0.796078}%
\pgfsetfillcolor{currentfill}%
\pgfsetlinewidth{0.501875pt}%
\definecolor{currentstroke}{rgb}{0.000000,0.000000,0.000000}%
\pgfsetstrokecolor{currentstroke}%
\pgfsetdash{}{0pt}%
\pgfsys@defobject{currentmarker}{\pgfqpoint{-0.034722in}{-0.034722in}}{\pgfqpoint{0.034722in}{0.034722in}}{%
\pgfpathmoveto{\pgfqpoint{-0.034722in}{-0.034722in}}%
\pgfpathlineto{\pgfqpoint{0.034722in}{-0.034722in}}%
\pgfpathlineto{\pgfqpoint{0.034722in}{0.034722in}}%
\pgfpathlineto{\pgfqpoint{-0.034722in}{0.034722in}}%
\pgfpathclose%
\pgfusepath{stroke,fill}%
}%
\begin{pgfscope}%
\pgfsys@transformshift{1.833336in}{0.132041in}%
\pgfsys@useobject{currentmarker}{}%
\end{pgfscope}%
\begin{pgfscope}%
\pgfsys@transformshift{2.058359in}{0.462115in}%
\pgfsys@useobject{currentmarker}{}%
\end{pgfscope}%
\begin{pgfscope}%
\pgfsys@transformshift{2.283382in}{0.846558in}%
\pgfsys@useobject{currentmarker}{}%
\end{pgfscope}%
\begin{pgfscope}%
\pgfsys@transformshift{2.508405in}{1.257717in}%
\pgfsys@useobject{currentmarker}{}%
\end{pgfscope}%
\begin{pgfscope}%
\pgfsys@transformshift{2.733429in}{1.657480in}%
\pgfsys@useobject{currentmarker}{}%
\end{pgfscope}%
\begin{pgfscope}%
\pgfsys@transformshift{2.958452in}{2.073652in}%
\pgfsys@useobject{currentmarker}{}%
\end{pgfscope}%
\begin{pgfscope}%
\pgfsys@transformshift{3.183475in}{2.469956in}%
\pgfsys@useobject{currentmarker}{}%
\end{pgfscope}%
\begin{pgfscope}%
\pgfsys@transformshift{3.408498in}{2.885012in}%
\pgfsys@useobject{currentmarker}{}%
\end{pgfscope}%
\end{pgfscope}%
\begin{pgfscope}%
\pgfpathrectangle{\pgfqpoint{0.708220in}{0.535823in}}{\pgfqpoint{5.141780in}{2.669453in}}%
\pgfusepath{clip}%
\pgfsetrectcap%
\pgfsetroundjoin%
\pgfsetlinewidth{1.003750pt}%
\definecolor{currentstroke}{rgb}{0.615686,0.007843,0.843137}%
\pgfsetstrokecolor{currentstroke}%
\pgfsetdash{}{0pt}%
\pgfpathmoveto{\pgfqpoint{2.027184in}{0.525823in}}%
\pgfpathlineto{\pgfqpoint{2.058359in}{0.571199in}}%
\pgfpathlineto{\pgfqpoint{2.283382in}{0.866486in}}%
\pgfpathlineto{\pgfqpoint{2.508405in}{1.219071in}}%
\pgfpathlineto{\pgfqpoint{2.733429in}{1.535650in}}%
\pgfpathlineto{\pgfqpoint{2.958452in}{1.852695in}}%
\pgfpathlineto{\pgfqpoint{3.183475in}{2.103243in}}%
\pgfpathlineto{\pgfqpoint{3.408498in}{2.465875in}}%
\pgfpathlineto{\pgfqpoint{3.633521in}{2.755438in}}%
\pgfpathlineto{\pgfqpoint{3.858545in}{3.018596in}}%
\pgfusepath{stroke}%
\end{pgfscope}%
\begin{pgfscope}%
\pgfpathrectangle{\pgfqpoint{0.708220in}{0.535823in}}{\pgfqpoint{5.141780in}{2.669453in}}%
\pgfusepath{clip}%
\pgfsetbuttcap%
\pgfsetroundjoin%
\definecolor{currentfill}{rgb}{0.615686,0.007843,0.843137}%
\pgfsetfillcolor{currentfill}%
\pgfsetlinewidth{0.501875pt}%
\definecolor{currentstroke}{rgb}{0.000000,0.000000,0.000000}%
\pgfsetstrokecolor{currentstroke}%
\pgfsetdash{}{0pt}%
\pgfsys@defobject{currentmarker}{\pgfqpoint{-0.034722in}{-0.034722in}}{\pgfqpoint{0.034722in}{0.034722in}}{%
\pgfpathmoveto{\pgfqpoint{0.000000in}{-0.034722in}}%
\pgfpathcurveto{\pgfqpoint{0.009208in}{-0.034722in}}{\pgfqpoint{0.018041in}{-0.031064in}}{\pgfqpoint{0.024552in}{-0.024552in}}%
\pgfpathcurveto{\pgfqpoint{0.031064in}{-0.018041in}}{\pgfqpoint{0.034722in}{-0.009208in}}{\pgfqpoint{0.034722in}{0.000000in}}%
\pgfpathcurveto{\pgfqpoint{0.034722in}{0.009208in}}{\pgfqpoint{0.031064in}{0.018041in}}{\pgfqpoint{0.024552in}{0.024552in}}%
\pgfpathcurveto{\pgfqpoint{0.018041in}{0.031064in}}{\pgfqpoint{0.009208in}{0.034722in}}{\pgfqpoint{0.000000in}{0.034722in}}%
\pgfpathcurveto{\pgfqpoint{-0.009208in}{0.034722in}}{\pgfqpoint{-0.018041in}{0.031064in}}{\pgfqpoint{-0.024552in}{0.024552in}}%
\pgfpathcurveto{\pgfqpoint{-0.031064in}{0.018041in}}{\pgfqpoint{-0.034722in}{0.009208in}}{\pgfqpoint{-0.034722in}{0.000000in}}%
\pgfpathcurveto{\pgfqpoint{-0.034722in}{-0.009208in}}{\pgfqpoint{-0.031064in}{-0.018041in}}{\pgfqpoint{-0.024552in}{-0.024552in}}%
\pgfpathcurveto{\pgfqpoint{-0.018041in}{-0.031064in}}{\pgfqpoint{-0.009208in}{-0.034722in}}{\pgfqpoint{0.000000in}{-0.034722in}}%
\pgfpathclose%
\pgfusepath{stroke,fill}%
}%
\begin{pgfscope}%
\pgfsys@transformshift{1.833336in}{0.243665in}%
\pgfsys@useobject{currentmarker}{}%
\end{pgfscope}%
\begin{pgfscope}%
\pgfsys@transformshift{2.058359in}{0.571199in}%
\pgfsys@useobject{currentmarker}{}%
\end{pgfscope}%
\begin{pgfscope}%
\pgfsys@transformshift{2.283382in}{0.866486in}%
\pgfsys@useobject{currentmarker}{}%
\end{pgfscope}%
\begin{pgfscope}%
\pgfsys@transformshift{2.508405in}{1.219071in}%
\pgfsys@useobject{currentmarker}{}%
\end{pgfscope}%
\begin{pgfscope}%
\pgfsys@transformshift{2.733429in}{1.535650in}%
\pgfsys@useobject{currentmarker}{}%
\end{pgfscope}%
\begin{pgfscope}%
\pgfsys@transformshift{2.958452in}{1.852695in}%
\pgfsys@useobject{currentmarker}{}%
\end{pgfscope}%
\begin{pgfscope}%
\pgfsys@transformshift{3.183475in}{2.103243in}%
\pgfsys@useobject{currentmarker}{}%
\end{pgfscope}%
\begin{pgfscope}%
\pgfsys@transformshift{3.408498in}{2.465875in}%
\pgfsys@useobject{currentmarker}{}%
\end{pgfscope}%
\begin{pgfscope}%
\pgfsys@transformshift{3.633521in}{2.755438in}%
\pgfsys@useobject{currentmarker}{}%
\end{pgfscope}%
\begin{pgfscope}%
\pgfsys@transformshift{3.858545in}{3.018596in}%
\pgfsys@useobject{currentmarker}{}%
\end{pgfscope}%
\end{pgfscope}%
\begin{pgfscope}%
\pgfpathrectangle{\pgfqpoint{0.708220in}{0.535823in}}{\pgfqpoint{5.141780in}{2.669453in}}%
\pgfusepath{clip}%
\pgfsetrectcap%
\pgfsetroundjoin%
\pgfsetlinewidth{1.003750pt}%
\definecolor{currentstroke}{rgb}{0.917647,0.372549,0.580392}%
\pgfsetstrokecolor{currentstroke}%
\pgfsetdash{}{0pt}%
\pgfpathmoveto{\pgfqpoint{2.685237in}{0.525823in}}%
\pgfpathlineto{\pgfqpoint{2.733429in}{0.563447in}}%
\pgfpathlineto{\pgfqpoint{2.958452in}{0.736719in}}%
\pgfpathlineto{\pgfqpoint{3.183475in}{0.978123in}}%
\pgfpathlineto{\pgfqpoint{3.408498in}{1.305836in}}%
\pgfpathlineto{\pgfqpoint{3.633521in}{1.559575in}}%
\pgfpathlineto{\pgfqpoint{3.858545in}{1.866610in}}%
\pgfpathlineto{\pgfqpoint{4.083568in}{2.126813in}}%
\pgfpathlineto{\pgfqpoint{4.308591in}{2.390895in}}%
\pgfpathlineto{\pgfqpoint{4.533614in}{2.643174in}}%
\pgfpathlineto{\pgfqpoint{4.758637in}{3.041851in}}%
\pgfusepath{stroke}%
\end{pgfscope}%
\begin{pgfscope}%
\pgfpathrectangle{\pgfqpoint{0.708220in}{0.535823in}}{\pgfqpoint{5.141780in}{2.669453in}}%
\pgfusepath{clip}%
\pgfsetbuttcap%
\pgfsetmiterjoin%
\definecolor{currentfill}{rgb}{0.917647,0.372549,0.580392}%
\pgfsetfillcolor{currentfill}%
\pgfsetlinewidth{0.501875pt}%
\definecolor{currentstroke}{rgb}{0.000000,0.000000,0.000000}%
\pgfsetstrokecolor{currentstroke}%
\pgfsetdash{}{0pt}%
\pgfsys@defobject{currentmarker}{\pgfqpoint{-0.049105in}{-0.049105in}}{\pgfqpoint{0.049105in}{0.049105in}}{%
\pgfpathmoveto{\pgfqpoint{0.000000in}{-0.049105in}}%
\pgfpathlineto{\pgfqpoint{0.049105in}{0.000000in}}%
\pgfpathlineto{\pgfqpoint{0.000000in}{0.049105in}}%
\pgfpathlineto{\pgfqpoint{-0.049105in}{0.000000in}}%
\pgfpathclose%
\pgfusepath{stroke,fill}%
}%
\begin{pgfscope}%
\pgfsys@transformshift{1.833336in}{0.270252in}%
\pgfsys@useobject{currentmarker}{}%
\end{pgfscope}%
\begin{pgfscope}%
\pgfsys@transformshift{2.058359in}{0.270252in}%
\pgfsys@useobject{currentmarker}{}%
\end{pgfscope}%
\begin{pgfscope}%
\pgfsys@transformshift{2.283382in}{0.270252in}%
\pgfsys@useobject{currentmarker}{}%
\end{pgfscope}%
\begin{pgfscope}%
\pgfsys@transformshift{2.508405in}{0.387769in}%
\pgfsys@useobject{currentmarker}{}%
\end{pgfscope}%
\begin{pgfscope}%
\pgfsys@transformshift{2.733429in}{0.563447in}%
\pgfsys@useobject{currentmarker}{}%
\end{pgfscope}%
\begin{pgfscope}%
\pgfsys@transformshift{2.958452in}{0.736719in}%
\pgfsys@useobject{currentmarker}{}%
\end{pgfscope}%
\begin{pgfscope}%
\pgfsys@transformshift{3.183475in}{0.978123in}%
\pgfsys@useobject{currentmarker}{}%
\end{pgfscope}%
\begin{pgfscope}%
\pgfsys@transformshift{3.408498in}{1.305836in}%
\pgfsys@useobject{currentmarker}{}%
\end{pgfscope}%
\begin{pgfscope}%
\pgfsys@transformshift{3.633521in}{1.559575in}%
\pgfsys@useobject{currentmarker}{}%
\end{pgfscope}%
\begin{pgfscope}%
\pgfsys@transformshift{3.858545in}{1.866610in}%
\pgfsys@useobject{currentmarker}{}%
\end{pgfscope}%
\begin{pgfscope}%
\pgfsys@transformshift{4.083568in}{2.126813in}%
\pgfsys@useobject{currentmarker}{}%
\end{pgfscope}%
\begin{pgfscope}%
\pgfsys@transformshift{4.308591in}{2.390895in}%
\pgfsys@useobject{currentmarker}{}%
\end{pgfscope}%
\begin{pgfscope}%
\pgfsys@transformshift{4.533614in}{2.643174in}%
\pgfsys@useobject{currentmarker}{}%
\end{pgfscope}%
\begin{pgfscope}%
\pgfsys@transformshift{4.758637in}{3.041851in}%
\pgfsys@useobject{currentmarker}{}%
\end{pgfscope}%
\end{pgfscope}%
\begin{pgfscope}%
\pgfpathrectangle{\pgfqpoint{0.708220in}{0.535823in}}{\pgfqpoint{5.141780in}{2.669453in}}%
\pgfusepath{clip}%
\pgfsetrectcap%
\pgfsetroundjoin%
\pgfsetlinewidth{1.003750pt}%
\definecolor{currentstroke}{rgb}{0.529412,0.462745,0.384314}%
\pgfsetstrokecolor{currentstroke}%
\pgfsetdash{}{0pt}%
\pgfpathmoveto{\pgfqpoint{1.920185in}{0.525823in}}%
\pgfpathlineto{\pgfqpoint{2.058359in}{0.573473in}}%
\pgfpathlineto{\pgfqpoint{2.283382in}{0.647919in}}%
\pgfpathlineto{\pgfqpoint{2.508405in}{0.725868in}}%
\pgfpathlineto{\pgfqpoint{2.733429in}{0.798513in}}%
\pgfpathlineto{\pgfqpoint{2.958452in}{0.899271in}}%
\pgfpathlineto{\pgfqpoint{3.183475in}{1.039475in}}%
\pgfpathlineto{\pgfqpoint{3.408498in}{1.399329in}}%
\pgfpathlineto{\pgfqpoint{3.633521in}{1.819793in}}%
\pgfpathlineto{\pgfqpoint{3.858545in}{2.283030in}}%
\pgfpathlineto{\pgfqpoint{4.083568in}{2.802781in}}%
\pgfusepath{stroke}%
\end{pgfscope}%
\begin{pgfscope}%
\pgfpathrectangle{\pgfqpoint{0.708220in}{0.535823in}}{\pgfqpoint{5.141780in}{2.669453in}}%
\pgfusepath{clip}%
\pgfsetbuttcap%
\pgfsetmiterjoin%
\definecolor{currentfill}{rgb}{0.529412,0.462745,0.384314}%
\pgfsetfillcolor{currentfill}%
\pgfsetlinewidth{0.501875pt}%
\definecolor{currentstroke}{rgb}{0.000000,0.000000,0.000000}%
\pgfsetstrokecolor{currentstroke}%
\pgfsetdash{}{0pt}%
\pgfsys@defobject{currentmarker}{\pgfqpoint{-0.034722in}{-0.034722in}}{\pgfqpoint{0.034722in}{0.034722in}}{%
\pgfpathmoveto{\pgfqpoint{-0.000000in}{-0.034722in}}%
\pgfpathlineto{\pgfqpoint{0.034722in}{0.034722in}}%
\pgfpathlineto{\pgfqpoint{-0.034722in}{0.034722in}}%
\pgfpathclose%
\pgfusepath{stroke,fill}%
}%
\begin{pgfscope}%
\pgfsys@transformshift{1.833336in}{0.495872in}%
\pgfsys@useobject{currentmarker}{}%
\end{pgfscope}%
\begin{pgfscope}%
\pgfsys@transformshift{2.058359in}{0.573473in}%
\pgfsys@useobject{currentmarker}{}%
\end{pgfscope}%
\begin{pgfscope}%
\pgfsys@transformshift{2.283382in}{0.647919in}%
\pgfsys@useobject{currentmarker}{}%
\end{pgfscope}%
\begin{pgfscope}%
\pgfsys@transformshift{2.508405in}{0.725868in}%
\pgfsys@useobject{currentmarker}{}%
\end{pgfscope}%
\begin{pgfscope}%
\pgfsys@transformshift{2.733429in}{0.798513in}%
\pgfsys@useobject{currentmarker}{}%
\end{pgfscope}%
\begin{pgfscope}%
\pgfsys@transformshift{2.958452in}{0.899271in}%
\pgfsys@useobject{currentmarker}{}%
\end{pgfscope}%
\begin{pgfscope}%
\pgfsys@transformshift{3.183475in}{1.039475in}%
\pgfsys@useobject{currentmarker}{}%
\end{pgfscope}%
\begin{pgfscope}%
\pgfsys@transformshift{3.408498in}{1.399329in}%
\pgfsys@useobject{currentmarker}{}%
\end{pgfscope}%
\begin{pgfscope}%
\pgfsys@transformshift{3.633521in}{1.819793in}%
\pgfsys@useobject{currentmarker}{}%
\end{pgfscope}%
\begin{pgfscope}%
\pgfsys@transformshift{3.858545in}{2.283030in}%
\pgfsys@useobject{currentmarker}{}%
\end{pgfscope}%
\begin{pgfscope}%
\pgfsys@transformshift{4.083568in}{2.802781in}%
\pgfsys@useobject{currentmarker}{}%
\end{pgfscope}%
\end{pgfscope}%
\begin{pgfscope}%
\pgfpathrectangle{\pgfqpoint{0.708220in}{0.535823in}}{\pgfqpoint{5.141780in}{2.669453in}}%
\pgfusepath{clip}%
\pgfsetrectcap%
\pgfsetroundjoin%
\pgfsetlinewidth{1.003750pt}%
\definecolor{currentstroke}{rgb}{0.611765,0.568627,0.274510}%
\pgfsetstrokecolor{currentstroke}%
\pgfsetdash{}{0pt}%
\pgfpathmoveto{\pgfqpoint{1.833336in}{0.533110in}}%
\pgfpathlineto{\pgfqpoint{2.058359in}{0.600165in}}%
\pgfpathlineto{\pgfqpoint{2.283382in}{0.656378in}}%
\pgfpathlineto{\pgfqpoint{2.508405in}{0.708555in}}%
\pgfpathlineto{\pgfqpoint{2.733429in}{0.755478in}}%
\pgfpathlineto{\pgfqpoint{2.958452in}{0.800865in}}%
\pgfpathlineto{\pgfqpoint{3.183475in}{0.845169in}}%
\pgfpathlineto{\pgfqpoint{3.408498in}{0.901495in}}%
\pgfpathlineto{\pgfqpoint{3.633521in}{0.983805in}}%
\pgfpathlineto{\pgfqpoint{3.858545in}{1.134783in}}%
\pgfpathlineto{\pgfqpoint{4.083568in}{1.319608in}}%
\pgfpathlineto{\pgfqpoint{4.308591in}{1.587480in}}%
\pgfpathlineto{\pgfqpoint{4.533614in}{1.922967in}}%
\pgfpathlineto{\pgfqpoint{4.758637in}{2.377229in}}%
\pgfpathlineto{\pgfqpoint{4.983661in}{2.606679in}}%
\pgfpathlineto{\pgfqpoint{5.208684in}{3.081033in}}%
\pgfusepath{stroke}%
\end{pgfscope}%
\begin{pgfscope}%
\pgfpathrectangle{\pgfqpoint{0.708220in}{0.535823in}}{\pgfqpoint{5.141780in}{2.669453in}}%
\pgfusepath{clip}%
\pgfsetbuttcap%
\pgfsetmiterjoin%
\definecolor{currentfill}{rgb}{0.611765,0.568627,0.274510}%
\pgfsetfillcolor{currentfill}%
\pgfsetlinewidth{0.501875pt}%
\definecolor{currentstroke}{rgb}{0.000000,0.000000,0.000000}%
\pgfsetstrokecolor{currentstroke}%
\pgfsetdash{}{0pt}%
\pgfsys@defobject{currentmarker}{\pgfqpoint{-0.034722in}{-0.034722in}}{\pgfqpoint{0.034722in}{0.034722in}}{%
\pgfpathmoveto{\pgfqpoint{-0.034722in}{0.000000in}}%
\pgfpathlineto{\pgfqpoint{0.034722in}{-0.034722in}}%
\pgfpathlineto{\pgfqpoint{0.034722in}{0.034722in}}%
\pgfpathclose%
\pgfusepath{stroke,fill}%
}%
\begin{pgfscope}%
\pgfsys@transformshift{1.833336in}{0.533110in}%
\pgfsys@useobject{currentmarker}{}%
\end{pgfscope}%
\begin{pgfscope}%
\pgfsys@transformshift{2.058359in}{0.600165in}%
\pgfsys@useobject{currentmarker}{}%
\end{pgfscope}%
\begin{pgfscope}%
\pgfsys@transformshift{2.283382in}{0.656378in}%
\pgfsys@useobject{currentmarker}{}%
\end{pgfscope}%
\begin{pgfscope}%
\pgfsys@transformshift{2.508405in}{0.708555in}%
\pgfsys@useobject{currentmarker}{}%
\end{pgfscope}%
\begin{pgfscope}%
\pgfsys@transformshift{2.733429in}{0.755478in}%
\pgfsys@useobject{currentmarker}{}%
\end{pgfscope}%
\begin{pgfscope}%
\pgfsys@transformshift{2.958452in}{0.800865in}%
\pgfsys@useobject{currentmarker}{}%
\end{pgfscope}%
\begin{pgfscope}%
\pgfsys@transformshift{3.183475in}{0.845169in}%
\pgfsys@useobject{currentmarker}{}%
\end{pgfscope}%
\begin{pgfscope}%
\pgfsys@transformshift{3.408498in}{0.901495in}%
\pgfsys@useobject{currentmarker}{}%
\end{pgfscope}%
\begin{pgfscope}%
\pgfsys@transformshift{3.633521in}{0.983805in}%
\pgfsys@useobject{currentmarker}{}%
\end{pgfscope}%
\begin{pgfscope}%
\pgfsys@transformshift{3.858545in}{1.134783in}%
\pgfsys@useobject{currentmarker}{}%
\end{pgfscope}%
\begin{pgfscope}%
\pgfsys@transformshift{4.083568in}{1.319608in}%
\pgfsys@useobject{currentmarker}{}%
\end{pgfscope}%
\begin{pgfscope}%
\pgfsys@transformshift{4.308591in}{1.587480in}%
\pgfsys@useobject{currentmarker}{}%
\end{pgfscope}%
\begin{pgfscope}%
\pgfsys@transformshift{4.533614in}{1.922967in}%
\pgfsys@useobject{currentmarker}{}%
\end{pgfscope}%
\begin{pgfscope}%
\pgfsys@transformshift{4.758637in}{2.377229in}%
\pgfsys@useobject{currentmarker}{}%
\end{pgfscope}%
\begin{pgfscope}%
\pgfsys@transformshift{4.983661in}{2.606679in}%
\pgfsys@useobject{currentmarker}{}%
\end{pgfscope}%
\begin{pgfscope}%
\pgfsys@transformshift{5.208684in}{3.081033in}%
\pgfsys@useobject{currentmarker}{}%
\end{pgfscope}%
\end{pgfscope}%
\begin{pgfscope}%
\pgfpathrectangle{\pgfqpoint{0.708220in}{0.535823in}}{\pgfqpoint{5.141780in}{2.669453in}}%
\pgfusepath{clip}%
\pgfsetrectcap%
\pgfsetroundjoin%
\pgfsetlinewidth{1.003750pt}%
\definecolor{currentstroke}{rgb}{0.780392,0.643137,0.254902}%
\pgfsetstrokecolor{currentstroke}%
\pgfsetdash{}{0pt}%
\pgfpathmoveto{\pgfqpoint{1.833336in}{0.607768in}}%
\pgfpathlineto{\pgfqpoint{2.058359in}{0.714594in}}%
\pgfpathlineto{\pgfqpoint{2.283382in}{0.810268in}}%
\pgfpathlineto{\pgfqpoint{2.508405in}{0.899028in}}%
\pgfpathlineto{\pgfqpoint{2.733429in}{0.976962in}}%
\pgfpathlineto{\pgfqpoint{2.958452in}{1.048705in}}%
\pgfpathlineto{\pgfqpoint{3.183475in}{1.117476in}}%
\pgfpathlineto{\pgfqpoint{3.408498in}{1.184622in}}%
\pgfpathlineto{\pgfqpoint{3.633521in}{1.247489in}}%
\pgfpathlineto{\pgfqpoint{3.858545in}{1.333602in}}%
\pgfpathlineto{\pgfqpoint{4.083568in}{1.407839in}}%
\pgfpathlineto{\pgfqpoint{4.308591in}{1.554250in}}%
\pgfpathlineto{\pgfqpoint{4.533614in}{1.754038in}}%
\pgfpathlineto{\pgfqpoint{4.758637in}{2.105422in}}%
\pgfpathlineto{\pgfqpoint{4.983661in}{2.186292in}}%
\pgfpathlineto{\pgfqpoint{5.208684in}{2.855359in}}%
\pgfusepath{stroke}%
\end{pgfscope}%
\begin{pgfscope}%
\pgfpathrectangle{\pgfqpoint{0.708220in}{0.535823in}}{\pgfqpoint{5.141780in}{2.669453in}}%
\pgfusepath{clip}%
\pgfsetbuttcap%
\pgfsetmiterjoin%
\definecolor{currentfill}{rgb}{0.780392,0.643137,0.254902}%
\pgfsetfillcolor{currentfill}%
\pgfsetlinewidth{0.501875pt}%
\definecolor{currentstroke}{rgb}{0.000000,0.000000,0.000000}%
\pgfsetstrokecolor{currentstroke}%
\pgfsetdash{}{0pt}%
\pgfsys@defobject{currentmarker}{\pgfqpoint{-0.034722in}{-0.034722in}}{\pgfqpoint{0.034722in}{0.034722in}}{%
\pgfpathmoveto{\pgfqpoint{0.034722in}{-0.000000in}}%
\pgfpathlineto{\pgfqpoint{-0.034722in}{0.034722in}}%
\pgfpathlineto{\pgfqpoint{-0.034722in}{-0.034722in}}%
\pgfpathclose%
\pgfusepath{stroke,fill}%
}%
\begin{pgfscope}%
\pgfsys@transformshift{1.833336in}{0.607768in}%
\pgfsys@useobject{currentmarker}{}%
\end{pgfscope}%
\begin{pgfscope}%
\pgfsys@transformshift{2.058359in}{0.714594in}%
\pgfsys@useobject{currentmarker}{}%
\end{pgfscope}%
\begin{pgfscope}%
\pgfsys@transformshift{2.283382in}{0.810268in}%
\pgfsys@useobject{currentmarker}{}%
\end{pgfscope}%
\begin{pgfscope}%
\pgfsys@transformshift{2.508405in}{0.899028in}%
\pgfsys@useobject{currentmarker}{}%
\end{pgfscope}%
\begin{pgfscope}%
\pgfsys@transformshift{2.733429in}{0.976962in}%
\pgfsys@useobject{currentmarker}{}%
\end{pgfscope}%
\begin{pgfscope}%
\pgfsys@transformshift{2.958452in}{1.048705in}%
\pgfsys@useobject{currentmarker}{}%
\end{pgfscope}%
\begin{pgfscope}%
\pgfsys@transformshift{3.183475in}{1.117476in}%
\pgfsys@useobject{currentmarker}{}%
\end{pgfscope}%
\begin{pgfscope}%
\pgfsys@transformshift{3.408498in}{1.184622in}%
\pgfsys@useobject{currentmarker}{}%
\end{pgfscope}%
\begin{pgfscope}%
\pgfsys@transformshift{3.633521in}{1.247489in}%
\pgfsys@useobject{currentmarker}{}%
\end{pgfscope}%
\begin{pgfscope}%
\pgfsys@transformshift{3.858545in}{1.333602in}%
\pgfsys@useobject{currentmarker}{}%
\end{pgfscope}%
\begin{pgfscope}%
\pgfsys@transformshift{4.083568in}{1.407839in}%
\pgfsys@useobject{currentmarker}{}%
\end{pgfscope}%
\begin{pgfscope}%
\pgfsys@transformshift{4.308591in}{1.554250in}%
\pgfsys@useobject{currentmarker}{}%
\end{pgfscope}%
\begin{pgfscope}%
\pgfsys@transformshift{4.533614in}{1.754038in}%
\pgfsys@useobject{currentmarker}{}%
\end{pgfscope}%
\begin{pgfscope}%
\pgfsys@transformshift{4.758637in}{2.105422in}%
\pgfsys@useobject{currentmarker}{}%
\end{pgfscope}%
\begin{pgfscope}%
\pgfsys@transformshift{4.983661in}{2.186292in}%
\pgfsys@useobject{currentmarker}{}%
\end{pgfscope}%
\begin{pgfscope}%
\pgfsys@transformshift{5.208684in}{2.855359in}%
\pgfsys@useobject{currentmarker}{}%
\end{pgfscope}%
\end{pgfscope}%
\begin{pgfscope}%
\pgfpathrectangle{\pgfqpoint{0.708220in}{0.535823in}}{\pgfqpoint{5.141780in}{2.669453in}}%
\pgfusepath{clip}%
\pgfsetrectcap%
\pgfsetroundjoin%
\pgfsetlinewidth{1.003750pt}%
\definecolor{currentstroke}{rgb}{1.000000,0.694118,0.305882}%
\pgfsetstrokecolor{currentstroke}%
\pgfsetdash{}{0pt}%
\pgfpathmoveto{\pgfqpoint{2.423238in}{0.525823in}}%
\pgfpathlineto{\pgfqpoint{2.508405in}{0.555088in}}%
\pgfpathlineto{\pgfqpoint{2.733429in}{0.713199in}}%
\pgfpathlineto{\pgfqpoint{2.958452in}{0.872472in}}%
\pgfpathlineto{\pgfqpoint{3.183475in}{1.125431in}}%
\pgfpathlineto{\pgfqpoint{3.408498in}{1.272615in}}%
\pgfpathlineto{\pgfqpoint{3.633521in}{1.392305in}}%
\pgfpathlineto{\pgfqpoint{3.858545in}{1.478523in}}%
\pgfpathlineto{\pgfqpoint{4.083568in}{1.557487in}}%
\pgfpathlineto{\pgfqpoint{4.308591in}{1.662572in}}%
\pgfpathlineto{\pgfqpoint{4.533614in}{1.734776in}}%
\pgfpathlineto{\pgfqpoint{4.758637in}{1.924610in}}%
\pgfpathlineto{\pgfqpoint{4.983661in}{2.079869in}}%
\pgfpathlineto{\pgfqpoint{5.208684in}{2.525204in}}%
\pgfpathlineto{\pgfqpoint{5.433707in}{2.672837in}}%
\pgfpathlineto{\pgfqpoint{5.658730in}{3.144706in}}%
\pgfusepath{stroke}%
\end{pgfscope}%
\begin{pgfscope}%
\pgfpathrectangle{\pgfqpoint{0.708220in}{0.535823in}}{\pgfqpoint{5.141780in}{2.669453in}}%
\pgfusepath{clip}%
\pgfsetbuttcap%
\pgfsetbeveljoin%
\definecolor{currentfill}{rgb}{1.000000,0.694118,0.305882}%
\pgfsetfillcolor{currentfill}%
\pgfsetlinewidth{0.501875pt}%
\definecolor{currentstroke}{rgb}{0.000000,0.000000,0.000000}%
\pgfsetstrokecolor{currentstroke}%
\pgfsetdash{}{0pt}%
\pgfsys@defobject{currentmarker}{\pgfqpoint{-0.033023in}{-0.028091in}}{\pgfqpoint{0.033023in}{0.034722in}}{%
\pgfpathmoveto{\pgfqpoint{0.000000in}{0.034722in}}%
\pgfpathlineto{\pgfqpoint{-0.007796in}{0.010730in}}%
\pgfpathlineto{\pgfqpoint{-0.033023in}{0.010730in}}%
\pgfpathlineto{\pgfqpoint{-0.012614in}{-0.004098in}}%
\pgfpathlineto{\pgfqpoint{-0.020409in}{-0.028091in}}%
\pgfpathlineto{\pgfqpoint{-0.000000in}{-0.013263in}}%
\pgfpathlineto{\pgfqpoint{0.020409in}{-0.028091in}}%
\pgfpathlineto{\pgfqpoint{0.012614in}{-0.004098in}}%
\pgfpathlineto{\pgfqpoint{0.033023in}{0.010730in}}%
\pgfpathlineto{\pgfqpoint{0.007796in}{0.010730in}}%
\pgfpathclose%
\pgfusepath{stroke,fill}%
}%
\begin{pgfscope}%
\pgfsys@transformshift{1.833336in}{0.359278in}%
\pgfsys@useobject{currentmarker}{}%
\end{pgfscope}%
\begin{pgfscope}%
\pgfsys@transformshift{2.058359in}{0.431153in}%
\pgfsys@useobject{currentmarker}{}%
\end{pgfscope}%
\begin{pgfscope}%
\pgfsys@transformshift{2.283382in}{0.477766in}%
\pgfsys@useobject{currentmarker}{}%
\end{pgfscope}%
\begin{pgfscope}%
\pgfsys@transformshift{2.508405in}{0.555088in}%
\pgfsys@useobject{currentmarker}{}%
\end{pgfscope}%
\begin{pgfscope}%
\pgfsys@transformshift{2.733429in}{0.713199in}%
\pgfsys@useobject{currentmarker}{}%
\end{pgfscope}%
\begin{pgfscope}%
\pgfsys@transformshift{2.958452in}{0.872472in}%
\pgfsys@useobject{currentmarker}{}%
\end{pgfscope}%
\begin{pgfscope}%
\pgfsys@transformshift{3.183475in}{1.125431in}%
\pgfsys@useobject{currentmarker}{}%
\end{pgfscope}%
\begin{pgfscope}%
\pgfsys@transformshift{3.408498in}{1.272615in}%
\pgfsys@useobject{currentmarker}{}%
\end{pgfscope}%
\begin{pgfscope}%
\pgfsys@transformshift{3.633521in}{1.392305in}%
\pgfsys@useobject{currentmarker}{}%
\end{pgfscope}%
\begin{pgfscope}%
\pgfsys@transformshift{3.858545in}{1.478523in}%
\pgfsys@useobject{currentmarker}{}%
\end{pgfscope}%
\begin{pgfscope}%
\pgfsys@transformshift{4.083568in}{1.557487in}%
\pgfsys@useobject{currentmarker}{}%
\end{pgfscope}%
\begin{pgfscope}%
\pgfsys@transformshift{4.308591in}{1.662572in}%
\pgfsys@useobject{currentmarker}{}%
\end{pgfscope}%
\begin{pgfscope}%
\pgfsys@transformshift{4.533614in}{1.734776in}%
\pgfsys@useobject{currentmarker}{}%
\end{pgfscope}%
\begin{pgfscope}%
\pgfsys@transformshift{4.758637in}{1.924610in}%
\pgfsys@useobject{currentmarker}{}%
\end{pgfscope}%
\begin{pgfscope}%
\pgfsys@transformshift{4.983661in}{2.079869in}%
\pgfsys@useobject{currentmarker}{}%
\end{pgfscope}%
\begin{pgfscope}%
\pgfsys@transformshift{5.208684in}{2.525204in}%
\pgfsys@useobject{currentmarker}{}%
\end{pgfscope}%
\begin{pgfscope}%
\pgfsys@transformshift{5.433707in}{2.672837in}%
\pgfsys@useobject{currentmarker}{}%
\end{pgfscope}%
\begin{pgfscope}%
\pgfsys@transformshift{5.658730in}{3.144706in}%
\pgfsys@useobject{currentmarker}{}%
\end{pgfscope}%
\end{pgfscope}%
\begin{pgfscope}%
\pgfsetrectcap%
\pgfsetmiterjoin%
\pgfsetlinewidth{0.803000pt}%
\definecolor{currentstroke}{rgb}{0.000000,0.000000,0.000000}%
\pgfsetstrokecolor{currentstroke}%
\pgfsetdash{}{0pt}%
\pgfpathmoveto{\pgfqpoint{0.708220in}{0.535823in}}%
\pgfpathlineto{\pgfqpoint{0.708220in}{3.205275in}}%
\pgfusepath{stroke}%
\end{pgfscope}%
\begin{pgfscope}%
\pgfsetrectcap%
\pgfsetmiterjoin%
\pgfsetlinewidth{0.803000pt}%
\definecolor{currentstroke}{rgb}{0.000000,0.000000,0.000000}%
\pgfsetstrokecolor{currentstroke}%
\pgfsetdash{}{0pt}%
\pgfpathmoveto{\pgfqpoint{5.850000in}{0.535823in}}%
\pgfpathlineto{\pgfqpoint{5.850000in}{3.205275in}}%
\pgfusepath{stroke}%
\end{pgfscope}%
\begin{pgfscope}%
\pgfsetrectcap%
\pgfsetmiterjoin%
\pgfsetlinewidth{0.803000pt}%
\definecolor{currentstroke}{rgb}{0.000000,0.000000,0.000000}%
\pgfsetstrokecolor{currentstroke}%
\pgfsetdash{}{0pt}%
\pgfpathmoveto{\pgfqpoint{0.708220in}{0.535823in}}%
\pgfpathlineto{\pgfqpoint{5.850000in}{0.535823in}}%
\pgfusepath{stroke}%
\end{pgfscope}%
\begin{pgfscope}%
\pgfsetrectcap%
\pgfsetmiterjoin%
\pgfsetlinewidth{0.803000pt}%
\definecolor{currentstroke}{rgb}{0.000000,0.000000,0.000000}%
\pgfsetstrokecolor{currentstroke}%
\pgfsetdash{}{0pt}%
\pgfpathmoveto{\pgfqpoint{0.708220in}{3.205275in}}%
\pgfpathlineto{\pgfqpoint{5.850000in}{3.205275in}}%
\pgfusepath{stroke}%
\end{pgfscope}%
\begin{pgfscope}%
\pgfsetrectcap%
\pgfsetroundjoin%
\pgfsetlinewidth{1.003750pt}%
\definecolor{currentstroke}{rgb}{0.866667,0.058824,0.058824}%
\pgfsetstrokecolor{currentstroke}%
\pgfsetdash{}{0pt}%
\pgfpathmoveto{\pgfqpoint{0.758220in}{3.111525in}}%
\pgfpathlineto{\pgfqpoint{1.008220in}{3.111525in}}%
\pgfusepath{stroke}%
\end{pgfscope}%
\begin{pgfscope}%
\pgfsetbuttcap%
\pgfsetmiterjoin%
\definecolor{currentfill}{rgb}{0.866667,0.058824,0.058824}%
\pgfsetfillcolor{currentfill}%
\pgfsetlinewidth{0.501875pt}%
\definecolor{currentstroke}{rgb}{0.000000,0.000000,0.000000}%
\pgfsetstrokecolor{currentstroke}%
\pgfsetdash{}{0pt}%
\pgfsys@defobject{currentmarker}{\pgfqpoint{-0.033023in}{-0.028091in}}{\pgfqpoint{0.033023in}{0.034722in}}{%
\pgfpathmoveto{\pgfqpoint{0.000000in}{0.034722in}}%
\pgfpathlineto{\pgfqpoint{-0.033023in}{0.010730in}}%
\pgfpathlineto{\pgfqpoint{-0.020409in}{-0.028091in}}%
\pgfpathlineto{\pgfqpoint{0.020409in}{-0.028091in}}%
\pgfpathlineto{\pgfqpoint{0.033023in}{0.010730in}}%
\pgfpathclose%
\pgfusepath{stroke,fill}%
}%
\begin{pgfscope}%
\pgfsys@transformshift{0.883220in}{3.111525in}%
\pgfsys@useobject{currentmarker}{}%
\end{pgfscope}%
\end{pgfscope}%
\begin{pgfscope}%
\definecolor{textcolor}{rgb}{0.000000,0.000000,0.000000}%
\pgfsetstrokecolor{textcolor}%
\pgfsetfillcolor{textcolor}%
\pgftext[x=1.033220in,y=3.067775in,left,base]{\color{textcolor}\rmfamily\fontsize{9.000000}{10.800000}\selectfont cachet}%
\end{pgfscope}%
\begin{pgfscope}%
\pgfsetrectcap%
\pgfsetroundjoin%
\pgfsetlinewidth{1.003750pt}%
\definecolor{currentstroke}{rgb}{0.000000,0.000000,0.200000}%
\pgfsetstrokecolor{currentstroke}%
\pgfsetdash{}{0pt}%
\pgfpathmoveto{\pgfqpoint{0.758220in}{2.949726in}}%
\pgfpathlineto{\pgfqpoint{1.008220in}{2.949726in}}%
\pgfusepath{stroke}%
\end{pgfscope}%
\begin{pgfscope}%
\pgfsetbuttcap%
\pgfsetmiterjoin%
\definecolor{currentfill}{rgb}{0.000000,0.000000,0.200000}%
\pgfsetfillcolor{currentfill}%
\pgfsetlinewidth{0.501875pt}%
\definecolor{currentstroke}{rgb}{0.000000,0.000000,0.000000}%
\pgfsetstrokecolor{currentstroke}%
\pgfsetdash{}{0pt}%
\pgfsys@defobject{currentmarker}{\pgfqpoint{-0.034722in}{-0.034722in}}{\pgfqpoint{0.034722in}{0.034722in}}{%
\pgfpathmoveto{\pgfqpoint{-0.011574in}{-0.034722in}}%
\pgfpathlineto{\pgfqpoint{0.011574in}{-0.034722in}}%
\pgfpathlineto{\pgfqpoint{0.011574in}{-0.011574in}}%
\pgfpathlineto{\pgfqpoint{0.034722in}{-0.011574in}}%
\pgfpathlineto{\pgfqpoint{0.034722in}{0.011574in}}%
\pgfpathlineto{\pgfqpoint{0.011574in}{0.011574in}}%
\pgfpathlineto{\pgfqpoint{0.011574in}{0.034722in}}%
\pgfpathlineto{\pgfqpoint{-0.011574in}{0.034722in}}%
\pgfpathlineto{\pgfqpoint{-0.011574in}{0.011574in}}%
\pgfpathlineto{\pgfqpoint{-0.034722in}{0.011574in}}%
\pgfpathlineto{\pgfqpoint{-0.034722in}{-0.011574in}}%
\pgfpathlineto{\pgfqpoint{-0.011574in}{-0.011574in}}%
\pgfpathclose%
\pgfusepath{stroke,fill}%
}%
\begin{pgfscope}%
\pgfsys@transformshift{0.883220in}{2.949726in}%
\pgfsys@useobject{currentmarker}{}%
\end{pgfscope}%
\end{pgfscope}%
\begin{pgfscope}%
\definecolor{textcolor}{rgb}{0.000000,0.000000,0.000000}%
\pgfsetstrokecolor{textcolor}%
\pgfsetfillcolor{textcolor}%
\pgftext[x=1.033220in,y=2.905976in,left,base]{\color{textcolor}\rmfamily\fontsize{9.000000}{10.800000}\selectfont dynQBF}%
\end{pgfscope}%
\begin{pgfscope}%
\pgfsetrectcap%
\pgfsetroundjoin%
\pgfsetlinewidth{1.003750pt}%
\definecolor{currentstroke}{rgb}{0.000000,0.000000,0.866667}%
\pgfsetstrokecolor{currentstroke}%
\pgfsetdash{}{0pt}%
\pgfpathmoveto{\pgfqpoint{0.758220in}{2.787926in}}%
\pgfpathlineto{\pgfqpoint{1.008220in}{2.787926in}}%
\pgfusepath{stroke}%
\end{pgfscope}%
\begin{pgfscope}%
\pgfsetbuttcap%
\pgfsetmiterjoin%
\definecolor{currentfill}{rgb}{0.000000,0.000000,0.866667}%
\pgfsetfillcolor{currentfill}%
\pgfsetlinewidth{0.501875pt}%
\definecolor{currentstroke}{rgb}{0.000000,0.000000,0.000000}%
\pgfsetstrokecolor{currentstroke}%
\pgfsetdash{}{0pt}%
\pgfsys@defobject{currentmarker}{\pgfqpoint{-0.029463in}{-0.049105in}}{\pgfqpoint{0.029463in}{0.049105in}}{%
\pgfpathmoveto{\pgfqpoint{0.000000in}{-0.049105in}}%
\pgfpathlineto{\pgfqpoint{0.029463in}{0.000000in}}%
\pgfpathlineto{\pgfqpoint{0.000000in}{0.049105in}}%
\pgfpathlineto{\pgfqpoint{-0.029463in}{0.000000in}}%
\pgfpathclose%
\pgfusepath{stroke,fill}%
}%
\begin{pgfscope}%
\pgfsys@transformshift{0.883220in}{2.787926in}%
\pgfsys@useobject{currentmarker}{}%
\end{pgfscope}%
\end{pgfscope}%
\begin{pgfscope}%
\definecolor{textcolor}{rgb}{0.000000,0.000000,0.000000}%
\pgfsetstrokecolor{textcolor}%
\pgfsetfillcolor{textcolor}%
\pgftext[x=1.033220in,y=2.744176in,left,base]{\color{textcolor}\rmfamily\fontsize{9.000000}{10.800000}\selectfont dynasp}%
\end{pgfscope}%
\begin{pgfscope}%
\pgfsetrectcap%
\pgfsetroundjoin%
\pgfsetlinewidth{1.003750pt}%
\definecolor{currentstroke}{rgb}{0.250980,0.231373,0.796078}%
\pgfsetstrokecolor{currentstroke}%
\pgfsetdash{}{0pt}%
\pgfpathmoveto{\pgfqpoint{0.758220in}{2.626126in}}%
\pgfpathlineto{\pgfqpoint{1.008220in}{2.626126in}}%
\pgfusepath{stroke}%
\end{pgfscope}%
\begin{pgfscope}%
\pgfsetbuttcap%
\pgfsetmiterjoin%
\definecolor{currentfill}{rgb}{0.250980,0.231373,0.796078}%
\pgfsetfillcolor{currentfill}%
\pgfsetlinewidth{0.501875pt}%
\definecolor{currentstroke}{rgb}{0.000000,0.000000,0.000000}%
\pgfsetstrokecolor{currentstroke}%
\pgfsetdash{}{0pt}%
\pgfsys@defobject{currentmarker}{\pgfqpoint{-0.034722in}{-0.034722in}}{\pgfqpoint{0.034722in}{0.034722in}}{%
\pgfpathmoveto{\pgfqpoint{-0.034722in}{-0.034722in}}%
\pgfpathlineto{\pgfqpoint{0.034722in}{-0.034722in}}%
\pgfpathlineto{\pgfqpoint{0.034722in}{0.034722in}}%
\pgfpathlineto{\pgfqpoint{-0.034722in}{0.034722in}}%
\pgfpathclose%
\pgfusepath{stroke,fill}%
}%
\begin{pgfscope}%
\pgfsys@transformshift{0.883220in}{2.626126in}%
\pgfsys@useobject{currentmarker}{}%
\end{pgfscope}%
\end{pgfscope}%
\begin{pgfscope}%
\definecolor{textcolor}{rgb}{0.000000,0.000000,0.000000}%
\pgfsetstrokecolor{textcolor}%
\pgfsetfillcolor{textcolor}%
\pgftext[x=1.033220in,y=2.582376in,left,base]{\color{textcolor}\rmfamily\fontsize{9.000000}{10.800000}\selectfont sharpSAT}%
\end{pgfscope}%
\begin{pgfscope}%
\pgfsetrectcap%
\pgfsetroundjoin%
\pgfsetlinewidth{1.003750pt}%
\definecolor{currentstroke}{rgb}{0.615686,0.007843,0.843137}%
\pgfsetstrokecolor{currentstroke}%
\pgfsetdash{}{0pt}%
\pgfpathmoveto{\pgfqpoint{0.758220in}{2.464327in}}%
\pgfpathlineto{\pgfqpoint{1.008220in}{2.464327in}}%
\pgfusepath{stroke}%
\end{pgfscope}%
\begin{pgfscope}%
\pgfsetbuttcap%
\pgfsetroundjoin%
\definecolor{currentfill}{rgb}{0.615686,0.007843,0.843137}%
\pgfsetfillcolor{currentfill}%
\pgfsetlinewidth{0.501875pt}%
\definecolor{currentstroke}{rgb}{0.000000,0.000000,0.000000}%
\pgfsetstrokecolor{currentstroke}%
\pgfsetdash{}{0pt}%
\pgfsys@defobject{currentmarker}{\pgfqpoint{-0.034722in}{-0.034722in}}{\pgfqpoint{0.034722in}{0.034722in}}{%
\pgfpathmoveto{\pgfqpoint{0.000000in}{-0.034722in}}%
\pgfpathcurveto{\pgfqpoint{0.009208in}{-0.034722in}}{\pgfqpoint{0.018041in}{-0.031064in}}{\pgfqpoint{0.024552in}{-0.024552in}}%
\pgfpathcurveto{\pgfqpoint{0.031064in}{-0.018041in}}{\pgfqpoint{0.034722in}{-0.009208in}}{\pgfqpoint{0.034722in}{0.000000in}}%
\pgfpathcurveto{\pgfqpoint{0.034722in}{0.009208in}}{\pgfqpoint{0.031064in}{0.018041in}}{\pgfqpoint{0.024552in}{0.024552in}}%
\pgfpathcurveto{\pgfqpoint{0.018041in}{0.031064in}}{\pgfqpoint{0.009208in}{0.034722in}}{\pgfqpoint{0.000000in}{0.034722in}}%
\pgfpathcurveto{\pgfqpoint{-0.009208in}{0.034722in}}{\pgfqpoint{-0.018041in}{0.031064in}}{\pgfqpoint{-0.024552in}{0.024552in}}%
\pgfpathcurveto{\pgfqpoint{-0.031064in}{0.018041in}}{\pgfqpoint{-0.034722in}{0.009208in}}{\pgfqpoint{-0.034722in}{0.000000in}}%
\pgfpathcurveto{\pgfqpoint{-0.034722in}{-0.009208in}}{\pgfqpoint{-0.031064in}{-0.018041in}}{\pgfqpoint{-0.024552in}{-0.024552in}}%
\pgfpathcurveto{\pgfqpoint{-0.018041in}{-0.031064in}}{\pgfqpoint{-0.009208in}{-0.034722in}}{\pgfqpoint{0.000000in}{-0.034722in}}%
\pgfpathclose%
\pgfusepath{stroke,fill}%
}%
\begin{pgfscope}%
\pgfsys@transformshift{0.883220in}{2.464327in}%
\pgfsys@useobject{currentmarker}{}%
\end{pgfscope}%
\end{pgfscope}%
\begin{pgfscope}%
\definecolor{textcolor}{rgb}{0.000000,0.000000,0.000000}%
\pgfsetstrokecolor{textcolor}%
\pgfsetfillcolor{textcolor}%
\pgftext[x=1.033220in,y=2.420577in,left,base]{\color{textcolor}\rmfamily\fontsize{9.000000}{10.800000}\selectfont d4}%
\end{pgfscope}%
\begin{pgfscope}%
\pgfsetrectcap%
\pgfsetroundjoin%
\pgfsetlinewidth{1.003750pt}%
\definecolor{currentstroke}{rgb}{0.917647,0.372549,0.580392}%
\pgfsetstrokecolor{currentstroke}%
\pgfsetdash{}{0pt}%
\pgfpathmoveto{\pgfqpoint{0.758220in}{2.302527in}}%
\pgfpathlineto{\pgfqpoint{1.008220in}{2.302527in}}%
\pgfusepath{stroke}%
\end{pgfscope}%
\begin{pgfscope}%
\pgfsetbuttcap%
\pgfsetmiterjoin%
\definecolor{currentfill}{rgb}{0.917647,0.372549,0.580392}%
\pgfsetfillcolor{currentfill}%
\pgfsetlinewidth{0.501875pt}%
\definecolor{currentstroke}{rgb}{0.000000,0.000000,0.000000}%
\pgfsetstrokecolor{currentstroke}%
\pgfsetdash{}{0pt}%
\pgfsys@defobject{currentmarker}{\pgfqpoint{-0.049105in}{-0.049105in}}{\pgfqpoint{0.049105in}{0.049105in}}{%
\pgfpathmoveto{\pgfqpoint{0.000000in}{-0.049105in}}%
\pgfpathlineto{\pgfqpoint{0.049105in}{0.000000in}}%
\pgfpathlineto{\pgfqpoint{0.000000in}{0.049105in}}%
\pgfpathlineto{\pgfqpoint{-0.049105in}{0.000000in}}%
\pgfpathclose%
\pgfusepath{stroke,fill}%
}%
\begin{pgfscope}%
\pgfsys@transformshift{0.883220in}{2.302527in}%
\pgfsys@useobject{currentmarker}{}%
\end{pgfscope}%
\end{pgfscope}%
\begin{pgfscope}%
\definecolor{textcolor}{rgb}{0.000000,0.000000,0.000000}%
\pgfsetstrokecolor{textcolor}%
\pgfsetfillcolor{textcolor}%
\pgftext[x=1.033220in,y=2.258777in,left,base]{\color{textcolor}\rmfamily\fontsize{9.000000}{10.800000}\selectfont miniC2D}%
\end{pgfscope}%
\begin{pgfscope}%
\pgfsetrectcap%
\pgfsetroundjoin%
\pgfsetlinewidth{1.003750pt}%
\definecolor{currentstroke}{rgb}{0.529412,0.462745,0.384314}%
\pgfsetstrokecolor{currentstroke}%
\pgfsetdash{}{0pt}%
\pgfpathmoveto{\pgfqpoint{0.758220in}{2.140728in}}%
\pgfpathlineto{\pgfqpoint{1.008220in}{2.140728in}}%
\pgfusepath{stroke}%
\end{pgfscope}%
\begin{pgfscope}%
\pgfsetbuttcap%
\pgfsetmiterjoin%
\definecolor{currentfill}{rgb}{0.529412,0.462745,0.384314}%
\pgfsetfillcolor{currentfill}%
\pgfsetlinewidth{0.501875pt}%
\definecolor{currentstroke}{rgb}{0.000000,0.000000,0.000000}%
\pgfsetstrokecolor{currentstroke}%
\pgfsetdash{}{0pt}%
\pgfsys@defobject{currentmarker}{\pgfqpoint{-0.034722in}{-0.034722in}}{\pgfqpoint{0.034722in}{0.034722in}}{%
\pgfpathmoveto{\pgfqpoint{-0.000000in}{-0.034722in}}%
\pgfpathlineto{\pgfqpoint{0.034722in}{0.034722in}}%
\pgfpathlineto{\pgfqpoint{-0.034722in}{0.034722in}}%
\pgfpathclose%
\pgfusepath{stroke,fill}%
}%
\begin{pgfscope}%
\pgfsys@transformshift{0.883220in}{2.140728in}%
\pgfsys@useobject{currentmarker}{}%
\end{pgfscope}%
\end{pgfscope}%
\begin{pgfscope}%
\definecolor{textcolor}{rgb}{0.000000,0.000000,0.000000}%
\pgfsetstrokecolor{textcolor}%
\pgfsetfillcolor{textcolor}%
\pgftext[x=1.033220in,y=2.096978in,left,base]{\color{textcolor}\rmfamily\fontsize{9.000000}{10.800000}\selectfont greedy}%
\end{pgfscope}%
\begin{pgfscope}%
\pgfsetrectcap%
\pgfsetroundjoin%
\pgfsetlinewidth{1.003750pt}%
\definecolor{currentstroke}{rgb}{0.611765,0.568627,0.274510}%
\pgfsetstrokecolor{currentstroke}%
\pgfsetdash{}{0pt}%
\pgfpathmoveto{\pgfqpoint{0.758220in}{1.978928in}}%
\pgfpathlineto{\pgfqpoint{1.008220in}{1.978928in}}%
\pgfusepath{stroke}%
\end{pgfscope}%
\begin{pgfscope}%
\pgfsetbuttcap%
\pgfsetmiterjoin%
\definecolor{currentfill}{rgb}{0.611765,0.568627,0.274510}%
\pgfsetfillcolor{currentfill}%
\pgfsetlinewidth{0.501875pt}%
\definecolor{currentstroke}{rgb}{0.000000,0.000000,0.000000}%
\pgfsetstrokecolor{currentstroke}%
\pgfsetdash{}{0pt}%
\pgfsys@defobject{currentmarker}{\pgfqpoint{-0.034722in}{-0.034722in}}{\pgfqpoint{0.034722in}{0.034722in}}{%
\pgfpathmoveto{\pgfqpoint{-0.034722in}{0.000000in}}%
\pgfpathlineto{\pgfqpoint{0.034722in}{-0.034722in}}%
\pgfpathlineto{\pgfqpoint{0.034722in}{0.034722in}}%
\pgfpathclose%
\pgfusepath{stroke,fill}%
}%
\begin{pgfscope}%
\pgfsys@transformshift{0.883220in}{1.978928in}%
\pgfsys@useobject{currentmarker}{}%
\end{pgfscope}%
\end{pgfscope}%
\begin{pgfscope}%
\definecolor{textcolor}{rgb}{0.000000,0.000000,0.000000}%
\pgfsetstrokecolor{textcolor}%
\pgfsetfillcolor{textcolor}%
\pgftext[x=1.033220in,y=1.935178in,left,base]{\color{textcolor}\rmfamily\fontsize{9.000000}{10.800000}\selectfont metis}%
\end{pgfscope}%
\begin{pgfscope}%
\pgfsetrectcap%
\pgfsetroundjoin%
\pgfsetlinewidth{1.003750pt}%
\definecolor{currentstroke}{rgb}{0.780392,0.643137,0.254902}%
\pgfsetstrokecolor{currentstroke}%
\pgfsetdash{}{0pt}%
\pgfpathmoveto{\pgfqpoint{0.758220in}{1.817129in}}%
\pgfpathlineto{\pgfqpoint{1.008220in}{1.817129in}}%
\pgfusepath{stroke}%
\end{pgfscope}%
\begin{pgfscope}%
\pgfsetbuttcap%
\pgfsetmiterjoin%
\definecolor{currentfill}{rgb}{0.780392,0.643137,0.254902}%
\pgfsetfillcolor{currentfill}%
\pgfsetlinewidth{0.501875pt}%
\definecolor{currentstroke}{rgb}{0.000000,0.000000,0.000000}%
\pgfsetstrokecolor{currentstroke}%
\pgfsetdash{}{0pt}%
\pgfsys@defobject{currentmarker}{\pgfqpoint{-0.034722in}{-0.034722in}}{\pgfqpoint{0.034722in}{0.034722in}}{%
\pgfpathmoveto{\pgfqpoint{0.034722in}{-0.000000in}}%
\pgfpathlineto{\pgfqpoint{-0.034722in}{0.034722in}}%
\pgfpathlineto{\pgfqpoint{-0.034722in}{-0.034722in}}%
\pgfpathclose%
\pgfusepath{stroke,fill}%
}%
\begin{pgfscope}%
\pgfsys@transformshift{0.883220in}{1.817129in}%
\pgfsys@useobject{currentmarker}{}%
\end{pgfscope}%
\end{pgfscope}%
\begin{pgfscope}%
\definecolor{textcolor}{rgb}{0.000000,0.000000,0.000000}%
\pgfsetstrokecolor{textcolor}%
\pgfsetfillcolor{textcolor}%
\pgftext[x=1.033220in,y=1.773379in,left,base]{\color{textcolor}\rmfamily\fontsize{9.000000}{10.800000}\selectfont GN}%
\end{pgfscope}%
\begin{pgfscope}%
\pgfsetrectcap%
\pgfsetroundjoin%
\pgfsetlinewidth{1.003750pt}%
\definecolor{currentstroke}{rgb}{1.000000,0.694118,0.305882}%
\pgfsetstrokecolor{currentstroke}%
\pgfsetdash{}{0pt}%
\pgfpathmoveto{\pgfqpoint{0.758220in}{1.655329in}}%
\pgfpathlineto{\pgfqpoint{1.008220in}{1.655329in}}%
\pgfusepath{stroke}%
\end{pgfscope}%
\begin{pgfscope}%
\pgfsetbuttcap%
\pgfsetbeveljoin%
\definecolor{currentfill}{rgb}{1.000000,0.694118,0.305882}%
\pgfsetfillcolor{currentfill}%
\pgfsetlinewidth{0.501875pt}%
\definecolor{currentstroke}{rgb}{0.000000,0.000000,0.000000}%
\pgfsetstrokecolor{currentstroke}%
\pgfsetdash{}{0pt}%
\pgfsys@defobject{currentmarker}{\pgfqpoint{-0.033023in}{-0.028091in}}{\pgfqpoint{0.033023in}{0.034722in}}{%
\pgfpathmoveto{\pgfqpoint{0.000000in}{0.034722in}}%
\pgfpathlineto{\pgfqpoint{-0.007796in}{0.010730in}}%
\pgfpathlineto{\pgfqpoint{-0.033023in}{0.010730in}}%
\pgfpathlineto{\pgfqpoint{-0.012614in}{-0.004098in}}%
\pgfpathlineto{\pgfqpoint{-0.020409in}{-0.028091in}}%
\pgfpathlineto{\pgfqpoint{-0.000000in}{-0.013263in}}%
\pgfpathlineto{\pgfqpoint{0.020409in}{-0.028091in}}%
\pgfpathlineto{\pgfqpoint{0.012614in}{-0.004098in}}%
\pgfpathlineto{\pgfqpoint{0.033023in}{0.010730in}}%
\pgfpathlineto{\pgfqpoint{0.007796in}{0.010730in}}%
\pgfpathclose%
\pgfusepath{stroke,fill}%
}%
\begin{pgfscope}%
\pgfsys@transformshift{0.883220in}{1.655329in}%
\pgfsys@useobject{currentmarker}{}%
\end{pgfscope}%
\end{pgfscope}%
\begin{pgfscope}%
\definecolor{textcolor}{rgb}{0.000000,0.000000,0.000000}%
\pgfsetstrokecolor{textcolor}%
\pgfsetfillcolor{textcolor}%
\pgftext[x=1.033220in,y=1.611579in,left,base]{\color{textcolor}\rmfamily\fontsize{9.000000}{10.800000}\selectfont LG+Flow}%
\end{pgfscope}%
\end{pgfpicture}%
\makeatother%
\endgroup%

%	\caption{\label{fig:cubic-time} Median runtime of various methods on counting the number of vertex covers of 100 randomly-sampled cubic graphs with $n$ vertices. Datapoints that ran out of time ($1000$ seconds) or memory (48 GB) are not shown. Our contribution \textbf{Line+Flow} is faster than the other methods on formulas counting vertex covers of large graphs.}
%\end{figure}

%\begin{figure}
%	\centering
%	%% Creator: Matplotlib, PGF backend
%%
%% To include the figure in your LaTeX document, write
%%   \input{<filename>.pgf}
%%
%% Make sure the required packages are loaded in your preamble
%%   \usepackage{pgf}
%%
%% and, on pdftex
%%   \usepackage[utf8]{inputenc}\DeclareUnicodeCharacter{2212}{-}
%%
%% or, on luatex and xetex
%%   \usepackage{unicode-math}
%%
%% Figures using additional raster images can only be included by \input if
%% they are in the same directory as the main LaTeX file. For loading figures
%% from other directories you can use the `import` package
%%   \usepackage{import}
%%
%% and then include the figures with
%%   \import{<path to file>}{<filename>.pgf}
%%
%% Matplotlib used the following preamble
%%   \usepackage[utf8x]{inputenc}
%%   \usepackage[T1]{fontenc}
%%
\begingroup%
\makeatletter%
\begin{pgfpicture}%
\pgfpathrectangle{\pgfpointorigin}{\pgfqpoint{6.000000in}{3.400000in}}%
\pgfusepath{use as bounding box, clip}%
\begin{pgfscope}%
\pgfsetbuttcap%
\pgfsetmiterjoin%
\definecolor{currentfill}{rgb}{1.000000,1.000000,1.000000}%
\pgfsetfillcolor{currentfill}%
\pgfsetlinewidth{0.000000pt}%
\definecolor{currentstroke}{rgb}{1.000000,1.000000,1.000000}%
\pgfsetstrokecolor{currentstroke}%
\pgfsetdash{}{0pt}%
\pgfpathmoveto{\pgfqpoint{0.000000in}{0.000000in}}%
\pgfpathlineto{\pgfqpoint{6.000000in}{0.000000in}}%
\pgfpathlineto{\pgfqpoint{6.000000in}{3.400000in}}%
\pgfpathlineto{\pgfqpoint{0.000000in}{3.400000in}}%
\pgfpathclose%
\pgfusepath{fill}%
\end{pgfscope}%
\begin{pgfscope}%
\pgfsetbuttcap%
\pgfsetmiterjoin%
\definecolor{currentfill}{rgb}{1.000000,1.000000,1.000000}%
\pgfsetfillcolor{currentfill}%
\pgfsetlinewidth{0.000000pt}%
\definecolor{currentstroke}{rgb}{0.000000,0.000000,0.000000}%
\pgfsetstrokecolor{currentstroke}%
\pgfsetstrokeopacity{0.000000}%
\pgfsetdash{}{0pt}%
\pgfpathmoveto{\pgfqpoint{0.553904in}{0.535823in}}%
\pgfpathlineto{\pgfqpoint{5.850000in}{0.535823in}}%
\pgfpathlineto{\pgfqpoint{5.850000in}{3.250000in}}%
\pgfpathlineto{\pgfqpoint{0.553904in}{3.250000in}}%
\pgfpathclose%
\pgfusepath{fill}%
\end{pgfscope}%
\begin{pgfscope}%
\pgfsetbuttcap%
\pgfsetroundjoin%
\definecolor{currentfill}{rgb}{0.000000,0.000000,0.000000}%
\pgfsetfillcolor{currentfill}%
\pgfsetlinewidth{0.803000pt}%
\definecolor{currentstroke}{rgb}{0.000000,0.000000,0.000000}%
\pgfsetstrokecolor{currentstroke}%
\pgfsetdash{}{0pt}%
\pgfsys@defobject{currentmarker}{\pgfqpoint{0.000000in}{-0.048611in}}{\pgfqpoint{0.000000in}{0.000000in}}{%
\pgfpathmoveto{\pgfqpoint{0.000000in}{0.000000in}}%
\pgfpathlineto{\pgfqpoint{0.000000in}{-0.048611in}}%
\pgfusepath{stroke,fill}%
}%
\begin{pgfscope}%
\pgfsys@transformshift{0.794636in}{0.535823in}%
\pgfsys@useobject{currentmarker}{}%
\end{pgfscope}%
\end{pgfscope}%
\begin{pgfscope}%
\definecolor{textcolor}{rgb}{0.000000,0.000000,0.000000}%
\pgfsetstrokecolor{textcolor}%
\pgfsetfillcolor{textcolor}%
\pgftext[x=0.794636in,y=0.438600in,,top]{\color{textcolor}\rmfamily\fontsize{9.000000}{10.800000}\selectfont \(\displaystyle {50}\)}%
\end{pgfscope}%
\begin{pgfscope}%
\pgfsetbuttcap%
\pgfsetroundjoin%
\definecolor{currentfill}{rgb}{0.000000,0.000000,0.000000}%
\pgfsetfillcolor{currentfill}%
\pgfsetlinewidth{0.803000pt}%
\definecolor{currentstroke}{rgb}{0.000000,0.000000,0.000000}%
\pgfsetstrokecolor{currentstroke}%
\pgfsetdash{}{0pt}%
\pgfsys@defobject{currentmarker}{\pgfqpoint{0.000000in}{-0.048611in}}{\pgfqpoint{0.000000in}{0.000000in}}{%
\pgfpathmoveto{\pgfqpoint{0.000000in}{0.000000in}}%
\pgfpathlineto{\pgfqpoint{0.000000in}{-0.048611in}}%
\pgfusepath{stroke,fill}%
}%
\begin{pgfscope}%
\pgfsys@transformshift{1.998294in}{0.535823in}%
\pgfsys@useobject{currentmarker}{}%
\end{pgfscope}%
\end{pgfscope}%
\begin{pgfscope}%
\definecolor{textcolor}{rgb}{0.000000,0.000000,0.000000}%
\pgfsetstrokecolor{textcolor}%
\pgfsetfillcolor{textcolor}%
\pgftext[x=1.998294in,y=0.438600in,,top]{\color{textcolor}\rmfamily\fontsize{9.000000}{10.800000}\selectfont \(\displaystyle {100}\)}%
\end{pgfscope}%
\begin{pgfscope}%
\pgfsetbuttcap%
\pgfsetroundjoin%
\definecolor{currentfill}{rgb}{0.000000,0.000000,0.000000}%
\pgfsetfillcolor{currentfill}%
\pgfsetlinewidth{0.803000pt}%
\definecolor{currentstroke}{rgb}{0.000000,0.000000,0.000000}%
\pgfsetstrokecolor{currentstroke}%
\pgfsetdash{}{0pt}%
\pgfsys@defobject{currentmarker}{\pgfqpoint{0.000000in}{-0.048611in}}{\pgfqpoint{0.000000in}{0.000000in}}{%
\pgfpathmoveto{\pgfqpoint{0.000000in}{0.000000in}}%
\pgfpathlineto{\pgfqpoint{0.000000in}{-0.048611in}}%
\pgfusepath{stroke,fill}%
}%
\begin{pgfscope}%
\pgfsys@transformshift{3.201952in}{0.535823in}%
\pgfsys@useobject{currentmarker}{}%
\end{pgfscope}%
\end{pgfscope}%
\begin{pgfscope}%
\definecolor{textcolor}{rgb}{0.000000,0.000000,0.000000}%
\pgfsetstrokecolor{textcolor}%
\pgfsetfillcolor{textcolor}%
\pgftext[x=3.201952in,y=0.438600in,,top]{\color{textcolor}\rmfamily\fontsize{9.000000}{10.800000}\selectfont \(\displaystyle {150}\)}%
\end{pgfscope}%
\begin{pgfscope}%
\pgfsetbuttcap%
\pgfsetroundjoin%
\definecolor{currentfill}{rgb}{0.000000,0.000000,0.000000}%
\pgfsetfillcolor{currentfill}%
\pgfsetlinewidth{0.803000pt}%
\definecolor{currentstroke}{rgb}{0.000000,0.000000,0.000000}%
\pgfsetstrokecolor{currentstroke}%
\pgfsetdash{}{0pt}%
\pgfsys@defobject{currentmarker}{\pgfqpoint{0.000000in}{-0.048611in}}{\pgfqpoint{0.000000in}{0.000000in}}{%
\pgfpathmoveto{\pgfqpoint{0.000000in}{0.000000in}}%
\pgfpathlineto{\pgfqpoint{0.000000in}{-0.048611in}}%
\pgfusepath{stroke,fill}%
}%
\begin{pgfscope}%
\pgfsys@transformshift{4.405610in}{0.535823in}%
\pgfsys@useobject{currentmarker}{}%
\end{pgfscope}%
\end{pgfscope}%
\begin{pgfscope}%
\definecolor{textcolor}{rgb}{0.000000,0.000000,0.000000}%
\pgfsetstrokecolor{textcolor}%
\pgfsetfillcolor{textcolor}%
\pgftext[x=4.405610in,y=0.438600in,,top]{\color{textcolor}\rmfamily\fontsize{9.000000}{10.800000}\selectfont \(\displaystyle {200}\)}%
\end{pgfscope}%
\begin{pgfscope}%
\pgfsetbuttcap%
\pgfsetroundjoin%
\definecolor{currentfill}{rgb}{0.000000,0.000000,0.000000}%
\pgfsetfillcolor{currentfill}%
\pgfsetlinewidth{0.803000pt}%
\definecolor{currentstroke}{rgb}{0.000000,0.000000,0.000000}%
\pgfsetstrokecolor{currentstroke}%
\pgfsetdash{}{0pt}%
\pgfsys@defobject{currentmarker}{\pgfqpoint{0.000000in}{-0.048611in}}{\pgfqpoint{0.000000in}{0.000000in}}{%
\pgfpathmoveto{\pgfqpoint{0.000000in}{0.000000in}}%
\pgfpathlineto{\pgfqpoint{0.000000in}{-0.048611in}}%
\pgfusepath{stroke,fill}%
}%
\begin{pgfscope}%
\pgfsys@transformshift{5.609268in}{0.535823in}%
\pgfsys@useobject{currentmarker}{}%
\end{pgfscope}%
\end{pgfscope}%
\begin{pgfscope}%
\definecolor{textcolor}{rgb}{0.000000,0.000000,0.000000}%
\pgfsetstrokecolor{textcolor}%
\pgfsetfillcolor{textcolor}%
\pgftext[x=5.609268in,y=0.438600in,,top]{\color{textcolor}\rmfamily\fontsize{9.000000}{10.800000}\selectfont \(\displaystyle {250}\)}%
\end{pgfscope}%
\begin{pgfscope}%
\definecolor{textcolor}{rgb}{0.000000,0.000000,0.000000}%
\pgfsetstrokecolor{textcolor}%
\pgfsetfillcolor{textcolor}%
\pgftext[x=3.201952in,y=0.272655in,,top]{\color{textcolor}\rmfamily\fontsize{10.000000}{12.000000}\selectfont \(\displaystyle n\): Number of vertices}%
\end{pgfscope}%
\begin{pgfscope}%
\pgfsetbuttcap%
\pgfsetroundjoin%
\definecolor{currentfill}{rgb}{0.000000,0.000000,0.000000}%
\pgfsetfillcolor{currentfill}%
\pgfsetlinewidth{0.803000pt}%
\definecolor{currentstroke}{rgb}{0.000000,0.000000,0.000000}%
\pgfsetstrokecolor{currentstroke}%
\pgfsetdash{}{0pt}%
\pgfsys@defobject{currentmarker}{\pgfqpoint{-0.048611in}{0.000000in}}{\pgfqpoint{-0.000000in}{0.000000in}}{%
\pgfpathmoveto{\pgfqpoint{-0.000000in}{0.000000in}}%
\pgfpathlineto{\pgfqpoint{-0.048611in}{0.000000in}}%
\pgfusepath{stroke,fill}%
}%
\begin{pgfscope}%
\pgfsys@transformshift{0.553904in}{0.719376in}%
\pgfsys@useobject{currentmarker}{}%
\end{pgfscope}%
\end{pgfscope}%
\begin{pgfscope}%
\definecolor{textcolor}{rgb}{0.000000,0.000000,0.000000}%
\pgfsetstrokecolor{textcolor}%
\pgfsetfillcolor{textcolor}%
\pgftext[x=0.328211in, y=0.676331in, left, base]{\color{textcolor}\rmfamily\fontsize{9.000000}{10.800000}\selectfont \(\displaystyle {10}\)}%
\end{pgfscope}%
\begin{pgfscope}%
\pgfsetbuttcap%
\pgfsetroundjoin%
\definecolor{currentfill}{rgb}{0.000000,0.000000,0.000000}%
\pgfsetfillcolor{currentfill}%
\pgfsetlinewidth{0.803000pt}%
\definecolor{currentstroke}{rgb}{0.000000,0.000000,0.000000}%
\pgfsetstrokecolor{currentstroke}%
\pgfsetdash{}{0pt}%
\pgfsys@defobject{currentmarker}{\pgfqpoint{-0.048611in}{0.000000in}}{\pgfqpoint{-0.000000in}{0.000000in}}{%
\pgfpathmoveto{\pgfqpoint{-0.000000in}{0.000000in}}%
\pgfpathlineto{\pgfqpoint{-0.048611in}{0.000000in}}%
\pgfusepath{stroke,fill}%
}%
\begin{pgfscope}%
\pgfsys@transformshift{0.553904in}{1.321189in}%
\pgfsys@useobject{currentmarker}{}%
\end{pgfscope}%
\end{pgfscope}%
\begin{pgfscope}%
\definecolor{textcolor}{rgb}{0.000000,0.000000,0.000000}%
\pgfsetstrokecolor{textcolor}%
\pgfsetfillcolor{textcolor}%
\pgftext[x=0.328211in, y=1.278144in, left, base]{\color{textcolor}\rmfamily\fontsize{9.000000}{10.800000}\selectfont \(\displaystyle {20}\)}%
\end{pgfscope}%
\begin{pgfscope}%
\pgfsetbuttcap%
\pgfsetroundjoin%
\definecolor{currentfill}{rgb}{0.000000,0.000000,0.000000}%
\pgfsetfillcolor{currentfill}%
\pgfsetlinewidth{0.803000pt}%
\definecolor{currentstroke}{rgb}{0.000000,0.000000,0.000000}%
\pgfsetstrokecolor{currentstroke}%
\pgfsetdash{}{0pt}%
\pgfsys@defobject{currentmarker}{\pgfqpoint{-0.048611in}{0.000000in}}{\pgfqpoint{-0.000000in}{0.000000in}}{%
\pgfpathmoveto{\pgfqpoint{-0.000000in}{0.000000in}}%
\pgfpathlineto{\pgfqpoint{-0.048611in}{0.000000in}}%
\pgfusepath{stroke,fill}%
}%
\begin{pgfscope}%
\pgfsys@transformshift{0.553904in}{1.923002in}%
\pgfsys@useobject{currentmarker}{}%
\end{pgfscope}%
\end{pgfscope}%
\begin{pgfscope}%
\definecolor{textcolor}{rgb}{0.000000,0.000000,0.000000}%
\pgfsetstrokecolor{textcolor}%
\pgfsetfillcolor{textcolor}%
\pgftext[x=0.328211in, y=1.879957in, left, base]{\color{textcolor}\rmfamily\fontsize{9.000000}{10.800000}\selectfont \(\displaystyle {30}\)}%
\end{pgfscope}%
\begin{pgfscope}%
\pgfsetbuttcap%
\pgfsetroundjoin%
\definecolor{currentfill}{rgb}{0.000000,0.000000,0.000000}%
\pgfsetfillcolor{currentfill}%
\pgfsetlinewidth{0.803000pt}%
\definecolor{currentstroke}{rgb}{0.000000,0.000000,0.000000}%
\pgfsetstrokecolor{currentstroke}%
\pgfsetdash{}{0pt}%
\pgfsys@defobject{currentmarker}{\pgfqpoint{-0.048611in}{0.000000in}}{\pgfqpoint{-0.000000in}{0.000000in}}{%
\pgfpathmoveto{\pgfqpoint{-0.000000in}{0.000000in}}%
\pgfpathlineto{\pgfqpoint{-0.048611in}{0.000000in}}%
\pgfusepath{stroke,fill}%
}%
\begin{pgfscope}%
\pgfsys@transformshift{0.553904in}{2.524815in}%
\pgfsys@useobject{currentmarker}{}%
\end{pgfscope}%
\end{pgfscope}%
\begin{pgfscope}%
\definecolor{textcolor}{rgb}{0.000000,0.000000,0.000000}%
\pgfsetstrokecolor{textcolor}%
\pgfsetfillcolor{textcolor}%
\pgftext[x=0.328211in, y=2.481770in, left, base]{\color{textcolor}\rmfamily\fontsize{9.000000}{10.800000}\selectfont \(\displaystyle {40}\)}%
\end{pgfscope}%
\begin{pgfscope}%
\pgfsetbuttcap%
\pgfsetroundjoin%
\definecolor{currentfill}{rgb}{0.000000,0.000000,0.000000}%
\pgfsetfillcolor{currentfill}%
\pgfsetlinewidth{0.803000pt}%
\definecolor{currentstroke}{rgb}{0.000000,0.000000,0.000000}%
\pgfsetstrokecolor{currentstroke}%
\pgfsetdash{}{0pt}%
\pgfsys@defobject{currentmarker}{\pgfqpoint{-0.048611in}{0.000000in}}{\pgfqpoint{-0.000000in}{0.000000in}}{%
\pgfpathmoveto{\pgfqpoint{-0.000000in}{0.000000in}}%
\pgfpathlineto{\pgfqpoint{-0.048611in}{0.000000in}}%
\pgfusepath{stroke,fill}%
}%
\begin{pgfscope}%
\pgfsys@transformshift{0.553904in}{3.126628in}%
\pgfsys@useobject{currentmarker}{}%
\end{pgfscope}%
\end{pgfscope}%
\begin{pgfscope}%
\definecolor{textcolor}{rgb}{0.000000,0.000000,0.000000}%
\pgfsetstrokecolor{textcolor}%
\pgfsetfillcolor{textcolor}%
\pgftext[x=0.328211in, y=3.083583in, left, base]{\color{textcolor}\rmfamily\fontsize{9.000000}{10.800000}\selectfont \(\displaystyle {50}\)}%
\end{pgfscope}%
\begin{pgfscope}%
\definecolor{textcolor}{rgb}{0.000000,0.000000,0.000000}%
\pgfsetstrokecolor{textcolor}%
\pgfsetfillcolor{textcolor}%
\pgftext[x=0.272655in,y=1.892911in,,bottom,rotate=90.000000]{\color{textcolor}\rmfamily\fontsize{10.000000}{12.000000}\selectfont Median max rank}%
\end{pgfscope}%
\begin{pgfscope}%
\pgfpathrectangle{\pgfqpoint{0.553904in}{0.535823in}}{\pgfqpoint{5.296096in}{2.714177in}}%
\pgfusepath{clip}%
\pgfsetrectcap%
\pgfsetroundjoin%
\pgfsetlinewidth{1.003750pt}%
\definecolor{currentstroke}{rgb}{0.529412,0.462745,0.384314}%
\pgfsetstrokecolor{currentstroke}%
\pgfsetdash{}{0pt}%
\pgfpathmoveto{\pgfqpoint{0.794636in}{0.779557in}}%
\pgfpathlineto{\pgfqpoint{1.035368in}{0.899920in}}%
\pgfpathlineto{\pgfqpoint{1.276099in}{0.960101in}}%
\pgfpathlineto{\pgfqpoint{1.516831in}{1.080464in}}%
\pgfpathlineto{\pgfqpoint{1.757562in}{1.200826in}}%
\pgfpathlineto{\pgfqpoint{1.998294in}{1.321189in}}%
\pgfpathlineto{\pgfqpoint{2.239026in}{1.441551in}}%
\pgfpathlineto{\pgfqpoint{2.479757in}{1.561914in}}%
\pgfpathlineto{\pgfqpoint{2.720489in}{1.682277in}}%
\pgfpathlineto{\pgfqpoint{2.961221in}{1.772549in}}%
\pgfpathlineto{\pgfqpoint{3.201952in}{1.923002in}}%
\pgfpathlineto{\pgfqpoint{3.442684in}{2.013274in}}%
\pgfpathlineto{\pgfqpoint{3.683415in}{2.103546in}}%
\pgfpathlineto{\pgfqpoint{3.924147in}{2.223909in}}%
\pgfpathlineto{\pgfqpoint{4.164879in}{2.344271in}}%
\pgfpathlineto{\pgfqpoint{4.405610in}{2.464634in}}%
\pgfpathlineto{\pgfqpoint{4.646342in}{2.584996in}}%
\pgfpathlineto{\pgfqpoint{4.887074in}{2.705359in}}%
\pgfpathlineto{\pgfqpoint{5.127805in}{2.885903in}}%
\pgfpathlineto{\pgfqpoint{5.368537in}{2.946084in}}%
\pgfpathlineto{\pgfqpoint{5.609268in}{3.126628in}}%
\pgfusepath{stroke}%
\end{pgfscope}%
\begin{pgfscope}%
\pgfpathrectangle{\pgfqpoint{0.553904in}{0.535823in}}{\pgfqpoint{5.296096in}{2.714177in}}%
\pgfusepath{clip}%
\pgfsetbuttcap%
\pgfsetmiterjoin%
\definecolor{currentfill}{rgb}{0.529412,0.462745,0.384314}%
\pgfsetfillcolor{currentfill}%
\pgfsetlinewidth{0.501875pt}%
\definecolor{currentstroke}{rgb}{0.000000,0.000000,0.000000}%
\pgfsetstrokecolor{currentstroke}%
\pgfsetdash{}{0pt}%
\pgfsys@defobject{currentmarker}{\pgfqpoint{-0.034722in}{-0.034722in}}{\pgfqpoint{0.034722in}{0.034722in}}{%
\pgfpathmoveto{\pgfqpoint{-0.000000in}{-0.034722in}}%
\pgfpathlineto{\pgfqpoint{0.034722in}{0.034722in}}%
\pgfpathlineto{\pgfqpoint{-0.034722in}{0.034722in}}%
\pgfpathclose%
\pgfusepath{stroke,fill}%
}%
\begin{pgfscope}%
\pgfsys@transformshift{0.794636in}{0.779557in}%
\pgfsys@useobject{currentmarker}{}%
\end{pgfscope}%
\begin{pgfscope}%
\pgfsys@transformshift{1.035368in}{0.899920in}%
\pgfsys@useobject{currentmarker}{}%
\end{pgfscope}%
\begin{pgfscope}%
\pgfsys@transformshift{1.276099in}{0.960101in}%
\pgfsys@useobject{currentmarker}{}%
\end{pgfscope}%
\begin{pgfscope}%
\pgfsys@transformshift{1.516831in}{1.080464in}%
\pgfsys@useobject{currentmarker}{}%
\end{pgfscope}%
\begin{pgfscope}%
\pgfsys@transformshift{1.757562in}{1.200826in}%
\pgfsys@useobject{currentmarker}{}%
\end{pgfscope}%
\begin{pgfscope}%
\pgfsys@transformshift{1.998294in}{1.321189in}%
\pgfsys@useobject{currentmarker}{}%
\end{pgfscope}%
\begin{pgfscope}%
\pgfsys@transformshift{2.239026in}{1.441551in}%
\pgfsys@useobject{currentmarker}{}%
\end{pgfscope}%
\begin{pgfscope}%
\pgfsys@transformshift{2.479757in}{1.561914in}%
\pgfsys@useobject{currentmarker}{}%
\end{pgfscope}%
\begin{pgfscope}%
\pgfsys@transformshift{2.720489in}{1.682277in}%
\pgfsys@useobject{currentmarker}{}%
\end{pgfscope}%
\begin{pgfscope}%
\pgfsys@transformshift{2.961221in}{1.772549in}%
\pgfsys@useobject{currentmarker}{}%
\end{pgfscope}%
\begin{pgfscope}%
\pgfsys@transformshift{3.201952in}{1.923002in}%
\pgfsys@useobject{currentmarker}{}%
\end{pgfscope}%
\begin{pgfscope}%
\pgfsys@transformshift{3.442684in}{2.013274in}%
\pgfsys@useobject{currentmarker}{}%
\end{pgfscope}%
\begin{pgfscope}%
\pgfsys@transformshift{3.683415in}{2.103546in}%
\pgfsys@useobject{currentmarker}{}%
\end{pgfscope}%
\begin{pgfscope}%
\pgfsys@transformshift{3.924147in}{2.223909in}%
\pgfsys@useobject{currentmarker}{}%
\end{pgfscope}%
\begin{pgfscope}%
\pgfsys@transformshift{4.164879in}{2.344271in}%
\pgfsys@useobject{currentmarker}{}%
\end{pgfscope}%
\begin{pgfscope}%
\pgfsys@transformshift{4.405610in}{2.464634in}%
\pgfsys@useobject{currentmarker}{}%
\end{pgfscope}%
\begin{pgfscope}%
\pgfsys@transformshift{4.646342in}{2.584996in}%
\pgfsys@useobject{currentmarker}{}%
\end{pgfscope}%
\begin{pgfscope}%
\pgfsys@transformshift{4.887074in}{2.705359in}%
\pgfsys@useobject{currentmarker}{}%
\end{pgfscope}%
\begin{pgfscope}%
\pgfsys@transformshift{5.127805in}{2.885903in}%
\pgfsys@useobject{currentmarker}{}%
\end{pgfscope}%
\begin{pgfscope}%
\pgfsys@transformshift{5.368537in}{2.946084in}%
\pgfsys@useobject{currentmarker}{}%
\end{pgfscope}%
\begin{pgfscope}%
\pgfsys@transformshift{5.609268in}{3.126628in}%
\pgfsys@useobject{currentmarker}{}%
\end{pgfscope}%
\end{pgfscope}%
\begin{pgfscope}%
\pgfpathrectangle{\pgfqpoint{0.553904in}{0.535823in}}{\pgfqpoint{5.296096in}{2.714177in}}%
\pgfusepath{clip}%
\pgfsetrectcap%
\pgfsetroundjoin%
\pgfsetlinewidth{1.003750pt}%
\definecolor{currentstroke}{rgb}{0.611765,0.568627,0.274510}%
\pgfsetstrokecolor{currentstroke}%
\pgfsetdash{}{0pt}%
\pgfpathmoveto{\pgfqpoint{0.794636in}{0.719376in}}%
\pgfpathlineto{\pgfqpoint{1.035368in}{0.779557in}}%
\pgfpathlineto{\pgfqpoint{1.276099in}{0.839738in}}%
\pgfpathlineto{\pgfqpoint{1.516831in}{0.899920in}}%
\pgfpathlineto{\pgfqpoint{1.757562in}{1.020282in}}%
\pgfpathlineto{\pgfqpoint{1.998294in}{1.080464in}}%
\pgfpathlineto{\pgfqpoint{2.239026in}{1.140645in}}%
\pgfpathlineto{\pgfqpoint{2.479757in}{1.200826in}}%
\pgfpathlineto{\pgfqpoint{2.720489in}{1.321189in}}%
\pgfpathlineto{\pgfqpoint{2.961221in}{1.411461in}}%
\pgfpathlineto{\pgfqpoint{3.201952in}{1.471642in}}%
\pgfpathlineto{\pgfqpoint{3.442684in}{1.561914in}}%
\pgfpathlineto{\pgfqpoint{3.683415in}{1.622095in}}%
\pgfpathlineto{\pgfqpoint{3.924147in}{1.742458in}}%
\pgfpathlineto{\pgfqpoint{4.164879in}{1.802639in}}%
\pgfpathlineto{\pgfqpoint{4.405610in}{1.862821in}}%
\pgfpathlineto{\pgfqpoint{4.646342in}{1.923002in}}%
\pgfpathlineto{\pgfqpoint{4.887074in}{2.043365in}}%
\pgfpathlineto{\pgfqpoint{5.127805in}{2.163727in}}%
\pgfpathlineto{\pgfqpoint{5.368537in}{2.223909in}}%
\pgfpathlineto{\pgfqpoint{5.609268in}{2.284090in}}%
\pgfusepath{stroke}%
\end{pgfscope}%
\begin{pgfscope}%
\pgfpathrectangle{\pgfqpoint{0.553904in}{0.535823in}}{\pgfqpoint{5.296096in}{2.714177in}}%
\pgfusepath{clip}%
\pgfsetbuttcap%
\pgfsetmiterjoin%
\definecolor{currentfill}{rgb}{0.611765,0.568627,0.274510}%
\pgfsetfillcolor{currentfill}%
\pgfsetlinewidth{0.501875pt}%
\definecolor{currentstroke}{rgb}{0.000000,0.000000,0.000000}%
\pgfsetstrokecolor{currentstroke}%
\pgfsetdash{}{0pt}%
\pgfsys@defobject{currentmarker}{\pgfqpoint{-0.034722in}{-0.034722in}}{\pgfqpoint{0.034722in}{0.034722in}}{%
\pgfpathmoveto{\pgfqpoint{-0.034722in}{0.000000in}}%
\pgfpathlineto{\pgfqpoint{0.034722in}{-0.034722in}}%
\pgfpathlineto{\pgfqpoint{0.034722in}{0.034722in}}%
\pgfpathclose%
\pgfusepath{stroke,fill}%
}%
\begin{pgfscope}%
\pgfsys@transformshift{0.794636in}{0.719376in}%
\pgfsys@useobject{currentmarker}{}%
\end{pgfscope}%
\begin{pgfscope}%
\pgfsys@transformshift{1.035368in}{0.779557in}%
\pgfsys@useobject{currentmarker}{}%
\end{pgfscope}%
\begin{pgfscope}%
\pgfsys@transformshift{1.276099in}{0.839738in}%
\pgfsys@useobject{currentmarker}{}%
\end{pgfscope}%
\begin{pgfscope}%
\pgfsys@transformshift{1.516831in}{0.899920in}%
\pgfsys@useobject{currentmarker}{}%
\end{pgfscope}%
\begin{pgfscope}%
\pgfsys@transformshift{1.757562in}{1.020282in}%
\pgfsys@useobject{currentmarker}{}%
\end{pgfscope}%
\begin{pgfscope}%
\pgfsys@transformshift{1.998294in}{1.080464in}%
\pgfsys@useobject{currentmarker}{}%
\end{pgfscope}%
\begin{pgfscope}%
\pgfsys@transformshift{2.239026in}{1.140645in}%
\pgfsys@useobject{currentmarker}{}%
\end{pgfscope}%
\begin{pgfscope}%
\pgfsys@transformshift{2.479757in}{1.200826in}%
\pgfsys@useobject{currentmarker}{}%
\end{pgfscope}%
\begin{pgfscope}%
\pgfsys@transformshift{2.720489in}{1.321189in}%
\pgfsys@useobject{currentmarker}{}%
\end{pgfscope}%
\begin{pgfscope}%
\pgfsys@transformshift{2.961221in}{1.411461in}%
\pgfsys@useobject{currentmarker}{}%
\end{pgfscope}%
\begin{pgfscope}%
\pgfsys@transformshift{3.201952in}{1.471642in}%
\pgfsys@useobject{currentmarker}{}%
\end{pgfscope}%
\begin{pgfscope}%
\pgfsys@transformshift{3.442684in}{1.561914in}%
\pgfsys@useobject{currentmarker}{}%
\end{pgfscope}%
\begin{pgfscope}%
\pgfsys@transformshift{3.683415in}{1.622095in}%
\pgfsys@useobject{currentmarker}{}%
\end{pgfscope}%
\begin{pgfscope}%
\pgfsys@transformshift{3.924147in}{1.742458in}%
\pgfsys@useobject{currentmarker}{}%
\end{pgfscope}%
\begin{pgfscope}%
\pgfsys@transformshift{4.164879in}{1.802639in}%
\pgfsys@useobject{currentmarker}{}%
\end{pgfscope}%
\begin{pgfscope}%
\pgfsys@transformshift{4.405610in}{1.862821in}%
\pgfsys@useobject{currentmarker}{}%
\end{pgfscope}%
\begin{pgfscope}%
\pgfsys@transformshift{4.646342in}{1.923002in}%
\pgfsys@useobject{currentmarker}{}%
\end{pgfscope}%
\begin{pgfscope}%
\pgfsys@transformshift{4.887074in}{2.043365in}%
\pgfsys@useobject{currentmarker}{}%
\end{pgfscope}%
\begin{pgfscope}%
\pgfsys@transformshift{5.127805in}{2.163727in}%
\pgfsys@useobject{currentmarker}{}%
\end{pgfscope}%
\begin{pgfscope}%
\pgfsys@transformshift{5.368537in}{2.223909in}%
\pgfsys@useobject{currentmarker}{}%
\end{pgfscope}%
\begin{pgfscope}%
\pgfsys@transformshift{5.609268in}{2.284090in}%
\pgfsys@useobject{currentmarker}{}%
\end{pgfscope}%
\end{pgfscope}%
\begin{pgfscope}%
\pgfpathrectangle{\pgfqpoint{0.553904in}{0.535823in}}{\pgfqpoint{5.296096in}{2.714177in}}%
\pgfusepath{clip}%
\pgfsetrectcap%
\pgfsetroundjoin%
\pgfsetlinewidth{1.003750pt}%
\definecolor{currentstroke}{rgb}{0.780392,0.643137,0.254902}%
\pgfsetstrokecolor{currentstroke}%
\pgfsetdash{}{0pt}%
\pgfpathmoveto{\pgfqpoint{0.794636in}{0.659194in}}%
\pgfpathlineto{\pgfqpoint{1.035368in}{0.779557in}}%
\pgfpathlineto{\pgfqpoint{1.276099in}{0.839738in}}%
\pgfpathlineto{\pgfqpoint{1.516831in}{0.899920in}}%
\pgfpathlineto{\pgfqpoint{1.757562in}{1.020282in}}%
\pgfpathlineto{\pgfqpoint{1.998294in}{1.080464in}}%
\pgfpathlineto{\pgfqpoint{2.239026in}{1.140645in}}%
\pgfpathlineto{\pgfqpoint{2.479757in}{1.261007in}}%
\pgfpathlineto{\pgfqpoint{2.720489in}{1.321189in}}%
\pgfpathlineto{\pgfqpoint{2.961221in}{1.441551in}}%
\pgfpathlineto{\pgfqpoint{3.201952in}{1.501733in}}%
\pgfpathlineto{\pgfqpoint{3.442684in}{1.561914in}}%
\pgfpathlineto{\pgfqpoint{3.683415in}{1.622095in}}%
\pgfpathlineto{\pgfqpoint{3.924147in}{1.742458in}}%
\pgfpathlineto{\pgfqpoint{4.164879in}{1.802639in}}%
\pgfpathlineto{\pgfqpoint{4.405610in}{1.923002in}}%
\pgfpathlineto{\pgfqpoint{4.646342in}{1.923002in}}%
\pgfpathlineto{\pgfqpoint{4.887074in}{2.043365in}}%
\pgfpathlineto{\pgfqpoint{5.127805in}{2.103546in}}%
\pgfpathlineto{\pgfqpoint{5.368537in}{2.223909in}}%
\pgfpathlineto{\pgfqpoint{5.609268in}{2.344271in}}%
\pgfusepath{stroke}%
\end{pgfscope}%
\begin{pgfscope}%
\pgfpathrectangle{\pgfqpoint{0.553904in}{0.535823in}}{\pgfqpoint{5.296096in}{2.714177in}}%
\pgfusepath{clip}%
\pgfsetbuttcap%
\pgfsetmiterjoin%
\definecolor{currentfill}{rgb}{0.780392,0.643137,0.254902}%
\pgfsetfillcolor{currentfill}%
\pgfsetlinewidth{0.501875pt}%
\definecolor{currentstroke}{rgb}{0.000000,0.000000,0.000000}%
\pgfsetstrokecolor{currentstroke}%
\pgfsetdash{}{0pt}%
\pgfsys@defobject{currentmarker}{\pgfqpoint{-0.034722in}{-0.034722in}}{\pgfqpoint{0.034722in}{0.034722in}}{%
\pgfpathmoveto{\pgfqpoint{0.034722in}{-0.000000in}}%
\pgfpathlineto{\pgfqpoint{-0.034722in}{0.034722in}}%
\pgfpathlineto{\pgfqpoint{-0.034722in}{-0.034722in}}%
\pgfpathclose%
\pgfusepath{stroke,fill}%
}%
\begin{pgfscope}%
\pgfsys@transformshift{0.794636in}{0.659194in}%
\pgfsys@useobject{currentmarker}{}%
\end{pgfscope}%
\begin{pgfscope}%
\pgfsys@transformshift{1.035368in}{0.779557in}%
\pgfsys@useobject{currentmarker}{}%
\end{pgfscope}%
\begin{pgfscope}%
\pgfsys@transformshift{1.276099in}{0.839738in}%
\pgfsys@useobject{currentmarker}{}%
\end{pgfscope}%
\begin{pgfscope}%
\pgfsys@transformshift{1.516831in}{0.899920in}%
\pgfsys@useobject{currentmarker}{}%
\end{pgfscope}%
\begin{pgfscope}%
\pgfsys@transformshift{1.757562in}{1.020282in}%
\pgfsys@useobject{currentmarker}{}%
\end{pgfscope}%
\begin{pgfscope}%
\pgfsys@transformshift{1.998294in}{1.080464in}%
\pgfsys@useobject{currentmarker}{}%
\end{pgfscope}%
\begin{pgfscope}%
\pgfsys@transformshift{2.239026in}{1.140645in}%
\pgfsys@useobject{currentmarker}{}%
\end{pgfscope}%
\begin{pgfscope}%
\pgfsys@transformshift{2.479757in}{1.261007in}%
\pgfsys@useobject{currentmarker}{}%
\end{pgfscope}%
\begin{pgfscope}%
\pgfsys@transformshift{2.720489in}{1.321189in}%
\pgfsys@useobject{currentmarker}{}%
\end{pgfscope}%
\begin{pgfscope}%
\pgfsys@transformshift{2.961221in}{1.441551in}%
\pgfsys@useobject{currentmarker}{}%
\end{pgfscope}%
\begin{pgfscope}%
\pgfsys@transformshift{3.201952in}{1.501733in}%
\pgfsys@useobject{currentmarker}{}%
\end{pgfscope}%
\begin{pgfscope}%
\pgfsys@transformshift{3.442684in}{1.561914in}%
\pgfsys@useobject{currentmarker}{}%
\end{pgfscope}%
\begin{pgfscope}%
\pgfsys@transformshift{3.683415in}{1.622095in}%
\pgfsys@useobject{currentmarker}{}%
\end{pgfscope}%
\begin{pgfscope}%
\pgfsys@transformshift{3.924147in}{1.742458in}%
\pgfsys@useobject{currentmarker}{}%
\end{pgfscope}%
\begin{pgfscope}%
\pgfsys@transformshift{4.164879in}{1.802639in}%
\pgfsys@useobject{currentmarker}{}%
\end{pgfscope}%
\begin{pgfscope}%
\pgfsys@transformshift{4.405610in}{1.923002in}%
\pgfsys@useobject{currentmarker}{}%
\end{pgfscope}%
\begin{pgfscope}%
\pgfsys@transformshift{4.646342in}{1.923002in}%
\pgfsys@useobject{currentmarker}{}%
\end{pgfscope}%
\begin{pgfscope}%
\pgfsys@transformshift{4.887074in}{2.043365in}%
\pgfsys@useobject{currentmarker}{}%
\end{pgfscope}%
\begin{pgfscope}%
\pgfsys@transformshift{5.127805in}{2.103546in}%
\pgfsys@useobject{currentmarker}{}%
\end{pgfscope}%
\begin{pgfscope}%
\pgfsys@transformshift{5.368537in}{2.223909in}%
\pgfsys@useobject{currentmarker}{}%
\end{pgfscope}%
\begin{pgfscope}%
\pgfsys@transformshift{5.609268in}{2.344271in}%
\pgfsys@useobject{currentmarker}{}%
\end{pgfscope}%
\end{pgfscope}%
\begin{pgfscope}%
\pgfpathrectangle{\pgfqpoint{0.553904in}{0.535823in}}{\pgfqpoint{5.296096in}{2.714177in}}%
\pgfusepath{clip}%
\pgfsetrectcap%
\pgfsetroundjoin%
\pgfsetlinewidth{1.003750pt}%
\definecolor{currentstroke}{rgb}{1.000000,0.694118,0.305882}%
\pgfsetstrokecolor{currentstroke}%
\pgfsetdash{}{0pt}%
\pgfpathmoveto{\pgfqpoint{0.794636in}{0.839738in}}%
\pgfpathlineto{\pgfqpoint{1.035368in}{0.960101in}}%
\pgfpathlineto{\pgfqpoint{1.276099in}{1.080464in}}%
\pgfpathlineto{\pgfqpoint{1.516831in}{1.200826in}}%
\pgfpathlineto{\pgfqpoint{1.757562in}{1.321189in}}%
\pgfpathlineto{\pgfqpoint{1.998294in}{1.381370in}}%
\pgfpathlineto{\pgfqpoint{2.239026in}{1.471642in}}%
\pgfpathlineto{\pgfqpoint{2.479757in}{1.501733in}}%
\pgfpathlineto{\pgfqpoint{2.720489in}{1.561914in}}%
\pgfpathlineto{\pgfqpoint{2.961221in}{1.561914in}}%
\pgfpathlineto{\pgfqpoint{3.201952in}{1.561914in}}%
\pgfpathlineto{\pgfqpoint{3.442684in}{1.561914in}}%
\pgfpathlineto{\pgfqpoint{3.683415in}{1.561914in}}%
\pgfpathlineto{\pgfqpoint{3.924147in}{1.622095in}}%
\pgfpathlineto{\pgfqpoint{4.164879in}{1.622095in}}%
\pgfpathlineto{\pgfqpoint{4.405610in}{1.682277in}}%
\pgfpathlineto{\pgfqpoint{4.646342in}{1.742458in}}%
\pgfpathlineto{\pgfqpoint{4.887074in}{1.802639in}}%
\pgfpathlineto{\pgfqpoint{5.127805in}{1.923002in}}%
\pgfpathlineto{\pgfqpoint{5.368537in}{1.953093in}}%
\pgfpathlineto{\pgfqpoint{5.609268in}{2.043365in}}%
\pgfusepath{stroke}%
\end{pgfscope}%
\begin{pgfscope}%
\pgfpathrectangle{\pgfqpoint{0.553904in}{0.535823in}}{\pgfqpoint{5.296096in}{2.714177in}}%
\pgfusepath{clip}%
\pgfsetbuttcap%
\pgfsetbeveljoin%
\definecolor{currentfill}{rgb}{1.000000,0.694118,0.305882}%
\pgfsetfillcolor{currentfill}%
\pgfsetlinewidth{0.501875pt}%
\definecolor{currentstroke}{rgb}{0.000000,0.000000,0.000000}%
\pgfsetstrokecolor{currentstroke}%
\pgfsetdash{}{0pt}%
\pgfsys@defobject{currentmarker}{\pgfqpoint{-0.033023in}{-0.028091in}}{\pgfqpoint{0.033023in}{0.034722in}}{%
\pgfpathmoveto{\pgfqpoint{0.000000in}{0.034722in}}%
\pgfpathlineto{\pgfqpoint{-0.007796in}{0.010730in}}%
\pgfpathlineto{\pgfqpoint{-0.033023in}{0.010730in}}%
\pgfpathlineto{\pgfqpoint{-0.012614in}{-0.004098in}}%
\pgfpathlineto{\pgfqpoint{-0.020409in}{-0.028091in}}%
\pgfpathlineto{\pgfqpoint{-0.000000in}{-0.013263in}}%
\pgfpathlineto{\pgfqpoint{0.020409in}{-0.028091in}}%
\pgfpathlineto{\pgfqpoint{0.012614in}{-0.004098in}}%
\pgfpathlineto{\pgfqpoint{0.033023in}{0.010730in}}%
\pgfpathlineto{\pgfqpoint{0.007796in}{0.010730in}}%
\pgfpathclose%
\pgfusepath{stroke,fill}%
}%
\begin{pgfscope}%
\pgfsys@transformshift{0.794636in}{0.839738in}%
\pgfsys@useobject{currentmarker}{}%
\end{pgfscope}%
\begin{pgfscope}%
\pgfsys@transformshift{1.035368in}{0.960101in}%
\pgfsys@useobject{currentmarker}{}%
\end{pgfscope}%
\begin{pgfscope}%
\pgfsys@transformshift{1.276099in}{1.080464in}%
\pgfsys@useobject{currentmarker}{}%
\end{pgfscope}%
\begin{pgfscope}%
\pgfsys@transformshift{1.516831in}{1.200826in}%
\pgfsys@useobject{currentmarker}{}%
\end{pgfscope}%
\begin{pgfscope}%
\pgfsys@transformshift{1.757562in}{1.321189in}%
\pgfsys@useobject{currentmarker}{}%
\end{pgfscope}%
\begin{pgfscope}%
\pgfsys@transformshift{1.998294in}{1.381370in}%
\pgfsys@useobject{currentmarker}{}%
\end{pgfscope}%
\begin{pgfscope}%
\pgfsys@transformshift{2.239026in}{1.471642in}%
\pgfsys@useobject{currentmarker}{}%
\end{pgfscope}%
\begin{pgfscope}%
\pgfsys@transformshift{2.479757in}{1.501733in}%
\pgfsys@useobject{currentmarker}{}%
\end{pgfscope}%
\begin{pgfscope}%
\pgfsys@transformshift{2.720489in}{1.561914in}%
\pgfsys@useobject{currentmarker}{}%
\end{pgfscope}%
\begin{pgfscope}%
\pgfsys@transformshift{2.961221in}{1.561914in}%
\pgfsys@useobject{currentmarker}{}%
\end{pgfscope}%
\begin{pgfscope}%
\pgfsys@transformshift{3.201952in}{1.561914in}%
\pgfsys@useobject{currentmarker}{}%
\end{pgfscope}%
\begin{pgfscope}%
\pgfsys@transformshift{3.442684in}{1.561914in}%
\pgfsys@useobject{currentmarker}{}%
\end{pgfscope}%
\begin{pgfscope}%
\pgfsys@transformshift{3.683415in}{1.561914in}%
\pgfsys@useobject{currentmarker}{}%
\end{pgfscope}%
\begin{pgfscope}%
\pgfsys@transformshift{3.924147in}{1.622095in}%
\pgfsys@useobject{currentmarker}{}%
\end{pgfscope}%
\begin{pgfscope}%
\pgfsys@transformshift{4.164879in}{1.622095in}%
\pgfsys@useobject{currentmarker}{}%
\end{pgfscope}%
\begin{pgfscope}%
\pgfsys@transformshift{4.405610in}{1.682277in}%
\pgfsys@useobject{currentmarker}{}%
\end{pgfscope}%
\begin{pgfscope}%
\pgfsys@transformshift{4.646342in}{1.742458in}%
\pgfsys@useobject{currentmarker}{}%
\end{pgfscope}%
\begin{pgfscope}%
\pgfsys@transformshift{4.887074in}{1.802639in}%
\pgfsys@useobject{currentmarker}{}%
\end{pgfscope}%
\begin{pgfscope}%
\pgfsys@transformshift{5.127805in}{1.923002in}%
\pgfsys@useobject{currentmarker}{}%
\end{pgfscope}%
\begin{pgfscope}%
\pgfsys@transformshift{5.368537in}{1.953093in}%
\pgfsys@useobject{currentmarker}{}%
\end{pgfscope}%
\begin{pgfscope}%
\pgfsys@transformshift{5.609268in}{2.043365in}%
\pgfsys@useobject{currentmarker}{}%
\end{pgfscope}%
\end{pgfscope}%
\begin{pgfscope}%
\pgfsetrectcap%
\pgfsetmiterjoin%
\pgfsetlinewidth{0.803000pt}%
\definecolor{currentstroke}{rgb}{0.000000,0.000000,0.000000}%
\pgfsetstrokecolor{currentstroke}%
\pgfsetdash{}{0pt}%
\pgfpathmoveto{\pgfqpoint{0.553904in}{0.535823in}}%
\pgfpathlineto{\pgfqpoint{0.553904in}{3.250000in}}%
\pgfusepath{stroke}%
\end{pgfscope}%
\begin{pgfscope}%
\pgfsetrectcap%
\pgfsetmiterjoin%
\pgfsetlinewidth{0.803000pt}%
\definecolor{currentstroke}{rgb}{0.000000,0.000000,0.000000}%
\pgfsetstrokecolor{currentstroke}%
\pgfsetdash{}{0pt}%
\pgfpathmoveto{\pgfqpoint{5.850000in}{0.535823in}}%
\pgfpathlineto{\pgfqpoint{5.850000in}{3.250000in}}%
\pgfusepath{stroke}%
\end{pgfscope}%
\begin{pgfscope}%
\pgfsetrectcap%
\pgfsetmiterjoin%
\pgfsetlinewidth{0.803000pt}%
\definecolor{currentstroke}{rgb}{0.000000,0.000000,0.000000}%
\pgfsetstrokecolor{currentstroke}%
\pgfsetdash{}{0pt}%
\pgfpathmoveto{\pgfqpoint{0.553904in}{0.535823in}}%
\pgfpathlineto{\pgfqpoint{5.850000in}{0.535823in}}%
\pgfusepath{stroke}%
\end{pgfscope}%
\begin{pgfscope}%
\pgfsetrectcap%
\pgfsetmiterjoin%
\pgfsetlinewidth{0.803000pt}%
\definecolor{currentstroke}{rgb}{0.000000,0.000000,0.000000}%
\pgfsetstrokecolor{currentstroke}%
\pgfsetdash{}{0pt}%
\pgfpathmoveto{\pgfqpoint{0.553904in}{3.250000in}}%
\pgfpathlineto{\pgfqpoint{5.850000in}{3.250000in}}%
\pgfusepath{stroke}%
\end{pgfscope}%
\begin{pgfscope}%
\pgfsetrectcap%
\pgfsetroundjoin%
\pgfsetlinewidth{1.003750pt}%
\definecolor{currentstroke}{rgb}{0.529412,0.462745,0.384314}%
\pgfsetstrokecolor{currentstroke}%
\pgfsetdash{}{0pt}%
\pgfpathmoveto{\pgfqpoint{0.603904in}{3.156250in}}%
\pgfpathlineto{\pgfqpoint{0.853904in}{3.156250in}}%
\pgfusepath{stroke}%
\end{pgfscope}%
\begin{pgfscope}%
\pgfsetbuttcap%
\pgfsetmiterjoin%
\definecolor{currentfill}{rgb}{0.529412,0.462745,0.384314}%
\pgfsetfillcolor{currentfill}%
\pgfsetlinewidth{0.501875pt}%
\definecolor{currentstroke}{rgb}{0.000000,0.000000,0.000000}%
\pgfsetstrokecolor{currentstroke}%
\pgfsetdash{}{0pt}%
\pgfsys@defobject{currentmarker}{\pgfqpoint{-0.034722in}{-0.034722in}}{\pgfqpoint{0.034722in}{0.034722in}}{%
\pgfpathmoveto{\pgfqpoint{-0.000000in}{-0.034722in}}%
\pgfpathlineto{\pgfqpoint{0.034722in}{0.034722in}}%
\pgfpathlineto{\pgfqpoint{-0.034722in}{0.034722in}}%
\pgfpathclose%
\pgfusepath{stroke,fill}%
}%
\begin{pgfscope}%
\pgfsys@transformshift{0.728904in}{3.156250in}%
\pgfsys@useobject{currentmarker}{}%
\end{pgfscope}%
\end{pgfscope}%
\begin{pgfscope}%
\definecolor{textcolor}{rgb}{0.000000,0.000000,0.000000}%
\pgfsetstrokecolor{textcolor}%
\pgfsetfillcolor{textcolor}%
\pgftext[x=0.878904in,y=3.112500in,left,base]{\color{textcolor}\rmfamily\fontsize{9.000000}{10.800000}\selectfont greedy}%
\end{pgfscope}%
\begin{pgfscope}%
\pgfsetrectcap%
\pgfsetroundjoin%
\pgfsetlinewidth{1.003750pt}%
\definecolor{currentstroke}{rgb}{0.611765,0.568627,0.274510}%
\pgfsetstrokecolor{currentstroke}%
\pgfsetdash{}{0pt}%
\pgfpathmoveto{\pgfqpoint{0.603904in}{2.994450in}}%
\pgfpathlineto{\pgfqpoint{0.853904in}{2.994450in}}%
\pgfusepath{stroke}%
\end{pgfscope}%
\begin{pgfscope}%
\pgfsetbuttcap%
\pgfsetmiterjoin%
\definecolor{currentfill}{rgb}{0.611765,0.568627,0.274510}%
\pgfsetfillcolor{currentfill}%
\pgfsetlinewidth{0.501875pt}%
\definecolor{currentstroke}{rgb}{0.000000,0.000000,0.000000}%
\pgfsetstrokecolor{currentstroke}%
\pgfsetdash{}{0pt}%
\pgfsys@defobject{currentmarker}{\pgfqpoint{-0.034722in}{-0.034722in}}{\pgfqpoint{0.034722in}{0.034722in}}{%
\pgfpathmoveto{\pgfqpoint{-0.034722in}{0.000000in}}%
\pgfpathlineto{\pgfqpoint{0.034722in}{-0.034722in}}%
\pgfpathlineto{\pgfqpoint{0.034722in}{0.034722in}}%
\pgfpathclose%
\pgfusepath{stroke,fill}%
}%
\begin{pgfscope}%
\pgfsys@transformshift{0.728904in}{2.994450in}%
\pgfsys@useobject{currentmarker}{}%
\end{pgfscope}%
\end{pgfscope}%
\begin{pgfscope}%
\definecolor{textcolor}{rgb}{0.000000,0.000000,0.000000}%
\pgfsetstrokecolor{textcolor}%
\pgfsetfillcolor{textcolor}%
\pgftext[x=0.878904in,y=2.950700in,left,base]{\color{textcolor}\rmfamily\fontsize{9.000000}{10.800000}\selectfont metis}%
\end{pgfscope}%
\begin{pgfscope}%
\pgfsetrectcap%
\pgfsetroundjoin%
\pgfsetlinewidth{1.003750pt}%
\definecolor{currentstroke}{rgb}{0.780392,0.643137,0.254902}%
\pgfsetstrokecolor{currentstroke}%
\pgfsetdash{}{0pt}%
\pgfpathmoveto{\pgfqpoint{0.603904in}{2.832651in}}%
\pgfpathlineto{\pgfqpoint{0.853904in}{2.832651in}}%
\pgfusepath{stroke}%
\end{pgfscope}%
\begin{pgfscope}%
\pgfsetbuttcap%
\pgfsetmiterjoin%
\definecolor{currentfill}{rgb}{0.780392,0.643137,0.254902}%
\pgfsetfillcolor{currentfill}%
\pgfsetlinewidth{0.501875pt}%
\definecolor{currentstroke}{rgb}{0.000000,0.000000,0.000000}%
\pgfsetstrokecolor{currentstroke}%
\pgfsetdash{}{0pt}%
\pgfsys@defobject{currentmarker}{\pgfqpoint{-0.034722in}{-0.034722in}}{\pgfqpoint{0.034722in}{0.034722in}}{%
\pgfpathmoveto{\pgfqpoint{0.034722in}{-0.000000in}}%
\pgfpathlineto{\pgfqpoint{-0.034722in}{0.034722in}}%
\pgfpathlineto{\pgfqpoint{-0.034722in}{-0.034722in}}%
\pgfpathclose%
\pgfusepath{stroke,fill}%
}%
\begin{pgfscope}%
\pgfsys@transformshift{0.728904in}{2.832651in}%
\pgfsys@useobject{currentmarker}{}%
\end{pgfscope}%
\end{pgfscope}%
\begin{pgfscope}%
\definecolor{textcolor}{rgb}{0.000000,0.000000,0.000000}%
\pgfsetstrokecolor{textcolor}%
\pgfsetfillcolor{textcolor}%
\pgftext[x=0.878904in,y=2.788901in,left,base]{\color{textcolor}\rmfamily\fontsize{9.000000}{10.800000}\selectfont GN}%
\end{pgfscope}%
\begin{pgfscope}%
\pgfsetrectcap%
\pgfsetroundjoin%
\pgfsetlinewidth{1.003750pt}%
\definecolor{currentstroke}{rgb}{1.000000,0.694118,0.305882}%
\pgfsetstrokecolor{currentstroke}%
\pgfsetdash{}{0pt}%
\pgfpathmoveto{\pgfqpoint{0.603904in}{2.670851in}}%
\pgfpathlineto{\pgfqpoint{0.853904in}{2.670851in}}%
\pgfusepath{stroke}%
\end{pgfscope}%
\begin{pgfscope}%
\pgfsetbuttcap%
\pgfsetbeveljoin%
\definecolor{currentfill}{rgb}{1.000000,0.694118,0.305882}%
\pgfsetfillcolor{currentfill}%
\pgfsetlinewidth{0.501875pt}%
\definecolor{currentstroke}{rgb}{0.000000,0.000000,0.000000}%
\pgfsetstrokecolor{currentstroke}%
\pgfsetdash{}{0pt}%
\pgfsys@defobject{currentmarker}{\pgfqpoint{-0.033023in}{-0.028091in}}{\pgfqpoint{0.033023in}{0.034722in}}{%
\pgfpathmoveto{\pgfqpoint{0.000000in}{0.034722in}}%
\pgfpathlineto{\pgfqpoint{-0.007796in}{0.010730in}}%
\pgfpathlineto{\pgfqpoint{-0.033023in}{0.010730in}}%
\pgfpathlineto{\pgfqpoint{-0.012614in}{-0.004098in}}%
\pgfpathlineto{\pgfqpoint{-0.020409in}{-0.028091in}}%
\pgfpathlineto{\pgfqpoint{-0.000000in}{-0.013263in}}%
\pgfpathlineto{\pgfqpoint{0.020409in}{-0.028091in}}%
\pgfpathlineto{\pgfqpoint{0.012614in}{-0.004098in}}%
\pgfpathlineto{\pgfqpoint{0.033023in}{0.010730in}}%
\pgfpathlineto{\pgfqpoint{0.007796in}{0.010730in}}%
\pgfpathclose%
\pgfusepath{stroke,fill}%
}%
\begin{pgfscope}%
\pgfsys@transformshift{0.728904in}{2.670851in}%
\pgfsys@useobject{currentmarker}{}%
\end{pgfscope}%
\end{pgfscope}%
\begin{pgfscope}%
\definecolor{textcolor}{rgb}{0.000000,0.000000,0.000000}%
\pgfsetstrokecolor{textcolor}%
\pgfsetfillcolor{textcolor}%
\pgftext[x=0.878904in,y=2.627101in,left,base]{\color{textcolor}\rmfamily\fontsize{9.000000}{10.800000}\selectfont LG+Flow}%
\end{pgfscope}%
\end{pgfpicture}%
\makeatother%
\endgroup%

%	\caption{\label{fig:cubic-rank} Median max rank of the contraction tree found by various tensor-based methods, on tensor networks that count the number of vertex covers of 100 randomly-sampled cubic graphs with $n$ vertices. Our contribution \textbf{Line+Flow} finds lower max-rank contraction trees than other methods when $n \geq 170$.}
%\end{figure}

\subsection{Unweighted Model Counting: Vertex Covers of Cubic Graphs}
\label{sec:tensors:experiments:cubic}

% (i.e., the number of sets of vertices where every edge of the graph is incident to at least one vertex) 
We first compare on benchmarks that count the number of vertex covers of randomly-generated cubic graphs \cite{KCMR18}. For each number of vertices $n \in \{50, 60, 70, \cdots, 250\}$ we randomly sample 100 connected cubic graphs using a Monte Carlo procedure \cite{VL05}. These benchmarks are monotone 2-CNF formulas in which every variable appears 3 times. We run each tool once on each benchmark with a timeout of 1000 seconds and record the wall-clock time taken.

Results on the runtime performance for these benchmarks are summarized in Figure \ref{fig:cubic-time}. For ease of presentation, we display only the best-performing of the \textbf{LG} and \textbf{FT} implementations: \textbf{LG+Flow}. We observe that tensor-based methods are fastest when $n \geq 110$. On large graphs ($n \geq 170$) our contribution \textbf{LG+Flow} is fastest and able to find the lowest max-rank contraction trees. \textbf{LG+Flow} is the only implementation able to solve at least 50 benchmarks within 1000 seconds when $n$ is $220$. We conclude that tensor-network-based approaches outperform state-of-the-art unweighted model counters on these benchmarks.

Results on the structural properties of these benchmarks are summarized in Figure \ref{fig:vertex-cover-width}. 
% Note that both \textbf{LG} and \textbf{FT} can be used to find carving decompositions on these benchmarks; although \textbf{FT} requires preprocessing to factor all tensors of order 4 or higher, each vertex in each cubic graph has exactly three incident edges and so there is no factoring required. 
For most large graphs $G$, we observe that the carving width of $G$ is smaller than the treewidth of $G$ which is smaller that the treewidth of $\Line{G}$. 
% Note that the carving width of $G$ is indeed smaller than the upper bound guaranteed by Theorem \ref{thm:carving-equiv-tree} of $width_t(\Line{G})+1$. In fact, the carving width of $G$ is smaller than the treewidth of $G$.

% width of the carving decompositions of $G$ found by \textbf{LG} are indeed smaller than the upper bound guaranteed by Theorem \ref{thm:carving-equiv-tree} of the tree decomposition width plus 1.

% In Figure \ref{fig:cubic-rank}, we present the median max rank of the contraction tree ultimately returned by each tensor-based method when counting each benchmark. We observe that \textbf{line-Flow} is consistently able to find better contraction trees than the other methods when $n \geq 180$. The flat performance of \textbf{line-Flow} when $n$ is between $100$ and $200$ is a result of the algorithm halting the online search for a contraction tree to immediately perform the contraction. % If \textbf{line-Flow} continues to improve a contraction tree for the full 5 minutes, we see (from the points labeled \textbf{line-Flow} [5 min]) that it consistently finds better contraction trees.

\begin{figure}[t]
	\centering
	%% Creator: Matplotlib, PGF backend
%%
%% To include the figure in your LaTeX document, write
%%   \input{<filename>.pgf}
%%
%% Make sure the required packages are loaded in your preamble
%%   \usepackage{pgf}
%%
%% and, on pdftex
%%   \usepackage[utf8]{inputenc}\DeclareUnicodeCharacter{2212}{-}
%%
%% or, on luatex and xetex
%%   \usepackage{unicode-math}
%%
%% Figures using additional raster images can only be included by \input if
%% they are in the same directory as the main LaTeX file. For loading figures
%% from other directories you can use the `import` package
%%   \usepackage{import}
%%
%% and then include the figures with
%%   \import{<path to file>}{<filename>.pgf}
%%
%% Matplotlib used the following preamble
%%   \usepackage[utf8x]{inputenc}
%%   \usepackage[T1]{fontenc}
%%
\begingroup%
\makeatletter%
\begin{pgfpicture}%
\pgfpathrectangle{\pgfpointorigin}{\pgfqpoint{6.400000in}{4.800000in}}%
\pgfusepath{use as bounding box, clip}%
\begin{pgfscope}%
\pgfsetbuttcap%
\pgfsetmiterjoin%
\definecolor{currentfill}{rgb}{1.000000,1.000000,1.000000}%
\pgfsetfillcolor{currentfill}%
\pgfsetlinewidth{0.000000pt}%
\definecolor{currentstroke}{rgb}{1.000000,1.000000,1.000000}%
\pgfsetstrokecolor{currentstroke}%
\pgfsetdash{}{0pt}%
\pgfpathmoveto{\pgfqpoint{0.000000in}{0.000000in}}%
\pgfpathlineto{\pgfqpoint{6.400000in}{0.000000in}}%
\pgfpathlineto{\pgfqpoint{6.400000in}{4.800000in}}%
\pgfpathlineto{\pgfqpoint{0.000000in}{4.800000in}}%
\pgfpathclose%
\pgfusepath{fill}%
\end{pgfscope}%
\begin{pgfscope}%
\pgfsetbuttcap%
\pgfsetmiterjoin%
\definecolor{currentfill}{rgb}{1.000000,1.000000,1.000000}%
\pgfsetfillcolor{currentfill}%
\pgfsetlinewidth{0.000000pt}%
\definecolor{currentstroke}{rgb}{0.000000,0.000000,0.000000}%
\pgfsetstrokecolor{currentstroke}%
\pgfsetstrokeopacity{0.000000}%
\pgfsetdash{}{0pt}%
\pgfpathmoveto{\pgfqpoint{0.725193in}{0.571603in}}%
\pgfpathlineto{\pgfqpoint{6.250000in}{0.571603in}}%
\pgfpathlineto{\pgfqpoint{6.250000in}{4.597238in}}%
\pgfpathlineto{\pgfqpoint{0.725193in}{4.597238in}}%
\pgfpathclose%
\pgfusepath{fill}%
\end{pgfscope}%
\begin{pgfscope}%
\pgfsetbuttcap%
\pgfsetroundjoin%
\definecolor{currentfill}{rgb}{0.000000,0.000000,0.000000}%
\pgfsetfillcolor{currentfill}%
\pgfsetlinewidth{0.803000pt}%
\definecolor{currentstroke}{rgb}{0.000000,0.000000,0.000000}%
\pgfsetstrokecolor{currentstroke}%
\pgfsetdash{}{0pt}%
\pgfsys@defobject{currentmarker}{\pgfqpoint{0.000000in}{-0.048611in}}{\pgfqpoint{0.000000in}{0.000000in}}{%
\pgfpathmoveto{\pgfqpoint{0.000000in}{0.000000in}}%
\pgfpathlineto{\pgfqpoint{0.000000in}{-0.048611in}}%
\pgfusepath{stroke,fill}%
}%
\begin{pgfscope}%
\pgfsys@transformshift{0.725193in}{0.571603in}%
\pgfsys@useobject{currentmarker}{}%
\end{pgfscope}%
\end{pgfscope}%
\begin{pgfscope}%
\definecolor{textcolor}{rgb}{0.000000,0.000000,0.000000}%
\pgfsetstrokecolor{textcolor}%
\pgfsetfillcolor{textcolor}%
\pgftext[x=0.725193in,y=0.474381in,,top]{\color{textcolor}\sffamily\fontsize{10.000000}{12.000000}\selectfont 0}%
\end{pgfscope}%
\begin{pgfscope}%
\pgfsetbuttcap%
\pgfsetroundjoin%
\definecolor{currentfill}{rgb}{0.000000,0.000000,0.000000}%
\pgfsetfillcolor{currentfill}%
\pgfsetlinewidth{0.803000pt}%
\definecolor{currentstroke}{rgb}{0.000000,0.000000,0.000000}%
\pgfsetstrokecolor{currentstroke}%
\pgfsetdash{}{0pt}%
\pgfsys@defobject{currentmarker}{\pgfqpoint{0.000000in}{-0.048611in}}{\pgfqpoint{0.000000in}{0.000000in}}{%
\pgfpathmoveto{\pgfqpoint{0.000000in}{0.000000in}}%
\pgfpathlineto{\pgfqpoint{0.000000in}{-0.048611in}}%
\pgfusepath{stroke,fill}%
}%
\begin{pgfscope}%
\pgfsys@transformshift{1.737990in}{0.571603in}%
\pgfsys@useobject{currentmarker}{}%
\end{pgfscope}%
\end{pgfscope}%
\begin{pgfscope}%
\definecolor{textcolor}{rgb}{0.000000,0.000000,0.000000}%
\pgfsetstrokecolor{textcolor}%
\pgfsetfillcolor{textcolor}%
\pgftext[x=1.737990in,y=0.474381in,,top]{\color{textcolor}\sffamily\fontsize{10.000000}{12.000000}\selectfont 200}%
\end{pgfscope}%
\begin{pgfscope}%
\pgfsetbuttcap%
\pgfsetroundjoin%
\definecolor{currentfill}{rgb}{0.000000,0.000000,0.000000}%
\pgfsetfillcolor{currentfill}%
\pgfsetlinewidth{0.803000pt}%
\definecolor{currentstroke}{rgb}{0.000000,0.000000,0.000000}%
\pgfsetstrokecolor{currentstroke}%
\pgfsetdash{}{0pt}%
\pgfsys@defobject{currentmarker}{\pgfqpoint{0.000000in}{-0.048611in}}{\pgfqpoint{0.000000in}{0.000000in}}{%
\pgfpathmoveto{\pgfqpoint{0.000000in}{0.000000in}}%
\pgfpathlineto{\pgfqpoint{0.000000in}{-0.048611in}}%
\pgfusepath{stroke,fill}%
}%
\begin{pgfscope}%
\pgfsys@transformshift{2.750787in}{0.571603in}%
\pgfsys@useobject{currentmarker}{}%
\end{pgfscope}%
\end{pgfscope}%
\begin{pgfscope}%
\definecolor{textcolor}{rgb}{0.000000,0.000000,0.000000}%
\pgfsetstrokecolor{textcolor}%
\pgfsetfillcolor{textcolor}%
\pgftext[x=2.750787in,y=0.474381in,,top]{\color{textcolor}\sffamily\fontsize{10.000000}{12.000000}\selectfont 400}%
\end{pgfscope}%
\begin{pgfscope}%
\pgfsetbuttcap%
\pgfsetroundjoin%
\definecolor{currentfill}{rgb}{0.000000,0.000000,0.000000}%
\pgfsetfillcolor{currentfill}%
\pgfsetlinewidth{0.803000pt}%
\definecolor{currentstroke}{rgb}{0.000000,0.000000,0.000000}%
\pgfsetstrokecolor{currentstroke}%
\pgfsetdash{}{0pt}%
\pgfsys@defobject{currentmarker}{\pgfqpoint{0.000000in}{-0.048611in}}{\pgfqpoint{0.000000in}{0.000000in}}{%
\pgfpathmoveto{\pgfqpoint{0.000000in}{0.000000in}}%
\pgfpathlineto{\pgfqpoint{0.000000in}{-0.048611in}}%
\pgfusepath{stroke,fill}%
}%
\begin{pgfscope}%
\pgfsys@transformshift{3.763584in}{0.571603in}%
\pgfsys@useobject{currentmarker}{}%
\end{pgfscope}%
\end{pgfscope}%
\begin{pgfscope}%
\definecolor{textcolor}{rgb}{0.000000,0.000000,0.000000}%
\pgfsetstrokecolor{textcolor}%
\pgfsetfillcolor{textcolor}%
\pgftext[x=3.763584in,y=0.474381in,,top]{\color{textcolor}\sffamily\fontsize{10.000000}{12.000000}\selectfont 600}%
\end{pgfscope}%
\begin{pgfscope}%
\pgfsetbuttcap%
\pgfsetroundjoin%
\definecolor{currentfill}{rgb}{0.000000,0.000000,0.000000}%
\pgfsetfillcolor{currentfill}%
\pgfsetlinewidth{0.803000pt}%
\definecolor{currentstroke}{rgb}{0.000000,0.000000,0.000000}%
\pgfsetstrokecolor{currentstroke}%
\pgfsetdash{}{0pt}%
\pgfsys@defobject{currentmarker}{\pgfqpoint{0.000000in}{-0.048611in}}{\pgfqpoint{0.000000in}{0.000000in}}{%
\pgfpathmoveto{\pgfqpoint{0.000000in}{0.000000in}}%
\pgfpathlineto{\pgfqpoint{0.000000in}{-0.048611in}}%
\pgfusepath{stroke,fill}%
}%
\begin{pgfscope}%
\pgfsys@transformshift{4.776381in}{0.571603in}%
\pgfsys@useobject{currentmarker}{}%
\end{pgfscope}%
\end{pgfscope}%
\begin{pgfscope}%
\definecolor{textcolor}{rgb}{0.000000,0.000000,0.000000}%
\pgfsetstrokecolor{textcolor}%
\pgfsetfillcolor{textcolor}%
\pgftext[x=4.776381in,y=0.474381in,,top]{\color{textcolor}\sffamily\fontsize{10.000000}{12.000000}\selectfont 800}%
\end{pgfscope}%
\begin{pgfscope}%
\pgfsetbuttcap%
\pgfsetroundjoin%
\definecolor{currentfill}{rgb}{0.000000,0.000000,0.000000}%
\pgfsetfillcolor{currentfill}%
\pgfsetlinewidth{0.803000pt}%
\definecolor{currentstroke}{rgb}{0.000000,0.000000,0.000000}%
\pgfsetstrokecolor{currentstroke}%
\pgfsetdash{}{0pt}%
\pgfsys@defobject{currentmarker}{\pgfqpoint{0.000000in}{-0.048611in}}{\pgfqpoint{0.000000in}{0.000000in}}{%
\pgfpathmoveto{\pgfqpoint{0.000000in}{0.000000in}}%
\pgfpathlineto{\pgfqpoint{0.000000in}{-0.048611in}}%
\pgfusepath{stroke,fill}%
}%
\begin{pgfscope}%
\pgfsys@transformshift{5.789177in}{0.571603in}%
\pgfsys@useobject{currentmarker}{}%
\end{pgfscope}%
\end{pgfscope}%
\begin{pgfscope}%
\definecolor{textcolor}{rgb}{0.000000,0.000000,0.000000}%
\pgfsetstrokecolor{textcolor}%
\pgfsetfillcolor{textcolor}%
\pgftext[x=5.789177in,y=0.474381in,,top]{\color{textcolor}\sffamily\fontsize{10.000000}{12.000000}\selectfont 1000}%
\end{pgfscope}%
\begin{pgfscope}%
\definecolor{textcolor}{rgb}{0.000000,0.000000,0.000000}%
\pgfsetstrokecolor{textcolor}%
\pgfsetfillcolor{textcolor}%
\pgftext[x=3.487597in,y=0.291542in,,top]{\color{textcolor}\sffamily\fontsize{10.000000}{12.000000}\selectfont Number of benchmarks solved}%
\end{pgfscope}%
\begin{pgfscope}%
\pgfsetbuttcap%
\pgfsetroundjoin%
\definecolor{currentfill}{rgb}{0.000000,0.000000,0.000000}%
\pgfsetfillcolor{currentfill}%
\pgfsetlinewidth{0.803000pt}%
\definecolor{currentstroke}{rgb}{0.000000,0.000000,0.000000}%
\pgfsetstrokecolor{currentstroke}%
\pgfsetdash{}{0pt}%
\pgfsys@defobject{currentmarker}{\pgfqpoint{-0.048611in}{0.000000in}}{\pgfqpoint{-0.000000in}{0.000000in}}{%
\pgfpathmoveto{\pgfqpoint{-0.000000in}{0.000000in}}%
\pgfpathlineto{\pgfqpoint{-0.048611in}{0.000000in}}%
\pgfusepath{stroke,fill}%
}%
\begin{pgfscope}%
\pgfsys@transformshift{0.725193in}{0.800207in}%
\pgfsys@useobject{currentmarker}{}%
\end{pgfscope}%
\end{pgfscope}%
\begin{pgfscope}%
\definecolor{textcolor}{rgb}{0.000000,0.000000,0.000000}%
\pgfsetstrokecolor{textcolor}%
\pgfsetfillcolor{textcolor}%
\pgftext[x=0.339968in, y=0.750065in, left, base]{\color{textcolor}\sffamily\fontsize{10.000000}{12.000000}\selectfont \(\displaystyle {10^{-2}}\)}%
\end{pgfscope}%
\begin{pgfscope}%
\pgfsetbuttcap%
\pgfsetroundjoin%
\definecolor{currentfill}{rgb}{0.000000,0.000000,0.000000}%
\pgfsetfillcolor{currentfill}%
\pgfsetlinewidth{0.803000pt}%
\definecolor{currentstroke}{rgb}{0.000000,0.000000,0.000000}%
\pgfsetstrokecolor{currentstroke}%
\pgfsetdash{}{0pt}%
\pgfsys@defobject{currentmarker}{\pgfqpoint{-0.048611in}{0.000000in}}{\pgfqpoint{-0.000000in}{0.000000in}}{%
\pgfpathmoveto{\pgfqpoint{-0.000000in}{0.000000in}}%
\pgfpathlineto{\pgfqpoint{-0.048611in}{0.000000in}}%
\pgfusepath{stroke,fill}%
}%
\begin{pgfscope}%
\pgfsys@transformshift{0.725193in}{1.559614in}%
\pgfsys@useobject{currentmarker}{}%
\end{pgfscope}%
\end{pgfscope}%
\begin{pgfscope}%
\definecolor{textcolor}{rgb}{0.000000,0.000000,0.000000}%
\pgfsetstrokecolor{textcolor}%
\pgfsetfillcolor{textcolor}%
\pgftext[x=0.339968in, y=1.509472in, left, base]{\color{textcolor}\sffamily\fontsize{10.000000}{12.000000}\selectfont \(\displaystyle {10^{-1}}\)}%
\end{pgfscope}%
\begin{pgfscope}%
\pgfsetbuttcap%
\pgfsetroundjoin%
\definecolor{currentfill}{rgb}{0.000000,0.000000,0.000000}%
\pgfsetfillcolor{currentfill}%
\pgfsetlinewidth{0.803000pt}%
\definecolor{currentstroke}{rgb}{0.000000,0.000000,0.000000}%
\pgfsetstrokecolor{currentstroke}%
\pgfsetdash{}{0pt}%
\pgfsys@defobject{currentmarker}{\pgfqpoint{-0.048611in}{0.000000in}}{\pgfqpoint{-0.000000in}{0.000000in}}{%
\pgfpathmoveto{\pgfqpoint{-0.000000in}{0.000000in}}%
\pgfpathlineto{\pgfqpoint{-0.048611in}{0.000000in}}%
\pgfusepath{stroke,fill}%
}%
\begin{pgfscope}%
\pgfsys@transformshift{0.725193in}{2.319020in}%
\pgfsys@useobject{currentmarker}{}%
\end{pgfscope}%
\end{pgfscope}%
\begin{pgfscope}%
\definecolor{textcolor}{rgb}{0.000000,0.000000,0.000000}%
\pgfsetstrokecolor{textcolor}%
\pgfsetfillcolor{textcolor}%
\pgftext[x=0.426774in, y=2.268878in, left, base]{\color{textcolor}\sffamily\fontsize{10.000000}{12.000000}\selectfont \(\displaystyle {10^{0}}\)}%
\end{pgfscope}%
\begin{pgfscope}%
\pgfsetbuttcap%
\pgfsetroundjoin%
\definecolor{currentfill}{rgb}{0.000000,0.000000,0.000000}%
\pgfsetfillcolor{currentfill}%
\pgfsetlinewidth{0.803000pt}%
\definecolor{currentstroke}{rgb}{0.000000,0.000000,0.000000}%
\pgfsetstrokecolor{currentstroke}%
\pgfsetdash{}{0pt}%
\pgfsys@defobject{currentmarker}{\pgfqpoint{-0.048611in}{0.000000in}}{\pgfqpoint{-0.000000in}{0.000000in}}{%
\pgfpathmoveto{\pgfqpoint{-0.000000in}{0.000000in}}%
\pgfpathlineto{\pgfqpoint{-0.048611in}{0.000000in}}%
\pgfusepath{stroke,fill}%
}%
\begin{pgfscope}%
\pgfsys@transformshift{0.725193in}{3.078426in}%
\pgfsys@useobject{currentmarker}{}%
\end{pgfscope}%
\end{pgfscope}%
\begin{pgfscope}%
\definecolor{textcolor}{rgb}{0.000000,0.000000,0.000000}%
\pgfsetstrokecolor{textcolor}%
\pgfsetfillcolor{textcolor}%
\pgftext[x=0.426774in, y=3.028284in, left, base]{\color{textcolor}\sffamily\fontsize{10.000000}{12.000000}\selectfont \(\displaystyle {10^{1}}\)}%
\end{pgfscope}%
\begin{pgfscope}%
\pgfsetbuttcap%
\pgfsetroundjoin%
\definecolor{currentfill}{rgb}{0.000000,0.000000,0.000000}%
\pgfsetfillcolor{currentfill}%
\pgfsetlinewidth{0.803000pt}%
\definecolor{currentstroke}{rgb}{0.000000,0.000000,0.000000}%
\pgfsetstrokecolor{currentstroke}%
\pgfsetdash{}{0pt}%
\pgfsys@defobject{currentmarker}{\pgfqpoint{-0.048611in}{0.000000in}}{\pgfqpoint{-0.000000in}{0.000000in}}{%
\pgfpathmoveto{\pgfqpoint{-0.000000in}{0.000000in}}%
\pgfpathlineto{\pgfqpoint{-0.048611in}{0.000000in}}%
\pgfusepath{stroke,fill}%
}%
\begin{pgfscope}%
\pgfsys@transformshift{0.725193in}{3.837832in}%
\pgfsys@useobject{currentmarker}{}%
\end{pgfscope}%
\end{pgfscope}%
\begin{pgfscope}%
\definecolor{textcolor}{rgb}{0.000000,0.000000,0.000000}%
\pgfsetstrokecolor{textcolor}%
\pgfsetfillcolor{textcolor}%
\pgftext[x=0.426774in, y=3.787690in, left, base]{\color{textcolor}\sffamily\fontsize{10.000000}{12.000000}\selectfont \(\displaystyle {10^{2}}\)}%
\end{pgfscope}%
\begin{pgfscope}%
\pgfsetbuttcap%
\pgfsetroundjoin%
\definecolor{currentfill}{rgb}{0.000000,0.000000,0.000000}%
\pgfsetfillcolor{currentfill}%
\pgfsetlinewidth{0.803000pt}%
\definecolor{currentstroke}{rgb}{0.000000,0.000000,0.000000}%
\pgfsetstrokecolor{currentstroke}%
\pgfsetdash{}{0pt}%
\pgfsys@defobject{currentmarker}{\pgfqpoint{-0.048611in}{0.000000in}}{\pgfqpoint{-0.000000in}{0.000000in}}{%
\pgfpathmoveto{\pgfqpoint{-0.000000in}{0.000000in}}%
\pgfpathlineto{\pgfqpoint{-0.048611in}{0.000000in}}%
\pgfusepath{stroke,fill}%
}%
\begin{pgfscope}%
\pgfsys@transformshift{0.725193in}{4.597238in}%
\pgfsys@useobject{currentmarker}{}%
\end{pgfscope}%
\end{pgfscope}%
\begin{pgfscope}%
\definecolor{textcolor}{rgb}{0.000000,0.000000,0.000000}%
\pgfsetstrokecolor{textcolor}%
\pgfsetfillcolor{textcolor}%
\pgftext[x=0.426774in, y=4.547096in, left, base]{\color{textcolor}\sffamily\fontsize{10.000000}{12.000000}\selectfont \(\displaystyle {10^{3}}\)}%
\end{pgfscope}%
\begin{pgfscope}%
\pgfsetbuttcap%
\pgfsetroundjoin%
\definecolor{currentfill}{rgb}{0.000000,0.000000,0.000000}%
\pgfsetfillcolor{currentfill}%
\pgfsetlinewidth{0.602250pt}%
\definecolor{currentstroke}{rgb}{0.000000,0.000000,0.000000}%
\pgfsetstrokecolor{currentstroke}%
\pgfsetdash{}{0pt}%
\pgfsys@defobject{currentmarker}{\pgfqpoint{-0.027778in}{0.000000in}}{\pgfqpoint{-0.000000in}{0.000000in}}{%
\pgfpathmoveto{\pgfqpoint{-0.000000in}{0.000000in}}%
\pgfpathlineto{\pgfqpoint{-0.027778in}{0.000000in}}%
\pgfusepath{stroke,fill}%
}%
\begin{pgfscope}%
\pgfsys@transformshift{0.725193in}{0.571603in}%
\pgfsys@useobject{currentmarker}{}%
\end{pgfscope}%
\end{pgfscope}%
\begin{pgfscope}%
\pgfsetbuttcap%
\pgfsetroundjoin%
\definecolor{currentfill}{rgb}{0.000000,0.000000,0.000000}%
\pgfsetfillcolor{currentfill}%
\pgfsetlinewidth{0.602250pt}%
\definecolor{currentstroke}{rgb}{0.000000,0.000000,0.000000}%
\pgfsetstrokecolor{currentstroke}%
\pgfsetdash{}{0pt}%
\pgfsys@defobject{currentmarker}{\pgfqpoint{-0.027778in}{0.000000in}}{\pgfqpoint{-0.000000in}{0.000000in}}{%
\pgfpathmoveto{\pgfqpoint{-0.000000in}{0.000000in}}%
\pgfpathlineto{\pgfqpoint{-0.027778in}{0.000000in}}%
\pgfusepath{stroke,fill}%
}%
\begin{pgfscope}%
\pgfsys@transformshift{0.725193in}{0.631734in}%
\pgfsys@useobject{currentmarker}{}%
\end{pgfscope}%
\end{pgfscope}%
\begin{pgfscope}%
\pgfsetbuttcap%
\pgfsetroundjoin%
\definecolor{currentfill}{rgb}{0.000000,0.000000,0.000000}%
\pgfsetfillcolor{currentfill}%
\pgfsetlinewidth{0.602250pt}%
\definecolor{currentstroke}{rgb}{0.000000,0.000000,0.000000}%
\pgfsetstrokecolor{currentstroke}%
\pgfsetdash{}{0pt}%
\pgfsys@defobject{currentmarker}{\pgfqpoint{-0.027778in}{0.000000in}}{\pgfqpoint{-0.000000in}{0.000000in}}{%
\pgfpathmoveto{\pgfqpoint{-0.000000in}{0.000000in}}%
\pgfpathlineto{\pgfqpoint{-0.027778in}{0.000000in}}%
\pgfusepath{stroke,fill}%
}%
\begin{pgfscope}%
\pgfsys@transformshift{0.725193in}{0.682574in}%
\pgfsys@useobject{currentmarker}{}%
\end{pgfscope}%
\end{pgfscope}%
\begin{pgfscope}%
\pgfsetbuttcap%
\pgfsetroundjoin%
\definecolor{currentfill}{rgb}{0.000000,0.000000,0.000000}%
\pgfsetfillcolor{currentfill}%
\pgfsetlinewidth{0.602250pt}%
\definecolor{currentstroke}{rgb}{0.000000,0.000000,0.000000}%
\pgfsetstrokecolor{currentstroke}%
\pgfsetdash{}{0pt}%
\pgfsys@defobject{currentmarker}{\pgfqpoint{-0.027778in}{0.000000in}}{\pgfqpoint{-0.000000in}{0.000000in}}{%
\pgfpathmoveto{\pgfqpoint{-0.000000in}{0.000000in}}%
\pgfpathlineto{\pgfqpoint{-0.027778in}{0.000000in}}%
\pgfusepath{stroke,fill}%
}%
\begin{pgfscope}%
\pgfsys@transformshift{0.725193in}{0.726613in}%
\pgfsys@useobject{currentmarker}{}%
\end{pgfscope}%
\end{pgfscope}%
\begin{pgfscope}%
\pgfsetbuttcap%
\pgfsetroundjoin%
\definecolor{currentfill}{rgb}{0.000000,0.000000,0.000000}%
\pgfsetfillcolor{currentfill}%
\pgfsetlinewidth{0.602250pt}%
\definecolor{currentstroke}{rgb}{0.000000,0.000000,0.000000}%
\pgfsetstrokecolor{currentstroke}%
\pgfsetdash{}{0pt}%
\pgfsys@defobject{currentmarker}{\pgfqpoint{-0.027778in}{0.000000in}}{\pgfqpoint{-0.000000in}{0.000000in}}{%
\pgfpathmoveto{\pgfqpoint{-0.000000in}{0.000000in}}%
\pgfpathlineto{\pgfqpoint{-0.027778in}{0.000000in}}%
\pgfusepath{stroke,fill}%
}%
\begin{pgfscope}%
\pgfsys@transformshift{0.725193in}{0.765459in}%
\pgfsys@useobject{currentmarker}{}%
\end{pgfscope}%
\end{pgfscope}%
\begin{pgfscope}%
\pgfsetbuttcap%
\pgfsetroundjoin%
\definecolor{currentfill}{rgb}{0.000000,0.000000,0.000000}%
\pgfsetfillcolor{currentfill}%
\pgfsetlinewidth{0.602250pt}%
\definecolor{currentstroke}{rgb}{0.000000,0.000000,0.000000}%
\pgfsetstrokecolor{currentstroke}%
\pgfsetdash{}{0pt}%
\pgfsys@defobject{currentmarker}{\pgfqpoint{-0.027778in}{0.000000in}}{\pgfqpoint{-0.000000in}{0.000000in}}{%
\pgfpathmoveto{\pgfqpoint{-0.000000in}{0.000000in}}%
\pgfpathlineto{\pgfqpoint{-0.027778in}{0.000000in}}%
\pgfusepath{stroke,fill}%
}%
\begin{pgfscope}%
\pgfsys@transformshift{0.725193in}{1.028811in}%
\pgfsys@useobject{currentmarker}{}%
\end{pgfscope}%
\end{pgfscope}%
\begin{pgfscope}%
\pgfsetbuttcap%
\pgfsetroundjoin%
\definecolor{currentfill}{rgb}{0.000000,0.000000,0.000000}%
\pgfsetfillcolor{currentfill}%
\pgfsetlinewidth{0.602250pt}%
\definecolor{currentstroke}{rgb}{0.000000,0.000000,0.000000}%
\pgfsetstrokecolor{currentstroke}%
\pgfsetdash{}{0pt}%
\pgfsys@defobject{currentmarker}{\pgfqpoint{-0.027778in}{0.000000in}}{\pgfqpoint{-0.000000in}{0.000000in}}{%
\pgfpathmoveto{\pgfqpoint{-0.000000in}{0.000000in}}%
\pgfpathlineto{\pgfqpoint{-0.027778in}{0.000000in}}%
\pgfusepath{stroke,fill}%
}%
\begin{pgfscope}%
\pgfsys@transformshift{0.725193in}{1.162536in}%
\pgfsys@useobject{currentmarker}{}%
\end{pgfscope}%
\end{pgfscope}%
\begin{pgfscope}%
\pgfsetbuttcap%
\pgfsetroundjoin%
\definecolor{currentfill}{rgb}{0.000000,0.000000,0.000000}%
\pgfsetfillcolor{currentfill}%
\pgfsetlinewidth{0.602250pt}%
\definecolor{currentstroke}{rgb}{0.000000,0.000000,0.000000}%
\pgfsetstrokecolor{currentstroke}%
\pgfsetdash{}{0pt}%
\pgfsys@defobject{currentmarker}{\pgfqpoint{-0.027778in}{0.000000in}}{\pgfqpoint{-0.000000in}{0.000000in}}{%
\pgfpathmoveto{\pgfqpoint{-0.000000in}{0.000000in}}%
\pgfpathlineto{\pgfqpoint{-0.027778in}{0.000000in}}%
\pgfusepath{stroke,fill}%
}%
\begin{pgfscope}%
\pgfsys@transformshift{0.725193in}{1.257415in}%
\pgfsys@useobject{currentmarker}{}%
\end{pgfscope}%
\end{pgfscope}%
\begin{pgfscope}%
\pgfsetbuttcap%
\pgfsetroundjoin%
\definecolor{currentfill}{rgb}{0.000000,0.000000,0.000000}%
\pgfsetfillcolor{currentfill}%
\pgfsetlinewidth{0.602250pt}%
\definecolor{currentstroke}{rgb}{0.000000,0.000000,0.000000}%
\pgfsetstrokecolor{currentstroke}%
\pgfsetdash{}{0pt}%
\pgfsys@defobject{currentmarker}{\pgfqpoint{-0.027778in}{0.000000in}}{\pgfqpoint{-0.000000in}{0.000000in}}{%
\pgfpathmoveto{\pgfqpoint{-0.000000in}{0.000000in}}%
\pgfpathlineto{\pgfqpoint{-0.027778in}{0.000000in}}%
\pgfusepath{stroke,fill}%
}%
\begin{pgfscope}%
\pgfsys@transformshift{0.725193in}{1.331010in}%
\pgfsys@useobject{currentmarker}{}%
\end{pgfscope}%
\end{pgfscope}%
\begin{pgfscope}%
\pgfsetbuttcap%
\pgfsetroundjoin%
\definecolor{currentfill}{rgb}{0.000000,0.000000,0.000000}%
\pgfsetfillcolor{currentfill}%
\pgfsetlinewidth{0.602250pt}%
\definecolor{currentstroke}{rgb}{0.000000,0.000000,0.000000}%
\pgfsetstrokecolor{currentstroke}%
\pgfsetdash{}{0pt}%
\pgfsys@defobject{currentmarker}{\pgfqpoint{-0.027778in}{0.000000in}}{\pgfqpoint{-0.000000in}{0.000000in}}{%
\pgfpathmoveto{\pgfqpoint{-0.000000in}{0.000000in}}%
\pgfpathlineto{\pgfqpoint{-0.027778in}{0.000000in}}%
\pgfusepath{stroke,fill}%
}%
\begin{pgfscope}%
\pgfsys@transformshift{0.725193in}{1.391140in}%
\pgfsys@useobject{currentmarker}{}%
\end{pgfscope}%
\end{pgfscope}%
\begin{pgfscope}%
\pgfsetbuttcap%
\pgfsetroundjoin%
\definecolor{currentfill}{rgb}{0.000000,0.000000,0.000000}%
\pgfsetfillcolor{currentfill}%
\pgfsetlinewidth{0.602250pt}%
\definecolor{currentstroke}{rgb}{0.000000,0.000000,0.000000}%
\pgfsetstrokecolor{currentstroke}%
\pgfsetdash{}{0pt}%
\pgfsys@defobject{currentmarker}{\pgfqpoint{-0.027778in}{0.000000in}}{\pgfqpoint{-0.000000in}{0.000000in}}{%
\pgfpathmoveto{\pgfqpoint{-0.000000in}{0.000000in}}%
\pgfpathlineto{\pgfqpoint{-0.027778in}{0.000000in}}%
\pgfusepath{stroke,fill}%
}%
\begin{pgfscope}%
\pgfsys@transformshift{0.725193in}{1.441980in}%
\pgfsys@useobject{currentmarker}{}%
\end{pgfscope}%
\end{pgfscope}%
\begin{pgfscope}%
\pgfsetbuttcap%
\pgfsetroundjoin%
\definecolor{currentfill}{rgb}{0.000000,0.000000,0.000000}%
\pgfsetfillcolor{currentfill}%
\pgfsetlinewidth{0.602250pt}%
\definecolor{currentstroke}{rgb}{0.000000,0.000000,0.000000}%
\pgfsetstrokecolor{currentstroke}%
\pgfsetdash{}{0pt}%
\pgfsys@defobject{currentmarker}{\pgfqpoint{-0.027778in}{0.000000in}}{\pgfqpoint{-0.000000in}{0.000000in}}{%
\pgfpathmoveto{\pgfqpoint{-0.000000in}{0.000000in}}%
\pgfpathlineto{\pgfqpoint{-0.027778in}{0.000000in}}%
\pgfusepath{stroke,fill}%
}%
\begin{pgfscope}%
\pgfsys@transformshift{0.725193in}{1.486020in}%
\pgfsys@useobject{currentmarker}{}%
\end{pgfscope}%
\end{pgfscope}%
\begin{pgfscope}%
\pgfsetbuttcap%
\pgfsetroundjoin%
\definecolor{currentfill}{rgb}{0.000000,0.000000,0.000000}%
\pgfsetfillcolor{currentfill}%
\pgfsetlinewidth{0.602250pt}%
\definecolor{currentstroke}{rgb}{0.000000,0.000000,0.000000}%
\pgfsetstrokecolor{currentstroke}%
\pgfsetdash{}{0pt}%
\pgfsys@defobject{currentmarker}{\pgfqpoint{-0.027778in}{0.000000in}}{\pgfqpoint{-0.000000in}{0.000000in}}{%
\pgfpathmoveto{\pgfqpoint{-0.000000in}{0.000000in}}%
\pgfpathlineto{\pgfqpoint{-0.027778in}{0.000000in}}%
\pgfusepath{stroke,fill}%
}%
\begin{pgfscope}%
\pgfsys@transformshift{0.725193in}{1.524865in}%
\pgfsys@useobject{currentmarker}{}%
\end{pgfscope}%
\end{pgfscope}%
\begin{pgfscope}%
\pgfsetbuttcap%
\pgfsetroundjoin%
\definecolor{currentfill}{rgb}{0.000000,0.000000,0.000000}%
\pgfsetfillcolor{currentfill}%
\pgfsetlinewidth{0.602250pt}%
\definecolor{currentstroke}{rgb}{0.000000,0.000000,0.000000}%
\pgfsetstrokecolor{currentstroke}%
\pgfsetdash{}{0pt}%
\pgfsys@defobject{currentmarker}{\pgfqpoint{-0.027778in}{0.000000in}}{\pgfqpoint{-0.000000in}{0.000000in}}{%
\pgfpathmoveto{\pgfqpoint{-0.000000in}{0.000000in}}%
\pgfpathlineto{\pgfqpoint{-0.027778in}{0.000000in}}%
\pgfusepath{stroke,fill}%
}%
\begin{pgfscope}%
\pgfsys@transformshift{0.725193in}{1.788218in}%
\pgfsys@useobject{currentmarker}{}%
\end{pgfscope}%
\end{pgfscope}%
\begin{pgfscope}%
\pgfsetbuttcap%
\pgfsetroundjoin%
\definecolor{currentfill}{rgb}{0.000000,0.000000,0.000000}%
\pgfsetfillcolor{currentfill}%
\pgfsetlinewidth{0.602250pt}%
\definecolor{currentstroke}{rgb}{0.000000,0.000000,0.000000}%
\pgfsetstrokecolor{currentstroke}%
\pgfsetdash{}{0pt}%
\pgfsys@defobject{currentmarker}{\pgfqpoint{-0.027778in}{0.000000in}}{\pgfqpoint{-0.000000in}{0.000000in}}{%
\pgfpathmoveto{\pgfqpoint{-0.000000in}{0.000000in}}%
\pgfpathlineto{\pgfqpoint{-0.027778in}{0.000000in}}%
\pgfusepath{stroke,fill}%
}%
\begin{pgfscope}%
\pgfsys@transformshift{0.725193in}{1.921942in}%
\pgfsys@useobject{currentmarker}{}%
\end{pgfscope}%
\end{pgfscope}%
\begin{pgfscope}%
\pgfsetbuttcap%
\pgfsetroundjoin%
\definecolor{currentfill}{rgb}{0.000000,0.000000,0.000000}%
\pgfsetfillcolor{currentfill}%
\pgfsetlinewidth{0.602250pt}%
\definecolor{currentstroke}{rgb}{0.000000,0.000000,0.000000}%
\pgfsetstrokecolor{currentstroke}%
\pgfsetdash{}{0pt}%
\pgfsys@defobject{currentmarker}{\pgfqpoint{-0.027778in}{0.000000in}}{\pgfqpoint{-0.000000in}{0.000000in}}{%
\pgfpathmoveto{\pgfqpoint{-0.000000in}{0.000000in}}%
\pgfpathlineto{\pgfqpoint{-0.027778in}{0.000000in}}%
\pgfusepath{stroke,fill}%
}%
\begin{pgfscope}%
\pgfsys@transformshift{0.725193in}{2.016822in}%
\pgfsys@useobject{currentmarker}{}%
\end{pgfscope}%
\end{pgfscope}%
\begin{pgfscope}%
\pgfsetbuttcap%
\pgfsetroundjoin%
\definecolor{currentfill}{rgb}{0.000000,0.000000,0.000000}%
\pgfsetfillcolor{currentfill}%
\pgfsetlinewidth{0.602250pt}%
\definecolor{currentstroke}{rgb}{0.000000,0.000000,0.000000}%
\pgfsetstrokecolor{currentstroke}%
\pgfsetdash{}{0pt}%
\pgfsys@defobject{currentmarker}{\pgfqpoint{-0.027778in}{0.000000in}}{\pgfqpoint{-0.000000in}{0.000000in}}{%
\pgfpathmoveto{\pgfqpoint{-0.000000in}{0.000000in}}%
\pgfpathlineto{\pgfqpoint{-0.027778in}{0.000000in}}%
\pgfusepath{stroke,fill}%
}%
\begin{pgfscope}%
\pgfsys@transformshift{0.725193in}{2.090416in}%
\pgfsys@useobject{currentmarker}{}%
\end{pgfscope}%
\end{pgfscope}%
\begin{pgfscope}%
\pgfsetbuttcap%
\pgfsetroundjoin%
\definecolor{currentfill}{rgb}{0.000000,0.000000,0.000000}%
\pgfsetfillcolor{currentfill}%
\pgfsetlinewidth{0.602250pt}%
\definecolor{currentstroke}{rgb}{0.000000,0.000000,0.000000}%
\pgfsetstrokecolor{currentstroke}%
\pgfsetdash{}{0pt}%
\pgfsys@defobject{currentmarker}{\pgfqpoint{-0.027778in}{0.000000in}}{\pgfqpoint{-0.000000in}{0.000000in}}{%
\pgfpathmoveto{\pgfqpoint{-0.000000in}{0.000000in}}%
\pgfpathlineto{\pgfqpoint{-0.027778in}{0.000000in}}%
\pgfusepath{stroke,fill}%
}%
\begin{pgfscope}%
\pgfsys@transformshift{0.725193in}{2.150546in}%
\pgfsys@useobject{currentmarker}{}%
\end{pgfscope}%
\end{pgfscope}%
\begin{pgfscope}%
\pgfsetbuttcap%
\pgfsetroundjoin%
\definecolor{currentfill}{rgb}{0.000000,0.000000,0.000000}%
\pgfsetfillcolor{currentfill}%
\pgfsetlinewidth{0.602250pt}%
\definecolor{currentstroke}{rgb}{0.000000,0.000000,0.000000}%
\pgfsetstrokecolor{currentstroke}%
\pgfsetdash{}{0pt}%
\pgfsys@defobject{currentmarker}{\pgfqpoint{-0.027778in}{0.000000in}}{\pgfqpoint{-0.000000in}{0.000000in}}{%
\pgfpathmoveto{\pgfqpoint{-0.000000in}{0.000000in}}%
\pgfpathlineto{\pgfqpoint{-0.027778in}{0.000000in}}%
\pgfusepath{stroke,fill}%
}%
\begin{pgfscope}%
\pgfsys@transformshift{0.725193in}{2.201386in}%
\pgfsys@useobject{currentmarker}{}%
\end{pgfscope}%
\end{pgfscope}%
\begin{pgfscope}%
\pgfsetbuttcap%
\pgfsetroundjoin%
\definecolor{currentfill}{rgb}{0.000000,0.000000,0.000000}%
\pgfsetfillcolor{currentfill}%
\pgfsetlinewidth{0.602250pt}%
\definecolor{currentstroke}{rgb}{0.000000,0.000000,0.000000}%
\pgfsetstrokecolor{currentstroke}%
\pgfsetdash{}{0pt}%
\pgfsys@defobject{currentmarker}{\pgfqpoint{-0.027778in}{0.000000in}}{\pgfqpoint{-0.000000in}{0.000000in}}{%
\pgfpathmoveto{\pgfqpoint{-0.000000in}{0.000000in}}%
\pgfpathlineto{\pgfqpoint{-0.027778in}{0.000000in}}%
\pgfusepath{stroke,fill}%
}%
\begin{pgfscope}%
\pgfsys@transformshift{0.725193in}{2.245426in}%
\pgfsys@useobject{currentmarker}{}%
\end{pgfscope}%
\end{pgfscope}%
\begin{pgfscope}%
\pgfsetbuttcap%
\pgfsetroundjoin%
\definecolor{currentfill}{rgb}{0.000000,0.000000,0.000000}%
\pgfsetfillcolor{currentfill}%
\pgfsetlinewidth{0.602250pt}%
\definecolor{currentstroke}{rgb}{0.000000,0.000000,0.000000}%
\pgfsetstrokecolor{currentstroke}%
\pgfsetdash{}{0pt}%
\pgfsys@defobject{currentmarker}{\pgfqpoint{-0.027778in}{0.000000in}}{\pgfqpoint{-0.000000in}{0.000000in}}{%
\pgfpathmoveto{\pgfqpoint{-0.000000in}{0.000000in}}%
\pgfpathlineto{\pgfqpoint{-0.027778in}{0.000000in}}%
\pgfusepath{stroke,fill}%
}%
\begin{pgfscope}%
\pgfsys@transformshift{0.725193in}{2.284271in}%
\pgfsys@useobject{currentmarker}{}%
\end{pgfscope}%
\end{pgfscope}%
\begin{pgfscope}%
\pgfsetbuttcap%
\pgfsetroundjoin%
\definecolor{currentfill}{rgb}{0.000000,0.000000,0.000000}%
\pgfsetfillcolor{currentfill}%
\pgfsetlinewidth{0.602250pt}%
\definecolor{currentstroke}{rgb}{0.000000,0.000000,0.000000}%
\pgfsetstrokecolor{currentstroke}%
\pgfsetdash{}{0pt}%
\pgfsys@defobject{currentmarker}{\pgfqpoint{-0.027778in}{0.000000in}}{\pgfqpoint{-0.000000in}{0.000000in}}{%
\pgfpathmoveto{\pgfqpoint{-0.000000in}{0.000000in}}%
\pgfpathlineto{\pgfqpoint{-0.027778in}{0.000000in}}%
\pgfusepath{stroke,fill}%
}%
\begin{pgfscope}%
\pgfsys@transformshift{0.725193in}{2.547624in}%
\pgfsys@useobject{currentmarker}{}%
\end{pgfscope}%
\end{pgfscope}%
\begin{pgfscope}%
\pgfsetbuttcap%
\pgfsetroundjoin%
\definecolor{currentfill}{rgb}{0.000000,0.000000,0.000000}%
\pgfsetfillcolor{currentfill}%
\pgfsetlinewidth{0.602250pt}%
\definecolor{currentstroke}{rgb}{0.000000,0.000000,0.000000}%
\pgfsetstrokecolor{currentstroke}%
\pgfsetdash{}{0pt}%
\pgfsys@defobject{currentmarker}{\pgfqpoint{-0.027778in}{0.000000in}}{\pgfqpoint{-0.000000in}{0.000000in}}{%
\pgfpathmoveto{\pgfqpoint{-0.000000in}{0.000000in}}%
\pgfpathlineto{\pgfqpoint{-0.027778in}{0.000000in}}%
\pgfusepath{stroke,fill}%
}%
\begin{pgfscope}%
\pgfsys@transformshift{0.725193in}{2.681349in}%
\pgfsys@useobject{currentmarker}{}%
\end{pgfscope}%
\end{pgfscope}%
\begin{pgfscope}%
\pgfsetbuttcap%
\pgfsetroundjoin%
\definecolor{currentfill}{rgb}{0.000000,0.000000,0.000000}%
\pgfsetfillcolor{currentfill}%
\pgfsetlinewidth{0.602250pt}%
\definecolor{currentstroke}{rgb}{0.000000,0.000000,0.000000}%
\pgfsetstrokecolor{currentstroke}%
\pgfsetdash{}{0pt}%
\pgfsys@defobject{currentmarker}{\pgfqpoint{-0.027778in}{0.000000in}}{\pgfqpoint{-0.000000in}{0.000000in}}{%
\pgfpathmoveto{\pgfqpoint{-0.000000in}{0.000000in}}%
\pgfpathlineto{\pgfqpoint{-0.027778in}{0.000000in}}%
\pgfusepath{stroke,fill}%
}%
\begin{pgfscope}%
\pgfsys@transformshift{0.725193in}{2.776228in}%
\pgfsys@useobject{currentmarker}{}%
\end{pgfscope}%
\end{pgfscope}%
\begin{pgfscope}%
\pgfsetbuttcap%
\pgfsetroundjoin%
\definecolor{currentfill}{rgb}{0.000000,0.000000,0.000000}%
\pgfsetfillcolor{currentfill}%
\pgfsetlinewidth{0.602250pt}%
\definecolor{currentstroke}{rgb}{0.000000,0.000000,0.000000}%
\pgfsetstrokecolor{currentstroke}%
\pgfsetdash{}{0pt}%
\pgfsys@defobject{currentmarker}{\pgfqpoint{-0.027778in}{0.000000in}}{\pgfqpoint{-0.000000in}{0.000000in}}{%
\pgfpathmoveto{\pgfqpoint{-0.000000in}{0.000000in}}%
\pgfpathlineto{\pgfqpoint{-0.027778in}{0.000000in}}%
\pgfusepath{stroke,fill}%
}%
\begin{pgfscope}%
\pgfsys@transformshift{0.725193in}{2.849822in}%
\pgfsys@useobject{currentmarker}{}%
\end{pgfscope}%
\end{pgfscope}%
\begin{pgfscope}%
\pgfsetbuttcap%
\pgfsetroundjoin%
\definecolor{currentfill}{rgb}{0.000000,0.000000,0.000000}%
\pgfsetfillcolor{currentfill}%
\pgfsetlinewidth{0.602250pt}%
\definecolor{currentstroke}{rgb}{0.000000,0.000000,0.000000}%
\pgfsetstrokecolor{currentstroke}%
\pgfsetdash{}{0pt}%
\pgfsys@defobject{currentmarker}{\pgfqpoint{-0.027778in}{0.000000in}}{\pgfqpoint{-0.000000in}{0.000000in}}{%
\pgfpathmoveto{\pgfqpoint{-0.000000in}{0.000000in}}%
\pgfpathlineto{\pgfqpoint{-0.027778in}{0.000000in}}%
\pgfusepath{stroke,fill}%
}%
\begin{pgfscope}%
\pgfsys@transformshift{0.725193in}{2.909953in}%
\pgfsys@useobject{currentmarker}{}%
\end{pgfscope}%
\end{pgfscope}%
\begin{pgfscope}%
\pgfsetbuttcap%
\pgfsetroundjoin%
\definecolor{currentfill}{rgb}{0.000000,0.000000,0.000000}%
\pgfsetfillcolor{currentfill}%
\pgfsetlinewidth{0.602250pt}%
\definecolor{currentstroke}{rgb}{0.000000,0.000000,0.000000}%
\pgfsetstrokecolor{currentstroke}%
\pgfsetdash{}{0pt}%
\pgfsys@defobject{currentmarker}{\pgfqpoint{-0.027778in}{0.000000in}}{\pgfqpoint{-0.000000in}{0.000000in}}{%
\pgfpathmoveto{\pgfqpoint{-0.000000in}{0.000000in}}%
\pgfpathlineto{\pgfqpoint{-0.027778in}{0.000000in}}%
\pgfusepath{stroke,fill}%
}%
\begin{pgfscope}%
\pgfsys@transformshift{0.725193in}{2.960793in}%
\pgfsys@useobject{currentmarker}{}%
\end{pgfscope}%
\end{pgfscope}%
\begin{pgfscope}%
\pgfsetbuttcap%
\pgfsetroundjoin%
\definecolor{currentfill}{rgb}{0.000000,0.000000,0.000000}%
\pgfsetfillcolor{currentfill}%
\pgfsetlinewidth{0.602250pt}%
\definecolor{currentstroke}{rgb}{0.000000,0.000000,0.000000}%
\pgfsetstrokecolor{currentstroke}%
\pgfsetdash{}{0pt}%
\pgfsys@defobject{currentmarker}{\pgfqpoint{-0.027778in}{0.000000in}}{\pgfqpoint{-0.000000in}{0.000000in}}{%
\pgfpathmoveto{\pgfqpoint{-0.000000in}{0.000000in}}%
\pgfpathlineto{\pgfqpoint{-0.027778in}{0.000000in}}%
\pgfusepath{stroke,fill}%
}%
\begin{pgfscope}%
\pgfsys@transformshift{0.725193in}{3.004832in}%
\pgfsys@useobject{currentmarker}{}%
\end{pgfscope}%
\end{pgfscope}%
\begin{pgfscope}%
\pgfsetbuttcap%
\pgfsetroundjoin%
\definecolor{currentfill}{rgb}{0.000000,0.000000,0.000000}%
\pgfsetfillcolor{currentfill}%
\pgfsetlinewidth{0.602250pt}%
\definecolor{currentstroke}{rgb}{0.000000,0.000000,0.000000}%
\pgfsetstrokecolor{currentstroke}%
\pgfsetdash{}{0pt}%
\pgfsys@defobject{currentmarker}{\pgfqpoint{-0.027778in}{0.000000in}}{\pgfqpoint{-0.000000in}{0.000000in}}{%
\pgfpathmoveto{\pgfqpoint{-0.000000in}{0.000000in}}%
\pgfpathlineto{\pgfqpoint{-0.027778in}{0.000000in}}%
\pgfusepath{stroke,fill}%
}%
\begin{pgfscope}%
\pgfsys@transformshift{0.725193in}{3.043678in}%
\pgfsys@useobject{currentmarker}{}%
\end{pgfscope}%
\end{pgfscope}%
\begin{pgfscope}%
\pgfsetbuttcap%
\pgfsetroundjoin%
\definecolor{currentfill}{rgb}{0.000000,0.000000,0.000000}%
\pgfsetfillcolor{currentfill}%
\pgfsetlinewidth{0.602250pt}%
\definecolor{currentstroke}{rgb}{0.000000,0.000000,0.000000}%
\pgfsetstrokecolor{currentstroke}%
\pgfsetdash{}{0pt}%
\pgfsys@defobject{currentmarker}{\pgfqpoint{-0.027778in}{0.000000in}}{\pgfqpoint{-0.000000in}{0.000000in}}{%
\pgfpathmoveto{\pgfqpoint{-0.000000in}{0.000000in}}%
\pgfpathlineto{\pgfqpoint{-0.027778in}{0.000000in}}%
\pgfusepath{stroke,fill}%
}%
\begin{pgfscope}%
\pgfsys@transformshift{0.725193in}{3.307030in}%
\pgfsys@useobject{currentmarker}{}%
\end{pgfscope}%
\end{pgfscope}%
\begin{pgfscope}%
\pgfsetbuttcap%
\pgfsetroundjoin%
\definecolor{currentfill}{rgb}{0.000000,0.000000,0.000000}%
\pgfsetfillcolor{currentfill}%
\pgfsetlinewidth{0.602250pt}%
\definecolor{currentstroke}{rgb}{0.000000,0.000000,0.000000}%
\pgfsetstrokecolor{currentstroke}%
\pgfsetdash{}{0pt}%
\pgfsys@defobject{currentmarker}{\pgfqpoint{-0.027778in}{0.000000in}}{\pgfqpoint{-0.000000in}{0.000000in}}{%
\pgfpathmoveto{\pgfqpoint{-0.000000in}{0.000000in}}%
\pgfpathlineto{\pgfqpoint{-0.027778in}{0.000000in}}%
\pgfusepath{stroke,fill}%
}%
\begin{pgfscope}%
\pgfsys@transformshift{0.725193in}{3.440755in}%
\pgfsys@useobject{currentmarker}{}%
\end{pgfscope}%
\end{pgfscope}%
\begin{pgfscope}%
\pgfsetbuttcap%
\pgfsetroundjoin%
\definecolor{currentfill}{rgb}{0.000000,0.000000,0.000000}%
\pgfsetfillcolor{currentfill}%
\pgfsetlinewidth{0.602250pt}%
\definecolor{currentstroke}{rgb}{0.000000,0.000000,0.000000}%
\pgfsetstrokecolor{currentstroke}%
\pgfsetdash{}{0pt}%
\pgfsys@defobject{currentmarker}{\pgfqpoint{-0.027778in}{0.000000in}}{\pgfqpoint{-0.000000in}{0.000000in}}{%
\pgfpathmoveto{\pgfqpoint{-0.000000in}{0.000000in}}%
\pgfpathlineto{\pgfqpoint{-0.027778in}{0.000000in}}%
\pgfusepath{stroke,fill}%
}%
\begin{pgfscope}%
\pgfsys@transformshift{0.725193in}{3.535634in}%
\pgfsys@useobject{currentmarker}{}%
\end{pgfscope}%
\end{pgfscope}%
\begin{pgfscope}%
\pgfsetbuttcap%
\pgfsetroundjoin%
\definecolor{currentfill}{rgb}{0.000000,0.000000,0.000000}%
\pgfsetfillcolor{currentfill}%
\pgfsetlinewidth{0.602250pt}%
\definecolor{currentstroke}{rgb}{0.000000,0.000000,0.000000}%
\pgfsetstrokecolor{currentstroke}%
\pgfsetdash{}{0pt}%
\pgfsys@defobject{currentmarker}{\pgfqpoint{-0.027778in}{0.000000in}}{\pgfqpoint{-0.000000in}{0.000000in}}{%
\pgfpathmoveto{\pgfqpoint{-0.000000in}{0.000000in}}%
\pgfpathlineto{\pgfqpoint{-0.027778in}{0.000000in}}%
\pgfusepath{stroke,fill}%
}%
\begin{pgfscope}%
\pgfsys@transformshift{0.725193in}{3.609228in}%
\pgfsys@useobject{currentmarker}{}%
\end{pgfscope}%
\end{pgfscope}%
\begin{pgfscope}%
\pgfsetbuttcap%
\pgfsetroundjoin%
\definecolor{currentfill}{rgb}{0.000000,0.000000,0.000000}%
\pgfsetfillcolor{currentfill}%
\pgfsetlinewidth{0.602250pt}%
\definecolor{currentstroke}{rgb}{0.000000,0.000000,0.000000}%
\pgfsetstrokecolor{currentstroke}%
\pgfsetdash{}{0pt}%
\pgfsys@defobject{currentmarker}{\pgfqpoint{-0.027778in}{0.000000in}}{\pgfqpoint{-0.000000in}{0.000000in}}{%
\pgfpathmoveto{\pgfqpoint{-0.000000in}{0.000000in}}%
\pgfpathlineto{\pgfqpoint{-0.027778in}{0.000000in}}%
\pgfusepath{stroke,fill}%
}%
\begin{pgfscope}%
\pgfsys@transformshift{0.725193in}{3.669359in}%
\pgfsys@useobject{currentmarker}{}%
\end{pgfscope}%
\end{pgfscope}%
\begin{pgfscope}%
\pgfsetbuttcap%
\pgfsetroundjoin%
\definecolor{currentfill}{rgb}{0.000000,0.000000,0.000000}%
\pgfsetfillcolor{currentfill}%
\pgfsetlinewidth{0.602250pt}%
\definecolor{currentstroke}{rgb}{0.000000,0.000000,0.000000}%
\pgfsetstrokecolor{currentstroke}%
\pgfsetdash{}{0pt}%
\pgfsys@defobject{currentmarker}{\pgfqpoint{-0.027778in}{0.000000in}}{\pgfqpoint{-0.000000in}{0.000000in}}{%
\pgfpathmoveto{\pgfqpoint{-0.000000in}{0.000000in}}%
\pgfpathlineto{\pgfqpoint{-0.027778in}{0.000000in}}%
\pgfusepath{stroke,fill}%
}%
\begin{pgfscope}%
\pgfsys@transformshift{0.725193in}{3.720199in}%
\pgfsys@useobject{currentmarker}{}%
\end{pgfscope}%
\end{pgfscope}%
\begin{pgfscope}%
\pgfsetbuttcap%
\pgfsetroundjoin%
\definecolor{currentfill}{rgb}{0.000000,0.000000,0.000000}%
\pgfsetfillcolor{currentfill}%
\pgfsetlinewidth{0.602250pt}%
\definecolor{currentstroke}{rgb}{0.000000,0.000000,0.000000}%
\pgfsetstrokecolor{currentstroke}%
\pgfsetdash{}{0pt}%
\pgfsys@defobject{currentmarker}{\pgfqpoint{-0.027778in}{0.000000in}}{\pgfqpoint{-0.000000in}{0.000000in}}{%
\pgfpathmoveto{\pgfqpoint{-0.000000in}{0.000000in}}%
\pgfpathlineto{\pgfqpoint{-0.027778in}{0.000000in}}%
\pgfusepath{stroke,fill}%
}%
\begin{pgfscope}%
\pgfsys@transformshift{0.725193in}{3.764238in}%
\pgfsys@useobject{currentmarker}{}%
\end{pgfscope}%
\end{pgfscope}%
\begin{pgfscope}%
\pgfsetbuttcap%
\pgfsetroundjoin%
\definecolor{currentfill}{rgb}{0.000000,0.000000,0.000000}%
\pgfsetfillcolor{currentfill}%
\pgfsetlinewidth{0.602250pt}%
\definecolor{currentstroke}{rgb}{0.000000,0.000000,0.000000}%
\pgfsetstrokecolor{currentstroke}%
\pgfsetdash{}{0pt}%
\pgfsys@defobject{currentmarker}{\pgfqpoint{-0.027778in}{0.000000in}}{\pgfqpoint{-0.000000in}{0.000000in}}{%
\pgfpathmoveto{\pgfqpoint{-0.000000in}{0.000000in}}%
\pgfpathlineto{\pgfqpoint{-0.027778in}{0.000000in}}%
\pgfusepath{stroke,fill}%
}%
\begin{pgfscope}%
\pgfsys@transformshift{0.725193in}{3.803084in}%
\pgfsys@useobject{currentmarker}{}%
\end{pgfscope}%
\end{pgfscope}%
\begin{pgfscope}%
\pgfsetbuttcap%
\pgfsetroundjoin%
\definecolor{currentfill}{rgb}{0.000000,0.000000,0.000000}%
\pgfsetfillcolor{currentfill}%
\pgfsetlinewidth{0.602250pt}%
\definecolor{currentstroke}{rgb}{0.000000,0.000000,0.000000}%
\pgfsetstrokecolor{currentstroke}%
\pgfsetdash{}{0pt}%
\pgfsys@defobject{currentmarker}{\pgfqpoint{-0.027778in}{0.000000in}}{\pgfqpoint{-0.000000in}{0.000000in}}{%
\pgfpathmoveto{\pgfqpoint{-0.000000in}{0.000000in}}%
\pgfpathlineto{\pgfqpoint{-0.027778in}{0.000000in}}%
\pgfusepath{stroke,fill}%
}%
\begin{pgfscope}%
\pgfsys@transformshift{0.725193in}{4.066436in}%
\pgfsys@useobject{currentmarker}{}%
\end{pgfscope}%
\end{pgfscope}%
\begin{pgfscope}%
\pgfsetbuttcap%
\pgfsetroundjoin%
\definecolor{currentfill}{rgb}{0.000000,0.000000,0.000000}%
\pgfsetfillcolor{currentfill}%
\pgfsetlinewidth{0.602250pt}%
\definecolor{currentstroke}{rgb}{0.000000,0.000000,0.000000}%
\pgfsetstrokecolor{currentstroke}%
\pgfsetdash{}{0pt}%
\pgfsys@defobject{currentmarker}{\pgfqpoint{-0.027778in}{0.000000in}}{\pgfqpoint{-0.000000in}{0.000000in}}{%
\pgfpathmoveto{\pgfqpoint{-0.000000in}{0.000000in}}%
\pgfpathlineto{\pgfqpoint{-0.027778in}{0.000000in}}%
\pgfusepath{stroke,fill}%
}%
\begin{pgfscope}%
\pgfsys@transformshift{0.725193in}{4.200161in}%
\pgfsys@useobject{currentmarker}{}%
\end{pgfscope}%
\end{pgfscope}%
\begin{pgfscope}%
\pgfsetbuttcap%
\pgfsetroundjoin%
\definecolor{currentfill}{rgb}{0.000000,0.000000,0.000000}%
\pgfsetfillcolor{currentfill}%
\pgfsetlinewidth{0.602250pt}%
\definecolor{currentstroke}{rgb}{0.000000,0.000000,0.000000}%
\pgfsetstrokecolor{currentstroke}%
\pgfsetdash{}{0pt}%
\pgfsys@defobject{currentmarker}{\pgfqpoint{-0.027778in}{0.000000in}}{\pgfqpoint{-0.000000in}{0.000000in}}{%
\pgfpathmoveto{\pgfqpoint{-0.000000in}{0.000000in}}%
\pgfpathlineto{\pgfqpoint{-0.027778in}{0.000000in}}%
\pgfusepath{stroke,fill}%
}%
\begin{pgfscope}%
\pgfsys@transformshift{0.725193in}{4.295040in}%
\pgfsys@useobject{currentmarker}{}%
\end{pgfscope}%
\end{pgfscope}%
\begin{pgfscope}%
\pgfsetbuttcap%
\pgfsetroundjoin%
\definecolor{currentfill}{rgb}{0.000000,0.000000,0.000000}%
\pgfsetfillcolor{currentfill}%
\pgfsetlinewidth{0.602250pt}%
\definecolor{currentstroke}{rgb}{0.000000,0.000000,0.000000}%
\pgfsetstrokecolor{currentstroke}%
\pgfsetdash{}{0pt}%
\pgfsys@defobject{currentmarker}{\pgfqpoint{-0.027778in}{0.000000in}}{\pgfqpoint{-0.000000in}{0.000000in}}{%
\pgfpathmoveto{\pgfqpoint{-0.000000in}{0.000000in}}%
\pgfpathlineto{\pgfqpoint{-0.027778in}{0.000000in}}%
\pgfusepath{stroke,fill}%
}%
\begin{pgfscope}%
\pgfsys@transformshift{0.725193in}{4.368634in}%
\pgfsys@useobject{currentmarker}{}%
\end{pgfscope}%
\end{pgfscope}%
\begin{pgfscope}%
\pgfsetbuttcap%
\pgfsetroundjoin%
\definecolor{currentfill}{rgb}{0.000000,0.000000,0.000000}%
\pgfsetfillcolor{currentfill}%
\pgfsetlinewidth{0.602250pt}%
\definecolor{currentstroke}{rgb}{0.000000,0.000000,0.000000}%
\pgfsetstrokecolor{currentstroke}%
\pgfsetdash{}{0pt}%
\pgfsys@defobject{currentmarker}{\pgfqpoint{-0.027778in}{0.000000in}}{\pgfqpoint{-0.000000in}{0.000000in}}{%
\pgfpathmoveto{\pgfqpoint{-0.000000in}{0.000000in}}%
\pgfpathlineto{\pgfqpoint{-0.027778in}{0.000000in}}%
\pgfusepath{stroke,fill}%
}%
\begin{pgfscope}%
\pgfsys@transformshift{0.725193in}{4.428765in}%
\pgfsys@useobject{currentmarker}{}%
\end{pgfscope}%
\end{pgfscope}%
\begin{pgfscope}%
\pgfsetbuttcap%
\pgfsetroundjoin%
\definecolor{currentfill}{rgb}{0.000000,0.000000,0.000000}%
\pgfsetfillcolor{currentfill}%
\pgfsetlinewidth{0.602250pt}%
\definecolor{currentstroke}{rgb}{0.000000,0.000000,0.000000}%
\pgfsetstrokecolor{currentstroke}%
\pgfsetdash{}{0pt}%
\pgfsys@defobject{currentmarker}{\pgfqpoint{-0.027778in}{0.000000in}}{\pgfqpoint{-0.000000in}{0.000000in}}{%
\pgfpathmoveto{\pgfqpoint{-0.000000in}{0.000000in}}%
\pgfpathlineto{\pgfqpoint{-0.027778in}{0.000000in}}%
\pgfusepath{stroke,fill}%
}%
\begin{pgfscope}%
\pgfsys@transformshift{0.725193in}{4.479605in}%
\pgfsys@useobject{currentmarker}{}%
\end{pgfscope}%
\end{pgfscope}%
\begin{pgfscope}%
\pgfsetbuttcap%
\pgfsetroundjoin%
\definecolor{currentfill}{rgb}{0.000000,0.000000,0.000000}%
\pgfsetfillcolor{currentfill}%
\pgfsetlinewidth{0.602250pt}%
\definecolor{currentstroke}{rgb}{0.000000,0.000000,0.000000}%
\pgfsetstrokecolor{currentstroke}%
\pgfsetdash{}{0pt}%
\pgfsys@defobject{currentmarker}{\pgfqpoint{-0.027778in}{0.000000in}}{\pgfqpoint{-0.000000in}{0.000000in}}{%
\pgfpathmoveto{\pgfqpoint{-0.000000in}{0.000000in}}%
\pgfpathlineto{\pgfqpoint{-0.027778in}{0.000000in}}%
\pgfusepath{stroke,fill}%
}%
\begin{pgfscope}%
\pgfsys@transformshift{0.725193in}{4.523644in}%
\pgfsys@useobject{currentmarker}{}%
\end{pgfscope}%
\end{pgfscope}%
\begin{pgfscope}%
\pgfsetbuttcap%
\pgfsetroundjoin%
\definecolor{currentfill}{rgb}{0.000000,0.000000,0.000000}%
\pgfsetfillcolor{currentfill}%
\pgfsetlinewidth{0.602250pt}%
\definecolor{currentstroke}{rgb}{0.000000,0.000000,0.000000}%
\pgfsetstrokecolor{currentstroke}%
\pgfsetdash{}{0pt}%
\pgfsys@defobject{currentmarker}{\pgfqpoint{-0.027778in}{0.000000in}}{\pgfqpoint{-0.000000in}{0.000000in}}{%
\pgfpathmoveto{\pgfqpoint{-0.000000in}{0.000000in}}%
\pgfpathlineto{\pgfqpoint{-0.027778in}{0.000000in}}%
\pgfusepath{stroke,fill}%
}%
\begin{pgfscope}%
\pgfsys@transformshift{0.725193in}{4.562490in}%
\pgfsys@useobject{currentmarker}{}%
\end{pgfscope}%
\end{pgfscope}%
\begin{pgfscope}%
\definecolor{textcolor}{rgb}{0.000000,0.000000,0.000000}%
\pgfsetstrokecolor{textcolor}%
\pgfsetfillcolor{textcolor}%
\pgftext[x=0.284413in,y=2.584421in,,bottom,rotate=90.000000]{\color{textcolor}\sffamily\fontsize{10.000000}{12.000000}\selectfont Longest solving time (s)}%
\end{pgfscope}%
\begin{pgfscope}%
\pgfpathrectangle{\pgfqpoint{0.725193in}{0.571603in}}{\pgfqpoint{5.524807in}{4.025635in}}%
\pgfusepath{clip}%
\pgfsetbuttcap%
\pgfsetroundjoin%
\pgfsetlinewidth{2.007500pt}%
\definecolor{currentstroke}{rgb}{1.000000,0.843137,0.000000}%
\pgfsetstrokecolor{currentstroke}%
\pgfsetdash{{7.400000pt}{3.200000pt}}{0.000000pt}%
\pgfpathmoveto{\pgfqpoint{0.725193in}{2.281323in}}%
\pgfpathlineto{\pgfqpoint{0.730257in}{2.286513in}}%
\pgfpathlineto{\pgfqpoint{0.740385in}{2.292270in}}%
\pgfpathlineto{\pgfqpoint{0.745449in}{2.297379in}}%
\pgfpathlineto{\pgfqpoint{0.755577in}{2.300082in}}%
\pgfpathlineto{\pgfqpoint{0.760641in}{2.307605in}}%
\pgfpathlineto{\pgfqpoint{0.765705in}{2.309583in}}%
\pgfpathlineto{\pgfqpoint{0.775833in}{2.328645in}}%
\pgfpathlineto{\pgfqpoint{0.780897in}{2.333607in}}%
\pgfpathlineto{\pgfqpoint{0.785961in}{2.334541in}}%
\pgfpathlineto{\pgfqpoint{0.791025in}{2.338096in}}%
\pgfpathlineto{\pgfqpoint{0.801153in}{2.362640in}}%
\pgfpathlineto{\pgfqpoint{0.806217in}{2.367899in}}%
\pgfpathlineto{\pgfqpoint{0.811281in}{2.369838in}}%
\pgfpathlineto{\pgfqpoint{0.816345in}{2.376014in}}%
\pgfpathlineto{\pgfqpoint{0.821409in}{2.376668in}}%
\pgfpathlineto{\pgfqpoint{0.831537in}{2.384435in}}%
\pgfpathlineto{\pgfqpoint{0.836601in}{2.384854in}}%
\pgfpathlineto{\pgfqpoint{0.841665in}{2.394248in}}%
\pgfpathlineto{\pgfqpoint{0.851793in}{2.394674in}}%
\pgfpathlineto{\pgfqpoint{0.856857in}{2.400444in}}%
\pgfpathlineto{\pgfqpoint{0.866985in}{2.406823in}}%
\pgfpathlineto{\pgfqpoint{0.872049in}{2.427132in}}%
\pgfpathlineto{\pgfqpoint{0.877113in}{2.428840in}}%
\pgfpathlineto{\pgfqpoint{0.882177in}{2.429275in}}%
\pgfpathlineto{\pgfqpoint{0.887241in}{2.467430in}}%
\pgfpathlineto{\pgfqpoint{0.892305in}{2.471055in}}%
\pgfpathlineto{\pgfqpoint{0.897369in}{2.483458in}}%
\pgfpathlineto{\pgfqpoint{0.917624in}{2.487475in}}%
\pgfpathlineto{\pgfqpoint{0.927752in}{2.506116in}}%
\pgfpathlineto{\pgfqpoint{0.937880in}{2.511362in}}%
\pgfpathlineto{\pgfqpoint{0.942944in}{2.511447in}}%
\pgfpathlineto{\pgfqpoint{0.948008in}{2.514899in}}%
\pgfpathlineto{\pgfqpoint{0.958136in}{2.515942in}}%
\pgfpathlineto{\pgfqpoint{0.968264in}{2.523736in}}%
\pgfpathlineto{\pgfqpoint{0.973328in}{2.523869in}}%
\pgfpathlineto{\pgfqpoint{0.978392in}{2.527499in}}%
\pgfpathlineto{\pgfqpoint{0.993584in}{2.531821in}}%
\pgfpathlineto{\pgfqpoint{1.013840in}{2.533743in}}%
\pgfpathlineto{\pgfqpoint{1.018904in}{2.535609in}}%
\pgfpathlineto{\pgfqpoint{1.023968in}{2.536043in}}%
\pgfpathlineto{\pgfqpoint{1.039160in}{2.542686in}}%
\pgfpathlineto{\pgfqpoint{1.049288in}{2.546595in}}%
\pgfpathlineto{\pgfqpoint{1.054352in}{2.546919in}}%
\pgfpathlineto{\pgfqpoint{1.059416in}{2.558309in}}%
\pgfpathlineto{\pgfqpoint{1.064480in}{2.558397in}}%
\pgfpathlineto{\pgfqpoint{1.069544in}{2.560559in}}%
\pgfpathlineto{\pgfqpoint{1.074608in}{2.565864in}}%
\pgfpathlineto{\pgfqpoint{1.079672in}{2.565940in}}%
\pgfpathlineto{\pgfqpoint{1.084736in}{2.569207in}}%
\pgfpathlineto{\pgfqpoint{1.089800in}{2.569843in}}%
\pgfpathlineto{\pgfqpoint{1.099928in}{2.573919in}}%
\pgfpathlineto{\pgfqpoint{1.104992in}{2.583100in}}%
\pgfpathlineto{\pgfqpoint{1.110056in}{2.608879in}}%
\pgfpathlineto{\pgfqpoint{1.115120in}{2.615770in}}%
\pgfpathlineto{\pgfqpoint{1.125248in}{2.621603in}}%
\pgfpathlineto{\pgfqpoint{1.135376in}{2.632011in}}%
\pgfpathlineto{\pgfqpoint{1.145504in}{2.636857in}}%
\pgfpathlineto{\pgfqpoint{1.150568in}{2.641897in}}%
\pgfpathlineto{\pgfqpoint{1.155632in}{2.662973in}}%
\pgfpathlineto{\pgfqpoint{1.160696in}{2.663723in}}%
\pgfpathlineto{\pgfqpoint{1.165760in}{2.665823in}}%
\pgfpathlineto{\pgfqpoint{1.175888in}{2.666708in}}%
\pgfpathlineto{\pgfqpoint{1.180952in}{2.669643in}}%
\pgfpathlineto{\pgfqpoint{1.186016in}{2.678255in}}%
\pgfpathlineto{\pgfqpoint{1.191080in}{2.680161in}}%
\pgfpathlineto{\pgfqpoint{1.196144in}{2.686351in}}%
\pgfpathlineto{\pgfqpoint{1.201208in}{2.687013in}}%
\pgfpathlineto{\pgfqpoint{1.206272in}{2.688877in}}%
\pgfpathlineto{\pgfqpoint{1.211336in}{2.693825in}}%
\pgfpathlineto{\pgfqpoint{1.216400in}{2.695541in}}%
\pgfpathlineto{\pgfqpoint{1.221464in}{2.699507in}}%
\pgfpathlineto{\pgfqpoint{1.226528in}{2.701193in}}%
\pgfpathlineto{\pgfqpoint{1.236655in}{2.707519in}}%
\pgfpathlineto{\pgfqpoint{1.241719in}{2.713831in}}%
\pgfpathlineto{\pgfqpoint{1.246783in}{2.716388in}}%
\pgfpathlineto{\pgfqpoint{1.251847in}{2.737138in}}%
\pgfpathlineto{\pgfqpoint{1.256911in}{2.739982in}}%
\pgfpathlineto{\pgfqpoint{1.261975in}{2.753560in}}%
\pgfpathlineto{\pgfqpoint{1.267039in}{2.758231in}}%
\pgfpathlineto{\pgfqpoint{1.282231in}{2.761362in}}%
\pgfpathlineto{\pgfqpoint{1.287295in}{2.766567in}}%
\pgfpathlineto{\pgfqpoint{1.292359in}{2.769935in}}%
\pgfpathlineto{\pgfqpoint{1.297423in}{2.775515in}}%
\pgfpathlineto{\pgfqpoint{1.302487in}{2.777178in}}%
\pgfpathlineto{\pgfqpoint{1.307551in}{2.777244in}}%
\pgfpathlineto{\pgfqpoint{1.317679in}{2.783043in}}%
\pgfpathlineto{\pgfqpoint{1.322743in}{2.784529in}}%
\pgfpathlineto{\pgfqpoint{1.327807in}{2.789807in}}%
\pgfpathlineto{\pgfqpoint{1.332871in}{2.791165in}}%
\pgfpathlineto{\pgfqpoint{1.337935in}{2.799298in}}%
\pgfpathlineto{\pgfqpoint{1.342999in}{2.802968in}}%
\pgfpathlineto{\pgfqpoint{1.353127in}{2.804178in}}%
\pgfpathlineto{\pgfqpoint{1.363255in}{2.819615in}}%
\pgfpathlineto{\pgfqpoint{1.368319in}{2.821689in}}%
\pgfpathlineto{\pgfqpoint{1.378447in}{2.832600in}}%
\pgfpathlineto{\pgfqpoint{1.398703in}{2.838823in}}%
\pgfpathlineto{\pgfqpoint{1.403767in}{2.840935in}}%
\pgfpathlineto{\pgfqpoint{1.408831in}{2.841507in}}%
\pgfpathlineto{\pgfqpoint{1.413895in}{2.846374in}}%
\pgfpathlineto{\pgfqpoint{1.418959in}{2.857406in}}%
\pgfpathlineto{\pgfqpoint{1.429087in}{2.858340in}}%
\pgfpathlineto{\pgfqpoint{1.434151in}{2.863398in}}%
\pgfpathlineto{\pgfqpoint{1.439215in}{2.871320in}}%
\pgfpathlineto{\pgfqpoint{1.464535in}{2.882656in}}%
\pgfpathlineto{\pgfqpoint{1.469599in}{2.887227in}}%
\pgfpathlineto{\pgfqpoint{1.474663in}{2.887678in}}%
\pgfpathlineto{\pgfqpoint{1.479727in}{2.895600in}}%
\pgfpathlineto{\pgfqpoint{1.484791in}{2.897211in}}%
\pgfpathlineto{\pgfqpoint{1.489855in}{2.904596in}}%
\pgfpathlineto{\pgfqpoint{1.494919in}{2.906609in}}%
\pgfpathlineto{\pgfqpoint{1.499983in}{2.920636in}}%
\pgfpathlineto{\pgfqpoint{1.505047in}{2.922500in}}%
\pgfpathlineto{\pgfqpoint{1.510111in}{2.930291in}}%
\pgfpathlineto{\pgfqpoint{1.515175in}{2.935372in}}%
\pgfpathlineto{\pgfqpoint{1.520239in}{2.937798in}}%
\pgfpathlineto{\pgfqpoint{1.525303in}{2.947685in}}%
\pgfpathlineto{\pgfqpoint{1.530367in}{2.947973in}}%
\pgfpathlineto{\pgfqpoint{1.535431in}{2.952011in}}%
\pgfpathlineto{\pgfqpoint{1.550623in}{2.954195in}}%
\pgfpathlineto{\pgfqpoint{1.555686in}{2.965190in}}%
\pgfpathlineto{\pgfqpoint{1.560750in}{2.967427in}}%
\pgfpathlineto{\pgfqpoint{1.565814in}{2.986913in}}%
\pgfpathlineto{\pgfqpoint{1.575942in}{3.009148in}}%
\pgfpathlineto{\pgfqpoint{1.581006in}{3.012543in}}%
\pgfpathlineto{\pgfqpoint{1.586070in}{3.026020in}}%
\pgfpathlineto{\pgfqpoint{1.596198in}{3.031801in}}%
\pgfpathlineto{\pgfqpoint{1.601262in}{3.036280in}}%
\pgfpathlineto{\pgfqpoint{1.606326in}{3.039119in}}%
\pgfpathlineto{\pgfqpoint{1.667094in}{3.049761in}}%
\pgfpathlineto{\pgfqpoint{1.672158in}{3.051086in}}%
\pgfpathlineto{\pgfqpoint{1.687350in}{3.052231in}}%
\pgfpathlineto{\pgfqpoint{1.692414in}{3.059382in}}%
\pgfpathlineto{\pgfqpoint{1.702542in}{3.061586in}}%
\pgfpathlineto{\pgfqpoint{1.712670in}{3.062632in}}%
\pgfpathlineto{\pgfqpoint{1.717734in}{3.064685in}}%
\pgfpathlineto{\pgfqpoint{1.727862in}{3.065096in}}%
\pgfpathlineto{\pgfqpoint{1.732926in}{3.067174in}}%
\pgfpathlineto{\pgfqpoint{1.737990in}{3.067194in}}%
\pgfpathlineto{\pgfqpoint{1.743054in}{3.069971in}}%
\pgfpathlineto{\pgfqpoint{1.748118in}{3.070479in}}%
\pgfpathlineto{\pgfqpoint{1.763310in}{3.078028in}}%
\pgfpathlineto{\pgfqpoint{1.768374in}{3.085229in}}%
\pgfpathlineto{\pgfqpoint{1.773438in}{3.086305in}}%
\pgfpathlineto{\pgfqpoint{1.778502in}{3.094663in}}%
\pgfpathlineto{\pgfqpoint{1.783566in}{3.111079in}}%
\pgfpathlineto{\pgfqpoint{1.788630in}{3.118549in}}%
\pgfpathlineto{\pgfqpoint{1.793694in}{3.120987in}}%
\pgfpathlineto{\pgfqpoint{1.798758in}{3.125530in}}%
\pgfpathlineto{\pgfqpoint{1.803822in}{3.128521in}}%
\pgfpathlineto{\pgfqpoint{1.813950in}{3.143852in}}%
\pgfpathlineto{\pgfqpoint{1.819014in}{3.144707in}}%
\pgfpathlineto{\pgfqpoint{1.824078in}{3.153306in}}%
\pgfpathlineto{\pgfqpoint{1.829142in}{3.155678in}}%
\pgfpathlineto{\pgfqpoint{1.834206in}{3.156318in}}%
\pgfpathlineto{\pgfqpoint{1.839270in}{3.161113in}}%
\pgfpathlineto{\pgfqpoint{1.844334in}{3.174083in}}%
\pgfpathlineto{\pgfqpoint{1.849398in}{3.192873in}}%
\pgfpathlineto{\pgfqpoint{1.854462in}{3.195166in}}%
\pgfpathlineto{\pgfqpoint{1.859526in}{3.236967in}}%
\pgfpathlineto{\pgfqpoint{1.864590in}{3.237604in}}%
\pgfpathlineto{\pgfqpoint{1.869654in}{3.243413in}}%
\pgfpathlineto{\pgfqpoint{1.874718in}{3.245893in}}%
\pgfpathlineto{\pgfqpoint{1.884845in}{3.247091in}}%
\pgfpathlineto{\pgfqpoint{1.889909in}{3.255046in}}%
\pgfpathlineto{\pgfqpoint{1.894973in}{3.271281in}}%
\pgfpathlineto{\pgfqpoint{1.900037in}{3.272493in}}%
\pgfpathlineto{\pgfqpoint{1.905101in}{3.277013in}}%
\pgfpathlineto{\pgfqpoint{1.915229in}{3.278473in}}%
\pgfpathlineto{\pgfqpoint{1.920293in}{3.278861in}}%
\pgfpathlineto{\pgfqpoint{1.930421in}{3.306691in}}%
\pgfpathlineto{\pgfqpoint{1.935485in}{3.308402in}}%
\pgfpathlineto{\pgfqpoint{1.950677in}{3.331091in}}%
\pgfpathlineto{\pgfqpoint{1.955741in}{3.331192in}}%
\pgfpathlineto{\pgfqpoint{1.965869in}{3.355847in}}%
\pgfpathlineto{\pgfqpoint{1.975997in}{3.357895in}}%
\pgfpathlineto{\pgfqpoint{1.981061in}{3.361303in}}%
\pgfpathlineto{\pgfqpoint{1.991189in}{3.361949in}}%
\pgfpathlineto{\pgfqpoint{2.001317in}{3.372306in}}%
\pgfpathlineto{\pgfqpoint{2.006381in}{3.372566in}}%
\pgfpathlineto{\pgfqpoint{2.011445in}{3.379045in}}%
\pgfpathlineto{\pgfqpoint{2.026637in}{3.380584in}}%
\pgfpathlineto{\pgfqpoint{2.041829in}{3.394508in}}%
\pgfpathlineto{\pgfqpoint{2.057021in}{3.399000in}}%
\pgfpathlineto{\pgfqpoint{2.067149in}{3.404107in}}%
\pgfpathlineto{\pgfqpoint{2.072213in}{3.412286in}}%
\pgfpathlineto{\pgfqpoint{2.077277in}{3.427373in}}%
\pgfpathlineto{\pgfqpoint{2.082341in}{3.430643in}}%
\pgfpathlineto{\pgfqpoint{2.087405in}{3.435510in}}%
\pgfpathlineto{\pgfqpoint{2.097533in}{3.437944in}}%
\pgfpathlineto{\pgfqpoint{2.102597in}{3.443967in}}%
\pgfpathlineto{\pgfqpoint{2.107661in}{3.453630in}}%
\pgfpathlineto{\pgfqpoint{2.117789in}{3.454435in}}%
\pgfpathlineto{\pgfqpoint{2.122853in}{3.464334in}}%
\pgfpathlineto{\pgfqpoint{2.138045in}{3.479002in}}%
\pgfpathlineto{\pgfqpoint{2.143109in}{3.488053in}}%
\pgfpathlineto{\pgfqpoint{2.148173in}{3.490912in}}%
\pgfpathlineto{\pgfqpoint{2.153237in}{3.495624in}}%
\pgfpathlineto{\pgfqpoint{2.158301in}{3.497845in}}%
\pgfpathlineto{\pgfqpoint{2.163365in}{3.502876in}}%
\pgfpathlineto{\pgfqpoint{2.168429in}{3.543678in}}%
\pgfpathlineto{\pgfqpoint{2.173493in}{3.547551in}}%
\pgfpathlineto{\pgfqpoint{2.188685in}{3.551655in}}%
\pgfpathlineto{\pgfqpoint{2.193749in}{3.555910in}}%
\pgfpathlineto{\pgfqpoint{2.198813in}{3.557126in}}%
\pgfpathlineto{\pgfqpoint{2.203876in}{3.561788in}}%
\pgfpathlineto{\pgfqpoint{2.208940in}{3.562459in}}%
\pgfpathlineto{\pgfqpoint{2.214004in}{3.566278in}}%
\pgfpathlineto{\pgfqpoint{2.219068in}{3.572379in}}%
\pgfpathlineto{\pgfqpoint{2.224132in}{3.573891in}}%
\pgfpathlineto{\pgfqpoint{2.234260in}{3.575074in}}%
\pgfpathlineto{\pgfqpoint{2.239324in}{3.576158in}}%
\pgfpathlineto{\pgfqpoint{2.249452in}{3.580537in}}%
\pgfpathlineto{\pgfqpoint{2.259580in}{3.581824in}}%
\pgfpathlineto{\pgfqpoint{2.264644in}{3.585901in}}%
\pgfpathlineto{\pgfqpoint{2.269708in}{3.588156in}}%
\pgfpathlineto{\pgfqpoint{2.274772in}{3.589171in}}%
\pgfpathlineto{\pgfqpoint{2.284900in}{3.599603in}}%
\pgfpathlineto{\pgfqpoint{2.300092in}{3.605557in}}%
\pgfpathlineto{\pgfqpoint{2.305156in}{3.611732in}}%
\pgfpathlineto{\pgfqpoint{2.310220in}{3.628123in}}%
\pgfpathlineto{\pgfqpoint{2.315284in}{3.653317in}}%
\pgfpathlineto{\pgfqpoint{2.320348in}{3.661340in}}%
\pgfpathlineto{\pgfqpoint{2.325412in}{3.673749in}}%
\pgfpathlineto{\pgfqpoint{2.330476in}{3.675729in}}%
\pgfpathlineto{\pgfqpoint{2.340604in}{3.682125in}}%
\pgfpathlineto{\pgfqpoint{2.345668in}{3.683362in}}%
\pgfpathlineto{\pgfqpoint{2.350732in}{3.687610in}}%
\pgfpathlineto{\pgfqpoint{2.360860in}{3.688722in}}%
\pgfpathlineto{\pgfqpoint{2.365924in}{3.700684in}}%
\pgfpathlineto{\pgfqpoint{2.370988in}{3.700833in}}%
\pgfpathlineto{\pgfqpoint{2.376052in}{3.703910in}}%
\pgfpathlineto{\pgfqpoint{2.381116in}{3.704014in}}%
\pgfpathlineto{\pgfqpoint{2.391244in}{3.726026in}}%
\pgfpathlineto{\pgfqpoint{2.396308in}{3.749925in}}%
\pgfpathlineto{\pgfqpoint{2.401372in}{3.763749in}}%
\pgfpathlineto{\pgfqpoint{2.431756in}{3.767202in}}%
\pgfpathlineto{\pgfqpoint{2.457076in}{3.773324in}}%
\pgfpathlineto{\pgfqpoint{2.462140in}{3.773547in}}%
\pgfpathlineto{\pgfqpoint{2.467204in}{3.774939in}}%
\pgfpathlineto{\pgfqpoint{2.472268in}{3.780275in}}%
\pgfpathlineto{\pgfqpoint{2.477332in}{3.782023in}}%
\pgfpathlineto{\pgfqpoint{2.487460in}{3.783390in}}%
\pgfpathlineto{\pgfqpoint{2.497588in}{3.790170in}}%
\pgfpathlineto{\pgfqpoint{2.502652in}{3.808999in}}%
\pgfpathlineto{\pgfqpoint{2.507716in}{3.812445in}}%
\pgfpathlineto{\pgfqpoint{2.522908in}{3.816041in}}%
\pgfpathlineto{\pgfqpoint{2.527971in}{3.821443in}}%
\pgfpathlineto{\pgfqpoint{2.533035in}{3.822567in}}%
\pgfpathlineto{\pgfqpoint{2.538099in}{3.825996in}}%
\pgfpathlineto{\pgfqpoint{2.548227in}{3.828248in}}%
\pgfpathlineto{\pgfqpoint{2.553291in}{3.831748in}}%
\pgfpathlineto{\pgfqpoint{2.563419in}{3.833553in}}%
\pgfpathlineto{\pgfqpoint{2.568483in}{3.834294in}}%
\pgfpathlineto{\pgfqpoint{2.573547in}{3.842750in}}%
\pgfpathlineto{\pgfqpoint{2.588739in}{3.848559in}}%
\pgfpathlineto{\pgfqpoint{2.603931in}{3.850121in}}%
\pgfpathlineto{\pgfqpoint{2.608995in}{3.855046in}}%
\pgfpathlineto{\pgfqpoint{2.614059in}{3.857081in}}%
\pgfpathlineto{\pgfqpoint{2.619123in}{3.863111in}}%
\pgfpathlineto{\pgfqpoint{2.624187in}{3.863684in}}%
\pgfpathlineto{\pgfqpoint{2.634315in}{3.866230in}}%
\pgfpathlineto{\pgfqpoint{2.639379in}{3.895420in}}%
\pgfpathlineto{\pgfqpoint{2.644443in}{3.898479in}}%
\pgfpathlineto{\pgfqpoint{2.654571in}{3.899506in}}%
\pgfpathlineto{\pgfqpoint{2.664699in}{3.901650in}}%
\pgfpathlineto{\pgfqpoint{2.669763in}{3.905894in}}%
\pgfpathlineto{\pgfqpoint{2.674827in}{3.913594in}}%
\pgfpathlineto{\pgfqpoint{2.679891in}{3.918737in}}%
\pgfpathlineto{\pgfqpoint{2.684955in}{3.963402in}}%
\pgfpathlineto{\pgfqpoint{2.690019in}{3.968296in}}%
\pgfpathlineto{\pgfqpoint{2.695083in}{3.970750in}}%
\pgfpathlineto{\pgfqpoint{2.705211in}{3.971674in}}%
\pgfpathlineto{\pgfqpoint{2.710275in}{3.980061in}}%
\pgfpathlineto{\pgfqpoint{2.720403in}{3.980473in}}%
\pgfpathlineto{\pgfqpoint{2.725467in}{3.981401in}}%
\pgfpathlineto{\pgfqpoint{2.730531in}{3.998085in}}%
\pgfpathlineto{\pgfqpoint{2.735595in}{3.999350in}}%
\pgfpathlineto{\pgfqpoint{2.740659in}{4.015266in}}%
\pgfpathlineto{\pgfqpoint{2.745723in}{4.060284in}}%
\pgfpathlineto{\pgfqpoint{2.750787in}{4.068690in}}%
\pgfpathlineto{\pgfqpoint{2.755851in}{4.074629in}}%
\pgfpathlineto{\pgfqpoint{2.765979in}{4.078090in}}%
\pgfpathlineto{\pgfqpoint{2.771043in}{4.090489in}}%
\pgfpathlineto{\pgfqpoint{2.786235in}{4.094067in}}%
\pgfpathlineto{\pgfqpoint{2.791299in}{4.095293in}}%
\pgfpathlineto{\pgfqpoint{2.801427in}{4.100639in}}%
\pgfpathlineto{\pgfqpoint{2.811555in}{4.101196in}}%
\pgfpathlineto{\pgfqpoint{2.821683in}{4.111435in}}%
\pgfpathlineto{\pgfqpoint{2.826747in}{4.111898in}}%
\pgfpathlineto{\pgfqpoint{2.831811in}{4.115487in}}%
\pgfpathlineto{\pgfqpoint{2.836875in}{4.120480in}}%
\pgfpathlineto{\pgfqpoint{2.841939in}{4.121561in}}%
\pgfpathlineto{\pgfqpoint{2.847003in}{4.126334in}}%
\pgfpathlineto{\pgfqpoint{2.852066in}{4.128973in}}%
\pgfpathlineto{\pgfqpoint{2.857130in}{4.129809in}}%
\pgfpathlineto{\pgfqpoint{2.862194in}{4.133476in}}%
\pgfpathlineto{\pgfqpoint{2.877386in}{4.149075in}}%
\pgfpathlineto{\pgfqpoint{2.882450in}{4.172316in}}%
\pgfpathlineto{\pgfqpoint{2.887514in}{4.182057in}}%
\pgfpathlineto{\pgfqpoint{2.892578in}{4.185847in}}%
\pgfpathlineto{\pgfqpoint{2.897642in}{4.187331in}}%
\pgfpathlineto{\pgfqpoint{2.902706in}{4.221761in}}%
\pgfpathlineto{\pgfqpoint{2.912834in}{4.237792in}}%
\pgfpathlineto{\pgfqpoint{2.917898in}{4.241451in}}%
\pgfpathlineto{\pgfqpoint{2.922962in}{4.246755in}}%
\pgfpathlineto{\pgfqpoint{2.928026in}{4.247206in}}%
\pgfpathlineto{\pgfqpoint{2.933090in}{4.250459in}}%
\pgfpathlineto{\pgfqpoint{2.938154in}{4.261768in}}%
\pgfpathlineto{\pgfqpoint{2.943218in}{4.265076in}}%
\pgfpathlineto{\pgfqpoint{2.948282in}{4.273361in}}%
\pgfpathlineto{\pgfqpoint{2.953346in}{4.351148in}}%
\pgfpathlineto{\pgfqpoint{2.963474in}{4.361488in}}%
\pgfpathlineto{\pgfqpoint{2.968538in}{4.362221in}}%
\pgfpathlineto{\pgfqpoint{2.978666in}{4.369667in}}%
\pgfpathlineto{\pgfqpoint{2.988794in}{4.371609in}}%
\pgfpathlineto{\pgfqpoint{2.998922in}{4.379987in}}%
\pgfpathlineto{\pgfqpoint{3.014114in}{4.383732in}}%
\pgfpathlineto{\pgfqpoint{3.034370in}{4.388619in}}%
\pgfpathlineto{\pgfqpoint{3.039434in}{4.390417in}}%
\pgfpathlineto{\pgfqpoint{3.044498in}{4.390904in}}%
\pgfpathlineto{\pgfqpoint{3.049562in}{4.397759in}}%
\pgfpathlineto{\pgfqpoint{3.054626in}{4.407693in}}%
\pgfpathlineto{\pgfqpoint{3.059690in}{4.408024in}}%
\pgfpathlineto{\pgfqpoint{3.064754in}{4.427202in}}%
\pgfpathlineto{\pgfqpoint{3.079946in}{4.429138in}}%
\pgfpathlineto{\pgfqpoint{3.095138in}{4.436934in}}%
\pgfpathlineto{\pgfqpoint{3.100202in}{4.442083in}}%
\pgfpathlineto{\pgfqpoint{3.110330in}{4.468458in}}%
\pgfpathlineto{\pgfqpoint{3.115394in}{4.470335in}}%
\pgfpathlineto{\pgfqpoint{3.120458in}{4.470438in}}%
\pgfpathlineto{\pgfqpoint{3.130586in}{4.482742in}}%
\pgfpathlineto{\pgfqpoint{3.135650in}{4.483760in}}%
\pgfpathlineto{\pgfqpoint{3.140714in}{4.486434in}}%
\pgfpathlineto{\pgfqpoint{3.145778in}{4.493606in}}%
\pgfpathlineto{\pgfqpoint{3.150842in}{4.493857in}}%
\pgfpathlineto{\pgfqpoint{3.155906in}{4.496827in}}%
\pgfpathlineto{\pgfqpoint{3.160970in}{4.498077in}}%
\pgfpathlineto{\pgfqpoint{3.166034in}{4.504994in}}%
\pgfpathlineto{\pgfqpoint{3.171098in}{4.505020in}}%
\pgfpathlineto{\pgfqpoint{3.176161in}{4.506585in}}%
\pgfpathlineto{\pgfqpoint{3.181225in}{4.511873in}}%
\pgfpathlineto{\pgfqpoint{3.186289in}{4.519236in}}%
\pgfpathlineto{\pgfqpoint{3.196417in}{4.524135in}}%
\pgfpathlineto{\pgfqpoint{3.211609in}{4.525074in}}%
\pgfpathlineto{\pgfqpoint{3.221737in}{4.550763in}}%
\pgfpathlineto{\pgfqpoint{3.226801in}{4.597238in}}%
\pgfpathlineto{\pgfqpoint{6.244936in}{4.597238in}}%
\pgfpathlineto{\pgfqpoint{6.244936in}{4.597238in}}%
\pgfusepath{stroke}%
\end{pgfscope}%
\begin{pgfscope}%
\pgfpathrectangle{\pgfqpoint{0.725193in}{0.571603in}}{\pgfqpoint{5.524807in}{4.025635in}}%
\pgfusepath{clip}%
\pgfsetbuttcap%
\pgfsetroundjoin%
\pgfsetlinewidth{2.007500pt}%
\definecolor{currentstroke}{rgb}{1.000000,0.694118,0.305882}%
\pgfsetstrokecolor{currentstroke}%
\pgfsetdash{{2.000000pt}{3.300000pt}}{0.000000pt}%
\pgfpathmoveto{\pgfqpoint{0.725193in}{1.925037in}}%
\pgfpathlineto{\pgfqpoint{0.730257in}{2.019130in}}%
\pgfpathlineto{\pgfqpoint{0.735321in}{2.021199in}}%
\pgfpathlineto{\pgfqpoint{0.740385in}{2.034050in}}%
\pgfpathlineto{\pgfqpoint{0.750513in}{2.036436in}}%
\pgfpathlineto{\pgfqpoint{0.755577in}{2.042795in}}%
\pgfpathlineto{\pgfqpoint{0.760641in}{2.043136in}}%
\pgfpathlineto{\pgfqpoint{0.765705in}{2.047944in}}%
\pgfpathlineto{\pgfqpoint{0.770769in}{2.060976in}}%
\pgfpathlineto{\pgfqpoint{0.775833in}{2.066641in}}%
\pgfpathlineto{\pgfqpoint{0.780897in}{2.080968in}}%
\pgfpathlineto{\pgfqpoint{0.785961in}{2.081638in}}%
\pgfpathlineto{\pgfqpoint{0.791025in}{2.086010in}}%
\pgfpathlineto{\pgfqpoint{0.796089in}{2.088038in}}%
\pgfpathlineto{\pgfqpoint{0.801153in}{2.088645in}}%
\pgfpathlineto{\pgfqpoint{0.806217in}{2.092800in}}%
\pgfpathlineto{\pgfqpoint{0.811281in}{2.102820in}}%
\pgfpathlineto{\pgfqpoint{0.816345in}{2.103844in}}%
\pgfpathlineto{\pgfqpoint{0.821409in}{2.106092in}}%
\pgfpathlineto{\pgfqpoint{0.831537in}{2.107356in}}%
\pgfpathlineto{\pgfqpoint{0.836601in}{2.117899in}}%
\pgfpathlineto{\pgfqpoint{0.841665in}{2.138323in}}%
\pgfpathlineto{\pgfqpoint{0.846729in}{2.143408in}}%
\pgfpathlineto{\pgfqpoint{0.851793in}{2.144514in}}%
\pgfpathlineto{\pgfqpoint{0.856857in}{2.152341in}}%
\pgfpathlineto{\pgfqpoint{0.861921in}{2.154231in}}%
\pgfpathlineto{\pgfqpoint{0.866985in}{2.158903in}}%
\pgfpathlineto{\pgfqpoint{0.882177in}{2.161732in}}%
\pgfpathlineto{\pgfqpoint{0.907496in}{2.168953in}}%
\pgfpathlineto{\pgfqpoint{0.922688in}{2.178129in}}%
\pgfpathlineto{\pgfqpoint{0.927752in}{2.179763in}}%
\pgfpathlineto{\pgfqpoint{0.932816in}{2.182960in}}%
\pgfpathlineto{\pgfqpoint{0.942944in}{2.193157in}}%
\pgfpathlineto{\pgfqpoint{0.948008in}{2.202885in}}%
\pgfpathlineto{\pgfqpoint{0.958136in}{2.205070in}}%
\pgfpathlineto{\pgfqpoint{0.973328in}{2.207861in}}%
\pgfpathlineto{\pgfqpoint{0.988520in}{2.214684in}}%
\pgfpathlineto{\pgfqpoint{1.013840in}{2.219859in}}%
\pgfpathlineto{\pgfqpoint{1.018904in}{2.225606in}}%
\pgfpathlineto{\pgfqpoint{1.023968in}{2.234177in}}%
\pgfpathlineto{\pgfqpoint{1.029032in}{2.235226in}}%
\pgfpathlineto{\pgfqpoint{1.034096in}{2.240755in}}%
\pgfpathlineto{\pgfqpoint{1.044224in}{2.268413in}}%
\pgfpathlineto{\pgfqpoint{1.054352in}{2.272751in}}%
\pgfpathlineto{\pgfqpoint{1.059416in}{2.280087in}}%
\pgfpathlineto{\pgfqpoint{1.074608in}{2.281875in}}%
\pgfpathlineto{\pgfqpoint{1.079672in}{2.285284in}}%
\pgfpathlineto{\pgfqpoint{1.084736in}{2.286806in}}%
\pgfpathlineto{\pgfqpoint{1.089800in}{2.290101in}}%
\pgfpathlineto{\pgfqpoint{1.094864in}{2.296252in}}%
\pgfpathlineto{\pgfqpoint{1.115120in}{2.303269in}}%
\pgfpathlineto{\pgfqpoint{1.120184in}{2.308916in}}%
\pgfpathlineto{\pgfqpoint{1.130312in}{2.310645in}}%
\pgfpathlineto{\pgfqpoint{1.140440in}{2.313398in}}%
\pgfpathlineto{\pgfqpoint{1.150568in}{2.316057in}}%
\pgfpathlineto{\pgfqpoint{1.160696in}{2.321684in}}%
\pgfpathlineto{\pgfqpoint{1.165760in}{2.324810in}}%
\pgfpathlineto{\pgfqpoint{1.170824in}{2.325017in}}%
\pgfpathlineto{\pgfqpoint{1.175888in}{2.326932in}}%
\pgfpathlineto{\pgfqpoint{1.180952in}{2.326989in}}%
\pgfpathlineto{\pgfqpoint{1.206272in}{2.334188in}}%
\pgfpathlineto{\pgfqpoint{1.211336in}{2.336886in}}%
\pgfpathlineto{\pgfqpoint{1.267039in}{2.346216in}}%
\pgfpathlineto{\pgfqpoint{1.277167in}{2.366184in}}%
\pgfpathlineto{\pgfqpoint{1.287295in}{2.368812in}}%
\pgfpathlineto{\pgfqpoint{1.292359in}{2.369112in}}%
\pgfpathlineto{\pgfqpoint{1.302487in}{2.378870in}}%
\pgfpathlineto{\pgfqpoint{1.307551in}{2.381391in}}%
\pgfpathlineto{\pgfqpoint{1.312615in}{2.382256in}}%
\pgfpathlineto{\pgfqpoint{1.317679in}{2.386128in}}%
\pgfpathlineto{\pgfqpoint{1.327807in}{2.387135in}}%
\pgfpathlineto{\pgfqpoint{1.337935in}{2.391176in}}%
\pgfpathlineto{\pgfqpoint{1.342999in}{2.393511in}}%
\pgfpathlineto{\pgfqpoint{1.353127in}{2.403259in}}%
\pgfpathlineto{\pgfqpoint{1.358191in}{2.403266in}}%
\pgfpathlineto{\pgfqpoint{1.368319in}{2.405261in}}%
\pgfpathlineto{\pgfqpoint{1.383511in}{2.406712in}}%
\pgfpathlineto{\pgfqpoint{1.388575in}{2.407510in}}%
\pgfpathlineto{\pgfqpoint{1.398703in}{2.412417in}}%
\pgfpathlineto{\pgfqpoint{1.429087in}{2.419006in}}%
\pgfpathlineto{\pgfqpoint{1.439215in}{2.425456in}}%
\pgfpathlineto{\pgfqpoint{1.449343in}{2.427651in}}%
\pgfpathlineto{\pgfqpoint{1.454407in}{2.434388in}}%
\pgfpathlineto{\pgfqpoint{1.464535in}{2.435593in}}%
\pgfpathlineto{\pgfqpoint{1.469599in}{2.439408in}}%
\pgfpathlineto{\pgfqpoint{1.474663in}{2.440540in}}%
\pgfpathlineto{\pgfqpoint{1.479727in}{2.446022in}}%
\pgfpathlineto{\pgfqpoint{1.484791in}{2.446499in}}%
\pgfpathlineto{\pgfqpoint{1.489855in}{2.449490in}}%
\pgfpathlineto{\pgfqpoint{1.499983in}{2.466737in}}%
\pgfpathlineto{\pgfqpoint{1.510111in}{2.472854in}}%
\pgfpathlineto{\pgfqpoint{1.515175in}{2.478806in}}%
\pgfpathlineto{\pgfqpoint{1.520239in}{2.482445in}}%
\pgfpathlineto{\pgfqpoint{1.525303in}{2.483303in}}%
\pgfpathlineto{\pgfqpoint{1.530367in}{2.495513in}}%
\pgfpathlineto{\pgfqpoint{1.540495in}{2.498674in}}%
\pgfpathlineto{\pgfqpoint{1.545559in}{2.499892in}}%
\pgfpathlineto{\pgfqpoint{1.550623in}{2.506016in}}%
\pgfpathlineto{\pgfqpoint{1.555686in}{2.508925in}}%
\pgfpathlineto{\pgfqpoint{1.560750in}{2.509534in}}%
\pgfpathlineto{\pgfqpoint{1.565814in}{2.513087in}}%
\pgfpathlineto{\pgfqpoint{1.570878in}{2.514890in}}%
\pgfpathlineto{\pgfqpoint{1.586070in}{2.524005in}}%
\pgfpathlineto{\pgfqpoint{1.591134in}{2.524836in}}%
\pgfpathlineto{\pgfqpoint{1.596198in}{2.528905in}}%
\pgfpathlineto{\pgfqpoint{1.606326in}{2.529386in}}%
\pgfpathlineto{\pgfqpoint{1.611390in}{2.533569in}}%
\pgfpathlineto{\pgfqpoint{1.616454in}{2.535018in}}%
\pgfpathlineto{\pgfqpoint{1.621518in}{2.544490in}}%
\pgfpathlineto{\pgfqpoint{1.626582in}{2.549556in}}%
\pgfpathlineto{\pgfqpoint{1.631646in}{2.557301in}}%
\pgfpathlineto{\pgfqpoint{1.636710in}{2.560583in}}%
\pgfpathlineto{\pgfqpoint{1.656966in}{2.564238in}}%
\pgfpathlineto{\pgfqpoint{1.672158in}{2.566926in}}%
\pgfpathlineto{\pgfqpoint{1.677222in}{2.574702in}}%
\pgfpathlineto{\pgfqpoint{1.682286in}{2.585941in}}%
\pgfpathlineto{\pgfqpoint{1.692414in}{2.586296in}}%
\pgfpathlineto{\pgfqpoint{1.697478in}{2.592365in}}%
\pgfpathlineto{\pgfqpoint{1.702542in}{2.593898in}}%
\pgfpathlineto{\pgfqpoint{1.707606in}{2.596602in}}%
\pgfpathlineto{\pgfqpoint{1.712670in}{2.601521in}}%
\pgfpathlineto{\pgfqpoint{1.717734in}{2.601867in}}%
\pgfpathlineto{\pgfqpoint{1.722798in}{2.605182in}}%
\pgfpathlineto{\pgfqpoint{1.727862in}{2.607056in}}%
\pgfpathlineto{\pgfqpoint{1.732926in}{2.611055in}}%
\pgfpathlineto{\pgfqpoint{1.737990in}{2.611280in}}%
\pgfpathlineto{\pgfqpoint{1.753182in}{2.616970in}}%
\pgfpathlineto{\pgfqpoint{1.758246in}{2.617124in}}%
\pgfpathlineto{\pgfqpoint{1.763310in}{2.619063in}}%
\pgfpathlineto{\pgfqpoint{1.768374in}{2.626498in}}%
\pgfpathlineto{\pgfqpoint{1.773438in}{2.627813in}}%
\pgfpathlineto{\pgfqpoint{1.778502in}{2.638731in}}%
\pgfpathlineto{\pgfqpoint{1.783566in}{2.638761in}}%
\pgfpathlineto{\pgfqpoint{1.793694in}{2.641831in}}%
\pgfpathlineto{\pgfqpoint{1.798758in}{2.645505in}}%
\pgfpathlineto{\pgfqpoint{1.808886in}{2.647624in}}%
\pgfpathlineto{\pgfqpoint{1.813950in}{2.656570in}}%
\pgfpathlineto{\pgfqpoint{1.819014in}{2.662608in}}%
\pgfpathlineto{\pgfqpoint{1.824078in}{2.662749in}}%
\pgfpathlineto{\pgfqpoint{1.839270in}{2.676021in}}%
\pgfpathlineto{\pgfqpoint{1.844334in}{2.682227in}}%
\pgfpathlineto{\pgfqpoint{1.859526in}{2.706569in}}%
\pgfpathlineto{\pgfqpoint{1.864590in}{2.709440in}}%
\pgfpathlineto{\pgfqpoint{1.869654in}{2.709995in}}%
\pgfpathlineto{\pgfqpoint{1.874718in}{2.716539in}}%
\pgfpathlineto{\pgfqpoint{1.879781in}{2.719804in}}%
\pgfpathlineto{\pgfqpoint{1.884845in}{2.725462in}}%
\pgfpathlineto{\pgfqpoint{1.894973in}{2.729412in}}%
\pgfpathlineto{\pgfqpoint{1.900037in}{2.729908in}}%
\pgfpathlineto{\pgfqpoint{1.910165in}{2.733492in}}%
\pgfpathlineto{\pgfqpoint{1.915229in}{2.757122in}}%
\pgfpathlineto{\pgfqpoint{1.930421in}{2.784240in}}%
\pgfpathlineto{\pgfqpoint{1.940549in}{2.785163in}}%
\pgfpathlineto{\pgfqpoint{1.945613in}{2.792927in}}%
\pgfpathlineto{\pgfqpoint{1.950677in}{2.797885in}}%
\pgfpathlineto{\pgfqpoint{1.955741in}{2.798062in}}%
\pgfpathlineto{\pgfqpoint{1.965869in}{2.802737in}}%
\pgfpathlineto{\pgfqpoint{1.975997in}{2.811366in}}%
\pgfpathlineto{\pgfqpoint{1.986125in}{2.811666in}}%
\pgfpathlineto{\pgfqpoint{1.991189in}{2.819977in}}%
\pgfpathlineto{\pgfqpoint{1.996253in}{2.821159in}}%
\pgfpathlineto{\pgfqpoint{2.001317in}{2.828134in}}%
\pgfpathlineto{\pgfqpoint{2.021573in}{2.831393in}}%
\pgfpathlineto{\pgfqpoint{2.026637in}{2.836659in}}%
\pgfpathlineto{\pgfqpoint{2.031701in}{2.837721in}}%
\pgfpathlineto{\pgfqpoint{2.036765in}{2.840860in}}%
\pgfpathlineto{\pgfqpoint{2.051957in}{2.842480in}}%
\pgfpathlineto{\pgfqpoint{2.062085in}{2.845298in}}%
\pgfpathlineto{\pgfqpoint{2.067149in}{2.848712in}}%
\pgfpathlineto{\pgfqpoint{2.077277in}{2.858597in}}%
\pgfpathlineto{\pgfqpoint{2.082341in}{2.859674in}}%
\pgfpathlineto{\pgfqpoint{2.092469in}{2.864720in}}%
\pgfpathlineto{\pgfqpoint{2.097533in}{2.865976in}}%
\pgfpathlineto{\pgfqpoint{2.102597in}{2.870107in}}%
\pgfpathlineto{\pgfqpoint{2.107661in}{2.872563in}}%
\pgfpathlineto{\pgfqpoint{2.112725in}{2.873167in}}%
\pgfpathlineto{\pgfqpoint{2.117789in}{2.875372in}}%
\pgfpathlineto{\pgfqpoint{2.122853in}{2.875381in}}%
\pgfpathlineto{\pgfqpoint{2.138045in}{2.884645in}}%
\pgfpathlineto{\pgfqpoint{2.148173in}{2.886774in}}%
\pgfpathlineto{\pgfqpoint{2.153237in}{2.894848in}}%
\pgfpathlineto{\pgfqpoint{2.158301in}{2.898456in}}%
\pgfpathlineto{\pgfqpoint{2.163365in}{2.910085in}}%
\pgfpathlineto{\pgfqpoint{2.168429in}{2.910748in}}%
\pgfpathlineto{\pgfqpoint{2.173493in}{2.916388in}}%
\pgfpathlineto{\pgfqpoint{2.178557in}{2.918983in}}%
\pgfpathlineto{\pgfqpoint{2.183621in}{2.930044in}}%
\pgfpathlineto{\pgfqpoint{2.188685in}{2.930263in}}%
\pgfpathlineto{\pgfqpoint{2.193749in}{2.934282in}}%
\pgfpathlineto{\pgfqpoint{2.198813in}{2.955310in}}%
\pgfpathlineto{\pgfqpoint{2.203876in}{2.957920in}}%
\pgfpathlineto{\pgfqpoint{2.208940in}{2.959234in}}%
\pgfpathlineto{\pgfqpoint{2.214004in}{2.962588in}}%
\pgfpathlineto{\pgfqpoint{2.219068in}{2.963499in}}%
\pgfpathlineto{\pgfqpoint{2.229196in}{2.971890in}}%
\pgfpathlineto{\pgfqpoint{2.234260in}{2.975078in}}%
\pgfpathlineto{\pgfqpoint{2.239324in}{2.975726in}}%
\pgfpathlineto{\pgfqpoint{2.244388in}{2.979750in}}%
\pgfpathlineto{\pgfqpoint{2.249452in}{2.996550in}}%
\pgfpathlineto{\pgfqpoint{2.259580in}{3.001292in}}%
\pgfpathlineto{\pgfqpoint{2.264644in}{3.013656in}}%
\pgfpathlineto{\pgfqpoint{2.274772in}{3.016425in}}%
\pgfpathlineto{\pgfqpoint{2.279836in}{3.025307in}}%
\pgfpathlineto{\pgfqpoint{2.300092in}{3.029587in}}%
\pgfpathlineto{\pgfqpoint{2.310220in}{3.038173in}}%
\pgfpathlineto{\pgfqpoint{2.320348in}{3.040997in}}%
\pgfpathlineto{\pgfqpoint{2.325412in}{3.041037in}}%
\pgfpathlineto{\pgfqpoint{2.330476in}{3.042374in}}%
\pgfpathlineto{\pgfqpoint{2.335540in}{3.048938in}}%
\pgfpathlineto{\pgfqpoint{2.340604in}{3.049008in}}%
\pgfpathlineto{\pgfqpoint{2.350732in}{3.052485in}}%
\pgfpathlineto{\pgfqpoint{2.355796in}{3.055590in}}%
\pgfpathlineto{\pgfqpoint{2.365924in}{3.056549in}}%
\pgfpathlineto{\pgfqpoint{2.376052in}{3.058282in}}%
\pgfpathlineto{\pgfqpoint{2.391244in}{3.060357in}}%
\pgfpathlineto{\pgfqpoint{2.401372in}{3.067497in}}%
\pgfpathlineto{\pgfqpoint{2.406436in}{3.069346in}}%
\pgfpathlineto{\pgfqpoint{2.416564in}{3.076183in}}%
\pgfpathlineto{\pgfqpoint{2.421628in}{3.094145in}}%
\pgfpathlineto{\pgfqpoint{2.431756in}{3.095408in}}%
\pgfpathlineto{\pgfqpoint{2.436820in}{3.095998in}}%
\pgfpathlineto{\pgfqpoint{2.441884in}{3.098789in}}%
\pgfpathlineto{\pgfqpoint{2.446948in}{3.106818in}}%
\pgfpathlineto{\pgfqpoint{2.452012in}{3.109008in}}%
\pgfpathlineto{\pgfqpoint{2.457076in}{3.109288in}}%
\pgfpathlineto{\pgfqpoint{2.462140in}{3.112301in}}%
\pgfpathlineto{\pgfqpoint{2.472268in}{3.114332in}}%
\pgfpathlineto{\pgfqpoint{2.482396in}{3.118381in}}%
\pgfpathlineto{\pgfqpoint{2.492524in}{3.123875in}}%
\pgfpathlineto{\pgfqpoint{2.497588in}{3.129374in}}%
\pgfpathlineto{\pgfqpoint{2.502652in}{3.132948in}}%
\pgfpathlineto{\pgfqpoint{2.512780in}{3.134195in}}%
\pgfpathlineto{\pgfqpoint{2.522908in}{3.135700in}}%
\pgfpathlineto{\pgfqpoint{2.527971in}{3.144194in}}%
\pgfpathlineto{\pgfqpoint{2.533035in}{3.148851in}}%
\pgfpathlineto{\pgfqpoint{2.538099in}{3.156121in}}%
\pgfpathlineto{\pgfqpoint{2.543163in}{3.156293in}}%
\pgfpathlineto{\pgfqpoint{2.548227in}{3.158851in}}%
\pgfpathlineto{\pgfqpoint{2.553291in}{3.162834in}}%
\pgfpathlineto{\pgfqpoint{2.558355in}{3.163299in}}%
\pgfpathlineto{\pgfqpoint{2.563419in}{3.166149in}}%
\pgfpathlineto{\pgfqpoint{2.573547in}{3.180153in}}%
\pgfpathlineto{\pgfqpoint{2.578611in}{3.180668in}}%
\pgfpathlineto{\pgfqpoint{2.583675in}{3.201490in}}%
\pgfpathlineto{\pgfqpoint{2.588739in}{3.204877in}}%
\pgfpathlineto{\pgfqpoint{2.593803in}{3.212394in}}%
\pgfpathlineto{\pgfqpoint{2.603931in}{3.233313in}}%
\pgfpathlineto{\pgfqpoint{2.608995in}{3.233362in}}%
\pgfpathlineto{\pgfqpoint{2.614059in}{3.260091in}}%
\pgfpathlineto{\pgfqpoint{2.619123in}{3.260712in}}%
\pgfpathlineto{\pgfqpoint{2.624187in}{3.267020in}}%
\pgfpathlineto{\pgfqpoint{2.629251in}{3.268725in}}%
\pgfpathlineto{\pgfqpoint{2.634315in}{3.275745in}}%
\pgfpathlineto{\pgfqpoint{2.639379in}{3.276134in}}%
\pgfpathlineto{\pgfqpoint{2.644443in}{3.280483in}}%
\pgfpathlineto{\pgfqpoint{2.654571in}{3.283702in}}%
\pgfpathlineto{\pgfqpoint{2.664699in}{3.291665in}}%
\pgfpathlineto{\pgfqpoint{2.669763in}{3.296909in}}%
\pgfpathlineto{\pgfqpoint{2.684955in}{3.299025in}}%
\pgfpathlineto{\pgfqpoint{2.690019in}{3.301456in}}%
\pgfpathlineto{\pgfqpoint{2.695083in}{3.301553in}}%
\pgfpathlineto{\pgfqpoint{2.700147in}{3.307911in}}%
\pgfpathlineto{\pgfqpoint{2.705211in}{3.324008in}}%
\pgfpathlineto{\pgfqpoint{2.710275in}{3.330621in}}%
\pgfpathlineto{\pgfqpoint{2.720403in}{3.333429in}}%
\pgfpathlineto{\pgfqpoint{2.730531in}{3.339180in}}%
\pgfpathlineto{\pgfqpoint{2.735595in}{3.339475in}}%
\pgfpathlineto{\pgfqpoint{2.740659in}{3.349936in}}%
\pgfpathlineto{\pgfqpoint{2.745723in}{3.351285in}}%
\pgfpathlineto{\pgfqpoint{2.750787in}{3.358912in}}%
\pgfpathlineto{\pgfqpoint{2.755851in}{3.361603in}}%
\pgfpathlineto{\pgfqpoint{2.760915in}{3.368201in}}%
\pgfpathlineto{\pgfqpoint{2.765979in}{3.370326in}}%
\pgfpathlineto{\pgfqpoint{2.771043in}{3.379738in}}%
\pgfpathlineto{\pgfqpoint{2.776107in}{3.383352in}}%
\pgfpathlineto{\pgfqpoint{2.781171in}{3.392016in}}%
\pgfpathlineto{\pgfqpoint{2.786235in}{3.407779in}}%
\pgfpathlineto{\pgfqpoint{2.791299in}{3.417961in}}%
\pgfpathlineto{\pgfqpoint{2.796363in}{3.418251in}}%
\pgfpathlineto{\pgfqpoint{2.801427in}{3.421443in}}%
\pgfpathlineto{\pgfqpoint{2.806491in}{3.421490in}}%
\pgfpathlineto{\pgfqpoint{2.811555in}{3.443331in}}%
\pgfpathlineto{\pgfqpoint{2.821683in}{3.447278in}}%
\pgfpathlineto{\pgfqpoint{2.826747in}{3.449361in}}%
\pgfpathlineto{\pgfqpoint{2.836875in}{3.458547in}}%
\pgfpathlineto{\pgfqpoint{2.841939in}{3.487156in}}%
\pgfpathlineto{\pgfqpoint{2.847003in}{3.490673in}}%
\pgfpathlineto{\pgfqpoint{2.852066in}{3.513044in}}%
\pgfpathlineto{\pgfqpoint{2.857130in}{3.522678in}}%
\pgfpathlineto{\pgfqpoint{2.862194in}{3.528268in}}%
\pgfpathlineto{\pgfqpoint{2.867258in}{3.531115in}}%
\pgfpathlineto{\pgfqpoint{2.872322in}{3.540607in}}%
\pgfpathlineto{\pgfqpoint{2.877386in}{3.542016in}}%
\pgfpathlineto{\pgfqpoint{2.882450in}{3.545765in}}%
\pgfpathlineto{\pgfqpoint{2.892578in}{3.546342in}}%
\pgfpathlineto{\pgfqpoint{2.902706in}{3.554104in}}%
\pgfpathlineto{\pgfqpoint{2.912834in}{3.554458in}}%
\pgfpathlineto{\pgfqpoint{2.917898in}{3.557952in}}%
\pgfpathlineto{\pgfqpoint{2.922962in}{3.559731in}}%
\pgfpathlineto{\pgfqpoint{2.928026in}{3.575220in}}%
\pgfpathlineto{\pgfqpoint{2.933090in}{3.578552in}}%
\pgfpathlineto{\pgfqpoint{2.938154in}{3.578837in}}%
\pgfpathlineto{\pgfqpoint{2.943218in}{3.583997in}}%
\pgfpathlineto{\pgfqpoint{2.948282in}{3.587457in}}%
\pgfpathlineto{\pgfqpoint{2.953346in}{3.597707in}}%
\pgfpathlineto{\pgfqpoint{2.958410in}{3.598488in}}%
\pgfpathlineto{\pgfqpoint{2.963474in}{3.608013in}}%
\pgfpathlineto{\pgfqpoint{2.968538in}{3.621528in}}%
\pgfpathlineto{\pgfqpoint{2.978666in}{3.628953in}}%
\pgfpathlineto{\pgfqpoint{2.983730in}{3.634509in}}%
\pgfpathlineto{\pgfqpoint{2.993858in}{3.636900in}}%
\pgfpathlineto{\pgfqpoint{2.998922in}{3.657041in}}%
\pgfpathlineto{\pgfqpoint{3.003986in}{3.660280in}}%
\pgfpathlineto{\pgfqpoint{3.009050in}{3.665938in}}%
\pgfpathlineto{\pgfqpoint{3.014114in}{3.678181in}}%
\pgfpathlineto{\pgfqpoint{3.029306in}{3.682545in}}%
\pgfpathlineto{\pgfqpoint{3.039434in}{3.686525in}}%
\pgfpathlineto{\pgfqpoint{3.049562in}{3.705210in}}%
\pgfpathlineto{\pgfqpoint{3.054626in}{3.712245in}}%
\pgfpathlineto{\pgfqpoint{3.059690in}{3.723342in}}%
\pgfpathlineto{\pgfqpoint{3.064754in}{3.729315in}}%
\pgfpathlineto{\pgfqpoint{3.069818in}{3.739030in}}%
\pgfpathlineto{\pgfqpoint{3.074882in}{3.742452in}}%
\pgfpathlineto{\pgfqpoint{3.079946in}{3.743468in}}%
\pgfpathlineto{\pgfqpoint{3.085010in}{3.766213in}}%
\pgfpathlineto{\pgfqpoint{3.095138in}{3.772907in}}%
\pgfpathlineto{\pgfqpoint{3.105266in}{3.774299in}}%
\pgfpathlineto{\pgfqpoint{3.110330in}{3.777942in}}%
\pgfpathlineto{\pgfqpoint{3.120458in}{3.780827in}}%
\pgfpathlineto{\pgfqpoint{3.130586in}{3.781229in}}%
\pgfpathlineto{\pgfqpoint{3.140714in}{3.783762in}}%
\pgfpathlineto{\pgfqpoint{3.145778in}{3.788815in}}%
\pgfpathlineto{\pgfqpoint{3.155906in}{3.793507in}}%
\pgfpathlineto{\pgfqpoint{3.160970in}{3.795098in}}%
\pgfpathlineto{\pgfqpoint{3.166034in}{3.798395in}}%
\pgfpathlineto{\pgfqpoint{3.176161in}{3.799848in}}%
\pgfpathlineto{\pgfqpoint{3.181225in}{3.818307in}}%
\pgfpathlineto{\pgfqpoint{3.186289in}{3.818935in}}%
\pgfpathlineto{\pgfqpoint{3.191353in}{3.827736in}}%
\pgfpathlineto{\pgfqpoint{3.196417in}{3.842527in}}%
\pgfpathlineto{\pgfqpoint{3.201481in}{3.843559in}}%
\pgfpathlineto{\pgfqpoint{3.206545in}{3.895236in}}%
\pgfpathlineto{\pgfqpoint{3.211609in}{3.903177in}}%
\pgfpathlineto{\pgfqpoint{3.216673in}{3.909009in}}%
\pgfpathlineto{\pgfqpoint{3.221737in}{3.912217in}}%
\pgfpathlineto{\pgfqpoint{3.236929in}{3.927557in}}%
\pgfpathlineto{\pgfqpoint{3.247057in}{3.929341in}}%
\pgfpathlineto{\pgfqpoint{3.257185in}{3.930345in}}%
\pgfpathlineto{\pgfqpoint{3.262249in}{3.930550in}}%
\pgfpathlineto{\pgfqpoint{3.267313in}{3.937588in}}%
\pgfpathlineto{\pgfqpoint{3.272377in}{3.941032in}}%
\pgfpathlineto{\pgfqpoint{3.287569in}{3.943748in}}%
\pgfpathlineto{\pgfqpoint{3.292633in}{3.949950in}}%
\pgfpathlineto{\pgfqpoint{3.297697in}{3.958122in}}%
\pgfpathlineto{\pgfqpoint{3.302761in}{3.961716in}}%
\pgfpathlineto{\pgfqpoint{3.307825in}{3.968210in}}%
\pgfpathlineto{\pgfqpoint{3.312889in}{3.978713in}}%
\pgfpathlineto{\pgfqpoint{3.317953in}{3.986231in}}%
\pgfpathlineto{\pgfqpoint{3.323017in}{3.986415in}}%
\pgfpathlineto{\pgfqpoint{3.328081in}{3.993303in}}%
\pgfpathlineto{\pgfqpoint{3.333145in}{4.010943in}}%
\pgfpathlineto{\pgfqpoint{3.338209in}{4.017028in}}%
\pgfpathlineto{\pgfqpoint{3.343273in}{4.018366in}}%
\pgfpathlineto{\pgfqpoint{3.348337in}{4.026804in}}%
\pgfpathlineto{\pgfqpoint{3.358465in}{4.037535in}}%
\pgfpathlineto{\pgfqpoint{3.363529in}{4.064153in}}%
\pgfpathlineto{\pgfqpoint{3.368593in}{4.072989in}}%
\pgfpathlineto{\pgfqpoint{3.383785in}{4.082312in}}%
\pgfpathlineto{\pgfqpoint{3.393913in}{4.083349in}}%
\pgfpathlineto{\pgfqpoint{3.398977in}{4.088315in}}%
\pgfpathlineto{\pgfqpoint{3.409105in}{4.089852in}}%
\pgfpathlineto{\pgfqpoint{3.424297in}{4.102840in}}%
\pgfpathlineto{\pgfqpoint{3.439489in}{4.105859in}}%
\pgfpathlineto{\pgfqpoint{3.444553in}{4.108084in}}%
\pgfpathlineto{\pgfqpoint{3.449617in}{4.108275in}}%
\pgfpathlineto{\pgfqpoint{3.454681in}{4.116944in}}%
\pgfpathlineto{\pgfqpoint{3.459745in}{4.129963in}}%
\pgfpathlineto{\pgfqpoint{3.464809in}{4.134442in}}%
\pgfpathlineto{\pgfqpoint{3.469873in}{4.136368in}}%
\pgfpathlineto{\pgfqpoint{3.474937in}{4.136517in}}%
\pgfpathlineto{\pgfqpoint{3.480001in}{4.145765in}}%
\pgfpathlineto{\pgfqpoint{3.485065in}{4.148837in}}%
\pgfpathlineto{\pgfqpoint{3.490129in}{4.149246in}}%
\pgfpathlineto{\pgfqpoint{3.495193in}{4.156003in}}%
\pgfpathlineto{\pgfqpoint{3.500256in}{4.179392in}}%
\pgfpathlineto{\pgfqpoint{3.505320in}{4.183709in}}%
\pgfpathlineto{\pgfqpoint{3.510384in}{4.185903in}}%
\pgfpathlineto{\pgfqpoint{3.520512in}{4.194626in}}%
\pgfpathlineto{\pgfqpoint{3.525576in}{4.197123in}}%
\pgfpathlineto{\pgfqpoint{3.530640in}{4.202467in}}%
\pgfpathlineto{\pgfqpoint{3.535704in}{4.204260in}}%
\pgfpathlineto{\pgfqpoint{3.545832in}{4.220537in}}%
\pgfpathlineto{\pgfqpoint{3.550896in}{4.224292in}}%
\pgfpathlineto{\pgfqpoint{3.555960in}{4.224555in}}%
\pgfpathlineto{\pgfqpoint{3.561024in}{4.245688in}}%
\pgfpathlineto{\pgfqpoint{3.566088in}{4.261678in}}%
\pgfpathlineto{\pgfqpoint{3.571152in}{4.269823in}}%
\pgfpathlineto{\pgfqpoint{3.576216in}{4.273350in}}%
\pgfpathlineto{\pgfqpoint{3.581280in}{4.274743in}}%
\pgfpathlineto{\pgfqpoint{3.591408in}{4.304500in}}%
\pgfpathlineto{\pgfqpoint{3.596472in}{4.305765in}}%
\pgfpathlineto{\pgfqpoint{3.606600in}{4.362527in}}%
\pgfpathlineto{\pgfqpoint{3.616728in}{4.374710in}}%
\pgfpathlineto{\pgfqpoint{3.621792in}{4.378091in}}%
\pgfpathlineto{\pgfqpoint{3.631920in}{4.409416in}}%
\pgfpathlineto{\pgfqpoint{3.636984in}{4.416322in}}%
\pgfpathlineto{\pgfqpoint{3.642048in}{4.427331in}}%
\pgfpathlineto{\pgfqpoint{3.652176in}{4.429728in}}%
\pgfpathlineto{\pgfqpoint{3.657240in}{4.430227in}}%
\pgfpathlineto{\pgfqpoint{3.667368in}{4.432881in}}%
\pgfpathlineto{\pgfqpoint{3.682560in}{4.436992in}}%
\pgfpathlineto{\pgfqpoint{3.707880in}{4.438619in}}%
\pgfpathlineto{\pgfqpoint{3.712944in}{4.445013in}}%
\pgfpathlineto{\pgfqpoint{3.723072in}{4.446824in}}%
\pgfpathlineto{\pgfqpoint{3.728136in}{4.449806in}}%
\pgfpathlineto{\pgfqpoint{3.748392in}{4.454389in}}%
\pgfpathlineto{\pgfqpoint{3.753456in}{4.463805in}}%
\pgfpathlineto{\pgfqpoint{3.758520in}{4.470028in}}%
\pgfpathlineto{\pgfqpoint{3.763584in}{4.470217in}}%
\pgfpathlineto{\pgfqpoint{3.768648in}{4.488286in}}%
\pgfpathlineto{\pgfqpoint{3.773712in}{4.501442in}}%
\pgfpathlineto{\pgfqpoint{3.778776in}{4.524319in}}%
\pgfpathlineto{\pgfqpoint{3.793968in}{4.527208in}}%
\pgfpathlineto{\pgfqpoint{3.799032in}{4.530922in}}%
\pgfpathlineto{\pgfqpoint{3.804096in}{4.539871in}}%
\pgfpathlineto{\pgfqpoint{3.809160in}{4.568011in}}%
\pgfpathlineto{\pgfqpoint{3.814224in}{4.568808in}}%
\pgfpathlineto{\pgfqpoint{3.819288in}{4.571676in}}%
\pgfpathlineto{\pgfqpoint{3.824351in}{4.597238in}}%
\pgfpathlineto{\pgfqpoint{6.244936in}{4.597238in}}%
\pgfpathlineto{\pgfqpoint{6.244936in}{4.597238in}}%
\pgfusepath{stroke}%
\end{pgfscope}%
\begin{pgfscope}%
\pgfpathrectangle{\pgfqpoint{0.725193in}{0.571603in}}{\pgfqpoint{5.524807in}{4.025635in}}%
\pgfusepath{clip}%
\pgfsetrectcap%
\pgfsetroundjoin%
\pgfsetlinewidth{2.007500pt}%
\definecolor{currentstroke}{rgb}{0.980392,0.529412,0.458824}%
\pgfsetstrokecolor{currentstroke}%
\pgfsetdash{}{0pt}%
\pgfpathmoveto{\pgfqpoint{0.725193in}{2.273495in}}%
\pgfpathlineto{\pgfqpoint{0.730257in}{2.340731in}}%
\pgfpathlineto{\pgfqpoint{0.735321in}{2.343232in}}%
\pgfpathlineto{\pgfqpoint{0.740385in}{2.349850in}}%
\pgfpathlineto{\pgfqpoint{0.745449in}{2.354097in}}%
\pgfpathlineto{\pgfqpoint{0.750513in}{2.371813in}}%
\pgfpathlineto{\pgfqpoint{0.755577in}{2.374314in}}%
\pgfpathlineto{\pgfqpoint{0.760641in}{2.439897in}}%
\pgfpathlineto{\pgfqpoint{0.765705in}{2.448261in}}%
\pgfpathlineto{\pgfqpoint{0.770769in}{2.449853in}}%
\pgfpathlineto{\pgfqpoint{0.780897in}{2.451423in}}%
\pgfpathlineto{\pgfqpoint{0.791025in}{2.456915in}}%
\pgfpathlineto{\pgfqpoint{0.796089in}{2.457510in}}%
\pgfpathlineto{\pgfqpoint{0.801153in}{2.459524in}}%
\pgfpathlineto{\pgfqpoint{0.806217in}{2.460103in}}%
\pgfpathlineto{\pgfqpoint{0.811281in}{2.465231in}}%
\pgfpathlineto{\pgfqpoint{0.816345in}{2.489410in}}%
\pgfpathlineto{\pgfqpoint{0.821409in}{2.495636in}}%
\pgfpathlineto{\pgfqpoint{0.841665in}{2.501074in}}%
\pgfpathlineto{\pgfqpoint{0.846729in}{2.507539in}}%
\pgfpathlineto{\pgfqpoint{0.851793in}{2.509494in}}%
\pgfpathlineto{\pgfqpoint{0.856857in}{2.518689in}}%
\pgfpathlineto{\pgfqpoint{0.861921in}{2.523230in}}%
\pgfpathlineto{\pgfqpoint{0.866985in}{2.525720in}}%
\pgfpathlineto{\pgfqpoint{0.872049in}{2.529825in}}%
\pgfpathlineto{\pgfqpoint{0.882177in}{2.531599in}}%
\pgfpathlineto{\pgfqpoint{0.907496in}{2.536621in}}%
\pgfpathlineto{\pgfqpoint{0.912560in}{2.540524in}}%
\pgfpathlineto{\pgfqpoint{0.922688in}{2.541933in}}%
\pgfpathlineto{\pgfqpoint{0.927752in}{2.543478in}}%
\pgfpathlineto{\pgfqpoint{0.932816in}{2.549919in}}%
\pgfpathlineto{\pgfqpoint{0.937880in}{2.551846in}}%
\pgfpathlineto{\pgfqpoint{0.942944in}{2.552350in}}%
\pgfpathlineto{\pgfqpoint{0.948008in}{2.555011in}}%
\pgfpathlineto{\pgfqpoint{0.953072in}{2.564324in}}%
\pgfpathlineto{\pgfqpoint{0.958136in}{2.564493in}}%
\pgfpathlineto{\pgfqpoint{0.973328in}{2.569038in}}%
\pgfpathlineto{\pgfqpoint{0.988520in}{2.570522in}}%
\pgfpathlineto{\pgfqpoint{0.993584in}{2.570831in}}%
\pgfpathlineto{\pgfqpoint{1.003712in}{2.578553in}}%
\pgfpathlineto{\pgfqpoint{1.008776in}{2.624165in}}%
\pgfpathlineto{\pgfqpoint{1.013840in}{2.631457in}}%
\pgfpathlineto{\pgfqpoint{1.018904in}{2.634828in}}%
\pgfpathlineto{\pgfqpoint{1.029032in}{2.637022in}}%
\pgfpathlineto{\pgfqpoint{1.034096in}{2.659537in}}%
\pgfpathlineto{\pgfqpoint{1.049288in}{2.663408in}}%
\pgfpathlineto{\pgfqpoint{1.059416in}{2.670157in}}%
\pgfpathlineto{\pgfqpoint{1.064480in}{2.675566in}}%
\pgfpathlineto{\pgfqpoint{1.069544in}{2.677734in}}%
\pgfpathlineto{\pgfqpoint{1.074608in}{2.678220in}}%
\pgfpathlineto{\pgfqpoint{1.079672in}{2.682864in}}%
\pgfpathlineto{\pgfqpoint{1.084736in}{2.691846in}}%
\pgfpathlineto{\pgfqpoint{1.094864in}{2.695699in}}%
\pgfpathlineto{\pgfqpoint{1.099928in}{2.698398in}}%
\pgfpathlineto{\pgfqpoint{1.104992in}{2.706548in}}%
\pgfpathlineto{\pgfqpoint{1.125248in}{2.710976in}}%
\pgfpathlineto{\pgfqpoint{1.135376in}{2.718979in}}%
\pgfpathlineto{\pgfqpoint{1.140440in}{2.729722in}}%
\pgfpathlineto{\pgfqpoint{1.145504in}{2.733256in}}%
\pgfpathlineto{\pgfqpoint{1.150568in}{2.752048in}}%
\pgfpathlineto{\pgfqpoint{1.155632in}{2.752811in}}%
\pgfpathlineto{\pgfqpoint{1.160696in}{2.758469in}}%
\pgfpathlineto{\pgfqpoint{1.165760in}{2.780077in}}%
\pgfpathlineto{\pgfqpoint{1.170824in}{2.784337in}}%
\pgfpathlineto{\pgfqpoint{1.175888in}{2.784476in}}%
\pgfpathlineto{\pgfqpoint{1.186016in}{2.788007in}}%
\pgfpathlineto{\pgfqpoint{1.196144in}{2.794612in}}%
\pgfpathlineto{\pgfqpoint{1.201208in}{2.797797in}}%
\pgfpathlineto{\pgfqpoint{1.206272in}{2.798690in}}%
\pgfpathlineto{\pgfqpoint{1.211336in}{2.806443in}}%
\pgfpathlineto{\pgfqpoint{1.221464in}{2.809804in}}%
\pgfpathlineto{\pgfqpoint{1.231591in}{2.813449in}}%
\pgfpathlineto{\pgfqpoint{1.236655in}{2.817196in}}%
\pgfpathlineto{\pgfqpoint{1.241719in}{2.817490in}}%
\pgfpathlineto{\pgfqpoint{1.246783in}{2.825029in}}%
\pgfpathlineto{\pgfqpoint{1.251847in}{2.825467in}}%
\pgfpathlineto{\pgfqpoint{1.261975in}{2.831571in}}%
\pgfpathlineto{\pgfqpoint{1.267039in}{2.836691in}}%
\pgfpathlineto{\pgfqpoint{1.272103in}{2.839469in}}%
\pgfpathlineto{\pgfqpoint{1.277167in}{2.846033in}}%
\pgfpathlineto{\pgfqpoint{1.282231in}{2.846572in}}%
\pgfpathlineto{\pgfqpoint{1.297423in}{2.864268in}}%
\pgfpathlineto{\pgfqpoint{1.302487in}{2.864623in}}%
\pgfpathlineto{\pgfqpoint{1.307551in}{2.866312in}}%
\pgfpathlineto{\pgfqpoint{1.312615in}{2.866808in}}%
\pgfpathlineto{\pgfqpoint{1.317679in}{2.873941in}}%
\pgfpathlineto{\pgfqpoint{1.322743in}{2.876479in}}%
\pgfpathlineto{\pgfqpoint{1.327807in}{2.881005in}}%
\pgfpathlineto{\pgfqpoint{1.332871in}{2.882066in}}%
\pgfpathlineto{\pgfqpoint{1.337935in}{2.900109in}}%
\pgfpathlineto{\pgfqpoint{1.342999in}{2.906806in}}%
\pgfpathlineto{\pgfqpoint{1.348063in}{2.909510in}}%
\pgfpathlineto{\pgfqpoint{1.353127in}{2.914798in}}%
\pgfpathlineto{\pgfqpoint{1.358191in}{2.917687in}}%
\pgfpathlineto{\pgfqpoint{1.363255in}{2.923980in}}%
\pgfpathlineto{\pgfqpoint{1.368319in}{2.925369in}}%
\pgfpathlineto{\pgfqpoint{1.373383in}{2.925434in}}%
\pgfpathlineto{\pgfqpoint{1.378447in}{2.929462in}}%
\pgfpathlineto{\pgfqpoint{1.388575in}{2.930448in}}%
\pgfpathlineto{\pgfqpoint{1.408831in}{2.938363in}}%
\pgfpathlineto{\pgfqpoint{1.413895in}{2.941152in}}%
\pgfpathlineto{\pgfqpoint{1.418959in}{2.947199in}}%
\pgfpathlineto{\pgfqpoint{1.424023in}{2.949857in}}%
\pgfpathlineto{\pgfqpoint{1.429087in}{2.955221in}}%
\pgfpathlineto{\pgfqpoint{1.434151in}{2.955406in}}%
\pgfpathlineto{\pgfqpoint{1.444279in}{2.959250in}}%
\pgfpathlineto{\pgfqpoint{1.449343in}{2.963027in}}%
\pgfpathlineto{\pgfqpoint{1.454407in}{2.965294in}}%
\pgfpathlineto{\pgfqpoint{1.459471in}{2.965937in}}%
\pgfpathlineto{\pgfqpoint{1.464535in}{2.968271in}}%
\pgfpathlineto{\pgfqpoint{1.474663in}{2.970254in}}%
\pgfpathlineto{\pgfqpoint{1.479727in}{2.973415in}}%
\pgfpathlineto{\pgfqpoint{1.484791in}{2.979659in}}%
\pgfpathlineto{\pgfqpoint{1.489855in}{2.983634in}}%
\pgfpathlineto{\pgfqpoint{1.494919in}{2.992490in}}%
\pgfpathlineto{\pgfqpoint{1.505047in}{2.997737in}}%
\pgfpathlineto{\pgfqpoint{1.510111in}{2.998408in}}%
\pgfpathlineto{\pgfqpoint{1.515175in}{3.004872in}}%
\pgfpathlineto{\pgfqpoint{1.520239in}{3.017510in}}%
\pgfpathlineto{\pgfqpoint{1.525303in}{3.018826in}}%
\pgfpathlineto{\pgfqpoint{1.535431in}{3.025109in}}%
\pgfpathlineto{\pgfqpoint{1.540495in}{3.025522in}}%
\pgfpathlineto{\pgfqpoint{1.545559in}{3.028885in}}%
\pgfpathlineto{\pgfqpoint{1.560750in}{3.031694in}}%
\pgfpathlineto{\pgfqpoint{1.565814in}{3.033262in}}%
\pgfpathlineto{\pgfqpoint{1.570878in}{3.036046in}}%
\pgfpathlineto{\pgfqpoint{1.606326in}{3.040336in}}%
\pgfpathlineto{\pgfqpoint{1.616454in}{3.042525in}}%
\pgfpathlineto{\pgfqpoint{1.631646in}{3.045189in}}%
\pgfpathlineto{\pgfqpoint{1.641774in}{3.047636in}}%
\pgfpathlineto{\pgfqpoint{1.651902in}{3.048371in}}%
\pgfpathlineto{\pgfqpoint{1.656966in}{3.050956in}}%
\pgfpathlineto{\pgfqpoint{1.662030in}{3.056733in}}%
\pgfpathlineto{\pgfqpoint{1.697478in}{3.074684in}}%
\pgfpathlineto{\pgfqpoint{1.702542in}{3.074716in}}%
\pgfpathlineto{\pgfqpoint{1.712670in}{3.081767in}}%
\pgfpathlineto{\pgfqpoint{1.722798in}{3.085670in}}%
\pgfpathlineto{\pgfqpoint{1.727862in}{3.090941in}}%
\pgfpathlineto{\pgfqpoint{1.743054in}{3.096032in}}%
\pgfpathlineto{\pgfqpoint{1.783566in}{3.099856in}}%
\pgfpathlineto{\pgfqpoint{1.788630in}{3.100172in}}%
\pgfpathlineto{\pgfqpoint{1.793694in}{3.101725in}}%
\pgfpathlineto{\pgfqpoint{1.798758in}{3.101770in}}%
\pgfpathlineto{\pgfqpoint{1.803822in}{3.103908in}}%
\pgfpathlineto{\pgfqpoint{1.819014in}{3.119152in}}%
\pgfpathlineto{\pgfqpoint{1.824078in}{3.119940in}}%
\pgfpathlineto{\pgfqpoint{1.829142in}{3.122685in}}%
\pgfpathlineto{\pgfqpoint{1.839270in}{3.125314in}}%
\pgfpathlineto{\pgfqpoint{1.844334in}{3.127181in}}%
\pgfpathlineto{\pgfqpoint{1.849398in}{3.139183in}}%
\pgfpathlineto{\pgfqpoint{1.854462in}{3.139807in}}%
\pgfpathlineto{\pgfqpoint{1.874718in}{3.149746in}}%
\pgfpathlineto{\pgfqpoint{1.879781in}{3.151303in}}%
\pgfpathlineto{\pgfqpoint{1.889909in}{3.158293in}}%
\pgfpathlineto{\pgfqpoint{1.894973in}{3.159227in}}%
\pgfpathlineto{\pgfqpoint{1.900037in}{3.161947in}}%
\pgfpathlineto{\pgfqpoint{1.920293in}{3.164208in}}%
\pgfpathlineto{\pgfqpoint{1.925357in}{3.165142in}}%
\pgfpathlineto{\pgfqpoint{1.930421in}{3.167293in}}%
\pgfpathlineto{\pgfqpoint{1.945613in}{3.168959in}}%
\pgfpathlineto{\pgfqpoint{1.950677in}{3.176335in}}%
\pgfpathlineto{\pgfqpoint{1.955741in}{3.185737in}}%
\pgfpathlineto{\pgfqpoint{1.975997in}{3.190118in}}%
\pgfpathlineto{\pgfqpoint{1.991189in}{3.191221in}}%
\pgfpathlineto{\pgfqpoint{1.996253in}{3.194854in}}%
\pgfpathlineto{\pgfqpoint{2.026637in}{3.202516in}}%
\pgfpathlineto{\pgfqpoint{2.031701in}{3.205456in}}%
\pgfpathlineto{\pgfqpoint{2.036765in}{3.205614in}}%
\pgfpathlineto{\pgfqpoint{2.046893in}{3.215307in}}%
\pgfpathlineto{\pgfqpoint{2.051957in}{3.217551in}}%
\pgfpathlineto{\pgfqpoint{2.057021in}{3.223235in}}%
\pgfpathlineto{\pgfqpoint{2.062085in}{3.225414in}}%
\pgfpathlineto{\pgfqpoint{2.067149in}{3.235713in}}%
\pgfpathlineto{\pgfqpoint{2.072213in}{3.264332in}}%
\pgfpathlineto{\pgfqpoint{2.082341in}{3.268192in}}%
\pgfpathlineto{\pgfqpoint{2.092469in}{3.270123in}}%
\pgfpathlineto{\pgfqpoint{2.102597in}{3.270803in}}%
\pgfpathlineto{\pgfqpoint{2.107661in}{3.273911in}}%
\pgfpathlineto{\pgfqpoint{2.112725in}{3.274949in}}%
\pgfpathlineto{\pgfqpoint{2.117789in}{3.288748in}}%
\pgfpathlineto{\pgfqpoint{2.143109in}{3.295880in}}%
\pgfpathlineto{\pgfqpoint{2.148173in}{3.296597in}}%
\pgfpathlineto{\pgfqpoint{2.163365in}{3.314642in}}%
\pgfpathlineto{\pgfqpoint{2.168429in}{3.315809in}}%
\pgfpathlineto{\pgfqpoint{2.173493in}{3.320754in}}%
\pgfpathlineto{\pgfqpoint{2.203876in}{3.324597in}}%
\pgfpathlineto{\pgfqpoint{2.219068in}{3.328547in}}%
\pgfpathlineto{\pgfqpoint{2.224132in}{3.335064in}}%
\pgfpathlineto{\pgfqpoint{2.229196in}{3.337102in}}%
\pgfpathlineto{\pgfqpoint{2.239324in}{3.343368in}}%
\pgfpathlineto{\pgfqpoint{2.244388in}{3.349020in}}%
\pgfpathlineto{\pgfqpoint{2.264644in}{3.360923in}}%
\pgfpathlineto{\pgfqpoint{2.269708in}{3.361541in}}%
\pgfpathlineto{\pgfqpoint{2.274772in}{3.371914in}}%
\pgfpathlineto{\pgfqpoint{2.279836in}{3.373157in}}%
\pgfpathlineto{\pgfqpoint{2.300092in}{3.388629in}}%
\pgfpathlineto{\pgfqpoint{2.315284in}{3.390109in}}%
\pgfpathlineto{\pgfqpoint{2.325412in}{3.392840in}}%
\pgfpathlineto{\pgfqpoint{2.335540in}{3.403313in}}%
\pgfpathlineto{\pgfqpoint{2.345668in}{3.405102in}}%
\pgfpathlineto{\pgfqpoint{2.350732in}{3.409309in}}%
\pgfpathlineto{\pgfqpoint{2.355796in}{3.410484in}}%
\pgfpathlineto{\pgfqpoint{2.360860in}{3.413957in}}%
\pgfpathlineto{\pgfqpoint{2.370988in}{3.416455in}}%
\pgfpathlineto{\pgfqpoint{2.386180in}{3.417629in}}%
\pgfpathlineto{\pgfqpoint{2.391244in}{3.417843in}}%
\pgfpathlineto{\pgfqpoint{2.401372in}{3.427388in}}%
\pgfpathlineto{\pgfqpoint{2.411500in}{3.428860in}}%
\pgfpathlineto{\pgfqpoint{2.416564in}{3.436081in}}%
\pgfpathlineto{\pgfqpoint{2.426692in}{3.436363in}}%
\pgfpathlineto{\pgfqpoint{2.431756in}{3.438318in}}%
\pgfpathlineto{\pgfqpoint{2.446948in}{3.438837in}}%
\pgfpathlineto{\pgfqpoint{2.457076in}{3.442628in}}%
\pgfpathlineto{\pgfqpoint{2.472268in}{3.444890in}}%
\pgfpathlineto{\pgfqpoint{2.477332in}{3.452469in}}%
\pgfpathlineto{\pgfqpoint{2.482396in}{3.453410in}}%
\pgfpathlineto{\pgfqpoint{2.487460in}{3.456868in}}%
\pgfpathlineto{\pgfqpoint{2.497588in}{3.460834in}}%
\pgfpathlineto{\pgfqpoint{2.502652in}{3.467751in}}%
\pgfpathlineto{\pgfqpoint{2.512780in}{3.467980in}}%
\pgfpathlineto{\pgfqpoint{2.522908in}{3.472331in}}%
\pgfpathlineto{\pgfqpoint{2.527971in}{3.476844in}}%
\pgfpathlineto{\pgfqpoint{2.533035in}{3.478453in}}%
\pgfpathlineto{\pgfqpoint{2.543163in}{3.479349in}}%
\pgfpathlineto{\pgfqpoint{2.548227in}{3.482032in}}%
\pgfpathlineto{\pgfqpoint{2.553291in}{3.483375in}}%
\pgfpathlineto{\pgfqpoint{2.558355in}{3.486923in}}%
\pgfpathlineto{\pgfqpoint{2.563419in}{3.487718in}}%
\pgfpathlineto{\pgfqpoint{2.568483in}{3.490459in}}%
\pgfpathlineto{\pgfqpoint{2.578611in}{3.492068in}}%
\pgfpathlineto{\pgfqpoint{2.583675in}{3.496451in}}%
\pgfpathlineto{\pgfqpoint{2.598867in}{3.499722in}}%
\pgfpathlineto{\pgfqpoint{2.603931in}{3.502899in}}%
\pgfpathlineto{\pgfqpoint{2.608995in}{3.503019in}}%
\pgfpathlineto{\pgfqpoint{2.614059in}{3.506959in}}%
\pgfpathlineto{\pgfqpoint{2.619123in}{3.506989in}}%
\pgfpathlineto{\pgfqpoint{2.624187in}{3.508903in}}%
\pgfpathlineto{\pgfqpoint{2.629251in}{3.521021in}}%
\pgfpathlineto{\pgfqpoint{2.634315in}{3.528455in}}%
\pgfpathlineto{\pgfqpoint{2.639379in}{3.532529in}}%
\pgfpathlineto{\pgfqpoint{2.644443in}{3.533799in}}%
\pgfpathlineto{\pgfqpoint{2.649507in}{3.533869in}}%
\pgfpathlineto{\pgfqpoint{2.659635in}{3.536703in}}%
\pgfpathlineto{\pgfqpoint{2.664699in}{3.538873in}}%
\pgfpathlineto{\pgfqpoint{2.684955in}{3.542307in}}%
\pgfpathlineto{\pgfqpoint{2.700147in}{3.545125in}}%
\pgfpathlineto{\pgfqpoint{2.705211in}{3.545757in}}%
\pgfpathlineto{\pgfqpoint{2.710275in}{3.549288in}}%
\pgfpathlineto{\pgfqpoint{2.715339in}{3.551047in}}%
\pgfpathlineto{\pgfqpoint{2.720403in}{3.554026in}}%
\pgfpathlineto{\pgfqpoint{2.725467in}{3.554568in}}%
\pgfpathlineto{\pgfqpoint{2.730531in}{3.556614in}}%
\pgfpathlineto{\pgfqpoint{2.781171in}{3.563722in}}%
\pgfpathlineto{\pgfqpoint{2.791299in}{3.579501in}}%
\pgfpathlineto{\pgfqpoint{2.801427in}{3.583266in}}%
\pgfpathlineto{\pgfqpoint{2.816619in}{3.584429in}}%
\pgfpathlineto{\pgfqpoint{2.826747in}{3.590227in}}%
\pgfpathlineto{\pgfqpoint{2.847003in}{3.592738in}}%
\pgfpathlineto{\pgfqpoint{2.862194in}{3.614614in}}%
\pgfpathlineto{\pgfqpoint{2.867258in}{3.615041in}}%
\pgfpathlineto{\pgfqpoint{2.872322in}{3.616626in}}%
\pgfpathlineto{\pgfqpoint{2.877386in}{3.621564in}}%
\pgfpathlineto{\pgfqpoint{2.887514in}{3.624469in}}%
\pgfpathlineto{\pgfqpoint{2.892578in}{3.631811in}}%
\pgfpathlineto{\pgfqpoint{2.902706in}{3.637172in}}%
\pgfpathlineto{\pgfqpoint{2.907770in}{3.639390in}}%
\pgfpathlineto{\pgfqpoint{2.912834in}{3.646550in}}%
\pgfpathlineto{\pgfqpoint{2.917898in}{3.648682in}}%
\pgfpathlineto{\pgfqpoint{2.922962in}{3.648795in}}%
\pgfpathlineto{\pgfqpoint{2.928026in}{3.651876in}}%
\pgfpathlineto{\pgfqpoint{2.953346in}{3.655409in}}%
\pgfpathlineto{\pgfqpoint{2.968538in}{3.657974in}}%
\pgfpathlineto{\pgfqpoint{2.973602in}{3.659713in}}%
\pgfpathlineto{\pgfqpoint{2.998922in}{3.661509in}}%
\pgfpathlineto{\pgfqpoint{3.003986in}{3.667602in}}%
\pgfpathlineto{\pgfqpoint{3.024242in}{3.673080in}}%
\pgfpathlineto{\pgfqpoint{3.029306in}{3.673876in}}%
\pgfpathlineto{\pgfqpoint{3.039434in}{3.678700in}}%
\pgfpathlineto{\pgfqpoint{3.044498in}{3.679024in}}%
\pgfpathlineto{\pgfqpoint{3.074882in}{3.692117in}}%
\pgfpathlineto{\pgfqpoint{3.079946in}{3.700413in}}%
\pgfpathlineto{\pgfqpoint{3.085010in}{3.729151in}}%
\pgfpathlineto{\pgfqpoint{3.090074in}{3.730798in}}%
\pgfpathlineto{\pgfqpoint{3.095138in}{3.735881in}}%
\pgfpathlineto{\pgfqpoint{3.100202in}{3.737259in}}%
\pgfpathlineto{\pgfqpoint{3.105266in}{3.744999in}}%
\pgfpathlineto{\pgfqpoint{3.110330in}{3.747330in}}%
\pgfpathlineto{\pgfqpoint{3.115394in}{3.757796in}}%
\pgfpathlineto{\pgfqpoint{3.120458in}{3.759434in}}%
\pgfpathlineto{\pgfqpoint{3.130586in}{3.768777in}}%
\pgfpathlineto{\pgfqpoint{3.135650in}{3.772635in}}%
\pgfpathlineto{\pgfqpoint{3.140714in}{3.782535in}}%
\pgfpathlineto{\pgfqpoint{3.171098in}{3.787253in}}%
\pgfpathlineto{\pgfqpoint{3.176161in}{3.800570in}}%
\pgfpathlineto{\pgfqpoint{3.181225in}{3.808346in}}%
\pgfpathlineto{\pgfqpoint{3.196417in}{3.810577in}}%
\pgfpathlineto{\pgfqpoint{3.201481in}{3.814725in}}%
\pgfpathlineto{\pgfqpoint{3.216673in}{3.822222in}}%
\pgfpathlineto{\pgfqpoint{3.221737in}{3.830364in}}%
\pgfpathlineto{\pgfqpoint{3.241993in}{3.834171in}}%
\pgfpathlineto{\pgfqpoint{3.257185in}{3.842876in}}%
\pgfpathlineto{\pgfqpoint{3.267313in}{3.845244in}}%
\pgfpathlineto{\pgfqpoint{3.272377in}{3.845441in}}%
\pgfpathlineto{\pgfqpoint{3.277441in}{3.854004in}}%
\pgfpathlineto{\pgfqpoint{3.282505in}{3.858316in}}%
\pgfpathlineto{\pgfqpoint{3.287569in}{3.859785in}}%
\pgfpathlineto{\pgfqpoint{3.292633in}{3.859944in}}%
\pgfpathlineto{\pgfqpoint{3.297697in}{3.861277in}}%
\pgfpathlineto{\pgfqpoint{3.307825in}{3.866455in}}%
\pgfpathlineto{\pgfqpoint{3.323017in}{3.867597in}}%
\pgfpathlineto{\pgfqpoint{3.333145in}{3.868182in}}%
\pgfpathlineto{\pgfqpoint{3.338209in}{3.869484in}}%
\pgfpathlineto{\pgfqpoint{3.343273in}{3.871969in}}%
\pgfpathlineto{\pgfqpoint{3.348337in}{3.876923in}}%
\pgfpathlineto{\pgfqpoint{3.363529in}{3.883703in}}%
\pgfpathlineto{\pgfqpoint{3.368593in}{3.883730in}}%
\pgfpathlineto{\pgfqpoint{3.388849in}{3.888976in}}%
\pgfpathlineto{\pgfqpoint{3.393913in}{3.892782in}}%
\pgfpathlineto{\pgfqpoint{3.398977in}{3.898104in}}%
\pgfpathlineto{\pgfqpoint{3.404041in}{3.899647in}}%
\pgfpathlineto{\pgfqpoint{3.409105in}{3.907509in}}%
\pgfpathlineto{\pgfqpoint{3.414169in}{3.910617in}}%
\pgfpathlineto{\pgfqpoint{3.419233in}{3.921126in}}%
\pgfpathlineto{\pgfqpoint{3.429361in}{3.925908in}}%
\pgfpathlineto{\pgfqpoint{3.434425in}{3.926695in}}%
\pgfpathlineto{\pgfqpoint{3.439489in}{3.932179in}}%
\pgfpathlineto{\pgfqpoint{3.449617in}{3.933425in}}%
\pgfpathlineto{\pgfqpoint{3.454681in}{3.937599in}}%
\pgfpathlineto{\pgfqpoint{3.459745in}{3.938816in}}%
\pgfpathlineto{\pgfqpoint{3.464809in}{3.942439in}}%
\pgfpathlineto{\pgfqpoint{3.474937in}{3.944872in}}%
\pgfpathlineto{\pgfqpoint{3.490129in}{3.947401in}}%
\pgfpathlineto{\pgfqpoint{3.495193in}{3.950862in}}%
\pgfpathlineto{\pgfqpoint{3.500256in}{3.951753in}}%
\pgfpathlineto{\pgfqpoint{3.505320in}{3.954340in}}%
\pgfpathlineto{\pgfqpoint{3.510384in}{3.955451in}}%
\pgfpathlineto{\pgfqpoint{3.515448in}{3.961076in}}%
\pgfpathlineto{\pgfqpoint{3.520512in}{3.962015in}}%
\pgfpathlineto{\pgfqpoint{3.525576in}{3.967850in}}%
\pgfpathlineto{\pgfqpoint{3.535704in}{3.969006in}}%
\pgfpathlineto{\pgfqpoint{3.540768in}{3.974089in}}%
\pgfpathlineto{\pgfqpoint{3.545832in}{3.983652in}}%
\pgfpathlineto{\pgfqpoint{3.550896in}{3.988296in}}%
\pgfpathlineto{\pgfqpoint{3.555960in}{3.990996in}}%
\pgfpathlineto{\pgfqpoint{3.561024in}{3.991693in}}%
\pgfpathlineto{\pgfqpoint{3.566088in}{3.996322in}}%
\pgfpathlineto{\pgfqpoint{3.576216in}{3.997348in}}%
\pgfpathlineto{\pgfqpoint{3.581280in}{3.998110in}}%
\pgfpathlineto{\pgfqpoint{3.591408in}{4.012166in}}%
\pgfpathlineto{\pgfqpoint{3.596472in}{4.014011in}}%
\pgfpathlineto{\pgfqpoint{3.601536in}{4.017183in}}%
\pgfpathlineto{\pgfqpoint{3.606600in}{4.024094in}}%
\pgfpathlineto{\pgfqpoint{3.611664in}{4.027696in}}%
\pgfpathlineto{\pgfqpoint{3.616728in}{4.034129in}}%
\pgfpathlineto{\pgfqpoint{3.621792in}{4.034483in}}%
\pgfpathlineto{\pgfqpoint{3.626856in}{4.046708in}}%
\pgfpathlineto{\pgfqpoint{3.636984in}{4.059393in}}%
\pgfpathlineto{\pgfqpoint{3.642048in}{4.062201in}}%
\pgfpathlineto{\pgfqpoint{3.647112in}{4.062219in}}%
\pgfpathlineto{\pgfqpoint{3.652176in}{4.067990in}}%
\pgfpathlineto{\pgfqpoint{3.657240in}{4.068509in}}%
\pgfpathlineto{\pgfqpoint{3.662304in}{4.071669in}}%
\pgfpathlineto{\pgfqpoint{3.667368in}{4.072224in}}%
\pgfpathlineto{\pgfqpoint{3.672432in}{4.077800in}}%
\pgfpathlineto{\pgfqpoint{3.677496in}{4.081647in}}%
\pgfpathlineto{\pgfqpoint{3.682560in}{4.082154in}}%
\pgfpathlineto{\pgfqpoint{3.687624in}{4.085773in}}%
\pgfpathlineto{\pgfqpoint{3.692688in}{4.092118in}}%
\pgfpathlineto{\pgfqpoint{3.697752in}{4.094906in}}%
\pgfpathlineto{\pgfqpoint{3.707880in}{4.098366in}}%
\pgfpathlineto{\pgfqpoint{3.712944in}{4.105262in}}%
\pgfpathlineto{\pgfqpoint{3.718008in}{4.106920in}}%
\pgfpathlineto{\pgfqpoint{3.723072in}{4.121531in}}%
\pgfpathlineto{\pgfqpoint{3.728136in}{4.130268in}}%
\pgfpathlineto{\pgfqpoint{3.738264in}{4.136372in}}%
\pgfpathlineto{\pgfqpoint{3.743328in}{4.138568in}}%
\pgfpathlineto{\pgfqpoint{3.748392in}{4.143677in}}%
\pgfpathlineto{\pgfqpoint{3.753456in}{4.144397in}}%
\pgfpathlineto{\pgfqpoint{3.763584in}{4.151007in}}%
\pgfpathlineto{\pgfqpoint{3.768648in}{4.152199in}}%
\pgfpathlineto{\pgfqpoint{3.773712in}{4.155122in}}%
\pgfpathlineto{\pgfqpoint{3.788904in}{4.160100in}}%
\pgfpathlineto{\pgfqpoint{3.793968in}{4.165190in}}%
\pgfpathlineto{\pgfqpoint{3.799032in}{4.179034in}}%
\pgfpathlineto{\pgfqpoint{3.804096in}{4.185225in}}%
\pgfpathlineto{\pgfqpoint{3.814224in}{4.188098in}}%
\pgfpathlineto{\pgfqpoint{3.819288in}{4.192454in}}%
\pgfpathlineto{\pgfqpoint{3.824351in}{4.198822in}}%
\pgfpathlineto{\pgfqpoint{3.829415in}{4.200896in}}%
\pgfpathlineto{\pgfqpoint{3.834479in}{4.205922in}}%
\pgfpathlineto{\pgfqpoint{3.839543in}{4.206737in}}%
\pgfpathlineto{\pgfqpoint{3.844607in}{4.208702in}}%
\pgfpathlineto{\pgfqpoint{3.859799in}{4.210337in}}%
\pgfpathlineto{\pgfqpoint{3.864863in}{4.216024in}}%
\pgfpathlineto{\pgfqpoint{3.869927in}{4.217729in}}%
\pgfpathlineto{\pgfqpoint{3.874991in}{4.224181in}}%
\pgfpathlineto{\pgfqpoint{3.880055in}{4.224443in}}%
\pgfpathlineto{\pgfqpoint{3.885119in}{4.225933in}}%
\pgfpathlineto{\pgfqpoint{3.890183in}{4.228938in}}%
\pgfpathlineto{\pgfqpoint{3.895247in}{4.245790in}}%
\pgfpathlineto{\pgfqpoint{3.900311in}{4.274238in}}%
\pgfpathlineto{\pgfqpoint{3.910439in}{4.306589in}}%
\pgfpathlineto{\pgfqpoint{3.915503in}{4.316708in}}%
\pgfpathlineto{\pgfqpoint{3.920567in}{4.319470in}}%
\pgfpathlineto{\pgfqpoint{3.930695in}{4.323056in}}%
\pgfpathlineto{\pgfqpoint{3.935759in}{4.346852in}}%
\pgfpathlineto{\pgfqpoint{3.945887in}{4.347465in}}%
\pgfpathlineto{\pgfqpoint{3.956015in}{4.351408in}}%
\pgfpathlineto{\pgfqpoint{3.961079in}{4.351975in}}%
\pgfpathlineto{\pgfqpoint{3.966143in}{4.359930in}}%
\pgfpathlineto{\pgfqpoint{3.971207in}{4.359957in}}%
\pgfpathlineto{\pgfqpoint{3.976271in}{4.361288in}}%
\pgfpathlineto{\pgfqpoint{3.981335in}{4.366692in}}%
\pgfpathlineto{\pgfqpoint{3.986399in}{4.383872in}}%
\pgfpathlineto{\pgfqpoint{4.001591in}{4.391435in}}%
\pgfpathlineto{\pgfqpoint{4.011719in}{4.398056in}}%
\pgfpathlineto{\pgfqpoint{4.031975in}{4.402262in}}%
\pgfpathlineto{\pgfqpoint{4.037039in}{4.406139in}}%
\pgfpathlineto{\pgfqpoint{4.047167in}{4.408140in}}%
\pgfpathlineto{\pgfqpoint{4.052231in}{4.428470in}}%
\pgfpathlineto{\pgfqpoint{4.057295in}{4.430759in}}%
\pgfpathlineto{\pgfqpoint{4.067423in}{4.431334in}}%
\pgfpathlineto{\pgfqpoint{4.072487in}{4.433994in}}%
\pgfpathlineto{\pgfqpoint{4.077551in}{4.439075in}}%
\pgfpathlineto{\pgfqpoint{4.087679in}{4.444051in}}%
\pgfpathlineto{\pgfqpoint{4.092743in}{4.444950in}}%
\pgfpathlineto{\pgfqpoint{4.097807in}{4.450863in}}%
\pgfpathlineto{\pgfqpoint{4.107935in}{4.458181in}}%
\pgfpathlineto{\pgfqpoint{4.112999in}{4.464528in}}%
\pgfpathlineto{\pgfqpoint{4.123127in}{4.469512in}}%
\pgfpathlineto{\pgfqpoint{4.128191in}{4.469620in}}%
\pgfpathlineto{\pgfqpoint{4.148446in}{4.476218in}}%
\pgfpathlineto{\pgfqpoint{4.153510in}{4.479036in}}%
\pgfpathlineto{\pgfqpoint{4.158574in}{4.485549in}}%
\pgfpathlineto{\pgfqpoint{4.163638in}{4.486455in}}%
\pgfpathlineto{\pgfqpoint{4.173766in}{4.495132in}}%
\pgfpathlineto{\pgfqpoint{4.183894in}{4.496412in}}%
\pgfpathlineto{\pgfqpoint{4.188958in}{4.504579in}}%
\pgfpathlineto{\pgfqpoint{4.194022in}{4.508347in}}%
\pgfpathlineto{\pgfqpoint{4.199086in}{4.513912in}}%
\pgfpathlineto{\pgfqpoint{4.204150in}{4.524155in}}%
\pgfpathlineto{\pgfqpoint{4.219342in}{4.531009in}}%
\pgfpathlineto{\pgfqpoint{4.224406in}{4.539657in}}%
\pgfpathlineto{\pgfqpoint{4.229470in}{4.541901in}}%
\pgfpathlineto{\pgfqpoint{4.234534in}{4.541937in}}%
\pgfpathlineto{\pgfqpoint{4.239598in}{4.543147in}}%
\pgfpathlineto{\pgfqpoint{4.244662in}{4.543193in}}%
\pgfpathlineto{\pgfqpoint{4.249726in}{4.545419in}}%
\pgfpathlineto{\pgfqpoint{4.254790in}{4.553076in}}%
\pgfpathlineto{\pgfqpoint{4.259854in}{4.580950in}}%
\pgfpathlineto{\pgfqpoint{4.264918in}{4.597238in}}%
\pgfpathlineto{\pgfqpoint{6.244936in}{4.597238in}}%
\pgfpathlineto{\pgfqpoint{6.244936in}{4.597238in}}%
\pgfusepath{stroke}%
\end{pgfscope}%
\begin{pgfscope}%
\pgfpathrectangle{\pgfqpoint{0.725193in}{0.571603in}}{\pgfqpoint{5.524807in}{4.025635in}}%
\pgfusepath{clip}%
\pgfsetbuttcap%
\pgfsetroundjoin%
\pgfsetlinewidth{2.007500pt}%
\definecolor{currentstroke}{rgb}{0.866667,0.058824,0.058824}%
\pgfsetstrokecolor{currentstroke}%
\pgfsetdash{{7.400000pt}{3.200000pt}}{0.000000pt}%
\pgfpathmoveto{\pgfqpoint{0.725193in}{0.765386in}}%
\pgfpathlineto{\pgfqpoint{0.801153in}{0.765422in}}%
\pgfpathlineto{\pgfqpoint{0.806217in}{0.800108in}}%
\pgfpathlineto{\pgfqpoint{1.348063in}{0.800174in}}%
\pgfpathlineto{\pgfqpoint{1.353127in}{0.831551in}}%
\pgfpathlineto{\pgfqpoint{1.535431in}{0.831611in}}%
\pgfpathlineto{\pgfqpoint{1.540495in}{0.860256in}}%
\pgfpathlineto{\pgfqpoint{1.662030in}{0.860283in}}%
\pgfpathlineto{\pgfqpoint{1.667094in}{0.886661in}}%
\pgfpathlineto{\pgfqpoint{1.737990in}{0.886686in}}%
\pgfpathlineto{\pgfqpoint{1.743054in}{0.911107in}}%
\pgfpathlineto{\pgfqpoint{1.778502in}{0.911131in}}%
\pgfpathlineto{\pgfqpoint{1.783566in}{0.933844in}}%
\pgfpathlineto{\pgfqpoint{1.813950in}{0.933888in}}%
\pgfpathlineto{\pgfqpoint{1.819014in}{0.955135in}}%
\pgfpathlineto{\pgfqpoint{1.854462in}{0.955176in}}%
\pgfpathlineto{\pgfqpoint{1.859526in}{0.975134in}}%
\pgfpathlineto{\pgfqpoint{1.894973in}{0.975173in}}%
\pgfpathlineto{\pgfqpoint{1.900037in}{0.993990in}}%
\pgfpathlineto{\pgfqpoint{1.925357in}{0.994008in}}%
\pgfpathlineto{\pgfqpoint{1.930421in}{1.011825in}}%
\pgfpathlineto{\pgfqpoint{1.955741in}{1.011843in}}%
\pgfpathlineto{\pgfqpoint{1.960805in}{1.028745in}}%
\pgfpathlineto{\pgfqpoint{1.970933in}{1.028762in}}%
\pgfpathlineto{\pgfqpoint{1.975997in}{1.044840in}}%
\pgfpathlineto{\pgfqpoint{1.981061in}{1.044840in}}%
\pgfpathlineto{\pgfqpoint{1.986125in}{1.060185in}}%
\pgfpathlineto{\pgfqpoint{1.996253in}{1.060200in}}%
\pgfpathlineto{\pgfqpoint{2.001317in}{1.074834in}}%
\pgfpathlineto{\pgfqpoint{2.011445in}{1.074863in}}%
\pgfpathlineto{\pgfqpoint{2.016509in}{1.088873in}}%
\pgfpathlineto{\pgfqpoint{2.041829in}{1.088901in}}%
\pgfpathlineto{\pgfqpoint{2.046893in}{1.102340in}}%
\pgfpathlineto{\pgfqpoint{2.062085in}{1.102353in}}%
\pgfpathlineto{\pgfqpoint{2.072213in}{1.127714in}}%
\pgfpathlineto{\pgfqpoint{2.092469in}{1.127739in}}%
\pgfpathlineto{\pgfqpoint{2.102597in}{1.151287in}}%
\pgfpathlineto{\pgfqpoint{2.127917in}{1.151310in}}%
\pgfpathlineto{\pgfqpoint{2.132981in}{1.173287in}}%
\pgfpathlineto{\pgfqpoint{2.153237in}{1.173308in}}%
\pgfpathlineto{\pgfqpoint{2.168429in}{1.203758in}}%
\pgfpathlineto{\pgfqpoint{2.183621in}{1.203767in}}%
\pgfpathlineto{\pgfqpoint{2.188685in}{1.213319in}}%
\pgfpathlineto{\pgfqpoint{2.193749in}{1.213319in}}%
\pgfpathlineto{\pgfqpoint{2.198813in}{1.222603in}}%
\pgfpathlineto{\pgfqpoint{2.203876in}{1.222612in}}%
\pgfpathlineto{\pgfqpoint{2.219068in}{1.249006in}}%
\pgfpathlineto{\pgfqpoint{2.234260in}{1.249015in}}%
\pgfpathlineto{\pgfqpoint{2.249452in}{1.273444in}}%
\pgfpathlineto{\pgfqpoint{2.254516in}{1.288797in}}%
\pgfpathlineto{\pgfqpoint{2.269708in}{1.288797in}}%
\pgfpathlineto{\pgfqpoint{2.274772in}{1.296210in}}%
\pgfpathlineto{\pgfqpoint{2.284900in}{1.296210in}}%
\pgfpathlineto{\pgfqpoint{2.300092in}{1.317491in}}%
\pgfpathlineto{\pgfqpoint{2.310220in}{1.317498in}}%
\pgfpathlineto{\pgfqpoint{2.315284in}{1.330950in}}%
\pgfpathlineto{\pgfqpoint{2.320348in}{1.337482in}}%
\pgfpathlineto{\pgfqpoint{2.330476in}{1.337489in}}%
\pgfpathlineto{\pgfqpoint{2.335540in}{1.343888in}}%
\pgfpathlineto{\pgfqpoint{2.340604in}{1.356331in}}%
\pgfpathlineto{\pgfqpoint{2.350732in}{1.356337in}}%
\pgfpathlineto{\pgfqpoint{2.365924in}{1.374166in}}%
\pgfpathlineto{\pgfqpoint{2.370988in}{1.374171in}}%
\pgfpathlineto{\pgfqpoint{2.381116in}{1.385547in}}%
\pgfpathlineto{\pgfqpoint{2.386180in}{1.385552in}}%
\pgfpathlineto{\pgfqpoint{2.396308in}{1.396538in}}%
\pgfpathlineto{\pgfqpoint{2.411500in}{1.396543in}}%
\pgfpathlineto{\pgfqpoint{2.416564in}{1.401896in}}%
\pgfpathlineto{\pgfqpoint{2.421628in}{1.401901in}}%
\pgfpathlineto{\pgfqpoint{2.426692in}{1.412369in}}%
\pgfpathlineto{\pgfqpoint{2.436820in}{1.422519in}}%
\pgfpathlineto{\pgfqpoint{2.441884in}{1.422524in}}%
\pgfpathlineto{\pgfqpoint{2.446948in}{1.437182in}}%
\pgfpathlineto{\pgfqpoint{2.457076in}{1.446607in}}%
\pgfpathlineto{\pgfqpoint{2.462140in}{1.446607in}}%
\pgfpathlineto{\pgfqpoint{2.467204in}{1.451216in}}%
\pgfpathlineto{\pgfqpoint{2.472268in}{1.451221in}}%
\pgfpathlineto{\pgfqpoint{2.477332in}{1.469055in}}%
\pgfpathlineto{\pgfqpoint{2.487460in}{1.477619in}}%
\pgfpathlineto{\pgfqpoint{2.492524in}{1.485966in}}%
\pgfpathlineto{\pgfqpoint{2.497588in}{1.498109in}}%
\pgfpathlineto{\pgfqpoint{2.512780in}{1.509818in}}%
\pgfpathlineto{\pgfqpoint{2.517844in}{1.509821in}}%
\pgfpathlineto{\pgfqpoint{2.527971in}{1.517397in}}%
\pgfpathlineto{\pgfqpoint{2.533035in}{1.524814in}}%
\pgfpathlineto{\pgfqpoint{2.543163in}{1.532060in}}%
\pgfpathlineto{\pgfqpoint{2.553291in}{1.552897in}}%
\pgfpathlineto{\pgfqpoint{2.558355in}{1.559561in}}%
\pgfpathlineto{\pgfqpoint{2.563419in}{1.575651in}}%
\pgfpathlineto{\pgfqpoint{2.573547in}{1.575655in}}%
\pgfpathlineto{\pgfqpoint{2.578611in}{1.578778in}}%
\pgfpathlineto{\pgfqpoint{2.583675in}{1.584944in}}%
\pgfpathlineto{\pgfqpoint{2.598867in}{1.593982in}}%
\pgfpathlineto{\pgfqpoint{2.608995in}{1.611344in}}%
\pgfpathlineto{\pgfqpoint{2.614059in}{1.616935in}}%
\pgfpathlineto{\pgfqpoint{2.619123in}{1.630505in}}%
\pgfpathlineto{\pgfqpoint{2.629251in}{1.630508in}}%
\pgfpathlineto{\pgfqpoint{2.634315in}{1.633158in}}%
\pgfpathlineto{\pgfqpoint{2.639379in}{1.638393in}}%
\pgfpathlineto{\pgfqpoint{2.644443in}{1.646092in}}%
\pgfpathlineto{\pgfqpoint{2.649507in}{1.658539in}}%
\pgfpathlineto{\pgfqpoint{2.654571in}{1.658541in}}%
\pgfpathlineto{\pgfqpoint{2.659635in}{1.668168in}}%
\pgfpathlineto{\pgfqpoint{2.669763in}{1.677526in}}%
\pgfpathlineto{\pgfqpoint{2.674827in}{1.679822in}}%
\pgfpathlineto{\pgfqpoint{2.679891in}{1.686624in}}%
\pgfpathlineto{\pgfqpoint{2.684955in}{1.695480in}}%
\pgfpathlineto{\pgfqpoint{2.690019in}{1.695480in}}%
\pgfpathlineto{\pgfqpoint{2.695083in}{1.697655in}}%
\pgfpathlineto{\pgfqpoint{2.705211in}{1.697657in}}%
\pgfpathlineto{\pgfqpoint{2.710275in}{1.699818in}}%
\pgfpathlineto{\pgfqpoint{2.715339in}{1.699820in}}%
\pgfpathlineto{\pgfqpoint{2.720403in}{1.712504in}}%
\pgfpathlineto{\pgfqpoint{2.725467in}{1.714572in}}%
\pgfpathlineto{\pgfqpoint{2.730531in}{1.714574in}}%
\pgfpathlineto{\pgfqpoint{2.745723in}{1.740336in}}%
\pgfpathlineto{\pgfqpoint{2.750787in}{1.744127in}}%
\pgfpathlineto{\pgfqpoint{2.755851in}{1.753418in}}%
\pgfpathlineto{\pgfqpoint{2.760915in}{1.755245in}}%
\pgfpathlineto{\pgfqpoint{2.765979in}{1.760668in}}%
\pgfpathlineto{\pgfqpoint{2.776107in}{1.764234in}}%
\pgfpathlineto{\pgfqpoint{2.781171in}{1.769508in}}%
\pgfpathlineto{\pgfqpoint{2.791299in}{1.772982in}}%
\pgfpathlineto{\pgfqpoint{2.801427in}{1.788165in}}%
\pgfpathlineto{\pgfqpoint{2.811555in}{1.802683in}}%
\pgfpathlineto{\pgfqpoint{2.826747in}{1.802684in}}%
\pgfpathlineto{\pgfqpoint{2.831811in}{1.804257in}}%
\pgfpathlineto{\pgfqpoint{2.841939in}{1.813549in}}%
\pgfpathlineto{\pgfqpoint{2.847003in}{1.813549in}}%
\pgfpathlineto{\pgfqpoint{2.857130in}{1.829932in}}%
\pgfpathlineto{\pgfqpoint{2.862194in}{1.849668in}}%
\pgfpathlineto{\pgfqpoint{2.867258in}{1.849669in}}%
\pgfpathlineto{\pgfqpoint{2.877386in}{1.852395in}}%
\pgfpathlineto{\pgfqpoint{2.882450in}{1.873425in}}%
\pgfpathlineto{\pgfqpoint{2.887514in}{1.873426in}}%
\pgfpathlineto{\pgfqpoint{2.892578in}{1.883457in}}%
\pgfpathlineto{\pgfqpoint{2.897642in}{1.883457in}}%
\pgfpathlineto{\pgfqpoint{2.902706in}{1.884690in}}%
\pgfpathlineto{\pgfqpoint{2.912834in}{1.893194in}}%
\pgfpathlineto{\pgfqpoint{2.922962in}{1.893196in}}%
\pgfpathlineto{\pgfqpoint{2.928026in}{1.894393in}}%
\pgfpathlineto{\pgfqpoint{2.933090in}{1.897959in}}%
\pgfpathlineto{\pgfqpoint{2.938154in}{1.908429in}}%
\pgfpathlineto{\pgfqpoint{2.943218in}{1.908429in}}%
\pgfpathlineto{\pgfqpoint{2.948282in}{1.916350in}}%
\pgfpathlineto{\pgfqpoint{2.958410in}{1.927344in}}%
\pgfpathlineto{\pgfqpoint{2.963474in}{1.930572in}}%
\pgfpathlineto{\pgfqpoint{2.968538in}{1.931642in}}%
\pgfpathlineto{\pgfqpoint{2.973602in}{1.948290in}}%
\pgfpathlineto{\pgfqpoint{2.978666in}{1.953325in}}%
\pgfpathlineto{\pgfqpoint{2.983730in}{1.956310in}}%
\pgfpathlineto{\pgfqpoint{2.993858in}{1.969893in}}%
\pgfpathlineto{\pgfqpoint{2.998922in}{1.974610in}}%
\pgfpathlineto{\pgfqpoint{3.003986in}{1.975546in}}%
\pgfpathlineto{\pgfqpoint{3.009050in}{1.987474in}}%
\pgfpathlineto{\pgfqpoint{3.014114in}{1.989272in}}%
\pgfpathlineto{\pgfqpoint{3.019178in}{1.989272in}}%
\pgfpathlineto{\pgfqpoint{3.024242in}{1.994605in}}%
\pgfpathlineto{\pgfqpoint{3.029306in}{2.019235in}}%
\pgfpathlineto{\pgfqpoint{3.034370in}{2.024109in}}%
\pgfpathlineto{\pgfqpoint{3.039434in}{2.024915in}}%
\pgfpathlineto{\pgfqpoint{3.044498in}{2.049701in}}%
\pgfpathlineto{\pgfqpoint{3.049562in}{2.054148in}}%
\pgfpathlineto{\pgfqpoint{3.054626in}{2.060708in}}%
\pgfpathlineto{\pgfqpoint{3.059690in}{2.062866in}}%
\pgfpathlineto{\pgfqpoint{3.069818in}{2.064296in}}%
\pgfpathlineto{\pgfqpoint{3.079946in}{2.100114in}}%
\pgfpathlineto{\pgfqpoint{3.085010in}{2.100754in}}%
\pgfpathlineto{\pgfqpoint{3.090074in}{2.108960in}}%
\pgfpathlineto{\pgfqpoint{3.095138in}{2.113295in}}%
\pgfpathlineto{\pgfqpoint{3.100202in}{2.113910in}}%
\pgfpathlineto{\pgfqpoint{3.110330in}{2.122399in}}%
\pgfpathlineto{\pgfqpoint{3.115394in}{2.124783in}}%
\pgfpathlineto{\pgfqpoint{3.120458in}{2.131256in}}%
\pgfpathlineto{\pgfqpoint{3.125522in}{2.132420in}}%
\pgfpathlineto{\pgfqpoint{3.130586in}{2.138176in}}%
\pgfpathlineto{\pgfqpoint{3.140714in}{2.143272in}}%
\pgfpathlineto{\pgfqpoint{3.145778in}{2.143833in}}%
\pgfpathlineto{\pgfqpoint{3.160970in}{2.155406in}}%
\pgfpathlineto{\pgfqpoint{3.166034in}{2.157565in}}%
\pgfpathlineto{\pgfqpoint{3.171098in}{2.161311in}}%
\pgfpathlineto{\pgfqpoint{3.176161in}{2.161842in}}%
\pgfpathlineto{\pgfqpoint{3.181225in}{2.181930in}}%
\pgfpathlineto{\pgfqpoint{3.196417in}{2.193227in}}%
\pgfpathlineto{\pgfqpoint{3.201481in}{2.194192in}}%
\pgfpathlineto{\pgfqpoint{3.206545in}{2.208329in}}%
\pgfpathlineto{\pgfqpoint{3.211609in}{2.211542in}}%
\pgfpathlineto{\pgfqpoint{3.216673in}{2.222327in}}%
\pgfpathlineto{\pgfqpoint{3.221737in}{2.229758in}}%
\pgfpathlineto{\pgfqpoint{3.226801in}{2.234479in}}%
\pgfpathlineto{\pgfqpoint{3.231865in}{2.264593in}}%
\pgfpathlineto{\pgfqpoint{3.236929in}{2.272281in}}%
\pgfpathlineto{\pgfqpoint{3.247057in}{2.310958in}}%
\pgfpathlineto{\pgfqpoint{3.252121in}{2.311296in}}%
\pgfpathlineto{\pgfqpoint{3.257185in}{2.316653in}}%
\pgfpathlineto{\pgfqpoint{3.262249in}{2.334117in}}%
\pgfpathlineto{\pgfqpoint{3.267313in}{2.336002in}}%
\pgfpathlineto{\pgfqpoint{3.272377in}{2.336315in}}%
\pgfpathlineto{\pgfqpoint{3.277441in}{2.340668in}}%
\pgfpathlineto{\pgfqpoint{3.282505in}{2.351002in}}%
\pgfpathlineto{\pgfqpoint{3.287569in}{2.352793in}}%
\pgfpathlineto{\pgfqpoint{3.292633in}{2.372437in}}%
\pgfpathlineto{\pgfqpoint{3.297697in}{2.372437in}}%
\pgfpathlineto{\pgfqpoint{3.302761in}{2.383196in}}%
\pgfpathlineto{\pgfqpoint{3.307825in}{2.389648in}}%
\pgfpathlineto{\pgfqpoint{3.312889in}{2.390180in}}%
\pgfpathlineto{\pgfqpoint{3.323017in}{2.409032in}}%
\pgfpathlineto{\pgfqpoint{3.333145in}{2.413767in}}%
\pgfpathlineto{\pgfqpoint{3.338209in}{2.423277in}}%
\pgfpathlineto{\pgfqpoint{3.343273in}{2.455976in}}%
\pgfpathlineto{\pgfqpoint{3.348337in}{2.459010in}}%
\pgfpathlineto{\pgfqpoint{3.353401in}{2.460087in}}%
\pgfpathlineto{\pgfqpoint{3.358465in}{2.481116in}}%
\pgfpathlineto{\pgfqpoint{3.363529in}{2.481519in}}%
\pgfpathlineto{\pgfqpoint{3.368593in}{2.484728in}}%
\pgfpathlineto{\pgfqpoint{3.373657in}{2.490854in}}%
\pgfpathlineto{\pgfqpoint{3.383785in}{2.542086in}}%
\pgfpathlineto{\pgfqpoint{3.393913in}{2.543258in}}%
\pgfpathlineto{\pgfqpoint{3.398977in}{2.556361in}}%
\pgfpathlineto{\pgfqpoint{3.404041in}{2.558282in}}%
\pgfpathlineto{\pgfqpoint{3.409105in}{2.566636in}}%
\pgfpathlineto{\pgfqpoint{3.414169in}{2.567724in}}%
\pgfpathlineto{\pgfqpoint{3.419233in}{2.576298in}}%
\pgfpathlineto{\pgfqpoint{3.424297in}{2.577053in}}%
\pgfpathlineto{\pgfqpoint{3.429361in}{2.590208in}}%
\pgfpathlineto{\pgfqpoint{3.434425in}{2.590642in}}%
\pgfpathlineto{\pgfqpoint{3.439489in}{2.608664in}}%
\pgfpathlineto{\pgfqpoint{3.444553in}{2.616117in}}%
\pgfpathlineto{\pgfqpoint{3.449617in}{2.618385in}}%
\pgfpathlineto{\pgfqpoint{3.454681in}{2.623534in}}%
\pgfpathlineto{\pgfqpoint{3.459745in}{2.640883in}}%
\pgfpathlineto{\pgfqpoint{3.464809in}{2.649227in}}%
\pgfpathlineto{\pgfqpoint{3.469873in}{2.652001in}}%
\pgfpathlineto{\pgfqpoint{3.474937in}{2.667147in}}%
\pgfpathlineto{\pgfqpoint{3.480001in}{2.669548in}}%
\pgfpathlineto{\pgfqpoint{3.485065in}{2.678539in}}%
\pgfpathlineto{\pgfqpoint{3.490129in}{2.684580in}}%
\pgfpathlineto{\pgfqpoint{3.500256in}{2.709518in}}%
\pgfpathlineto{\pgfqpoint{3.505320in}{2.709922in}}%
\pgfpathlineto{\pgfqpoint{3.510384in}{2.712733in}}%
\pgfpathlineto{\pgfqpoint{3.515448in}{2.742527in}}%
\pgfpathlineto{\pgfqpoint{3.520512in}{2.746249in}}%
\pgfpathlineto{\pgfqpoint{3.525576in}{2.747601in}}%
\pgfpathlineto{\pgfqpoint{3.530640in}{2.756472in}}%
\pgfpathlineto{\pgfqpoint{3.535704in}{2.761682in}}%
\pgfpathlineto{\pgfqpoint{3.540768in}{2.776837in}}%
\pgfpathlineto{\pgfqpoint{3.545832in}{2.798106in}}%
\pgfpathlineto{\pgfqpoint{3.566088in}{2.820835in}}%
\pgfpathlineto{\pgfqpoint{3.571152in}{2.839386in}}%
\pgfpathlineto{\pgfqpoint{3.576216in}{2.863404in}}%
\pgfpathlineto{\pgfqpoint{3.581280in}{2.864415in}}%
\pgfpathlineto{\pgfqpoint{3.591408in}{2.875093in}}%
\pgfpathlineto{\pgfqpoint{3.596472in}{2.877649in}}%
\pgfpathlineto{\pgfqpoint{3.601536in}{2.882940in}}%
\pgfpathlineto{\pgfqpoint{3.606600in}{2.883715in}}%
\pgfpathlineto{\pgfqpoint{3.611664in}{2.896325in}}%
\pgfpathlineto{\pgfqpoint{3.616728in}{2.898608in}}%
\pgfpathlineto{\pgfqpoint{3.621792in}{2.906644in}}%
\pgfpathlineto{\pgfqpoint{3.626856in}{2.910836in}}%
\pgfpathlineto{\pgfqpoint{3.636984in}{2.928237in}}%
\pgfpathlineto{\pgfqpoint{3.647112in}{2.933243in}}%
\pgfpathlineto{\pgfqpoint{3.657240in}{2.939683in}}%
\pgfpathlineto{\pgfqpoint{3.662304in}{2.948113in}}%
\pgfpathlineto{\pgfqpoint{3.667368in}{2.948943in}}%
\pgfpathlineto{\pgfqpoint{3.672432in}{2.952634in}}%
\pgfpathlineto{\pgfqpoint{3.687624in}{2.980936in}}%
\pgfpathlineto{\pgfqpoint{3.692688in}{3.000090in}}%
\pgfpathlineto{\pgfqpoint{3.697752in}{3.004658in}}%
\pgfpathlineto{\pgfqpoint{3.702816in}{3.005194in}}%
\pgfpathlineto{\pgfqpoint{3.707880in}{3.027327in}}%
\pgfpathlineto{\pgfqpoint{3.712944in}{3.033317in}}%
\pgfpathlineto{\pgfqpoint{3.723072in}{3.039794in}}%
\pgfpathlineto{\pgfqpoint{3.728136in}{3.041458in}}%
\pgfpathlineto{\pgfqpoint{3.738264in}{3.049331in}}%
\pgfpathlineto{\pgfqpoint{3.743328in}{3.056704in}}%
\pgfpathlineto{\pgfqpoint{3.753456in}{3.061529in}}%
\pgfpathlineto{\pgfqpoint{3.758520in}{3.073759in}}%
\pgfpathlineto{\pgfqpoint{3.768648in}{3.077682in}}%
\pgfpathlineto{\pgfqpoint{3.773712in}{3.077848in}}%
\pgfpathlineto{\pgfqpoint{3.778776in}{3.086293in}}%
\pgfpathlineto{\pgfqpoint{3.783840in}{3.087771in}}%
\pgfpathlineto{\pgfqpoint{3.788904in}{3.093775in}}%
\pgfpathlineto{\pgfqpoint{3.793968in}{3.106948in}}%
\pgfpathlineto{\pgfqpoint{3.799032in}{3.113536in}}%
\pgfpathlineto{\pgfqpoint{3.804096in}{3.116899in}}%
\pgfpathlineto{\pgfqpoint{3.809160in}{3.124099in}}%
\pgfpathlineto{\pgfqpoint{3.824351in}{3.138892in}}%
\pgfpathlineto{\pgfqpoint{3.829415in}{3.138974in}}%
\pgfpathlineto{\pgfqpoint{3.834479in}{3.147747in}}%
\pgfpathlineto{\pgfqpoint{3.839543in}{3.153969in}}%
\pgfpathlineto{\pgfqpoint{3.849671in}{3.157050in}}%
\pgfpathlineto{\pgfqpoint{3.854735in}{3.166120in}}%
\pgfpathlineto{\pgfqpoint{3.859799in}{3.169215in}}%
\pgfpathlineto{\pgfqpoint{3.864863in}{3.174433in}}%
\pgfpathlineto{\pgfqpoint{3.869927in}{3.175590in}}%
\pgfpathlineto{\pgfqpoint{3.874991in}{3.179761in}}%
\pgfpathlineto{\pgfqpoint{3.880055in}{3.181985in}}%
\pgfpathlineto{\pgfqpoint{3.885119in}{3.193186in}}%
\pgfpathlineto{\pgfqpoint{3.890183in}{3.193884in}}%
\pgfpathlineto{\pgfqpoint{3.895247in}{3.206469in}}%
\pgfpathlineto{\pgfqpoint{3.900311in}{3.215644in}}%
\pgfpathlineto{\pgfqpoint{3.905375in}{3.233860in}}%
\pgfpathlineto{\pgfqpoint{3.910439in}{3.237156in}}%
\pgfpathlineto{\pgfqpoint{3.915503in}{3.238092in}}%
\pgfpathlineto{\pgfqpoint{3.920567in}{3.251200in}}%
\pgfpathlineto{\pgfqpoint{3.930695in}{3.252525in}}%
\pgfpathlineto{\pgfqpoint{3.935759in}{3.260767in}}%
\pgfpathlineto{\pgfqpoint{3.940823in}{3.284303in}}%
\pgfpathlineto{\pgfqpoint{3.945887in}{3.290254in}}%
\pgfpathlineto{\pgfqpoint{3.950951in}{3.294341in}}%
\pgfpathlineto{\pgfqpoint{3.956015in}{3.300586in}}%
\pgfpathlineto{\pgfqpoint{3.966143in}{3.356952in}}%
\pgfpathlineto{\pgfqpoint{3.971207in}{3.361162in}}%
\pgfpathlineto{\pgfqpoint{3.981335in}{3.362725in}}%
\pgfpathlineto{\pgfqpoint{3.986399in}{3.378576in}}%
\pgfpathlineto{\pgfqpoint{3.991463in}{3.383281in}}%
\pgfpathlineto{\pgfqpoint{3.996527in}{3.412223in}}%
\pgfpathlineto{\pgfqpoint{4.001591in}{3.426771in}}%
\pgfpathlineto{\pgfqpoint{4.016783in}{3.431859in}}%
\pgfpathlineto{\pgfqpoint{4.021847in}{3.458821in}}%
\pgfpathlineto{\pgfqpoint{4.031975in}{3.490242in}}%
\pgfpathlineto{\pgfqpoint{4.037039in}{3.494985in}}%
\pgfpathlineto{\pgfqpoint{4.042103in}{3.504444in}}%
\pgfpathlineto{\pgfqpoint{4.047167in}{3.510985in}}%
\pgfpathlineto{\pgfqpoint{4.052231in}{3.558545in}}%
\pgfpathlineto{\pgfqpoint{4.057295in}{3.568738in}}%
\pgfpathlineto{\pgfqpoint{4.062359in}{3.569936in}}%
\pgfpathlineto{\pgfqpoint{4.067423in}{3.572364in}}%
\pgfpathlineto{\pgfqpoint{4.072487in}{3.590164in}}%
\pgfpathlineto{\pgfqpoint{4.077551in}{3.592948in}}%
\pgfpathlineto{\pgfqpoint{4.082615in}{3.605764in}}%
\pgfpathlineto{\pgfqpoint{4.087679in}{3.606722in}}%
\pgfpathlineto{\pgfqpoint{4.092743in}{3.645114in}}%
\pgfpathlineto{\pgfqpoint{4.097807in}{3.652901in}}%
\pgfpathlineto{\pgfqpoint{4.102871in}{3.657735in}}%
\pgfpathlineto{\pgfqpoint{4.107935in}{3.664201in}}%
\pgfpathlineto{\pgfqpoint{4.112999in}{3.675905in}}%
\pgfpathlineto{\pgfqpoint{4.118063in}{3.677421in}}%
\pgfpathlineto{\pgfqpoint{4.123127in}{3.681704in}}%
\pgfpathlineto{\pgfqpoint{4.128191in}{3.688142in}}%
\pgfpathlineto{\pgfqpoint{4.133255in}{3.688298in}}%
\pgfpathlineto{\pgfqpoint{4.138319in}{3.711024in}}%
\pgfpathlineto{\pgfqpoint{4.143383in}{3.740653in}}%
\pgfpathlineto{\pgfqpoint{4.148446in}{3.751647in}}%
\pgfpathlineto{\pgfqpoint{4.163638in}{3.767323in}}%
\pgfpathlineto{\pgfqpoint{4.173766in}{3.769974in}}%
\pgfpathlineto{\pgfqpoint{4.178830in}{3.770998in}}%
\pgfpathlineto{\pgfqpoint{4.183894in}{3.774684in}}%
\pgfpathlineto{\pgfqpoint{4.188958in}{3.804025in}}%
\pgfpathlineto{\pgfqpoint{4.194022in}{3.816472in}}%
\pgfpathlineto{\pgfqpoint{4.199086in}{3.817221in}}%
\pgfpathlineto{\pgfqpoint{4.204150in}{3.827937in}}%
\pgfpathlineto{\pgfqpoint{4.209214in}{3.846754in}}%
\pgfpathlineto{\pgfqpoint{4.214278in}{3.848013in}}%
\pgfpathlineto{\pgfqpoint{4.219342in}{3.863980in}}%
\pgfpathlineto{\pgfqpoint{4.224406in}{3.870317in}}%
\pgfpathlineto{\pgfqpoint{4.229470in}{3.887171in}}%
\pgfpathlineto{\pgfqpoint{4.234534in}{3.890694in}}%
\pgfpathlineto{\pgfqpoint{4.239598in}{3.892479in}}%
\pgfpathlineto{\pgfqpoint{4.244662in}{3.904626in}}%
\pgfpathlineto{\pgfqpoint{4.249726in}{3.905731in}}%
\pgfpathlineto{\pgfqpoint{4.254790in}{3.914379in}}%
\pgfpathlineto{\pgfqpoint{4.259854in}{3.917167in}}%
\pgfpathlineto{\pgfqpoint{4.269982in}{3.939478in}}%
\pgfpathlineto{\pgfqpoint{4.275046in}{3.940793in}}%
\pgfpathlineto{\pgfqpoint{4.280110in}{3.944277in}}%
\pgfpathlineto{\pgfqpoint{4.285174in}{3.962229in}}%
\pgfpathlineto{\pgfqpoint{4.290238in}{3.964694in}}%
\pgfpathlineto{\pgfqpoint{4.295302in}{3.968977in}}%
\pgfpathlineto{\pgfqpoint{4.300366in}{3.970229in}}%
\pgfpathlineto{\pgfqpoint{4.305430in}{3.970290in}}%
\pgfpathlineto{\pgfqpoint{4.310494in}{3.971713in}}%
\pgfpathlineto{\pgfqpoint{4.315558in}{3.978321in}}%
\pgfpathlineto{\pgfqpoint{4.320622in}{3.982041in}}%
\pgfpathlineto{\pgfqpoint{4.325686in}{4.005720in}}%
\pgfpathlineto{\pgfqpoint{4.330750in}{4.015761in}}%
\pgfpathlineto{\pgfqpoint{4.335814in}{4.021240in}}%
\pgfpathlineto{\pgfqpoint{4.340878in}{4.031501in}}%
\pgfpathlineto{\pgfqpoint{4.345942in}{4.037838in}}%
\pgfpathlineto{\pgfqpoint{4.351006in}{4.047570in}}%
\pgfpathlineto{\pgfqpoint{4.356070in}{4.054023in}}%
\pgfpathlineto{\pgfqpoint{4.361134in}{4.063653in}}%
\pgfpathlineto{\pgfqpoint{4.366198in}{4.076777in}}%
\pgfpathlineto{\pgfqpoint{4.371262in}{4.106426in}}%
\pgfpathlineto{\pgfqpoint{4.376326in}{4.116805in}}%
\pgfpathlineto{\pgfqpoint{4.381390in}{4.135347in}}%
\pgfpathlineto{\pgfqpoint{4.386454in}{4.170360in}}%
\pgfpathlineto{\pgfqpoint{4.391518in}{4.184638in}}%
\pgfpathlineto{\pgfqpoint{4.396582in}{4.185769in}}%
\pgfpathlineto{\pgfqpoint{4.401646in}{4.190273in}}%
\pgfpathlineto{\pgfqpoint{4.411774in}{4.243792in}}%
\pgfpathlineto{\pgfqpoint{4.416838in}{4.245909in}}%
\pgfpathlineto{\pgfqpoint{4.421902in}{4.251847in}}%
\pgfpathlineto{\pgfqpoint{4.426966in}{4.290970in}}%
\pgfpathlineto{\pgfqpoint{4.432030in}{4.298110in}}%
\pgfpathlineto{\pgfqpoint{4.437094in}{4.301478in}}%
\pgfpathlineto{\pgfqpoint{4.442158in}{4.319910in}}%
\pgfpathlineto{\pgfqpoint{4.447222in}{4.368279in}}%
\pgfpathlineto{\pgfqpoint{4.452286in}{4.372763in}}%
\pgfpathlineto{\pgfqpoint{4.457350in}{4.380787in}}%
\pgfpathlineto{\pgfqpoint{4.462414in}{4.403377in}}%
\pgfpathlineto{\pgfqpoint{4.467478in}{4.415244in}}%
\pgfpathlineto{\pgfqpoint{4.472541in}{4.430220in}}%
\pgfpathlineto{\pgfqpoint{4.477605in}{4.450151in}}%
\pgfpathlineto{\pgfqpoint{4.482669in}{4.454641in}}%
\pgfpathlineto{\pgfqpoint{4.492797in}{4.487518in}}%
\pgfpathlineto{\pgfqpoint{4.497861in}{4.493087in}}%
\pgfpathlineto{\pgfqpoint{4.507989in}{4.497612in}}%
\pgfpathlineto{\pgfqpoint{4.513053in}{4.498457in}}%
\pgfpathlineto{\pgfqpoint{4.518117in}{4.522037in}}%
\pgfpathlineto{\pgfqpoint{4.523181in}{4.525160in}}%
\pgfpathlineto{\pgfqpoint{4.528245in}{4.549170in}}%
\pgfpathlineto{\pgfqpoint{4.533309in}{4.550611in}}%
\pgfpathlineto{\pgfqpoint{4.538373in}{4.568735in}}%
\pgfpathlineto{\pgfqpoint{4.548501in}{4.575794in}}%
\pgfpathlineto{\pgfqpoint{4.553565in}{4.590802in}}%
\pgfpathlineto{\pgfqpoint{4.558629in}{4.597238in}}%
\pgfpathlineto{\pgfqpoint{6.244936in}{4.597238in}}%
\pgfpathlineto{\pgfqpoint{6.244936in}{4.597238in}}%
\pgfusepath{stroke}%
\end{pgfscope}%
\begin{pgfscope}%
\pgfpathrectangle{\pgfqpoint{0.725193in}{0.571603in}}{\pgfqpoint{5.524807in}{4.025635in}}%
\pgfusepath{clip}%
\pgfsetbuttcap%
\pgfsetroundjoin%
\pgfsetlinewidth{2.007500pt}%
\definecolor{currentstroke}{rgb}{0.917647,0.372549,0.580392}%
\pgfsetstrokecolor{currentstroke}%
\pgfsetdash{{2.000000pt}{3.300000pt}}{0.000000pt}%
\pgfpathmoveto{\pgfqpoint{0.725193in}{1.162536in}}%
\pgfpathlineto{\pgfqpoint{0.730257in}{1.162536in}}%
\pgfpathlineto{\pgfqpoint{0.735321in}{1.257415in}}%
\pgfpathlineto{\pgfqpoint{1.292359in}{1.257415in}}%
\pgfpathlineto{\pgfqpoint{1.297423in}{1.331010in}}%
\pgfpathlineto{\pgfqpoint{1.707606in}{1.331010in}}%
\pgfpathlineto{\pgfqpoint{1.712670in}{1.391140in}}%
\pgfpathlineto{\pgfqpoint{2.102597in}{1.391140in}}%
\pgfpathlineto{\pgfqpoint{2.107661in}{1.441980in}}%
\pgfpathlineto{\pgfqpoint{2.289964in}{1.441980in}}%
\pgfpathlineto{\pgfqpoint{2.295028in}{1.486020in}}%
\pgfpathlineto{\pgfqpoint{2.370988in}{1.486020in}}%
\pgfpathlineto{\pgfqpoint{2.376052in}{1.524865in}}%
\pgfpathlineto{\pgfqpoint{2.436820in}{1.524865in}}%
\pgfpathlineto{\pgfqpoint{2.441884in}{1.559614in}}%
\pgfpathlineto{\pgfqpoint{2.517844in}{1.559614in}}%
\pgfpathlineto{\pgfqpoint{2.522908in}{1.591047in}}%
\pgfpathlineto{\pgfqpoint{2.538099in}{1.591047in}}%
\pgfpathlineto{\pgfqpoint{2.543163in}{1.619744in}}%
\pgfpathlineto{\pgfqpoint{2.583675in}{1.619744in}}%
\pgfpathlineto{\pgfqpoint{2.588739in}{1.646143in}}%
\pgfpathlineto{\pgfqpoint{2.624187in}{1.646143in}}%
\pgfpathlineto{\pgfqpoint{2.629251in}{1.670584in}}%
\pgfpathlineto{\pgfqpoint{2.659635in}{1.670584in}}%
\pgfpathlineto{\pgfqpoint{2.664699in}{1.693338in}}%
\pgfpathlineto{\pgfqpoint{2.690019in}{1.693338in}}%
\pgfpathlineto{\pgfqpoint{2.695083in}{1.714624in}}%
\pgfpathlineto{\pgfqpoint{2.715339in}{1.714624in}}%
\pgfpathlineto{\pgfqpoint{2.720403in}{1.734618in}}%
\pgfpathlineto{\pgfqpoint{2.750787in}{1.734618in}}%
\pgfpathlineto{\pgfqpoint{2.755851in}{1.753469in}}%
\pgfpathlineto{\pgfqpoint{2.765979in}{1.753469in}}%
\pgfpathlineto{\pgfqpoint{2.771043in}{1.771301in}}%
\pgfpathlineto{\pgfqpoint{2.781171in}{1.771301in}}%
\pgfpathlineto{\pgfqpoint{2.786235in}{1.788218in}}%
\pgfpathlineto{\pgfqpoint{2.811555in}{1.788218in}}%
\pgfpathlineto{\pgfqpoint{2.816619in}{1.804309in}}%
\pgfpathlineto{\pgfqpoint{2.841939in}{1.804309in}}%
\pgfpathlineto{\pgfqpoint{2.847003in}{1.819652in}}%
\pgfpathlineto{\pgfqpoint{2.857130in}{1.819652in}}%
\pgfpathlineto{\pgfqpoint{2.862194in}{1.834312in}}%
\pgfpathlineto{\pgfqpoint{2.882450in}{1.834312in}}%
\pgfpathlineto{\pgfqpoint{2.887514in}{1.848348in}}%
\pgfpathlineto{\pgfqpoint{2.892578in}{1.848348in}}%
\pgfpathlineto{\pgfqpoint{2.897642in}{1.861812in}}%
\pgfpathlineto{\pgfqpoint{2.917898in}{1.861812in}}%
\pgfpathlineto{\pgfqpoint{2.922962in}{1.874747in}}%
\pgfpathlineto{\pgfqpoint{2.928026in}{1.874747in}}%
\pgfpathlineto{\pgfqpoint{2.933090in}{1.887194in}}%
\pgfpathlineto{\pgfqpoint{2.943218in}{1.887194in}}%
\pgfpathlineto{\pgfqpoint{2.948282in}{1.899188in}}%
\pgfpathlineto{\pgfqpoint{2.958410in}{1.899188in}}%
\pgfpathlineto{\pgfqpoint{2.963474in}{1.910762in}}%
\pgfpathlineto{\pgfqpoint{2.983730in}{1.910762in}}%
\pgfpathlineto{\pgfqpoint{2.988794in}{1.921942in}}%
\pgfpathlineto{\pgfqpoint{2.998922in}{1.921942in}}%
\pgfpathlineto{\pgfqpoint{3.009050in}{1.943228in}}%
\pgfpathlineto{\pgfqpoint{3.014114in}{1.943228in}}%
\pgfpathlineto{\pgfqpoint{3.019178in}{1.953376in}}%
\pgfpathlineto{\pgfqpoint{3.024242in}{1.953376in}}%
\pgfpathlineto{\pgfqpoint{3.029306in}{1.963222in}}%
\pgfpathlineto{\pgfqpoint{3.049562in}{1.963222in}}%
\pgfpathlineto{\pgfqpoint{3.059690in}{1.982073in}}%
\pgfpathlineto{\pgfqpoint{3.069818in}{1.982073in}}%
\pgfpathlineto{\pgfqpoint{3.074882in}{1.999905in}}%
\pgfpathlineto{\pgfqpoint{3.090074in}{2.024965in}}%
\pgfpathlineto{\pgfqpoint{3.095138in}{2.024965in}}%
\pgfpathlineto{\pgfqpoint{3.100202in}{2.040673in}}%
\pgfpathlineto{\pgfqpoint{3.105266in}{2.048256in}}%
\pgfpathlineto{\pgfqpoint{3.115394in}{2.048256in}}%
\pgfpathlineto{\pgfqpoint{3.120458in}{2.055667in}}%
\pgfpathlineto{\pgfqpoint{3.130586in}{2.096947in}}%
\pgfpathlineto{\pgfqpoint{3.135650in}{2.096947in}}%
\pgfpathlineto{\pgfqpoint{3.140714in}{2.103351in}}%
\pgfpathlineto{\pgfqpoint{3.145778in}{2.103351in}}%
\pgfpathlineto{\pgfqpoint{3.150842in}{2.109633in}}%
\pgfpathlineto{\pgfqpoint{3.160970in}{2.133630in}}%
\pgfpathlineto{\pgfqpoint{3.166034in}{2.139366in}}%
\pgfpathlineto{\pgfqpoint{3.171098in}{2.139366in}}%
\pgfpathlineto{\pgfqpoint{3.176161in}{2.145003in}}%
\pgfpathlineto{\pgfqpoint{3.181225in}{2.155998in}}%
\pgfpathlineto{\pgfqpoint{3.186289in}{2.161361in}}%
\pgfpathlineto{\pgfqpoint{3.191353in}{2.161361in}}%
\pgfpathlineto{\pgfqpoint{3.196417in}{2.166638in}}%
\pgfpathlineto{\pgfqpoint{3.201481in}{2.166638in}}%
\pgfpathlineto{\pgfqpoint{3.206545in}{2.171832in}}%
\pgfpathlineto{\pgfqpoint{3.211609in}{2.181980in}}%
\pgfpathlineto{\pgfqpoint{3.221737in}{2.181980in}}%
\pgfpathlineto{\pgfqpoint{3.231865in}{2.191826in}}%
\pgfpathlineto{\pgfqpoint{3.241993in}{2.191826in}}%
\pgfpathlineto{\pgfqpoint{3.247057in}{2.215226in}}%
\pgfpathlineto{\pgfqpoint{3.252121in}{2.219714in}}%
\pgfpathlineto{\pgfqpoint{3.257185in}{2.219714in}}%
\pgfpathlineto{\pgfqpoint{3.262249in}{2.232820in}}%
\pgfpathlineto{\pgfqpoint{3.267313in}{2.232820in}}%
\pgfpathlineto{\pgfqpoint{3.277441in}{2.241277in}}%
\pgfpathlineto{\pgfqpoint{3.282505in}{2.265420in}}%
\pgfpathlineto{\pgfqpoint{3.287569in}{2.265420in}}%
\pgfpathlineto{\pgfqpoint{3.292633in}{2.280586in}}%
\pgfpathlineto{\pgfqpoint{3.302761in}{2.287916in}}%
\pgfpathlineto{\pgfqpoint{3.307825in}{2.287916in}}%
\pgfpathlineto{\pgfqpoint{3.317953in}{2.302103in}}%
\pgfpathlineto{\pgfqpoint{3.323017in}{2.305556in}}%
\pgfpathlineto{\pgfqpoint{3.328081in}{2.312357in}}%
\pgfpathlineto{\pgfqpoint{3.338209in}{2.319020in}}%
\pgfpathlineto{\pgfqpoint{3.343273in}{2.338237in}}%
\pgfpathlineto{\pgfqpoint{3.348337in}{2.347442in}}%
\pgfpathlineto{\pgfqpoint{3.353401in}{2.353438in}}%
\pgfpathlineto{\pgfqpoint{3.358465in}{2.353438in}}%
\pgfpathlineto{\pgfqpoint{3.368593in}{2.359328in}}%
\pgfpathlineto{\pgfqpoint{3.373657in}{2.365114in}}%
\pgfpathlineto{\pgfqpoint{3.378721in}{2.373607in}}%
\pgfpathlineto{\pgfqpoint{3.383785in}{2.376391in}}%
\pgfpathlineto{\pgfqpoint{3.393913in}{2.405549in}}%
\pgfpathlineto{\pgfqpoint{3.398977in}{2.408076in}}%
\pgfpathlineto{\pgfqpoint{3.409105in}{2.429990in}}%
\pgfpathlineto{\pgfqpoint{3.414169in}{2.432338in}}%
\pgfpathlineto{\pgfqpoint{3.424297in}{2.432338in}}%
\pgfpathlineto{\pgfqpoint{3.434425in}{2.436983in}}%
\pgfpathlineto{\pgfqpoint{3.439489in}{2.443830in}}%
\pgfpathlineto{\pgfqpoint{3.444553in}{2.443830in}}%
\pgfpathlineto{\pgfqpoint{3.449617in}{2.446082in}}%
\pgfpathlineto{\pgfqpoint{3.454681in}{2.446082in}}%
\pgfpathlineto{\pgfqpoint{3.459745in}{2.452745in}}%
\pgfpathlineto{\pgfqpoint{3.469873in}{2.457113in}}%
\pgfpathlineto{\pgfqpoint{3.480001in}{2.465680in}}%
\pgfpathlineto{\pgfqpoint{3.485065in}{2.467787in}}%
\pgfpathlineto{\pgfqpoint{3.490129in}{2.467787in}}%
\pgfpathlineto{\pgfqpoint{3.495193in}{2.486171in}}%
\pgfpathlineto{\pgfqpoint{3.500256in}{2.486171in}}%
\pgfpathlineto{\pgfqpoint{3.505320in}{2.492078in}}%
\pgfpathlineto{\pgfqpoint{3.510384in}{2.494024in}}%
\pgfpathlineto{\pgfqpoint{3.515448in}{2.494024in}}%
\pgfpathlineto{\pgfqpoint{3.520512in}{2.501694in}}%
\pgfpathlineto{\pgfqpoint{3.525576in}{2.503584in}}%
\pgfpathlineto{\pgfqpoint{3.545832in}{2.525458in}}%
\pgfpathlineto{\pgfqpoint{3.550896in}{2.527217in}}%
\pgfpathlineto{\pgfqpoint{3.555960in}{2.532438in}}%
\pgfpathlineto{\pgfqpoint{3.561024in}{2.535874in}}%
\pgfpathlineto{\pgfqpoint{3.566088in}{2.537578in}}%
\pgfpathlineto{\pgfqpoint{3.571152in}{2.540961in}}%
\pgfpathlineto{\pgfqpoint{3.576216in}{2.542639in}}%
\pgfpathlineto{\pgfqpoint{3.581280in}{2.542639in}}%
\pgfpathlineto{\pgfqpoint{3.586344in}{2.547624in}}%
\pgfpathlineto{\pgfqpoint{3.591408in}{2.550906in}}%
\pgfpathlineto{\pgfqpoint{3.596472in}{2.560559in}}%
\pgfpathlineto{\pgfqpoint{3.601536in}{2.576046in}}%
\pgfpathlineto{\pgfqpoint{3.606600in}{2.580553in}}%
\pgfpathlineto{\pgfqpoint{3.611664in}{2.580553in}}%
\pgfpathlineto{\pgfqpoint{3.616728in}{2.582042in}}%
\pgfpathlineto{\pgfqpoint{3.621792in}{2.585000in}}%
\pgfpathlineto{\pgfqpoint{3.642048in}{2.618569in}}%
\pgfpathlineto{\pgfqpoint{3.647112in}{2.621218in}}%
\pgfpathlineto{\pgfqpoint{3.657240in}{2.634153in}}%
\pgfpathlineto{\pgfqpoint{3.662304in}{2.634153in}}%
\pgfpathlineto{\pgfqpoint{3.672432in}{2.660942in}}%
\pgfpathlineto{\pgfqpoint{3.677496in}{2.662109in}}%
\pgfpathlineto{\pgfqpoint{3.692688in}{2.667885in}}%
\pgfpathlineto{\pgfqpoint{3.697752in}{2.676922in}}%
\pgfpathlineto{\pgfqpoint{3.702816in}{2.678034in}}%
\pgfpathlineto{\pgfqpoint{3.707880in}{2.681349in}}%
\pgfpathlineto{\pgfqpoint{3.712944in}{2.687880in}}%
\pgfpathlineto{\pgfqpoint{3.718008in}{2.688956in}}%
\pgfpathlineto{\pgfqpoint{3.723072in}{2.702634in}}%
\pgfpathlineto{\pgfqpoint{3.733200in}{2.703663in}}%
\pgfpathlineto{\pgfqpoint{3.738264in}{2.707747in}}%
\pgfpathlineto{\pgfqpoint{3.743328in}{2.715767in}}%
\pgfpathlineto{\pgfqpoint{3.753456in}{2.752294in}}%
\pgfpathlineto{\pgfqpoint{3.758520in}{2.753179in}}%
\pgfpathlineto{\pgfqpoint{3.768648in}{2.767031in}}%
\pgfpathlineto{\pgfqpoint{3.773712in}{2.768723in}}%
\pgfpathlineto{\pgfqpoint{3.778776in}{2.775402in}}%
\pgfpathlineto{\pgfqpoint{3.783840in}{2.787574in}}%
\pgfpathlineto{\pgfqpoint{3.793968in}{2.793104in}}%
\pgfpathlineto{\pgfqpoint{3.799032in}{2.794667in}}%
\pgfpathlineto{\pgfqpoint{3.804096in}{2.803893in}}%
\pgfpathlineto{\pgfqpoint{3.809160in}{2.816536in}}%
\pgfpathlineto{\pgfqpoint{3.824351in}{2.830816in}}%
\pgfpathlineto{\pgfqpoint{3.829415in}{2.845840in}}%
\pgfpathlineto{\pgfqpoint{3.839543in}{2.853104in}}%
\pgfpathlineto{\pgfqpoint{3.844607in}{2.860849in}}%
\pgfpathlineto{\pgfqpoint{3.849671in}{2.864023in}}%
\pgfpathlineto{\pgfqpoint{3.854735in}{2.864654in}}%
\pgfpathlineto{\pgfqpoint{3.859799in}{2.866541in}}%
\pgfpathlineto{\pgfqpoint{3.864863in}{2.870901in}}%
\pgfpathlineto{\pgfqpoint{3.869927in}{2.883646in}}%
\pgfpathlineto{\pgfqpoint{3.874991in}{2.885427in}}%
\pgfpathlineto{\pgfqpoint{3.885119in}{2.897633in}}%
\pgfpathlineto{\pgfqpoint{3.890183in}{2.899340in}}%
\pgfpathlineto{\pgfqpoint{3.895247in}{2.903290in}}%
\pgfpathlineto{\pgfqpoint{3.900311in}{2.905526in}}%
\pgfpathlineto{\pgfqpoint{3.905375in}{2.906082in}}%
\pgfpathlineto{\pgfqpoint{3.910439in}{2.910502in}}%
\pgfpathlineto{\pgfqpoint{3.915503in}{2.911050in}}%
\pgfpathlineto{\pgfqpoint{3.920567in}{2.914863in}}%
\pgfpathlineto{\pgfqpoint{3.925631in}{2.916484in}}%
\pgfpathlineto{\pgfqpoint{3.935759in}{2.927611in}}%
\pgfpathlineto{\pgfqpoint{3.940823in}{2.938375in}}%
\pgfpathlineto{\pgfqpoint{3.945887in}{2.941886in}}%
\pgfpathlineto{\pgfqpoint{3.950951in}{2.942882in}}%
\pgfpathlineto{\pgfqpoint{3.961079in}{2.952201in}}%
\pgfpathlineto{\pgfqpoint{3.966143in}{2.956525in}}%
\pgfpathlineto{\pgfqpoint{3.976271in}{2.957953in}}%
\pgfpathlineto{\pgfqpoint{3.981335in}{2.968246in}}%
\pgfpathlineto{\pgfqpoint{3.991463in}{2.975985in}}%
\pgfpathlineto{\pgfqpoint{3.996527in}{2.976884in}}%
\pgfpathlineto{\pgfqpoint{4.001591in}{2.983547in}}%
\pgfpathlineto{\pgfqpoint{4.021847in}{2.987915in}}%
\pgfpathlineto{\pgfqpoint{4.037039in}{2.998589in}}%
\pgfpathlineto{\pgfqpoint{4.042103in}{3.008114in}}%
\pgfpathlineto{\pgfqpoint{4.047167in}{3.011767in}}%
\pgfpathlineto{\pgfqpoint{4.052231in}{3.013378in}}%
\pgfpathlineto{\pgfqpoint{4.062359in}{3.024049in}}%
\pgfpathlineto{\pgfqpoint{4.067423in}{3.027146in}}%
\pgfpathlineto{\pgfqpoint{4.072487in}{3.037015in}}%
\pgfpathlineto{\pgfqpoint{4.082615in}{3.039992in}}%
\pgfpathlineto{\pgfqpoint{4.087679in}{3.045140in}}%
\pgfpathlineto{\pgfqpoint{4.092743in}{3.046233in}}%
\pgfpathlineto{\pgfqpoint{4.097807in}{3.052714in}}%
\pgfpathlineto{\pgfqpoint{4.102871in}{3.066334in}}%
\pgfpathlineto{\pgfqpoint{4.112999in}{3.068720in}}%
\pgfpathlineto{\pgfqpoint{4.118063in}{3.069060in}}%
\pgfpathlineto{\pgfqpoint{4.123127in}{3.083986in}}%
\pgfpathlineto{\pgfqpoint{4.128191in}{3.107754in}}%
\pgfpathlineto{\pgfqpoint{4.133255in}{3.117857in}}%
\pgfpathlineto{\pgfqpoint{4.138319in}{3.121351in}}%
\pgfpathlineto{\pgfqpoint{4.148446in}{3.125380in}}%
\pgfpathlineto{\pgfqpoint{4.153510in}{3.134964in}}%
\pgfpathlineto{\pgfqpoint{4.158574in}{3.136074in}}%
\pgfpathlineto{\pgfqpoint{4.163638in}{3.150433in}}%
\pgfpathlineto{\pgfqpoint{4.168702in}{3.154648in}}%
\pgfpathlineto{\pgfqpoint{4.178830in}{3.188925in}}%
\pgfpathlineto{\pgfqpoint{4.183894in}{3.192445in}}%
\pgfpathlineto{\pgfqpoint{4.188958in}{3.198916in}}%
\pgfpathlineto{\pgfqpoint{4.194022in}{3.199602in}}%
\pgfpathlineto{\pgfqpoint{4.199086in}{3.204814in}}%
\pgfpathlineto{\pgfqpoint{4.209214in}{3.207947in}}%
\pgfpathlineto{\pgfqpoint{4.214278in}{3.222113in}}%
\pgfpathlineto{\pgfqpoint{4.224406in}{3.223178in}}%
\pgfpathlineto{\pgfqpoint{4.234534in}{3.228661in}}%
\pgfpathlineto{\pgfqpoint{4.239598in}{3.238955in}}%
\pgfpathlineto{\pgfqpoint{4.244662in}{3.240169in}}%
\pgfpathlineto{\pgfqpoint{4.249726in}{3.259390in}}%
\pgfpathlineto{\pgfqpoint{4.259854in}{3.270996in}}%
\pgfpathlineto{\pgfqpoint{4.264918in}{3.280783in}}%
\pgfpathlineto{\pgfqpoint{4.275046in}{3.328573in}}%
\pgfpathlineto{\pgfqpoint{4.300366in}{3.355838in}}%
\pgfpathlineto{\pgfqpoint{4.305430in}{3.356406in}}%
\pgfpathlineto{\pgfqpoint{4.315558in}{3.363846in}}%
\pgfpathlineto{\pgfqpoint{4.320622in}{3.373557in}}%
\pgfpathlineto{\pgfqpoint{4.325686in}{3.374634in}}%
\pgfpathlineto{\pgfqpoint{4.330750in}{3.379302in}}%
\pgfpathlineto{\pgfqpoint{4.335814in}{3.382072in}}%
\pgfpathlineto{\pgfqpoint{4.345942in}{3.383383in}}%
\pgfpathlineto{\pgfqpoint{4.351006in}{3.400215in}}%
\pgfpathlineto{\pgfqpoint{4.356070in}{3.423954in}}%
\pgfpathlineto{\pgfqpoint{4.361134in}{3.434428in}}%
\pgfpathlineto{\pgfqpoint{4.366198in}{3.437329in}}%
\pgfpathlineto{\pgfqpoint{4.376326in}{3.440205in}}%
\pgfpathlineto{\pgfqpoint{4.381390in}{3.446098in}}%
\pgfpathlineto{\pgfqpoint{4.386454in}{3.454851in}}%
\pgfpathlineto{\pgfqpoint{4.391518in}{3.455167in}}%
\pgfpathlineto{\pgfqpoint{4.396582in}{3.478131in}}%
\pgfpathlineto{\pgfqpoint{4.401646in}{3.481549in}}%
\pgfpathlineto{\pgfqpoint{4.406710in}{3.490273in}}%
\pgfpathlineto{\pgfqpoint{4.416838in}{3.491029in}}%
\pgfpathlineto{\pgfqpoint{4.421902in}{3.493474in}}%
\pgfpathlineto{\pgfqpoint{4.426966in}{3.504349in}}%
\pgfpathlineto{\pgfqpoint{4.442158in}{3.515490in}}%
\pgfpathlineto{\pgfqpoint{4.447222in}{3.516715in}}%
\pgfpathlineto{\pgfqpoint{4.462414in}{3.529055in}}%
\pgfpathlineto{\pgfqpoint{4.467478in}{3.536869in}}%
\pgfpathlineto{\pgfqpoint{4.472541in}{3.537115in}}%
\pgfpathlineto{\pgfqpoint{4.482669in}{3.544662in}}%
\pgfpathlineto{\pgfqpoint{4.492797in}{3.545543in}}%
\pgfpathlineto{\pgfqpoint{4.497861in}{3.549203in}}%
\pgfpathlineto{\pgfqpoint{4.502925in}{3.551411in}}%
\pgfpathlineto{\pgfqpoint{4.507989in}{3.551725in}}%
\pgfpathlineto{\pgfqpoint{4.523181in}{3.556868in}}%
\pgfpathlineto{\pgfqpoint{4.528245in}{3.565339in}}%
\pgfpathlineto{\pgfqpoint{4.533309in}{3.565565in}}%
\pgfpathlineto{\pgfqpoint{4.538373in}{3.573231in}}%
\pgfpathlineto{\pgfqpoint{4.543437in}{3.576525in}}%
\pgfpathlineto{\pgfqpoint{4.548501in}{3.587133in}}%
\pgfpathlineto{\pgfqpoint{4.558629in}{3.593766in}}%
\pgfpathlineto{\pgfqpoint{4.563693in}{3.602363in}}%
\pgfpathlineto{\pgfqpoint{4.568757in}{3.623682in}}%
\pgfpathlineto{\pgfqpoint{4.573821in}{3.630616in}}%
\pgfpathlineto{\pgfqpoint{4.583949in}{3.633815in}}%
\pgfpathlineto{\pgfqpoint{4.589013in}{3.637590in}}%
\pgfpathlineto{\pgfqpoint{4.594077in}{3.638858in}}%
\pgfpathlineto{\pgfqpoint{4.599141in}{3.645484in}}%
\pgfpathlineto{\pgfqpoint{4.604205in}{3.654576in}}%
\pgfpathlineto{\pgfqpoint{4.619397in}{3.664988in}}%
\pgfpathlineto{\pgfqpoint{4.624461in}{3.673456in}}%
\pgfpathlineto{\pgfqpoint{4.634589in}{3.679907in}}%
\pgfpathlineto{\pgfqpoint{4.639653in}{3.689767in}}%
\pgfpathlineto{\pgfqpoint{4.644717in}{3.713007in}}%
\pgfpathlineto{\pgfqpoint{4.649781in}{3.721140in}}%
\pgfpathlineto{\pgfqpoint{4.654845in}{3.722453in}}%
\pgfpathlineto{\pgfqpoint{4.659909in}{3.727790in}}%
\pgfpathlineto{\pgfqpoint{4.664973in}{3.728434in}}%
\pgfpathlineto{\pgfqpoint{4.670037in}{3.730313in}}%
\pgfpathlineto{\pgfqpoint{4.675101in}{3.743612in}}%
\pgfpathlineto{\pgfqpoint{4.680165in}{3.745799in}}%
\pgfpathlineto{\pgfqpoint{4.685229in}{3.752274in}}%
\pgfpathlineto{\pgfqpoint{4.690293in}{3.753000in}}%
\pgfpathlineto{\pgfqpoint{4.695357in}{3.756311in}}%
\pgfpathlineto{\pgfqpoint{4.700421in}{3.785472in}}%
\pgfpathlineto{\pgfqpoint{4.705485in}{3.791561in}}%
\pgfpathlineto{\pgfqpoint{4.710549in}{3.792320in}}%
\pgfpathlineto{\pgfqpoint{4.720677in}{3.807633in}}%
\pgfpathlineto{\pgfqpoint{4.725741in}{3.807777in}}%
\pgfpathlineto{\pgfqpoint{4.730805in}{3.825535in}}%
\pgfpathlineto{\pgfqpoint{4.735869in}{3.826799in}}%
\pgfpathlineto{\pgfqpoint{4.740933in}{3.832781in}}%
\pgfpathlineto{\pgfqpoint{4.745997in}{3.833851in}}%
\pgfpathlineto{\pgfqpoint{4.751061in}{3.854300in}}%
\pgfpathlineto{\pgfqpoint{4.761189in}{3.861131in}}%
\pgfpathlineto{\pgfqpoint{4.766253in}{3.891300in}}%
\pgfpathlineto{\pgfqpoint{4.771317in}{3.905463in}}%
\pgfpathlineto{\pgfqpoint{4.776381in}{3.905570in}}%
\pgfpathlineto{\pgfqpoint{4.781445in}{3.909760in}}%
\pgfpathlineto{\pgfqpoint{4.786509in}{3.916610in}}%
\pgfpathlineto{\pgfqpoint{4.791573in}{3.917440in}}%
\pgfpathlineto{\pgfqpoint{4.796636in}{3.949132in}}%
\pgfpathlineto{\pgfqpoint{4.806764in}{3.970478in}}%
\pgfpathlineto{\pgfqpoint{4.811828in}{3.976771in}}%
\pgfpathlineto{\pgfqpoint{4.816892in}{4.000765in}}%
\pgfpathlineto{\pgfqpoint{4.821956in}{4.042910in}}%
\pgfpathlineto{\pgfqpoint{4.827020in}{4.043228in}}%
\pgfpathlineto{\pgfqpoint{4.832084in}{4.059150in}}%
\pgfpathlineto{\pgfqpoint{4.842212in}{4.078673in}}%
\pgfpathlineto{\pgfqpoint{4.847276in}{4.084766in}}%
\pgfpathlineto{\pgfqpoint{4.852340in}{4.085374in}}%
\pgfpathlineto{\pgfqpoint{4.857404in}{4.088828in}}%
\pgfpathlineto{\pgfqpoint{4.862468in}{4.089597in}}%
\pgfpathlineto{\pgfqpoint{4.867532in}{4.094222in}}%
\pgfpathlineto{\pgfqpoint{4.872596in}{4.101508in}}%
\pgfpathlineto{\pgfqpoint{4.877660in}{4.106482in}}%
\pgfpathlineto{\pgfqpoint{4.882724in}{4.120380in}}%
\pgfpathlineto{\pgfqpoint{4.887788in}{4.120786in}}%
\pgfpathlineto{\pgfqpoint{4.892852in}{4.126210in}}%
\pgfpathlineto{\pgfqpoint{4.897916in}{4.157626in}}%
\pgfpathlineto{\pgfqpoint{4.902980in}{4.162110in}}%
\pgfpathlineto{\pgfqpoint{4.908044in}{4.164483in}}%
\pgfpathlineto{\pgfqpoint{4.913108in}{4.173091in}}%
\pgfpathlineto{\pgfqpoint{4.918172in}{4.174663in}}%
\pgfpathlineto{\pgfqpoint{4.923236in}{4.180303in}}%
\pgfpathlineto{\pgfqpoint{4.928300in}{4.196580in}}%
\pgfpathlineto{\pgfqpoint{4.933364in}{4.202243in}}%
\pgfpathlineto{\pgfqpoint{4.938428in}{4.203083in}}%
\pgfpathlineto{\pgfqpoint{4.943492in}{4.224748in}}%
\pgfpathlineto{\pgfqpoint{4.953620in}{4.240644in}}%
\pgfpathlineto{\pgfqpoint{4.958684in}{4.242989in}}%
\pgfpathlineto{\pgfqpoint{4.963748in}{4.243847in}}%
\pgfpathlineto{\pgfqpoint{4.984004in}{4.287467in}}%
\pgfpathlineto{\pgfqpoint{4.989068in}{4.310164in}}%
\pgfpathlineto{\pgfqpoint{4.994132in}{4.320148in}}%
\pgfpathlineto{\pgfqpoint{4.999196in}{4.321436in}}%
\pgfpathlineto{\pgfqpoint{5.004260in}{4.368364in}}%
\pgfpathlineto{\pgfqpoint{5.009324in}{4.373506in}}%
\pgfpathlineto{\pgfqpoint{5.014388in}{4.386742in}}%
\pgfpathlineto{\pgfqpoint{5.019452in}{4.388852in}}%
\pgfpathlineto{\pgfqpoint{5.024516in}{4.405433in}}%
\pgfpathlineto{\pgfqpoint{5.029580in}{4.413913in}}%
\pgfpathlineto{\pgfqpoint{5.039708in}{4.415502in}}%
\pgfpathlineto{\pgfqpoint{5.044772in}{4.425084in}}%
\pgfpathlineto{\pgfqpoint{5.049836in}{4.442202in}}%
\pgfpathlineto{\pgfqpoint{5.054900in}{4.445202in}}%
\pgfpathlineto{\pgfqpoint{5.059964in}{4.445683in}}%
\pgfpathlineto{\pgfqpoint{5.065028in}{4.460064in}}%
\pgfpathlineto{\pgfqpoint{5.070092in}{4.467283in}}%
\pgfpathlineto{\pgfqpoint{5.075156in}{4.468187in}}%
\pgfpathlineto{\pgfqpoint{5.080220in}{4.471960in}}%
\pgfpathlineto{\pgfqpoint{5.085284in}{4.477260in}}%
\pgfpathlineto{\pgfqpoint{5.090348in}{4.493463in}}%
\pgfpathlineto{\pgfqpoint{5.095412in}{4.501342in}}%
\pgfpathlineto{\pgfqpoint{5.100476in}{4.506328in}}%
\pgfpathlineto{\pgfqpoint{5.110604in}{4.511163in}}%
\pgfpathlineto{\pgfqpoint{5.115668in}{4.520380in}}%
\pgfpathlineto{\pgfqpoint{5.120731in}{4.522806in}}%
\pgfpathlineto{\pgfqpoint{5.125795in}{4.539806in}}%
\pgfpathlineto{\pgfqpoint{5.130859in}{4.544422in}}%
\pgfpathlineto{\pgfqpoint{5.135923in}{4.545670in}}%
\pgfpathlineto{\pgfqpoint{5.140987in}{4.564416in}}%
\pgfpathlineto{\pgfqpoint{5.146051in}{4.572953in}}%
\pgfpathlineto{\pgfqpoint{5.151115in}{4.579139in}}%
\pgfpathlineto{\pgfqpoint{5.156179in}{4.597238in}}%
\pgfpathlineto{\pgfqpoint{6.244936in}{4.597238in}}%
\pgfpathlineto{\pgfqpoint{6.244936in}{4.597238in}}%
\pgfusepath{stroke}%
\end{pgfscope}%
\begin{pgfscope}%
\pgfpathrectangle{\pgfqpoint{0.725193in}{0.571603in}}{\pgfqpoint{5.524807in}{4.025635in}}%
\pgfusepath{clip}%
\pgfsetrectcap%
\pgfsetroundjoin%
\pgfsetlinewidth{2.007500pt}%
\definecolor{currentstroke}{rgb}{0.615686,0.007843,0.843137}%
\pgfsetstrokecolor{currentstroke}%
\pgfsetdash{}{0pt}%
\pgfpathmoveto{\pgfqpoint{0.859077in}{0.561603in}}%
\pgfpathlineto{\pgfqpoint{0.861921in}{0.613815in}}%
\pgfpathlineto{\pgfqpoint{0.866985in}{0.794384in}}%
\pgfpathlineto{\pgfqpoint{0.872049in}{0.837870in}}%
\pgfpathlineto{\pgfqpoint{0.877113in}{0.838214in}}%
\pgfpathlineto{\pgfqpoint{0.882177in}{0.851528in}}%
\pgfpathlineto{\pgfqpoint{0.887241in}{0.854169in}}%
\pgfpathlineto{\pgfqpoint{0.892305in}{0.855462in}}%
\pgfpathlineto{\pgfqpoint{0.897369in}{0.858270in}}%
\pgfpathlineto{\pgfqpoint{0.917624in}{0.861087in}}%
\pgfpathlineto{\pgfqpoint{0.932816in}{0.861772in}}%
\pgfpathlineto{\pgfqpoint{0.937880in}{0.864378in}}%
\pgfpathlineto{\pgfqpoint{0.948008in}{0.865328in}}%
\pgfpathlineto{\pgfqpoint{0.953072in}{0.866777in}}%
\pgfpathlineto{\pgfqpoint{0.973328in}{0.868162in}}%
\pgfpathlineto{\pgfqpoint{1.003712in}{0.870814in}}%
\pgfpathlineto{\pgfqpoint{1.044224in}{0.873135in}}%
\pgfpathlineto{\pgfqpoint{1.191080in}{0.883351in}}%
\pgfpathlineto{\pgfqpoint{1.211336in}{0.884619in}}%
\pgfpathlineto{\pgfqpoint{1.221464in}{0.885689in}}%
\pgfpathlineto{\pgfqpoint{1.267039in}{0.887631in}}%
\pgfpathlineto{\pgfqpoint{1.277167in}{0.889370in}}%
\pgfpathlineto{\pgfqpoint{1.297423in}{0.889879in}}%
\pgfpathlineto{\pgfqpoint{1.317679in}{0.892790in}}%
\pgfpathlineto{\pgfqpoint{1.342999in}{0.893993in}}%
\pgfpathlineto{\pgfqpoint{1.358191in}{0.895870in}}%
\pgfpathlineto{\pgfqpoint{1.368319in}{0.897056in}}%
\pgfpathlineto{\pgfqpoint{1.373383in}{0.899090in}}%
\pgfpathlineto{\pgfqpoint{1.403767in}{0.901898in}}%
\pgfpathlineto{\pgfqpoint{1.413895in}{0.903316in}}%
\pgfpathlineto{\pgfqpoint{1.439215in}{0.906202in}}%
\pgfpathlineto{\pgfqpoint{1.444279in}{0.907721in}}%
\pgfpathlineto{\pgfqpoint{1.454407in}{0.908384in}}%
\pgfpathlineto{\pgfqpoint{1.459471in}{0.910592in}}%
\pgfpathlineto{\pgfqpoint{1.494919in}{0.914023in}}%
\pgfpathlineto{\pgfqpoint{1.499983in}{0.914657in}}%
\pgfpathlineto{\pgfqpoint{1.505047in}{0.917749in}}%
\pgfpathlineto{\pgfqpoint{1.515175in}{0.918755in}}%
\pgfpathlineto{\pgfqpoint{1.520239in}{0.920704in}}%
\pgfpathlineto{\pgfqpoint{1.530367in}{0.921559in}}%
\pgfpathlineto{\pgfqpoint{1.535431in}{0.925648in}}%
\pgfpathlineto{\pgfqpoint{1.555686in}{0.928624in}}%
\pgfpathlineto{\pgfqpoint{1.560750in}{0.930478in}}%
\pgfpathlineto{\pgfqpoint{1.565814in}{0.930758in}}%
\pgfpathlineto{\pgfqpoint{1.570878in}{0.934118in}}%
\pgfpathlineto{\pgfqpoint{1.581006in}{0.935999in}}%
\pgfpathlineto{\pgfqpoint{1.606326in}{0.939032in}}%
\pgfpathlineto{\pgfqpoint{1.611390in}{0.940413in}}%
\pgfpathlineto{\pgfqpoint{1.641774in}{0.941792in}}%
\pgfpathlineto{\pgfqpoint{1.667094in}{0.946468in}}%
\pgfpathlineto{\pgfqpoint{1.672158in}{0.946514in}}%
\pgfpathlineto{\pgfqpoint{1.677222in}{0.955419in}}%
\pgfpathlineto{\pgfqpoint{1.687350in}{0.956356in}}%
\pgfpathlineto{\pgfqpoint{1.692414in}{0.958587in}}%
\pgfpathlineto{\pgfqpoint{1.697478in}{0.958820in}}%
\pgfpathlineto{\pgfqpoint{1.702542in}{0.960450in}}%
\pgfpathlineto{\pgfqpoint{1.707606in}{0.966004in}}%
\pgfpathlineto{\pgfqpoint{1.712670in}{0.966280in}}%
\pgfpathlineto{\pgfqpoint{1.722798in}{0.969764in}}%
\pgfpathlineto{\pgfqpoint{1.727862in}{0.970144in}}%
\pgfpathlineto{\pgfqpoint{1.732926in}{0.973034in}}%
\pgfpathlineto{\pgfqpoint{1.748118in}{0.974665in}}%
\pgfpathlineto{\pgfqpoint{1.758246in}{0.977892in}}%
\pgfpathlineto{\pgfqpoint{1.763310in}{0.978121in}}%
\pgfpathlineto{\pgfqpoint{1.768374in}{0.979927in}}%
\pgfpathlineto{\pgfqpoint{1.773438in}{0.985713in}}%
\pgfpathlineto{\pgfqpoint{1.783566in}{0.990340in}}%
\pgfpathlineto{\pgfqpoint{1.793694in}{0.997161in}}%
\pgfpathlineto{\pgfqpoint{1.808886in}{0.998965in}}%
\pgfpathlineto{\pgfqpoint{1.813950in}{0.999318in}}%
\pgfpathlineto{\pgfqpoint{1.824078in}{1.003767in}}%
\pgfpathlineto{\pgfqpoint{1.834206in}{1.006198in}}%
\pgfpathlineto{\pgfqpoint{1.839270in}{1.009845in}}%
\pgfpathlineto{\pgfqpoint{1.849398in}{1.010519in}}%
\pgfpathlineto{\pgfqpoint{1.854462in}{1.013901in}}%
\pgfpathlineto{\pgfqpoint{1.864590in}{1.017139in}}%
\pgfpathlineto{\pgfqpoint{1.869654in}{1.023404in}}%
\pgfpathlineto{\pgfqpoint{1.874718in}{1.023931in}}%
\pgfpathlineto{\pgfqpoint{1.879781in}{1.026008in}}%
\pgfpathlineto{\pgfqpoint{1.884845in}{1.026313in}}%
\pgfpathlineto{\pgfqpoint{1.889909in}{1.029969in}}%
\pgfpathlineto{\pgfqpoint{1.910165in}{1.032051in}}%
\pgfpathlineto{\pgfqpoint{1.915229in}{1.042213in}}%
\pgfpathlineto{\pgfqpoint{1.920293in}{1.042262in}}%
\pgfpathlineto{\pgfqpoint{1.930421in}{1.053100in}}%
\pgfpathlineto{\pgfqpoint{1.940549in}{1.057937in}}%
\pgfpathlineto{\pgfqpoint{1.945613in}{1.058088in}}%
\pgfpathlineto{\pgfqpoint{1.950677in}{1.060035in}}%
\pgfpathlineto{\pgfqpoint{1.955741in}{1.063317in}}%
\pgfpathlineto{\pgfqpoint{1.960805in}{1.070663in}}%
\pgfpathlineto{\pgfqpoint{1.965869in}{1.071428in}}%
\pgfpathlineto{\pgfqpoint{1.970933in}{1.094638in}}%
\pgfpathlineto{\pgfqpoint{1.975997in}{1.097243in}}%
\pgfpathlineto{\pgfqpoint{1.981061in}{1.098410in}}%
\pgfpathlineto{\pgfqpoint{1.991189in}{1.103807in}}%
\pgfpathlineto{\pgfqpoint{1.996253in}{1.105487in}}%
\pgfpathlineto{\pgfqpoint{2.001317in}{1.110042in}}%
\pgfpathlineto{\pgfqpoint{2.006381in}{1.110506in}}%
\pgfpathlineto{\pgfqpoint{2.011445in}{1.113132in}}%
\pgfpathlineto{\pgfqpoint{2.016509in}{1.113479in}}%
\pgfpathlineto{\pgfqpoint{2.021573in}{1.115441in}}%
\pgfpathlineto{\pgfqpoint{2.026637in}{1.119261in}}%
\pgfpathlineto{\pgfqpoint{2.031701in}{1.119670in}}%
\pgfpathlineto{\pgfqpoint{2.036765in}{1.126375in}}%
\pgfpathlineto{\pgfqpoint{2.051957in}{1.131465in}}%
\pgfpathlineto{\pgfqpoint{2.057021in}{1.132908in}}%
\pgfpathlineto{\pgfqpoint{2.062085in}{1.140863in}}%
\pgfpathlineto{\pgfqpoint{2.072213in}{1.148091in}}%
\pgfpathlineto{\pgfqpoint{2.082341in}{1.158323in}}%
\pgfpathlineto{\pgfqpoint{2.087405in}{1.164642in}}%
\pgfpathlineto{\pgfqpoint{2.092469in}{1.164881in}}%
\pgfpathlineto{\pgfqpoint{2.097533in}{1.167430in}}%
\pgfpathlineto{\pgfqpoint{2.102597in}{1.167481in}}%
\pgfpathlineto{\pgfqpoint{2.107661in}{1.174696in}}%
\pgfpathlineto{\pgfqpoint{2.112725in}{1.178404in}}%
\pgfpathlineto{\pgfqpoint{2.117789in}{1.184828in}}%
\pgfpathlineto{\pgfqpoint{2.127917in}{1.186241in}}%
\pgfpathlineto{\pgfqpoint{2.132981in}{1.189259in}}%
\pgfpathlineto{\pgfqpoint{2.148173in}{1.191348in}}%
\pgfpathlineto{\pgfqpoint{2.153237in}{1.193484in}}%
\pgfpathlineto{\pgfqpoint{2.158301in}{1.193491in}}%
\pgfpathlineto{\pgfqpoint{2.163365in}{1.195989in}}%
\pgfpathlineto{\pgfqpoint{2.168429in}{1.196240in}}%
\pgfpathlineto{\pgfqpoint{2.173493in}{1.199794in}}%
\pgfpathlineto{\pgfqpoint{2.178557in}{1.206420in}}%
\pgfpathlineto{\pgfqpoint{2.183621in}{1.206514in}}%
\pgfpathlineto{\pgfqpoint{2.188685in}{1.212225in}}%
\pgfpathlineto{\pgfqpoint{2.193749in}{1.215979in}}%
\pgfpathlineto{\pgfqpoint{2.198813in}{1.221755in}}%
\pgfpathlineto{\pgfqpoint{2.203876in}{1.229628in}}%
\pgfpathlineto{\pgfqpoint{2.208940in}{1.233941in}}%
\pgfpathlineto{\pgfqpoint{2.214004in}{1.234188in}}%
\pgfpathlineto{\pgfqpoint{2.224132in}{1.241465in}}%
\pgfpathlineto{\pgfqpoint{2.229196in}{1.245934in}}%
\pgfpathlineto{\pgfqpoint{2.234260in}{1.254659in}}%
\pgfpathlineto{\pgfqpoint{2.244388in}{1.255685in}}%
\pgfpathlineto{\pgfqpoint{2.249452in}{1.263515in}}%
\pgfpathlineto{\pgfqpoint{2.254516in}{1.266705in}}%
\pgfpathlineto{\pgfqpoint{2.264644in}{1.287561in}}%
\pgfpathlineto{\pgfqpoint{2.269708in}{1.289951in}}%
\pgfpathlineto{\pgfqpoint{2.274772in}{1.294668in}}%
\pgfpathlineto{\pgfqpoint{2.279836in}{1.295117in}}%
\pgfpathlineto{\pgfqpoint{2.284900in}{1.297890in}}%
\pgfpathlineto{\pgfqpoint{2.289964in}{1.298518in}}%
\pgfpathlineto{\pgfqpoint{2.295028in}{1.310443in}}%
\pgfpathlineto{\pgfqpoint{2.300092in}{1.312720in}}%
\pgfpathlineto{\pgfqpoint{2.305156in}{1.318385in}}%
\pgfpathlineto{\pgfqpoint{2.315284in}{1.324241in}}%
\pgfpathlineto{\pgfqpoint{2.320348in}{1.328030in}}%
\pgfpathlineto{\pgfqpoint{2.325412in}{1.328325in}}%
\pgfpathlineto{\pgfqpoint{2.330476in}{1.338622in}}%
\pgfpathlineto{\pgfqpoint{2.335540in}{1.342449in}}%
\pgfpathlineto{\pgfqpoint{2.340604in}{1.344071in}}%
\pgfpathlineto{\pgfqpoint{2.355796in}{1.344649in}}%
\pgfpathlineto{\pgfqpoint{2.360860in}{1.348600in}}%
\pgfpathlineto{\pgfqpoint{2.370988in}{1.350021in}}%
\pgfpathlineto{\pgfqpoint{2.376052in}{1.351172in}}%
\pgfpathlineto{\pgfqpoint{2.381116in}{1.361542in}}%
\pgfpathlineto{\pgfqpoint{2.386180in}{1.365493in}}%
\pgfpathlineto{\pgfqpoint{2.391244in}{1.365518in}}%
\pgfpathlineto{\pgfqpoint{2.396308in}{1.367963in}}%
\pgfpathlineto{\pgfqpoint{2.401372in}{1.382394in}}%
\pgfpathlineto{\pgfqpoint{2.411500in}{1.388806in}}%
\pgfpathlineto{\pgfqpoint{2.416564in}{1.389687in}}%
\pgfpathlineto{\pgfqpoint{2.436820in}{1.410895in}}%
\pgfpathlineto{\pgfqpoint{2.446948in}{1.426182in}}%
\pgfpathlineto{\pgfqpoint{2.452012in}{1.429681in}}%
\pgfpathlineto{\pgfqpoint{2.457076in}{1.430206in}}%
\pgfpathlineto{\pgfqpoint{2.462140in}{1.433925in}}%
\pgfpathlineto{\pgfqpoint{2.467204in}{1.434570in}}%
\pgfpathlineto{\pgfqpoint{2.472268in}{1.441911in}}%
\pgfpathlineto{\pgfqpoint{2.477332in}{1.444615in}}%
\pgfpathlineto{\pgfqpoint{2.482396in}{1.445624in}}%
\pgfpathlineto{\pgfqpoint{2.487460in}{1.452991in}}%
\pgfpathlineto{\pgfqpoint{2.502652in}{1.464428in}}%
\pgfpathlineto{\pgfqpoint{2.512780in}{1.468123in}}%
\pgfpathlineto{\pgfqpoint{2.517844in}{1.477681in}}%
\pgfpathlineto{\pgfqpoint{2.533035in}{1.481353in}}%
\pgfpathlineto{\pgfqpoint{2.543163in}{1.483560in}}%
\pgfpathlineto{\pgfqpoint{2.548227in}{1.491552in}}%
\pgfpathlineto{\pgfqpoint{2.553291in}{1.495126in}}%
\pgfpathlineto{\pgfqpoint{2.563419in}{1.496742in}}%
\pgfpathlineto{\pgfqpoint{2.568483in}{1.502056in}}%
\pgfpathlineto{\pgfqpoint{2.573547in}{1.502557in}}%
\pgfpathlineto{\pgfqpoint{2.583675in}{1.511578in}}%
\pgfpathlineto{\pgfqpoint{2.588739in}{1.512919in}}%
\pgfpathlineto{\pgfqpoint{2.598867in}{1.529241in}}%
\pgfpathlineto{\pgfqpoint{2.603931in}{1.531348in}}%
\pgfpathlineto{\pgfqpoint{2.608995in}{1.544472in}}%
\pgfpathlineto{\pgfqpoint{2.614059in}{1.545562in}}%
\pgfpathlineto{\pgfqpoint{2.619123in}{1.560575in}}%
\pgfpathlineto{\pgfqpoint{2.624187in}{1.565131in}}%
\pgfpathlineto{\pgfqpoint{2.629251in}{1.566035in}}%
\pgfpathlineto{\pgfqpoint{2.634315in}{1.571879in}}%
\pgfpathlineto{\pgfqpoint{2.639379in}{1.573394in}}%
\pgfpathlineto{\pgfqpoint{2.644443in}{1.578731in}}%
\pgfpathlineto{\pgfqpoint{2.649507in}{1.592186in}}%
\pgfpathlineto{\pgfqpoint{2.654571in}{1.592614in}}%
\pgfpathlineto{\pgfqpoint{2.659635in}{1.594804in}}%
\pgfpathlineto{\pgfqpoint{2.664699in}{1.595244in}}%
\pgfpathlineto{\pgfqpoint{2.669763in}{1.597509in}}%
\pgfpathlineto{\pgfqpoint{2.674827in}{1.615698in}}%
\pgfpathlineto{\pgfqpoint{2.679891in}{1.619913in}}%
\pgfpathlineto{\pgfqpoint{2.684955in}{1.639356in}}%
\pgfpathlineto{\pgfqpoint{2.690019in}{1.640665in}}%
\pgfpathlineto{\pgfqpoint{2.695083in}{1.646974in}}%
\pgfpathlineto{\pgfqpoint{2.700147in}{1.647092in}}%
\pgfpathlineto{\pgfqpoint{2.705211in}{1.648741in}}%
\pgfpathlineto{\pgfqpoint{2.710275in}{1.654682in}}%
\pgfpathlineto{\pgfqpoint{2.715339in}{1.663030in}}%
\pgfpathlineto{\pgfqpoint{2.720403in}{1.667736in}}%
\pgfpathlineto{\pgfqpoint{2.725467in}{1.685983in}}%
\pgfpathlineto{\pgfqpoint{2.730531in}{1.692263in}}%
\pgfpathlineto{\pgfqpoint{2.740659in}{1.695423in}}%
\pgfpathlineto{\pgfqpoint{2.745723in}{1.697279in}}%
\pgfpathlineto{\pgfqpoint{2.750787in}{1.703057in}}%
\pgfpathlineto{\pgfqpoint{2.760915in}{1.707845in}}%
\pgfpathlineto{\pgfqpoint{2.771043in}{1.708965in}}%
\pgfpathlineto{\pgfqpoint{2.781171in}{1.720257in}}%
\pgfpathlineto{\pgfqpoint{2.786235in}{1.724480in}}%
\pgfpathlineto{\pgfqpoint{2.791299in}{1.725436in}}%
\pgfpathlineto{\pgfqpoint{2.796363in}{1.738255in}}%
\pgfpathlineto{\pgfqpoint{2.801427in}{1.746147in}}%
\pgfpathlineto{\pgfqpoint{2.806491in}{1.758465in}}%
\pgfpathlineto{\pgfqpoint{2.811555in}{1.760027in}}%
\pgfpathlineto{\pgfqpoint{2.816619in}{1.763862in}}%
\pgfpathlineto{\pgfqpoint{2.821683in}{1.769152in}}%
\pgfpathlineto{\pgfqpoint{2.826747in}{1.769429in}}%
\pgfpathlineto{\pgfqpoint{2.831811in}{1.773338in}}%
\pgfpathlineto{\pgfqpoint{2.841939in}{1.777736in}}%
\pgfpathlineto{\pgfqpoint{2.852066in}{1.778579in}}%
\pgfpathlineto{\pgfqpoint{2.857130in}{1.786461in}}%
\pgfpathlineto{\pgfqpoint{2.862194in}{1.790849in}}%
\pgfpathlineto{\pgfqpoint{2.867258in}{1.791243in}}%
\pgfpathlineto{\pgfqpoint{2.872322in}{1.792885in}}%
\pgfpathlineto{\pgfqpoint{2.882450in}{1.794892in}}%
\pgfpathlineto{\pgfqpoint{2.907770in}{1.812826in}}%
\pgfpathlineto{\pgfqpoint{2.928026in}{1.815941in}}%
\pgfpathlineto{\pgfqpoint{2.933090in}{1.821081in}}%
\pgfpathlineto{\pgfqpoint{2.938154in}{1.838110in}}%
\pgfpathlineto{\pgfqpoint{2.948282in}{1.849617in}}%
\pgfpathlineto{\pgfqpoint{2.958410in}{1.851055in}}%
\pgfpathlineto{\pgfqpoint{2.963474in}{1.853624in}}%
\pgfpathlineto{\pgfqpoint{2.968538in}{1.862724in}}%
\pgfpathlineto{\pgfqpoint{2.973602in}{1.864015in}}%
\pgfpathlineto{\pgfqpoint{2.978666in}{1.870475in}}%
\pgfpathlineto{\pgfqpoint{2.983730in}{1.872119in}}%
\pgfpathlineto{\pgfqpoint{2.993858in}{1.889853in}}%
\pgfpathlineto{\pgfqpoint{2.998922in}{1.893459in}}%
\pgfpathlineto{\pgfqpoint{3.003986in}{1.893607in}}%
\pgfpathlineto{\pgfqpoint{3.009050in}{1.895151in}}%
\pgfpathlineto{\pgfqpoint{3.014114in}{1.908908in}}%
\pgfpathlineto{\pgfqpoint{3.019178in}{1.917980in}}%
\pgfpathlineto{\pgfqpoint{3.024242in}{1.919537in}}%
\pgfpathlineto{\pgfqpoint{3.029306in}{1.927509in}}%
\pgfpathlineto{\pgfqpoint{3.039434in}{1.928371in}}%
\pgfpathlineto{\pgfqpoint{3.044498in}{1.931122in}}%
\pgfpathlineto{\pgfqpoint{3.049562in}{1.936591in}}%
\pgfpathlineto{\pgfqpoint{3.054626in}{1.947424in}}%
\pgfpathlineto{\pgfqpoint{3.059690in}{1.953778in}}%
\pgfpathlineto{\pgfqpoint{3.064754in}{1.957564in}}%
\pgfpathlineto{\pgfqpoint{3.069818in}{1.958860in}}%
\pgfpathlineto{\pgfqpoint{3.074882in}{1.970081in}}%
\pgfpathlineto{\pgfqpoint{3.079946in}{1.977687in}}%
\pgfpathlineto{\pgfqpoint{3.085010in}{1.979648in}}%
\pgfpathlineto{\pgfqpoint{3.090074in}{1.998447in}}%
\pgfpathlineto{\pgfqpoint{3.095138in}{2.003391in}}%
\pgfpathlineto{\pgfqpoint{3.105266in}{2.009316in}}%
\pgfpathlineto{\pgfqpoint{3.110330in}{2.024680in}}%
\pgfpathlineto{\pgfqpoint{3.115394in}{2.034672in}}%
\pgfpathlineto{\pgfqpoint{3.120458in}{2.039596in}}%
\pgfpathlineto{\pgfqpoint{3.125522in}{2.041102in}}%
\pgfpathlineto{\pgfqpoint{3.135650in}{2.051778in}}%
\pgfpathlineto{\pgfqpoint{3.145778in}{2.054929in}}%
\pgfpathlineto{\pgfqpoint{3.150842in}{2.055345in}}%
\pgfpathlineto{\pgfqpoint{3.155906in}{2.060346in}}%
\pgfpathlineto{\pgfqpoint{3.160970in}{2.071333in}}%
\pgfpathlineto{\pgfqpoint{3.166034in}{2.088938in}}%
\pgfpathlineto{\pgfqpoint{3.176161in}{2.099566in}}%
\pgfpathlineto{\pgfqpoint{3.181225in}{2.100042in}}%
\pgfpathlineto{\pgfqpoint{3.191353in}{2.102718in}}%
\pgfpathlineto{\pgfqpoint{3.196417in}{2.103071in}}%
\pgfpathlineto{\pgfqpoint{3.201481in}{2.109344in}}%
\pgfpathlineto{\pgfqpoint{3.206545in}{2.110130in}}%
\pgfpathlineto{\pgfqpoint{3.211609in}{2.114108in}}%
\pgfpathlineto{\pgfqpoint{3.216673in}{2.121124in}}%
\pgfpathlineto{\pgfqpoint{3.236929in}{2.128493in}}%
\pgfpathlineto{\pgfqpoint{3.241993in}{2.135418in}}%
\pgfpathlineto{\pgfqpoint{3.247057in}{2.138542in}}%
\pgfpathlineto{\pgfqpoint{3.252121in}{2.150830in}}%
\pgfpathlineto{\pgfqpoint{3.257185in}{2.152176in}}%
\pgfpathlineto{\pgfqpoint{3.262249in}{2.154765in}}%
\pgfpathlineto{\pgfqpoint{3.272377in}{2.161499in}}%
\pgfpathlineto{\pgfqpoint{3.287569in}{2.166187in}}%
\pgfpathlineto{\pgfqpoint{3.297697in}{2.185837in}}%
\pgfpathlineto{\pgfqpoint{3.307825in}{2.192859in}}%
\pgfpathlineto{\pgfqpoint{3.317953in}{2.196153in}}%
\pgfpathlineto{\pgfqpoint{3.323017in}{2.201077in}}%
\pgfpathlineto{\pgfqpoint{3.338209in}{2.204612in}}%
\pgfpathlineto{\pgfqpoint{3.343273in}{2.210679in}}%
\pgfpathlineto{\pgfqpoint{3.348337in}{2.227279in}}%
\pgfpathlineto{\pgfqpoint{3.353401in}{2.235168in}}%
\pgfpathlineto{\pgfqpoint{3.358465in}{2.236992in}}%
\pgfpathlineto{\pgfqpoint{3.363529in}{2.240593in}}%
\pgfpathlineto{\pgfqpoint{3.368593in}{2.246307in}}%
\pgfpathlineto{\pgfqpoint{3.378721in}{2.248608in}}%
\pgfpathlineto{\pgfqpoint{3.383785in}{2.252124in}}%
\pgfpathlineto{\pgfqpoint{3.388849in}{2.254181in}}%
\pgfpathlineto{\pgfqpoint{3.398977in}{2.272696in}}%
\pgfpathlineto{\pgfqpoint{3.409105in}{2.273849in}}%
\pgfpathlineto{\pgfqpoint{3.414169in}{2.279061in}}%
\pgfpathlineto{\pgfqpoint{3.429361in}{2.280381in}}%
\pgfpathlineto{\pgfqpoint{3.434425in}{2.288937in}}%
\pgfpathlineto{\pgfqpoint{3.439489in}{2.290435in}}%
\pgfpathlineto{\pgfqpoint{3.454681in}{2.298616in}}%
\pgfpathlineto{\pgfqpoint{3.459745in}{2.303910in}}%
\pgfpathlineto{\pgfqpoint{3.464809in}{2.304545in}}%
\pgfpathlineto{\pgfqpoint{3.469873in}{2.319682in}}%
\pgfpathlineto{\pgfqpoint{3.474937in}{2.321086in}}%
\pgfpathlineto{\pgfqpoint{3.480001in}{2.324320in}}%
\pgfpathlineto{\pgfqpoint{3.485065in}{2.326175in}}%
\pgfpathlineto{\pgfqpoint{3.500256in}{2.342022in}}%
\pgfpathlineto{\pgfqpoint{3.505320in}{2.352345in}}%
\pgfpathlineto{\pgfqpoint{3.510384in}{2.366572in}}%
\pgfpathlineto{\pgfqpoint{3.515448in}{2.373498in}}%
\pgfpathlineto{\pgfqpoint{3.520512in}{2.387611in}}%
\pgfpathlineto{\pgfqpoint{3.525576in}{2.394804in}}%
\pgfpathlineto{\pgfqpoint{3.530640in}{2.397599in}}%
\pgfpathlineto{\pgfqpoint{3.535704in}{2.410652in}}%
\pgfpathlineto{\pgfqpoint{3.550896in}{2.425894in}}%
\pgfpathlineto{\pgfqpoint{3.561024in}{2.429068in}}%
\pgfpathlineto{\pgfqpoint{3.566088in}{2.437336in}}%
\pgfpathlineto{\pgfqpoint{3.571152in}{2.469080in}}%
\pgfpathlineto{\pgfqpoint{3.591408in}{2.472777in}}%
\pgfpathlineto{\pgfqpoint{3.596472in}{2.476127in}}%
\pgfpathlineto{\pgfqpoint{3.606600in}{2.484732in}}%
\pgfpathlineto{\pgfqpoint{3.611664in}{2.487022in}}%
\pgfpathlineto{\pgfqpoint{3.616728in}{2.496041in}}%
\pgfpathlineto{\pgfqpoint{3.621792in}{2.508705in}}%
\pgfpathlineto{\pgfqpoint{3.626856in}{2.516526in}}%
\pgfpathlineto{\pgfqpoint{3.631920in}{2.530928in}}%
\pgfpathlineto{\pgfqpoint{3.636984in}{2.532699in}}%
\pgfpathlineto{\pgfqpoint{3.647112in}{2.532790in}}%
\pgfpathlineto{\pgfqpoint{3.657240in}{2.539313in}}%
\pgfpathlineto{\pgfqpoint{3.662304in}{2.542240in}}%
\pgfpathlineto{\pgfqpoint{3.672432in}{2.545571in}}%
\pgfpathlineto{\pgfqpoint{3.677496in}{2.548801in}}%
\pgfpathlineto{\pgfqpoint{3.682560in}{2.554267in}}%
\pgfpathlineto{\pgfqpoint{3.687624in}{2.557267in}}%
\pgfpathlineto{\pgfqpoint{3.692688in}{2.573483in}}%
\pgfpathlineto{\pgfqpoint{3.702816in}{2.587277in}}%
\pgfpathlineto{\pgfqpoint{3.707880in}{2.588864in}}%
\pgfpathlineto{\pgfqpoint{3.718008in}{2.603478in}}%
\pgfpathlineto{\pgfqpoint{3.723072in}{2.607703in}}%
\pgfpathlineto{\pgfqpoint{3.733200in}{2.623203in}}%
\pgfpathlineto{\pgfqpoint{3.738264in}{2.626635in}}%
\pgfpathlineto{\pgfqpoint{3.743328in}{2.646633in}}%
\pgfpathlineto{\pgfqpoint{3.753456in}{2.658600in}}%
\pgfpathlineto{\pgfqpoint{3.758520in}{2.668038in}}%
\pgfpathlineto{\pgfqpoint{3.768648in}{2.669844in}}%
\pgfpathlineto{\pgfqpoint{3.773712in}{2.674251in}}%
\pgfpathlineto{\pgfqpoint{3.778776in}{2.683470in}}%
\pgfpathlineto{\pgfqpoint{3.783840in}{2.688563in}}%
\pgfpathlineto{\pgfqpoint{3.788904in}{2.691544in}}%
\pgfpathlineto{\pgfqpoint{3.793968in}{2.692710in}}%
\pgfpathlineto{\pgfqpoint{3.804096in}{2.705656in}}%
\pgfpathlineto{\pgfqpoint{3.809160in}{2.714460in}}%
\pgfpathlineto{\pgfqpoint{3.824351in}{2.732071in}}%
\pgfpathlineto{\pgfqpoint{3.834479in}{2.739555in}}%
\pgfpathlineto{\pgfqpoint{3.844607in}{2.760935in}}%
\pgfpathlineto{\pgfqpoint{3.849671in}{2.765855in}}%
\pgfpathlineto{\pgfqpoint{3.854735in}{2.776399in}}%
\pgfpathlineto{\pgfqpoint{3.859799in}{2.776472in}}%
\pgfpathlineto{\pgfqpoint{3.864863in}{2.792433in}}%
\pgfpathlineto{\pgfqpoint{3.874991in}{2.802579in}}%
\pgfpathlineto{\pgfqpoint{3.885119in}{2.810313in}}%
\pgfpathlineto{\pgfqpoint{3.890183in}{2.812300in}}%
\pgfpathlineto{\pgfqpoint{3.900311in}{2.831186in}}%
\pgfpathlineto{\pgfqpoint{3.905375in}{2.845516in}}%
\pgfpathlineto{\pgfqpoint{3.910439in}{2.856367in}}%
\pgfpathlineto{\pgfqpoint{3.915503in}{2.859755in}}%
\pgfpathlineto{\pgfqpoint{3.930695in}{2.861498in}}%
\pgfpathlineto{\pgfqpoint{3.935759in}{2.867920in}}%
\pgfpathlineto{\pgfqpoint{3.940823in}{2.869539in}}%
\pgfpathlineto{\pgfqpoint{3.945887in}{2.906146in}}%
\pgfpathlineto{\pgfqpoint{3.950951in}{2.908326in}}%
\pgfpathlineto{\pgfqpoint{3.956015in}{2.909254in}}%
\pgfpathlineto{\pgfqpoint{3.961079in}{2.919267in}}%
\pgfpathlineto{\pgfqpoint{3.971207in}{2.929783in}}%
\pgfpathlineto{\pgfqpoint{3.976271in}{2.930692in}}%
\pgfpathlineto{\pgfqpoint{3.981335in}{2.938866in}}%
\pgfpathlineto{\pgfqpoint{3.986399in}{2.944699in}}%
\pgfpathlineto{\pgfqpoint{3.996527in}{2.948587in}}%
\pgfpathlineto{\pgfqpoint{4.001591in}{2.963924in}}%
\pgfpathlineto{\pgfqpoint{4.011719in}{2.978804in}}%
\pgfpathlineto{\pgfqpoint{4.016783in}{2.979161in}}%
\pgfpathlineto{\pgfqpoint{4.026911in}{2.982489in}}%
\pgfpathlineto{\pgfqpoint{4.031975in}{2.994912in}}%
\pgfpathlineto{\pgfqpoint{4.037039in}{2.997968in}}%
\pgfpathlineto{\pgfqpoint{4.042103in}{2.998675in}}%
\pgfpathlineto{\pgfqpoint{4.057295in}{3.006910in}}%
\pgfpathlineto{\pgfqpoint{4.067423in}{3.007254in}}%
\pgfpathlineto{\pgfqpoint{4.072487in}{3.010246in}}%
\pgfpathlineto{\pgfqpoint{4.077551in}{3.018736in}}%
\pgfpathlineto{\pgfqpoint{4.082615in}{3.021573in}}%
\pgfpathlineto{\pgfqpoint{4.087679in}{3.032079in}}%
\pgfpathlineto{\pgfqpoint{4.097807in}{3.032727in}}%
\pgfpathlineto{\pgfqpoint{4.102871in}{3.044175in}}%
\pgfpathlineto{\pgfqpoint{4.112999in}{3.054122in}}%
\pgfpathlineto{\pgfqpoint{4.118063in}{3.061664in}}%
\pgfpathlineto{\pgfqpoint{4.123127in}{3.065023in}}%
\pgfpathlineto{\pgfqpoint{4.128191in}{3.066008in}}%
\pgfpathlineto{\pgfqpoint{4.133255in}{3.082910in}}%
\pgfpathlineto{\pgfqpoint{4.148446in}{3.086514in}}%
\pgfpathlineto{\pgfqpoint{4.153510in}{3.104976in}}%
\pgfpathlineto{\pgfqpoint{4.163638in}{3.106577in}}%
\pgfpathlineto{\pgfqpoint{4.168702in}{3.116625in}}%
\pgfpathlineto{\pgfqpoint{4.183894in}{3.139258in}}%
\pgfpathlineto{\pgfqpoint{4.188958in}{3.140099in}}%
\pgfpathlineto{\pgfqpoint{4.204150in}{3.154075in}}%
\pgfpathlineto{\pgfqpoint{4.209214in}{3.154909in}}%
\pgfpathlineto{\pgfqpoint{4.214278in}{3.162775in}}%
\pgfpathlineto{\pgfqpoint{4.234534in}{3.183329in}}%
\pgfpathlineto{\pgfqpoint{4.249726in}{3.184247in}}%
\pgfpathlineto{\pgfqpoint{4.254790in}{3.185355in}}%
\pgfpathlineto{\pgfqpoint{4.259854in}{3.196267in}}%
\pgfpathlineto{\pgfqpoint{4.269982in}{3.196980in}}%
\pgfpathlineto{\pgfqpoint{4.275046in}{3.198079in}}%
\pgfpathlineto{\pgfqpoint{4.285174in}{3.210886in}}%
\pgfpathlineto{\pgfqpoint{4.290238in}{3.227993in}}%
\pgfpathlineto{\pgfqpoint{4.295302in}{3.229812in}}%
\pgfpathlineto{\pgfqpoint{4.305430in}{3.231597in}}%
\pgfpathlineto{\pgfqpoint{4.310494in}{3.232249in}}%
\pgfpathlineto{\pgfqpoint{4.325686in}{3.240706in}}%
\pgfpathlineto{\pgfqpoint{4.330750in}{3.248405in}}%
\pgfpathlineto{\pgfqpoint{4.335814in}{3.249100in}}%
\pgfpathlineto{\pgfqpoint{4.340878in}{3.260343in}}%
\pgfpathlineto{\pgfqpoint{4.345942in}{3.260515in}}%
\pgfpathlineto{\pgfqpoint{4.351006in}{3.265784in}}%
\pgfpathlineto{\pgfqpoint{4.356070in}{3.267652in}}%
\pgfpathlineto{\pgfqpoint{4.366198in}{3.274448in}}%
\pgfpathlineto{\pgfqpoint{4.371262in}{3.282796in}}%
\pgfpathlineto{\pgfqpoint{4.376326in}{3.286563in}}%
\pgfpathlineto{\pgfqpoint{4.381390in}{3.306169in}}%
\pgfpathlineto{\pgfqpoint{4.386454in}{3.306577in}}%
\pgfpathlineto{\pgfqpoint{4.391518in}{3.315981in}}%
\pgfpathlineto{\pgfqpoint{4.401646in}{3.316864in}}%
\pgfpathlineto{\pgfqpoint{4.411774in}{3.323153in}}%
\pgfpathlineto{\pgfqpoint{4.416838in}{3.324110in}}%
\pgfpathlineto{\pgfqpoint{4.421902in}{3.330844in}}%
\pgfpathlineto{\pgfqpoint{4.426966in}{3.340181in}}%
\pgfpathlineto{\pgfqpoint{4.432030in}{3.343181in}}%
\pgfpathlineto{\pgfqpoint{4.437094in}{3.343677in}}%
\pgfpathlineto{\pgfqpoint{4.442158in}{3.352049in}}%
\pgfpathlineto{\pgfqpoint{4.452286in}{3.357463in}}%
\pgfpathlineto{\pgfqpoint{4.457350in}{3.360386in}}%
\pgfpathlineto{\pgfqpoint{4.462414in}{3.375053in}}%
\pgfpathlineto{\pgfqpoint{4.467478in}{3.377370in}}%
\pgfpathlineto{\pgfqpoint{4.477605in}{3.379400in}}%
\pgfpathlineto{\pgfqpoint{4.482669in}{3.383188in}}%
\pgfpathlineto{\pgfqpoint{4.487733in}{3.389228in}}%
\pgfpathlineto{\pgfqpoint{4.497861in}{3.392032in}}%
\pgfpathlineto{\pgfqpoint{4.507989in}{3.393273in}}%
\pgfpathlineto{\pgfqpoint{4.523181in}{3.404888in}}%
\pgfpathlineto{\pgfqpoint{4.528245in}{3.405127in}}%
\pgfpathlineto{\pgfqpoint{4.538373in}{3.414159in}}%
\pgfpathlineto{\pgfqpoint{4.543437in}{3.434613in}}%
\pgfpathlineto{\pgfqpoint{4.548501in}{3.441265in}}%
\pgfpathlineto{\pgfqpoint{4.553565in}{3.464443in}}%
\pgfpathlineto{\pgfqpoint{4.558629in}{3.482284in}}%
\pgfpathlineto{\pgfqpoint{4.568757in}{3.487236in}}%
\pgfpathlineto{\pgfqpoint{4.573821in}{3.507565in}}%
\pgfpathlineto{\pgfqpoint{4.583949in}{3.515289in}}%
\pgfpathlineto{\pgfqpoint{4.589013in}{3.518058in}}%
\pgfpathlineto{\pgfqpoint{4.594077in}{3.529745in}}%
\pgfpathlineto{\pgfqpoint{4.604205in}{3.535567in}}%
\pgfpathlineto{\pgfqpoint{4.609269in}{3.539750in}}%
\pgfpathlineto{\pgfqpoint{4.614333in}{3.553289in}}%
\pgfpathlineto{\pgfqpoint{4.619397in}{3.555909in}}%
\pgfpathlineto{\pgfqpoint{4.624461in}{3.561543in}}%
\pgfpathlineto{\pgfqpoint{4.629525in}{3.565148in}}%
\pgfpathlineto{\pgfqpoint{4.634589in}{3.566576in}}%
\pgfpathlineto{\pgfqpoint{4.639653in}{3.599982in}}%
\pgfpathlineto{\pgfqpoint{4.644717in}{3.604250in}}%
\pgfpathlineto{\pgfqpoint{4.649781in}{3.617272in}}%
\pgfpathlineto{\pgfqpoint{4.654845in}{3.617573in}}%
\pgfpathlineto{\pgfqpoint{4.664973in}{3.628620in}}%
\pgfpathlineto{\pgfqpoint{4.670037in}{3.638979in}}%
\pgfpathlineto{\pgfqpoint{4.680165in}{3.640412in}}%
\pgfpathlineto{\pgfqpoint{4.685229in}{3.643195in}}%
\pgfpathlineto{\pgfqpoint{4.690293in}{3.651962in}}%
\pgfpathlineto{\pgfqpoint{4.695357in}{3.667400in}}%
\pgfpathlineto{\pgfqpoint{4.700421in}{3.670515in}}%
\pgfpathlineto{\pgfqpoint{4.705485in}{3.675691in}}%
\pgfpathlineto{\pgfqpoint{4.710549in}{3.683086in}}%
\pgfpathlineto{\pgfqpoint{4.715613in}{3.684643in}}%
\pgfpathlineto{\pgfqpoint{4.720677in}{3.695803in}}%
\pgfpathlineto{\pgfqpoint{4.725741in}{3.701461in}}%
\pgfpathlineto{\pgfqpoint{4.730805in}{3.703822in}}%
\pgfpathlineto{\pgfqpoint{4.735869in}{3.712114in}}%
\pgfpathlineto{\pgfqpoint{4.751061in}{3.717890in}}%
\pgfpathlineto{\pgfqpoint{4.761189in}{3.719701in}}%
\pgfpathlineto{\pgfqpoint{4.766253in}{3.731964in}}%
\pgfpathlineto{\pgfqpoint{4.771317in}{3.736015in}}%
\pgfpathlineto{\pgfqpoint{4.781445in}{3.739384in}}%
\pgfpathlineto{\pgfqpoint{4.786509in}{3.743446in}}%
\pgfpathlineto{\pgfqpoint{4.791573in}{3.750423in}}%
\pgfpathlineto{\pgfqpoint{4.796636in}{3.762617in}}%
\pgfpathlineto{\pgfqpoint{4.801700in}{3.766414in}}%
\pgfpathlineto{\pgfqpoint{4.806764in}{3.775781in}}%
\pgfpathlineto{\pgfqpoint{4.811828in}{3.776155in}}%
\pgfpathlineto{\pgfqpoint{4.816892in}{3.790217in}}%
\pgfpathlineto{\pgfqpoint{4.821956in}{3.794232in}}%
\pgfpathlineto{\pgfqpoint{4.827020in}{3.795836in}}%
\pgfpathlineto{\pgfqpoint{4.842212in}{3.809613in}}%
\pgfpathlineto{\pgfqpoint{4.847276in}{3.816389in}}%
\pgfpathlineto{\pgfqpoint{4.852340in}{3.826183in}}%
\pgfpathlineto{\pgfqpoint{4.857404in}{3.828937in}}%
\pgfpathlineto{\pgfqpoint{4.862468in}{3.829476in}}%
\pgfpathlineto{\pgfqpoint{4.867532in}{3.831152in}}%
\pgfpathlineto{\pgfqpoint{4.872596in}{3.839104in}}%
\pgfpathlineto{\pgfqpoint{4.877660in}{3.840455in}}%
\pgfpathlineto{\pgfqpoint{4.882724in}{3.851076in}}%
\pgfpathlineto{\pgfqpoint{4.887788in}{3.858831in}}%
\pgfpathlineto{\pgfqpoint{4.892852in}{3.873315in}}%
\pgfpathlineto{\pgfqpoint{4.897916in}{3.882240in}}%
\pgfpathlineto{\pgfqpoint{4.902980in}{3.884091in}}%
\pgfpathlineto{\pgfqpoint{4.908044in}{3.890183in}}%
\pgfpathlineto{\pgfqpoint{4.913108in}{3.901530in}}%
\pgfpathlineto{\pgfqpoint{4.923236in}{3.906519in}}%
\pgfpathlineto{\pgfqpoint{4.928300in}{3.913210in}}%
\pgfpathlineto{\pgfqpoint{4.933364in}{3.924810in}}%
\pgfpathlineto{\pgfqpoint{4.938428in}{3.930099in}}%
\pgfpathlineto{\pgfqpoint{4.943492in}{3.931990in}}%
\pgfpathlineto{\pgfqpoint{4.948556in}{3.935357in}}%
\pgfpathlineto{\pgfqpoint{4.953620in}{3.953765in}}%
\pgfpathlineto{\pgfqpoint{4.963748in}{3.976893in}}%
\pgfpathlineto{\pgfqpoint{4.968812in}{3.986880in}}%
\pgfpathlineto{\pgfqpoint{4.978940in}{3.994951in}}%
\pgfpathlineto{\pgfqpoint{4.984004in}{4.001754in}}%
\pgfpathlineto{\pgfqpoint{4.989068in}{4.003991in}}%
\pgfpathlineto{\pgfqpoint{4.999196in}{4.026880in}}%
\pgfpathlineto{\pgfqpoint{5.014388in}{4.031125in}}%
\pgfpathlineto{\pgfqpoint{5.019452in}{4.039600in}}%
\pgfpathlineto{\pgfqpoint{5.024516in}{4.073356in}}%
\pgfpathlineto{\pgfqpoint{5.034644in}{4.078307in}}%
\pgfpathlineto{\pgfqpoint{5.044772in}{4.085232in}}%
\pgfpathlineto{\pgfqpoint{5.054900in}{4.088854in}}%
\pgfpathlineto{\pgfqpoint{5.059964in}{4.089464in}}%
\pgfpathlineto{\pgfqpoint{5.065028in}{4.092826in}}%
\pgfpathlineto{\pgfqpoint{5.070092in}{4.093643in}}%
\pgfpathlineto{\pgfqpoint{5.075156in}{4.103410in}}%
\pgfpathlineto{\pgfqpoint{5.080220in}{4.127518in}}%
\pgfpathlineto{\pgfqpoint{5.085284in}{4.128684in}}%
\pgfpathlineto{\pgfqpoint{5.095412in}{4.145776in}}%
\pgfpathlineto{\pgfqpoint{5.100476in}{4.159831in}}%
\pgfpathlineto{\pgfqpoint{5.105540in}{4.160478in}}%
\pgfpathlineto{\pgfqpoint{5.110604in}{4.163134in}}%
\pgfpathlineto{\pgfqpoint{5.120731in}{4.172635in}}%
\pgfpathlineto{\pgfqpoint{5.125795in}{4.172965in}}%
\pgfpathlineto{\pgfqpoint{5.130859in}{4.195154in}}%
\pgfpathlineto{\pgfqpoint{5.140987in}{4.201268in}}%
\pgfpathlineto{\pgfqpoint{5.146051in}{4.222775in}}%
\pgfpathlineto{\pgfqpoint{5.151115in}{4.227766in}}%
\pgfpathlineto{\pgfqpoint{5.156179in}{4.244521in}}%
\pgfpathlineto{\pgfqpoint{5.161243in}{4.281674in}}%
\pgfpathlineto{\pgfqpoint{5.171371in}{4.283628in}}%
\pgfpathlineto{\pgfqpoint{5.176435in}{4.287451in}}%
\pgfpathlineto{\pgfqpoint{5.181499in}{4.296009in}}%
\pgfpathlineto{\pgfqpoint{5.191627in}{4.300591in}}%
\pgfpathlineto{\pgfqpoint{5.196691in}{4.300689in}}%
\pgfpathlineto{\pgfqpoint{5.201755in}{4.304305in}}%
\pgfpathlineto{\pgfqpoint{5.206819in}{4.312240in}}%
\pgfpathlineto{\pgfqpoint{5.211883in}{4.316146in}}%
\pgfpathlineto{\pgfqpoint{5.227075in}{4.334241in}}%
\pgfpathlineto{\pgfqpoint{5.232139in}{4.334584in}}%
\pgfpathlineto{\pgfqpoint{5.237203in}{4.339926in}}%
\pgfpathlineto{\pgfqpoint{5.257459in}{4.374998in}}%
\pgfpathlineto{\pgfqpoint{5.262523in}{4.375237in}}%
\pgfpathlineto{\pgfqpoint{5.267587in}{4.377412in}}%
\pgfpathlineto{\pgfqpoint{5.272651in}{4.380803in}}%
\pgfpathlineto{\pgfqpoint{5.277715in}{4.390402in}}%
\pgfpathlineto{\pgfqpoint{5.287843in}{4.391192in}}%
\pgfpathlineto{\pgfqpoint{5.292907in}{4.403732in}}%
\pgfpathlineto{\pgfqpoint{5.297971in}{4.407147in}}%
\pgfpathlineto{\pgfqpoint{5.313163in}{4.420764in}}%
\pgfpathlineto{\pgfqpoint{5.318227in}{4.430670in}}%
\pgfpathlineto{\pgfqpoint{5.328355in}{4.438238in}}%
\pgfpathlineto{\pgfqpoint{5.333419in}{4.440867in}}%
\pgfpathlineto{\pgfqpoint{5.338483in}{4.450292in}}%
\pgfpathlineto{\pgfqpoint{5.343547in}{4.474945in}}%
\pgfpathlineto{\pgfqpoint{5.353675in}{4.480219in}}%
\pgfpathlineto{\pgfqpoint{5.358739in}{4.482381in}}%
\pgfpathlineto{\pgfqpoint{5.363803in}{4.501635in}}%
\pgfpathlineto{\pgfqpoint{5.368867in}{4.502004in}}%
\pgfpathlineto{\pgfqpoint{5.373931in}{4.517632in}}%
\pgfpathlineto{\pgfqpoint{5.389123in}{4.545569in}}%
\pgfpathlineto{\pgfqpoint{5.394187in}{4.553534in}}%
\pgfpathlineto{\pgfqpoint{5.404315in}{4.557539in}}%
\pgfpathlineto{\pgfqpoint{5.409379in}{4.561093in}}%
\pgfpathlineto{\pgfqpoint{5.419507in}{4.562348in}}%
\pgfpathlineto{\pgfqpoint{5.424571in}{4.572293in}}%
\pgfpathlineto{\pgfqpoint{5.429635in}{4.573479in}}%
\pgfpathlineto{\pgfqpoint{5.434699in}{4.580132in}}%
\pgfpathlineto{\pgfqpoint{5.439763in}{4.591336in}}%
\pgfpathlineto{\pgfqpoint{5.444826in}{4.591670in}}%
\pgfpathlineto{\pgfqpoint{5.449890in}{4.595562in}}%
\pgfpathlineto{\pgfqpoint{5.454954in}{4.597238in}}%
\pgfpathlineto{\pgfqpoint{6.244936in}{4.597238in}}%
\pgfpathlineto{\pgfqpoint{6.244936in}{4.597238in}}%
\pgfusepath{stroke}%
\end{pgfscope}%
\begin{pgfscope}%
\pgfpathrectangle{\pgfqpoint{0.725193in}{0.571603in}}{\pgfqpoint{5.524807in}{4.025635in}}%
\pgfusepath{clip}%
\pgfsetbuttcap%
\pgfsetroundjoin%
\pgfsetlinewidth{2.007500pt}%
\definecolor{currentstroke}{rgb}{0.000000,0.000000,1.000000}%
\pgfsetstrokecolor{currentstroke}%
\pgfsetdash{{7.400000pt}{3.200000pt}}{0.000000pt}%
\pgfpathmoveto{\pgfqpoint{0.859077in}{0.561603in}}%
\pgfpathlineto{\pgfqpoint{0.861921in}{0.613815in}}%
\pgfpathlineto{\pgfqpoint{0.866985in}{0.765386in}}%
\pgfpathlineto{\pgfqpoint{0.942944in}{0.765422in}}%
\pgfpathlineto{\pgfqpoint{0.948008in}{0.794384in}}%
\pgfpathlineto{\pgfqpoint{0.953072in}{0.800108in}}%
\pgfpathlineto{\pgfqpoint{1.494919in}{0.800174in}}%
\pgfpathlineto{\pgfqpoint{1.499983in}{0.831551in}}%
\pgfpathlineto{\pgfqpoint{1.682286in}{0.831611in}}%
\pgfpathlineto{\pgfqpoint{1.687350in}{0.860256in}}%
\pgfpathlineto{\pgfqpoint{1.808886in}{0.860283in}}%
\pgfpathlineto{\pgfqpoint{1.813950in}{0.884619in}}%
\pgfpathlineto{\pgfqpoint{1.819014in}{0.886661in}}%
\pgfpathlineto{\pgfqpoint{1.879781in}{0.886686in}}%
\pgfpathlineto{\pgfqpoint{1.884845in}{0.911107in}}%
\pgfpathlineto{\pgfqpoint{1.915229in}{0.911131in}}%
\pgfpathlineto{\pgfqpoint{1.920293in}{0.914657in}}%
\pgfpathlineto{\pgfqpoint{1.925357in}{0.933844in}}%
\pgfpathlineto{\pgfqpoint{1.955741in}{0.933888in}}%
\pgfpathlineto{\pgfqpoint{1.960805in}{0.955135in}}%
\pgfpathlineto{\pgfqpoint{1.986125in}{0.955176in}}%
\pgfpathlineto{\pgfqpoint{1.991189in}{0.974665in}}%
\pgfpathlineto{\pgfqpoint{2.006381in}{0.975154in}}%
\pgfpathlineto{\pgfqpoint{2.026637in}{0.975173in}}%
\pgfpathlineto{\pgfqpoint{2.031701in}{0.993990in}}%
\pgfpathlineto{\pgfqpoint{2.057021in}{0.994008in}}%
\pgfpathlineto{\pgfqpoint{2.062085in}{1.011825in}}%
\pgfpathlineto{\pgfqpoint{2.087405in}{1.011843in}}%
\pgfpathlineto{\pgfqpoint{2.092469in}{1.017139in}}%
\pgfpathlineto{\pgfqpoint{2.097533in}{1.028745in}}%
\pgfpathlineto{\pgfqpoint{2.107661in}{1.028762in}}%
\pgfpathlineto{\pgfqpoint{2.112725in}{1.030466in}}%
\pgfpathlineto{\pgfqpoint{2.117789in}{1.044840in}}%
\pgfpathlineto{\pgfqpoint{2.122853in}{1.044840in}}%
\pgfpathlineto{\pgfqpoint{2.127917in}{1.055404in}}%
\pgfpathlineto{\pgfqpoint{2.132981in}{1.060035in}}%
\pgfpathlineto{\pgfqpoint{2.138045in}{1.060185in}}%
\pgfpathlineto{\pgfqpoint{2.143109in}{1.063317in}}%
\pgfpathlineto{\pgfqpoint{2.148173in}{1.070663in}}%
\pgfpathlineto{\pgfqpoint{2.153237in}{1.074834in}}%
\pgfpathlineto{\pgfqpoint{2.163365in}{1.074863in}}%
\pgfpathlineto{\pgfqpoint{2.168429in}{1.088873in}}%
\pgfpathlineto{\pgfqpoint{2.173493in}{1.088887in}}%
\pgfpathlineto{\pgfqpoint{2.178557in}{1.102340in}}%
\pgfpathlineto{\pgfqpoint{2.193749in}{1.102353in}}%
\pgfpathlineto{\pgfqpoint{2.198813in}{1.115277in}}%
\pgfpathlineto{\pgfqpoint{2.203876in}{1.115441in}}%
\pgfpathlineto{\pgfqpoint{2.208940in}{1.119261in}}%
\pgfpathlineto{\pgfqpoint{2.214004in}{1.119670in}}%
\pgfpathlineto{\pgfqpoint{2.219068in}{1.126375in}}%
\pgfpathlineto{\pgfqpoint{2.224132in}{1.127714in}}%
\pgfpathlineto{\pgfqpoint{2.239324in}{1.127739in}}%
\pgfpathlineto{\pgfqpoint{2.249452in}{1.151287in}}%
\pgfpathlineto{\pgfqpoint{2.269708in}{1.151310in}}%
\pgfpathlineto{\pgfqpoint{2.274772in}{1.173297in}}%
\pgfpathlineto{\pgfqpoint{2.289964in}{1.173308in}}%
\pgfpathlineto{\pgfqpoint{2.300092in}{1.183760in}}%
\pgfpathlineto{\pgfqpoint{2.310220in}{1.203758in}}%
\pgfpathlineto{\pgfqpoint{2.325412in}{1.203767in}}%
\pgfpathlineto{\pgfqpoint{2.330476in}{1.206420in}}%
\pgfpathlineto{\pgfqpoint{2.335540in}{1.206514in}}%
\pgfpathlineto{\pgfqpoint{2.340604in}{1.212225in}}%
\pgfpathlineto{\pgfqpoint{2.345668in}{1.213319in}}%
\pgfpathlineto{\pgfqpoint{2.350732in}{1.215979in}}%
\pgfpathlineto{\pgfqpoint{2.355796in}{1.222603in}}%
\pgfpathlineto{\pgfqpoint{2.360860in}{1.222612in}}%
\pgfpathlineto{\pgfqpoint{2.365924in}{1.229628in}}%
\pgfpathlineto{\pgfqpoint{2.370988in}{1.231641in}}%
\pgfpathlineto{\pgfqpoint{2.381116in}{1.249006in}}%
\pgfpathlineto{\pgfqpoint{2.396308in}{1.249015in}}%
\pgfpathlineto{\pgfqpoint{2.401372in}{1.257366in}}%
\pgfpathlineto{\pgfqpoint{2.406436in}{1.263515in}}%
\pgfpathlineto{\pgfqpoint{2.411500in}{1.265511in}}%
\pgfpathlineto{\pgfqpoint{2.416564in}{1.273444in}}%
\pgfpathlineto{\pgfqpoint{2.421628in}{1.288797in}}%
\pgfpathlineto{\pgfqpoint{2.436820in}{1.288797in}}%
\pgfpathlineto{\pgfqpoint{2.441884in}{1.296210in}}%
\pgfpathlineto{\pgfqpoint{2.446948in}{1.298518in}}%
\pgfpathlineto{\pgfqpoint{2.452012in}{1.303460in}}%
\pgfpathlineto{\pgfqpoint{2.457076in}{1.310443in}}%
\pgfpathlineto{\pgfqpoint{2.462140in}{1.310540in}}%
\pgfpathlineto{\pgfqpoint{2.467204in}{1.317491in}}%
\pgfpathlineto{\pgfqpoint{2.482396in}{1.318385in}}%
\pgfpathlineto{\pgfqpoint{2.487460in}{1.330950in}}%
\pgfpathlineto{\pgfqpoint{2.517844in}{1.331010in}}%
\pgfpathlineto{\pgfqpoint{2.522908in}{1.337482in}}%
\pgfpathlineto{\pgfqpoint{2.527971in}{1.338622in}}%
\pgfpathlineto{\pgfqpoint{2.533035in}{1.342449in}}%
\pgfpathlineto{\pgfqpoint{2.538099in}{1.343888in}}%
\pgfpathlineto{\pgfqpoint{2.548227in}{1.344119in}}%
\pgfpathlineto{\pgfqpoint{2.553291in}{1.348922in}}%
\pgfpathlineto{\pgfqpoint{2.558355in}{1.351172in}}%
\pgfpathlineto{\pgfqpoint{2.568483in}{1.374166in}}%
\pgfpathlineto{\pgfqpoint{2.573547in}{1.374171in}}%
\pgfpathlineto{\pgfqpoint{2.578611in}{1.379902in}}%
\pgfpathlineto{\pgfqpoint{2.583675in}{1.391140in}}%
\pgfpathlineto{\pgfqpoint{2.624187in}{1.391140in}}%
\pgfpathlineto{\pgfqpoint{2.629251in}{1.396538in}}%
\pgfpathlineto{\pgfqpoint{2.634315in}{1.410895in}}%
\pgfpathlineto{\pgfqpoint{2.639379in}{1.412369in}}%
\pgfpathlineto{\pgfqpoint{2.644443in}{1.417483in}}%
\pgfpathlineto{\pgfqpoint{2.654571in}{1.441924in}}%
\pgfpathlineto{\pgfqpoint{2.659635in}{1.441980in}}%
\pgfpathlineto{\pgfqpoint{2.664699in}{1.444615in}}%
\pgfpathlineto{\pgfqpoint{2.669763in}{1.445624in}}%
\pgfpathlineto{\pgfqpoint{2.674827in}{1.451216in}}%
\pgfpathlineto{\pgfqpoint{2.679891in}{1.451221in}}%
\pgfpathlineto{\pgfqpoint{2.684955in}{1.457248in}}%
\pgfpathlineto{\pgfqpoint{2.690019in}{1.477681in}}%
\pgfpathlineto{\pgfqpoint{2.695083in}{1.483560in}}%
\pgfpathlineto{\pgfqpoint{2.700147in}{1.486020in}}%
\pgfpathlineto{\pgfqpoint{2.715339in}{1.486020in}}%
\pgfpathlineto{\pgfqpoint{2.730531in}{1.502056in}}%
\pgfpathlineto{\pgfqpoint{2.735595in}{1.502557in}}%
\pgfpathlineto{\pgfqpoint{2.750787in}{1.511578in}}%
\pgfpathlineto{\pgfqpoint{2.755851in}{1.513631in}}%
\pgfpathlineto{\pgfqpoint{2.760915in}{1.517397in}}%
\pgfpathlineto{\pgfqpoint{2.771043in}{1.532060in}}%
\pgfpathlineto{\pgfqpoint{2.776107in}{1.542645in}}%
\pgfpathlineto{\pgfqpoint{2.781171in}{1.559561in}}%
\pgfpathlineto{\pgfqpoint{2.811555in}{1.560575in}}%
\pgfpathlineto{\pgfqpoint{2.816619in}{1.566035in}}%
\pgfpathlineto{\pgfqpoint{2.821683in}{1.575655in}}%
\pgfpathlineto{\pgfqpoint{2.826747in}{1.578731in}}%
\pgfpathlineto{\pgfqpoint{2.831811in}{1.591047in}}%
\pgfpathlineto{\pgfqpoint{2.836875in}{1.592186in}}%
\pgfpathlineto{\pgfqpoint{2.841939in}{1.597509in}}%
\pgfpathlineto{\pgfqpoint{2.847003in}{1.616935in}}%
\pgfpathlineto{\pgfqpoint{2.852066in}{1.619744in}}%
\pgfpathlineto{\pgfqpoint{2.867258in}{1.619744in}}%
\pgfpathlineto{\pgfqpoint{2.872322in}{1.630505in}}%
\pgfpathlineto{\pgfqpoint{2.877386in}{1.630508in}}%
\pgfpathlineto{\pgfqpoint{2.882450in}{1.646143in}}%
\pgfpathlineto{\pgfqpoint{2.897642in}{1.647092in}}%
\pgfpathlineto{\pgfqpoint{2.902706in}{1.658539in}}%
\pgfpathlineto{\pgfqpoint{2.907770in}{1.658541in}}%
\pgfpathlineto{\pgfqpoint{2.912834in}{1.668168in}}%
\pgfpathlineto{\pgfqpoint{2.917898in}{1.670584in}}%
\pgfpathlineto{\pgfqpoint{2.928026in}{1.670584in}}%
\pgfpathlineto{\pgfqpoint{2.933090in}{1.672880in}}%
\pgfpathlineto{\pgfqpoint{2.948282in}{1.693338in}}%
\pgfpathlineto{\pgfqpoint{2.958410in}{1.693408in}}%
\pgfpathlineto{\pgfqpoint{2.963474in}{1.695480in}}%
\pgfpathlineto{\pgfqpoint{2.968538in}{1.695480in}}%
\pgfpathlineto{\pgfqpoint{2.973602in}{1.699818in}}%
\pgfpathlineto{\pgfqpoint{2.978666in}{1.707845in}}%
\pgfpathlineto{\pgfqpoint{2.983730in}{1.708466in}}%
\pgfpathlineto{\pgfqpoint{2.988794in}{1.714572in}}%
\pgfpathlineto{\pgfqpoint{2.998922in}{1.714624in}}%
\pgfpathlineto{\pgfqpoint{3.003986in}{1.720257in}}%
\pgfpathlineto{\pgfqpoint{3.009050in}{1.734618in}}%
\pgfpathlineto{\pgfqpoint{3.024242in}{1.734618in}}%
\pgfpathlineto{\pgfqpoint{3.029306in}{1.753418in}}%
\pgfpathlineto{\pgfqpoint{3.034370in}{1.762456in}}%
\pgfpathlineto{\pgfqpoint{3.044498in}{1.764234in}}%
\pgfpathlineto{\pgfqpoint{3.049562in}{1.769429in}}%
\pgfpathlineto{\pgfqpoint{3.054626in}{1.771301in}}%
\pgfpathlineto{\pgfqpoint{3.064754in}{1.771301in}}%
\pgfpathlineto{\pgfqpoint{3.069818in}{1.778579in}}%
\pgfpathlineto{\pgfqpoint{3.074882in}{1.781503in}}%
\pgfpathlineto{\pgfqpoint{3.079946in}{1.786461in}}%
\pgfpathlineto{\pgfqpoint{3.085010in}{1.788218in}}%
\pgfpathlineto{\pgfqpoint{3.090074in}{1.788218in}}%
\pgfpathlineto{\pgfqpoint{3.095138in}{1.792885in}}%
\pgfpathlineto{\pgfqpoint{3.100202in}{1.794892in}}%
\pgfpathlineto{\pgfqpoint{3.105266in}{1.804309in}}%
\pgfpathlineto{\pgfqpoint{3.110330in}{1.804309in}}%
\pgfpathlineto{\pgfqpoint{3.115394in}{1.808936in}}%
\pgfpathlineto{\pgfqpoint{3.125522in}{1.812826in}}%
\pgfpathlineto{\pgfqpoint{3.130586in}{1.819652in}}%
\pgfpathlineto{\pgfqpoint{3.140714in}{1.819652in}}%
\pgfpathlineto{\pgfqpoint{3.145778in}{1.821096in}}%
\pgfpathlineto{\pgfqpoint{3.150842in}{1.834312in}}%
\pgfpathlineto{\pgfqpoint{3.166034in}{1.834312in}}%
\pgfpathlineto{\pgfqpoint{3.171098in}{1.848348in}}%
\pgfpathlineto{\pgfqpoint{3.181225in}{1.850553in}}%
\pgfpathlineto{\pgfqpoint{3.186289in}{1.861812in}}%
\pgfpathlineto{\pgfqpoint{3.191353in}{1.861812in}}%
\pgfpathlineto{\pgfqpoint{3.196417in}{1.870475in}}%
\pgfpathlineto{\pgfqpoint{3.206545in}{1.874747in}}%
\pgfpathlineto{\pgfqpoint{3.211609in}{1.883457in}}%
\pgfpathlineto{\pgfqpoint{3.216673in}{1.883457in}}%
\pgfpathlineto{\pgfqpoint{3.221737in}{1.887194in}}%
\pgfpathlineto{\pgfqpoint{3.226801in}{1.887194in}}%
\pgfpathlineto{\pgfqpoint{3.231865in}{1.889578in}}%
\pgfpathlineto{\pgfqpoint{3.236929in}{1.889853in}}%
\pgfpathlineto{\pgfqpoint{3.241993in}{1.893195in}}%
\pgfpathlineto{\pgfqpoint{3.247057in}{1.893459in}}%
\pgfpathlineto{\pgfqpoint{3.252121in}{1.899188in}}%
\pgfpathlineto{\pgfqpoint{3.257185in}{1.908429in}}%
\pgfpathlineto{\pgfqpoint{3.262249in}{1.908908in}}%
\pgfpathlineto{\pgfqpoint{3.267313in}{1.919537in}}%
\pgfpathlineto{\pgfqpoint{3.272377in}{1.921942in}}%
\pgfpathlineto{\pgfqpoint{3.277441in}{1.927552in}}%
\pgfpathlineto{\pgfqpoint{3.287569in}{1.956310in}}%
\pgfpathlineto{\pgfqpoint{3.292633in}{1.958860in}}%
\pgfpathlineto{\pgfqpoint{3.297697in}{1.970081in}}%
\pgfpathlineto{\pgfqpoint{3.307825in}{1.979648in}}%
\pgfpathlineto{\pgfqpoint{3.312889in}{1.982073in}}%
\pgfpathlineto{\pgfqpoint{3.317953in}{1.998447in}}%
\pgfpathlineto{\pgfqpoint{3.323017in}{2.009316in}}%
\pgfpathlineto{\pgfqpoint{3.333145in}{2.039596in}}%
\pgfpathlineto{\pgfqpoint{3.338209in}{2.040673in}}%
\pgfpathlineto{\pgfqpoint{3.343273in}{2.048256in}}%
\pgfpathlineto{\pgfqpoint{3.348337in}{2.048256in}}%
\pgfpathlineto{\pgfqpoint{3.353401in}{2.051778in}}%
\pgfpathlineto{\pgfqpoint{3.363529in}{2.088938in}}%
\pgfpathlineto{\pgfqpoint{3.368593in}{2.096947in}}%
\pgfpathlineto{\pgfqpoint{3.373657in}{2.101690in}}%
\pgfpathlineto{\pgfqpoint{3.378721in}{2.102718in}}%
\pgfpathlineto{\pgfqpoint{3.383785in}{2.109633in}}%
\pgfpathlineto{\pgfqpoint{3.388849in}{2.113295in}}%
\pgfpathlineto{\pgfqpoint{3.393913in}{2.114108in}}%
\pgfpathlineto{\pgfqpoint{3.409105in}{2.127110in}}%
\pgfpathlineto{\pgfqpoint{3.414169in}{2.131256in}}%
\pgfpathlineto{\pgfqpoint{3.419233in}{2.138176in}}%
\pgfpathlineto{\pgfqpoint{3.429361in}{2.143272in}}%
\pgfpathlineto{\pgfqpoint{3.434425in}{2.143408in}}%
\pgfpathlineto{\pgfqpoint{3.439489in}{2.151593in}}%
\pgfpathlineto{\pgfqpoint{3.449617in}{2.161499in}}%
\pgfpathlineto{\pgfqpoint{3.464809in}{2.166187in}}%
\pgfpathlineto{\pgfqpoint{3.469873in}{2.166638in}}%
\pgfpathlineto{\pgfqpoint{3.480001in}{2.176526in}}%
\pgfpathlineto{\pgfqpoint{3.485065in}{2.178129in}}%
\pgfpathlineto{\pgfqpoint{3.490129in}{2.181930in}}%
\pgfpathlineto{\pgfqpoint{3.495193in}{2.181980in}}%
\pgfpathlineto{\pgfqpoint{3.500256in}{2.188539in}}%
\pgfpathlineto{\pgfqpoint{3.505320in}{2.188853in}}%
\pgfpathlineto{\pgfqpoint{3.510384in}{2.191826in}}%
\pgfpathlineto{\pgfqpoint{3.520512in}{2.194463in}}%
\pgfpathlineto{\pgfqpoint{3.525576in}{2.201077in}}%
\pgfpathlineto{\pgfqpoint{3.550896in}{2.205976in}}%
\pgfpathlineto{\pgfqpoint{3.561024in}{2.210202in}}%
\pgfpathlineto{\pgfqpoint{3.566088in}{2.210679in}}%
\pgfpathlineto{\pgfqpoint{3.571152in}{2.215226in}}%
\pgfpathlineto{\pgfqpoint{3.576216in}{2.216951in}}%
\pgfpathlineto{\pgfqpoint{3.581280in}{2.219859in}}%
\pgfpathlineto{\pgfqpoint{3.586344in}{2.225606in}}%
\pgfpathlineto{\pgfqpoint{3.596472in}{2.229758in}}%
\pgfpathlineto{\pgfqpoint{3.601536in}{2.232820in}}%
\pgfpathlineto{\pgfqpoint{3.611664in}{2.235168in}}%
\pgfpathlineto{\pgfqpoint{3.616728in}{2.235226in}}%
\pgfpathlineto{\pgfqpoint{3.621792in}{2.240755in}}%
\pgfpathlineto{\pgfqpoint{3.626856in}{2.241277in}}%
\pgfpathlineto{\pgfqpoint{3.631920in}{2.264593in}}%
\pgfpathlineto{\pgfqpoint{3.636984in}{2.272696in}}%
\pgfpathlineto{\pgfqpoint{3.642048in}{2.273849in}}%
\pgfpathlineto{\pgfqpoint{3.647112in}{2.279061in}}%
\pgfpathlineto{\pgfqpoint{3.652176in}{2.280459in}}%
\pgfpathlineto{\pgfqpoint{3.657240in}{2.280586in}}%
\pgfpathlineto{\pgfqpoint{3.662304in}{2.288937in}}%
\pgfpathlineto{\pgfqpoint{3.667368in}{2.290435in}}%
\pgfpathlineto{\pgfqpoint{3.672432in}{2.295086in}}%
\pgfpathlineto{\pgfqpoint{3.677496in}{2.296252in}}%
\pgfpathlineto{\pgfqpoint{3.682560in}{2.298616in}}%
\pgfpathlineto{\pgfqpoint{3.687624in}{2.303910in}}%
\pgfpathlineto{\pgfqpoint{3.692688in}{2.305556in}}%
\pgfpathlineto{\pgfqpoint{3.697752in}{2.312357in}}%
\pgfpathlineto{\pgfqpoint{3.707880in}{2.313398in}}%
\pgfpathlineto{\pgfqpoint{3.712944in}{2.315705in}}%
\pgfpathlineto{\pgfqpoint{3.723072in}{2.316653in}}%
\pgfpathlineto{\pgfqpoint{3.728136in}{2.319682in}}%
\pgfpathlineto{\pgfqpoint{3.758520in}{2.326989in}}%
\pgfpathlineto{\pgfqpoint{3.783840in}{2.334117in}}%
\pgfpathlineto{\pgfqpoint{3.788904in}{2.336002in}}%
\pgfpathlineto{\pgfqpoint{3.793968in}{2.336535in}}%
\pgfpathlineto{\pgfqpoint{3.799032in}{2.338445in}}%
\pgfpathlineto{\pgfqpoint{3.804096in}{2.338928in}}%
\pgfpathlineto{\pgfqpoint{3.809160in}{2.342022in}}%
\pgfpathlineto{\pgfqpoint{3.814224in}{2.351002in}}%
\pgfpathlineto{\pgfqpoint{3.819288in}{2.353438in}}%
\pgfpathlineto{\pgfqpoint{3.824351in}{2.366184in}}%
\pgfpathlineto{\pgfqpoint{3.834479in}{2.368812in}}%
\pgfpathlineto{\pgfqpoint{3.839543in}{2.369112in}}%
\pgfpathlineto{\pgfqpoint{3.844607in}{2.374253in}}%
\pgfpathlineto{\pgfqpoint{3.849671in}{2.376014in}}%
\pgfpathlineto{\pgfqpoint{3.854735in}{2.376391in}}%
\pgfpathlineto{\pgfqpoint{3.859799in}{2.378870in}}%
\pgfpathlineto{\pgfqpoint{3.864863in}{2.386297in}}%
\pgfpathlineto{\pgfqpoint{3.869927in}{2.388677in}}%
\pgfpathlineto{\pgfqpoint{3.874991in}{2.389648in}}%
\pgfpathlineto{\pgfqpoint{3.885119in}{2.405261in}}%
\pgfpathlineto{\pgfqpoint{3.905375in}{2.410135in}}%
\pgfpathlineto{\pgfqpoint{3.910439in}{2.413767in}}%
\pgfpathlineto{\pgfqpoint{3.925631in}{2.419006in}}%
\pgfpathlineto{\pgfqpoint{3.930695in}{2.421532in}}%
\pgfpathlineto{\pgfqpoint{3.935759in}{2.421982in}}%
\pgfpathlineto{\pgfqpoint{3.940823in}{2.426300in}}%
\pgfpathlineto{\pgfqpoint{3.950951in}{2.427651in}}%
\pgfpathlineto{\pgfqpoint{3.956015in}{2.432338in}}%
\pgfpathlineto{\pgfqpoint{3.961079in}{2.432338in}}%
\pgfpathlineto{\pgfqpoint{3.966143in}{2.435593in}}%
\pgfpathlineto{\pgfqpoint{3.976271in}{2.439408in}}%
\pgfpathlineto{\pgfqpoint{3.981335in}{2.446022in}}%
\pgfpathlineto{\pgfqpoint{3.991463in}{2.446499in}}%
\pgfpathlineto{\pgfqpoint{3.996527in}{2.454424in}}%
\pgfpathlineto{\pgfqpoint{4.006655in}{2.455976in}}%
\pgfpathlineto{\pgfqpoint{4.011719in}{2.466737in}}%
\pgfpathlineto{\pgfqpoint{4.016783in}{2.469080in}}%
\pgfpathlineto{\pgfqpoint{4.026911in}{2.469725in}}%
\pgfpathlineto{\pgfqpoint{4.037039in}{2.472854in}}%
\pgfpathlineto{\pgfqpoint{4.042103in}{2.482445in}}%
\pgfpathlineto{\pgfqpoint{4.052231in}{2.486171in}}%
\pgfpathlineto{\pgfqpoint{4.057295in}{2.492078in}}%
\pgfpathlineto{\pgfqpoint{4.082615in}{2.501694in}}%
\pgfpathlineto{\pgfqpoint{4.087679in}{2.509190in}}%
\pgfpathlineto{\pgfqpoint{4.092743in}{2.509534in}}%
\pgfpathlineto{\pgfqpoint{4.097807in}{2.513087in}}%
\pgfpathlineto{\pgfqpoint{4.102871in}{2.520568in}}%
\pgfpathlineto{\pgfqpoint{4.112999in}{2.528905in}}%
\pgfpathlineto{\pgfqpoint{4.118063in}{2.528908in}}%
\pgfpathlineto{\pgfqpoint{4.123127in}{2.532624in}}%
\pgfpathlineto{\pgfqpoint{4.133255in}{2.535018in}}%
\pgfpathlineto{\pgfqpoint{4.138319in}{2.536623in}}%
\pgfpathlineto{\pgfqpoint{4.143383in}{2.542086in}}%
\pgfpathlineto{\pgfqpoint{4.148446in}{2.542240in}}%
\pgfpathlineto{\pgfqpoint{4.158574in}{2.544490in}}%
\pgfpathlineto{\pgfqpoint{4.163638in}{2.549556in}}%
\pgfpathlineto{\pgfqpoint{4.168702in}{2.560583in}}%
\pgfpathlineto{\pgfqpoint{4.178830in}{2.561536in}}%
\pgfpathlineto{\pgfqpoint{4.183894in}{2.564238in}}%
\pgfpathlineto{\pgfqpoint{4.194022in}{2.566373in}}%
\pgfpathlineto{\pgfqpoint{4.199086in}{2.569203in}}%
\pgfpathlineto{\pgfqpoint{4.204150in}{2.574702in}}%
\pgfpathlineto{\pgfqpoint{4.219342in}{2.580553in}}%
\pgfpathlineto{\pgfqpoint{4.224406in}{2.586253in}}%
\pgfpathlineto{\pgfqpoint{4.229470in}{2.590208in}}%
\pgfpathlineto{\pgfqpoint{4.234534in}{2.601521in}}%
\pgfpathlineto{\pgfqpoint{4.249726in}{2.619063in}}%
\pgfpathlineto{\pgfqpoint{4.254790in}{2.627749in}}%
\pgfpathlineto{\pgfqpoint{4.259854in}{2.634153in}}%
\pgfpathlineto{\pgfqpoint{4.264918in}{2.634153in}}%
\pgfpathlineto{\pgfqpoint{4.269982in}{2.640082in}}%
\pgfpathlineto{\pgfqpoint{4.280110in}{2.641831in}}%
\pgfpathlineto{\pgfqpoint{4.285174in}{2.649227in}}%
\pgfpathlineto{\pgfqpoint{4.290238in}{2.660942in}}%
\pgfpathlineto{\pgfqpoint{4.295302in}{2.669153in}}%
\pgfpathlineto{\pgfqpoint{4.300366in}{2.669844in}}%
\pgfpathlineto{\pgfqpoint{4.315558in}{2.688563in}}%
\pgfpathlineto{\pgfqpoint{4.320622in}{2.688956in}}%
\pgfpathlineto{\pgfqpoint{4.325686in}{2.698123in}}%
\pgfpathlineto{\pgfqpoint{4.330750in}{2.699018in}}%
\pgfpathlineto{\pgfqpoint{4.345942in}{2.709440in}}%
\pgfpathlineto{\pgfqpoint{4.361134in}{2.709995in}}%
\pgfpathlineto{\pgfqpoint{4.371262in}{2.729908in}}%
\pgfpathlineto{\pgfqpoint{4.376326in}{2.731566in}}%
\pgfpathlineto{\pgfqpoint{4.381390in}{2.747601in}}%
\pgfpathlineto{\pgfqpoint{4.386454in}{2.756472in}}%
\pgfpathlineto{\pgfqpoint{4.396582in}{2.760935in}}%
\pgfpathlineto{\pgfqpoint{4.401646in}{2.761682in}}%
\pgfpathlineto{\pgfqpoint{4.406710in}{2.767031in}}%
\pgfpathlineto{\pgfqpoint{4.416838in}{2.787574in}}%
\pgfpathlineto{\pgfqpoint{4.426966in}{2.793104in}}%
\pgfpathlineto{\pgfqpoint{4.432030in}{2.794667in}}%
\pgfpathlineto{\pgfqpoint{4.442158in}{2.799856in}}%
\pgfpathlineto{\pgfqpoint{4.452286in}{2.806443in}}%
\pgfpathlineto{\pgfqpoint{4.457350in}{2.811366in}}%
\pgfpathlineto{\pgfqpoint{4.462414in}{2.811507in}}%
\pgfpathlineto{\pgfqpoint{4.467478in}{2.819615in}}%
\pgfpathlineto{\pgfqpoint{4.477605in}{2.821159in}}%
\pgfpathlineto{\pgfqpoint{4.482669in}{2.841048in}}%
\pgfpathlineto{\pgfqpoint{4.487733in}{2.842480in}}%
\pgfpathlineto{\pgfqpoint{4.492797in}{2.845840in}}%
\pgfpathlineto{\pgfqpoint{4.497861in}{2.846374in}}%
\pgfpathlineto{\pgfqpoint{4.513053in}{2.854108in}}%
\pgfpathlineto{\pgfqpoint{4.518117in}{2.860267in}}%
\pgfpathlineto{\pgfqpoint{4.523181in}{2.860747in}}%
\pgfpathlineto{\pgfqpoint{4.528245in}{2.863404in}}%
\pgfpathlineto{\pgfqpoint{4.533309in}{2.864720in}}%
\pgfpathlineto{\pgfqpoint{4.548501in}{2.882229in}}%
\pgfpathlineto{\pgfqpoint{4.558629in}{2.883715in}}%
\pgfpathlineto{\pgfqpoint{4.563693in}{2.891295in}}%
\pgfpathlineto{\pgfqpoint{4.568757in}{2.908326in}}%
\pgfpathlineto{\pgfqpoint{4.573821in}{2.910085in}}%
\pgfpathlineto{\pgfqpoint{4.578885in}{2.914863in}}%
\pgfpathlineto{\pgfqpoint{4.589013in}{2.919267in}}%
\pgfpathlineto{\pgfqpoint{4.594077in}{2.927611in}}%
\pgfpathlineto{\pgfqpoint{4.599141in}{2.930044in}}%
\pgfpathlineto{\pgfqpoint{4.604205in}{2.930263in}}%
\pgfpathlineto{\pgfqpoint{4.609269in}{2.933243in}}%
\pgfpathlineto{\pgfqpoint{4.614333in}{2.934282in}}%
\pgfpathlineto{\pgfqpoint{4.619397in}{2.937798in}}%
\pgfpathlineto{\pgfqpoint{4.629525in}{2.939683in}}%
\pgfpathlineto{\pgfqpoint{4.634589in}{2.952201in}}%
\pgfpathlineto{\pgfqpoint{4.649781in}{2.962588in}}%
\pgfpathlineto{\pgfqpoint{4.654845in}{2.968198in}}%
\pgfpathlineto{\pgfqpoint{4.664973in}{2.975726in}}%
\pgfpathlineto{\pgfqpoint{4.670037in}{2.976884in}}%
\pgfpathlineto{\pgfqpoint{4.675101in}{2.980936in}}%
\pgfpathlineto{\pgfqpoint{4.680165in}{2.980993in}}%
\pgfpathlineto{\pgfqpoint{4.685229in}{2.983547in}}%
\pgfpathlineto{\pgfqpoint{4.690293in}{2.991798in}}%
\pgfpathlineto{\pgfqpoint{4.695357in}{2.996550in}}%
\pgfpathlineto{\pgfqpoint{4.700421in}{2.999241in}}%
\pgfpathlineto{\pgfqpoint{4.705485in}{3.000090in}}%
\pgfpathlineto{\pgfqpoint{4.710549in}{3.002791in}}%
\pgfpathlineto{\pgfqpoint{4.715613in}{3.007254in}}%
\pgfpathlineto{\pgfqpoint{4.720677in}{3.008114in}}%
\pgfpathlineto{\pgfqpoint{4.725741in}{3.013378in}}%
\pgfpathlineto{\pgfqpoint{4.735869in}{3.014853in}}%
\pgfpathlineto{\pgfqpoint{4.740933in}{3.026297in}}%
\pgfpathlineto{\pgfqpoint{4.745997in}{3.027911in}}%
\pgfpathlineto{\pgfqpoint{4.751061in}{3.037077in}}%
\pgfpathlineto{\pgfqpoint{4.756125in}{3.040329in}}%
\pgfpathlineto{\pgfqpoint{4.761189in}{3.045140in}}%
\pgfpathlineto{\pgfqpoint{4.771317in}{3.046233in}}%
\pgfpathlineto{\pgfqpoint{4.776381in}{3.052485in}}%
\pgfpathlineto{\pgfqpoint{4.781445in}{3.052714in}}%
\pgfpathlineto{\pgfqpoint{4.786509in}{3.056880in}}%
\pgfpathlineto{\pgfqpoint{4.791573in}{3.058635in}}%
\pgfpathlineto{\pgfqpoint{4.796636in}{3.061664in}}%
\pgfpathlineto{\pgfqpoint{4.801700in}{3.062632in}}%
\pgfpathlineto{\pgfqpoint{4.806764in}{3.067700in}}%
\pgfpathlineto{\pgfqpoint{4.811828in}{3.069060in}}%
\pgfpathlineto{\pgfqpoint{4.821956in}{3.083986in}}%
\pgfpathlineto{\pgfqpoint{4.827020in}{3.083999in}}%
\pgfpathlineto{\pgfqpoint{4.832084in}{3.086514in}}%
\pgfpathlineto{\pgfqpoint{4.837148in}{3.087771in}}%
\pgfpathlineto{\pgfqpoint{4.842212in}{3.094145in}}%
\pgfpathlineto{\pgfqpoint{4.847276in}{3.094300in}}%
\pgfpathlineto{\pgfqpoint{4.852340in}{3.098789in}}%
\pgfpathlineto{\pgfqpoint{4.857404in}{3.119152in}}%
\pgfpathlineto{\pgfqpoint{4.862468in}{3.123875in}}%
\pgfpathlineto{\pgfqpoint{4.867532in}{3.124099in}}%
\pgfpathlineto{\pgfqpoint{4.872596in}{3.129374in}}%
\pgfpathlineto{\pgfqpoint{4.877660in}{3.132948in}}%
\pgfpathlineto{\pgfqpoint{4.892852in}{3.135700in}}%
\pgfpathlineto{\pgfqpoint{4.897916in}{3.136074in}}%
\pgfpathlineto{\pgfqpoint{4.902980in}{3.154909in}}%
\pgfpathlineto{\pgfqpoint{4.908044in}{3.162775in}}%
\pgfpathlineto{\pgfqpoint{4.913108in}{3.163299in}}%
\pgfpathlineto{\pgfqpoint{4.923236in}{3.183459in}}%
\pgfpathlineto{\pgfqpoint{4.928300in}{3.204814in}}%
\pgfpathlineto{\pgfqpoint{4.933364in}{3.204877in}}%
\pgfpathlineto{\pgfqpoint{4.938428in}{3.215644in}}%
\pgfpathlineto{\pgfqpoint{4.943492in}{3.229812in}}%
\pgfpathlineto{\pgfqpoint{4.948556in}{3.232249in}}%
\pgfpathlineto{\pgfqpoint{4.958684in}{3.233362in}}%
\pgfpathlineto{\pgfqpoint{4.968812in}{3.260767in}}%
\pgfpathlineto{\pgfqpoint{4.973876in}{3.268725in}}%
\pgfpathlineto{\pgfqpoint{4.978940in}{3.270996in}}%
\pgfpathlineto{\pgfqpoint{4.984004in}{3.282796in}}%
\pgfpathlineto{\pgfqpoint{4.989068in}{3.284303in}}%
\pgfpathlineto{\pgfqpoint{4.994132in}{3.299025in}}%
\pgfpathlineto{\pgfqpoint{4.999196in}{3.301456in}}%
\pgfpathlineto{\pgfqpoint{5.004260in}{3.301553in}}%
\pgfpathlineto{\pgfqpoint{5.009324in}{3.304713in}}%
\pgfpathlineto{\pgfqpoint{5.014388in}{3.306169in}}%
\pgfpathlineto{\pgfqpoint{5.019452in}{3.324110in}}%
\pgfpathlineto{\pgfqpoint{5.024516in}{3.328573in}}%
\pgfpathlineto{\pgfqpoint{5.029580in}{3.330844in}}%
\pgfpathlineto{\pgfqpoint{5.034644in}{3.331855in}}%
\pgfpathlineto{\pgfqpoint{5.039708in}{3.339475in}}%
\pgfpathlineto{\pgfqpoint{5.044772in}{3.343181in}}%
\pgfpathlineto{\pgfqpoint{5.049836in}{3.349936in}}%
\pgfpathlineto{\pgfqpoint{5.054900in}{3.351285in}}%
\pgfpathlineto{\pgfqpoint{5.059964in}{3.358912in}}%
\pgfpathlineto{\pgfqpoint{5.080220in}{3.378458in}}%
\pgfpathlineto{\pgfqpoint{5.085284in}{3.379738in}}%
\pgfpathlineto{\pgfqpoint{5.090348in}{3.382203in}}%
\pgfpathlineto{\pgfqpoint{5.095412in}{3.383352in}}%
\pgfpathlineto{\pgfqpoint{5.100476in}{3.389228in}}%
\pgfpathlineto{\pgfqpoint{5.110604in}{3.390785in}}%
\pgfpathlineto{\pgfqpoint{5.115668in}{3.421490in}}%
\pgfpathlineto{\pgfqpoint{5.120731in}{3.434428in}}%
\pgfpathlineto{\pgfqpoint{5.125795in}{3.434613in}}%
\pgfpathlineto{\pgfqpoint{5.130859in}{3.438637in}}%
\pgfpathlineto{\pgfqpoint{5.135923in}{3.440205in}}%
\pgfpathlineto{\pgfqpoint{5.146051in}{3.445860in}}%
\pgfpathlineto{\pgfqpoint{5.151115in}{3.467777in}}%
\pgfpathlineto{\pgfqpoint{5.156179in}{3.478131in}}%
\pgfpathlineto{\pgfqpoint{5.161243in}{3.494985in}}%
\pgfpathlineto{\pgfqpoint{5.171371in}{3.504444in}}%
\pgfpathlineto{\pgfqpoint{5.186563in}{3.513044in}}%
\pgfpathlineto{\pgfqpoint{5.196691in}{3.518058in}}%
\pgfpathlineto{\pgfqpoint{5.206819in}{3.528455in}}%
\pgfpathlineto{\pgfqpoint{5.211883in}{3.545543in}}%
\pgfpathlineto{\pgfqpoint{5.216947in}{3.545765in}}%
\pgfpathlineto{\pgfqpoint{5.222011in}{3.561543in}}%
\pgfpathlineto{\pgfqpoint{5.227075in}{3.565339in}}%
\pgfpathlineto{\pgfqpoint{5.232139in}{3.565565in}}%
\pgfpathlineto{\pgfqpoint{5.237203in}{3.573231in}}%
\pgfpathlineto{\pgfqpoint{5.242267in}{3.575220in}}%
\pgfpathlineto{\pgfqpoint{5.247331in}{3.583997in}}%
\pgfpathlineto{\pgfqpoint{5.257459in}{3.621528in}}%
\pgfpathlineto{\pgfqpoint{5.267587in}{3.628620in}}%
\pgfpathlineto{\pgfqpoint{5.272651in}{3.628953in}}%
\pgfpathlineto{\pgfqpoint{5.277715in}{3.638858in}}%
\pgfpathlineto{\pgfqpoint{5.282779in}{3.638979in}}%
\pgfpathlineto{\pgfqpoint{5.287843in}{3.640412in}}%
\pgfpathlineto{\pgfqpoint{5.292907in}{3.645484in}}%
\pgfpathlineto{\pgfqpoint{5.297971in}{3.652901in}}%
\pgfpathlineto{\pgfqpoint{5.303035in}{3.657735in}}%
\pgfpathlineto{\pgfqpoint{5.308099in}{3.660280in}}%
\pgfpathlineto{\pgfqpoint{5.313163in}{3.660767in}}%
\pgfpathlineto{\pgfqpoint{5.323291in}{3.665938in}}%
\pgfpathlineto{\pgfqpoint{5.328355in}{3.678700in}}%
\pgfpathlineto{\pgfqpoint{5.333419in}{3.713007in}}%
\pgfpathlineto{\pgfqpoint{5.338483in}{3.714395in}}%
\pgfpathlineto{\pgfqpoint{5.343547in}{3.722453in}}%
\pgfpathlineto{\pgfqpoint{5.348611in}{3.723342in}}%
\pgfpathlineto{\pgfqpoint{5.353675in}{3.743446in}}%
\pgfpathlineto{\pgfqpoint{5.358739in}{3.745799in}}%
\pgfpathlineto{\pgfqpoint{5.363803in}{3.751647in}}%
\pgfpathlineto{\pgfqpoint{5.368867in}{3.752274in}}%
\pgfpathlineto{\pgfqpoint{5.373931in}{3.769974in}}%
\pgfpathlineto{\pgfqpoint{5.378995in}{3.776155in}}%
\pgfpathlineto{\pgfqpoint{5.384059in}{3.794232in}}%
\pgfpathlineto{\pgfqpoint{5.399251in}{3.807777in}}%
\pgfpathlineto{\pgfqpoint{5.404315in}{3.848013in}}%
\pgfpathlineto{\pgfqpoint{5.409379in}{3.863980in}}%
\pgfpathlineto{\pgfqpoint{5.414443in}{3.890183in}}%
\pgfpathlineto{\pgfqpoint{5.419507in}{3.895236in}}%
\pgfpathlineto{\pgfqpoint{5.424571in}{3.903177in}}%
\pgfpathlineto{\pgfqpoint{5.429635in}{3.916610in}}%
\pgfpathlineto{\pgfqpoint{5.439763in}{3.925908in}}%
\pgfpathlineto{\pgfqpoint{5.454954in}{3.930099in}}%
\pgfpathlineto{\pgfqpoint{5.460018in}{3.953765in}}%
\pgfpathlineto{\pgfqpoint{5.465082in}{3.962015in}}%
\pgfpathlineto{\pgfqpoint{5.475210in}{3.968210in}}%
\pgfpathlineto{\pgfqpoint{5.480274in}{3.990904in}}%
\pgfpathlineto{\pgfqpoint{5.485338in}{4.005720in}}%
\pgfpathlineto{\pgfqpoint{5.490402in}{4.010943in}}%
\pgfpathlineto{\pgfqpoint{5.495466in}{4.026880in}}%
\pgfpathlineto{\pgfqpoint{5.500530in}{4.031125in}}%
\pgfpathlineto{\pgfqpoint{5.510658in}{4.034483in}}%
\pgfpathlineto{\pgfqpoint{5.515722in}{4.053360in}}%
\pgfpathlineto{\pgfqpoint{5.520786in}{4.062201in}}%
\pgfpathlineto{\pgfqpoint{5.525850in}{4.078307in}}%
\pgfpathlineto{\pgfqpoint{5.530914in}{4.081902in}}%
\pgfpathlineto{\pgfqpoint{5.535978in}{4.092826in}}%
\pgfpathlineto{\pgfqpoint{5.546106in}{4.096259in}}%
\pgfpathlineto{\pgfqpoint{5.561298in}{4.103609in}}%
\pgfpathlineto{\pgfqpoint{5.566362in}{4.136372in}}%
\pgfpathlineto{\pgfqpoint{5.571426in}{4.157626in}}%
\pgfpathlineto{\pgfqpoint{5.581554in}{4.162110in}}%
\pgfpathlineto{\pgfqpoint{5.586618in}{4.170360in}}%
\pgfpathlineto{\pgfqpoint{5.591682in}{4.186722in}}%
\pgfpathlineto{\pgfqpoint{5.601810in}{4.198822in}}%
\pgfpathlineto{\pgfqpoint{5.606874in}{4.200896in}}%
\pgfpathlineto{\pgfqpoint{5.611938in}{4.206737in}}%
\pgfpathlineto{\pgfqpoint{5.617002in}{4.208926in}}%
\pgfpathlineto{\pgfqpoint{5.622066in}{4.224181in}}%
\pgfpathlineto{\pgfqpoint{5.627130in}{4.224748in}}%
\pgfpathlineto{\pgfqpoint{5.632194in}{4.274743in}}%
\pgfpathlineto{\pgfqpoint{5.637258in}{4.281674in}}%
\pgfpathlineto{\pgfqpoint{5.647386in}{4.283628in}}%
\pgfpathlineto{\pgfqpoint{5.657514in}{4.298726in}}%
\pgfpathlineto{\pgfqpoint{5.662578in}{4.321436in}}%
\pgfpathlineto{\pgfqpoint{5.667642in}{4.334241in}}%
\pgfpathlineto{\pgfqpoint{5.672706in}{4.361288in}}%
\pgfpathlineto{\pgfqpoint{5.677770in}{4.362527in}}%
\pgfpathlineto{\pgfqpoint{5.687898in}{4.374998in}}%
\pgfpathlineto{\pgfqpoint{5.692962in}{4.375237in}}%
\pgfpathlineto{\pgfqpoint{5.698026in}{4.377412in}}%
\pgfpathlineto{\pgfqpoint{5.703090in}{4.394868in}}%
\pgfpathlineto{\pgfqpoint{5.708154in}{4.398056in}}%
\pgfpathlineto{\pgfqpoint{5.728410in}{4.402262in}}%
\pgfpathlineto{\pgfqpoint{5.748666in}{4.407441in}}%
\pgfpathlineto{\pgfqpoint{5.753730in}{4.415244in}}%
\pgfpathlineto{\pgfqpoint{5.758794in}{4.420764in}}%
\pgfpathlineto{\pgfqpoint{5.768921in}{4.440867in}}%
\pgfpathlineto{\pgfqpoint{5.773985in}{4.445202in}}%
\pgfpathlineto{\pgfqpoint{5.779049in}{4.454641in}}%
\pgfpathlineto{\pgfqpoint{5.799305in}{4.513912in}}%
\pgfpathlineto{\pgfqpoint{5.804369in}{4.524155in}}%
\pgfpathlineto{\pgfqpoint{5.814497in}{4.531009in}}%
\pgfpathlineto{\pgfqpoint{5.819561in}{4.539806in}}%
\pgfpathlineto{\pgfqpoint{5.824625in}{4.541901in}}%
\pgfpathlineto{\pgfqpoint{5.829689in}{4.541937in}}%
\pgfpathlineto{\pgfqpoint{5.834753in}{4.543147in}}%
\pgfpathlineto{\pgfqpoint{5.839817in}{4.543193in}}%
\pgfpathlineto{\pgfqpoint{5.844881in}{4.545419in}}%
\pgfpathlineto{\pgfqpoint{5.849945in}{4.545569in}}%
\pgfpathlineto{\pgfqpoint{5.855009in}{4.549170in}}%
\pgfpathlineto{\pgfqpoint{5.865137in}{4.553076in}}%
\pgfpathlineto{\pgfqpoint{5.870201in}{4.561093in}}%
\pgfpathlineto{\pgfqpoint{5.875265in}{4.573479in}}%
\pgfpathlineto{\pgfqpoint{5.880329in}{4.591670in}}%
\pgfpathlineto{\pgfqpoint{5.885393in}{4.597238in}}%
\pgfpathlineto{\pgfqpoint{6.244936in}{4.597238in}}%
\pgfpathlineto{\pgfqpoint{6.244936in}{4.597238in}}%
\pgfusepath{stroke}%
\end{pgfscope}%
\begin{pgfscope}%
\pgfsetrectcap%
\pgfsetmiterjoin%
\pgfsetlinewidth{0.803000pt}%
\definecolor{currentstroke}{rgb}{0.000000,0.000000,0.000000}%
\pgfsetstrokecolor{currentstroke}%
\pgfsetdash{}{0pt}%
\pgfpathmoveto{\pgfqpoint{0.725193in}{0.571603in}}%
\pgfpathlineto{\pgfqpoint{0.725193in}{4.597238in}}%
\pgfusepath{stroke}%
\end{pgfscope}%
\begin{pgfscope}%
\pgfsetrectcap%
\pgfsetmiterjoin%
\pgfsetlinewidth{0.803000pt}%
\definecolor{currentstroke}{rgb}{0.000000,0.000000,0.000000}%
\pgfsetstrokecolor{currentstroke}%
\pgfsetdash{}{0pt}%
\pgfpathmoveto{\pgfqpoint{6.250000in}{0.571603in}}%
\pgfpathlineto{\pgfqpoint{6.250000in}{4.597238in}}%
\pgfusepath{stroke}%
\end{pgfscope}%
\begin{pgfscope}%
\pgfsetrectcap%
\pgfsetmiterjoin%
\pgfsetlinewidth{0.803000pt}%
\definecolor{currentstroke}{rgb}{0.000000,0.000000,0.000000}%
\pgfsetstrokecolor{currentstroke}%
\pgfsetdash{}{0pt}%
\pgfpathmoveto{\pgfqpoint{0.725193in}{0.571603in}}%
\pgfpathlineto{\pgfqpoint{6.250000in}{0.571603in}}%
\pgfusepath{stroke}%
\end{pgfscope}%
\begin{pgfscope}%
\pgfsetrectcap%
\pgfsetmiterjoin%
\pgfsetlinewidth{0.803000pt}%
\definecolor{currentstroke}{rgb}{0.000000,0.000000,0.000000}%
\pgfsetstrokecolor{currentstroke}%
\pgfsetdash{}{0pt}%
\pgfpathmoveto{\pgfqpoint{0.725193in}{4.597238in}}%
\pgfpathlineto{\pgfqpoint{6.250000in}{4.597238in}}%
\pgfusepath{stroke}%
\end{pgfscope}%
\begin{pgfscope}%
\pgfsetbuttcap%
\pgfsetroundjoin%
\pgfsetlinewidth{2.007500pt}%
\definecolor{currentstroke}{rgb}{1.000000,0.843137,0.000000}%
\pgfsetstrokecolor{currentstroke}%
\pgfsetdash{{7.400000pt}{3.200000pt}}{0.000000pt}%
\pgfpathmoveto{\pgfqpoint{0.780749in}{4.490010in}}%
\pgfpathlineto{\pgfqpoint{1.058526in}{4.490010in}}%
\pgfusepath{stroke}%
\end{pgfscope}%
\begin{pgfscope}%
\definecolor{textcolor}{rgb}{0.000000,0.000000,0.000000}%
\pgfsetstrokecolor{textcolor}%
\pgfsetfillcolor{textcolor}%
\pgftext[x=1.086304in,y=4.441399in,left,base]{\color{textcolor}\sffamily\fontsize{10.000000}{12.000000}\selectfont FT+htd}%
\end{pgfscope}%
\begin{pgfscope}%
\pgfsetbuttcap%
\pgfsetroundjoin%
\pgfsetlinewidth{2.007500pt}%
\definecolor{currentstroke}{rgb}{1.000000,0.694118,0.305882}%
\pgfsetstrokecolor{currentstroke}%
\pgfsetdash{{2.000000pt}{3.300000pt}}{0.000000pt}%
\pgfpathmoveto{\pgfqpoint{0.780749in}{4.307171in}}%
\pgfpathlineto{\pgfqpoint{1.058526in}{4.307171in}}%
\pgfusepath{stroke}%
\end{pgfscope}%
\begin{pgfscope}%
\definecolor{textcolor}{rgb}{0.000000,0.000000,0.000000}%
\pgfsetstrokecolor{textcolor}%
\pgfsetfillcolor{textcolor}%
\pgftext[x=1.086304in,y=4.258559in,left,base]{\color{textcolor}\sffamily\fontsize{10.000000}{12.000000}\selectfont FT+Flow}%
\end{pgfscope}%
\begin{pgfscope}%
\pgfsetrectcap%
\pgfsetroundjoin%
\pgfsetlinewidth{2.007500pt}%
\definecolor{currentstroke}{rgb}{0.980392,0.529412,0.458824}%
\pgfsetstrokecolor{currentstroke}%
\pgfsetdash{}{0pt}%
\pgfpathmoveto{\pgfqpoint{0.780749in}{4.124331in}}%
\pgfpathlineto{\pgfqpoint{1.058526in}{4.124331in}}%
\pgfusepath{stroke}%
\end{pgfscope}%
\begin{pgfscope}%
\definecolor{textcolor}{rgb}{0.000000,0.000000,0.000000}%
\pgfsetstrokecolor{textcolor}%
\pgfsetfillcolor{textcolor}%
\pgftext[x=1.086304in,y=4.075720in,left,base]{\color{textcolor}\sffamily\fontsize{10.000000}{12.000000}\selectfont FT+Tamaki}%
\end{pgfscope}%
\begin{pgfscope}%
\pgfsetbuttcap%
\pgfsetroundjoin%
\pgfsetlinewidth{2.007500pt}%
\definecolor{currentstroke}{rgb}{0.866667,0.058824,0.058824}%
\pgfsetstrokecolor{currentstroke}%
\pgfsetdash{{7.400000pt}{3.200000pt}}{0.000000pt}%
\pgfpathmoveto{\pgfqpoint{0.780749in}{3.941492in}}%
\pgfpathlineto{\pgfqpoint{1.058526in}{3.941492in}}%
\pgfusepath{stroke}%
\end{pgfscope}%
\begin{pgfscope}%
\definecolor{textcolor}{rgb}{0.000000,0.000000,0.000000}%
\pgfsetstrokecolor{textcolor}%
\pgfsetfillcolor{textcolor}%
\pgftext[x=1.086304in,y=3.892881in,left,base]{\color{textcolor}\sffamily\fontsize{10.000000}{12.000000}\selectfont cachet}%
\end{pgfscope}%
\begin{pgfscope}%
\pgfsetbuttcap%
\pgfsetroundjoin%
\pgfsetlinewidth{2.007500pt}%
\definecolor{currentstroke}{rgb}{0.917647,0.372549,0.580392}%
\pgfsetstrokecolor{currentstroke}%
\pgfsetdash{{2.000000pt}{3.300000pt}}{0.000000pt}%
\pgfpathmoveto{\pgfqpoint{0.780749in}{3.758653in}}%
\pgfpathlineto{\pgfqpoint{1.058526in}{3.758653in}}%
\pgfusepath{stroke}%
\end{pgfscope}%
\begin{pgfscope}%
\definecolor{textcolor}{rgb}{0.000000,0.000000,0.000000}%
\pgfsetstrokecolor{textcolor}%
\pgfsetfillcolor{textcolor}%
\pgftext[x=1.086304in,y=3.710042in,left,base]{\color{textcolor}\sffamily\fontsize{10.000000}{12.000000}\selectfont miniC2D}%
\end{pgfscope}%
\begin{pgfscope}%
\pgfsetrectcap%
\pgfsetroundjoin%
\pgfsetlinewidth{2.007500pt}%
\definecolor{currentstroke}{rgb}{0.615686,0.007843,0.843137}%
\pgfsetstrokecolor{currentstroke}%
\pgfsetdash{}{0pt}%
\pgfpathmoveto{\pgfqpoint{0.780749in}{3.575813in}}%
\pgfpathlineto{\pgfqpoint{1.058526in}{3.575813in}}%
\pgfusepath{stroke}%
\end{pgfscope}%
\begin{pgfscope}%
\definecolor{textcolor}{rgb}{0.000000,0.000000,0.000000}%
\pgfsetstrokecolor{textcolor}%
\pgfsetfillcolor{textcolor}%
\pgftext[x=1.086304in,y=3.527202in,left,base]{\color{textcolor}\sffamily\fontsize{10.000000}{12.000000}\selectfont d4}%
\end{pgfscope}%
\begin{pgfscope}%
\pgfsetbuttcap%
\pgfsetroundjoin%
\pgfsetlinewidth{2.007500pt}%
\definecolor{currentstroke}{rgb}{0.000000,0.000000,1.000000}%
\pgfsetstrokecolor{currentstroke}%
\pgfsetdash{{7.400000pt}{3.200000pt}}{0.000000pt}%
\pgfpathmoveto{\pgfqpoint{0.780749in}{3.392974in}}%
\pgfpathlineto{\pgfqpoint{1.058526in}{3.392974in}}%
\pgfusepath{stroke}%
\end{pgfscope}%
\begin{pgfscope}%
\definecolor{textcolor}{rgb}{0.000000,0.000000,0.000000}%
\pgfsetstrokecolor{textcolor}%
\pgfsetfillcolor{textcolor}%
\pgftext[x=1.086304in,y=3.344363in,left,base]{\color{textcolor}\sffamily\fontsize{10.000000}{12.000000}\selectfont VBS}%
\end{pgfscope}%
\end{pgfpicture}%
\makeatother%
\endgroup%

	\caption{\label{fig:cachet-cactus} A cactus plot of the number of benchmarks solved by various methods out of 1091 probabilistic inference benchmarks. Although our contributions \textbf{FT+*} solve fewer benchmarks than the existing weighted model counters \tool{cachet}, \tool{miniC2D}, and \tool{d4}, they improve the virtual best solver on 231 benchmarks. Note that \tool{dynQBF}, \tool{dynasp}, and \tool{SharpSAT} are unweighted model counters and so cannot solve these weighted benchmarks.}
\end{figure}

\begin{figure}[t]
	\centering
	%% Creator: Matplotlib, PGF backend
%%
%% To include the figure in your LaTeX document, write
%%   \input{<filename>.pgf}
%%
%% Make sure the required packages are loaded in your preamble
%%   \usepackage{pgf}
%%
%% and, on pdftex
%%   \usepackage[utf8]{inputenc}\DeclareUnicodeCharacter{2212}{-}
%%
%% or, on luatex and xetex
%%   \usepackage{unicode-math}
%%
%% Figures using additional raster images can only be included by \input if
%% they are in the same directory as the main LaTeX file. For loading figures
%% from other directories you can use the `import` package
%%   \usepackage{import}
%%
%% and then include the figures with
%%   \import{<path to file>}{<filename>.pgf}
%%
%% Matplotlib used the following preamble
%%   \usepackage[utf8x]{inputenc}
%%   \usepackage[T1]{fontenc}
%%
\begingroup%
\makeatletter%
\begin{pgfpicture}%
\pgfpathrectangle{\pgfpointorigin}{\pgfqpoint{6.000000in}{3.400000in}}%
\pgfusepath{use as bounding box, clip}%
\begin{pgfscope}%
\pgfsetbuttcap%
\pgfsetmiterjoin%
\definecolor{currentfill}{rgb}{1.000000,1.000000,1.000000}%
\pgfsetfillcolor{currentfill}%
\pgfsetlinewidth{0.000000pt}%
\definecolor{currentstroke}{rgb}{1.000000,1.000000,1.000000}%
\pgfsetstrokecolor{currentstroke}%
\pgfsetdash{}{0pt}%
\pgfpathmoveto{\pgfqpoint{0.000000in}{0.000000in}}%
\pgfpathlineto{\pgfqpoint{6.000000in}{0.000000in}}%
\pgfpathlineto{\pgfqpoint{6.000000in}{3.400000in}}%
\pgfpathlineto{\pgfqpoint{0.000000in}{3.400000in}}%
\pgfpathclose%
\pgfusepath{fill}%
\end{pgfscope}%
\begin{pgfscope}%
\pgfsetbuttcap%
\pgfsetmiterjoin%
\definecolor{currentfill}{rgb}{1.000000,1.000000,1.000000}%
\pgfsetfillcolor{currentfill}%
\pgfsetlinewidth{0.000000pt}%
\definecolor{currentstroke}{rgb}{0.000000,0.000000,0.000000}%
\pgfsetstrokecolor{currentstroke}%
\pgfsetstrokeopacity{0.000000}%
\pgfsetdash{}{0pt}%
\pgfpathmoveto{\pgfqpoint{0.682376in}{0.535823in}}%
\pgfpathlineto{\pgfqpoint{5.785764in}{0.535823in}}%
\pgfpathlineto{\pgfqpoint{5.785764in}{3.250000in}}%
\pgfpathlineto{\pgfqpoint{0.682376in}{3.250000in}}%
\pgfpathclose%
\pgfusepath{fill}%
\end{pgfscope}%
\begin{pgfscope}%
\pgfsetbuttcap%
\pgfsetroundjoin%
\definecolor{currentfill}{rgb}{0.000000,0.000000,0.000000}%
\pgfsetfillcolor{currentfill}%
\pgfsetlinewidth{0.803000pt}%
\definecolor{currentstroke}{rgb}{0.000000,0.000000,0.000000}%
\pgfsetstrokecolor{currentstroke}%
\pgfsetdash{}{0pt}%
\pgfsys@defobject{currentmarker}{\pgfqpoint{0.000000in}{-0.048611in}}{\pgfqpoint{0.000000in}{0.000000in}}{%
\pgfpathmoveto{\pgfqpoint{0.000000in}{0.000000in}}%
\pgfpathlineto{\pgfqpoint{0.000000in}{-0.048611in}}%
\pgfusepath{stroke,fill}%
}%
\begin{pgfscope}%
\pgfsys@transformshift{0.806849in}{0.535823in}%
\pgfsys@useobject{currentmarker}{}%
\end{pgfscope}%
\end{pgfscope}%
\begin{pgfscope}%
\definecolor{textcolor}{rgb}{0.000000,0.000000,0.000000}%
\pgfsetstrokecolor{textcolor}%
\pgfsetfillcolor{textcolor}%
\pgftext[x=0.806849in,y=0.438600in,,top]{\color{textcolor}\rmfamily\fontsize{9.000000}{10.800000}\selectfont \(\displaystyle {10}\)}%
\end{pgfscope}%
\begin{pgfscope}%
\pgfsetbuttcap%
\pgfsetroundjoin%
\definecolor{currentfill}{rgb}{0.000000,0.000000,0.000000}%
\pgfsetfillcolor{currentfill}%
\pgfsetlinewidth{0.803000pt}%
\definecolor{currentstroke}{rgb}{0.000000,0.000000,0.000000}%
\pgfsetstrokecolor{currentstroke}%
\pgfsetdash{}{0pt}%
\pgfsys@defobject{currentmarker}{\pgfqpoint{0.000000in}{-0.048611in}}{\pgfqpoint{0.000000in}{0.000000in}}{%
\pgfpathmoveto{\pgfqpoint{0.000000in}{0.000000in}}%
\pgfpathlineto{\pgfqpoint{0.000000in}{-0.048611in}}%
\pgfusepath{stroke,fill}%
}%
\begin{pgfscope}%
\pgfsys@transformshift{1.429213in}{0.535823in}%
\pgfsys@useobject{currentmarker}{}%
\end{pgfscope}%
\end{pgfscope}%
\begin{pgfscope}%
\definecolor{textcolor}{rgb}{0.000000,0.000000,0.000000}%
\pgfsetstrokecolor{textcolor}%
\pgfsetfillcolor{textcolor}%
\pgftext[x=1.429213in,y=0.438600in,,top]{\color{textcolor}\rmfamily\fontsize{9.000000}{10.800000}\selectfont \(\displaystyle {15}\)}%
\end{pgfscope}%
\begin{pgfscope}%
\pgfsetbuttcap%
\pgfsetroundjoin%
\definecolor{currentfill}{rgb}{0.000000,0.000000,0.000000}%
\pgfsetfillcolor{currentfill}%
\pgfsetlinewidth{0.803000pt}%
\definecolor{currentstroke}{rgb}{0.000000,0.000000,0.000000}%
\pgfsetstrokecolor{currentstroke}%
\pgfsetdash{}{0pt}%
\pgfsys@defobject{currentmarker}{\pgfqpoint{0.000000in}{-0.048611in}}{\pgfqpoint{0.000000in}{0.000000in}}{%
\pgfpathmoveto{\pgfqpoint{0.000000in}{0.000000in}}%
\pgfpathlineto{\pgfqpoint{0.000000in}{-0.048611in}}%
\pgfusepath{stroke,fill}%
}%
\begin{pgfscope}%
\pgfsys@transformshift{2.051578in}{0.535823in}%
\pgfsys@useobject{currentmarker}{}%
\end{pgfscope}%
\end{pgfscope}%
\begin{pgfscope}%
\definecolor{textcolor}{rgb}{0.000000,0.000000,0.000000}%
\pgfsetstrokecolor{textcolor}%
\pgfsetfillcolor{textcolor}%
\pgftext[x=2.051578in,y=0.438600in,,top]{\color{textcolor}\rmfamily\fontsize{9.000000}{10.800000}\selectfont \(\displaystyle {20}\)}%
\end{pgfscope}%
\begin{pgfscope}%
\pgfsetbuttcap%
\pgfsetroundjoin%
\definecolor{currentfill}{rgb}{0.000000,0.000000,0.000000}%
\pgfsetfillcolor{currentfill}%
\pgfsetlinewidth{0.803000pt}%
\definecolor{currentstroke}{rgb}{0.000000,0.000000,0.000000}%
\pgfsetstrokecolor{currentstroke}%
\pgfsetdash{}{0pt}%
\pgfsys@defobject{currentmarker}{\pgfqpoint{0.000000in}{-0.048611in}}{\pgfqpoint{0.000000in}{0.000000in}}{%
\pgfpathmoveto{\pgfqpoint{0.000000in}{0.000000in}}%
\pgfpathlineto{\pgfqpoint{0.000000in}{-0.048611in}}%
\pgfusepath{stroke,fill}%
}%
\begin{pgfscope}%
\pgfsys@transformshift{2.673942in}{0.535823in}%
\pgfsys@useobject{currentmarker}{}%
\end{pgfscope}%
\end{pgfscope}%
\begin{pgfscope}%
\definecolor{textcolor}{rgb}{0.000000,0.000000,0.000000}%
\pgfsetstrokecolor{textcolor}%
\pgfsetfillcolor{textcolor}%
\pgftext[x=2.673942in,y=0.438600in,,top]{\color{textcolor}\rmfamily\fontsize{9.000000}{10.800000}\selectfont \(\displaystyle {25}\)}%
\end{pgfscope}%
\begin{pgfscope}%
\pgfsetbuttcap%
\pgfsetroundjoin%
\definecolor{currentfill}{rgb}{0.000000,0.000000,0.000000}%
\pgfsetfillcolor{currentfill}%
\pgfsetlinewidth{0.803000pt}%
\definecolor{currentstroke}{rgb}{0.000000,0.000000,0.000000}%
\pgfsetstrokecolor{currentstroke}%
\pgfsetdash{}{0pt}%
\pgfsys@defobject{currentmarker}{\pgfqpoint{0.000000in}{-0.048611in}}{\pgfqpoint{0.000000in}{0.000000in}}{%
\pgfpathmoveto{\pgfqpoint{0.000000in}{0.000000in}}%
\pgfpathlineto{\pgfqpoint{0.000000in}{-0.048611in}}%
\pgfusepath{stroke,fill}%
}%
\begin{pgfscope}%
\pgfsys@transformshift{3.296306in}{0.535823in}%
\pgfsys@useobject{currentmarker}{}%
\end{pgfscope}%
\end{pgfscope}%
\begin{pgfscope}%
\definecolor{textcolor}{rgb}{0.000000,0.000000,0.000000}%
\pgfsetstrokecolor{textcolor}%
\pgfsetfillcolor{textcolor}%
\pgftext[x=3.296306in,y=0.438600in,,top]{\color{textcolor}\rmfamily\fontsize{9.000000}{10.800000}\selectfont \(\displaystyle {30}\)}%
\end{pgfscope}%
\begin{pgfscope}%
\pgfsetbuttcap%
\pgfsetroundjoin%
\definecolor{currentfill}{rgb}{0.000000,0.000000,0.000000}%
\pgfsetfillcolor{currentfill}%
\pgfsetlinewidth{0.803000pt}%
\definecolor{currentstroke}{rgb}{0.000000,0.000000,0.000000}%
\pgfsetstrokecolor{currentstroke}%
\pgfsetdash{}{0pt}%
\pgfsys@defobject{currentmarker}{\pgfqpoint{0.000000in}{-0.048611in}}{\pgfqpoint{0.000000in}{0.000000in}}{%
\pgfpathmoveto{\pgfqpoint{0.000000in}{0.000000in}}%
\pgfpathlineto{\pgfqpoint{0.000000in}{-0.048611in}}%
\pgfusepath{stroke,fill}%
}%
\begin{pgfscope}%
\pgfsys@transformshift{3.918671in}{0.535823in}%
\pgfsys@useobject{currentmarker}{}%
\end{pgfscope}%
\end{pgfscope}%
\begin{pgfscope}%
\definecolor{textcolor}{rgb}{0.000000,0.000000,0.000000}%
\pgfsetstrokecolor{textcolor}%
\pgfsetfillcolor{textcolor}%
\pgftext[x=3.918671in,y=0.438600in,,top]{\color{textcolor}\rmfamily\fontsize{9.000000}{10.800000}\selectfont \(\displaystyle {35}\)}%
\end{pgfscope}%
\begin{pgfscope}%
\pgfsetbuttcap%
\pgfsetroundjoin%
\definecolor{currentfill}{rgb}{0.000000,0.000000,0.000000}%
\pgfsetfillcolor{currentfill}%
\pgfsetlinewidth{0.803000pt}%
\definecolor{currentstroke}{rgb}{0.000000,0.000000,0.000000}%
\pgfsetstrokecolor{currentstroke}%
\pgfsetdash{}{0pt}%
\pgfsys@defobject{currentmarker}{\pgfqpoint{0.000000in}{-0.048611in}}{\pgfqpoint{0.000000in}{0.000000in}}{%
\pgfpathmoveto{\pgfqpoint{0.000000in}{0.000000in}}%
\pgfpathlineto{\pgfqpoint{0.000000in}{-0.048611in}}%
\pgfusepath{stroke,fill}%
}%
\begin{pgfscope}%
\pgfsys@transformshift{4.541035in}{0.535823in}%
\pgfsys@useobject{currentmarker}{}%
\end{pgfscope}%
\end{pgfscope}%
\begin{pgfscope}%
\definecolor{textcolor}{rgb}{0.000000,0.000000,0.000000}%
\pgfsetstrokecolor{textcolor}%
\pgfsetfillcolor{textcolor}%
\pgftext[x=4.541035in,y=0.438600in,,top]{\color{textcolor}\rmfamily\fontsize{9.000000}{10.800000}\selectfont \(\displaystyle {40}\)}%
\end{pgfscope}%
\begin{pgfscope}%
\pgfsetbuttcap%
\pgfsetroundjoin%
\definecolor{currentfill}{rgb}{0.000000,0.000000,0.000000}%
\pgfsetfillcolor{currentfill}%
\pgfsetlinewidth{0.803000pt}%
\definecolor{currentstroke}{rgb}{0.000000,0.000000,0.000000}%
\pgfsetstrokecolor{currentstroke}%
\pgfsetdash{}{0pt}%
\pgfsys@defobject{currentmarker}{\pgfqpoint{0.000000in}{-0.048611in}}{\pgfqpoint{0.000000in}{0.000000in}}{%
\pgfpathmoveto{\pgfqpoint{0.000000in}{0.000000in}}%
\pgfpathlineto{\pgfqpoint{0.000000in}{-0.048611in}}%
\pgfusepath{stroke,fill}%
}%
\begin{pgfscope}%
\pgfsys@transformshift{5.163400in}{0.535823in}%
\pgfsys@useobject{currentmarker}{}%
\end{pgfscope}%
\end{pgfscope}%
\begin{pgfscope}%
\definecolor{textcolor}{rgb}{0.000000,0.000000,0.000000}%
\pgfsetstrokecolor{textcolor}%
\pgfsetfillcolor{textcolor}%
\pgftext[x=5.163400in,y=0.438600in,,top]{\color{textcolor}\rmfamily\fontsize{9.000000}{10.800000}\selectfont \(\displaystyle {45}\)}%
\end{pgfscope}%
\begin{pgfscope}%
\pgfsetbuttcap%
\pgfsetroundjoin%
\definecolor{currentfill}{rgb}{0.000000,0.000000,0.000000}%
\pgfsetfillcolor{currentfill}%
\pgfsetlinewidth{0.803000pt}%
\definecolor{currentstroke}{rgb}{0.000000,0.000000,0.000000}%
\pgfsetstrokecolor{currentstroke}%
\pgfsetdash{}{0pt}%
\pgfsys@defobject{currentmarker}{\pgfqpoint{0.000000in}{-0.048611in}}{\pgfqpoint{0.000000in}{0.000000in}}{%
\pgfpathmoveto{\pgfqpoint{0.000000in}{0.000000in}}%
\pgfpathlineto{\pgfqpoint{0.000000in}{-0.048611in}}%
\pgfusepath{stroke,fill}%
}%
\begin{pgfscope}%
\pgfsys@transformshift{5.785764in}{0.535823in}%
\pgfsys@useobject{currentmarker}{}%
\end{pgfscope}%
\end{pgfscope}%
\begin{pgfscope}%
\definecolor{textcolor}{rgb}{0.000000,0.000000,0.000000}%
\pgfsetstrokecolor{textcolor}%
\pgfsetfillcolor{textcolor}%
\pgftext[x=5.785764in,y=0.438600in,,top]{\color{textcolor}\rmfamily\fontsize{9.000000}{10.800000}\selectfont \(\displaystyle {50}\)}%
\end{pgfscope}%
\begin{pgfscope}%
\definecolor{textcolor}{rgb}{0.000000,0.000000,0.000000}%
\pgfsetstrokecolor{textcolor}%
\pgfsetfillcolor{textcolor}%
\pgftext[x=3.234070in,y=0.272655in,,top]{\color{textcolor}\rmfamily\fontsize{10.000000}{12.000000}\selectfont Upper bound on carving width}%
\end{pgfscope}%
\begin{pgfscope}%
\pgfsetbuttcap%
\pgfsetroundjoin%
\definecolor{currentfill}{rgb}{0.000000,0.000000,0.000000}%
\pgfsetfillcolor{currentfill}%
\pgfsetlinewidth{0.803000pt}%
\definecolor{currentstroke}{rgb}{0.000000,0.000000,0.000000}%
\pgfsetstrokecolor{currentstroke}%
\pgfsetdash{}{0pt}%
\pgfsys@defobject{currentmarker}{\pgfqpoint{-0.048611in}{0.000000in}}{\pgfqpoint{-0.000000in}{0.000000in}}{%
\pgfpathmoveto{\pgfqpoint{-0.000000in}{0.000000in}}%
\pgfpathlineto{\pgfqpoint{-0.048611in}{0.000000in}}%
\pgfusepath{stroke,fill}%
}%
\begin{pgfscope}%
\pgfsys@transformshift{0.682376in}{0.535823in}%
\pgfsys@useobject{currentmarker}{}%
\end{pgfscope}%
\end{pgfscope}%
\begin{pgfscope}%
\definecolor{textcolor}{rgb}{0.000000,0.000000,0.000000}%
\pgfsetstrokecolor{textcolor}%
\pgfsetfillcolor{textcolor}%
\pgftext[x=0.520918in, y=0.492778in, left, base]{\color{textcolor}\rmfamily\fontsize{9.000000}{10.800000}\selectfont \(\displaystyle {0}\)}%
\end{pgfscope}%
\begin{pgfscope}%
\pgfsetbuttcap%
\pgfsetroundjoin%
\definecolor{currentfill}{rgb}{0.000000,0.000000,0.000000}%
\pgfsetfillcolor{currentfill}%
\pgfsetlinewidth{0.803000pt}%
\definecolor{currentstroke}{rgb}{0.000000,0.000000,0.000000}%
\pgfsetstrokecolor{currentstroke}%
\pgfsetdash{}{0pt}%
\pgfsys@defobject{currentmarker}{\pgfqpoint{-0.048611in}{0.000000in}}{\pgfqpoint{-0.000000in}{0.000000in}}{%
\pgfpathmoveto{\pgfqpoint{-0.000000in}{0.000000in}}%
\pgfpathlineto{\pgfqpoint{-0.048611in}{0.000000in}}%
\pgfusepath{stroke,fill}%
}%
\begin{pgfscope}%
\pgfsys@transformshift{0.682376in}{1.029309in}%
\pgfsys@useobject{currentmarker}{}%
\end{pgfscope}%
\end{pgfscope}%
\begin{pgfscope}%
\definecolor{textcolor}{rgb}{0.000000,0.000000,0.000000}%
\pgfsetstrokecolor{textcolor}%
\pgfsetfillcolor{textcolor}%
\pgftext[x=0.392446in, y=0.986264in, left, base]{\color{textcolor}\rmfamily\fontsize{9.000000}{10.800000}\selectfont \(\displaystyle {200}\)}%
\end{pgfscope}%
\begin{pgfscope}%
\pgfsetbuttcap%
\pgfsetroundjoin%
\definecolor{currentfill}{rgb}{0.000000,0.000000,0.000000}%
\pgfsetfillcolor{currentfill}%
\pgfsetlinewidth{0.803000pt}%
\definecolor{currentstroke}{rgb}{0.000000,0.000000,0.000000}%
\pgfsetstrokecolor{currentstroke}%
\pgfsetdash{}{0pt}%
\pgfsys@defobject{currentmarker}{\pgfqpoint{-0.048611in}{0.000000in}}{\pgfqpoint{-0.000000in}{0.000000in}}{%
\pgfpathmoveto{\pgfqpoint{-0.000000in}{0.000000in}}%
\pgfpathlineto{\pgfqpoint{-0.048611in}{0.000000in}}%
\pgfusepath{stroke,fill}%
}%
\begin{pgfscope}%
\pgfsys@transformshift{0.682376in}{1.522796in}%
\pgfsys@useobject{currentmarker}{}%
\end{pgfscope}%
\end{pgfscope}%
\begin{pgfscope}%
\definecolor{textcolor}{rgb}{0.000000,0.000000,0.000000}%
\pgfsetstrokecolor{textcolor}%
\pgfsetfillcolor{textcolor}%
\pgftext[x=0.392446in, y=1.479751in, left, base]{\color{textcolor}\rmfamily\fontsize{9.000000}{10.800000}\selectfont \(\displaystyle {400}\)}%
\end{pgfscope}%
\begin{pgfscope}%
\pgfsetbuttcap%
\pgfsetroundjoin%
\definecolor{currentfill}{rgb}{0.000000,0.000000,0.000000}%
\pgfsetfillcolor{currentfill}%
\pgfsetlinewidth{0.803000pt}%
\definecolor{currentstroke}{rgb}{0.000000,0.000000,0.000000}%
\pgfsetstrokecolor{currentstroke}%
\pgfsetdash{}{0pt}%
\pgfsys@defobject{currentmarker}{\pgfqpoint{-0.048611in}{0.000000in}}{\pgfqpoint{-0.000000in}{0.000000in}}{%
\pgfpathmoveto{\pgfqpoint{-0.000000in}{0.000000in}}%
\pgfpathlineto{\pgfqpoint{-0.048611in}{0.000000in}}%
\pgfusepath{stroke,fill}%
}%
\begin{pgfscope}%
\pgfsys@transformshift{0.682376in}{2.016283in}%
\pgfsys@useobject{currentmarker}{}%
\end{pgfscope}%
\end{pgfscope}%
\begin{pgfscope}%
\definecolor{textcolor}{rgb}{0.000000,0.000000,0.000000}%
\pgfsetstrokecolor{textcolor}%
\pgfsetfillcolor{textcolor}%
\pgftext[x=0.392446in, y=1.973238in, left, base]{\color{textcolor}\rmfamily\fontsize{9.000000}{10.800000}\selectfont \(\displaystyle {600}\)}%
\end{pgfscope}%
\begin{pgfscope}%
\pgfsetbuttcap%
\pgfsetroundjoin%
\definecolor{currentfill}{rgb}{0.000000,0.000000,0.000000}%
\pgfsetfillcolor{currentfill}%
\pgfsetlinewidth{0.803000pt}%
\definecolor{currentstroke}{rgb}{0.000000,0.000000,0.000000}%
\pgfsetstrokecolor{currentstroke}%
\pgfsetdash{}{0pt}%
\pgfsys@defobject{currentmarker}{\pgfqpoint{-0.048611in}{0.000000in}}{\pgfqpoint{-0.000000in}{0.000000in}}{%
\pgfpathmoveto{\pgfqpoint{-0.000000in}{0.000000in}}%
\pgfpathlineto{\pgfqpoint{-0.048611in}{0.000000in}}%
\pgfusepath{stroke,fill}%
}%
\begin{pgfscope}%
\pgfsys@transformshift{0.682376in}{2.509770in}%
\pgfsys@useobject{currentmarker}{}%
\end{pgfscope}%
\end{pgfscope}%
\begin{pgfscope}%
\definecolor{textcolor}{rgb}{0.000000,0.000000,0.000000}%
\pgfsetstrokecolor{textcolor}%
\pgfsetfillcolor{textcolor}%
\pgftext[x=0.392446in, y=2.466725in, left, base]{\color{textcolor}\rmfamily\fontsize{9.000000}{10.800000}\selectfont \(\displaystyle {800}\)}%
\end{pgfscope}%
\begin{pgfscope}%
\pgfsetbuttcap%
\pgfsetroundjoin%
\definecolor{currentfill}{rgb}{0.000000,0.000000,0.000000}%
\pgfsetfillcolor{currentfill}%
\pgfsetlinewidth{0.803000pt}%
\definecolor{currentstroke}{rgb}{0.000000,0.000000,0.000000}%
\pgfsetstrokecolor{currentstroke}%
\pgfsetdash{}{0pt}%
\pgfsys@defobject{currentmarker}{\pgfqpoint{-0.048611in}{0.000000in}}{\pgfqpoint{-0.000000in}{0.000000in}}{%
\pgfpathmoveto{\pgfqpoint{-0.000000in}{0.000000in}}%
\pgfpathlineto{\pgfqpoint{-0.048611in}{0.000000in}}%
\pgfusepath{stroke,fill}%
}%
\begin{pgfscope}%
\pgfsys@transformshift{0.682376in}{3.003257in}%
\pgfsys@useobject{currentmarker}{}%
\end{pgfscope}%
\end{pgfscope}%
\begin{pgfscope}%
\definecolor{textcolor}{rgb}{0.000000,0.000000,0.000000}%
\pgfsetstrokecolor{textcolor}%
\pgfsetfillcolor{textcolor}%
\pgftext[x=0.328211in, y=2.960212in, left, base]{\color{textcolor}\rmfamily\fontsize{9.000000}{10.800000}\selectfont \(\displaystyle {1000}\)}%
\end{pgfscope}%
\begin{pgfscope}%
\definecolor{textcolor}{rgb}{0.000000,0.000000,0.000000}%
\pgfsetstrokecolor{textcolor}%
\pgfsetfillcolor{textcolor}%
\pgftext[x=0.272655in,y=1.892911in,,bottom,rotate=90.000000]{\color{textcolor}\rmfamily\fontsize{10.000000}{12.000000}\selectfont Number of solved benchmarks}%
\end{pgfscope}%
\begin{pgfscope}%
\pgfpathrectangle{\pgfqpoint{0.682376in}{0.535823in}}{\pgfqpoint{5.103389in}{2.714177in}}%
\pgfusepath{clip}%
\pgfsetbuttcap%
\pgfsetroundjoin%
\pgfsetlinewidth{2.007500pt}%
\definecolor{currentstroke}{rgb}{0.000000,0.000000,0.000000}%
\pgfsetstrokecolor{currentstroke}%
\pgfsetdash{{2.000000pt}{3.300000pt}}{0.000000pt}%
\pgfpathmoveto{\pgfqpoint{0.806849in}{0.595041in}}%
\pgfpathlineto{\pgfqpoint{0.931321in}{0.607378in}}%
\pgfpathlineto{\pgfqpoint{1.055794in}{0.632053in}}%
\pgfpathlineto{\pgfqpoint{1.180267in}{0.681401in}}%
\pgfpathlineto{\pgfqpoint{1.304740in}{0.698673in}}%
\pgfpathlineto{\pgfqpoint{1.429213in}{0.757892in}}%
\pgfpathlineto{\pgfqpoint{1.553686in}{0.829447in}}%
\pgfpathlineto{\pgfqpoint{1.678159in}{0.881263in}}%
\pgfpathlineto{\pgfqpoint{1.802632in}{0.972558in}}%
\pgfpathlineto{\pgfqpoint{1.927105in}{1.049049in}}%
\pgfpathlineto{\pgfqpoint{2.051578in}{1.199562in}}%
\pgfpathlineto{\pgfqpoint{2.176050in}{1.342674in}}%
\pgfpathlineto{\pgfqpoint{2.300523in}{1.505524in}}%
\pgfpathlineto{\pgfqpoint{2.424996in}{1.643700in}}%
\pgfpathlineto{\pgfqpoint{2.549469in}{1.813953in}}%
\pgfpathlineto{\pgfqpoint{2.673942in}{1.971869in}}%
\pgfpathlineto{\pgfqpoint{2.798415in}{2.147057in}}%
\pgfpathlineto{\pgfqpoint{2.922888in}{2.235885in}}%
\pgfpathlineto{\pgfqpoint{3.047361in}{2.381463in}}%
\pgfpathlineto{\pgfqpoint{3.171834in}{2.445617in}}%
\pgfpathlineto{\pgfqpoint{3.296306in}{2.494965in}}%
\pgfpathlineto{\pgfqpoint{3.420779in}{2.531977in}}%
\pgfpathlineto{\pgfqpoint{3.545252in}{2.546781in}}%
\pgfpathlineto{\pgfqpoint{3.669725in}{2.564053in}}%
\pgfpathlineto{\pgfqpoint{4.167617in}{2.633142in}}%
\pgfpathlineto{\pgfqpoint{4.292090in}{2.729371in}}%
\pgfpathlineto{\pgfqpoint{4.416562in}{2.833004in}}%
\pgfpathlineto{\pgfqpoint{4.541035in}{2.973647in}}%
\pgfpathlineto{\pgfqpoint{4.665508in}{3.060008in}}%
\pgfpathlineto{\pgfqpoint{4.789981in}{3.114291in}}%
\pgfpathlineto{\pgfqpoint{4.914454in}{3.134031in}}%
\pgfpathlineto{\pgfqpoint{5.038927in}{3.138965in}}%
\pgfpathlineto{\pgfqpoint{5.163400in}{3.163640in}}%
\pgfpathlineto{\pgfqpoint{5.795764in}{3.169338in}}%
\pgfusepath{stroke}%
\end{pgfscope}%
\begin{pgfscope}%
\pgfpathrectangle{\pgfqpoint{0.682376in}{0.535823in}}{\pgfqpoint{5.103389in}{2.714177in}}%
\pgfusepath{clip}%
\pgfsetbuttcap%
\pgfsetroundjoin%
\pgfsetlinewidth{2.007500pt}%
\definecolor{currentstroke}{rgb}{1.000000,0.843137,0.000000}%
\pgfsetstrokecolor{currentstroke}%
\pgfsetdash{{7.400000pt}{3.200000pt}}{0.000000pt}%
\pgfpathmoveto{\pgfqpoint{0.806849in}{0.595041in}}%
\pgfpathlineto{\pgfqpoint{0.931321in}{0.607378in}}%
\pgfpathlineto{\pgfqpoint{1.055794in}{0.632053in}}%
\pgfpathlineto{\pgfqpoint{1.180267in}{0.681401in}}%
\pgfpathlineto{\pgfqpoint{1.304740in}{0.698673in}}%
\pgfpathlineto{\pgfqpoint{1.429213in}{0.757892in}}%
\pgfpathlineto{\pgfqpoint{1.553686in}{0.829447in}}%
\pgfpathlineto{\pgfqpoint{1.678159in}{0.881263in}}%
\pgfpathlineto{\pgfqpoint{1.802632in}{0.972558in}}%
\pgfpathlineto{\pgfqpoint{1.927105in}{1.049049in}}%
\pgfpathlineto{\pgfqpoint{2.051578in}{1.199562in}}%
\pgfpathlineto{\pgfqpoint{2.176050in}{1.342674in}}%
\pgfpathlineto{\pgfqpoint{2.300523in}{1.463578in}}%
\pgfpathlineto{\pgfqpoint{2.424996in}{1.535133in}}%
\pgfpathlineto{\pgfqpoint{2.549469in}{1.638766in}}%
\pgfpathlineto{\pgfqpoint{2.673942in}{1.705386in}}%
\pgfpathlineto{\pgfqpoint{2.798415in}{1.734996in}}%
\pgfpathlineto{\pgfqpoint{2.922888in}{1.752268in}}%
\pgfusepath{stroke}%
\end{pgfscope}%
\begin{pgfscope}%
\pgfpathrectangle{\pgfqpoint{0.682376in}{0.535823in}}{\pgfqpoint{5.103389in}{2.714177in}}%
\pgfusepath{clip}%
\pgfsetbuttcap%
\pgfsetroundjoin%
\pgfsetlinewidth{2.007500pt}%
\definecolor{currentstroke}{rgb}{1.000000,0.694118,0.305882}%
\pgfsetstrokecolor{currentstroke}%
\pgfsetdash{{2.000000pt}{3.300000pt}}{0.000000pt}%
\pgfpathmoveto{\pgfqpoint{0.806849in}{0.595041in}}%
\pgfpathlineto{\pgfqpoint{0.931321in}{0.607378in}}%
\pgfpathlineto{\pgfqpoint{1.055794in}{0.632053in}}%
\pgfpathlineto{\pgfqpoint{1.180267in}{0.681401in}}%
\pgfpathlineto{\pgfqpoint{1.304740in}{0.698673in}}%
\pgfpathlineto{\pgfqpoint{1.429213in}{0.757892in}}%
\pgfpathlineto{\pgfqpoint{1.553686in}{0.829447in}}%
\pgfpathlineto{\pgfqpoint{1.678159in}{0.881263in}}%
\pgfpathlineto{\pgfqpoint{1.802632in}{0.972558in}}%
\pgfpathlineto{\pgfqpoint{1.927105in}{1.049049in}}%
\pgfpathlineto{\pgfqpoint{2.051578in}{1.199562in}}%
\pgfpathlineto{\pgfqpoint{2.176050in}{1.342674in}}%
\pgfpathlineto{\pgfqpoint{2.300523in}{1.505524in}}%
\pgfpathlineto{\pgfqpoint{2.424996in}{1.638766in}}%
\pgfpathlineto{\pgfqpoint{2.549469in}{1.796681in}}%
\pgfpathlineto{\pgfqpoint{2.673942in}{1.905248in}}%
\pgfpathlineto{\pgfqpoint{2.798415in}{1.976804in}}%
\pgfpathlineto{\pgfqpoint{2.922888in}{2.003946in}}%
\pgfpathlineto{\pgfqpoint{3.047361in}{2.031088in}}%
\pgfpathlineto{\pgfqpoint{3.171834in}{2.043425in}}%
\pgfusepath{stroke}%
\end{pgfscope}%
\begin{pgfscope}%
\pgfpathrectangle{\pgfqpoint{0.682376in}{0.535823in}}{\pgfqpoint{5.103389in}{2.714177in}}%
\pgfusepath{clip}%
\pgfsetrectcap%
\pgfsetroundjoin%
\pgfsetlinewidth{2.007500pt}%
\definecolor{currentstroke}{rgb}{0.980392,0.529412,0.458824}%
\pgfsetstrokecolor{currentstroke}%
\pgfsetdash{}{0pt}%
\pgfpathmoveto{\pgfqpoint{0.806849in}{0.595041in}}%
\pgfpathlineto{\pgfqpoint{0.931321in}{0.607378in}}%
\pgfpathlineto{\pgfqpoint{1.055794in}{0.632053in}}%
\pgfpathlineto{\pgfqpoint{1.180267in}{0.681401in}}%
\pgfpathlineto{\pgfqpoint{1.304740in}{0.698673in}}%
\pgfpathlineto{\pgfqpoint{1.429213in}{0.757892in}}%
\pgfpathlineto{\pgfqpoint{1.553686in}{0.829447in}}%
\pgfpathlineto{\pgfqpoint{1.678159in}{0.881263in}}%
\pgfpathlineto{\pgfqpoint{1.802632in}{0.972558in}}%
\pgfpathlineto{\pgfqpoint{1.927105in}{1.049049in}}%
\pgfpathlineto{\pgfqpoint{2.051578in}{1.199562in}}%
\pgfpathlineto{\pgfqpoint{2.176050in}{1.342674in}}%
\pgfpathlineto{\pgfqpoint{2.300523in}{1.505524in}}%
\pgfpathlineto{\pgfqpoint{2.424996in}{1.643700in}}%
\pgfpathlineto{\pgfqpoint{2.549469in}{1.809019in}}%
\pgfpathlineto{\pgfqpoint{2.673942in}{1.961999in}}%
\pgfpathlineto{\pgfqpoint{2.798415in}{2.129785in}}%
\pgfpathlineto{\pgfqpoint{2.922888in}{2.213678in}}%
\pgfpathlineto{\pgfqpoint{3.047361in}{2.243287in}}%
\pgfpathlineto{\pgfqpoint{3.171834in}{2.255624in}}%
\pgfpathlineto{\pgfqpoint{3.296306in}{2.258092in}}%
\pgfusepath{stroke}%
\end{pgfscope}%
\begin{pgfscope}%
\pgfpathrectangle{\pgfqpoint{0.682376in}{0.535823in}}{\pgfqpoint{5.103389in}{2.714177in}}%
\pgfusepath{clip}%
\pgfsetbuttcap%
\pgfsetroundjoin%
\pgfsetlinewidth{2.007500pt}%
\definecolor{currentstroke}{rgb}{0.866667,0.058824,0.058824}%
\pgfsetstrokecolor{currentstroke}%
\pgfsetdash{{7.400000pt}{3.200000pt}}{0.000000pt}%
\pgfpathmoveto{\pgfqpoint{0.806849in}{0.577769in}}%
\pgfpathlineto{\pgfqpoint{0.931321in}{0.585171in}}%
\pgfpathlineto{\pgfqpoint{1.055794in}{0.599976in}}%
\pgfpathlineto{\pgfqpoint{1.180267in}{0.634520in}}%
\pgfpathlineto{\pgfqpoint{1.304740in}{0.649325in}}%
\pgfpathlineto{\pgfqpoint{1.429213in}{0.671532in}}%
\pgfpathlineto{\pgfqpoint{1.553686in}{0.703608in}}%
\pgfpathlineto{\pgfqpoint{1.678159in}{0.728282in}}%
\pgfpathlineto{\pgfqpoint{1.802632in}{0.755424in}}%
\pgfpathlineto{\pgfqpoint{1.927105in}{0.799838in}}%
\pgfpathlineto{\pgfqpoint{2.051578in}{0.918275in}}%
\pgfpathlineto{\pgfqpoint{2.176050in}{1.024375in}}%
\pgfpathlineto{\pgfqpoint{2.300523in}{1.150214in}}%
\pgfpathlineto{\pgfqpoint{2.424996in}{1.253846in}}%
\pgfpathlineto{\pgfqpoint{2.549469in}{1.394490in}}%
\pgfpathlineto{\pgfqpoint{2.673942in}{1.532666in}}%
\pgfpathlineto{\pgfqpoint{2.798415in}{1.636298in}}%
\pgfpathlineto{\pgfqpoint{2.922888in}{1.717724in}}%
\pgfpathlineto{\pgfqpoint{3.047361in}{1.850965in}}%
\pgfpathlineto{\pgfqpoint{3.171834in}{1.910183in}}%
\pgfpathlineto{\pgfqpoint{3.296306in}{1.954597in}}%
\pgfpathlineto{\pgfqpoint{3.420779in}{1.986674in}}%
\pgfpathlineto{\pgfqpoint{3.545252in}{1.991609in}}%
\pgfpathlineto{\pgfqpoint{4.167617in}{2.028620in}}%
\pgfpathlineto{\pgfqpoint{4.292090in}{2.102643in}}%
\pgfpathlineto{\pgfqpoint{4.416562in}{2.184069in}}%
\pgfpathlineto{\pgfqpoint{4.541035in}{2.280298in}}%
\pgfpathlineto{\pgfqpoint{4.665508in}{2.329647in}}%
\pgfpathlineto{\pgfqpoint{4.789981in}{2.371594in}}%
\pgfpathlineto{\pgfqpoint{4.914454in}{2.388866in}}%
\pgfpathlineto{\pgfqpoint{5.038927in}{2.391333in}}%
\pgfpathlineto{\pgfqpoint{5.795764in}{2.392583in}}%
\pgfusepath{stroke}%
\end{pgfscope}%
\begin{pgfscope}%
\pgfpathrectangle{\pgfqpoint{0.682376in}{0.535823in}}{\pgfqpoint{5.103389in}{2.714177in}}%
\pgfusepath{clip}%
\pgfsetbuttcap%
\pgfsetroundjoin%
\pgfsetlinewidth{2.007500pt}%
\definecolor{currentstroke}{rgb}{0.917647,0.372549,0.580392}%
\pgfsetstrokecolor{currentstroke}%
\pgfsetdash{{2.000000pt}{3.300000pt}}{0.000000pt}%
\pgfpathmoveto{\pgfqpoint{0.806849in}{0.595041in}}%
\pgfpathlineto{\pgfqpoint{0.931321in}{0.607378in}}%
\pgfpathlineto{\pgfqpoint{1.055794in}{0.632053in}}%
\pgfpathlineto{\pgfqpoint{1.180267in}{0.681401in}}%
\pgfpathlineto{\pgfqpoint{1.304740in}{0.698673in}}%
\pgfpathlineto{\pgfqpoint{1.429213in}{0.755424in}}%
\pgfpathlineto{\pgfqpoint{1.553686in}{0.817110in}}%
\pgfpathlineto{\pgfqpoint{1.678159in}{0.866459in}}%
\pgfpathlineto{\pgfqpoint{1.802632in}{0.938014in}}%
\pgfpathlineto{\pgfqpoint{1.927105in}{0.997233in}}%
\pgfpathlineto{\pgfqpoint{2.051578in}{1.125539in}}%
\pgfpathlineto{\pgfqpoint{2.176050in}{1.251378in}}%
\pgfpathlineto{\pgfqpoint{2.300523in}{1.392022in}}%
\pgfpathlineto{\pgfqpoint{2.424996in}{1.505524in}}%
\pgfpathlineto{\pgfqpoint{2.549469in}{1.660973in}}%
\pgfpathlineto{\pgfqpoint{2.673942in}{1.801616in}}%
\pgfpathlineto{\pgfqpoint{2.798415in}{1.929923in}}%
\pgfpathlineto{\pgfqpoint{2.922888in}{2.013816in}}%
\pgfpathlineto{\pgfqpoint{3.047361in}{2.154459in}}%
\pgfpathlineto{\pgfqpoint{3.171834in}{2.211210in}}%
\pgfpathlineto{\pgfqpoint{3.296306in}{2.250689in}}%
\pgfpathlineto{\pgfqpoint{3.420779in}{2.282766in}}%
\pgfpathlineto{\pgfqpoint{3.545252in}{2.287701in}}%
\pgfpathlineto{\pgfqpoint{3.669725in}{2.295103in}}%
\pgfpathlineto{\pgfqpoint{4.167617in}{2.334582in}}%
\pgfpathlineto{\pgfqpoint{4.292090in}{2.408605in}}%
\pgfpathlineto{\pgfqpoint{4.416562in}{2.487563in}}%
\pgfpathlineto{\pgfqpoint{4.541035in}{2.581325in}}%
\pgfpathlineto{\pgfqpoint{4.665508in}{2.625739in}}%
\pgfpathlineto{\pgfqpoint{4.789981in}{2.662751in}}%
\pgfpathlineto{\pgfqpoint{4.914454in}{2.680023in}}%
\pgfpathlineto{\pgfqpoint{5.038927in}{2.682490in}}%
\pgfpathlineto{\pgfqpoint{5.795764in}{2.683740in}}%
\pgfusepath{stroke}%
\end{pgfscope}%
\begin{pgfscope}%
\pgfpathrectangle{\pgfqpoint{0.682376in}{0.535823in}}{\pgfqpoint{5.103389in}{2.714177in}}%
\pgfusepath{clip}%
\pgfsetrectcap%
\pgfsetroundjoin%
\pgfsetlinewidth{2.007500pt}%
\definecolor{currentstroke}{rgb}{0.615686,0.007843,0.843137}%
\pgfsetstrokecolor{currentstroke}%
\pgfsetdash{}{0pt}%
\pgfpathmoveto{\pgfqpoint{0.806849in}{0.595041in}}%
\pgfpathlineto{\pgfqpoint{0.931321in}{0.607378in}}%
\pgfpathlineto{\pgfqpoint{1.055794in}{0.632053in}}%
\pgfpathlineto{\pgfqpoint{1.180267in}{0.681401in}}%
\pgfpathlineto{\pgfqpoint{1.304740in}{0.698673in}}%
\pgfpathlineto{\pgfqpoint{1.429213in}{0.755424in}}%
\pgfpathlineto{\pgfqpoint{1.553686in}{0.812175in}}%
\pgfpathlineto{\pgfqpoint{1.678159in}{0.856589in}}%
\pgfpathlineto{\pgfqpoint{1.802632in}{0.928145in}}%
\pgfpathlineto{\pgfqpoint{1.927105in}{0.997233in}}%
\pgfpathlineto{\pgfqpoint{2.051578in}{1.130474in}}%
\pgfpathlineto{\pgfqpoint{2.176050in}{1.258781in}}%
\pgfpathlineto{\pgfqpoint{2.300523in}{1.404359in}}%
\pgfpathlineto{\pgfqpoint{2.424996in}{1.530199in}}%
\pgfpathlineto{\pgfqpoint{2.549469in}{1.693049in}}%
\pgfpathlineto{\pgfqpoint{2.673942in}{1.838628in}}%
\pgfpathlineto{\pgfqpoint{2.798415in}{1.974337in}}%
\pgfpathlineto{\pgfqpoint{2.922888in}{2.055762in}}%
\pgfpathlineto{\pgfqpoint{3.047361in}{2.191471in}}%
\pgfpathlineto{\pgfqpoint{3.171834in}{2.245754in}}%
\pgfpathlineto{\pgfqpoint{3.296306in}{2.287701in}}%
\pgfpathlineto{\pgfqpoint{3.420779in}{2.319777in}}%
\pgfpathlineto{\pgfqpoint{3.545252in}{2.332115in}}%
\pgfpathlineto{\pgfqpoint{3.669725in}{2.349387in}}%
\pgfpathlineto{\pgfqpoint{4.167617in}{2.408605in}}%
\pgfpathlineto{\pgfqpoint{4.292090in}{2.485095in}}%
\pgfpathlineto{\pgfqpoint{4.416562in}{2.564053in}}%
\pgfpathlineto{\pgfqpoint{4.541035in}{2.660283in}}%
\pgfpathlineto{\pgfqpoint{4.665508in}{2.729371in}}%
\pgfpathlineto{\pgfqpoint{4.789981in}{2.768850in}}%
\pgfpathlineto{\pgfqpoint{4.914454in}{2.786122in}}%
\pgfpathlineto{\pgfqpoint{5.038927in}{2.788590in}}%
\pgfpathlineto{\pgfqpoint{5.163400in}{2.805862in}}%
\pgfpathlineto{\pgfqpoint{5.795764in}{2.808141in}}%
\pgfusepath{stroke}%
\end{pgfscope}%
\begin{pgfscope}%
\pgfsetrectcap%
\pgfsetmiterjoin%
\pgfsetlinewidth{0.803000pt}%
\definecolor{currentstroke}{rgb}{0.000000,0.000000,0.000000}%
\pgfsetstrokecolor{currentstroke}%
\pgfsetdash{}{0pt}%
\pgfpathmoveto{\pgfqpoint{0.682376in}{0.535823in}}%
\pgfpathlineto{\pgfqpoint{0.682376in}{3.250000in}}%
\pgfusepath{stroke}%
\end{pgfscope}%
\begin{pgfscope}%
\pgfsetrectcap%
\pgfsetmiterjoin%
\pgfsetlinewidth{0.803000pt}%
\definecolor{currentstroke}{rgb}{0.000000,0.000000,0.000000}%
\pgfsetstrokecolor{currentstroke}%
\pgfsetdash{}{0pt}%
\pgfpathmoveto{\pgfqpoint{5.785764in}{0.535823in}}%
\pgfpathlineto{\pgfqpoint{5.785764in}{3.250000in}}%
\pgfusepath{stroke}%
\end{pgfscope}%
\begin{pgfscope}%
\pgfsetrectcap%
\pgfsetmiterjoin%
\pgfsetlinewidth{0.803000pt}%
\definecolor{currentstroke}{rgb}{0.000000,0.000000,0.000000}%
\pgfsetstrokecolor{currentstroke}%
\pgfsetdash{}{0pt}%
\pgfpathmoveto{\pgfqpoint{0.682376in}{0.535823in}}%
\pgfpathlineto{\pgfqpoint{5.785764in}{0.535823in}}%
\pgfusepath{stroke}%
\end{pgfscope}%
\begin{pgfscope}%
\pgfsetrectcap%
\pgfsetmiterjoin%
\pgfsetlinewidth{0.803000pt}%
\definecolor{currentstroke}{rgb}{0.000000,0.000000,0.000000}%
\pgfsetstrokecolor{currentstroke}%
\pgfsetdash{}{0pt}%
\pgfpathmoveto{\pgfqpoint{0.682376in}{3.250000in}}%
\pgfpathlineto{\pgfqpoint{5.785764in}{3.250000in}}%
\pgfusepath{stroke}%
\end{pgfscope}%
\begin{pgfscope}%
\pgfsetbuttcap%
\pgfsetroundjoin%
\pgfsetlinewidth{2.007500pt}%
\definecolor{currentstroke}{rgb}{0.000000,0.000000,0.000000}%
\pgfsetstrokecolor{currentstroke}%
\pgfsetdash{{2.000000pt}{3.300000pt}}{0.000000pt}%
\pgfpathmoveto{\pgfqpoint{0.732376in}{3.156250in}}%
\pgfpathlineto{\pgfqpoint{0.982376in}{3.156250in}}%
\pgfusepath{stroke}%
\end{pgfscope}%
\begin{pgfscope}%
\definecolor{textcolor}{rgb}{0.000000,0.000000,0.000000}%
\pgfsetstrokecolor{textcolor}%
\pgfsetfillcolor{textcolor}%
\pgftext[x=1.007376in,y=3.112500in,left,base]{\color{textcolor}\rmfamily\fontsize{9.000000}{10.800000}\selectfont All benchmarks}%
\end{pgfscope}%
\begin{pgfscope}%
\pgfsetbuttcap%
\pgfsetroundjoin%
\pgfsetlinewidth{2.007500pt}%
\definecolor{currentstroke}{rgb}{1.000000,0.843137,0.000000}%
\pgfsetstrokecolor{currentstroke}%
\pgfsetdash{{7.400000pt}{3.200000pt}}{0.000000pt}%
\pgfpathmoveto{\pgfqpoint{0.732376in}{2.994450in}}%
\pgfpathlineto{\pgfqpoint{0.982376in}{2.994450in}}%
\pgfusepath{stroke}%
\end{pgfscope}%
\begin{pgfscope}%
\definecolor{textcolor}{rgb}{0.000000,0.000000,0.000000}%
\pgfsetstrokecolor{textcolor}%
\pgfsetfillcolor{textcolor}%
\pgftext[x=1.007376in,y=2.950700in,left,base]{\color{textcolor}\rmfamily\fontsize{9.000000}{10.800000}\selectfont FT+htd}%
\end{pgfscope}%
\begin{pgfscope}%
\pgfsetbuttcap%
\pgfsetroundjoin%
\pgfsetlinewidth{2.007500pt}%
\definecolor{currentstroke}{rgb}{1.000000,0.694118,0.305882}%
\pgfsetstrokecolor{currentstroke}%
\pgfsetdash{{2.000000pt}{3.300000pt}}{0.000000pt}%
\pgfpathmoveto{\pgfqpoint{0.732376in}{2.832651in}}%
\pgfpathlineto{\pgfqpoint{0.982376in}{2.832651in}}%
\pgfusepath{stroke}%
\end{pgfscope}%
\begin{pgfscope}%
\definecolor{textcolor}{rgb}{0.000000,0.000000,0.000000}%
\pgfsetstrokecolor{textcolor}%
\pgfsetfillcolor{textcolor}%
\pgftext[x=1.007376in,y=2.788901in,left,base]{\color{textcolor}\rmfamily\fontsize{9.000000}{10.800000}\selectfont FT+Flow}%
\end{pgfscope}%
\begin{pgfscope}%
\pgfsetrectcap%
\pgfsetroundjoin%
\pgfsetlinewidth{2.007500pt}%
\definecolor{currentstroke}{rgb}{0.980392,0.529412,0.458824}%
\pgfsetstrokecolor{currentstroke}%
\pgfsetdash{}{0pt}%
\pgfpathmoveto{\pgfqpoint{0.732376in}{2.670851in}}%
\pgfpathlineto{\pgfqpoint{0.982376in}{2.670851in}}%
\pgfusepath{stroke}%
\end{pgfscope}%
\begin{pgfscope}%
\definecolor{textcolor}{rgb}{0.000000,0.000000,0.000000}%
\pgfsetstrokecolor{textcolor}%
\pgfsetfillcolor{textcolor}%
\pgftext[x=1.007376in,y=2.627101in,left,base]{\color{textcolor}\rmfamily\fontsize{9.000000}{10.800000}\selectfont FT+Tamaki}%
\end{pgfscope}%
\begin{pgfscope}%
\pgfsetbuttcap%
\pgfsetroundjoin%
\pgfsetlinewidth{2.007500pt}%
\definecolor{currentstroke}{rgb}{0.866667,0.058824,0.058824}%
\pgfsetstrokecolor{currentstroke}%
\pgfsetdash{{7.400000pt}{3.200000pt}}{0.000000pt}%
\pgfpathmoveto{\pgfqpoint{0.732376in}{2.509052in}}%
\pgfpathlineto{\pgfqpoint{0.982376in}{2.509052in}}%
\pgfusepath{stroke}%
\end{pgfscope}%
\begin{pgfscope}%
\definecolor{textcolor}{rgb}{0.000000,0.000000,0.000000}%
\pgfsetstrokecolor{textcolor}%
\pgfsetfillcolor{textcolor}%
\pgftext[x=1.007376in,y=2.465302in,left,base]{\color{textcolor}\rmfamily\fontsize{9.000000}{10.800000}\selectfont cachet}%
\end{pgfscope}%
\begin{pgfscope}%
\pgfsetbuttcap%
\pgfsetroundjoin%
\pgfsetlinewidth{2.007500pt}%
\definecolor{currentstroke}{rgb}{0.917647,0.372549,0.580392}%
\pgfsetstrokecolor{currentstroke}%
\pgfsetdash{{2.000000pt}{3.300000pt}}{0.000000pt}%
\pgfpathmoveto{\pgfqpoint{0.732376in}{2.347252in}}%
\pgfpathlineto{\pgfqpoint{0.982376in}{2.347252in}}%
\pgfusepath{stroke}%
\end{pgfscope}%
\begin{pgfscope}%
\definecolor{textcolor}{rgb}{0.000000,0.000000,0.000000}%
\pgfsetstrokecolor{textcolor}%
\pgfsetfillcolor{textcolor}%
\pgftext[x=1.007376in,y=2.303502in,left,base]{\color{textcolor}\rmfamily\fontsize{9.000000}{10.800000}\selectfont miniC2D}%
\end{pgfscope}%
\begin{pgfscope}%
\pgfsetrectcap%
\pgfsetroundjoin%
\pgfsetlinewidth{2.007500pt}%
\definecolor{currentstroke}{rgb}{0.615686,0.007843,0.843137}%
\pgfsetstrokecolor{currentstroke}%
\pgfsetdash{}{0pt}%
\pgfpathmoveto{\pgfqpoint{0.732376in}{2.185453in}}%
\pgfpathlineto{\pgfqpoint{0.982376in}{2.185453in}}%
\pgfusepath{stroke}%
\end{pgfscope}%
\begin{pgfscope}%
\definecolor{textcolor}{rgb}{0.000000,0.000000,0.000000}%
\pgfsetstrokecolor{textcolor}%
\pgfsetfillcolor{textcolor}%
\pgftext[x=1.007376in,y=2.141703in,left,base]{\color{textcolor}\rmfamily\fontsize{9.000000}{10.800000}\selectfont d4}%
\end{pgfscope}%
\end{pgfpicture}%
\makeatother%
\endgroup%

	\caption{\label{fig:cachet-carving-cactus} A plot of the number of benchmarks solved by various methods organized by carving width. Each $(x,y)$ data point indicates that the corresponding tool was able to solve $y$ benchmarks whose carving width (after \textbf{FT}-preprocessing) was at most $x$. Our approach \textbf{FT+Tamaki} can solve almost all benchmarks with carving width below 27 (unlike existing model counters, which fail on many benchmarks with small carving width) and no benchmarks with carving width above 30.}
\end{figure}

\subsection{Weighted Model Counting: Exact Inference}
\label{sec:tensors:experiments:cachet}
We next compare on a set of weighted model counting benchmarks from Sang, Beame, and Kautz \shortcite{SBK05}. These 1091 benchmarks are formulas whose weighted model count corresponds to exact inference on Bayesian networks. We compare against the weighted model counters \tool{cachet} \cite{SBK05}, \tool{miniC2D} \cite{OD15} and \tool{d4} \cite{LM17}. Since these benchmarks are weighted, we cannot compare against tools that can only perform unweighted model counting (\tool{dynQBF} \cite{CW16}, \tool{dynasp} \cite{FHMW17} and \tool{SharpSAT} \cite{Thurley2006}). We run each tool once on each benchmark with a timeout of 1000 seconds and record the wall-clock time taken.

We first evaluate numerical accuracy, since our approach uses 64-bit double precision floats: on all benchmarks that \tool{miniC2D} also finishes, the weighted model count returned by our approaches agrees within $10^{-3}$.

We next evaluate runtime performance. Results on these benchmarks are summarized in Figure \ref{fig:cachet-cactus}. \textbf{FT+Tamaki} is able to solve the most benchmarks of all tensor-based methods. Our implementations of \textbf{FT} each solve fewer benchmarks than \tool{cachet}, \tool{miniC2D}, and \tool{d4}. Nevertheless, \textbf{FT+*} are together able to solve 231 benchmarks faster than existing counters (\textbf{FT+Tamaki} is fastest on 50, \textbf{FT+Flow} is fastest on 175, and \textbf{FT+htd} is fastest on 6), including 62 benchmarks on which \tool{cachet}, \tool{miniC2D}, and \tool{d4} all time out. This significantly improves the virtual best solver (VBS) when \textbf{FT+*} are included. We conclude that \textbf{FT} is useful as part of a portfolio of weighted model counters.

% A more detailed analysis is available in the appendix.

\begin{figure}
	\centering
	%% Creator: Matplotlib, PGF backend
%%
%% To include the figure in your LaTeX document, write
%%   \input{<filename>.pgf}
%%
%% Make sure the required packages are loaded in your preamble
%%   \usepackage{pgf}
%%
%% and, on pdftex
%%   \usepackage[utf8]{inputenc}\DeclareUnicodeCharacter{2212}{-}
%%
%% or, on luatex and xetex
%%   \usepackage{unicode-math}
%%
%% Figures using additional raster images can only be included by \input if
%% they are in the same directory as the main LaTeX file. For loading figures
%% from other directories you can use the `import` package
%%   \usepackage{import}
%%
%% and then include the figures with
%%   \import{<path to file>}{<filename>.pgf}
%%
%% Matplotlib used the following preamble
%%   \usepackage[utf8x]{inputenc}
%%   \usepackage[T1]{fontenc}
%%
\begingroup%
\makeatletter%
\begin{pgfpicture}%
\pgfpathrectangle{\pgfpointorigin}{\pgfqpoint{6.000000in}{6.100000in}}%
\pgfusepath{use as bounding box, clip}%
\begin{pgfscope}%
\pgfsetbuttcap%
\pgfsetmiterjoin%
\definecolor{currentfill}{rgb}{1.000000,1.000000,1.000000}%
\pgfsetfillcolor{currentfill}%
\pgfsetlinewidth{0.000000pt}%
\definecolor{currentstroke}{rgb}{1.000000,1.000000,1.000000}%
\pgfsetstrokecolor{currentstroke}%
\pgfsetdash{}{0pt}%
\pgfpathmoveto{\pgfqpoint{0.000000in}{0.000000in}}%
\pgfpathlineto{\pgfqpoint{6.000000in}{0.000000in}}%
\pgfpathlineto{\pgfqpoint{6.000000in}{6.100000in}}%
\pgfpathlineto{\pgfqpoint{0.000000in}{6.100000in}}%
\pgfpathclose%
\pgfusepath{fill}%
\end{pgfscope}%
\begin{pgfscope}%
\pgfsetbuttcap%
\pgfsetmiterjoin%
\definecolor{currentfill}{rgb}{1.000000,1.000000,1.000000}%
\pgfsetfillcolor{currentfill}%
\pgfsetlinewidth{0.000000pt}%
\definecolor{currentstroke}{rgb}{0.000000,0.000000,0.000000}%
\pgfsetstrokecolor{currentstroke}%
\pgfsetstrokeopacity{0.000000}%
\pgfsetdash{}{0pt}%
\pgfpathmoveto{\pgfqpoint{0.708220in}{4.502489in}}%
\pgfpathlineto{\pgfqpoint{5.850000in}{4.502489in}}%
\pgfpathlineto{\pgfqpoint{5.850000in}{5.905275in}}%
\pgfpathlineto{\pgfqpoint{0.708220in}{5.905275in}}%
\pgfpathclose%
\pgfusepath{fill}%
\end{pgfscope}%
\begin{pgfscope}%
\pgfsetbuttcap%
\pgfsetroundjoin%
\definecolor{currentfill}{rgb}{0.000000,0.000000,0.000000}%
\pgfsetfillcolor{currentfill}%
\pgfsetlinewidth{0.803000pt}%
\definecolor{currentstroke}{rgb}{0.000000,0.000000,0.000000}%
\pgfsetstrokecolor{currentstroke}%
\pgfsetdash{}{0pt}%
\pgfsys@defobject{currentmarker}{\pgfqpoint{0.000000in}{-0.048611in}}{\pgfqpoint{0.000000in}{0.000000in}}{%
\pgfpathmoveto{\pgfqpoint{0.000000in}{0.000000in}}%
\pgfpathlineto{\pgfqpoint{0.000000in}{-0.048611in}}%
\pgfusepath{stroke,fill}%
}%
\begin{pgfscope}%
\pgfsys@transformshift{0.708220in}{4.502489in}%
\pgfsys@useobject{currentmarker}{}%
\end{pgfscope}%
\end{pgfscope}%
\begin{pgfscope}%
\definecolor{textcolor}{rgb}{0.000000,0.000000,0.000000}%
\pgfsetstrokecolor{textcolor}%
\pgfsetfillcolor{textcolor}%
\pgftext[x=0.708220in,y=4.405267in,,top]{\color{textcolor}\rmfamily\fontsize{9.000000}{10.800000}\selectfont \(\displaystyle {0}\)}%
\end{pgfscope}%
\begin{pgfscope}%
\pgfsetbuttcap%
\pgfsetroundjoin%
\definecolor{currentfill}{rgb}{0.000000,0.000000,0.000000}%
\pgfsetfillcolor{currentfill}%
\pgfsetlinewidth{0.803000pt}%
\definecolor{currentstroke}{rgb}{0.000000,0.000000,0.000000}%
\pgfsetstrokecolor{currentstroke}%
\pgfsetdash{}{0pt}%
\pgfsys@defobject{currentmarker}{\pgfqpoint{0.000000in}{-0.048611in}}{\pgfqpoint{0.000000in}{0.000000in}}{%
\pgfpathmoveto{\pgfqpoint{0.000000in}{0.000000in}}%
\pgfpathlineto{\pgfqpoint{0.000000in}{-0.048611in}}%
\pgfusepath{stroke,fill}%
}%
\begin{pgfscope}%
\pgfsys@transformshift{1.650801in}{4.502489in}%
\pgfsys@useobject{currentmarker}{}%
\end{pgfscope}%
\end{pgfscope}%
\begin{pgfscope}%
\definecolor{textcolor}{rgb}{0.000000,0.000000,0.000000}%
\pgfsetstrokecolor{textcolor}%
\pgfsetfillcolor{textcolor}%
\pgftext[x=1.650801in,y=4.405267in,,top]{\color{textcolor}\rmfamily\fontsize{9.000000}{10.800000}\selectfont \(\displaystyle {200}\)}%
\end{pgfscope}%
\begin{pgfscope}%
\pgfsetbuttcap%
\pgfsetroundjoin%
\definecolor{currentfill}{rgb}{0.000000,0.000000,0.000000}%
\pgfsetfillcolor{currentfill}%
\pgfsetlinewidth{0.803000pt}%
\definecolor{currentstroke}{rgb}{0.000000,0.000000,0.000000}%
\pgfsetstrokecolor{currentstroke}%
\pgfsetdash{}{0pt}%
\pgfsys@defobject{currentmarker}{\pgfqpoint{0.000000in}{-0.048611in}}{\pgfqpoint{0.000000in}{0.000000in}}{%
\pgfpathmoveto{\pgfqpoint{0.000000in}{0.000000in}}%
\pgfpathlineto{\pgfqpoint{0.000000in}{-0.048611in}}%
\pgfusepath{stroke,fill}%
}%
\begin{pgfscope}%
\pgfsys@transformshift{2.593382in}{4.502489in}%
\pgfsys@useobject{currentmarker}{}%
\end{pgfscope}%
\end{pgfscope}%
\begin{pgfscope}%
\definecolor{textcolor}{rgb}{0.000000,0.000000,0.000000}%
\pgfsetstrokecolor{textcolor}%
\pgfsetfillcolor{textcolor}%
\pgftext[x=2.593382in,y=4.405267in,,top]{\color{textcolor}\rmfamily\fontsize{9.000000}{10.800000}\selectfont \(\displaystyle {400}\)}%
\end{pgfscope}%
\begin{pgfscope}%
\pgfsetbuttcap%
\pgfsetroundjoin%
\definecolor{currentfill}{rgb}{0.000000,0.000000,0.000000}%
\pgfsetfillcolor{currentfill}%
\pgfsetlinewidth{0.803000pt}%
\definecolor{currentstroke}{rgb}{0.000000,0.000000,0.000000}%
\pgfsetstrokecolor{currentstroke}%
\pgfsetdash{}{0pt}%
\pgfsys@defobject{currentmarker}{\pgfqpoint{0.000000in}{-0.048611in}}{\pgfqpoint{0.000000in}{0.000000in}}{%
\pgfpathmoveto{\pgfqpoint{0.000000in}{0.000000in}}%
\pgfpathlineto{\pgfqpoint{0.000000in}{-0.048611in}}%
\pgfusepath{stroke,fill}%
}%
\begin{pgfscope}%
\pgfsys@transformshift{3.535963in}{4.502489in}%
\pgfsys@useobject{currentmarker}{}%
\end{pgfscope}%
\end{pgfscope}%
\begin{pgfscope}%
\definecolor{textcolor}{rgb}{0.000000,0.000000,0.000000}%
\pgfsetstrokecolor{textcolor}%
\pgfsetfillcolor{textcolor}%
\pgftext[x=3.535963in,y=4.405267in,,top]{\color{textcolor}\rmfamily\fontsize{9.000000}{10.800000}\selectfont \(\displaystyle {600}\)}%
\end{pgfscope}%
\begin{pgfscope}%
\pgfsetbuttcap%
\pgfsetroundjoin%
\definecolor{currentfill}{rgb}{0.000000,0.000000,0.000000}%
\pgfsetfillcolor{currentfill}%
\pgfsetlinewidth{0.803000pt}%
\definecolor{currentstroke}{rgb}{0.000000,0.000000,0.000000}%
\pgfsetstrokecolor{currentstroke}%
\pgfsetdash{}{0pt}%
\pgfsys@defobject{currentmarker}{\pgfqpoint{0.000000in}{-0.048611in}}{\pgfqpoint{0.000000in}{0.000000in}}{%
\pgfpathmoveto{\pgfqpoint{0.000000in}{0.000000in}}%
\pgfpathlineto{\pgfqpoint{0.000000in}{-0.048611in}}%
\pgfusepath{stroke,fill}%
}%
\begin{pgfscope}%
\pgfsys@transformshift{4.478544in}{4.502489in}%
\pgfsys@useobject{currentmarker}{}%
\end{pgfscope}%
\end{pgfscope}%
\begin{pgfscope}%
\definecolor{textcolor}{rgb}{0.000000,0.000000,0.000000}%
\pgfsetstrokecolor{textcolor}%
\pgfsetfillcolor{textcolor}%
\pgftext[x=4.478544in,y=4.405267in,,top]{\color{textcolor}\rmfamily\fontsize{9.000000}{10.800000}\selectfont \(\displaystyle {800}\)}%
\end{pgfscope}%
\begin{pgfscope}%
\pgfsetbuttcap%
\pgfsetroundjoin%
\definecolor{currentfill}{rgb}{0.000000,0.000000,0.000000}%
\pgfsetfillcolor{currentfill}%
\pgfsetlinewidth{0.803000pt}%
\definecolor{currentstroke}{rgb}{0.000000,0.000000,0.000000}%
\pgfsetstrokecolor{currentstroke}%
\pgfsetdash{}{0pt}%
\pgfsys@defobject{currentmarker}{\pgfqpoint{0.000000in}{-0.048611in}}{\pgfqpoint{0.000000in}{0.000000in}}{%
\pgfpathmoveto{\pgfqpoint{0.000000in}{0.000000in}}%
\pgfpathlineto{\pgfqpoint{0.000000in}{-0.048611in}}%
\pgfusepath{stroke,fill}%
}%
\begin{pgfscope}%
\pgfsys@transformshift{5.421126in}{4.502489in}%
\pgfsys@useobject{currentmarker}{}%
\end{pgfscope}%
\end{pgfscope}%
\begin{pgfscope}%
\definecolor{textcolor}{rgb}{0.000000,0.000000,0.000000}%
\pgfsetstrokecolor{textcolor}%
\pgfsetfillcolor{textcolor}%
\pgftext[x=5.421126in,y=4.405267in,,top]{\color{textcolor}\rmfamily\fontsize{9.000000}{10.800000}\selectfont \(\displaystyle {1000}\)}%
\end{pgfscope}%
\begin{pgfscope}%
\definecolor{textcolor}{rgb}{0.000000,0.000000,0.000000}%
\pgfsetstrokecolor{textcolor}%
\pgfsetfillcolor{textcolor}%
\pgftext[x=3.279110in,y=4.239322in,,top]{\color{textcolor}\rmfamily\fontsize{10.000000}{12.000000}\selectfont Number of benchmarks solved}%
\end{pgfscope}%
\begin{pgfscope}%
\pgfsetbuttcap%
\pgfsetroundjoin%
\definecolor{currentfill}{rgb}{0.000000,0.000000,0.000000}%
\pgfsetfillcolor{currentfill}%
\pgfsetlinewidth{0.803000pt}%
\definecolor{currentstroke}{rgb}{0.000000,0.000000,0.000000}%
\pgfsetstrokecolor{currentstroke}%
\pgfsetdash{}{0pt}%
\pgfsys@defobject{currentmarker}{\pgfqpoint{-0.048611in}{0.000000in}}{\pgfqpoint{-0.000000in}{0.000000in}}{%
\pgfpathmoveto{\pgfqpoint{-0.000000in}{0.000000in}}%
\pgfpathlineto{\pgfqpoint{-0.048611in}{0.000000in}}%
\pgfusepath{stroke,fill}%
}%
\begin{pgfscope}%
\pgfsys@transformshift{0.708220in}{4.502489in}%
\pgfsys@useobject{currentmarker}{}%
\end{pgfscope}%
\end{pgfscope}%
\begin{pgfscope}%
\definecolor{textcolor}{rgb}{0.000000,0.000000,0.000000}%
\pgfsetstrokecolor{textcolor}%
\pgfsetfillcolor{textcolor}%
\pgftext[x=0.344411in, y=4.457765in, left, base]{\color{textcolor}\rmfamily\fontsize{9.000000}{10.800000}\selectfont \(\displaystyle {10^{-1}}\)}%
\end{pgfscope}%
\begin{pgfscope}%
\pgfsetbuttcap%
\pgfsetroundjoin%
\definecolor{currentfill}{rgb}{0.000000,0.000000,0.000000}%
\pgfsetfillcolor{currentfill}%
\pgfsetlinewidth{0.803000pt}%
\definecolor{currentstroke}{rgb}{0.000000,0.000000,0.000000}%
\pgfsetstrokecolor{currentstroke}%
\pgfsetdash{}{0pt}%
\pgfsys@defobject{currentmarker}{\pgfqpoint{-0.048611in}{0.000000in}}{\pgfqpoint{-0.000000in}{0.000000in}}{%
\pgfpathmoveto{\pgfqpoint{-0.000000in}{0.000000in}}%
\pgfpathlineto{\pgfqpoint{-0.048611in}{0.000000in}}%
\pgfusepath{stroke,fill}%
}%
\begin{pgfscope}%
\pgfsys@transformshift{0.708220in}{4.853186in}%
\pgfsys@useobject{currentmarker}{}%
\end{pgfscope}%
\end{pgfscope}%
\begin{pgfscope}%
\definecolor{textcolor}{rgb}{0.000000,0.000000,0.000000}%
\pgfsetstrokecolor{textcolor}%
\pgfsetfillcolor{textcolor}%
\pgftext[x=0.424657in, y=4.808461in, left, base]{\color{textcolor}\rmfamily\fontsize{9.000000}{10.800000}\selectfont \(\displaystyle {10^{0}}\)}%
\end{pgfscope}%
\begin{pgfscope}%
\pgfsetbuttcap%
\pgfsetroundjoin%
\definecolor{currentfill}{rgb}{0.000000,0.000000,0.000000}%
\pgfsetfillcolor{currentfill}%
\pgfsetlinewidth{0.803000pt}%
\definecolor{currentstroke}{rgb}{0.000000,0.000000,0.000000}%
\pgfsetstrokecolor{currentstroke}%
\pgfsetdash{}{0pt}%
\pgfsys@defobject{currentmarker}{\pgfqpoint{-0.048611in}{0.000000in}}{\pgfqpoint{-0.000000in}{0.000000in}}{%
\pgfpathmoveto{\pgfqpoint{-0.000000in}{0.000000in}}%
\pgfpathlineto{\pgfqpoint{-0.048611in}{0.000000in}}%
\pgfusepath{stroke,fill}%
}%
\begin{pgfscope}%
\pgfsys@transformshift{0.708220in}{5.203882in}%
\pgfsys@useobject{currentmarker}{}%
\end{pgfscope}%
\end{pgfscope}%
\begin{pgfscope}%
\definecolor{textcolor}{rgb}{0.000000,0.000000,0.000000}%
\pgfsetstrokecolor{textcolor}%
\pgfsetfillcolor{textcolor}%
\pgftext[x=0.424657in, y=5.159157in, left, base]{\color{textcolor}\rmfamily\fontsize{9.000000}{10.800000}\selectfont \(\displaystyle {10^{1}}\)}%
\end{pgfscope}%
\begin{pgfscope}%
\pgfsetbuttcap%
\pgfsetroundjoin%
\definecolor{currentfill}{rgb}{0.000000,0.000000,0.000000}%
\pgfsetfillcolor{currentfill}%
\pgfsetlinewidth{0.803000pt}%
\definecolor{currentstroke}{rgb}{0.000000,0.000000,0.000000}%
\pgfsetstrokecolor{currentstroke}%
\pgfsetdash{}{0pt}%
\pgfsys@defobject{currentmarker}{\pgfqpoint{-0.048611in}{0.000000in}}{\pgfqpoint{-0.000000in}{0.000000in}}{%
\pgfpathmoveto{\pgfqpoint{-0.000000in}{0.000000in}}%
\pgfpathlineto{\pgfqpoint{-0.048611in}{0.000000in}}%
\pgfusepath{stroke,fill}%
}%
\begin{pgfscope}%
\pgfsys@transformshift{0.708220in}{5.554579in}%
\pgfsys@useobject{currentmarker}{}%
\end{pgfscope}%
\end{pgfscope}%
\begin{pgfscope}%
\definecolor{textcolor}{rgb}{0.000000,0.000000,0.000000}%
\pgfsetstrokecolor{textcolor}%
\pgfsetfillcolor{textcolor}%
\pgftext[x=0.424657in, y=5.509854in, left, base]{\color{textcolor}\rmfamily\fontsize{9.000000}{10.800000}\selectfont \(\displaystyle {10^{2}}\)}%
\end{pgfscope}%
\begin{pgfscope}%
\pgfsetbuttcap%
\pgfsetroundjoin%
\definecolor{currentfill}{rgb}{0.000000,0.000000,0.000000}%
\pgfsetfillcolor{currentfill}%
\pgfsetlinewidth{0.803000pt}%
\definecolor{currentstroke}{rgb}{0.000000,0.000000,0.000000}%
\pgfsetstrokecolor{currentstroke}%
\pgfsetdash{}{0pt}%
\pgfsys@defobject{currentmarker}{\pgfqpoint{-0.048611in}{0.000000in}}{\pgfqpoint{-0.000000in}{0.000000in}}{%
\pgfpathmoveto{\pgfqpoint{-0.000000in}{0.000000in}}%
\pgfpathlineto{\pgfqpoint{-0.048611in}{0.000000in}}%
\pgfusepath{stroke,fill}%
}%
\begin{pgfscope}%
\pgfsys@transformshift{0.708220in}{5.905275in}%
\pgfsys@useobject{currentmarker}{}%
\end{pgfscope}%
\end{pgfscope}%
\begin{pgfscope}%
\definecolor{textcolor}{rgb}{0.000000,0.000000,0.000000}%
\pgfsetstrokecolor{textcolor}%
\pgfsetfillcolor{textcolor}%
\pgftext[x=0.424657in, y=5.860550in, left, base]{\color{textcolor}\rmfamily\fontsize{9.000000}{10.800000}\selectfont \(\displaystyle {10^{3}}\)}%
\end{pgfscope}%
\begin{pgfscope}%
\pgfsetbuttcap%
\pgfsetroundjoin%
\definecolor{currentfill}{rgb}{0.000000,0.000000,0.000000}%
\pgfsetfillcolor{currentfill}%
\pgfsetlinewidth{0.602250pt}%
\definecolor{currentstroke}{rgb}{0.000000,0.000000,0.000000}%
\pgfsetstrokecolor{currentstroke}%
\pgfsetdash{}{0pt}%
\pgfsys@defobject{currentmarker}{\pgfqpoint{-0.027778in}{0.000000in}}{\pgfqpoint{-0.000000in}{0.000000in}}{%
\pgfpathmoveto{\pgfqpoint{-0.000000in}{0.000000in}}%
\pgfpathlineto{\pgfqpoint{-0.027778in}{0.000000in}}%
\pgfusepath{stroke,fill}%
}%
\begin{pgfscope}%
\pgfsys@transformshift{0.708220in}{4.608059in}%
\pgfsys@useobject{currentmarker}{}%
\end{pgfscope}%
\end{pgfscope}%
\begin{pgfscope}%
\pgfsetbuttcap%
\pgfsetroundjoin%
\definecolor{currentfill}{rgb}{0.000000,0.000000,0.000000}%
\pgfsetfillcolor{currentfill}%
\pgfsetlinewidth{0.602250pt}%
\definecolor{currentstroke}{rgb}{0.000000,0.000000,0.000000}%
\pgfsetstrokecolor{currentstroke}%
\pgfsetdash{}{0pt}%
\pgfsys@defobject{currentmarker}{\pgfqpoint{-0.027778in}{0.000000in}}{\pgfqpoint{-0.000000in}{0.000000in}}{%
\pgfpathmoveto{\pgfqpoint{-0.000000in}{0.000000in}}%
\pgfpathlineto{\pgfqpoint{-0.027778in}{0.000000in}}%
\pgfusepath{stroke,fill}%
}%
\begin{pgfscope}%
\pgfsys@transformshift{0.708220in}{4.669814in}%
\pgfsys@useobject{currentmarker}{}%
\end{pgfscope}%
\end{pgfscope}%
\begin{pgfscope}%
\pgfsetbuttcap%
\pgfsetroundjoin%
\definecolor{currentfill}{rgb}{0.000000,0.000000,0.000000}%
\pgfsetfillcolor{currentfill}%
\pgfsetlinewidth{0.602250pt}%
\definecolor{currentstroke}{rgb}{0.000000,0.000000,0.000000}%
\pgfsetstrokecolor{currentstroke}%
\pgfsetdash{}{0pt}%
\pgfsys@defobject{currentmarker}{\pgfqpoint{-0.027778in}{0.000000in}}{\pgfqpoint{-0.000000in}{0.000000in}}{%
\pgfpathmoveto{\pgfqpoint{-0.000000in}{0.000000in}}%
\pgfpathlineto{\pgfqpoint{-0.027778in}{0.000000in}}%
\pgfusepath{stroke,fill}%
}%
\begin{pgfscope}%
\pgfsys@transformshift{0.708220in}{4.713630in}%
\pgfsys@useobject{currentmarker}{}%
\end{pgfscope}%
\end{pgfscope}%
\begin{pgfscope}%
\pgfsetbuttcap%
\pgfsetroundjoin%
\definecolor{currentfill}{rgb}{0.000000,0.000000,0.000000}%
\pgfsetfillcolor{currentfill}%
\pgfsetlinewidth{0.602250pt}%
\definecolor{currentstroke}{rgb}{0.000000,0.000000,0.000000}%
\pgfsetstrokecolor{currentstroke}%
\pgfsetdash{}{0pt}%
\pgfsys@defobject{currentmarker}{\pgfqpoint{-0.027778in}{0.000000in}}{\pgfqpoint{-0.000000in}{0.000000in}}{%
\pgfpathmoveto{\pgfqpoint{-0.000000in}{0.000000in}}%
\pgfpathlineto{\pgfqpoint{-0.027778in}{0.000000in}}%
\pgfusepath{stroke,fill}%
}%
\begin{pgfscope}%
\pgfsys@transformshift{0.708220in}{4.747616in}%
\pgfsys@useobject{currentmarker}{}%
\end{pgfscope}%
\end{pgfscope}%
\begin{pgfscope}%
\pgfsetbuttcap%
\pgfsetroundjoin%
\definecolor{currentfill}{rgb}{0.000000,0.000000,0.000000}%
\pgfsetfillcolor{currentfill}%
\pgfsetlinewidth{0.602250pt}%
\definecolor{currentstroke}{rgb}{0.000000,0.000000,0.000000}%
\pgfsetstrokecolor{currentstroke}%
\pgfsetdash{}{0pt}%
\pgfsys@defobject{currentmarker}{\pgfqpoint{-0.027778in}{0.000000in}}{\pgfqpoint{-0.000000in}{0.000000in}}{%
\pgfpathmoveto{\pgfqpoint{-0.000000in}{0.000000in}}%
\pgfpathlineto{\pgfqpoint{-0.027778in}{0.000000in}}%
\pgfusepath{stroke,fill}%
}%
\begin{pgfscope}%
\pgfsys@transformshift{0.708220in}{4.775384in}%
\pgfsys@useobject{currentmarker}{}%
\end{pgfscope}%
\end{pgfscope}%
\begin{pgfscope}%
\pgfsetbuttcap%
\pgfsetroundjoin%
\definecolor{currentfill}{rgb}{0.000000,0.000000,0.000000}%
\pgfsetfillcolor{currentfill}%
\pgfsetlinewidth{0.602250pt}%
\definecolor{currentstroke}{rgb}{0.000000,0.000000,0.000000}%
\pgfsetstrokecolor{currentstroke}%
\pgfsetdash{}{0pt}%
\pgfsys@defobject{currentmarker}{\pgfqpoint{-0.027778in}{0.000000in}}{\pgfqpoint{-0.000000in}{0.000000in}}{%
\pgfpathmoveto{\pgfqpoint{-0.000000in}{0.000000in}}%
\pgfpathlineto{\pgfqpoint{-0.027778in}{0.000000in}}%
\pgfusepath{stroke,fill}%
}%
\begin{pgfscope}%
\pgfsys@transformshift{0.708220in}{4.798862in}%
\pgfsys@useobject{currentmarker}{}%
\end{pgfscope}%
\end{pgfscope}%
\begin{pgfscope}%
\pgfsetbuttcap%
\pgfsetroundjoin%
\definecolor{currentfill}{rgb}{0.000000,0.000000,0.000000}%
\pgfsetfillcolor{currentfill}%
\pgfsetlinewidth{0.602250pt}%
\definecolor{currentstroke}{rgb}{0.000000,0.000000,0.000000}%
\pgfsetstrokecolor{currentstroke}%
\pgfsetdash{}{0pt}%
\pgfsys@defobject{currentmarker}{\pgfqpoint{-0.027778in}{0.000000in}}{\pgfqpoint{-0.000000in}{0.000000in}}{%
\pgfpathmoveto{\pgfqpoint{-0.000000in}{0.000000in}}%
\pgfpathlineto{\pgfqpoint{-0.027778in}{0.000000in}}%
\pgfusepath{stroke,fill}%
}%
\begin{pgfscope}%
\pgfsys@transformshift{0.708220in}{4.819200in}%
\pgfsys@useobject{currentmarker}{}%
\end{pgfscope}%
\end{pgfscope}%
\begin{pgfscope}%
\pgfsetbuttcap%
\pgfsetroundjoin%
\definecolor{currentfill}{rgb}{0.000000,0.000000,0.000000}%
\pgfsetfillcolor{currentfill}%
\pgfsetlinewidth{0.602250pt}%
\definecolor{currentstroke}{rgb}{0.000000,0.000000,0.000000}%
\pgfsetstrokecolor{currentstroke}%
\pgfsetdash{}{0pt}%
\pgfsys@defobject{currentmarker}{\pgfqpoint{-0.027778in}{0.000000in}}{\pgfqpoint{-0.000000in}{0.000000in}}{%
\pgfpathmoveto{\pgfqpoint{-0.000000in}{0.000000in}}%
\pgfpathlineto{\pgfqpoint{-0.027778in}{0.000000in}}%
\pgfusepath{stroke,fill}%
}%
\begin{pgfscope}%
\pgfsys@transformshift{0.708220in}{4.837139in}%
\pgfsys@useobject{currentmarker}{}%
\end{pgfscope}%
\end{pgfscope}%
\begin{pgfscope}%
\pgfsetbuttcap%
\pgfsetroundjoin%
\definecolor{currentfill}{rgb}{0.000000,0.000000,0.000000}%
\pgfsetfillcolor{currentfill}%
\pgfsetlinewidth{0.602250pt}%
\definecolor{currentstroke}{rgb}{0.000000,0.000000,0.000000}%
\pgfsetstrokecolor{currentstroke}%
\pgfsetdash{}{0pt}%
\pgfsys@defobject{currentmarker}{\pgfqpoint{-0.027778in}{0.000000in}}{\pgfqpoint{-0.000000in}{0.000000in}}{%
\pgfpathmoveto{\pgfqpoint{-0.000000in}{0.000000in}}%
\pgfpathlineto{\pgfqpoint{-0.027778in}{0.000000in}}%
\pgfusepath{stroke,fill}%
}%
\begin{pgfscope}%
\pgfsys@transformshift{0.708220in}{4.958756in}%
\pgfsys@useobject{currentmarker}{}%
\end{pgfscope}%
\end{pgfscope}%
\begin{pgfscope}%
\pgfsetbuttcap%
\pgfsetroundjoin%
\definecolor{currentfill}{rgb}{0.000000,0.000000,0.000000}%
\pgfsetfillcolor{currentfill}%
\pgfsetlinewidth{0.602250pt}%
\definecolor{currentstroke}{rgb}{0.000000,0.000000,0.000000}%
\pgfsetstrokecolor{currentstroke}%
\pgfsetdash{}{0pt}%
\pgfsys@defobject{currentmarker}{\pgfqpoint{-0.027778in}{0.000000in}}{\pgfqpoint{-0.000000in}{0.000000in}}{%
\pgfpathmoveto{\pgfqpoint{-0.000000in}{0.000000in}}%
\pgfpathlineto{\pgfqpoint{-0.027778in}{0.000000in}}%
\pgfusepath{stroke,fill}%
}%
\begin{pgfscope}%
\pgfsys@transformshift{0.708220in}{5.020511in}%
\pgfsys@useobject{currentmarker}{}%
\end{pgfscope}%
\end{pgfscope}%
\begin{pgfscope}%
\pgfsetbuttcap%
\pgfsetroundjoin%
\definecolor{currentfill}{rgb}{0.000000,0.000000,0.000000}%
\pgfsetfillcolor{currentfill}%
\pgfsetlinewidth{0.602250pt}%
\definecolor{currentstroke}{rgb}{0.000000,0.000000,0.000000}%
\pgfsetstrokecolor{currentstroke}%
\pgfsetdash{}{0pt}%
\pgfsys@defobject{currentmarker}{\pgfqpoint{-0.027778in}{0.000000in}}{\pgfqpoint{-0.000000in}{0.000000in}}{%
\pgfpathmoveto{\pgfqpoint{-0.000000in}{0.000000in}}%
\pgfpathlineto{\pgfqpoint{-0.027778in}{0.000000in}}%
\pgfusepath{stroke,fill}%
}%
\begin{pgfscope}%
\pgfsys@transformshift{0.708220in}{5.064326in}%
\pgfsys@useobject{currentmarker}{}%
\end{pgfscope}%
\end{pgfscope}%
\begin{pgfscope}%
\pgfsetbuttcap%
\pgfsetroundjoin%
\definecolor{currentfill}{rgb}{0.000000,0.000000,0.000000}%
\pgfsetfillcolor{currentfill}%
\pgfsetlinewidth{0.602250pt}%
\definecolor{currentstroke}{rgb}{0.000000,0.000000,0.000000}%
\pgfsetstrokecolor{currentstroke}%
\pgfsetdash{}{0pt}%
\pgfsys@defobject{currentmarker}{\pgfqpoint{-0.027778in}{0.000000in}}{\pgfqpoint{-0.000000in}{0.000000in}}{%
\pgfpathmoveto{\pgfqpoint{-0.000000in}{0.000000in}}%
\pgfpathlineto{\pgfqpoint{-0.027778in}{0.000000in}}%
\pgfusepath{stroke,fill}%
}%
\begin{pgfscope}%
\pgfsys@transformshift{0.708220in}{5.098312in}%
\pgfsys@useobject{currentmarker}{}%
\end{pgfscope}%
\end{pgfscope}%
\begin{pgfscope}%
\pgfsetbuttcap%
\pgfsetroundjoin%
\definecolor{currentfill}{rgb}{0.000000,0.000000,0.000000}%
\pgfsetfillcolor{currentfill}%
\pgfsetlinewidth{0.602250pt}%
\definecolor{currentstroke}{rgb}{0.000000,0.000000,0.000000}%
\pgfsetstrokecolor{currentstroke}%
\pgfsetdash{}{0pt}%
\pgfsys@defobject{currentmarker}{\pgfqpoint{-0.027778in}{0.000000in}}{\pgfqpoint{-0.000000in}{0.000000in}}{%
\pgfpathmoveto{\pgfqpoint{-0.000000in}{0.000000in}}%
\pgfpathlineto{\pgfqpoint{-0.027778in}{0.000000in}}%
\pgfusepath{stroke,fill}%
}%
\begin{pgfscope}%
\pgfsys@transformshift{0.708220in}{5.126081in}%
\pgfsys@useobject{currentmarker}{}%
\end{pgfscope}%
\end{pgfscope}%
\begin{pgfscope}%
\pgfsetbuttcap%
\pgfsetroundjoin%
\definecolor{currentfill}{rgb}{0.000000,0.000000,0.000000}%
\pgfsetfillcolor{currentfill}%
\pgfsetlinewidth{0.602250pt}%
\definecolor{currentstroke}{rgb}{0.000000,0.000000,0.000000}%
\pgfsetstrokecolor{currentstroke}%
\pgfsetdash{}{0pt}%
\pgfsys@defobject{currentmarker}{\pgfqpoint{-0.027778in}{0.000000in}}{\pgfqpoint{-0.000000in}{0.000000in}}{%
\pgfpathmoveto{\pgfqpoint{-0.000000in}{0.000000in}}%
\pgfpathlineto{\pgfqpoint{-0.027778in}{0.000000in}}%
\pgfusepath{stroke,fill}%
}%
\begin{pgfscope}%
\pgfsys@transformshift{0.708220in}{5.149559in}%
\pgfsys@useobject{currentmarker}{}%
\end{pgfscope}%
\end{pgfscope}%
\begin{pgfscope}%
\pgfsetbuttcap%
\pgfsetroundjoin%
\definecolor{currentfill}{rgb}{0.000000,0.000000,0.000000}%
\pgfsetfillcolor{currentfill}%
\pgfsetlinewidth{0.602250pt}%
\definecolor{currentstroke}{rgb}{0.000000,0.000000,0.000000}%
\pgfsetstrokecolor{currentstroke}%
\pgfsetdash{}{0pt}%
\pgfsys@defobject{currentmarker}{\pgfqpoint{-0.027778in}{0.000000in}}{\pgfqpoint{-0.000000in}{0.000000in}}{%
\pgfpathmoveto{\pgfqpoint{-0.000000in}{0.000000in}}%
\pgfpathlineto{\pgfqpoint{-0.027778in}{0.000000in}}%
\pgfusepath{stroke,fill}%
}%
\begin{pgfscope}%
\pgfsys@transformshift{0.708220in}{5.169896in}%
\pgfsys@useobject{currentmarker}{}%
\end{pgfscope}%
\end{pgfscope}%
\begin{pgfscope}%
\pgfsetbuttcap%
\pgfsetroundjoin%
\definecolor{currentfill}{rgb}{0.000000,0.000000,0.000000}%
\pgfsetfillcolor{currentfill}%
\pgfsetlinewidth{0.602250pt}%
\definecolor{currentstroke}{rgb}{0.000000,0.000000,0.000000}%
\pgfsetstrokecolor{currentstroke}%
\pgfsetdash{}{0pt}%
\pgfsys@defobject{currentmarker}{\pgfqpoint{-0.027778in}{0.000000in}}{\pgfqpoint{-0.000000in}{0.000000in}}{%
\pgfpathmoveto{\pgfqpoint{-0.000000in}{0.000000in}}%
\pgfpathlineto{\pgfqpoint{-0.027778in}{0.000000in}}%
\pgfusepath{stroke,fill}%
}%
\begin{pgfscope}%
\pgfsys@transformshift{0.708220in}{5.187835in}%
\pgfsys@useobject{currentmarker}{}%
\end{pgfscope}%
\end{pgfscope}%
\begin{pgfscope}%
\pgfsetbuttcap%
\pgfsetroundjoin%
\definecolor{currentfill}{rgb}{0.000000,0.000000,0.000000}%
\pgfsetfillcolor{currentfill}%
\pgfsetlinewidth{0.602250pt}%
\definecolor{currentstroke}{rgb}{0.000000,0.000000,0.000000}%
\pgfsetstrokecolor{currentstroke}%
\pgfsetdash{}{0pt}%
\pgfsys@defobject{currentmarker}{\pgfqpoint{-0.027778in}{0.000000in}}{\pgfqpoint{-0.000000in}{0.000000in}}{%
\pgfpathmoveto{\pgfqpoint{-0.000000in}{0.000000in}}%
\pgfpathlineto{\pgfqpoint{-0.027778in}{0.000000in}}%
\pgfusepath{stroke,fill}%
}%
\begin{pgfscope}%
\pgfsys@transformshift{0.708220in}{5.309452in}%
\pgfsys@useobject{currentmarker}{}%
\end{pgfscope}%
\end{pgfscope}%
\begin{pgfscope}%
\pgfsetbuttcap%
\pgfsetroundjoin%
\definecolor{currentfill}{rgb}{0.000000,0.000000,0.000000}%
\pgfsetfillcolor{currentfill}%
\pgfsetlinewidth{0.602250pt}%
\definecolor{currentstroke}{rgb}{0.000000,0.000000,0.000000}%
\pgfsetstrokecolor{currentstroke}%
\pgfsetdash{}{0pt}%
\pgfsys@defobject{currentmarker}{\pgfqpoint{-0.027778in}{0.000000in}}{\pgfqpoint{-0.000000in}{0.000000in}}{%
\pgfpathmoveto{\pgfqpoint{-0.000000in}{0.000000in}}%
\pgfpathlineto{\pgfqpoint{-0.027778in}{0.000000in}}%
\pgfusepath{stroke,fill}%
}%
\begin{pgfscope}%
\pgfsys@transformshift{0.708220in}{5.371207in}%
\pgfsys@useobject{currentmarker}{}%
\end{pgfscope}%
\end{pgfscope}%
\begin{pgfscope}%
\pgfsetbuttcap%
\pgfsetroundjoin%
\definecolor{currentfill}{rgb}{0.000000,0.000000,0.000000}%
\pgfsetfillcolor{currentfill}%
\pgfsetlinewidth{0.602250pt}%
\definecolor{currentstroke}{rgb}{0.000000,0.000000,0.000000}%
\pgfsetstrokecolor{currentstroke}%
\pgfsetdash{}{0pt}%
\pgfsys@defobject{currentmarker}{\pgfqpoint{-0.027778in}{0.000000in}}{\pgfqpoint{-0.000000in}{0.000000in}}{%
\pgfpathmoveto{\pgfqpoint{-0.000000in}{0.000000in}}%
\pgfpathlineto{\pgfqpoint{-0.027778in}{0.000000in}}%
\pgfusepath{stroke,fill}%
}%
\begin{pgfscope}%
\pgfsys@transformshift{0.708220in}{5.415023in}%
\pgfsys@useobject{currentmarker}{}%
\end{pgfscope}%
\end{pgfscope}%
\begin{pgfscope}%
\pgfsetbuttcap%
\pgfsetroundjoin%
\definecolor{currentfill}{rgb}{0.000000,0.000000,0.000000}%
\pgfsetfillcolor{currentfill}%
\pgfsetlinewidth{0.602250pt}%
\definecolor{currentstroke}{rgb}{0.000000,0.000000,0.000000}%
\pgfsetstrokecolor{currentstroke}%
\pgfsetdash{}{0pt}%
\pgfsys@defobject{currentmarker}{\pgfqpoint{-0.027778in}{0.000000in}}{\pgfqpoint{-0.000000in}{0.000000in}}{%
\pgfpathmoveto{\pgfqpoint{-0.000000in}{0.000000in}}%
\pgfpathlineto{\pgfqpoint{-0.027778in}{0.000000in}}%
\pgfusepath{stroke,fill}%
}%
\begin{pgfscope}%
\pgfsys@transformshift{0.708220in}{5.449009in}%
\pgfsys@useobject{currentmarker}{}%
\end{pgfscope}%
\end{pgfscope}%
\begin{pgfscope}%
\pgfsetbuttcap%
\pgfsetroundjoin%
\definecolor{currentfill}{rgb}{0.000000,0.000000,0.000000}%
\pgfsetfillcolor{currentfill}%
\pgfsetlinewidth{0.602250pt}%
\definecolor{currentstroke}{rgb}{0.000000,0.000000,0.000000}%
\pgfsetstrokecolor{currentstroke}%
\pgfsetdash{}{0pt}%
\pgfsys@defobject{currentmarker}{\pgfqpoint{-0.027778in}{0.000000in}}{\pgfqpoint{-0.000000in}{0.000000in}}{%
\pgfpathmoveto{\pgfqpoint{-0.000000in}{0.000000in}}%
\pgfpathlineto{\pgfqpoint{-0.027778in}{0.000000in}}%
\pgfusepath{stroke,fill}%
}%
\begin{pgfscope}%
\pgfsys@transformshift{0.708220in}{5.476777in}%
\pgfsys@useobject{currentmarker}{}%
\end{pgfscope}%
\end{pgfscope}%
\begin{pgfscope}%
\pgfsetbuttcap%
\pgfsetroundjoin%
\definecolor{currentfill}{rgb}{0.000000,0.000000,0.000000}%
\pgfsetfillcolor{currentfill}%
\pgfsetlinewidth{0.602250pt}%
\definecolor{currentstroke}{rgb}{0.000000,0.000000,0.000000}%
\pgfsetstrokecolor{currentstroke}%
\pgfsetdash{}{0pt}%
\pgfsys@defobject{currentmarker}{\pgfqpoint{-0.027778in}{0.000000in}}{\pgfqpoint{-0.000000in}{0.000000in}}{%
\pgfpathmoveto{\pgfqpoint{-0.000000in}{0.000000in}}%
\pgfpathlineto{\pgfqpoint{-0.027778in}{0.000000in}}%
\pgfusepath{stroke,fill}%
}%
\begin{pgfscope}%
\pgfsys@transformshift{0.708220in}{5.500255in}%
\pgfsys@useobject{currentmarker}{}%
\end{pgfscope}%
\end{pgfscope}%
\begin{pgfscope}%
\pgfsetbuttcap%
\pgfsetroundjoin%
\definecolor{currentfill}{rgb}{0.000000,0.000000,0.000000}%
\pgfsetfillcolor{currentfill}%
\pgfsetlinewidth{0.602250pt}%
\definecolor{currentstroke}{rgb}{0.000000,0.000000,0.000000}%
\pgfsetstrokecolor{currentstroke}%
\pgfsetdash{}{0pt}%
\pgfsys@defobject{currentmarker}{\pgfqpoint{-0.027778in}{0.000000in}}{\pgfqpoint{-0.000000in}{0.000000in}}{%
\pgfpathmoveto{\pgfqpoint{-0.000000in}{0.000000in}}%
\pgfpathlineto{\pgfqpoint{-0.027778in}{0.000000in}}%
\pgfusepath{stroke,fill}%
}%
\begin{pgfscope}%
\pgfsys@transformshift{0.708220in}{5.520593in}%
\pgfsys@useobject{currentmarker}{}%
\end{pgfscope}%
\end{pgfscope}%
\begin{pgfscope}%
\pgfsetbuttcap%
\pgfsetroundjoin%
\definecolor{currentfill}{rgb}{0.000000,0.000000,0.000000}%
\pgfsetfillcolor{currentfill}%
\pgfsetlinewidth{0.602250pt}%
\definecolor{currentstroke}{rgb}{0.000000,0.000000,0.000000}%
\pgfsetstrokecolor{currentstroke}%
\pgfsetdash{}{0pt}%
\pgfsys@defobject{currentmarker}{\pgfqpoint{-0.027778in}{0.000000in}}{\pgfqpoint{-0.000000in}{0.000000in}}{%
\pgfpathmoveto{\pgfqpoint{-0.000000in}{0.000000in}}%
\pgfpathlineto{\pgfqpoint{-0.027778in}{0.000000in}}%
\pgfusepath{stroke,fill}%
}%
\begin{pgfscope}%
\pgfsys@transformshift{0.708220in}{5.538532in}%
\pgfsys@useobject{currentmarker}{}%
\end{pgfscope}%
\end{pgfscope}%
\begin{pgfscope}%
\pgfsetbuttcap%
\pgfsetroundjoin%
\definecolor{currentfill}{rgb}{0.000000,0.000000,0.000000}%
\pgfsetfillcolor{currentfill}%
\pgfsetlinewidth{0.602250pt}%
\definecolor{currentstroke}{rgb}{0.000000,0.000000,0.000000}%
\pgfsetstrokecolor{currentstroke}%
\pgfsetdash{}{0pt}%
\pgfsys@defobject{currentmarker}{\pgfqpoint{-0.027778in}{0.000000in}}{\pgfqpoint{-0.000000in}{0.000000in}}{%
\pgfpathmoveto{\pgfqpoint{-0.000000in}{0.000000in}}%
\pgfpathlineto{\pgfqpoint{-0.027778in}{0.000000in}}%
\pgfusepath{stroke,fill}%
}%
\begin{pgfscope}%
\pgfsys@transformshift{0.708220in}{5.660149in}%
\pgfsys@useobject{currentmarker}{}%
\end{pgfscope}%
\end{pgfscope}%
\begin{pgfscope}%
\pgfsetbuttcap%
\pgfsetroundjoin%
\definecolor{currentfill}{rgb}{0.000000,0.000000,0.000000}%
\pgfsetfillcolor{currentfill}%
\pgfsetlinewidth{0.602250pt}%
\definecolor{currentstroke}{rgb}{0.000000,0.000000,0.000000}%
\pgfsetstrokecolor{currentstroke}%
\pgfsetdash{}{0pt}%
\pgfsys@defobject{currentmarker}{\pgfqpoint{-0.027778in}{0.000000in}}{\pgfqpoint{-0.000000in}{0.000000in}}{%
\pgfpathmoveto{\pgfqpoint{-0.000000in}{0.000000in}}%
\pgfpathlineto{\pgfqpoint{-0.027778in}{0.000000in}}%
\pgfusepath{stroke,fill}%
}%
\begin{pgfscope}%
\pgfsys@transformshift{0.708220in}{5.721903in}%
\pgfsys@useobject{currentmarker}{}%
\end{pgfscope}%
\end{pgfscope}%
\begin{pgfscope}%
\pgfsetbuttcap%
\pgfsetroundjoin%
\definecolor{currentfill}{rgb}{0.000000,0.000000,0.000000}%
\pgfsetfillcolor{currentfill}%
\pgfsetlinewidth{0.602250pt}%
\definecolor{currentstroke}{rgb}{0.000000,0.000000,0.000000}%
\pgfsetstrokecolor{currentstroke}%
\pgfsetdash{}{0pt}%
\pgfsys@defobject{currentmarker}{\pgfqpoint{-0.027778in}{0.000000in}}{\pgfqpoint{-0.000000in}{0.000000in}}{%
\pgfpathmoveto{\pgfqpoint{-0.000000in}{0.000000in}}%
\pgfpathlineto{\pgfqpoint{-0.027778in}{0.000000in}}%
\pgfusepath{stroke,fill}%
}%
\begin{pgfscope}%
\pgfsys@transformshift{0.708220in}{5.765719in}%
\pgfsys@useobject{currentmarker}{}%
\end{pgfscope}%
\end{pgfscope}%
\begin{pgfscope}%
\pgfsetbuttcap%
\pgfsetroundjoin%
\definecolor{currentfill}{rgb}{0.000000,0.000000,0.000000}%
\pgfsetfillcolor{currentfill}%
\pgfsetlinewidth{0.602250pt}%
\definecolor{currentstroke}{rgb}{0.000000,0.000000,0.000000}%
\pgfsetstrokecolor{currentstroke}%
\pgfsetdash{}{0pt}%
\pgfsys@defobject{currentmarker}{\pgfqpoint{-0.027778in}{0.000000in}}{\pgfqpoint{-0.000000in}{0.000000in}}{%
\pgfpathmoveto{\pgfqpoint{-0.000000in}{0.000000in}}%
\pgfpathlineto{\pgfqpoint{-0.027778in}{0.000000in}}%
\pgfusepath{stroke,fill}%
}%
\begin{pgfscope}%
\pgfsys@transformshift{0.708220in}{5.799705in}%
\pgfsys@useobject{currentmarker}{}%
\end{pgfscope}%
\end{pgfscope}%
\begin{pgfscope}%
\pgfsetbuttcap%
\pgfsetroundjoin%
\definecolor{currentfill}{rgb}{0.000000,0.000000,0.000000}%
\pgfsetfillcolor{currentfill}%
\pgfsetlinewidth{0.602250pt}%
\definecolor{currentstroke}{rgb}{0.000000,0.000000,0.000000}%
\pgfsetstrokecolor{currentstroke}%
\pgfsetdash{}{0pt}%
\pgfsys@defobject{currentmarker}{\pgfqpoint{-0.027778in}{0.000000in}}{\pgfqpoint{-0.000000in}{0.000000in}}{%
\pgfpathmoveto{\pgfqpoint{-0.000000in}{0.000000in}}%
\pgfpathlineto{\pgfqpoint{-0.027778in}{0.000000in}}%
\pgfusepath{stroke,fill}%
}%
\begin{pgfscope}%
\pgfsys@transformshift{0.708220in}{5.827474in}%
\pgfsys@useobject{currentmarker}{}%
\end{pgfscope}%
\end{pgfscope}%
\begin{pgfscope}%
\pgfsetbuttcap%
\pgfsetroundjoin%
\definecolor{currentfill}{rgb}{0.000000,0.000000,0.000000}%
\pgfsetfillcolor{currentfill}%
\pgfsetlinewidth{0.602250pt}%
\definecolor{currentstroke}{rgb}{0.000000,0.000000,0.000000}%
\pgfsetstrokecolor{currentstroke}%
\pgfsetdash{}{0pt}%
\pgfsys@defobject{currentmarker}{\pgfqpoint{-0.027778in}{0.000000in}}{\pgfqpoint{-0.000000in}{0.000000in}}{%
\pgfpathmoveto{\pgfqpoint{-0.000000in}{0.000000in}}%
\pgfpathlineto{\pgfqpoint{-0.027778in}{0.000000in}}%
\pgfusepath{stroke,fill}%
}%
\begin{pgfscope}%
\pgfsys@transformshift{0.708220in}{5.850952in}%
\pgfsys@useobject{currentmarker}{}%
\end{pgfscope}%
\end{pgfscope}%
\begin{pgfscope}%
\pgfsetbuttcap%
\pgfsetroundjoin%
\definecolor{currentfill}{rgb}{0.000000,0.000000,0.000000}%
\pgfsetfillcolor{currentfill}%
\pgfsetlinewidth{0.602250pt}%
\definecolor{currentstroke}{rgb}{0.000000,0.000000,0.000000}%
\pgfsetstrokecolor{currentstroke}%
\pgfsetdash{}{0pt}%
\pgfsys@defobject{currentmarker}{\pgfqpoint{-0.027778in}{0.000000in}}{\pgfqpoint{-0.000000in}{0.000000in}}{%
\pgfpathmoveto{\pgfqpoint{-0.000000in}{0.000000in}}%
\pgfpathlineto{\pgfqpoint{-0.027778in}{0.000000in}}%
\pgfusepath{stroke,fill}%
}%
\begin{pgfscope}%
\pgfsys@transformshift{0.708220in}{5.871289in}%
\pgfsys@useobject{currentmarker}{}%
\end{pgfscope}%
\end{pgfscope}%
\begin{pgfscope}%
\pgfsetbuttcap%
\pgfsetroundjoin%
\definecolor{currentfill}{rgb}{0.000000,0.000000,0.000000}%
\pgfsetfillcolor{currentfill}%
\pgfsetlinewidth{0.602250pt}%
\definecolor{currentstroke}{rgb}{0.000000,0.000000,0.000000}%
\pgfsetstrokecolor{currentstroke}%
\pgfsetdash{}{0pt}%
\pgfsys@defobject{currentmarker}{\pgfqpoint{-0.027778in}{0.000000in}}{\pgfqpoint{-0.000000in}{0.000000in}}{%
\pgfpathmoveto{\pgfqpoint{-0.000000in}{0.000000in}}%
\pgfpathlineto{\pgfqpoint{-0.027778in}{0.000000in}}%
\pgfusepath{stroke,fill}%
}%
\begin{pgfscope}%
\pgfsys@transformshift{0.708220in}{5.889228in}%
\pgfsys@useobject{currentmarker}{}%
\end{pgfscope}%
\end{pgfscope}%
\begin{pgfscope}%
\definecolor{textcolor}{rgb}{0.000000,0.000000,0.000000}%
\pgfsetstrokecolor{textcolor}%
\pgfsetfillcolor{textcolor}%
\pgftext[x=0.288855in,y=5.203882in,,bottom,rotate=90.000000]{\color{textcolor}\rmfamily\fontsize{10.000000}{12.000000}\selectfont Longest solving time (s)}%
\end{pgfscope}%
\begin{pgfscope}%
\pgfpathrectangle{\pgfqpoint{0.708220in}{4.502489in}}{\pgfqpoint{5.141780in}{1.402786in}}%
\pgfusepath{clip}%
\pgfsetbuttcap%
\pgfsetroundjoin%
\pgfsetlinewidth{2.007500pt}%
\definecolor{currentstroke}{rgb}{1.000000,0.843137,0.000000}%
\pgfsetstrokecolor{currentstroke}%
\pgfsetdash{{7.400000pt}{3.200000pt}}{0.000000pt}%
\pgfpathmoveto{\pgfqpoint{0.708220in}{4.728093in}}%
\pgfpathlineto{\pgfqpoint{0.840181in}{4.730743in}}%
\pgfpathlineto{\pgfqpoint{1.028697in}{4.733603in}}%
\pgfpathlineto{\pgfqpoint{1.047549in}{4.734714in}}%
\pgfpathlineto{\pgfqpoint{1.052262in}{4.736246in}}%
\pgfpathlineto{\pgfqpoint{1.056975in}{4.741986in}}%
\pgfpathlineto{\pgfqpoint{1.075827in}{4.743023in}}%
\pgfpathlineto{\pgfqpoint{1.080539in}{4.744282in}}%
\pgfpathlineto{\pgfqpoint{1.287907in}{4.746791in}}%
\pgfpathlineto{\pgfqpoint{1.391591in}{4.749779in}}%
\pgfpathlineto{\pgfqpoint{1.396304in}{4.750497in}}%
\pgfpathlineto{\pgfqpoint{1.401017in}{4.752957in}}%
\pgfpathlineto{\pgfqpoint{1.410443in}{4.753751in}}%
\pgfpathlineto{\pgfqpoint{1.415156in}{4.779399in}}%
\pgfpathlineto{\pgfqpoint{1.825179in}{4.787107in}}%
\pgfpathlineto{\pgfqpoint{1.829891in}{4.789497in}}%
\pgfpathlineto{\pgfqpoint{2.008982in}{4.794406in}}%
\pgfpathlineto{\pgfqpoint{2.013695in}{4.798304in}}%
\pgfpathlineto{\pgfqpoint{2.023121in}{4.800954in}}%
\pgfpathlineto{\pgfqpoint{2.027833in}{4.803733in}}%
\pgfpathlineto{\pgfqpoint{2.032546in}{4.803797in}}%
\pgfpathlineto{\pgfqpoint{2.037259in}{4.806875in}}%
\pgfpathlineto{\pgfqpoint{2.056111in}{4.810233in}}%
\pgfpathlineto{\pgfqpoint{2.093814in}{4.812058in}}%
\pgfpathlineto{\pgfqpoint{2.112666in}{4.815764in}}%
\pgfpathlineto{\pgfqpoint{2.131517in}{4.817141in}}%
\pgfpathlineto{\pgfqpoint{2.155082in}{4.818294in}}%
\pgfpathlineto{\pgfqpoint{2.183359in}{4.819316in}}%
\pgfpathlineto{\pgfqpoint{2.197498in}{4.819959in}}%
\pgfpathlineto{\pgfqpoint{2.202211in}{4.823680in}}%
\pgfpathlineto{\pgfqpoint{2.216350in}{4.828305in}}%
\pgfpathlineto{\pgfqpoint{2.230488in}{4.832002in}}%
\pgfpathlineto{\pgfqpoint{2.235201in}{4.841129in}}%
\pgfpathlineto{\pgfqpoint{2.244627in}{4.842570in}}%
\pgfpathlineto{\pgfqpoint{2.287043in}{4.844250in}}%
\pgfpathlineto{\pgfqpoint{2.315321in}{4.845354in}}%
\pgfpathlineto{\pgfqpoint{2.348311in}{4.846367in}}%
\pgfpathlineto{\pgfqpoint{2.362450in}{4.847399in}}%
\pgfpathlineto{\pgfqpoint{2.381301in}{4.850163in}}%
\pgfpathlineto{\pgfqpoint{2.442569in}{4.854016in}}%
\pgfpathlineto{\pgfqpoint{2.480272in}{4.855327in}}%
\pgfpathlineto{\pgfqpoint{2.489698in}{4.856251in}}%
\pgfpathlineto{\pgfqpoint{2.499124in}{4.857776in}}%
\pgfpathlineto{\pgfqpoint{2.503837in}{4.865914in}}%
\pgfpathlineto{\pgfqpoint{2.508550in}{4.866521in}}%
\pgfpathlineto{\pgfqpoint{2.522689in}{4.872495in}}%
\pgfpathlineto{\pgfqpoint{2.536827in}{4.884964in}}%
\pgfpathlineto{\pgfqpoint{2.541540in}{4.887568in}}%
\pgfpathlineto{\pgfqpoint{2.546253in}{4.899982in}}%
\pgfpathlineto{\pgfqpoint{2.555679in}{4.905137in}}%
\pgfpathlineto{\pgfqpoint{2.579243in}{4.909524in}}%
\pgfpathlineto{\pgfqpoint{2.583956in}{4.910189in}}%
\pgfpathlineto{\pgfqpoint{2.598095in}{4.918885in}}%
\pgfpathlineto{\pgfqpoint{2.602808in}{4.924026in}}%
\pgfpathlineto{\pgfqpoint{2.607521in}{4.924348in}}%
\pgfpathlineto{\pgfqpoint{2.612234in}{4.930718in}}%
\pgfpathlineto{\pgfqpoint{2.616947in}{4.934489in}}%
\pgfpathlineto{\pgfqpoint{2.621660in}{4.935886in}}%
\pgfpathlineto{\pgfqpoint{2.626372in}{4.941875in}}%
\pgfpathlineto{\pgfqpoint{2.631085in}{4.944295in}}%
\pgfpathlineto{\pgfqpoint{2.649937in}{4.948507in}}%
\pgfpathlineto{\pgfqpoint{2.659363in}{4.954463in}}%
\pgfpathlineto{\pgfqpoint{2.664076in}{4.966451in}}%
\pgfpathlineto{\pgfqpoint{2.668789in}{4.969696in}}%
\pgfpathlineto{\pgfqpoint{2.673502in}{4.970547in}}%
\pgfpathlineto{\pgfqpoint{2.678214in}{4.977131in}}%
\pgfpathlineto{\pgfqpoint{2.692353in}{4.978196in}}%
\pgfpathlineto{\pgfqpoint{2.697066in}{4.979358in}}%
\pgfpathlineto{\pgfqpoint{2.701779in}{4.982506in}}%
\pgfpathlineto{\pgfqpoint{2.711205in}{4.996799in}}%
\pgfpathlineto{\pgfqpoint{2.734769in}{5.003732in}}%
\pgfpathlineto{\pgfqpoint{2.739482in}{5.010823in}}%
\pgfpathlineto{\pgfqpoint{2.744195in}{5.012004in}}%
\pgfpathlineto{\pgfqpoint{2.748908in}{5.014470in}}%
\pgfpathlineto{\pgfqpoint{2.758334in}{5.016505in}}%
\pgfpathlineto{\pgfqpoint{2.772473in}{5.017695in}}%
\pgfpathlineto{\pgfqpoint{2.777185in}{5.018364in}}%
\pgfpathlineto{\pgfqpoint{2.781898in}{5.030951in}}%
\pgfpathlineto{\pgfqpoint{2.791324in}{5.031892in}}%
\pgfpathlineto{\pgfqpoint{2.800750in}{5.032096in}}%
\pgfpathlineto{\pgfqpoint{2.810176in}{5.033993in}}%
\pgfpathlineto{\pgfqpoint{2.824315in}{5.035463in}}%
\pgfpathlineto{\pgfqpoint{2.833740in}{5.037040in}}%
\pgfpathlineto{\pgfqpoint{2.862018in}{5.047038in}}%
\pgfpathlineto{\pgfqpoint{2.866731in}{5.047976in}}%
\pgfpathlineto{\pgfqpoint{2.871444in}{5.061184in}}%
\pgfpathlineto{\pgfqpoint{2.899721in}{5.063773in}}%
\pgfpathlineto{\pgfqpoint{2.909147in}{5.065832in}}%
\pgfpathlineto{\pgfqpoint{2.913860in}{5.068138in}}%
\pgfpathlineto{\pgfqpoint{2.927998in}{5.070539in}}%
\pgfpathlineto{\pgfqpoint{2.932711in}{5.072630in}}%
\pgfpathlineto{\pgfqpoint{2.956276in}{5.075468in}}%
\pgfpathlineto{\pgfqpoint{2.960989in}{5.077471in}}%
\pgfpathlineto{\pgfqpoint{2.965702in}{5.078108in}}%
\pgfpathlineto{\pgfqpoint{2.970415in}{5.087551in}}%
\pgfpathlineto{\pgfqpoint{2.975128in}{5.089559in}}%
\pgfpathlineto{\pgfqpoint{2.979840in}{5.095771in}}%
\pgfpathlineto{\pgfqpoint{2.989266in}{5.096192in}}%
\pgfpathlineto{\pgfqpoint{2.998692in}{5.097585in}}%
\pgfpathlineto{\pgfqpoint{3.003405in}{5.097884in}}%
\pgfpathlineto{\pgfqpoint{3.008118in}{5.100699in}}%
\pgfpathlineto{\pgfqpoint{3.022257in}{5.103300in}}%
\pgfpathlineto{\pgfqpoint{3.026969in}{5.103714in}}%
\pgfpathlineto{\pgfqpoint{3.031682in}{5.109335in}}%
\pgfpathlineto{\pgfqpoint{3.055247in}{5.111412in}}%
\pgfpathlineto{\pgfqpoint{3.064673in}{5.112052in}}%
\pgfpathlineto{\pgfqpoint{3.078811in}{5.112871in}}%
\pgfpathlineto{\pgfqpoint{3.083524in}{5.113355in}}%
\pgfpathlineto{\pgfqpoint{3.088237in}{5.127018in}}%
\pgfpathlineto{\pgfqpoint{3.102376in}{5.128712in}}%
\pgfpathlineto{\pgfqpoint{3.130653in}{5.130606in}}%
\pgfpathlineto{\pgfqpoint{3.149505in}{5.133991in}}%
\pgfpathlineto{\pgfqpoint{3.154218in}{5.134624in}}%
\pgfpathlineto{\pgfqpoint{3.168357in}{5.139708in}}%
\pgfpathlineto{\pgfqpoint{3.173070in}{5.140310in}}%
\pgfpathlineto{\pgfqpoint{3.182495in}{5.144101in}}%
\pgfpathlineto{\pgfqpoint{3.187208in}{5.153376in}}%
\pgfpathlineto{\pgfqpoint{3.215486in}{5.158673in}}%
\pgfpathlineto{\pgfqpoint{3.220199in}{5.158929in}}%
\pgfpathlineto{\pgfqpoint{3.224912in}{5.161465in}}%
\pgfpathlineto{\pgfqpoint{3.239050in}{5.163633in}}%
\pgfpathlineto{\pgfqpoint{3.243763in}{5.167473in}}%
\pgfpathlineto{\pgfqpoint{3.272041in}{5.170108in}}%
\pgfpathlineto{\pgfqpoint{3.276753in}{5.172136in}}%
\pgfpathlineto{\pgfqpoint{3.295605in}{5.174024in}}%
\pgfpathlineto{\pgfqpoint{3.305031in}{5.176195in}}%
\pgfpathlineto{\pgfqpoint{3.309744in}{5.185076in}}%
\pgfpathlineto{\pgfqpoint{3.323883in}{5.187135in}}%
\pgfpathlineto{\pgfqpoint{3.375724in}{5.192222in}}%
\pgfpathlineto{\pgfqpoint{3.380437in}{5.196896in}}%
\pgfpathlineto{\pgfqpoint{3.399289in}{5.205656in}}%
\pgfpathlineto{\pgfqpoint{3.418141in}{5.207271in}}%
\pgfpathlineto{\pgfqpoint{3.427566in}{5.209342in}}%
\pgfpathlineto{\pgfqpoint{3.432279in}{5.220026in}}%
\pgfpathlineto{\pgfqpoint{3.436992in}{5.221898in}}%
\pgfpathlineto{\pgfqpoint{3.451131in}{5.222209in}}%
\pgfpathlineto{\pgfqpoint{3.465270in}{5.225683in}}%
\pgfpathlineto{\pgfqpoint{3.484121in}{5.229624in}}%
\pgfpathlineto{\pgfqpoint{3.488834in}{5.233686in}}%
\pgfpathlineto{\pgfqpoint{3.512399in}{5.234915in}}%
\pgfpathlineto{\pgfqpoint{3.545389in}{5.236403in}}%
\pgfpathlineto{\pgfqpoint{3.568954in}{5.237799in}}%
\pgfpathlineto{\pgfqpoint{3.578379in}{5.251450in}}%
\pgfpathlineto{\pgfqpoint{3.592518in}{5.252691in}}%
\pgfpathlineto{\pgfqpoint{3.597231in}{5.255469in}}%
\pgfpathlineto{\pgfqpoint{3.700915in}{5.267026in}}%
\pgfpathlineto{\pgfqpoint{3.705628in}{5.281272in}}%
\pgfpathlineto{\pgfqpoint{3.710341in}{5.281523in}}%
\pgfpathlineto{\pgfqpoint{3.724480in}{5.287589in}}%
\pgfpathlineto{\pgfqpoint{3.743331in}{5.289402in}}%
\pgfpathlineto{\pgfqpoint{3.762183in}{5.291591in}}%
\pgfpathlineto{\pgfqpoint{3.790460in}{5.294110in}}%
\pgfpathlineto{\pgfqpoint{3.795173in}{5.294187in}}%
\pgfpathlineto{\pgfqpoint{3.799886in}{5.302941in}}%
\pgfpathlineto{\pgfqpoint{3.818738in}{5.304320in}}%
\pgfpathlineto{\pgfqpoint{3.828163in}{5.316469in}}%
\pgfpathlineto{\pgfqpoint{3.837589in}{5.319308in}}%
\pgfpathlineto{\pgfqpoint{3.936560in}{5.326424in}}%
\pgfpathlineto{\pgfqpoint{3.945986in}{5.331410in}}%
\pgfpathlineto{\pgfqpoint{3.960125in}{5.331856in}}%
\pgfpathlineto{\pgfqpoint{3.964838in}{5.334060in}}%
\pgfpathlineto{\pgfqpoint{3.978976in}{5.335049in}}%
\pgfpathlineto{\pgfqpoint{3.993115in}{5.338146in}}%
\pgfpathlineto{\pgfqpoint{3.997828in}{5.341903in}}%
\pgfpathlineto{\pgfqpoint{4.002541in}{5.348261in}}%
\pgfpathlineto{\pgfqpoint{4.035531in}{5.352773in}}%
\pgfpathlineto{\pgfqpoint{4.040244in}{5.355705in}}%
\pgfpathlineto{\pgfqpoint{4.049670in}{5.365359in}}%
\pgfpathlineto{\pgfqpoint{4.054383in}{5.374613in}}%
\pgfpathlineto{\pgfqpoint{4.063809in}{5.376621in}}%
\pgfpathlineto{\pgfqpoint{4.096799in}{5.383236in}}%
\pgfpathlineto{\pgfqpoint{4.101512in}{5.386951in}}%
\pgfpathlineto{\pgfqpoint{4.110938in}{5.387581in}}%
\pgfpathlineto{\pgfqpoint{4.120364in}{5.391786in}}%
\pgfpathlineto{\pgfqpoint{4.129789in}{5.392886in}}%
\pgfpathlineto{\pgfqpoint{4.134502in}{5.407725in}}%
\pgfpathlineto{\pgfqpoint{4.143928in}{5.905275in}}%
\pgfpathlineto{\pgfqpoint{5.845287in}{5.905275in}}%
\pgfpathlineto{\pgfqpoint{5.845287in}{5.905275in}}%
\pgfusepath{stroke}%
\end{pgfscope}%
\begin{pgfscope}%
\pgfpathrectangle{\pgfqpoint{0.708220in}{4.502489in}}{\pgfqpoint{5.141780in}{1.402786in}}%
\pgfusepath{clip}%
\pgfsetbuttcap%
\pgfsetroundjoin%
\pgfsetlinewidth{2.007500pt}%
\definecolor{currentstroke}{rgb}{1.000000,0.694118,0.305882}%
\pgfsetstrokecolor{currentstroke}%
\pgfsetdash{{2.000000pt}{3.300000pt}}{0.000000pt}%
\pgfpathmoveto{\pgfqpoint{0.708220in}{4.575686in}}%
\pgfpathlineto{\pgfqpoint{0.722359in}{4.577027in}}%
\pgfpathlineto{\pgfqpoint{1.372740in}{4.583679in}}%
\pgfpathlineto{\pgfqpoint{1.386878in}{4.584619in}}%
\pgfpathlineto{\pgfqpoint{1.391591in}{4.584630in}}%
\pgfpathlineto{\pgfqpoint{1.396304in}{4.586758in}}%
\pgfpathlineto{\pgfqpoint{1.401017in}{4.594530in}}%
\pgfpathlineto{\pgfqpoint{1.405730in}{4.596867in}}%
\pgfpathlineto{\pgfqpoint{1.419869in}{4.599099in}}%
\pgfpathlineto{\pgfqpoint{1.443433in}{4.599687in}}%
\pgfpathlineto{\pgfqpoint{1.716782in}{4.602148in}}%
\pgfpathlineto{\pgfqpoint{1.839317in}{4.603322in}}%
\pgfpathlineto{\pgfqpoint{1.886446in}{4.603649in}}%
\pgfpathlineto{\pgfqpoint{1.975991in}{4.604887in}}%
\pgfpathlineto{\pgfqpoint{1.990130in}{4.605564in}}%
\pgfpathlineto{\pgfqpoint{1.999556in}{4.608550in}}%
\pgfpathlineto{\pgfqpoint{2.013695in}{4.618676in}}%
\pgfpathlineto{\pgfqpoint{2.070250in}{4.620492in}}%
\pgfpathlineto{\pgfqpoint{2.183359in}{4.623892in}}%
\pgfpathlineto{\pgfqpoint{2.206924in}{4.626010in}}%
\pgfpathlineto{\pgfqpoint{2.211637in}{4.629424in}}%
\pgfpathlineto{\pgfqpoint{2.216350in}{4.647832in}}%
\pgfpathlineto{\pgfqpoint{2.230488in}{4.650708in}}%
\pgfpathlineto{\pgfqpoint{2.239914in}{4.653906in}}%
\pgfpathlineto{\pgfqpoint{2.258766in}{4.658518in}}%
\pgfpathlineto{\pgfqpoint{2.263479in}{4.671626in}}%
\pgfpathlineto{\pgfqpoint{2.268192in}{4.674134in}}%
\pgfpathlineto{\pgfqpoint{2.277617in}{4.674964in}}%
\pgfpathlineto{\pgfqpoint{2.291756in}{4.680857in}}%
\pgfpathlineto{\pgfqpoint{2.310608in}{4.681963in}}%
\pgfpathlineto{\pgfqpoint{2.315321in}{4.683356in}}%
\pgfpathlineto{\pgfqpoint{2.348311in}{4.685825in}}%
\pgfpathlineto{\pgfqpoint{2.357737in}{4.688192in}}%
\pgfpathlineto{\pgfqpoint{2.362450in}{4.696546in}}%
\pgfpathlineto{\pgfqpoint{2.371876in}{4.701279in}}%
\pgfpathlineto{\pgfqpoint{2.381301in}{4.702312in}}%
\pgfpathlineto{\pgfqpoint{2.489698in}{4.705440in}}%
\pgfpathlineto{\pgfqpoint{2.565105in}{4.707508in}}%
\pgfpathlineto{\pgfqpoint{2.593382in}{4.708585in}}%
\pgfpathlineto{\pgfqpoint{2.635798in}{4.710876in}}%
\pgfpathlineto{\pgfqpoint{2.659363in}{4.715253in}}%
\pgfpathlineto{\pgfqpoint{2.664076in}{4.716954in}}%
\pgfpathlineto{\pgfqpoint{2.668789in}{4.717077in}}%
\pgfpathlineto{\pgfqpoint{2.682927in}{4.720175in}}%
\pgfpathlineto{\pgfqpoint{2.687640in}{4.726657in}}%
\pgfpathlineto{\pgfqpoint{2.697066in}{4.727461in}}%
\pgfpathlineto{\pgfqpoint{2.706492in}{4.729110in}}%
\pgfpathlineto{\pgfqpoint{2.711205in}{4.729173in}}%
\pgfpathlineto{\pgfqpoint{2.715918in}{4.731567in}}%
\pgfpathlineto{\pgfqpoint{2.725344in}{4.733241in}}%
\pgfpathlineto{\pgfqpoint{2.730056in}{4.738342in}}%
\pgfpathlineto{\pgfqpoint{2.739482in}{4.739897in}}%
\pgfpathlineto{\pgfqpoint{2.744195in}{4.758295in}}%
\pgfpathlineto{\pgfqpoint{2.753621in}{4.761653in}}%
\pgfpathlineto{\pgfqpoint{2.772473in}{4.762463in}}%
\pgfpathlineto{\pgfqpoint{2.777185in}{4.764980in}}%
\pgfpathlineto{\pgfqpoint{2.781898in}{4.765126in}}%
\pgfpathlineto{\pgfqpoint{2.786611in}{4.767501in}}%
\pgfpathlineto{\pgfqpoint{2.796037in}{4.767627in}}%
\pgfpathlineto{\pgfqpoint{2.810176in}{4.772374in}}%
\pgfpathlineto{\pgfqpoint{2.819602in}{4.773276in}}%
\pgfpathlineto{\pgfqpoint{2.843166in}{4.775744in}}%
\pgfpathlineto{\pgfqpoint{2.847879in}{4.777443in}}%
\pgfpathlineto{\pgfqpoint{2.857305in}{4.778405in}}%
\pgfpathlineto{\pgfqpoint{2.862018in}{4.780090in}}%
\pgfpathlineto{\pgfqpoint{2.904434in}{4.783203in}}%
\pgfpathlineto{\pgfqpoint{2.909147in}{4.786076in}}%
\pgfpathlineto{\pgfqpoint{2.923286in}{4.787254in}}%
\pgfpathlineto{\pgfqpoint{2.927998in}{4.788671in}}%
\pgfpathlineto{\pgfqpoint{2.932711in}{4.791749in}}%
\pgfpathlineto{\pgfqpoint{2.942137in}{4.793884in}}%
\pgfpathlineto{\pgfqpoint{2.951563in}{4.794105in}}%
\pgfpathlineto{\pgfqpoint{2.960989in}{4.796525in}}%
\pgfpathlineto{\pgfqpoint{2.975128in}{4.797634in}}%
\pgfpathlineto{\pgfqpoint{2.979840in}{4.799422in}}%
\pgfpathlineto{\pgfqpoint{2.993979in}{4.801065in}}%
\pgfpathlineto{\pgfqpoint{2.998692in}{4.803094in}}%
\pgfpathlineto{\pgfqpoint{3.003405in}{4.803538in}}%
\pgfpathlineto{\pgfqpoint{3.008118in}{4.805437in}}%
\pgfpathlineto{\pgfqpoint{3.036395in}{4.809773in}}%
\pgfpathlineto{\pgfqpoint{3.045821in}{4.812528in}}%
\pgfpathlineto{\pgfqpoint{3.055247in}{4.814006in}}%
\pgfpathlineto{\pgfqpoint{3.064673in}{4.814702in}}%
\pgfpathlineto{\pgfqpoint{3.097663in}{4.820512in}}%
\pgfpathlineto{\pgfqpoint{3.130653in}{4.822504in}}%
\pgfpathlineto{\pgfqpoint{3.140079in}{4.824887in}}%
\pgfpathlineto{\pgfqpoint{3.163644in}{4.826436in}}%
\pgfpathlineto{\pgfqpoint{3.177782in}{4.827325in}}%
\pgfpathlineto{\pgfqpoint{3.182495in}{4.829725in}}%
\pgfpathlineto{\pgfqpoint{3.187208in}{4.839898in}}%
\pgfpathlineto{\pgfqpoint{3.201347in}{4.840718in}}%
\pgfpathlineto{\pgfqpoint{3.210773in}{4.842516in}}%
\pgfpathlineto{\pgfqpoint{3.215486in}{4.842704in}}%
\pgfpathlineto{\pgfqpoint{3.220199in}{4.844139in}}%
\pgfpathlineto{\pgfqpoint{3.229624in}{4.845618in}}%
\pgfpathlineto{\pgfqpoint{3.239050in}{4.847691in}}%
\pgfpathlineto{\pgfqpoint{3.243763in}{4.849798in}}%
\pgfpathlineto{\pgfqpoint{3.267328in}{4.851319in}}%
\pgfpathlineto{\pgfqpoint{3.276753in}{4.852407in}}%
\pgfpathlineto{\pgfqpoint{3.281466in}{4.854128in}}%
\pgfpathlineto{\pgfqpoint{3.286179in}{4.854661in}}%
\pgfpathlineto{\pgfqpoint{3.290892in}{4.856700in}}%
\pgfpathlineto{\pgfqpoint{3.323883in}{4.861520in}}%
\pgfpathlineto{\pgfqpoint{3.333308in}{4.863837in}}%
\pgfpathlineto{\pgfqpoint{3.352160in}{4.864811in}}%
\pgfpathlineto{\pgfqpoint{3.380437in}{4.866933in}}%
\pgfpathlineto{\pgfqpoint{3.436992in}{4.871169in}}%
\pgfpathlineto{\pgfqpoint{3.441705in}{4.872127in}}%
\pgfpathlineto{\pgfqpoint{3.446418in}{4.875000in}}%
\pgfpathlineto{\pgfqpoint{3.469983in}{4.880187in}}%
\pgfpathlineto{\pgfqpoint{3.498260in}{4.882437in}}%
\pgfpathlineto{\pgfqpoint{3.512399in}{4.883546in}}%
\pgfpathlineto{\pgfqpoint{3.526537in}{4.885272in}}%
\pgfpathlineto{\pgfqpoint{3.531250in}{4.897268in}}%
\pgfpathlineto{\pgfqpoint{3.568954in}{4.903702in}}%
\pgfpathlineto{\pgfqpoint{3.583092in}{4.904901in}}%
\pgfpathlineto{\pgfqpoint{3.611370in}{4.906567in}}%
\pgfpathlineto{\pgfqpoint{3.634934in}{4.907689in}}%
\pgfpathlineto{\pgfqpoint{3.649073in}{4.909217in}}%
\pgfpathlineto{\pgfqpoint{3.653786in}{4.909617in}}%
\pgfpathlineto{\pgfqpoint{3.658499in}{4.913111in}}%
\pgfpathlineto{\pgfqpoint{3.667925in}{4.914023in}}%
\pgfpathlineto{\pgfqpoint{3.677350in}{4.915221in}}%
\pgfpathlineto{\pgfqpoint{3.700915in}{4.917629in}}%
\pgfpathlineto{\pgfqpoint{3.719767in}{4.920669in}}%
\pgfpathlineto{\pgfqpoint{3.724480in}{4.926393in}}%
\pgfpathlineto{\pgfqpoint{3.729192in}{4.927383in}}%
\pgfpathlineto{\pgfqpoint{3.733905in}{4.930561in}}%
\pgfpathlineto{\pgfqpoint{3.738618in}{4.940078in}}%
\pgfpathlineto{\pgfqpoint{3.762183in}{4.941663in}}%
\pgfpathlineto{\pgfqpoint{3.781034in}{4.943376in}}%
\pgfpathlineto{\pgfqpoint{3.790460in}{4.944487in}}%
\pgfpathlineto{\pgfqpoint{3.799886in}{4.945672in}}%
\pgfpathlineto{\pgfqpoint{3.823451in}{4.947287in}}%
\pgfpathlineto{\pgfqpoint{3.828163in}{4.949276in}}%
\pgfpathlineto{\pgfqpoint{3.832876in}{4.955403in}}%
\pgfpathlineto{\pgfqpoint{3.837589in}{4.959151in}}%
\pgfpathlineto{\pgfqpoint{3.842302in}{4.960205in}}%
\pgfpathlineto{\pgfqpoint{3.847015in}{4.974420in}}%
\pgfpathlineto{\pgfqpoint{3.851728in}{4.982712in}}%
\pgfpathlineto{\pgfqpoint{3.856441in}{4.984178in}}%
\pgfpathlineto{\pgfqpoint{3.861154in}{4.991753in}}%
\pgfpathlineto{\pgfqpoint{3.865867in}{4.993084in}}%
\pgfpathlineto{\pgfqpoint{3.870580in}{5.005690in}}%
\pgfpathlineto{\pgfqpoint{3.880005in}{5.008592in}}%
\pgfpathlineto{\pgfqpoint{3.884718in}{5.021462in}}%
\pgfpathlineto{\pgfqpoint{3.889431in}{5.026788in}}%
\pgfpathlineto{\pgfqpoint{3.912996in}{5.029357in}}%
\pgfpathlineto{\pgfqpoint{3.922422in}{5.032609in}}%
\pgfpathlineto{\pgfqpoint{3.927134in}{5.036632in}}%
\pgfpathlineto{\pgfqpoint{3.941273in}{5.037816in}}%
\pgfpathlineto{\pgfqpoint{3.945986in}{5.039755in}}%
\pgfpathlineto{\pgfqpoint{3.955412in}{5.040048in}}%
\pgfpathlineto{\pgfqpoint{3.960125in}{5.046514in}}%
\pgfpathlineto{\pgfqpoint{3.964838in}{5.048731in}}%
\pgfpathlineto{\pgfqpoint{3.978976in}{5.048916in}}%
\pgfpathlineto{\pgfqpoint{3.988402in}{5.050454in}}%
\pgfpathlineto{\pgfqpoint{3.997828in}{5.058586in}}%
\pgfpathlineto{\pgfqpoint{4.002541in}{5.059113in}}%
\pgfpathlineto{\pgfqpoint{4.007254in}{5.061944in}}%
\pgfpathlineto{\pgfqpoint{4.030818in}{5.066156in}}%
\pgfpathlineto{\pgfqpoint{4.044957in}{5.072083in}}%
\pgfpathlineto{\pgfqpoint{4.049670in}{5.076490in}}%
\pgfpathlineto{\pgfqpoint{4.054383in}{5.076807in}}%
\pgfpathlineto{\pgfqpoint{4.059096in}{5.078407in}}%
\pgfpathlineto{\pgfqpoint{4.073235in}{5.089313in}}%
\pgfpathlineto{\pgfqpoint{4.077947in}{5.091116in}}%
\pgfpathlineto{\pgfqpoint{4.082660in}{5.144930in}}%
\pgfpathlineto{\pgfqpoint{4.087373in}{5.154558in}}%
\pgfpathlineto{\pgfqpoint{4.092086in}{5.157608in}}%
\pgfpathlineto{\pgfqpoint{4.096799in}{5.165826in}}%
\pgfpathlineto{\pgfqpoint{4.106225in}{5.203822in}}%
\pgfpathlineto{\pgfqpoint{4.110938in}{5.203848in}}%
\pgfpathlineto{\pgfqpoint{4.115651in}{5.209878in}}%
\pgfpathlineto{\pgfqpoint{4.120364in}{5.228144in}}%
\pgfpathlineto{\pgfqpoint{4.125077in}{5.230046in}}%
\pgfpathlineto{\pgfqpoint{4.129789in}{5.244814in}}%
\pgfpathlineto{\pgfqpoint{4.134502in}{5.306157in}}%
\pgfpathlineto{\pgfqpoint{4.139215in}{5.451029in}}%
\pgfpathlineto{\pgfqpoint{4.143928in}{5.467669in}}%
\pgfpathlineto{\pgfqpoint{4.148641in}{5.514647in}}%
\pgfpathlineto{\pgfqpoint{4.153354in}{5.770257in}}%
\pgfpathlineto{\pgfqpoint{4.158067in}{5.779182in}}%
\pgfpathlineto{\pgfqpoint{4.167493in}{5.905275in}}%
\pgfpathlineto{\pgfqpoint{5.845287in}{5.905275in}}%
\pgfpathlineto{\pgfqpoint{5.845287in}{5.905275in}}%
\pgfusepath{stroke}%
\end{pgfscope}%
\begin{pgfscope}%
\pgfpathrectangle{\pgfqpoint{0.708220in}{4.502489in}}{\pgfqpoint{5.141780in}{1.402786in}}%
\pgfusepath{clip}%
\pgfsetrectcap%
\pgfsetroundjoin%
\pgfsetlinewidth{2.007500pt}%
\definecolor{currentstroke}{rgb}{0.980392,0.529412,0.458824}%
\pgfsetstrokecolor{currentstroke}%
\pgfsetdash{}{0pt}%
\pgfpathmoveto{\pgfqpoint{0.708220in}{4.782864in}}%
\pgfpathlineto{\pgfqpoint{0.717646in}{4.784557in}}%
\pgfpathlineto{\pgfqpoint{0.727071in}{4.785703in}}%
\pgfpathlineto{\pgfqpoint{0.821330in}{4.788240in}}%
\pgfpathlineto{\pgfqpoint{0.849607in}{4.789212in}}%
\pgfpathlineto{\pgfqpoint{0.906162in}{4.790401in}}%
\pgfpathlineto{\pgfqpoint{0.939152in}{4.791266in}}%
\pgfpathlineto{\pgfqpoint{0.990994in}{4.792296in}}%
\pgfpathlineto{\pgfqpoint{1.000420in}{4.793180in}}%
\pgfpathlineto{\pgfqpoint{1.221927in}{4.797417in}}%
\pgfpathlineto{\pgfqpoint{1.302046in}{4.799714in}}%
\pgfpathlineto{\pgfqpoint{1.320898in}{4.800566in}}%
\pgfpathlineto{\pgfqpoint{1.358601in}{4.801713in}}%
\pgfpathlineto{\pgfqpoint{1.405730in}{4.805834in}}%
\pgfpathlineto{\pgfqpoint{1.410443in}{4.808929in}}%
\pgfpathlineto{\pgfqpoint{1.495275in}{4.813067in}}%
\pgfpathlineto{\pgfqpoint{1.740346in}{4.816720in}}%
\pgfpathlineto{\pgfqpoint{1.811040in}{4.817949in}}%
\pgfpathlineto{\pgfqpoint{1.858169in}{4.818976in}}%
\pgfpathlineto{\pgfqpoint{1.924150in}{4.820525in}}%
\pgfpathlineto{\pgfqpoint{1.975991in}{4.822050in}}%
\pgfpathlineto{\pgfqpoint{2.018408in}{4.823531in}}%
\pgfpathlineto{\pgfqpoint{2.023121in}{4.823635in}}%
\pgfpathlineto{\pgfqpoint{2.032546in}{4.825882in}}%
\pgfpathlineto{\pgfqpoint{2.065537in}{4.827575in}}%
\pgfpathlineto{\pgfqpoint{2.070250in}{4.838260in}}%
\pgfpathlineto{\pgfqpoint{2.098527in}{4.840163in}}%
\pgfpathlineto{\pgfqpoint{2.206924in}{4.841781in}}%
\pgfpathlineto{\pgfqpoint{2.239914in}{4.842815in}}%
\pgfpathlineto{\pgfqpoint{2.348311in}{4.844648in}}%
\pgfpathlineto{\pgfqpoint{2.357737in}{4.846244in}}%
\pgfpathlineto{\pgfqpoint{2.390727in}{4.849211in}}%
\pgfpathlineto{\pgfqpoint{2.395440in}{4.868461in}}%
\pgfpathlineto{\pgfqpoint{2.400153in}{4.878142in}}%
\pgfpathlineto{\pgfqpoint{2.409579in}{4.880398in}}%
\pgfpathlineto{\pgfqpoint{2.414292in}{4.883980in}}%
\pgfpathlineto{\pgfqpoint{2.423718in}{4.885284in}}%
\pgfpathlineto{\pgfqpoint{2.428430in}{4.885538in}}%
\pgfpathlineto{\pgfqpoint{2.433143in}{4.891260in}}%
\pgfpathlineto{\pgfqpoint{2.437856in}{4.891946in}}%
\pgfpathlineto{\pgfqpoint{2.442569in}{4.902909in}}%
\pgfpathlineto{\pgfqpoint{2.461421in}{4.907644in}}%
\pgfpathlineto{\pgfqpoint{2.466134in}{4.915293in}}%
\pgfpathlineto{\pgfqpoint{2.470847in}{4.919742in}}%
\pgfpathlineto{\pgfqpoint{2.489698in}{4.925709in}}%
\pgfpathlineto{\pgfqpoint{2.499124in}{4.934684in}}%
\pgfpathlineto{\pgfqpoint{2.508550in}{4.937375in}}%
\pgfpathlineto{\pgfqpoint{2.522689in}{4.952833in}}%
\pgfpathlineto{\pgfqpoint{2.527401in}{4.953445in}}%
\pgfpathlineto{\pgfqpoint{2.532114in}{4.959200in}}%
\pgfpathlineto{\pgfqpoint{2.536827in}{4.959723in}}%
\pgfpathlineto{\pgfqpoint{2.541540in}{4.978508in}}%
\pgfpathlineto{\pgfqpoint{2.546253in}{4.984614in}}%
\pgfpathlineto{\pgfqpoint{2.560392in}{4.985964in}}%
\pgfpathlineto{\pgfqpoint{2.565105in}{4.988989in}}%
\pgfpathlineto{\pgfqpoint{2.569818in}{5.001477in}}%
\pgfpathlineto{\pgfqpoint{2.574531in}{5.006797in}}%
\pgfpathlineto{\pgfqpoint{2.579243in}{5.007296in}}%
\pgfpathlineto{\pgfqpoint{2.583956in}{5.011990in}}%
\pgfpathlineto{\pgfqpoint{2.593382in}{5.014442in}}%
\pgfpathlineto{\pgfqpoint{2.602808in}{5.018340in}}%
\pgfpathlineto{\pgfqpoint{2.607521in}{5.018908in}}%
\pgfpathlineto{\pgfqpoint{2.616947in}{5.022120in}}%
\pgfpathlineto{\pgfqpoint{2.626372in}{5.022806in}}%
\pgfpathlineto{\pgfqpoint{2.640511in}{5.027057in}}%
\pgfpathlineto{\pgfqpoint{2.645224in}{5.027171in}}%
\pgfpathlineto{\pgfqpoint{2.654650in}{5.033352in}}%
\pgfpathlineto{\pgfqpoint{2.659363in}{5.034056in}}%
\pgfpathlineto{\pgfqpoint{2.664076in}{5.041378in}}%
\pgfpathlineto{\pgfqpoint{2.673502in}{5.042804in}}%
\pgfpathlineto{\pgfqpoint{2.687640in}{5.044519in}}%
\pgfpathlineto{\pgfqpoint{2.697066in}{5.046568in}}%
\pgfpathlineto{\pgfqpoint{2.701779in}{5.052607in}}%
\pgfpathlineto{\pgfqpoint{2.715918in}{5.055986in}}%
\pgfpathlineto{\pgfqpoint{2.725344in}{5.058585in}}%
\pgfpathlineto{\pgfqpoint{2.753621in}{5.060806in}}%
\pgfpathlineto{\pgfqpoint{2.772473in}{5.064228in}}%
\pgfpathlineto{\pgfqpoint{2.781898in}{5.065799in}}%
\pgfpathlineto{\pgfqpoint{2.786611in}{5.069207in}}%
\pgfpathlineto{\pgfqpoint{2.800750in}{5.071568in}}%
\pgfpathlineto{\pgfqpoint{2.805463in}{5.080531in}}%
\pgfpathlineto{\pgfqpoint{2.810176in}{5.081175in}}%
\pgfpathlineto{\pgfqpoint{2.814889in}{5.087024in}}%
\pgfpathlineto{\pgfqpoint{2.829027in}{5.089740in}}%
\pgfpathlineto{\pgfqpoint{2.833740in}{5.094182in}}%
\pgfpathlineto{\pgfqpoint{2.847879in}{5.101269in}}%
\pgfpathlineto{\pgfqpoint{2.852592in}{5.108880in}}%
\pgfpathlineto{\pgfqpoint{2.857305in}{5.112251in}}%
\pgfpathlineto{\pgfqpoint{2.866731in}{5.114486in}}%
\pgfpathlineto{\pgfqpoint{2.871444in}{5.118678in}}%
\pgfpathlineto{\pgfqpoint{2.880869in}{5.121465in}}%
\pgfpathlineto{\pgfqpoint{2.885582in}{5.127137in}}%
\pgfpathlineto{\pgfqpoint{2.895008in}{5.127963in}}%
\pgfpathlineto{\pgfqpoint{2.904434in}{5.137468in}}%
\pgfpathlineto{\pgfqpoint{2.909147in}{5.138078in}}%
\pgfpathlineto{\pgfqpoint{2.913860in}{5.140653in}}%
\pgfpathlineto{\pgfqpoint{2.918573in}{5.144521in}}%
\pgfpathlineto{\pgfqpoint{2.937424in}{5.150046in}}%
\pgfpathlineto{\pgfqpoint{2.951563in}{5.160917in}}%
\pgfpathlineto{\pgfqpoint{2.956276in}{5.160931in}}%
\pgfpathlineto{\pgfqpoint{2.960989in}{5.169021in}}%
\pgfpathlineto{\pgfqpoint{2.965702in}{5.169775in}}%
\pgfpathlineto{\pgfqpoint{2.970415in}{5.173472in}}%
\pgfpathlineto{\pgfqpoint{2.975128in}{5.175751in}}%
\pgfpathlineto{\pgfqpoint{2.984553in}{5.176842in}}%
\pgfpathlineto{\pgfqpoint{2.989266in}{5.181123in}}%
\pgfpathlineto{\pgfqpoint{2.993979in}{5.181536in}}%
\pgfpathlineto{\pgfqpoint{3.003405in}{5.190928in}}%
\pgfpathlineto{\pgfqpoint{3.012831in}{5.193292in}}%
\pgfpathlineto{\pgfqpoint{3.022257in}{5.199180in}}%
\pgfpathlineto{\pgfqpoint{3.026969in}{5.206216in}}%
\pgfpathlineto{\pgfqpoint{3.045821in}{5.208564in}}%
\pgfpathlineto{\pgfqpoint{3.055247in}{5.209872in}}%
\pgfpathlineto{\pgfqpoint{3.059960in}{5.215031in}}%
\pgfpathlineto{\pgfqpoint{3.069386in}{5.229936in}}%
\pgfpathlineto{\pgfqpoint{3.078811in}{5.231748in}}%
\pgfpathlineto{\pgfqpoint{3.083524in}{5.234361in}}%
\pgfpathlineto{\pgfqpoint{3.088237in}{5.235454in}}%
\pgfpathlineto{\pgfqpoint{3.092950in}{5.239846in}}%
\pgfpathlineto{\pgfqpoint{3.111802in}{5.241679in}}%
\pgfpathlineto{\pgfqpoint{3.121228in}{5.243354in}}%
\pgfpathlineto{\pgfqpoint{3.125940in}{5.247499in}}%
\pgfpathlineto{\pgfqpoint{3.182495in}{5.255927in}}%
\pgfpathlineto{\pgfqpoint{3.196634in}{5.256969in}}%
\pgfpathlineto{\pgfqpoint{3.201347in}{5.265553in}}%
\pgfpathlineto{\pgfqpoint{3.206060in}{5.270108in}}%
\pgfpathlineto{\pgfqpoint{3.210773in}{5.285113in}}%
\pgfpathlineto{\pgfqpoint{3.220199in}{5.289499in}}%
\pgfpathlineto{\pgfqpoint{3.224912in}{5.289920in}}%
\pgfpathlineto{\pgfqpoint{3.229624in}{5.292237in}}%
\pgfpathlineto{\pgfqpoint{3.234337in}{5.298023in}}%
\pgfpathlineto{\pgfqpoint{3.239050in}{5.299223in}}%
\pgfpathlineto{\pgfqpoint{3.243763in}{5.299246in}}%
\pgfpathlineto{\pgfqpoint{3.253189in}{5.304047in}}%
\pgfpathlineto{\pgfqpoint{3.257902in}{5.304406in}}%
\pgfpathlineto{\pgfqpoint{3.262615in}{5.305881in}}%
\pgfpathlineto{\pgfqpoint{3.267328in}{5.310073in}}%
\pgfpathlineto{\pgfqpoint{3.290892in}{5.312052in}}%
\pgfpathlineto{\pgfqpoint{3.305031in}{5.317707in}}%
\pgfpathlineto{\pgfqpoint{3.309744in}{5.318149in}}%
\pgfpathlineto{\pgfqpoint{3.314457in}{5.321515in}}%
\pgfpathlineto{\pgfqpoint{3.323883in}{5.322919in}}%
\pgfpathlineto{\pgfqpoint{3.328595in}{5.332500in}}%
\pgfpathlineto{\pgfqpoint{3.333308in}{5.338982in}}%
\pgfpathlineto{\pgfqpoint{3.352160in}{5.343563in}}%
\pgfpathlineto{\pgfqpoint{3.356873in}{5.343773in}}%
\pgfpathlineto{\pgfqpoint{3.371012in}{5.350171in}}%
\pgfpathlineto{\pgfqpoint{3.375724in}{5.350292in}}%
\pgfpathlineto{\pgfqpoint{3.385150in}{5.352359in}}%
\pgfpathlineto{\pgfqpoint{3.418141in}{5.356872in}}%
\pgfpathlineto{\pgfqpoint{3.427566in}{5.357624in}}%
\pgfpathlineto{\pgfqpoint{3.432279in}{5.359442in}}%
\pgfpathlineto{\pgfqpoint{3.441705in}{5.360211in}}%
\pgfpathlineto{\pgfqpoint{3.451131in}{5.364862in}}%
\pgfpathlineto{\pgfqpoint{3.465270in}{5.367127in}}%
\pgfpathlineto{\pgfqpoint{3.469983in}{5.378033in}}%
\pgfpathlineto{\pgfqpoint{3.484121in}{5.390601in}}%
\pgfpathlineto{\pgfqpoint{3.488834in}{5.390962in}}%
\pgfpathlineto{\pgfqpoint{3.493547in}{5.394073in}}%
\pgfpathlineto{\pgfqpoint{3.498260in}{5.394826in}}%
\pgfpathlineto{\pgfqpoint{3.502973in}{5.397158in}}%
\pgfpathlineto{\pgfqpoint{3.507686in}{5.397300in}}%
\pgfpathlineto{\pgfqpoint{3.517112in}{5.407208in}}%
\pgfpathlineto{\pgfqpoint{3.521825in}{5.407661in}}%
\pgfpathlineto{\pgfqpoint{3.526537in}{5.410083in}}%
\pgfpathlineto{\pgfqpoint{3.535963in}{5.411620in}}%
\pgfpathlineto{\pgfqpoint{3.545389in}{5.413550in}}%
\pgfpathlineto{\pgfqpoint{3.550102in}{5.417132in}}%
\pgfpathlineto{\pgfqpoint{3.554815in}{5.418925in}}%
\pgfpathlineto{\pgfqpoint{3.568954in}{5.420351in}}%
\pgfpathlineto{\pgfqpoint{3.583092in}{5.427734in}}%
\pgfpathlineto{\pgfqpoint{3.587805in}{5.442225in}}%
\pgfpathlineto{\pgfqpoint{3.592518in}{5.447128in}}%
\pgfpathlineto{\pgfqpoint{3.630221in}{5.455121in}}%
\pgfpathlineto{\pgfqpoint{3.634934in}{5.458362in}}%
\pgfpathlineto{\pgfqpoint{3.677350in}{5.462222in}}%
\pgfpathlineto{\pgfqpoint{3.682063in}{5.463075in}}%
\pgfpathlineto{\pgfqpoint{3.686776in}{5.468746in}}%
\pgfpathlineto{\pgfqpoint{3.691489in}{5.469064in}}%
\pgfpathlineto{\pgfqpoint{3.696202in}{5.471304in}}%
\pgfpathlineto{\pgfqpoint{3.719767in}{5.472976in}}%
\pgfpathlineto{\pgfqpoint{3.724480in}{5.478201in}}%
\pgfpathlineto{\pgfqpoint{3.729192in}{5.479731in}}%
\pgfpathlineto{\pgfqpoint{3.733905in}{5.487667in}}%
\pgfpathlineto{\pgfqpoint{3.738618in}{5.492955in}}%
\pgfpathlineto{\pgfqpoint{3.743331in}{5.494997in}}%
\pgfpathlineto{\pgfqpoint{3.752757in}{5.500871in}}%
\pgfpathlineto{\pgfqpoint{3.757470in}{5.501279in}}%
\pgfpathlineto{\pgfqpoint{3.762183in}{5.503357in}}%
\pgfpathlineto{\pgfqpoint{3.766896in}{5.503841in}}%
\pgfpathlineto{\pgfqpoint{3.771609in}{5.505590in}}%
\pgfpathlineto{\pgfqpoint{3.776321in}{5.505598in}}%
\pgfpathlineto{\pgfqpoint{3.781034in}{5.510552in}}%
\pgfpathlineto{\pgfqpoint{3.795173in}{5.512266in}}%
\pgfpathlineto{\pgfqpoint{3.799886in}{5.514675in}}%
\pgfpathlineto{\pgfqpoint{3.809312in}{5.515917in}}%
\pgfpathlineto{\pgfqpoint{3.814025in}{5.519111in}}%
\pgfpathlineto{\pgfqpoint{3.818738in}{5.520602in}}%
\pgfpathlineto{\pgfqpoint{3.823451in}{5.528692in}}%
\pgfpathlineto{\pgfqpoint{3.828163in}{5.529032in}}%
\pgfpathlineto{\pgfqpoint{3.832876in}{5.537438in}}%
\pgfpathlineto{\pgfqpoint{3.842302in}{5.545872in}}%
\pgfpathlineto{\pgfqpoint{3.870580in}{5.549601in}}%
\pgfpathlineto{\pgfqpoint{3.875293in}{5.552425in}}%
\pgfpathlineto{\pgfqpoint{3.903570in}{5.556148in}}%
\pgfpathlineto{\pgfqpoint{3.908283in}{5.558193in}}%
\pgfpathlineto{\pgfqpoint{3.917709in}{5.559687in}}%
\pgfpathlineto{\pgfqpoint{3.922422in}{5.561107in}}%
\pgfpathlineto{\pgfqpoint{3.931847in}{5.565982in}}%
\pgfpathlineto{\pgfqpoint{3.945986in}{5.568091in}}%
\pgfpathlineto{\pgfqpoint{3.950699in}{5.569725in}}%
\pgfpathlineto{\pgfqpoint{3.955412in}{5.569938in}}%
\pgfpathlineto{\pgfqpoint{3.964838in}{5.572827in}}%
\pgfpathlineto{\pgfqpoint{3.969551in}{5.574072in}}%
\pgfpathlineto{\pgfqpoint{3.978976in}{5.583038in}}%
\pgfpathlineto{\pgfqpoint{3.993115in}{5.585451in}}%
\pgfpathlineto{\pgfqpoint{3.997828in}{5.588475in}}%
\pgfpathlineto{\pgfqpoint{4.040244in}{5.593912in}}%
\pgfpathlineto{\pgfqpoint{4.044957in}{5.598248in}}%
\pgfpathlineto{\pgfqpoint{4.054383in}{5.599291in}}%
\pgfpathlineto{\pgfqpoint{4.059096in}{5.599574in}}%
\pgfpathlineto{\pgfqpoint{4.068522in}{5.603776in}}%
\pgfpathlineto{\pgfqpoint{4.077947in}{5.605228in}}%
\pgfpathlineto{\pgfqpoint{4.082660in}{5.616673in}}%
\pgfpathlineto{\pgfqpoint{4.096799in}{5.623299in}}%
\pgfpathlineto{\pgfqpoint{4.101512in}{5.623587in}}%
\pgfpathlineto{\pgfqpoint{4.110938in}{5.627149in}}%
\pgfpathlineto{\pgfqpoint{4.125077in}{5.629305in}}%
\pgfpathlineto{\pgfqpoint{4.129789in}{5.632533in}}%
\pgfpathlineto{\pgfqpoint{4.139215in}{5.634912in}}%
\pgfpathlineto{\pgfqpoint{4.143928in}{5.638689in}}%
\pgfpathlineto{\pgfqpoint{4.153354in}{5.640046in}}%
\pgfpathlineto{\pgfqpoint{4.162780in}{5.642504in}}%
\pgfpathlineto{\pgfqpoint{4.167493in}{5.642527in}}%
\pgfpathlineto{\pgfqpoint{4.172206in}{5.654851in}}%
\pgfpathlineto{\pgfqpoint{4.186344in}{5.659429in}}%
\pgfpathlineto{\pgfqpoint{4.195770in}{5.661319in}}%
\pgfpathlineto{\pgfqpoint{4.209909in}{5.662819in}}%
\pgfpathlineto{\pgfqpoint{4.224048in}{5.663800in}}%
\pgfpathlineto{\pgfqpoint{4.228760in}{5.664012in}}%
\pgfpathlineto{\pgfqpoint{4.233473in}{5.667883in}}%
\pgfpathlineto{\pgfqpoint{4.247612in}{5.670380in}}%
\pgfpathlineto{\pgfqpoint{4.252325in}{5.676552in}}%
\pgfpathlineto{\pgfqpoint{4.261751in}{5.679536in}}%
\pgfpathlineto{\pgfqpoint{4.266464in}{5.683464in}}%
\pgfpathlineto{\pgfqpoint{4.271177in}{5.691831in}}%
\pgfpathlineto{\pgfqpoint{4.275889in}{5.693992in}}%
\pgfpathlineto{\pgfqpoint{4.280602in}{5.694141in}}%
\pgfpathlineto{\pgfqpoint{4.285315in}{5.698921in}}%
\pgfpathlineto{\pgfqpoint{4.299454in}{5.700812in}}%
\pgfpathlineto{\pgfqpoint{4.304167in}{5.700978in}}%
\pgfpathlineto{\pgfqpoint{4.308880in}{5.702832in}}%
\pgfpathlineto{\pgfqpoint{4.313593in}{5.703061in}}%
\pgfpathlineto{\pgfqpoint{4.318306in}{5.708475in}}%
\pgfpathlineto{\pgfqpoint{4.327731in}{5.710915in}}%
\pgfpathlineto{\pgfqpoint{4.332444in}{5.710991in}}%
\pgfpathlineto{\pgfqpoint{4.337157in}{5.712733in}}%
\pgfpathlineto{\pgfqpoint{4.356009in}{5.715359in}}%
\pgfpathlineto{\pgfqpoint{4.370148in}{5.730409in}}%
\pgfpathlineto{\pgfqpoint{4.379573in}{5.736077in}}%
\pgfpathlineto{\pgfqpoint{4.384286in}{5.742667in}}%
\pgfpathlineto{\pgfqpoint{4.388999in}{5.742709in}}%
\pgfpathlineto{\pgfqpoint{4.393712in}{5.744583in}}%
\pgfpathlineto{\pgfqpoint{4.398425in}{5.744848in}}%
\pgfpathlineto{\pgfqpoint{4.403138in}{5.752797in}}%
\pgfpathlineto{\pgfqpoint{4.407851in}{5.753141in}}%
\pgfpathlineto{\pgfqpoint{4.412564in}{5.757406in}}%
\pgfpathlineto{\pgfqpoint{4.417277in}{5.770061in}}%
\pgfpathlineto{\pgfqpoint{4.421990in}{5.870479in}}%
\pgfpathlineto{\pgfqpoint{4.426702in}{5.905275in}}%
\pgfpathlineto{\pgfqpoint{5.845287in}{5.905275in}}%
\pgfpathlineto{\pgfqpoint{5.845287in}{5.905275in}}%
\pgfusepath{stroke}%
\end{pgfscope}%
\begin{pgfscope}%
\pgfsetrectcap%
\pgfsetmiterjoin%
\pgfsetlinewidth{0.803000pt}%
\definecolor{currentstroke}{rgb}{0.000000,0.000000,0.000000}%
\pgfsetstrokecolor{currentstroke}%
\pgfsetdash{}{0pt}%
\pgfpathmoveto{\pgfqpoint{0.708220in}{4.502489in}}%
\pgfpathlineto{\pgfqpoint{0.708220in}{5.905275in}}%
\pgfusepath{stroke}%
\end{pgfscope}%
\begin{pgfscope}%
\pgfsetrectcap%
\pgfsetmiterjoin%
\pgfsetlinewidth{0.803000pt}%
\definecolor{currentstroke}{rgb}{0.000000,0.000000,0.000000}%
\pgfsetstrokecolor{currentstroke}%
\pgfsetdash{}{0pt}%
\pgfpathmoveto{\pgfqpoint{5.850000in}{4.502489in}}%
\pgfpathlineto{\pgfqpoint{5.850000in}{5.905275in}}%
\pgfusepath{stroke}%
\end{pgfscope}%
\begin{pgfscope}%
\pgfsetrectcap%
\pgfsetmiterjoin%
\pgfsetlinewidth{0.803000pt}%
\definecolor{currentstroke}{rgb}{0.000000,0.000000,0.000000}%
\pgfsetstrokecolor{currentstroke}%
\pgfsetdash{}{0pt}%
\pgfpathmoveto{\pgfqpoint{0.708220in}{4.502489in}}%
\pgfpathlineto{\pgfqpoint{5.850000in}{4.502489in}}%
\pgfusepath{stroke}%
\end{pgfscope}%
\begin{pgfscope}%
\pgfsetrectcap%
\pgfsetmiterjoin%
\pgfsetlinewidth{0.803000pt}%
\definecolor{currentstroke}{rgb}{0.000000,0.000000,0.000000}%
\pgfsetstrokecolor{currentstroke}%
\pgfsetdash{}{0pt}%
\pgfpathmoveto{\pgfqpoint{0.708220in}{5.905275in}}%
\pgfpathlineto{\pgfqpoint{5.850000in}{5.905275in}}%
\pgfusepath{stroke}%
\end{pgfscope}%
\begin{pgfscope}%
\pgfsetbuttcap%
\pgfsetroundjoin%
\pgfsetlinewidth{2.007500pt}%
\definecolor{currentstroke}{rgb}{1.000000,0.843137,0.000000}%
\pgfsetstrokecolor{currentstroke}%
\pgfsetdash{{7.400000pt}{3.200000pt}}{0.000000pt}%
\pgfpathmoveto{\pgfqpoint{4.827505in}{4.944138in}}%
\pgfpathlineto{\pgfqpoint{5.077505in}{4.944138in}}%
\pgfusepath{stroke}%
\end{pgfscope}%
\begin{pgfscope}%
\definecolor{textcolor}{rgb}{0.000000,0.000000,0.000000}%
\pgfsetstrokecolor{textcolor}%
\pgfsetfillcolor{textcolor}%
\pgftext[x=5.102505in,y=4.900388in,left,base]{\color{textcolor}\rmfamily\fontsize{9.000000}{10.800000}\selectfont FT+htd}%
\end{pgfscope}%
\begin{pgfscope}%
\pgfsetbuttcap%
\pgfsetroundjoin%
\pgfsetlinewidth{2.007500pt}%
\definecolor{currentstroke}{rgb}{1.000000,0.694118,0.305882}%
\pgfsetstrokecolor{currentstroke}%
\pgfsetdash{{2.000000pt}{3.300000pt}}{0.000000pt}%
\pgfpathmoveto{\pgfqpoint{4.827505in}{4.782338in}}%
\pgfpathlineto{\pgfqpoint{5.077505in}{4.782338in}}%
\pgfusepath{stroke}%
\end{pgfscope}%
\begin{pgfscope}%
\definecolor{textcolor}{rgb}{0.000000,0.000000,0.000000}%
\pgfsetstrokecolor{textcolor}%
\pgfsetfillcolor{textcolor}%
\pgftext[x=5.102505in,y=4.738588in,left,base]{\color{textcolor}\rmfamily\fontsize{9.000000}{10.800000}\selectfont FT+Flow}%
\end{pgfscope}%
\begin{pgfscope}%
\pgfsetrectcap%
\pgfsetroundjoin%
\pgfsetlinewidth{2.007500pt}%
\definecolor{currentstroke}{rgb}{0.980392,0.529412,0.458824}%
\pgfsetstrokecolor{currentstroke}%
\pgfsetdash{}{0pt}%
\pgfpathmoveto{\pgfqpoint{4.827505in}{4.620539in}}%
\pgfpathlineto{\pgfqpoint{5.077505in}{4.620539in}}%
\pgfusepath{stroke}%
\end{pgfscope}%
\begin{pgfscope}%
\definecolor{textcolor}{rgb}{0.000000,0.000000,0.000000}%
\pgfsetstrokecolor{textcolor}%
\pgfsetfillcolor{textcolor}%
\pgftext[x=5.102505in,y=4.576789in,left,base]{\color{textcolor}\rmfamily\fontsize{9.000000}{10.800000}\selectfont FT+Tamaki}%
\end{pgfscope}%
\begin{pgfscope}%
\pgfsetbuttcap%
\pgfsetmiterjoin%
\definecolor{currentfill}{rgb}{1.000000,1.000000,1.000000}%
\pgfsetfillcolor{currentfill}%
\pgfsetlinewidth{0.000000pt}%
\definecolor{currentstroke}{rgb}{0.000000,0.000000,0.000000}%
\pgfsetstrokecolor{currentstroke}%
\pgfsetstrokeopacity{0.000000}%
\pgfsetdash{}{0pt}%
\pgfpathmoveto{\pgfqpoint{0.708220in}{2.519156in}}%
\pgfpathlineto{\pgfqpoint{5.850000in}{2.519156in}}%
\pgfpathlineto{\pgfqpoint{5.850000in}{3.921942in}}%
\pgfpathlineto{\pgfqpoint{0.708220in}{3.921942in}}%
\pgfpathclose%
\pgfusepath{fill}%
\end{pgfscope}%
\begin{pgfscope}%
\pgfsetbuttcap%
\pgfsetroundjoin%
\definecolor{currentfill}{rgb}{0.000000,0.000000,0.000000}%
\pgfsetfillcolor{currentfill}%
\pgfsetlinewidth{0.803000pt}%
\definecolor{currentstroke}{rgb}{0.000000,0.000000,0.000000}%
\pgfsetstrokecolor{currentstroke}%
\pgfsetdash{}{0pt}%
\pgfsys@defobject{currentmarker}{\pgfqpoint{0.000000in}{-0.048611in}}{\pgfqpoint{0.000000in}{0.000000in}}{%
\pgfpathmoveto{\pgfqpoint{0.000000in}{0.000000in}}%
\pgfpathlineto{\pgfqpoint{0.000000in}{-0.048611in}}%
\pgfusepath{stroke,fill}%
}%
\begin{pgfscope}%
\pgfsys@transformshift{0.708220in}{2.519156in}%
\pgfsys@useobject{currentmarker}{}%
\end{pgfscope}%
\end{pgfscope}%
\begin{pgfscope}%
\definecolor{textcolor}{rgb}{0.000000,0.000000,0.000000}%
\pgfsetstrokecolor{textcolor}%
\pgfsetfillcolor{textcolor}%
\pgftext[x=0.708220in,y=2.421934in,,top]{\color{textcolor}\rmfamily\fontsize{9.000000}{10.800000}\selectfont \(\displaystyle {0}\)}%
\end{pgfscope}%
\begin{pgfscope}%
\pgfsetbuttcap%
\pgfsetroundjoin%
\definecolor{currentfill}{rgb}{0.000000,0.000000,0.000000}%
\pgfsetfillcolor{currentfill}%
\pgfsetlinewidth{0.803000pt}%
\definecolor{currentstroke}{rgb}{0.000000,0.000000,0.000000}%
\pgfsetstrokecolor{currentstroke}%
\pgfsetdash{}{0pt}%
\pgfsys@defobject{currentmarker}{\pgfqpoint{0.000000in}{-0.048611in}}{\pgfqpoint{0.000000in}{0.000000in}}{%
\pgfpathmoveto{\pgfqpoint{0.000000in}{0.000000in}}%
\pgfpathlineto{\pgfqpoint{0.000000in}{-0.048611in}}%
\pgfusepath{stroke,fill}%
}%
\begin{pgfscope}%
\pgfsys@transformshift{1.650801in}{2.519156in}%
\pgfsys@useobject{currentmarker}{}%
\end{pgfscope}%
\end{pgfscope}%
\begin{pgfscope}%
\definecolor{textcolor}{rgb}{0.000000,0.000000,0.000000}%
\pgfsetstrokecolor{textcolor}%
\pgfsetfillcolor{textcolor}%
\pgftext[x=1.650801in,y=2.421934in,,top]{\color{textcolor}\rmfamily\fontsize{9.000000}{10.800000}\selectfont \(\displaystyle {200}\)}%
\end{pgfscope}%
\begin{pgfscope}%
\pgfsetbuttcap%
\pgfsetroundjoin%
\definecolor{currentfill}{rgb}{0.000000,0.000000,0.000000}%
\pgfsetfillcolor{currentfill}%
\pgfsetlinewidth{0.803000pt}%
\definecolor{currentstroke}{rgb}{0.000000,0.000000,0.000000}%
\pgfsetstrokecolor{currentstroke}%
\pgfsetdash{}{0pt}%
\pgfsys@defobject{currentmarker}{\pgfqpoint{0.000000in}{-0.048611in}}{\pgfqpoint{0.000000in}{0.000000in}}{%
\pgfpathmoveto{\pgfqpoint{0.000000in}{0.000000in}}%
\pgfpathlineto{\pgfqpoint{0.000000in}{-0.048611in}}%
\pgfusepath{stroke,fill}%
}%
\begin{pgfscope}%
\pgfsys@transformshift{2.593382in}{2.519156in}%
\pgfsys@useobject{currentmarker}{}%
\end{pgfscope}%
\end{pgfscope}%
\begin{pgfscope}%
\definecolor{textcolor}{rgb}{0.000000,0.000000,0.000000}%
\pgfsetstrokecolor{textcolor}%
\pgfsetfillcolor{textcolor}%
\pgftext[x=2.593382in,y=2.421934in,,top]{\color{textcolor}\rmfamily\fontsize{9.000000}{10.800000}\selectfont \(\displaystyle {400}\)}%
\end{pgfscope}%
\begin{pgfscope}%
\pgfsetbuttcap%
\pgfsetroundjoin%
\definecolor{currentfill}{rgb}{0.000000,0.000000,0.000000}%
\pgfsetfillcolor{currentfill}%
\pgfsetlinewidth{0.803000pt}%
\definecolor{currentstroke}{rgb}{0.000000,0.000000,0.000000}%
\pgfsetstrokecolor{currentstroke}%
\pgfsetdash{}{0pt}%
\pgfsys@defobject{currentmarker}{\pgfqpoint{0.000000in}{-0.048611in}}{\pgfqpoint{0.000000in}{0.000000in}}{%
\pgfpathmoveto{\pgfqpoint{0.000000in}{0.000000in}}%
\pgfpathlineto{\pgfqpoint{0.000000in}{-0.048611in}}%
\pgfusepath{stroke,fill}%
}%
\begin{pgfscope}%
\pgfsys@transformshift{3.535963in}{2.519156in}%
\pgfsys@useobject{currentmarker}{}%
\end{pgfscope}%
\end{pgfscope}%
\begin{pgfscope}%
\definecolor{textcolor}{rgb}{0.000000,0.000000,0.000000}%
\pgfsetstrokecolor{textcolor}%
\pgfsetfillcolor{textcolor}%
\pgftext[x=3.535963in,y=2.421934in,,top]{\color{textcolor}\rmfamily\fontsize{9.000000}{10.800000}\selectfont \(\displaystyle {600}\)}%
\end{pgfscope}%
\begin{pgfscope}%
\pgfsetbuttcap%
\pgfsetroundjoin%
\definecolor{currentfill}{rgb}{0.000000,0.000000,0.000000}%
\pgfsetfillcolor{currentfill}%
\pgfsetlinewidth{0.803000pt}%
\definecolor{currentstroke}{rgb}{0.000000,0.000000,0.000000}%
\pgfsetstrokecolor{currentstroke}%
\pgfsetdash{}{0pt}%
\pgfsys@defobject{currentmarker}{\pgfqpoint{0.000000in}{-0.048611in}}{\pgfqpoint{0.000000in}{0.000000in}}{%
\pgfpathmoveto{\pgfqpoint{0.000000in}{0.000000in}}%
\pgfpathlineto{\pgfqpoint{0.000000in}{-0.048611in}}%
\pgfusepath{stroke,fill}%
}%
\begin{pgfscope}%
\pgfsys@transformshift{4.478544in}{2.519156in}%
\pgfsys@useobject{currentmarker}{}%
\end{pgfscope}%
\end{pgfscope}%
\begin{pgfscope}%
\definecolor{textcolor}{rgb}{0.000000,0.000000,0.000000}%
\pgfsetstrokecolor{textcolor}%
\pgfsetfillcolor{textcolor}%
\pgftext[x=4.478544in,y=2.421934in,,top]{\color{textcolor}\rmfamily\fontsize{9.000000}{10.800000}\selectfont \(\displaystyle {800}\)}%
\end{pgfscope}%
\begin{pgfscope}%
\pgfsetbuttcap%
\pgfsetroundjoin%
\definecolor{currentfill}{rgb}{0.000000,0.000000,0.000000}%
\pgfsetfillcolor{currentfill}%
\pgfsetlinewidth{0.803000pt}%
\definecolor{currentstroke}{rgb}{0.000000,0.000000,0.000000}%
\pgfsetstrokecolor{currentstroke}%
\pgfsetdash{}{0pt}%
\pgfsys@defobject{currentmarker}{\pgfqpoint{0.000000in}{-0.048611in}}{\pgfqpoint{0.000000in}{0.000000in}}{%
\pgfpathmoveto{\pgfqpoint{0.000000in}{0.000000in}}%
\pgfpathlineto{\pgfqpoint{0.000000in}{-0.048611in}}%
\pgfusepath{stroke,fill}%
}%
\begin{pgfscope}%
\pgfsys@transformshift{5.421126in}{2.519156in}%
\pgfsys@useobject{currentmarker}{}%
\end{pgfscope}%
\end{pgfscope}%
\begin{pgfscope}%
\definecolor{textcolor}{rgb}{0.000000,0.000000,0.000000}%
\pgfsetstrokecolor{textcolor}%
\pgfsetfillcolor{textcolor}%
\pgftext[x=5.421126in,y=2.421934in,,top]{\color{textcolor}\rmfamily\fontsize{9.000000}{10.800000}\selectfont \(\displaystyle {1000}\)}%
\end{pgfscope}%
\begin{pgfscope}%
\definecolor{textcolor}{rgb}{0.000000,0.000000,0.000000}%
\pgfsetstrokecolor{textcolor}%
\pgfsetfillcolor{textcolor}%
\pgftext[x=3.279110in,y=2.255988in,,top]{\color{textcolor}\rmfamily\fontsize{10.000000}{12.000000}\selectfont Number of benchmarks solved}%
\end{pgfscope}%
\begin{pgfscope}%
\pgfsetbuttcap%
\pgfsetroundjoin%
\definecolor{currentfill}{rgb}{0.000000,0.000000,0.000000}%
\pgfsetfillcolor{currentfill}%
\pgfsetlinewidth{0.803000pt}%
\definecolor{currentstroke}{rgb}{0.000000,0.000000,0.000000}%
\pgfsetstrokecolor{currentstroke}%
\pgfsetdash{}{0pt}%
\pgfsys@defobject{currentmarker}{\pgfqpoint{-0.048611in}{0.000000in}}{\pgfqpoint{-0.000000in}{0.000000in}}{%
\pgfpathmoveto{\pgfqpoint{-0.000000in}{0.000000in}}%
\pgfpathlineto{\pgfqpoint{-0.048611in}{0.000000in}}%
\pgfusepath{stroke,fill}%
}%
\begin{pgfscope}%
\pgfsys@transformshift{0.708220in}{2.519156in}%
\pgfsys@useobject{currentmarker}{}%
\end{pgfscope}%
\end{pgfscope}%
\begin{pgfscope}%
\definecolor{textcolor}{rgb}{0.000000,0.000000,0.000000}%
\pgfsetstrokecolor{textcolor}%
\pgfsetfillcolor{textcolor}%
\pgftext[x=0.344411in, y=2.474431in, left, base]{\color{textcolor}\rmfamily\fontsize{9.000000}{10.800000}\selectfont \(\displaystyle {10^{-1}}\)}%
\end{pgfscope}%
\begin{pgfscope}%
\pgfsetbuttcap%
\pgfsetroundjoin%
\definecolor{currentfill}{rgb}{0.000000,0.000000,0.000000}%
\pgfsetfillcolor{currentfill}%
\pgfsetlinewidth{0.803000pt}%
\definecolor{currentstroke}{rgb}{0.000000,0.000000,0.000000}%
\pgfsetstrokecolor{currentstroke}%
\pgfsetdash{}{0pt}%
\pgfsys@defobject{currentmarker}{\pgfqpoint{-0.048611in}{0.000000in}}{\pgfqpoint{-0.000000in}{0.000000in}}{%
\pgfpathmoveto{\pgfqpoint{-0.000000in}{0.000000in}}%
\pgfpathlineto{\pgfqpoint{-0.048611in}{0.000000in}}%
\pgfusepath{stroke,fill}%
}%
\begin{pgfscope}%
\pgfsys@transformshift{0.708220in}{2.869852in}%
\pgfsys@useobject{currentmarker}{}%
\end{pgfscope}%
\end{pgfscope}%
\begin{pgfscope}%
\definecolor{textcolor}{rgb}{0.000000,0.000000,0.000000}%
\pgfsetstrokecolor{textcolor}%
\pgfsetfillcolor{textcolor}%
\pgftext[x=0.424657in, y=2.825128in, left, base]{\color{textcolor}\rmfamily\fontsize{9.000000}{10.800000}\selectfont \(\displaystyle {10^{0}}\)}%
\end{pgfscope}%
\begin{pgfscope}%
\pgfsetbuttcap%
\pgfsetroundjoin%
\definecolor{currentfill}{rgb}{0.000000,0.000000,0.000000}%
\pgfsetfillcolor{currentfill}%
\pgfsetlinewidth{0.803000pt}%
\definecolor{currentstroke}{rgb}{0.000000,0.000000,0.000000}%
\pgfsetstrokecolor{currentstroke}%
\pgfsetdash{}{0pt}%
\pgfsys@defobject{currentmarker}{\pgfqpoint{-0.048611in}{0.000000in}}{\pgfqpoint{-0.000000in}{0.000000in}}{%
\pgfpathmoveto{\pgfqpoint{-0.000000in}{0.000000in}}%
\pgfpathlineto{\pgfqpoint{-0.048611in}{0.000000in}}%
\pgfusepath{stroke,fill}%
}%
\begin{pgfscope}%
\pgfsys@transformshift{0.708220in}{3.220549in}%
\pgfsys@useobject{currentmarker}{}%
\end{pgfscope}%
\end{pgfscope}%
\begin{pgfscope}%
\definecolor{textcolor}{rgb}{0.000000,0.000000,0.000000}%
\pgfsetstrokecolor{textcolor}%
\pgfsetfillcolor{textcolor}%
\pgftext[x=0.424657in, y=3.175824in, left, base]{\color{textcolor}\rmfamily\fontsize{9.000000}{10.800000}\selectfont \(\displaystyle {10^{1}}\)}%
\end{pgfscope}%
\begin{pgfscope}%
\pgfsetbuttcap%
\pgfsetroundjoin%
\definecolor{currentfill}{rgb}{0.000000,0.000000,0.000000}%
\pgfsetfillcolor{currentfill}%
\pgfsetlinewidth{0.803000pt}%
\definecolor{currentstroke}{rgb}{0.000000,0.000000,0.000000}%
\pgfsetstrokecolor{currentstroke}%
\pgfsetdash{}{0pt}%
\pgfsys@defobject{currentmarker}{\pgfqpoint{-0.048611in}{0.000000in}}{\pgfqpoint{-0.000000in}{0.000000in}}{%
\pgfpathmoveto{\pgfqpoint{-0.000000in}{0.000000in}}%
\pgfpathlineto{\pgfqpoint{-0.048611in}{0.000000in}}%
\pgfusepath{stroke,fill}%
}%
\begin{pgfscope}%
\pgfsys@transformshift{0.708220in}{3.571245in}%
\pgfsys@useobject{currentmarker}{}%
\end{pgfscope}%
\end{pgfscope}%
\begin{pgfscope}%
\definecolor{textcolor}{rgb}{0.000000,0.000000,0.000000}%
\pgfsetstrokecolor{textcolor}%
\pgfsetfillcolor{textcolor}%
\pgftext[x=0.424657in, y=3.526521in, left, base]{\color{textcolor}\rmfamily\fontsize{9.000000}{10.800000}\selectfont \(\displaystyle {10^{2}}\)}%
\end{pgfscope}%
\begin{pgfscope}%
\pgfsetbuttcap%
\pgfsetroundjoin%
\definecolor{currentfill}{rgb}{0.000000,0.000000,0.000000}%
\pgfsetfillcolor{currentfill}%
\pgfsetlinewidth{0.803000pt}%
\definecolor{currentstroke}{rgb}{0.000000,0.000000,0.000000}%
\pgfsetstrokecolor{currentstroke}%
\pgfsetdash{}{0pt}%
\pgfsys@defobject{currentmarker}{\pgfqpoint{-0.048611in}{0.000000in}}{\pgfqpoint{-0.000000in}{0.000000in}}{%
\pgfpathmoveto{\pgfqpoint{-0.000000in}{0.000000in}}%
\pgfpathlineto{\pgfqpoint{-0.048611in}{0.000000in}}%
\pgfusepath{stroke,fill}%
}%
\begin{pgfscope}%
\pgfsys@transformshift{0.708220in}{3.921942in}%
\pgfsys@useobject{currentmarker}{}%
\end{pgfscope}%
\end{pgfscope}%
\begin{pgfscope}%
\definecolor{textcolor}{rgb}{0.000000,0.000000,0.000000}%
\pgfsetstrokecolor{textcolor}%
\pgfsetfillcolor{textcolor}%
\pgftext[x=0.424657in, y=3.877217in, left, base]{\color{textcolor}\rmfamily\fontsize{9.000000}{10.800000}\selectfont \(\displaystyle {10^{3}}\)}%
\end{pgfscope}%
\begin{pgfscope}%
\pgfsetbuttcap%
\pgfsetroundjoin%
\definecolor{currentfill}{rgb}{0.000000,0.000000,0.000000}%
\pgfsetfillcolor{currentfill}%
\pgfsetlinewidth{0.602250pt}%
\definecolor{currentstroke}{rgb}{0.000000,0.000000,0.000000}%
\pgfsetstrokecolor{currentstroke}%
\pgfsetdash{}{0pt}%
\pgfsys@defobject{currentmarker}{\pgfqpoint{-0.027778in}{0.000000in}}{\pgfqpoint{-0.000000in}{0.000000in}}{%
\pgfpathmoveto{\pgfqpoint{-0.000000in}{0.000000in}}%
\pgfpathlineto{\pgfqpoint{-0.027778in}{0.000000in}}%
\pgfusepath{stroke,fill}%
}%
\begin{pgfscope}%
\pgfsys@transformshift{0.708220in}{2.624726in}%
\pgfsys@useobject{currentmarker}{}%
\end{pgfscope}%
\end{pgfscope}%
\begin{pgfscope}%
\pgfsetbuttcap%
\pgfsetroundjoin%
\definecolor{currentfill}{rgb}{0.000000,0.000000,0.000000}%
\pgfsetfillcolor{currentfill}%
\pgfsetlinewidth{0.602250pt}%
\definecolor{currentstroke}{rgb}{0.000000,0.000000,0.000000}%
\pgfsetstrokecolor{currentstroke}%
\pgfsetdash{}{0pt}%
\pgfsys@defobject{currentmarker}{\pgfqpoint{-0.027778in}{0.000000in}}{\pgfqpoint{-0.000000in}{0.000000in}}{%
\pgfpathmoveto{\pgfqpoint{-0.000000in}{0.000000in}}%
\pgfpathlineto{\pgfqpoint{-0.027778in}{0.000000in}}%
\pgfusepath{stroke,fill}%
}%
\begin{pgfscope}%
\pgfsys@transformshift{0.708220in}{2.686481in}%
\pgfsys@useobject{currentmarker}{}%
\end{pgfscope}%
\end{pgfscope}%
\begin{pgfscope}%
\pgfsetbuttcap%
\pgfsetroundjoin%
\definecolor{currentfill}{rgb}{0.000000,0.000000,0.000000}%
\pgfsetfillcolor{currentfill}%
\pgfsetlinewidth{0.602250pt}%
\definecolor{currentstroke}{rgb}{0.000000,0.000000,0.000000}%
\pgfsetstrokecolor{currentstroke}%
\pgfsetdash{}{0pt}%
\pgfsys@defobject{currentmarker}{\pgfqpoint{-0.027778in}{0.000000in}}{\pgfqpoint{-0.000000in}{0.000000in}}{%
\pgfpathmoveto{\pgfqpoint{-0.000000in}{0.000000in}}%
\pgfpathlineto{\pgfqpoint{-0.027778in}{0.000000in}}%
\pgfusepath{stroke,fill}%
}%
\begin{pgfscope}%
\pgfsys@transformshift{0.708220in}{2.730296in}%
\pgfsys@useobject{currentmarker}{}%
\end{pgfscope}%
\end{pgfscope}%
\begin{pgfscope}%
\pgfsetbuttcap%
\pgfsetroundjoin%
\definecolor{currentfill}{rgb}{0.000000,0.000000,0.000000}%
\pgfsetfillcolor{currentfill}%
\pgfsetlinewidth{0.602250pt}%
\definecolor{currentstroke}{rgb}{0.000000,0.000000,0.000000}%
\pgfsetstrokecolor{currentstroke}%
\pgfsetdash{}{0pt}%
\pgfsys@defobject{currentmarker}{\pgfqpoint{-0.027778in}{0.000000in}}{\pgfqpoint{-0.000000in}{0.000000in}}{%
\pgfpathmoveto{\pgfqpoint{-0.000000in}{0.000000in}}%
\pgfpathlineto{\pgfqpoint{-0.027778in}{0.000000in}}%
\pgfusepath{stroke,fill}%
}%
\begin{pgfscope}%
\pgfsys@transformshift{0.708220in}{2.764282in}%
\pgfsys@useobject{currentmarker}{}%
\end{pgfscope}%
\end{pgfscope}%
\begin{pgfscope}%
\pgfsetbuttcap%
\pgfsetroundjoin%
\definecolor{currentfill}{rgb}{0.000000,0.000000,0.000000}%
\pgfsetfillcolor{currentfill}%
\pgfsetlinewidth{0.602250pt}%
\definecolor{currentstroke}{rgb}{0.000000,0.000000,0.000000}%
\pgfsetstrokecolor{currentstroke}%
\pgfsetdash{}{0pt}%
\pgfsys@defobject{currentmarker}{\pgfqpoint{-0.027778in}{0.000000in}}{\pgfqpoint{-0.000000in}{0.000000in}}{%
\pgfpathmoveto{\pgfqpoint{-0.000000in}{0.000000in}}%
\pgfpathlineto{\pgfqpoint{-0.027778in}{0.000000in}}%
\pgfusepath{stroke,fill}%
}%
\begin{pgfscope}%
\pgfsys@transformshift{0.708220in}{2.792051in}%
\pgfsys@useobject{currentmarker}{}%
\end{pgfscope}%
\end{pgfscope}%
\begin{pgfscope}%
\pgfsetbuttcap%
\pgfsetroundjoin%
\definecolor{currentfill}{rgb}{0.000000,0.000000,0.000000}%
\pgfsetfillcolor{currentfill}%
\pgfsetlinewidth{0.602250pt}%
\definecolor{currentstroke}{rgb}{0.000000,0.000000,0.000000}%
\pgfsetstrokecolor{currentstroke}%
\pgfsetdash{}{0pt}%
\pgfsys@defobject{currentmarker}{\pgfqpoint{-0.027778in}{0.000000in}}{\pgfqpoint{-0.000000in}{0.000000in}}{%
\pgfpathmoveto{\pgfqpoint{-0.000000in}{0.000000in}}%
\pgfpathlineto{\pgfqpoint{-0.027778in}{0.000000in}}%
\pgfusepath{stroke,fill}%
}%
\begin{pgfscope}%
\pgfsys@transformshift{0.708220in}{2.815529in}%
\pgfsys@useobject{currentmarker}{}%
\end{pgfscope}%
\end{pgfscope}%
\begin{pgfscope}%
\pgfsetbuttcap%
\pgfsetroundjoin%
\definecolor{currentfill}{rgb}{0.000000,0.000000,0.000000}%
\pgfsetfillcolor{currentfill}%
\pgfsetlinewidth{0.602250pt}%
\definecolor{currentstroke}{rgb}{0.000000,0.000000,0.000000}%
\pgfsetstrokecolor{currentstroke}%
\pgfsetdash{}{0pt}%
\pgfsys@defobject{currentmarker}{\pgfqpoint{-0.027778in}{0.000000in}}{\pgfqpoint{-0.000000in}{0.000000in}}{%
\pgfpathmoveto{\pgfqpoint{-0.000000in}{0.000000in}}%
\pgfpathlineto{\pgfqpoint{-0.027778in}{0.000000in}}%
\pgfusepath{stroke,fill}%
}%
\begin{pgfscope}%
\pgfsys@transformshift{0.708220in}{2.835866in}%
\pgfsys@useobject{currentmarker}{}%
\end{pgfscope}%
\end{pgfscope}%
\begin{pgfscope}%
\pgfsetbuttcap%
\pgfsetroundjoin%
\definecolor{currentfill}{rgb}{0.000000,0.000000,0.000000}%
\pgfsetfillcolor{currentfill}%
\pgfsetlinewidth{0.602250pt}%
\definecolor{currentstroke}{rgb}{0.000000,0.000000,0.000000}%
\pgfsetstrokecolor{currentstroke}%
\pgfsetdash{}{0pt}%
\pgfsys@defobject{currentmarker}{\pgfqpoint{-0.027778in}{0.000000in}}{\pgfqpoint{-0.000000in}{0.000000in}}{%
\pgfpathmoveto{\pgfqpoint{-0.000000in}{0.000000in}}%
\pgfpathlineto{\pgfqpoint{-0.027778in}{0.000000in}}%
\pgfusepath{stroke,fill}%
}%
\begin{pgfscope}%
\pgfsys@transformshift{0.708220in}{2.853805in}%
\pgfsys@useobject{currentmarker}{}%
\end{pgfscope}%
\end{pgfscope}%
\begin{pgfscope}%
\pgfsetbuttcap%
\pgfsetroundjoin%
\definecolor{currentfill}{rgb}{0.000000,0.000000,0.000000}%
\pgfsetfillcolor{currentfill}%
\pgfsetlinewidth{0.602250pt}%
\definecolor{currentstroke}{rgb}{0.000000,0.000000,0.000000}%
\pgfsetstrokecolor{currentstroke}%
\pgfsetdash{}{0pt}%
\pgfsys@defobject{currentmarker}{\pgfqpoint{-0.027778in}{0.000000in}}{\pgfqpoint{-0.000000in}{0.000000in}}{%
\pgfpathmoveto{\pgfqpoint{-0.000000in}{0.000000in}}%
\pgfpathlineto{\pgfqpoint{-0.027778in}{0.000000in}}%
\pgfusepath{stroke,fill}%
}%
\begin{pgfscope}%
\pgfsys@transformshift{0.708220in}{2.975423in}%
\pgfsys@useobject{currentmarker}{}%
\end{pgfscope}%
\end{pgfscope}%
\begin{pgfscope}%
\pgfsetbuttcap%
\pgfsetroundjoin%
\definecolor{currentfill}{rgb}{0.000000,0.000000,0.000000}%
\pgfsetfillcolor{currentfill}%
\pgfsetlinewidth{0.602250pt}%
\definecolor{currentstroke}{rgb}{0.000000,0.000000,0.000000}%
\pgfsetstrokecolor{currentstroke}%
\pgfsetdash{}{0pt}%
\pgfsys@defobject{currentmarker}{\pgfqpoint{-0.027778in}{0.000000in}}{\pgfqpoint{-0.000000in}{0.000000in}}{%
\pgfpathmoveto{\pgfqpoint{-0.000000in}{0.000000in}}%
\pgfpathlineto{\pgfqpoint{-0.027778in}{0.000000in}}%
\pgfusepath{stroke,fill}%
}%
\begin{pgfscope}%
\pgfsys@transformshift{0.708220in}{3.037177in}%
\pgfsys@useobject{currentmarker}{}%
\end{pgfscope}%
\end{pgfscope}%
\begin{pgfscope}%
\pgfsetbuttcap%
\pgfsetroundjoin%
\definecolor{currentfill}{rgb}{0.000000,0.000000,0.000000}%
\pgfsetfillcolor{currentfill}%
\pgfsetlinewidth{0.602250pt}%
\definecolor{currentstroke}{rgb}{0.000000,0.000000,0.000000}%
\pgfsetstrokecolor{currentstroke}%
\pgfsetdash{}{0pt}%
\pgfsys@defobject{currentmarker}{\pgfqpoint{-0.027778in}{0.000000in}}{\pgfqpoint{-0.000000in}{0.000000in}}{%
\pgfpathmoveto{\pgfqpoint{-0.000000in}{0.000000in}}%
\pgfpathlineto{\pgfqpoint{-0.027778in}{0.000000in}}%
\pgfusepath{stroke,fill}%
}%
\begin{pgfscope}%
\pgfsys@transformshift{0.708220in}{3.080993in}%
\pgfsys@useobject{currentmarker}{}%
\end{pgfscope}%
\end{pgfscope}%
\begin{pgfscope}%
\pgfsetbuttcap%
\pgfsetroundjoin%
\definecolor{currentfill}{rgb}{0.000000,0.000000,0.000000}%
\pgfsetfillcolor{currentfill}%
\pgfsetlinewidth{0.602250pt}%
\definecolor{currentstroke}{rgb}{0.000000,0.000000,0.000000}%
\pgfsetstrokecolor{currentstroke}%
\pgfsetdash{}{0pt}%
\pgfsys@defobject{currentmarker}{\pgfqpoint{-0.027778in}{0.000000in}}{\pgfqpoint{-0.000000in}{0.000000in}}{%
\pgfpathmoveto{\pgfqpoint{-0.000000in}{0.000000in}}%
\pgfpathlineto{\pgfqpoint{-0.027778in}{0.000000in}}%
\pgfusepath{stroke,fill}%
}%
\begin{pgfscope}%
\pgfsys@transformshift{0.708220in}{3.114979in}%
\pgfsys@useobject{currentmarker}{}%
\end{pgfscope}%
\end{pgfscope}%
\begin{pgfscope}%
\pgfsetbuttcap%
\pgfsetroundjoin%
\definecolor{currentfill}{rgb}{0.000000,0.000000,0.000000}%
\pgfsetfillcolor{currentfill}%
\pgfsetlinewidth{0.602250pt}%
\definecolor{currentstroke}{rgb}{0.000000,0.000000,0.000000}%
\pgfsetstrokecolor{currentstroke}%
\pgfsetdash{}{0pt}%
\pgfsys@defobject{currentmarker}{\pgfqpoint{-0.027778in}{0.000000in}}{\pgfqpoint{-0.000000in}{0.000000in}}{%
\pgfpathmoveto{\pgfqpoint{-0.000000in}{0.000000in}}%
\pgfpathlineto{\pgfqpoint{-0.027778in}{0.000000in}}%
\pgfusepath{stroke,fill}%
}%
\begin{pgfscope}%
\pgfsys@transformshift{0.708220in}{3.142747in}%
\pgfsys@useobject{currentmarker}{}%
\end{pgfscope}%
\end{pgfscope}%
\begin{pgfscope}%
\pgfsetbuttcap%
\pgfsetroundjoin%
\definecolor{currentfill}{rgb}{0.000000,0.000000,0.000000}%
\pgfsetfillcolor{currentfill}%
\pgfsetlinewidth{0.602250pt}%
\definecolor{currentstroke}{rgb}{0.000000,0.000000,0.000000}%
\pgfsetstrokecolor{currentstroke}%
\pgfsetdash{}{0pt}%
\pgfsys@defobject{currentmarker}{\pgfqpoint{-0.027778in}{0.000000in}}{\pgfqpoint{-0.000000in}{0.000000in}}{%
\pgfpathmoveto{\pgfqpoint{-0.000000in}{0.000000in}}%
\pgfpathlineto{\pgfqpoint{-0.027778in}{0.000000in}}%
\pgfusepath{stroke,fill}%
}%
\begin{pgfscope}%
\pgfsys@transformshift{0.708220in}{3.166225in}%
\pgfsys@useobject{currentmarker}{}%
\end{pgfscope}%
\end{pgfscope}%
\begin{pgfscope}%
\pgfsetbuttcap%
\pgfsetroundjoin%
\definecolor{currentfill}{rgb}{0.000000,0.000000,0.000000}%
\pgfsetfillcolor{currentfill}%
\pgfsetlinewidth{0.602250pt}%
\definecolor{currentstroke}{rgb}{0.000000,0.000000,0.000000}%
\pgfsetstrokecolor{currentstroke}%
\pgfsetdash{}{0pt}%
\pgfsys@defobject{currentmarker}{\pgfqpoint{-0.027778in}{0.000000in}}{\pgfqpoint{-0.000000in}{0.000000in}}{%
\pgfpathmoveto{\pgfqpoint{-0.000000in}{0.000000in}}%
\pgfpathlineto{\pgfqpoint{-0.027778in}{0.000000in}}%
\pgfusepath{stroke,fill}%
}%
\begin{pgfscope}%
\pgfsys@transformshift{0.708220in}{3.186563in}%
\pgfsys@useobject{currentmarker}{}%
\end{pgfscope}%
\end{pgfscope}%
\begin{pgfscope}%
\pgfsetbuttcap%
\pgfsetroundjoin%
\definecolor{currentfill}{rgb}{0.000000,0.000000,0.000000}%
\pgfsetfillcolor{currentfill}%
\pgfsetlinewidth{0.602250pt}%
\definecolor{currentstroke}{rgb}{0.000000,0.000000,0.000000}%
\pgfsetstrokecolor{currentstroke}%
\pgfsetdash{}{0pt}%
\pgfsys@defobject{currentmarker}{\pgfqpoint{-0.027778in}{0.000000in}}{\pgfqpoint{-0.000000in}{0.000000in}}{%
\pgfpathmoveto{\pgfqpoint{-0.000000in}{0.000000in}}%
\pgfpathlineto{\pgfqpoint{-0.027778in}{0.000000in}}%
\pgfusepath{stroke,fill}%
}%
\begin{pgfscope}%
\pgfsys@transformshift{0.708220in}{3.204502in}%
\pgfsys@useobject{currentmarker}{}%
\end{pgfscope}%
\end{pgfscope}%
\begin{pgfscope}%
\pgfsetbuttcap%
\pgfsetroundjoin%
\definecolor{currentfill}{rgb}{0.000000,0.000000,0.000000}%
\pgfsetfillcolor{currentfill}%
\pgfsetlinewidth{0.602250pt}%
\definecolor{currentstroke}{rgb}{0.000000,0.000000,0.000000}%
\pgfsetstrokecolor{currentstroke}%
\pgfsetdash{}{0pt}%
\pgfsys@defobject{currentmarker}{\pgfqpoint{-0.027778in}{0.000000in}}{\pgfqpoint{-0.000000in}{0.000000in}}{%
\pgfpathmoveto{\pgfqpoint{-0.000000in}{0.000000in}}%
\pgfpathlineto{\pgfqpoint{-0.027778in}{0.000000in}}%
\pgfusepath{stroke,fill}%
}%
\begin{pgfscope}%
\pgfsys@transformshift{0.708220in}{3.326119in}%
\pgfsys@useobject{currentmarker}{}%
\end{pgfscope}%
\end{pgfscope}%
\begin{pgfscope}%
\pgfsetbuttcap%
\pgfsetroundjoin%
\definecolor{currentfill}{rgb}{0.000000,0.000000,0.000000}%
\pgfsetfillcolor{currentfill}%
\pgfsetlinewidth{0.602250pt}%
\definecolor{currentstroke}{rgb}{0.000000,0.000000,0.000000}%
\pgfsetstrokecolor{currentstroke}%
\pgfsetdash{}{0pt}%
\pgfsys@defobject{currentmarker}{\pgfqpoint{-0.027778in}{0.000000in}}{\pgfqpoint{-0.000000in}{0.000000in}}{%
\pgfpathmoveto{\pgfqpoint{-0.000000in}{0.000000in}}%
\pgfpathlineto{\pgfqpoint{-0.027778in}{0.000000in}}%
\pgfusepath{stroke,fill}%
}%
\begin{pgfscope}%
\pgfsys@transformshift{0.708220in}{3.387874in}%
\pgfsys@useobject{currentmarker}{}%
\end{pgfscope}%
\end{pgfscope}%
\begin{pgfscope}%
\pgfsetbuttcap%
\pgfsetroundjoin%
\definecolor{currentfill}{rgb}{0.000000,0.000000,0.000000}%
\pgfsetfillcolor{currentfill}%
\pgfsetlinewidth{0.602250pt}%
\definecolor{currentstroke}{rgb}{0.000000,0.000000,0.000000}%
\pgfsetstrokecolor{currentstroke}%
\pgfsetdash{}{0pt}%
\pgfsys@defobject{currentmarker}{\pgfqpoint{-0.027778in}{0.000000in}}{\pgfqpoint{-0.000000in}{0.000000in}}{%
\pgfpathmoveto{\pgfqpoint{-0.000000in}{0.000000in}}%
\pgfpathlineto{\pgfqpoint{-0.027778in}{0.000000in}}%
\pgfusepath{stroke,fill}%
}%
\begin{pgfscope}%
\pgfsys@transformshift{0.708220in}{3.431689in}%
\pgfsys@useobject{currentmarker}{}%
\end{pgfscope}%
\end{pgfscope}%
\begin{pgfscope}%
\pgfsetbuttcap%
\pgfsetroundjoin%
\definecolor{currentfill}{rgb}{0.000000,0.000000,0.000000}%
\pgfsetfillcolor{currentfill}%
\pgfsetlinewidth{0.602250pt}%
\definecolor{currentstroke}{rgb}{0.000000,0.000000,0.000000}%
\pgfsetstrokecolor{currentstroke}%
\pgfsetdash{}{0pt}%
\pgfsys@defobject{currentmarker}{\pgfqpoint{-0.027778in}{0.000000in}}{\pgfqpoint{-0.000000in}{0.000000in}}{%
\pgfpathmoveto{\pgfqpoint{-0.000000in}{0.000000in}}%
\pgfpathlineto{\pgfqpoint{-0.027778in}{0.000000in}}%
\pgfusepath{stroke,fill}%
}%
\begin{pgfscope}%
\pgfsys@transformshift{0.708220in}{3.465675in}%
\pgfsys@useobject{currentmarker}{}%
\end{pgfscope}%
\end{pgfscope}%
\begin{pgfscope}%
\pgfsetbuttcap%
\pgfsetroundjoin%
\definecolor{currentfill}{rgb}{0.000000,0.000000,0.000000}%
\pgfsetfillcolor{currentfill}%
\pgfsetlinewidth{0.602250pt}%
\definecolor{currentstroke}{rgb}{0.000000,0.000000,0.000000}%
\pgfsetstrokecolor{currentstroke}%
\pgfsetdash{}{0pt}%
\pgfsys@defobject{currentmarker}{\pgfqpoint{-0.027778in}{0.000000in}}{\pgfqpoint{-0.000000in}{0.000000in}}{%
\pgfpathmoveto{\pgfqpoint{-0.000000in}{0.000000in}}%
\pgfpathlineto{\pgfqpoint{-0.027778in}{0.000000in}}%
\pgfusepath{stroke,fill}%
}%
\begin{pgfscope}%
\pgfsys@transformshift{0.708220in}{3.493444in}%
\pgfsys@useobject{currentmarker}{}%
\end{pgfscope}%
\end{pgfscope}%
\begin{pgfscope}%
\pgfsetbuttcap%
\pgfsetroundjoin%
\definecolor{currentfill}{rgb}{0.000000,0.000000,0.000000}%
\pgfsetfillcolor{currentfill}%
\pgfsetlinewidth{0.602250pt}%
\definecolor{currentstroke}{rgb}{0.000000,0.000000,0.000000}%
\pgfsetstrokecolor{currentstroke}%
\pgfsetdash{}{0pt}%
\pgfsys@defobject{currentmarker}{\pgfqpoint{-0.027778in}{0.000000in}}{\pgfqpoint{-0.000000in}{0.000000in}}{%
\pgfpathmoveto{\pgfqpoint{-0.000000in}{0.000000in}}%
\pgfpathlineto{\pgfqpoint{-0.027778in}{0.000000in}}%
\pgfusepath{stroke,fill}%
}%
\begin{pgfscope}%
\pgfsys@transformshift{0.708220in}{3.516922in}%
\pgfsys@useobject{currentmarker}{}%
\end{pgfscope}%
\end{pgfscope}%
\begin{pgfscope}%
\pgfsetbuttcap%
\pgfsetroundjoin%
\definecolor{currentfill}{rgb}{0.000000,0.000000,0.000000}%
\pgfsetfillcolor{currentfill}%
\pgfsetlinewidth{0.602250pt}%
\definecolor{currentstroke}{rgb}{0.000000,0.000000,0.000000}%
\pgfsetstrokecolor{currentstroke}%
\pgfsetdash{}{0pt}%
\pgfsys@defobject{currentmarker}{\pgfqpoint{-0.027778in}{0.000000in}}{\pgfqpoint{-0.000000in}{0.000000in}}{%
\pgfpathmoveto{\pgfqpoint{-0.000000in}{0.000000in}}%
\pgfpathlineto{\pgfqpoint{-0.027778in}{0.000000in}}%
\pgfusepath{stroke,fill}%
}%
\begin{pgfscope}%
\pgfsys@transformshift{0.708220in}{3.537259in}%
\pgfsys@useobject{currentmarker}{}%
\end{pgfscope}%
\end{pgfscope}%
\begin{pgfscope}%
\pgfsetbuttcap%
\pgfsetroundjoin%
\definecolor{currentfill}{rgb}{0.000000,0.000000,0.000000}%
\pgfsetfillcolor{currentfill}%
\pgfsetlinewidth{0.602250pt}%
\definecolor{currentstroke}{rgb}{0.000000,0.000000,0.000000}%
\pgfsetstrokecolor{currentstroke}%
\pgfsetdash{}{0pt}%
\pgfsys@defobject{currentmarker}{\pgfqpoint{-0.027778in}{0.000000in}}{\pgfqpoint{-0.000000in}{0.000000in}}{%
\pgfpathmoveto{\pgfqpoint{-0.000000in}{0.000000in}}%
\pgfpathlineto{\pgfqpoint{-0.027778in}{0.000000in}}%
\pgfusepath{stroke,fill}%
}%
\begin{pgfscope}%
\pgfsys@transformshift{0.708220in}{3.555198in}%
\pgfsys@useobject{currentmarker}{}%
\end{pgfscope}%
\end{pgfscope}%
\begin{pgfscope}%
\pgfsetbuttcap%
\pgfsetroundjoin%
\definecolor{currentfill}{rgb}{0.000000,0.000000,0.000000}%
\pgfsetfillcolor{currentfill}%
\pgfsetlinewidth{0.602250pt}%
\definecolor{currentstroke}{rgb}{0.000000,0.000000,0.000000}%
\pgfsetstrokecolor{currentstroke}%
\pgfsetdash{}{0pt}%
\pgfsys@defobject{currentmarker}{\pgfqpoint{-0.027778in}{0.000000in}}{\pgfqpoint{-0.000000in}{0.000000in}}{%
\pgfpathmoveto{\pgfqpoint{-0.000000in}{0.000000in}}%
\pgfpathlineto{\pgfqpoint{-0.027778in}{0.000000in}}%
\pgfusepath{stroke,fill}%
}%
\begin{pgfscope}%
\pgfsys@transformshift{0.708220in}{3.676816in}%
\pgfsys@useobject{currentmarker}{}%
\end{pgfscope}%
\end{pgfscope}%
\begin{pgfscope}%
\pgfsetbuttcap%
\pgfsetroundjoin%
\definecolor{currentfill}{rgb}{0.000000,0.000000,0.000000}%
\pgfsetfillcolor{currentfill}%
\pgfsetlinewidth{0.602250pt}%
\definecolor{currentstroke}{rgb}{0.000000,0.000000,0.000000}%
\pgfsetstrokecolor{currentstroke}%
\pgfsetdash{}{0pt}%
\pgfsys@defobject{currentmarker}{\pgfqpoint{-0.027778in}{0.000000in}}{\pgfqpoint{-0.000000in}{0.000000in}}{%
\pgfpathmoveto{\pgfqpoint{-0.000000in}{0.000000in}}%
\pgfpathlineto{\pgfqpoint{-0.027778in}{0.000000in}}%
\pgfusepath{stroke,fill}%
}%
\begin{pgfscope}%
\pgfsys@transformshift{0.708220in}{3.738570in}%
\pgfsys@useobject{currentmarker}{}%
\end{pgfscope}%
\end{pgfscope}%
\begin{pgfscope}%
\pgfsetbuttcap%
\pgfsetroundjoin%
\definecolor{currentfill}{rgb}{0.000000,0.000000,0.000000}%
\pgfsetfillcolor{currentfill}%
\pgfsetlinewidth{0.602250pt}%
\definecolor{currentstroke}{rgb}{0.000000,0.000000,0.000000}%
\pgfsetstrokecolor{currentstroke}%
\pgfsetdash{}{0pt}%
\pgfsys@defobject{currentmarker}{\pgfqpoint{-0.027778in}{0.000000in}}{\pgfqpoint{-0.000000in}{0.000000in}}{%
\pgfpathmoveto{\pgfqpoint{-0.000000in}{0.000000in}}%
\pgfpathlineto{\pgfqpoint{-0.027778in}{0.000000in}}%
\pgfusepath{stroke,fill}%
}%
\begin{pgfscope}%
\pgfsys@transformshift{0.708220in}{3.782386in}%
\pgfsys@useobject{currentmarker}{}%
\end{pgfscope}%
\end{pgfscope}%
\begin{pgfscope}%
\pgfsetbuttcap%
\pgfsetroundjoin%
\definecolor{currentfill}{rgb}{0.000000,0.000000,0.000000}%
\pgfsetfillcolor{currentfill}%
\pgfsetlinewidth{0.602250pt}%
\definecolor{currentstroke}{rgb}{0.000000,0.000000,0.000000}%
\pgfsetstrokecolor{currentstroke}%
\pgfsetdash{}{0pt}%
\pgfsys@defobject{currentmarker}{\pgfqpoint{-0.027778in}{0.000000in}}{\pgfqpoint{-0.000000in}{0.000000in}}{%
\pgfpathmoveto{\pgfqpoint{-0.000000in}{0.000000in}}%
\pgfpathlineto{\pgfqpoint{-0.027778in}{0.000000in}}%
\pgfusepath{stroke,fill}%
}%
\begin{pgfscope}%
\pgfsys@transformshift{0.708220in}{3.816372in}%
\pgfsys@useobject{currentmarker}{}%
\end{pgfscope}%
\end{pgfscope}%
\begin{pgfscope}%
\pgfsetbuttcap%
\pgfsetroundjoin%
\definecolor{currentfill}{rgb}{0.000000,0.000000,0.000000}%
\pgfsetfillcolor{currentfill}%
\pgfsetlinewidth{0.602250pt}%
\definecolor{currentstroke}{rgb}{0.000000,0.000000,0.000000}%
\pgfsetstrokecolor{currentstroke}%
\pgfsetdash{}{0pt}%
\pgfsys@defobject{currentmarker}{\pgfqpoint{-0.027778in}{0.000000in}}{\pgfqpoint{-0.000000in}{0.000000in}}{%
\pgfpathmoveto{\pgfqpoint{-0.000000in}{0.000000in}}%
\pgfpathlineto{\pgfqpoint{-0.027778in}{0.000000in}}%
\pgfusepath{stroke,fill}%
}%
\begin{pgfscope}%
\pgfsys@transformshift{0.708220in}{3.844140in}%
\pgfsys@useobject{currentmarker}{}%
\end{pgfscope}%
\end{pgfscope}%
\begin{pgfscope}%
\pgfsetbuttcap%
\pgfsetroundjoin%
\definecolor{currentfill}{rgb}{0.000000,0.000000,0.000000}%
\pgfsetfillcolor{currentfill}%
\pgfsetlinewidth{0.602250pt}%
\definecolor{currentstroke}{rgb}{0.000000,0.000000,0.000000}%
\pgfsetstrokecolor{currentstroke}%
\pgfsetdash{}{0pt}%
\pgfsys@defobject{currentmarker}{\pgfqpoint{-0.027778in}{0.000000in}}{\pgfqpoint{-0.000000in}{0.000000in}}{%
\pgfpathmoveto{\pgfqpoint{-0.000000in}{0.000000in}}%
\pgfpathlineto{\pgfqpoint{-0.027778in}{0.000000in}}%
\pgfusepath{stroke,fill}%
}%
\begin{pgfscope}%
\pgfsys@transformshift{0.708220in}{3.867618in}%
\pgfsys@useobject{currentmarker}{}%
\end{pgfscope}%
\end{pgfscope}%
\begin{pgfscope}%
\pgfsetbuttcap%
\pgfsetroundjoin%
\definecolor{currentfill}{rgb}{0.000000,0.000000,0.000000}%
\pgfsetfillcolor{currentfill}%
\pgfsetlinewidth{0.602250pt}%
\definecolor{currentstroke}{rgb}{0.000000,0.000000,0.000000}%
\pgfsetstrokecolor{currentstroke}%
\pgfsetdash{}{0pt}%
\pgfsys@defobject{currentmarker}{\pgfqpoint{-0.027778in}{0.000000in}}{\pgfqpoint{-0.000000in}{0.000000in}}{%
\pgfpathmoveto{\pgfqpoint{-0.000000in}{0.000000in}}%
\pgfpathlineto{\pgfqpoint{-0.027778in}{0.000000in}}%
\pgfusepath{stroke,fill}%
}%
\begin{pgfscope}%
\pgfsys@transformshift{0.708220in}{3.887956in}%
\pgfsys@useobject{currentmarker}{}%
\end{pgfscope}%
\end{pgfscope}%
\begin{pgfscope}%
\pgfsetbuttcap%
\pgfsetroundjoin%
\definecolor{currentfill}{rgb}{0.000000,0.000000,0.000000}%
\pgfsetfillcolor{currentfill}%
\pgfsetlinewidth{0.602250pt}%
\definecolor{currentstroke}{rgb}{0.000000,0.000000,0.000000}%
\pgfsetstrokecolor{currentstroke}%
\pgfsetdash{}{0pt}%
\pgfsys@defobject{currentmarker}{\pgfqpoint{-0.027778in}{0.000000in}}{\pgfqpoint{-0.000000in}{0.000000in}}{%
\pgfpathmoveto{\pgfqpoint{-0.000000in}{0.000000in}}%
\pgfpathlineto{\pgfqpoint{-0.027778in}{0.000000in}}%
\pgfusepath{stroke,fill}%
}%
\begin{pgfscope}%
\pgfsys@transformshift{0.708220in}{3.905895in}%
\pgfsys@useobject{currentmarker}{}%
\end{pgfscope}%
\end{pgfscope}%
\begin{pgfscope}%
\definecolor{textcolor}{rgb}{0.000000,0.000000,0.000000}%
\pgfsetstrokecolor{textcolor}%
\pgfsetfillcolor{textcolor}%
\pgftext[x=0.288855in,y=3.220549in,,bottom,rotate=90.000000]{\color{textcolor}\rmfamily\fontsize{10.000000}{12.000000}\selectfont Longest solving time (s)}%
\end{pgfscope}%
\begin{pgfscope}%
\pgfpathrectangle{\pgfqpoint{0.708220in}{2.519156in}}{\pgfqpoint{5.141780in}{1.402786in}}%
\pgfusepath{clip}%
\pgfsetbuttcap%
\pgfsetroundjoin%
\pgfsetlinewidth{2.007500pt}%
\definecolor{currentstroke}{rgb}{1.000000,0.843137,0.000000}%
\pgfsetstrokecolor{currentstroke}%
\pgfsetdash{{7.400000pt}{3.200000pt}}{0.000000pt}%
\pgfpathmoveto{\pgfqpoint{0.708220in}{2.744893in}}%
\pgfpathlineto{\pgfqpoint{0.712933in}{2.746168in}}%
\pgfpathlineto{\pgfqpoint{0.783626in}{2.747478in}}%
\pgfpathlineto{\pgfqpoint{0.826043in}{2.748628in}}%
\pgfpathlineto{\pgfqpoint{0.877884in}{2.750430in}}%
\pgfpathlineto{\pgfqpoint{0.882597in}{2.752913in}}%
\pgfpathlineto{\pgfqpoint{0.887310in}{2.758652in}}%
\pgfpathlineto{\pgfqpoint{0.892023in}{2.760981in}}%
\pgfpathlineto{\pgfqpoint{0.934439in}{2.762072in}}%
\pgfpathlineto{\pgfqpoint{1.023985in}{2.765622in}}%
\pgfpathlineto{\pgfqpoint{1.038123in}{2.770291in}}%
\pgfpathlineto{\pgfqpoint{1.042836in}{2.770418in}}%
\pgfpathlineto{\pgfqpoint{1.047549in}{2.787417in}}%
\pgfpathlineto{\pgfqpoint{1.127668in}{2.790471in}}%
\pgfpathlineto{\pgfqpoint{1.207788in}{2.798015in}}%
\pgfpathlineto{\pgfqpoint{1.212501in}{2.800565in}}%
\pgfpathlineto{\pgfqpoint{1.217214in}{2.810008in}}%
\pgfpathlineto{\pgfqpoint{1.221927in}{2.814623in}}%
\pgfpathlineto{\pgfqpoint{1.236065in}{2.816190in}}%
\pgfpathlineto{\pgfqpoint{1.240778in}{2.817620in}}%
\pgfpathlineto{\pgfqpoint{1.245491in}{2.820400in}}%
\pgfpathlineto{\pgfqpoint{1.283194in}{2.826859in}}%
\pgfpathlineto{\pgfqpoint{1.386878in}{2.832008in}}%
\pgfpathlineto{\pgfqpoint{1.405730in}{2.835870in}}%
\pgfpathlineto{\pgfqpoint{1.438720in}{2.838138in}}%
\pgfpathlineto{\pgfqpoint{1.443433in}{2.839645in}}%
\pgfpathlineto{\pgfqpoint{1.452859in}{2.857796in}}%
\pgfpathlineto{\pgfqpoint{1.457572in}{2.858965in}}%
\pgfpathlineto{\pgfqpoint{1.466998in}{2.862806in}}%
\pgfpathlineto{\pgfqpoint{1.471711in}{2.868488in}}%
\pgfpathlineto{\pgfqpoint{1.485849in}{2.869178in}}%
\pgfpathlineto{\pgfqpoint{1.490562in}{2.873739in}}%
\pgfpathlineto{\pgfqpoint{1.499988in}{2.899054in}}%
\pgfpathlineto{\pgfqpoint{1.509414in}{2.904235in}}%
\pgfpathlineto{\pgfqpoint{1.514127in}{2.916648in}}%
\pgfpathlineto{\pgfqpoint{1.523553in}{2.921804in}}%
\pgfpathlineto{\pgfqpoint{1.547117in}{2.926191in}}%
\pgfpathlineto{\pgfqpoint{1.551830in}{2.926855in}}%
\pgfpathlineto{\pgfqpoint{1.565969in}{2.935551in}}%
\pgfpathlineto{\pgfqpoint{1.570682in}{2.940693in}}%
\pgfpathlineto{\pgfqpoint{1.575395in}{2.941014in}}%
\pgfpathlineto{\pgfqpoint{1.580107in}{2.947385in}}%
\pgfpathlineto{\pgfqpoint{1.584820in}{2.951156in}}%
\pgfpathlineto{\pgfqpoint{1.589533in}{2.952553in}}%
\pgfpathlineto{\pgfqpoint{1.594246in}{2.958542in}}%
\pgfpathlineto{\pgfqpoint{1.598959in}{2.960962in}}%
\pgfpathlineto{\pgfqpoint{1.617811in}{2.965174in}}%
\pgfpathlineto{\pgfqpoint{1.627236in}{2.971130in}}%
\pgfpathlineto{\pgfqpoint{1.631949in}{2.983117in}}%
\pgfpathlineto{\pgfqpoint{1.636662in}{2.986363in}}%
\pgfpathlineto{\pgfqpoint{1.641375in}{2.987213in}}%
\pgfpathlineto{\pgfqpoint{1.646088in}{2.993798in}}%
\pgfpathlineto{\pgfqpoint{1.660227in}{2.994863in}}%
\pgfpathlineto{\pgfqpoint{1.664940in}{2.996024in}}%
\pgfpathlineto{\pgfqpoint{1.669653in}{2.999173in}}%
\pgfpathlineto{\pgfqpoint{1.679078in}{3.013466in}}%
\pgfpathlineto{\pgfqpoint{1.702643in}{3.020399in}}%
\pgfpathlineto{\pgfqpoint{1.707356in}{3.027490in}}%
\pgfpathlineto{\pgfqpoint{1.712069in}{3.028671in}}%
\pgfpathlineto{\pgfqpoint{1.716782in}{3.031137in}}%
\pgfpathlineto{\pgfqpoint{1.726207in}{3.033171in}}%
\pgfpathlineto{\pgfqpoint{1.740346in}{3.034362in}}%
\pgfpathlineto{\pgfqpoint{1.745059in}{3.035031in}}%
\pgfpathlineto{\pgfqpoint{1.749772in}{3.047618in}}%
\pgfpathlineto{\pgfqpoint{1.759198in}{3.048559in}}%
\pgfpathlineto{\pgfqpoint{1.768624in}{3.048763in}}%
\pgfpathlineto{\pgfqpoint{1.778049in}{3.050660in}}%
\pgfpathlineto{\pgfqpoint{1.792188in}{3.052130in}}%
\pgfpathlineto{\pgfqpoint{1.801614in}{3.053706in}}%
\pgfpathlineto{\pgfqpoint{1.829891in}{3.063705in}}%
\pgfpathlineto{\pgfqpoint{1.834604in}{3.064642in}}%
\pgfpathlineto{\pgfqpoint{1.839317in}{3.077851in}}%
\pgfpathlineto{\pgfqpoint{1.867595in}{3.080439in}}%
\pgfpathlineto{\pgfqpoint{1.877020in}{3.082499in}}%
\pgfpathlineto{\pgfqpoint{1.881733in}{3.084805in}}%
\pgfpathlineto{\pgfqpoint{1.895872in}{3.087206in}}%
\pgfpathlineto{\pgfqpoint{1.900585in}{3.089297in}}%
\pgfpathlineto{\pgfqpoint{1.924150in}{3.092135in}}%
\pgfpathlineto{\pgfqpoint{1.928862in}{3.094138in}}%
\pgfpathlineto{\pgfqpoint{1.933575in}{3.094774in}}%
\pgfpathlineto{\pgfqpoint{1.938288in}{3.104218in}}%
\pgfpathlineto{\pgfqpoint{1.943001in}{3.106226in}}%
\pgfpathlineto{\pgfqpoint{1.947714in}{3.112438in}}%
\pgfpathlineto{\pgfqpoint{1.957140in}{3.112858in}}%
\pgfpathlineto{\pgfqpoint{1.966566in}{3.114252in}}%
\pgfpathlineto{\pgfqpoint{1.971279in}{3.114551in}}%
\pgfpathlineto{\pgfqpoint{1.975991in}{3.117366in}}%
\pgfpathlineto{\pgfqpoint{1.990130in}{3.119966in}}%
\pgfpathlineto{\pgfqpoint{1.994843in}{3.120380in}}%
\pgfpathlineto{\pgfqpoint{1.999556in}{3.126002in}}%
\pgfpathlineto{\pgfqpoint{2.023121in}{3.128078in}}%
\pgfpathlineto{\pgfqpoint{2.032546in}{3.128718in}}%
\pgfpathlineto{\pgfqpoint{2.046685in}{3.129538in}}%
\pgfpathlineto{\pgfqpoint{2.051398in}{3.130022in}}%
\pgfpathlineto{\pgfqpoint{2.056111in}{3.143684in}}%
\pgfpathlineto{\pgfqpoint{2.070250in}{3.145379in}}%
\pgfpathlineto{\pgfqpoint{2.098527in}{3.147273in}}%
\pgfpathlineto{\pgfqpoint{2.117379in}{3.150657in}}%
\pgfpathlineto{\pgfqpoint{2.122092in}{3.151291in}}%
\pgfpathlineto{\pgfqpoint{2.136230in}{3.156374in}}%
\pgfpathlineto{\pgfqpoint{2.140943in}{3.156976in}}%
\pgfpathlineto{\pgfqpoint{2.150369in}{3.160767in}}%
\pgfpathlineto{\pgfqpoint{2.155082in}{3.170042in}}%
\pgfpathlineto{\pgfqpoint{2.183359in}{3.175340in}}%
\pgfpathlineto{\pgfqpoint{2.188072in}{3.175595in}}%
\pgfpathlineto{\pgfqpoint{2.192785in}{3.178132in}}%
\pgfpathlineto{\pgfqpoint{2.206924in}{3.180300in}}%
\pgfpathlineto{\pgfqpoint{2.211637in}{3.184140in}}%
\pgfpathlineto{\pgfqpoint{2.239914in}{3.186775in}}%
\pgfpathlineto{\pgfqpoint{2.244627in}{3.188803in}}%
\pgfpathlineto{\pgfqpoint{2.263479in}{3.190690in}}%
\pgfpathlineto{\pgfqpoint{2.272905in}{3.192862in}}%
\pgfpathlineto{\pgfqpoint{2.277617in}{3.201743in}}%
\pgfpathlineto{\pgfqpoint{2.291756in}{3.203801in}}%
\pgfpathlineto{\pgfqpoint{2.343598in}{3.208889in}}%
\pgfpathlineto{\pgfqpoint{2.348311in}{3.213562in}}%
\pgfpathlineto{\pgfqpoint{2.367163in}{3.222322in}}%
\pgfpathlineto{\pgfqpoint{2.386014in}{3.223938in}}%
\pgfpathlineto{\pgfqpoint{2.395440in}{3.226009in}}%
\pgfpathlineto{\pgfqpoint{2.400153in}{3.236693in}}%
\pgfpathlineto{\pgfqpoint{2.404866in}{3.238565in}}%
\pgfpathlineto{\pgfqpoint{2.419005in}{3.238876in}}%
\pgfpathlineto{\pgfqpoint{2.433143in}{3.242349in}}%
\pgfpathlineto{\pgfqpoint{2.451995in}{3.246291in}}%
\pgfpathlineto{\pgfqpoint{2.456708in}{3.250352in}}%
\pgfpathlineto{\pgfqpoint{2.480272in}{3.251581in}}%
\pgfpathlineto{\pgfqpoint{2.513263in}{3.253070in}}%
\pgfpathlineto{\pgfqpoint{2.536827in}{3.254465in}}%
\pgfpathlineto{\pgfqpoint{2.546253in}{3.268117in}}%
\pgfpathlineto{\pgfqpoint{2.560392in}{3.269357in}}%
\pgfpathlineto{\pgfqpoint{2.565105in}{3.272136in}}%
\pgfpathlineto{\pgfqpoint{2.579243in}{3.274752in}}%
\pgfpathlineto{\pgfqpoint{2.583956in}{3.277687in}}%
\pgfpathlineto{\pgfqpoint{2.593382in}{3.278920in}}%
\pgfpathlineto{\pgfqpoint{2.626372in}{3.283692in}}%
\pgfpathlineto{\pgfqpoint{2.631085in}{3.297938in}}%
\pgfpathlineto{\pgfqpoint{2.635798in}{3.305261in}}%
\pgfpathlineto{\pgfqpoint{2.640511in}{3.307958in}}%
\pgfpathlineto{\pgfqpoint{2.659363in}{3.310469in}}%
\pgfpathlineto{\pgfqpoint{2.668789in}{3.310853in}}%
\pgfpathlineto{\pgfqpoint{2.673502in}{3.319608in}}%
\pgfpathlineto{\pgfqpoint{2.687640in}{3.320966in}}%
\pgfpathlineto{\pgfqpoint{2.692353in}{3.320987in}}%
\pgfpathlineto{\pgfqpoint{2.701779in}{3.326503in}}%
\pgfpathlineto{\pgfqpoint{2.706492in}{3.336797in}}%
\pgfpathlineto{\pgfqpoint{2.711205in}{3.338097in}}%
\pgfpathlineto{\pgfqpoint{2.715918in}{3.338177in}}%
\pgfpathlineto{\pgfqpoint{2.720631in}{3.339825in}}%
\pgfpathlineto{\pgfqpoint{2.730056in}{3.341323in}}%
\pgfpathlineto{\pgfqpoint{2.734769in}{3.342830in}}%
\pgfpathlineto{\pgfqpoint{2.739482in}{3.372605in}}%
\pgfpathlineto{\pgfqpoint{2.744195in}{3.390170in}}%
\pgfpathlineto{\pgfqpoint{2.748908in}{3.392738in}}%
\pgfpathlineto{\pgfqpoint{2.758334in}{3.404249in}}%
\pgfpathlineto{\pgfqpoint{2.763047in}{3.420772in}}%
\pgfpathlineto{\pgfqpoint{2.767760in}{3.424416in}}%
\pgfpathlineto{\pgfqpoint{2.772473in}{3.433537in}}%
\pgfpathlineto{\pgfqpoint{2.777185in}{3.438334in}}%
\pgfpathlineto{\pgfqpoint{2.786611in}{3.686124in}}%
\pgfpathlineto{\pgfqpoint{2.791324in}{3.921942in}}%
\pgfpathlineto{\pgfqpoint{5.845287in}{3.921942in}}%
\pgfpathlineto{\pgfqpoint{5.845287in}{3.921942in}}%
\pgfusepath{stroke}%
\end{pgfscope}%
\begin{pgfscope}%
\pgfpathrectangle{\pgfqpoint{0.708220in}{2.519156in}}{\pgfqpoint{5.141780in}{1.402786in}}%
\pgfusepath{clip}%
\pgfsetbuttcap%
\pgfsetroundjoin%
\pgfsetlinewidth{2.007500pt}%
\definecolor{currentstroke}{rgb}{1.000000,0.694118,0.305882}%
\pgfsetstrokecolor{currentstroke}%
\pgfsetdash{{2.000000pt}{3.300000pt}}{0.000000pt}%
\pgfpathmoveto{\pgfqpoint{0.708220in}{2.592353in}}%
\pgfpathlineto{\pgfqpoint{0.727071in}{2.593926in}}%
\pgfpathlineto{\pgfqpoint{0.920301in}{2.596598in}}%
\pgfpathlineto{\pgfqpoint{1.052262in}{2.598148in}}%
\pgfpathlineto{\pgfqpoint{1.085252in}{2.598719in}}%
\pgfpathlineto{\pgfqpoint{1.165372in}{2.601297in}}%
\pgfpathlineto{\pgfqpoint{1.170085in}{2.613533in}}%
\pgfpathlineto{\pgfqpoint{1.184223in}{2.616267in}}%
\pgfpathlineto{\pgfqpoint{1.198362in}{2.617419in}}%
\pgfpathlineto{\pgfqpoint{1.203075in}{2.619027in}}%
\pgfpathlineto{\pgfqpoint{1.207788in}{2.619161in}}%
\pgfpathlineto{\pgfqpoint{1.212501in}{2.621495in}}%
\pgfpathlineto{\pgfqpoint{1.217214in}{2.621554in}}%
\pgfpathlineto{\pgfqpoint{1.221927in}{2.635583in}}%
\pgfpathlineto{\pgfqpoint{1.226639in}{2.646090in}}%
\pgfpathlineto{\pgfqpoint{1.231352in}{2.664498in}}%
\pgfpathlineto{\pgfqpoint{1.245491in}{2.667375in}}%
\pgfpathlineto{\pgfqpoint{1.254917in}{2.670572in}}%
\pgfpathlineto{\pgfqpoint{1.264343in}{2.672651in}}%
\pgfpathlineto{\pgfqpoint{1.283194in}{2.673548in}}%
\pgfpathlineto{\pgfqpoint{1.306759in}{2.674710in}}%
\pgfpathlineto{\pgfqpoint{1.358601in}{2.675758in}}%
\pgfpathlineto{\pgfqpoint{1.372740in}{2.676974in}}%
\pgfpathlineto{\pgfqpoint{1.401017in}{2.678175in}}%
\pgfpathlineto{\pgfqpoint{1.410443in}{2.680894in}}%
\pgfpathlineto{\pgfqpoint{1.415156in}{2.688293in}}%
\pgfpathlineto{\pgfqpoint{1.419869in}{2.690800in}}%
\pgfpathlineto{\pgfqpoint{1.429294in}{2.691630in}}%
\pgfpathlineto{\pgfqpoint{1.443433in}{2.697524in}}%
\pgfpathlineto{\pgfqpoint{1.528265in}{2.700470in}}%
\pgfpathlineto{\pgfqpoint{1.570682in}{2.702026in}}%
\pgfpathlineto{\pgfqpoint{1.580107in}{2.702830in}}%
\pgfpathlineto{\pgfqpoint{1.598959in}{2.703133in}}%
\pgfpathlineto{\pgfqpoint{1.603672in}{2.704829in}}%
\pgfpathlineto{\pgfqpoint{1.608385in}{2.704859in}}%
\pgfpathlineto{\pgfqpoint{1.613098in}{2.712685in}}%
\pgfpathlineto{\pgfqpoint{1.617811in}{2.713213in}}%
\pgfpathlineto{\pgfqpoint{1.622524in}{2.715653in}}%
\pgfpathlineto{\pgfqpoint{1.627236in}{2.720339in}}%
\pgfpathlineto{\pgfqpoint{1.646088in}{2.722003in}}%
\pgfpathlineto{\pgfqpoint{1.660227in}{2.725044in}}%
\pgfpathlineto{\pgfqpoint{1.669653in}{2.726384in}}%
\pgfpathlineto{\pgfqpoint{1.679078in}{2.728261in}}%
\pgfpathlineto{\pgfqpoint{1.688504in}{2.729377in}}%
\pgfpathlineto{\pgfqpoint{1.693217in}{2.730087in}}%
\pgfpathlineto{\pgfqpoint{1.697930in}{2.733621in}}%
\pgfpathlineto{\pgfqpoint{1.702643in}{2.733744in}}%
\pgfpathlineto{\pgfqpoint{1.712069in}{2.735734in}}%
\pgfpathlineto{\pgfqpoint{1.716782in}{2.743323in}}%
\pgfpathlineto{\pgfqpoint{1.726207in}{2.744128in}}%
\pgfpathlineto{\pgfqpoint{1.735633in}{2.745777in}}%
\pgfpathlineto{\pgfqpoint{1.740346in}{2.745839in}}%
\pgfpathlineto{\pgfqpoint{1.745059in}{2.748233in}}%
\pgfpathlineto{\pgfqpoint{1.754485in}{2.749908in}}%
\pgfpathlineto{\pgfqpoint{1.759198in}{2.755009in}}%
\pgfpathlineto{\pgfqpoint{1.768624in}{2.756564in}}%
\pgfpathlineto{\pgfqpoint{1.773337in}{2.774962in}}%
\pgfpathlineto{\pgfqpoint{1.782762in}{2.778320in}}%
\pgfpathlineto{\pgfqpoint{1.801614in}{2.779130in}}%
\pgfpathlineto{\pgfqpoint{1.806327in}{2.781647in}}%
\pgfpathlineto{\pgfqpoint{1.811040in}{2.781793in}}%
\pgfpathlineto{\pgfqpoint{1.815753in}{2.784167in}}%
\pgfpathlineto{\pgfqpoint{1.825179in}{2.784294in}}%
\pgfpathlineto{\pgfqpoint{1.839317in}{2.789040in}}%
\pgfpathlineto{\pgfqpoint{1.848743in}{2.789943in}}%
\pgfpathlineto{\pgfqpoint{1.872308in}{2.792411in}}%
\pgfpathlineto{\pgfqpoint{1.877020in}{2.794109in}}%
\pgfpathlineto{\pgfqpoint{1.886446in}{2.795072in}}%
\pgfpathlineto{\pgfqpoint{1.891159in}{2.796757in}}%
\pgfpathlineto{\pgfqpoint{1.933575in}{2.799870in}}%
\pgfpathlineto{\pgfqpoint{1.938288in}{2.802743in}}%
\pgfpathlineto{\pgfqpoint{1.952427in}{2.803920in}}%
\pgfpathlineto{\pgfqpoint{1.957140in}{2.805337in}}%
\pgfpathlineto{\pgfqpoint{1.961853in}{2.808416in}}%
\pgfpathlineto{\pgfqpoint{1.971279in}{2.810551in}}%
\pgfpathlineto{\pgfqpoint{1.980704in}{2.810772in}}%
\pgfpathlineto{\pgfqpoint{1.990130in}{2.813192in}}%
\pgfpathlineto{\pgfqpoint{2.004269in}{2.814301in}}%
\pgfpathlineto{\pgfqpoint{2.008982in}{2.816089in}}%
\pgfpathlineto{\pgfqpoint{2.023121in}{2.817732in}}%
\pgfpathlineto{\pgfqpoint{2.027833in}{2.819761in}}%
\pgfpathlineto{\pgfqpoint{2.032546in}{2.820204in}}%
\pgfpathlineto{\pgfqpoint{2.037259in}{2.822104in}}%
\pgfpathlineto{\pgfqpoint{2.065537in}{2.826440in}}%
\pgfpathlineto{\pgfqpoint{2.074963in}{2.829195in}}%
\pgfpathlineto{\pgfqpoint{2.084388in}{2.830672in}}%
\pgfpathlineto{\pgfqpoint{2.093814in}{2.831368in}}%
\pgfpathlineto{\pgfqpoint{2.126804in}{2.837179in}}%
\pgfpathlineto{\pgfqpoint{2.159795in}{2.839170in}}%
\pgfpathlineto{\pgfqpoint{2.169221in}{2.841554in}}%
\pgfpathlineto{\pgfqpoint{2.183359in}{2.842902in}}%
\pgfpathlineto{\pgfqpoint{2.202211in}{2.843992in}}%
\pgfpathlineto{\pgfqpoint{2.206924in}{2.846392in}}%
\pgfpathlineto{\pgfqpoint{2.211637in}{2.856565in}}%
\pgfpathlineto{\pgfqpoint{2.225776in}{2.857385in}}%
\pgfpathlineto{\pgfqpoint{2.239914in}{2.859183in}}%
\pgfpathlineto{\pgfqpoint{2.244627in}{2.859371in}}%
\pgfpathlineto{\pgfqpoint{2.249340in}{2.860806in}}%
\pgfpathlineto{\pgfqpoint{2.254053in}{2.861018in}}%
\pgfpathlineto{\pgfqpoint{2.268192in}{2.866464in}}%
\pgfpathlineto{\pgfqpoint{2.291756in}{2.869074in}}%
\pgfpathlineto{\pgfqpoint{2.301182in}{2.873367in}}%
\pgfpathlineto{\pgfqpoint{2.310608in}{2.877176in}}%
\pgfpathlineto{\pgfqpoint{2.338885in}{2.881297in}}%
\pgfpathlineto{\pgfqpoint{2.353024in}{2.882213in}}%
\pgfpathlineto{\pgfqpoint{2.367163in}{2.884807in}}%
\pgfpathlineto{\pgfqpoint{2.386014in}{2.885776in}}%
\pgfpathlineto{\pgfqpoint{2.414292in}{2.897052in}}%
\pgfpathlineto{\pgfqpoint{2.433143in}{2.900213in}}%
\pgfpathlineto{\pgfqpoint{2.437856in}{2.917345in}}%
\pgfpathlineto{\pgfqpoint{2.447282in}{2.920368in}}%
\pgfpathlineto{\pgfqpoint{2.456708in}{2.921568in}}%
\pgfpathlineto{\pgfqpoint{2.480272in}{2.924356in}}%
\pgfpathlineto{\pgfqpoint{2.484985in}{2.925368in}}%
\pgfpathlineto{\pgfqpoint{2.489698in}{2.929984in}}%
\pgfpathlineto{\pgfqpoint{2.499124in}{2.931888in}}%
\pgfpathlineto{\pgfqpoint{2.503837in}{2.936804in}}%
\pgfpathlineto{\pgfqpoint{2.508550in}{2.958373in}}%
\pgfpathlineto{\pgfqpoint{2.513263in}{2.958633in}}%
\pgfpathlineto{\pgfqpoint{2.517976in}{2.961500in}}%
\pgfpathlineto{\pgfqpoint{2.527401in}{2.962629in}}%
\pgfpathlineto{\pgfqpoint{2.532114in}{2.972866in}}%
\pgfpathlineto{\pgfqpoint{2.546253in}{2.979112in}}%
\pgfpathlineto{\pgfqpoint{2.550966in}{2.984915in}}%
\pgfpathlineto{\pgfqpoint{2.569818in}{2.987540in}}%
\pgfpathlineto{\pgfqpoint{2.579243in}{2.990192in}}%
\pgfpathlineto{\pgfqpoint{2.588669in}{2.997912in}}%
\pgfpathlineto{\pgfqpoint{2.598095in}{3.002249in}}%
\pgfpathlineto{\pgfqpoint{2.602808in}{3.003749in}}%
\pgfpathlineto{\pgfqpoint{2.607521in}{3.009751in}}%
\pgfpathlineto{\pgfqpoint{2.612234in}{3.009805in}}%
\pgfpathlineto{\pgfqpoint{2.616947in}{3.015238in}}%
\pgfpathlineto{\pgfqpoint{2.626372in}{3.016769in}}%
\pgfpathlineto{\pgfqpoint{2.631085in}{3.019931in}}%
\pgfpathlineto{\pgfqpoint{2.635798in}{3.019979in}}%
\pgfpathlineto{\pgfqpoint{2.649937in}{3.026986in}}%
\pgfpathlineto{\pgfqpoint{2.654650in}{3.027014in}}%
\pgfpathlineto{\pgfqpoint{2.659363in}{3.040612in}}%
\pgfpathlineto{\pgfqpoint{2.668789in}{3.041631in}}%
\pgfpathlineto{\pgfqpoint{2.673502in}{3.043602in}}%
\pgfpathlineto{\pgfqpoint{2.678214in}{3.056398in}}%
\pgfpathlineto{\pgfqpoint{2.687640in}{3.058187in}}%
\pgfpathlineto{\pgfqpoint{2.697066in}{3.061157in}}%
\pgfpathlineto{\pgfqpoint{2.706492in}{3.062147in}}%
\pgfpathlineto{\pgfqpoint{2.715918in}{3.065397in}}%
\pgfpathlineto{\pgfqpoint{2.720631in}{3.065586in}}%
\pgfpathlineto{\pgfqpoint{2.725344in}{3.107786in}}%
\pgfpathlineto{\pgfqpoint{2.730056in}{3.120491in}}%
\pgfpathlineto{\pgfqpoint{2.734769in}{3.171049in}}%
\pgfpathlineto{\pgfqpoint{2.739482in}{3.180100in}}%
\pgfpathlineto{\pgfqpoint{2.744195in}{3.194991in}}%
\pgfpathlineto{\pgfqpoint{2.748908in}{3.250082in}}%
\pgfpathlineto{\pgfqpoint{2.753621in}{3.256391in}}%
\pgfpathlineto{\pgfqpoint{2.758334in}{3.257889in}}%
\pgfpathlineto{\pgfqpoint{2.763047in}{3.842887in}}%
\pgfpathlineto{\pgfqpoint{2.767760in}{3.843284in}}%
\pgfpathlineto{\pgfqpoint{2.772473in}{3.921942in}}%
\pgfpathlineto{\pgfqpoint{5.845287in}{3.921942in}}%
\pgfpathlineto{\pgfqpoint{5.845287in}{3.921942in}}%
\pgfusepath{stroke}%
\end{pgfscope}%
\begin{pgfscope}%
\pgfpathrectangle{\pgfqpoint{0.708220in}{2.519156in}}{\pgfqpoint{5.141780in}{1.402786in}}%
\pgfusepath{clip}%
\pgfsetrectcap%
\pgfsetroundjoin%
\pgfsetlinewidth{2.007500pt}%
\definecolor{currentstroke}{rgb}{0.980392,0.529412,0.458824}%
\pgfsetstrokecolor{currentstroke}%
\pgfsetdash{}{0pt}%
\pgfpathmoveto{\pgfqpoint{0.708220in}{2.799531in}}%
\pgfpathlineto{\pgfqpoint{0.717646in}{2.801224in}}%
\pgfpathlineto{\pgfqpoint{0.727071in}{2.802450in}}%
\pgfpathlineto{\pgfqpoint{0.764775in}{2.804251in}}%
\pgfpathlineto{\pgfqpoint{0.802478in}{2.805358in}}%
\pgfpathlineto{\pgfqpoint{0.830755in}{2.806539in}}%
\pgfpathlineto{\pgfqpoint{0.901449in}{2.808963in}}%
\pgfpathlineto{\pgfqpoint{0.910875in}{2.810009in}}%
\pgfpathlineto{\pgfqpoint{0.929726in}{2.810729in}}%
\pgfpathlineto{\pgfqpoint{1.198362in}{2.817891in}}%
\pgfpathlineto{\pgfqpoint{1.250204in}{2.821982in}}%
\pgfpathlineto{\pgfqpoint{1.254917in}{2.825596in}}%
\pgfpathlineto{\pgfqpoint{1.264343in}{2.826258in}}%
\pgfpathlineto{\pgfqpoint{1.269056in}{2.828461in}}%
\pgfpathlineto{\pgfqpoint{1.278481in}{2.829481in}}%
\pgfpathlineto{\pgfqpoint{1.306759in}{2.830881in}}%
\pgfpathlineto{\pgfqpoint{1.316185in}{2.832100in}}%
\pgfpathlineto{\pgfqpoint{1.349175in}{2.833966in}}%
\pgfpathlineto{\pgfqpoint{1.443433in}{2.840302in}}%
\pgfpathlineto{\pgfqpoint{1.448146in}{2.842888in}}%
\pgfpathlineto{\pgfqpoint{1.452859in}{2.858564in}}%
\pgfpathlineto{\pgfqpoint{1.462285in}{2.860840in}}%
\pgfpathlineto{\pgfqpoint{1.466998in}{2.861199in}}%
\pgfpathlineto{\pgfqpoint{1.471711in}{2.865041in}}%
\pgfpathlineto{\pgfqpoint{1.485849in}{2.869370in}}%
\pgfpathlineto{\pgfqpoint{1.490562in}{2.869656in}}%
\pgfpathlineto{\pgfqpoint{1.495275in}{2.873604in}}%
\pgfpathlineto{\pgfqpoint{1.499988in}{2.880314in}}%
\pgfpathlineto{\pgfqpoint{1.504701in}{2.889409in}}%
\pgfpathlineto{\pgfqpoint{1.509414in}{2.890450in}}%
\pgfpathlineto{\pgfqpoint{1.514127in}{2.896020in}}%
\pgfpathlineto{\pgfqpoint{1.523553in}{2.898600in}}%
\pgfpathlineto{\pgfqpoint{1.528265in}{2.900707in}}%
\pgfpathlineto{\pgfqpoint{1.532978in}{2.915405in}}%
\pgfpathlineto{\pgfqpoint{1.537691in}{2.919575in}}%
\pgfpathlineto{\pgfqpoint{1.556543in}{2.924310in}}%
\pgfpathlineto{\pgfqpoint{1.561256in}{2.931959in}}%
\pgfpathlineto{\pgfqpoint{1.565969in}{2.936408in}}%
\pgfpathlineto{\pgfqpoint{1.570682in}{2.937871in}}%
\pgfpathlineto{\pgfqpoint{1.575395in}{2.938003in}}%
\pgfpathlineto{\pgfqpoint{1.589533in}{2.944198in}}%
\pgfpathlineto{\pgfqpoint{1.598959in}{2.951351in}}%
\pgfpathlineto{\pgfqpoint{1.608385in}{2.954042in}}%
\pgfpathlineto{\pgfqpoint{1.613098in}{2.959569in}}%
\pgfpathlineto{\pgfqpoint{1.617811in}{2.959801in}}%
\pgfpathlineto{\pgfqpoint{1.627236in}{2.969500in}}%
\pgfpathlineto{\pgfqpoint{1.631949in}{2.970112in}}%
\pgfpathlineto{\pgfqpoint{1.636662in}{2.975867in}}%
\pgfpathlineto{\pgfqpoint{1.641375in}{2.976390in}}%
\pgfpathlineto{\pgfqpoint{1.646088in}{2.995175in}}%
\pgfpathlineto{\pgfqpoint{1.650801in}{2.999802in}}%
\pgfpathlineto{\pgfqpoint{1.664940in}{3.001818in}}%
\pgfpathlineto{\pgfqpoint{1.721495in}{3.007781in}}%
\pgfpathlineto{\pgfqpoint{1.726207in}{3.009975in}}%
\pgfpathlineto{\pgfqpoint{1.735633in}{3.018126in}}%
\pgfpathlineto{\pgfqpoint{1.745059in}{3.019043in}}%
\pgfpathlineto{\pgfqpoint{1.749772in}{3.022071in}}%
\pgfpathlineto{\pgfqpoint{1.759198in}{3.023963in}}%
\pgfpathlineto{\pgfqpoint{1.763911in}{3.028657in}}%
\pgfpathlineto{\pgfqpoint{1.782762in}{3.032311in}}%
\pgfpathlineto{\pgfqpoint{1.792188in}{3.036649in}}%
\pgfpathlineto{\pgfqpoint{1.829891in}{3.043838in}}%
\pgfpathlineto{\pgfqpoint{1.834604in}{3.045982in}}%
\pgfpathlineto{\pgfqpoint{1.839317in}{3.046280in}}%
\pgfpathlineto{\pgfqpoint{1.844030in}{3.049752in}}%
\pgfpathlineto{\pgfqpoint{1.853456in}{3.050723in}}%
\pgfpathlineto{\pgfqpoint{1.862882in}{3.055020in}}%
\pgfpathlineto{\pgfqpoint{1.867595in}{3.060232in}}%
\pgfpathlineto{\pgfqpoint{1.872308in}{3.062762in}}%
\pgfpathlineto{\pgfqpoint{1.886446in}{3.064460in}}%
\pgfpathlineto{\pgfqpoint{1.891159in}{3.071220in}}%
\pgfpathlineto{\pgfqpoint{1.895872in}{3.074025in}}%
\pgfpathlineto{\pgfqpoint{1.900585in}{3.080070in}}%
\pgfpathlineto{\pgfqpoint{1.910011in}{3.083853in}}%
\pgfpathlineto{\pgfqpoint{1.919437in}{3.086819in}}%
\pgfpathlineto{\pgfqpoint{1.924150in}{3.097841in}}%
\pgfpathlineto{\pgfqpoint{1.928862in}{3.102963in}}%
\pgfpathlineto{\pgfqpoint{1.933575in}{3.104762in}}%
\pgfpathlineto{\pgfqpoint{1.938288in}{3.110849in}}%
\pgfpathlineto{\pgfqpoint{1.943001in}{3.125379in}}%
\pgfpathlineto{\pgfqpoint{1.947714in}{3.130487in}}%
\pgfpathlineto{\pgfqpoint{1.952427in}{3.131153in}}%
\pgfpathlineto{\pgfqpoint{1.957140in}{3.135345in}}%
\pgfpathlineto{\pgfqpoint{1.966566in}{3.138132in}}%
\pgfpathlineto{\pgfqpoint{1.971279in}{3.143803in}}%
\pgfpathlineto{\pgfqpoint{1.980704in}{3.144630in}}%
\pgfpathlineto{\pgfqpoint{1.990130in}{3.149243in}}%
\pgfpathlineto{\pgfqpoint{1.994843in}{3.154135in}}%
\pgfpathlineto{\pgfqpoint{2.008982in}{3.162396in}}%
\pgfpathlineto{\pgfqpoint{2.013695in}{3.164475in}}%
\pgfpathlineto{\pgfqpoint{2.018408in}{3.165215in}}%
\pgfpathlineto{\pgfqpoint{2.023121in}{3.170262in}}%
\pgfpathlineto{\pgfqpoint{2.027833in}{3.173116in}}%
\pgfpathlineto{\pgfqpoint{2.032546in}{3.177584in}}%
\pgfpathlineto{\pgfqpoint{2.037259in}{3.177598in}}%
\pgfpathlineto{\pgfqpoint{2.041972in}{3.185688in}}%
\pgfpathlineto{\pgfqpoint{2.046685in}{3.186442in}}%
\pgfpathlineto{\pgfqpoint{2.060824in}{3.197790in}}%
\pgfpathlineto{\pgfqpoint{2.070250in}{3.207595in}}%
\pgfpathlineto{\pgfqpoint{2.079675in}{3.209959in}}%
\pgfpathlineto{\pgfqpoint{2.089101in}{3.215847in}}%
\pgfpathlineto{\pgfqpoint{2.093814in}{3.222882in}}%
\pgfpathlineto{\pgfqpoint{2.112666in}{3.225230in}}%
\pgfpathlineto{\pgfqpoint{2.122092in}{3.226539in}}%
\pgfpathlineto{\pgfqpoint{2.126804in}{3.231698in}}%
\pgfpathlineto{\pgfqpoint{2.136230in}{3.246602in}}%
\pgfpathlineto{\pgfqpoint{2.145656in}{3.248415in}}%
\pgfpathlineto{\pgfqpoint{2.150369in}{3.251028in}}%
\pgfpathlineto{\pgfqpoint{2.155082in}{3.252121in}}%
\pgfpathlineto{\pgfqpoint{2.159795in}{3.256512in}}%
\pgfpathlineto{\pgfqpoint{2.178646in}{3.258346in}}%
\pgfpathlineto{\pgfqpoint{2.188072in}{3.260021in}}%
\pgfpathlineto{\pgfqpoint{2.192785in}{3.264165in}}%
\pgfpathlineto{\pgfqpoint{2.249340in}{3.272445in}}%
\pgfpathlineto{\pgfqpoint{2.268192in}{3.273636in}}%
\pgfpathlineto{\pgfqpoint{2.277617in}{3.275265in}}%
\pgfpathlineto{\pgfqpoint{2.287043in}{3.282220in}}%
\pgfpathlineto{\pgfqpoint{2.291756in}{3.287656in}}%
\pgfpathlineto{\pgfqpoint{2.305895in}{3.289127in}}%
\pgfpathlineto{\pgfqpoint{2.310608in}{3.291384in}}%
\pgfpathlineto{\pgfqpoint{2.315321in}{3.292029in}}%
\pgfpathlineto{\pgfqpoint{2.320034in}{3.304041in}}%
\pgfpathlineto{\pgfqpoint{2.324747in}{3.306165in}}%
\pgfpathlineto{\pgfqpoint{2.329459in}{3.306587in}}%
\pgfpathlineto{\pgfqpoint{2.348311in}{3.315890in}}%
\pgfpathlineto{\pgfqpoint{2.353024in}{3.315913in}}%
\pgfpathlineto{\pgfqpoint{2.357737in}{3.317831in}}%
\pgfpathlineto{\pgfqpoint{2.362450in}{3.318341in}}%
\pgfpathlineto{\pgfqpoint{2.371876in}{3.322547in}}%
\pgfpathlineto{\pgfqpoint{2.376588in}{3.322557in}}%
\pgfpathlineto{\pgfqpoint{2.381301in}{3.326739in}}%
\pgfpathlineto{\pgfqpoint{2.404866in}{3.328719in}}%
\pgfpathlineto{\pgfqpoint{2.414292in}{3.332071in}}%
\pgfpathlineto{\pgfqpoint{2.419005in}{3.338182in}}%
\pgfpathlineto{\pgfqpoint{2.428430in}{3.341350in}}%
\pgfpathlineto{\pgfqpoint{2.437856in}{3.355648in}}%
\pgfpathlineto{\pgfqpoint{2.456708in}{3.360229in}}%
\pgfpathlineto{\pgfqpoint{2.466134in}{3.361444in}}%
\pgfpathlineto{\pgfqpoint{2.470847in}{3.362900in}}%
\pgfpathlineto{\pgfqpoint{2.475560in}{3.365754in}}%
\pgfpathlineto{\pgfqpoint{2.494411in}{3.369790in}}%
\pgfpathlineto{\pgfqpoint{2.499124in}{3.369912in}}%
\pgfpathlineto{\pgfqpoint{2.503837in}{3.371507in}}%
\pgfpathlineto{\pgfqpoint{2.532114in}{3.374291in}}%
\pgfpathlineto{\pgfqpoint{2.536827in}{3.376108in}}%
\pgfpathlineto{\pgfqpoint{2.550966in}{3.376878in}}%
\pgfpathlineto{\pgfqpoint{2.560392in}{3.381529in}}%
\pgfpathlineto{\pgfqpoint{2.574531in}{3.383794in}}%
\pgfpathlineto{\pgfqpoint{2.579243in}{3.386895in}}%
\pgfpathlineto{\pgfqpoint{2.583956in}{3.394700in}}%
\pgfpathlineto{\pgfqpoint{2.588669in}{3.398430in}}%
\pgfpathlineto{\pgfqpoint{2.593382in}{3.407268in}}%
\pgfpathlineto{\pgfqpoint{2.598095in}{3.407628in}}%
\pgfpathlineto{\pgfqpoint{2.602808in}{3.410740in}}%
\pgfpathlineto{\pgfqpoint{2.607521in}{3.411492in}}%
\pgfpathlineto{\pgfqpoint{2.612234in}{3.413824in}}%
\pgfpathlineto{\pgfqpoint{2.616947in}{3.418434in}}%
\pgfpathlineto{\pgfqpoint{2.626372in}{3.423874in}}%
\pgfpathlineto{\pgfqpoint{2.635798in}{3.424719in}}%
\pgfpathlineto{\pgfqpoint{2.640511in}{3.426749in}}%
\pgfpathlineto{\pgfqpoint{2.649937in}{3.428287in}}%
\pgfpathlineto{\pgfqpoint{2.668789in}{3.433799in}}%
\pgfpathlineto{\pgfqpoint{2.673502in}{3.436128in}}%
\pgfpathlineto{\pgfqpoint{2.682927in}{3.437017in}}%
\pgfpathlineto{\pgfqpoint{2.687640in}{3.443821in}}%
\pgfpathlineto{\pgfqpoint{2.697066in}{3.445924in}}%
\pgfpathlineto{\pgfqpoint{2.701779in}{3.455072in}}%
\pgfpathlineto{\pgfqpoint{2.706492in}{3.458872in}}%
\pgfpathlineto{\pgfqpoint{2.715918in}{3.459825in}}%
\pgfpathlineto{\pgfqpoint{2.720631in}{3.463794in}}%
\pgfpathlineto{\pgfqpoint{2.734769in}{3.465608in}}%
\pgfpathlineto{\pgfqpoint{2.744195in}{3.466921in}}%
\pgfpathlineto{\pgfqpoint{2.767760in}{3.471788in}}%
\pgfpathlineto{\pgfqpoint{2.772473in}{3.475029in}}%
\pgfpathlineto{\pgfqpoint{2.814889in}{3.478888in}}%
\pgfpathlineto{\pgfqpoint{2.819602in}{3.479742in}}%
\pgfpathlineto{\pgfqpoint{2.824315in}{3.485412in}}%
\pgfpathlineto{\pgfqpoint{2.829027in}{3.485731in}}%
\pgfpathlineto{\pgfqpoint{2.833740in}{3.487971in}}%
\pgfpathlineto{\pgfqpoint{2.857305in}{3.489643in}}%
\pgfpathlineto{\pgfqpoint{2.862018in}{3.494868in}}%
\pgfpathlineto{\pgfqpoint{2.866731in}{3.496397in}}%
\pgfpathlineto{\pgfqpoint{2.876156in}{3.512173in}}%
\pgfpathlineto{\pgfqpoint{2.880869in}{3.514594in}}%
\pgfpathlineto{\pgfqpoint{2.885582in}{3.514997in}}%
\pgfpathlineto{\pgfqpoint{2.895008in}{3.520024in}}%
\pgfpathlineto{\pgfqpoint{2.899721in}{3.520036in}}%
\pgfpathlineto{\pgfqpoint{2.913860in}{3.524886in}}%
\pgfpathlineto{\pgfqpoint{2.918573in}{3.524888in}}%
\pgfpathlineto{\pgfqpoint{2.923286in}{3.527219in}}%
\pgfpathlineto{\pgfqpoint{2.927998in}{3.531342in}}%
\pgfpathlineto{\pgfqpoint{2.951563in}{3.533182in}}%
\pgfpathlineto{\pgfqpoint{2.956276in}{3.537269in}}%
\pgfpathlineto{\pgfqpoint{2.960989in}{3.538000in}}%
\pgfpathlineto{\pgfqpoint{2.965702in}{3.545359in}}%
\pgfpathlineto{\pgfqpoint{2.970415in}{3.545699in}}%
\pgfpathlineto{\pgfqpoint{2.975128in}{3.554105in}}%
\pgfpathlineto{\pgfqpoint{2.984553in}{3.562539in}}%
\pgfpathlineto{\pgfqpoint{2.989266in}{3.563111in}}%
\pgfpathlineto{\pgfqpoint{2.993979in}{3.565085in}}%
\pgfpathlineto{\pgfqpoint{3.017544in}{3.568362in}}%
\pgfpathlineto{\pgfqpoint{3.022257in}{3.572322in}}%
\pgfpathlineto{\pgfqpoint{3.064673in}{3.577774in}}%
\pgfpathlineto{\pgfqpoint{3.078811in}{3.582911in}}%
\pgfpathlineto{\pgfqpoint{3.083524in}{3.583073in}}%
\pgfpathlineto{\pgfqpoint{3.107089in}{3.589494in}}%
\pgfpathlineto{\pgfqpoint{3.111802in}{3.599705in}}%
\pgfpathlineto{\pgfqpoint{3.125940in}{3.602118in}}%
\pgfpathlineto{\pgfqpoint{3.130653in}{3.607615in}}%
\pgfpathlineto{\pgfqpoint{3.140079in}{3.608068in}}%
\pgfpathlineto{\pgfqpoint{3.144792in}{3.610528in}}%
\pgfpathlineto{\pgfqpoint{3.149505in}{3.615913in}}%
\pgfpathlineto{\pgfqpoint{3.154218in}{3.616241in}}%
\pgfpathlineto{\pgfqpoint{3.158931in}{3.620774in}}%
\pgfpathlineto{\pgfqpoint{3.163644in}{3.621894in}}%
\pgfpathlineto{\pgfqpoint{3.168357in}{3.698200in}}%
\pgfpathlineto{\pgfqpoint{3.173070in}{3.706196in}}%
\pgfpathlineto{\pgfqpoint{3.177782in}{3.710821in}}%
\pgfpathlineto{\pgfqpoint{3.182495in}{3.713305in}}%
\pgfpathlineto{\pgfqpoint{3.187208in}{3.714091in}}%
\pgfpathlineto{\pgfqpoint{3.191921in}{3.718637in}}%
\pgfpathlineto{\pgfqpoint{3.201347in}{3.719025in}}%
\pgfpathlineto{\pgfqpoint{3.210773in}{3.723886in}}%
\pgfpathlineto{\pgfqpoint{3.215486in}{3.728542in}}%
\pgfpathlineto{\pgfqpoint{3.224912in}{3.730545in}}%
\pgfpathlineto{\pgfqpoint{3.248476in}{3.735665in}}%
\pgfpathlineto{\pgfqpoint{3.253189in}{3.738164in}}%
\pgfpathlineto{\pgfqpoint{3.262615in}{3.750773in}}%
\pgfpathlineto{\pgfqpoint{3.267328in}{3.752714in}}%
\pgfpathlineto{\pgfqpoint{3.272041in}{3.758843in}}%
\pgfpathlineto{\pgfqpoint{3.276753in}{3.760407in}}%
\pgfpathlineto{\pgfqpoint{3.281466in}{3.767709in}}%
\pgfpathlineto{\pgfqpoint{3.295605in}{3.773576in}}%
\pgfpathlineto{\pgfqpoint{3.300318in}{3.780746in}}%
\pgfpathlineto{\pgfqpoint{3.309744in}{3.785079in}}%
\pgfpathlineto{\pgfqpoint{3.319170in}{3.791712in}}%
\pgfpathlineto{\pgfqpoint{3.323883in}{3.793397in}}%
\pgfpathlineto{\pgfqpoint{3.338021in}{3.794941in}}%
\pgfpathlineto{\pgfqpoint{3.347447in}{3.799615in}}%
\pgfpathlineto{\pgfqpoint{3.352160in}{3.799873in}}%
\pgfpathlineto{\pgfqpoint{3.361586in}{3.803581in}}%
\pgfpathlineto{\pgfqpoint{3.366299in}{3.814802in}}%
\pgfpathlineto{\pgfqpoint{3.371012in}{3.849600in}}%
\pgfpathlineto{\pgfqpoint{3.375724in}{3.849904in}}%
\pgfpathlineto{\pgfqpoint{3.380437in}{3.852897in}}%
\pgfpathlineto{\pgfqpoint{3.385150in}{3.862535in}}%
\pgfpathlineto{\pgfqpoint{3.389863in}{3.876768in}}%
\pgfpathlineto{\pgfqpoint{3.394576in}{3.878420in}}%
\pgfpathlineto{\pgfqpoint{3.399289in}{3.921942in}}%
\pgfpathlineto{\pgfqpoint{5.845287in}{3.921942in}}%
\pgfpathlineto{\pgfqpoint{5.845287in}{3.921942in}}%
\pgfusepath{stroke}%
\end{pgfscope}%
\begin{pgfscope}%
\pgfsetrectcap%
\pgfsetmiterjoin%
\pgfsetlinewidth{0.803000pt}%
\definecolor{currentstroke}{rgb}{0.000000,0.000000,0.000000}%
\pgfsetstrokecolor{currentstroke}%
\pgfsetdash{}{0pt}%
\pgfpathmoveto{\pgfqpoint{0.708220in}{2.519156in}}%
\pgfpathlineto{\pgfqpoint{0.708220in}{3.921942in}}%
\pgfusepath{stroke}%
\end{pgfscope}%
\begin{pgfscope}%
\pgfsetrectcap%
\pgfsetmiterjoin%
\pgfsetlinewidth{0.803000pt}%
\definecolor{currentstroke}{rgb}{0.000000,0.000000,0.000000}%
\pgfsetstrokecolor{currentstroke}%
\pgfsetdash{}{0pt}%
\pgfpathmoveto{\pgfqpoint{5.850000in}{2.519156in}}%
\pgfpathlineto{\pgfqpoint{5.850000in}{3.921942in}}%
\pgfusepath{stroke}%
\end{pgfscope}%
\begin{pgfscope}%
\pgfsetrectcap%
\pgfsetmiterjoin%
\pgfsetlinewidth{0.803000pt}%
\definecolor{currentstroke}{rgb}{0.000000,0.000000,0.000000}%
\pgfsetstrokecolor{currentstroke}%
\pgfsetdash{}{0pt}%
\pgfpathmoveto{\pgfqpoint{0.708220in}{2.519156in}}%
\pgfpathlineto{\pgfqpoint{5.850000in}{2.519156in}}%
\pgfusepath{stroke}%
\end{pgfscope}%
\begin{pgfscope}%
\pgfsetrectcap%
\pgfsetmiterjoin%
\pgfsetlinewidth{0.803000pt}%
\definecolor{currentstroke}{rgb}{0.000000,0.000000,0.000000}%
\pgfsetstrokecolor{currentstroke}%
\pgfsetdash{}{0pt}%
\pgfpathmoveto{\pgfqpoint{0.708220in}{3.921942in}}%
\pgfpathlineto{\pgfqpoint{5.850000in}{3.921942in}}%
\pgfusepath{stroke}%
\end{pgfscope}%
\begin{pgfscope}%
\pgfsetbuttcap%
\pgfsetroundjoin%
\pgfsetlinewidth{2.007500pt}%
\definecolor{currentstroke}{rgb}{1.000000,0.843137,0.000000}%
\pgfsetstrokecolor{currentstroke}%
\pgfsetdash{{7.400000pt}{3.200000pt}}{0.000000pt}%
\pgfpathmoveto{\pgfqpoint{4.827505in}{2.960805in}}%
\pgfpathlineto{\pgfqpoint{5.077505in}{2.960805in}}%
\pgfusepath{stroke}%
\end{pgfscope}%
\begin{pgfscope}%
\definecolor{textcolor}{rgb}{0.000000,0.000000,0.000000}%
\pgfsetstrokecolor{textcolor}%
\pgfsetfillcolor{textcolor}%
\pgftext[x=5.102505in,y=2.917055in,left,base]{\color{textcolor}\rmfamily\fontsize{9.000000}{10.800000}\selectfont FT+htd}%
\end{pgfscope}%
\begin{pgfscope}%
\pgfsetbuttcap%
\pgfsetroundjoin%
\pgfsetlinewidth{2.007500pt}%
\definecolor{currentstroke}{rgb}{1.000000,0.694118,0.305882}%
\pgfsetstrokecolor{currentstroke}%
\pgfsetdash{{2.000000pt}{3.300000pt}}{0.000000pt}%
\pgfpathmoveto{\pgfqpoint{4.827505in}{2.799005in}}%
\pgfpathlineto{\pgfqpoint{5.077505in}{2.799005in}}%
\pgfusepath{stroke}%
\end{pgfscope}%
\begin{pgfscope}%
\definecolor{textcolor}{rgb}{0.000000,0.000000,0.000000}%
\pgfsetstrokecolor{textcolor}%
\pgfsetfillcolor{textcolor}%
\pgftext[x=5.102505in,y=2.755255in,left,base]{\color{textcolor}\rmfamily\fontsize{9.000000}{10.800000}\selectfont FT+Flow}%
\end{pgfscope}%
\begin{pgfscope}%
\pgfsetrectcap%
\pgfsetroundjoin%
\pgfsetlinewidth{2.007500pt}%
\definecolor{currentstroke}{rgb}{0.980392,0.529412,0.458824}%
\pgfsetstrokecolor{currentstroke}%
\pgfsetdash{}{0pt}%
\pgfpathmoveto{\pgfqpoint{4.827505in}{2.637206in}}%
\pgfpathlineto{\pgfqpoint{5.077505in}{2.637206in}}%
\pgfusepath{stroke}%
\end{pgfscope}%
\begin{pgfscope}%
\definecolor{textcolor}{rgb}{0.000000,0.000000,0.000000}%
\pgfsetstrokecolor{textcolor}%
\pgfsetfillcolor{textcolor}%
\pgftext[x=5.102505in,y=2.593456in,left,base]{\color{textcolor}\rmfamily\fontsize{9.000000}{10.800000}\selectfont FT+Tamaki}%
\end{pgfscope}%
\begin{pgfscope}%
\pgfsetbuttcap%
\pgfsetmiterjoin%
\definecolor{currentfill}{rgb}{1.000000,1.000000,1.000000}%
\pgfsetfillcolor{currentfill}%
\pgfsetlinewidth{0.000000pt}%
\definecolor{currentstroke}{rgb}{0.000000,0.000000,0.000000}%
\pgfsetstrokecolor{currentstroke}%
\pgfsetstrokeopacity{0.000000}%
\pgfsetdash{}{0pt}%
\pgfpathmoveto{\pgfqpoint{0.708220in}{0.535823in}}%
\pgfpathlineto{\pgfqpoint{5.850000in}{0.535823in}}%
\pgfpathlineto{\pgfqpoint{5.850000in}{1.938609in}}%
\pgfpathlineto{\pgfqpoint{0.708220in}{1.938609in}}%
\pgfpathclose%
\pgfusepath{fill}%
\end{pgfscope}%
\begin{pgfscope}%
\pgfsetbuttcap%
\pgfsetroundjoin%
\definecolor{currentfill}{rgb}{0.000000,0.000000,0.000000}%
\pgfsetfillcolor{currentfill}%
\pgfsetlinewidth{0.803000pt}%
\definecolor{currentstroke}{rgb}{0.000000,0.000000,0.000000}%
\pgfsetstrokecolor{currentstroke}%
\pgfsetdash{}{0pt}%
\pgfsys@defobject{currentmarker}{\pgfqpoint{0.000000in}{-0.048611in}}{\pgfqpoint{0.000000in}{0.000000in}}{%
\pgfpathmoveto{\pgfqpoint{0.000000in}{0.000000in}}%
\pgfpathlineto{\pgfqpoint{0.000000in}{-0.048611in}}%
\pgfusepath{stroke,fill}%
}%
\begin{pgfscope}%
\pgfsys@transformshift{0.708220in}{0.535823in}%
\pgfsys@useobject{currentmarker}{}%
\end{pgfscope}%
\end{pgfscope}%
\begin{pgfscope}%
\definecolor{textcolor}{rgb}{0.000000,0.000000,0.000000}%
\pgfsetstrokecolor{textcolor}%
\pgfsetfillcolor{textcolor}%
\pgftext[x=0.708220in,y=0.438600in,,top]{\color{textcolor}\rmfamily\fontsize{9.000000}{10.800000}\selectfont \(\displaystyle {0}\)}%
\end{pgfscope}%
\begin{pgfscope}%
\pgfsetbuttcap%
\pgfsetroundjoin%
\definecolor{currentfill}{rgb}{0.000000,0.000000,0.000000}%
\pgfsetfillcolor{currentfill}%
\pgfsetlinewidth{0.803000pt}%
\definecolor{currentstroke}{rgb}{0.000000,0.000000,0.000000}%
\pgfsetstrokecolor{currentstroke}%
\pgfsetdash{}{0pt}%
\pgfsys@defobject{currentmarker}{\pgfqpoint{0.000000in}{-0.048611in}}{\pgfqpoint{0.000000in}{0.000000in}}{%
\pgfpathmoveto{\pgfqpoint{0.000000in}{0.000000in}}%
\pgfpathlineto{\pgfqpoint{0.000000in}{-0.048611in}}%
\pgfusepath{stroke,fill}%
}%
\begin{pgfscope}%
\pgfsys@transformshift{1.650801in}{0.535823in}%
\pgfsys@useobject{currentmarker}{}%
\end{pgfscope}%
\end{pgfscope}%
\begin{pgfscope}%
\definecolor{textcolor}{rgb}{0.000000,0.000000,0.000000}%
\pgfsetstrokecolor{textcolor}%
\pgfsetfillcolor{textcolor}%
\pgftext[x=1.650801in,y=0.438600in,,top]{\color{textcolor}\rmfamily\fontsize{9.000000}{10.800000}\selectfont \(\displaystyle {200}\)}%
\end{pgfscope}%
\begin{pgfscope}%
\pgfsetbuttcap%
\pgfsetroundjoin%
\definecolor{currentfill}{rgb}{0.000000,0.000000,0.000000}%
\pgfsetfillcolor{currentfill}%
\pgfsetlinewidth{0.803000pt}%
\definecolor{currentstroke}{rgb}{0.000000,0.000000,0.000000}%
\pgfsetstrokecolor{currentstroke}%
\pgfsetdash{}{0pt}%
\pgfsys@defobject{currentmarker}{\pgfqpoint{0.000000in}{-0.048611in}}{\pgfqpoint{0.000000in}{0.000000in}}{%
\pgfpathmoveto{\pgfqpoint{0.000000in}{0.000000in}}%
\pgfpathlineto{\pgfqpoint{0.000000in}{-0.048611in}}%
\pgfusepath{stroke,fill}%
}%
\begin{pgfscope}%
\pgfsys@transformshift{2.593382in}{0.535823in}%
\pgfsys@useobject{currentmarker}{}%
\end{pgfscope}%
\end{pgfscope}%
\begin{pgfscope}%
\definecolor{textcolor}{rgb}{0.000000,0.000000,0.000000}%
\pgfsetstrokecolor{textcolor}%
\pgfsetfillcolor{textcolor}%
\pgftext[x=2.593382in,y=0.438600in,,top]{\color{textcolor}\rmfamily\fontsize{9.000000}{10.800000}\selectfont \(\displaystyle {400}\)}%
\end{pgfscope}%
\begin{pgfscope}%
\pgfsetbuttcap%
\pgfsetroundjoin%
\definecolor{currentfill}{rgb}{0.000000,0.000000,0.000000}%
\pgfsetfillcolor{currentfill}%
\pgfsetlinewidth{0.803000pt}%
\definecolor{currentstroke}{rgb}{0.000000,0.000000,0.000000}%
\pgfsetstrokecolor{currentstroke}%
\pgfsetdash{}{0pt}%
\pgfsys@defobject{currentmarker}{\pgfqpoint{0.000000in}{-0.048611in}}{\pgfqpoint{0.000000in}{0.000000in}}{%
\pgfpathmoveto{\pgfqpoint{0.000000in}{0.000000in}}%
\pgfpathlineto{\pgfqpoint{0.000000in}{-0.048611in}}%
\pgfusepath{stroke,fill}%
}%
\begin{pgfscope}%
\pgfsys@transformshift{3.535963in}{0.535823in}%
\pgfsys@useobject{currentmarker}{}%
\end{pgfscope}%
\end{pgfscope}%
\begin{pgfscope}%
\definecolor{textcolor}{rgb}{0.000000,0.000000,0.000000}%
\pgfsetstrokecolor{textcolor}%
\pgfsetfillcolor{textcolor}%
\pgftext[x=3.535963in,y=0.438600in,,top]{\color{textcolor}\rmfamily\fontsize{9.000000}{10.800000}\selectfont \(\displaystyle {600}\)}%
\end{pgfscope}%
\begin{pgfscope}%
\pgfsetbuttcap%
\pgfsetroundjoin%
\definecolor{currentfill}{rgb}{0.000000,0.000000,0.000000}%
\pgfsetfillcolor{currentfill}%
\pgfsetlinewidth{0.803000pt}%
\definecolor{currentstroke}{rgb}{0.000000,0.000000,0.000000}%
\pgfsetstrokecolor{currentstroke}%
\pgfsetdash{}{0pt}%
\pgfsys@defobject{currentmarker}{\pgfqpoint{0.000000in}{-0.048611in}}{\pgfqpoint{0.000000in}{0.000000in}}{%
\pgfpathmoveto{\pgfqpoint{0.000000in}{0.000000in}}%
\pgfpathlineto{\pgfqpoint{0.000000in}{-0.048611in}}%
\pgfusepath{stroke,fill}%
}%
\begin{pgfscope}%
\pgfsys@transformshift{4.478544in}{0.535823in}%
\pgfsys@useobject{currentmarker}{}%
\end{pgfscope}%
\end{pgfscope}%
\begin{pgfscope}%
\definecolor{textcolor}{rgb}{0.000000,0.000000,0.000000}%
\pgfsetstrokecolor{textcolor}%
\pgfsetfillcolor{textcolor}%
\pgftext[x=4.478544in,y=0.438600in,,top]{\color{textcolor}\rmfamily\fontsize{9.000000}{10.800000}\selectfont \(\displaystyle {800}\)}%
\end{pgfscope}%
\begin{pgfscope}%
\pgfsetbuttcap%
\pgfsetroundjoin%
\definecolor{currentfill}{rgb}{0.000000,0.000000,0.000000}%
\pgfsetfillcolor{currentfill}%
\pgfsetlinewidth{0.803000pt}%
\definecolor{currentstroke}{rgb}{0.000000,0.000000,0.000000}%
\pgfsetstrokecolor{currentstroke}%
\pgfsetdash{}{0pt}%
\pgfsys@defobject{currentmarker}{\pgfqpoint{0.000000in}{-0.048611in}}{\pgfqpoint{0.000000in}{0.000000in}}{%
\pgfpathmoveto{\pgfqpoint{0.000000in}{0.000000in}}%
\pgfpathlineto{\pgfqpoint{0.000000in}{-0.048611in}}%
\pgfusepath{stroke,fill}%
}%
\begin{pgfscope}%
\pgfsys@transformshift{5.421126in}{0.535823in}%
\pgfsys@useobject{currentmarker}{}%
\end{pgfscope}%
\end{pgfscope}%
\begin{pgfscope}%
\definecolor{textcolor}{rgb}{0.000000,0.000000,0.000000}%
\pgfsetstrokecolor{textcolor}%
\pgfsetfillcolor{textcolor}%
\pgftext[x=5.421126in,y=0.438600in,,top]{\color{textcolor}\rmfamily\fontsize{9.000000}{10.800000}\selectfont \(\displaystyle {1000}\)}%
\end{pgfscope}%
\begin{pgfscope}%
\definecolor{textcolor}{rgb}{0.000000,0.000000,0.000000}%
\pgfsetstrokecolor{textcolor}%
\pgfsetfillcolor{textcolor}%
\pgftext[x=3.279110in,y=0.272655in,,top]{\color{textcolor}\rmfamily\fontsize{10.000000}{12.000000}\selectfont Number of benchmarks solved}%
\end{pgfscope}%
\begin{pgfscope}%
\pgfsetbuttcap%
\pgfsetroundjoin%
\definecolor{currentfill}{rgb}{0.000000,0.000000,0.000000}%
\pgfsetfillcolor{currentfill}%
\pgfsetlinewidth{0.803000pt}%
\definecolor{currentstroke}{rgb}{0.000000,0.000000,0.000000}%
\pgfsetstrokecolor{currentstroke}%
\pgfsetdash{}{0pt}%
\pgfsys@defobject{currentmarker}{\pgfqpoint{-0.048611in}{0.000000in}}{\pgfqpoint{-0.000000in}{0.000000in}}{%
\pgfpathmoveto{\pgfqpoint{-0.000000in}{0.000000in}}%
\pgfpathlineto{\pgfqpoint{-0.048611in}{0.000000in}}%
\pgfusepath{stroke,fill}%
}%
\begin{pgfscope}%
\pgfsys@transformshift{0.708220in}{0.535823in}%
\pgfsys@useobject{currentmarker}{}%
\end{pgfscope}%
\end{pgfscope}%
\begin{pgfscope}%
\definecolor{textcolor}{rgb}{0.000000,0.000000,0.000000}%
\pgfsetstrokecolor{textcolor}%
\pgfsetfillcolor{textcolor}%
\pgftext[x=0.344411in, y=0.491098in, left, base]{\color{textcolor}\rmfamily\fontsize{9.000000}{10.800000}\selectfont \(\displaystyle {10^{-1}}\)}%
\end{pgfscope}%
\begin{pgfscope}%
\pgfsetbuttcap%
\pgfsetroundjoin%
\definecolor{currentfill}{rgb}{0.000000,0.000000,0.000000}%
\pgfsetfillcolor{currentfill}%
\pgfsetlinewidth{0.803000pt}%
\definecolor{currentstroke}{rgb}{0.000000,0.000000,0.000000}%
\pgfsetstrokecolor{currentstroke}%
\pgfsetdash{}{0pt}%
\pgfsys@defobject{currentmarker}{\pgfqpoint{-0.048611in}{0.000000in}}{\pgfqpoint{-0.000000in}{0.000000in}}{%
\pgfpathmoveto{\pgfqpoint{-0.000000in}{0.000000in}}%
\pgfpathlineto{\pgfqpoint{-0.048611in}{0.000000in}}%
\pgfusepath{stroke,fill}%
}%
\begin{pgfscope}%
\pgfsys@transformshift{0.708220in}{0.886519in}%
\pgfsys@useobject{currentmarker}{}%
\end{pgfscope}%
\end{pgfscope}%
\begin{pgfscope}%
\definecolor{textcolor}{rgb}{0.000000,0.000000,0.000000}%
\pgfsetstrokecolor{textcolor}%
\pgfsetfillcolor{textcolor}%
\pgftext[x=0.424657in, y=0.841794in, left, base]{\color{textcolor}\rmfamily\fontsize{9.000000}{10.800000}\selectfont \(\displaystyle {10^{0}}\)}%
\end{pgfscope}%
\begin{pgfscope}%
\pgfsetbuttcap%
\pgfsetroundjoin%
\definecolor{currentfill}{rgb}{0.000000,0.000000,0.000000}%
\pgfsetfillcolor{currentfill}%
\pgfsetlinewidth{0.803000pt}%
\definecolor{currentstroke}{rgb}{0.000000,0.000000,0.000000}%
\pgfsetstrokecolor{currentstroke}%
\pgfsetdash{}{0pt}%
\pgfsys@defobject{currentmarker}{\pgfqpoint{-0.048611in}{0.000000in}}{\pgfqpoint{-0.000000in}{0.000000in}}{%
\pgfpathmoveto{\pgfqpoint{-0.000000in}{0.000000in}}%
\pgfpathlineto{\pgfqpoint{-0.048611in}{0.000000in}}%
\pgfusepath{stroke,fill}%
}%
\begin{pgfscope}%
\pgfsys@transformshift{0.708220in}{1.237216in}%
\pgfsys@useobject{currentmarker}{}%
\end{pgfscope}%
\end{pgfscope}%
\begin{pgfscope}%
\definecolor{textcolor}{rgb}{0.000000,0.000000,0.000000}%
\pgfsetstrokecolor{textcolor}%
\pgfsetfillcolor{textcolor}%
\pgftext[x=0.424657in, y=1.192491in, left, base]{\color{textcolor}\rmfamily\fontsize{9.000000}{10.800000}\selectfont \(\displaystyle {10^{1}}\)}%
\end{pgfscope}%
\begin{pgfscope}%
\pgfsetbuttcap%
\pgfsetroundjoin%
\definecolor{currentfill}{rgb}{0.000000,0.000000,0.000000}%
\pgfsetfillcolor{currentfill}%
\pgfsetlinewidth{0.803000pt}%
\definecolor{currentstroke}{rgb}{0.000000,0.000000,0.000000}%
\pgfsetstrokecolor{currentstroke}%
\pgfsetdash{}{0pt}%
\pgfsys@defobject{currentmarker}{\pgfqpoint{-0.048611in}{0.000000in}}{\pgfqpoint{-0.000000in}{0.000000in}}{%
\pgfpathmoveto{\pgfqpoint{-0.000000in}{0.000000in}}%
\pgfpathlineto{\pgfqpoint{-0.048611in}{0.000000in}}%
\pgfusepath{stroke,fill}%
}%
\begin{pgfscope}%
\pgfsys@transformshift{0.708220in}{1.587912in}%
\pgfsys@useobject{currentmarker}{}%
\end{pgfscope}%
\end{pgfscope}%
\begin{pgfscope}%
\definecolor{textcolor}{rgb}{0.000000,0.000000,0.000000}%
\pgfsetstrokecolor{textcolor}%
\pgfsetfillcolor{textcolor}%
\pgftext[x=0.424657in, y=1.543187in, left, base]{\color{textcolor}\rmfamily\fontsize{9.000000}{10.800000}\selectfont \(\displaystyle {10^{2}}\)}%
\end{pgfscope}%
\begin{pgfscope}%
\pgfsetbuttcap%
\pgfsetroundjoin%
\definecolor{currentfill}{rgb}{0.000000,0.000000,0.000000}%
\pgfsetfillcolor{currentfill}%
\pgfsetlinewidth{0.803000pt}%
\definecolor{currentstroke}{rgb}{0.000000,0.000000,0.000000}%
\pgfsetstrokecolor{currentstroke}%
\pgfsetdash{}{0pt}%
\pgfsys@defobject{currentmarker}{\pgfqpoint{-0.048611in}{0.000000in}}{\pgfqpoint{-0.000000in}{0.000000in}}{%
\pgfpathmoveto{\pgfqpoint{-0.000000in}{0.000000in}}%
\pgfpathlineto{\pgfqpoint{-0.048611in}{0.000000in}}%
\pgfusepath{stroke,fill}%
}%
\begin{pgfscope}%
\pgfsys@transformshift{0.708220in}{1.938609in}%
\pgfsys@useobject{currentmarker}{}%
\end{pgfscope}%
\end{pgfscope}%
\begin{pgfscope}%
\definecolor{textcolor}{rgb}{0.000000,0.000000,0.000000}%
\pgfsetstrokecolor{textcolor}%
\pgfsetfillcolor{textcolor}%
\pgftext[x=0.424657in, y=1.893884in, left, base]{\color{textcolor}\rmfamily\fontsize{9.000000}{10.800000}\selectfont \(\displaystyle {10^{3}}\)}%
\end{pgfscope}%
\begin{pgfscope}%
\pgfsetbuttcap%
\pgfsetroundjoin%
\definecolor{currentfill}{rgb}{0.000000,0.000000,0.000000}%
\pgfsetfillcolor{currentfill}%
\pgfsetlinewidth{0.602250pt}%
\definecolor{currentstroke}{rgb}{0.000000,0.000000,0.000000}%
\pgfsetstrokecolor{currentstroke}%
\pgfsetdash{}{0pt}%
\pgfsys@defobject{currentmarker}{\pgfqpoint{-0.027778in}{0.000000in}}{\pgfqpoint{-0.000000in}{0.000000in}}{%
\pgfpathmoveto{\pgfqpoint{-0.000000in}{0.000000in}}%
\pgfpathlineto{\pgfqpoint{-0.027778in}{0.000000in}}%
\pgfusepath{stroke,fill}%
}%
\begin{pgfscope}%
\pgfsys@transformshift{0.708220in}{0.641393in}%
\pgfsys@useobject{currentmarker}{}%
\end{pgfscope}%
\end{pgfscope}%
\begin{pgfscope}%
\pgfsetbuttcap%
\pgfsetroundjoin%
\definecolor{currentfill}{rgb}{0.000000,0.000000,0.000000}%
\pgfsetfillcolor{currentfill}%
\pgfsetlinewidth{0.602250pt}%
\definecolor{currentstroke}{rgb}{0.000000,0.000000,0.000000}%
\pgfsetstrokecolor{currentstroke}%
\pgfsetdash{}{0pt}%
\pgfsys@defobject{currentmarker}{\pgfqpoint{-0.027778in}{0.000000in}}{\pgfqpoint{-0.000000in}{0.000000in}}{%
\pgfpathmoveto{\pgfqpoint{-0.000000in}{0.000000in}}%
\pgfpathlineto{\pgfqpoint{-0.027778in}{0.000000in}}%
\pgfusepath{stroke,fill}%
}%
\begin{pgfscope}%
\pgfsys@transformshift{0.708220in}{0.703147in}%
\pgfsys@useobject{currentmarker}{}%
\end{pgfscope}%
\end{pgfscope}%
\begin{pgfscope}%
\pgfsetbuttcap%
\pgfsetroundjoin%
\definecolor{currentfill}{rgb}{0.000000,0.000000,0.000000}%
\pgfsetfillcolor{currentfill}%
\pgfsetlinewidth{0.602250pt}%
\definecolor{currentstroke}{rgb}{0.000000,0.000000,0.000000}%
\pgfsetstrokecolor{currentstroke}%
\pgfsetdash{}{0pt}%
\pgfsys@defobject{currentmarker}{\pgfqpoint{-0.027778in}{0.000000in}}{\pgfqpoint{-0.000000in}{0.000000in}}{%
\pgfpathmoveto{\pgfqpoint{-0.000000in}{0.000000in}}%
\pgfpathlineto{\pgfqpoint{-0.027778in}{0.000000in}}%
\pgfusepath{stroke,fill}%
}%
\begin{pgfscope}%
\pgfsys@transformshift{0.708220in}{0.746963in}%
\pgfsys@useobject{currentmarker}{}%
\end{pgfscope}%
\end{pgfscope}%
\begin{pgfscope}%
\pgfsetbuttcap%
\pgfsetroundjoin%
\definecolor{currentfill}{rgb}{0.000000,0.000000,0.000000}%
\pgfsetfillcolor{currentfill}%
\pgfsetlinewidth{0.602250pt}%
\definecolor{currentstroke}{rgb}{0.000000,0.000000,0.000000}%
\pgfsetstrokecolor{currentstroke}%
\pgfsetdash{}{0pt}%
\pgfsys@defobject{currentmarker}{\pgfqpoint{-0.027778in}{0.000000in}}{\pgfqpoint{-0.000000in}{0.000000in}}{%
\pgfpathmoveto{\pgfqpoint{-0.000000in}{0.000000in}}%
\pgfpathlineto{\pgfqpoint{-0.027778in}{0.000000in}}%
\pgfusepath{stroke,fill}%
}%
\begin{pgfscope}%
\pgfsys@transformshift{0.708220in}{0.780949in}%
\pgfsys@useobject{currentmarker}{}%
\end{pgfscope}%
\end{pgfscope}%
\begin{pgfscope}%
\pgfsetbuttcap%
\pgfsetroundjoin%
\definecolor{currentfill}{rgb}{0.000000,0.000000,0.000000}%
\pgfsetfillcolor{currentfill}%
\pgfsetlinewidth{0.602250pt}%
\definecolor{currentstroke}{rgb}{0.000000,0.000000,0.000000}%
\pgfsetstrokecolor{currentstroke}%
\pgfsetdash{}{0pt}%
\pgfsys@defobject{currentmarker}{\pgfqpoint{-0.027778in}{0.000000in}}{\pgfqpoint{-0.000000in}{0.000000in}}{%
\pgfpathmoveto{\pgfqpoint{-0.000000in}{0.000000in}}%
\pgfpathlineto{\pgfqpoint{-0.027778in}{0.000000in}}%
\pgfusepath{stroke,fill}%
}%
\begin{pgfscope}%
\pgfsys@transformshift{0.708220in}{0.808718in}%
\pgfsys@useobject{currentmarker}{}%
\end{pgfscope}%
\end{pgfscope}%
\begin{pgfscope}%
\pgfsetbuttcap%
\pgfsetroundjoin%
\definecolor{currentfill}{rgb}{0.000000,0.000000,0.000000}%
\pgfsetfillcolor{currentfill}%
\pgfsetlinewidth{0.602250pt}%
\definecolor{currentstroke}{rgb}{0.000000,0.000000,0.000000}%
\pgfsetstrokecolor{currentstroke}%
\pgfsetdash{}{0pt}%
\pgfsys@defobject{currentmarker}{\pgfqpoint{-0.027778in}{0.000000in}}{\pgfqpoint{-0.000000in}{0.000000in}}{%
\pgfpathmoveto{\pgfqpoint{-0.000000in}{0.000000in}}%
\pgfpathlineto{\pgfqpoint{-0.027778in}{0.000000in}}%
\pgfusepath{stroke,fill}%
}%
\begin{pgfscope}%
\pgfsys@transformshift{0.708220in}{0.832196in}%
\pgfsys@useobject{currentmarker}{}%
\end{pgfscope}%
\end{pgfscope}%
\begin{pgfscope}%
\pgfsetbuttcap%
\pgfsetroundjoin%
\definecolor{currentfill}{rgb}{0.000000,0.000000,0.000000}%
\pgfsetfillcolor{currentfill}%
\pgfsetlinewidth{0.602250pt}%
\definecolor{currentstroke}{rgb}{0.000000,0.000000,0.000000}%
\pgfsetstrokecolor{currentstroke}%
\pgfsetdash{}{0pt}%
\pgfsys@defobject{currentmarker}{\pgfqpoint{-0.027778in}{0.000000in}}{\pgfqpoint{-0.000000in}{0.000000in}}{%
\pgfpathmoveto{\pgfqpoint{-0.000000in}{0.000000in}}%
\pgfpathlineto{\pgfqpoint{-0.027778in}{0.000000in}}%
\pgfusepath{stroke,fill}%
}%
\begin{pgfscope}%
\pgfsys@transformshift{0.708220in}{0.852533in}%
\pgfsys@useobject{currentmarker}{}%
\end{pgfscope}%
\end{pgfscope}%
\begin{pgfscope}%
\pgfsetbuttcap%
\pgfsetroundjoin%
\definecolor{currentfill}{rgb}{0.000000,0.000000,0.000000}%
\pgfsetfillcolor{currentfill}%
\pgfsetlinewidth{0.602250pt}%
\definecolor{currentstroke}{rgb}{0.000000,0.000000,0.000000}%
\pgfsetstrokecolor{currentstroke}%
\pgfsetdash{}{0pt}%
\pgfsys@defobject{currentmarker}{\pgfqpoint{-0.027778in}{0.000000in}}{\pgfqpoint{-0.000000in}{0.000000in}}{%
\pgfpathmoveto{\pgfqpoint{-0.000000in}{0.000000in}}%
\pgfpathlineto{\pgfqpoint{-0.027778in}{0.000000in}}%
\pgfusepath{stroke,fill}%
}%
\begin{pgfscope}%
\pgfsys@transformshift{0.708220in}{0.870472in}%
\pgfsys@useobject{currentmarker}{}%
\end{pgfscope}%
\end{pgfscope}%
\begin{pgfscope}%
\pgfsetbuttcap%
\pgfsetroundjoin%
\definecolor{currentfill}{rgb}{0.000000,0.000000,0.000000}%
\pgfsetfillcolor{currentfill}%
\pgfsetlinewidth{0.602250pt}%
\definecolor{currentstroke}{rgb}{0.000000,0.000000,0.000000}%
\pgfsetstrokecolor{currentstroke}%
\pgfsetdash{}{0pt}%
\pgfsys@defobject{currentmarker}{\pgfqpoint{-0.027778in}{0.000000in}}{\pgfqpoint{-0.000000in}{0.000000in}}{%
\pgfpathmoveto{\pgfqpoint{-0.000000in}{0.000000in}}%
\pgfpathlineto{\pgfqpoint{-0.027778in}{0.000000in}}%
\pgfusepath{stroke,fill}%
}%
\begin{pgfscope}%
\pgfsys@transformshift{0.708220in}{0.992089in}%
\pgfsys@useobject{currentmarker}{}%
\end{pgfscope}%
\end{pgfscope}%
\begin{pgfscope}%
\pgfsetbuttcap%
\pgfsetroundjoin%
\definecolor{currentfill}{rgb}{0.000000,0.000000,0.000000}%
\pgfsetfillcolor{currentfill}%
\pgfsetlinewidth{0.602250pt}%
\definecolor{currentstroke}{rgb}{0.000000,0.000000,0.000000}%
\pgfsetstrokecolor{currentstroke}%
\pgfsetdash{}{0pt}%
\pgfsys@defobject{currentmarker}{\pgfqpoint{-0.027778in}{0.000000in}}{\pgfqpoint{-0.000000in}{0.000000in}}{%
\pgfpathmoveto{\pgfqpoint{-0.000000in}{0.000000in}}%
\pgfpathlineto{\pgfqpoint{-0.027778in}{0.000000in}}%
\pgfusepath{stroke,fill}%
}%
\begin{pgfscope}%
\pgfsys@transformshift{0.708220in}{1.053844in}%
\pgfsys@useobject{currentmarker}{}%
\end{pgfscope}%
\end{pgfscope}%
\begin{pgfscope}%
\pgfsetbuttcap%
\pgfsetroundjoin%
\definecolor{currentfill}{rgb}{0.000000,0.000000,0.000000}%
\pgfsetfillcolor{currentfill}%
\pgfsetlinewidth{0.602250pt}%
\definecolor{currentstroke}{rgb}{0.000000,0.000000,0.000000}%
\pgfsetstrokecolor{currentstroke}%
\pgfsetdash{}{0pt}%
\pgfsys@defobject{currentmarker}{\pgfqpoint{-0.027778in}{0.000000in}}{\pgfqpoint{-0.000000in}{0.000000in}}{%
\pgfpathmoveto{\pgfqpoint{-0.000000in}{0.000000in}}%
\pgfpathlineto{\pgfqpoint{-0.027778in}{0.000000in}}%
\pgfusepath{stroke,fill}%
}%
\begin{pgfscope}%
\pgfsys@transformshift{0.708220in}{1.097659in}%
\pgfsys@useobject{currentmarker}{}%
\end{pgfscope}%
\end{pgfscope}%
\begin{pgfscope}%
\pgfsetbuttcap%
\pgfsetroundjoin%
\definecolor{currentfill}{rgb}{0.000000,0.000000,0.000000}%
\pgfsetfillcolor{currentfill}%
\pgfsetlinewidth{0.602250pt}%
\definecolor{currentstroke}{rgb}{0.000000,0.000000,0.000000}%
\pgfsetstrokecolor{currentstroke}%
\pgfsetdash{}{0pt}%
\pgfsys@defobject{currentmarker}{\pgfqpoint{-0.027778in}{0.000000in}}{\pgfqpoint{-0.000000in}{0.000000in}}{%
\pgfpathmoveto{\pgfqpoint{-0.000000in}{0.000000in}}%
\pgfpathlineto{\pgfqpoint{-0.027778in}{0.000000in}}%
\pgfusepath{stroke,fill}%
}%
\begin{pgfscope}%
\pgfsys@transformshift{0.708220in}{1.131645in}%
\pgfsys@useobject{currentmarker}{}%
\end{pgfscope}%
\end{pgfscope}%
\begin{pgfscope}%
\pgfsetbuttcap%
\pgfsetroundjoin%
\definecolor{currentfill}{rgb}{0.000000,0.000000,0.000000}%
\pgfsetfillcolor{currentfill}%
\pgfsetlinewidth{0.602250pt}%
\definecolor{currentstroke}{rgb}{0.000000,0.000000,0.000000}%
\pgfsetstrokecolor{currentstroke}%
\pgfsetdash{}{0pt}%
\pgfsys@defobject{currentmarker}{\pgfqpoint{-0.027778in}{0.000000in}}{\pgfqpoint{-0.000000in}{0.000000in}}{%
\pgfpathmoveto{\pgfqpoint{-0.000000in}{0.000000in}}%
\pgfpathlineto{\pgfqpoint{-0.027778in}{0.000000in}}%
\pgfusepath{stroke,fill}%
}%
\begin{pgfscope}%
\pgfsys@transformshift{0.708220in}{1.159414in}%
\pgfsys@useobject{currentmarker}{}%
\end{pgfscope}%
\end{pgfscope}%
\begin{pgfscope}%
\pgfsetbuttcap%
\pgfsetroundjoin%
\definecolor{currentfill}{rgb}{0.000000,0.000000,0.000000}%
\pgfsetfillcolor{currentfill}%
\pgfsetlinewidth{0.602250pt}%
\definecolor{currentstroke}{rgb}{0.000000,0.000000,0.000000}%
\pgfsetstrokecolor{currentstroke}%
\pgfsetdash{}{0pt}%
\pgfsys@defobject{currentmarker}{\pgfqpoint{-0.027778in}{0.000000in}}{\pgfqpoint{-0.000000in}{0.000000in}}{%
\pgfpathmoveto{\pgfqpoint{-0.000000in}{0.000000in}}%
\pgfpathlineto{\pgfqpoint{-0.027778in}{0.000000in}}%
\pgfusepath{stroke,fill}%
}%
\begin{pgfscope}%
\pgfsys@transformshift{0.708220in}{1.182892in}%
\pgfsys@useobject{currentmarker}{}%
\end{pgfscope}%
\end{pgfscope}%
\begin{pgfscope}%
\pgfsetbuttcap%
\pgfsetroundjoin%
\definecolor{currentfill}{rgb}{0.000000,0.000000,0.000000}%
\pgfsetfillcolor{currentfill}%
\pgfsetlinewidth{0.602250pt}%
\definecolor{currentstroke}{rgb}{0.000000,0.000000,0.000000}%
\pgfsetstrokecolor{currentstroke}%
\pgfsetdash{}{0pt}%
\pgfsys@defobject{currentmarker}{\pgfqpoint{-0.027778in}{0.000000in}}{\pgfqpoint{-0.000000in}{0.000000in}}{%
\pgfpathmoveto{\pgfqpoint{-0.000000in}{0.000000in}}%
\pgfpathlineto{\pgfqpoint{-0.027778in}{0.000000in}}%
\pgfusepath{stroke,fill}%
}%
\begin{pgfscope}%
\pgfsys@transformshift{0.708220in}{1.203230in}%
\pgfsys@useobject{currentmarker}{}%
\end{pgfscope}%
\end{pgfscope}%
\begin{pgfscope}%
\pgfsetbuttcap%
\pgfsetroundjoin%
\definecolor{currentfill}{rgb}{0.000000,0.000000,0.000000}%
\pgfsetfillcolor{currentfill}%
\pgfsetlinewidth{0.602250pt}%
\definecolor{currentstroke}{rgb}{0.000000,0.000000,0.000000}%
\pgfsetstrokecolor{currentstroke}%
\pgfsetdash{}{0pt}%
\pgfsys@defobject{currentmarker}{\pgfqpoint{-0.027778in}{0.000000in}}{\pgfqpoint{-0.000000in}{0.000000in}}{%
\pgfpathmoveto{\pgfqpoint{-0.000000in}{0.000000in}}%
\pgfpathlineto{\pgfqpoint{-0.027778in}{0.000000in}}%
\pgfusepath{stroke,fill}%
}%
\begin{pgfscope}%
\pgfsys@transformshift{0.708220in}{1.221169in}%
\pgfsys@useobject{currentmarker}{}%
\end{pgfscope}%
\end{pgfscope}%
\begin{pgfscope}%
\pgfsetbuttcap%
\pgfsetroundjoin%
\definecolor{currentfill}{rgb}{0.000000,0.000000,0.000000}%
\pgfsetfillcolor{currentfill}%
\pgfsetlinewidth{0.602250pt}%
\definecolor{currentstroke}{rgb}{0.000000,0.000000,0.000000}%
\pgfsetstrokecolor{currentstroke}%
\pgfsetdash{}{0pt}%
\pgfsys@defobject{currentmarker}{\pgfqpoint{-0.027778in}{0.000000in}}{\pgfqpoint{-0.000000in}{0.000000in}}{%
\pgfpathmoveto{\pgfqpoint{-0.000000in}{0.000000in}}%
\pgfpathlineto{\pgfqpoint{-0.027778in}{0.000000in}}%
\pgfusepath{stroke,fill}%
}%
\begin{pgfscope}%
\pgfsys@transformshift{0.708220in}{1.342786in}%
\pgfsys@useobject{currentmarker}{}%
\end{pgfscope}%
\end{pgfscope}%
\begin{pgfscope}%
\pgfsetbuttcap%
\pgfsetroundjoin%
\definecolor{currentfill}{rgb}{0.000000,0.000000,0.000000}%
\pgfsetfillcolor{currentfill}%
\pgfsetlinewidth{0.602250pt}%
\definecolor{currentstroke}{rgb}{0.000000,0.000000,0.000000}%
\pgfsetstrokecolor{currentstroke}%
\pgfsetdash{}{0pt}%
\pgfsys@defobject{currentmarker}{\pgfqpoint{-0.027778in}{0.000000in}}{\pgfqpoint{-0.000000in}{0.000000in}}{%
\pgfpathmoveto{\pgfqpoint{-0.000000in}{0.000000in}}%
\pgfpathlineto{\pgfqpoint{-0.027778in}{0.000000in}}%
\pgfusepath{stroke,fill}%
}%
\begin{pgfscope}%
\pgfsys@transformshift{0.708220in}{1.404540in}%
\pgfsys@useobject{currentmarker}{}%
\end{pgfscope}%
\end{pgfscope}%
\begin{pgfscope}%
\pgfsetbuttcap%
\pgfsetroundjoin%
\definecolor{currentfill}{rgb}{0.000000,0.000000,0.000000}%
\pgfsetfillcolor{currentfill}%
\pgfsetlinewidth{0.602250pt}%
\definecolor{currentstroke}{rgb}{0.000000,0.000000,0.000000}%
\pgfsetstrokecolor{currentstroke}%
\pgfsetdash{}{0pt}%
\pgfsys@defobject{currentmarker}{\pgfqpoint{-0.027778in}{0.000000in}}{\pgfqpoint{-0.000000in}{0.000000in}}{%
\pgfpathmoveto{\pgfqpoint{-0.000000in}{0.000000in}}%
\pgfpathlineto{\pgfqpoint{-0.027778in}{0.000000in}}%
\pgfusepath{stroke,fill}%
}%
\begin{pgfscope}%
\pgfsys@transformshift{0.708220in}{1.448356in}%
\pgfsys@useobject{currentmarker}{}%
\end{pgfscope}%
\end{pgfscope}%
\begin{pgfscope}%
\pgfsetbuttcap%
\pgfsetroundjoin%
\definecolor{currentfill}{rgb}{0.000000,0.000000,0.000000}%
\pgfsetfillcolor{currentfill}%
\pgfsetlinewidth{0.602250pt}%
\definecolor{currentstroke}{rgb}{0.000000,0.000000,0.000000}%
\pgfsetstrokecolor{currentstroke}%
\pgfsetdash{}{0pt}%
\pgfsys@defobject{currentmarker}{\pgfqpoint{-0.027778in}{0.000000in}}{\pgfqpoint{-0.000000in}{0.000000in}}{%
\pgfpathmoveto{\pgfqpoint{-0.000000in}{0.000000in}}%
\pgfpathlineto{\pgfqpoint{-0.027778in}{0.000000in}}%
\pgfusepath{stroke,fill}%
}%
\begin{pgfscope}%
\pgfsys@transformshift{0.708220in}{1.482342in}%
\pgfsys@useobject{currentmarker}{}%
\end{pgfscope}%
\end{pgfscope}%
\begin{pgfscope}%
\pgfsetbuttcap%
\pgfsetroundjoin%
\definecolor{currentfill}{rgb}{0.000000,0.000000,0.000000}%
\pgfsetfillcolor{currentfill}%
\pgfsetlinewidth{0.602250pt}%
\definecolor{currentstroke}{rgb}{0.000000,0.000000,0.000000}%
\pgfsetstrokecolor{currentstroke}%
\pgfsetdash{}{0pt}%
\pgfsys@defobject{currentmarker}{\pgfqpoint{-0.027778in}{0.000000in}}{\pgfqpoint{-0.000000in}{0.000000in}}{%
\pgfpathmoveto{\pgfqpoint{-0.000000in}{0.000000in}}%
\pgfpathlineto{\pgfqpoint{-0.027778in}{0.000000in}}%
\pgfusepath{stroke,fill}%
}%
\begin{pgfscope}%
\pgfsys@transformshift{0.708220in}{1.510110in}%
\pgfsys@useobject{currentmarker}{}%
\end{pgfscope}%
\end{pgfscope}%
\begin{pgfscope}%
\pgfsetbuttcap%
\pgfsetroundjoin%
\definecolor{currentfill}{rgb}{0.000000,0.000000,0.000000}%
\pgfsetfillcolor{currentfill}%
\pgfsetlinewidth{0.602250pt}%
\definecolor{currentstroke}{rgb}{0.000000,0.000000,0.000000}%
\pgfsetstrokecolor{currentstroke}%
\pgfsetdash{}{0pt}%
\pgfsys@defobject{currentmarker}{\pgfqpoint{-0.027778in}{0.000000in}}{\pgfqpoint{-0.000000in}{0.000000in}}{%
\pgfpathmoveto{\pgfqpoint{-0.000000in}{0.000000in}}%
\pgfpathlineto{\pgfqpoint{-0.027778in}{0.000000in}}%
\pgfusepath{stroke,fill}%
}%
\begin{pgfscope}%
\pgfsys@transformshift{0.708220in}{1.533588in}%
\pgfsys@useobject{currentmarker}{}%
\end{pgfscope}%
\end{pgfscope}%
\begin{pgfscope}%
\pgfsetbuttcap%
\pgfsetroundjoin%
\definecolor{currentfill}{rgb}{0.000000,0.000000,0.000000}%
\pgfsetfillcolor{currentfill}%
\pgfsetlinewidth{0.602250pt}%
\definecolor{currentstroke}{rgb}{0.000000,0.000000,0.000000}%
\pgfsetstrokecolor{currentstroke}%
\pgfsetdash{}{0pt}%
\pgfsys@defobject{currentmarker}{\pgfqpoint{-0.027778in}{0.000000in}}{\pgfqpoint{-0.000000in}{0.000000in}}{%
\pgfpathmoveto{\pgfqpoint{-0.000000in}{0.000000in}}%
\pgfpathlineto{\pgfqpoint{-0.027778in}{0.000000in}}%
\pgfusepath{stroke,fill}%
}%
\begin{pgfscope}%
\pgfsys@transformshift{0.708220in}{1.553926in}%
\pgfsys@useobject{currentmarker}{}%
\end{pgfscope}%
\end{pgfscope}%
\begin{pgfscope}%
\pgfsetbuttcap%
\pgfsetroundjoin%
\definecolor{currentfill}{rgb}{0.000000,0.000000,0.000000}%
\pgfsetfillcolor{currentfill}%
\pgfsetlinewidth{0.602250pt}%
\definecolor{currentstroke}{rgb}{0.000000,0.000000,0.000000}%
\pgfsetstrokecolor{currentstroke}%
\pgfsetdash{}{0pt}%
\pgfsys@defobject{currentmarker}{\pgfqpoint{-0.027778in}{0.000000in}}{\pgfqpoint{-0.000000in}{0.000000in}}{%
\pgfpathmoveto{\pgfqpoint{-0.000000in}{0.000000in}}%
\pgfpathlineto{\pgfqpoint{-0.027778in}{0.000000in}}%
\pgfusepath{stroke,fill}%
}%
\begin{pgfscope}%
\pgfsys@transformshift{0.708220in}{1.571865in}%
\pgfsys@useobject{currentmarker}{}%
\end{pgfscope}%
\end{pgfscope}%
\begin{pgfscope}%
\pgfsetbuttcap%
\pgfsetroundjoin%
\definecolor{currentfill}{rgb}{0.000000,0.000000,0.000000}%
\pgfsetfillcolor{currentfill}%
\pgfsetlinewidth{0.602250pt}%
\definecolor{currentstroke}{rgb}{0.000000,0.000000,0.000000}%
\pgfsetstrokecolor{currentstroke}%
\pgfsetdash{}{0pt}%
\pgfsys@defobject{currentmarker}{\pgfqpoint{-0.027778in}{0.000000in}}{\pgfqpoint{-0.000000in}{0.000000in}}{%
\pgfpathmoveto{\pgfqpoint{-0.000000in}{0.000000in}}%
\pgfpathlineto{\pgfqpoint{-0.027778in}{0.000000in}}%
\pgfusepath{stroke,fill}%
}%
\begin{pgfscope}%
\pgfsys@transformshift{0.708220in}{1.693482in}%
\pgfsys@useobject{currentmarker}{}%
\end{pgfscope}%
\end{pgfscope}%
\begin{pgfscope}%
\pgfsetbuttcap%
\pgfsetroundjoin%
\definecolor{currentfill}{rgb}{0.000000,0.000000,0.000000}%
\pgfsetfillcolor{currentfill}%
\pgfsetlinewidth{0.602250pt}%
\definecolor{currentstroke}{rgb}{0.000000,0.000000,0.000000}%
\pgfsetstrokecolor{currentstroke}%
\pgfsetdash{}{0pt}%
\pgfsys@defobject{currentmarker}{\pgfqpoint{-0.027778in}{0.000000in}}{\pgfqpoint{-0.000000in}{0.000000in}}{%
\pgfpathmoveto{\pgfqpoint{-0.000000in}{0.000000in}}%
\pgfpathlineto{\pgfqpoint{-0.027778in}{0.000000in}}%
\pgfusepath{stroke,fill}%
}%
\begin{pgfscope}%
\pgfsys@transformshift{0.708220in}{1.755237in}%
\pgfsys@useobject{currentmarker}{}%
\end{pgfscope}%
\end{pgfscope}%
\begin{pgfscope}%
\pgfsetbuttcap%
\pgfsetroundjoin%
\definecolor{currentfill}{rgb}{0.000000,0.000000,0.000000}%
\pgfsetfillcolor{currentfill}%
\pgfsetlinewidth{0.602250pt}%
\definecolor{currentstroke}{rgb}{0.000000,0.000000,0.000000}%
\pgfsetstrokecolor{currentstroke}%
\pgfsetdash{}{0pt}%
\pgfsys@defobject{currentmarker}{\pgfqpoint{-0.027778in}{0.000000in}}{\pgfqpoint{-0.000000in}{0.000000in}}{%
\pgfpathmoveto{\pgfqpoint{-0.000000in}{0.000000in}}%
\pgfpathlineto{\pgfqpoint{-0.027778in}{0.000000in}}%
\pgfusepath{stroke,fill}%
}%
\begin{pgfscope}%
\pgfsys@transformshift{0.708220in}{1.799052in}%
\pgfsys@useobject{currentmarker}{}%
\end{pgfscope}%
\end{pgfscope}%
\begin{pgfscope}%
\pgfsetbuttcap%
\pgfsetroundjoin%
\definecolor{currentfill}{rgb}{0.000000,0.000000,0.000000}%
\pgfsetfillcolor{currentfill}%
\pgfsetlinewidth{0.602250pt}%
\definecolor{currentstroke}{rgb}{0.000000,0.000000,0.000000}%
\pgfsetstrokecolor{currentstroke}%
\pgfsetdash{}{0pt}%
\pgfsys@defobject{currentmarker}{\pgfqpoint{-0.027778in}{0.000000in}}{\pgfqpoint{-0.000000in}{0.000000in}}{%
\pgfpathmoveto{\pgfqpoint{-0.000000in}{0.000000in}}%
\pgfpathlineto{\pgfqpoint{-0.027778in}{0.000000in}}%
\pgfusepath{stroke,fill}%
}%
\begin{pgfscope}%
\pgfsys@transformshift{0.708220in}{1.833038in}%
\pgfsys@useobject{currentmarker}{}%
\end{pgfscope}%
\end{pgfscope}%
\begin{pgfscope}%
\pgfsetbuttcap%
\pgfsetroundjoin%
\definecolor{currentfill}{rgb}{0.000000,0.000000,0.000000}%
\pgfsetfillcolor{currentfill}%
\pgfsetlinewidth{0.602250pt}%
\definecolor{currentstroke}{rgb}{0.000000,0.000000,0.000000}%
\pgfsetstrokecolor{currentstroke}%
\pgfsetdash{}{0pt}%
\pgfsys@defobject{currentmarker}{\pgfqpoint{-0.027778in}{0.000000in}}{\pgfqpoint{-0.000000in}{0.000000in}}{%
\pgfpathmoveto{\pgfqpoint{-0.000000in}{0.000000in}}%
\pgfpathlineto{\pgfqpoint{-0.027778in}{0.000000in}}%
\pgfusepath{stroke,fill}%
}%
\begin{pgfscope}%
\pgfsys@transformshift{0.708220in}{1.860807in}%
\pgfsys@useobject{currentmarker}{}%
\end{pgfscope}%
\end{pgfscope}%
\begin{pgfscope}%
\pgfsetbuttcap%
\pgfsetroundjoin%
\definecolor{currentfill}{rgb}{0.000000,0.000000,0.000000}%
\pgfsetfillcolor{currentfill}%
\pgfsetlinewidth{0.602250pt}%
\definecolor{currentstroke}{rgb}{0.000000,0.000000,0.000000}%
\pgfsetstrokecolor{currentstroke}%
\pgfsetdash{}{0pt}%
\pgfsys@defobject{currentmarker}{\pgfqpoint{-0.027778in}{0.000000in}}{\pgfqpoint{-0.000000in}{0.000000in}}{%
\pgfpathmoveto{\pgfqpoint{-0.000000in}{0.000000in}}%
\pgfpathlineto{\pgfqpoint{-0.027778in}{0.000000in}}%
\pgfusepath{stroke,fill}%
}%
\begin{pgfscope}%
\pgfsys@transformshift{0.708220in}{1.884285in}%
\pgfsys@useobject{currentmarker}{}%
\end{pgfscope}%
\end{pgfscope}%
\begin{pgfscope}%
\pgfsetbuttcap%
\pgfsetroundjoin%
\definecolor{currentfill}{rgb}{0.000000,0.000000,0.000000}%
\pgfsetfillcolor{currentfill}%
\pgfsetlinewidth{0.602250pt}%
\definecolor{currentstroke}{rgb}{0.000000,0.000000,0.000000}%
\pgfsetstrokecolor{currentstroke}%
\pgfsetdash{}{0pt}%
\pgfsys@defobject{currentmarker}{\pgfqpoint{-0.027778in}{0.000000in}}{\pgfqpoint{-0.000000in}{0.000000in}}{%
\pgfpathmoveto{\pgfqpoint{-0.000000in}{0.000000in}}%
\pgfpathlineto{\pgfqpoint{-0.027778in}{0.000000in}}%
\pgfusepath{stroke,fill}%
}%
\begin{pgfscope}%
\pgfsys@transformshift{0.708220in}{1.904623in}%
\pgfsys@useobject{currentmarker}{}%
\end{pgfscope}%
\end{pgfscope}%
\begin{pgfscope}%
\pgfsetbuttcap%
\pgfsetroundjoin%
\definecolor{currentfill}{rgb}{0.000000,0.000000,0.000000}%
\pgfsetfillcolor{currentfill}%
\pgfsetlinewidth{0.602250pt}%
\definecolor{currentstroke}{rgb}{0.000000,0.000000,0.000000}%
\pgfsetstrokecolor{currentstroke}%
\pgfsetdash{}{0pt}%
\pgfsys@defobject{currentmarker}{\pgfqpoint{-0.027778in}{0.000000in}}{\pgfqpoint{-0.000000in}{0.000000in}}{%
\pgfpathmoveto{\pgfqpoint{-0.000000in}{0.000000in}}%
\pgfpathlineto{\pgfqpoint{-0.027778in}{0.000000in}}%
\pgfusepath{stroke,fill}%
}%
\begin{pgfscope}%
\pgfsys@transformshift{0.708220in}{1.922562in}%
\pgfsys@useobject{currentmarker}{}%
\end{pgfscope}%
\end{pgfscope}%
\begin{pgfscope}%
\definecolor{textcolor}{rgb}{0.000000,0.000000,0.000000}%
\pgfsetstrokecolor{textcolor}%
\pgfsetfillcolor{textcolor}%
\pgftext[x=0.288855in,y=1.237216in,,bottom,rotate=90.000000]{\color{textcolor}\rmfamily\fontsize{10.000000}{12.000000}\selectfont Longest solving time (s)}%
\end{pgfscope}%
\begin{pgfscope}%
\pgfpathrectangle{\pgfqpoint{0.708220in}{0.535823in}}{\pgfqpoint{5.141780in}{1.402786in}}%
\pgfusepath{clip}%
\pgfsetbuttcap%
\pgfsetroundjoin%
\pgfsetlinewidth{2.007500pt}%
\definecolor{currentstroke}{rgb}{1.000000,0.843137,0.000000}%
\pgfsetstrokecolor{currentstroke}%
\pgfsetdash{{7.400000pt}{3.200000pt}}{0.000000pt}%
\pgfpathmoveto{\pgfqpoint{0.708220in}{0.804867in}}%
\pgfpathlineto{\pgfqpoint{0.712933in}{0.811364in}}%
\pgfpathlineto{\pgfqpoint{0.717646in}{0.831637in}}%
\pgfpathlineto{\pgfqpoint{0.727071in}{0.834287in}}%
\pgfpathlineto{\pgfqpoint{0.736497in}{0.840208in}}%
\pgfpathlineto{\pgfqpoint{0.745923in}{0.841367in}}%
\pgfpathlineto{\pgfqpoint{0.750636in}{0.844515in}}%
\pgfpathlineto{\pgfqpoint{0.760062in}{0.845068in}}%
\pgfpathlineto{\pgfqpoint{0.764775in}{0.865336in}}%
\pgfpathlineto{\pgfqpoint{0.769488in}{0.874462in}}%
\pgfpathlineto{\pgfqpoint{0.774201in}{0.875631in}}%
\pgfpathlineto{\pgfqpoint{0.783626in}{0.879473in}}%
\pgfpathlineto{\pgfqpoint{0.788339in}{0.885155in}}%
\pgfpathlineto{\pgfqpoint{0.802478in}{0.885844in}}%
\pgfpathlineto{\pgfqpoint{0.807191in}{0.890406in}}%
\pgfpathlineto{\pgfqpoint{0.816617in}{0.915721in}}%
\pgfpathlineto{\pgfqpoint{0.821330in}{0.918298in}}%
\pgfpathlineto{\pgfqpoint{0.826043in}{0.933315in}}%
\pgfpathlineto{\pgfqpoint{0.835468in}{0.938470in}}%
\pgfpathlineto{\pgfqpoint{0.859033in}{0.942858in}}%
\pgfpathlineto{\pgfqpoint{0.863746in}{0.943522in}}%
\pgfpathlineto{\pgfqpoint{0.877884in}{0.952218in}}%
\pgfpathlineto{\pgfqpoint{0.882597in}{0.957360in}}%
\pgfpathlineto{\pgfqpoint{0.887310in}{0.957681in}}%
\pgfpathlineto{\pgfqpoint{0.892023in}{0.964052in}}%
\pgfpathlineto{\pgfqpoint{0.896736in}{0.967822in}}%
\pgfpathlineto{\pgfqpoint{0.901449in}{0.969220in}}%
\pgfpathlineto{\pgfqpoint{0.906162in}{0.975209in}}%
\pgfpathlineto{\pgfqpoint{0.910875in}{0.977628in}}%
\pgfpathlineto{\pgfqpoint{0.929726in}{0.981840in}}%
\pgfpathlineto{\pgfqpoint{0.939152in}{0.987797in}}%
\pgfpathlineto{\pgfqpoint{0.943865in}{0.999784in}}%
\pgfpathlineto{\pgfqpoint{0.948578in}{1.003030in}}%
\pgfpathlineto{\pgfqpoint{0.953291in}{1.003880in}}%
\pgfpathlineto{\pgfqpoint{0.958004in}{1.010464in}}%
\pgfpathlineto{\pgfqpoint{0.972143in}{1.011530in}}%
\pgfpathlineto{\pgfqpoint{0.976855in}{1.012691in}}%
\pgfpathlineto{\pgfqpoint{0.981568in}{1.015839in}}%
\pgfpathlineto{\pgfqpoint{0.990994in}{1.030132in}}%
\pgfpathlineto{\pgfqpoint{1.014559in}{1.037065in}}%
\pgfpathlineto{\pgfqpoint{1.019272in}{1.044156in}}%
\pgfpathlineto{\pgfqpoint{1.023985in}{1.045337in}}%
\pgfpathlineto{\pgfqpoint{1.028697in}{1.047803in}}%
\pgfpathlineto{\pgfqpoint{1.038123in}{1.049838in}}%
\pgfpathlineto{\pgfqpoint{1.052262in}{1.051029in}}%
\pgfpathlineto{\pgfqpoint{1.056975in}{1.051698in}}%
\pgfpathlineto{\pgfqpoint{1.061688in}{1.064284in}}%
\pgfpathlineto{\pgfqpoint{1.071114in}{1.065226in}}%
\pgfpathlineto{\pgfqpoint{1.080539in}{1.065429in}}%
\pgfpathlineto{\pgfqpoint{1.089965in}{1.067327in}}%
\pgfpathlineto{\pgfqpoint{1.104104in}{1.068796in}}%
\pgfpathlineto{\pgfqpoint{1.113530in}{1.070373in}}%
\pgfpathlineto{\pgfqpoint{1.141807in}{1.080372in}}%
\pgfpathlineto{\pgfqpoint{1.146520in}{1.081309in}}%
\pgfpathlineto{\pgfqpoint{1.151233in}{1.094518in}}%
\pgfpathlineto{\pgfqpoint{1.179510in}{1.097106in}}%
\pgfpathlineto{\pgfqpoint{1.188936in}{1.099165in}}%
\pgfpathlineto{\pgfqpoint{1.193649in}{1.101471in}}%
\pgfpathlineto{\pgfqpoint{1.207788in}{1.103872in}}%
\pgfpathlineto{\pgfqpoint{1.212501in}{1.105964in}}%
\pgfpathlineto{\pgfqpoint{1.236065in}{1.108802in}}%
\pgfpathlineto{\pgfqpoint{1.240778in}{1.110804in}}%
\pgfpathlineto{\pgfqpoint{1.245491in}{1.111441in}}%
\pgfpathlineto{\pgfqpoint{1.250204in}{1.120884in}}%
\pgfpathlineto{\pgfqpoint{1.254917in}{1.122892in}}%
\pgfpathlineto{\pgfqpoint{1.259630in}{1.129104in}}%
\pgfpathlineto{\pgfqpoint{1.269056in}{1.129525in}}%
\pgfpathlineto{\pgfqpoint{1.278481in}{1.130919in}}%
\pgfpathlineto{\pgfqpoint{1.283194in}{1.131218in}}%
\pgfpathlineto{\pgfqpoint{1.287907in}{1.134033in}}%
\pgfpathlineto{\pgfqpoint{1.302046in}{1.136633in}}%
\pgfpathlineto{\pgfqpoint{1.306759in}{1.137047in}}%
\pgfpathlineto{\pgfqpoint{1.311472in}{1.142669in}}%
\pgfpathlineto{\pgfqpoint{1.335036in}{1.144745in}}%
\pgfpathlineto{\pgfqpoint{1.344462in}{1.145385in}}%
\pgfpathlineto{\pgfqpoint{1.353888in}{1.146205in}}%
\pgfpathlineto{\pgfqpoint{1.358601in}{1.146689in}}%
\pgfpathlineto{\pgfqpoint{1.363314in}{1.163240in}}%
\pgfpathlineto{\pgfqpoint{1.368027in}{1.163521in}}%
\pgfpathlineto{\pgfqpoint{1.382165in}{1.167324in}}%
\pgfpathlineto{\pgfqpoint{1.386878in}{1.167958in}}%
\pgfpathlineto{\pgfqpoint{1.410443in}{1.177434in}}%
\pgfpathlineto{\pgfqpoint{1.415156in}{1.185222in}}%
\pgfpathlineto{\pgfqpoint{1.419869in}{1.189278in}}%
\pgfpathlineto{\pgfqpoint{1.424582in}{1.189326in}}%
\pgfpathlineto{\pgfqpoint{1.438720in}{1.195905in}}%
\pgfpathlineto{\pgfqpoint{1.443433in}{1.201637in}}%
\pgfpathlineto{\pgfqpoint{1.457572in}{1.206980in}}%
\pgfpathlineto{\pgfqpoint{1.471711in}{1.209000in}}%
\pgfpathlineto{\pgfqpoint{1.476423in}{1.210578in}}%
\pgfpathlineto{\pgfqpoint{1.481136in}{1.222256in}}%
\pgfpathlineto{\pgfqpoint{1.490562in}{1.225555in}}%
\pgfpathlineto{\pgfqpoint{1.495275in}{1.227079in}}%
\pgfpathlineto{\pgfqpoint{1.499988in}{1.232578in}}%
\pgfpathlineto{\pgfqpoint{1.504701in}{1.234960in}}%
\pgfpathlineto{\pgfqpoint{1.509414in}{1.244768in}}%
\pgfpathlineto{\pgfqpoint{1.514127in}{1.247995in}}%
\pgfpathlineto{\pgfqpoint{1.518840in}{1.248179in}}%
\pgfpathlineto{\pgfqpoint{1.528265in}{1.253981in}}%
\pgfpathlineto{\pgfqpoint{1.532978in}{1.254664in}}%
\pgfpathlineto{\pgfqpoint{1.537691in}{1.266334in}}%
\pgfpathlineto{\pgfqpoint{1.542404in}{1.270138in}}%
\pgfpathlineto{\pgfqpoint{1.547117in}{1.279001in}}%
\pgfpathlineto{\pgfqpoint{1.551830in}{1.305786in}}%
\pgfpathlineto{\pgfqpoint{1.556543in}{1.310741in}}%
\pgfpathlineto{\pgfqpoint{1.565969in}{1.316493in}}%
\pgfpathlineto{\pgfqpoint{1.570682in}{1.321949in}}%
\pgfpathlineto{\pgfqpoint{1.575395in}{1.343219in}}%
\pgfpathlineto{\pgfqpoint{1.580107in}{1.346888in}}%
\pgfpathlineto{\pgfqpoint{1.584820in}{1.353518in}}%
\pgfpathlineto{\pgfqpoint{1.594246in}{1.377254in}}%
\pgfpathlineto{\pgfqpoint{1.603672in}{1.481793in}}%
\pgfpathlineto{\pgfqpoint{1.608385in}{1.926740in}}%
\pgfpathlineto{\pgfqpoint{1.613098in}{1.938609in}}%
\pgfpathlineto{\pgfqpoint{5.845287in}{1.938609in}}%
\pgfpathlineto{\pgfqpoint{5.845287in}{1.938609in}}%
\pgfusepath{stroke}%
\end{pgfscope}%
\begin{pgfscope}%
\pgfpathrectangle{\pgfqpoint{0.708220in}{0.535823in}}{\pgfqpoint{5.141780in}{1.402786in}}%
\pgfusepath{clip}%
\pgfsetbuttcap%
\pgfsetroundjoin%
\pgfsetlinewidth{2.007500pt}%
\definecolor{currentstroke}{rgb}{1.000000,0.694118,0.305882}%
\pgfsetstrokecolor{currentstroke}%
\pgfsetdash{{2.000000pt}{3.300000pt}}{0.000000pt}%
\pgfpathmoveto{\pgfqpoint{0.708220in}{0.638220in}}%
\pgfpathlineto{\pgfqpoint{0.712933in}{0.681165in}}%
\pgfpathlineto{\pgfqpoint{0.727071in}{0.684042in}}%
\pgfpathlineto{\pgfqpoint{0.736497in}{0.687239in}}%
\pgfpathlineto{\pgfqpoint{0.745923in}{0.688772in}}%
\pgfpathlineto{\pgfqpoint{0.755349in}{0.689831in}}%
\pgfpathlineto{\pgfqpoint{0.807191in}{0.691885in}}%
\pgfpathlineto{\pgfqpoint{0.826043in}{0.695171in}}%
\pgfpathlineto{\pgfqpoint{0.830755in}{0.695494in}}%
\pgfpathlineto{\pgfqpoint{0.835468in}{0.704959in}}%
\pgfpathlineto{\pgfqpoint{0.840181in}{0.707467in}}%
\pgfpathlineto{\pgfqpoint{0.849607in}{0.708297in}}%
\pgfpathlineto{\pgfqpoint{0.863746in}{0.714191in}}%
\pgfpathlineto{\pgfqpoint{0.868459in}{0.714922in}}%
\pgfpathlineto{\pgfqpoint{0.873172in}{0.716896in}}%
\pgfpathlineto{\pgfqpoint{0.877884in}{0.721525in}}%
\pgfpathlineto{\pgfqpoint{0.882597in}{0.729879in}}%
\pgfpathlineto{\pgfqpoint{0.887310in}{0.732319in}}%
\pgfpathlineto{\pgfqpoint{0.892023in}{0.737005in}}%
\pgfpathlineto{\pgfqpoint{0.906162in}{0.738670in}}%
\pgfpathlineto{\pgfqpoint{0.910875in}{0.741710in}}%
\pgfpathlineto{\pgfqpoint{0.920301in}{0.743051in}}%
\pgfpathlineto{\pgfqpoint{0.929726in}{0.744927in}}%
\pgfpathlineto{\pgfqpoint{0.939152in}{0.746044in}}%
\pgfpathlineto{\pgfqpoint{0.943865in}{0.746754in}}%
\pgfpathlineto{\pgfqpoint{0.948578in}{0.750288in}}%
\pgfpathlineto{\pgfqpoint{0.953291in}{0.750410in}}%
\pgfpathlineto{\pgfqpoint{0.962717in}{0.752401in}}%
\pgfpathlineto{\pgfqpoint{0.967430in}{0.759990in}}%
\pgfpathlineto{\pgfqpoint{0.976855in}{0.760794in}}%
\pgfpathlineto{\pgfqpoint{0.986281in}{0.762443in}}%
\pgfpathlineto{\pgfqpoint{0.990994in}{0.762506in}}%
\pgfpathlineto{\pgfqpoint{0.995707in}{0.764900in}}%
\pgfpathlineto{\pgfqpoint{1.005133in}{0.766574in}}%
\pgfpathlineto{\pgfqpoint{1.009846in}{0.771675in}}%
\pgfpathlineto{\pgfqpoint{1.019272in}{0.773230in}}%
\pgfpathlineto{\pgfqpoint{1.023985in}{0.791628in}}%
\pgfpathlineto{\pgfqpoint{1.033410in}{0.794987in}}%
\pgfpathlineto{\pgfqpoint{1.047549in}{0.795796in}}%
\pgfpathlineto{\pgfqpoint{1.052262in}{0.798314in}}%
\pgfpathlineto{\pgfqpoint{1.056975in}{0.798459in}}%
\pgfpathlineto{\pgfqpoint{1.061688in}{0.800834in}}%
\pgfpathlineto{\pgfqpoint{1.071114in}{0.800961in}}%
\pgfpathlineto{\pgfqpoint{1.085252in}{0.805707in}}%
\pgfpathlineto{\pgfqpoint{1.094678in}{0.806609in}}%
\pgfpathlineto{\pgfqpoint{1.099391in}{0.808268in}}%
\pgfpathlineto{\pgfqpoint{1.113530in}{0.809077in}}%
\pgfpathlineto{\pgfqpoint{1.127668in}{0.813423in}}%
\pgfpathlineto{\pgfqpoint{1.170085in}{0.816536in}}%
\pgfpathlineto{\pgfqpoint{1.174798in}{0.819410in}}%
\pgfpathlineto{\pgfqpoint{1.188936in}{0.820587in}}%
\pgfpathlineto{\pgfqpoint{1.193649in}{0.822004in}}%
\pgfpathlineto{\pgfqpoint{1.203075in}{0.829858in}}%
\pgfpathlineto{\pgfqpoint{1.217214in}{0.830967in}}%
\pgfpathlineto{\pgfqpoint{1.221927in}{0.832756in}}%
\pgfpathlineto{\pgfqpoint{1.236065in}{0.834398in}}%
\pgfpathlineto{\pgfqpoint{1.240778in}{0.839396in}}%
\pgfpathlineto{\pgfqpoint{1.250204in}{0.840337in}}%
\pgfpathlineto{\pgfqpoint{1.259630in}{0.844791in}}%
\pgfpathlineto{\pgfqpoint{1.283194in}{0.850383in}}%
\pgfpathlineto{\pgfqpoint{1.287907in}{0.852429in}}%
\pgfpathlineto{\pgfqpoint{1.292620in}{0.852559in}}%
\pgfpathlineto{\pgfqpoint{1.302046in}{0.854419in}}%
\pgfpathlineto{\pgfqpoint{1.311472in}{0.854720in}}%
\pgfpathlineto{\pgfqpoint{1.320898in}{0.858220in}}%
\pgfpathlineto{\pgfqpoint{1.325611in}{0.858309in}}%
\pgfpathlineto{\pgfqpoint{1.330323in}{0.860659in}}%
\pgfpathlineto{\pgfqpoint{1.335036in}{0.882956in}}%
\pgfpathlineto{\pgfqpoint{1.339749in}{0.883520in}}%
\pgfpathlineto{\pgfqpoint{1.344462in}{0.897238in}}%
\pgfpathlineto{\pgfqpoint{1.349175in}{0.901563in}}%
\pgfpathlineto{\pgfqpoint{1.353888in}{0.902090in}}%
\pgfpathlineto{\pgfqpoint{1.368027in}{0.908334in}}%
\pgfpathlineto{\pgfqpoint{1.372740in}{0.910264in}}%
\pgfpathlineto{\pgfqpoint{1.377452in}{0.923929in}}%
\pgfpathlineto{\pgfqpoint{1.396304in}{0.928247in}}%
\pgfpathlineto{\pgfqpoint{1.401017in}{0.935495in}}%
\pgfpathlineto{\pgfqpoint{1.405730in}{0.935501in}}%
\pgfpathlineto{\pgfqpoint{1.410443in}{0.937769in}}%
\pgfpathlineto{\pgfqpoint{1.415156in}{0.938478in}}%
\pgfpathlineto{\pgfqpoint{1.419869in}{0.956348in}}%
\pgfpathlineto{\pgfqpoint{1.429294in}{0.961320in}}%
\pgfpathlineto{\pgfqpoint{1.438720in}{0.962580in}}%
\pgfpathlineto{\pgfqpoint{1.443433in}{0.962834in}}%
\pgfpathlineto{\pgfqpoint{1.452859in}{0.968993in}}%
\pgfpathlineto{\pgfqpoint{1.457572in}{0.969268in}}%
\pgfpathlineto{\pgfqpoint{1.462285in}{0.972491in}}%
\pgfpathlineto{\pgfqpoint{1.466998in}{0.979480in}}%
\pgfpathlineto{\pgfqpoint{1.476423in}{0.981802in}}%
\pgfpathlineto{\pgfqpoint{1.481136in}{0.999351in}}%
\pgfpathlineto{\pgfqpoint{1.485849in}{1.001244in}}%
\pgfpathlineto{\pgfqpoint{1.499988in}{1.002736in}}%
\pgfpathlineto{\pgfqpoint{1.504701in}{1.005996in}}%
\pgfpathlineto{\pgfqpoint{1.509414in}{1.013311in}}%
\pgfpathlineto{\pgfqpoint{1.514127in}{1.041239in}}%
\pgfpathlineto{\pgfqpoint{1.518840in}{1.049146in}}%
\pgfpathlineto{\pgfqpoint{1.523553in}{1.060231in}}%
\pgfpathlineto{\pgfqpoint{1.528265in}{1.938609in}}%
\pgfpathlineto{\pgfqpoint{5.845287in}{1.938609in}}%
\pgfpathlineto{\pgfqpoint{5.845287in}{1.938609in}}%
\pgfusepath{stroke}%
\end{pgfscope}%
\begin{pgfscope}%
\pgfpathrectangle{\pgfqpoint{0.708220in}{0.535823in}}{\pgfqpoint{5.141780in}{1.402786in}}%
\pgfusepath{clip}%
\pgfsetrectcap%
\pgfsetroundjoin%
\pgfsetlinewidth{2.007500pt}%
\definecolor{currentstroke}{rgb}{0.980392,0.529412,0.458824}%
\pgfsetstrokecolor{currentstroke}%
\pgfsetdash{}{0pt}%
\pgfpathmoveto{\pgfqpoint{0.708220in}{0.830704in}}%
\pgfpathlineto{\pgfqpoint{0.712933in}{0.832616in}}%
\pgfpathlineto{\pgfqpoint{0.717646in}{0.832715in}}%
\pgfpathlineto{\pgfqpoint{0.727071in}{0.842924in}}%
\pgfpathlineto{\pgfqpoint{0.731784in}{0.913732in}}%
\pgfpathlineto{\pgfqpoint{0.736497in}{0.936242in}}%
\pgfpathlineto{\pgfqpoint{0.755349in}{0.940977in}}%
\pgfpathlineto{\pgfqpoint{0.760062in}{0.948626in}}%
\pgfpathlineto{\pgfqpoint{0.764775in}{0.953075in}}%
\pgfpathlineto{\pgfqpoint{0.774201in}{0.956554in}}%
\pgfpathlineto{\pgfqpoint{0.778913in}{0.959043in}}%
\pgfpathlineto{\pgfqpoint{0.788339in}{0.968018in}}%
\pgfpathlineto{\pgfqpoint{0.802478in}{0.971514in}}%
\pgfpathlineto{\pgfqpoint{0.816617in}{0.986167in}}%
\pgfpathlineto{\pgfqpoint{0.821330in}{0.986778in}}%
\pgfpathlineto{\pgfqpoint{0.826043in}{0.990653in}}%
\pgfpathlineto{\pgfqpoint{0.830755in}{0.992533in}}%
\pgfpathlineto{\pgfqpoint{0.849607in}{0.995729in}}%
\pgfpathlineto{\pgfqpoint{0.854320in}{0.998778in}}%
\pgfpathlineto{\pgfqpoint{0.859033in}{1.010459in}}%
\pgfpathlineto{\pgfqpoint{0.868459in}{1.011842in}}%
\pgfpathlineto{\pgfqpoint{0.873172in}{1.015972in}}%
\pgfpathlineto{\pgfqpoint{0.877884in}{1.017948in}}%
\pgfpathlineto{\pgfqpoint{0.892023in}{1.019297in}}%
\pgfpathlineto{\pgfqpoint{0.896736in}{1.022322in}}%
\pgfpathlineto{\pgfqpoint{0.901449in}{1.034811in}}%
\pgfpathlineto{\pgfqpoint{0.906162in}{1.040130in}}%
\pgfpathlineto{\pgfqpoint{0.910875in}{1.040630in}}%
\pgfpathlineto{\pgfqpoint{0.915588in}{1.045324in}}%
\pgfpathlineto{\pgfqpoint{0.925014in}{1.047775in}}%
\pgfpathlineto{\pgfqpoint{0.929726in}{1.051674in}}%
\pgfpathlineto{\pgfqpoint{0.939152in}{1.055453in}}%
\pgfpathlineto{\pgfqpoint{0.948578in}{1.055821in}}%
\pgfpathlineto{\pgfqpoint{0.958004in}{1.060504in}}%
\pgfpathlineto{\pgfqpoint{0.962717in}{1.066685in}}%
\pgfpathlineto{\pgfqpoint{0.967430in}{1.067390in}}%
\pgfpathlineto{\pgfqpoint{0.972143in}{1.076898in}}%
\pgfpathlineto{\pgfqpoint{0.976855in}{1.079429in}}%
\pgfpathlineto{\pgfqpoint{0.981568in}{1.079901in}}%
\pgfpathlineto{\pgfqpoint{0.986281in}{1.087887in}}%
\pgfpathlineto{\pgfqpoint{0.990994in}{1.090691in}}%
\pgfpathlineto{\pgfqpoint{0.995707in}{1.096737in}}%
\pgfpathlineto{\pgfqpoint{1.009846in}{1.103486in}}%
\pgfpathlineto{\pgfqpoint{1.014559in}{1.114508in}}%
\pgfpathlineto{\pgfqpoint{1.023985in}{1.127516in}}%
\pgfpathlineto{\pgfqpoint{1.028697in}{1.147153in}}%
\pgfpathlineto{\pgfqpoint{1.033410in}{1.147820in}}%
\pgfpathlineto{\pgfqpoint{1.038123in}{1.152011in}}%
\pgfpathlineto{\pgfqpoint{1.047549in}{1.154799in}}%
\pgfpathlineto{\pgfqpoint{1.052262in}{1.160470in}}%
\pgfpathlineto{\pgfqpoint{1.061688in}{1.161297in}}%
\pgfpathlineto{\pgfqpoint{1.075827in}{1.173987in}}%
\pgfpathlineto{\pgfqpoint{1.080539in}{1.177854in}}%
\pgfpathlineto{\pgfqpoint{1.094678in}{1.181882in}}%
\pgfpathlineto{\pgfqpoint{1.099391in}{1.186928in}}%
\pgfpathlineto{\pgfqpoint{1.104104in}{1.189783in}}%
\pgfpathlineto{\pgfqpoint{1.108817in}{1.194251in}}%
\pgfpathlineto{\pgfqpoint{1.113530in}{1.194264in}}%
\pgfpathlineto{\pgfqpoint{1.118243in}{1.202355in}}%
\pgfpathlineto{\pgfqpoint{1.122956in}{1.203109in}}%
\pgfpathlineto{\pgfqpoint{1.137094in}{1.214456in}}%
\pgfpathlineto{\pgfqpoint{1.146520in}{1.224262in}}%
\pgfpathlineto{\pgfqpoint{1.155946in}{1.226626in}}%
\pgfpathlineto{\pgfqpoint{1.165372in}{1.232513in}}%
\pgfpathlineto{\pgfqpoint{1.170085in}{1.239549in}}%
\pgfpathlineto{\pgfqpoint{1.188936in}{1.241897in}}%
\pgfpathlineto{\pgfqpoint{1.198362in}{1.243205in}}%
\pgfpathlineto{\pgfqpoint{1.203075in}{1.248364in}}%
\pgfpathlineto{\pgfqpoint{1.212501in}{1.263269in}}%
\pgfpathlineto{\pgfqpoint{1.221927in}{1.265081in}}%
\pgfpathlineto{\pgfqpoint{1.226639in}{1.267695in}}%
\pgfpathlineto{\pgfqpoint{1.231352in}{1.268787in}}%
\pgfpathlineto{\pgfqpoint{1.236065in}{1.272479in}}%
\pgfpathlineto{\pgfqpoint{1.245491in}{1.273370in}}%
\pgfpathlineto{\pgfqpoint{1.254917in}{1.274016in}}%
\pgfpathlineto{\pgfqpoint{1.269056in}{1.276687in}}%
\pgfpathlineto{\pgfqpoint{1.273769in}{1.280832in}}%
\pgfpathlineto{\pgfqpoint{1.297333in}{1.283768in}}%
\pgfpathlineto{\pgfqpoint{1.311472in}{1.285198in}}%
\pgfpathlineto{\pgfqpoint{1.316185in}{1.286576in}}%
\pgfpathlineto{\pgfqpoint{1.330323in}{1.287702in}}%
\pgfpathlineto{\pgfqpoint{1.344462in}{1.289261in}}%
\pgfpathlineto{\pgfqpoint{1.358601in}{1.290302in}}%
\pgfpathlineto{\pgfqpoint{1.363314in}{1.296241in}}%
\pgfpathlineto{\pgfqpoint{1.368027in}{1.298886in}}%
\pgfpathlineto{\pgfqpoint{1.372740in}{1.307492in}}%
\pgfpathlineto{\pgfqpoint{1.377452in}{1.309365in}}%
\pgfpathlineto{\pgfqpoint{1.382165in}{1.320708in}}%
\pgfpathlineto{\pgfqpoint{1.386878in}{1.322832in}}%
\pgfpathlineto{\pgfqpoint{1.391591in}{1.323253in}}%
\pgfpathlineto{\pgfqpoint{1.396304in}{1.332556in}}%
\pgfpathlineto{\pgfqpoint{1.401017in}{1.332580in}}%
\pgfpathlineto{\pgfqpoint{1.415156in}{1.339214in}}%
\pgfpathlineto{\pgfqpoint{1.419869in}{1.339438in}}%
\pgfpathlineto{\pgfqpoint{1.424582in}{1.343874in}}%
\pgfpathlineto{\pgfqpoint{1.429294in}{1.345386in}}%
\pgfpathlineto{\pgfqpoint{1.434007in}{1.345489in}}%
\pgfpathlineto{\pgfqpoint{1.438720in}{1.351082in}}%
\pgfpathlineto{\pgfqpoint{1.443433in}{1.351107in}}%
\pgfpathlineto{\pgfqpoint{1.452859in}{1.354848in}}%
\pgfpathlineto{\pgfqpoint{1.462285in}{1.364014in}}%
\pgfpathlineto{\pgfqpoint{1.466998in}{1.372170in}}%
\pgfpathlineto{\pgfqpoint{1.471711in}{1.372315in}}%
\pgfpathlineto{\pgfqpoint{1.476423in}{1.373680in}}%
\pgfpathlineto{\pgfqpoint{1.485849in}{1.380603in}}%
\pgfpathlineto{\pgfqpoint{1.499988in}{1.383626in}}%
\pgfpathlineto{\pgfqpoint{1.528265in}{1.387852in}}%
\pgfpathlineto{\pgfqpoint{1.532978in}{1.388478in}}%
\pgfpathlineto{\pgfqpoint{1.537691in}{1.392775in}}%
\pgfpathlineto{\pgfqpoint{1.542404in}{1.393263in}}%
\pgfpathlineto{\pgfqpoint{1.556543in}{1.398844in}}%
\pgfpathlineto{\pgfqpoint{1.575395in}{1.400460in}}%
\pgfpathlineto{\pgfqpoint{1.580107in}{1.407076in}}%
\pgfpathlineto{\pgfqpoint{1.589533in}{1.415097in}}%
\pgfpathlineto{\pgfqpoint{1.594246in}{1.442562in}}%
\pgfpathlineto{\pgfqpoint{1.598959in}{1.443416in}}%
\pgfpathlineto{\pgfqpoint{1.603672in}{1.462719in}}%
\pgfpathlineto{\pgfqpoint{1.608385in}{1.471789in}}%
\pgfpathlineto{\pgfqpoint{1.613098in}{1.472395in}}%
\pgfpathlineto{\pgfqpoint{1.617811in}{1.474125in}}%
\pgfpathlineto{\pgfqpoint{1.622524in}{1.480461in}}%
\pgfpathlineto{\pgfqpoint{1.627236in}{1.481424in}}%
\pgfpathlineto{\pgfqpoint{1.631949in}{1.485127in}}%
\pgfpathlineto{\pgfqpoint{1.636662in}{1.486268in}}%
\pgfpathlineto{\pgfqpoint{1.641375in}{1.488800in}}%
\pgfpathlineto{\pgfqpoint{1.650801in}{1.490896in}}%
\pgfpathlineto{\pgfqpoint{1.655514in}{1.495220in}}%
\pgfpathlineto{\pgfqpoint{1.660227in}{1.504927in}}%
\pgfpathlineto{\pgfqpoint{1.664940in}{1.511535in}}%
\pgfpathlineto{\pgfqpoint{1.669653in}{1.520159in}}%
\pgfpathlineto{\pgfqpoint{1.679078in}{1.522618in}}%
\pgfpathlineto{\pgfqpoint{1.683791in}{1.526075in}}%
\pgfpathlineto{\pgfqpoint{1.688504in}{1.526682in}}%
\pgfpathlineto{\pgfqpoint{1.693217in}{1.528945in}}%
\pgfpathlineto{\pgfqpoint{1.702643in}{1.529871in}}%
\pgfpathlineto{\pgfqpoint{1.707356in}{1.534511in}}%
\pgfpathlineto{\pgfqpoint{1.712069in}{1.546864in}}%
\pgfpathlineto{\pgfqpoint{1.716782in}{1.549193in}}%
\pgfpathlineto{\pgfqpoint{1.721495in}{1.575044in}}%
\pgfpathlineto{\pgfqpoint{1.726207in}{1.584319in}}%
\pgfpathlineto{\pgfqpoint{1.730920in}{1.589049in}}%
\pgfpathlineto{\pgfqpoint{1.735633in}{1.597536in}}%
\pgfpathlineto{\pgfqpoint{1.740346in}{1.599955in}}%
\pgfpathlineto{\pgfqpoint{1.745059in}{1.606194in}}%
\pgfpathlineto{\pgfqpoint{1.754485in}{1.606866in}}%
\pgfpathlineto{\pgfqpoint{1.763911in}{1.610345in}}%
\pgfpathlineto{\pgfqpoint{1.773337in}{1.610999in}}%
\pgfpathlineto{\pgfqpoint{1.782762in}{1.613770in}}%
\pgfpathlineto{\pgfqpoint{1.787475in}{1.616015in}}%
\pgfpathlineto{\pgfqpoint{1.811040in}{1.619074in}}%
\pgfpathlineto{\pgfqpoint{1.825179in}{1.631734in}}%
\pgfpathlineto{\pgfqpoint{1.829891in}{1.640946in}}%
\pgfpathlineto{\pgfqpoint{1.839317in}{1.642891in}}%
\pgfpathlineto{\pgfqpoint{1.844030in}{1.658513in}}%
\pgfpathlineto{\pgfqpoint{1.848743in}{1.662369in}}%
\pgfpathlineto{\pgfqpoint{1.853456in}{1.671113in}}%
\pgfpathlineto{\pgfqpoint{1.858169in}{1.673373in}}%
\pgfpathlineto{\pgfqpoint{1.862882in}{1.673390in}}%
\pgfpathlineto{\pgfqpoint{1.867595in}{1.677757in}}%
\pgfpathlineto{\pgfqpoint{1.891159in}{1.681358in}}%
\pgfpathlineto{\pgfqpoint{1.895872in}{1.685101in}}%
\pgfpathlineto{\pgfqpoint{1.900585in}{1.696887in}}%
\pgfpathlineto{\pgfqpoint{1.905298in}{1.698460in}}%
\pgfpathlineto{\pgfqpoint{1.914724in}{1.703558in}}%
\pgfpathlineto{\pgfqpoint{1.919437in}{1.706482in}}%
\pgfpathlineto{\pgfqpoint{1.924150in}{1.724729in}}%
\pgfpathlineto{\pgfqpoint{1.928862in}{1.799534in}}%
\pgfpathlineto{\pgfqpoint{1.933575in}{1.938609in}}%
\pgfpathlineto{\pgfqpoint{5.845287in}{1.938609in}}%
\pgfpathlineto{\pgfqpoint{5.845287in}{1.938609in}}%
\pgfusepath{stroke}%
\end{pgfscope}%
\begin{pgfscope}%
\pgfsetrectcap%
\pgfsetmiterjoin%
\pgfsetlinewidth{0.803000pt}%
\definecolor{currentstroke}{rgb}{0.000000,0.000000,0.000000}%
\pgfsetstrokecolor{currentstroke}%
\pgfsetdash{}{0pt}%
\pgfpathmoveto{\pgfqpoint{0.708220in}{0.535823in}}%
\pgfpathlineto{\pgfqpoint{0.708220in}{1.938609in}}%
\pgfusepath{stroke}%
\end{pgfscope}%
\begin{pgfscope}%
\pgfsetrectcap%
\pgfsetmiterjoin%
\pgfsetlinewidth{0.803000pt}%
\definecolor{currentstroke}{rgb}{0.000000,0.000000,0.000000}%
\pgfsetstrokecolor{currentstroke}%
\pgfsetdash{}{0pt}%
\pgfpathmoveto{\pgfqpoint{5.850000in}{0.535823in}}%
\pgfpathlineto{\pgfqpoint{5.850000in}{1.938609in}}%
\pgfusepath{stroke}%
\end{pgfscope}%
\begin{pgfscope}%
\pgfsetrectcap%
\pgfsetmiterjoin%
\pgfsetlinewidth{0.803000pt}%
\definecolor{currentstroke}{rgb}{0.000000,0.000000,0.000000}%
\pgfsetstrokecolor{currentstroke}%
\pgfsetdash{}{0pt}%
\pgfpathmoveto{\pgfqpoint{0.708220in}{0.535823in}}%
\pgfpathlineto{\pgfqpoint{5.850000in}{0.535823in}}%
\pgfusepath{stroke}%
\end{pgfscope}%
\begin{pgfscope}%
\pgfsetrectcap%
\pgfsetmiterjoin%
\pgfsetlinewidth{0.803000pt}%
\definecolor{currentstroke}{rgb}{0.000000,0.000000,0.000000}%
\pgfsetstrokecolor{currentstroke}%
\pgfsetdash{}{0pt}%
\pgfpathmoveto{\pgfqpoint{0.708220in}{1.938609in}}%
\pgfpathlineto{\pgfqpoint{5.850000in}{1.938609in}}%
\pgfusepath{stroke}%
\end{pgfscope}%
\begin{pgfscope}%
\pgfsetbuttcap%
\pgfsetroundjoin%
\pgfsetlinewidth{2.007500pt}%
\definecolor{currentstroke}{rgb}{1.000000,0.843137,0.000000}%
\pgfsetstrokecolor{currentstroke}%
\pgfsetdash{{7.400000pt}{3.200000pt}}{0.000000pt}%
\pgfpathmoveto{\pgfqpoint{4.827505in}{0.977471in}}%
\pgfpathlineto{\pgfqpoint{5.077505in}{0.977471in}}%
\pgfusepath{stroke}%
\end{pgfscope}%
\begin{pgfscope}%
\definecolor{textcolor}{rgb}{0.000000,0.000000,0.000000}%
\pgfsetstrokecolor{textcolor}%
\pgfsetfillcolor{textcolor}%
\pgftext[x=5.102505in,y=0.933721in,left,base]{\color{textcolor}\rmfamily\fontsize{9.000000}{10.800000}\selectfont FT+htd}%
\end{pgfscope}%
\begin{pgfscope}%
\pgfsetbuttcap%
\pgfsetroundjoin%
\pgfsetlinewidth{2.007500pt}%
\definecolor{currentstroke}{rgb}{1.000000,0.694118,0.305882}%
\pgfsetstrokecolor{currentstroke}%
\pgfsetdash{{2.000000pt}{3.300000pt}}{0.000000pt}%
\pgfpathmoveto{\pgfqpoint{4.827505in}{0.815672in}}%
\pgfpathlineto{\pgfqpoint{5.077505in}{0.815672in}}%
\pgfusepath{stroke}%
\end{pgfscope}%
\begin{pgfscope}%
\definecolor{textcolor}{rgb}{0.000000,0.000000,0.000000}%
\pgfsetstrokecolor{textcolor}%
\pgfsetfillcolor{textcolor}%
\pgftext[x=5.102505in,y=0.771922in,left,base]{\color{textcolor}\rmfamily\fontsize{9.000000}{10.800000}\selectfont FT+Flow}%
\end{pgfscope}%
\begin{pgfscope}%
\pgfsetrectcap%
\pgfsetroundjoin%
\pgfsetlinewidth{2.007500pt}%
\definecolor{currentstroke}{rgb}{0.980392,0.529412,0.458824}%
\pgfsetstrokecolor{currentstroke}%
\pgfsetdash{}{0pt}%
\pgfpathmoveto{\pgfqpoint{4.827505in}{0.653872in}}%
\pgfpathlineto{\pgfqpoint{5.077505in}{0.653872in}}%
\pgfusepath{stroke}%
\end{pgfscope}%
\begin{pgfscope}%
\definecolor{textcolor}{rgb}{0.000000,0.000000,0.000000}%
\pgfsetstrokecolor{textcolor}%
\pgfsetfillcolor{textcolor}%
\pgftext[x=5.102505in,y=0.610122in,left,base]{\color{textcolor}\rmfamily\fontsize{9.000000}{10.800000}\selectfont FT+Tamaki}%
\end{pgfscope}%
\end{pgfpicture}%
\makeatother%
\endgroup%

	\caption{\label{fig:solver-analysis} The number of probabilistic-inference benchmarks (out of 1091) for which \textbf{FT+Tamaki}, \textbf{FT+Flow}, and \textbf{FT+htd} were able to find a contraction tree whose max-rank was no larger than (top) 30, (middle) 25, or (bottom) 20 within the indicated time.}
\end{figure}

The existing tensor-based methods (\textbf{LG}, \textbf{greedy}, \textbf{metis}, and \textbf{GN}) that do not perform factoring were only able to count a single benchmark from this set within 1000 seconds. We observe that most of these benchmarks have a variable that appears many times, which significantly hinders tensor-based methods that do not perform factoring. This justifies our motivation for \textbf{FT} in Section \ref{sec:tensors:preprocessing}.

We next evaluate the structural properties of benchmarks for which \textbf{FT} outperforms other approaches. In Figure \ref{fig:cachet-carving-cactus}, we organize the number of benchmarks completed for each tool by carving width after \textbf{FT}-preprocessing. We observe that all tensor-based methods perform best on benchmarks with small carving width. In particular, \textbf{FT+Tamaki} was able to solve almost all benchmarks whose width is below 27. On the other hand, existing tools do not heavily rely on structural properties and so solve fewer low-width benchmarks than \textbf{FT+Tamaki} but significantly more high-width benchmarks. We conclude that tensor-network-based approaches perform well on benchmark instances of low carving width (after  \textbf{FT}-preprocessing).

Finally, we are interested in explaining the relative performance of the tensor-based methods \textbf{FT+Tamaki}, \textbf{FT+Flow}, and \textbf{FT+htd} on these benchmarks. To do this, we analyze the quality of the contraction trees they produce over time. Specifically, we rerun each implementation of \textbf{FT} for 1000 seconds with the contraction step disabled (i.e. remove step 3 of Algorithm \ref{alg:wmc}). Each implementation of \textbf{FT} is an online solver and so produces a sequence of contraction trees over time. For each contraction tree produced on each benchmark, we record the max-rank and time of production.

Results of this experiment are summarized in Figure \ref{fig:solver-analysis}. 
\textbf{FT+Flow} is able to find more contraction trees of small max-rank within 10 seconds than the other methods, while \textbf{FT+Tamaki} is able to find more contraction trees of small max-rank within 1000 seconds than the other methods. This matches our previous observations that, among the tensor-based methods, \textbf{FT+Flow} was the fastest method on the most benchmarks while \textbf{FT+Tamaki} was able to solve the most benchmarks after 1000 seconds.

% Moreover, we observe that the quality of the discovered contraction trees does not significantly improve on most benchmarks after 4 seconds for \textbf{FT+Flow}, 20 seconds for \textbf{FT+htd}, and 500 seconds for \textbf{FT+Tamaki}. This suggests that these methods are not likely to discover significantly better contract

We conclude from the experiments in Section \ref{sec:tensors:experiments:cubic} and Section \ref{sec:tensors:experiments:cachet} that both \textbf{LG} and \textbf{FT} are useful as part of a portfolio of model counters.

% Although \pkg{Tamaki} placed above \pkg{FlowCutter} in the PACE 2017 competition, our implementations based on \pkg{FlowCutter} outperformed our implementations based on \pkg{Tamaki} on both sets of benchmarks. This suggests that tensor-based methods might be improved by developing specialized decomposition solvers.