\section{Execution Phase: Performing the Projected Valuation}
\label{sec:procount:execution}

In the execution phase, we are given a CNF formula $\phi$ over variables $X \cup Y$, an $\{X, Y\}$-graded project-join tree $(T, r, \gamma, \pi)$ of $\phi$, and a literal-weight function $W$ over $X$.
The goal is to compute the $W$-projected-valuation $g^W_r$ using Definition \ref{def:graded_valuation}.
Several data structures can be used for the pseudo-Boolean functions that occur while using Definition \ref{def:graded_valuation}.

In Chapter \ref{ch:dpmc}, \emph{algebraic decision diagrams (ADDs)} were used for computing $W$-valuations.
An ADD is a compact representation of a pseudo-Boolean function as a directed acyclic graph \cite{bahar1997algebraic}.
ADDs were found to outperform tensors in the execution phase for model counting on a single-core, especially on benchmarks where the width of the project-join tree was above 30.
We observe in Section \ref{sec:procount:experiments:planners} that for projected counting most benchmarks have large width, since requiring the project-join trees to be graded often leads to an increase in minimum width.
In this work, we therefore use ADDs to compute $W$-projected-valuations.

This work was done by Vu H. N. Phan and is not a contribution of this dissertation. We refer the reader to \cite{phan2021phd} for the complete details.
