\section{Experimental Evaluation}
\label{sec:procount:experiments}

To implement our projected model counter \procount, we modify the unprojected model counter \dpmc{}, which is based on ungraded project-join trees \cite{dudek2020dpmc}.
The \dpmc{} framework includes: 
(1) the \Lg{} planner that uses tree-decomposition techniques, 
(2) the \htb{} planner that uses constraint-satisfaction heuristics, and
(3) the \dmc{} executor that uses \emph{algebraic decision diagrams (ADDs)}.
We generalize these three components to support graded project-join trees and projected model counting.

We conduct three computational experiments to address three following research questions.
\begin{enumerate}
    \item[RQ1] In the planning phase, how do tree-decomposition techniques compare with constraint-satisfaction heuristics?
    \item[RQ2] In the execution phase, how do different ADD variable orders compare?
    \item[RQ3] How does \procount{} compare to other exact weighted projected tools?
\end{enumerate}

To answer RQ1, in Experiment 1 we compare the planner \Lg{} (which uses tree decompositions) to \htb{} (which uses constraint-satisfaction heuristics).
\Lg{} uses the tree decomposers \flowcutter{} \cite{strasser2017computing}, \htd{} \cite{AMW17}, and \tamaki{} \cite{Tamaki17}.
\htb{} implements four heuristics for variable ordering: maximal-cardinality search (\mcs{}) \cite{tarjan1984simple}, lexicographic search for perfect/minimal orders (\lexp/\lexm{}) \cite{koster2001treewidth}, and min-fill (\minfill{}) \cite{dechter03}.
\htb{} also implements two clause-ordering heuristics: bucket elimination (\be) \cite{dechter99} and Bouquet's Method (\bm) \cite{bouquet1999gestion}%
% as well as two clause clustering heuristics \List{} and \tree{} \cite{DPV20}
.

To answer RQ2, in Experiment 2 we compare variable-ordering heuristics for the ADD-based executor \dmc.
An ADD \cite{bahar1997algebraic} is a directed acyclic graph that compactly represents a pseudo-Boolean function.
% Each internal node of an ADD corresponds to an input variable of the function.
An ADD requires a variable order, which strongly influences the compactness of the ADD.
\dmc{} implements four variable-ordering heuristics (see above): \mcs, \lexp, \lexm, and \minfill{}.

To answer RQ3, in Experiment 3 we compare \procount{} to state-of-the-art exact weighted projected model counters \dfp{} \cite{lagniez2019recursive}, \projmc{} \cite{lagniez2019recursive}, and \ssat{} \cite{lee2017solving}.
% We do not consider tools that are probabilistic, approximate, or unweighted.

%\dfp{} compiles Boolean formulas into decision decomposable negation normal form,
%\projmc{} uses disjunctive decomposition,
%and \ssat{} solves random-exist stochastic SAT by combining counting with SAT techniques.

We use \benchmarks{} CNF benchmarks gathered from two families.
The first family contains \wapsBenchmarks{} formulas and was used for weighted projected sampling \cite{gupta2019waps}.
For each benchmark in this family, a positive literal $x$ has weight $0 < W_x(\set{x}) < 1$, and a negative literal $\neg x$ has weight $W(\emptyset) = 1 - W_x(\set{x})$.
The second family contains \birdBenchmarks{} formulas and was used for unweighted projected model counting \cite{soos2019bird}.
We add weights to this family by randomly assigning $W_x(\set{x}) = 0.4$ and $W_x(\emptyset) = 0.6$ or vice versa to each variable $x$.
All \benchmarks{} benchmarks are satisfiable, as verified by the SAT solver \sat{} \cite{soos2009extending}.

We run all experiments on single CPU cores of a Linux cluster with Intel Xeon E5-2650v2 processors (2.60-GHz) and 30 GB of RAM.
All code and data are available (\url{https://github.com/vardigroup/DPMC/tree/v2.0.0}).

\noindent
% \the\columnwidth    \\ % 347.12354pt
% \the\textwidth      \\ % 347.12354pt
% \the\linewidth      \\ % 347.12354pt
% \the\hsize          \\ % 347.12354pt

%%%%%%%%%%%%%%%%%%%%%%%%%%%%%%%%%%%%%%%%%%%%%%%%%%%%%%%%%%%%%%%%%%%%%%%%%%%%%%%%

\subsection{Experiment 1: Comparing Planners}

In this experiment, we run all configurations of the planners \Lg{} and \htb{} once on each CNF benchmark with a timeout of 100 seconds.
We present results in Figure \ref{figPlanning}.
Each point $(x, y)$ on a plotted curve indicates that, within $x$ seconds on each of $y$ benchmarks, the first graded project-join tree produced by the corresponding planner has width at most \maxWidth{}.
We choose \maxWidth{} because previous work shows that executors do not handle larger project-join trees well \cite{DDV19,dudek2020dpmc}. % JD: This is more true for tensors than for ADDs, let's just leave it out
% Figure \ref{figPlanning} is qualitatively similar for other width cutoffs.

While \Lg{} is an \emph{anytime} tool that produces several trees (of decreasing widths) for each benchmark, for all experiments in the main paper, we use only the first tree produced on each benchmark.
We find that this does not significantly affect the performance of \procount{}; see Appendix \ref{appendix:exp} for more detail.

The tree-decomposition-based planner \Lg{} outputs more low-width trees than the constraint-satisfaction-based planner \htb{}.
Moreover, for \Lg{}, the tree decomposer \flowcutter{} is faster than \htd{} and \tamaki{}.
Thus we use \Lg{} with \flowcutter{} in \procount{} for later experiments.
\begin{figure}[t]
    \centering
    %% Creator: Matplotlib, PGF backend
%%
%% To include the figure in your LaTeX document, write
%%   \input{<filename>.pgf}
%%
%% Make sure the required packages are loaded in your preamble
%%   \usepackage{pgf}
%%
%% and, on pdftex
%%   \usepackage[utf8]{inputenc}\DeclareUnicodeCharacter{2212}{-}
%%
%% or, on luatex and xetex
%%   \usepackage{unicode-math}
%%
%% Figures using additional raster images can only be included by \input if
%% they are in the same directory as the main LaTeX file. For loading figures
%% from other directories you can use the `import` package
%%   \usepackage{import}
%%
%% and then include the figures with
%%   \import{<path to file>}{<filename>.pgf}
%%
%% Matplotlib used the following preamble
%%   \usepackage{fontspec}
%%   \setmainfont{DejaVuSerif.ttf}[Path=/home/vhp1/.local/lib/python3.8/site-packages/matplotlib/mpl-data/fonts/ttf/]
%%   \setsansfont{DejaVuSans.ttf}[Path=/home/vhp1/.local/lib/python3.8/site-packages/matplotlib/mpl-data/fonts/ttf/]
%%   \setmonofont{DejaVuSansMono.ttf}[Path=/home/vhp1/.local/lib/python3.8/site-packages/matplotlib/mpl-data/fonts/ttf/]
%%
\begingroup%
\makeatletter%
\begin{pgfpicture}%
\pgfpathrectangle{\pgfpointorigin}{\pgfqpoint{4.820041in}{1.610194in}}%
\pgfusepath{use as bounding box, clip}%
\begin{pgfscope}%
\pgfsetbuttcap%
\pgfsetmiterjoin%
\pgfsetlinewidth{0.000000pt}%
\definecolor{currentstroke}{rgb}{1.000000,1.000000,1.000000}%
\pgfsetstrokecolor{currentstroke}%
\pgfsetstrokeopacity{0.000000}%
\pgfsetdash{}{0pt}%
\pgfpathmoveto{\pgfqpoint{0.000000in}{0.000000in}}%
\pgfpathlineto{\pgfqpoint{4.820041in}{0.000000in}}%
\pgfpathlineto{\pgfqpoint{4.820041in}{1.610194in}}%
\pgfpathlineto{\pgfqpoint{0.000000in}{1.610194in}}%
\pgfpathclose%
\pgfusepath{}%
\end{pgfscope}%
\begin{pgfscope}%
\pgfsetbuttcap%
\pgfsetmiterjoin%
\definecolor{currentfill}{rgb}{1.000000,1.000000,1.000000}%
\pgfsetfillcolor{currentfill}%
\pgfsetlinewidth{0.000000pt}%
\definecolor{currentstroke}{rgb}{0.000000,0.000000,0.000000}%
\pgfsetstrokecolor{currentstroke}%
\pgfsetstrokeopacity{0.000000}%
\pgfsetdash{}{0pt}%
\pgfpathmoveto{\pgfqpoint{0.537394in}{0.467838in}}%
\pgfpathlineto{\pgfqpoint{4.632078in}{0.467838in}}%
\pgfpathlineto{\pgfqpoint{4.632078in}{1.465092in}}%
\pgfpathlineto{\pgfqpoint{0.537394in}{1.465092in}}%
\pgfpathclose%
\pgfusepath{fill}%
\end{pgfscope}%
\begin{pgfscope}%
\pgfsetbuttcap%
\pgfsetroundjoin%
\definecolor{currentfill}{rgb}{0.000000,0.000000,0.000000}%
\pgfsetfillcolor{currentfill}%
\pgfsetlinewidth{0.803000pt}%
\definecolor{currentstroke}{rgb}{0.000000,0.000000,0.000000}%
\pgfsetstrokecolor{currentstroke}%
\pgfsetdash{}{0pt}%
\pgfsys@defobject{currentmarker}{\pgfqpoint{0.000000in}{-0.048611in}}{\pgfqpoint{0.000000in}{0.000000in}}{%
\pgfpathmoveto{\pgfqpoint{0.000000in}{0.000000in}}%
\pgfpathlineto{\pgfqpoint{0.000000in}{-0.048611in}}%
\pgfusepath{stroke,fill}%
}%
\begin{pgfscope}%
\pgfsys@transformshift{0.537394in}{0.467838in}%
\pgfsys@useobject{currentmarker}{}%
\end{pgfscope}%
\end{pgfscope}%
\begin{pgfscope}%
\definecolor{textcolor}{rgb}{0.000000,0.000000,0.000000}%
\pgfsetstrokecolor{textcolor}%
\pgfsetfillcolor{textcolor}%
\pgftext[x=0.537394in,y=0.370616in,,top]{\color{textcolor}\sffamily\fontsize{8.000000}{9.600000}\selectfont \(\displaystyle {10^{-3}}\)}%
\end{pgfscope}%
\begin{pgfscope}%
\pgfsetbuttcap%
\pgfsetroundjoin%
\definecolor{currentfill}{rgb}{0.000000,0.000000,0.000000}%
\pgfsetfillcolor{currentfill}%
\pgfsetlinewidth{0.803000pt}%
\definecolor{currentstroke}{rgb}{0.000000,0.000000,0.000000}%
\pgfsetstrokecolor{currentstroke}%
\pgfsetdash{}{0pt}%
\pgfsys@defobject{currentmarker}{\pgfqpoint{0.000000in}{-0.048611in}}{\pgfqpoint{0.000000in}{0.000000in}}{%
\pgfpathmoveto{\pgfqpoint{0.000000in}{0.000000in}}%
\pgfpathlineto{\pgfqpoint{0.000000in}{-0.048611in}}%
\pgfusepath{stroke,fill}%
}%
\begin{pgfscope}%
\pgfsys@transformshift{1.356331in}{0.467838in}%
\pgfsys@useobject{currentmarker}{}%
\end{pgfscope}%
\end{pgfscope}%
\begin{pgfscope}%
\definecolor{textcolor}{rgb}{0.000000,0.000000,0.000000}%
\pgfsetstrokecolor{textcolor}%
\pgfsetfillcolor{textcolor}%
\pgftext[x=1.356331in,y=0.370616in,,top]{\color{textcolor}\sffamily\fontsize{8.000000}{9.600000}\selectfont \(\displaystyle {10^{-2}}\)}%
\end{pgfscope}%
\begin{pgfscope}%
\pgfsetbuttcap%
\pgfsetroundjoin%
\definecolor{currentfill}{rgb}{0.000000,0.000000,0.000000}%
\pgfsetfillcolor{currentfill}%
\pgfsetlinewidth{0.803000pt}%
\definecolor{currentstroke}{rgb}{0.000000,0.000000,0.000000}%
\pgfsetstrokecolor{currentstroke}%
\pgfsetdash{}{0pt}%
\pgfsys@defobject{currentmarker}{\pgfqpoint{0.000000in}{-0.048611in}}{\pgfqpoint{0.000000in}{0.000000in}}{%
\pgfpathmoveto{\pgfqpoint{0.000000in}{0.000000in}}%
\pgfpathlineto{\pgfqpoint{0.000000in}{-0.048611in}}%
\pgfusepath{stroke,fill}%
}%
\begin{pgfscope}%
\pgfsys@transformshift{2.175268in}{0.467838in}%
\pgfsys@useobject{currentmarker}{}%
\end{pgfscope}%
\end{pgfscope}%
\begin{pgfscope}%
\definecolor{textcolor}{rgb}{0.000000,0.000000,0.000000}%
\pgfsetstrokecolor{textcolor}%
\pgfsetfillcolor{textcolor}%
\pgftext[x=2.175268in,y=0.370616in,,top]{\color{textcolor}\sffamily\fontsize{8.000000}{9.600000}\selectfont \(\displaystyle {10^{-1}}\)}%
\end{pgfscope}%
\begin{pgfscope}%
\pgfsetbuttcap%
\pgfsetroundjoin%
\definecolor{currentfill}{rgb}{0.000000,0.000000,0.000000}%
\pgfsetfillcolor{currentfill}%
\pgfsetlinewidth{0.803000pt}%
\definecolor{currentstroke}{rgb}{0.000000,0.000000,0.000000}%
\pgfsetstrokecolor{currentstroke}%
\pgfsetdash{}{0pt}%
\pgfsys@defobject{currentmarker}{\pgfqpoint{0.000000in}{-0.048611in}}{\pgfqpoint{0.000000in}{0.000000in}}{%
\pgfpathmoveto{\pgfqpoint{0.000000in}{0.000000in}}%
\pgfpathlineto{\pgfqpoint{0.000000in}{-0.048611in}}%
\pgfusepath{stroke,fill}%
}%
\begin{pgfscope}%
\pgfsys@transformshift{2.994204in}{0.467838in}%
\pgfsys@useobject{currentmarker}{}%
\end{pgfscope}%
\end{pgfscope}%
\begin{pgfscope}%
\definecolor{textcolor}{rgb}{0.000000,0.000000,0.000000}%
\pgfsetstrokecolor{textcolor}%
\pgfsetfillcolor{textcolor}%
\pgftext[x=2.994204in,y=0.370616in,,top]{\color{textcolor}\sffamily\fontsize{8.000000}{9.600000}\selectfont \(\displaystyle {10^{0}}\)}%
\end{pgfscope}%
\begin{pgfscope}%
\pgfsetbuttcap%
\pgfsetroundjoin%
\definecolor{currentfill}{rgb}{0.000000,0.000000,0.000000}%
\pgfsetfillcolor{currentfill}%
\pgfsetlinewidth{0.803000pt}%
\definecolor{currentstroke}{rgb}{0.000000,0.000000,0.000000}%
\pgfsetstrokecolor{currentstroke}%
\pgfsetdash{}{0pt}%
\pgfsys@defobject{currentmarker}{\pgfqpoint{0.000000in}{-0.048611in}}{\pgfqpoint{0.000000in}{0.000000in}}{%
\pgfpathmoveto{\pgfqpoint{0.000000in}{0.000000in}}%
\pgfpathlineto{\pgfqpoint{0.000000in}{-0.048611in}}%
\pgfusepath{stroke,fill}%
}%
\begin{pgfscope}%
\pgfsys@transformshift{3.813141in}{0.467838in}%
\pgfsys@useobject{currentmarker}{}%
\end{pgfscope}%
\end{pgfscope}%
\begin{pgfscope}%
\definecolor{textcolor}{rgb}{0.000000,0.000000,0.000000}%
\pgfsetstrokecolor{textcolor}%
\pgfsetfillcolor{textcolor}%
\pgftext[x=3.813141in,y=0.370616in,,top]{\color{textcolor}\sffamily\fontsize{8.000000}{9.600000}\selectfont \(\displaystyle {10^{1}}\)}%
\end{pgfscope}%
\begin{pgfscope}%
\pgfsetbuttcap%
\pgfsetroundjoin%
\definecolor{currentfill}{rgb}{0.000000,0.000000,0.000000}%
\pgfsetfillcolor{currentfill}%
\pgfsetlinewidth{0.803000pt}%
\definecolor{currentstroke}{rgb}{0.000000,0.000000,0.000000}%
\pgfsetstrokecolor{currentstroke}%
\pgfsetdash{}{0pt}%
\pgfsys@defobject{currentmarker}{\pgfqpoint{0.000000in}{-0.048611in}}{\pgfqpoint{0.000000in}{0.000000in}}{%
\pgfpathmoveto{\pgfqpoint{0.000000in}{0.000000in}}%
\pgfpathlineto{\pgfqpoint{0.000000in}{-0.048611in}}%
\pgfusepath{stroke,fill}%
}%
\begin{pgfscope}%
\pgfsys@transformshift{4.632078in}{0.467838in}%
\pgfsys@useobject{currentmarker}{}%
\end{pgfscope}%
\end{pgfscope}%
\begin{pgfscope}%
\definecolor{textcolor}{rgb}{0.000000,0.000000,0.000000}%
\pgfsetstrokecolor{textcolor}%
\pgfsetfillcolor{textcolor}%
\pgftext[x=4.632078in,y=0.370616in,,top]{\color{textcolor}\sffamily\fontsize{8.000000}{9.600000}\selectfont \(\displaystyle {10^{2}}\)}%
\end{pgfscope}%
\begin{pgfscope}%
\pgfsetbuttcap%
\pgfsetroundjoin%
\definecolor{currentfill}{rgb}{0.000000,0.000000,0.000000}%
\pgfsetfillcolor{currentfill}%
\pgfsetlinewidth{0.602250pt}%
\definecolor{currentstroke}{rgb}{0.000000,0.000000,0.000000}%
\pgfsetstrokecolor{currentstroke}%
\pgfsetdash{}{0pt}%
\pgfsys@defobject{currentmarker}{\pgfqpoint{0.000000in}{-0.027778in}}{\pgfqpoint{0.000000in}{0.000000in}}{%
\pgfpathmoveto{\pgfqpoint{0.000000in}{0.000000in}}%
\pgfpathlineto{\pgfqpoint{0.000000in}{-0.027778in}}%
\pgfusepath{stroke,fill}%
}%
\begin{pgfscope}%
\pgfsys@transformshift{0.783918in}{0.467838in}%
\pgfsys@useobject{currentmarker}{}%
\end{pgfscope}%
\end{pgfscope}%
\begin{pgfscope}%
\pgfsetbuttcap%
\pgfsetroundjoin%
\definecolor{currentfill}{rgb}{0.000000,0.000000,0.000000}%
\pgfsetfillcolor{currentfill}%
\pgfsetlinewidth{0.602250pt}%
\definecolor{currentstroke}{rgb}{0.000000,0.000000,0.000000}%
\pgfsetstrokecolor{currentstroke}%
\pgfsetdash{}{0pt}%
\pgfsys@defobject{currentmarker}{\pgfqpoint{0.000000in}{-0.027778in}}{\pgfqpoint{0.000000in}{0.000000in}}{%
\pgfpathmoveto{\pgfqpoint{0.000000in}{0.000000in}}%
\pgfpathlineto{\pgfqpoint{0.000000in}{-0.027778in}}%
\pgfusepath{stroke,fill}%
}%
\begin{pgfscope}%
\pgfsys@transformshift{0.928126in}{0.467838in}%
\pgfsys@useobject{currentmarker}{}%
\end{pgfscope}%
\end{pgfscope}%
\begin{pgfscope}%
\pgfsetbuttcap%
\pgfsetroundjoin%
\definecolor{currentfill}{rgb}{0.000000,0.000000,0.000000}%
\pgfsetfillcolor{currentfill}%
\pgfsetlinewidth{0.602250pt}%
\definecolor{currentstroke}{rgb}{0.000000,0.000000,0.000000}%
\pgfsetstrokecolor{currentstroke}%
\pgfsetdash{}{0pt}%
\pgfsys@defobject{currentmarker}{\pgfqpoint{0.000000in}{-0.027778in}}{\pgfqpoint{0.000000in}{0.000000in}}{%
\pgfpathmoveto{\pgfqpoint{0.000000in}{0.000000in}}%
\pgfpathlineto{\pgfqpoint{0.000000in}{-0.027778in}}%
\pgfusepath{stroke,fill}%
}%
\begin{pgfscope}%
\pgfsys@transformshift{1.030443in}{0.467838in}%
\pgfsys@useobject{currentmarker}{}%
\end{pgfscope}%
\end{pgfscope}%
\begin{pgfscope}%
\pgfsetbuttcap%
\pgfsetroundjoin%
\definecolor{currentfill}{rgb}{0.000000,0.000000,0.000000}%
\pgfsetfillcolor{currentfill}%
\pgfsetlinewidth{0.602250pt}%
\definecolor{currentstroke}{rgb}{0.000000,0.000000,0.000000}%
\pgfsetstrokecolor{currentstroke}%
\pgfsetdash{}{0pt}%
\pgfsys@defobject{currentmarker}{\pgfqpoint{0.000000in}{-0.027778in}}{\pgfqpoint{0.000000in}{0.000000in}}{%
\pgfpathmoveto{\pgfqpoint{0.000000in}{0.000000in}}%
\pgfpathlineto{\pgfqpoint{0.000000in}{-0.027778in}}%
\pgfusepath{stroke,fill}%
}%
\begin{pgfscope}%
\pgfsys@transformshift{1.109806in}{0.467838in}%
\pgfsys@useobject{currentmarker}{}%
\end{pgfscope}%
\end{pgfscope}%
\begin{pgfscope}%
\pgfsetbuttcap%
\pgfsetroundjoin%
\definecolor{currentfill}{rgb}{0.000000,0.000000,0.000000}%
\pgfsetfillcolor{currentfill}%
\pgfsetlinewidth{0.602250pt}%
\definecolor{currentstroke}{rgb}{0.000000,0.000000,0.000000}%
\pgfsetstrokecolor{currentstroke}%
\pgfsetdash{}{0pt}%
\pgfsys@defobject{currentmarker}{\pgfqpoint{0.000000in}{-0.027778in}}{\pgfqpoint{0.000000in}{0.000000in}}{%
\pgfpathmoveto{\pgfqpoint{0.000000in}{0.000000in}}%
\pgfpathlineto{\pgfqpoint{0.000000in}{-0.027778in}}%
\pgfusepath{stroke,fill}%
}%
\begin{pgfscope}%
\pgfsys@transformshift{1.174651in}{0.467838in}%
\pgfsys@useobject{currentmarker}{}%
\end{pgfscope}%
\end{pgfscope}%
\begin{pgfscope}%
\pgfsetbuttcap%
\pgfsetroundjoin%
\definecolor{currentfill}{rgb}{0.000000,0.000000,0.000000}%
\pgfsetfillcolor{currentfill}%
\pgfsetlinewidth{0.602250pt}%
\definecolor{currentstroke}{rgb}{0.000000,0.000000,0.000000}%
\pgfsetstrokecolor{currentstroke}%
\pgfsetdash{}{0pt}%
\pgfsys@defobject{currentmarker}{\pgfqpoint{0.000000in}{-0.027778in}}{\pgfqpoint{0.000000in}{0.000000in}}{%
\pgfpathmoveto{\pgfqpoint{0.000000in}{0.000000in}}%
\pgfpathlineto{\pgfqpoint{0.000000in}{-0.027778in}}%
\pgfusepath{stroke,fill}%
}%
\begin{pgfscope}%
\pgfsys@transformshift{1.229476in}{0.467838in}%
\pgfsys@useobject{currentmarker}{}%
\end{pgfscope}%
\end{pgfscope}%
\begin{pgfscope}%
\pgfsetbuttcap%
\pgfsetroundjoin%
\definecolor{currentfill}{rgb}{0.000000,0.000000,0.000000}%
\pgfsetfillcolor{currentfill}%
\pgfsetlinewidth{0.602250pt}%
\definecolor{currentstroke}{rgb}{0.000000,0.000000,0.000000}%
\pgfsetstrokecolor{currentstroke}%
\pgfsetdash{}{0pt}%
\pgfsys@defobject{currentmarker}{\pgfqpoint{0.000000in}{-0.027778in}}{\pgfqpoint{0.000000in}{0.000000in}}{%
\pgfpathmoveto{\pgfqpoint{0.000000in}{0.000000in}}%
\pgfpathlineto{\pgfqpoint{0.000000in}{-0.027778in}}%
\pgfusepath{stroke,fill}%
}%
\begin{pgfscope}%
\pgfsys@transformshift{1.276968in}{0.467838in}%
\pgfsys@useobject{currentmarker}{}%
\end{pgfscope}%
\end{pgfscope}%
\begin{pgfscope}%
\pgfsetbuttcap%
\pgfsetroundjoin%
\definecolor{currentfill}{rgb}{0.000000,0.000000,0.000000}%
\pgfsetfillcolor{currentfill}%
\pgfsetlinewidth{0.602250pt}%
\definecolor{currentstroke}{rgb}{0.000000,0.000000,0.000000}%
\pgfsetstrokecolor{currentstroke}%
\pgfsetdash{}{0pt}%
\pgfsys@defobject{currentmarker}{\pgfqpoint{0.000000in}{-0.027778in}}{\pgfqpoint{0.000000in}{0.000000in}}{%
\pgfpathmoveto{\pgfqpoint{0.000000in}{0.000000in}}%
\pgfpathlineto{\pgfqpoint{0.000000in}{-0.027778in}}%
\pgfusepath{stroke,fill}%
}%
\begin{pgfscope}%
\pgfsys@transformshift{1.318858in}{0.467838in}%
\pgfsys@useobject{currentmarker}{}%
\end{pgfscope}%
\end{pgfscope}%
\begin{pgfscope}%
\pgfsetbuttcap%
\pgfsetroundjoin%
\definecolor{currentfill}{rgb}{0.000000,0.000000,0.000000}%
\pgfsetfillcolor{currentfill}%
\pgfsetlinewidth{0.602250pt}%
\definecolor{currentstroke}{rgb}{0.000000,0.000000,0.000000}%
\pgfsetstrokecolor{currentstroke}%
\pgfsetdash{}{0pt}%
\pgfsys@defobject{currentmarker}{\pgfqpoint{0.000000in}{-0.027778in}}{\pgfqpoint{0.000000in}{0.000000in}}{%
\pgfpathmoveto{\pgfqpoint{0.000000in}{0.000000in}}%
\pgfpathlineto{\pgfqpoint{0.000000in}{-0.027778in}}%
\pgfusepath{stroke,fill}%
}%
\begin{pgfscope}%
\pgfsys@transformshift{1.602855in}{0.467838in}%
\pgfsys@useobject{currentmarker}{}%
\end{pgfscope}%
\end{pgfscope}%
\begin{pgfscope}%
\pgfsetbuttcap%
\pgfsetroundjoin%
\definecolor{currentfill}{rgb}{0.000000,0.000000,0.000000}%
\pgfsetfillcolor{currentfill}%
\pgfsetlinewidth{0.602250pt}%
\definecolor{currentstroke}{rgb}{0.000000,0.000000,0.000000}%
\pgfsetstrokecolor{currentstroke}%
\pgfsetdash{}{0pt}%
\pgfsys@defobject{currentmarker}{\pgfqpoint{0.000000in}{-0.027778in}}{\pgfqpoint{0.000000in}{0.000000in}}{%
\pgfpathmoveto{\pgfqpoint{0.000000in}{0.000000in}}%
\pgfpathlineto{\pgfqpoint{0.000000in}{-0.027778in}}%
\pgfusepath{stroke,fill}%
}%
\begin{pgfscope}%
\pgfsys@transformshift{1.747063in}{0.467838in}%
\pgfsys@useobject{currentmarker}{}%
\end{pgfscope}%
\end{pgfscope}%
\begin{pgfscope}%
\pgfsetbuttcap%
\pgfsetroundjoin%
\definecolor{currentfill}{rgb}{0.000000,0.000000,0.000000}%
\pgfsetfillcolor{currentfill}%
\pgfsetlinewidth{0.602250pt}%
\definecolor{currentstroke}{rgb}{0.000000,0.000000,0.000000}%
\pgfsetstrokecolor{currentstroke}%
\pgfsetdash{}{0pt}%
\pgfsys@defobject{currentmarker}{\pgfqpoint{0.000000in}{-0.027778in}}{\pgfqpoint{0.000000in}{0.000000in}}{%
\pgfpathmoveto{\pgfqpoint{0.000000in}{0.000000in}}%
\pgfpathlineto{\pgfqpoint{0.000000in}{-0.027778in}}%
\pgfusepath{stroke,fill}%
}%
\begin{pgfscope}%
\pgfsys@transformshift{1.849380in}{0.467838in}%
\pgfsys@useobject{currentmarker}{}%
\end{pgfscope}%
\end{pgfscope}%
\begin{pgfscope}%
\pgfsetbuttcap%
\pgfsetroundjoin%
\definecolor{currentfill}{rgb}{0.000000,0.000000,0.000000}%
\pgfsetfillcolor{currentfill}%
\pgfsetlinewidth{0.602250pt}%
\definecolor{currentstroke}{rgb}{0.000000,0.000000,0.000000}%
\pgfsetstrokecolor{currentstroke}%
\pgfsetdash{}{0pt}%
\pgfsys@defobject{currentmarker}{\pgfqpoint{0.000000in}{-0.027778in}}{\pgfqpoint{0.000000in}{0.000000in}}{%
\pgfpathmoveto{\pgfqpoint{0.000000in}{0.000000in}}%
\pgfpathlineto{\pgfqpoint{0.000000in}{-0.027778in}}%
\pgfusepath{stroke,fill}%
}%
\begin{pgfscope}%
\pgfsys@transformshift{1.928743in}{0.467838in}%
\pgfsys@useobject{currentmarker}{}%
\end{pgfscope}%
\end{pgfscope}%
\begin{pgfscope}%
\pgfsetbuttcap%
\pgfsetroundjoin%
\definecolor{currentfill}{rgb}{0.000000,0.000000,0.000000}%
\pgfsetfillcolor{currentfill}%
\pgfsetlinewidth{0.602250pt}%
\definecolor{currentstroke}{rgb}{0.000000,0.000000,0.000000}%
\pgfsetstrokecolor{currentstroke}%
\pgfsetdash{}{0pt}%
\pgfsys@defobject{currentmarker}{\pgfqpoint{0.000000in}{-0.027778in}}{\pgfqpoint{0.000000in}{0.000000in}}{%
\pgfpathmoveto{\pgfqpoint{0.000000in}{0.000000in}}%
\pgfpathlineto{\pgfqpoint{0.000000in}{-0.027778in}}%
\pgfusepath{stroke,fill}%
}%
\begin{pgfscope}%
\pgfsys@transformshift{1.993587in}{0.467838in}%
\pgfsys@useobject{currentmarker}{}%
\end{pgfscope}%
\end{pgfscope}%
\begin{pgfscope}%
\pgfsetbuttcap%
\pgfsetroundjoin%
\definecolor{currentfill}{rgb}{0.000000,0.000000,0.000000}%
\pgfsetfillcolor{currentfill}%
\pgfsetlinewidth{0.602250pt}%
\definecolor{currentstroke}{rgb}{0.000000,0.000000,0.000000}%
\pgfsetstrokecolor{currentstroke}%
\pgfsetdash{}{0pt}%
\pgfsys@defobject{currentmarker}{\pgfqpoint{0.000000in}{-0.027778in}}{\pgfqpoint{0.000000in}{0.000000in}}{%
\pgfpathmoveto{\pgfqpoint{0.000000in}{0.000000in}}%
\pgfpathlineto{\pgfqpoint{0.000000in}{-0.027778in}}%
\pgfusepath{stroke,fill}%
}%
\begin{pgfscope}%
\pgfsys@transformshift{2.048413in}{0.467838in}%
\pgfsys@useobject{currentmarker}{}%
\end{pgfscope}%
\end{pgfscope}%
\begin{pgfscope}%
\pgfsetbuttcap%
\pgfsetroundjoin%
\definecolor{currentfill}{rgb}{0.000000,0.000000,0.000000}%
\pgfsetfillcolor{currentfill}%
\pgfsetlinewidth{0.602250pt}%
\definecolor{currentstroke}{rgb}{0.000000,0.000000,0.000000}%
\pgfsetstrokecolor{currentstroke}%
\pgfsetdash{}{0pt}%
\pgfsys@defobject{currentmarker}{\pgfqpoint{0.000000in}{-0.027778in}}{\pgfqpoint{0.000000in}{0.000000in}}{%
\pgfpathmoveto{\pgfqpoint{0.000000in}{0.000000in}}%
\pgfpathlineto{\pgfqpoint{0.000000in}{-0.027778in}}%
\pgfusepath{stroke,fill}%
}%
\begin{pgfscope}%
\pgfsys@transformshift{2.095904in}{0.467838in}%
\pgfsys@useobject{currentmarker}{}%
\end{pgfscope}%
\end{pgfscope}%
\begin{pgfscope}%
\pgfsetbuttcap%
\pgfsetroundjoin%
\definecolor{currentfill}{rgb}{0.000000,0.000000,0.000000}%
\pgfsetfillcolor{currentfill}%
\pgfsetlinewidth{0.602250pt}%
\definecolor{currentstroke}{rgb}{0.000000,0.000000,0.000000}%
\pgfsetstrokecolor{currentstroke}%
\pgfsetdash{}{0pt}%
\pgfsys@defobject{currentmarker}{\pgfqpoint{0.000000in}{-0.027778in}}{\pgfqpoint{0.000000in}{0.000000in}}{%
\pgfpathmoveto{\pgfqpoint{0.000000in}{0.000000in}}%
\pgfpathlineto{\pgfqpoint{0.000000in}{-0.027778in}}%
\pgfusepath{stroke,fill}%
}%
\begin{pgfscope}%
\pgfsys@transformshift{2.137795in}{0.467838in}%
\pgfsys@useobject{currentmarker}{}%
\end{pgfscope}%
\end{pgfscope}%
\begin{pgfscope}%
\pgfsetbuttcap%
\pgfsetroundjoin%
\definecolor{currentfill}{rgb}{0.000000,0.000000,0.000000}%
\pgfsetfillcolor{currentfill}%
\pgfsetlinewidth{0.602250pt}%
\definecolor{currentstroke}{rgb}{0.000000,0.000000,0.000000}%
\pgfsetstrokecolor{currentstroke}%
\pgfsetdash{}{0pt}%
\pgfsys@defobject{currentmarker}{\pgfqpoint{0.000000in}{-0.027778in}}{\pgfqpoint{0.000000in}{0.000000in}}{%
\pgfpathmoveto{\pgfqpoint{0.000000in}{0.000000in}}%
\pgfpathlineto{\pgfqpoint{0.000000in}{-0.027778in}}%
\pgfusepath{stroke,fill}%
}%
\begin{pgfscope}%
\pgfsys@transformshift{2.421792in}{0.467838in}%
\pgfsys@useobject{currentmarker}{}%
\end{pgfscope}%
\end{pgfscope}%
\begin{pgfscope}%
\pgfsetbuttcap%
\pgfsetroundjoin%
\definecolor{currentfill}{rgb}{0.000000,0.000000,0.000000}%
\pgfsetfillcolor{currentfill}%
\pgfsetlinewidth{0.602250pt}%
\definecolor{currentstroke}{rgb}{0.000000,0.000000,0.000000}%
\pgfsetstrokecolor{currentstroke}%
\pgfsetdash{}{0pt}%
\pgfsys@defobject{currentmarker}{\pgfqpoint{0.000000in}{-0.027778in}}{\pgfqpoint{0.000000in}{0.000000in}}{%
\pgfpathmoveto{\pgfqpoint{0.000000in}{0.000000in}}%
\pgfpathlineto{\pgfqpoint{0.000000in}{-0.027778in}}%
\pgfusepath{stroke,fill}%
}%
\begin{pgfscope}%
\pgfsys@transformshift{2.566000in}{0.467838in}%
\pgfsys@useobject{currentmarker}{}%
\end{pgfscope}%
\end{pgfscope}%
\begin{pgfscope}%
\pgfsetbuttcap%
\pgfsetroundjoin%
\definecolor{currentfill}{rgb}{0.000000,0.000000,0.000000}%
\pgfsetfillcolor{currentfill}%
\pgfsetlinewidth{0.602250pt}%
\definecolor{currentstroke}{rgb}{0.000000,0.000000,0.000000}%
\pgfsetstrokecolor{currentstroke}%
\pgfsetdash{}{0pt}%
\pgfsys@defobject{currentmarker}{\pgfqpoint{0.000000in}{-0.027778in}}{\pgfqpoint{0.000000in}{0.000000in}}{%
\pgfpathmoveto{\pgfqpoint{0.000000in}{0.000000in}}%
\pgfpathlineto{\pgfqpoint{0.000000in}{-0.027778in}}%
\pgfusepath{stroke,fill}%
}%
\begin{pgfscope}%
\pgfsys@transformshift{2.668317in}{0.467838in}%
\pgfsys@useobject{currentmarker}{}%
\end{pgfscope}%
\end{pgfscope}%
\begin{pgfscope}%
\pgfsetbuttcap%
\pgfsetroundjoin%
\definecolor{currentfill}{rgb}{0.000000,0.000000,0.000000}%
\pgfsetfillcolor{currentfill}%
\pgfsetlinewidth{0.602250pt}%
\definecolor{currentstroke}{rgb}{0.000000,0.000000,0.000000}%
\pgfsetstrokecolor{currentstroke}%
\pgfsetdash{}{0pt}%
\pgfsys@defobject{currentmarker}{\pgfqpoint{0.000000in}{-0.027778in}}{\pgfqpoint{0.000000in}{0.000000in}}{%
\pgfpathmoveto{\pgfqpoint{0.000000in}{0.000000in}}%
\pgfpathlineto{\pgfqpoint{0.000000in}{-0.027778in}}%
\pgfusepath{stroke,fill}%
}%
\begin{pgfscope}%
\pgfsys@transformshift{2.747680in}{0.467838in}%
\pgfsys@useobject{currentmarker}{}%
\end{pgfscope}%
\end{pgfscope}%
\begin{pgfscope}%
\pgfsetbuttcap%
\pgfsetroundjoin%
\definecolor{currentfill}{rgb}{0.000000,0.000000,0.000000}%
\pgfsetfillcolor{currentfill}%
\pgfsetlinewidth{0.602250pt}%
\definecolor{currentstroke}{rgb}{0.000000,0.000000,0.000000}%
\pgfsetstrokecolor{currentstroke}%
\pgfsetdash{}{0pt}%
\pgfsys@defobject{currentmarker}{\pgfqpoint{0.000000in}{-0.027778in}}{\pgfqpoint{0.000000in}{0.000000in}}{%
\pgfpathmoveto{\pgfqpoint{0.000000in}{0.000000in}}%
\pgfpathlineto{\pgfqpoint{0.000000in}{-0.027778in}}%
\pgfusepath{stroke,fill}%
}%
\begin{pgfscope}%
\pgfsys@transformshift{2.812524in}{0.467838in}%
\pgfsys@useobject{currentmarker}{}%
\end{pgfscope}%
\end{pgfscope}%
\begin{pgfscope}%
\pgfsetbuttcap%
\pgfsetroundjoin%
\definecolor{currentfill}{rgb}{0.000000,0.000000,0.000000}%
\pgfsetfillcolor{currentfill}%
\pgfsetlinewidth{0.602250pt}%
\definecolor{currentstroke}{rgb}{0.000000,0.000000,0.000000}%
\pgfsetstrokecolor{currentstroke}%
\pgfsetdash{}{0pt}%
\pgfsys@defobject{currentmarker}{\pgfqpoint{0.000000in}{-0.027778in}}{\pgfqpoint{0.000000in}{0.000000in}}{%
\pgfpathmoveto{\pgfqpoint{0.000000in}{0.000000in}}%
\pgfpathlineto{\pgfqpoint{0.000000in}{-0.027778in}}%
\pgfusepath{stroke,fill}%
}%
\begin{pgfscope}%
\pgfsys@transformshift{2.867349in}{0.467838in}%
\pgfsys@useobject{currentmarker}{}%
\end{pgfscope}%
\end{pgfscope}%
\begin{pgfscope}%
\pgfsetbuttcap%
\pgfsetroundjoin%
\definecolor{currentfill}{rgb}{0.000000,0.000000,0.000000}%
\pgfsetfillcolor{currentfill}%
\pgfsetlinewidth{0.602250pt}%
\definecolor{currentstroke}{rgb}{0.000000,0.000000,0.000000}%
\pgfsetstrokecolor{currentstroke}%
\pgfsetdash{}{0pt}%
\pgfsys@defobject{currentmarker}{\pgfqpoint{0.000000in}{-0.027778in}}{\pgfqpoint{0.000000in}{0.000000in}}{%
\pgfpathmoveto{\pgfqpoint{0.000000in}{0.000000in}}%
\pgfpathlineto{\pgfqpoint{0.000000in}{-0.027778in}}%
\pgfusepath{stroke,fill}%
}%
\begin{pgfscope}%
\pgfsys@transformshift{2.914841in}{0.467838in}%
\pgfsys@useobject{currentmarker}{}%
\end{pgfscope}%
\end{pgfscope}%
\begin{pgfscope}%
\pgfsetbuttcap%
\pgfsetroundjoin%
\definecolor{currentfill}{rgb}{0.000000,0.000000,0.000000}%
\pgfsetfillcolor{currentfill}%
\pgfsetlinewidth{0.602250pt}%
\definecolor{currentstroke}{rgb}{0.000000,0.000000,0.000000}%
\pgfsetstrokecolor{currentstroke}%
\pgfsetdash{}{0pt}%
\pgfsys@defobject{currentmarker}{\pgfqpoint{0.000000in}{-0.027778in}}{\pgfqpoint{0.000000in}{0.000000in}}{%
\pgfpathmoveto{\pgfqpoint{0.000000in}{0.000000in}}%
\pgfpathlineto{\pgfqpoint{0.000000in}{-0.027778in}}%
\pgfusepath{stroke,fill}%
}%
\begin{pgfscope}%
\pgfsys@transformshift{2.956732in}{0.467838in}%
\pgfsys@useobject{currentmarker}{}%
\end{pgfscope}%
\end{pgfscope}%
\begin{pgfscope}%
\pgfsetbuttcap%
\pgfsetroundjoin%
\definecolor{currentfill}{rgb}{0.000000,0.000000,0.000000}%
\pgfsetfillcolor{currentfill}%
\pgfsetlinewidth{0.602250pt}%
\definecolor{currentstroke}{rgb}{0.000000,0.000000,0.000000}%
\pgfsetstrokecolor{currentstroke}%
\pgfsetdash{}{0pt}%
\pgfsys@defobject{currentmarker}{\pgfqpoint{0.000000in}{-0.027778in}}{\pgfqpoint{0.000000in}{0.000000in}}{%
\pgfpathmoveto{\pgfqpoint{0.000000in}{0.000000in}}%
\pgfpathlineto{\pgfqpoint{0.000000in}{-0.027778in}}%
\pgfusepath{stroke,fill}%
}%
\begin{pgfscope}%
\pgfsys@transformshift{3.240729in}{0.467838in}%
\pgfsys@useobject{currentmarker}{}%
\end{pgfscope}%
\end{pgfscope}%
\begin{pgfscope}%
\pgfsetbuttcap%
\pgfsetroundjoin%
\definecolor{currentfill}{rgb}{0.000000,0.000000,0.000000}%
\pgfsetfillcolor{currentfill}%
\pgfsetlinewidth{0.602250pt}%
\definecolor{currentstroke}{rgb}{0.000000,0.000000,0.000000}%
\pgfsetstrokecolor{currentstroke}%
\pgfsetdash{}{0pt}%
\pgfsys@defobject{currentmarker}{\pgfqpoint{0.000000in}{-0.027778in}}{\pgfqpoint{0.000000in}{0.000000in}}{%
\pgfpathmoveto{\pgfqpoint{0.000000in}{0.000000in}}%
\pgfpathlineto{\pgfqpoint{0.000000in}{-0.027778in}}%
\pgfusepath{stroke,fill}%
}%
\begin{pgfscope}%
\pgfsys@transformshift{3.384936in}{0.467838in}%
\pgfsys@useobject{currentmarker}{}%
\end{pgfscope}%
\end{pgfscope}%
\begin{pgfscope}%
\pgfsetbuttcap%
\pgfsetroundjoin%
\definecolor{currentfill}{rgb}{0.000000,0.000000,0.000000}%
\pgfsetfillcolor{currentfill}%
\pgfsetlinewidth{0.602250pt}%
\definecolor{currentstroke}{rgb}{0.000000,0.000000,0.000000}%
\pgfsetstrokecolor{currentstroke}%
\pgfsetdash{}{0pt}%
\pgfsys@defobject{currentmarker}{\pgfqpoint{0.000000in}{-0.027778in}}{\pgfqpoint{0.000000in}{0.000000in}}{%
\pgfpathmoveto{\pgfqpoint{0.000000in}{0.000000in}}%
\pgfpathlineto{\pgfqpoint{0.000000in}{-0.027778in}}%
\pgfusepath{stroke,fill}%
}%
\begin{pgfscope}%
\pgfsys@transformshift{3.487253in}{0.467838in}%
\pgfsys@useobject{currentmarker}{}%
\end{pgfscope}%
\end{pgfscope}%
\begin{pgfscope}%
\pgfsetbuttcap%
\pgfsetroundjoin%
\definecolor{currentfill}{rgb}{0.000000,0.000000,0.000000}%
\pgfsetfillcolor{currentfill}%
\pgfsetlinewidth{0.602250pt}%
\definecolor{currentstroke}{rgb}{0.000000,0.000000,0.000000}%
\pgfsetstrokecolor{currentstroke}%
\pgfsetdash{}{0pt}%
\pgfsys@defobject{currentmarker}{\pgfqpoint{0.000000in}{-0.027778in}}{\pgfqpoint{0.000000in}{0.000000in}}{%
\pgfpathmoveto{\pgfqpoint{0.000000in}{0.000000in}}%
\pgfpathlineto{\pgfqpoint{0.000000in}{-0.027778in}}%
\pgfusepath{stroke,fill}%
}%
\begin{pgfscope}%
\pgfsys@transformshift{3.566617in}{0.467838in}%
\pgfsys@useobject{currentmarker}{}%
\end{pgfscope}%
\end{pgfscope}%
\begin{pgfscope}%
\pgfsetbuttcap%
\pgfsetroundjoin%
\definecolor{currentfill}{rgb}{0.000000,0.000000,0.000000}%
\pgfsetfillcolor{currentfill}%
\pgfsetlinewidth{0.602250pt}%
\definecolor{currentstroke}{rgb}{0.000000,0.000000,0.000000}%
\pgfsetstrokecolor{currentstroke}%
\pgfsetdash{}{0pt}%
\pgfsys@defobject{currentmarker}{\pgfqpoint{0.000000in}{-0.027778in}}{\pgfqpoint{0.000000in}{0.000000in}}{%
\pgfpathmoveto{\pgfqpoint{0.000000in}{0.000000in}}%
\pgfpathlineto{\pgfqpoint{0.000000in}{-0.027778in}}%
\pgfusepath{stroke,fill}%
}%
\begin{pgfscope}%
\pgfsys@transformshift{3.631461in}{0.467838in}%
\pgfsys@useobject{currentmarker}{}%
\end{pgfscope}%
\end{pgfscope}%
\begin{pgfscope}%
\pgfsetbuttcap%
\pgfsetroundjoin%
\definecolor{currentfill}{rgb}{0.000000,0.000000,0.000000}%
\pgfsetfillcolor{currentfill}%
\pgfsetlinewidth{0.602250pt}%
\definecolor{currentstroke}{rgb}{0.000000,0.000000,0.000000}%
\pgfsetstrokecolor{currentstroke}%
\pgfsetdash{}{0pt}%
\pgfsys@defobject{currentmarker}{\pgfqpoint{0.000000in}{-0.027778in}}{\pgfqpoint{0.000000in}{0.000000in}}{%
\pgfpathmoveto{\pgfqpoint{0.000000in}{0.000000in}}%
\pgfpathlineto{\pgfqpoint{0.000000in}{-0.027778in}}%
\pgfusepath{stroke,fill}%
}%
\begin{pgfscope}%
\pgfsys@transformshift{3.686286in}{0.467838in}%
\pgfsys@useobject{currentmarker}{}%
\end{pgfscope}%
\end{pgfscope}%
\begin{pgfscope}%
\pgfsetbuttcap%
\pgfsetroundjoin%
\definecolor{currentfill}{rgb}{0.000000,0.000000,0.000000}%
\pgfsetfillcolor{currentfill}%
\pgfsetlinewidth{0.602250pt}%
\definecolor{currentstroke}{rgb}{0.000000,0.000000,0.000000}%
\pgfsetstrokecolor{currentstroke}%
\pgfsetdash{}{0pt}%
\pgfsys@defobject{currentmarker}{\pgfqpoint{0.000000in}{-0.027778in}}{\pgfqpoint{0.000000in}{0.000000in}}{%
\pgfpathmoveto{\pgfqpoint{0.000000in}{0.000000in}}%
\pgfpathlineto{\pgfqpoint{0.000000in}{-0.027778in}}%
\pgfusepath{stroke,fill}%
}%
\begin{pgfscope}%
\pgfsys@transformshift{3.733778in}{0.467838in}%
\pgfsys@useobject{currentmarker}{}%
\end{pgfscope}%
\end{pgfscope}%
\begin{pgfscope}%
\pgfsetbuttcap%
\pgfsetroundjoin%
\definecolor{currentfill}{rgb}{0.000000,0.000000,0.000000}%
\pgfsetfillcolor{currentfill}%
\pgfsetlinewidth{0.602250pt}%
\definecolor{currentstroke}{rgb}{0.000000,0.000000,0.000000}%
\pgfsetstrokecolor{currentstroke}%
\pgfsetdash{}{0pt}%
\pgfsys@defobject{currentmarker}{\pgfqpoint{0.000000in}{-0.027778in}}{\pgfqpoint{0.000000in}{0.000000in}}{%
\pgfpathmoveto{\pgfqpoint{0.000000in}{0.000000in}}%
\pgfpathlineto{\pgfqpoint{0.000000in}{-0.027778in}}%
\pgfusepath{stroke,fill}%
}%
\begin{pgfscope}%
\pgfsys@transformshift{3.775669in}{0.467838in}%
\pgfsys@useobject{currentmarker}{}%
\end{pgfscope}%
\end{pgfscope}%
\begin{pgfscope}%
\pgfsetbuttcap%
\pgfsetroundjoin%
\definecolor{currentfill}{rgb}{0.000000,0.000000,0.000000}%
\pgfsetfillcolor{currentfill}%
\pgfsetlinewidth{0.602250pt}%
\definecolor{currentstroke}{rgb}{0.000000,0.000000,0.000000}%
\pgfsetstrokecolor{currentstroke}%
\pgfsetdash{}{0pt}%
\pgfsys@defobject{currentmarker}{\pgfqpoint{0.000000in}{-0.027778in}}{\pgfqpoint{0.000000in}{0.000000in}}{%
\pgfpathmoveto{\pgfqpoint{0.000000in}{0.000000in}}%
\pgfpathlineto{\pgfqpoint{0.000000in}{-0.027778in}}%
\pgfusepath{stroke,fill}%
}%
\begin{pgfscope}%
\pgfsys@transformshift{4.059666in}{0.467838in}%
\pgfsys@useobject{currentmarker}{}%
\end{pgfscope}%
\end{pgfscope}%
\begin{pgfscope}%
\pgfsetbuttcap%
\pgfsetroundjoin%
\definecolor{currentfill}{rgb}{0.000000,0.000000,0.000000}%
\pgfsetfillcolor{currentfill}%
\pgfsetlinewidth{0.602250pt}%
\definecolor{currentstroke}{rgb}{0.000000,0.000000,0.000000}%
\pgfsetstrokecolor{currentstroke}%
\pgfsetdash{}{0pt}%
\pgfsys@defobject{currentmarker}{\pgfqpoint{0.000000in}{-0.027778in}}{\pgfqpoint{0.000000in}{0.000000in}}{%
\pgfpathmoveto{\pgfqpoint{0.000000in}{0.000000in}}%
\pgfpathlineto{\pgfqpoint{0.000000in}{-0.027778in}}%
\pgfusepath{stroke,fill}%
}%
\begin{pgfscope}%
\pgfsys@transformshift{4.203873in}{0.467838in}%
\pgfsys@useobject{currentmarker}{}%
\end{pgfscope}%
\end{pgfscope}%
\begin{pgfscope}%
\pgfsetbuttcap%
\pgfsetroundjoin%
\definecolor{currentfill}{rgb}{0.000000,0.000000,0.000000}%
\pgfsetfillcolor{currentfill}%
\pgfsetlinewidth{0.602250pt}%
\definecolor{currentstroke}{rgb}{0.000000,0.000000,0.000000}%
\pgfsetstrokecolor{currentstroke}%
\pgfsetdash{}{0pt}%
\pgfsys@defobject{currentmarker}{\pgfqpoint{0.000000in}{-0.027778in}}{\pgfqpoint{0.000000in}{0.000000in}}{%
\pgfpathmoveto{\pgfqpoint{0.000000in}{0.000000in}}%
\pgfpathlineto{\pgfqpoint{0.000000in}{-0.027778in}}%
\pgfusepath{stroke,fill}%
}%
\begin{pgfscope}%
\pgfsys@transformshift{4.306190in}{0.467838in}%
\pgfsys@useobject{currentmarker}{}%
\end{pgfscope}%
\end{pgfscope}%
\begin{pgfscope}%
\pgfsetbuttcap%
\pgfsetroundjoin%
\definecolor{currentfill}{rgb}{0.000000,0.000000,0.000000}%
\pgfsetfillcolor{currentfill}%
\pgfsetlinewidth{0.602250pt}%
\definecolor{currentstroke}{rgb}{0.000000,0.000000,0.000000}%
\pgfsetstrokecolor{currentstroke}%
\pgfsetdash{}{0pt}%
\pgfsys@defobject{currentmarker}{\pgfqpoint{0.000000in}{-0.027778in}}{\pgfqpoint{0.000000in}{0.000000in}}{%
\pgfpathmoveto{\pgfqpoint{0.000000in}{0.000000in}}%
\pgfpathlineto{\pgfqpoint{0.000000in}{-0.027778in}}%
\pgfusepath{stroke,fill}%
}%
\begin{pgfscope}%
\pgfsys@transformshift{4.385553in}{0.467838in}%
\pgfsys@useobject{currentmarker}{}%
\end{pgfscope}%
\end{pgfscope}%
\begin{pgfscope}%
\pgfsetbuttcap%
\pgfsetroundjoin%
\definecolor{currentfill}{rgb}{0.000000,0.000000,0.000000}%
\pgfsetfillcolor{currentfill}%
\pgfsetlinewidth{0.602250pt}%
\definecolor{currentstroke}{rgb}{0.000000,0.000000,0.000000}%
\pgfsetstrokecolor{currentstroke}%
\pgfsetdash{}{0pt}%
\pgfsys@defobject{currentmarker}{\pgfqpoint{0.000000in}{-0.027778in}}{\pgfqpoint{0.000000in}{0.000000in}}{%
\pgfpathmoveto{\pgfqpoint{0.000000in}{0.000000in}}%
\pgfpathlineto{\pgfqpoint{0.000000in}{-0.027778in}}%
\pgfusepath{stroke,fill}%
}%
\begin{pgfscope}%
\pgfsys@transformshift{4.450398in}{0.467838in}%
\pgfsys@useobject{currentmarker}{}%
\end{pgfscope}%
\end{pgfscope}%
\begin{pgfscope}%
\pgfsetbuttcap%
\pgfsetroundjoin%
\definecolor{currentfill}{rgb}{0.000000,0.000000,0.000000}%
\pgfsetfillcolor{currentfill}%
\pgfsetlinewidth{0.602250pt}%
\definecolor{currentstroke}{rgb}{0.000000,0.000000,0.000000}%
\pgfsetstrokecolor{currentstroke}%
\pgfsetdash{}{0pt}%
\pgfsys@defobject{currentmarker}{\pgfqpoint{0.000000in}{-0.027778in}}{\pgfqpoint{0.000000in}{0.000000in}}{%
\pgfpathmoveto{\pgfqpoint{0.000000in}{0.000000in}}%
\pgfpathlineto{\pgfqpoint{0.000000in}{-0.027778in}}%
\pgfusepath{stroke,fill}%
}%
\begin{pgfscope}%
\pgfsys@transformshift{4.505223in}{0.467838in}%
\pgfsys@useobject{currentmarker}{}%
\end{pgfscope}%
\end{pgfscope}%
\begin{pgfscope}%
\pgfsetbuttcap%
\pgfsetroundjoin%
\definecolor{currentfill}{rgb}{0.000000,0.000000,0.000000}%
\pgfsetfillcolor{currentfill}%
\pgfsetlinewidth{0.602250pt}%
\definecolor{currentstroke}{rgb}{0.000000,0.000000,0.000000}%
\pgfsetstrokecolor{currentstroke}%
\pgfsetdash{}{0pt}%
\pgfsys@defobject{currentmarker}{\pgfqpoint{0.000000in}{-0.027778in}}{\pgfqpoint{0.000000in}{0.000000in}}{%
\pgfpathmoveto{\pgfqpoint{0.000000in}{0.000000in}}%
\pgfpathlineto{\pgfqpoint{0.000000in}{-0.027778in}}%
\pgfusepath{stroke,fill}%
}%
\begin{pgfscope}%
\pgfsys@transformshift{4.552715in}{0.467838in}%
\pgfsys@useobject{currentmarker}{}%
\end{pgfscope}%
\end{pgfscope}%
\begin{pgfscope}%
\pgfsetbuttcap%
\pgfsetroundjoin%
\definecolor{currentfill}{rgb}{0.000000,0.000000,0.000000}%
\pgfsetfillcolor{currentfill}%
\pgfsetlinewidth{0.602250pt}%
\definecolor{currentstroke}{rgb}{0.000000,0.000000,0.000000}%
\pgfsetstrokecolor{currentstroke}%
\pgfsetdash{}{0pt}%
\pgfsys@defobject{currentmarker}{\pgfqpoint{0.000000in}{-0.027778in}}{\pgfqpoint{0.000000in}{0.000000in}}{%
\pgfpathmoveto{\pgfqpoint{0.000000in}{0.000000in}}%
\pgfpathlineto{\pgfqpoint{0.000000in}{-0.027778in}}%
\pgfusepath{stroke,fill}%
}%
\begin{pgfscope}%
\pgfsys@transformshift{4.594605in}{0.467838in}%
\pgfsys@useobject{currentmarker}{}%
\end{pgfscope}%
\end{pgfscope}%
\begin{pgfscope}%
\definecolor{textcolor}{rgb}{0.000000,0.000000,0.000000}%
\pgfsetstrokecolor{textcolor}%
\pgfsetfillcolor{textcolor}%
\pgftext[x=2.584736in,y=0.207530in,,top]{\color{textcolor}\sffamily\fontsize{8.000000}{9.600000}\selectfont Longest solving time (seconds)}%
\end{pgfscope}%
\begin{pgfscope}%
\pgfsetbuttcap%
\pgfsetroundjoin%
\definecolor{currentfill}{rgb}{0.000000,0.000000,0.000000}%
\pgfsetfillcolor{currentfill}%
\pgfsetlinewidth{0.803000pt}%
\definecolor{currentstroke}{rgb}{0.000000,0.000000,0.000000}%
\pgfsetstrokecolor{currentstroke}%
\pgfsetdash{}{0pt}%
\pgfsys@defobject{currentmarker}{\pgfqpoint{-0.048611in}{0.000000in}}{\pgfqpoint{-0.000000in}{0.000000in}}{%
\pgfpathmoveto{\pgfqpoint{-0.000000in}{0.000000in}}%
\pgfpathlineto{\pgfqpoint{-0.048611in}{0.000000in}}%
\pgfusepath{stroke,fill}%
}%
\begin{pgfscope}%
\pgfsys@transformshift{0.537394in}{0.467838in}%
\pgfsys@useobject{currentmarker}{}%
\end{pgfscope}%
\end{pgfscope}%
\begin{pgfscope}%
\definecolor{textcolor}{rgb}{0.000000,0.000000,0.000000}%
\pgfsetstrokecolor{textcolor}%
\pgfsetfillcolor{textcolor}%
\pgftext[x=0.381143in, y=0.425629in, left, base]{\color{textcolor}\sffamily\fontsize{8.000000}{9.600000}\selectfont \(\displaystyle {0}\)}%
\end{pgfscope}%
\begin{pgfscope}%
\pgfsetbuttcap%
\pgfsetroundjoin%
\definecolor{currentfill}{rgb}{0.000000,0.000000,0.000000}%
\pgfsetfillcolor{currentfill}%
\pgfsetlinewidth{0.803000pt}%
\definecolor{currentstroke}{rgb}{0.000000,0.000000,0.000000}%
\pgfsetstrokecolor{currentstroke}%
\pgfsetdash{}{0pt}%
\pgfsys@defobject{currentmarker}{\pgfqpoint{-0.048611in}{0.000000in}}{\pgfqpoint{-0.000000in}{0.000000in}}{%
\pgfpathmoveto{\pgfqpoint{-0.000000in}{0.000000in}}%
\pgfpathlineto{\pgfqpoint{-0.048611in}{0.000000in}}%
\pgfusepath{stroke,fill}%
}%
\begin{pgfscope}%
\pgfsys@transformshift{0.537394in}{0.717152in}%
\pgfsys@useobject{currentmarker}{}%
\end{pgfscope}%
\end{pgfscope}%
\begin{pgfscope}%
\definecolor{textcolor}{rgb}{0.000000,0.000000,0.000000}%
\pgfsetstrokecolor{textcolor}%
\pgfsetfillcolor{textcolor}%
\pgftext[x=0.263086in, y=0.674943in, left, base]{\color{textcolor}\sffamily\fontsize{8.000000}{9.600000}\selectfont \(\displaystyle {100}\)}%
\end{pgfscope}%
\begin{pgfscope}%
\pgfsetbuttcap%
\pgfsetroundjoin%
\definecolor{currentfill}{rgb}{0.000000,0.000000,0.000000}%
\pgfsetfillcolor{currentfill}%
\pgfsetlinewidth{0.803000pt}%
\definecolor{currentstroke}{rgb}{0.000000,0.000000,0.000000}%
\pgfsetstrokecolor{currentstroke}%
\pgfsetdash{}{0pt}%
\pgfsys@defobject{currentmarker}{\pgfqpoint{-0.048611in}{0.000000in}}{\pgfqpoint{-0.000000in}{0.000000in}}{%
\pgfpathmoveto{\pgfqpoint{-0.000000in}{0.000000in}}%
\pgfpathlineto{\pgfqpoint{-0.048611in}{0.000000in}}%
\pgfusepath{stroke,fill}%
}%
\begin{pgfscope}%
\pgfsys@transformshift{0.537394in}{0.966465in}%
\pgfsys@useobject{currentmarker}{}%
\end{pgfscope}%
\end{pgfscope}%
\begin{pgfscope}%
\definecolor{textcolor}{rgb}{0.000000,0.000000,0.000000}%
\pgfsetstrokecolor{textcolor}%
\pgfsetfillcolor{textcolor}%
\pgftext[x=0.263086in, y=0.924256in, left, base]{\color{textcolor}\sffamily\fontsize{8.000000}{9.600000}\selectfont \(\displaystyle {200}\)}%
\end{pgfscope}%
\begin{pgfscope}%
\pgfsetbuttcap%
\pgfsetroundjoin%
\definecolor{currentfill}{rgb}{0.000000,0.000000,0.000000}%
\pgfsetfillcolor{currentfill}%
\pgfsetlinewidth{0.803000pt}%
\definecolor{currentstroke}{rgb}{0.000000,0.000000,0.000000}%
\pgfsetstrokecolor{currentstroke}%
\pgfsetdash{}{0pt}%
\pgfsys@defobject{currentmarker}{\pgfqpoint{-0.048611in}{0.000000in}}{\pgfqpoint{-0.000000in}{0.000000in}}{%
\pgfpathmoveto{\pgfqpoint{-0.000000in}{0.000000in}}%
\pgfpathlineto{\pgfqpoint{-0.048611in}{0.000000in}}%
\pgfusepath{stroke,fill}%
}%
\begin{pgfscope}%
\pgfsys@transformshift{0.537394in}{1.215779in}%
\pgfsys@useobject{currentmarker}{}%
\end{pgfscope}%
\end{pgfscope}%
\begin{pgfscope}%
\definecolor{textcolor}{rgb}{0.000000,0.000000,0.000000}%
\pgfsetstrokecolor{textcolor}%
\pgfsetfillcolor{textcolor}%
\pgftext[x=0.263086in, y=1.173569in, left, base]{\color{textcolor}\sffamily\fontsize{8.000000}{9.600000}\selectfont \(\displaystyle {300}\)}%
\end{pgfscope}%
\begin{pgfscope}%
\pgfsetbuttcap%
\pgfsetroundjoin%
\definecolor{currentfill}{rgb}{0.000000,0.000000,0.000000}%
\pgfsetfillcolor{currentfill}%
\pgfsetlinewidth{0.803000pt}%
\definecolor{currentstroke}{rgb}{0.000000,0.000000,0.000000}%
\pgfsetstrokecolor{currentstroke}%
\pgfsetdash{}{0pt}%
\pgfsys@defobject{currentmarker}{\pgfqpoint{-0.048611in}{0.000000in}}{\pgfqpoint{-0.000000in}{0.000000in}}{%
\pgfpathmoveto{\pgfqpoint{-0.000000in}{0.000000in}}%
\pgfpathlineto{\pgfqpoint{-0.048611in}{0.000000in}}%
\pgfusepath{stroke,fill}%
}%
\begin{pgfscope}%
\pgfsys@transformshift{0.537394in}{1.465092in}%
\pgfsys@useobject{currentmarker}{}%
\end{pgfscope}%
\end{pgfscope}%
\begin{pgfscope}%
\definecolor{textcolor}{rgb}{0.000000,0.000000,0.000000}%
\pgfsetstrokecolor{textcolor}%
\pgfsetfillcolor{textcolor}%
\pgftext[x=0.263086in, y=1.422883in, left, base]{\color{textcolor}\sffamily\fontsize{8.000000}{9.600000}\selectfont \(\displaystyle {400}\)}%
\end{pgfscope}%
\begin{pgfscope}%
\definecolor{textcolor}{rgb}{0.000000,0.000000,0.000000}%
\pgfsetstrokecolor{textcolor}%
\pgfsetfillcolor{textcolor}%
\pgftext[x=0.207530in,y=0.966465in,,bottom,rotate=90.000000]{\color{textcolor}\sffamily\fontsize{8.000000}{9.600000}\selectfont Benchmarks solved}%
\end{pgfscope}%
\begin{pgfscope}%
\pgfpathrectangle{\pgfqpoint{0.537394in}{0.467838in}}{\pgfqpoint{4.094684in}{0.997254in}}%
\pgfusepath{clip}%
\pgfsetrectcap%
\pgfsetroundjoin%
\pgfsetlinewidth{1.003750pt}%
\definecolor{currentstroke}{rgb}{0.121569,0.466667,0.705882}%
\pgfsetstrokecolor{currentstroke}%
\pgfsetdash{}{0pt}%
\pgfpathmoveto{\pgfqpoint{0.537394in}{0.482797in}}%
\pgfpathlineto{\pgfqpoint{0.623180in}{0.485290in}}%
\pgfpathlineto{\pgfqpoint{0.652299in}{0.487783in}}%
\pgfpathlineto{\pgfqpoint{1.155535in}{0.490277in}}%
\pgfpathlineto{\pgfqpoint{1.161909in}{0.492770in}}%
\pgfpathlineto{\pgfqpoint{1.184532in}{0.495263in}}%
\pgfpathlineto{\pgfqpoint{1.193365in}{0.497756in}}%
\pgfpathlineto{\pgfqpoint{1.223585in}{0.500249in}}%
\pgfpathlineto{\pgfqpoint{1.256641in}{0.502742in}}%
\pgfpathlineto{\pgfqpoint{1.281347in}{0.505235in}}%
\pgfpathlineto{\pgfqpoint{1.292678in}{0.507728in}}%
\pgfpathlineto{\pgfqpoint{1.295722in}{0.510222in}}%
\pgfpathlineto{\pgfqpoint{1.301003in}{0.512715in}}%
\pgfpathlineto{\pgfqpoint{1.303519in}{0.515208in}}%
\pgfpathlineto{\pgfqpoint{1.320749in}{0.517701in}}%
\pgfpathlineto{\pgfqpoint{1.332065in}{0.520194in}}%
\pgfpathlineto{\pgfqpoint{1.342626in}{0.522687in}}%
\pgfpathlineto{\pgfqpoint{1.345267in}{0.525180in}}%
\pgfpathlineto{\pgfqpoint{1.355900in}{0.527674in}}%
\pgfpathlineto{\pgfqpoint{1.367340in}{0.530167in}}%
\pgfpathlineto{\pgfqpoint{1.368308in}{0.532660in}}%
\pgfpathlineto{\pgfqpoint{1.378260in}{0.535153in}}%
\pgfpathlineto{\pgfqpoint{1.387535in}{0.537646in}}%
\pgfpathlineto{\pgfqpoint{1.390917in}{0.540139in}}%
\pgfpathlineto{\pgfqpoint{1.397462in}{0.542632in}}%
\pgfpathlineto{\pgfqpoint{1.399096in}{0.545125in}}%
\pgfpathlineto{\pgfqpoint{1.407754in}{0.547619in}}%
\pgfpathlineto{\pgfqpoint{1.413951in}{0.550112in}}%
\pgfpathlineto{\pgfqpoint{1.421942in}{0.552605in}}%
\pgfpathlineto{\pgfqpoint{1.425962in}{0.555098in}}%
\pgfpathlineto{\pgfqpoint{1.430414in}{0.557591in}}%
\pgfpathlineto{\pgfqpoint{1.432837in}{0.560084in}}%
\pgfpathlineto{\pgfqpoint{1.435947in}{0.562577in}}%
\pgfpathlineto{\pgfqpoint{1.436441in}{0.565071in}}%
\pgfpathlineto{\pgfqpoint{1.436765in}{0.567564in}}%
\pgfpathlineto{\pgfqpoint{1.441966in}{0.570057in}}%
\pgfpathlineto{\pgfqpoint{1.443158in}{0.572550in}}%
\pgfpathlineto{\pgfqpoint{1.445457in}{0.575043in}}%
\pgfpathlineto{\pgfqpoint{1.446690in}{0.577536in}}%
\pgfpathlineto{\pgfqpoint{1.447142in}{0.580029in}}%
\pgfpathlineto{\pgfqpoint{1.450932in}{0.582523in}}%
\pgfpathlineto{\pgfqpoint{1.451534in}{0.585016in}}%
\pgfpathlineto{\pgfqpoint{1.453453in}{0.587509in}}%
\pgfpathlineto{\pgfqpoint{1.456447in}{0.590002in}}%
\pgfpathlineto{\pgfqpoint{1.456641in}{0.592495in}}%
\pgfpathlineto{\pgfqpoint{1.458359in}{0.594988in}}%
\pgfpathlineto{\pgfqpoint{1.458743in}{0.597481in}}%
\pgfpathlineto{\pgfqpoint{1.469429in}{0.599974in}}%
\pgfpathlineto{\pgfqpoint{1.471863in}{0.602468in}}%
\pgfpathlineto{\pgfqpoint{1.474679in}{0.604961in}}%
\pgfpathlineto{\pgfqpoint{1.474871in}{0.607454in}}%
\pgfpathlineto{\pgfqpoint{1.476853in}{0.609947in}}%
\pgfpathlineto{\pgfqpoint{1.478267in}{0.612440in}}%
\pgfpathlineto{\pgfqpoint{1.480719in}{0.614933in}}%
\pgfpathlineto{\pgfqpoint{1.483143in}{0.617426in}}%
\pgfpathlineto{\pgfqpoint{1.483941in}{0.619920in}}%
\pgfpathlineto{\pgfqpoint{1.488954in}{0.622413in}}%
\pgfpathlineto{\pgfqpoint{1.494438in}{0.624906in}}%
\pgfpathlineto{\pgfqpoint{1.499118in}{0.627399in}}%
\pgfpathlineto{\pgfqpoint{1.509457in}{0.629892in}}%
\pgfpathlineto{\pgfqpoint{1.517462in}{0.632385in}}%
\pgfpathlineto{\pgfqpoint{1.521267in}{0.634878in}}%
\pgfpathlineto{\pgfqpoint{1.526991in}{0.637371in}}%
\pgfpathlineto{\pgfqpoint{1.537478in}{0.639865in}}%
\pgfpathlineto{\pgfqpoint{1.544394in}{0.642358in}}%
\pgfpathlineto{\pgfqpoint{1.548674in}{0.644851in}}%
\pgfpathlineto{\pgfqpoint{1.549197in}{0.647344in}}%
\pgfpathlineto{\pgfqpoint{1.550089in}{0.649837in}}%
\pgfpathlineto{\pgfqpoint{1.550778in}{0.652330in}}%
\pgfpathlineto{\pgfqpoint{1.563149in}{0.654823in}}%
\pgfpathlineto{\pgfqpoint{1.569719in}{0.657317in}}%
\pgfpathlineto{\pgfqpoint{1.575970in}{0.659810in}}%
\pgfpathlineto{\pgfqpoint{1.581049in}{0.662303in}}%
\pgfpathlineto{\pgfqpoint{1.581427in}{0.664796in}}%
\pgfpathlineto{\pgfqpoint{1.585031in}{0.667289in}}%
\pgfpathlineto{\pgfqpoint{1.588977in}{0.669782in}}%
\pgfpathlineto{\pgfqpoint{1.590273in}{0.672275in}}%
\pgfpathlineto{\pgfqpoint{1.604962in}{0.674768in}}%
\pgfpathlineto{\pgfqpoint{1.607996in}{0.677262in}}%
\pgfpathlineto{\pgfqpoint{1.608017in}{0.679755in}}%
\pgfpathlineto{\pgfqpoint{1.609584in}{0.682248in}}%
\pgfpathlineto{\pgfqpoint{1.613808in}{0.684741in}}%
\pgfpathlineto{\pgfqpoint{1.637540in}{0.687234in}}%
\pgfpathlineto{\pgfqpoint{1.644115in}{0.689727in}}%
\pgfpathlineto{\pgfqpoint{1.648692in}{0.692220in}}%
\pgfpathlineto{\pgfqpoint{1.665201in}{0.694714in}}%
\pgfpathlineto{\pgfqpoint{1.679582in}{0.697207in}}%
\pgfpathlineto{\pgfqpoint{1.681824in}{0.699700in}}%
\pgfpathlineto{\pgfqpoint{1.691189in}{0.702193in}}%
\pgfpathlineto{\pgfqpoint{1.691711in}{0.704686in}}%
\pgfpathlineto{\pgfqpoint{1.724243in}{0.707179in}}%
\pgfpathlineto{\pgfqpoint{1.731394in}{0.709672in}}%
\pgfpathlineto{\pgfqpoint{1.733611in}{0.712165in}}%
\pgfpathlineto{\pgfqpoint{1.782049in}{0.714659in}}%
\pgfpathlineto{\pgfqpoint{1.793838in}{0.717152in}}%
\pgfpathlineto{\pgfqpoint{1.828569in}{0.719645in}}%
\pgfpathlineto{\pgfqpoint{1.833818in}{0.722138in}}%
\pgfpathlineto{\pgfqpoint{1.840259in}{0.724631in}}%
\pgfpathlineto{\pgfqpoint{1.845773in}{0.727124in}}%
\pgfpathlineto{\pgfqpoint{1.865641in}{0.729617in}}%
\pgfpathlineto{\pgfqpoint{1.908000in}{0.732111in}}%
\pgfpathlineto{\pgfqpoint{1.913583in}{0.734604in}}%
\pgfusepath{stroke}%
\end{pgfscope}%
\begin{pgfscope}%
\pgfpathrectangle{\pgfqpoint{0.537394in}{0.467838in}}{\pgfqpoint{4.094684in}{0.997254in}}%
\pgfusepath{clip}%
\pgfsetrectcap%
\pgfsetroundjoin%
\pgfsetlinewidth{1.003750pt}%
\definecolor{currentstroke}{rgb}{1.000000,0.498039,0.054902}%
\pgfsetstrokecolor{currentstroke}%
\pgfsetdash{}{0pt}%
\pgfpathmoveto{\pgfqpoint{1.587682in}{0.470331in}}%
\pgfpathlineto{\pgfqpoint{1.606153in}{0.472825in}}%
\pgfpathlineto{\pgfqpoint{1.610623in}{0.475318in}}%
\pgfpathlineto{\pgfqpoint{1.610786in}{0.477811in}}%
\pgfpathlineto{\pgfqpoint{1.614743in}{0.480304in}}%
\pgfpathlineto{\pgfqpoint{1.614896in}{0.482797in}}%
\pgfpathlineto{\pgfqpoint{1.618575in}{0.485290in}}%
\pgfpathlineto{\pgfqpoint{1.620140in}{0.487783in}}%
\pgfpathlineto{\pgfqpoint{1.624884in}{0.490277in}}%
\pgfpathlineto{\pgfqpoint{1.645141in}{0.492770in}}%
\pgfpathlineto{\pgfqpoint{1.650265in}{0.495263in}}%
\pgfpathlineto{\pgfqpoint{1.653662in}{0.497756in}}%
\pgfpathlineto{\pgfqpoint{1.655956in}{0.500249in}}%
\pgfpathlineto{\pgfqpoint{1.656218in}{0.502742in}}%
\pgfpathlineto{\pgfqpoint{1.656255in}{0.505235in}}%
\pgfpathlineto{\pgfqpoint{1.656703in}{0.507728in}}%
\pgfpathlineto{\pgfqpoint{1.658222in}{0.510222in}}%
\pgfpathlineto{\pgfqpoint{1.660531in}{0.512715in}}%
\pgfpathlineto{\pgfqpoint{1.660700in}{0.515208in}}%
\pgfpathlineto{\pgfqpoint{1.660728in}{0.517701in}}%
\pgfpathlineto{\pgfqpoint{1.662344in}{0.520194in}}%
\pgfpathlineto{\pgfqpoint{1.663169in}{0.522687in}}%
\pgfpathlineto{\pgfqpoint{1.663604in}{0.525180in}}%
\pgfpathlineto{\pgfqpoint{1.664528in}{0.527674in}}%
\pgfpathlineto{\pgfqpoint{1.666994in}{0.530167in}}%
\pgfpathlineto{\pgfqpoint{1.667537in}{0.532660in}}%
\pgfpathlineto{\pgfqpoint{1.670047in}{0.535153in}}%
\pgfpathlineto{\pgfqpoint{1.672522in}{0.537646in}}%
\pgfpathlineto{\pgfqpoint{1.673325in}{0.540139in}}%
\pgfpathlineto{\pgfqpoint{1.673396in}{0.542632in}}%
\pgfpathlineto{\pgfqpoint{1.674456in}{0.545125in}}%
\pgfpathlineto{\pgfqpoint{1.675032in}{0.547619in}}%
\pgfpathlineto{\pgfqpoint{1.678001in}{0.550112in}}%
\pgfpathlineto{\pgfqpoint{1.681131in}{0.552605in}}%
\pgfpathlineto{\pgfqpoint{1.684848in}{0.555098in}}%
\pgfpathlineto{\pgfqpoint{1.687347in}{0.557591in}}%
\pgfpathlineto{\pgfqpoint{1.689643in}{0.560084in}}%
\pgfpathlineto{\pgfqpoint{1.691249in}{0.562577in}}%
\pgfpathlineto{\pgfqpoint{1.692737in}{0.565071in}}%
\pgfpathlineto{\pgfqpoint{1.693043in}{0.567564in}}%
\pgfpathlineto{\pgfqpoint{1.695298in}{0.570057in}}%
\pgfpathlineto{\pgfqpoint{1.695580in}{0.572550in}}%
\pgfpathlineto{\pgfqpoint{1.695930in}{0.575043in}}%
\pgfpathlineto{\pgfqpoint{1.696780in}{0.577536in}}%
\pgfpathlineto{\pgfqpoint{1.701227in}{0.580029in}}%
\pgfpathlineto{\pgfqpoint{1.703074in}{0.582523in}}%
\pgfpathlineto{\pgfqpoint{1.708703in}{0.585016in}}%
\pgfpathlineto{\pgfqpoint{1.709323in}{0.587509in}}%
\pgfpathlineto{\pgfqpoint{1.710241in}{0.590002in}}%
\pgfpathlineto{\pgfqpoint{1.710427in}{0.592495in}}%
\pgfpathlineto{\pgfqpoint{1.713202in}{0.594988in}}%
\pgfpathlineto{\pgfqpoint{1.715938in}{0.597481in}}%
\pgfpathlineto{\pgfqpoint{1.716185in}{0.599974in}}%
\pgfpathlineto{\pgfqpoint{1.723054in}{0.602468in}}%
\pgfpathlineto{\pgfqpoint{1.723943in}{0.604961in}}%
\pgfpathlineto{\pgfqpoint{1.726644in}{0.607454in}}%
\pgfpathlineto{\pgfqpoint{1.728389in}{0.609947in}}%
\pgfpathlineto{\pgfqpoint{1.731767in}{0.612440in}}%
\pgfpathlineto{\pgfqpoint{1.732259in}{0.614933in}}%
\pgfpathlineto{\pgfqpoint{1.733018in}{0.617426in}}%
\pgfpathlineto{\pgfqpoint{1.746308in}{0.619920in}}%
\pgfpathlineto{\pgfqpoint{1.750463in}{0.622413in}}%
\pgfpathlineto{\pgfqpoint{1.751782in}{0.624906in}}%
\pgfpathlineto{\pgfqpoint{1.753319in}{0.627399in}}%
\pgfpathlineto{\pgfqpoint{1.758139in}{0.629892in}}%
\pgfpathlineto{\pgfqpoint{1.768082in}{0.632385in}}%
\pgfpathlineto{\pgfqpoint{1.783506in}{0.634878in}}%
\pgfpathlineto{\pgfqpoint{1.784600in}{0.637371in}}%
\pgfpathlineto{\pgfqpoint{1.785287in}{0.639865in}}%
\pgfpathlineto{\pgfqpoint{1.799146in}{0.642358in}}%
\pgfpathlineto{\pgfqpoint{1.844278in}{0.644851in}}%
\pgfpathlineto{\pgfqpoint{1.862830in}{0.647344in}}%
\pgfpathlineto{\pgfqpoint{1.967921in}{0.649837in}}%
\pgfusepath{stroke}%
\end{pgfscope}%
\begin{pgfscope}%
\pgfpathrectangle{\pgfqpoint{0.537394in}{0.467838in}}{\pgfqpoint{4.094684in}{0.997254in}}%
\pgfusepath{clip}%
\pgfsetrectcap%
\pgfsetroundjoin%
\pgfsetlinewidth{1.003750pt}%
\definecolor{currentstroke}{rgb}{0.172549,0.627451,0.172549}%
\pgfsetstrokecolor{currentstroke}%
\pgfsetdash{}{0pt}%
\pgfpathmoveto{\pgfqpoint{2.629336in}{0.470331in}}%
\pgfpathlineto{\pgfqpoint{2.630777in}{0.472825in}}%
\pgfpathlineto{\pgfqpoint{2.632740in}{0.475318in}}%
\pgfpathlineto{\pgfqpoint{2.634423in}{0.477811in}}%
\pgfpathlineto{\pgfqpoint{2.634777in}{0.480304in}}%
\pgfpathlineto{\pgfqpoint{2.638446in}{0.482797in}}%
\pgfpathlineto{\pgfqpoint{2.639466in}{0.485290in}}%
\pgfpathlineto{\pgfqpoint{2.640442in}{0.487783in}}%
\pgfpathlineto{\pgfqpoint{2.645044in}{0.490277in}}%
\pgfpathlineto{\pgfqpoint{2.646381in}{0.492770in}}%
\pgfpathlineto{\pgfqpoint{2.648579in}{0.495263in}}%
\pgfpathlineto{\pgfqpoint{2.649997in}{0.497756in}}%
\pgfpathlineto{\pgfqpoint{2.650508in}{0.500249in}}%
\pgfpathlineto{\pgfqpoint{2.651662in}{0.502742in}}%
\pgfpathlineto{\pgfqpoint{2.652057in}{0.505235in}}%
\pgfpathlineto{\pgfqpoint{2.653167in}{0.507728in}}%
\pgfpathlineto{\pgfqpoint{2.659152in}{0.510222in}}%
\pgfpathlineto{\pgfqpoint{2.689255in}{0.512715in}}%
\pgfpathlineto{\pgfqpoint{2.703416in}{0.515208in}}%
\pgfpathlineto{\pgfqpoint{2.769193in}{0.517701in}}%
\pgfpathlineto{\pgfqpoint{2.771843in}{0.520194in}}%
\pgfpathlineto{\pgfqpoint{2.774546in}{0.522687in}}%
\pgfpathlineto{\pgfqpoint{2.777503in}{0.525180in}}%
\pgfpathlineto{\pgfqpoint{2.780019in}{0.527674in}}%
\pgfpathlineto{\pgfqpoint{2.780519in}{0.530167in}}%
\pgfpathlineto{\pgfqpoint{2.782335in}{0.532660in}}%
\pgfpathlineto{\pgfqpoint{2.791381in}{0.535153in}}%
\pgfpathlineto{\pgfqpoint{2.793183in}{0.537646in}}%
\pgfpathlineto{\pgfqpoint{2.797060in}{0.540139in}}%
\pgfpathlineto{\pgfqpoint{2.800837in}{0.542632in}}%
\pgfpathlineto{\pgfqpoint{2.806680in}{0.545125in}}%
\pgfpathlineto{\pgfqpoint{2.808580in}{0.547619in}}%
\pgfpathlineto{\pgfqpoint{2.808834in}{0.550112in}}%
\pgfpathlineto{\pgfqpoint{2.817504in}{0.552605in}}%
\pgfpathlineto{\pgfqpoint{2.817930in}{0.555098in}}%
\pgfpathlineto{\pgfqpoint{2.820964in}{0.557591in}}%
\pgfpathlineto{\pgfqpoint{2.821953in}{0.560084in}}%
\pgfpathlineto{\pgfqpoint{2.826339in}{0.562577in}}%
\pgfpathlineto{\pgfqpoint{2.827180in}{0.565071in}}%
\pgfpathlineto{\pgfqpoint{2.833731in}{0.567564in}}%
\pgfpathlineto{\pgfqpoint{2.833919in}{0.570057in}}%
\pgfpathlineto{\pgfqpoint{2.836498in}{0.572550in}}%
\pgfpathlineto{\pgfqpoint{2.841966in}{0.575043in}}%
\pgfpathlineto{\pgfqpoint{2.847268in}{0.577536in}}%
\pgfpathlineto{\pgfqpoint{2.849184in}{0.580029in}}%
\pgfpathlineto{\pgfqpoint{2.850101in}{0.582523in}}%
\pgfpathlineto{\pgfqpoint{2.850370in}{0.585016in}}%
\pgfpathlineto{\pgfqpoint{2.856100in}{0.587509in}}%
\pgfpathlineto{\pgfqpoint{2.861029in}{0.590002in}}%
\pgfpathlineto{\pgfqpoint{2.864223in}{0.592495in}}%
\pgfpathlineto{\pgfqpoint{2.870729in}{0.594988in}}%
\pgfpathlineto{\pgfqpoint{2.870856in}{0.597481in}}%
\pgfpathlineto{\pgfqpoint{2.872796in}{0.599974in}}%
\pgfpathlineto{\pgfqpoint{2.873479in}{0.602468in}}%
\pgfpathlineto{\pgfqpoint{2.884455in}{0.604961in}}%
\pgfpathlineto{\pgfqpoint{2.890655in}{0.607454in}}%
\pgfpathlineto{\pgfqpoint{2.896335in}{0.609947in}}%
\pgfpathlineto{\pgfqpoint{2.900271in}{0.612440in}}%
\pgfpathlineto{\pgfqpoint{2.900505in}{0.614933in}}%
\pgfpathlineto{\pgfqpoint{2.902393in}{0.617426in}}%
\pgfpathlineto{\pgfqpoint{2.904381in}{0.619920in}}%
\pgfpathlineto{\pgfqpoint{2.912657in}{0.622413in}}%
\pgfpathlineto{\pgfqpoint{2.946906in}{0.624906in}}%
\pgfpathlineto{\pgfqpoint{2.954005in}{0.627399in}}%
\pgfpathlineto{\pgfqpoint{2.955367in}{0.629892in}}%
\pgfpathlineto{\pgfqpoint{2.964918in}{0.632385in}}%
\pgfpathlineto{\pgfqpoint{2.969194in}{0.634878in}}%
\pgfpathlineto{\pgfqpoint{2.978289in}{0.637371in}}%
\pgfpathlineto{\pgfqpoint{2.978306in}{0.639865in}}%
\pgfpathlineto{\pgfqpoint{2.980802in}{0.642358in}}%
\pgfpathlineto{\pgfqpoint{2.985517in}{0.644851in}}%
\pgfpathlineto{\pgfqpoint{3.005528in}{0.647344in}}%
\pgfpathlineto{\pgfqpoint{3.009720in}{0.649837in}}%
\pgfpathlineto{\pgfqpoint{3.013024in}{0.652330in}}%
\pgfpathlineto{\pgfqpoint{3.021095in}{0.654823in}}%
\pgfpathlineto{\pgfqpoint{3.040409in}{0.657317in}}%
\pgfpathlineto{\pgfqpoint{3.115390in}{0.659810in}}%
\pgfpathlineto{\pgfqpoint{3.146666in}{0.662303in}}%
\pgfpathlineto{\pgfqpoint{3.179080in}{0.664796in}}%
\pgfpathlineto{\pgfqpoint{3.197795in}{0.667289in}}%
\pgfpathlineto{\pgfqpoint{3.199165in}{0.669782in}}%
\pgfpathlineto{\pgfqpoint{3.213815in}{0.672275in}}%
\pgfpathlineto{\pgfqpoint{3.242407in}{0.674768in}}%
\pgfusepath{stroke}%
\end{pgfscope}%
\begin{pgfscope}%
\pgfpathrectangle{\pgfqpoint{0.537394in}{0.467838in}}{\pgfqpoint{4.094684in}{0.997254in}}%
\pgfusepath{clip}%
\pgfsetbuttcap%
\pgfsetroundjoin%
\pgfsetlinewidth{1.003750pt}%
\definecolor{currentstroke}{rgb}{0.839216,0.152941,0.156863}%
\pgfsetstrokecolor{currentstroke}%
\pgfsetdash{{3.700000pt}{1.600000pt}}{0.000000pt}%
\pgfpathmoveto{\pgfqpoint{1.109806in}{0.480304in}}%
\pgfpathlineto{\pgfqpoint{1.174651in}{0.485290in}}%
\pgfpathlineto{\pgfqpoint{1.229476in}{0.490277in}}%
\pgfpathlineto{\pgfqpoint{1.318858in}{0.497756in}}%
\pgfpathlineto{\pgfqpoint{1.356331in}{0.500249in}}%
\pgfpathlineto{\pgfqpoint{1.390229in}{0.502742in}}%
\pgfpathlineto{\pgfqpoint{1.421175in}{0.505235in}}%
\pgfpathlineto{\pgfqpoint{1.476000in}{0.507728in}}%
\pgfpathlineto{\pgfqpoint{1.500538in}{0.512715in}}%
\pgfpathlineto{\pgfqpoint{1.565383in}{0.515208in}}%
\pgfpathlineto{\pgfqpoint{1.584612in}{0.517701in}}%
\pgfpathlineto{\pgfqpoint{1.636753in}{0.525180in}}%
\pgfpathlineto{\pgfqpoint{1.652563in}{0.530167in}}%
\pgfpathlineto{\pgfqpoint{1.667700in}{0.532660in}}%
\pgfpathlineto{\pgfqpoint{1.696168in}{0.535153in}}%
\pgfpathlineto{\pgfqpoint{1.709590in}{0.540139in}}%
\pgfpathlineto{\pgfqpoint{1.722525in}{0.545125in}}%
\pgfpathlineto{\pgfqpoint{1.849380in}{0.547619in}}%
\pgfusepath{stroke}%
\end{pgfscope}%
\begin{pgfscope}%
\pgfpathrectangle{\pgfqpoint{0.537394in}{0.467838in}}{\pgfqpoint{4.094684in}{0.997254in}}%
\pgfusepath{clip}%
\pgfsetbuttcap%
\pgfsetroundjoin%
\pgfsetlinewidth{1.003750pt}%
\definecolor{currentstroke}{rgb}{0.580392,0.403922,0.741176}%
\pgfsetstrokecolor{currentstroke}%
\pgfsetdash{{3.700000pt}{1.600000pt}}{0.000000pt}%
\pgfpathmoveto{\pgfqpoint{1.109806in}{0.477811in}}%
\pgfpathlineto{\pgfqpoint{1.174651in}{0.482797in}}%
\pgfpathlineto{\pgfqpoint{1.276968in}{0.485290in}}%
\pgfpathlineto{\pgfqpoint{1.318858in}{0.487783in}}%
\pgfpathlineto{\pgfqpoint{1.390229in}{0.490277in}}%
\pgfpathlineto{\pgfqpoint{1.421175in}{0.492770in}}%
\pgfpathlineto{\pgfqpoint{1.449643in}{0.497756in}}%
\pgfpathlineto{\pgfqpoint{1.476000in}{0.505235in}}%
\pgfpathlineto{\pgfqpoint{1.500538in}{0.510222in}}%
\pgfpathlineto{\pgfqpoint{1.523492in}{0.517701in}}%
\pgfpathlineto{\pgfqpoint{1.545054in}{0.520194in}}%
\pgfpathlineto{\pgfqpoint{1.565383in}{0.530167in}}%
\pgfpathlineto{\pgfqpoint{1.584612in}{0.535153in}}%
\pgfpathlineto{\pgfqpoint{1.620208in}{0.540139in}}%
\pgfpathlineto{\pgfqpoint{1.652563in}{0.545125in}}%
\pgfpathlineto{\pgfqpoint{1.709590in}{0.550112in}}%
\pgfpathlineto{\pgfqpoint{1.747063in}{0.552605in}}%
\pgfpathlineto{\pgfqpoint{1.770017in}{0.555098in}}%
\pgfpathlineto{\pgfqpoint{1.811907in}{0.557591in}}%
\pgfpathlineto{\pgfqpoint{1.831137in}{0.560084in}}%
\pgfpathlineto{\pgfqpoint{1.840375in}{0.562577in}}%
\pgfpathlineto{\pgfqpoint{1.866732in}{0.565071in}}%
\pgfpathlineto{\pgfqpoint{1.875101in}{0.567564in}}%
\pgfpathlineto{\pgfqpoint{1.899087in}{0.570057in}}%
\pgfpathlineto{\pgfqpoint{1.914224in}{0.572550in}}%
\pgfpathlineto{\pgfqpoint{1.921558in}{0.575043in}}%
\pgfpathlineto{\pgfqpoint{1.987610in}{0.577536in}}%
\pgfpathlineto{\pgfqpoint{1.993587in}{0.580029in}}%
\pgfpathlineto{\pgfqpoint{2.016541in}{0.582523in}}%
\pgfpathlineto{\pgfqpoint{2.038103in}{0.585016in}}%
\pgfpathlineto{\pgfqpoint{2.043295in}{0.587509in}}%
\pgfpathlineto{\pgfqpoint{2.095904in}{0.590002in}}%
\pgfpathlineto{\pgfqpoint{2.121626in}{0.592495in}}%
\pgfpathlineto{\pgfqpoint{2.164434in}{0.594988in}}%
\pgfusepath{stroke}%
\end{pgfscope}%
\begin{pgfscope}%
\pgfsetrectcap%
\pgfsetmiterjoin%
\pgfsetlinewidth{0.803000pt}%
\definecolor{currentstroke}{rgb}{0.000000,0.000000,0.000000}%
\pgfsetstrokecolor{currentstroke}%
\pgfsetdash{}{0pt}%
\pgfpathmoveto{\pgfqpoint{0.537394in}{0.467838in}}%
\pgfpathlineto{\pgfqpoint{0.537394in}{1.465092in}}%
\pgfusepath{stroke}%
\end{pgfscope}%
\begin{pgfscope}%
\pgfsetrectcap%
\pgfsetmiterjoin%
\pgfsetlinewidth{0.803000pt}%
\definecolor{currentstroke}{rgb}{0.000000,0.000000,0.000000}%
\pgfsetstrokecolor{currentstroke}%
\pgfsetdash{}{0pt}%
\pgfpathmoveto{\pgfqpoint{4.632078in}{0.467838in}}%
\pgfpathlineto{\pgfqpoint{4.632078in}{1.465092in}}%
\pgfusepath{stroke}%
\end{pgfscope}%
\begin{pgfscope}%
\pgfsetrectcap%
\pgfsetmiterjoin%
\pgfsetlinewidth{0.803000pt}%
\definecolor{currentstroke}{rgb}{0.000000,0.000000,0.000000}%
\pgfsetstrokecolor{currentstroke}%
\pgfsetdash{}{0pt}%
\pgfpathmoveto{\pgfqpoint{0.537394in}{0.467838in}}%
\pgfpathlineto{\pgfqpoint{4.632078in}{0.467838in}}%
\pgfusepath{stroke}%
\end{pgfscope}%
\begin{pgfscope}%
\pgfsetrectcap%
\pgfsetmiterjoin%
\pgfsetlinewidth{0.803000pt}%
\definecolor{currentstroke}{rgb}{0.000000,0.000000,0.000000}%
\pgfsetstrokecolor{currentstroke}%
\pgfsetdash{}{0pt}%
\pgfpathmoveto{\pgfqpoint{0.537394in}{1.465092in}}%
\pgfpathlineto{\pgfqpoint{4.632078in}{1.465092in}}%
\pgfusepath{stroke}%
\end{pgfscope}%
\begin{pgfscope}%
\pgfsetbuttcap%
\pgfsetmiterjoin%
\definecolor{currentfill}{rgb}{1.000000,1.000000,1.000000}%
\pgfsetfillcolor{currentfill}%
\pgfsetfillopacity{0.800000}%
\pgfsetlinewidth{1.003750pt}%
\definecolor{currentstroke}{rgb}{0.800000,0.800000,0.800000}%
\pgfsetstrokecolor{currentstroke}%
\pgfsetstrokeopacity{0.800000}%
\pgfsetdash{}{0pt}%
\pgfpathmoveto{\pgfqpoint{3.361234in}{0.560774in}}%
\pgfpathlineto{\pgfqpoint{4.554300in}{0.560774in}}%
\pgfpathquadraticcurveto{\pgfqpoint{4.576522in}{0.560774in}}{\pgfqpoint{4.576522in}{0.582996in}}%
\pgfpathlineto{\pgfqpoint{4.576522in}{1.387314in}}%
\pgfpathquadraticcurveto{\pgfqpoint{4.576522in}{1.409536in}}{\pgfqpoint{4.554300in}{1.409536in}}%
\pgfpathlineto{\pgfqpoint{3.361234in}{1.409536in}}%
\pgfpathquadraticcurveto{\pgfqpoint{3.339012in}{1.409536in}}{\pgfqpoint{3.339012in}{1.387314in}}%
\pgfpathlineto{\pgfqpoint{3.339012in}{0.582996in}}%
\pgfpathquadraticcurveto{\pgfqpoint{3.339012in}{0.560774in}}{\pgfqpoint{3.361234in}{0.560774in}}%
\pgfpathclose%
\pgfusepath{stroke,fill}%
\end{pgfscope}%
\begin{pgfscope}%
\pgfsetrectcap%
\pgfsetroundjoin%
\pgfsetlinewidth{1.003750pt}%
\definecolor{currentstroke}{rgb}{0.121569,0.466667,0.705882}%
\pgfsetstrokecolor{currentstroke}%
\pgfsetdash{}{0pt}%
\pgfpathmoveto{\pgfqpoint{3.383456in}{1.319562in}}%
\pgfpathlineto{\pgfqpoint{3.605678in}{1.319562in}}%
\pgfusepath{stroke}%
\end{pgfscope}%
\begin{pgfscope}%
\definecolor{textcolor}{rgb}{0.000000,0.000000,0.000000}%
\pgfsetstrokecolor{textcolor}%
\pgfsetfillcolor{textcolor}%
\pgftext[x=3.694567in,y=1.280674in,left,base]{\color{textcolor}\sffamily\fontsize{8.000000}{9.600000}\selectfont LG(FlowCutter)}%
\end{pgfscope}%
\begin{pgfscope}%
\pgfsetrectcap%
\pgfsetroundjoin%
\pgfsetlinewidth{1.003750pt}%
\definecolor{currentstroke}{rgb}{1.000000,0.498039,0.054902}%
\pgfsetstrokecolor{currentstroke}%
\pgfsetdash{}{0pt}%
\pgfpathmoveto{\pgfqpoint{3.383456in}{1.156477in}}%
\pgfpathlineto{\pgfqpoint{3.605678in}{1.156477in}}%
\pgfusepath{stroke}%
\end{pgfscope}%
\begin{pgfscope}%
\definecolor{textcolor}{rgb}{0.000000,0.000000,0.000000}%
\pgfsetstrokecolor{textcolor}%
\pgfsetfillcolor{textcolor}%
\pgftext[x=3.694567in,y=1.117588in,left,base]{\color{textcolor}\sffamily\fontsize{8.000000}{9.600000}\selectfont LG(htd)}%
\end{pgfscope}%
\begin{pgfscope}%
\pgfsetrectcap%
\pgfsetroundjoin%
\pgfsetlinewidth{1.003750pt}%
\definecolor{currentstroke}{rgb}{0.172549,0.627451,0.172549}%
\pgfsetstrokecolor{currentstroke}%
\pgfsetdash{}{0pt}%
\pgfpathmoveto{\pgfqpoint{3.383456in}{0.993391in}}%
\pgfpathlineto{\pgfqpoint{3.605678in}{0.993391in}}%
\pgfusepath{stroke}%
\end{pgfscope}%
\begin{pgfscope}%
\definecolor{textcolor}{rgb}{0.000000,0.000000,0.000000}%
\pgfsetstrokecolor{textcolor}%
\pgfsetfillcolor{textcolor}%
\pgftext[x=3.694567in,y=0.954502in,left,base]{\color{textcolor}\sffamily\fontsize{8.000000}{9.600000}\selectfont LG(Tamaki)}%
\end{pgfscope}%
\begin{pgfscope}%
\pgfsetbuttcap%
\pgfsetroundjoin%
\pgfsetlinewidth{1.003750pt}%
\definecolor{currentstroke}{rgb}{0.839216,0.152941,0.156863}%
\pgfsetstrokecolor{currentstroke}%
\pgfsetdash{{3.700000pt}{1.600000pt}}{0.000000pt}%
\pgfpathmoveto{\pgfqpoint{3.383456in}{0.830305in}}%
\pgfpathlineto{\pgfqpoint{3.605678in}{0.830305in}}%
\pgfusepath{stroke}%
\end{pgfscope}%
\begin{pgfscope}%
\definecolor{textcolor}{rgb}{0.000000,0.000000,0.000000}%
\pgfsetstrokecolor{textcolor}%
\pgfsetfillcolor{textcolor}%
\pgftext[x=3.694567in,y=0.791416in,left,base]{\color{textcolor}\sffamily\fontsize{8.000000}{9.600000}\selectfont HTB(MCS, BE)}%
\end{pgfscope}%
\begin{pgfscope}%
\pgfsetbuttcap%
\pgfsetroundjoin%
\pgfsetlinewidth{1.003750pt}%
\definecolor{currentstroke}{rgb}{0.580392,0.403922,0.741176}%
\pgfsetstrokecolor{currentstroke}%
\pgfsetdash{{3.700000pt}{1.600000pt}}{0.000000pt}%
\pgfpathmoveto{\pgfqpoint{3.383456in}{0.667219in}}%
\pgfpathlineto{\pgfqpoint{3.605678in}{0.667219in}}%
\pgfusepath{stroke}%
\end{pgfscope}%
\begin{pgfscope}%
\definecolor{textcolor}{rgb}{0.000000,0.000000,0.000000}%
\pgfsetstrokecolor{textcolor}%
\pgfsetfillcolor{textcolor}%
\pgftext[x=3.694567in,y=0.628330in,left,base]{\color{textcolor}\sffamily\fontsize{8.000000}{9.600000}\selectfont HTB(MCS, BM)}%
\end{pgfscope}%
\end{pgfpicture}%
\makeatother%
\endgroup%

    \caption{
        Experiment 1 compares the tree-decomposition-based planner \Lg{} to the constraint-satisfaction-based planner \htb{}.
	    A planner ``solves'' a benchmark when it finds a project-join tree of width \maxWidth{} or lower.
        % \Lg{} invokes a tree decomposer (\flowcutter{}, \htd{}, or \tamaki{}).
        % \Lg{} is an \emph{anytime} tool that produces several trees (of decreasing widths) for each benchmark.
        % We only show an \Lg{} data point if the first tree produced has width at most \maxWidth{}%
        % (we discard an \Lg{} data point when the first tree has width over \maxWidth, even if a later tree has width at most \maxWidth)%
        % .
        % \htb{} requires a variable-ordering heuristic (\mcs{}, \lexp, \lexm, or \minfill{}) and a clause-ordering heuristic (\be{} or \bm).
        For \htb, we only show the variable-ordering heuristic \mcs{}; the \lexp{}, \lexm{}, and \minfill{} curves are qualitatively similar.
    }
    \label{figPlanning}
\end{figure}

%%%%%%%%%%%%%%%%%%%%%%%%%%%%%%%%%%%%%%%%%%%%%%%%%%%%%%%%%%%%%%%%%%%%%%%%%%%%%%%%

\subsection{Experiment 2: Comparing Execution Heuristics}

In this experiment, we take the 346 graded project-join trees produced by \Lg{} with \flowcutter{} in Experiment 1 % (107 trees of widths 1-30 and 239 trees of widths 31-99)
and run \dmc{} once for 100 seconds with each of four ADD variable-ordering heuristics. 
We present the execution time of each heuristic (excluding planning time) in Figure \ref{figExecution}. 
We observe that \mcs{} and \lexp{} outperform \lexm{} and \minfill{}.
We use \dmc{} with \mcs{} in \procount{} for Experiment 3.
\begin{figure}[t]
    \centering
    %% Creator: Matplotlib, PGF backend
%%
%% To include the figure in your LaTeX document, write
%%   \input{<filename>.pgf}
%%
%% Make sure the required packages are loaded in your preamble
%%   \usepackage{pgf}
%%
%% and, on pdftex
%%   \usepackage[utf8]{inputenc}\DeclareUnicodeCharacter{2212}{-}
%%
%% or, on luatex and xetex
%%   \usepackage{unicode-math}
%%
%% Figures using additional raster images can only be included by \input if
%% they are in the same directory as the main LaTeX file. For loading figures
%% from other directories you can use the `import` package
%%   \usepackage{import}
%%
%% and then include the figures with
%%   \import{<path to file>}{<filename>.pgf}
%%
%% Matplotlib used the following preamble
%%   \usepackage{fontspec}
%%   \setmainfont{DejaVuSerif.ttf}[Path=/home/vhp1/.local/lib/python3.8/site-packages/matplotlib/mpl-data/fonts/ttf/]
%%   \setsansfont{DejaVuSans.ttf}[Path=/home/vhp1/.local/lib/python3.8/site-packages/matplotlib/mpl-data/fonts/ttf/]
%%   \setmonofont{DejaVuSansMono.ttf}[Path=/home/vhp1/.local/lib/python3.8/site-packages/matplotlib/mpl-data/fonts/ttf/]
%%
\begingroup%
\makeatletter%
\begin{pgfpicture}%
\pgfpathrectangle{\pgfpointorigin}{\pgfqpoint{4.820041in}{1.610194in}}%
\pgfusepath{use as bounding box, clip}%
\begin{pgfscope}%
\pgfsetbuttcap%
\pgfsetmiterjoin%
\pgfsetlinewidth{0.000000pt}%
\definecolor{currentstroke}{rgb}{1.000000,1.000000,1.000000}%
\pgfsetstrokecolor{currentstroke}%
\pgfsetstrokeopacity{0.000000}%
\pgfsetdash{}{0pt}%
\pgfpathmoveto{\pgfqpoint{0.000000in}{0.000000in}}%
\pgfpathlineto{\pgfqpoint{4.820041in}{0.000000in}}%
\pgfpathlineto{\pgfqpoint{4.820041in}{1.610194in}}%
\pgfpathlineto{\pgfqpoint{0.000000in}{1.610194in}}%
\pgfpathclose%
\pgfusepath{}%
\end{pgfscope}%
\begin{pgfscope}%
\pgfsetbuttcap%
\pgfsetmiterjoin%
\definecolor{currentfill}{rgb}{1.000000,1.000000,1.000000}%
\pgfsetfillcolor{currentfill}%
\pgfsetlinewidth{0.000000pt}%
\definecolor{currentstroke}{rgb}{0.000000,0.000000,0.000000}%
\pgfsetstrokecolor{currentstroke}%
\pgfsetstrokeopacity{0.000000}%
\pgfsetdash{}{0pt}%
\pgfpathmoveto{\pgfqpoint{0.537394in}{0.467838in}}%
\pgfpathlineto{\pgfqpoint{4.632078in}{0.467838in}}%
\pgfpathlineto{\pgfqpoint{4.632078in}{1.465092in}}%
\pgfpathlineto{\pgfqpoint{0.537394in}{1.465092in}}%
\pgfpathclose%
\pgfusepath{fill}%
\end{pgfscope}%
\begin{pgfscope}%
\pgfsetbuttcap%
\pgfsetroundjoin%
\definecolor{currentfill}{rgb}{0.000000,0.000000,0.000000}%
\pgfsetfillcolor{currentfill}%
\pgfsetlinewidth{0.803000pt}%
\definecolor{currentstroke}{rgb}{0.000000,0.000000,0.000000}%
\pgfsetstrokecolor{currentstroke}%
\pgfsetdash{}{0pt}%
\pgfsys@defobject{currentmarker}{\pgfqpoint{0.000000in}{-0.048611in}}{\pgfqpoint{0.000000in}{0.000000in}}{%
\pgfpathmoveto{\pgfqpoint{0.000000in}{0.000000in}}%
\pgfpathlineto{\pgfqpoint{0.000000in}{-0.048611in}}%
\pgfusepath{stroke,fill}%
}%
\begin{pgfscope}%
\pgfsys@transformshift{0.537394in}{0.467838in}%
\pgfsys@useobject{currentmarker}{}%
\end{pgfscope}%
\end{pgfscope}%
\begin{pgfscope}%
\definecolor{textcolor}{rgb}{0.000000,0.000000,0.000000}%
\pgfsetstrokecolor{textcolor}%
\pgfsetfillcolor{textcolor}%
\pgftext[x=0.537394in,y=0.370616in,,top]{\color{textcolor}\sffamily\fontsize{8.000000}{9.600000}\selectfont \(\displaystyle {10^{-3}}\)}%
\end{pgfscope}%
\begin{pgfscope}%
\pgfsetbuttcap%
\pgfsetroundjoin%
\definecolor{currentfill}{rgb}{0.000000,0.000000,0.000000}%
\pgfsetfillcolor{currentfill}%
\pgfsetlinewidth{0.803000pt}%
\definecolor{currentstroke}{rgb}{0.000000,0.000000,0.000000}%
\pgfsetstrokecolor{currentstroke}%
\pgfsetdash{}{0pt}%
\pgfsys@defobject{currentmarker}{\pgfqpoint{0.000000in}{-0.048611in}}{\pgfqpoint{0.000000in}{0.000000in}}{%
\pgfpathmoveto{\pgfqpoint{0.000000in}{0.000000in}}%
\pgfpathlineto{\pgfqpoint{0.000000in}{-0.048611in}}%
\pgfusepath{stroke,fill}%
}%
\begin{pgfscope}%
\pgfsys@transformshift{1.356331in}{0.467838in}%
\pgfsys@useobject{currentmarker}{}%
\end{pgfscope}%
\end{pgfscope}%
\begin{pgfscope}%
\definecolor{textcolor}{rgb}{0.000000,0.000000,0.000000}%
\pgfsetstrokecolor{textcolor}%
\pgfsetfillcolor{textcolor}%
\pgftext[x=1.356331in,y=0.370616in,,top]{\color{textcolor}\sffamily\fontsize{8.000000}{9.600000}\selectfont \(\displaystyle {10^{-2}}\)}%
\end{pgfscope}%
\begin{pgfscope}%
\pgfsetbuttcap%
\pgfsetroundjoin%
\definecolor{currentfill}{rgb}{0.000000,0.000000,0.000000}%
\pgfsetfillcolor{currentfill}%
\pgfsetlinewidth{0.803000pt}%
\definecolor{currentstroke}{rgb}{0.000000,0.000000,0.000000}%
\pgfsetstrokecolor{currentstroke}%
\pgfsetdash{}{0pt}%
\pgfsys@defobject{currentmarker}{\pgfqpoint{0.000000in}{-0.048611in}}{\pgfqpoint{0.000000in}{0.000000in}}{%
\pgfpathmoveto{\pgfqpoint{0.000000in}{0.000000in}}%
\pgfpathlineto{\pgfqpoint{0.000000in}{-0.048611in}}%
\pgfusepath{stroke,fill}%
}%
\begin{pgfscope}%
\pgfsys@transformshift{2.175268in}{0.467838in}%
\pgfsys@useobject{currentmarker}{}%
\end{pgfscope}%
\end{pgfscope}%
\begin{pgfscope}%
\definecolor{textcolor}{rgb}{0.000000,0.000000,0.000000}%
\pgfsetstrokecolor{textcolor}%
\pgfsetfillcolor{textcolor}%
\pgftext[x=2.175268in,y=0.370616in,,top]{\color{textcolor}\sffamily\fontsize{8.000000}{9.600000}\selectfont \(\displaystyle {10^{-1}}\)}%
\end{pgfscope}%
\begin{pgfscope}%
\pgfsetbuttcap%
\pgfsetroundjoin%
\definecolor{currentfill}{rgb}{0.000000,0.000000,0.000000}%
\pgfsetfillcolor{currentfill}%
\pgfsetlinewidth{0.803000pt}%
\definecolor{currentstroke}{rgb}{0.000000,0.000000,0.000000}%
\pgfsetstrokecolor{currentstroke}%
\pgfsetdash{}{0pt}%
\pgfsys@defobject{currentmarker}{\pgfqpoint{0.000000in}{-0.048611in}}{\pgfqpoint{0.000000in}{0.000000in}}{%
\pgfpathmoveto{\pgfqpoint{0.000000in}{0.000000in}}%
\pgfpathlineto{\pgfqpoint{0.000000in}{-0.048611in}}%
\pgfusepath{stroke,fill}%
}%
\begin{pgfscope}%
\pgfsys@transformshift{2.994204in}{0.467838in}%
\pgfsys@useobject{currentmarker}{}%
\end{pgfscope}%
\end{pgfscope}%
\begin{pgfscope}%
\definecolor{textcolor}{rgb}{0.000000,0.000000,0.000000}%
\pgfsetstrokecolor{textcolor}%
\pgfsetfillcolor{textcolor}%
\pgftext[x=2.994204in,y=0.370616in,,top]{\color{textcolor}\sffamily\fontsize{8.000000}{9.600000}\selectfont \(\displaystyle {10^{0}}\)}%
\end{pgfscope}%
\begin{pgfscope}%
\pgfsetbuttcap%
\pgfsetroundjoin%
\definecolor{currentfill}{rgb}{0.000000,0.000000,0.000000}%
\pgfsetfillcolor{currentfill}%
\pgfsetlinewidth{0.803000pt}%
\definecolor{currentstroke}{rgb}{0.000000,0.000000,0.000000}%
\pgfsetstrokecolor{currentstroke}%
\pgfsetdash{}{0pt}%
\pgfsys@defobject{currentmarker}{\pgfqpoint{0.000000in}{-0.048611in}}{\pgfqpoint{0.000000in}{0.000000in}}{%
\pgfpathmoveto{\pgfqpoint{0.000000in}{0.000000in}}%
\pgfpathlineto{\pgfqpoint{0.000000in}{-0.048611in}}%
\pgfusepath{stroke,fill}%
}%
\begin{pgfscope}%
\pgfsys@transformshift{3.813141in}{0.467838in}%
\pgfsys@useobject{currentmarker}{}%
\end{pgfscope}%
\end{pgfscope}%
\begin{pgfscope}%
\definecolor{textcolor}{rgb}{0.000000,0.000000,0.000000}%
\pgfsetstrokecolor{textcolor}%
\pgfsetfillcolor{textcolor}%
\pgftext[x=3.813141in,y=0.370616in,,top]{\color{textcolor}\sffamily\fontsize{8.000000}{9.600000}\selectfont \(\displaystyle {10^{1}}\)}%
\end{pgfscope}%
\begin{pgfscope}%
\pgfsetbuttcap%
\pgfsetroundjoin%
\definecolor{currentfill}{rgb}{0.000000,0.000000,0.000000}%
\pgfsetfillcolor{currentfill}%
\pgfsetlinewidth{0.803000pt}%
\definecolor{currentstroke}{rgb}{0.000000,0.000000,0.000000}%
\pgfsetstrokecolor{currentstroke}%
\pgfsetdash{}{0pt}%
\pgfsys@defobject{currentmarker}{\pgfqpoint{0.000000in}{-0.048611in}}{\pgfqpoint{0.000000in}{0.000000in}}{%
\pgfpathmoveto{\pgfqpoint{0.000000in}{0.000000in}}%
\pgfpathlineto{\pgfqpoint{0.000000in}{-0.048611in}}%
\pgfusepath{stroke,fill}%
}%
\begin{pgfscope}%
\pgfsys@transformshift{4.632078in}{0.467838in}%
\pgfsys@useobject{currentmarker}{}%
\end{pgfscope}%
\end{pgfscope}%
\begin{pgfscope}%
\definecolor{textcolor}{rgb}{0.000000,0.000000,0.000000}%
\pgfsetstrokecolor{textcolor}%
\pgfsetfillcolor{textcolor}%
\pgftext[x=4.632078in,y=0.370616in,,top]{\color{textcolor}\sffamily\fontsize{8.000000}{9.600000}\selectfont \(\displaystyle {10^{2}}\)}%
\end{pgfscope}%
\begin{pgfscope}%
\pgfsetbuttcap%
\pgfsetroundjoin%
\definecolor{currentfill}{rgb}{0.000000,0.000000,0.000000}%
\pgfsetfillcolor{currentfill}%
\pgfsetlinewidth{0.602250pt}%
\definecolor{currentstroke}{rgb}{0.000000,0.000000,0.000000}%
\pgfsetstrokecolor{currentstroke}%
\pgfsetdash{}{0pt}%
\pgfsys@defobject{currentmarker}{\pgfqpoint{0.000000in}{-0.027778in}}{\pgfqpoint{0.000000in}{0.000000in}}{%
\pgfpathmoveto{\pgfqpoint{0.000000in}{0.000000in}}%
\pgfpathlineto{\pgfqpoint{0.000000in}{-0.027778in}}%
\pgfusepath{stroke,fill}%
}%
\begin{pgfscope}%
\pgfsys@transformshift{0.783918in}{0.467838in}%
\pgfsys@useobject{currentmarker}{}%
\end{pgfscope}%
\end{pgfscope}%
\begin{pgfscope}%
\pgfsetbuttcap%
\pgfsetroundjoin%
\definecolor{currentfill}{rgb}{0.000000,0.000000,0.000000}%
\pgfsetfillcolor{currentfill}%
\pgfsetlinewidth{0.602250pt}%
\definecolor{currentstroke}{rgb}{0.000000,0.000000,0.000000}%
\pgfsetstrokecolor{currentstroke}%
\pgfsetdash{}{0pt}%
\pgfsys@defobject{currentmarker}{\pgfqpoint{0.000000in}{-0.027778in}}{\pgfqpoint{0.000000in}{0.000000in}}{%
\pgfpathmoveto{\pgfqpoint{0.000000in}{0.000000in}}%
\pgfpathlineto{\pgfqpoint{0.000000in}{-0.027778in}}%
\pgfusepath{stroke,fill}%
}%
\begin{pgfscope}%
\pgfsys@transformshift{0.928126in}{0.467838in}%
\pgfsys@useobject{currentmarker}{}%
\end{pgfscope}%
\end{pgfscope}%
\begin{pgfscope}%
\pgfsetbuttcap%
\pgfsetroundjoin%
\definecolor{currentfill}{rgb}{0.000000,0.000000,0.000000}%
\pgfsetfillcolor{currentfill}%
\pgfsetlinewidth{0.602250pt}%
\definecolor{currentstroke}{rgb}{0.000000,0.000000,0.000000}%
\pgfsetstrokecolor{currentstroke}%
\pgfsetdash{}{0pt}%
\pgfsys@defobject{currentmarker}{\pgfqpoint{0.000000in}{-0.027778in}}{\pgfqpoint{0.000000in}{0.000000in}}{%
\pgfpathmoveto{\pgfqpoint{0.000000in}{0.000000in}}%
\pgfpathlineto{\pgfqpoint{0.000000in}{-0.027778in}}%
\pgfusepath{stroke,fill}%
}%
\begin{pgfscope}%
\pgfsys@transformshift{1.030443in}{0.467838in}%
\pgfsys@useobject{currentmarker}{}%
\end{pgfscope}%
\end{pgfscope}%
\begin{pgfscope}%
\pgfsetbuttcap%
\pgfsetroundjoin%
\definecolor{currentfill}{rgb}{0.000000,0.000000,0.000000}%
\pgfsetfillcolor{currentfill}%
\pgfsetlinewidth{0.602250pt}%
\definecolor{currentstroke}{rgb}{0.000000,0.000000,0.000000}%
\pgfsetstrokecolor{currentstroke}%
\pgfsetdash{}{0pt}%
\pgfsys@defobject{currentmarker}{\pgfqpoint{0.000000in}{-0.027778in}}{\pgfqpoint{0.000000in}{0.000000in}}{%
\pgfpathmoveto{\pgfqpoint{0.000000in}{0.000000in}}%
\pgfpathlineto{\pgfqpoint{0.000000in}{-0.027778in}}%
\pgfusepath{stroke,fill}%
}%
\begin{pgfscope}%
\pgfsys@transformshift{1.109806in}{0.467838in}%
\pgfsys@useobject{currentmarker}{}%
\end{pgfscope}%
\end{pgfscope}%
\begin{pgfscope}%
\pgfsetbuttcap%
\pgfsetroundjoin%
\definecolor{currentfill}{rgb}{0.000000,0.000000,0.000000}%
\pgfsetfillcolor{currentfill}%
\pgfsetlinewidth{0.602250pt}%
\definecolor{currentstroke}{rgb}{0.000000,0.000000,0.000000}%
\pgfsetstrokecolor{currentstroke}%
\pgfsetdash{}{0pt}%
\pgfsys@defobject{currentmarker}{\pgfqpoint{0.000000in}{-0.027778in}}{\pgfqpoint{0.000000in}{0.000000in}}{%
\pgfpathmoveto{\pgfqpoint{0.000000in}{0.000000in}}%
\pgfpathlineto{\pgfqpoint{0.000000in}{-0.027778in}}%
\pgfusepath{stroke,fill}%
}%
\begin{pgfscope}%
\pgfsys@transformshift{1.174651in}{0.467838in}%
\pgfsys@useobject{currentmarker}{}%
\end{pgfscope}%
\end{pgfscope}%
\begin{pgfscope}%
\pgfsetbuttcap%
\pgfsetroundjoin%
\definecolor{currentfill}{rgb}{0.000000,0.000000,0.000000}%
\pgfsetfillcolor{currentfill}%
\pgfsetlinewidth{0.602250pt}%
\definecolor{currentstroke}{rgb}{0.000000,0.000000,0.000000}%
\pgfsetstrokecolor{currentstroke}%
\pgfsetdash{}{0pt}%
\pgfsys@defobject{currentmarker}{\pgfqpoint{0.000000in}{-0.027778in}}{\pgfqpoint{0.000000in}{0.000000in}}{%
\pgfpathmoveto{\pgfqpoint{0.000000in}{0.000000in}}%
\pgfpathlineto{\pgfqpoint{0.000000in}{-0.027778in}}%
\pgfusepath{stroke,fill}%
}%
\begin{pgfscope}%
\pgfsys@transformshift{1.229476in}{0.467838in}%
\pgfsys@useobject{currentmarker}{}%
\end{pgfscope}%
\end{pgfscope}%
\begin{pgfscope}%
\pgfsetbuttcap%
\pgfsetroundjoin%
\definecolor{currentfill}{rgb}{0.000000,0.000000,0.000000}%
\pgfsetfillcolor{currentfill}%
\pgfsetlinewidth{0.602250pt}%
\definecolor{currentstroke}{rgb}{0.000000,0.000000,0.000000}%
\pgfsetstrokecolor{currentstroke}%
\pgfsetdash{}{0pt}%
\pgfsys@defobject{currentmarker}{\pgfqpoint{0.000000in}{-0.027778in}}{\pgfqpoint{0.000000in}{0.000000in}}{%
\pgfpathmoveto{\pgfqpoint{0.000000in}{0.000000in}}%
\pgfpathlineto{\pgfqpoint{0.000000in}{-0.027778in}}%
\pgfusepath{stroke,fill}%
}%
\begin{pgfscope}%
\pgfsys@transformshift{1.276968in}{0.467838in}%
\pgfsys@useobject{currentmarker}{}%
\end{pgfscope}%
\end{pgfscope}%
\begin{pgfscope}%
\pgfsetbuttcap%
\pgfsetroundjoin%
\definecolor{currentfill}{rgb}{0.000000,0.000000,0.000000}%
\pgfsetfillcolor{currentfill}%
\pgfsetlinewidth{0.602250pt}%
\definecolor{currentstroke}{rgb}{0.000000,0.000000,0.000000}%
\pgfsetstrokecolor{currentstroke}%
\pgfsetdash{}{0pt}%
\pgfsys@defobject{currentmarker}{\pgfqpoint{0.000000in}{-0.027778in}}{\pgfqpoint{0.000000in}{0.000000in}}{%
\pgfpathmoveto{\pgfqpoint{0.000000in}{0.000000in}}%
\pgfpathlineto{\pgfqpoint{0.000000in}{-0.027778in}}%
\pgfusepath{stroke,fill}%
}%
\begin{pgfscope}%
\pgfsys@transformshift{1.318858in}{0.467838in}%
\pgfsys@useobject{currentmarker}{}%
\end{pgfscope}%
\end{pgfscope}%
\begin{pgfscope}%
\pgfsetbuttcap%
\pgfsetroundjoin%
\definecolor{currentfill}{rgb}{0.000000,0.000000,0.000000}%
\pgfsetfillcolor{currentfill}%
\pgfsetlinewidth{0.602250pt}%
\definecolor{currentstroke}{rgb}{0.000000,0.000000,0.000000}%
\pgfsetstrokecolor{currentstroke}%
\pgfsetdash{}{0pt}%
\pgfsys@defobject{currentmarker}{\pgfqpoint{0.000000in}{-0.027778in}}{\pgfqpoint{0.000000in}{0.000000in}}{%
\pgfpathmoveto{\pgfqpoint{0.000000in}{0.000000in}}%
\pgfpathlineto{\pgfqpoint{0.000000in}{-0.027778in}}%
\pgfusepath{stroke,fill}%
}%
\begin{pgfscope}%
\pgfsys@transformshift{1.602855in}{0.467838in}%
\pgfsys@useobject{currentmarker}{}%
\end{pgfscope}%
\end{pgfscope}%
\begin{pgfscope}%
\pgfsetbuttcap%
\pgfsetroundjoin%
\definecolor{currentfill}{rgb}{0.000000,0.000000,0.000000}%
\pgfsetfillcolor{currentfill}%
\pgfsetlinewidth{0.602250pt}%
\definecolor{currentstroke}{rgb}{0.000000,0.000000,0.000000}%
\pgfsetstrokecolor{currentstroke}%
\pgfsetdash{}{0pt}%
\pgfsys@defobject{currentmarker}{\pgfqpoint{0.000000in}{-0.027778in}}{\pgfqpoint{0.000000in}{0.000000in}}{%
\pgfpathmoveto{\pgfqpoint{0.000000in}{0.000000in}}%
\pgfpathlineto{\pgfqpoint{0.000000in}{-0.027778in}}%
\pgfusepath{stroke,fill}%
}%
\begin{pgfscope}%
\pgfsys@transformshift{1.747063in}{0.467838in}%
\pgfsys@useobject{currentmarker}{}%
\end{pgfscope}%
\end{pgfscope}%
\begin{pgfscope}%
\pgfsetbuttcap%
\pgfsetroundjoin%
\definecolor{currentfill}{rgb}{0.000000,0.000000,0.000000}%
\pgfsetfillcolor{currentfill}%
\pgfsetlinewidth{0.602250pt}%
\definecolor{currentstroke}{rgb}{0.000000,0.000000,0.000000}%
\pgfsetstrokecolor{currentstroke}%
\pgfsetdash{}{0pt}%
\pgfsys@defobject{currentmarker}{\pgfqpoint{0.000000in}{-0.027778in}}{\pgfqpoint{0.000000in}{0.000000in}}{%
\pgfpathmoveto{\pgfqpoint{0.000000in}{0.000000in}}%
\pgfpathlineto{\pgfqpoint{0.000000in}{-0.027778in}}%
\pgfusepath{stroke,fill}%
}%
\begin{pgfscope}%
\pgfsys@transformshift{1.849380in}{0.467838in}%
\pgfsys@useobject{currentmarker}{}%
\end{pgfscope}%
\end{pgfscope}%
\begin{pgfscope}%
\pgfsetbuttcap%
\pgfsetroundjoin%
\definecolor{currentfill}{rgb}{0.000000,0.000000,0.000000}%
\pgfsetfillcolor{currentfill}%
\pgfsetlinewidth{0.602250pt}%
\definecolor{currentstroke}{rgb}{0.000000,0.000000,0.000000}%
\pgfsetstrokecolor{currentstroke}%
\pgfsetdash{}{0pt}%
\pgfsys@defobject{currentmarker}{\pgfqpoint{0.000000in}{-0.027778in}}{\pgfqpoint{0.000000in}{0.000000in}}{%
\pgfpathmoveto{\pgfqpoint{0.000000in}{0.000000in}}%
\pgfpathlineto{\pgfqpoint{0.000000in}{-0.027778in}}%
\pgfusepath{stroke,fill}%
}%
\begin{pgfscope}%
\pgfsys@transformshift{1.928743in}{0.467838in}%
\pgfsys@useobject{currentmarker}{}%
\end{pgfscope}%
\end{pgfscope}%
\begin{pgfscope}%
\pgfsetbuttcap%
\pgfsetroundjoin%
\definecolor{currentfill}{rgb}{0.000000,0.000000,0.000000}%
\pgfsetfillcolor{currentfill}%
\pgfsetlinewidth{0.602250pt}%
\definecolor{currentstroke}{rgb}{0.000000,0.000000,0.000000}%
\pgfsetstrokecolor{currentstroke}%
\pgfsetdash{}{0pt}%
\pgfsys@defobject{currentmarker}{\pgfqpoint{0.000000in}{-0.027778in}}{\pgfqpoint{0.000000in}{0.000000in}}{%
\pgfpathmoveto{\pgfqpoint{0.000000in}{0.000000in}}%
\pgfpathlineto{\pgfqpoint{0.000000in}{-0.027778in}}%
\pgfusepath{stroke,fill}%
}%
\begin{pgfscope}%
\pgfsys@transformshift{1.993587in}{0.467838in}%
\pgfsys@useobject{currentmarker}{}%
\end{pgfscope}%
\end{pgfscope}%
\begin{pgfscope}%
\pgfsetbuttcap%
\pgfsetroundjoin%
\definecolor{currentfill}{rgb}{0.000000,0.000000,0.000000}%
\pgfsetfillcolor{currentfill}%
\pgfsetlinewidth{0.602250pt}%
\definecolor{currentstroke}{rgb}{0.000000,0.000000,0.000000}%
\pgfsetstrokecolor{currentstroke}%
\pgfsetdash{}{0pt}%
\pgfsys@defobject{currentmarker}{\pgfqpoint{0.000000in}{-0.027778in}}{\pgfqpoint{0.000000in}{0.000000in}}{%
\pgfpathmoveto{\pgfqpoint{0.000000in}{0.000000in}}%
\pgfpathlineto{\pgfqpoint{0.000000in}{-0.027778in}}%
\pgfusepath{stroke,fill}%
}%
\begin{pgfscope}%
\pgfsys@transformshift{2.048413in}{0.467838in}%
\pgfsys@useobject{currentmarker}{}%
\end{pgfscope}%
\end{pgfscope}%
\begin{pgfscope}%
\pgfsetbuttcap%
\pgfsetroundjoin%
\definecolor{currentfill}{rgb}{0.000000,0.000000,0.000000}%
\pgfsetfillcolor{currentfill}%
\pgfsetlinewidth{0.602250pt}%
\definecolor{currentstroke}{rgb}{0.000000,0.000000,0.000000}%
\pgfsetstrokecolor{currentstroke}%
\pgfsetdash{}{0pt}%
\pgfsys@defobject{currentmarker}{\pgfqpoint{0.000000in}{-0.027778in}}{\pgfqpoint{0.000000in}{0.000000in}}{%
\pgfpathmoveto{\pgfqpoint{0.000000in}{0.000000in}}%
\pgfpathlineto{\pgfqpoint{0.000000in}{-0.027778in}}%
\pgfusepath{stroke,fill}%
}%
\begin{pgfscope}%
\pgfsys@transformshift{2.095904in}{0.467838in}%
\pgfsys@useobject{currentmarker}{}%
\end{pgfscope}%
\end{pgfscope}%
\begin{pgfscope}%
\pgfsetbuttcap%
\pgfsetroundjoin%
\definecolor{currentfill}{rgb}{0.000000,0.000000,0.000000}%
\pgfsetfillcolor{currentfill}%
\pgfsetlinewidth{0.602250pt}%
\definecolor{currentstroke}{rgb}{0.000000,0.000000,0.000000}%
\pgfsetstrokecolor{currentstroke}%
\pgfsetdash{}{0pt}%
\pgfsys@defobject{currentmarker}{\pgfqpoint{0.000000in}{-0.027778in}}{\pgfqpoint{0.000000in}{0.000000in}}{%
\pgfpathmoveto{\pgfqpoint{0.000000in}{0.000000in}}%
\pgfpathlineto{\pgfqpoint{0.000000in}{-0.027778in}}%
\pgfusepath{stroke,fill}%
}%
\begin{pgfscope}%
\pgfsys@transformshift{2.137795in}{0.467838in}%
\pgfsys@useobject{currentmarker}{}%
\end{pgfscope}%
\end{pgfscope}%
\begin{pgfscope}%
\pgfsetbuttcap%
\pgfsetroundjoin%
\definecolor{currentfill}{rgb}{0.000000,0.000000,0.000000}%
\pgfsetfillcolor{currentfill}%
\pgfsetlinewidth{0.602250pt}%
\definecolor{currentstroke}{rgb}{0.000000,0.000000,0.000000}%
\pgfsetstrokecolor{currentstroke}%
\pgfsetdash{}{0pt}%
\pgfsys@defobject{currentmarker}{\pgfqpoint{0.000000in}{-0.027778in}}{\pgfqpoint{0.000000in}{0.000000in}}{%
\pgfpathmoveto{\pgfqpoint{0.000000in}{0.000000in}}%
\pgfpathlineto{\pgfqpoint{0.000000in}{-0.027778in}}%
\pgfusepath{stroke,fill}%
}%
\begin{pgfscope}%
\pgfsys@transformshift{2.421792in}{0.467838in}%
\pgfsys@useobject{currentmarker}{}%
\end{pgfscope}%
\end{pgfscope}%
\begin{pgfscope}%
\pgfsetbuttcap%
\pgfsetroundjoin%
\definecolor{currentfill}{rgb}{0.000000,0.000000,0.000000}%
\pgfsetfillcolor{currentfill}%
\pgfsetlinewidth{0.602250pt}%
\definecolor{currentstroke}{rgb}{0.000000,0.000000,0.000000}%
\pgfsetstrokecolor{currentstroke}%
\pgfsetdash{}{0pt}%
\pgfsys@defobject{currentmarker}{\pgfqpoint{0.000000in}{-0.027778in}}{\pgfqpoint{0.000000in}{0.000000in}}{%
\pgfpathmoveto{\pgfqpoint{0.000000in}{0.000000in}}%
\pgfpathlineto{\pgfqpoint{0.000000in}{-0.027778in}}%
\pgfusepath{stroke,fill}%
}%
\begin{pgfscope}%
\pgfsys@transformshift{2.566000in}{0.467838in}%
\pgfsys@useobject{currentmarker}{}%
\end{pgfscope}%
\end{pgfscope}%
\begin{pgfscope}%
\pgfsetbuttcap%
\pgfsetroundjoin%
\definecolor{currentfill}{rgb}{0.000000,0.000000,0.000000}%
\pgfsetfillcolor{currentfill}%
\pgfsetlinewidth{0.602250pt}%
\definecolor{currentstroke}{rgb}{0.000000,0.000000,0.000000}%
\pgfsetstrokecolor{currentstroke}%
\pgfsetdash{}{0pt}%
\pgfsys@defobject{currentmarker}{\pgfqpoint{0.000000in}{-0.027778in}}{\pgfqpoint{0.000000in}{0.000000in}}{%
\pgfpathmoveto{\pgfqpoint{0.000000in}{0.000000in}}%
\pgfpathlineto{\pgfqpoint{0.000000in}{-0.027778in}}%
\pgfusepath{stroke,fill}%
}%
\begin{pgfscope}%
\pgfsys@transformshift{2.668317in}{0.467838in}%
\pgfsys@useobject{currentmarker}{}%
\end{pgfscope}%
\end{pgfscope}%
\begin{pgfscope}%
\pgfsetbuttcap%
\pgfsetroundjoin%
\definecolor{currentfill}{rgb}{0.000000,0.000000,0.000000}%
\pgfsetfillcolor{currentfill}%
\pgfsetlinewidth{0.602250pt}%
\definecolor{currentstroke}{rgb}{0.000000,0.000000,0.000000}%
\pgfsetstrokecolor{currentstroke}%
\pgfsetdash{}{0pt}%
\pgfsys@defobject{currentmarker}{\pgfqpoint{0.000000in}{-0.027778in}}{\pgfqpoint{0.000000in}{0.000000in}}{%
\pgfpathmoveto{\pgfqpoint{0.000000in}{0.000000in}}%
\pgfpathlineto{\pgfqpoint{0.000000in}{-0.027778in}}%
\pgfusepath{stroke,fill}%
}%
\begin{pgfscope}%
\pgfsys@transformshift{2.747680in}{0.467838in}%
\pgfsys@useobject{currentmarker}{}%
\end{pgfscope}%
\end{pgfscope}%
\begin{pgfscope}%
\pgfsetbuttcap%
\pgfsetroundjoin%
\definecolor{currentfill}{rgb}{0.000000,0.000000,0.000000}%
\pgfsetfillcolor{currentfill}%
\pgfsetlinewidth{0.602250pt}%
\definecolor{currentstroke}{rgb}{0.000000,0.000000,0.000000}%
\pgfsetstrokecolor{currentstroke}%
\pgfsetdash{}{0pt}%
\pgfsys@defobject{currentmarker}{\pgfqpoint{0.000000in}{-0.027778in}}{\pgfqpoint{0.000000in}{0.000000in}}{%
\pgfpathmoveto{\pgfqpoint{0.000000in}{0.000000in}}%
\pgfpathlineto{\pgfqpoint{0.000000in}{-0.027778in}}%
\pgfusepath{stroke,fill}%
}%
\begin{pgfscope}%
\pgfsys@transformshift{2.812524in}{0.467838in}%
\pgfsys@useobject{currentmarker}{}%
\end{pgfscope}%
\end{pgfscope}%
\begin{pgfscope}%
\pgfsetbuttcap%
\pgfsetroundjoin%
\definecolor{currentfill}{rgb}{0.000000,0.000000,0.000000}%
\pgfsetfillcolor{currentfill}%
\pgfsetlinewidth{0.602250pt}%
\definecolor{currentstroke}{rgb}{0.000000,0.000000,0.000000}%
\pgfsetstrokecolor{currentstroke}%
\pgfsetdash{}{0pt}%
\pgfsys@defobject{currentmarker}{\pgfqpoint{0.000000in}{-0.027778in}}{\pgfqpoint{0.000000in}{0.000000in}}{%
\pgfpathmoveto{\pgfqpoint{0.000000in}{0.000000in}}%
\pgfpathlineto{\pgfqpoint{0.000000in}{-0.027778in}}%
\pgfusepath{stroke,fill}%
}%
\begin{pgfscope}%
\pgfsys@transformshift{2.867349in}{0.467838in}%
\pgfsys@useobject{currentmarker}{}%
\end{pgfscope}%
\end{pgfscope}%
\begin{pgfscope}%
\pgfsetbuttcap%
\pgfsetroundjoin%
\definecolor{currentfill}{rgb}{0.000000,0.000000,0.000000}%
\pgfsetfillcolor{currentfill}%
\pgfsetlinewidth{0.602250pt}%
\definecolor{currentstroke}{rgb}{0.000000,0.000000,0.000000}%
\pgfsetstrokecolor{currentstroke}%
\pgfsetdash{}{0pt}%
\pgfsys@defobject{currentmarker}{\pgfqpoint{0.000000in}{-0.027778in}}{\pgfqpoint{0.000000in}{0.000000in}}{%
\pgfpathmoveto{\pgfqpoint{0.000000in}{0.000000in}}%
\pgfpathlineto{\pgfqpoint{0.000000in}{-0.027778in}}%
\pgfusepath{stroke,fill}%
}%
\begin{pgfscope}%
\pgfsys@transformshift{2.914841in}{0.467838in}%
\pgfsys@useobject{currentmarker}{}%
\end{pgfscope}%
\end{pgfscope}%
\begin{pgfscope}%
\pgfsetbuttcap%
\pgfsetroundjoin%
\definecolor{currentfill}{rgb}{0.000000,0.000000,0.000000}%
\pgfsetfillcolor{currentfill}%
\pgfsetlinewidth{0.602250pt}%
\definecolor{currentstroke}{rgb}{0.000000,0.000000,0.000000}%
\pgfsetstrokecolor{currentstroke}%
\pgfsetdash{}{0pt}%
\pgfsys@defobject{currentmarker}{\pgfqpoint{0.000000in}{-0.027778in}}{\pgfqpoint{0.000000in}{0.000000in}}{%
\pgfpathmoveto{\pgfqpoint{0.000000in}{0.000000in}}%
\pgfpathlineto{\pgfqpoint{0.000000in}{-0.027778in}}%
\pgfusepath{stroke,fill}%
}%
\begin{pgfscope}%
\pgfsys@transformshift{2.956732in}{0.467838in}%
\pgfsys@useobject{currentmarker}{}%
\end{pgfscope}%
\end{pgfscope}%
\begin{pgfscope}%
\pgfsetbuttcap%
\pgfsetroundjoin%
\definecolor{currentfill}{rgb}{0.000000,0.000000,0.000000}%
\pgfsetfillcolor{currentfill}%
\pgfsetlinewidth{0.602250pt}%
\definecolor{currentstroke}{rgb}{0.000000,0.000000,0.000000}%
\pgfsetstrokecolor{currentstroke}%
\pgfsetdash{}{0pt}%
\pgfsys@defobject{currentmarker}{\pgfqpoint{0.000000in}{-0.027778in}}{\pgfqpoint{0.000000in}{0.000000in}}{%
\pgfpathmoveto{\pgfqpoint{0.000000in}{0.000000in}}%
\pgfpathlineto{\pgfqpoint{0.000000in}{-0.027778in}}%
\pgfusepath{stroke,fill}%
}%
\begin{pgfscope}%
\pgfsys@transformshift{3.240729in}{0.467838in}%
\pgfsys@useobject{currentmarker}{}%
\end{pgfscope}%
\end{pgfscope}%
\begin{pgfscope}%
\pgfsetbuttcap%
\pgfsetroundjoin%
\definecolor{currentfill}{rgb}{0.000000,0.000000,0.000000}%
\pgfsetfillcolor{currentfill}%
\pgfsetlinewidth{0.602250pt}%
\definecolor{currentstroke}{rgb}{0.000000,0.000000,0.000000}%
\pgfsetstrokecolor{currentstroke}%
\pgfsetdash{}{0pt}%
\pgfsys@defobject{currentmarker}{\pgfqpoint{0.000000in}{-0.027778in}}{\pgfqpoint{0.000000in}{0.000000in}}{%
\pgfpathmoveto{\pgfqpoint{0.000000in}{0.000000in}}%
\pgfpathlineto{\pgfqpoint{0.000000in}{-0.027778in}}%
\pgfusepath{stroke,fill}%
}%
\begin{pgfscope}%
\pgfsys@transformshift{3.384936in}{0.467838in}%
\pgfsys@useobject{currentmarker}{}%
\end{pgfscope}%
\end{pgfscope}%
\begin{pgfscope}%
\pgfsetbuttcap%
\pgfsetroundjoin%
\definecolor{currentfill}{rgb}{0.000000,0.000000,0.000000}%
\pgfsetfillcolor{currentfill}%
\pgfsetlinewidth{0.602250pt}%
\definecolor{currentstroke}{rgb}{0.000000,0.000000,0.000000}%
\pgfsetstrokecolor{currentstroke}%
\pgfsetdash{}{0pt}%
\pgfsys@defobject{currentmarker}{\pgfqpoint{0.000000in}{-0.027778in}}{\pgfqpoint{0.000000in}{0.000000in}}{%
\pgfpathmoveto{\pgfqpoint{0.000000in}{0.000000in}}%
\pgfpathlineto{\pgfqpoint{0.000000in}{-0.027778in}}%
\pgfusepath{stroke,fill}%
}%
\begin{pgfscope}%
\pgfsys@transformshift{3.487253in}{0.467838in}%
\pgfsys@useobject{currentmarker}{}%
\end{pgfscope}%
\end{pgfscope}%
\begin{pgfscope}%
\pgfsetbuttcap%
\pgfsetroundjoin%
\definecolor{currentfill}{rgb}{0.000000,0.000000,0.000000}%
\pgfsetfillcolor{currentfill}%
\pgfsetlinewidth{0.602250pt}%
\definecolor{currentstroke}{rgb}{0.000000,0.000000,0.000000}%
\pgfsetstrokecolor{currentstroke}%
\pgfsetdash{}{0pt}%
\pgfsys@defobject{currentmarker}{\pgfqpoint{0.000000in}{-0.027778in}}{\pgfqpoint{0.000000in}{0.000000in}}{%
\pgfpathmoveto{\pgfqpoint{0.000000in}{0.000000in}}%
\pgfpathlineto{\pgfqpoint{0.000000in}{-0.027778in}}%
\pgfusepath{stroke,fill}%
}%
\begin{pgfscope}%
\pgfsys@transformshift{3.566617in}{0.467838in}%
\pgfsys@useobject{currentmarker}{}%
\end{pgfscope}%
\end{pgfscope}%
\begin{pgfscope}%
\pgfsetbuttcap%
\pgfsetroundjoin%
\definecolor{currentfill}{rgb}{0.000000,0.000000,0.000000}%
\pgfsetfillcolor{currentfill}%
\pgfsetlinewidth{0.602250pt}%
\definecolor{currentstroke}{rgb}{0.000000,0.000000,0.000000}%
\pgfsetstrokecolor{currentstroke}%
\pgfsetdash{}{0pt}%
\pgfsys@defobject{currentmarker}{\pgfqpoint{0.000000in}{-0.027778in}}{\pgfqpoint{0.000000in}{0.000000in}}{%
\pgfpathmoveto{\pgfqpoint{0.000000in}{0.000000in}}%
\pgfpathlineto{\pgfqpoint{0.000000in}{-0.027778in}}%
\pgfusepath{stroke,fill}%
}%
\begin{pgfscope}%
\pgfsys@transformshift{3.631461in}{0.467838in}%
\pgfsys@useobject{currentmarker}{}%
\end{pgfscope}%
\end{pgfscope}%
\begin{pgfscope}%
\pgfsetbuttcap%
\pgfsetroundjoin%
\definecolor{currentfill}{rgb}{0.000000,0.000000,0.000000}%
\pgfsetfillcolor{currentfill}%
\pgfsetlinewidth{0.602250pt}%
\definecolor{currentstroke}{rgb}{0.000000,0.000000,0.000000}%
\pgfsetstrokecolor{currentstroke}%
\pgfsetdash{}{0pt}%
\pgfsys@defobject{currentmarker}{\pgfqpoint{0.000000in}{-0.027778in}}{\pgfqpoint{0.000000in}{0.000000in}}{%
\pgfpathmoveto{\pgfqpoint{0.000000in}{0.000000in}}%
\pgfpathlineto{\pgfqpoint{0.000000in}{-0.027778in}}%
\pgfusepath{stroke,fill}%
}%
\begin{pgfscope}%
\pgfsys@transformshift{3.686286in}{0.467838in}%
\pgfsys@useobject{currentmarker}{}%
\end{pgfscope}%
\end{pgfscope}%
\begin{pgfscope}%
\pgfsetbuttcap%
\pgfsetroundjoin%
\definecolor{currentfill}{rgb}{0.000000,0.000000,0.000000}%
\pgfsetfillcolor{currentfill}%
\pgfsetlinewidth{0.602250pt}%
\definecolor{currentstroke}{rgb}{0.000000,0.000000,0.000000}%
\pgfsetstrokecolor{currentstroke}%
\pgfsetdash{}{0pt}%
\pgfsys@defobject{currentmarker}{\pgfqpoint{0.000000in}{-0.027778in}}{\pgfqpoint{0.000000in}{0.000000in}}{%
\pgfpathmoveto{\pgfqpoint{0.000000in}{0.000000in}}%
\pgfpathlineto{\pgfqpoint{0.000000in}{-0.027778in}}%
\pgfusepath{stroke,fill}%
}%
\begin{pgfscope}%
\pgfsys@transformshift{3.733778in}{0.467838in}%
\pgfsys@useobject{currentmarker}{}%
\end{pgfscope}%
\end{pgfscope}%
\begin{pgfscope}%
\pgfsetbuttcap%
\pgfsetroundjoin%
\definecolor{currentfill}{rgb}{0.000000,0.000000,0.000000}%
\pgfsetfillcolor{currentfill}%
\pgfsetlinewidth{0.602250pt}%
\definecolor{currentstroke}{rgb}{0.000000,0.000000,0.000000}%
\pgfsetstrokecolor{currentstroke}%
\pgfsetdash{}{0pt}%
\pgfsys@defobject{currentmarker}{\pgfqpoint{0.000000in}{-0.027778in}}{\pgfqpoint{0.000000in}{0.000000in}}{%
\pgfpathmoveto{\pgfqpoint{0.000000in}{0.000000in}}%
\pgfpathlineto{\pgfqpoint{0.000000in}{-0.027778in}}%
\pgfusepath{stroke,fill}%
}%
\begin{pgfscope}%
\pgfsys@transformshift{3.775669in}{0.467838in}%
\pgfsys@useobject{currentmarker}{}%
\end{pgfscope}%
\end{pgfscope}%
\begin{pgfscope}%
\pgfsetbuttcap%
\pgfsetroundjoin%
\definecolor{currentfill}{rgb}{0.000000,0.000000,0.000000}%
\pgfsetfillcolor{currentfill}%
\pgfsetlinewidth{0.602250pt}%
\definecolor{currentstroke}{rgb}{0.000000,0.000000,0.000000}%
\pgfsetstrokecolor{currentstroke}%
\pgfsetdash{}{0pt}%
\pgfsys@defobject{currentmarker}{\pgfqpoint{0.000000in}{-0.027778in}}{\pgfqpoint{0.000000in}{0.000000in}}{%
\pgfpathmoveto{\pgfqpoint{0.000000in}{0.000000in}}%
\pgfpathlineto{\pgfqpoint{0.000000in}{-0.027778in}}%
\pgfusepath{stroke,fill}%
}%
\begin{pgfscope}%
\pgfsys@transformshift{4.059666in}{0.467838in}%
\pgfsys@useobject{currentmarker}{}%
\end{pgfscope}%
\end{pgfscope}%
\begin{pgfscope}%
\pgfsetbuttcap%
\pgfsetroundjoin%
\definecolor{currentfill}{rgb}{0.000000,0.000000,0.000000}%
\pgfsetfillcolor{currentfill}%
\pgfsetlinewidth{0.602250pt}%
\definecolor{currentstroke}{rgb}{0.000000,0.000000,0.000000}%
\pgfsetstrokecolor{currentstroke}%
\pgfsetdash{}{0pt}%
\pgfsys@defobject{currentmarker}{\pgfqpoint{0.000000in}{-0.027778in}}{\pgfqpoint{0.000000in}{0.000000in}}{%
\pgfpathmoveto{\pgfqpoint{0.000000in}{0.000000in}}%
\pgfpathlineto{\pgfqpoint{0.000000in}{-0.027778in}}%
\pgfusepath{stroke,fill}%
}%
\begin{pgfscope}%
\pgfsys@transformshift{4.203873in}{0.467838in}%
\pgfsys@useobject{currentmarker}{}%
\end{pgfscope}%
\end{pgfscope}%
\begin{pgfscope}%
\pgfsetbuttcap%
\pgfsetroundjoin%
\definecolor{currentfill}{rgb}{0.000000,0.000000,0.000000}%
\pgfsetfillcolor{currentfill}%
\pgfsetlinewidth{0.602250pt}%
\definecolor{currentstroke}{rgb}{0.000000,0.000000,0.000000}%
\pgfsetstrokecolor{currentstroke}%
\pgfsetdash{}{0pt}%
\pgfsys@defobject{currentmarker}{\pgfqpoint{0.000000in}{-0.027778in}}{\pgfqpoint{0.000000in}{0.000000in}}{%
\pgfpathmoveto{\pgfqpoint{0.000000in}{0.000000in}}%
\pgfpathlineto{\pgfqpoint{0.000000in}{-0.027778in}}%
\pgfusepath{stroke,fill}%
}%
\begin{pgfscope}%
\pgfsys@transformshift{4.306190in}{0.467838in}%
\pgfsys@useobject{currentmarker}{}%
\end{pgfscope}%
\end{pgfscope}%
\begin{pgfscope}%
\pgfsetbuttcap%
\pgfsetroundjoin%
\definecolor{currentfill}{rgb}{0.000000,0.000000,0.000000}%
\pgfsetfillcolor{currentfill}%
\pgfsetlinewidth{0.602250pt}%
\definecolor{currentstroke}{rgb}{0.000000,0.000000,0.000000}%
\pgfsetstrokecolor{currentstroke}%
\pgfsetdash{}{0pt}%
\pgfsys@defobject{currentmarker}{\pgfqpoint{0.000000in}{-0.027778in}}{\pgfqpoint{0.000000in}{0.000000in}}{%
\pgfpathmoveto{\pgfqpoint{0.000000in}{0.000000in}}%
\pgfpathlineto{\pgfqpoint{0.000000in}{-0.027778in}}%
\pgfusepath{stroke,fill}%
}%
\begin{pgfscope}%
\pgfsys@transformshift{4.385553in}{0.467838in}%
\pgfsys@useobject{currentmarker}{}%
\end{pgfscope}%
\end{pgfscope}%
\begin{pgfscope}%
\pgfsetbuttcap%
\pgfsetroundjoin%
\definecolor{currentfill}{rgb}{0.000000,0.000000,0.000000}%
\pgfsetfillcolor{currentfill}%
\pgfsetlinewidth{0.602250pt}%
\definecolor{currentstroke}{rgb}{0.000000,0.000000,0.000000}%
\pgfsetstrokecolor{currentstroke}%
\pgfsetdash{}{0pt}%
\pgfsys@defobject{currentmarker}{\pgfqpoint{0.000000in}{-0.027778in}}{\pgfqpoint{0.000000in}{0.000000in}}{%
\pgfpathmoveto{\pgfqpoint{0.000000in}{0.000000in}}%
\pgfpathlineto{\pgfqpoint{0.000000in}{-0.027778in}}%
\pgfusepath{stroke,fill}%
}%
\begin{pgfscope}%
\pgfsys@transformshift{4.450398in}{0.467838in}%
\pgfsys@useobject{currentmarker}{}%
\end{pgfscope}%
\end{pgfscope}%
\begin{pgfscope}%
\pgfsetbuttcap%
\pgfsetroundjoin%
\definecolor{currentfill}{rgb}{0.000000,0.000000,0.000000}%
\pgfsetfillcolor{currentfill}%
\pgfsetlinewidth{0.602250pt}%
\definecolor{currentstroke}{rgb}{0.000000,0.000000,0.000000}%
\pgfsetstrokecolor{currentstroke}%
\pgfsetdash{}{0pt}%
\pgfsys@defobject{currentmarker}{\pgfqpoint{0.000000in}{-0.027778in}}{\pgfqpoint{0.000000in}{0.000000in}}{%
\pgfpathmoveto{\pgfqpoint{0.000000in}{0.000000in}}%
\pgfpathlineto{\pgfqpoint{0.000000in}{-0.027778in}}%
\pgfusepath{stroke,fill}%
}%
\begin{pgfscope}%
\pgfsys@transformshift{4.505223in}{0.467838in}%
\pgfsys@useobject{currentmarker}{}%
\end{pgfscope}%
\end{pgfscope}%
\begin{pgfscope}%
\pgfsetbuttcap%
\pgfsetroundjoin%
\definecolor{currentfill}{rgb}{0.000000,0.000000,0.000000}%
\pgfsetfillcolor{currentfill}%
\pgfsetlinewidth{0.602250pt}%
\definecolor{currentstroke}{rgb}{0.000000,0.000000,0.000000}%
\pgfsetstrokecolor{currentstroke}%
\pgfsetdash{}{0pt}%
\pgfsys@defobject{currentmarker}{\pgfqpoint{0.000000in}{-0.027778in}}{\pgfqpoint{0.000000in}{0.000000in}}{%
\pgfpathmoveto{\pgfqpoint{0.000000in}{0.000000in}}%
\pgfpathlineto{\pgfqpoint{0.000000in}{-0.027778in}}%
\pgfusepath{stroke,fill}%
}%
\begin{pgfscope}%
\pgfsys@transformshift{4.552715in}{0.467838in}%
\pgfsys@useobject{currentmarker}{}%
\end{pgfscope}%
\end{pgfscope}%
\begin{pgfscope}%
\pgfsetbuttcap%
\pgfsetroundjoin%
\definecolor{currentfill}{rgb}{0.000000,0.000000,0.000000}%
\pgfsetfillcolor{currentfill}%
\pgfsetlinewidth{0.602250pt}%
\definecolor{currentstroke}{rgb}{0.000000,0.000000,0.000000}%
\pgfsetstrokecolor{currentstroke}%
\pgfsetdash{}{0pt}%
\pgfsys@defobject{currentmarker}{\pgfqpoint{0.000000in}{-0.027778in}}{\pgfqpoint{0.000000in}{0.000000in}}{%
\pgfpathmoveto{\pgfqpoint{0.000000in}{0.000000in}}%
\pgfpathlineto{\pgfqpoint{0.000000in}{-0.027778in}}%
\pgfusepath{stroke,fill}%
}%
\begin{pgfscope}%
\pgfsys@transformshift{4.594605in}{0.467838in}%
\pgfsys@useobject{currentmarker}{}%
\end{pgfscope}%
\end{pgfscope}%
\begin{pgfscope}%
\definecolor{textcolor}{rgb}{0.000000,0.000000,0.000000}%
\pgfsetstrokecolor{textcolor}%
\pgfsetfillcolor{textcolor}%
\pgftext[x=2.584736in,y=0.207530in,,top]{\color{textcolor}\sffamily\fontsize{8.000000}{9.600000}\selectfont Longest solving time (seconds)}%
\end{pgfscope}%
\begin{pgfscope}%
\pgfsetbuttcap%
\pgfsetroundjoin%
\definecolor{currentfill}{rgb}{0.000000,0.000000,0.000000}%
\pgfsetfillcolor{currentfill}%
\pgfsetlinewidth{0.803000pt}%
\definecolor{currentstroke}{rgb}{0.000000,0.000000,0.000000}%
\pgfsetstrokecolor{currentstroke}%
\pgfsetdash{}{0pt}%
\pgfsys@defobject{currentmarker}{\pgfqpoint{-0.048611in}{0.000000in}}{\pgfqpoint{-0.000000in}{0.000000in}}{%
\pgfpathmoveto{\pgfqpoint{-0.000000in}{0.000000in}}%
\pgfpathlineto{\pgfqpoint{-0.048611in}{0.000000in}}%
\pgfusepath{stroke,fill}%
}%
\begin{pgfscope}%
\pgfsys@transformshift{0.537394in}{0.467838in}%
\pgfsys@useobject{currentmarker}{}%
\end{pgfscope}%
\end{pgfscope}%
\begin{pgfscope}%
\definecolor{textcolor}{rgb}{0.000000,0.000000,0.000000}%
\pgfsetstrokecolor{textcolor}%
\pgfsetfillcolor{textcolor}%
\pgftext[x=0.381143in, y=0.425629in, left, base]{\color{textcolor}\sffamily\fontsize{8.000000}{9.600000}\selectfont \(\displaystyle {0}\)}%
\end{pgfscope}%
\begin{pgfscope}%
\pgfsetbuttcap%
\pgfsetroundjoin%
\definecolor{currentfill}{rgb}{0.000000,0.000000,0.000000}%
\pgfsetfillcolor{currentfill}%
\pgfsetlinewidth{0.803000pt}%
\definecolor{currentstroke}{rgb}{0.000000,0.000000,0.000000}%
\pgfsetstrokecolor{currentstroke}%
\pgfsetdash{}{0pt}%
\pgfsys@defobject{currentmarker}{\pgfqpoint{-0.048611in}{0.000000in}}{\pgfqpoint{-0.000000in}{0.000000in}}{%
\pgfpathmoveto{\pgfqpoint{-0.000000in}{0.000000in}}%
\pgfpathlineto{\pgfqpoint{-0.048611in}{0.000000in}}%
\pgfusepath{stroke,fill}%
}%
\begin{pgfscope}%
\pgfsys@transformshift{0.537394in}{0.717152in}%
\pgfsys@useobject{currentmarker}{}%
\end{pgfscope}%
\end{pgfscope}%
\begin{pgfscope}%
\definecolor{textcolor}{rgb}{0.000000,0.000000,0.000000}%
\pgfsetstrokecolor{textcolor}%
\pgfsetfillcolor{textcolor}%
\pgftext[x=0.263086in, y=0.674943in, left, base]{\color{textcolor}\sffamily\fontsize{8.000000}{9.600000}\selectfont \(\displaystyle {100}\)}%
\end{pgfscope}%
\begin{pgfscope}%
\pgfsetbuttcap%
\pgfsetroundjoin%
\definecolor{currentfill}{rgb}{0.000000,0.000000,0.000000}%
\pgfsetfillcolor{currentfill}%
\pgfsetlinewidth{0.803000pt}%
\definecolor{currentstroke}{rgb}{0.000000,0.000000,0.000000}%
\pgfsetstrokecolor{currentstroke}%
\pgfsetdash{}{0pt}%
\pgfsys@defobject{currentmarker}{\pgfqpoint{-0.048611in}{0.000000in}}{\pgfqpoint{-0.000000in}{0.000000in}}{%
\pgfpathmoveto{\pgfqpoint{-0.000000in}{0.000000in}}%
\pgfpathlineto{\pgfqpoint{-0.048611in}{0.000000in}}%
\pgfusepath{stroke,fill}%
}%
\begin{pgfscope}%
\pgfsys@transformshift{0.537394in}{0.966465in}%
\pgfsys@useobject{currentmarker}{}%
\end{pgfscope}%
\end{pgfscope}%
\begin{pgfscope}%
\definecolor{textcolor}{rgb}{0.000000,0.000000,0.000000}%
\pgfsetstrokecolor{textcolor}%
\pgfsetfillcolor{textcolor}%
\pgftext[x=0.263086in, y=0.924256in, left, base]{\color{textcolor}\sffamily\fontsize{8.000000}{9.600000}\selectfont \(\displaystyle {200}\)}%
\end{pgfscope}%
\begin{pgfscope}%
\pgfsetbuttcap%
\pgfsetroundjoin%
\definecolor{currentfill}{rgb}{0.000000,0.000000,0.000000}%
\pgfsetfillcolor{currentfill}%
\pgfsetlinewidth{0.803000pt}%
\definecolor{currentstroke}{rgb}{0.000000,0.000000,0.000000}%
\pgfsetstrokecolor{currentstroke}%
\pgfsetdash{}{0pt}%
\pgfsys@defobject{currentmarker}{\pgfqpoint{-0.048611in}{0.000000in}}{\pgfqpoint{-0.000000in}{0.000000in}}{%
\pgfpathmoveto{\pgfqpoint{-0.000000in}{0.000000in}}%
\pgfpathlineto{\pgfqpoint{-0.048611in}{0.000000in}}%
\pgfusepath{stroke,fill}%
}%
\begin{pgfscope}%
\pgfsys@transformshift{0.537394in}{1.215779in}%
\pgfsys@useobject{currentmarker}{}%
\end{pgfscope}%
\end{pgfscope}%
\begin{pgfscope}%
\definecolor{textcolor}{rgb}{0.000000,0.000000,0.000000}%
\pgfsetstrokecolor{textcolor}%
\pgfsetfillcolor{textcolor}%
\pgftext[x=0.263086in, y=1.173569in, left, base]{\color{textcolor}\sffamily\fontsize{8.000000}{9.600000}\selectfont \(\displaystyle {300}\)}%
\end{pgfscope}%
\begin{pgfscope}%
\pgfsetbuttcap%
\pgfsetroundjoin%
\definecolor{currentfill}{rgb}{0.000000,0.000000,0.000000}%
\pgfsetfillcolor{currentfill}%
\pgfsetlinewidth{0.803000pt}%
\definecolor{currentstroke}{rgb}{0.000000,0.000000,0.000000}%
\pgfsetstrokecolor{currentstroke}%
\pgfsetdash{}{0pt}%
\pgfsys@defobject{currentmarker}{\pgfqpoint{-0.048611in}{0.000000in}}{\pgfqpoint{-0.000000in}{0.000000in}}{%
\pgfpathmoveto{\pgfqpoint{-0.000000in}{0.000000in}}%
\pgfpathlineto{\pgfqpoint{-0.048611in}{0.000000in}}%
\pgfusepath{stroke,fill}%
}%
\begin{pgfscope}%
\pgfsys@transformshift{0.537394in}{1.465092in}%
\pgfsys@useobject{currentmarker}{}%
\end{pgfscope}%
\end{pgfscope}%
\begin{pgfscope}%
\definecolor{textcolor}{rgb}{0.000000,0.000000,0.000000}%
\pgfsetstrokecolor{textcolor}%
\pgfsetfillcolor{textcolor}%
\pgftext[x=0.263086in, y=1.422883in, left, base]{\color{textcolor}\sffamily\fontsize{8.000000}{9.600000}\selectfont \(\displaystyle {400}\)}%
\end{pgfscope}%
\begin{pgfscope}%
\definecolor{textcolor}{rgb}{0.000000,0.000000,0.000000}%
\pgfsetstrokecolor{textcolor}%
\pgfsetfillcolor{textcolor}%
\pgftext[x=0.207530in,y=0.966465in,,bottom,rotate=90.000000]{\color{textcolor}\sffamily\fontsize{8.000000}{9.600000}\selectfont Benchmarks solved}%
\end{pgfscope}%
\begin{pgfscope}%
\pgfpathrectangle{\pgfqpoint{0.537394in}{0.467838in}}{\pgfqpoint{4.094684in}{0.997254in}}%
\pgfusepath{clip}%
\pgfsetrectcap%
\pgfsetroundjoin%
\pgfsetlinewidth{1.003750pt}%
\definecolor{currentstroke}{rgb}{0.121569,0.466667,0.705882}%
\pgfsetstrokecolor{currentstroke}%
\pgfsetdash{}{0pt}%
\pgfpathmoveto{\pgfqpoint{1.229476in}{0.475318in}}%
\pgfpathlineto{\pgfqpoint{1.318858in}{0.477811in}}%
\pgfpathlineto{\pgfqpoint{1.356331in}{0.485290in}}%
\pgfpathlineto{\pgfqpoint{1.390229in}{0.497756in}}%
\pgfpathlineto{\pgfqpoint{1.421175in}{0.505235in}}%
\pgfpathlineto{\pgfqpoint{1.449643in}{0.510222in}}%
\pgfpathlineto{\pgfqpoint{1.476000in}{0.525180in}}%
\pgfpathlineto{\pgfqpoint{1.500538in}{0.530167in}}%
\pgfpathlineto{\pgfqpoint{1.523492in}{0.540139in}}%
\pgfpathlineto{\pgfqpoint{1.565383in}{0.550112in}}%
\pgfpathlineto{\pgfqpoint{1.584612in}{0.555098in}}%
\pgfpathlineto{\pgfqpoint{1.602855in}{0.567564in}}%
\pgfpathlineto{\pgfqpoint{1.620208in}{0.572550in}}%
\pgfpathlineto{\pgfqpoint{1.636753in}{0.587509in}}%
\pgfpathlineto{\pgfqpoint{1.652563in}{0.594988in}}%
\pgfpathlineto{\pgfqpoint{1.667700in}{0.602468in}}%
\pgfpathlineto{\pgfqpoint{1.682218in}{0.614933in}}%
\pgfpathlineto{\pgfqpoint{1.696168in}{0.624906in}}%
\pgfpathlineto{\pgfqpoint{1.709590in}{0.639865in}}%
\pgfpathlineto{\pgfqpoint{1.722525in}{0.652330in}}%
\pgfpathlineto{\pgfqpoint{1.735005in}{0.664796in}}%
\pgfpathlineto{\pgfqpoint{1.758725in}{0.679755in}}%
\pgfpathlineto{\pgfqpoint{1.770017in}{0.692220in}}%
\pgfpathlineto{\pgfqpoint{1.780961in}{0.697207in}}%
\pgfpathlineto{\pgfqpoint{1.791578in}{0.704686in}}%
\pgfpathlineto{\pgfqpoint{1.801888in}{0.722138in}}%
\pgfpathlineto{\pgfqpoint{1.811907in}{0.734604in}}%
\pgfpathlineto{\pgfqpoint{1.821652in}{0.749563in}}%
\pgfpathlineto{\pgfqpoint{1.840375in}{0.754549in}}%
\pgfpathlineto{\pgfqpoint{1.849380in}{0.759535in}}%
\pgfpathlineto{\pgfqpoint{1.866732in}{0.774494in}}%
\pgfpathlineto{\pgfqpoint{1.875101in}{0.784466in}}%
\pgfpathlineto{\pgfqpoint{1.883278in}{0.791946in}}%
\pgfpathlineto{\pgfqpoint{1.899087in}{0.796932in}}%
\pgfpathlineto{\pgfqpoint{1.906736in}{0.804411in}}%
\pgfpathlineto{\pgfqpoint{1.914224in}{0.806905in}}%
\pgfpathlineto{\pgfqpoint{1.921558in}{0.814384in}}%
\pgfpathlineto{\pgfqpoint{1.935786in}{0.819370in}}%
\pgfpathlineto{\pgfqpoint{1.962641in}{0.829343in}}%
\pgfpathlineto{\pgfqpoint{1.969049in}{0.834329in}}%
\pgfpathlineto{\pgfqpoint{1.981530in}{0.836822in}}%
\pgfpathlineto{\pgfqpoint{1.987610in}{0.841808in}}%
\pgfpathlineto{\pgfqpoint{1.993587in}{0.844302in}}%
\pgfpathlineto{\pgfqpoint{2.005249in}{0.846795in}}%
\pgfpathlineto{\pgfqpoint{2.016541in}{0.856767in}}%
\pgfpathlineto{\pgfqpoint{2.022055in}{0.859260in}}%
\pgfpathlineto{\pgfqpoint{2.027485in}{0.864247in}}%
\pgfpathlineto{\pgfqpoint{2.032834in}{0.866740in}}%
\pgfpathlineto{\pgfqpoint{2.043295in}{0.876712in}}%
\pgfpathlineto{\pgfqpoint{2.058432in}{0.879205in}}%
\pgfpathlineto{\pgfqpoint{2.068177in}{0.886685in}}%
\pgfpathlineto{\pgfqpoint{2.072951in}{0.889178in}}%
\pgfpathlineto{\pgfqpoint{2.100323in}{0.896657in}}%
\pgfpathlineto{\pgfqpoint{2.113257in}{0.904137in}}%
\pgfpathlineto{\pgfqpoint{2.117466in}{0.909123in}}%
\pgfpathlineto{\pgfqpoint{2.121626in}{0.911616in}}%
\pgfpathlineto{\pgfqpoint{2.129802in}{0.914109in}}%
\pgfpathlineto{\pgfqpoint{2.133821in}{0.919096in}}%
\pgfpathlineto{\pgfqpoint{2.145612in}{0.926575in}}%
\pgfpathlineto{\pgfqpoint{2.175268in}{0.929068in}}%
\pgfpathlineto{\pgfqpoint{2.182311in}{0.931561in}}%
\pgfpathlineto{\pgfqpoint{2.189217in}{0.936548in}}%
\pgfpathlineto{\pgfqpoint{2.195991in}{0.939041in}}%
\pgfpathlineto{\pgfqpoint{2.209166in}{0.941534in}}%
\pgfpathlineto{\pgfqpoint{2.212384in}{0.944027in}}%
\pgfpathlineto{\pgfqpoint{2.221869in}{0.946520in}}%
\pgfpathlineto{\pgfqpoint{2.234134in}{0.956493in}}%
\pgfpathlineto{\pgfqpoint{2.240112in}{0.958986in}}%
\pgfpathlineto{\pgfqpoint{2.254631in}{0.963972in}}%
\pgfpathlineto{\pgfqpoint{2.263066in}{0.966465in}}%
\pgfpathlineto{\pgfqpoint{2.274010in}{0.968958in}}%
\pgfpathlineto{\pgfqpoint{2.289820in}{0.971451in}}%
\pgfpathlineto{\pgfqpoint{2.292388in}{0.973945in}}%
\pgfpathlineto{\pgfqpoint{2.333424in}{0.976438in}}%
\pgfpathlineto{\pgfqpoint{2.342429in}{0.981424in}}%
\pgfpathlineto{\pgfqpoint{2.349036in}{0.988903in}}%
\pgfpathlineto{\pgfqpoint{2.355522in}{0.991397in}}%
\pgfpathlineto{\pgfqpoint{2.359782in}{0.993890in}}%
\pgfpathlineto{\pgfqpoint{2.361892in}{0.996383in}}%
\pgfpathlineto{\pgfqpoint{2.372262in}{0.998876in}}%
\pgfpathlineto{\pgfqpoint{2.397889in}{1.001369in}}%
\pgfpathlineto{\pgfqpoint{2.414607in}{1.003862in}}%
\pgfpathlineto{\pgfqpoint{2.421792in}{1.006355in}}%
\pgfpathlineto{\pgfqpoint{2.440834in}{1.008848in}}%
\pgfpathlineto{\pgfqpoint{2.452442in}{1.011342in}}%
\pgfpathlineto{\pgfqpoint{2.469950in}{1.013835in}}%
\pgfpathlineto{\pgfqpoint{2.498299in}{1.016328in}}%
\pgfpathlineto{\pgfqpoint{2.523219in}{1.018821in}}%
\pgfpathlineto{\pgfqpoint{2.541462in}{1.021314in}}%
\pgfpathlineto{\pgfqpoint{2.552714in}{1.023807in}}%
\pgfpathlineto{\pgfqpoint{2.562425in}{1.026300in}}%
\pgfpathlineto{\pgfqpoint{2.573043in}{1.028794in}}%
\pgfpathlineto{\pgfqpoint{2.611560in}{1.031287in}}%
\pgfpathlineto{\pgfqpoint{2.615707in}{1.033780in}}%
\pgfpathlineto{\pgfqpoint{2.657483in}{1.036273in}}%
\pgfpathlineto{\pgfqpoint{2.675360in}{1.038766in}}%
\pgfpathlineto{\pgfqpoint{2.733161in}{1.041259in}}%
\pgfpathlineto{\pgfqpoint{2.758883in}{1.043752in}}%
\pgfpathlineto{\pgfqpoint{2.765032in}{1.046245in}}%
\pgfpathlineto{\pgfqpoint{2.789886in}{1.048739in}}%
\pgfpathlineto{\pgfqpoint{2.803520in}{1.051232in}}%
\pgfpathlineto{\pgfqpoint{2.842630in}{1.053725in}}%
\pgfpathlineto{\pgfqpoint{2.853888in}{1.056218in}}%
\pgfpathlineto{\pgfqpoint{2.982637in}{1.058711in}}%
\pgfpathlineto{\pgfqpoint{2.990989in}{1.061204in}}%
\pgfpathlineto{\pgfqpoint{2.992064in}{1.063697in}}%
\pgfpathlineto{\pgfqpoint{3.105910in}{1.066191in}}%
\pgfpathlineto{\pgfqpoint{3.165345in}{1.068684in}}%
\pgfpathlineto{\pgfqpoint{3.174245in}{1.071177in}}%
\pgfpathlineto{\pgfqpoint{3.181880in}{1.073670in}}%
\pgfpathlineto{\pgfqpoint{3.185221in}{1.076163in}}%
\pgfpathlineto{\pgfqpoint{3.349434in}{1.078656in}}%
\pgfpathlineto{\pgfqpoint{3.480793in}{1.081149in}}%
\pgfpathlineto{\pgfqpoint{3.507138in}{1.083642in}}%
\pgfpathlineto{\pgfqpoint{3.555930in}{1.086136in}}%
\pgfpathlineto{\pgfqpoint{3.558122in}{1.088629in}}%
\pgfpathlineto{\pgfqpoint{3.651569in}{1.091122in}}%
\pgfpathlineto{\pgfqpoint{3.669804in}{1.093615in}}%
\pgfpathlineto{\pgfqpoint{3.691181in}{1.096108in}}%
\pgfpathlineto{\pgfqpoint{3.696058in}{1.098601in}}%
\pgfpathlineto{\pgfqpoint{3.705039in}{1.101094in}}%
\pgfpathlineto{\pgfqpoint{3.833059in}{1.103588in}}%
\pgfpathlineto{\pgfqpoint{3.923858in}{1.106081in}}%
\pgfpathlineto{\pgfqpoint{4.024851in}{1.108574in}}%
\pgfpathlineto{\pgfqpoint{4.175414in}{1.111067in}}%
\pgfpathlineto{\pgfqpoint{4.311766in}{1.113560in}}%
\pgfpathlineto{\pgfqpoint{4.348586in}{1.116053in}}%
\pgfpathlineto{\pgfqpoint{4.391500in}{1.118546in}}%
\pgfpathlineto{\pgfqpoint{4.487557in}{1.121040in}}%
\pgfpathlineto{\pgfqpoint{4.543350in}{1.123533in}}%
\pgfpathlineto{\pgfqpoint{4.552968in}{1.126026in}}%
\pgfpathlineto{\pgfqpoint{4.552968in}{1.126026in}}%
\pgfusepath{stroke}%
\end{pgfscope}%
\begin{pgfscope}%
\pgfpathrectangle{\pgfqpoint{0.537394in}{0.467838in}}{\pgfqpoint{4.094684in}{0.997254in}}%
\pgfusepath{clip}%
\pgfsetrectcap%
\pgfsetroundjoin%
\pgfsetlinewidth{1.003750pt}%
\definecolor{currentstroke}{rgb}{1.000000,0.498039,0.054902}%
\pgfsetstrokecolor{currentstroke}%
\pgfsetdash{}{0pt}%
\pgfpathmoveto{\pgfqpoint{0.928126in}{0.480304in}}%
\pgfpathlineto{\pgfqpoint{1.030443in}{0.485290in}}%
\pgfpathlineto{\pgfqpoint{1.109806in}{0.490277in}}%
\pgfpathlineto{\pgfqpoint{1.229476in}{0.492770in}}%
\pgfpathlineto{\pgfqpoint{1.276968in}{0.500249in}}%
\pgfpathlineto{\pgfqpoint{1.318858in}{0.502742in}}%
\pgfpathlineto{\pgfqpoint{1.356331in}{0.507728in}}%
\pgfpathlineto{\pgfqpoint{1.390229in}{0.522687in}}%
\pgfpathlineto{\pgfqpoint{1.421175in}{0.527674in}}%
\pgfpathlineto{\pgfqpoint{1.449643in}{0.532660in}}%
\pgfpathlineto{\pgfqpoint{1.476000in}{0.550112in}}%
\pgfpathlineto{\pgfqpoint{1.500538in}{0.557591in}}%
\pgfpathlineto{\pgfqpoint{1.523492in}{0.572550in}}%
\pgfpathlineto{\pgfqpoint{1.545054in}{0.585016in}}%
\pgfpathlineto{\pgfqpoint{1.565383in}{0.592495in}}%
\pgfpathlineto{\pgfqpoint{1.584612in}{0.607454in}}%
\pgfpathlineto{\pgfqpoint{1.602855in}{0.609947in}}%
\pgfpathlineto{\pgfqpoint{1.620208in}{0.632385in}}%
\pgfpathlineto{\pgfqpoint{1.652563in}{0.642358in}}%
\pgfpathlineto{\pgfqpoint{1.667700in}{0.657317in}}%
\pgfpathlineto{\pgfqpoint{1.696168in}{0.667289in}}%
\pgfpathlineto{\pgfqpoint{1.709590in}{0.677262in}}%
\pgfpathlineto{\pgfqpoint{1.722525in}{0.689727in}}%
\pgfpathlineto{\pgfqpoint{1.735005in}{0.699700in}}%
\pgfpathlineto{\pgfqpoint{1.747063in}{0.707179in}}%
\pgfpathlineto{\pgfqpoint{1.758725in}{0.717152in}}%
\pgfpathlineto{\pgfqpoint{1.770017in}{0.729617in}}%
\pgfpathlineto{\pgfqpoint{1.780961in}{0.739590in}}%
\pgfpathlineto{\pgfqpoint{1.791578in}{0.744576in}}%
\pgfpathlineto{\pgfqpoint{1.821652in}{0.754549in}}%
\pgfpathlineto{\pgfqpoint{1.840375in}{0.759535in}}%
\pgfpathlineto{\pgfqpoint{1.858162in}{0.774494in}}%
\pgfpathlineto{\pgfqpoint{1.866732in}{0.779480in}}%
\pgfpathlineto{\pgfqpoint{1.891270in}{0.786960in}}%
\pgfpathlineto{\pgfqpoint{1.899087in}{0.789453in}}%
\pgfpathlineto{\pgfqpoint{1.906736in}{0.794439in}}%
\pgfpathlineto{\pgfqpoint{1.921558in}{0.796932in}}%
\pgfpathlineto{\pgfqpoint{1.942692in}{0.804411in}}%
\pgfpathlineto{\pgfqpoint{1.962641in}{0.811891in}}%
\pgfpathlineto{\pgfqpoint{1.969049in}{0.814384in}}%
\pgfpathlineto{\pgfqpoint{1.981530in}{0.816877in}}%
\pgfpathlineto{\pgfqpoint{2.005249in}{0.819370in}}%
\pgfpathlineto{\pgfqpoint{2.010940in}{0.821863in}}%
\pgfpathlineto{\pgfqpoint{2.016541in}{0.826850in}}%
\pgfpathlineto{\pgfqpoint{2.022055in}{0.829343in}}%
\pgfpathlineto{\pgfqpoint{2.027485in}{0.836822in}}%
\pgfpathlineto{\pgfqpoint{2.048413in}{0.846795in}}%
\pgfpathlineto{\pgfqpoint{2.058432in}{0.854274in}}%
\pgfpathlineto{\pgfqpoint{2.068177in}{0.856767in}}%
\pgfpathlineto{\pgfqpoint{2.082311in}{0.864247in}}%
\pgfpathlineto{\pgfqpoint{2.091431in}{0.866740in}}%
\pgfpathlineto{\pgfqpoint{2.108998in}{0.876712in}}%
\pgfpathlineto{\pgfqpoint{2.113257in}{0.879205in}}%
\pgfpathlineto{\pgfqpoint{2.121626in}{0.881699in}}%
\pgfpathlineto{\pgfqpoint{2.125738in}{0.886685in}}%
\pgfpathlineto{\pgfqpoint{2.137795in}{0.889178in}}%
\pgfpathlineto{\pgfqpoint{2.141725in}{0.891671in}}%
\pgfpathlineto{\pgfqpoint{2.145612in}{0.896657in}}%
\pgfpathlineto{\pgfqpoint{2.149457in}{0.904137in}}%
\pgfpathlineto{\pgfqpoint{2.157025in}{0.909123in}}%
\pgfpathlineto{\pgfqpoint{2.171693in}{0.911616in}}%
\pgfpathlineto{\pgfqpoint{2.175268in}{0.914109in}}%
\pgfpathlineto{\pgfqpoint{2.178806in}{0.921589in}}%
\pgfpathlineto{\pgfqpoint{2.182311in}{0.926575in}}%
\pgfpathlineto{\pgfqpoint{2.192620in}{0.929068in}}%
\pgfpathlineto{\pgfqpoint{2.209166in}{0.931561in}}%
\pgfpathlineto{\pgfqpoint{2.212384in}{0.934054in}}%
\pgfpathlineto{\pgfqpoint{2.240112in}{0.936548in}}%
\pgfpathlineto{\pgfqpoint{2.248894in}{0.939041in}}%
\pgfpathlineto{\pgfqpoint{2.254631in}{0.941534in}}%
\pgfpathlineto{\pgfqpoint{2.276694in}{0.944027in}}%
\pgfpathlineto{\pgfqpoint{2.279358in}{0.946520in}}%
\pgfpathlineto{\pgfqpoint{2.319475in}{0.951506in}}%
\pgfpathlineto{\pgfqpoint{2.321838in}{0.954000in}}%
\pgfpathlineto{\pgfqpoint{2.337955in}{0.956493in}}%
\pgfpathlineto{\pgfqpoint{2.342429in}{0.958986in}}%
\pgfpathlineto{\pgfqpoint{2.344645in}{0.961479in}}%
\pgfpathlineto{\pgfqpoint{2.386290in}{0.963972in}}%
\pgfpathlineto{\pgfqpoint{2.390198in}{0.968958in}}%
\pgfpathlineto{\pgfqpoint{2.403549in}{0.971451in}}%
\pgfpathlineto{\pgfqpoint{2.412788in}{0.973945in}}%
\pgfpathlineto{\pgfqpoint{2.420009in}{0.976438in}}%
\pgfpathlineto{\pgfqpoint{2.435741in}{0.978931in}}%
\pgfpathlineto{\pgfqpoint{2.462099in}{0.981424in}}%
\pgfpathlineto{\pgfqpoint{2.463683in}{0.983917in}}%
\pgfpathlineto{\pgfqpoint{2.469950in}{0.986410in}}%
\pgfpathlineto{\pgfqpoint{2.473043in}{0.988903in}}%
\pgfpathlineto{\pgfqpoint{2.482163in}{0.991397in}}%
\pgfpathlineto{\pgfqpoint{2.503989in}{0.996383in}}%
\pgfpathlineto{\pgfqpoint{2.510977in}{0.998876in}}%
\pgfpathlineto{\pgfqpoint{2.512358in}{1.003862in}}%
\pgfpathlineto{\pgfqpoint{2.515104in}{1.006355in}}%
\pgfpathlineto{\pgfqpoint{2.525883in}{1.008848in}}%
\pgfpathlineto{\pgfqpoint{2.531152in}{1.011342in}}%
\pgfpathlineto{\pgfqpoint{2.542730in}{1.013835in}}%
\pgfpathlineto{\pgfqpoint{2.547757in}{1.016328in}}%
\pgfpathlineto{\pgfqpoint{2.555167in}{1.018821in}}%
\pgfpathlineto{\pgfqpoint{2.613640in}{1.021314in}}%
\pgfpathlineto{\pgfqpoint{2.616737in}{1.023807in}}%
\pgfpathlineto{\pgfqpoint{2.657483in}{1.026300in}}%
\pgfpathlineto{\pgfqpoint{2.669205in}{1.028794in}}%
\pgfpathlineto{\pgfqpoint{2.673612in}{1.031287in}}%
\pgfpathlineto{\pgfqpoint{2.691548in}{1.033780in}}%
\pgfpathlineto{\pgfqpoint{2.715697in}{1.036273in}}%
\pgfpathlineto{\pgfqpoint{2.733161in}{1.038766in}}%
\pgfpathlineto{\pgfqpoint{2.817235in}{1.041259in}}%
\pgfpathlineto{\pgfqpoint{2.927934in}{1.043752in}}%
\pgfpathlineto{\pgfqpoint{2.938489in}{1.046245in}}%
\pgfpathlineto{\pgfqpoint{2.950754in}{1.048739in}}%
\pgfpathlineto{\pgfqpoint{3.059641in}{1.051232in}}%
\pgfpathlineto{\pgfqpoint{3.080889in}{1.053725in}}%
\pgfpathlineto{\pgfqpoint{3.250896in}{1.056218in}}%
\pgfpathlineto{\pgfqpoint{3.259940in}{1.058711in}}%
\pgfpathlineto{\pgfqpoint{3.261621in}{1.061204in}}%
\pgfpathlineto{\pgfqpoint{3.370788in}{1.063697in}}%
\pgfpathlineto{\pgfqpoint{3.420233in}{1.066191in}}%
\pgfpathlineto{\pgfqpoint{3.429243in}{1.068684in}}%
\pgfpathlineto{\pgfqpoint{3.448494in}{1.071177in}}%
\pgfpathlineto{\pgfqpoint{3.511483in}{1.073670in}}%
\pgfpathlineto{\pgfqpoint{3.597984in}{1.076163in}}%
\pgfpathlineto{\pgfqpoint{3.606287in}{1.078656in}}%
\pgfpathlineto{\pgfqpoint{3.744420in}{1.081149in}}%
\pgfpathlineto{\pgfqpoint{3.744679in}{1.083642in}}%
\pgfpathlineto{\pgfqpoint{3.745239in}{1.086136in}}%
\pgfpathlineto{\pgfqpoint{4.049382in}{1.088629in}}%
\pgfpathlineto{\pgfqpoint{4.091553in}{1.091122in}}%
\pgfpathlineto{\pgfqpoint{4.120306in}{1.093615in}}%
\pgfpathlineto{\pgfqpoint{4.140420in}{1.096108in}}%
\pgfpathlineto{\pgfqpoint{4.145346in}{1.098601in}}%
\pgfpathlineto{\pgfqpoint{4.158826in}{1.101094in}}%
\pgfpathlineto{\pgfqpoint{4.180755in}{1.103588in}}%
\pgfpathlineto{\pgfqpoint{4.236097in}{1.106081in}}%
\pgfpathlineto{\pgfqpoint{4.275770in}{1.108574in}}%
\pgfpathlineto{\pgfqpoint{4.291495in}{1.111067in}}%
\pgfpathlineto{\pgfqpoint{4.311407in}{1.113560in}}%
\pgfpathlineto{\pgfqpoint{4.322619in}{1.116053in}}%
\pgfpathlineto{\pgfqpoint{4.405195in}{1.118546in}}%
\pgfpathlineto{\pgfqpoint{4.415629in}{1.121040in}}%
\pgfpathlineto{\pgfqpoint{4.419393in}{1.123533in}}%
\pgfpathlineto{\pgfqpoint{4.419393in}{1.123533in}}%
\pgfusepath{stroke}%
\end{pgfscope}%
\begin{pgfscope}%
\pgfpathrectangle{\pgfqpoint{0.537394in}{0.467838in}}{\pgfqpoint{4.094684in}{0.997254in}}%
\pgfusepath{clip}%
\pgfsetbuttcap%
\pgfsetroundjoin%
\pgfsetlinewidth{1.003750pt}%
\definecolor{currentstroke}{rgb}{0.172549,0.627451,0.172549}%
\pgfsetstrokecolor{currentstroke}%
\pgfsetdash{{3.700000pt}{1.600000pt}}{0.000000pt}%
\pgfpathmoveto{\pgfqpoint{0.928126in}{0.485290in}}%
\pgfpathlineto{\pgfqpoint{1.030443in}{0.495263in}}%
\pgfpathlineto{\pgfqpoint{1.109806in}{0.497756in}}%
\pgfpathlineto{\pgfqpoint{1.174651in}{0.502742in}}%
\pgfpathlineto{\pgfqpoint{1.229476in}{0.505235in}}%
\pgfpathlineto{\pgfqpoint{1.276968in}{0.507728in}}%
\pgfpathlineto{\pgfqpoint{1.390229in}{0.515208in}}%
\pgfpathlineto{\pgfqpoint{1.421175in}{0.517701in}}%
\pgfpathlineto{\pgfqpoint{1.449643in}{0.522687in}}%
\pgfpathlineto{\pgfqpoint{1.500538in}{0.527674in}}%
\pgfpathlineto{\pgfqpoint{1.584612in}{0.542632in}}%
\pgfpathlineto{\pgfqpoint{1.620208in}{0.547619in}}%
\pgfpathlineto{\pgfqpoint{1.667700in}{0.550112in}}%
\pgfpathlineto{\pgfqpoint{1.682218in}{0.552605in}}%
\pgfpathlineto{\pgfqpoint{1.735005in}{0.555098in}}%
\pgfpathlineto{\pgfqpoint{1.801888in}{0.557591in}}%
\pgfpathlineto{\pgfqpoint{1.858162in}{0.565071in}}%
\pgfpathlineto{\pgfqpoint{1.866732in}{0.567564in}}%
\pgfpathlineto{\pgfqpoint{1.928743in}{0.570057in}}%
\pgfpathlineto{\pgfqpoint{1.942692in}{0.572550in}}%
\pgfpathlineto{\pgfqpoint{1.981530in}{0.575043in}}%
\pgfpathlineto{\pgfqpoint{2.063338in}{0.577536in}}%
\pgfpathlineto{\pgfqpoint{2.068177in}{0.582523in}}%
\pgfpathlineto{\pgfqpoint{2.125738in}{0.585016in}}%
\pgfpathlineto{\pgfqpoint{2.129802in}{0.587509in}}%
\pgfpathlineto{\pgfqpoint{2.153261in}{0.590002in}}%
\pgfpathlineto{\pgfqpoint{2.157025in}{0.592495in}}%
\pgfpathlineto{\pgfqpoint{2.168082in}{0.597481in}}%
\pgfpathlineto{\pgfqpoint{2.205917in}{0.599974in}}%
\pgfpathlineto{\pgfqpoint{2.260276in}{0.602468in}}%
\pgfpathlineto{\pgfqpoint{2.276694in}{0.604961in}}%
\pgfpathlineto{\pgfqpoint{2.328835in}{0.607454in}}%
\pgfpathlineto{\pgfqpoint{2.346847in}{0.609947in}}%
\pgfpathlineto{\pgfqpoint{2.454070in}{0.612440in}}%
\pgfpathlineto{\pgfqpoint{2.468394in}{0.614933in}}%
\pgfpathlineto{\pgfqpoint{2.473043in}{0.617426in}}%
\pgfpathlineto{\pgfqpoint{2.502575in}{0.619920in}}%
\pgfpathlineto{\pgfqpoint{2.503989in}{0.624906in}}%
\pgfpathlineto{\pgfqpoint{2.523219in}{0.627399in}}%
\pgfpathlineto{\pgfqpoint{2.541462in}{0.632385in}}%
\pgfpathlineto{\pgfqpoint{2.547757in}{0.634878in}}%
\pgfpathlineto{\pgfqpoint{2.550244in}{0.637371in}}%
\pgfpathlineto{\pgfqpoint{2.586724in}{0.639865in}}%
\pgfpathlineto{\pgfqpoint{2.587840in}{0.642358in}}%
\pgfpathlineto{\pgfqpoint{2.591169in}{0.644851in}}%
\pgfpathlineto{\pgfqpoint{2.593372in}{0.647344in}}%
\pgfpathlineto{\pgfqpoint{2.594468in}{0.649837in}}%
\pgfpathlineto{\pgfqpoint{2.596650in}{0.652330in}}%
\pgfpathlineto{\pgfqpoint{2.598818in}{0.657317in}}%
\pgfpathlineto{\pgfqpoint{2.617763in}{0.659810in}}%
\pgfpathlineto{\pgfqpoint{2.622851in}{0.664796in}}%
\pgfpathlineto{\pgfqpoint{2.629855in}{0.669782in}}%
\pgfpathlineto{\pgfqpoint{2.639626in}{0.672275in}}%
\pgfpathlineto{\pgfqpoint{2.684821in}{0.674768in}}%
\pgfpathlineto{\pgfqpoint{2.749100in}{0.677262in}}%
\pgfpathlineto{\pgfqpoint{2.749807in}{0.679755in}}%
\pgfpathlineto{\pgfqpoint{2.760258in}{0.682248in}}%
\pgfpathlineto{\pgfqpoint{2.801079in}{0.684741in}}%
\pgfpathlineto{\pgfqpoint{2.808350in}{0.687234in}}%
\pgfpathlineto{\pgfqpoint{2.822461in}{0.689727in}}%
\pgfpathlineto{\pgfqpoint{2.850707in}{0.692220in}}%
\pgfpathlineto{\pgfqpoint{2.868364in}{0.694714in}}%
\pgfpathlineto{\pgfqpoint{2.872895in}{0.697207in}}%
\pgfpathlineto{\pgfqpoint{2.882274in}{0.699700in}}%
\pgfpathlineto{\pgfqpoint{2.882761in}{0.702193in}}%
\pgfpathlineto{\pgfqpoint{2.888073in}{0.704686in}}%
\pgfpathlineto{\pgfqpoint{2.894721in}{0.707179in}}%
\pgfpathlineto{\pgfqpoint{2.903090in}{0.709672in}}%
\pgfpathlineto{\pgfqpoint{2.907202in}{0.712165in}}%
\pgfpathlineto{\pgfqpoint{2.910367in}{0.714659in}}%
\pgfpathlineto{\pgfqpoint{2.910817in}{0.719645in}}%
\pgfpathlineto{\pgfqpoint{2.911267in}{0.722138in}}%
\pgfpathlineto{\pgfqpoint{2.912611in}{0.724631in}}%
\pgfpathlineto{\pgfqpoint{2.925354in}{0.727124in}}%
\pgfpathlineto{\pgfqpoint{2.926216in}{0.729617in}}%
\pgfpathlineto{\pgfqpoint{2.926647in}{0.732111in}}%
\pgfpathlineto{\pgfqpoint{2.929644in}{0.734604in}}%
\pgfpathlineto{\pgfqpoint{2.935145in}{0.737097in}}%
\pgfpathlineto{\pgfqpoint{2.937239in}{0.739590in}}%
\pgfpathlineto{\pgfqpoint{2.938905in}{0.742083in}}%
\pgfpathlineto{\pgfqpoint{2.945899in}{0.744576in}}%
\pgfpathlineto{\pgfqpoint{2.955544in}{0.747069in}}%
\pgfpathlineto{\pgfqpoint{2.958309in}{0.749563in}}%
\pgfpathlineto{\pgfqpoint{2.961443in}{0.752056in}}%
\pgfpathlineto{\pgfqpoint{2.962222in}{0.754549in}}%
\pgfpathlineto{\pgfqpoint{2.969920in}{0.757042in}}%
\pgfpathlineto{\pgfqpoint{2.973331in}{0.759535in}}%
\pgfpathlineto{\pgfqpoint{2.974085in}{0.764521in}}%
\pgfpathlineto{\pgfqpoint{2.975587in}{0.769508in}}%
\pgfpathlineto{\pgfqpoint{2.978201in}{0.772001in}}%
\pgfpathlineto{\pgfqpoint{2.984104in}{0.774494in}}%
\pgfpathlineto{\pgfqpoint{2.986292in}{0.776987in}}%
\pgfpathlineto{\pgfqpoint{2.988106in}{0.779480in}}%
\pgfpathlineto{\pgfqpoint{3.005752in}{0.781973in}}%
\pgfpathlineto{\pgfqpoint{3.006440in}{0.784466in}}%
\pgfpathlineto{\pgfqpoint{3.007811in}{0.786960in}}%
\pgfpathlineto{\pgfqpoint{3.010879in}{0.794439in}}%
\pgfpathlineto{\pgfqpoint{3.011218in}{0.796932in}}%
\pgfpathlineto{\pgfqpoint{3.015599in}{0.799425in}}%
\pgfpathlineto{\pgfqpoint{3.015933in}{0.801918in}}%
\pgfpathlineto{\pgfqpoint{3.017935in}{0.804411in}}%
\pgfpathlineto{\pgfqpoint{3.054875in}{0.806905in}}%
\pgfpathlineto{\pgfqpoint{3.076966in}{0.809398in}}%
\pgfpathlineto{\pgfqpoint{3.079213in}{0.811891in}}%
\pgfpathlineto{\pgfqpoint{3.101203in}{0.814384in}}%
\pgfpathlineto{\pgfqpoint{3.101466in}{0.816877in}}%
\pgfpathlineto{\pgfqpoint{3.104348in}{0.819370in}}%
\pgfpathlineto{\pgfqpoint{3.108756in}{0.821863in}}%
\pgfpathlineto{\pgfqpoint{3.136510in}{0.824357in}}%
\pgfpathlineto{\pgfqpoint{3.136748in}{0.826850in}}%
\pgfpathlineto{\pgfqpoint{3.139832in}{0.829343in}}%
\pgfpathlineto{\pgfqpoint{3.140775in}{0.839315in}}%
\pgfpathlineto{\pgfqpoint{3.142420in}{0.841808in}}%
\pgfpathlineto{\pgfqpoint{3.172956in}{0.844302in}}%
\pgfpathlineto{\pgfqpoint{3.184597in}{0.846795in}}%
\pgfpathlineto{\pgfqpoint{3.189149in}{0.849288in}}%
\pgfpathlineto{\pgfqpoint{3.189355in}{0.851781in}}%
\pgfpathlineto{\pgfqpoint{3.190995in}{0.854274in}}%
\pgfpathlineto{\pgfqpoint{3.193846in}{0.856767in}}%
\pgfpathlineto{\pgfqpoint{3.237693in}{0.859260in}}%
\pgfpathlineto{\pgfqpoint{3.281353in}{0.864247in}}%
\pgfpathlineto{\pgfqpoint{3.322502in}{0.866740in}}%
\pgfpathlineto{\pgfqpoint{3.330605in}{0.869233in}}%
\pgfpathlineto{\pgfqpoint{3.331433in}{0.871726in}}%
\pgfpathlineto{\pgfqpoint{3.331570in}{0.874219in}}%
\pgfpathlineto{\pgfqpoint{3.333082in}{0.876712in}}%
\pgfpathlineto{\pgfqpoint{3.351264in}{0.879205in}}%
\pgfpathlineto{\pgfqpoint{3.352305in}{0.881699in}}%
\pgfpathlineto{\pgfqpoint{3.352694in}{0.886685in}}%
\pgfpathlineto{\pgfqpoint{3.352824in}{0.889178in}}%
\pgfpathlineto{\pgfqpoint{3.357337in}{0.891671in}}%
\pgfpathlineto{\pgfqpoint{3.382438in}{0.894164in}}%
\pgfpathlineto{\pgfqpoint{3.384343in}{0.896657in}}%
\pgfpathlineto{\pgfqpoint{3.395679in}{0.899151in}}%
\pgfpathlineto{\pgfqpoint{3.421198in}{0.901644in}}%
\pgfpathlineto{\pgfqpoint{3.422587in}{0.904137in}}%
\pgfpathlineto{\pgfqpoint{3.457114in}{0.906630in}}%
\pgfpathlineto{\pgfqpoint{3.458659in}{0.909123in}}%
\pgfpathlineto{\pgfqpoint{3.464963in}{0.911616in}}%
\pgfpathlineto{\pgfqpoint{3.465436in}{0.914109in}}%
\pgfpathlineto{\pgfqpoint{3.470784in}{0.916602in}}%
\pgfpathlineto{\pgfqpoint{3.471436in}{0.919096in}}%
\pgfpathlineto{\pgfqpoint{3.501032in}{0.921589in}}%
\pgfpathlineto{\pgfqpoint{3.508982in}{0.924082in}}%
\pgfpathlineto{\pgfqpoint{3.510068in}{0.926575in}}%
\pgfpathlineto{\pgfqpoint{3.512727in}{0.929068in}}%
\pgfpathlineto{\pgfqpoint{3.516104in}{0.934054in}}%
\pgfpathlineto{\pgfqpoint{3.609454in}{0.936548in}}%
\pgfpathlineto{\pgfqpoint{3.616757in}{0.939041in}}%
\pgfpathlineto{\pgfqpoint{3.640879in}{0.941534in}}%
\pgfpathlineto{\pgfqpoint{3.642377in}{0.944027in}}%
\pgfpathlineto{\pgfqpoint{3.642951in}{0.946520in}}%
\pgfpathlineto{\pgfqpoint{3.645638in}{0.949013in}}%
\pgfpathlineto{\pgfqpoint{3.646776in}{0.951506in}}%
\pgfpathlineto{\pgfqpoint{3.647059in}{0.954000in}}%
\pgfpathlineto{\pgfqpoint{3.655303in}{0.956493in}}%
\pgfpathlineto{\pgfqpoint{3.655801in}{0.958986in}}%
\pgfpathlineto{\pgfqpoint{3.657127in}{0.961479in}}%
\pgfpathlineto{\pgfqpoint{3.659381in}{0.963972in}}%
\pgfpathlineto{\pgfqpoint{3.659436in}{0.966465in}}%
\pgfpathlineto{\pgfqpoint{3.672402in}{0.968958in}}%
\pgfpathlineto{\pgfqpoint{3.674614in}{0.971451in}}%
\pgfpathlineto{\pgfqpoint{3.683532in}{0.973945in}}%
\pgfpathlineto{\pgfqpoint{3.683737in}{0.976438in}}%
\pgfpathlineto{\pgfqpoint{3.691531in}{0.981424in}}%
\pgfpathlineto{\pgfqpoint{3.692781in}{0.983917in}}%
\pgfpathlineto{\pgfqpoint{3.707106in}{0.986410in}}%
\pgfpathlineto{\pgfqpoint{3.739248in}{0.988903in}}%
\pgfpathlineto{\pgfqpoint{3.740559in}{0.991397in}}%
\pgfpathlineto{\pgfqpoint{3.787751in}{0.993890in}}%
\pgfpathlineto{\pgfqpoint{3.788819in}{0.996383in}}%
\pgfpathlineto{\pgfqpoint{3.793435in}{0.998876in}}%
\pgfpathlineto{\pgfqpoint{3.794561in}{1.001369in}}%
\pgfpathlineto{\pgfqpoint{3.799695in}{1.003862in}}%
\pgfpathlineto{\pgfqpoint{3.800875in}{1.006355in}}%
\pgfpathlineto{\pgfqpoint{3.810428in}{1.008848in}}%
\pgfpathlineto{\pgfqpoint{3.811858in}{1.011342in}}%
\pgfpathlineto{\pgfqpoint{3.832386in}{1.013835in}}%
\pgfpathlineto{\pgfqpoint{3.836173in}{1.016328in}}%
\pgfpathlineto{\pgfqpoint{3.859743in}{1.018821in}}%
\pgfpathlineto{\pgfqpoint{3.924977in}{1.021314in}}%
\pgfpathlineto{\pgfqpoint{3.925936in}{1.023807in}}%
\pgfpathlineto{\pgfqpoint{3.929621in}{1.026300in}}%
\pgfpathlineto{\pgfqpoint{3.929826in}{1.028794in}}%
\pgfpathlineto{\pgfqpoint{3.980947in}{1.031287in}}%
\pgfpathlineto{\pgfqpoint{3.982098in}{1.033780in}}%
\pgfpathlineto{\pgfqpoint{4.012499in}{1.036273in}}%
\pgfpathlineto{\pgfqpoint{4.012702in}{1.038766in}}%
\pgfpathlineto{\pgfqpoint{4.013857in}{1.041259in}}%
\pgfpathlineto{\pgfqpoint{4.015914in}{1.043752in}}%
\pgfpathlineto{\pgfqpoint{4.017319in}{1.046245in}}%
\pgfpathlineto{\pgfqpoint{4.020510in}{1.048739in}}%
\pgfpathlineto{\pgfqpoint{4.025204in}{1.051232in}}%
\pgfpathlineto{\pgfqpoint{4.026416in}{1.053725in}}%
\pgfpathlineto{\pgfqpoint{4.045684in}{1.056218in}}%
\pgfpathlineto{\pgfqpoint{4.048007in}{1.058711in}}%
\pgfpathlineto{\pgfqpoint{4.050351in}{1.063697in}}%
\pgfpathlineto{\pgfqpoint{4.051190in}{1.066191in}}%
\pgfpathlineto{\pgfqpoint{4.056558in}{1.068684in}}%
\pgfpathlineto{\pgfqpoint{4.063557in}{1.071177in}}%
\pgfpathlineto{\pgfqpoint{4.089417in}{1.073670in}}%
\pgfpathlineto{\pgfqpoint{4.119691in}{1.076163in}}%
\pgfpathlineto{\pgfqpoint{4.130608in}{1.078656in}}%
\pgfpathlineto{\pgfqpoint{4.133697in}{1.081149in}}%
\pgfpathlineto{\pgfqpoint{4.184968in}{1.083642in}}%
\pgfpathlineto{\pgfqpoint{4.227660in}{1.086136in}}%
\pgfpathlineto{\pgfqpoint{4.234980in}{1.088629in}}%
\pgfpathlineto{\pgfqpoint{4.235392in}{1.091122in}}%
\pgfpathlineto{\pgfqpoint{4.239341in}{1.093615in}}%
\pgfpathlineto{\pgfqpoint{4.248169in}{1.096108in}}%
\pgfpathlineto{\pgfqpoint{4.278059in}{1.098601in}}%
\pgfpathlineto{\pgfqpoint{4.281115in}{1.101094in}}%
\pgfpathlineto{\pgfqpoint{4.344003in}{1.103588in}}%
\pgfpathlineto{\pgfqpoint{4.357710in}{1.106081in}}%
\pgfpathlineto{\pgfqpoint{4.382482in}{1.108574in}}%
\pgfpathlineto{\pgfqpoint{4.382847in}{1.111067in}}%
\pgfpathlineto{\pgfqpoint{4.386342in}{1.113560in}}%
\pgfpathlineto{\pgfqpoint{4.393279in}{1.116053in}}%
\pgfpathlineto{\pgfqpoint{4.394911in}{1.118546in}}%
\pgfpathlineto{\pgfqpoint{4.405451in}{1.121040in}}%
\pgfpathlineto{\pgfqpoint{4.469755in}{1.123533in}}%
\pgfpathlineto{\pgfqpoint{4.477034in}{1.126026in}}%
\pgfpathlineto{\pgfqpoint{4.477297in}{1.128519in}}%
\pgfpathlineto{\pgfqpoint{4.519182in}{1.131012in}}%
\pgfpathlineto{\pgfqpoint{4.519182in}{1.131012in}}%
\pgfusepath{stroke}%
\end{pgfscope}%
\begin{pgfscope}%
\pgfpathrectangle{\pgfqpoint{0.537394in}{0.467838in}}{\pgfqpoint{4.094684in}{0.997254in}}%
\pgfusepath{clip}%
\pgfsetbuttcap%
\pgfsetroundjoin%
\pgfsetlinewidth{1.003750pt}%
\definecolor{currentstroke}{rgb}{0.839216,0.152941,0.156863}%
\pgfsetstrokecolor{currentstroke}%
\pgfsetdash{{3.700000pt}{1.600000pt}}{0.000000pt}%
\pgfpathmoveto{\pgfqpoint{0.928126in}{0.495263in}}%
\pgfpathlineto{\pgfqpoint{1.030443in}{0.512715in}}%
\pgfpathlineto{\pgfqpoint{1.109806in}{0.525180in}}%
\pgfpathlineto{\pgfqpoint{1.174651in}{0.532660in}}%
\pgfpathlineto{\pgfqpoint{1.229476in}{0.545125in}}%
\pgfpathlineto{\pgfqpoint{1.276968in}{0.547619in}}%
\pgfpathlineto{\pgfqpoint{1.318858in}{0.557591in}}%
\pgfpathlineto{\pgfqpoint{1.421175in}{0.565071in}}%
\pgfpathlineto{\pgfqpoint{1.476000in}{0.570057in}}%
\pgfpathlineto{\pgfqpoint{1.523492in}{0.575043in}}%
\pgfpathlineto{\pgfqpoint{1.565383in}{0.580029in}}%
\pgfpathlineto{\pgfqpoint{1.584612in}{0.587509in}}%
\pgfpathlineto{\pgfqpoint{1.620208in}{0.590002in}}%
\pgfpathlineto{\pgfqpoint{1.652563in}{0.599974in}}%
\pgfpathlineto{\pgfqpoint{1.667700in}{0.607454in}}%
\pgfpathlineto{\pgfqpoint{1.682218in}{0.612440in}}%
\pgfpathlineto{\pgfqpoint{1.709590in}{0.614933in}}%
\pgfpathlineto{\pgfqpoint{1.722525in}{0.617426in}}%
\pgfpathlineto{\pgfqpoint{1.758725in}{0.622413in}}%
\pgfpathlineto{\pgfqpoint{1.791578in}{0.624906in}}%
\pgfpathlineto{\pgfqpoint{1.811907in}{0.634878in}}%
\pgfpathlineto{\pgfqpoint{1.821652in}{0.637371in}}%
\pgfpathlineto{\pgfqpoint{1.831137in}{0.644851in}}%
\pgfpathlineto{\pgfqpoint{1.840375in}{0.647344in}}%
\pgfpathlineto{\pgfqpoint{1.849380in}{0.652330in}}%
\pgfpathlineto{\pgfqpoint{1.858162in}{0.662303in}}%
\pgfpathlineto{\pgfqpoint{1.875101in}{0.667289in}}%
\pgfpathlineto{\pgfqpoint{1.883278in}{0.677262in}}%
\pgfpathlineto{\pgfqpoint{1.899087in}{0.682248in}}%
\pgfpathlineto{\pgfqpoint{1.914224in}{0.684741in}}%
\pgfpathlineto{\pgfqpoint{1.935786in}{0.692220in}}%
\pgfpathlineto{\pgfqpoint{1.942692in}{0.697207in}}%
\pgfpathlineto{\pgfqpoint{1.949467in}{0.699700in}}%
\pgfpathlineto{\pgfqpoint{1.981530in}{0.704686in}}%
\pgfpathlineto{\pgfqpoint{1.987610in}{0.707179in}}%
\pgfpathlineto{\pgfqpoint{1.999466in}{0.709672in}}%
\pgfpathlineto{\pgfqpoint{2.005249in}{0.717152in}}%
\pgfpathlineto{\pgfqpoint{2.058432in}{0.722138in}}%
\pgfpathlineto{\pgfqpoint{2.072951in}{0.727124in}}%
\pgfpathlineto{\pgfqpoint{2.091431in}{0.732111in}}%
\pgfpathlineto{\pgfqpoint{2.095904in}{0.734604in}}%
\pgfpathlineto{\pgfqpoint{2.104687in}{0.737097in}}%
\pgfpathlineto{\pgfqpoint{2.108998in}{0.742083in}}%
\pgfpathlineto{\pgfqpoint{2.121626in}{0.749563in}}%
\pgfpathlineto{\pgfqpoint{2.137795in}{0.752056in}}%
\pgfpathlineto{\pgfqpoint{2.145612in}{0.762028in}}%
\pgfpathlineto{\pgfqpoint{2.164434in}{0.767014in}}%
\pgfpathlineto{\pgfqpoint{2.178806in}{0.772001in}}%
\pgfpathlineto{\pgfqpoint{2.185780in}{0.779480in}}%
\pgfpathlineto{\pgfqpoint{2.189217in}{0.784466in}}%
\pgfpathlineto{\pgfqpoint{2.245991in}{0.786960in}}%
\pgfpathlineto{\pgfqpoint{2.248894in}{0.789453in}}%
\pgfpathlineto{\pgfqpoint{2.314701in}{0.791946in}}%
\pgfpathlineto{\pgfqpoint{2.324186in}{0.794439in}}%
\pgfpathlineto{\pgfqpoint{2.331137in}{0.796932in}}%
\pgfpathlineto{\pgfqpoint{2.337955in}{0.804411in}}%
\pgfpathlineto{\pgfqpoint{2.342429in}{0.806905in}}%
\pgfpathlineto{\pgfqpoint{2.351211in}{0.809398in}}%
\pgfpathlineto{\pgfqpoint{2.397889in}{0.811891in}}%
\pgfpathlineto{\pgfqpoint{2.414607in}{0.814384in}}%
\pgfpathlineto{\pgfqpoint{2.420009in}{0.816877in}}%
\pgfpathlineto{\pgfqpoint{2.423566in}{0.819370in}}%
\pgfpathlineto{\pgfqpoint{2.435741in}{0.821863in}}%
\pgfpathlineto{\pgfqpoint{2.437447in}{0.824357in}}%
\pgfpathlineto{\pgfqpoint{2.462099in}{0.826850in}}%
\pgfpathlineto{\pgfqpoint{2.474579in}{0.829343in}}%
\pgfpathlineto{\pgfqpoint{2.483660in}{0.831836in}}%
\pgfpathlineto{\pgfqpoint{2.517830in}{0.834329in}}%
\pgfpathlineto{\pgfqpoint{2.520534in}{0.839315in}}%
\pgfpathlineto{\pgfqpoint{2.546507in}{0.841808in}}%
\pgfpathlineto{\pgfqpoint{2.552714in}{0.844302in}}%
\pgfpathlineto{\pgfqpoint{2.562425in}{0.846795in}}%
\pgfpathlineto{\pgfqpoint{2.564812in}{0.849288in}}%
\pgfpathlineto{\pgfqpoint{2.566000in}{0.851781in}}%
\pgfpathlineto{\pgfqpoint{2.577662in}{0.854274in}}%
\pgfpathlineto{\pgfqpoint{2.584480in}{0.856767in}}%
\pgfpathlineto{\pgfqpoint{2.585603in}{0.859260in}}%
\pgfpathlineto{\pgfqpoint{2.595560in}{0.861754in}}%
\pgfpathlineto{\pgfqpoint{2.607363in}{0.864247in}}%
\pgfpathlineto{\pgfqpoint{2.611560in}{0.866740in}}%
\pgfpathlineto{\pgfqpoint{2.613640in}{0.871726in}}%
\pgfpathlineto{\pgfqpoint{2.615707in}{0.874219in}}%
\pgfpathlineto{\pgfqpoint{2.642506in}{0.876712in}}%
\pgfpathlineto{\pgfqpoint{2.650074in}{0.879205in}}%
\pgfpathlineto{\pgfqpoint{2.651008in}{0.881699in}}%
\pgfpathlineto{\pgfqpoint{2.654723in}{0.884192in}}%
\pgfpathlineto{\pgfqpoint{2.670090in}{0.886685in}}%
\pgfpathlineto{\pgfqpoint{2.691548in}{0.889178in}}%
\pgfpathlineto{\pgfqpoint{2.703022in}{0.891671in}}%
\pgfpathlineto{\pgfqpoint{2.727183in}{0.894164in}}%
\pgfpathlineto{\pgfqpoint{2.754723in}{0.896657in}}%
\pgfpathlineto{\pgfqpoint{2.762312in}{0.899151in}}%
\pgfpathlineto{\pgfqpoint{2.769074in}{0.901644in}}%
\pgfpathlineto{\pgfqpoint{2.779633in}{0.904137in}}%
\pgfpathlineto{\pgfqpoint{2.798005in}{0.906630in}}%
\pgfpathlineto{\pgfqpoint{2.803520in}{0.909123in}}%
\pgfpathlineto{\pgfqpoint{2.862232in}{0.911616in}}%
\pgfpathlineto{\pgfqpoint{2.883247in}{0.914109in}}%
\pgfpathlineto{\pgfqpoint{2.884218in}{0.916602in}}%
\pgfpathlineto{\pgfqpoint{2.886151in}{0.919096in}}%
\pgfpathlineto{\pgfqpoint{2.886632in}{0.921589in}}%
\pgfpathlineto{\pgfqpoint{2.891413in}{0.926575in}}%
\pgfpathlineto{\pgfqpoint{2.891887in}{0.929068in}}%
\pgfpathlineto{\pgfqpoint{2.901247in}{0.934054in}}%
\pgfpathlineto{\pgfqpoint{2.903090in}{0.936548in}}%
\pgfpathlineto{\pgfqpoint{2.903549in}{0.939041in}}%
\pgfpathlineto{\pgfqpoint{2.909014in}{0.944027in}}%
\pgfpathlineto{\pgfqpoint{2.917057in}{0.946520in}}%
\pgfpathlineto{\pgfqpoint{2.927934in}{0.951506in}}%
\pgfpathlineto{\pgfqpoint{2.929644in}{0.954000in}}%
\pgfpathlineto{\pgfqpoint{2.946306in}{0.956493in}}%
\pgfpathlineto{\pgfqpoint{2.978944in}{0.958986in}}%
\pgfpathlineto{\pgfqpoint{2.991706in}{0.961479in}}%
\pgfpathlineto{\pgfqpoint{3.000200in}{0.963972in}}%
\pgfpathlineto{\pgfqpoint{3.017602in}{0.966465in}}%
\pgfpathlineto{\pgfqpoint{3.056073in}{0.968958in}}%
\pgfpathlineto{\pgfqpoint{3.085046in}{0.971451in}}%
\pgfpathlineto{\pgfqpoint{3.144291in}{0.973945in}}%
\pgfpathlineto{\pgfqpoint{3.144524in}{0.976438in}}%
\pgfpathlineto{\pgfqpoint{3.163140in}{0.978931in}}%
\pgfpathlineto{\pgfqpoint{3.187707in}{0.981424in}}%
\pgfpathlineto{\pgfqpoint{3.321228in}{0.983917in}}%
\pgfpathlineto{\pgfqpoint{3.327832in}{0.986410in}}%
\pgfpathlineto{\pgfqpoint{3.329914in}{0.988903in}}%
\pgfpathlineto{\pgfqpoint{3.339606in}{0.991397in}}%
\pgfpathlineto{\pgfqpoint{3.355539in}{0.993890in}}%
\pgfpathlineto{\pgfqpoint{3.388475in}{0.996383in}}%
\pgfpathlineto{\pgfqpoint{3.415042in}{0.998876in}}%
\pgfpathlineto{\pgfqpoint{3.424819in}{1.001369in}}%
\pgfpathlineto{\pgfqpoint{3.438948in}{1.003862in}}%
\pgfpathlineto{\pgfqpoint{3.440371in}{1.006355in}}%
\pgfpathlineto{\pgfqpoint{3.476237in}{1.008848in}}%
\pgfpathlineto{\pgfqpoint{3.477701in}{1.011342in}}%
\pgfpathlineto{\pgfqpoint{3.481878in}{1.013835in}}%
\pgfpathlineto{\pgfqpoint{3.493074in}{1.016328in}}%
\pgfpathlineto{\pgfqpoint{3.551950in}{1.018821in}}%
\pgfpathlineto{\pgfqpoint{3.578508in}{1.023807in}}%
\pgfpathlineto{\pgfqpoint{3.633824in}{1.026300in}}%
\pgfpathlineto{\pgfqpoint{3.647739in}{1.028794in}}%
\pgfpathlineto{\pgfqpoint{3.651905in}{1.031287in}}%
\pgfpathlineto{\pgfqpoint{3.659436in}{1.033780in}}%
\pgfpathlineto{\pgfqpoint{3.676290in}{1.036273in}}%
\pgfpathlineto{\pgfqpoint{3.743340in}{1.038766in}}%
\pgfpathlineto{\pgfqpoint{3.877926in}{1.041259in}}%
\pgfpathlineto{\pgfqpoint{3.880290in}{1.043752in}}%
\pgfpathlineto{\pgfqpoint{3.884127in}{1.046245in}}%
\pgfpathlineto{\pgfqpoint{3.989150in}{1.048739in}}%
\pgfpathlineto{\pgfqpoint{4.000481in}{1.051232in}}%
\pgfpathlineto{\pgfqpoint{4.009194in}{1.053725in}}%
\pgfpathlineto{\pgfqpoint{4.171761in}{1.056218in}}%
\pgfpathlineto{\pgfqpoint{4.191632in}{1.058711in}}%
\pgfpathlineto{\pgfqpoint{4.449615in}{1.061204in}}%
\pgfpathlineto{\pgfqpoint{4.455559in}{1.063697in}}%
\pgfpathlineto{\pgfqpoint{4.476103in}{1.066191in}}%
\pgfpathlineto{\pgfqpoint{4.580095in}{1.068684in}}%
\pgfpathlineto{\pgfqpoint{4.580103in}{1.071177in}}%
\pgfpathlineto{\pgfqpoint{4.590132in}{1.073670in}}%
\pgfpathlineto{\pgfqpoint{4.607935in}{1.076163in}}%
\pgfpathlineto{\pgfqpoint{4.627781in}{1.078656in}}%
\pgfpathlineto{\pgfqpoint{4.627781in}{1.078656in}}%
\pgfusepath{stroke}%
\end{pgfscope}%
\begin{pgfscope}%
\pgfsetrectcap%
\pgfsetmiterjoin%
\pgfsetlinewidth{0.803000pt}%
\definecolor{currentstroke}{rgb}{0.000000,0.000000,0.000000}%
\pgfsetstrokecolor{currentstroke}%
\pgfsetdash{}{0pt}%
\pgfpathmoveto{\pgfqpoint{0.537394in}{0.467838in}}%
\pgfpathlineto{\pgfqpoint{0.537394in}{1.465092in}}%
\pgfusepath{stroke}%
\end{pgfscope}%
\begin{pgfscope}%
\pgfsetrectcap%
\pgfsetmiterjoin%
\pgfsetlinewidth{0.803000pt}%
\definecolor{currentstroke}{rgb}{0.000000,0.000000,0.000000}%
\pgfsetstrokecolor{currentstroke}%
\pgfsetdash{}{0pt}%
\pgfpathmoveto{\pgfqpoint{4.632078in}{0.467838in}}%
\pgfpathlineto{\pgfqpoint{4.632078in}{1.465092in}}%
\pgfusepath{stroke}%
\end{pgfscope}%
\begin{pgfscope}%
\pgfsetrectcap%
\pgfsetmiterjoin%
\pgfsetlinewidth{0.803000pt}%
\definecolor{currentstroke}{rgb}{0.000000,0.000000,0.000000}%
\pgfsetstrokecolor{currentstroke}%
\pgfsetdash{}{0pt}%
\pgfpathmoveto{\pgfqpoint{0.537394in}{0.467838in}}%
\pgfpathlineto{\pgfqpoint{4.632078in}{0.467838in}}%
\pgfusepath{stroke}%
\end{pgfscope}%
\begin{pgfscope}%
\pgfsetrectcap%
\pgfsetmiterjoin%
\pgfsetlinewidth{0.803000pt}%
\definecolor{currentstroke}{rgb}{0.000000,0.000000,0.000000}%
\pgfsetstrokecolor{currentstroke}%
\pgfsetdash{}{0pt}%
\pgfpathmoveto{\pgfqpoint{0.537394in}{1.465092in}}%
\pgfpathlineto{\pgfqpoint{4.632078in}{1.465092in}}%
\pgfusepath{stroke}%
\end{pgfscope}%
\begin{pgfscope}%
\pgfsetbuttcap%
\pgfsetmiterjoin%
\definecolor{currentfill}{rgb}{1.000000,1.000000,1.000000}%
\pgfsetfillcolor{currentfill}%
\pgfsetfillopacity{0.800000}%
\pgfsetlinewidth{1.003750pt}%
\definecolor{currentstroke}{rgb}{0.800000,0.800000,0.800000}%
\pgfsetstrokecolor{currentstroke}%
\pgfsetstrokeopacity{0.800000}%
\pgfsetdash{}{0pt}%
\pgfpathmoveto{\pgfqpoint{0.615172in}{0.723860in}}%
\pgfpathlineto{\pgfqpoint{1.560408in}{0.723860in}}%
\pgfpathquadraticcurveto{\pgfqpoint{1.582630in}{0.723860in}}{\pgfqpoint{1.582630in}{0.746082in}}%
\pgfpathlineto{\pgfqpoint{1.582630in}{1.387314in}}%
\pgfpathquadraticcurveto{\pgfqpoint{1.582630in}{1.409536in}}{\pgfqpoint{1.560408in}{1.409536in}}%
\pgfpathlineto{\pgfqpoint{0.615172in}{1.409536in}}%
\pgfpathquadraticcurveto{\pgfqpoint{0.592949in}{1.409536in}}{\pgfqpoint{0.592949in}{1.387314in}}%
\pgfpathlineto{\pgfqpoint{0.592949in}{0.746082in}}%
\pgfpathquadraticcurveto{\pgfqpoint{0.592949in}{0.723860in}}{\pgfqpoint{0.615172in}{0.723860in}}%
\pgfpathclose%
\pgfusepath{stroke,fill}%
\end{pgfscope}%
\begin{pgfscope}%
\pgfsetrectcap%
\pgfsetroundjoin%
\pgfsetlinewidth{1.003750pt}%
\definecolor{currentstroke}{rgb}{0.121569,0.466667,0.705882}%
\pgfsetstrokecolor{currentstroke}%
\pgfsetdash{}{0pt}%
\pgfpathmoveto{\pgfqpoint{0.637394in}{1.319562in}}%
\pgfpathlineto{\pgfqpoint{0.859616in}{1.319562in}}%
\pgfusepath{stroke}%
\end{pgfscope}%
\begin{pgfscope}%
\definecolor{textcolor}{rgb}{0.000000,0.000000,0.000000}%
\pgfsetstrokecolor{textcolor}%
\pgfsetfillcolor{textcolor}%
\pgftext[x=0.948505in,y=1.280674in,left,base]{\color{textcolor}\sffamily\fontsize{8.000000}{9.600000}\selectfont DMC(MCS)}%
\end{pgfscope}%
\begin{pgfscope}%
\pgfsetrectcap%
\pgfsetroundjoin%
\pgfsetlinewidth{1.003750pt}%
\definecolor{currentstroke}{rgb}{1.000000,0.498039,0.054902}%
\pgfsetstrokecolor{currentstroke}%
\pgfsetdash{}{0pt}%
\pgfpathmoveto{\pgfqpoint{0.637394in}{1.156477in}}%
\pgfpathlineto{\pgfqpoint{0.859616in}{1.156477in}}%
\pgfusepath{stroke}%
\end{pgfscope}%
\begin{pgfscope}%
\definecolor{textcolor}{rgb}{0.000000,0.000000,0.000000}%
\pgfsetstrokecolor{textcolor}%
\pgfsetfillcolor{textcolor}%
\pgftext[x=0.948505in,y=1.117588in,left,base]{\color{textcolor}\sffamily\fontsize{8.000000}{9.600000}\selectfont DMC(LP)}%
\end{pgfscope}%
\begin{pgfscope}%
\pgfsetbuttcap%
\pgfsetroundjoin%
\pgfsetlinewidth{1.003750pt}%
\definecolor{currentstroke}{rgb}{0.172549,0.627451,0.172549}%
\pgfsetstrokecolor{currentstroke}%
\pgfsetdash{{3.700000pt}{1.600000pt}}{0.000000pt}%
\pgfpathmoveto{\pgfqpoint{0.637394in}{0.993391in}}%
\pgfpathlineto{\pgfqpoint{0.859616in}{0.993391in}}%
\pgfusepath{stroke}%
\end{pgfscope}%
\begin{pgfscope}%
\definecolor{textcolor}{rgb}{0.000000,0.000000,0.000000}%
\pgfsetstrokecolor{textcolor}%
\pgfsetfillcolor{textcolor}%
\pgftext[x=0.948505in,y=0.954502in,left,base]{\color{textcolor}\sffamily\fontsize{8.000000}{9.600000}\selectfont DMC(LM)}%
\end{pgfscope}%
\begin{pgfscope}%
\pgfsetbuttcap%
\pgfsetroundjoin%
\pgfsetlinewidth{1.003750pt}%
\definecolor{currentstroke}{rgb}{0.839216,0.152941,0.156863}%
\pgfsetstrokecolor{currentstroke}%
\pgfsetdash{{3.700000pt}{1.600000pt}}{0.000000pt}%
\pgfpathmoveto{\pgfqpoint{0.637394in}{0.830305in}}%
\pgfpathlineto{\pgfqpoint{0.859616in}{0.830305in}}%
\pgfusepath{stroke}%
\end{pgfscope}%
\begin{pgfscope}%
\definecolor{textcolor}{rgb}{0.000000,0.000000,0.000000}%
\pgfsetstrokecolor{textcolor}%
\pgfsetfillcolor{textcolor}%
\pgftext[x=0.948505in,y=0.791416in,left,base]{\color{textcolor}\sffamily\fontsize{8.000000}{9.600000}\selectfont DMC(MF)}%
\end{pgfscope}%
\end{pgfpicture}%
\makeatother%
\endgroup%

    \caption{
        Experiment 2 compares variable-ordering heuristics (\mcs{}, \lexp, \lexm, and \minfill{}) for the ADD-based executor \dmc.
        \mcs{} and \lexp{} are significantly faster than \lexm{} and \minfill{}.
        % The graded project-join trees here were produced by the planner \Lg{} with the tree decomposer \flowcutter{} from Experiment 1. %JD: Don't need to repeat this
    }
    \label{figExecution}
\end{figure}

%%%%%%%%%%%%%%%%%%%%%%%%%%%%%%%%%%%%%%%%%%%%%%%%%%%%%%%%%%%%%%%%%%%%%%%%%%%%%%%%

\subsection{Experiment 3: Comparing Weighted Projected Model Counters}

Informed by Experiments 1 and 2, we choose \Lg{} with \flowcutter{} as the planner and \dmc{} with \mcs{} as the executor for our framework \procount.
We compare \procount{} with the weighted projected model counters \dfp{}, \projmc{}, and \ssat{}.
Since all benchmarks are satisfiable
% (checked by the SAT solver \sat{} \cite{soos2009extending})
with positive literal weights, the model counts must be positive.
Thus, for all tools, we exclude outputs that are zero (possible floating-point underflow).
We are confident that the remaining results are correct.
Differences in model counts among tools are less than $10^{-6}$.

Figure \ref{figSolving} shows the performance of \procount{}, \dfp{}, \projmc{}, and \ssat{} with a 1000-second timeout. 
Additional statistics are given in Table \ref{tableSolving}. 
Of \benchmarks{} benchmarks, 390 are solved by at least one of four tools.
\procount{} achieves the shortest solving time on 131 benchmarks, including \dpmcUniqueBenchmarks{} solved by none of the other three tools.
Between the two \emph{virtual best solvers} in Figure \ref{figSolving}, \vbs1 (all four tools) is significantly faster than \vbs0 (three existing tools, without \procount).
\begin{figure}[t]
    \centering
    %% Creator: Matplotlib, PGF backend
%%
%% To include the figure in your LaTeX document, write
%%   \input{<filename>.pgf}
%%
%% Make sure the required packages are loaded in your preamble
%%   \usepackage{pgf}
%%
%% and, on pdftex
%%   \usepackage[utf8]{inputenc}\DeclareUnicodeCharacter{2212}{-}
%%
%% or, on luatex and xetex
%%   \usepackage{unicode-math}
%%
%% Figures using additional raster images can only be included by \input if
%% they are in the same directory as the main LaTeX file. For loading figures
%% from other directories you can use the `import` package
%%   \usepackage{import}
%%
%% and then include the figures with
%%   \import{<path to file>}{<filename>.pgf}
%%
%% Matplotlib used the following preamble
%%   \usepackage{fontspec}
%%   \setmainfont{DejaVuSerif.ttf}[Path=/home/vhp1/.local/lib/python3.8/site-packages/matplotlib/mpl-data/fonts/ttf/]
%%   \setsansfont{DejaVuSans.ttf}[Path=/home/vhp1/.local/lib/python3.8/site-packages/matplotlib/mpl-data/fonts/ttf/]
%%   \setmonofont{DejaVuSansMono.ttf}[Path=/home/vhp1/.local/lib/python3.8/site-packages/matplotlib/mpl-data/fonts/ttf/]
%%
\begingroup%
\makeatletter%
\begin{pgfpicture}%
\pgfpathrectangle{\pgfpointorigin}{\pgfqpoint{4.820041in}{2.205653in}}%
\pgfusepath{use as bounding box, clip}%
\begin{pgfscope}%
\pgfsetbuttcap%
\pgfsetmiterjoin%
\pgfsetlinewidth{0.000000pt}%
\definecolor{currentstroke}{rgb}{1.000000,1.000000,1.000000}%
\pgfsetstrokecolor{currentstroke}%
\pgfsetstrokeopacity{0.000000}%
\pgfsetdash{}{0pt}%
\pgfpathmoveto{\pgfqpoint{0.000000in}{0.000000in}}%
\pgfpathlineto{\pgfqpoint{4.820041in}{0.000000in}}%
\pgfpathlineto{\pgfqpoint{4.820041in}{2.205653in}}%
\pgfpathlineto{\pgfqpoint{0.000000in}{2.205653in}}%
\pgfpathclose%
\pgfusepath{}%
\end{pgfscope}%
\begin{pgfscope}%
\pgfsetbuttcap%
\pgfsetmiterjoin%
\definecolor{currentfill}{rgb}{1.000000,1.000000,1.000000}%
\pgfsetfillcolor{currentfill}%
\pgfsetlinewidth{0.000000pt}%
\definecolor{currentstroke}{rgb}{0.000000,0.000000,0.000000}%
\pgfsetstrokecolor{currentstroke}%
\pgfsetstrokeopacity{0.000000}%
\pgfsetdash{}{0pt}%
\pgfpathmoveto{\pgfqpoint{0.537394in}{0.467838in}}%
\pgfpathlineto{\pgfqpoint{4.632078in}{0.467838in}}%
\pgfpathlineto{\pgfqpoint{4.632078in}{2.063444in}}%
\pgfpathlineto{\pgfqpoint{0.537394in}{2.063444in}}%
\pgfpathclose%
\pgfusepath{fill}%
\end{pgfscope}%
\begin{pgfscope}%
\pgfsetbuttcap%
\pgfsetroundjoin%
\definecolor{currentfill}{rgb}{0.000000,0.000000,0.000000}%
\pgfsetfillcolor{currentfill}%
\pgfsetlinewidth{0.803000pt}%
\definecolor{currentstroke}{rgb}{0.000000,0.000000,0.000000}%
\pgfsetstrokecolor{currentstroke}%
\pgfsetdash{}{0pt}%
\pgfsys@defobject{currentmarker}{\pgfqpoint{0.000000in}{-0.048611in}}{\pgfqpoint{0.000000in}{0.000000in}}{%
\pgfpathmoveto{\pgfqpoint{0.000000in}{0.000000in}}%
\pgfpathlineto{\pgfqpoint{0.000000in}{-0.048611in}}%
\pgfusepath{stroke,fill}%
}%
\begin{pgfscope}%
\pgfsys@transformshift{0.537394in}{0.467838in}%
\pgfsys@useobject{currentmarker}{}%
\end{pgfscope}%
\end{pgfscope}%
\begin{pgfscope}%
\definecolor{textcolor}{rgb}{0.000000,0.000000,0.000000}%
\pgfsetstrokecolor{textcolor}%
\pgfsetfillcolor{textcolor}%
\pgftext[x=0.537394in,y=0.370616in,,top]{\color{textcolor}\sffamily\fontsize{8.000000}{9.600000}\selectfont \(\displaystyle {10^{-3}}\)}%
\end{pgfscope}%
\begin{pgfscope}%
\pgfsetbuttcap%
\pgfsetroundjoin%
\definecolor{currentfill}{rgb}{0.000000,0.000000,0.000000}%
\pgfsetfillcolor{currentfill}%
\pgfsetlinewidth{0.803000pt}%
\definecolor{currentstroke}{rgb}{0.000000,0.000000,0.000000}%
\pgfsetstrokecolor{currentstroke}%
\pgfsetdash{}{0pt}%
\pgfsys@defobject{currentmarker}{\pgfqpoint{0.000000in}{-0.048611in}}{\pgfqpoint{0.000000in}{0.000000in}}{%
\pgfpathmoveto{\pgfqpoint{0.000000in}{0.000000in}}%
\pgfpathlineto{\pgfqpoint{0.000000in}{-0.048611in}}%
\pgfusepath{stroke,fill}%
}%
\begin{pgfscope}%
\pgfsys@transformshift{1.219841in}{0.467838in}%
\pgfsys@useobject{currentmarker}{}%
\end{pgfscope}%
\end{pgfscope}%
\begin{pgfscope}%
\definecolor{textcolor}{rgb}{0.000000,0.000000,0.000000}%
\pgfsetstrokecolor{textcolor}%
\pgfsetfillcolor{textcolor}%
\pgftext[x=1.219841in,y=0.370616in,,top]{\color{textcolor}\sffamily\fontsize{8.000000}{9.600000}\selectfont \(\displaystyle {10^{-2}}\)}%
\end{pgfscope}%
\begin{pgfscope}%
\pgfsetbuttcap%
\pgfsetroundjoin%
\definecolor{currentfill}{rgb}{0.000000,0.000000,0.000000}%
\pgfsetfillcolor{currentfill}%
\pgfsetlinewidth{0.803000pt}%
\definecolor{currentstroke}{rgb}{0.000000,0.000000,0.000000}%
\pgfsetstrokecolor{currentstroke}%
\pgfsetdash{}{0pt}%
\pgfsys@defobject{currentmarker}{\pgfqpoint{0.000000in}{-0.048611in}}{\pgfqpoint{0.000000in}{0.000000in}}{%
\pgfpathmoveto{\pgfqpoint{0.000000in}{0.000000in}}%
\pgfpathlineto{\pgfqpoint{0.000000in}{-0.048611in}}%
\pgfusepath{stroke,fill}%
}%
\begin{pgfscope}%
\pgfsys@transformshift{1.902289in}{0.467838in}%
\pgfsys@useobject{currentmarker}{}%
\end{pgfscope}%
\end{pgfscope}%
\begin{pgfscope}%
\definecolor{textcolor}{rgb}{0.000000,0.000000,0.000000}%
\pgfsetstrokecolor{textcolor}%
\pgfsetfillcolor{textcolor}%
\pgftext[x=1.902289in,y=0.370616in,,top]{\color{textcolor}\sffamily\fontsize{8.000000}{9.600000}\selectfont \(\displaystyle {10^{-1}}\)}%
\end{pgfscope}%
\begin{pgfscope}%
\pgfsetbuttcap%
\pgfsetroundjoin%
\definecolor{currentfill}{rgb}{0.000000,0.000000,0.000000}%
\pgfsetfillcolor{currentfill}%
\pgfsetlinewidth{0.803000pt}%
\definecolor{currentstroke}{rgb}{0.000000,0.000000,0.000000}%
\pgfsetstrokecolor{currentstroke}%
\pgfsetdash{}{0pt}%
\pgfsys@defobject{currentmarker}{\pgfqpoint{0.000000in}{-0.048611in}}{\pgfqpoint{0.000000in}{0.000000in}}{%
\pgfpathmoveto{\pgfqpoint{0.000000in}{0.000000in}}%
\pgfpathlineto{\pgfqpoint{0.000000in}{-0.048611in}}%
\pgfusepath{stroke,fill}%
}%
\begin{pgfscope}%
\pgfsys@transformshift{2.584736in}{0.467838in}%
\pgfsys@useobject{currentmarker}{}%
\end{pgfscope}%
\end{pgfscope}%
\begin{pgfscope}%
\definecolor{textcolor}{rgb}{0.000000,0.000000,0.000000}%
\pgfsetstrokecolor{textcolor}%
\pgfsetfillcolor{textcolor}%
\pgftext[x=2.584736in,y=0.370616in,,top]{\color{textcolor}\sffamily\fontsize{8.000000}{9.600000}\selectfont \(\displaystyle {10^{0}}\)}%
\end{pgfscope}%
\begin{pgfscope}%
\pgfsetbuttcap%
\pgfsetroundjoin%
\definecolor{currentfill}{rgb}{0.000000,0.000000,0.000000}%
\pgfsetfillcolor{currentfill}%
\pgfsetlinewidth{0.803000pt}%
\definecolor{currentstroke}{rgb}{0.000000,0.000000,0.000000}%
\pgfsetstrokecolor{currentstroke}%
\pgfsetdash{}{0pt}%
\pgfsys@defobject{currentmarker}{\pgfqpoint{0.000000in}{-0.048611in}}{\pgfqpoint{0.000000in}{0.000000in}}{%
\pgfpathmoveto{\pgfqpoint{0.000000in}{0.000000in}}%
\pgfpathlineto{\pgfqpoint{0.000000in}{-0.048611in}}%
\pgfusepath{stroke,fill}%
}%
\begin{pgfscope}%
\pgfsys@transformshift{3.267183in}{0.467838in}%
\pgfsys@useobject{currentmarker}{}%
\end{pgfscope}%
\end{pgfscope}%
\begin{pgfscope}%
\definecolor{textcolor}{rgb}{0.000000,0.000000,0.000000}%
\pgfsetstrokecolor{textcolor}%
\pgfsetfillcolor{textcolor}%
\pgftext[x=3.267183in,y=0.370616in,,top]{\color{textcolor}\sffamily\fontsize{8.000000}{9.600000}\selectfont \(\displaystyle {10^{1}}\)}%
\end{pgfscope}%
\begin{pgfscope}%
\pgfsetbuttcap%
\pgfsetroundjoin%
\definecolor{currentfill}{rgb}{0.000000,0.000000,0.000000}%
\pgfsetfillcolor{currentfill}%
\pgfsetlinewidth{0.803000pt}%
\definecolor{currentstroke}{rgb}{0.000000,0.000000,0.000000}%
\pgfsetstrokecolor{currentstroke}%
\pgfsetdash{}{0pt}%
\pgfsys@defobject{currentmarker}{\pgfqpoint{0.000000in}{-0.048611in}}{\pgfqpoint{0.000000in}{0.000000in}}{%
\pgfpathmoveto{\pgfqpoint{0.000000in}{0.000000in}}%
\pgfpathlineto{\pgfqpoint{0.000000in}{-0.048611in}}%
\pgfusepath{stroke,fill}%
}%
\begin{pgfscope}%
\pgfsys@transformshift{3.949631in}{0.467838in}%
\pgfsys@useobject{currentmarker}{}%
\end{pgfscope}%
\end{pgfscope}%
\begin{pgfscope}%
\definecolor{textcolor}{rgb}{0.000000,0.000000,0.000000}%
\pgfsetstrokecolor{textcolor}%
\pgfsetfillcolor{textcolor}%
\pgftext[x=3.949631in,y=0.370616in,,top]{\color{textcolor}\sffamily\fontsize{8.000000}{9.600000}\selectfont \(\displaystyle {10^{2}}\)}%
\end{pgfscope}%
\begin{pgfscope}%
\pgfsetbuttcap%
\pgfsetroundjoin%
\definecolor{currentfill}{rgb}{0.000000,0.000000,0.000000}%
\pgfsetfillcolor{currentfill}%
\pgfsetlinewidth{0.803000pt}%
\definecolor{currentstroke}{rgb}{0.000000,0.000000,0.000000}%
\pgfsetstrokecolor{currentstroke}%
\pgfsetdash{}{0pt}%
\pgfsys@defobject{currentmarker}{\pgfqpoint{0.000000in}{-0.048611in}}{\pgfqpoint{0.000000in}{0.000000in}}{%
\pgfpathmoveto{\pgfqpoint{0.000000in}{0.000000in}}%
\pgfpathlineto{\pgfqpoint{0.000000in}{-0.048611in}}%
\pgfusepath{stroke,fill}%
}%
\begin{pgfscope}%
\pgfsys@transformshift{4.632078in}{0.467838in}%
\pgfsys@useobject{currentmarker}{}%
\end{pgfscope}%
\end{pgfscope}%
\begin{pgfscope}%
\definecolor{textcolor}{rgb}{0.000000,0.000000,0.000000}%
\pgfsetstrokecolor{textcolor}%
\pgfsetfillcolor{textcolor}%
\pgftext[x=4.632078in,y=0.370616in,,top]{\color{textcolor}\sffamily\fontsize{8.000000}{9.600000}\selectfont \(\displaystyle {10^{3}}\)}%
\end{pgfscope}%
\begin{pgfscope}%
\pgfsetbuttcap%
\pgfsetroundjoin%
\definecolor{currentfill}{rgb}{0.000000,0.000000,0.000000}%
\pgfsetfillcolor{currentfill}%
\pgfsetlinewidth{0.602250pt}%
\definecolor{currentstroke}{rgb}{0.000000,0.000000,0.000000}%
\pgfsetstrokecolor{currentstroke}%
\pgfsetdash{}{0pt}%
\pgfsys@defobject{currentmarker}{\pgfqpoint{0.000000in}{-0.027778in}}{\pgfqpoint{0.000000in}{0.000000in}}{%
\pgfpathmoveto{\pgfqpoint{0.000000in}{0.000000in}}%
\pgfpathlineto{\pgfqpoint{0.000000in}{-0.027778in}}%
\pgfusepath{stroke,fill}%
}%
\begin{pgfscope}%
\pgfsys@transformshift{0.742831in}{0.467838in}%
\pgfsys@useobject{currentmarker}{}%
\end{pgfscope}%
\end{pgfscope}%
\begin{pgfscope}%
\pgfsetbuttcap%
\pgfsetroundjoin%
\definecolor{currentfill}{rgb}{0.000000,0.000000,0.000000}%
\pgfsetfillcolor{currentfill}%
\pgfsetlinewidth{0.602250pt}%
\definecolor{currentstroke}{rgb}{0.000000,0.000000,0.000000}%
\pgfsetstrokecolor{currentstroke}%
\pgfsetdash{}{0pt}%
\pgfsys@defobject{currentmarker}{\pgfqpoint{0.000000in}{-0.027778in}}{\pgfqpoint{0.000000in}{0.000000in}}{%
\pgfpathmoveto{\pgfqpoint{0.000000in}{0.000000in}}%
\pgfpathlineto{\pgfqpoint{0.000000in}{-0.027778in}}%
\pgfusepath{stroke,fill}%
}%
\begin{pgfscope}%
\pgfsys@transformshift{0.863004in}{0.467838in}%
\pgfsys@useobject{currentmarker}{}%
\end{pgfscope}%
\end{pgfscope}%
\begin{pgfscope}%
\pgfsetbuttcap%
\pgfsetroundjoin%
\definecolor{currentfill}{rgb}{0.000000,0.000000,0.000000}%
\pgfsetfillcolor{currentfill}%
\pgfsetlinewidth{0.602250pt}%
\definecolor{currentstroke}{rgb}{0.000000,0.000000,0.000000}%
\pgfsetstrokecolor{currentstroke}%
\pgfsetdash{}{0pt}%
\pgfsys@defobject{currentmarker}{\pgfqpoint{0.000000in}{-0.027778in}}{\pgfqpoint{0.000000in}{0.000000in}}{%
\pgfpathmoveto{\pgfqpoint{0.000000in}{0.000000in}}%
\pgfpathlineto{\pgfqpoint{0.000000in}{-0.027778in}}%
\pgfusepath{stroke,fill}%
}%
\begin{pgfscope}%
\pgfsys@transformshift{0.948268in}{0.467838in}%
\pgfsys@useobject{currentmarker}{}%
\end{pgfscope}%
\end{pgfscope}%
\begin{pgfscope}%
\pgfsetbuttcap%
\pgfsetroundjoin%
\definecolor{currentfill}{rgb}{0.000000,0.000000,0.000000}%
\pgfsetfillcolor{currentfill}%
\pgfsetlinewidth{0.602250pt}%
\definecolor{currentstroke}{rgb}{0.000000,0.000000,0.000000}%
\pgfsetstrokecolor{currentstroke}%
\pgfsetdash{}{0pt}%
\pgfsys@defobject{currentmarker}{\pgfqpoint{0.000000in}{-0.027778in}}{\pgfqpoint{0.000000in}{0.000000in}}{%
\pgfpathmoveto{\pgfqpoint{0.000000in}{0.000000in}}%
\pgfpathlineto{\pgfqpoint{0.000000in}{-0.027778in}}%
\pgfusepath{stroke,fill}%
}%
\begin{pgfscope}%
\pgfsys@transformshift{1.014404in}{0.467838in}%
\pgfsys@useobject{currentmarker}{}%
\end{pgfscope}%
\end{pgfscope}%
\begin{pgfscope}%
\pgfsetbuttcap%
\pgfsetroundjoin%
\definecolor{currentfill}{rgb}{0.000000,0.000000,0.000000}%
\pgfsetfillcolor{currentfill}%
\pgfsetlinewidth{0.602250pt}%
\definecolor{currentstroke}{rgb}{0.000000,0.000000,0.000000}%
\pgfsetstrokecolor{currentstroke}%
\pgfsetdash{}{0pt}%
\pgfsys@defobject{currentmarker}{\pgfqpoint{0.000000in}{-0.027778in}}{\pgfqpoint{0.000000in}{0.000000in}}{%
\pgfpathmoveto{\pgfqpoint{0.000000in}{0.000000in}}%
\pgfpathlineto{\pgfqpoint{0.000000in}{-0.027778in}}%
\pgfusepath{stroke,fill}%
}%
\begin{pgfscope}%
\pgfsys@transformshift{1.068441in}{0.467838in}%
\pgfsys@useobject{currentmarker}{}%
\end{pgfscope}%
\end{pgfscope}%
\begin{pgfscope}%
\pgfsetbuttcap%
\pgfsetroundjoin%
\definecolor{currentfill}{rgb}{0.000000,0.000000,0.000000}%
\pgfsetfillcolor{currentfill}%
\pgfsetlinewidth{0.602250pt}%
\definecolor{currentstroke}{rgb}{0.000000,0.000000,0.000000}%
\pgfsetstrokecolor{currentstroke}%
\pgfsetdash{}{0pt}%
\pgfsys@defobject{currentmarker}{\pgfqpoint{0.000000in}{-0.027778in}}{\pgfqpoint{0.000000in}{0.000000in}}{%
\pgfpathmoveto{\pgfqpoint{0.000000in}{0.000000in}}%
\pgfpathlineto{\pgfqpoint{0.000000in}{-0.027778in}}%
\pgfusepath{stroke,fill}%
}%
\begin{pgfscope}%
\pgfsys@transformshift{1.114129in}{0.467838in}%
\pgfsys@useobject{currentmarker}{}%
\end{pgfscope}%
\end{pgfscope}%
\begin{pgfscope}%
\pgfsetbuttcap%
\pgfsetroundjoin%
\definecolor{currentfill}{rgb}{0.000000,0.000000,0.000000}%
\pgfsetfillcolor{currentfill}%
\pgfsetlinewidth{0.602250pt}%
\definecolor{currentstroke}{rgb}{0.000000,0.000000,0.000000}%
\pgfsetstrokecolor{currentstroke}%
\pgfsetdash{}{0pt}%
\pgfsys@defobject{currentmarker}{\pgfqpoint{0.000000in}{-0.027778in}}{\pgfqpoint{0.000000in}{0.000000in}}{%
\pgfpathmoveto{\pgfqpoint{0.000000in}{0.000000in}}%
\pgfpathlineto{\pgfqpoint{0.000000in}{-0.027778in}}%
\pgfusepath{stroke,fill}%
}%
\begin{pgfscope}%
\pgfsys@transformshift{1.153705in}{0.467838in}%
\pgfsys@useobject{currentmarker}{}%
\end{pgfscope}%
\end{pgfscope}%
\begin{pgfscope}%
\pgfsetbuttcap%
\pgfsetroundjoin%
\definecolor{currentfill}{rgb}{0.000000,0.000000,0.000000}%
\pgfsetfillcolor{currentfill}%
\pgfsetlinewidth{0.602250pt}%
\definecolor{currentstroke}{rgb}{0.000000,0.000000,0.000000}%
\pgfsetstrokecolor{currentstroke}%
\pgfsetdash{}{0pt}%
\pgfsys@defobject{currentmarker}{\pgfqpoint{0.000000in}{-0.027778in}}{\pgfqpoint{0.000000in}{0.000000in}}{%
\pgfpathmoveto{\pgfqpoint{0.000000in}{0.000000in}}%
\pgfpathlineto{\pgfqpoint{0.000000in}{-0.027778in}}%
\pgfusepath{stroke,fill}%
}%
\begin{pgfscope}%
\pgfsys@transformshift{1.188614in}{0.467838in}%
\pgfsys@useobject{currentmarker}{}%
\end{pgfscope}%
\end{pgfscope}%
\begin{pgfscope}%
\pgfsetbuttcap%
\pgfsetroundjoin%
\definecolor{currentfill}{rgb}{0.000000,0.000000,0.000000}%
\pgfsetfillcolor{currentfill}%
\pgfsetlinewidth{0.602250pt}%
\definecolor{currentstroke}{rgb}{0.000000,0.000000,0.000000}%
\pgfsetstrokecolor{currentstroke}%
\pgfsetdash{}{0pt}%
\pgfsys@defobject{currentmarker}{\pgfqpoint{0.000000in}{-0.027778in}}{\pgfqpoint{0.000000in}{0.000000in}}{%
\pgfpathmoveto{\pgfqpoint{0.000000in}{0.000000in}}%
\pgfpathlineto{\pgfqpoint{0.000000in}{-0.027778in}}%
\pgfusepath{stroke,fill}%
}%
\begin{pgfscope}%
\pgfsys@transformshift{1.425278in}{0.467838in}%
\pgfsys@useobject{currentmarker}{}%
\end{pgfscope}%
\end{pgfscope}%
\begin{pgfscope}%
\pgfsetbuttcap%
\pgfsetroundjoin%
\definecolor{currentfill}{rgb}{0.000000,0.000000,0.000000}%
\pgfsetfillcolor{currentfill}%
\pgfsetlinewidth{0.602250pt}%
\definecolor{currentstroke}{rgb}{0.000000,0.000000,0.000000}%
\pgfsetstrokecolor{currentstroke}%
\pgfsetdash{}{0pt}%
\pgfsys@defobject{currentmarker}{\pgfqpoint{0.000000in}{-0.027778in}}{\pgfqpoint{0.000000in}{0.000000in}}{%
\pgfpathmoveto{\pgfqpoint{0.000000in}{0.000000in}}%
\pgfpathlineto{\pgfqpoint{0.000000in}{-0.027778in}}%
\pgfusepath{stroke,fill}%
}%
\begin{pgfscope}%
\pgfsys@transformshift{1.545451in}{0.467838in}%
\pgfsys@useobject{currentmarker}{}%
\end{pgfscope}%
\end{pgfscope}%
\begin{pgfscope}%
\pgfsetbuttcap%
\pgfsetroundjoin%
\definecolor{currentfill}{rgb}{0.000000,0.000000,0.000000}%
\pgfsetfillcolor{currentfill}%
\pgfsetlinewidth{0.602250pt}%
\definecolor{currentstroke}{rgb}{0.000000,0.000000,0.000000}%
\pgfsetstrokecolor{currentstroke}%
\pgfsetdash{}{0pt}%
\pgfsys@defobject{currentmarker}{\pgfqpoint{0.000000in}{-0.027778in}}{\pgfqpoint{0.000000in}{0.000000in}}{%
\pgfpathmoveto{\pgfqpoint{0.000000in}{0.000000in}}%
\pgfpathlineto{\pgfqpoint{0.000000in}{-0.027778in}}%
\pgfusepath{stroke,fill}%
}%
\begin{pgfscope}%
\pgfsys@transformshift{1.630715in}{0.467838in}%
\pgfsys@useobject{currentmarker}{}%
\end{pgfscope}%
\end{pgfscope}%
\begin{pgfscope}%
\pgfsetbuttcap%
\pgfsetroundjoin%
\definecolor{currentfill}{rgb}{0.000000,0.000000,0.000000}%
\pgfsetfillcolor{currentfill}%
\pgfsetlinewidth{0.602250pt}%
\definecolor{currentstroke}{rgb}{0.000000,0.000000,0.000000}%
\pgfsetstrokecolor{currentstroke}%
\pgfsetdash{}{0pt}%
\pgfsys@defobject{currentmarker}{\pgfqpoint{0.000000in}{-0.027778in}}{\pgfqpoint{0.000000in}{0.000000in}}{%
\pgfpathmoveto{\pgfqpoint{0.000000in}{0.000000in}}%
\pgfpathlineto{\pgfqpoint{0.000000in}{-0.027778in}}%
\pgfusepath{stroke,fill}%
}%
\begin{pgfscope}%
\pgfsys@transformshift{1.696851in}{0.467838in}%
\pgfsys@useobject{currentmarker}{}%
\end{pgfscope}%
\end{pgfscope}%
\begin{pgfscope}%
\pgfsetbuttcap%
\pgfsetroundjoin%
\definecolor{currentfill}{rgb}{0.000000,0.000000,0.000000}%
\pgfsetfillcolor{currentfill}%
\pgfsetlinewidth{0.602250pt}%
\definecolor{currentstroke}{rgb}{0.000000,0.000000,0.000000}%
\pgfsetstrokecolor{currentstroke}%
\pgfsetdash{}{0pt}%
\pgfsys@defobject{currentmarker}{\pgfqpoint{0.000000in}{-0.027778in}}{\pgfqpoint{0.000000in}{0.000000in}}{%
\pgfpathmoveto{\pgfqpoint{0.000000in}{0.000000in}}%
\pgfpathlineto{\pgfqpoint{0.000000in}{-0.027778in}}%
\pgfusepath{stroke,fill}%
}%
\begin{pgfscope}%
\pgfsys@transformshift{1.750888in}{0.467838in}%
\pgfsys@useobject{currentmarker}{}%
\end{pgfscope}%
\end{pgfscope}%
\begin{pgfscope}%
\pgfsetbuttcap%
\pgfsetroundjoin%
\definecolor{currentfill}{rgb}{0.000000,0.000000,0.000000}%
\pgfsetfillcolor{currentfill}%
\pgfsetlinewidth{0.602250pt}%
\definecolor{currentstroke}{rgb}{0.000000,0.000000,0.000000}%
\pgfsetstrokecolor{currentstroke}%
\pgfsetdash{}{0pt}%
\pgfsys@defobject{currentmarker}{\pgfqpoint{0.000000in}{-0.027778in}}{\pgfqpoint{0.000000in}{0.000000in}}{%
\pgfpathmoveto{\pgfqpoint{0.000000in}{0.000000in}}%
\pgfpathlineto{\pgfqpoint{0.000000in}{-0.027778in}}%
\pgfusepath{stroke,fill}%
}%
\begin{pgfscope}%
\pgfsys@transformshift{1.796576in}{0.467838in}%
\pgfsys@useobject{currentmarker}{}%
\end{pgfscope}%
\end{pgfscope}%
\begin{pgfscope}%
\pgfsetbuttcap%
\pgfsetroundjoin%
\definecolor{currentfill}{rgb}{0.000000,0.000000,0.000000}%
\pgfsetfillcolor{currentfill}%
\pgfsetlinewidth{0.602250pt}%
\definecolor{currentstroke}{rgb}{0.000000,0.000000,0.000000}%
\pgfsetstrokecolor{currentstroke}%
\pgfsetdash{}{0pt}%
\pgfsys@defobject{currentmarker}{\pgfqpoint{0.000000in}{-0.027778in}}{\pgfqpoint{0.000000in}{0.000000in}}{%
\pgfpathmoveto{\pgfqpoint{0.000000in}{0.000000in}}%
\pgfpathlineto{\pgfqpoint{0.000000in}{-0.027778in}}%
\pgfusepath{stroke,fill}%
}%
\begin{pgfscope}%
\pgfsys@transformshift{1.836153in}{0.467838in}%
\pgfsys@useobject{currentmarker}{}%
\end{pgfscope}%
\end{pgfscope}%
\begin{pgfscope}%
\pgfsetbuttcap%
\pgfsetroundjoin%
\definecolor{currentfill}{rgb}{0.000000,0.000000,0.000000}%
\pgfsetfillcolor{currentfill}%
\pgfsetlinewidth{0.602250pt}%
\definecolor{currentstroke}{rgb}{0.000000,0.000000,0.000000}%
\pgfsetstrokecolor{currentstroke}%
\pgfsetdash{}{0pt}%
\pgfsys@defobject{currentmarker}{\pgfqpoint{0.000000in}{-0.027778in}}{\pgfqpoint{0.000000in}{0.000000in}}{%
\pgfpathmoveto{\pgfqpoint{0.000000in}{0.000000in}}%
\pgfpathlineto{\pgfqpoint{0.000000in}{-0.027778in}}%
\pgfusepath{stroke,fill}%
}%
\begin{pgfscope}%
\pgfsys@transformshift{1.871062in}{0.467838in}%
\pgfsys@useobject{currentmarker}{}%
\end{pgfscope}%
\end{pgfscope}%
\begin{pgfscope}%
\pgfsetbuttcap%
\pgfsetroundjoin%
\definecolor{currentfill}{rgb}{0.000000,0.000000,0.000000}%
\pgfsetfillcolor{currentfill}%
\pgfsetlinewidth{0.602250pt}%
\definecolor{currentstroke}{rgb}{0.000000,0.000000,0.000000}%
\pgfsetstrokecolor{currentstroke}%
\pgfsetdash{}{0pt}%
\pgfsys@defobject{currentmarker}{\pgfqpoint{0.000000in}{-0.027778in}}{\pgfqpoint{0.000000in}{0.000000in}}{%
\pgfpathmoveto{\pgfqpoint{0.000000in}{0.000000in}}%
\pgfpathlineto{\pgfqpoint{0.000000in}{-0.027778in}}%
\pgfusepath{stroke,fill}%
}%
\begin{pgfscope}%
\pgfsys@transformshift{2.107726in}{0.467838in}%
\pgfsys@useobject{currentmarker}{}%
\end{pgfscope}%
\end{pgfscope}%
\begin{pgfscope}%
\pgfsetbuttcap%
\pgfsetroundjoin%
\definecolor{currentfill}{rgb}{0.000000,0.000000,0.000000}%
\pgfsetfillcolor{currentfill}%
\pgfsetlinewidth{0.602250pt}%
\definecolor{currentstroke}{rgb}{0.000000,0.000000,0.000000}%
\pgfsetstrokecolor{currentstroke}%
\pgfsetdash{}{0pt}%
\pgfsys@defobject{currentmarker}{\pgfqpoint{0.000000in}{-0.027778in}}{\pgfqpoint{0.000000in}{0.000000in}}{%
\pgfpathmoveto{\pgfqpoint{0.000000in}{0.000000in}}%
\pgfpathlineto{\pgfqpoint{0.000000in}{-0.027778in}}%
\pgfusepath{stroke,fill}%
}%
\begin{pgfscope}%
\pgfsys@transformshift{2.227899in}{0.467838in}%
\pgfsys@useobject{currentmarker}{}%
\end{pgfscope}%
\end{pgfscope}%
\begin{pgfscope}%
\pgfsetbuttcap%
\pgfsetroundjoin%
\definecolor{currentfill}{rgb}{0.000000,0.000000,0.000000}%
\pgfsetfillcolor{currentfill}%
\pgfsetlinewidth{0.602250pt}%
\definecolor{currentstroke}{rgb}{0.000000,0.000000,0.000000}%
\pgfsetstrokecolor{currentstroke}%
\pgfsetdash{}{0pt}%
\pgfsys@defobject{currentmarker}{\pgfqpoint{0.000000in}{-0.027778in}}{\pgfqpoint{0.000000in}{0.000000in}}{%
\pgfpathmoveto{\pgfqpoint{0.000000in}{0.000000in}}%
\pgfpathlineto{\pgfqpoint{0.000000in}{-0.027778in}}%
\pgfusepath{stroke,fill}%
}%
\begin{pgfscope}%
\pgfsys@transformshift{2.313163in}{0.467838in}%
\pgfsys@useobject{currentmarker}{}%
\end{pgfscope}%
\end{pgfscope}%
\begin{pgfscope}%
\pgfsetbuttcap%
\pgfsetroundjoin%
\definecolor{currentfill}{rgb}{0.000000,0.000000,0.000000}%
\pgfsetfillcolor{currentfill}%
\pgfsetlinewidth{0.602250pt}%
\definecolor{currentstroke}{rgb}{0.000000,0.000000,0.000000}%
\pgfsetstrokecolor{currentstroke}%
\pgfsetdash{}{0pt}%
\pgfsys@defobject{currentmarker}{\pgfqpoint{0.000000in}{-0.027778in}}{\pgfqpoint{0.000000in}{0.000000in}}{%
\pgfpathmoveto{\pgfqpoint{0.000000in}{0.000000in}}%
\pgfpathlineto{\pgfqpoint{0.000000in}{-0.027778in}}%
\pgfusepath{stroke,fill}%
}%
\begin{pgfscope}%
\pgfsys@transformshift{2.379299in}{0.467838in}%
\pgfsys@useobject{currentmarker}{}%
\end{pgfscope}%
\end{pgfscope}%
\begin{pgfscope}%
\pgfsetbuttcap%
\pgfsetroundjoin%
\definecolor{currentfill}{rgb}{0.000000,0.000000,0.000000}%
\pgfsetfillcolor{currentfill}%
\pgfsetlinewidth{0.602250pt}%
\definecolor{currentstroke}{rgb}{0.000000,0.000000,0.000000}%
\pgfsetstrokecolor{currentstroke}%
\pgfsetdash{}{0pt}%
\pgfsys@defobject{currentmarker}{\pgfqpoint{0.000000in}{-0.027778in}}{\pgfqpoint{0.000000in}{0.000000in}}{%
\pgfpathmoveto{\pgfqpoint{0.000000in}{0.000000in}}%
\pgfpathlineto{\pgfqpoint{0.000000in}{-0.027778in}}%
\pgfusepath{stroke,fill}%
}%
\begin{pgfscope}%
\pgfsys@transformshift{2.433336in}{0.467838in}%
\pgfsys@useobject{currentmarker}{}%
\end{pgfscope}%
\end{pgfscope}%
\begin{pgfscope}%
\pgfsetbuttcap%
\pgfsetroundjoin%
\definecolor{currentfill}{rgb}{0.000000,0.000000,0.000000}%
\pgfsetfillcolor{currentfill}%
\pgfsetlinewidth{0.602250pt}%
\definecolor{currentstroke}{rgb}{0.000000,0.000000,0.000000}%
\pgfsetstrokecolor{currentstroke}%
\pgfsetdash{}{0pt}%
\pgfsys@defobject{currentmarker}{\pgfqpoint{0.000000in}{-0.027778in}}{\pgfqpoint{0.000000in}{0.000000in}}{%
\pgfpathmoveto{\pgfqpoint{0.000000in}{0.000000in}}%
\pgfpathlineto{\pgfqpoint{0.000000in}{-0.027778in}}%
\pgfusepath{stroke,fill}%
}%
\begin{pgfscope}%
\pgfsys@transformshift{2.479023in}{0.467838in}%
\pgfsys@useobject{currentmarker}{}%
\end{pgfscope}%
\end{pgfscope}%
\begin{pgfscope}%
\pgfsetbuttcap%
\pgfsetroundjoin%
\definecolor{currentfill}{rgb}{0.000000,0.000000,0.000000}%
\pgfsetfillcolor{currentfill}%
\pgfsetlinewidth{0.602250pt}%
\definecolor{currentstroke}{rgb}{0.000000,0.000000,0.000000}%
\pgfsetstrokecolor{currentstroke}%
\pgfsetdash{}{0pt}%
\pgfsys@defobject{currentmarker}{\pgfqpoint{0.000000in}{-0.027778in}}{\pgfqpoint{0.000000in}{0.000000in}}{%
\pgfpathmoveto{\pgfqpoint{0.000000in}{0.000000in}}%
\pgfpathlineto{\pgfqpoint{0.000000in}{-0.027778in}}%
\pgfusepath{stroke,fill}%
}%
\begin{pgfscope}%
\pgfsys@transformshift{2.518600in}{0.467838in}%
\pgfsys@useobject{currentmarker}{}%
\end{pgfscope}%
\end{pgfscope}%
\begin{pgfscope}%
\pgfsetbuttcap%
\pgfsetroundjoin%
\definecolor{currentfill}{rgb}{0.000000,0.000000,0.000000}%
\pgfsetfillcolor{currentfill}%
\pgfsetlinewidth{0.602250pt}%
\definecolor{currentstroke}{rgb}{0.000000,0.000000,0.000000}%
\pgfsetstrokecolor{currentstroke}%
\pgfsetdash{}{0pt}%
\pgfsys@defobject{currentmarker}{\pgfqpoint{0.000000in}{-0.027778in}}{\pgfqpoint{0.000000in}{0.000000in}}{%
\pgfpathmoveto{\pgfqpoint{0.000000in}{0.000000in}}%
\pgfpathlineto{\pgfqpoint{0.000000in}{-0.027778in}}%
\pgfusepath{stroke,fill}%
}%
\begin{pgfscope}%
\pgfsys@transformshift{2.553509in}{0.467838in}%
\pgfsys@useobject{currentmarker}{}%
\end{pgfscope}%
\end{pgfscope}%
\begin{pgfscope}%
\pgfsetbuttcap%
\pgfsetroundjoin%
\definecolor{currentfill}{rgb}{0.000000,0.000000,0.000000}%
\pgfsetfillcolor{currentfill}%
\pgfsetlinewidth{0.602250pt}%
\definecolor{currentstroke}{rgb}{0.000000,0.000000,0.000000}%
\pgfsetstrokecolor{currentstroke}%
\pgfsetdash{}{0pt}%
\pgfsys@defobject{currentmarker}{\pgfqpoint{0.000000in}{-0.027778in}}{\pgfqpoint{0.000000in}{0.000000in}}{%
\pgfpathmoveto{\pgfqpoint{0.000000in}{0.000000in}}%
\pgfpathlineto{\pgfqpoint{0.000000in}{-0.027778in}}%
\pgfusepath{stroke,fill}%
}%
\begin{pgfscope}%
\pgfsys@transformshift{2.790173in}{0.467838in}%
\pgfsys@useobject{currentmarker}{}%
\end{pgfscope}%
\end{pgfscope}%
\begin{pgfscope}%
\pgfsetbuttcap%
\pgfsetroundjoin%
\definecolor{currentfill}{rgb}{0.000000,0.000000,0.000000}%
\pgfsetfillcolor{currentfill}%
\pgfsetlinewidth{0.602250pt}%
\definecolor{currentstroke}{rgb}{0.000000,0.000000,0.000000}%
\pgfsetstrokecolor{currentstroke}%
\pgfsetdash{}{0pt}%
\pgfsys@defobject{currentmarker}{\pgfqpoint{0.000000in}{-0.027778in}}{\pgfqpoint{0.000000in}{0.000000in}}{%
\pgfpathmoveto{\pgfqpoint{0.000000in}{0.000000in}}%
\pgfpathlineto{\pgfqpoint{0.000000in}{-0.027778in}}%
\pgfusepath{stroke,fill}%
}%
\begin{pgfscope}%
\pgfsys@transformshift{2.910346in}{0.467838in}%
\pgfsys@useobject{currentmarker}{}%
\end{pgfscope}%
\end{pgfscope}%
\begin{pgfscope}%
\pgfsetbuttcap%
\pgfsetroundjoin%
\definecolor{currentfill}{rgb}{0.000000,0.000000,0.000000}%
\pgfsetfillcolor{currentfill}%
\pgfsetlinewidth{0.602250pt}%
\definecolor{currentstroke}{rgb}{0.000000,0.000000,0.000000}%
\pgfsetstrokecolor{currentstroke}%
\pgfsetdash{}{0pt}%
\pgfsys@defobject{currentmarker}{\pgfqpoint{0.000000in}{-0.027778in}}{\pgfqpoint{0.000000in}{0.000000in}}{%
\pgfpathmoveto{\pgfqpoint{0.000000in}{0.000000in}}%
\pgfpathlineto{\pgfqpoint{0.000000in}{-0.027778in}}%
\pgfusepath{stroke,fill}%
}%
\begin{pgfscope}%
\pgfsys@transformshift{2.995610in}{0.467838in}%
\pgfsys@useobject{currentmarker}{}%
\end{pgfscope}%
\end{pgfscope}%
\begin{pgfscope}%
\pgfsetbuttcap%
\pgfsetroundjoin%
\definecolor{currentfill}{rgb}{0.000000,0.000000,0.000000}%
\pgfsetfillcolor{currentfill}%
\pgfsetlinewidth{0.602250pt}%
\definecolor{currentstroke}{rgb}{0.000000,0.000000,0.000000}%
\pgfsetstrokecolor{currentstroke}%
\pgfsetdash{}{0pt}%
\pgfsys@defobject{currentmarker}{\pgfqpoint{0.000000in}{-0.027778in}}{\pgfqpoint{0.000000in}{0.000000in}}{%
\pgfpathmoveto{\pgfqpoint{0.000000in}{0.000000in}}%
\pgfpathlineto{\pgfqpoint{0.000000in}{-0.027778in}}%
\pgfusepath{stroke,fill}%
}%
\begin{pgfscope}%
\pgfsys@transformshift{3.061746in}{0.467838in}%
\pgfsys@useobject{currentmarker}{}%
\end{pgfscope}%
\end{pgfscope}%
\begin{pgfscope}%
\pgfsetbuttcap%
\pgfsetroundjoin%
\definecolor{currentfill}{rgb}{0.000000,0.000000,0.000000}%
\pgfsetfillcolor{currentfill}%
\pgfsetlinewidth{0.602250pt}%
\definecolor{currentstroke}{rgb}{0.000000,0.000000,0.000000}%
\pgfsetstrokecolor{currentstroke}%
\pgfsetdash{}{0pt}%
\pgfsys@defobject{currentmarker}{\pgfqpoint{0.000000in}{-0.027778in}}{\pgfqpoint{0.000000in}{0.000000in}}{%
\pgfpathmoveto{\pgfqpoint{0.000000in}{0.000000in}}%
\pgfpathlineto{\pgfqpoint{0.000000in}{-0.027778in}}%
\pgfusepath{stroke,fill}%
}%
\begin{pgfscope}%
\pgfsys@transformshift{3.115783in}{0.467838in}%
\pgfsys@useobject{currentmarker}{}%
\end{pgfscope}%
\end{pgfscope}%
\begin{pgfscope}%
\pgfsetbuttcap%
\pgfsetroundjoin%
\definecolor{currentfill}{rgb}{0.000000,0.000000,0.000000}%
\pgfsetfillcolor{currentfill}%
\pgfsetlinewidth{0.602250pt}%
\definecolor{currentstroke}{rgb}{0.000000,0.000000,0.000000}%
\pgfsetstrokecolor{currentstroke}%
\pgfsetdash{}{0pt}%
\pgfsys@defobject{currentmarker}{\pgfqpoint{0.000000in}{-0.027778in}}{\pgfqpoint{0.000000in}{0.000000in}}{%
\pgfpathmoveto{\pgfqpoint{0.000000in}{0.000000in}}%
\pgfpathlineto{\pgfqpoint{0.000000in}{-0.027778in}}%
\pgfusepath{stroke,fill}%
}%
\begin{pgfscope}%
\pgfsys@transformshift{3.161471in}{0.467838in}%
\pgfsys@useobject{currentmarker}{}%
\end{pgfscope}%
\end{pgfscope}%
\begin{pgfscope}%
\pgfsetbuttcap%
\pgfsetroundjoin%
\definecolor{currentfill}{rgb}{0.000000,0.000000,0.000000}%
\pgfsetfillcolor{currentfill}%
\pgfsetlinewidth{0.602250pt}%
\definecolor{currentstroke}{rgb}{0.000000,0.000000,0.000000}%
\pgfsetstrokecolor{currentstroke}%
\pgfsetdash{}{0pt}%
\pgfsys@defobject{currentmarker}{\pgfqpoint{0.000000in}{-0.027778in}}{\pgfqpoint{0.000000in}{0.000000in}}{%
\pgfpathmoveto{\pgfqpoint{0.000000in}{0.000000in}}%
\pgfpathlineto{\pgfqpoint{0.000000in}{-0.027778in}}%
\pgfusepath{stroke,fill}%
}%
\begin{pgfscope}%
\pgfsys@transformshift{3.201047in}{0.467838in}%
\pgfsys@useobject{currentmarker}{}%
\end{pgfscope}%
\end{pgfscope}%
\begin{pgfscope}%
\pgfsetbuttcap%
\pgfsetroundjoin%
\definecolor{currentfill}{rgb}{0.000000,0.000000,0.000000}%
\pgfsetfillcolor{currentfill}%
\pgfsetlinewidth{0.602250pt}%
\definecolor{currentstroke}{rgb}{0.000000,0.000000,0.000000}%
\pgfsetstrokecolor{currentstroke}%
\pgfsetdash{}{0pt}%
\pgfsys@defobject{currentmarker}{\pgfqpoint{0.000000in}{-0.027778in}}{\pgfqpoint{0.000000in}{0.000000in}}{%
\pgfpathmoveto{\pgfqpoint{0.000000in}{0.000000in}}%
\pgfpathlineto{\pgfqpoint{0.000000in}{-0.027778in}}%
\pgfusepath{stroke,fill}%
}%
\begin{pgfscope}%
\pgfsys@transformshift{3.235956in}{0.467838in}%
\pgfsys@useobject{currentmarker}{}%
\end{pgfscope}%
\end{pgfscope}%
\begin{pgfscope}%
\pgfsetbuttcap%
\pgfsetroundjoin%
\definecolor{currentfill}{rgb}{0.000000,0.000000,0.000000}%
\pgfsetfillcolor{currentfill}%
\pgfsetlinewidth{0.602250pt}%
\definecolor{currentstroke}{rgb}{0.000000,0.000000,0.000000}%
\pgfsetstrokecolor{currentstroke}%
\pgfsetdash{}{0pt}%
\pgfsys@defobject{currentmarker}{\pgfqpoint{0.000000in}{-0.027778in}}{\pgfqpoint{0.000000in}{0.000000in}}{%
\pgfpathmoveto{\pgfqpoint{0.000000in}{0.000000in}}%
\pgfpathlineto{\pgfqpoint{0.000000in}{-0.027778in}}%
\pgfusepath{stroke,fill}%
}%
\begin{pgfscope}%
\pgfsys@transformshift{3.472620in}{0.467838in}%
\pgfsys@useobject{currentmarker}{}%
\end{pgfscope}%
\end{pgfscope}%
\begin{pgfscope}%
\pgfsetbuttcap%
\pgfsetroundjoin%
\definecolor{currentfill}{rgb}{0.000000,0.000000,0.000000}%
\pgfsetfillcolor{currentfill}%
\pgfsetlinewidth{0.602250pt}%
\definecolor{currentstroke}{rgb}{0.000000,0.000000,0.000000}%
\pgfsetstrokecolor{currentstroke}%
\pgfsetdash{}{0pt}%
\pgfsys@defobject{currentmarker}{\pgfqpoint{0.000000in}{-0.027778in}}{\pgfqpoint{0.000000in}{0.000000in}}{%
\pgfpathmoveto{\pgfqpoint{0.000000in}{0.000000in}}%
\pgfpathlineto{\pgfqpoint{0.000000in}{-0.027778in}}%
\pgfusepath{stroke,fill}%
}%
\begin{pgfscope}%
\pgfsys@transformshift{3.592793in}{0.467838in}%
\pgfsys@useobject{currentmarker}{}%
\end{pgfscope}%
\end{pgfscope}%
\begin{pgfscope}%
\pgfsetbuttcap%
\pgfsetroundjoin%
\definecolor{currentfill}{rgb}{0.000000,0.000000,0.000000}%
\pgfsetfillcolor{currentfill}%
\pgfsetlinewidth{0.602250pt}%
\definecolor{currentstroke}{rgb}{0.000000,0.000000,0.000000}%
\pgfsetstrokecolor{currentstroke}%
\pgfsetdash{}{0pt}%
\pgfsys@defobject{currentmarker}{\pgfqpoint{0.000000in}{-0.027778in}}{\pgfqpoint{0.000000in}{0.000000in}}{%
\pgfpathmoveto{\pgfqpoint{0.000000in}{0.000000in}}%
\pgfpathlineto{\pgfqpoint{0.000000in}{-0.027778in}}%
\pgfusepath{stroke,fill}%
}%
\begin{pgfscope}%
\pgfsys@transformshift{3.678058in}{0.467838in}%
\pgfsys@useobject{currentmarker}{}%
\end{pgfscope}%
\end{pgfscope}%
\begin{pgfscope}%
\pgfsetbuttcap%
\pgfsetroundjoin%
\definecolor{currentfill}{rgb}{0.000000,0.000000,0.000000}%
\pgfsetfillcolor{currentfill}%
\pgfsetlinewidth{0.602250pt}%
\definecolor{currentstroke}{rgb}{0.000000,0.000000,0.000000}%
\pgfsetstrokecolor{currentstroke}%
\pgfsetdash{}{0pt}%
\pgfsys@defobject{currentmarker}{\pgfqpoint{0.000000in}{-0.027778in}}{\pgfqpoint{0.000000in}{0.000000in}}{%
\pgfpathmoveto{\pgfqpoint{0.000000in}{0.000000in}}%
\pgfpathlineto{\pgfqpoint{0.000000in}{-0.027778in}}%
\pgfusepath{stroke,fill}%
}%
\begin{pgfscope}%
\pgfsys@transformshift{3.744193in}{0.467838in}%
\pgfsys@useobject{currentmarker}{}%
\end{pgfscope}%
\end{pgfscope}%
\begin{pgfscope}%
\pgfsetbuttcap%
\pgfsetroundjoin%
\definecolor{currentfill}{rgb}{0.000000,0.000000,0.000000}%
\pgfsetfillcolor{currentfill}%
\pgfsetlinewidth{0.602250pt}%
\definecolor{currentstroke}{rgb}{0.000000,0.000000,0.000000}%
\pgfsetstrokecolor{currentstroke}%
\pgfsetdash{}{0pt}%
\pgfsys@defobject{currentmarker}{\pgfqpoint{0.000000in}{-0.027778in}}{\pgfqpoint{0.000000in}{0.000000in}}{%
\pgfpathmoveto{\pgfqpoint{0.000000in}{0.000000in}}%
\pgfpathlineto{\pgfqpoint{0.000000in}{-0.027778in}}%
\pgfusepath{stroke,fill}%
}%
\begin{pgfscope}%
\pgfsys@transformshift{3.798231in}{0.467838in}%
\pgfsys@useobject{currentmarker}{}%
\end{pgfscope}%
\end{pgfscope}%
\begin{pgfscope}%
\pgfsetbuttcap%
\pgfsetroundjoin%
\definecolor{currentfill}{rgb}{0.000000,0.000000,0.000000}%
\pgfsetfillcolor{currentfill}%
\pgfsetlinewidth{0.602250pt}%
\definecolor{currentstroke}{rgb}{0.000000,0.000000,0.000000}%
\pgfsetstrokecolor{currentstroke}%
\pgfsetdash{}{0pt}%
\pgfsys@defobject{currentmarker}{\pgfqpoint{0.000000in}{-0.027778in}}{\pgfqpoint{0.000000in}{0.000000in}}{%
\pgfpathmoveto{\pgfqpoint{0.000000in}{0.000000in}}%
\pgfpathlineto{\pgfqpoint{0.000000in}{-0.027778in}}%
\pgfusepath{stroke,fill}%
}%
\begin{pgfscope}%
\pgfsys@transformshift{3.843918in}{0.467838in}%
\pgfsys@useobject{currentmarker}{}%
\end{pgfscope}%
\end{pgfscope}%
\begin{pgfscope}%
\pgfsetbuttcap%
\pgfsetroundjoin%
\definecolor{currentfill}{rgb}{0.000000,0.000000,0.000000}%
\pgfsetfillcolor{currentfill}%
\pgfsetlinewidth{0.602250pt}%
\definecolor{currentstroke}{rgb}{0.000000,0.000000,0.000000}%
\pgfsetstrokecolor{currentstroke}%
\pgfsetdash{}{0pt}%
\pgfsys@defobject{currentmarker}{\pgfqpoint{0.000000in}{-0.027778in}}{\pgfqpoint{0.000000in}{0.000000in}}{%
\pgfpathmoveto{\pgfqpoint{0.000000in}{0.000000in}}%
\pgfpathlineto{\pgfqpoint{0.000000in}{-0.027778in}}%
\pgfusepath{stroke,fill}%
}%
\begin{pgfscope}%
\pgfsys@transformshift{3.883495in}{0.467838in}%
\pgfsys@useobject{currentmarker}{}%
\end{pgfscope}%
\end{pgfscope}%
\begin{pgfscope}%
\pgfsetbuttcap%
\pgfsetroundjoin%
\definecolor{currentfill}{rgb}{0.000000,0.000000,0.000000}%
\pgfsetfillcolor{currentfill}%
\pgfsetlinewidth{0.602250pt}%
\definecolor{currentstroke}{rgb}{0.000000,0.000000,0.000000}%
\pgfsetstrokecolor{currentstroke}%
\pgfsetdash{}{0pt}%
\pgfsys@defobject{currentmarker}{\pgfqpoint{0.000000in}{-0.027778in}}{\pgfqpoint{0.000000in}{0.000000in}}{%
\pgfpathmoveto{\pgfqpoint{0.000000in}{0.000000in}}%
\pgfpathlineto{\pgfqpoint{0.000000in}{-0.027778in}}%
\pgfusepath{stroke,fill}%
}%
\begin{pgfscope}%
\pgfsys@transformshift{3.918404in}{0.467838in}%
\pgfsys@useobject{currentmarker}{}%
\end{pgfscope}%
\end{pgfscope}%
\begin{pgfscope}%
\pgfsetbuttcap%
\pgfsetroundjoin%
\definecolor{currentfill}{rgb}{0.000000,0.000000,0.000000}%
\pgfsetfillcolor{currentfill}%
\pgfsetlinewidth{0.602250pt}%
\definecolor{currentstroke}{rgb}{0.000000,0.000000,0.000000}%
\pgfsetstrokecolor{currentstroke}%
\pgfsetdash{}{0pt}%
\pgfsys@defobject{currentmarker}{\pgfqpoint{0.000000in}{-0.027778in}}{\pgfqpoint{0.000000in}{0.000000in}}{%
\pgfpathmoveto{\pgfqpoint{0.000000in}{0.000000in}}%
\pgfpathlineto{\pgfqpoint{0.000000in}{-0.027778in}}%
\pgfusepath{stroke,fill}%
}%
\begin{pgfscope}%
\pgfsys@transformshift{4.155068in}{0.467838in}%
\pgfsys@useobject{currentmarker}{}%
\end{pgfscope}%
\end{pgfscope}%
\begin{pgfscope}%
\pgfsetbuttcap%
\pgfsetroundjoin%
\definecolor{currentfill}{rgb}{0.000000,0.000000,0.000000}%
\pgfsetfillcolor{currentfill}%
\pgfsetlinewidth{0.602250pt}%
\definecolor{currentstroke}{rgb}{0.000000,0.000000,0.000000}%
\pgfsetstrokecolor{currentstroke}%
\pgfsetdash{}{0pt}%
\pgfsys@defobject{currentmarker}{\pgfqpoint{0.000000in}{-0.027778in}}{\pgfqpoint{0.000000in}{0.000000in}}{%
\pgfpathmoveto{\pgfqpoint{0.000000in}{0.000000in}}%
\pgfpathlineto{\pgfqpoint{0.000000in}{-0.027778in}}%
\pgfusepath{stroke,fill}%
}%
\begin{pgfscope}%
\pgfsys@transformshift{4.275241in}{0.467838in}%
\pgfsys@useobject{currentmarker}{}%
\end{pgfscope}%
\end{pgfscope}%
\begin{pgfscope}%
\pgfsetbuttcap%
\pgfsetroundjoin%
\definecolor{currentfill}{rgb}{0.000000,0.000000,0.000000}%
\pgfsetfillcolor{currentfill}%
\pgfsetlinewidth{0.602250pt}%
\definecolor{currentstroke}{rgb}{0.000000,0.000000,0.000000}%
\pgfsetstrokecolor{currentstroke}%
\pgfsetdash{}{0pt}%
\pgfsys@defobject{currentmarker}{\pgfqpoint{0.000000in}{-0.027778in}}{\pgfqpoint{0.000000in}{0.000000in}}{%
\pgfpathmoveto{\pgfqpoint{0.000000in}{0.000000in}}%
\pgfpathlineto{\pgfqpoint{0.000000in}{-0.027778in}}%
\pgfusepath{stroke,fill}%
}%
\begin{pgfscope}%
\pgfsys@transformshift{4.360505in}{0.467838in}%
\pgfsys@useobject{currentmarker}{}%
\end{pgfscope}%
\end{pgfscope}%
\begin{pgfscope}%
\pgfsetbuttcap%
\pgfsetroundjoin%
\definecolor{currentfill}{rgb}{0.000000,0.000000,0.000000}%
\pgfsetfillcolor{currentfill}%
\pgfsetlinewidth{0.602250pt}%
\definecolor{currentstroke}{rgb}{0.000000,0.000000,0.000000}%
\pgfsetstrokecolor{currentstroke}%
\pgfsetdash{}{0pt}%
\pgfsys@defobject{currentmarker}{\pgfqpoint{0.000000in}{-0.027778in}}{\pgfqpoint{0.000000in}{0.000000in}}{%
\pgfpathmoveto{\pgfqpoint{0.000000in}{0.000000in}}%
\pgfpathlineto{\pgfqpoint{0.000000in}{-0.027778in}}%
\pgfusepath{stroke,fill}%
}%
\begin{pgfscope}%
\pgfsys@transformshift{4.426641in}{0.467838in}%
\pgfsys@useobject{currentmarker}{}%
\end{pgfscope}%
\end{pgfscope}%
\begin{pgfscope}%
\pgfsetbuttcap%
\pgfsetroundjoin%
\definecolor{currentfill}{rgb}{0.000000,0.000000,0.000000}%
\pgfsetfillcolor{currentfill}%
\pgfsetlinewidth{0.602250pt}%
\definecolor{currentstroke}{rgb}{0.000000,0.000000,0.000000}%
\pgfsetstrokecolor{currentstroke}%
\pgfsetdash{}{0pt}%
\pgfsys@defobject{currentmarker}{\pgfqpoint{0.000000in}{-0.027778in}}{\pgfqpoint{0.000000in}{0.000000in}}{%
\pgfpathmoveto{\pgfqpoint{0.000000in}{0.000000in}}%
\pgfpathlineto{\pgfqpoint{0.000000in}{-0.027778in}}%
\pgfusepath{stroke,fill}%
}%
\begin{pgfscope}%
\pgfsys@transformshift{4.480678in}{0.467838in}%
\pgfsys@useobject{currentmarker}{}%
\end{pgfscope}%
\end{pgfscope}%
\begin{pgfscope}%
\pgfsetbuttcap%
\pgfsetroundjoin%
\definecolor{currentfill}{rgb}{0.000000,0.000000,0.000000}%
\pgfsetfillcolor{currentfill}%
\pgfsetlinewidth{0.602250pt}%
\definecolor{currentstroke}{rgb}{0.000000,0.000000,0.000000}%
\pgfsetstrokecolor{currentstroke}%
\pgfsetdash{}{0pt}%
\pgfsys@defobject{currentmarker}{\pgfqpoint{0.000000in}{-0.027778in}}{\pgfqpoint{0.000000in}{0.000000in}}{%
\pgfpathmoveto{\pgfqpoint{0.000000in}{0.000000in}}%
\pgfpathlineto{\pgfqpoint{0.000000in}{-0.027778in}}%
\pgfusepath{stroke,fill}%
}%
\begin{pgfscope}%
\pgfsys@transformshift{4.526366in}{0.467838in}%
\pgfsys@useobject{currentmarker}{}%
\end{pgfscope}%
\end{pgfscope}%
\begin{pgfscope}%
\pgfsetbuttcap%
\pgfsetroundjoin%
\definecolor{currentfill}{rgb}{0.000000,0.000000,0.000000}%
\pgfsetfillcolor{currentfill}%
\pgfsetlinewidth{0.602250pt}%
\definecolor{currentstroke}{rgb}{0.000000,0.000000,0.000000}%
\pgfsetstrokecolor{currentstroke}%
\pgfsetdash{}{0pt}%
\pgfsys@defobject{currentmarker}{\pgfqpoint{0.000000in}{-0.027778in}}{\pgfqpoint{0.000000in}{0.000000in}}{%
\pgfpathmoveto{\pgfqpoint{0.000000in}{0.000000in}}%
\pgfpathlineto{\pgfqpoint{0.000000in}{-0.027778in}}%
\pgfusepath{stroke,fill}%
}%
\begin{pgfscope}%
\pgfsys@transformshift{4.565942in}{0.467838in}%
\pgfsys@useobject{currentmarker}{}%
\end{pgfscope}%
\end{pgfscope}%
\begin{pgfscope}%
\pgfsetbuttcap%
\pgfsetroundjoin%
\definecolor{currentfill}{rgb}{0.000000,0.000000,0.000000}%
\pgfsetfillcolor{currentfill}%
\pgfsetlinewidth{0.602250pt}%
\definecolor{currentstroke}{rgb}{0.000000,0.000000,0.000000}%
\pgfsetstrokecolor{currentstroke}%
\pgfsetdash{}{0pt}%
\pgfsys@defobject{currentmarker}{\pgfqpoint{0.000000in}{-0.027778in}}{\pgfqpoint{0.000000in}{0.000000in}}{%
\pgfpathmoveto{\pgfqpoint{0.000000in}{0.000000in}}%
\pgfpathlineto{\pgfqpoint{0.000000in}{-0.027778in}}%
\pgfusepath{stroke,fill}%
}%
\begin{pgfscope}%
\pgfsys@transformshift{4.600851in}{0.467838in}%
\pgfsys@useobject{currentmarker}{}%
\end{pgfscope}%
\end{pgfscope}%
\begin{pgfscope}%
\definecolor{textcolor}{rgb}{0.000000,0.000000,0.000000}%
\pgfsetstrokecolor{textcolor}%
\pgfsetfillcolor{textcolor}%
\pgftext[x=2.584736in,y=0.207530in,,top]{\color{textcolor}\sffamily\fontsize{8.000000}{9.600000}\selectfont Longest solving time (seconds)}%
\end{pgfscope}%
\begin{pgfscope}%
\pgfsetbuttcap%
\pgfsetroundjoin%
\definecolor{currentfill}{rgb}{0.000000,0.000000,0.000000}%
\pgfsetfillcolor{currentfill}%
\pgfsetlinewidth{0.803000pt}%
\definecolor{currentstroke}{rgb}{0.000000,0.000000,0.000000}%
\pgfsetstrokecolor{currentstroke}%
\pgfsetdash{}{0pt}%
\pgfsys@defobject{currentmarker}{\pgfqpoint{-0.048611in}{0.000000in}}{\pgfqpoint{-0.000000in}{0.000000in}}{%
\pgfpathmoveto{\pgfqpoint{-0.000000in}{0.000000in}}%
\pgfpathlineto{\pgfqpoint{-0.048611in}{0.000000in}}%
\pgfusepath{stroke,fill}%
}%
\begin{pgfscope}%
\pgfsys@transformshift{0.537394in}{0.467838in}%
\pgfsys@useobject{currentmarker}{}%
\end{pgfscope}%
\end{pgfscope}%
\begin{pgfscope}%
\definecolor{textcolor}{rgb}{0.000000,0.000000,0.000000}%
\pgfsetstrokecolor{textcolor}%
\pgfsetfillcolor{textcolor}%
\pgftext[x=0.381143in, y=0.425629in, left, base]{\color{textcolor}\sffamily\fontsize{8.000000}{9.600000}\selectfont \(\displaystyle {0}\)}%
\end{pgfscope}%
\begin{pgfscope}%
\pgfsetbuttcap%
\pgfsetroundjoin%
\definecolor{currentfill}{rgb}{0.000000,0.000000,0.000000}%
\pgfsetfillcolor{currentfill}%
\pgfsetlinewidth{0.803000pt}%
\definecolor{currentstroke}{rgb}{0.000000,0.000000,0.000000}%
\pgfsetstrokecolor{currentstroke}%
\pgfsetdash{}{0pt}%
\pgfsys@defobject{currentmarker}{\pgfqpoint{-0.048611in}{0.000000in}}{\pgfqpoint{-0.000000in}{0.000000in}}{%
\pgfpathmoveto{\pgfqpoint{-0.000000in}{0.000000in}}%
\pgfpathlineto{\pgfqpoint{-0.048611in}{0.000000in}}%
\pgfusepath{stroke,fill}%
}%
\begin{pgfscope}%
\pgfsys@transformshift{0.537394in}{0.866740in}%
\pgfsys@useobject{currentmarker}{}%
\end{pgfscope}%
\end{pgfscope}%
\begin{pgfscope}%
\definecolor{textcolor}{rgb}{0.000000,0.000000,0.000000}%
\pgfsetstrokecolor{textcolor}%
\pgfsetfillcolor{textcolor}%
\pgftext[x=0.263086in, y=0.824531in, left, base]{\color{textcolor}\sffamily\fontsize{8.000000}{9.600000}\selectfont \(\displaystyle {100}\)}%
\end{pgfscope}%
\begin{pgfscope}%
\pgfsetbuttcap%
\pgfsetroundjoin%
\definecolor{currentfill}{rgb}{0.000000,0.000000,0.000000}%
\pgfsetfillcolor{currentfill}%
\pgfsetlinewidth{0.803000pt}%
\definecolor{currentstroke}{rgb}{0.000000,0.000000,0.000000}%
\pgfsetstrokecolor{currentstroke}%
\pgfsetdash{}{0pt}%
\pgfsys@defobject{currentmarker}{\pgfqpoint{-0.048611in}{0.000000in}}{\pgfqpoint{-0.000000in}{0.000000in}}{%
\pgfpathmoveto{\pgfqpoint{-0.000000in}{0.000000in}}%
\pgfpathlineto{\pgfqpoint{-0.048611in}{0.000000in}}%
\pgfusepath{stroke,fill}%
}%
\begin{pgfscope}%
\pgfsys@transformshift{0.537394in}{1.265641in}%
\pgfsys@useobject{currentmarker}{}%
\end{pgfscope}%
\end{pgfscope}%
\begin{pgfscope}%
\definecolor{textcolor}{rgb}{0.000000,0.000000,0.000000}%
\pgfsetstrokecolor{textcolor}%
\pgfsetfillcolor{textcolor}%
\pgftext[x=0.263086in, y=1.223432in, left, base]{\color{textcolor}\sffamily\fontsize{8.000000}{9.600000}\selectfont \(\displaystyle {200}\)}%
\end{pgfscope}%
\begin{pgfscope}%
\pgfsetbuttcap%
\pgfsetroundjoin%
\definecolor{currentfill}{rgb}{0.000000,0.000000,0.000000}%
\pgfsetfillcolor{currentfill}%
\pgfsetlinewidth{0.803000pt}%
\definecolor{currentstroke}{rgb}{0.000000,0.000000,0.000000}%
\pgfsetstrokecolor{currentstroke}%
\pgfsetdash{}{0pt}%
\pgfsys@defobject{currentmarker}{\pgfqpoint{-0.048611in}{0.000000in}}{\pgfqpoint{-0.000000in}{0.000000in}}{%
\pgfpathmoveto{\pgfqpoint{-0.000000in}{0.000000in}}%
\pgfpathlineto{\pgfqpoint{-0.048611in}{0.000000in}}%
\pgfusepath{stroke,fill}%
}%
\begin{pgfscope}%
\pgfsys@transformshift{0.537394in}{1.664543in}%
\pgfsys@useobject{currentmarker}{}%
\end{pgfscope}%
\end{pgfscope}%
\begin{pgfscope}%
\definecolor{textcolor}{rgb}{0.000000,0.000000,0.000000}%
\pgfsetstrokecolor{textcolor}%
\pgfsetfillcolor{textcolor}%
\pgftext[x=0.263086in, y=1.622334in, left, base]{\color{textcolor}\sffamily\fontsize{8.000000}{9.600000}\selectfont \(\displaystyle {300}\)}%
\end{pgfscope}%
\begin{pgfscope}%
\pgfsetbuttcap%
\pgfsetroundjoin%
\definecolor{currentfill}{rgb}{0.000000,0.000000,0.000000}%
\pgfsetfillcolor{currentfill}%
\pgfsetlinewidth{0.803000pt}%
\definecolor{currentstroke}{rgb}{0.000000,0.000000,0.000000}%
\pgfsetstrokecolor{currentstroke}%
\pgfsetdash{}{0pt}%
\pgfsys@defobject{currentmarker}{\pgfqpoint{-0.048611in}{0.000000in}}{\pgfqpoint{-0.000000in}{0.000000in}}{%
\pgfpathmoveto{\pgfqpoint{-0.000000in}{0.000000in}}%
\pgfpathlineto{\pgfqpoint{-0.048611in}{0.000000in}}%
\pgfusepath{stroke,fill}%
}%
\begin{pgfscope}%
\pgfsys@transformshift{0.537394in}{2.063444in}%
\pgfsys@useobject{currentmarker}{}%
\end{pgfscope}%
\end{pgfscope}%
\begin{pgfscope}%
\definecolor{textcolor}{rgb}{0.000000,0.000000,0.000000}%
\pgfsetstrokecolor{textcolor}%
\pgfsetfillcolor{textcolor}%
\pgftext[x=0.263086in, y=2.021235in, left, base]{\color{textcolor}\sffamily\fontsize{8.000000}{9.600000}\selectfont \(\displaystyle {400}\)}%
\end{pgfscope}%
\begin{pgfscope}%
\definecolor{textcolor}{rgb}{0.000000,0.000000,0.000000}%
\pgfsetstrokecolor{textcolor}%
\pgfsetfillcolor{textcolor}%
\pgftext[x=0.207530in,y=1.265641in,,bottom,rotate=90.000000]{\color{textcolor}\sffamily\fontsize{8.000000}{9.600000}\selectfont Benchmarks solved}%
\end{pgfscope}%
\begin{pgfscope}%
\pgfpathrectangle{\pgfqpoint{0.537394in}{0.467838in}}{\pgfqpoint{4.094684in}{1.595606in}}%
\pgfusepath{clip}%
\pgfsetrectcap%
\pgfsetroundjoin%
\pgfsetlinewidth{1.003750pt}%
\definecolor{currentstroke}{rgb}{0.121569,0.466667,0.705882}%
\pgfsetstrokecolor{currentstroke}%
\pgfsetdash{}{0pt}%
\pgfpathmoveto{\pgfqpoint{1.658964in}{0.471827in}}%
\pgfpathlineto{\pgfqpoint{1.750888in}{0.475816in}}%
\pgfpathlineto{\pgfqpoint{1.890190in}{0.479805in}}%
\pgfpathlineto{\pgfqpoint{1.896301in}{0.487783in}}%
\pgfpathlineto{\pgfqpoint{1.899310in}{0.499750in}}%
\pgfpathlineto{\pgfqpoint{1.908158in}{0.547619in}}%
\pgfpathlineto{\pgfqpoint{1.911049in}{0.559586in}}%
\pgfpathlineto{\pgfqpoint{1.916749in}{0.575542in}}%
\pgfpathlineto{\pgfqpoint{1.919558in}{0.587509in}}%
\pgfpathlineto{\pgfqpoint{1.922341in}{0.591498in}}%
\pgfpathlineto{\pgfqpoint{1.927830in}{0.615432in}}%
\pgfpathlineto{\pgfqpoint{1.933219in}{0.647344in}}%
\pgfpathlineto{\pgfqpoint{1.935877in}{0.651333in}}%
\pgfpathlineto{\pgfqpoint{1.938512in}{0.663300in}}%
\pgfpathlineto{\pgfqpoint{1.941123in}{0.703190in}}%
\pgfpathlineto{\pgfqpoint{1.943712in}{0.715157in}}%
\pgfpathlineto{\pgfqpoint{1.948822in}{0.747069in}}%
\pgfpathlineto{\pgfqpoint{1.953845in}{0.755047in}}%
\pgfpathlineto{\pgfqpoint{1.956326in}{0.763025in}}%
\pgfpathlineto{\pgfqpoint{1.961225in}{0.794938in}}%
\pgfpathlineto{\pgfqpoint{1.966044in}{0.818872in}}%
\pgfpathlineto{\pgfqpoint{1.968425in}{0.826850in}}%
\pgfpathlineto{\pgfqpoint{1.970786in}{0.850784in}}%
\pgfpathlineto{\pgfqpoint{1.977760in}{0.862751in}}%
\pgfpathlineto{\pgfqpoint{1.982320in}{0.894663in}}%
\pgfpathlineto{\pgfqpoint{1.984574in}{0.918597in}}%
\pgfpathlineto{\pgfqpoint{1.986811in}{0.930564in}}%
\pgfpathlineto{\pgfqpoint{1.989031in}{0.950509in}}%
\pgfpathlineto{\pgfqpoint{1.993422in}{0.958487in}}%
\pgfpathlineto{\pgfqpoint{1.995593in}{0.970454in}}%
\pgfpathlineto{\pgfqpoint{1.997749in}{0.974443in}}%
\pgfpathlineto{\pgfqpoint{2.006217in}{0.978432in}}%
\pgfpathlineto{\pgfqpoint{2.008297in}{0.982421in}}%
\pgfpathlineto{\pgfqpoint{2.014451in}{0.990399in}}%
\pgfpathlineto{\pgfqpoint{2.016474in}{0.994388in}}%
\pgfpathlineto{\pgfqpoint{2.022462in}{1.002366in}}%
\pgfpathlineto{\pgfqpoint{2.028331in}{1.026300in}}%
\pgfpathlineto{\pgfqpoint{2.030262in}{1.030289in}}%
\pgfpathlineto{\pgfqpoint{2.034086in}{1.034278in}}%
\pgfpathlineto{\pgfqpoint{2.039732in}{1.058213in}}%
\pgfpathlineto{\pgfqpoint{2.043436in}{1.062202in}}%
\pgfpathlineto{\pgfqpoint{2.054281in}{1.078158in}}%
\pgfpathlineto{\pgfqpoint{2.056050in}{1.086136in}}%
\pgfpathlineto{\pgfqpoint{2.059558in}{1.090125in}}%
\pgfpathlineto{\pgfqpoint{2.066451in}{1.094114in}}%
\pgfpathlineto{\pgfqpoint{2.071517in}{1.106081in}}%
\pgfpathlineto{\pgfqpoint{2.074847in}{1.110070in}}%
\pgfpathlineto{\pgfqpoint{2.076499in}{1.114059in}}%
\pgfpathlineto{\pgfqpoint{2.079774in}{1.118048in}}%
\pgfpathlineto{\pgfqpoint{2.081398in}{1.126026in}}%
\pgfpathlineto{\pgfqpoint{2.083013in}{1.130015in}}%
\pgfpathlineto{\pgfqpoint{2.087806in}{1.134004in}}%
\pgfpathlineto{\pgfqpoint{2.089387in}{1.137993in}}%
\pgfpathlineto{\pgfqpoint{2.090959in}{1.149960in}}%
\pgfpathlineto{\pgfqpoint{2.094079in}{1.153949in}}%
\pgfpathlineto{\pgfqpoint{2.098698in}{1.161927in}}%
\pgfpathlineto{\pgfqpoint{2.106240in}{1.181872in}}%
\pgfpathlineto{\pgfqpoint{2.107726in}{1.189850in}}%
\pgfpathlineto{\pgfqpoint{2.112138in}{1.193839in}}%
\pgfpathlineto{\pgfqpoint{2.115044in}{1.197828in}}%
\pgfpathlineto{\pgfqpoint{2.116486in}{1.205806in}}%
\pgfpathlineto{\pgfqpoint{2.117922in}{1.209795in}}%
\pgfpathlineto{\pgfqpoint{2.120772in}{1.213784in}}%
\pgfpathlineto{\pgfqpoint{2.126390in}{1.217773in}}%
\pgfpathlineto{\pgfqpoint{2.130536in}{1.221762in}}%
\pgfpathlineto{\pgfqpoint{2.131905in}{1.225751in}}%
\pgfpathlineto{\pgfqpoint{2.133267in}{1.233729in}}%
\pgfpathlineto{\pgfqpoint{2.138656in}{1.241707in}}%
\pgfpathlineto{\pgfqpoint{2.143949in}{1.245696in}}%
\pgfpathlineto{\pgfqpoint{2.147857in}{1.249685in}}%
\pgfpathlineto{\pgfqpoint{2.149149in}{1.257663in}}%
\pgfpathlineto{\pgfqpoint{2.164222in}{1.261652in}}%
\pgfpathlineto{\pgfqpoint{2.172674in}{1.265641in}}%
\pgfpathlineto{\pgfqpoint{2.175045in}{1.269630in}}%
\pgfpathlineto{\pgfqpoint{2.179731in}{1.273619in}}%
\pgfpathlineto{\pgfqpoint{2.184344in}{1.281597in}}%
\pgfpathlineto{\pgfqpoint{2.188886in}{1.285586in}}%
\pgfpathlineto{\pgfqpoint{2.191132in}{1.289575in}}%
\pgfpathlineto{\pgfqpoint{2.198859in}{1.293564in}}%
\pgfpathlineto{\pgfqpoint{2.201030in}{1.297553in}}%
\pgfpathlineto{\pgfqpoint{2.202110in}{1.301542in}}%
\pgfpathlineto{\pgfqpoint{2.236659in}{1.309520in}}%
\pgfpathlineto{\pgfqpoint{2.237617in}{1.313509in}}%
\pgfpathlineto{\pgfqpoint{2.245169in}{1.321488in}}%
\pgfpathlineto{\pgfqpoint{2.252533in}{1.325477in}}%
\pgfpathlineto{\pgfqpoint{2.277790in}{1.329466in}}%
\pgfpathlineto{\pgfqpoint{2.282758in}{1.337444in}}%
\pgfpathlineto{\pgfqpoint{2.287643in}{1.341433in}}%
\pgfpathlineto{\pgfqpoint{2.296396in}{1.345422in}}%
\pgfpathlineto{\pgfqpoint{2.297960in}{1.349411in}}%
\pgfpathlineto{\pgfqpoint{2.301064in}{1.353400in}}%
\pgfpathlineto{\pgfqpoint{2.319758in}{1.357389in}}%
\pgfpathlineto{\pgfqpoint{2.322642in}{1.365367in}}%
\pgfpathlineto{\pgfqpoint{2.324787in}{1.369356in}}%
\pgfpathlineto{\pgfqpoint{2.335973in}{1.373345in}}%
\pgfpathlineto{\pgfqpoint{2.348730in}{1.377334in}}%
\pgfpathlineto{\pgfqpoint{2.372099in}{1.381323in}}%
\pgfpathlineto{\pgfqpoint{2.372706in}{1.385312in}}%
\pgfpathlineto{\pgfqpoint{2.415522in}{1.389301in}}%
\pgfpathlineto{\pgfqpoint{2.440172in}{1.393290in}}%
\pgfpathlineto{\pgfqpoint{2.457515in}{1.397279in}}%
\pgfpathlineto{\pgfqpoint{2.469559in}{1.401268in}}%
\pgfpathlineto{\pgfqpoint{2.506501in}{1.405257in}}%
\pgfpathlineto{\pgfqpoint{2.525557in}{1.409246in}}%
\pgfpathlineto{\pgfqpoint{2.560345in}{1.413235in}}%
\pgfpathlineto{\pgfqpoint{2.606997in}{1.417224in}}%
\pgfpathlineto{\pgfqpoint{2.615132in}{1.421213in}}%
\pgfpathlineto{\pgfqpoint{2.695881in}{1.425202in}}%
\pgfpathlineto{\pgfqpoint{2.756632in}{1.429191in}}%
\pgfpathlineto{\pgfqpoint{2.760424in}{1.433180in}}%
\pgfpathlineto{\pgfqpoint{2.787643in}{1.437169in}}%
\pgfpathlineto{\pgfqpoint{2.882719in}{1.441158in}}%
\pgfpathlineto{\pgfqpoint{2.996202in}{1.445147in}}%
\pgfpathlineto{\pgfqpoint{3.017320in}{1.449136in}}%
\pgfpathlineto{\pgfqpoint{3.074508in}{1.453125in}}%
\pgfpathlineto{\pgfqpoint{3.100685in}{1.457114in}}%
\pgfpathlineto{\pgfqpoint{3.135235in}{1.461103in}}%
\pgfpathlineto{\pgfqpoint{3.140917in}{1.465092in}}%
\pgfpathlineto{\pgfqpoint{3.162991in}{1.469081in}}%
\pgfpathlineto{\pgfqpoint{3.181880in}{1.473070in}}%
\pgfpathlineto{\pgfqpoint{3.201898in}{1.477059in}}%
\pgfpathlineto{\pgfqpoint{3.283921in}{1.481048in}}%
\pgfpathlineto{\pgfqpoint{3.372195in}{1.485037in}}%
\pgfpathlineto{\pgfqpoint{3.375607in}{1.489026in}}%
\pgfpathlineto{\pgfqpoint{3.508082in}{1.493015in}}%
\pgfpathlineto{\pgfqpoint{3.522165in}{1.497004in}}%
\pgfpathlineto{\pgfqpoint{3.572641in}{1.500993in}}%
\pgfpathlineto{\pgfqpoint{3.672915in}{1.504982in}}%
\pgfpathlineto{\pgfqpoint{3.759691in}{1.508971in}}%
\pgfpathlineto{\pgfqpoint{3.873415in}{1.516949in}}%
\pgfpathlineto{\pgfqpoint{3.896693in}{1.520938in}}%
\pgfpathlineto{\pgfqpoint{3.905645in}{1.524927in}}%
\pgfpathlineto{\pgfqpoint{3.941293in}{1.528916in}}%
\pgfpathlineto{\pgfqpoint{4.128458in}{1.532905in}}%
\pgfpathlineto{\pgfqpoint{4.147919in}{1.536894in}}%
\pgfpathlineto{\pgfqpoint{4.236809in}{1.540883in}}%
\pgfpathlineto{\pgfqpoint{4.260284in}{1.544872in}}%
\pgfpathlineto{\pgfqpoint{4.265468in}{1.548861in}}%
\pgfpathlineto{\pgfqpoint{4.282259in}{1.552850in}}%
\pgfpathlineto{\pgfqpoint{4.297977in}{1.556839in}}%
\pgfpathlineto{\pgfqpoint{4.298604in}{1.560828in}}%
\pgfpathlineto{\pgfqpoint{4.313846in}{1.564817in}}%
\pgfpathlineto{\pgfqpoint{4.316529in}{1.568806in}}%
\pgfpathlineto{\pgfqpoint{4.339683in}{1.572795in}}%
\pgfpathlineto{\pgfqpoint{4.391607in}{1.576784in}}%
\pgfpathlineto{\pgfqpoint{4.392447in}{1.580773in}}%
\pgfpathlineto{\pgfqpoint{4.402074in}{1.584762in}}%
\pgfpathlineto{\pgfqpoint{4.406327in}{1.588751in}}%
\pgfpathlineto{\pgfqpoint{4.573389in}{1.596730in}}%
\pgfpathlineto{\pgfqpoint{4.573389in}{1.596730in}}%
\pgfusepath{stroke}%
\end{pgfscope}%
\begin{pgfscope}%
\pgfpathrectangle{\pgfqpoint{0.537394in}{0.467838in}}{\pgfqpoint{4.094684in}{1.595606in}}%
\pgfusepath{clip}%
\pgfsetrectcap%
\pgfsetroundjoin%
\pgfsetlinewidth{1.003750pt}%
\definecolor{currentstroke}{rgb}{1.000000,0.498039,0.054902}%
\pgfsetstrokecolor{currentstroke}%
\pgfsetdash{}{0pt}%
\pgfpathmoveto{\pgfqpoint{0.537394in}{0.535652in}}%
\pgfpathlineto{\pgfqpoint{0.540929in}{0.539641in}}%
\pgfpathlineto{\pgfqpoint{0.543263in}{0.543630in}}%
\pgfpathlineto{\pgfqpoint{0.549018in}{0.547619in}}%
\pgfpathlineto{\pgfqpoint{0.625240in}{0.551608in}}%
\pgfpathlineto{\pgfqpoint{0.654388in}{0.555597in}}%
\pgfpathlineto{\pgfqpoint{0.701556in}{0.559586in}}%
\pgfpathlineto{\pgfqpoint{0.723387in}{0.563575in}}%
\pgfpathlineto{\pgfqpoint{0.729029in}{0.567564in}}%
\pgfpathlineto{\pgfqpoint{0.731041in}{0.571553in}}%
\pgfpathlineto{\pgfqpoint{0.732425in}{0.575542in}}%
\pgfpathlineto{\pgfqpoint{0.732579in}{0.579531in}}%
\pgfpathlineto{\pgfqpoint{0.736238in}{0.583520in}}%
\pgfpathlineto{\pgfqpoint{0.737749in}{0.587509in}}%
\pgfpathlineto{\pgfqpoint{0.739553in}{0.591498in}}%
\pgfpathlineto{\pgfqpoint{0.746073in}{0.595487in}}%
\pgfpathlineto{\pgfqpoint{0.746805in}{0.599476in}}%
\pgfpathlineto{\pgfqpoint{0.747098in}{0.603465in}}%
\pgfpathlineto{\pgfqpoint{0.749136in}{0.607454in}}%
\pgfpathlineto{\pgfqpoint{0.752310in}{0.615432in}}%
\pgfpathlineto{\pgfqpoint{0.756302in}{0.619421in}}%
\pgfpathlineto{\pgfqpoint{0.760939in}{0.627399in}}%
\pgfpathlineto{\pgfqpoint{0.761217in}{0.631388in}}%
\pgfpathlineto{\pgfqpoint{0.820933in}{0.635377in}}%
\pgfpathlineto{\pgfqpoint{0.821616in}{0.639366in}}%
\pgfpathlineto{\pgfqpoint{0.846447in}{0.643355in}}%
\pgfpathlineto{\pgfqpoint{0.861518in}{0.647344in}}%
\pgfpathlineto{\pgfqpoint{0.862905in}{0.651333in}}%
\pgfpathlineto{\pgfqpoint{0.878873in}{0.655322in}}%
\pgfpathlineto{\pgfqpoint{0.881297in}{0.659311in}}%
\pgfpathlineto{\pgfqpoint{0.881483in}{0.663300in}}%
\pgfpathlineto{\pgfqpoint{0.884990in}{0.667289in}}%
\pgfpathlineto{\pgfqpoint{0.902445in}{0.671278in}}%
\pgfpathlineto{\pgfqpoint{0.902531in}{0.675267in}}%
\pgfpathlineto{\pgfqpoint{0.916712in}{0.679256in}}%
\pgfpathlineto{\pgfqpoint{0.924841in}{0.683245in}}%
\pgfpathlineto{\pgfqpoint{0.927952in}{0.687234in}}%
\pgfpathlineto{\pgfqpoint{0.935628in}{0.691223in}}%
\pgfpathlineto{\pgfqpoint{0.968113in}{0.695212in}}%
\pgfpathlineto{\pgfqpoint{0.988011in}{0.699201in}}%
\pgfpathlineto{\pgfqpoint{0.994611in}{0.703190in}}%
\pgfpathlineto{\pgfqpoint{0.995813in}{0.707179in}}%
\pgfpathlineto{\pgfqpoint{1.003845in}{0.711168in}}%
\pgfpathlineto{\pgfqpoint{1.009322in}{0.715157in}}%
\pgfpathlineto{\pgfqpoint{1.016942in}{0.719146in}}%
\pgfpathlineto{\pgfqpoint{1.025629in}{0.723135in}}%
\pgfpathlineto{\pgfqpoint{1.053447in}{0.727124in}}%
\pgfpathlineto{\pgfqpoint{1.066757in}{0.731113in}}%
\pgfpathlineto{\pgfqpoint{1.073923in}{0.735102in}}%
\pgfpathlineto{\pgfqpoint{1.083747in}{0.739091in}}%
\pgfpathlineto{\pgfqpoint{1.087940in}{0.743080in}}%
\pgfpathlineto{\pgfqpoint{1.097855in}{0.747069in}}%
\pgfpathlineto{\pgfqpoint{1.106190in}{0.751058in}}%
\pgfpathlineto{\pgfqpoint{1.123218in}{0.755047in}}%
\pgfpathlineto{\pgfqpoint{1.130839in}{0.759036in}}%
\pgfpathlineto{\pgfqpoint{1.133151in}{0.763025in}}%
\pgfpathlineto{\pgfqpoint{1.137995in}{0.767014in}}%
\pgfpathlineto{\pgfqpoint{1.140640in}{0.771003in}}%
\pgfpathlineto{\pgfqpoint{1.153483in}{0.774992in}}%
\pgfpathlineto{\pgfqpoint{1.153853in}{0.778981in}}%
\pgfpathlineto{\pgfqpoint{1.157935in}{0.782971in}}%
\pgfpathlineto{\pgfqpoint{1.169293in}{0.794938in}}%
\pgfpathlineto{\pgfqpoint{1.170765in}{0.798927in}}%
\pgfpathlineto{\pgfqpoint{1.172995in}{0.802916in}}%
\pgfpathlineto{\pgfqpoint{1.178191in}{0.806905in}}%
\pgfpathlineto{\pgfqpoint{1.181886in}{0.810894in}}%
\pgfpathlineto{\pgfqpoint{1.186930in}{0.814883in}}%
\pgfpathlineto{\pgfqpoint{1.200618in}{0.818872in}}%
\pgfpathlineto{\pgfqpoint{1.208236in}{0.822861in}}%
\pgfpathlineto{\pgfqpoint{1.223962in}{0.826850in}}%
\pgfpathlineto{\pgfqpoint{1.226552in}{0.830839in}}%
\pgfpathlineto{\pgfqpoint{1.258207in}{0.838817in}}%
\pgfpathlineto{\pgfqpoint{1.268193in}{0.842806in}}%
\pgfpathlineto{\pgfqpoint{1.280280in}{0.846795in}}%
\pgfpathlineto{\pgfqpoint{1.293076in}{0.850784in}}%
\pgfpathlineto{\pgfqpoint{1.295726in}{0.854773in}}%
\pgfpathlineto{\pgfqpoint{1.301092in}{0.858762in}}%
\pgfpathlineto{\pgfqpoint{1.319397in}{0.862751in}}%
\pgfpathlineto{\pgfqpoint{1.330375in}{0.866740in}}%
\pgfpathlineto{\pgfqpoint{1.339480in}{0.870729in}}%
\pgfpathlineto{\pgfqpoint{1.341197in}{0.874718in}}%
\pgfpathlineto{\pgfqpoint{1.342395in}{0.878707in}}%
\pgfpathlineto{\pgfqpoint{1.361081in}{0.882696in}}%
\pgfpathlineto{\pgfqpoint{1.370000in}{0.886685in}}%
\pgfpathlineto{\pgfqpoint{1.388299in}{0.890674in}}%
\pgfpathlineto{\pgfqpoint{1.409295in}{0.894663in}}%
\pgfpathlineto{\pgfqpoint{1.415807in}{0.898652in}}%
\pgfpathlineto{\pgfqpoint{1.418927in}{0.902641in}}%
\pgfpathlineto{\pgfqpoint{1.422838in}{0.906630in}}%
\pgfpathlineto{\pgfqpoint{1.425308in}{0.910619in}}%
\pgfpathlineto{\pgfqpoint{1.425826in}{0.914608in}}%
\pgfpathlineto{\pgfqpoint{1.454361in}{0.918597in}}%
\pgfpathlineto{\pgfqpoint{1.474284in}{0.922586in}}%
\pgfpathlineto{\pgfqpoint{1.475787in}{0.926575in}}%
\pgfpathlineto{\pgfqpoint{1.509087in}{0.930564in}}%
\pgfpathlineto{\pgfqpoint{1.515888in}{0.934553in}}%
\pgfpathlineto{\pgfqpoint{1.521404in}{0.938542in}}%
\pgfpathlineto{\pgfqpoint{1.521842in}{0.942531in}}%
\pgfpathlineto{\pgfqpoint{1.523081in}{0.946520in}}%
\pgfpathlineto{\pgfqpoint{1.529447in}{0.950509in}}%
\pgfpathlineto{\pgfqpoint{1.531111in}{0.954498in}}%
\pgfpathlineto{\pgfqpoint{1.535577in}{0.958487in}}%
\pgfpathlineto{\pgfqpoint{1.541353in}{0.966465in}}%
\pgfpathlineto{\pgfqpoint{1.544422in}{0.970454in}}%
\pgfpathlineto{\pgfqpoint{1.566665in}{0.974443in}}%
\pgfpathlineto{\pgfqpoint{1.594365in}{0.978432in}}%
\pgfpathlineto{\pgfqpoint{1.597307in}{0.982421in}}%
\pgfpathlineto{\pgfqpoint{1.603804in}{0.986410in}}%
\pgfpathlineto{\pgfqpoint{1.625166in}{0.990399in}}%
\pgfpathlineto{\pgfqpoint{1.628424in}{0.994388in}}%
\pgfpathlineto{\pgfqpoint{1.636701in}{1.002366in}}%
\pgfpathlineto{\pgfqpoint{1.639771in}{1.006355in}}%
\pgfpathlineto{\pgfqpoint{1.649797in}{1.010344in}}%
\pgfpathlineto{\pgfqpoint{1.651577in}{1.014333in}}%
\pgfpathlineto{\pgfqpoint{1.664251in}{1.018322in}}%
\pgfpathlineto{\pgfqpoint{1.672634in}{1.022311in}}%
\pgfpathlineto{\pgfqpoint{1.677014in}{1.026300in}}%
\pgfpathlineto{\pgfqpoint{1.717817in}{1.034278in}}%
\pgfpathlineto{\pgfqpoint{1.722143in}{1.042256in}}%
\pgfpathlineto{\pgfqpoint{1.722252in}{1.046245in}}%
\pgfpathlineto{\pgfqpoint{1.724161in}{1.050234in}}%
\pgfpathlineto{\pgfqpoint{1.724917in}{1.054224in}}%
\pgfpathlineto{\pgfqpoint{1.730419in}{1.058213in}}%
\pgfpathlineto{\pgfqpoint{1.747355in}{1.062202in}}%
\pgfpathlineto{\pgfqpoint{1.757657in}{1.066191in}}%
\pgfpathlineto{\pgfqpoint{1.763688in}{1.070180in}}%
\pgfpathlineto{\pgfqpoint{1.764449in}{1.074169in}}%
\pgfpathlineto{\pgfqpoint{1.771144in}{1.078158in}}%
\pgfpathlineto{\pgfqpoint{1.781975in}{1.082147in}}%
\pgfpathlineto{\pgfqpoint{1.787679in}{1.086136in}}%
\pgfpathlineto{\pgfqpoint{1.789372in}{1.090125in}}%
\pgfpathlineto{\pgfqpoint{1.796707in}{1.094114in}}%
\pgfpathlineto{\pgfqpoint{1.800400in}{1.098103in}}%
\pgfpathlineto{\pgfqpoint{1.809058in}{1.102092in}}%
\pgfpathlineto{\pgfqpoint{1.819586in}{1.106081in}}%
\pgfpathlineto{\pgfqpoint{1.834384in}{1.110070in}}%
\pgfpathlineto{\pgfqpoint{1.835804in}{1.114059in}}%
\pgfpathlineto{\pgfqpoint{1.865722in}{1.118048in}}%
\pgfpathlineto{\pgfqpoint{1.884186in}{1.122037in}}%
\pgfpathlineto{\pgfqpoint{1.887834in}{1.126026in}}%
\pgfpathlineto{\pgfqpoint{1.935133in}{1.130015in}}%
\pgfpathlineto{\pgfqpoint{1.935597in}{1.134004in}}%
\pgfpathlineto{\pgfqpoint{1.944831in}{1.137993in}}%
\pgfpathlineto{\pgfqpoint{1.947883in}{1.141982in}}%
\pgfpathlineto{\pgfqpoint{1.958307in}{1.145971in}}%
\pgfpathlineto{\pgfqpoint{1.961654in}{1.149960in}}%
\pgfpathlineto{\pgfqpoint{1.967096in}{1.153949in}}%
\pgfpathlineto{\pgfqpoint{1.970993in}{1.157938in}}%
\pgfpathlineto{\pgfqpoint{1.979599in}{1.161927in}}%
\pgfpathlineto{\pgfqpoint{1.981820in}{1.165916in}}%
\pgfpathlineto{\pgfqpoint{2.024040in}{1.169905in}}%
\pgfpathlineto{\pgfqpoint{2.031000in}{1.173894in}}%
\pgfpathlineto{\pgfqpoint{2.034942in}{1.177883in}}%
\pgfpathlineto{\pgfqpoint{2.047689in}{1.181872in}}%
\pgfpathlineto{\pgfqpoint{2.062060in}{1.185861in}}%
\pgfpathlineto{\pgfqpoint{2.067978in}{1.189850in}}%
\pgfpathlineto{\pgfqpoint{2.069872in}{1.193839in}}%
\pgfpathlineto{\pgfqpoint{2.070427in}{1.197828in}}%
\pgfpathlineto{\pgfqpoint{2.081867in}{1.201817in}}%
\pgfpathlineto{\pgfqpoint{2.087311in}{1.205806in}}%
\pgfpathlineto{\pgfqpoint{2.101794in}{1.209795in}}%
\pgfpathlineto{\pgfqpoint{2.102606in}{1.213784in}}%
\pgfpathlineto{\pgfqpoint{2.110541in}{1.217773in}}%
\pgfpathlineto{\pgfqpoint{2.111213in}{1.221762in}}%
\pgfpathlineto{\pgfqpoint{2.115801in}{1.225751in}}%
\pgfpathlineto{\pgfqpoint{2.116080in}{1.229740in}}%
\pgfpathlineto{\pgfqpoint{2.146020in}{1.237718in}}%
\pgfpathlineto{\pgfqpoint{2.153274in}{1.241707in}}%
\pgfpathlineto{\pgfqpoint{2.172748in}{1.245696in}}%
\pgfpathlineto{\pgfqpoint{2.173632in}{1.249685in}}%
\pgfpathlineto{\pgfqpoint{2.182870in}{1.253674in}}%
\pgfpathlineto{\pgfqpoint{2.202128in}{1.257663in}}%
\pgfpathlineto{\pgfqpoint{2.252138in}{1.261652in}}%
\pgfpathlineto{\pgfqpoint{2.252633in}{1.265641in}}%
\pgfpathlineto{\pgfqpoint{2.254918in}{1.269630in}}%
\pgfpathlineto{\pgfqpoint{2.256259in}{1.273619in}}%
\pgfpathlineto{\pgfqpoint{2.266416in}{1.277608in}}%
\pgfpathlineto{\pgfqpoint{2.279885in}{1.281597in}}%
\pgfpathlineto{\pgfqpoint{2.280145in}{1.285586in}}%
\pgfpathlineto{\pgfqpoint{2.280698in}{1.289575in}}%
\pgfpathlineto{\pgfqpoint{2.286279in}{1.293564in}}%
\pgfpathlineto{\pgfqpoint{2.348026in}{1.297553in}}%
\pgfpathlineto{\pgfqpoint{2.356239in}{1.301542in}}%
\pgfpathlineto{\pgfqpoint{2.356311in}{1.305531in}}%
\pgfpathlineto{\pgfqpoint{2.373416in}{1.309520in}}%
\pgfpathlineto{\pgfqpoint{2.373986in}{1.313509in}}%
\pgfpathlineto{\pgfqpoint{2.388270in}{1.317498in}}%
\pgfpathlineto{\pgfqpoint{2.405466in}{1.321488in}}%
\pgfpathlineto{\pgfqpoint{2.415539in}{1.325477in}}%
\pgfpathlineto{\pgfqpoint{2.424067in}{1.329466in}}%
\pgfpathlineto{\pgfqpoint{2.456914in}{1.333455in}}%
\pgfpathlineto{\pgfqpoint{2.458554in}{1.337444in}}%
\pgfpathlineto{\pgfqpoint{2.459396in}{1.341433in}}%
\pgfpathlineto{\pgfqpoint{2.472267in}{1.345422in}}%
\pgfpathlineto{\pgfqpoint{2.473046in}{1.349411in}}%
\pgfpathlineto{\pgfqpoint{2.545434in}{1.353400in}}%
\pgfpathlineto{\pgfqpoint{2.564093in}{1.357389in}}%
\pgfpathlineto{\pgfqpoint{2.588644in}{1.361378in}}%
\pgfpathlineto{\pgfqpoint{2.594991in}{1.365367in}}%
\pgfpathlineto{\pgfqpoint{2.595722in}{1.369356in}}%
\pgfpathlineto{\pgfqpoint{2.596991in}{1.373345in}}%
\pgfpathlineto{\pgfqpoint{2.618791in}{1.377334in}}%
\pgfpathlineto{\pgfqpoint{2.619450in}{1.381323in}}%
\pgfpathlineto{\pgfqpoint{2.629335in}{1.385312in}}%
\pgfpathlineto{\pgfqpoint{2.633614in}{1.389301in}}%
\pgfpathlineto{\pgfqpoint{2.635773in}{1.393290in}}%
\pgfpathlineto{\pgfqpoint{2.636355in}{1.401268in}}%
\pgfpathlineto{\pgfqpoint{2.645452in}{1.405257in}}%
\pgfpathlineto{\pgfqpoint{2.646069in}{1.409246in}}%
\pgfpathlineto{\pgfqpoint{2.650775in}{1.417224in}}%
\pgfpathlineto{\pgfqpoint{2.764295in}{1.421213in}}%
\pgfpathlineto{\pgfqpoint{2.764870in}{1.425202in}}%
\pgfpathlineto{\pgfqpoint{2.777383in}{1.429191in}}%
\pgfpathlineto{\pgfqpoint{2.813877in}{1.433180in}}%
\pgfpathlineto{\pgfqpoint{2.830016in}{1.437169in}}%
\pgfpathlineto{\pgfqpoint{2.843996in}{1.441158in}}%
\pgfpathlineto{\pgfqpoint{2.859991in}{1.445147in}}%
\pgfpathlineto{\pgfqpoint{2.899051in}{1.449136in}}%
\pgfpathlineto{\pgfqpoint{2.905154in}{1.453125in}}%
\pgfpathlineto{\pgfqpoint{2.906941in}{1.457114in}}%
\pgfpathlineto{\pgfqpoint{2.936506in}{1.461103in}}%
\pgfpathlineto{\pgfqpoint{2.947631in}{1.469081in}}%
\pgfpathlineto{\pgfqpoint{2.950680in}{1.477059in}}%
\pgfpathlineto{\pgfqpoint{2.959625in}{1.481048in}}%
\pgfpathlineto{\pgfqpoint{2.973557in}{1.485037in}}%
\pgfpathlineto{\pgfqpoint{2.979306in}{1.489026in}}%
\pgfpathlineto{\pgfqpoint{2.980632in}{1.493015in}}%
\pgfpathlineto{\pgfqpoint{3.004341in}{1.497004in}}%
\pgfpathlineto{\pgfqpoint{3.007674in}{1.500993in}}%
\pgfpathlineto{\pgfqpoint{3.012362in}{1.508971in}}%
\pgfpathlineto{\pgfqpoint{3.034549in}{1.512960in}}%
\pgfpathlineto{\pgfqpoint{3.040680in}{1.516949in}}%
\pgfpathlineto{\pgfqpoint{3.069901in}{1.520938in}}%
\pgfpathlineto{\pgfqpoint{3.078405in}{1.524927in}}%
\pgfpathlineto{\pgfqpoint{3.078765in}{1.528916in}}%
\pgfpathlineto{\pgfqpoint{3.098213in}{1.532905in}}%
\pgfpathlineto{\pgfqpoint{3.098700in}{1.536894in}}%
\pgfpathlineto{\pgfqpoint{3.099415in}{1.540883in}}%
\pgfpathlineto{\pgfqpoint{3.189598in}{1.548861in}}%
\pgfpathlineto{\pgfqpoint{3.204360in}{1.552850in}}%
\pgfpathlineto{\pgfqpoint{3.228707in}{1.556839in}}%
\pgfpathlineto{\pgfqpoint{3.229381in}{1.560828in}}%
\pgfpathlineto{\pgfqpoint{3.236152in}{1.564817in}}%
\pgfpathlineto{\pgfqpoint{3.327564in}{1.568806in}}%
\pgfpathlineto{\pgfqpoint{3.334106in}{1.572795in}}%
\pgfpathlineto{\pgfqpoint{3.358348in}{1.576784in}}%
\pgfpathlineto{\pgfqpoint{3.371532in}{1.580773in}}%
\pgfpathlineto{\pgfqpoint{3.395640in}{1.584762in}}%
\pgfpathlineto{\pgfqpoint{3.409241in}{1.588751in}}%
\pgfpathlineto{\pgfqpoint{3.413234in}{1.592741in}}%
\pgfpathlineto{\pgfqpoint{3.420807in}{1.596730in}}%
\pgfpathlineto{\pgfqpoint{3.507387in}{1.600719in}}%
\pgfpathlineto{\pgfqpoint{3.510202in}{1.604708in}}%
\pgfpathlineto{\pgfqpoint{3.518922in}{1.608697in}}%
\pgfpathlineto{\pgfqpoint{3.522272in}{1.612686in}}%
\pgfpathlineto{\pgfqpoint{3.589253in}{1.616675in}}%
\pgfpathlineto{\pgfqpoint{3.600664in}{1.620664in}}%
\pgfpathlineto{\pgfqpoint{3.605163in}{1.624653in}}%
\pgfpathlineto{\pgfqpoint{3.627137in}{1.628642in}}%
\pgfpathlineto{\pgfqpoint{3.652046in}{1.632631in}}%
\pgfpathlineto{\pgfqpoint{3.652542in}{1.636620in}}%
\pgfpathlineto{\pgfqpoint{3.654311in}{1.640609in}}%
\pgfpathlineto{\pgfqpoint{3.657121in}{1.648587in}}%
\pgfpathlineto{\pgfqpoint{3.726205in}{1.652576in}}%
\pgfpathlineto{\pgfqpoint{3.727318in}{1.656565in}}%
\pgfpathlineto{\pgfqpoint{3.740276in}{1.660554in}}%
\pgfpathlineto{\pgfqpoint{3.742861in}{1.664543in}}%
\pgfpathlineto{\pgfqpoint{3.779730in}{1.668532in}}%
\pgfpathlineto{\pgfqpoint{3.805105in}{1.672521in}}%
\pgfpathlineto{\pgfqpoint{3.810779in}{1.676510in}}%
\pgfpathlineto{\pgfqpoint{3.881660in}{1.680499in}}%
\pgfpathlineto{\pgfqpoint{3.881812in}{1.684488in}}%
\pgfpathlineto{\pgfqpoint{3.890978in}{1.688477in}}%
\pgfpathlineto{\pgfqpoint{3.891729in}{1.692466in}}%
\pgfpathlineto{\pgfqpoint{3.958314in}{1.696455in}}%
\pgfpathlineto{\pgfqpoint{3.964192in}{1.704433in}}%
\pgfpathlineto{\pgfqpoint{3.991108in}{1.708422in}}%
\pgfpathlineto{\pgfqpoint{3.995548in}{1.712411in}}%
\pgfpathlineto{\pgfqpoint{3.997605in}{1.716400in}}%
\pgfpathlineto{\pgfqpoint{4.003495in}{1.720389in}}%
\pgfpathlineto{\pgfqpoint{4.040950in}{1.724378in}}%
\pgfpathlineto{\pgfqpoint{4.069642in}{1.728367in}}%
\pgfpathlineto{\pgfqpoint{4.074603in}{1.732356in}}%
\pgfpathlineto{\pgfqpoint{4.076124in}{1.736345in}}%
\pgfpathlineto{\pgfqpoint{4.102029in}{1.740334in}}%
\pgfpathlineto{\pgfqpoint{4.149646in}{1.744323in}}%
\pgfpathlineto{\pgfqpoint{4.172457in}{1.748312in}}%
\pgfpathlineto{\pgfqpoint{4.183988in}{1.752301in}}%
\pgfpathlineto{\pgfqpoint{4.186390in}{1.756290in}}%
\pgfpathlineto{\pgfqpoint{4.197995in}{1.760279in}}%
\pgfpathlineto{\pgfqpoint{4.213330in}{1.764268in}}%
\pgfpathlineto{\pgfqpoint{4.265982in}{1.768257in}}%
\pgfpathlineto{\pgfqpoint{4.268074in}{1.772246in}}%
\pgfpathlineto{\pgfqpoint{4.284052in}{1.776235in}}%
\pgfpathlineto{\pgfqpoint{4.289585in}{1.780224in}}%
\pgfpathlineto{\pgfqpoint{4.290302in}{1.784213in}}%
\pgfpathlineto{\pgfqpoint{4.290618in}{1.788202in}}%
\pgfpathlineto{\pgfqpoint{4.304571in}{1.792191in}}%
\pgfpathlineto{\pgfqpoint{4.337329in}{1.796180in}}%
\pgfpathlineto{\pgfqpoint{4.362388in}{1.800169in}}%
\pgfpathlineto{\pgfqpoint{4.363279in}{1.804158in}}%
\pgfpathlineto{\pgfqpoint{4.364971in}{1.808147in}}%
\pgfpathlineto{\pgfqpoint{4.424704in}{1.812136in}}%
\pgfpathlineto{\pgfqpoint{4.438966in}{1.816125in}}%
\pgfpathlineto{\pgfqpoint{4.448353in}{1.820114in}}%
\pgfpathlineto{\pgfqpoint{4.469375in}{1.824103in}}%
\pgfpathlineto{\pgfqpoint{4.472714in}{1.828092in}}%
\pgfpathlineto{\pgfqpoint{4.530858in}{1.832081in}}%
\pgfpathlineto{\pgfqpoint{4.538327in}{1.836070in}}%
\pgfpathlineto{\pgfqpoint{4.538787in}{1.840059in}}%
\pgfpathlineto{\pgfqpoint{4.595785in}{1.844048in}}%
\pgfpathlineto{\pgfqpoint{4.595785in}{1.844048in}}%
\pgfusepath{stroke}%
\end{pgfscope}%
\begin{pgfscope}%
\pgfpathrectangle{\pgfqpoint{0.537394in}{0.467838in}}{\pgfqpoint{4.094684in}{1.595606in}}%
\pgfusepath{clip}%
\pgfsetrectcap%
\pgfsetroundjoin%
\pgfsetlinewidth{1.003750pt}%
\definecolor{currentstroke}{rgb}{0.172549,0.627451,0.172549}%
\pgfsetstrokecolor{currentstroke}%
\pgfsetdash{}{0pt}%
\pgfpathmoveto{\pgfqpoint{1.342180in}{0.471827in}}%
\pgfpathlineto{\pgfqpoint{1.350420in}{0.475816in}}%
\pgfpathlineto{\pgfqpoint{1.353759in}{0.483794in}}%
\pgfpathlineto{\pgfqpoint{1.355845in}{0.491772in}}%
\pgfpathlineto{\pgfqpoint{1.358549in}{0.495761in}}%
\pgfpathlineto{\pgfqpoint{1.359365in}{0.499750in}}%
\pgfpathlineto{\pgfqpoint{1.367471in}{0.503739in}}%
\pgfpathlineto{\pgfqpoint{1.367903in}{0.507728in}}%
\pgfpathlineto{\pgfqpoint{1.371851in}{0.511717in}}%
\pgfpathlineto{\pgfqpoint{1.374008in}{0.515707in}}%
\pgfpathlineto{\pgfqpoint{1.380318in}{0.519696in}}%
\pgfpathlineto{\pgfqpoint{1.383219in}{0.523685in}}%
\pgfpathlineto{\pgfqpoint{1.386531in}{0.527674in}}%
\pgfpathlineto{\pgfqpoint{1.387660in}{0.531663in}}%
\pgfpathlineto{\pgfqpoint{1.392185in}{0.535652in}}%
\pgfpathlineto{\pgfqpoint{1.401546in}{0.539641in}}%
\pgfpathlineto{\pgfqpoint{1.402588in}{0.547619in}}%
\pgfpathlineto{\pgfqpoint{1.407302in}{0.551608in}}%
\pgfpathlineto{\pgfqpoint{1.410979in}{0.555597in}}%
\pgfpathlineto{\pgfqpoint{1.417683in}{0.559586in}}%
\pgfpathlineto{\pgfqpoint{1.436974in}{0.563575in}}%
\pgfpathlineto{\pgfqpoint{1.438523in}{0.567564in}}%
\pgfpathlineto{\pgfqpoint{1.438976in}{0.571553in}}%
\pgfpathlineto{\pgfqpoint{1.439047in}{0.575542in}}%
\pgfpathlineto{\pgfqpoint{1.441301in}{0.579531in}}%
\pgfpathlineto{\pgfqpoint{1.441554in}{0.583520in}}%
\pgfpathlineto{\pgfqpoint{1.442269in}{0.587509in}}%
\pgfpathlineto{\pgfqpoint{1.442618in}{0.591498in}}%
\pgfpathlineto{\pgfqpoint{1.445373in}{0.595487in}}%
\pgfpathlineto{\pgfqpoint{1.449321in}{0.599476in}}%
\pgfpathlineto{\pgfqpoint{1.450562in}{0.603465in}}%
\pgfpathlineto{\pgfqpoint{1.452636in}{0.607454in}}%
\pgfpathlineto{\pgfqpoint{1.454307in}{0.611443in}}%
\pgfpathlineto{\pgfqpoint{1.454804in}{0.615432in}}%
\pgfpathlineto{\pgfqpoint{1.457807in}{0.619421in}}%
\pgfpathlineto{\pgfqpoint{1.463267in}{0.623410in}}%
\pgfpathlineto{\pgfqpoint{1.475137in}{0.627399in}}%
\pgfpathlineto{\pgfqpoint{1.478846in}{0.631388in}}%
\pgfpathlineto{\pgfqpoint{1.479217in}{0.635377in}}%
\pgfpathlineto{\pgfqpoint{1.480720in}{0.639366in}}%
\pgfpathlineto{\pgfqpoint{1.482985in}{0.643355in}}%
\pgfpathlineto{\pgfqpoint{1.487500in}{0.647344in}}%
\pgfpathlineto{\pgfqpoint{1.492397in}{0.655322in}}%
\pgfpathlineto{\pgfqpoint{1.495172in}{0.659311in}}%
\pgfpathlineto{\pgfqpoint{1.495395in}{0.663300in}}%
\pgfpathlineto{\pgfqpoint{1.495991in}{0.667289in}}%
\pgfpathlineto{\pgfqpoint{1.499368in}{0.671278in}}%
\pgfpathlineto{\pgfqpoint{1.505027in}{0.675267in}}%
\pgfpathlineto{\pgfqpoint{1.508393in}{0.679256in}}%
\pgfpathlineto{\pgfqpoint{1.521532in}{0.683245in}}%
\pgfpathlineto{\pgfqpoint{1.526618in}{0.687234in}}%
\pgfpathlineto{\pgfqpoint{1.537156in}{0.691223in}}%
\pgfpathlineto{\pgfqpoint{1.542782in}{0.695212in}}%
\pgfpathlineto{\pgfqpoint{1.563140in}{0.699201in}}%
\pgfpathlineto{\pgfqpoint{1.584901in}{0.703190in}}%
\pgfpathlineto{\pgfqpoint{1.585945in}{0.707179in}}%
\pgfpathlineto{\pgfqpoint{1.589636in}{0.711168in}}%
\pgfpathlineto{\pgfqpoint{1.590164in}{0.715157in}}%
\pgfpathlineto{\pgfqpoint{1.604938in}{0.719146in}}%
\pgfpathlineto{\pgfqpoint{1.605236in}{0.723135in}}%
\pgfpathlineto{\pgfqpoint{1.612455in}{0.727124in}}%
\pgfpathlineto{\pgfqpoint{1.617224in}{0.731113in}}%
\pgfpathlineto{\pgfqpoint{1.631315in}{0.735102in}}%
\pgfpathlineto{\pgfqpoint{1.633643in}{0.739091in}}%
\pgfpathlineto{\pgfqpoint{1.641105in}{0.743080in}}%
\pgfpathlineto{\pgfqpoint{1.650464in}{0.747069in}}%
\pgfpathlineto{\pgfqpoint{1.652384in}{0.751058in}}%
\pgfpathlineto{\pgfqpoint{1.666952in}{0.755047in}}%
\pgfpathlineto{\pgfqpoint{1.671926in}{0.759036in}}%
\pgfpathlineto{\pgfqpoint{1.672164in}{0.763025in}}%
\pgfpathlineto{\pgfqpoint{1.675107in}{0.767014in}}%
\pgfpathlineto{\pgfqpoint{1.675559in}{0.771003in}}%
\pgfpathlineto{\pgfqpoint{1.679124in}{0.774992in}}%
\pgfpathlineto{\pgfqpoint{1.679231in}{0.778981in}}%
\pgfpathlineto{\pgfqpoint{1.682832in}{0.786960in}}%
\pgfpathlineto{\pgfqpoint{1.685240in}{0.790949in}}%
\pgfpathlineto{\pgfqpoint{1.686489in}{0.794938in}}%
\pgfpathlineto{\pgfqpoint{1.688337in}{0.798927in}}%
\pgfpathlineto{\pgfqpoint{1.693459in}{0.802916in}}%
\pgfpathlineto{\pgfqpoint{1.711803in}{0.806905in}}%
\pgfpathlineto{\pgfqpoint{1.723251in}{0.810894in}}%
\pgfpathlineto{\pgfqpoint{1.727809in}{0.814883in}}%
\pgfpathlineto{\pgfqpoint{1.739560in}{0.818872in}}%
\pgfpathlineto{\pgfqpoint{1.740032in}{0.822861in}}%
\pgfpathlineto{\pgfqpoint{1.741881in}{0.826850in}}%
\pgfpathlineto{\pgfqpoint{1.742161in}{0.830839in}}%
\pgfpathlineto{\pgfqpoint{1.743876in}{0.834828in}}%
\pgfpathlineto{\pgfqpoint{1.766349in}{0.838817in}}%
\pgfpathlineto{\pgfqpoint{1.769637in}{0.842806in}}%
\pgfpathlineto{\pgfqpoint{1.772769in}{0.850784in}}%
\pgfpathlineto{\pgfqpoint{1.803237in}{0.854773in}}%
\pgfpathlineto{\pgfqpoint{1.803440in}{0.858762in}}%
\pgfpathlineto{\pgfqpoint{1.822040in}{0.862751in}}%
\pgfpathlineto{\pgfqpoint{1.829032in}{0.866740in}}%
\pgfpathlineto{\pgfqpoint{1.829628in}{0.870729in}}%
\pgfpathlineto{\pgfqpoint{1.836153in}{0.874718in}}%
\pgfpathlineto{\pgfqpoint{1.837929in}{0.878707in}}%
\pgfpathlineto{\pgfqpoint{1.838301in}{0.882696in}}%
\pgfpathlineto{\pgfqpoint{1.841313in}{0.886685in}}%
\pgfpathlineto{\pgfqpoint{1.845506in}{0.890674in}}%
\pgfpathlineto{\pgfqpoint{1.845908in}{0.898652in}}%
\pgfpathlineto{\pgfqpoint{1.855599in}{0.902641in}}%
\pgfpathlineto{\pgfqpoint{1.857097in}{0.906630in}}%
\pgfpathlineto{\pgfqpoint{1.861327in}{0.910619in}}%
\pgfpathlineto{\pgfqpoint{1.864724in}{0.914608in}}%
\pgfpathlineto{\pgfqpoint{1.868986in}{0.918597in}}%
\pgfpathlineto{\pgfqpoint{1.883729in}{0.922586in}}%
\pgfpathlineto{\pgfqpoint{1.886411in}{0.926575in}}%
\pgfpathlineto{\pgfqpoint{1.891594in}{0.930564in}}%
\pgfpathlineto{\pgfqpoint{1.893682in}{0.934553in}}%
\pgfpathlineto{\pgfqpoint{1.896252in}{0.938542in}}%
\pgfpathlineto{\pgfqpoint{1.900687in}{0.942531in}}%
\pgfpathlineto{\pgfqpoint{1.910565in}{0.946520in}}%
\pgfpathlineto{\pgfqpoint{1.912381in}{0.950509in}}%
\pgfpathlineto{\pgfqpoint{1.918022in}{0.954498in}}%
\pgfpathlineto{\pgfqpoint{1.925362in}{0.958487in}}%
\pgfpathlineto{\pgfqpoint{1.930486in}{0.962476in}}%
\pgfpathlineto{\pgfqpoint{1.931237in}{0.966465in}}%
\pgfpathlineto{\pgfqpoint{1.934567in}{0.970454in}}%
\pgfpathlineto{\pgfqpoint{1.936963in}{0.974443in}}%
\pgfpathlineto{\pgfqpoint{1.938415in}{0.978432in}}%
\pgfpathlineto{\pgfqpoint{1.940207in}{0.982421in}}%
\pgfpathlineto{\pgfqpoint{1.957098in}{0.986410in}}%
\pgfpathlineto{\pgfqpoint{1.959062in}{0.990399in}}%
\pgfpathlineto{\pgfqpoint{1.974277in}{0.994388in}}%
\pgfpathlineto{\pgfqpoint{1.991772in}{0.998377in}}%
\pgfpathlineto{\pgfqpoint{2.003342in}{1.002366in}}%
\pgfpathlineto{\pgfqpoint{2.003344in}{1.006355in}}%
\pgfpathlineto{\pgfqpoint{2.017759in}{1.010344in}}%
\pgfpathlineto{\pgfqpoint{2.058732in}{1.014333in}}%
\pgfpathlineto{\pgfqpoint{2.059401in}{1.018322in}}%
\pgfpathlineto{\pgfqpoint{2.065119in}{1.022311in}}%
\pgfpathlineto{\pgfqpoint{2.069545in}{1.026300in}}%
\pgfpathlineto{\pgfqpoint{2.070434in}{1.030289in}}%
\pgfpathlineto{\pgfqpoint{2.081951in}{1.034278in}}%
\pgfpathlineto{\pgfqpoint{2.085849in}{1.038267in}}%
\pgfpathlineto{\pgfqpoint{2.088014in}{1.042256in}}%
\pgfpathlineto{\pgfqpoint{2.096951in}{1.046245in}}%
\pgfpathlineto{\pgfqpoint{2.102975in}{1.050234in}}%
\pgfpathlineto{\pgfqpoint{2.123679in}{1.054224in}}%
\pgfpathlineto{\pgfqpoint{2.128792in}{1.058213in}}%
\pgfpathlineto{\pgfqpoint{2.130518in}{1.062202in}}%
\pgfpathlineto{\pgfqpoint{2.140450in}{1.066191in}}%
\pgfpathlineto{\pgfqpoint{2.141929in}{1.070180in}}%
\pgfpathlineto{\pgfqpoint{2.146676in}{1.074169in}}%
\pgfpathlineto{\pgfqpoint{2.153049in}{1.078158in}}%
\pgfpathlineto{\pgfqpoint{2.166133in}{1.082147in}}%
\pgfpathlineto{\pgfqpoint{2.167989in}{1.086136in}}%
\pgfpathlineto{\pgfqpoint{2.202769in}{1.090125in}}%
\pgfpathlineto{\pgfqpoint{2.204486in}{1.094114in}}%
\pgfpathlineto{\pgfqpoint{2.253758in}{1.102092in}}%
\pgfpathlineto{\pgfqpoint{2.262080in}{1.106081in}}%
\pgfpathlineto{\pgfqpoint{2.277607in}{1.110070in}}%
\pgfpathlineto{\pgfqpoint{2.278920in}{1.114059in}}%
\pgfpathlineto{\pgfqpoint{2.286192in}{1.118048in}}%
\pgfpathlineto{\pgfqpoint{2.286953in}{1.122037in}}%
\pgfpathlineto{\pgfqpoint{2.309664in}{1.126026in}}%
\pgfpathlineto{\pgfqpoint{2.313320in}{1.130015in}}%
\pgfpathlineto{\pgfqpoint{2.317962in}{1.134004in}}%
\pgfpathlineto{\pgfqpoint{2.319053in}{1.137993in}}%
\pgfpathlineto{\pgfqpoint{2.319865in}{1.141982in}}%
\pgfpathlineto{\pgfqpoint{2.320257in}{1.149960in}}%
\pgfpathlineto{\pgfqpoint{2.321138in}{1.153949in}}%
\pgfpathlineto{\pgfqpoint{2.326883in}{1.157938in}}%
\pgfpathlineto{\pgfqpoint{2.328811in}{1.161927in}}%
\pgfpathlineto{\pgfqpoint{2.352062in}{1.165916in}}%
\pgfpathlineto{\pgfqpoint{2.362161in}{1.173894in}}%
\pgfpathlineto{\pgfqpoint{2.423065in}{1.177883in}}%
\pgfpathlineto{\pgfqpoint{2.423273in}{1.181872in}}%
\pgfpathlineto{\pgfqpoint{2.438977in}{1.185861in}}%
\pgfpathlineto{\pgfqpoint{2.439051in}{1.189850in}}%
\pgfpathlineto{\pgfqpoint{2.527849in}{1.193839in}}%
\pgfpathlineto{\pgfqpoint{2.532714in}{1.197828in}}%
\pgfpathlineto{\pgfqpoint{2.576660in}{1.205806in}}%
\pgfpathlineto{\pgfqpoint{2.589179in}{1.209795in}}%
\pgfpathlineto{\pgfqpoint{2.605856in}{1.213784in}}%
\pgfpathlineto{\pgfqpoint{2.609206in}{1.217773in}}%
\pgfpathlineto{\pgfqpoint{2.618253in}{1.221762in}}%
\pgfpathlineto{\pgfqpoint{2.631412in}{1.225751in}}%
\pgfpathlineto{\pgfqpoint{2.683592in}{1.229740in}}%
\pgfpathlineto{\pgfqpoint{2.698674in}{1.233729in}}%
\pgfpathlineto{\pgfqpoint{2.701069in}{1.237718in}}%
\pgfpathlineto{\pgfqpoint{2.717112in}{1.241707in}}%
\pgfpathlineto{\pgfqpoint{2.756558in}{1.245696in}}%
\pgfpathlineto{\pgfqpoint{2.759829in}{1.249685in}}%
\pgfpathlineto{\pgfqpoint{2.760965in}{1.253674in}}%
\pgfpathlineto{\pgfqpoint{2.762342in}{1.257663in}}%
\pgfpathlineto{\pgfqpoint{2.817255in}{1.261652in}}%
\pgfpathlineto{\pgfqpoint{2.818709in}{1.265641in}}%
\pgfpathlineto{\pgfqpoint{2.825033in}{1.269630in}}%
\pgfpathlineto{\pgfqpoint{2.842029in}{1.273619in}}%
\pgfpathlineto{\pgfqpoint{2.850723in}{1.277608in}}%
\pgfpathlineto{\pgfqpoint{2.852021in}{1.281597in}}%
\pgfpathlineto{\pgfqpoint{2.865127in}{1.285586in}}%
\pgfpathlineto{\pgfqpoint{2.870434in}{1.289575in}}%
\pgfpathlineto{\pgfqpoint{2.871772in}{1.293564in}}%
\pgfpathlineto{\pgfqpoint{2.871967in}{1.297553in}}%
\pgfpathlineto{\pgfqpoint{2.882868in}{1.301542in}}%
\pgfpathlineto{\pgfqpoint{2.909067in}{1.305531in}}%
\pgfpathlineto{\pgfqpoint{2.940462in}{1.309520in}}%
\pgfpathlineto{\pgfqpoint{2.974255in}{1.313509in}}%
\pgfpathlineto{\pgfqpoint{2.974603in}{1.317498in}}%
\pgfpathlineto{\pgfqpoint{3.048358in}{1.325477in}}%
\pgfpathlineto{\pgfqpoint{3.048807in}{1.329466in}}%
\pgfpathlineto{\pgfqpoint{3.073088in}{1.333455in}}%
\pgfpathlineto{\pgfqpoint{3.161366in}{1.337444in}}%
\pgfpathlineto{\pgfqpoint{3.285689in}{1.341433in}}%
\pgfpathlineto{\pgfqpoint{3.323349in}{1.345422in}}%
\pgfpathlineto{\pgfqpoint{3.323610in}{1.349411in}}%
\pgfpathlineto{\pgfqpoint{3.327674in}{1.353400in}}%
\pgfpathlineto{\pgfqpoint{3.329328in}{1.357389in}}%
\pgfpathlineto{\pgfqpoint{3.366325in}{1.361378in}}%
\pgfpathlineto{\pgfqpoint{3.368372in}{1.365367in}}%
\pgfpathlineto{\pgfqpoint{3.384222in}{1.369356in}}%
\pgfpathlineto{\pgfqpoint{3.393233in}{1.373345in}}%
\pgfpathlineto{\pgfqpoint{3.396060in}{1.377334in}}%
\pgfpathlineto{\pgfqpoint{3.451292in}{1.381323in}}%
\pgfpathlineto{\pgfqpoint{3.481816in}{1.385312in}}%
\pgfpathlineto{\pgfqpoint{3.485158in}{1.389301in}}%
\pgfpathlineto{\pgfqpoint{3.496707in}{1.393290in}}%
\pgfpathlineto{\pgfqpoint{3.497651in}{1.397279in}}%
\pgfpathlineto{\pgfqpoint{3.508348in}{1.401268in}}%
\pgfpathlineto{\pgfqpoint{3.519751in}{1.405257in}}%
\pgfpathlineto{\pgfqpoint{3.519827in}{1.409246in}}%
\pgfpathlineto{\pgfqpoint{3.608897in}{1.413235in}}%
\pgfpathlineto{\pgfqpoint{3.624164in}{1.417224in}}%
\pgfpathlineto{\pgfqpoint{3.641850in}{1.421213in}}%
\pgfpathlineto{\pgfqpoint{3.658010in}{1.425202in}}%
\pgfpathlineto{\pgfqpoint{3.659676in}{1.429191in}}%
\pgfpathlineto{\pgfqpoint{3.705596in}{1.433180in}}%
\pgfpathlineto{\pgfqpoint{3.710094in}{1.437169in}}%
\pgfpathlineto{\pgfqpoint{3.780590in}{1.441158in}}%
\pgfpathlineto{\pgfqpoint{3.817843in}{1.449136in}}%
\pgfpathlineto{\pgfqpoint{3.829157in}{1.457114in}}%
\pgfpathlineto{\pgfqpoint{3.836489in}{1.461103in}}%
\pgfpathlineto{\pgfqpoint{3.863371in}{1.465092in}}%
\pgfpathlineto{\pgfqpoint{3.984326in}{1.469081in}}%
\pgfpathlineto{\pgfqpoint{4.063822in}{1.473070in}}%
\pgfpathlineto{\pgfqpoint{4.111616in}{1.477059in}}%
\pgfpathlineto{\pgfqpoint{4.112533in}{1.481048in}}%
\pgfpathlineto{\pgfqpoint{4.123748in}{1.485037in}}%
\pgfpathlineto{\pgfqpoint{4.140888in}{1.489026in}}%
\pgfpathlineto{\pgfqpoint{4.147863in}{1.493015in}}%
\pgfpathlineto{\pgfqpoint{4.176247in}{1.497004in}}%
\pgfpathlineto{\pgfqpoint{4.183698in}{1.500993in}}%
\pgfpathlineto{\pgfqpoint{4.201968in}{1.504982in}}%
\pgfpathlineto{\pgfqpoint{4.204720in}{1.508971in}}%
\pgfpathlineto{\pgfqpoint{4.240207in}{1.512960in}}%
\pgfpathlineto{\pgfqpoint{4.240947in}{1.516949in}}%
\pgfpathlineto{\pgfqpoint{4.243705in}{1.520938in}}%
\pgfpathlineto{\pgfqpoint{4.245946in}{1.524927in}}%
\pgfpathlineto{\pgfqpoint{4.265760in}{1.528916in}}%
\pgfpathlineto{\pgfqpoint{4.292201in}{1.532905in}}%
\pgfpathlineto{\pgfqpoint{4.320895in}{1.536894in}}%
\pgfpathlineto{\pgfqpoint{4.328583in}{1.540883in}}%
\pgfpathlineto{\pgfqpoint{4.346740in}{1.544872in}}%
\pgfpathlineto{\pgfqpoint{4.350532in}{1.548861in}}%
\pgfpathlineto{\pgfqpoint{4.371670in}{1.552850in}}%
\pgfpathlineto{\pgfqpoint{4.521094in}{1.556839in}}%
\pgfpathlineto{\pgfqpoint{4.545738in}{1.560828in}}%
\pgfpathlineto{\pgfqpoint{4.563427in}{1.564817in}}%
\pgfpathlineto{\pgfqpoint{4.563427in}{1.564817in}}%
\pgfusepath{stroke}%
\end{pgfscope}%
\begin{pgfscope}%
\pgfpathrectangle{\pgfqpoint{0.537394in}{0.467838in}}{\pgfqpoint{4.094684in}{1.595606in}}%
\pgfusepath{clip}%
\pgfsetrectcap%
\pgfsetroundjoin%
\pgfsetlinewidth{1.003750pt}%
\definecolor{currentstroke}{rgb}{0.839216,0.152941,0.156863}%
\pgfsetstrokecolor{currentstroke}%
\pgfsetdash{}{0pt}%
\pgfpathmoveto{\pgfqpoint{0.537394in}{0.511717in}}%
\pgfpathlineto{\pgfqpoint{1.219841in}{0.579531in}}%
\pgfpathlineto{\pgfqpoint{1.425278in}{0.611443in}}%
\pgfpathlineto{\pgfqpoint{1.630715in}{0.619421in}}%
\pgfpathlineto{\pgfqpoint{1.696851in}{0.623410in}}%
\pgfpathlineto{\pgfqpoint{1.902289in}{0.627399in}}%
\pgfpathlineto{\pgfqpoint{1.956326in}{0.635377in}}%
\pgfpathlineto{\pgfqpoint{2.002013in}{0.639366in}}%
\pgfpathlineto{\pgfqpoint{2.022462in}{0.647344in}}%
\pgfpathlineto{\pgfqpoint{2.041590in}{0.651333in}}%
\pgfpathlineto{\pgfqpoint{2.185486in}{0.659311in}}%
\pgfpathlineto{\pgfqpoint{2.196672in}{0.663300in}}%
\pgfpathlineto{\pgfqpoint{2.297960in}{0.667289in}}%
\pgfpathlineto{\pgfqpoint{2.334597in}{0.675267in}}%
\pgfpathlineto{\pgfqpoint{2.341411in}{0.679256in}}%
\pgfpathlineto{\pgfqpoint{2.423288in}{0.683245in}}%
\pgfpathlineto{\pgfqpoint{2.433336in}{0.691223in}}%
\pgfpathlineto{\pgfqpoint{2.438235in}{0.699201in}}%
\pgfpathlineto{\pgfqpoint{2.443054in}{0.703190in}}%
\pgfpathlineto{\pgfqpoint{2.452464in}{0.707179in}}%
\pgfpathlineto{\pgfqpoint{2.457059in}{0.711168in}}%
\pgfpathlineto{\pgfqpoint{2.479023in}{0.715157in}}%
\pgfpathlineto{\pgfqpoint{2.499472in}{0.719146in}}%
\pgfpathlineto{\pgfqpoint{2.618325in}{0.723135in}}%
\pgfpathlineto{\pgfqpoint{2.653234in}{0.727124in}}%
\pgfpathlineto{\pgfqpoint{2.655576in}{0.731113in}}%
\pgfpathlineto{\pgfqpoint{2.657901in}{0.735102in}}%
\pgfpathlineto{\pgfqpoint{2.671478in}{0.739091in}}%
\pgfpathlineto{\pgfqpoint{2.675869in}{0.743080in}}%
\pgfpathlineto{\pgfqpoint{2.750597in}{0.747069in}}%
\pgfpathlineto{\pgfqpoint{2.757295in}{0.751058in}}%
\pgfpathlineto{\pgfqpoint{2.758946in}{0.755047in}}%
\pgfpathlineto{\pgfqpoint{2.771834in}{0.759036in}}%
\pgfpathlineto{\pgfqpoint{2.781145in}{0.767014in}}%
\pgfpathlineto{\pgfqpoint{2.808838in}{0.771003in}}%
\pgfpathlineto{\pgfqpoint{2.815715in}{0.774992in}}%
\pgfpathlineto{\pgfqpoint{2.895144in}{0.778981in}}%
\pgfpathlineto{\pgfqpoint{2.900298in}{0.782971in}}%
\pgfpathlineto{\pgfqpoint{2.912315in}{0.786960in}}%
\pgfpathlineto{\pgfqpoint{2.933156in}{0.790949in}}%
\pgfpathlineto{\pgfqpoint{2.943935in}{0.794938in}}%
\pgfpathlineto{\pgfqpoint{2.944816in}{0.798927in}}%
\pgfpathlineto{\pgfqpoint{2.954335in}{0.802916in}}%
\pgfpathlineto{\pgfqpoint{2.968471in}{0.806905in}}%
\pgfpathlineto{\pgfqpoint{2.970091in}{0.810894in}}%
\pgfpathlineto{\pgfqpoint{2.994868in}{0.814883in}}%
\pgfpathlineto{\pgfqpoint{3.161047in}{0.818872in}}%
\pgfpathlineto{\pgfqpoint{3.167340in}{0.822861in}}%
\pgfpathlineto{\pgfqpoint{3.174314in}{0.826850in}}%
\pgfpathlineto{\pgfqpoint{3.177941in}{0.830839in}}%
\pgfpathlineto{\pgfqpoint{3.228959in}{0.834828in}}%
\pgfpathlineto{\pgfqpoint{3.233642in}{0.838817in}}%
\pgfpathlineto{\pgfqpoint{3.240855in}{0.846795in}}%
\pgfpathlineto{\pgfqpoint{3.241502in}{0.850784in}}%
\pgfpathlineto{\pgfqpoint{3.328780in}{0.854773in}}%
\pgfpathlineto{\pgfqpoint{3.362214in}{0.858762in}}%
\pgfpathlineto{\pgfqpoint{3.425670in}{0.862751in}}%
\pgfpathlineto{\pgfqpoint{3.436914in}{0.866740in}}%
\pgfpathlineto{\pgfqpoint{3.438082in}{0.870729in}}%
\pgfpathlineto{\pgfqpoint{3.446778in}{0.874718in}}%
\pgfpathlineto{\pgfqpoint{3.451271in}{0.878707in}}%
\pgfpathlineto{\pgfqpoint{3.454754in}{0.882696in}}%
\pgfpathlineto{\pgfqpoint{3.456793in}{0.886685in}}%
\pgfpathlineto{\pgfqpoint{3.457730in}{0.890674in}}%
\pgfpathlineto{\pgfqpoint{3.490589in}{0.894663in}}%
\pgfpathlineto{\pgfqpoint{3.504484in}{0.898652in}}%
\pgfpathlineto{\pgfqpoint{3.527890in}{0.902641in}}%
\pgfpathlineto{\pgfqpoint{3.575504in}{0.906630in}}%
\pgfpathlineto{\pgfqpoint{3.575922in}{0.910619in}}%
\pgfpathlineto{\pgfqpoint{3.576654in}{0.914608in}}%
\pgfpathlineto{\pgfqpoint{3.758993in}{0.918597in}}%
\pgfpathlineto{\pgfqpoint{3.812973in}{0.922586in}}%
\pgfpathlineto{\pgfqpoint{3.820354in}{0.926575in}}%
\pgfpathlineto{\pgfqpoint{3.821315in}{0.930564in}}%
\pgfpathlineto{\pgfqpoint{3.831863in}{0.934553in}}%
\pgfpathlineto{\pgfqpoint{3.864919in}{0.938542in}}%
\pgfpathlineto{\pgfqpoint{3.887871in}{0.942531in}}%
\pgfpathlineto{\pgfqpoint{3.916554in}{0.946520in}}%
\pgfpathlineto{\pgfqpoint{4.002728in}{0.950509in}}%
\pgfpathlineto{\pgfqpoint{4.026523in}{0.954498in}}%
\pgfpathlineto{\pgfqpoint{4.040437in}{0.958487in}}%
\pgfpathlineto{\pgfqpoint{4.113622in}{0.962476in}}%
\pgfpathlineto{\pgfqpoint{4.114847in}{0.966465in}}%
\pgfpathlineto{\pgfqpoint{4.130435in}{0.970454in}}%
\pgfpathlineto{\pgfqpoint{4.130934in}{0.974443in}}%
\pgfpathlineto{\pgfqpoint{4.224903in}{0.978432in}}%
\pgfpathlineto{\pgfqpoint{4.231903in}{0.982421in}}%
\pgfpathlineto{\pgfqpoint{4.232463in}{0.986410in}}%
\pgfpathlineto{\pgfqpoint{4.251985in}{0.990399in}}%
\pgfpathlineto{\pgfqpoint{4.265601in}{0.994388in}}%
\pgfpathlineto{\pgfqpoint{4.271883in}{0.998377in}}%
\pgfpathlineto{\pgfqpoint{4.292110in}{1.002366in}}%
\pgfpathlineto{\pgfqpoint{4.324514in}{1.006355in}}%
\pgfpathlineto{\pgfqpoint{4.439250in}{1.010344in}}%
\pgfpathlineto{\pgfqpoint{4.440468in}{1.014333in}}%
\pgfpathlineto{\pgfqpoint{4.508935in}{1.018322in}}%
\pgfpathlineto{\pgfqpoint{4.509586in}{1.022311in}}%
\pgfpathlineto{\pgfqpoint{4.513410in}{1.026300in}}%
\pgfpathlineto{\pgfqpoint{4.536361in}{1.030289in}}%
\pgfpathlineto{\pgfqpoint{4.542210in}{1.034278in}}%
\pgfpathlineto{\pgfqpoint{4.543699in}{1.038267in}}%
\pgfpathlineto{\pgfqpoint{4.556903in}{1.042256in}}%
\pgfpathlineto{\pgfqpoint{4.559527in}{1.046245in}}%
\pgfpathlineto{\pgfqpoint{4.560223in}{1.050234in}}%
\pgfpathlineto{\pgfqpoint{4.566061in}{1.054224in}}%
\pgfpathlineto{\pgfqpoint{4.566800in}{1.058213in}}%
\pgfpathlineto{\pgfqpoint{4.567339in}{1.062202in}}%
\pgfpathlineto{\pgfqpoint{4.576560in}{1.066191in}}%
\pgfpathlineto{\pgfqpoint{4.580339in}{1.070180in}}%
\pgfpathlineto{\pgfqpoint{4.581417in}{1.074169in}}%
\pgfpathlineto{\pgfqpoint{4.618726in}{1.078158in}}%
\pgfpathlineto{\pgfqpoint{4.619788in}{1.082147in}}%
\pgfusepath{stroke}%
\end{pgfscope}%
\begin{pgfscope}%
\pgfpathrectangle{\pgfqpoint{0.537394in}{0.467838in}}{\pgfqpoint{4.094684in}{1.595606in}}%
\pgfusepath{clip}%
\pgfsetbuttcap%
\pgfsetroundjoin%
\pgfsetlinewidth{1.003750pt}%
\definecolor{currentstroke}{rgb}{0.580392,0.403922,0.741176}%
\pgfsetstrokecolor{currentstroke}%
\pgfsetdash{{3.700000pt}{1.600000pt}}{0.000000pt}%
\pgfpathmoveto{\pgfqpoint{0.537394in}{0.543630in}}%
\pgfpathlineto{\pgfqpoint{0.549018in}{0.547619in}}%
\pgfpathlineto{\pgfqpoint{0.625240in}{0.551608in}}%
\pgfpathlineto{\pgfqpoint{0.654388in}{0.555597in}}%
\pgfpathlineto{\pgfqpoint{0.701556in}{0.559586in}}%
\pgfpathlineto{\pgfqpoint{0.723387in}{0.563575in}}%
\pgfpathlineto{\pgfqpoint{0.729029in}{0.567564in}}%
\pgfpathlineto{\pgfqpoint{0.731041in}{0.571553in}}%
\pgfpathlineto{\pgfqpoint{0.732425in}{0.575542in}}%
\pgfpathlineto{\pgfqpoint{0.732579in}{0.579531in}}%
\pgfpathlineto{\pgfqpoint{0.736238in}{0.583520in}}%
\pgfpathlineto{\pgfqpoint{0.737749in}{0.587509in}}%
\pgfpathlineto{\pgfqpoint{0.739553in}{0.591498in}}%
\pgfpathlineto{\pgfqpoint{0.746073in}{0.595487in}}%
\pgfpathlineto{\pgfqpoint{0.746805in}{0.599476in}}%
\pgfpathlineto{\pgfqpoint{0.747098in}{0.603465in}}%
\pgfpathlineto{\pgfqpoint{0.749136in}{0.607454in}}%
\pgfpathlineto{\pgfqpoint{0.752310in}{0.615432in}}%
\pgfpathlineto{\pgfqpoint{0.756302in}{0.619421in}}%
\pgfpathlineto{\pgfqpoint{0.760939in}{0.627399in}}%
\pgfpathlineto{\pgfqpoint{0.761217in}{0.631388in}}%
\pgfpathlineto{\pgfqpoint{0.820933in}{0.635377in}}%
\pgfpathlineto{\pgfqpoint{0.821616in}{0.639366in}}%
\pgfpathlineto{\pgfqpoint{0.846447in}{0.643355in}}%
\pgfpathlineto{\pgfqpoint{0.861518in}{0.647344in}}%
\pgfpathlineto{\pgfqpoint{0.862905in}{0.651333in}}%
\pgfpathlineto{\pgfqpoint{0.878873in}{0.655322in}}%
\pgfpathlineto{\pgfqpoint{0.881297in}{0.659311in}}%
\pgfpathlineto{\pgfqpoint{0.881483in}{0.663300in}}%
\pgfpathlineto{\pgfqpoint{0.884990in}{0.667289in}}%
\pgfpathlineto{\pgfqpoint{0.902445in}{0.671278in}}%
\pgfpathlineto{\pgfqpoint{0.902531in}{0.675267in}}%
\pgfpathlineto{\pgfqpoint{0.916712in}{0.679256in}}%
\pgfpathlineto{\pgfqpoint{0.924841in}{0.683245in}}%
\pgfpathlineto{\pgfqpoint{0.927952in}{0.687234in}}%
\pgfpathlineto{\pgfqpoint{0.935628in}{0.691223in}}%
\pgfpathlineto{\pgfqpoint{0.968113in}{0.695212in}}%
\pgfpathlineto{\pgfqpoint{0.988011in}{0.699201in}}%
\pgfpathlineto{\pgfqpoint{0.994611in}{0.703190in}}%
\pgfpathlineto{\pgfqpoint{0.995813in}{0.707179in}}%
\pgfpathlineto{\pgfqpoint{1.003845in}{0.711168in}}%
\pgfpathlineto{\pgfqpoint{1.009322in}{0.715157in}}%
\pgfpathlineto{\pgfqpoint{1.016942in}{0.719146in}}%
\pgfpathlineto{\pgfqpoint{1.025629in}{0.723135in}}%
\pgfpathlineto{\pgfqpoint{1.053447in}{0.727124in}}%
\pgfpathlineto{\pgfqpoint{1.066757in}{0.731113in}}%
\pgfpathlineto{\pgfqpoint{1.073923in}{0.735102in}}%
\pgfpathlineto{\pgfqpoint{1.083747in}{0.739091in}}%
\pgfpathlineto{\pgfqpoint{1.087940in}{0.743080in}}%
\pgfpathlineto{\pgfqpoint{1.097855in}{0.747069in}}%
\pgfpathlineto{\pgfqpoint{1.106190in}{0.751058in}}%
\pgfpathlineto{\pgfqpoint{1.123218in}{0.755047in}}%
\pgfpathlineto{\pgfqpoint{1.130839in}{0.759036in}}%
\pgfpathlineto{\pgfqpoint{1.133151in}{0.763025in}}%
\pgfpathlineto{\pgfqpoint{1.137995in}{0.767014in}}%
\pgfpathlineto{\pgfqpoint{1.140640in}{0.771003in}}%
\pgfpathlineto{\pgfqpoint{1.153483in}{0.774992in}}%
\pgfpathlineto{\pgfqpoint{1.153853in}{0.778981in}}%
\pgfpathlineto{\pgfqpoint{1.157935in}{0.782971in}}%
\pgfpathlineto{\pgfqpoint{1.169293in}{0.794938in}}%
\pgfpathlineto{\pgfqpoint{1.170765in}{0.798927in}}%
\pgfpathlineto{\pgfqpoint{1.172995in}{0.802916in}}%
\pgfpathlineto{\pgfqpoint{1.178191in}{0.806905in}}%
\pgfpathlineto{\pgfqpoint{1.181886in}{0.810894in}}%
\pgfpathlineto{\pgfqpoint{1.186930in}{0.814883in}}%
\pgfpathlineto{\pgfqpoint{1.200618in}{0.818872in}}%
\pgfpathlineto{\pgfqpoint{1.208236in}{0.822861in}}%
\pgfpathlineto{\pgfqpoint{1.219841in}{0.826850in}}%
\pgfpathlineto{\pgfqpoint{1.223962in}{0.830839in}}%
\pgfpathlineto{\pgfqpoint{1.226552in}{0.834828in}}%
\pgfpathlineto{\pgfqpoint{1.258207in}{0.842806in}}%
\pgfpathlineto{\pgfqpoint{1.268193in}{0.846795in}}%
\pgfpathlineto{\pgfqpoint{1.280280in}{0.850784in}}%
\pgfpathlineto{\pgfqpoint{1.293076in}{0.854773in}}%
\pgfpathlineto{\pgfqpoint{1.295726in}{0.858762in}}%
\pgfpathlineto{\pgfqpoint{1.301092in}{0.862751in}}%
\pgfpathlineto{\pgfqpoint{1.319397in}{0.866740in}}%
\pgfpathlineto{\pgfqpoint{1.330375in}{0.870729in}}%
\pgfpathlineto{\pgfqpoint{1.339480in}{0.874718in}}%
\pgfpathlineto{\pgfqpoint{1.341197in}{0.878707in}}%
\pgfpathlineto{\pgfqpoint{1.342395in}{0.882696in}}%
\pgfpathlineto{\pgfqpoint{1.361081in}{0.886685in}}%
\pgfpathlineto{\pgfqpoint{1.370000in}{0.890674in}}%
\pgfpathlineto{\pgfqpoint{1.388299in}{0.894663in}}%
\pgfpathlineto{\pgfqpoint{1.402588in}{0.898652in}}%
\pgfpathlineto{\pgfqpoint{1.415807in}{0.906630in}}%
\pgfpathlineto{\pgfqpoint{1.418927in}{0.910619in}}%
\pgfpathlineto{\pgfqpoint{1.422838in}{0.914608in}}%
\pgfpathlineto{\pgfqpoint{1.425308in}{0.918597in}}%
\pgfpathlineto{\pgfqpoint{1.425826in}{0.922586in}}%
\pgfpathlineto{\pgfqpoint{1.454361in}{0.926575in}}%
\pgfpathlineto{\pgfqpoint{1.474284in}{0.930564in}}%
\pgfpathlineto{\pgfqpoint{1.475787in}{0.934553in}}%
\pgfpathlineto{\pgfqpoint{1.509087in}{0.938542in}}%
\pgfpathlineto{\pgfqpoint{1.515888in}{0.942531in}}%
\pgfpathlineto{\pgfqpoint{1.521404in}{0.946520in}}%
\pgfpathlineto{\pgfqpoint{1.521842in}{0.950509in}}%
\pgfpathlineto{\pgfqpoint{1.523081in}{0.954498in}}%
\pgfpathlineto{\pgfqpoint{1.529447in}{0.958487in}}%
\pgfpathlineto{\pgfqpoint{1.531111in}{0.962476in}}%
\pgfpathlineto{\pgfqpoint{1.535577in}{0.966465in}}%
\pgfpathlineto{\pgfqpoint{1.541353in}{0.974443in}}%
\pgfpathlineto{\pgfqpoint{1.544422in}{0.978432in}}%
\pgfpathlineto{\pgfqpoint{1.566665in}{0.982421in}}%
\pgfpathlineto{\pgfqpoint{1.594365in}{0.986410in}}%
\pgfpathlineto{\pgfqpoint{1.597307in}{0.990399in}}%
\pgfpathlineto{\pgfqpoint{1.603804in}{0.994388in}}%
\pgfpathlineto{\pgfqpoint{1.625166in}{0.998377in}}%
\pgfpathlineto{\pgfqpoint{1.628424in}{1.002366in}}%
\pgfpathlineto{\pgfqpoint{1.630715in}{1.006355in}}%
\pgfpathlineto{\pgfqpoint{1.632592in}{1.010344in}}%
\pgfpathlineto{\pgfqpoint{1.636701in}{1.014333in}}%
\pgfpathlineto{\pgfqpoint{1.639771in}{1.018322in}}%
\pgfpathlineto{\pgfqpoint{1.641105in}{1.022311in}}%
\pgfpathlineto{\pgfqpoint{1.649797in}{1.026300in}}%
\pgfpathlineto{\pgfqpoint{1.651577in}{1.030289in}}%
\pgfpathlineto{\pgfqpoint{1.664251in}{1.034278in}}%
\pgfpathlineto{\pgfqpoint{1.672634in}{1.038267in}}%
\pgfpathlineto{\pgfqpoint{1.697094in}{1.042256in}}%
\pgfpathlineto{\pgfqpoint{1.717817in}{1.046245in}}%
\pgfpathlineto{\pgfqpoint{1.722143in}{1.054224in}}%
\pgfpathlineto{\pgfqpoint{1.722252in}{1.058213in}}%
\pgfpathlineto{\pgfqpoint{1.724161in}{1.062202in}}%
\pgfpathlineto{\pgfqpoint{1.724917in}{1.066191in}}%
\pgfpathlineto{\pgfqpoint{1.730419in}{1.070180in}}%
\pgfpathlineto{\pgfqpoint{1.747355in}{1.074169in}}%
\pgfpathlineto{\pgfqpoint{1.757657in}{1.078158in}}%
\pgfpathlineto{\pgfqpoint{1.763688in}{1.082147in}}%
\pgfpathlineto{\pgfqpoint{1.764449in}{1.086136in}}%
\pgfpathlineto{\pgfqpoint{1.771144in}{1.090125in}}%
\pgfpathlineto{\pgfqpoint{1.781975in}{1.094114in}}%
\pgfpathlineto{\pgfqpoint{1.787679in}{1.098103in}}%
\pgfpathlineto{\pgfqpoint{1.789372in}{1.102092in}}%
\pgfpathlineto{\pgfqpoint{1.796707in}{1.106081in}}%
\pgfpathlineto{\pgfqpoint{1.800400in}{1.110070in}}%
\pgfpathlineto{\pgfqpoint{1.809058in}{1.114059in}}%
\pgfpathlineto{\pgfqpoint{1.819586in}{1.118048in}}%
\pgfpathlineto{\pgfqpoint{1.834384in}{1.122037in}}%
\pgfpathlineto{\pgfqpoint{1.835804in}{1.126026in}}%
\pgfpathlineto{\pgfqpoint{1.865722in}{1.130015in}}%
\pgfpathlineto{\pgfqpoint{1.884186in}{1.134004in}}%
\pgfpathlineto{\pgfqpoint{1.887834in}{1.137993in}}%
\pgfpathlineto{\pgfqpoint{1.900687in}{1.141982in}}%
\pgfpathlineto{\pgfqpoint{1.910565in}{1.145971in}}%
\pgfpathlineto{\pgfqpoint{1.935133in}{1.149960in}}%
\pgfpathlineto{\pgfqpoint{1.935597in}{1.153949in}}%
\pgfpathlineto{\pgfqpoint{1.944831in}{1.157938in}}%
\pgfpathlineto{\pgfqpoint{1.947883in}{1.161927in}}%
\pgfpathlineto{\pgfqpoint{1.956326in}{1.165916in}}%
\pgfpathlineto{\pgfqpoint{1.958307in}{1.169905in}}%
\pgfpathlineto{\pgfqpoint{1.961654in}{1.173894in}}%
\pgfpathlineto{\pgfqpoint{1.967096in}{1.177883in}}%
\pgfpathlineto{\pgfqpoint{1.981820in}{1.181872in}}%
\pgfpathlineto{\pgfqpoint{2.002013in}{1.185861in}}%
\pgfpathlineto{\pgfqpoint{2.031000in}{1.189850in}}%
\pgfpathlineto{\pgfqpoint{2.034942in}{1.193839in}}%
\pgfpathlineto{\pgfqpoint{2.047689in}{1.197828in}}%
\pgfpathlineto{\pgfqpoint{2.069872in}{1.201817in}}%
\pgfpathlineto{\pgfqpoint{2.070427in}{1.205806in}}%
\pgfpathlineto{\pgfqpoint{2.085849in}{1.209795in}}%
\pgfpathlineto{\pgfqpoint{2.087311in}{1.213784in}}%
\pgfpathlineto{\pgfqpoint{2.096951in}{1.217773in}}%
\pgfpathlineto{\pgfqpoint{2.101794in}{1.221762in}}%
\pgfpathlineto{\pgfqpoint{2.102606in}{1.225751in}}%
\pgfpathlineto{\pgfqpoint{2.115801in}{1.229740in}}%
\pgfpathlineto{\pgfqpoint{2.116080in}{1.233729in}}%
\pgfpathlineto{\pgfqpoint{2.146020in}{1.241707in}}%
\pgfpathlineto{\pgfqpoint{2.153274in}{1.245696in}}%
\pgfpathlineto{\pgfqpoint{2.172748in}{1.249685in}}%
\pgfpathlineto{\pgfqpoint{2.173632in}{1.253674in}}%
\pgfpathlineto{\pgfqpoint{2.182870in}{1.257663in}}%
\pgfpathlineto{\pgfqpoint{2.202128in}{1.261652in}}%
\pgfpathlineto{\pgfqpoint{2.252138in}{1.265641in}}%
\pgfpathlineto{\pgfqpoint{2.252633in}{1.269630in}}%
\pgfpathlineto{\pgfqpoint{2.254918in}{1.273619in}}%
\pgfpathlineto{\pgfqpoint{2.256259in}{1.277608in}}%
\pgfpathlineto{\pgfqpoint{2.266416in}{1.281597in}}%
\pgfpathlineto{\pgfqpoint{2.278920in}{1.285586in}}%
\pgfpathlineto{\pgfqpoint{2.279885in}{1.289575in}}%
\pgfpathlineto{\pgfqpoint{2.280145in}{1.293564in}}%
\pgfpathlineto{\pgfqpoint{2.280698in}{1.297553in}}%
\pgfpathlineto{\pgfqpoint{2.286192in}{1.301542in}}%
\pgfpathlineto{\pgfqpoint{2.356239in}{1.305531in}}%
\pgfpathlineto{\pgfqpoint{2.356311in}{1.309520in}}%
\pgfpathlineto{\pgfqpoint{2.373416in}{1.313509in}}%
\pgfpathlineto{\pgfqpoint{2.373986in}{1.317498in}}%
\pgfpathlineto{\pgfqpoint{2.388270in}{1.321488in}}%
\pgfpathlineto{\pgfqpoint{2.405466in}{1.325477in}}%
\pgfpathlineto{\pgfqpoint{2.415539in}{1.329466in}}%
\pgfpathlineto{\pgfqpoint{2.424067in}{1.333455in}}%
\pgfpathlineto{\pgfqpoint{2.443054in}{1.337444in}}%
\pgfpathlineto{\pgfqpoint{2.456914in}{1.341433in}}%
\pgfpathlineto{\pgfqpoint{2.458554in}{1.345422in}}%
\pgfpathlineto{\pgfqpoint{2.459396in}{1.349411in}}%
\pgfpathlineto{\pgfqpoint{2.472267in}{1.353400in}}%
\pgfpathlineto{\pgfqpoint{2.473046in}{1.357389in}}%
\pgfpathlineto{\pgfqpoint{2.545434in}{1.361378in}}%
\pgfpathlineto{\pgfqpoint{2.564093in}{1.365367in}}%
\pgfpathlineto{\pgfqpoint{2.588644in}{1.373345in}}%
\pgfpathlineto{\pgfqpoint{2.589179in}{1.377334in}}%
\pgfpathlineto{\pgfqpoint{2.594991in}{1.381323in}}%
\pgfpathlineto{\pgfqpoint{2.595722in}{1.385312in}}%
\pgfpathlineto{\pgfqpoint{2.596991in}{1.389301in}}%
\pgfpathlineto{\pgfqpoint{2.605856in}{1.393290in}}%
\pgfpathlineto{\pgfqpoint{2.618791in}{1.397279in}}%
\pgfpathlineto{\pgfqpoint{2.619450in}{1.401268in}}%
\pgfpathlineto{\pgfqpoint{2.629335in}{1.405257in}}%
\pgfpathlineto{\pgfqpoint{2.633614in}{1.409246in}}%
\pgfpathlineto{\pgfqpoint{2.635773in}{1.413235in}}%
\pgfpathlineto{\pgfqpoint{2.636355in}{1.421213in}}%
\pgfpathlineto{\pgfqpoint{2.645452in}{1.425202in}}%
\pgfpathlineto{\pgfqpoint{2.646069in}{1.429191in}}%
\pgfpathlineto{\pgfqpoint{2.650775in}{1.437169in}}%
\pgfpathlineto{\pgfqpoint{2.683592in}{1.441158in}}%
\pgfpathlineto{\pgfqpoint{2.777383in}{1.445147in}}%
\pgfpathlineto{\pgfqpoint{2.813877in}{1.449136in}}%
\pgfpathlineto{\pgfqpoint{2.830016in}{1.453125in}}%
\pgfpathlineto{\pgfqpoint{2.843996in}{1.457114in}}%
\pgfpathlineto{\pgfqpoint{2.859991in}{1.461103in}}%
\pgfpathlineto{\pgfqpoint{2.870434in}{1.465092in}}%
\pgfpathlineto{\pgfqpoint{2.871967in}{1.469081in}}%
\pgfpathlineto{\pgfqpoint{2.899051in}{1.473070in}}%
\pgfpathlineto{\pgfqpoint{2.905154in}{1.477059in}}%
\pgfpathlineto{\pgfqpoint{2.906941in}{1.481048in}}%
\pgfpathlineto{\pgfqpoint{2.936506in}{1.485037in}}%
\pgfpathlineto{\pgfqpoint{2.940462in}{1.489026in}}%
\pgfpathlineto{\pgfqpoint{2.942212in}{1.493015in}}%
\pgfpathlineto{\pgfqpoint{2.947631in}{1.497004in}}%
\pgfpathlineto{\pgfqpoint{2.950680in}{1.504982in}}%
\pgfpathlineto{\pgfqpoint{2.959625in}{1.508971in}}%
\pgfpathlineto{\pgfqpoint{2.973557in}{1.512960in}}%
\pgfpathlineto{\pgfqpoint{2.979306in}{1.516949in}}%
\pgfpathlineto{\pgfqpoint{2.980632in}{1.520938in}}%
\pgfpathlineto{\pgfqpoint{3.012362in}{1.524927in}}%
\pgfpathlineto{\pgfqpoint{3.034549in}{1.528916in}}%
\pgfpathlineto{\pgfqpoint{3.040680in}{1.532905in}}%
\pgfpathlineto{\pgfqpoint{3.048358in}{1.536894in}}%
\pgfpathlineto{\pgfqpoint{3.048807in}{1.540883in}}%
\pgfpathlineto{\pgfqpoint{3.073088in}{1.544872in}}%
\pgfpathlineto{\pgfqpoint{3.078405in}{1.548861in}}%
\pgfpathlineto{\pgfqpoint{3.078765in}{1.552850in}}%
\pgfpathlineto{\pgfqpoint{3.098213in}{1.556839in}}%
\pgfpathlineto{\pgfqpoint{3.098700in}{1.560828in}}%
\pgfpathlineto{\pgfqpoint{3.099415in}{1.564817in}}%
\pgfpathlineto{\pgfqpoint{3.189598in}{1.572795in}}%
\pgfpathlineto{\pgfqpoint{3.204360in}{1.576784in}}%
\pgfpathlineto{\pgfqpoint{3.228707in}{1.580773in}}%
\pgfpathlineto{\pgfqpoint{3.229381in}{1.584762in}}%
\pgfpathlineto{\pgfqpoint{3.236152in}{1.588751in}}%
\pgfpathlineto{\pgfqpoint{3.323349in}{1.592741in}}%
\pgfpathlineto{\pgfqpoint{3.323610in}{1.596730in}}%
\pgfpathlineto{\pgfqpoint{3.327564in}{1.600719in}}%
\pgfpathlineto{\pgfqpoint{3.327674in}{1.604708in}}%
\pgfpathlineto{\pgfqpoint{3.329328in}{1.608697in}}%
\pgfpathlineto{\pgfqpoint{3.334106in}{1.612686in}}%
\pgfpathlineto{\pgfqpoint{3.371532in}{1.616675in}}%
\pgfpathlineto{\pgfqpoint{3.393233in}{1.620664in}}%
\pgfpathlineto{\pgfqpoint{3.396060in}{1.624653in}}%
\pgfpathlineto{\pgfqpoint{3.507387in}{1.628642in}}%
\pgfpathlineto{\pgfqpoint{3.510202in}{1.632631in}}%
\pgfpathlineto{\pgfqpoint{3.519751in}{1.636620in}}%
\pgfpathlineto{\pgfqpoint{3.519827in}{1.640609in}}%
\pgfpathlineto{\pgfqpoint{3.589253in}{1.644598in}}%
\pgfpathlineto{\pgfqpoint{3.600664in}{1.648587in}}%
\pgfpathlineto{\pgfqpoint{3.605163in}{1.652576in}}%
\pgfpathlineto{\pgfqpoint{3.627137in}{1.656565in}}%
\pgfpathlineto{\pgfqpoint{3.652046in}{1.660554in}}%
\pgfpathlineto{\pgfqpoint{3.652542in}{1.664543in}}%
\pgfpathlineto{\pgfqpoint{3.655665in}{1.668532in}}%
\pgfpathlineto{\pgfqpoint{3.740276in}{1.672521in}}%
\pgfpathlineto{\pgfqpoint{3.742861in}{1.676510in}}%
\pgfpathlineto{\pgfqpoint{3.779730in}{1.680499in}}%
\pgfpathlineto{\pgfqpoint{3.805105in}{1.684488in}}%
\pgfpathlineto{\pgfqpoint{3.810779in}{1.688477in}}%
\pgfpathlineto{\pgfqpoint{3.881660in}{1.692466in}}%
\pgfpathlineto{\pgfqpoint{3.881812in}{1.696455in}}%
\pgfpathlineto{\pgfqpoint{3.890978in}{1.700444in}}%
\pgfpathlineto{\pgfqpoint{3.891729in}{1.704433in}}%
\pgfpathlineto{\pgfqpoint{3.964192in}{1.708422in}}%
\pgfpathlineto{\pgfqpoint{3.991108in}{1.712411in}}%
\pgfpathlineto{\pgfqpoint{3.995548in}{1.716400in}}%
\pgfpathlineto{\pgfqpoint{3.997605in}{1.720389in}}%
\pgfpathlineto{\pgfqpoint{4.003495in}{1.724378in}}%
\pgfpathlineto{\pgfqpoint{4.040950in}{1.728367in}}%
\pgfpathlineto{\pgfqpoint{4.069642in}{1.732356in}}%
\pgfpathlineto{\pgfqpoint{4.074603in}{1.736345in}}%
\pgfpathlineto{\pgfqpoint{4.076124in}{1.740334in}}%
\pgfpathlineto{\pgfqpoint{4.102029in}{1.744323in}}%
\pgfpathlineto{\pgfqpoint{4.147863in}{1.748312in}}%
\pgfpathlineto{\pgfqpoint{4.149646in}{1.752301in}}%
\pgfpathlineto{\pgfqpoint{4.172457in}{1.756290in}}%
\pgfpathlineto{\pgfqpoint{4.176247in}{1.760279in}}%
\pgfpathlineto{\pgfqpoint{4.183698in}{1.764268in}}%
\pgfpathlineto{\pgfqpoint{4.183988in}{1.768257in}}%
\pgfpathlineto{\pgfqpoint{4.186390in}{1.772246in}}%
\pgfpathlineto{\pgfqpoint{4.197995in}{1.776235in}}%
\pgfpathlineto{\pgfqpoint{4.204720in}{1.780224in}}%
\pgfpathlineto{\pgfqpoint{4.213330in}{1.784213in}}%
\pgfpathlineto{\pgfqpoint{4.265982in}{1.788202in}}%
\pgfpathlineto{\pgfqpoint{4.268074in}{1.792191in}}%
\pgfpathlineto{\pgfqpoint{4.284052in}{1.796180in}}%
\pgfpathlineto{\pgfqpoint{4.304571in}{1.800169in}}%
\pgfpathlineto{\pgfqpoint{4.337329in}{1.804158in}}%
\pgfpathlineto{\pgfqpoint{4.362388in}{1.808147in}}%
\pgfpathlineto{\pgfqpoint{4.363279in}{1.812136in}}%
\pgfpathlineto{\pgfqpoint{4.364971in}{1.816125in}}%
\pgfpathlineto{\pgfqpoint{4.424704in}{1.820114in}}%
\pgfpathlineto{\pgfqpoint{4.438966in}{1.824103in}}%
\pgfpathlineto{\pgfqpoint{4.448353in}{1.828092in}}%
\pgfpathlineto{\pgfqpoint{4.472714in}{1.832081in}}%
\pgfpathlineto{\pgfqpoint{4.530858in}{1.836070in}}%
\pgfpathlineto{\pgfqpoint{4.538327in}{1.840059in}}%
\pgfpathlineto{\pgfqpoint{4.538787in}{1.844048in}}%
\pgfpathlineto{\pgfqpoint{4.595785in}{1.848037in}}%
\pgfpathlineto{\pgfqpoint{4.595785in}{1.848037in}}%
\pgfusepath{stroke}%
\end{pgfscope}%
\begin{pgfscope}%
\pgfpathrectangle{\pgfqpoint{0.537394in}{0.467838in}}{\pgfqpoint{4.094684in}{1.595606in}}%
\pgfusepath{clip}%
\pgfsetbuttcap%
\pgfsetroundjoin%
\pgfsetlinewidth{1.003750pt}%
\definecolor{currentstroke}{rgb}{0.549020,0.337255,0.294118}%
\pgfsetstrokecolor{currentstroke}%
\pgfsetdash{{1.000000pt}{1.650000pt}}{0.000000pt}%
\pgfpathmoveto{\pgfqpoint{0.537394in}{0.543630in}}%
\pgfpathlineto{\pgfqpoint{0.549018in}{0.547619in}}%
\pgfpathlineto{\pgfqpoint{0.625240in}{0.551608in}}%
\pgfpathlineto{\pgfqpoint{0.654388in}{0.555597in}}%
\pgfpathlineto{\pgfqpoint{0.701556in}{0.559586in}}%
\pgfpathlineto{\pgfqpoint{0.723387in}{0.563575in}}%
\pgfpathlineto{\pgfqpoint{0.729029in}{0.567564in}}%
\pgfpathlineto{\pgfqpoint{0.731041in}{0.571553in}}%
\pgfpathlineto{\pgfqpoint{0.732425in}{0.575542in}}%
\pgfpathlineto{\pgfqpoint{0.732579in}{0.579531in}}%
\pgfpathlineto{\pgfqpoint{0.736238in}{0.583520in}}%
\pgfpathlineto{\pgfqpoint{0.737749in}{0.587509in}}%
\pgfpathlineto{\pgfqpoint{0.739553in}{0.591498in}}%
\pgfpathlineto{\pgfqpoint{0.746073in}{0.595487in}}%
\pgfpathlineto{\pgfqpoint{0.746805in}{0.599476in}}%
\pgfpathlineto{\pgfqpoint{0.747098in}{0.603465in}}%
\pgfpathlineto{\pgfqpoint{0.749136in}{0.607454in}}%
\pgfpathlineto{\pgfqpoint{0.752310in}{0.615432in}}%
\pgfpathlineto{\pgfqpoint{0.756302in}{0.619421in}}%
\pgfpathlineto{\pgfqpoint{0.760939in}{0.627399in}}%
\pgfpathlineto{\pgfqpoint{0.761217in}{0.631388in}}%
\pgfpathlineto{\pgfqpoint{0.820933in}{0.635377in}}%
\pgfpathlineto{\pgfqpoint{0.821616in}{0.639366in}}%
\pgfpathlineto{\pgfqpoint{0.846447in}{0.643355in}}%
\pgfpathlineto{\pgfqpoint{0.861518in}{0.647344in}}%
\pgfpathlineto{\pgfqpoint{0.862905in}{0.651333in}}%
\pgfpathlineto{\pgfqpoint{0.878873in}{0.655322in}}%
\pgfpathlineto{\pgfqpoint{0.881297in}{0.659311in}}%
\pgfpathlineto{\pgfqpoint{0.881483in}{0.663300in}}%
\pgfpathlineto{\pgfqpoint{0.884990in}{0.667289in}}%
\pgfpathlineto{\pgfqpoint{0.902445in}{0.671278in}}%
\pgfpathlineto{\pgfqpoint{0.902531in}{0.675267in}}%
\pgfpathlineto{\pgfqpoint{0.916712in}{0.679256in}}%
\pgfpathlineto{\pgfqpoint{0.924841in}{0.683245in}}%
\pgfpathlineto{\pgfqpoint{0.927952in}{0.687234in}}%
\pgfpathlineto{\pgfqpoint{0.935628in}{0.691223in}}%
\pgfpathlineto{\pgfqpoint{0.968113in}{0.695212in}}%
\pgfpathlineto{\pgfqpoint{0.988011in}{0.699201in}}%
\pgfpathlineto{\pgfqpoint{0.994611in}{0.703190in}}%
\pgfpathlineto{\pgfqpoint{0.995813in}{0.707179in}}%
\pgfpathlineto{\pgfqpoint{1.003845in}{0.711168in}}%
\pgfpathlineto{\pgfqpoint{1.009322in}{0.715157in}}%
\pgfpathlineto{\pgfqpoint{1.016942in}{0.719146in}}%
\pgfpathlineto{\pgfqpoint{1.025629in}{0.723135in}}%
\pgfpathlineto{\pgfqpoint{1.053447in}{0.727124in}}%
\pgfpathlineto{\pgfqpoint{1.066757in}{0.731113in}}%
\pgfpathlineto{\pgfqpoint{1.073923in}{0.735102in}}%
\pgfpathlineto{\pgfqpoint{1.083747in}{0.739091in}}%
\pgfpathlineto{\pgfqpoint{1.087940in}{0.743080in}}%
\pgfpathlineto{\pgfqpoint{1.097855in}{0.747069in}}%
\pgfpathlineto{\pgfqpoint{1.106190in}{0.751058in}}%
\pgfpathlineto{\pgfqpoint{1.123218in}{0.755047in}}%
\pgfpathlineto{\pgfqpoint{1.130839in}{0.759036in}}%
\pgfpathlineto{\pgfqpoint{1.133151in}{0.763025in}}%
\pgfpathlineto{\pgfqpoint{1.137995in}{0.767014in}}%
\pgfpathlineto{\pgfqpoint{1.140640in}{0.771003in}}%
\pgfpathlineto{\pgfqpoint{1.153483in}{0.774992in}}%
\pgfpathlineto{\pgfqpoint{1.153853in}{0.778981in}}%
\pgfpathlineto{\pgfqpoint{1.157935in}{0.782971in}}%
\pgfpathlineto{\pgfqpoint{1.169293in}{0.794938in}}%
\pgfpathlineto{\pgfqpoint{1.170765in}{0.798927in}}%
\pgfpathlineto{\pgfqpoint{1.172995in}{0.802916in}}%
\pgfpathlineto{\pgfqpoint{1.178191in}{0.806905in}}%
\pgfpathlineto{\pgfqpoint{1.181886in}{0.810894in}}%
\pgfpathlineto{\pgfqpoint{1.186930in}{0.814883in}}%
\pgfpathlineto{\pgfqpoint{1.200618in}{0.818872in}}%
\pgfpathlineto{\pgfqpoint{1.208236in}{0.822861in}}%
\pgfpathlineto{\pgfqpoint{1.219841in}{0.826850in}}%
\pgfpathlineto{\pgfqpoint{1.223962in}{0.830839in}}%
\pgfpathlineto{\pgfqpoint{1.226552in}{0.834828in}}%
\pgfpathlineto{\pgfqpoint{1.258207in}{0.842806in}}%
\pgfpathlineto{\pgfqpoint{1.268193in}{0.846795in}}%
\pgfpathlineto{\pgfqpoint{1.280280in}{0.850784in}}%
\pgfpathlineto{\pgfqpoint{1.293076in}{0.854773in}}%
\pgfpathlineto{\pgfqpoint{1.295726in}{0.858762in}}%
\pgfpathlineto{\pgfqpoint{1.301092in}{0.862751in}}%
\pgfpathlineto{\pgfqpoint{1.319397in}{0.866740in}}%
\pgfpathlineto{\pgfqpoint{1.330375in}{0.870729in}}%
\pgfpathlineto{\pgfqpoint{1.339480in}{0.874718in}}%
\pgfpathlineto{\pgfqpoint{1.341197in}{0.878707in}}%
\pgfpathlineto{\pgfqpoint{1.342395in}{0.882696in}}%
\pgfpathlineto{\pgfqpoint{1.361081in}{0.886685in}}%
\pgfpathlineto{\pgfqpoint{1.370000in}{0.890674in}}%
\pgfpathlineto{\pgfqpoint{1.388299in}{0.894663in}}%
\pgfpathlineto{\pgfqpoint{1.402588in}{0.898652in}}%
\pgfpathlineto{\pgfqpoint{1.415807in}{0.906630in}}%
\pgfpathlineto{\pgfqpoint{1.418927in}{0.910619in}}%
\pgfpathlineto{\pgfqpoint{1.422838in}{0.914608in}}%
\pgfpathlineto{\pgfqpoint{1.425308in}{0.918597in}}%
\pgfpathlineto{\pgfqpoint{1.425826in}{0.922586in}}%
\pgfpathlineto{\pgfqpoint{1.454361in}{0.926575in}}%
\pgfpathlineto{\pgfqpoint{1.474284in}{0.930564in}}%
\pgfpathlineto{\pgfqpoint{1.475787in}{0.934553in}}%
\pgfpathlineto{\pgfqpoint{1.509087in}{0.938542in}}%
\pgfpathlineto{\pgfqpoint{1.515888in}{0.942531in}}%
\pgfpathlineto{\pgfqpoint{1.521404in}{0.946520in}}%
\pgfpathlineto{\pgfqpoint{1.521842in}{0.950509in}}%
\pgfpathlineto{\pgfqpoint{1.523081in}{0.954498in}}%
\pgfpathlineto{\pgfqpoint{1.529447in}{0.958487in}}%
\pgfpathlineto{\pgfqpoint{1.531111in}{0.962476in}}%
\pgfpathlineto{\pgfqpoint{1.535577in}{0.966465in}}%
\pgfpathlineto{\pgfqpoint{1.541353in}{0.974443in}}%
\pgfpathlineto{\pgfqpoint{1.544422in}{0.978432in}}%
\pgfpathlineto{\pgfqpoint{1.566665in}{0.982421in}}%
\pgfpathlineto{\pgfqpoint{1.594365in}{0.986410in}}%
\pgfpathlineto{\pgfqpoint{1.597307in}{0.990399in}}%
\pgfpathlineto{\pgfqpoint{1.603804in}{0.994388in}}%
\pgfpathlineto{\pgfqpoint{1.625166in}{0.998377in}}%
\pgfpathlineto{\pgfqpoint{1.628424in}{1.002366in}}%
\pgfpathlineto{\pgfqpoint{1.630715in}{1.006355in}}%
\pgfpathlineto{\pgfqpoint{1.632592in}{1.010344in}}%
\pgfpathlineto{\pgfqpoint{1.636701in}{1.014333in}}%
\pgfpathlineto{\pgfqpoint{1.639771in}{1.018322in}}%
\pgfpathlineto{\pgfqpoint{1.641105in}{1.022311in}}%
\pgfpathlineto{\pgfqpoint{1.649797in}{1.026300in}}%
\pgfpathlineto{\pgfqpoint{1.651577in}{1.030289in}}%
\pgfpathlineto{\pgfqpoint{1.658964in}{1.034278in}}%
\pgfpathlineto{\pgfqpoint{1.664251in}{1.038267in}}%
\pgfpathlineto{\pgfqpoint{1.672634in}{1.042256in}}%
\pgfpathlineto{\pgfqpoint{1.697094in}{1.046245in}}%
\pgfpathlineto{\pgfqpoint{1.717817in}{1.050234in}}%
\pgfpathlineto{\pgfqpoint{1.722143in}{1.058213in}}%
\pgfpathlineto{\pgfqpoint{1.722252in}{1.062202in}}%
\pgfpathlineto{\pgfqpoint{1.724161in}{1.066191in}}%
\pgfpathlineto{\pgfqpoint{1.724917in}{1.070180in}}%
\pgfpathlineto{\pgfqpoint{1.730419in}{1.074169in}}%
\pgfpathlineto{\pgfqpoint{1.747355in}{1.078158in}}%
\pgfpathlineto{\pgfqpoint{1.757657in}{1.082147in}}%
\pgfpathlineto{\pgfqpoint{1.763688in}{1.086136in}}%
\pgfpathlineto{\pgfqpoint{1.764449in}{1.090125in}}%
\pgfpathlineto{\pgfqpoint{1.771144in}{1.094114in}}%
\pgfpathlineto{\pgfqpoint{1.781975in}{1.098103in}}%
\pgfpathlineto{\pgfqpoint{1.787679in}{1.102092in}}%
\pgfpathlineto{\pgfqpoint{1.789372in}{1.106081in}}%
\pgfpathlineto{\pgfqpoint{1.796707in}{1.110070in}}%
\pgfpathlineto{\pgfqpoint{1.800400in}{1.114059in}}%
\pgfpathlineto{\pgfqpoint{1.809058in}{1.118048in}}%
\pgfpathlineto{\pgfqpoint{1.819586in}{1.122037in}}%
\pgfpathlineto{\pgfqpoint{1.834384in}{1.126026in}}%
\pgfpathlineto{\pgfqpoint{1.835804in}{1.130015in}}%
\pgfpathlineto{\pgfqpoint{1.865722in}{1.134004in}}%
\pgfpathlineto{\pgfqpoint{1.884186in}{1.137993in}}%
\pgfpathlineto{\pgfqpoint{1.887834in}{1.141982in}}%
\pgfpathlineto{\pgfqpoint{1.900687in}{1.145971in}}%
\pgfpathlineto{\pgfqpoint{1.910565in}{1.149960in}}%
\pgfpathlineto{\pgfqpoint{1.935133in}{1.153949in}}%
\pgfpathlineto{\pgfqpoint{1.935597in}{1.157938in}}%
\pgfpathlineto{\pgfqpoint{1.944831in}{1.161927in}}%
\pgfpathlineto{\pgfqpoint{1.947883in}{1.165916in}}%
\pgfpathlineto{\pgfqpoint{1.948822in}{1.169905in}}%
\pgfpathlineto{\pgfqpoint{1.956326in}{1.173894in}}%
\pgfpathlineto{\pgfqpoint{1.958307in}{1.177883in}}%
\pgfpathlineto{\pgfqpoint{1.961225in}{1.185861in}}%
\pgfpathlineto{\pgfqpoint{1.961654in}{1.189850in}}%
\pgfpathlineto{\pgfqpoint{1.963644in}{1.197828in}}%
\pgfpathlineto{\pgfqpoint{1.967096in}{1.201817in}}%
\pgfpathlineto{\pgfqpoint{1.968425in}{1.205806in}}%
\pgfpathlineto{\pgfqpoint{1.970786in}{1.213784in}}%
\pgfpathlineto{\pgfqpoint{1.977760in}{1.217773in}}%
\pgfpathlineto{\pgfqpoint{1.980049in}{1.225751in}}%
\pgfpathlineto{\pgfqpoint{1.982320in}{1.229740in}}%
\pgfpathlineto{\pgfqpoint{1.986811in}{1.245696in}}%
\pgfpathlineto{\pgfqpoint{1.989031in}{1.257663in}}%
\pgfpathlineto{\pgfqpoint{1.993422in}{1.265641in}}%
\pgfpathlineto{\pgfqpoint{1.995593in}{1.277608in}}%
\pgfpathlineto{\pgfqpoint{2.002013in}{1.281597in}}%
\pgfpathlineto{\pgfqpoint{2.006217in}{1.285586in}}%
\pgfpathlineto{\pgfqpoint{2.022462in}{1.293564in}}%
\pgfpathlineto{\pgfqpoint{2.026387in}{1.305531in}}%
\pgfpathlineto{\pgfqpoint{2.028331in}{1.313509in}}%
\pgfpathlineto{\pgfqpoint{2.030262in}{1.317498in}}%
\pgfpathlineto{\pgfqpoint{2.034086in}{1.321488in}}%
\pgfpathlineto{\pgfqpoint{2.035980in}{1.329466in}}%
\pgfpathlineto{\pgfqpoint{2.039732in}{1.337444in}}%
\pgfpathlineto{\pgfqpoint{2.047689in}{1.341433in}}%
\pgfpathlineto{\pgfqpoint{2.054281in}{1.345422in}}%
\pgfpathlineto{\pgfqpoint{2.066451in}{1.349411in}}%
\pgfpathlineto{\pgfqpoint{2.069838in}{1.353400in}}%
\pgfpathlineto{\pgfqpoint{2.074847in}{1.357389in}}%
\pgfpathlineto{\pgfqpoint{2.087806in}{1.365367in}}%
\pgfpathlineto{\pgfqpoint{2.089387in}{1.369356in}}%
\pgfpathlineto{\pgfqpoint{2.090959in}{1.377334in}}%
\pgfpathlineto{\pgfqpoint{2.094079in}{1.381323in}}%
\pgfpathlineto{\pgfqpoint{2.098698in}{1.385312in}}%
\pgfpathlineto{\pgfqpoint{2.102606in}{1.389301in}}%
\pgfpathlineto{\pgfqpoint{2.103246in}{1.393290in}}%
\pgfpathlineto{\pgfqpoint{2.106240in}{1.397279in}}%
\pgfpathlineto{\pgfqpoint{2.107726in}{1.405257in}}%
\pgfpathlineto{\pgfqpoint{2.115044in}{1.409246in}}%
\pgfpathlineto{\pgfqpoint{2.115801in}{1.413235in}}%
\pgfpathlineto{\pgfqpoint{2.116080in}{1.417224in}}%
\pgfpathlineto{\pgfqpoint{2.117922in}{1.421213in}}%
\pgfpathlineto{\pgfqpoint{2.120772in}{1.425202in}}%
\pgfpathlineto{\pgfqpoint{2.130536in}{1.429191in}}%
\pgfpathlineto{\pgfqpoint{2.131043in}{1.433180in}}%
\pgfpathlineto{\pgfqpoint{2.131905in}{1.437169in}}%
\pgfpathlineto{\pgfqpoint{2.133267in}{1.445147in}}%
\pgfpathlineto{\pgfqpoint{2.143949in}{1.453125in}}%
\pgfpathlineto{\pgfqpoint{2.147857in}{1.461103in}}%
\pgfpathlineto{\pgfqpoint{2.149149in}{1.469081in}}%
\pgfpathlineto{\pgfqpoint{2.153274in}{1.473070in}}%
\pgfpathlineto{\pgfqpoint{2.164222in}{1.477059in}}%
\pgfpathlineto{\pgfqpoint{2.172748in}{1.481048in}}%
\pgfpathlineto{\pgfqpoint{2.173632in}{1.485037in}}%
\pgfpathlineto{\pgfqpoint{2.175045in}{1.489026in}}%
\pgfpathlineto{\pgfqpoint{2.182870in}{1.493015in}}%
\pgfpathlineto{\pgfqpoint{2.184344in}{1.497004in}}%
\pgfpathlineto{\pgfqpoint{2.188886in}{1.500993in}}%
\pgfpathlineto{\pgfqpoint{2.191132in}{1.504982in}}%
\pgfpathlineto{\pgfqpoint{2.198859in}{1.508971in}}%
\pgfpathlineto{\pgfqpoint{2.201030in}{1.512960in}}%
\pgfpathlineto{\pgfqpoint{2.202110in}{1.516949in}}%
\pgfpathlineto{\pgfqpoint{2.202128in}{1.520938in}}%
\pgfpathlineto{\pgfqpoint{2.241417in}{1.524927in}}%
\pgfpathlineto{\pgfqpoint{2.245169in}{1.528916in}}%
\pgfpathlineto{\pgfqpoint{2.252138in}{1.532905in}}%
\pgfpathlineto{\pgfqpoint{2.252533in}{1.536894in}}%
\pgfpathlineto{\pgfqpoint{2.277790in}{1.540883in}}%
\pgfpathlineto{\pgfqpoint{2.278920in}{1.544872in}}%
\pgfpathlineto{\pgfqpoint{2.286192in}{1.548861in}}%
\pgfpathlineto{\pgfqpoint{2.296396in}{1.552850in}}%
\pgfpathlineto{\pgfqpoint{2.297960in}{1.556839in}}%
\pgfpathlineto{\pgfqpoint{2.301064in}{1.560828in}}%
\pgfpathlineto{\pgfqpoint{2.319758in}{1.564817in}}%
\pgfpathlineto{\pgfqpoint{2.322642in}{1.568806in}}%
\pgfpathlineto{\pgfqpoint{2.335973in}{1.572795in}}%
\pgfpathlineto{\pgfqpoint{2.372099in}{1.576784in}}%
\pgfpathlineto{\pgfqpoint{2.372706in}{1.580773in}}%
\pgfpathlineto{\pgfqpoint{2.405466in}{1.584762in}}%
\pgfpathlineto{\pgfqpoint{2.415522in}{1.588751in}}%
\pgfpathlineto{\pgfqpoint{2.415539in}{1.592741in}}%
\pgfpathlineto{\pgfqpoint{2.424067in}{1.596730in}}%
\pgfpathlineto{\pgfqpoint{2.440172in}{1.600719in}}%
\pgfpathlineto{\pgfqpoint{2.443054in}{1.604708in}}%
\pgfpathlineto{\pgfqpoint{2.457515in}{1.608697in}}%
\pgfpathlineto{\pgfqpoint{2.469559in}{1.612686in}}%
\pgfpathlineto{\pgfqpoint{2.506501in}{1.616675in}}%
\pgfpathlineto{\pgfqpoint{2.545434in}{1.624653in}}%
\pgfpathlineto{\pgfqpoint{2.560345in}{1.628642in}}%
\pgfpathlineto{\pgfqpoint{2.564093in}{1.632631in}}%
\pgfpathlineto{\pgfqpoint{2.576660in}{1.636620in}}%
\pgfpathlineto{\pgfqpoint{2.629335in}{1.640609in}}%
\pgfpathlineto{\pgfqpoint{2.636088in}{1.644598in}}%
\pgfpathlineto{\pgfqpoint{2.636355in}{1.648587in}}%
\pgfpathlineto{\pgfqpoint{2.645452in}{1.652576in}}%
\pgfpathlineto{\pgfqpoint{2.646069in}{1.656565in}}%
\pgfpathlineto{\pgfqpoint{2.695881in}{1.660554in}}%
\pgfpathlineto{\pgfqpoint{2.756632in}{1.664543in}}%
\pgfpathlineto{\pgfqpoint{2.760424in}{1.668532in}}%
\pgfpathlineto{\pgfqpoint{2.777383in}{1.672521in}}%
\pgfpathlineto{\pgfqpoint{2.787643in}{1.676510in}}%
\pgfpathlineto{\pgfqpoint{2.859991in}{1.680499in}}%
\pgfpathlineto{\pgfqpoint{2.882719in}{1.684488in}}%
\pgfpathlineto{\pgfqpoint{2.936506in}{1.688477in}}%
\pgfpathlineto{\pgfqpoint{2.942212in}{1.692466in}}%
\pgfpathlineto{\pgfqpoint{2.959625in}{1.696455in}}%
\pgfpathlineto{\pgfqpoint{3.012362in}{1.700444in}}%
\pgfpathlineto{\pgfqpoint{3.034549in}{1.704433in}}%
\pgfpathlineto{\pgfqpoint{3.040680in}{1.708422in}}%
\pgfpathlineto{\pgfqpoint{3.073088in}{1.712411in}}%
\pgfpathlineto{\pgfqpoint{3.074508in}{1.716400in}}%
\pgfpathlineto{\pgfqpoint{3.078405in}{1.720389in}}%
\pgfpathlineto{\pgfqpoint{3.078765in}{1.724378in}}%
\pgfpathlineto{\pgfqpoint{3.135235in}{1.728367in}}%
\pgfpathlineto{\pgfqpoint{3.140917in}{1.732356in}}%
\pgfpathlineto{\pgfqpoint{3.145187in}{1.736345in}}%
\pgfpathlineto{\pgfqpoint{3.162991in}{1.740334in}}%
\pgfpathlineto{\pgfqpoint{3.181880in}{1.744323in}}%
\pgfpathlineto{\pgfqpoint{3.189598in}{1.748312in}}%
\pgfpathlineto{\pgfqpoint{3.201898in}{1.752301in}}%
\pgfpathlineto{\pgfqpoint{3.204360in}{1.756290in}}%
\pgfpathlineto{\pgfqpoint{3.236152in}{1.760279in}}%
\pgfpathlineto{\pgfqpoint{3.283921in}{1.764268in}}%
\pgfpathlineto{\pgfqpoint{3.327564in}{1.768257in}}%
\pgfpathlineto{\pgfqpoint{3.334106in}{1.772246in}}%
\pgfpathlineto{\pgfqpoint{3.371532in}{1.776235in}}%
\pgfpathlineto{\pgfqpoint{3.372195in}{1.780224in}}%
\pgfpathlineto{\pgfqpoint{3.375607in}{1.784213in}}%
\pgfpathlineto{\pgfqpoint{3.507387in}{1.788202in}}%
\pgfpathlineto{\pgfqpoint{3.508082in}{1.792191in}}%
\pgfpathlineto{\pgfqpoint{3.510202in}{1.796180in}}%
\pgfpathlineto{\pgfqpoint{3.572641in}{1.800169in}}%
\pgfpathlineto{\pgfqpoint{3.600664in}{1.804158in}}%
\pgfpathlineto{\pgfqpoint{3.605163in}{1.808147in}}%
\pgfpathlineto{\pgfqpoint{3.627137in}{1.812136in}}%
\pgfpathlineto{\pgfqpoint{3.672915in}{1.816125in}}%
\pgfpathlineto{\pgfqpoint{3.759691in}{1.820114in}}%
\pgfpathlineto{\pgfqpoint{3.779730in}{1.824103in}}%
\pgfpathlineto{\pgfqpoint{3.805105in}{1.828092in}}%
\pgfpathlineto{\pgfqpoint{3.817886in}{1.832081in}}%
\pgfpathlineto{\pgfqpoint{3.873415in}{1.836070in}}%
\pgfpathlineto{\pgfqpoint{3.881660in}{1.840059in}}%
\pgfpathlineto{\pgfqpoint{3.881812in}{1.844048in}}%
\pgfpathlineto{\pgfqpoint{3.896693in}{1.848037in}}%
\pgfpathlineto{\pgfqpoint{3.905645in}{1.852026in}}%
\pgfpathlineto{\pgfqpoint{3.941293in}{1.856015in}}%
\pgfpathlineto{\pgfqpoint{3.991108in}{1.860005in}}%
\pgfpathlineto{\pgfqpoint{3.995548in}{1.863994in}}%
\pgfpathlineto{\pgfqpoint{3.997605in}{1.867983in}}%
\pgfpathlineto{\pgfqpoint{4.003495in}{1.871972in}}%
\pgfpathlineto{\pgfqpoint{4.069642in}{1.875961in}}%
\pgfpathlineto{\pgfqpoint{4.074603in}{1.879950in}}%
\pgfpathlineto{\pgfqpoint{4.076124in}{1.883939in}}%
\pgfpathlineto{\pgfqpoint{4.128458in}{1.891917in}}%
\pgfpathlineto{\pgfqpoint{4.147919in}{1.895906in}}%
\pgfpathlineto{\pgfqpoint{4.149646in}{1.899895in}}%
\pgfpathlineto{\pgfqpoint{4.172457in}{1.903884in}}%
\pgfpathlineto{\pgfqpoint{4.183988in}{1.907873in}}%
\pgfpathlineto{\pgfqpoint{4.186390in}{1.911862in}}%
\pgfpathlineto{\pgfqpoint{4.197995in}{1.915851in}}%
\pgfpathlineto{\pgfqpoint{4.213330in}{1.919840in}}%
\pgfpathlineto{\pgfqpoint{4.260284in}{1.927818in}}%
\pgfpathlineto{\pgfqpoint{4.265468in}{1.931807in}}%
\pgfpathlineto{\pgfqpoint{4.265982in}{1.935796in}}%
\pgfpathlineto{\pgfqpoint{4.268074in}{1.939785in}}%
\pgfpathlineto{\pgfqpoint{4.282259in}{1.943774in}}%
\pgfpathlineto{\pgfqpoint{4.284052in}{1.947763in}}%
\pgfpathlineto{\pgfqpoint{4.297977in}{1.951752in}}%
\pgfpathlineto{\pgfqpoint{4.298604in}{1.955741in}}%
\pgfpathlineto{\pgfqpoint{4.304571in}{1.959730in}}%
\pgfpathlineto{\pgfqpoint{4.313846in}{1.963719in}}%
\pgfpathlineto{\pgfqpoint{4.316529in}{1.967708in}}%
\pgfpathlineto{\pgfqpoint{4.337329in}{1.971697in}}%
\pgfpathlineto{\pgfqpoint{4.362388in}{1.975686in}}%
\pgfpathlineto{\pgfqpoint{4.363279in}{1.979675in}}%
\pgfpathlineto{\pgfqpoint{4.392447in}{1.983664in}}%
\pgfpathlineto{\pgfqpoint{4.402074in}{1.987653in}}%
\pgfpathlineto{\pgfqpoint{4.406327in}{1.991642in}}%
\pgfpathlineto{\pgfqpoint{4.424704in}{1.995631in}}%
\pgfpathlineto{\pgfqpoint{4.472714in}{2.003609in}}%
\pgfpathlineto{\pgfqpoint{4.491110in}{2.007598in}}%
\pgfpathlineto{\pgfqpoint{4.530858in}{2.011587in}}%
\pgfpathlineto{\pgfqpoint{4.538327in}{2.015576in}}%
\pgfpathlineto{\pgfqpoint{4.538787in}{2.019565in}}%
\pgfpathlineto{\pgfqpoint{4.573389in}{2.023554in}}%
\pgfpathlineto{\pgfqpoint{4.573389in}{2.023554in}}%
\pgfusepath{stroke}%
\end{pgfscope}%
\begin{pgfscope}%
\pgfsetrectcap%
\pgfsetmiterjoin%
\pgfsetlinewidth{0.803000pt}%
\definecolor{currentstroke}{rgb}{0.000000,0.000000,0.000000}%
\pgfsetstrokecolor{currentstroke}%
\pgfsetdash{}{0pt}%
\pgfpathmoveto{\pgfqpoint{0.537394in}{0.467838in}}%
\pgfpathlineto{\pgfqpoint{0.537394in}{2.063444in}}%
\pgfusepath{stroke}%
\end{pgfscope}%
\begin{pgfscope}%
\pgfsetrectcap%
\pgfsetmiterjoin%
\pgfsetlinewidth{0.803000pt}%
\definecolor{currentstroke}{rgb}{0.000000,0.000000,0.000000}%
\pgfsetstrokecolor{currentstroke}%
\pgfsetdash{}{0pt}%
\pgfpathmoveto{\pgfqpoint{4.632078in}{0.467838in}}%
\pgfpathlineto{\pgfqpoint{4.632078in}{2.063444in}}%
\pgfusepath{stroke}%
\end{pgfscope}%
\begin{pgfscope}%
\pgfsetrectcap%
\pgfsetmiterjoin%
\pgfsetlinewidth{0.803000pt}%
\definecolor{currentstroke}{rgb}{0.000000,0.000000,0.000000}%
\pgfsetstrokecolor{currentstroke}%
\pgfsetdash{}{0pt}%
\pgfpathmoveto{\pgfqpoint{0.537394in}{0.467838in}}%
\pgfpathlineto{\pgfqpoint{4.632078in}{0.467838in}}%
\pgfusepath{stroke}%
\end{pgfscope}%
\begin{pgfscope}%
\pgfsetrectcap%
\pgfsetmiterjoin%
\pgfsetlinewidth{0.803000pt}%
\definecolor{currentstroke}{rgb}{0.000000,0.000000,0.000000}%
\pgfsetstrokecolor{currentstroke}%
\pgfsetdash{}{0pt}%
\pgfpathmoveto{\pgfqpoint{0.537394in}{2.063444in}}%
\pgfpathlineto{\pgfqpoint{4.632078in}{2.063444in}}%
\pgfusepath{stroke}%
\end{pgfscope}%
\begin{pgfscope}%
\pgfsetbuttcap%
\pgfsetmiterjoin%
\definecolor{currentfill}{rgb}{1.000000,1.000000,1.000000}%
\pgfsetfillcolor{currentfill}%
\pgfsetfillopacity{0.800000}%
\pgfsetlinewidth{1.003750pt}%
\definecolor{currentstroke}{rgb}{0.800000,0.800000,0.800000}%
\pgfsetstrokecolor{currentstroke}%
\pgfsetstrokeopacity{0.800000}%
\pgfsetdash{}{0pt}%
\pgfpathmoveto{\pgfqpoint{0.615172in}{0.996040in}}%
\pgfpathlineto{\pgfqpoint{1.476966in}{0.996040in}}%
\pgfpathquadraticcurveto{\pgfqpoint{1.499189in}{0.996040in}}{\pgfqpoint{1.499189in}{1.018263in}}%
\pgfpathlineto{\pgfqpoint{1.499189in}{1.985666in}}%
\pgfpathquadraticcurveto{\pgfqpoint{1.499189in}{2.007889in}}{\pgfqpoint{1.476966in}{2.007889in}}%
\pgfpathlineto{\pgfqpoint{0.615172in}{2.007889in}}%
\pgfpathquadraticcurveto{\pgfqpoint{0.592949in}{2.007889in}}{\pgfqpoint{0.592949in}{1.985666in}}%
\pgfpathlineto{\pgfqpoint{0.592949in}{1.018263in}}%
\pgfpathquadraticcurveto{\pgfqpoint{0.592949in}{0.996040in}}{\pgfqpoint{0.615172in}{0.996040in}}%
\pgfpathclose%
\pgfusepath{stroke,fill}%
\end{pgfscope}%
\begin{pgfscope}%
\pgfsetrectcap%
\pgfsetroundjoin%
\pgfsetlinewidth{1.003750pt}%
\definecolor{currentstroke}{rgb}{0.121569,0.466667,0.705882}%
\pgfsetstrokecolor{currentstroke}%
\pgfsetdash{}{0pt}%
\pgfpathmoveto{\pgfqpoint{0.637394in}{1.917915in}}%
\pgfpathlineto{\pgfqpoint{0.859616in}{1.917915in}}%
\pgfusepath{stroke}%
\end{pgfscope}%
\begin{pgfscope}%
\definecolor{textcolor}{rgb}{0.000000,0.000000,0.000000}%
\pgfsetstrokecolor{textcolor}%
\pgfsetfillcolor{textcolor}%
\pgftext[x=0.948505in,y=1.879026in,left,base]{\color{textcolor}\sffamily\fontsize{8.000000}{9.600000}\selectfont ProCount}%
\end{pgfscope}%
\begin{pgfscope}%
\pgfsetrectcap%
\pgfsetroundjoin%
\pgfsetlinewidth{1.003750pt}%
\definecolor{currentstroke}{rgb}{1.000000,0.498039,0.054902}%
\pgfsetstrokecolor{currentstroke}%
\pgfsetdash{}{0pt}%
\pgfpathmoveto{\pgfqpoint{0.637394in}{1.754829in}}%
\pgfpathlineto{\pgfqpoint{0.859616in}{1.754829in}}%
\pgfusepath{stroke}%
\end{pgfscope}%
\begin{pgfscope}%
\definecolor{textcolor}{rgb}{0.000000,0.000000,0.000000}%
\pgfsetstrokecolor{textcolor}%
\pgfsetfillcolor{textcolor}%
\pgftext[x=0.948505in,y=1.715940in,left,base]{\color{textcolor}\sffamily\fontsize{8.000000}{9.600000}\selectfont D4\textsubscript{P}}%
\end{pgfscope}%
\begin{pgfscope}%
\pgfsetrectcap%
\pgfsetroundjoin%
\pgfsetlinewidth{1.003750pt}%
\definecolor{currentstroke}{rgb}{0.172549,0.627451,0.172549}%
\pgfsetstrokecolor{currentstroke}%
\pgfsetdash{}{0pt}%
\pgfpathmoveto{\pgfqpoint{0.637394in}{1.591743in}}%
\pgfpathlineto{\pgfqpoint{0.859616in}{1.591743in}}%
\pgfusepath{stroke}%
\end{pgfscope}%
\begin{pgfscope}%
\definecolor{textcolor}{rgb}{0.000000,0.000000,0.000000}%
\pgfsetstrokecolor{textcolor}%
\pgfsetfillcolor{textcolor}%
\pgftext[x=0.948505in,y=1.552854in,left,base]{\color{textcolor}\sffamily\fontsize{8.000000}{9.600000}\selectfont projMC}%
\end{pgfscope}%
\begin{pgfscope}%
\pgfsetrectcap%
\pgfsetroundjoin%
\pgfsetlinewidth{1.003750pt}%
\definecolor{currentstroke}{rgb}{0.839216,0.152941,0.156863}%
\pgfsetstrokecolor{currentstroke}%
\pgfsetdash{}{0pt}%
\pgfpathmoveto{\pgfqpoint{0.637394in}{1.428657in}}%
\pgfpathlineto{\pgfqpoint{0.859616in}{1.428657in}}%
\pgfusepath{stroke}%
\end{pgfscope}%
\begin{pgfscope}%
\definecolor{textcolor}{rgb}{0.000000,0.000000,0.000000}%
\pgfsetstrokecolor{textcolor}%
\pgfsetfillcolor{textcolor}%
\pgftext[x=0.948505in,y=1.389768in,left,base]{\color{textcolor}\sffamily\fontsize{8.000000}{9.600000}\selectfont reSSAT}%
\end{pgfscope}%
\begin{pgfscope}%
\pgfsetbuttcap%
\pgfsetroundjoin%
\pgfsetlinewidth{1.003750pt}%
\definecolor{currentstroke}{rgb}{0.580392,0.403922,0.741176}%
\pgfsetstrokecolor{currentstroke}%
\pgfsetdash{{3.700000pt}{1.600000pt}}{0.000000pt}%
\pgfpathmoveto{\pgfqpoint{0.637394in}{1.265571in}}%
\pgfpathlineto{\pgfqpoint{0.859616in}{1.265571in}}%
\pgfusepath{stroke}%
\end{pgfscope}%
\begin{pgfscope}%
\definecolor{textcolor}{rgb}{0.000000,0.000000,0.000000}%
\pgfsetstrokecolor{textcolor}%
\pgfsetfillcolor{textcolor}%
\pgftext[x=0.948505in,y=1.226682in,left,base]{\color{textcolor}\sffamily\fontsize{8.000000}{9.600000}\selectfont VBS0}%
\end{pgfscope}%
\begin{pgfscope}%
\pgfsetbuttcap%
\pgfsetroundjoin%
\pgfsetlinewidth{1.003750pt}%
\definecolor{currentstroke}{rgb}{0.549020,0.337255,0.294118}%
\pgfsetstrokecolor{currentstroke}%
\pgfsetdash{{1.000000pt}{1.650000pt}}{0.000000pt}%
\pgfpathmoveto{\pgfqpoint{0.637394in}{1.102486in}}%
\pgfpathlineto{\pgfqpoint{0.859616in}{1.102486in}}%
\pgfusepath{stroke}%
\end{pgfscope}%
\begin{pgfscope}%
\definecolor{textcolor}{rgb}{0.000000,0.000000,0.000000}%
\pgfsetstrokecolor{textcolor}%
\pgfsetfillcolor{textcolor}%
\pgftext[x=0.948505in,y=1.063597in,left,base]{\color{textcolor}\sffamily\fontsize{8.000000}{9.600000}\selectfont VBS1}%
\end{pgfscope}%
\end{pgfpicture}%
\makeatother%
\endgroup%

    \caption{
        Experiment 3 compares our framework \procount{} to the state-of-the-art exact weighted projected model counters \dfp, \projmc{}, and \ssat{}.
        \vbs0 is the virtual best solver of the three existing tools, excluding \procount.
        \vbs1 includes all four tools.
        Adding \procount{} significantly improves the portfolio of projected model counters.
    }
    \label{figSolving}
\end{figure}
\begin{table}[t]
    \centering
    \caption{
        Experiment 3 compares our framework \procount{} to the state-of-the-art exact weighted projected model counters \dfp{}, \projmc{}, and \ssat{}. 
        For each solver, the PAR-2 score is the cumulative solving time of completed benchmarks plus twice the 1000-second timeout for each unsolved benchmark.
        There are \solvedBenchmarks{} benchmarks solved by at least one of four tools.
        By including \procount, the portfolio of tools solves \dpmcUniqueBenchmarks{} more benchmarks and achieves shorter solving time on 87 other benchmarks.
    }
    \begin{tabular}{|l|r|r|r|r|} \hline
        \multirow{2}{*}{Tool} & \multicolumn{3}{c|}{Number of benchmarks solved (of \benchmarks)} & \multirow{2}{*}{PAR-2 score} \\ \cline{2-4}
        & By no other & In shortest time & In total & \\ \hline
        \procount & \dpmcUniqueBenchmarks & \dpmcFastestBenchmarks & 283 & 1139215 \\ \hline
        \dfp{} & 50 & 235 & 345 & 1021809 \\ \hline
        \projmc{} & 0 & 8 & 275 & 1157018 \\ \hline
        \ssat{} & 1 & 16 & 154 & 1408853 \\ \hline
        \vbs0 & - & - & 346 & 1018784 \\ \hline
        \vbs1 & - & - & 390 & 933494 \\ \hline
    \end{tabular}
    \label{tableSolving}
\end{table}


%%%%%%%%%%%%%%%%%%%%%%%%%%%%%%%%%%%%%%%%%%%%%%%%%%%%%%%%%%%%%%%%%%%%%%%%%%%%%%%%

\subsubsection{Project-Join Tree Width and Computation Time}

To identify which type of benchmarks can be solved efficiently by \procount{}, we study how the performance of each projected model counter varies with the widths of graded project-join trees.
In particular, for each benchmark, we consider the width of the first graded project-join tree produced by the planner \Lg{} in Experiment 1.
% For those 346 project-join trees, the widths range from 1 to 99. Each such project-join tree was produced in less than 29 seconds. % JD: This info was already in 5.2
Figure \ref{time_vs_width} shows how these widths relate to PAR-2 scores of projected model counters. 
\procount{} seems to be the fastest solver on instances for which there exist graded project-join trees of widths between 50 and 100.
\begin{figure}[t]
    \centering
    %% Creator: Matplotlib, PGF backend
%%
%% To include the figure in your LaTeX document, write
%%   \input{<filename>.pgf}
%%
%% Make sure the required packages are loaded in your preamble
%%   \usepackage{pgf}
%%
%% and, on pdftex
%%   \usepackage[utf8]{inputenc}\DeclareUnicodeCharacter{2212}{-}
%%
%% or, on luatex and xetex
%%   \usepackage{unicode-math}
%%
%% Figures using additional raster images can only be included by \input if
%% they are in the same directory as the main LaTeX file. For loading figures
%% from other directories you can use the `import` package
%%   \usepackage{import}
%%
%% and then include the figures with
%%   \import{<path to file>}{<filename>.pgf}
%%
%% Matplotlib used the following preamble
%%   \usepackage[utf8x]{inputenc}
%%   \usepackage[T1]{fontenc}
%%
\begingroup%
\makeatletter%
\begin{pgfpicture}%
\pgfpathrectangle{\pgfpointorigin}{\pgfqpoint{6.000000in}{2.500000in}}%
\pgfusepath{use as bounding box, clip}%
\begin{pgfscope}%
\pgfsetbuttcap%
\pgfsetmiterjoin%
\definecolor{currentfill}{rgb}{1.000000,1.000000,1.000000}%
\pgfsetfillcolor{currentfill}%
\pgfsetlinewidth{0.000000pt}%
\definecolor{currentstroke}{rgb}{1.000000,1.000000,1.000000}%
\pgfsetstrokecolor{currentstroke}%
\pgfsetdash{}{0pt}%
\pgfpathmoveto{\pgfqpoint{0.000000in}{0.000000in}}%
\pgfpathlineto{\pgfqpoint{6.000000in}{0.000000in}}%
\pgfpathlineto{\pgfqpoint{6.000000in}{2.500000in}}%
\pgfpathlineto{\pgfqpoint{0.000000in}{2.500000in}}%
\pgfpathclose%
\pgfusepath{fill}%
\end{pgfscope}%
\begin{pgfscope}%
\pgfsetbuttcap%
\pgfsetmiterjoin%
\definecolor{currentfill}{rgb}{1.000000,1.000000,1.000000}%
\pgfsetfillcolor{currentfill}%
\pgfsetlinewidth{0.000000pt}%
\definecolor{currentstroke}{rgb}{0.000000,0.000000,0.000000}%
\pgfsetstrokecolor{currentstroke}%
\pgfsetstrokeopacity{0.000000}%
\pgfsetdash{}{0pt}%
\pgfpathmoveto{\pgfqpoint{0.682376in}{0.535823in}}%
\pgfpathlineto{\pgfqpoint{5.850000in}{0.535823in}}%
\pgfpathlineto{\pgfqpoint{5.850000in}{2.350000in}}%
\pgfpathlineto{\pgfqpoint{0.682376in}{2.350000in}}%
\pgfpathclose%
\pgfusepath{fill}%
\end{pgfscope}%
\begin{pgfscope}%
\pgfsetbuttcap%
\pgfsetroundjoin%
\definecolor{currentfill}{rgb}{0.000000,0.000000,0.000000}%
\pgfsetfillcolor{currentfill}%
\pgfsetlinewidth{0.803000pt}%
\definecolor{currentstroke}{rgb}{0.000000,0.000000,0.000000}%
\pgfsetstrokecolor{currentstroke}%
\pgfsetdash{}{0pt}%
\pgfsys@defobject{currentmarker}{\pgfqpoint{0.000000in}{-0.048611in}}{\pgfqpoint{0.000000in}{0.000000in}}{%
\pgfpathmoveto{\pgfqpoint{0.000000in}{0.000000in}}%
\pgfpathlineto{\pgfqpoint{0.000000in}{-0.048611in}}%
\pgfusepath{stroke,fill}%
}%
\begin{pgfscope}%
\pgfsys@transformshift{1.274066in}{0.535823in}%
\pgfsys@useobject{currentmarker}{}%
\end{pgfscope}%
\end{pgfscope}%
\begin{pgfscope}%
\definecolor{textcolor}{rgb}{0.000000,0.000000,0.000000}%
\pgfsetstrokecolor{textcolor}%
\pgfsetfillcolor{textcolor}%
\pgftext[x=1.274066in,y=0.438600in,,top]{\color{textcolor}\rmfamily\fontsize{9.000000}{10.800000}\selectfont \(\displaystyle {20}\)}%
\end{pgfscope}%
\begin{pgfscope}%
\pgfsetbuttcap%
\pgfsetroundjoin%
\definecolor{currentfill}{rgb}{0.000000,0.000000,0.000000}%
\pgfsetfillcolor{currentfill}%
\pgfsetlinewidth{0.803000pt}%
\definecolor{currentstroke}{rgb}{0.000000,0.000000,0.000000}%
\pgfsetstrokecolor{currentstroke}%
\pgfsetdash{}{0pt}%
\pgfsys@defobject{currentmarker}{\pgfqpoint{0.000000in}{-0.048611in}}{\pgfqpoint{0.000000in}{0.000000in}}{%
\pgfpathmoveto{\pgfqpoint{0.000000in}{0.000000in}}%
\pgfpathlineto{\pgfqpoint{0.000000in}{-0.048611in}}%
\pgfusepath{stroke,fill}%
}%
\begin{pgfscope}%
\pgfsys@transformshift{1.868729in}{0.535823in}%
\pgfsys@useobject{currentmarker}{}%
\end{pgfscope}%
\end{pgfscope}%
\begin{pgfscope}%
\definecolor{textcolor}{rgb}{0.000000,0.000000,0.000000}%
\pgfsetstrokecolor{textcolor}%
\pgfsetfillcolor{textcolor}%
\pgftext[x=1.868729in,y=0.438600in,,top]{\color{textcolor}\rmfamily\fontsize{9.000000}{10.800000}\selectfont \(\displaystyle {30}\)}%
\end{pgfscope}%
\begin{pgfscope}%
\pgfsetbuttcap%
\pgfsetroundjoin%
\definecolor{currentfill}{rgb}{0.000000,0.000000,0.000000}%
\pgfsetfillcolor{currentfill}%
\pgfsetlinewidth{0.803000pt}%
\definecolor{currentstroke}{rgb}{0.000000,0.000000,0.000000}%
\pgfsetstrokecolor{currentstroke}%
\pgfsetdash{}{0pt}%
\pgfsys@defobject{currentmarker}{\pgfqpoint{0.000000in}{-0.048611in}}{\pgfqpoint{0.000000in}{0.000000in}}{%
\pgfpathmoveto{\pgfqpoint{0.000000in}{0.000000in}}%
\pgfpathlineto{\pgfqpoint{0.000000in}{-0.048611in}}%
\pgfusepath{stroke,fill}%
}%
\begin{pgfscope}%
\pgfsys@transformshift{2.463392in}{0.535823in}%
\pgfsys@useobject{currentmarker}{}%
\end{pgfscope}%
\end{pgfscope}%
\begin{pgfscope}%
\definecolor{textcolor}{rgb}{0.000000,0.000000,0.000000}%
\pgfsetstrokecolor{textcolor}%
\pgfsetfillcolor{textcolor}%
\pgftext[x=2.463392in,y=0.438600in,,top]{\color{textcolor}\rmfamily\fontsize{9.000000}{10.800000}\selectfont \(\displaystyle {40}\)}%
\end{pgfscope}%
\begin{pgfscope}%
\pgfsetbuttcap%
\pgfsetroundjoin%
\definecolor{currentfill}{rgb}{0.000000,0.000000,0.000000}%
\pgfsetfillcolor{currentfill}%
\pgfsetlinewidth{0.803000pt}%
\definecolor{currentstroke}{rgb}{0.000000,0.000000,0.000000}%
\pgfsetstrokecolor{currentstroke}%
\pgfsetdash{}{0pt}%
\pgfsys@defobject{currentmarker}{\pgfqpoint{0.000000in}{-0.048611in}}{\pgfqpoint{0.000000in}{0.000000in}}{%
\pgfpathmoveto{\pgfqpoint{0.000000in}{0.000000in}}%
\pgfpathlineto{\pgfqpoint{0.000000in}{-0.048611in}}%
\pgfusepath{stroke,fill}%
}%
\begin{pgfscope}%
\pgfsys@transformshift{3.058056in}{0.535823in}%
\pgfsys@useobject{currentmarker}{}%
\end{pgfscope}%
\end{pgfscope}%
\begin{pgfscope}%
\definecolor{textcolor}{rgb}{0.000000,0.000000,0.000000}%
\pgfsetstrokecolor{textcolor}%
\pgfsetfillcolor{textcolor}%
\pgftext[x=3.058056in,y=0.438600in,,top]{\color{textcolor}\rmfamily\fontsize{9.000000}{10.800000}\selectfont \(\displaystyle {50}\)}%
\end{pgfscope}%
\begin{pgfscope}%
\pgfsetbuttcap%
\pgfsetroundjoin%
\definecolor{currentfill}{rgb}{0.000000,0.000000,0.000000}%
\pgfsetfillcolor{currentfill}%
\pgfsetlinewidth{0.803000pt}%
\definecolor{currentstroke}{rgb}{0.000000,0.000000,0.000000}%
\pgfsetstrokecolor{currentstroke}%
\pgfsetdash{}{0pt}%
\pgfsys@defobject{currentmarker}{\pgfqpoint{0.000000in}{-0.048611in}}{\pgfqpoint{0.000000in}{0.000000in}}{%
\pgfpathmoveto{\pgfqpoint{0.000000in}{0.000000in}}%
\pgfpathlineto{\pgfqpoint{0.000000in}{-0.048611in}}%
\pgfusepath{stroke,fill}%
}%
\begin{pgfscope}%
\pgfsys@transformshift{3.652719in}{0.535823in}%
\pgfsys@useobject{currentmarker}{}%
\end{pgfscope}%
\end{pgfscope}%
\begin{pgfscope}%
\definecolor{textcolor}{rgb}{0.000000,0.000000,0.000000}%
\pgfsetstrokecolor{textcolor}%
\pgfsetfillcolor{textcolor}%
\pgftext[x=3.652719in,y=0.438600in,,top]{\color{textcolor}\rmfamily\fontsize{9.000000}{10.800000}\selectfont \(\displaystyle {60}\)}%
\end{pgfscope}%
\begin{pgfscope}%
\pgfsetbuttcap%
\pgfsetroundjoin%
\definecolor{currentfill}{rgb}{0.000000,0.000000,0.000000}%
\pgfsetfillcolor{currentfill}%
\pgfsetlinewidth{0.803000pt}%
\definecolor{currentstroke}{rgb}{0.000000,0.000000,0.000000}%
\pgfsetstrokecolor{currentstroke}%
\pgfsetdash{}{0pt}%
\pgfsys@defobject{currentmarker}{\pgfqpoint{0.000000in}{-0.048611in}}{\pgfqpoint{0.000000in}{0.000000in}}{%
\pgfpathmoveto{\pgfqpoint{0.000000in}{0.000000in}}%
\pgfpathlineto{\pgfqpoint{0.000000in}{-0.048611in}}%
\pgfusepath{stroke,fill}%
}%
\begin{pgfscope}%
\pgfsys@transformshift{4.247382in}{0.535823in}%
\pgfsys@useobject{currentmarker}{}%
\end{pgfscope}%
\end{pgfscope}%
\begin{pgfscope}%
\definecolor{textcolor}{rgb}{0.000000,0.000000,0.000000}%
\pgfsetstrokecolor{textcolor}%
\pgfsetfillcolor{textcolor}%
\pgftext[x=4.247382in,y=0.438600in,,top]{\color{textcolor}\rmfamily\fontsize{9.000000}{10.800000}\selectfont \(\displaystyle {70}\)}%
\end{pgfscope}%
\begin{pgfscope}%
\pgfsetbuttcap%
\pgfsetroundjoin%
\definecolor{currentfill}{rgb}{0.000000,0.000000,0.000000}%
\pgfsetfillcolor{currentfill}%
\pgfsetlinewidth{0.803000pt}%
\definecolor{currentstroke}{rgb}{0.000000,0.000000,0.000000}%
\pgfsetstrokecolor{currentstroke}%
\pgfsetdash{}{0pt}%
\pgfsys@defobject{currentmarker}{\pgfqpoint{0.000000in}{-0.048611in}}{\pgfqpoint{0.000000in}{0.000000in}}{%
\pgfpathmoveto{\pgfqpoint{0.000000in}{0.000000in}}%
\pgfpathlineto{\pgfqpoint{0.000000in}{-0.048611in}}%
\pgfusepath{stroke,fill}%
}%
\begin{pgfscope}%
\pgfsys@transformshift{4.842046in}{0.535823in}%
\pgfsys@useobject{currentmarker}{}%
\end{pgfscope}%
\end{pgfscope}%
\begin{pgfscope}%
\definecolor{textcolor}{rgb}{0.000000,0.000000,0.000000}%
\pgfsetstrokecolor{textcolor}%
\pgfsetfillcolor{textcolor}%
\pgftext[x=4.842046in,y=0.438600in,,top]{\color{textcolor}\rmfamily\fontsize{9.000000}{10.800000}\selectfont \(\displaystyle {80}\)}%
\end{pgfscope}%
\begin{pgfscope}%
\pgfsetbuttcap%
\pgfsetroundjoin%
\definecolor{currentfill}{rgb}{0.000000,0.000000,0.000000}%
\pgfsetfillcolor{currentfill}%
\pgfsetlinewidth{0.803000pt}%
\definecolor{currentstroke}{rgb}{0.000000,0.000000,0.000000}%
\pgfsetstrokecolor{currentstroke}%
\pgfsetdash{}{0pt}%
\pgfsys@defobject{currentmarker}{\pgfqpoint{0.000000in}{-0.048611in}}{\pgfqpoint{0.000000in}{0.000000in}}{%
\pgfpathmoveto{\pgfqpoint{0.000000in}{0.000000in}}%
\pgfpathlineto{\pgfqpoint{0.000000in}{-0.048611in}}%
\pgfusepath{stroke,fill}%
}%
\begin{pgfscope}%
\pgfsys@transformshift{5.436709in}{0.535823in}%
\pgfsys@useobject{currentmarker}{}%
\end{pgfscope}%
\end{pgfscope}%
\begin{pgfscope}%
\definecolor{textcolor}{rgb}{0.000000,0.000000,0.000000}%
\pgfsetstrokecolor{textcolor}%
\pgfsetfillcolor{textcolor}%
\pgftext[x=5.436709in,y=0.438600in,,top]{\color{textcolor}\rmfamily\fontsize{9.000000}{10.800000}\selectfont \(\displaystyle {90}\)}%
\end{pgfscope}%
\begin{pgfscope}%
\definecolor{textcolor}{rgb}{0.000000,0.000000,0.000000}%
\pgfsetstrokecolor{textcolor}%
\pgfsetfillcolor{textcolor}%
\pgftext[x=3.266188in,y=0.272655in,,top]{\color{textcolor}\rmfamily\fontsize{10.000000}{12.000000}\selectfont Mean of 10 project-join tree widths}%
\end{pgfscope}%
\begin{pgfscope}%
\pgfsetbuttcap%
\pgfsetroundjoin%
\definecolor{currentfill}{rgb}{0.000000,0.000000,0.000000}%
\pgfsetfillcolor{currentfill}%
\pgfsetlinewidth{0.803000pt}%
\definecolor{currentstroke}{rgb}{0.000000,0.000000,0.000000}%
\pgfsetstrokecolor{currentstroke}%
\pgfsetdash{}{0pt}%
\pgfsys@defobject{currentmarker}{\pgfqpoint{-0.048611in}{0.000000in}}{\pgfqpoint{-0.000000in}{0.000000in}}{%
\pgfpathmoveto{\pgfqpoint{-0.000000in}{0.000000in}}%
\pgfpathlineto{\pgfqpoint{-0.048611in}{0.000000in}}%
\pgfusepath{stroke,fill}%
}%
\begin{pgfscope}%
\pgfsys@transformshift{0.682376in}{0.618191in}%
\pgfsys@useobject{currentmarker}{}%
\end{pgfscope}%
\end{pgfscope}%
\begin{pgfscope}%
\definecolor{textcolor}{rgb}{0.000000,0.000000,0.000000}%
\pgfsetstrokecolor{textcolor}%
\pgfsetfillcolor{textcolor}%
\pgftext[x=0.520918in, y=0.575146in, left, base]{\color{textcolor}\rmfamily\fontsize{9.000000}{10.800000}\selectfont \(\displaystyle {0}\)}%
\end{pgfscope}%
\begin{pgfscope}%
\pgfsetbuttcap%
\pgfsetroundjoin%
\definecolor{currentfill}{rgb}{0.000000,0.000000,0.000000}%
\pgfsetfillcolor{currentfill}%
\pgfsetlinewidth{0.803000pt}%
\definecolor{currentstroke}{rgb}{0.000000,0.000000,0.000000}%
\pgfsetstrokecolor{currentstroke}%
\pgfsetdash{}{0pt}%
\pgfsys@defobject{currentmarker}{\pgfqpoint{-0.048611in}{0.000000in}}{\pgfqpoint{-0.000000in}{0.000000in}}{%
\pgfpathmoveto{\pgfqpoint{-0.000000in}{0.000000in}}%
\pgfpathlineto{\pgfqpoint{-0.048611in}{0.000000in}}%
\pgfusepath{stroke,fill}%
}%
\begin{pgfscope}%
\pgfsys@transformshift{0.682376in}{1.030528in}%
\pgfsys@useobject{currentmarker}{}%
\end{pgfscope}%
\end{pgfscope}%
\begin{pgfscope}%
\definecolor{textcolor}{rgb}{0.000000,0.000000,0.000000}%
\pgfsetstrokecolor{textcolor}%
\pgfsetfillcolor{textcolor}%
\pgftext[x=0.392446in, y=0.987483in, left, base]{\color{textcolor}\rmfamily\fontsize{9.000000}{10.800000}\selectfont \(\displaystyle {500}\)}%
\end{pgfscope}%
\begin{pgfscope}%
\pgfsetbuttcap%
\pgfsetroundjoin%
\definecolor{currentfill}{rgb}{0.000000,0.000000,0.000000}%
\pgfsetfillcolor{currentfill}%
\pgfsetlinewidth{0.803000pt}%
\definecolor{currentstroke}{rgb}{0.000000,0.000000,0.000000}%
\pgfsetstrokecolor{currentstroke}%
\pgfsetdash{}{0pt}%
\pgfsys@defobject{currentmarker}{\pgfqpoint{-0.048611in}{0.000000in}}{\pgfqpoint{-0.000000in}{0.000000in}}{%
\pgfpathmoveto{\pgfqpoint{-0.000000in}{0.000000in}}%
\pgfpathlineto{\pgfqpoint{-0.048611in}{0.000000in}}%
\pgfusepath{stroke,fill}%
}%
\begin{pgfscope}%
\pgfsys@transformshift{0.682376in}{1.442864in}%
\pgfsys@useobject{currentmarker}{}%
\end{pgfscope}%
\end{pgfscope}%
\begin{pgfscope}%
\definecolor{textcolor}{rgb}{0.000000,0.000000,0.000000}%
\pgfsetstrokecolor{textcolor}%
\pgfsetfillcolor{textcolor}%
\pgftext[x=0.328211in, y=1.399819in, left, base]{\color{textcolor}\rmfamily\fontsize{9.000000}{10.800000}\selectfont \(\displaystyle {1000}\)}%
\end{pgfscope}%
\begin{pgfscope}%
\pgfsetbuttcap%
\pgfsetroundjoin%
\definecolor{currentfill}{rgb}{0.000000,0.000000,0.000000}%
\pgfsetfillcolor{currentfill}%
\pgfsetlinewidth{0.803000pt}%
\definecolor{currentstroke}{rgb}{0.000000,0.000000,0.000000}%
\pgfsetstrokecolor{currentstroke}%
\pgfsetdash{}{0pt}%
\pgfsys@defobject{currentmarker}{\pgfqpoint{-0.048611in}{0.000000in}}{\pgfqpoint{-0.000000in}{0.000000in}}{%
\pgfpathmoveto{\pgfqpoint{-0.000000in}{0.000000in}}%
\pgfpathlineto{\pgfqpoint{-0.048611in}{0.000000in}}%
\pgfusepath{stroke,fill}%
}%
\begin{pgfscope}%
\pgfsys@transformshift{0.682376in}{1.855201in}%
\pgfsys@useobject{currentmarker}{}%
\end{pgfscope}%
\end{pgfscope}%
\begin{pgfscope}%
\definecolor{textcolor}{rgb}{0.000000,0.000000,0.000000}%
\pgfsetstrokecolor{textcolor}%
\pgfsetfillcolor{textcolor}%
\pgftext[x=0.328211in, y=1.812156in, left, base]{\color{textcolor}\rmfamily\fontsize{9.000000}{10.800000}\selectfont \(\displaystyle {1500}\)}%
\end{pgfscope}%
\begin{pgfscope}%
\pgfsetbuttcap%
\pgfsetroundjoin%
\definecolor{currentfill}{rgb}{0.000000,0.000000,0.000000}%
\pgfsetfillcolor{currentfill}%
\pgfsetlinewidth{0.803000pt}%
\definecolor{currentstroke}{rgb}{0.000000,0.000000,0.000000}%
\pgfsetstrokecolor{currentstroke}%
\pgfsetdash{}{0pt}%
\pgfsys@defobject{currentmarker}{\pgfqpoint{-0.048611in}{0.000000in}}{\pgfqpoint{-0.000000in}{0.000000in}}{%
\pgfpathmoveto{\pgfqpoint{-0.000000in}{0.000000in}}%
\pgfpathlineto{\pgfqpoint{-0.048611in}{0.000000in}}%
\pgfusepath{stroke,fill}%
}%
\begin{pgfscope}%
\pgfsys@transformshift{0.682376in}{2.267537in}%
\pgfsys@useobject{currentmarker}{}%
\end{pgfscope}%
\end{pgfscope}%
\begin{pgfscope}%
\definecolor{textcolor}{rgb}{0.000000,0.000000,0.000000}%
\pgfsetstrokecolor{textcolor}%
\pgfsetfillcolor{textcolor}%
\pgftext[x=0.328211in, y=2.224492in, left, base]{\color{textcolor}\rmfamily\fontsize{9.000000}{10.800000}\selectfont \(\displaystyle {2000}\)}%
\end{pgfscope}%
\begin{pgfscope}%
\definecolor{textcolor}{rgb}{0.000000,0.000000,0.000000}%
\pgfsetstrokecolor{textcolor}%
\pgfsetfillcolor{textcolor}%
\pgftext[x=0.272655in,y=1.442911in,,bottom,rotate=90.000000]{\color{textcolor}\rmfamily\fontsize{10.000000}{12.000000}\selectfont Mean PAR-2 score of 10 widths}%
\end{pgfscope}%
\begin{pgfscope}%
\pgfpathrectangle{\pgfqpoint{0.682376in}{0.535823in}}{\pgfqpoint{5.167624in}{1.814177in}}%
\pgfusepath{clip}%
\pgfsetrectcap%
\pgfsetroundjoin%
\pgfsetlinewidth{1.003750pt}%
\definecolor{currentstroke}{rgb}{0.121569,0.466667,0.705882}%
\pgfsetstrokecolor{currentstroke}%
\pgfsetdash{}{0pt}%
\pgfpathmoveto{\pgfqpoint{0.917268in}{0.618291in}}%
\pgfpathlineto{\pgfqpoint{0.976734in}{0.618286in}}%
\pgfpathlineto{\pgfqpoint{1.036200in}{0.618286in}}%
\pgfpathlineto{\pgfqpoint{1.095667in}{0.618285in}}%
\pgfpathlineto{\pgfqpoint{1.155133in}{0.618286in}}%
\pgfpathlineto{\pgfqpoint{1.214599in}{0.618287in}}%
\pgfpathlineto{\pgfqpoint{1.274066in}{0.642191in}}%
\pgfpathlineto{\pgfqpoint{1.333532in}{0.640882in}}%
\pgfpathlineto{\pgfqpoint{1.392998in}{0.662271in}}%
\pgfpathlineto{\pgfqpoint{1.452465in}{0.686069in}}%
\pgfpathlineto{\pgfqpoint{1.511931in}{0.693259in}}%
\pgfpathlineto{\pgfqpoint{1.571397in}{0.708251in}}%
\pgfpathlineto{\pgfqpoint{1.630864in}{0.745161in}}%
\pgfpathlineto{\pgfqpoint{1.690330in}{0.740465in}}%
\pgfpathlineto{\pgfqpoint{1.749796in}{0.742775in}}%
\pgfpathlineto{\pgfqpoint{1.809263in}{0.771015in}}%
\pgfpathlineto{\pgfqpoint{1.868729in}{0.773899in}}%
\pgfpathlineto{\pgfqpoint{1.928195in}{0.824476in}}%
\pgfpathlineto{\pgfqpoint{1.987662in}{0.838216in}}%
\pgfpathlineto{\pgfqpoint{2.047128in}{0.838218in}}%
\pgfpathlineto{\pgfqpoint{2.106594in}{0.776898in}}%
\pgfpathlineto{\pgfqpoint{2.166061in}{0.805026in}}%
\pgfpathlineto{\pgfqpoint{2.225527in}{0.805027in}}%
\pgfpathlineto{\pgfqpoint{2.284993in}{0.768251in}}%
\pgfpathlineto{\pgfqpoint{2.344460in}{0.773908in}}%
\pgfpathlineto{\pgfqpoint{2.403926in}{0.795030in}}%
\pgfpathlineto{\pgfqpoint{2.463392in}{0.858213in}}%
\pgfpathlineto{\pgfqpoint{2.522859in}{0.878731in}}%
\pgfpathlineto{\pgfqpoint{2.582325in}{0.909366in}}%
\pgfpathlineto{\pgfqpoint{2.641791in}{1.039411in}}%
\pgfpathlineto{\pgfqpoint{2.701258in}{1.030639in}}%
\pgfpathlineto{\pgfqpoint{2.760724in}{1.060812in}}%
\pgfpathlineto{\pgfqpoint{2.820190in}{1.021490in}}%
\pgfpathlineto{\pgfqpoint{2.879657in}{1.039432in}}%
\pgfpathlineto{\pgfqpoint{2.939123in}{1.030674in}}%
\pgfpathlineto{\pgfqpoint{2.998589in}{1.030741in}}%
\pgfpathlineto{\pgfqpoint{3.058056in}{1.022384in}}%
\pgfpathlineto{\pgfqpoint{3.117522in}{0.934311in}}%
\pgfpathlineto{\pgfqpoint{3.176988in}{0.932658in}}%
\pgfpathlineto{\pgfqpoint{3.236455in}{0.955854in}}%
\pgfpathlineto{\pgfqpoint{3.295921in}{0.854125in}}%
\pgfpathlineto{\pgfqpoint{3.355387in}{0.859876in}}%
\pgfpathlineto{\pgfqpoint{3.414854in}{0.901288in}}%
\pgfpathlineto{\pgfqpoint{3.474320in}{0.902991in}}%
\pgfpathlineto{\pgfqpoint{3.533786in}{0.905497in}}%
\pgfpathlineto{\pgfqpoint{3.593253in}{0.965304in}}%
\pgfpathlineto{\pgfqpoint{3.652719in}{0.974674in}}%
\pgfpathlineto{\pgfqpoint{3.712185in}{1.019080in}}%
\pgfpathlineto{\pgfqpoint{3.771652in}{0.979892in}}%
\pgfpathlineto{\pgfqpoint{3.831118in}{0.958880in}}%
\pgfpathlineto{\pgfqpoint{3.890584in}{0.882692in}}%
\pgfpathlineto{\pgfqpoint{3.950051in}{0.938975in}}%
\pgfpathlineto{\pgfqpoint{4.009517in}{0.899359in}}%
\pgfpathlineto{\pgfqpoint{4.068983in}{0.954036in}}%
\pgfpathlineto{\pgfqpoint{4.128450in}{0.961560in}}%
\pgfpathlineto{\pgfqpoint{4.187916in}{0.975500in}}%
\pgfpathlineto{\pgfqpoint{4.247382in}{1.012976in}}%
\pgfpathlineto{\pgfqpoint{4.306849in}{1.031670in}}%
\pgfpathlineto{\pgfqpoint{4.366315in}{1.041751in}}%
\pgfpathlineto{\pgfqpoint{4.425781in}{1.167931in}}%
\pgfpathlineto{\pgfqpoint{4.485248in}{1.266345in}}%
\pgfpathlineto{\pgfqpoint{4.544714in}{1.184097in}}%
\pgfpathlineto{\pgfqpoint{4.604180in}{1.184103in}}%
\pgfpathlineto{\pgfqpoint{4.663647in}{1.301034in}}%
\pgfpathlineto{\pgfqpoint{4.723113in}{1.288345in}}%
\pgfpathlineto{\pgfqpoint{4.782579in}{1.198347in}}%
\pgfpathlineto{\pgfqpoint{4.842046in}{1.152537in}}%
\pgfpathlineto{\pgfqpoint{4.901512in}{1.120680in}}%
\pgfpathlineto{\pgfqpoint{4.960978in}{1.135458in}}%
\pgfpathlineto{\pgfqpoint{5.020445in}{1.198352in}}%
\pgfpathlineto{\pgfqpoint{5.079911in}{1.093673in}}%
\pgfpathlineto{\pgfqpoint{5.139377in}{1.081674in}}%
\pgfpathlineto{\pgfqpoint{5.198844in}{1.281600in}}%
\pgfpathlineto{\pgfqpoint{5.258310in}{1.238530in}}%
\pgfpathlineto{\pgfqpoint{5.317776in}{1.247464in}}%
\pgfpathlineto{\pgfqpoint{5.377243in}{1.231894in}}%
\pgfpathlineto{\pgfqpoint{5.436709in}{1.431683in}}%
\pgfpathlineto{\pgfqpoint{5.496175in}{1.468651in}}%
\pgfpathlineto{\pgfqpoint{5.555642in}{1.564517in}}%
\pgfpathlineto{\pgfqpoint{5.615108in}{1.654981in}}%
\pgfusepath{stroke}%
\end{pgfscope}%
\begin{pgfscope}%
\pgfpathrectangle{\pgfqpoint{0.682376in}{0.535823in}}{\pgfqpoint{5.167624in}{1.814177in}}%
\pgfusepath{clip}%
\pgfsetrectcap%
\pgfsetroundjoin%
\pgfsetlinewidth{1.003750pt}%
\definecolor{currentstroke}{rgb}{1.000000,0.498039,0.054902}%
\pgfsetstrokecolor{currentstroke}%
\pgfsetdash{}{0pt}%
\pgfpathmoveto{\pgfqpoint{0.917268in}{0.618471in}}%
\pgfpathlineto{\pgfqpoint{0.976734in}{0.618478in}}%
\pgfpathlineto{\pgfqpoint{1.036200in}{0.618471in}}%
\pgfpathlineto{\pgfqpoint{1.095667in}{0.618442in}}%
\pgfpathlineto{\pgfqpoint{1.155133in}{0.618448in}}%
\pgfpathlineto{\pgfqpoint{1.214599in}{0.618429in}}%
\pgfpathlineto{\pgfqpoint{1.274066in}{0.619149in}}%
\pgfpathlineto{\pgfqpoint{1.333532in}{0.620519in}}%
\pgfpathlineto{\pgfqpoint{1.392998in}{0.621272in}}%
\pgfpathlineto{\pgfqpoint{1.452465in}{0.621357in}}%
\pgfpathlineto{\pgfqpoint{1.511931in}{0.621683in}}%
\pgfpathlineto{\pgfqpoint{1.571397in}{0.622186in}}%
\pgfpathlineto{\pgfqpoint{1.630864in}{0.622421in}}%
\pgfpathlineto{\pgfqpoint{1.690330in}{0.622304in}}%
\pgfpathlineto{\pgfqpoint{1.749796in}{0.622383in}}%
\pgfpathlineto{\pgfqpoint{1.809263in}{0.622306in}}%
\pgfpathlineto{\pgfqpoint{1.868729in}{0.622383in}}%
\pgfpathlineto{\pgfqpoint{1.928195in}{0.623555in}}%
\pgfpathlineto{\pgfqpoint{1.987662in}{0.626726in}}%
\pgfpathlineto{\pgfqpoint{2.047128in}{0.628471in}}%
\pgfpathlineto{\pgfqpoint{2.106594in}{0.627091in}}%
\pgfpathlineto{\pgfqpoint{2.166061in}{0.626933in}}%
\pgfpathlineto{\pgfqpoint{2.225527in}{0.626948in}}%
\pgfpathlineto{\pgfqpoint{2.284993in}{0.626707in}}%
\pgfpathlineto{\pgfqpoint{2.344460in}{0.626988in}}%
\pgfpathlineto{\pgfqpoint{2.403926in}{0.634982in}}%
\pgfpathlineto{\pgfqpoint{2.463392in}{0.705918in}}%
\pgfpathlineto{\pgfqpoint{2.522859in}{0.732021in}}%
\pgfpathlineto{\pgfqpoint{2.582325in}{0.744502in}}%
\pgfpathlineto{\pgfqpoint{2.641791in}{0.861821in}}%
\pgfpathlineto{\pgfqpoint{2.701258in}{0.877187in}}%
\pgfpathlineto{\pgfqpoint{2.760724in}{0.915376in}}%
\pgfpathlineto{\pgfqpoint{2.820190in}{0.889488in}}%
\pgfpathlineto{\pgfqpoint{2.879657in}{0.877938in}}%
\pgfpathlineto{\pgfqpoint{2.939123in}{1.009878in}}%
\pgfpathlineto{\pgfqpoint{2.998589in}{1.009921in}}%
\pgfpathlineto{\pgfqpoint{3.058056in}{1.194273in}}%
\pgfpathlineto{\pgfqpoint{3.117522in}{1.136141in}}%
\pgfpathlineto{\pgfqpoint{3.176988in}{1.158197in}}%
\pgfpathlineto{\pgfqpoint{3.236455in}{1.144479in}}%
\pgfpathlineto{\pgfqpoint{3.295921in}{1.122315in}}%
\pgfpathlineto{\pgfqpoint{3.355387in}{1.132539in}}%
\pgfpathlineto{\pgfqpoint{3.414854in}{1.198170in}}%
\pgfpathlineto{\pgfqpoint{3.474320in}{1.320387in}}%
\pgfpathlineto{\pgfqpoint{3.533786in}{1.381113in}}%
\pgfpathlineto{\pgfqpoint{3.593253in}{1.265266in}}%
\pgfpathlineto{\pgfqpoint{3.652719in}{1.234977in}}%
\pgfpathlineto{\pgfqpoint{3.712185in}{0.979374in}}%
\pgfpathlineto{\pgfqpoint{3.771652in}{1.015150in}}%
\pgfpathlineto{\pgfqpoint{3.831118in}{1.014238in}}%
\pgfpathlineto{\pgfqpoint{3.890584in}{1.060962in}}%
\pgfpathlineto{\pgfqpoint{3.950051in}{1.025506in}}%
\pgfpathlineto{\pgfqpoint{4.009517in}{1.183547in}}%
\pgfpathlineto{\pgfqpoint{4.068983in}{1.228698in}}%
\pgfpathlineto{\pgfqpoint{4.128450in}{1.275890in}}%
\pgfpathlineto{\pgfqpoint{4.187916in}{1.265751in}}%
\pgfpathlineto{\pgfqpoint{4.247382in}{1.303181in}}%
\pgfpathlineto{\pgfqpoint{4.306849in}{1.333594in}}%
\pgfpathlineto{\pgfqpoint{4.366315in}{1.385441in}}%
\pgfpathlineto{\pgfqpoint{4.425781in}{1.441168in}}%
\pgfpathlineto{\pgfqpoint{4.485248in}{1.468078in}}%
\pgfpathlineto{\pgfqpoint{4.544714in}{1.591141in}}%
\pgfpathlineto{\pgfqpoint{4.604180in}{1.591134in}}%
\pgfpathlineto{\pgfqpoint{4.663647in}{1.522982in}}%
\pgfpathlineto{\pgfqpoint{4.723113in}{1.522985in}}%
\pgfpathlineto{\pgfqpoint{4.782579in}{1.443159in}}%
\pgfpathlineto{\pgfqpoint{4.842046in}{1.443166in}}%
\pgfpathlineto{\pgfqpoint{4.901512in}{1.419612in}}%
\pgfpathlineto{\pgfqpoint{4.960978in}{1.443182in}}%
\pgfpathlineto{\pgfqpoint{5.020445in}{1.488979in}}%
\pgfpathlineto{\pgfqpoint{5.079911in}{1.510088in}}%
\pgfpathlineto{\pgfqpoint{5.139377in}{1.530002in}}%
\pgfpathlineto{\pgfqpoint{5.198844in}{1.443174in}}%
\pgfpathlineto{\pgfqpoint{5.258310in}{1.663039in}}%
\pgfpathlineto{\pgfqpoint{5.317776in}{1.717994in}}%
\pgfpathlineto{\pgfqpoint{5.377243in}{1.737638in}}%
\pgfpathlineto{\pgfqpoint{5.436709in}{1.909247in}}%
\pgfpathlineto{\pgfqpoint{5.496175in}{1.967918in}}%
\pgfpathlineto{\pgfqpoint{5.555642in}{1.872374in}}%
\pgfpathlineto{\pgfqpoint{5.615108in}{1.824154in}}%
\pgfusepath{stroke}%
\end{pgfscope}%
\begin{pgfscope}%
\pgfpathrectangle{\pgfqpoint{0.682376in}{0.535823in}}{\pgfqpoint{5.167624in}{1.814177in}}%
\pgfusepath{clip}%
\pgfsetrectcap%
\pgfsetroundjoin%
\pgfsetlinewidth{1.003750pt}%
\definecolor{currentstroke}{rgb}{0.172549,0.627451,0.172549}%
\pgfsetstrokecolor{currentstroke}%
\pgfsetdash{}{0pt}%
\pgfpathmoveto{\pgfqpoint{0.917268in}{0.618455in}}%
\pgfpathlineto{\pgfqpoint{0.976734in}{0.618463in}}%
\pgfpathlineto{\pgfqpoint{1.036200in}{0.618458in}}%
\pgfpathlineto{\pgfqpoint{1.095667in}{0.618437in}}%
\pgfpathlineto{\pgfqpoint{1.155133in}{0.618444in}}%
\pgfpathlineto{\pgfqpoint{1.214599in}{0.618435in}}%
\pgfpathlineto{\pgfqpoint{1.274066in}{0.618859in}}%
\pgfpathlineto{\pgfqpoint{1.333532in}{0.619374in}}%
\pgfpathlineto{\pgfqpoint{1.392998in}{0.619859in}}%
\pgfpathlineto{\pgfqpoint{1.452465in}{0.619934in}}%
\pgfpathlineto{\pgfqpoint{1.511931in}{0.620124in}}%
\pgfpathlineto{\pgfqpoint{1.571397in}{0.620362in}}%
\pgfpathlineto{\pgfqpoint{1.630864in}{0.620502in}}%
\pgfpathlineto{\pgfqpoint{1.690330in}{0.621135in}}%
\pgfpathlineto{\pgfqpoint{1.749796in}{0.621191in}}%
\pgfpathlineto{\pgfqpoint{1.809263in}{0.621138in}}%
\pgfpathlineto{\pgfqpoint{1.868729in}{0.621190in}}%
\pgfpathlineto{\pgfqpoint{1.928195in}{0.625031in}}%
\pgfpathlineto{\pgfqpoint{1.987662in}{0.660494in}}%
\pgfpathlineto{\pgfqpoint{2.047128in}{0.696867in}}%
\pgfpathlineto{\pgfqpoint{2.106594in}{0.686247in}}%
\pgfpathlineto{\pgfqpoint{2.166061in}{0.685209in}}%
\pgfpathlineto{\pgfqpoint{2.225527in}{0.697549in}}%
\pgfpathlineto{\pgfqpoint{2.284993in}{0.697624in}}%
\pgfpathlineto{\pgfqpoint{2.344460in}{0.699901in}}%
\pgfpathlineto{\pgfqpoint{2.403926in}{0.733304in}}%
\pgfpathlineto{\pgfqpoint{2.463392in}{0.825356in}}%
\pgfpathlineto{\pgfqpoint{2.522859in}{0.864208in}}%
\pgfpathlineto{\pgfqpoint{2.582325in}{0.890243in}}%
\pgfpathlineto{\pgfqpoint{2.641791in}{1.021668in}}%
\pgfpathlineto{\pgfqpoint{2.701258in}{1.011386in}}%
\pgfpathlineto{\pgfqpoint{2.760724in}{1.069809in}}%
\pgfpathlineto{\pgfqpoint{2.820190in}{1.035146in}}%
\pgfpathlineto{\pgfqpoint{2.879657in}{1.003503in}}%
\pgfpathlineto{\pgfqpoint{2.939123in}{1.129509in}}%
\pgfpathlineto{\pgfqpoint{2.998589in}{1.132771in}}%
\pgfpathlineto{\pgfqpoint{3.058056in}{1.284992in}}%
\pgfpathlineto{\pgfqpoint{3.117522in}{1.212857in}}%
\pgfpathlineto{\pgfqpoint{3.176988in}{1.221051in}}%
\pgfpathlineto{\pgfqpoint{3.236455in}{1.197301in}}%
\pgfpathlineto{\pgfqpoint{3.295921in}{1.142489in}}%
\pgfpathlineto{\pgfqpoint{3.355387in}{1.154966in}}%
\pgfpathlineto{\pgfqpoint{3.414854in}{1.212229in}}%
\pgfpathlineto{\pgfqpoint{3.474320in}{1.404208in}}%
\pgfpathlineto{\pgfqpoint{3.533786in}{1.517928in}}%
\pgfpathlineto{\pgfqpoint{3.593253in}{1.454110in}}%
\pgfpathlineto{\pgfqpoint{3.652719in}{1.623603in}}%
\pgfpathlineto{\pgfqpoint{3.712185in}{1.604569in}}%
\pgfpathlineto{\pgfqpoint{3.771652in}{1.762571in}}%
\pgfpathlineto{\pgfqpoint{3.831118in}{1.794131in}}%
\pgfpathlineto{\pgfqpoint{3.890584in}{1.783118in}}%
\pgfpathlineto{\pgfqpoint{3.950051in}{1.769050in}}%
\pgfpathlineto{\pgfqpoint{4.009517in}{1.785484in}}%
\pgfpathlineto{\pgfqpoint{4.068983in}{1.766633in}}%
\pgfpathlineto{\pgfqpoint{4.128450in}{1.778202in}}%
\pgfpathlineto{\pgfqpoint{4.187916in}{1.755965in}}%
\pgfpathlineto{\pgfqpoint{4.247382in}{1.780562in}}%
\pgfpathlineto{\pgfqpoint{4.306849in}{1.678848in}}%
\pgfpathlineto{\pgfqpoint{4.366315in}{1.643871in}}%
\pgfpathlineto{\pgfqpoint{4.425781in}{1.518305in}}%
\pgfpathlineto{\pgfqpoint{4.485248in}{1.468349in}}%
\pgfpathlineto{\pgfqpoint{4.544714in}{1.592978in}}%
\pgfpathlineto{\pgfqpoint{4.604180in}{1.592899in}}%
\pgfpathlineto{\pgfqpoint{4.663647in}{1.525176in}}%
\pgfpathlineto{\pgfqpoint{4.723113in}{1.525214in}}%
\pgfpathlineto{\pgfqpoint{4.782579in}{1.445156in}}%
\pgfpathlineto{\pgfqpoint{4.842046in}{1.490963in}}%
\pgfpathlineto{\pgfqpoint{4.901512in}{1.468776in}}%
\pgfpathlineto{\pgfqpoint{4.960978in}{1.542283in}}%
\pgfpathlineto{\pgfqpoint{5.020445in}{1.582575in}}%
\pgfpathlineto{\pgfqpoint{5.079911in}{1.645657in}}%
\pgfpathlineto{\pgfqpoint{5.139377in}{1.705405in}}%
\pgfpathlineto{\pgfqpoint{5.198844in}{1.678859in}}%
\pgfpathlineto{\pgfqpoint{5.258310in}{1.882935in}}%
\pgfpathlineto{\pgfqpoint{5.317776in}{1.917899in}}%
\pgfpathlineto{\pgfqpoint{5.377243in}{1.973145in}}%
\pgfpathlineto{\pgfqpoint{5.436709in}{2.195830in}}%
\pgfpathlineto{\pgfqpoint{5.496175in}{2.192570in}}%
\pgfpathlineto{\pgfqpoint{5.555642in}{2.069792in}}%
\pgfpathlineto{\pgfqpoint{5.615108in}{1.951649in}}%
\pgfusepath{stroke}%
\end{pgfscope}%
\begin{pgfscope}%
\pgfpathrectangle{\pgfqpoint{0.682376in}{0.535823in}}{\pgfqpoint{5.167624in}{1.814177in}}%
\pgfusepath{clip}%
\pgfsetrectcap%
\pgfsetroundjoin%
\pgfsetlinewidth{1.003750pt}%
\definecolor{currentstroke}{rgb}{0.839216,0.152941,0.156863}%
\pgfsetstrokecolor{currentstroke}%
\pgfsetdash{}{0pt}%
\pgfpathmoveto{\pgfqpoint{0.917268in}{1.090265in}}%
\pgfpathlineto{\pgfqpoint{0.976734in}{1.102001in}}%
\pgfpathlineto{\pgfqpoint{1.036200in}{1.090487in}}%
\pgfpathlineto{\pgfqpoint{1.095667in}{1.044996in}}%
\pgfpathlineto{\pgfqpoint{1.155133in}{1.055736in}}%
\pgfpathlineto{\pgfqpoint{1.214599in}{1.021093in}}%
\pgfpathlineto{\pgfqpoint{1.274066in}{1.177014in}}%
\pgfpathlineto{\pgfqpoint{1.333532in}{1.254477in}}%
\pgfpathlineto{\pgfqpoint{1.392998in}{1.303483in}}%
\pgfpathlineto{\pgfqpoint{1.452465in}{1.344893in}}%
\pgfpathlineto{\pgfqpoint{1.511931in}{1.425179in}}%
\pgfpathlineto{\pgfqpoint{1.571397in}{1.228060in}}%
\pgfpathlineto{\pgfqpoint{1.630864in}{1.263464in}}%
\pgfpathlineto{\pgfqpoint{1.690330in}{1.275006in}}%
\pgfpathlineto{\pgfqpoint{1.749796in}{1.330593in}}%
\pgfpathlineto{\pgfqpoint{1.809263in}{1.336391in}}%
\pgfpathlineto{\pgfqpoint{1.868729in}{1.442962in}}%
\pgfpathlineto{\pgfqpoint{1.928195in}{1.309273in}}%
\pgfpathlineto{\pgfqpoint{1.987662in}{1.460323in}}%
\pgfpathlineto{\pgfqpoint{2.047128in}{1.496975in}}%
\pgfpathlineto{\pgfqpoint{2.106594in}{1.600760in}}%
\pgfpathlineto{\pgfqpoint{2.166061in}{1.702045in}}%
\pgfpathlineto{\pgfqpoint{2.225527in}{1.731773in}}%
\pgfpathlineto{\pgfqpoint{2.284993in}{1.840846in}}%
\pgfpathlineto{\pgfqpoint{2.344460in}{1.881990in}}%
\pgfpathlineto{\pgfqpoint{2.403926in}{1.975585in}}%
\pgfpathlineto{\pgfqpoint{2.463392in}{2.070361in}}%
\pgfpathlineto{\pgfqpoint{2.522859in}{2.077279in}}%
\pgfpathlineto{\pgfqpoint{2.582325in}{2.235313in}}%
\pgfpathlineto{\pgfqpoint{2.641791in}{2.232570in}}%
\pgfpathlineto{\pgfqpoint{2.701258in}{2.230186in}}%
\pgfpathlineto{\pgfqpoint{2.760724in}{2.267537in}}%
\pgfpathlineto{\pgfqpoint{2.820190in}{2.011771in}}%
\pgfpathlineto{\pgfqpoint{2.879657in}{1.917640in}}%
\pgfpathlineto{\pgfqpoint{2.939123in}{1.856850in}}%
\pgfpathlineto{\pgfqpoint{2.998589in}{1.856850in}}%
\pgfpathlineto{\pgfqpoint{3.058056in}{1.831622in}}%
\pgfpathlineto{\pgfqpoint{3.117522in}{1.813072in}}%
\pgfpathlineto{\pgfqpoint{3.176988in}{1.758969in}}%
\pgfpathlineto{\pgfqpoint{3.236455in}{1.707120in}}%
\pgfpathlineto{\pgfqpoint{3.295921in}{1.563321in}}%
\pgfpathlineto{\pgfqpoint{3.355387in}{1.546145in}}%
\pgfpathlineto{\pgfqpoint{3.414854in}{1.422477in}}%
\pgfpathlineto{\pgfqpoint{3.474320in}{1.644517in}}%
\pgfpathlineto{\pgfqpoint{3.533786in}{1.757857in}}%
\pgfpathlineto{\pgfqpoint{3.593253in}{1.787832in}}%
\pgfpathlineto{\pgfqpoint{3.652719in}{1.856361in}}%
\pgfpathlineto{\pgfqpoint{3.712185in}{1.806945in}}%
\pgfpathlineto{\pgfqpoint{3.771652in}{1.883711in}}%
\pgfpathlineto{\pgfqpoint{3.831118in}{1.907700in}}%
\pgfpathlineto{\pgfqpoint{3.890584in}{2.032786in}}%
\pgfpathlineto{\pgfqpoint{3.950051in}{2.122043in}}%
\pgfpathlineto{\pgfqpoint{4.009517in}{2.149757in}}%
\pgfpathlineto{\pgfqpoint{4.068983in}{2.162286in}}%
\pgfpathlineto{\pgfqpoint{4.128450in}{2.159998in}}%
\pgfpathlineto{\pgfqpoint{4.187916in}{2.155629in}}%
\pgfpathlineto{\pgfqpoint{4.247382in}{2.155629in}}%
\pgfpathlineto{\pgfqpoint{4.306849in}{2.078603in}}%
\pgfpathlineto{\pgfqpoint{4.366315in}{2.097632in}}%
\pgfpathlineto{\pgfqpoint{4.425781in}{2.027659in}}%
\pgfpathlineto{\pgfqpoint{4.485248in}{1.979295in}}%
\pgfpathlineto{\pgfqpoint{4.544714in}{2.023640in}}%
\pgfpathlineto{\pgfqpoint{4.604180in}{2.024107in}}%
\pgfpathlineto{\pgfqpoint{4.663647in}{1.961287in}}%
\pgfpathlineto{\pgfqpoint{4.723113in}{1.961287in}}%
\pgfpathlineto{\pgfqpoint{4.782579in}{1.840748in}}%
\pgfpathlineto{\pgfqpoint{4.842046in}{1.885915in}}%
\pgfpathlineto{\pgfqpoint{4.901512in}{1.875011in}}%
\pgfpathlineto{\pgfqpoint{4.960978in}{1.952033in}}%
\pgfpathlineto{\pgfqpoint{5.020445in}{1.969561in}}%
\pgfpathlineto{\pgfqpoint{5.079911in}{2.021654in}}%
\pgfpathlineto{\pgfqpoint{5.139377in}{2.070126in}}%
\pgfpathlineto{\pgfqpoint{5.198844in}{1.940728in}}%
\pgfpathlineto{\pgfqpoint{5.258310in}{2.016881in}}%
\pgfpathlineto{\pgfqpoint{5.317776in}{2.039668in}}%
\pgfpathlineto{\pgfqpoint{5.377243in}{1.998977in}}%
\pgfpathlineto{\pgfqpoint{5.436709in}{2.195840in}}%
\pgfpathlineto{\pgfqpoint{5.496175in}{2.192581in}}%
\pgfpathlineto{\pgfqpoint{5.555642in}{2.161199in}}%
\pgfpathlineto{\pgfqpoint{5.615108in}{2.061394in}}%
\pgfusepath{stroke}%
\end{pgfscope}%
\begin{pgfscope}%
\pgfsetrectcap%
\pgfsetmiterjoin%
\pgfsetlinewidth{0.803000pt}%
\definecolor{currentstroke}{rgb}{0.000000,0.000000,0.000000}%
\pgfsetstrokecolor{currentstroke}%
\pgfsetdash{}{0pt}%
\pgfpathmoveto{\pgfqpoint{0.682376in}{0.535823in}}%
\pgfpathlineto{\pgfqpoint{0.682376in}{2.350000in}}%
\pgfusepath{stroke}%
\end{pgfscope}%
\begin{pgfscope}%
\pgfsetrectcap%
\pgfsetmiterjoin%
\pgfsetlinewidth{0.803000pt}%
\definecolor{currentstroke}{rgb}{0.000000,0.000000,0.000000}%
\pgfsetstrokecolor{currentstroke}%
\pgfsetdash{}{0pt}%
\pgfpathmoveto{\pgfqpoint{5.850000in}{0.535823in}}%
\pgfpathlineto{\pgfqpoint{5.850000in}{2.350000in}}%
\pgfusepath{stroke}%
\end{pgfscope}%
\begin{pgfscope}%
\pgfsetrectcap%
\pgfsetmiterjoin%
\pgfsetlinewidth{0.803000pt}%
\definecolor{currentstroke}{rgb}{0.000000,0.000000,0.000000}%
\pgfsetstrokecolor{currentstroke}%
\pgfsetdash{}{0pt}%
\pgfpathmoveto{\pgfqpoint{0.682376in}{0.535823in}}%
\pgfpathlineto{\pgfqpoint{5.850000in}{0.535823in}}%
\pgfusepath{stroke}%
\end{pgfscope}%
\begin{pgfscope}%
\pgfsetrectcap%
\pgfsetmiterjoin%
\pgfsetlinewidth{0.803000pt}%
\definecolor{currentstroke}{rgb}{0.000000,0.000000,0.000000}%
\pgfsetstrokecolor{currentstroke}%
\pgfsetdash{}{0pt}%
\pgfpathmoveto{\pgfqpoint{0.682376in}{2.350000in}}%
\pgfpathlineto{\pgfqpoint{5.850000in}{2.350000in}}%
\pgfusepath{stroke}%
\end{pgfscope}%
\begin{pgfscope}%
\pgfsetrectcap%
\pgfsetroundjoin%
\pgfsetlinewidth{1.003750pt}%
\definecolor{currentstroke}{rgb}{0.121569,0.466667,0.705882}%
\pgfsetstrokecolor{currentstroke}%
\pgfsetdash{}{0pt}%
\pgfpathmoveto{\pgfqpoint{0.732376in}{2.256250in}}%
\pgfpathlineto{\pgfqpoint{0.982376in}{2.256250in}}%
\pgfusepath{stroke}%
\end{pgfscope}%
\begin{pgfscope}%
\definecolor{textcolor}{rgb}{0.000000,0.000000,0.000000}%
\pgfsetstrokecolor{textcolor}%
\pgfsetfillcolor{textcolor}%
\pgftext[x=1.007376in,y=2.212500in,left,base]{\color{textcolor}\rmfamily\fontsize{9.000000}{10.800000}\selectfont ProCount}%
\end{pgfscope}%
\begin{pgfscope}%
\pgfsetrectcap%
\pgfsetroundjoin%
\pgfsetlinewidth{1.003750pt}%
\definecolor{currentstroke}{rgb}{1.000000,0.498039,0.054902}%
\pgfsetstrokecolor{currentstroke}%
\pgfsetdash{}{0pt}%
\pgfpathmoveto{\pgfqpoint{0.732376in}{2.094450in}}%
\pgfpathlineto{\pgfqpoint{0.982376in}{2.094450in}}%
\pgfusepath{stroke}%
\end{pgfscope}%
\begin{pgfscope}%
\definecolor{textcolor}{rgb}{0.000000,0.000000,0.000000}%
\pgfsetstrokecolor{textcolor}%
\pgfsetfillcolor{textcolor}%
\pgftext[x=1.007376in,y=2.050700in,left,base]{\color{textcolor}\rmfamily\fontsize{9.000000}{10.800000}\selectfont D4\textsubscript{P}}%
\end{pgfscope}%
\begin{pgfscope}%
\pgfsetrectcap%
\pgfsetroundjoin%
\pgfsetlinewidth{1.003750pt}%
\definecolor{currentstroke}{rgb}{0.172549,0.627451,0.172549}%
\pgfsetstrokecolor{currentstroke}%
\pgfsetdash{}{0pt}%
\pgfpathmoveto{\pgfqpoint{0.732376in}{1.932651in}}%
\pgfpathlineto{\pgfqpoint{0.982376in}{1.932651in}}%
\pgfusepath{stroke}%
\end{pgfscope}%
\begin{pgfscope}%
\definecolor{textcolor}{rgb}{0.000000,0.000000,0.000000}%
\pgfsetstrokecolor{textcolor}%
\pgfsetfillcolor{textcolor}%
\pgftext[x=1.007376in,y=1.888901in,left,base]{\color{textcolor}\rmfamily\fontsize{9.000000}{10.800000}\selectfont projMC}%
\end{pgfscope}%
\begin{pgfscope}%
\pgfsetrectcap%
\pgfsetroundjoin%
\pgfsetlinewidth{1.003750pt}%
\definecolor{currentstroke}{rgb}{0.839216,0.152941,0.156863}%
\pgfsetstrokecolor{currentstroke}%
\pgfsetdash{}{0pt}%
\pgfpathmoveto{\pgfqpoint{0.732376in}{1.770851in}}%
\pgfpathlineto{\pgfqpoint{0.982376in}{1.770851in}}%
\pgfusepath{stroke}%
\end{pgfscope}%
\begin{pgfscope}%
\definecolor{textcolor}{rgb}{0.000000,0.000000,0.000000}%
\pgfsetstrokecolor{textcolor}%
\pgfsetfillcolor{textcolor}%
\pgftext[x=1.007376in,y=1.727101in,left,base]{\color{textcolor}\rmfamily\fontsize{9.000000}{10.800000}\selectfont reSSAT}%
\end{pgfscope}%
\end{pgfpicture}%
\makeatother%
\endgroup%

    \caption{
        A plot of mean PAR-2 scores (in seconds) against mean project-join tree widths.
        On this plot, each projected model counter (\procount{}, \dfp{}, \projmc, or \ssat) is represented by a curve, on which a point $(x, y)$ indicates that $x$ is the central moving average of 10 consecutive project-join tree widths ($1 \le w_1 < w_2 < \ldots < w_{10} \le 99$) and $y$ is the average PAR-2 score of the benchmarks whose project-join trees have widths $w$ where $w_1 \le w \le w_{10}$.
        We observe that the performance of \procount{} degrades as the project-join tree width increases.
        However, \procount{} tends to be the fastest solver on benchmarks whose graded project-join trees have widths roughly between 50 and 100.
    }
    \label{time_vs_width}
\end{figure}
