





\section{Additional Experimental Evaluation}
\label{appendix:exp}
% Our \procount{} source code and experimental data are available in a public repository (\url{https://github.com/vardigroup/DPMC/tree/v2.0.0/experiments}).

%%%%%%%%%%%%%%%%%%%%%%%%%%%%%%%%%%%%%%%%%%%%%%%%%%%%%%%%%%%%%%%%%%%%%%%%%%%%%%%%

\subsection{Experiment 1: Comparing Planners}

Figure \ref{figPlanningA} illustrates how the planner \Lg{} compares to the planner \htb{} across several settings.
\Lg{} is an \emph{anytime} tool, which outputs better and better project-join trees the longer it runs.
In Experiment 1, within 100 seconds, each \Lg{} setting may produce several project-join trees (of decreasing widths) for a single benchmark.
Figure \ref{figPlanningA} plots the time of the first such project-join tree of width at most \maxWidth.
\htb{} is a \emph{one-shot} tool, which outputs only one project-join tree for each benchmark.
\begin{figure}[H]
    \centering
    %% Creator: Matplotlib, PGF backend
%%
%% To include the figure in your LaTeX document, write
%%   \input{<filename>.pgf}
%%
%% Make sure the required packages are loaded in your preamble
%%   \usepackage{pgf}
%%
%% and, on pdftex
%%   \usepackage[utf8]{inputenc}\DeclareUnicodeCharacter{2212}{-}
%%
%% or, on luatex and xetex
%%   \usepackage{unicode-math}
%%
%% Figures using additional raster images can only be included by \input if
%% they are in the same directory as the main LaTeX file. For loading figures
%% from other directories you can use the `import` package
%%   \usepackage{import}
%%
%% and then include the figures with
%%   \import{<path to file>}{<filename>.pgf}
%%
%% Matplotlib used the following preamble
%%   \usepackage{fontspec}
%%   \setmainfont{DejaVuSerif.ttf}[Path=/home/vhp1/.local/lib/python3.8/site-packages/matplotlib/mpl-data/fonts/ttf/]
%%   \setsansfont{DejaVuSans.ttf}[Path=/home/vhp1/.local/lib/python3.8/site-packages/matplotlib/mpl-data/fonts/ttf/]
%%   \setmonofont{DejaVuSansMono.ttf}[Path=/home/vhp1/.local/lib/python3.8/site-packages/matplotlib/mpl-data/fonts/ttf/]
%%
\begingroup%
\makeatletter%
\begin{pgfpicture}%
\pgfpathrectangle{\pgfpointorigin}{\pgfqpoint{4.820041in}{3.402358in}}%
\pgfusepath{use as bounding box, clip}%
\begin{pgfscope}%
\pgfsetbuttcap%
\pgfsetmiterjoin%
\pgfsetlinewidth{0.000000pt}%
\definecolor{currentstroke}{rgb}{1.000000,1.000000,1.000000}%
\pgfsetstrokecolor{currentstroke}%
\pgfsetstrokeopacity{0.000000}%
\pgfsetdash{}{0pt}%
\pgfpathmoveto{\pgfqpoint{0.000000in}{0.000000in}}%
\pgfpathlineto{\pgfqpoint{4.820041in}{0.000000in}}%
\pgfpathlineto{\pgfqpoint{4.820041in}{3.402358in}}%
\pgfpathlineto{\pgfqpoint{0.000000in}{3.402358in}}%
\pgfpathclose%
\pgfusepath{}%
\end{pgfscope}%
\begin{pgfscope}%
\pgfsetbuttcap%
\pgfsetmiterjoin%
\definecolor{currentfill}{rgb}{1.000000,1.000000,1.000000}%
\pgfsetfillcolor{currentfill}%
\pgfsetlinewidth{0.000000pt}%
\definecolor{currentstroke}{rgb}{0.000000,0.000000,0.000000}%
\pgfsetstrokecolor{currentstroke}%
\pgfsetstrokeopacity{0.000000}%
\pgfsetdash{}{0pt}%
\pgfpathmoveto{\pgfqpoint{0.537394in}{0.467838in}}%
\pgfpathlineto{\pgfqpoint{4.632078in}{0.467838in}}%
\pgfpathlineto{\pgfqpoint{4.632078in}{3.260149in}}%
\pgfpathlineto{\pgfqpoint{0.537394in}{3.260149in}}%
\pgfpathclose%
\pgfusepath{fill}%
\end{pgfscope}%
\begin{pgfscope}%
\pgfsetbuttcap%
\pgfsetroundjoin%
\definecolor{currentfill}{rgb}{0.000000,0.000000,0.000000}%
\pgfsetfillcolor{currentfill}%
\pgfsetlinewidth{0.803000pt}%
\definecolor{currentstroke}{rgb}{0.000000,0.000000,0.000000}%
\pgfsetstrokecolor{currentstroke}%
\pgfsetdash{}{0pt}%
\pgfsys@defobject{currentmarker}{\pgfqpoint{0.000000in}{-0.048611in}}{\pgfqpoint{0.000000in}{0.000000in}}{%
\pgfpathmoveto{\pgfqpoint{0.000000in}{0.000000in}}%
\pgfpathlineto{\pgfqpoint{0.000000in}{-0.048611in}}%
\pgfusepath{stroke,fill}%
}%
\begin{pgfscope}%
\pgfsys@transformshift{0.537394in}{0.467838in}%
\pgfsys@useobject{currentmarker}{}%
\end{pgfscope}%
\end{pgfscope}%
\begin{pgfscope}%
\definecolor{textcolor}{rgb}{0.000000,0.000000,0.000000}%
\pgfsetstrokecolor{textcolor}%
\pgfsetfillcolor{textcolor}%
\pgftext[x=0.537394in,y=0.370616in,,top]{\color{textcolor}\sffamily\fontsize{8.000000}{9.600000}\selectfont \(\displaystyle {10^{-3}}\)}%
\end{pgfscope}%
\begin{pgfscope}%
\pgfsetbuttcap%
\pgfsetroundjoin%
\definecolor{currentfill}{rgb}{0.000000,0.000000,0.000000}%
\pgfsetfillcolor{currentfill}%
\pgfsetlinewidth{0.803000pt}%
\definecolor{currentstroke}{rgb}{0.000000,0.000000,0.000000}%
\pgfsetstrokecolor{currentstroke}%
\pgfsetdash{}{0pt}%
\pgfsys@defobject{currentmarker}{\pgfqpoint{0.000000in}{-0.048611in}}{\pgfqpoint{0.000000in}{0.000000in}}{%
\pgfpathmoveto{\pgfqpoint{0.000000in}{0.000000in}}%
\pgfpathlineto{\pgfqpoint{0.000000in}{-0.048611in}}%
\pgfusepath{stroke,fill}%
}%
\begin{pgfscope}%
\pgfsys@transformshift{1.356331in}{0.467838in}%
\pgfsys@useobject{currentmarker}{}%
\end{pgfscope}%
\end{pgfscope}%
\begin{pgfscope}%
\definecolor{textcolor}{rgb}{0.000000,0.000000,0.000000}%
\pgfsetstrokecolor{textcolor}%
\pgfsetfillcolor{textcolor}%
\pgftext[x=1.356331in,y=0.370616in,,top]{\color{textcolor}\sffamily\fontsize{8.000000}{9.600000}\selectfont \(\displaystyle {10^{-2}}\)}%
\end{pgfscope}%
\begin{pgfscope}%
\pgfsetbuttcap%
\pgfsetroundjoin%
\definecolor{currentfill}{rgb}{0.000000,0.000000,0.000000}%
\pgfsetfillcolor{currentfill}%
\pgfsetlinewidth{0.803000pt}%
\definecolor{currentstroke}{rgb}{0.000000,0.000000,0.000000}%
\pgfsetstrokecolor{currentstroke}%
\pgfsetdash{}{0pt}%
\pgfsys@defobject{currentmarker}{\pgfqpoint{0.000000in}{-0.048611in}}{\pgfqpoint{0.000000in}{0.000000in}}{%
\pgfpathmoveto{\pgfqpoint{0.000000in}{0.000000in}}%
\pgfpathlineto{\pgfqpoint{0.000000in}{-0.048611in}}%
\pgfusepath{stroke,fill}%
}%
\begin{pgfscope}%
\pgfsys@transformshift{2.175268in}{0.467838in}%
\pgfsys@useobject{currentmarker}{}%
\end{pgfscope}%
\end{pgfscope}%
\begin{pgfscope}%
\definecolor{textcolor}{rgb}{0.000000,0.000000,0.000000}%
\pgfsetstrokecolor{textcolor}%
\pgfsetfillcolor{textcolor}%
\pgftext[x=2.175268in,y=0.370616in,,top]{\color{textcolor}\sffamily\fontsize{8.000000}{9.600000}\selectfont \(\displaystyle {10^{-1}}\)}%
\end{pgfscope}%
\begin{pgfscope}%
\pgfsetbuttcap%
\pgfsetroundjoin%
\definecolor{currentfill}{rgb}{0.000000,0.000000,0.000000}%
\pgfsetfillcolor{currentfill}%
\pgfsetlinewidth{0.803000pt}%
\definecolor{currentstroke}{rgb}{0.000000,0.000000,0.000000}%
\pgfsetstrokecolor{currentstroke}%
\pgfsetdash{}{0pt}%
\pgfsys@defobject{currentmarker}{\pgfqpoint{0.000000in}{-0.048611in}}{\pgfqpoint{0.000000in}{0.000000in}}{%
\pgfpathmoveto{\pgfqpoint{0.000000in}{0.000000in}}%
\pgfpathlineto{\pgfqpoint{0.000000in}{-0.048611in}}%
\pgfusepath{stroke,fill}%
}%
\begin{pgfscope}%
\pgfsys@transformshift{2.994204in}{0.467838in}%
\pgfsys@useobject{currentmarker}{}%
\end{pgfscope}%
\end{pgfscope}%
\begin{pgfscope}%
\definecolor{textcolor}{rgb}{0.000000,0.000000,0.000000}%
\pgfsetstrokecolor{textcolor}%
\pgfsetfillcolor{textcolor}%
\pgftext[x=2.994204in,y=0.370616in,,top]{\color{textcolor}\sffamily\fontsize{8.000000}{9.600000}\selectfont \(\displaystyle {10^{0}}\)}%
\end{pgfscope}%
\begin{pgfscope}%
\pgfsetbuttcap%
\pgfsetroundjoin%
\definecolor{currentfill}{rgb}{0.000000,0.000000,0.000000}%
\pgfsetfillcolor{currentfill}%
\pgfsetlinewidth{0.803000pt}%
\definecolor{currentstroke}{rgb}{0.000000,0.000000,0.000000}%
\pgfsetstrokecolor{currentstroke}%
\pgfsetdash{}{0pt}%
\pgfsys@defobject{currentmarker}{\pgfqpoint{0.000000in}{-0.048611in}}{\pgfqpoint{0.000000in}{0.000000in}}{%
\pgfpathmoveto{\pgfqpoint{0.000000in}{0.000000in}}%
\pgfpathlineto{\pgfqpoint{0.000000in}{-0.048611in}}%
\pgfusepath{stroke,fill}%
}%
\begin{pgfscope}%
\pgfsys@transformshift{3.813141in}{0.467838in}%
\pgfsys@useobject{currentmarker}{}%
\end{pgfscope}%
\end{pgfscope}%
\begin{pgfscope}%
\definecolor{textcolor}{rgb}{0.000000,0.000000,0.000000}%
\pgfsetstrokecolor{textcolor}%
\pgfsetfillcolor{textcolor}%
\pgftext[x=3.813141in,y=0.370616in,,top]{\color{textcolor}\sffamily\fontsize{8.000000}{9.600000}\selectfont \(\displaystyle {10^{1}}\)}%
\end{pgfscope}%
\begin{pgfscope}%
\pgfsetbuttcap%
\pgfsetroundjoin%
\definecolor{currentfill}{rgb}{0.000000,0.000000,0.000000}%
\pgfsetfillcolor{currentfill}%
\pgfsetlinewidth{0.803000pt}%
\definecolor{currentstroke}{rgb}{0.000000,0.000000,0.000000}%
\pgfsetstrokecolor{currentstroke}%
\pgfsetdash{}{0pt}%
\pgfsys@defobject{currentmarker}{\pgfqpoint{0.000000in}{-0.048611in}}{\pgfqpoint{0.000000in}{0.000000in}}{%
\pgfpathmoveto{\pgfqpoint{0.000000in}{0.000000in}}%
\pgfpathlineto{\pgfqpoint{0.000000in}{-0.048611in}}%
\pgfusepath{stroke,fill}%
}%
\begin{pgfscope}%
\pgfsys@transformshift{4.632078in}{0.467838in}%
\pgfsys@useobject{currentmarker}{}%
\end{pgfscope}%
\end{pgfscope}%
\begin{pgfscope}%
\definecolor{textcolor}{rgb}{0.000000,0.000000,0.000000}%
\pgfsetstrokecolor{textcolor}%
\pgfsetfillcolor{textcolor}%
\pgftext[x=4.632078in,y=0.370616in,,top]{\color{textcolor}\sffamily\fontsize{8.000000}{9.600000}\selectfont \(\displaystyle {10^{2}}\)}%
\end{pgfscope}%
\begin{pgfscope}%
\pgfsetbuttcap%
\pgfsetroundjoin%
\definecolor{currentfill}{rgb}{0.000000,0.000000,0.000000}%
\pgfsetfillcolor{currentfill}%
\pgfsetlinewidth{0.602250pt}%
\definecolor{currentstroke}{rgb}{0.000000,0.000000,0.000000}%
\pgfsetstrokecolor{currentstroke}%
\pgfsetdash{}{0pt}%
\pgfsys@defobject{currentmarker}{\pgfqpoint{0.000000in}{-0.027778in}}{\pgfqpoint{0.000000in}{0.000000in}}{%
\pgfpathmoveto{\pgfqpoint{0.000000in}{0.000000in}}%
\pgfpathlineto{\pgfqpoint{0.000000in}{-0.027778in}}%
\pgfusepath{stroke,fill}%
}%
\begin{pgfscope}%
\pgfsys@transformshift{0.783918in}{0.467838in}%
\pgfsys@useobject{currentmarker}{}%
\end{pgfscope}%
\end{pgfscope}%
\begin{pgfscope}%
\pgfsetbuttcap%
\pgfsetroundjoin%
\definecolor{currentfill}{rgb}{0.000000,0.000000,0.000000}%
\pgfsetfillcolor{currentfill}%
\pgfsetlinewidth{0.602250pt}%
\definecolor{currentstroke}{rgb}{0.000000,0.000000,0.000000}%
\pgfsetstrokecolor{currentstroke}%
\pgfsetdash{}{0pt}%
\pgfsys@defobject{currentmarker}{\pgfqpoint{0.000000in}{-0.027778in}}{\pgfqpoint{0.000000in}{0.000000in}}{%
\pgfpathmoveto{\pgfqpoint{0.000000in}{0.000000in}}%
\pgfpathlineto{\pgfqpoint{0.000000in}{-0.027778in}}%
\pgfusepath{stroke,fill}%
}%
\begin{pgfscope}%
\pgfsys@transformshift{0.928126in}{0.467838in}%
\pgfsys@useobject{currentmarker}{}%
\end{pgfscope}%
\end{pgfscope}%
\begin{pgfscope}%
\pgfsetbuttcap%
\pgfsetroundjoin%
\definecolor{currentfill}{rgb}{0.000000,0.000000,0.000000}%
\pgfsetfillcolor{currentfill}%
\pgfsetlinewidth{0.602250pt}%
\definecolor{currentstroke}{rgb}{0.000000,0.000000,0.000000}%
\pgfsetstrokecolor{currentstroke}%
\pgfsetdash{}{0pt}%
\pgfsys@defobject{currentmarker}{\pgfqpoint{0.000000in}{-0.027778in}}{\pgfqpoint{0.000000in}{0.000000in}}{%
\pgfpathmoveto{\pgfqpoint{0.000000in}{0.000000in}}%
\pgfpathlineto{\pgfqpoint{0.000000in}{-0.027778in}}%
\pgfusepath{stroke,fill}%
}%
\begin{pgfscope}%
\pgfsys@transformshift{1.030443in}{0.467838in}%
\pgfsys@useobject{currentmarker}{}%
\end{pgfscope}%
\end{pgfscope}%
\begin{pgfscope}%
\pgfsetbuttcap%
\pgfsetroundjoin%
\definecolor{currentfill}{rgb}{0.000000,0.000000,0.000000}%
\pgfsetfillcolor{currentfill}%
\pgfsetlinewidth{0.602250pt}%
\definecolor{currentstroke}{rgb}{0.000000,0.000000,0.000000}%
\pgfsetstrokecolor{currentstroke}%
\pgfsetdash{}{0pt}%
\pgfsys@defobject{currentmarker}{\pgfqpoint{0.000000in}{-0.027778in}}{\pgfqpoint{0.000000in}{0.000000in}}{%
\pgfpathmoveto{\pgfqpoint{0.000000in}{0.000000in}}%
\pgfpathlineto{\pgfqpoint{0.000000in}{-0.027778in}}%
\pgfusepath{stroke,fill}%
}%
\begin{pgfscope}%
\pgfsys@transformshift{1.109806in}{0.467838in}%
\pgfsys@useobject{currentmarker}{}%
\end{pgfscope}%
\end{pgfscope}%
\begin{pgfscope}%
\pgfsetbuttcap%
\pgfsetroundjoin%
\definecolor{currentfill}{rgb}{0.000000,0.000000,0.000000}%
\pgfsetfillcolor{currentfill}%
\pgfsetlinewidth{0.602250pt}%
\definecolor{currentstroke}{rgb}{0.000000,0.000000,0.000000}%
\pgfsetstrokecolor{currentstroke}%
\pgfsetdash{}{0pt}%
\pgfsys@defobject{currentmarker}{\pgfqpoint{0.000000in}{-0.027778in}}{\pgfqpoint{0.000000in}{0.000000in}}{%
\pgfpathmoveto{\pgfqpoint{0.000000in}{0.000000in}}%
\pgfpathlineto{\pgfqpoint{0.000000in}{-0.027778in}}%
\pgfusepath{stroke,fill}%
}%
\begin{pgfscope}%
\pgfsys@transformshift{1.174651in}{0.467838in}%
\pgfsys@useobject{currentmarker}{}%
\end{pgfscope}%
\end{pgfscope}%
\begin{pgfscope}%
\pgfsetbuttcap%
\pgfsetroundjoin%
\definecolor{currentfill}{rgb}{0.000000,0.000000,0.000000}%
\pgfsetfillcolor{currentfill}%
\pgfsetlinewidth{0.602250pt}%
\definecolor{currentstroke}{rgb}{0.000000,0.000000,0.000000}%
\pgfsetstrokecolor{currentstroke}%
\pgfsetdash{}{0pt}%
\pgfsys@defobject{currentmarker}{\pgfqpoint{0.000000in}{-0.027778in}}{\pgfqpoint{0.000000in}{0.000000in}}{%
\pgfpathmoveto{\pgfqpoint{0.000000in}{0.000000in}}%
\pgfpathlineto{\pgfqpoint{0.000000in}{-0.027778in}}%
\pgfusepath{stroke,fill}%
}%
\begin{pgfscope}%
\pgfsys@transformshift{1.229476in}{0.467838in}%
\pgfsys@useobject{currentmarker}{}%
\end{pgfscope}%
\end{pgfscope}%
\begin{pgfscope}%
\pgfsetbuttcap%
\pgfsetroundjoin%
\definecolor{currentfill}{rgb}{0.000000,0.000000,0.000000}%
\pgfsetfillcolor{currentfill}%
\pgfsetlinewidth{0.602250pt}%
\definecolor{currentstroke}{rgb}{0.000000,0.000000,0.000000}%
\pgfsetstrokecolor{currentstroke}%
\pgfsetdash{}{0pt}%
\pgfsys@defobject{currentmarker}{\pgfqpoint{0.000000in}{-0.027778in}}{\pgfqpoint{0.000000in}{0.000000in}}{%
\pgfpathmoveto{\pgfqpoint{0.000000in}{0.000000in}}%
\pgfpathlineto{\pgfqpoint{0.000000in}{-0.027778in}}%
\pgfusepath{stroke,fill}%
}%
\begin{pgfscope}%
\pgfsys@transformshift{1.276968in}{0.467838in}%
\pgfsys@useobject{currentmarker}{}%
\end{pgfscope}%
\end{pgfscope}%
\begin{pgfscope}%
\pgfsetbuttcap%
\pgfsetroundjoin%
\definecolor{currentfill}{rgb}{0.000000,0.000000,0.000000}%
\pgfsetfillcolor{currentfill}%
\pgfsetlinewidth{0.602250pt}%
\definecolor{currentstroke}{rgb}{0.000000,0.000000,0.000000}%
\pgfsetstrokecolor{currentstroke}%
\pgfsetdash{}{0pt}%
\pgfsys@defobject{currentmarker}{\pgfqpoint{0.000000in}{-0.027778in}}{\pgfqpoint{0.000000in}{0.000000in}}{%
\pgfpathmoveto{\pgfqpoint{0.000000in}{0.000000in}}%
\pgfpathlineto{\pgfqpoint{0.000000in}{-0.027778in}}%
\pgfusepath{stroke,fill}%
}%
\begin{pgfscope}%
\pgfsys@transformshift{1.318858in}{0.467838in}%
\pgfsys@useobject{currentmarker}{}%
\end{pgfscope}%
\end{pgfscope}%
\begin{pgfscope}%
\pgfsetbuttcap%
\pgfsetroundjoin%
\definecolor{currentfill}{rgb}{0.000000,0.000000,0.000000}%
\pgfsetfillcolor{currentfill}%
\pgfsetlinewidth{0.602250pt}%
\definecolor{currentstroke}{rgb}{0.000000,0.000000,0.000000}%
\pgfsetstrokecolor{currentstroke}%
\pgfsetdash{}{0pt}%
\pgfsys@defobject{currentmarker}{\pgfqpoint{0.000000in}{-0.027778in}}{\pgfqpoint{0.000000in}{0.000000in}}{%
\pgfpathmoveto{\pgfqpoint{0.000000in}{0.000000in}}%
\pgfpathlineto{\pgfqpoint{0.000000in}{-0.027778in}}%
\pgfusepath{stroke,fill}%
}%
\begin{pgfscope}%
\pgfsys@transformshift{1.602855in}{0.467838in}%
\pgfsys@useobject{currentmarker}{}%
\end{pgfscope}%
\end{pgfscope}%
\begin{pgfscope}%
\pgfsetbuttcap%
\pgfsetroundjoin%
\definecolor{currentfill}{rgb}{0.000000,0.000000,0.000000}%
\pgfsetfillcolor{currentfill}%
\pgfsetlinewidth{0.602250pt}%
\definecolor{currentstroke}{rgb}{0.000000,0.000000,0.000000}%
\pgfsetstrokecolor{currentstroke}%
\pgfsetdash{}{0pt}%
\pgfsys@defobject{currentmarker}{\pgfqpoint{0.000000in}{-0.027778in}}{\pgfqpoint{0.000000in}{0.000000in}}{%
\pgfpathmoveto{\pgfqpoint{0.000000in}{0.000000in}}%
\pgfpathlineto{\pgfqpoint{0.000000in}{-0.027778in}}%
\pgfusepath{stroke,fill}%
}%
\begin{pgfscope}%
\pgfsys@transformshift{1.747063in}{0.467838in}%
\pgfsys@useobject{currentmarker}{}%
\end{pgfscope}%
\end{pgfscope}%
\begin{pgfscope}%
\pgfsetbuttcap%
\pgfsetroundjoin%
\definecolor{currentfill}{rgb}{0.000000,0.000000,0.000000}%
\pgfsetfillcolor{currentfill}%
\pgfsetlinewidth{0.602250pt}%
\definecolor{currentstroke}{rgb}{0.000000,0.000000,0.000000}%
\pgfsetstrokecolor{currentstroke}%
\pgfsetdash{}{0pt}%
\pgfsys@defobject{currentmarker}{\pgfqpoint{0.000000in}{-0.027778in}}{\pgfqpoint{0.000000in}{0.000000in}}{%
\pgfpathmoveto{\pgfqpoint{0.000000in}{0.000000in}}%
\pgfpathlineto{\pgfqpoint{0.000000in}{-0.027778in}}%
\pgfusepath{stroke,fill}%
}%
\begin{pgfscope}%
\pgfsys@transformshift{1.849380in}{0.467838in}%
\pgfsys@useobject{currentmarker}{}%
\end{pgfscope}%
\end{pgfscope}%
\begin{pgfscope}%
\pgfsetbuttcap%
\pgfsetroundjoin%
\definecolor{currentfill}{rgb}{0.000000,0.000000,0.000000}%
\pgfsetfillcolor{currentfill}%
\pgfsetlinewidth{0.602250pt}%
\definecolor{currentstroke}{rgb}{0.000000,0.000000,0.000000}%
\pgfsetstrokecolor{currentstroke}%
\pgfsetdash{}{0pt}%
\pgfsys@defobject{currentmarker}{\pgfqpoint{0.000000in}{-0.027778in}}{\pgfqpoint{0.000000in}{0.000000in}}{%
\pgfpathmoveto{\pgfqpoint{0.000000in}{0.000000in}}%
\pgfpathlineto{\pgfqpoint{0.000000in}{-0.027778in}}%
\pgfusepath{stroke,fill}%
}%
\begin{pgfscope}%
\pgfsys@transformshift{1.928743in}{0.467838in}%
\pgfsys@useobject{currentmarker}{}%
\end{pgfscope}%
\end{pgfscope}%
\begin{pgfscope}%
\pgfsetbuttcap%
\pgfsetroundjoin%
\definecolor{currentfill}{rgb}{0.000000,0.000000,0.000000}%
\pgfsetfillcolor{currentfill}%
\pgfsetlinewidth{0.602250pt}%
\definecolor{currentstroke}{rgb}{0.000000,0.000000,0.000000}%
\pgfsetstrokecolor{currentstroke}%
\pgfsetdash{}{0pt}%
\pgfsys@defobject{currentmarker}{\pgfqpoint{0.000000in}{-0.027778in}}{\pgfqpoint{0.000000in}{0.000000in}}{%
\pgfpathmoveto{\pgfqpoint{0.000000in}{0.000000in}}%
\pgfpathlineto{\pgfqpoint{0.000000in}{-0.027778in}}%
\pgfusepath{stroke,fill}%
}%
\begin{pgfscope}%
\pgfsys@transformshift{1.993587in}{0.467838in}%
\pgfsys@useobject{currentmarker}{}%
\end{pgfscope}%
\end{pgfscope}%
\begin{pgfscope}%
\pgfsetbuttcap%
\pgfsetroundjoin%
\definecolor{currentfill}{rgb}{0.000000,0.000000,0.000000}%
\pgfsetfillcolor{currentfill}%
\pgfsetlinewidth{0.602250pt}%
\definecolor{currentstroke}{rgb}{0.000000,0.000000,0.000000}%
\pgfsetstrokecolor{currentstroke}%
\pgfsetdash{}{0pt}%
\pgfsys@defobject{currentmarker}{\pgfqpoint{0.000000in}{-0.027778in}}{\pgfqpoint{0.000000in}{0.000000in}}{%
\pgfpathmoveto{\pgfqpoint{0.000000in}{0.000000in}}%
\pgfpathlineto{\pgfqpoint{0.000000in}{-0.027778in}}%
\pgfusepath{stroke,fill}%
}%
\begin{pgfscope}%
\pgfsys@transformshift{2.048413in}{0.467838in}%
\pgfsys@useobject{currentmarker}{}%
\end{pgfscope}%
\end{pgfscope}%
\begin{pgfscope}%
\pgfsetbuttcap%
\pgfsetroundjoin%
\definecolor{currentfill}{rgb}{0.000000,0.000000,0.000000}%
\pgfsetfillcolor{currentfill}%
\pgfsetlinewidth{0.602250pt}%
\definecolor{currentstroke}{rgb}{0.000000,0.000000,0.000000}%
\pgfsetstrokecolor{currentstroke}%
\pgfsetdash{}{0pt}%
\pgfsys@defobject{currentmarker}{\pgfqpoint{0.000000in}{-0.027778in}}{\pgfqpoint{0.000000in}{0.000000in}}{%
\pgfpathmoveto{\pgfqpoint{0.000000in}{0.000000in}}%
\pgfpathlineto{\pgfqpoint{0.000000in}{-0.027778in}}%
\pgfusepath{stroke,fill}%
}%
\begin{pgfscope}%
\pgfsys@transformshift{2.095904in}{0.467838in}%
\pgfsys@useobject{currentmarker}{}%
\end{pgfscope}%
\end{pgfscope}%
\begin{pgfscope}%
\pgfsetbuttcap%
\pgfsetroundjoin%
\definecolor{currentfill}{rgb}{0.000000,0.000000,0.000000}%
\pgfsetfillcolor{currentfill}%
\pgfsetlinewidth{0.602250pt}%
\definecolor{currentstroke}{rgb}{0.000000,0.000000,0.000000}%
\pgfsetstrokecolor{currentstroke}%
\pgfsetdash{}{0pt}%
\pgfsys@defobject{currentmarker}{\pgfqpoint{0.000000in}{-0.027778in}}{\pgfqpoint{0.000000in}{0.000000in}}{%
\pgfpathmoveto{\pgfqpoint{0.000000in}{0.000000in}}%
\pgfpathlineto{\pgfqpoint{0.000000in}{-0.027778in}}%
\pgfusepath{stroke,fill}%
}%
\begin{pgfscope}%
\pgfsys@transformshift{2.137795in}{0.467838in}%
\pgfsys@useobject{currentmarker}{}%
\end{pgfscope}%
\end{pgfscope}%
\begin{pgfscope}%
\pgfsetbuttcap%
\pgfsetroundjoin%
\definecolor{currentfill}{rgb}{0.000000,0.000000,0.000000}%
\pgfsetfillcolor{currentfill}%
\pgfsetlinewidth{0.602250pt}%
\definecolor{currentstroke}{rgb}{0.000000,0.000000,0.000000}%
\pgfsetstrokecolor{currentstroke}%
\pgfsetdash{}{0pt}%
\pgfsys@defobject{currentmarker}{\pgfqpoint{0.000000in}{-0.027778in}}{\pgfqpoint{0.000000in}{0.000000in}}{%
\pgfpathmoveto{\pgfqpoint{0.000000in}{0.000000in}}%
\pgfpathlineto{\pgfqpoint{0.000000in}{-0.027778in}}%
\pgfusepath{stroke,fill}%
}%
\begin{pgfscope}%
\pgfsys@transformshift{2.421792in}{0.467838in}%
\pgfsys@useobject{currentmarker}{}%
\end{pgfscope}%
\end{pgfscope}%
\begin{pgfscope}%
\pgfsetbuttcap%
\pgfsetroundjoin%
\definecolor{currentfill}{rgb}{0.000000,0.000000,0.000000}%
\pgfsetfillcolor{currentfill}%
\pgfsetlinewidth{0.602250pt}%
\definecolor{currentstroke}{rgb}{0.000000,0.000000,0.000000}%
\pgfsetstrokecolor{currentstroke}%
\pgfsetdash{}{0pt}%
\pgfsys@defobject{currentmarker}{\pgfqpoint{0.000000in}{-0.027778in}}{\pgfqpoint{0.000000in}{0.000000in}}{%
\pgfpathmoveto{\pgfqpoint{0.000000in}{0.000000in}}%
\pgfpathlineto{\pgfqpoint{0.000000in}{-0.027778in}}%
\pgfusepath{stroke,fill}%
}%
\begin{pgfscope}%
\pgfsys@transformshift{2.566000in}{0.467838in}%
\pgfsys@useobject{currentmarker}{}%
\end{pgfscope}%
\end{pgfscope}%
\begin{pgfscope}%
\pgfsetbuttcap%
\pgfsetroundjoin%
\definecolor{currentfill}{rgb}{0.000000,0.000000,0.000000}%
\pgfsetfillcolor{currentfill}%
\pgfsetlinewidth{0.602250pt}%
\definecolor{currentstroke}{rgb}{0.000000,0.000000,0.000000}%
\pgfsetstrokecolor{currentstroke}%
\pgfsetdash{}{0pt}%
\pgfsys@defobject{currentmarker}{\pgfqpoint{0.000000in}{-0.027778in}}{\pgfqpoint{0.000000in}{0.000000in}}{%
\pgfpathmoveto{\pgfqpoint{0.000000in}{0.000000in}}%
\pgfpathlineto{\pgfqpoint{0.000000in}{-0.027778in}}%
\pgfusepath{stroke,fill}%
}%
\begin{pgfscope}%
\pgfsys@transformshift{2.668317in}{0.467838in}%
\pgfsys@useobject{currentmarker}{}%
\end{pgfscope}%
\end{pgfscope}%
\begin{pgfscope}%
\pgfsetbuttcap%
\pgfsetroundjoin%
\definecolor{currentfill}{rgb}{0.000000,0.000000,0.000000}%
\pgfsetfillcolor{currentfill}%
\pgfsetlinewidth{0.602250pt}%
\definecolor{currentstroke}{rgb}{0.000000,0.000000,0.000000}%
\pgfsetstrokecolor{currentstroke}%
\pgfsetdash{}{0pt}%
\pgfsys@defobject{currentmarker}{\pgfqpoint{0.000000in}{-0.027778in}}{\pgfqpoint{0.000000in}{0.000000in}}{%
\pgfpathmoveto{\pgfqpoint{0.000000in}{0.000000in}}%
\pgfpathlineto{\pgfqpoint{0.000000in}{-0.027778in}}%
\pgfusepath{stroke,fill}%
}%
\begin{pgfscope}%
\pgfsys@transformshift{2.747680in}{0.467838in}%
\pgfsys@useobject{currentmarker}{}%
\end{pgfscope}%
\end{pgfscope}%
\begin{pgfscope}%
\pgfsetbuttcap%
\pgfsetroundjoin%
\definecolor{currentfill}{rgb}{0.000000,0.000000,0.000000}%
\pgfsetfillcolor{currentfill}%
\pgfsetlinewidth{0.602250pt}%
\definecolor{currentstroke}{rgb}{0.000000,0.000000,0.000000}%
\pgfsetstrokecolor{currentstroke}%
\pgfsetdash{}{0pt}%
\pgfsys@defobject{currentmarker}{\pgfqpoint{0.000000in}{-0.027778in}}{\pgfqpoint{0.000000in}{0.000000in}}{%
\pgfpathmoveto{\pgfqpoint{0.000000in}{0.000000in}}%
\pgfpathlineto{\pgfqpoint{0.000000in}{-0.027778in}}%
\pgfusepath{stroke,fill}%
}%
\begin{pgfscope}%
\pgfsys@transformshift{2.812524in}{0.467838in}%
\pgfsys@useobject{currentmarker}{}%
\end{pgfscope}%
\end{pgfscope}%
\begin{pgfscope}%
\pgfsetbuttcap%
\pgfsetroundjoin%
\definecolor{currentfill}{rgb}{0.000000,0.000000,0.000000}%
\pgfsetfillcolor{currentfill}%
\pgfsetlinewidth{0.602250pt}%
\definecolor{currentstroke}{rgb}{0.000000,0.000000,0.000000}%
\pgfsetstrokecolor{currentstroke}%
\pgfsetdash{}{0pt}%
\pgfsys@defobject{currentmarker}{\pgfqpoint{0.000000in}{-0.027778in}}{\pgfqpoint{0.000000in}{0.000000in}}{%
\pgfpathmoveto{\pgfqpoint{0.000000in}{0.000000in}}%
\pgfpathlineto{\pgfqpoint{0.000000in}{-0.027778in}}%
\pgfusepath{stroke,fill}%
}%
\begin{pgfscope}%
\pgfsys@transformshift{2.867349in}{0.467838in}%
\pgfsys@useobject{currentmarker}{}%
\end{pgfscope}%
\end{pgfscope}%
\begin{pgfscope}%
\pgfsetbuttcap%
\pgfsetroundjoin%
\definecolor{currentfill}{rgb}{0.000000,0.000000,0.000000}%
\pgfsetfillcolor{currentfill}%
\pgfsetlinewidth{0.602250pt}%
\definecolor{currentstroke}{rgb}{0.000000,0.000000,0.000000}%
\pgfsetstrokecolor{currentstroke}%
\pgfsetdash{}{0pt}%
\pgfsys@defobject{currentmarker}{\pgfqpoint{0.000000in}{-0.027778in}}{\pgfqpoint{0.000000in}{0.000000in}}{%
\pgfpathmoveto{\pgfqpoint{0.000000in}{0.000000in}}%
\pgfpathlineto{\pgfqpoint{0.000000in}{-0.027778in}}%
\pgfusepath{stroke,fill}%
}%
\begin{pgfscope}%
\pgfsys@transformshift{2.914841in}{0.467838in}%
\pgfsys@useobject{currentmarker}{}%
\end{pgfscope}%
\end{pgfscope}%
\begin{pgfscope}%
\pgfsetbuttcap%
\pgfsetroundjoin%
\definecolor{currentfill}{rgb}{0.000000,0.000000,0.000000}%
\pgfsetfillcolor{currentfill}%
\pgfsetlinewidth{0.602250pt}%
\definecolor{currentstroke}{rgb}{0.000000,0.000000,0.000000}%
\pgfsetstrokecolor{currentstroke}%
\pgfsetdash{}{0pt}%
\pgfsys@defobject{currentmarker}{\pgfqpoint{0.000000in}{-0.027778in}}{\pgfqpoint{0.000000in}{0.000000in}}{%
\pgfpathmoveto{\pgfqpoint{0.000000in}{0.000000in}}%
\pgfpathlineto{\pgfqpoint{0.000000in}{-0.027778in}}%
\pgfusepath{stroke,fill}%
}%
\begin{pgfscope}%
\pgfsys@transformshift{2.956732in}{0.467838in}%
\pgfsys@useobject{currentmarker}{}%
\end{pgfscope}%
\end{pgfscope}%
\begin{pgfscope}%
\pgfsetbuttcap%
\pgfsetroundjoin%
\definecolor{currentfill}{rgb}{0.000000,0.000000,0.000000}%
\pgfsetfillcolor{currentfill}%
\pgfsetlinewidth{0.602250pt}%
\definecolor{currentstroke}{rgb}{0.000000,0.000000,0.000000}%
\pgfsetstrokecolor{currentstroke}%
\pgfsetdash{}{0pt}%
\pgfsys@defobject{currentmarker}{\pgfqpoint{0.000000in}{-0.027778in}}{\pgfqpoint{0.000000in}{0.000000in}}{%
\pgfpathmoveto{\pgfqpoint{0.000000in}{0.000000in}}%
\pgfpathlineto{\pgfqpoint{0.000000in}{-0.027778in}}%
\pgfusepath{stroke,fill}%
}%
\begin{pgfscope}%
\pgfsys@transformshift{3.240729in}{0.467838in}%
\pgfsys@useobject{currentmarker}{}%
\end{pgfscope}%
\end{pgfscope}%
\begin{pgfscope}%
\pgfsetbuttcap%
\pgfsetroundjoin%
\definecolor{currentfill}{rgb}{0.000000,0.000000,0.000000}%
\pgfsetfillcolor{currentfill}%
\pgfsetlinewidth{0.602250pt}%
\definecolor{currentstroke}{rgb}{0.000000,0.000000,0.000000}%
\pgfsetstrokecolor{currentstroke}%
\pgfsetdash{}{0pt}%
\pgfsys@defobject{currentmarker}{\pgfqpoint{0.000000in}{-0.027778in}}{\pgfqpoint{0.000000in}{0.000000in}}{%
\pgfpathmoveto{\pgfqpoint{0.000000in}{0.000000in}}%
\pgfpathlineto{\pgfqpoint{0.000000in}{-0.027778in}}%
\pgfusepath{stroke,fill}%
}%
\begin{pgfscope}%
\pgfsys@transformshift{3.384936in}{0.467838in}%
\pgfsys@useobject{currentmarker}{}%
\end{pgfscope}%
\end{pgfscope}%
\begin{pgfscope}%
\pgfsetbuttcap%
\pgfsetroundjoin%
\definecolor{currentfill}{rgb}{0.000000,0.000000,0.000000}%
\pgfsetfillcolor{currentfill}%
\pgfsetlinewidth{0.602250pt}%
\definecolor{currentstroke}{rgb}{0.000000,0.000000,0.000000}%
\pgfsetstrokecolor{currentstroke}%
\pgfsetdash{}{0pt}%
\pgfsys@defobject{currentmarker}{\pgfqpoint{0.000000in}{-0.027778in}}{\pgfqpoint{0.000000in}{0.000000in}}{%
\pgfpathmoveto{\pgfqpoint{0.000000in}{0.000000in}}%
\pgfpathlineto{\pgfqpoint{0.000000in}{-0.027778in}}%
\pgfusepath{stroke,fill}%
}%
\begin{pgfscope}%
\pgfsys@transformshift{3.487253in}{0.467838in}%
\pgfsys@useobject{currentmarker}{}%
\end{pgfscope}%
\end{pgfscope}%
\begin{pgfscope}%
\pgfsetbuttcap%
\pgfsetroundjoin%
\definecolor{currentfill}{rgb}{0.000000,0.000000,0.000000}%
\pgfsetfillcolor{currentfill}%
\pgfsetlinewidth{0.602250pt}%
\definecolor{currentstroke}{rgb}{0.000000,0.000000,0.000000}%
\pgfsetstrokecolor{currentstroke}%
\pgfsetdash{}{0pt}%
\pgfsys@defobject{currentmarker}{\pgfqpoint{0.000000in}{-0.027778in}}{\pgfqpoint{0.000000in}{0.000000in}}{%
\pgfpathmoveto{\pgfqpoint{0.000000in}{0.000000in}}%
\pgfpathlineto{\pgfqpoint{0.000000in}{-0.027778in}}%
\pgfusepath{stroke,fill}%
}%
\begin{pgfscope}%
\pgfsys@transformshift{3.566617in}{0.467838in}%
\pgfsys@useobject{currentmarker}{}%
\end{pgfscope}%
\end{pgfscope}%
\begin{pgfscope}%
\pgfsetbuttcap%
\pgfsetroundjoin%
\definecolor{currentfill}{rgb}{0.000000,0.000000,0.000000}%
\pgfsetfillcolor{currentfill}%
\pgfsetlinewidth{0.602250pt}%
\definecolor{currentstroke}{rgb}{0.000000,0.000000,0.000000}%
\pgfsetstrokecolor{currentstroke}%
\pgfsetdash{}{0pt}%
\pgfsys@defobject{currentmarker}{\pgfqpoint{0.000000in}{-0.027778in}}{\pgfqpoint{0.000000in}{0.000000in}}{%
\pgfpathmoveto{\pgfqpoint{0.000000in}{0.000000in}}%
\pgfpathlineto{\pgfqpoint{0.000000in}{-0.027778in}}%
\pgfusepath{stroke,fill}%
}%
\begin{pgfscope}%
\pgfsys@transformshift{3.631461in}{0.467838in}%
\pgfsys@useobject{currentmarker}{}%
\end{pgfscope}%
\end{pgfscope}%
\begin{pgfscope}%
\pgfsetbuttcap%
\pgfsetroundjoin%
\definecolor{currentfill}{rgb}{0.000000,0.000000,0.000000}%
\pgfsetfillcolor{currentfill}%
\pgfsetlinewidth{0.602250pt}%
\definecolor{currentstroke}{rgb}{0.000000,0.000000,0.000000}%
\pgfsetstrokecolor{currentstroke}%
\pgfsetdash{}{0pt}%
\pgfsys@defobject{currentmarker}{\pgfqpoint{0.000000in}{-0.027778in}}{\pgfqpoint{0.000000in}{0.000000in}}{%
\pgfpathmoveto{\pgfqpoint{0.000000in}{0.000000in}}%
\pgfpathlineto{\pgfqpoint{0.000000in}{-0.027778in}}%
\pgfusepath{stroke,fill}%
}%
\begin{pgfscope}%
\pgfsys@transformshift{3.686286in}{0.467838in}%
\pgfsys@useobject{currentmarker}{}%
\end{pgfscope}%
\end{pgfscope}%
\begin{pgfscope}%
\pgfsetbuttcap%
\pgfsetroundjoin%
\definecolor{currentfill}{rgb}{0.000000,0.000000,0.000000}%
\pgfsetfillcolor{currentfill}%
\pgfsetlinewidth{0.602250pt}%
\definecolor{currentstroke}{rgb}{0.000000,0.000000,0.000000}%
\pgfsetstrokecolor{currentstroke}%
\pgfsetdash{}{0pt}%
\pgfsys@defobject{currentmarker}{\pgfqpoint{0.000000in}{-0.027778in}}{\pgfqpoint{0.000000in}{0.000000in}}{%
\pgfpathmoveto{\pgfqpoint{0.000000in}{0.000000in}}%
\pgfpathlineto{\pgfqpoint{0.000000in}{-0.027778in}}%
\pgfusepath{stroke,fill}%
}%
\begin{pgfscope}%
\pgfsys@transformshift{3.733778in}{0.467838in}%
\pgfsys@useobject{currentmarker}{}%
\end{pgfscope}%
\end{pgfscope}%
\begin{pgfscope}%
\pgfsetbuttcap%
\pgfsetroundjoin%
\definecolor{currentfill}{rgb}{0.000000,0.000000,0.000000}%
\pgfsetfillcolor{currentfill}%
\pgfsetlinewidth{0.602250pt}%
\definecolor{currentstroke}{rgb}{0.000000,0.000000,0.000000}%
\pgfsetstrokecolor{currentstroke}%
\pgfsetdash{}{0pt}%
\pgfsys@defobject{currentmarker}{\pgfqpoint{0.000000in}{-0.027778in}}{\pgfqpoint{0.000000in}{0.000000in}}{%
\pgfpathmoveto{\pgfqpoint{0.000000in}{0.000000in}}%
\pgfpathlineto{\pgfqpoint{0.000000in}{-0.027778in}}%
\pgfusepath{stroke,fill}%
}%
\begin{pgfscope}%
\pgfsys@transformshift{3.775669in}{0.467838in}%
\pgfsys@useobject{currentmarker}{}%
\end{pgfscope}%
\end{pgfscope}%
\begin{pgfscope}%
\pgfsetbuttcap%
\pgfsetroundjoin%
\definecolor{currentfill}{rgb}{0.000000,0.000000,0.000000}%
\pgfsetfillcolor{currentfill}%
\pgfsetlinewidth{0.602250pt}%
\definecolor{currentstroke}{rgb}{0.000000,0.000000,0.000000}%
\pgfsetstrokecolor{currentstroke}%
\pgfsetdash{}{0pt}%
\pgfsys@defobject{currentmarker}{\pgfqpoint{0.000000in}{-0.027778in}}{\pgfqpoint{0.000000in}{0.000000in}}{%
\pgfpathmoveto{\pgfqpoint{0.000000in}{0.000000in}}%
\pgfpathlineto{\pgfqpoint{0.000000in}{-0.027778in}}%
\pgfusepath{stroke,fill}%
}%
\begin{pgfscope}%
\pgfsys@transformshift{4.059666in}{0.467838in}%
\pgfsys@useobject{currentmarker}{}%
\end{pgfscope}%
\end{pgfscope}%
\begin{pgfscope}%
\pgfsetbuttcap%
\pgfsetroundjoin%
\definecolor{currentfill}{rgb}{0.000000,0.000000,0.000000}%
\pgfsetfillcolor{currentfill}%
\pgfsetlinewidth{0.602250pt}%
\definecolor{currentstroke}{rgb}{0.000000,0.000000,0.000000}%
\pgfsetstrokecolor{currentstroke}%
\pgfsetdash{}{0pt}%
\pgfsys@defobject{currentmarker}{\pgfqpoint{0.000000in}{-0.027778in}}{\pgfqpoint{0.000000in}{0.000000in}}{%
\pgfpathmoveto{\pgfqpoint{0.000000in}{0.000000in}}%
\pgfpathlineto{\pgfqpoint{0.000000in}{-0.027778in}}%
\pgfusepath{stroke,fill}%
}%
\begin{pgfscope}%
\pgfsys@transformshift{4.203873in}{0.467838in}%
\pgfsys@useobject{currentmarker}{}%
\end{pgfscope}%
\end{pgfscope}%
\begin{pgfscope}%
\pgfsetbuttcap%
\pgfsetroundjoin%
\definecolor{currentfill}{rgb}{0.000000,0.000000,0.000000}%
\pgfsetfillcolor{currentfill}%
\pgfsetlinewidth{0.602250pt}%
\definecolor{currentstroke}{rgb}{0.000000,0.000000,0.000000}%
\pgfsetstrokecolor{currentstroke}%
\pgfsetdash{}{0pt}%
\pgfsys@defobject{currentmarker}{\pgfqpoint{0.000000in}{-0.027778in}}{\pgfqpoint{0.000000in}{0.000000in}}{%
\pgfpathmoveto{\pgfqpoint{0.000000in}{0.000000in}}%
\pgfpathlineto{\pgfqpoint{0.000000in}{-0.027778in}}%
\pgfusepath{stroke,fill}%
}%
\begin{pgfscope}%
\pgfsys@transformshift{4.306190in}{0.467838in}%
\pgfsys@useobject{currentmarker}{}%
\end{pgfscope}%
\end{pgfscope}%
\begin{pgfscope}%
\pgfsetbuttcap%
\pgfsetroundjoin%
\definecolor{currentfill}{rgb}{0.000000,0.000000,0.000000}%
\pgfsetfillcolor{currentfill}%
\pgfsetlinewidth{0.602250pt}%
\definecolor{currentstroke}{rgb}{0.000000,0.000000,0.000000}%
\pgfsetstrokecolor{currentstroke}%
\pgfsetdash{}{0pt}%
\pgfsys@defobject{currentmarker}{\pgfqpoint{0.000000in}{-0.027778in}}{\pgfqpoint{0.000000in}{0.000000in}}{%
\pgfpathmoveto{\pgfqpoint{0.000000in}{0.000000in}}%
\pgfpathlineto{\pgfqpoint{0.000000in}{-0.027778in}}%
\pgfusepath{stroke,fill}%
}%
\begin{pgfscope}%
\pgfsys@transformshift{4.385553in}{0.467838in}%
\pgfsys@useobject{currentmarker}{}%
\end{pgfscope}%
\end{pgfscope}%
\begin{pgfscope}%
\pgfsetbuttcap%
\pgfsetroundjoin%
\definecolor{currentfill}{rgb}{0.000000,0.000000,0.000000}%
\pgfsetfillcolor{currentfill}%
\pgfsetlinewidth{0.602250pt}%
\definecolor{currentstroke}{rgb}{0.000000,0.000000,0.000000}%
\pgfsetstrokecolor{currentstroke}%
\pgfsetdash{}{0pt}%
\pgfsys@defobject{currentmarker}{\pgfqpoint{0.000000in}{-0.027778in}}{\pgfqpoint{0.000000in}{0.000000in}}{%
\pgfpathmoveto{\pgfqpoint{0.000000in}{0.000000in}}%
\pgfpathlineto{\pgfqpoint{0.000000in}{-0.027778in}}%
\pgfusepath{stroke,fill}%
}%
\begin{pgfscope}%
\pgfsys@transformshift{4.450398in}{0.467838in}%
\pgfsys@useobject{currentmarker}{}%
\end{pgfscope}%
\end{pgfscope}%
\begin{pgfscope}%
\pgfsetbuttcap%
\pgfsetroundjoin%
\definecolor{currentfill}{rgb}{0.000000,0.000000,0.000000}%
\pgfsetfillcolor{currentfill}%
\pgfsetlinewidth{0.602250pt}%
\definecolor{currentstroke}{rgb}{0.000000,0.000000,0.000000}%
\pgfsetstrokecolor{currentstroke}%
\pgfsetdash{}{0pt}%
\pgfsys@defobject{currentmarker}{\pgfqpoint{0.000000in}{-0.027778in}}{\pgfqpoint{0.000000in}{0.000000in}}{%
\pgfpathmoveto{\pgfqpoint{0.000000in}{0.000000in}}%
\pgfpathlineto{\pgfqpoint{0.000000in}{-0.027778in}}%
\pgfusepath{stroke,fill}%
}%
\begin{pgfscope}%
\pgfsys@transformshift{4.505223in}{0.467838in}%
\pgfsys@useobject{currentmarker}{}%
\end{pgfscope}%
\end{pgfscope}%
\begin{pgfscope}%
\pgfsetbuttcap%
\pgfsetroundjoin%
\definecolor{currentfill}{rgb}{0.000000,0.000000,0.000000}%
\pgfsetfillcolor{currentfill}%
\pgfsetlinewidth{0.602250pt}%
\definecolor{currentstroke}{rgb}{0.000000,0.000000,0.000000}%
\pgfsetstrokecolor{currentstroke}%
\pgfsetdash{}{0pt}%
\pgfsys@defobject{currentmarker}{\pgfqpoint{0.000000in}{-0.027778in}}{\pgfqpoint{0.000000in}{0.000000in}}{%
\pgfpathmoveto{\pgfqpoint{0.000000in}{0.000000in}}%
\pgfpathlineto{\pgfqpoint{0.000000in}{-0.027778in}}%
\pgfusepath{stroke,fill}%
}%
\begin{pgfscope}%
\pgfsys@transformshift{4.552715in}{0.467838in}%
\pgfsys@useobject{currentmarker}{}%
\end{pgfscope}%
\end{pgfscope}%
\begin{pgfscope}%
\pgfsetbuttcap%
\pgfsetroundjoin%
\definecolor{currentfill}{rgb}{0.000000,0.000000,0.000000}%
\pgfsetfillcolor{currentfill}%
\pgfsetlinewidth{0.602250pt}%
\definecolor{currentstroke}{rgb}{0.000000,0.000000,0.000000}%
\pgfsetstrokecolor{currentstroke}%
\pgfsetdash{}{0pt}%
\pgfsys@defobject{currentmarker}{\pgfqpoint{0.000000in}{-0.027778in}}{\pgfqpoint{0.000000in}{0.000000in}}{%
\pgfpathmoveto{\pgfqpoint{0.000000in}{0.000000in}}%
\pgfpathlineto{\pgfqpoint{0.000000in}{-0.027778in}}%
\pgfusepath{stroke,fill}%
}%
\begin{pgfscope}%
\pgfsys@transformshift{4.594605in}{0.467838in}%
\pgfsys@useobject{currentmarker}{}%
\end{pgfscope}%
\end{pgfscope}%
\begin{pgfscope}%
\definecolor{textcolor}{rgb}{0.000000,0.000000,0.000000}%
\pgfsetstrokecolor{textcolor}%
\pgfsetfillcolor{textcolor}%
\pgftext[x=2.584736in,y=0.207530in,,top]{\color{textcolor}\sffamily\fontsize{8.000000}{9.600000}\selectfont Longest solving time (seconds)}%
\end{pgfscope}%
\begin{pgfscope}%
\pgfsetbuttcap%
\pgfsetroundjoin%
\definecolor{currentfill}{rgb}{0.000000,0.000000,0.000000}%
\pgfsetfillcolor{currentfill}%
\pgfsetlinewidth{0.803000pt}%
\definecolor{currentstroke}{rgb}{0.000000,0.000000,0.000000}%
\pgfsetstrokecolor{currentstroke}%
\pgfsetdash{}{0pt}%
\pgfsys@defobject{currentmarker}{\pgfqpoint{-0.048611in}{0.000000in}}{\pgfqpoint{-0.000000in}{0.000000in}}{%
\pgfpathmoveto{\pgfqpoint{-0.000000in}{0.000000in}}%
\pgfpathlineto{\pgfqpoint{-0.048611in}{0.000000in}}%
\pgfusepath{stroke,fill}%
}%
\begin{pgfscope}%
\pgfsys@transformshift{0.537394in}{0.467838in}%
\pgfsys@useobject{currentmarker}{}%
\end{pgfscope}%
\end{pgfscope}%
\begin{pgfscope}%
\definecolor{textcolor}{rgb}{0.000000,0.000000,0.000000}%
\pgfsetstrokecolor{textcolor}%
\pgfsetfillcolor{textcolor}%
\pgftext[x=0.381143in, y=0.425629in, left, base]{\color{textcolor}\sffamily\fontsize{8.000000}{9.600000}\selectfont \(\displaystyle {0}\)}%
\end{pgfscope}%
\begin{pgfscope}%
\pgfsetbuttcap%
\pgfsetroundjoin%
\definecolor{currentfill}{rgb}{0.000000,0.000000,0.000000}%
\pgfsetfillcolor{currentfill}%
\pgfsetlinewidth{0.803000pt}%
\definecolor{currentstroke}{rgb}{0.000000,0.000000,0.000000}%
\pgfsetstrokecolor{currentstroke}%
\pgfsetdash{}{0pt}%
\pgfsys@defobject{currentmarker}{\pgfqpoint{-0.048611in}{0.000000in}}{\pgfqpoint{-0.000000in}{0.000000in}}{%
\pgfpathmoveto{\pgfqpoint{-0.000000in}{0.000000in}}%
\pgfpathlineto{\pgfqpoint{-0.048611in}{0.000000in}}%
\pgfusepath{stroke,fill}%
}%
\begin{pgfscope}%
\pgfsys@transformshift{0.537394in}{0.816877in}%
\pgfsys@useobject{currentmarker}{}%
\end{pgfscope}%
\end{pgfscope}%
\begin{pgfscope}%
\definecolor{textcolor}{rgb}{0.000000,0.000000,0.000000}%
\pgfsetstrokecolor{textcolor}%
\pgfsetfillcolor{textcolor}%
\pgftext[x=0.322114in, y=0.774668in, left, base]{\color{textcolor}\sffamily\fontsize{8.000000}{9.600000}\selectfont \(\displaystyle {50}\)}%
\end{pgfscope}%
\begin{pgfscope}%
\pgfsetbuttcap%
\pgfsetroundjoin%
\definecolor{currentfill}{rgb}{0.000000,0.000000,0.000000}%
\pgfsetfillcolor{currentfill}%
\pgfsetlinewidth{0.803000pt}%
\definecolor{currentstroke}{rgb}{0.000000,0.000000,0.000000}%
\pgfsetstrokecolor{currentstroke}%
\pgfsetdash{}{0pt}%
\pgfsys@defobject{currentmarker}{\pgfqpoint{-0.048611in}{0.000000in}}{\pgfqpoint{-0.000000in}{0.000000in}}{%
\pgfpathmoveto{\pgfqpoint{-0.000000in}{0.000000in}}%
\pgfpathlineto{\pgfqpoint{-0.048611in}{0.000000in}}%
\pgfusepath{stroke,fill}%
}%
\begin{pgfscope}%
\pgfsys@transformshift{0.537394in}{1.165916in}%
\pgfsys@useobject{currentmarker}{}%
\end{pgfscope}%
\end{pgfscope}%
\begin{pgfscope}%
\definecolor{textcolor}{rgb}{0.000000,0.000000,0.000000}%
\pgfsetstrokecolor{textcolor}%
\pgfsetfillcolor{textcolor}%
\pgftext[x=0.263086in, y=1.123707in, left, base]{\color{textcolor}\sffamily\fontsize{8.000000}{9.600000}\selectfont \(\displaystyle {100}\)}%
\end{pgfscope}%
\begin{pgfscope}%
\pgfsetbuttcap%
\pgfsetroundjoin%
\definecolor{currentfill}{rgb}{0.000000,0.000000,0.000000}%
\pgfsetfillcolor{currentfill}%
\pgfsetlinewidth{0.803000pt}%
\definecolor{currentstroke}{rgb}{0.000000,0.000000,0.000000}%
\pgfsetstrokecolor{currentstroke}%
\pgfsetdash{}{0pt}%
\pgfsys@defobject{currentmarker}{\pgfqpoint{-0.048611in}{0.000000in}}{\pgfqpoint{-0.000000in}{0.000000in}}{%
\pgfpathmoveto{\pgfqpoint{-0.000000in}{0.000000in}}%
\pgfpathlineto{\pgfqpoint{-0.048611in}{0.000000in}}%
\pgfusepath{stroke,fill}%
}%
\begin{pgfscope}%
\pgfsys@transformshift{0.537394in}{1.514955in}%
\pgfsys@useobject{currentmarker}{}%
\end{pgfscope}%
\end{pgfscope}%
\begin{pgfscope}%
\definecolor{textcolor}{rgb}{0.000000,0.000000,0.000000}%
\pgfsetstrokecolor{textcolor}%
\pgfsetfillcolor{textcolor}%
\pgftext[x=0.263086in, y=1.472745in, left, base]{\color{textcolor}\sffamily\fontsize{8.000000}{9.600000}\selectfont \(\displaystyle {150}\)}%
\end{pgfscope}%
\begin{pgfscope}%
\pgfsetbuttcap%
\pgfsetroundjoin%
\definecolor{currentfill}{rgb}{0.000000,0.000000,0.000000}%
\pgfsetfillcolor{currentfill}%
\pgfsetlinewidth{0.803000pt}%
\definecolor{currentstroke}{rgb}{0.000000,0.000000,0.000000}%
\pgfsetstrokecolor{currentstroke}%
\pgfsetdash{}{0pt}%
\pgfsys@defobject{currentmarker}{\pgfqpoint{-0.048611in}{0.000000in}}{\pgfqpoint{-0.000000in}{0.000000in}}{%
\pgfpathmoveto{\pgfqpoint{-0.000000in}{0.000000in}}%
\pgfpathlineto{\pgfqpoint{-0.048611in}{0.000000in}}%
\pgfusepath{stroke,fill}%
}%
\begin{pgfscope}%
\pgfsys@transformshift{0.537394in}{1.863994in}%
\pgfsys@useobject{currentmarker}{}%
\end{pgfscope}%
\end{pgfscope}%
\begin{pgfscope}%
\definecolor{textcolor}{rgb}{0.000000,0.000000,0.000000}%
\pgfsetstrokecolor{textcolor}%
\pgfsetfillcolor{textcolor}%
\pgftext[x=0.263086in, y=1.821784in, left, base]{\color{textcolor}\sffamily\fontsize{8.000000}{9.600000}\selectfont \(\displaystyle {200}\)}%
\end{pgfscope}%
\begin{pgfscope}%
\pgfsetbuttcap%
\pgfsetroundjoin%
\definecolor{currentfill}{rgb}{0.000000,0.000000,0.000000}%
\pgfsetfillcolor{currentfill}%
\pgfsetlinewidth{0.803000pt}%
\definecolor{currentstroke}{rgb}{0.000000,0.000000,0.000000}%
\pgfsetstrokecolor{currentstroke}%
\pgfsetdash{}{0pt}%
\pgfsys@defobject{currentmarker}{\pgfqpoint{-0.048611in}{0.000000in}}{\pgfqpoint{-0.000000in}{0.000000in}}{%
\pgfpathmoveto{\pgfqpoint{-0.000000in}{0.000000in}}%
\pgfpathlineto{\pgfqpoint{-0.048611in}{0.000000in}}%
\pgfusepath{stroke,fill}%
}%
\begin{pgfscope}%
\pgfsys@transformshift{0.537394in}{2.213032in}%
\pgfsys@useobject{currentmarker}{}%
\end{pgfscope}%
\end{pgfscope}%
\begin{pgfscope}%
\definecolor{textcolor}{rgb}{0.000000,0.000000,0.000000}%
\pgfsetstrokecolor{textcolor}%
\pgfsetfillcolor{textcolor}%
\pgftext[x=0.263086in, y=2.170823in, left, base]{\color{textcolor}\sffamily\fontsize{8.000000}{9.600000}\selectfont \(\displaystyle {250}\)}%
\end{pgfscope}%
\begin{pgfscope}%
\pgfsetbuttcap%
\pgfsetroundjoin%
\definecolor{currentfill}{rgb}{0.000000,0.000000,0.000000}%
\pgfsetfillcolor{currentfill}%
\pgfsetlinewidth{0.803000pt}%
\definecolor{currentstroke}{rgb}{0.000000,0.000000,0.000000}%
\pgfsetstrokecolor{currentstroke}%
\pgfsetdash{}{0pt}%
\pgfsys@defobject{currentmarker}{\pgfqpoint{-0.048611in}{0.000000in}}{\pgfqpoint{-0.000000in}{0.000000in}}{%
\pgfpathmoveto{\pgfqpoint{-0.000000in}{0.000000in}}%
\pgfpathlineto{\pgfqpoint{-0.048611in}{0.000000in}}%
\pgfusepath{stroke,fill}%
}%
\begin{pgfscope}%
\pgfsys@transformshift{0.537394in}{2.562071in}%
\pgfsys@useobject{currentmarker}{}%
\end{pgfscope}%
\end{pgfscope}%
\begin{pgfscope}%
\definecolor{textcolor}{rgb}{0.000000,0.000000,0.000000}%
\pgfsetstrokecolor{textcolor}%
\pgfsetfillcolor{textcolor}%
\pgftext[x=0.263086in, y=2.519862in, left, base]{\color{textcolor}\sffamily\fontsize{8.000000}{9.600000}\selectfont \(\displaystyle {300}\)}%
\end{pgfscope}%
\begin{pgfscope}%
\pgfsetbuttcap%
\pgfsetroundjoin%
\definecolor{currentfill}{rgb}{0.000000,0.000000,0.000000}%
\pgfsetfillcolor{currentfill}%
\pgfsetlinewidth{0.803000pt}%
\definecolor{currentstroke}{rgb}{0.000000,0.000000,0.000000}%
\pgfsetstrokecolor{currentstroke}%
\pgfsetdash{}{0pt}%
\pgfsys@defobject{currentmarker}{\pgfqpoint{-0.048611in}{0.000000in}}{\pgfqpoint{-0.000000in}{0.000000in}}{%
\pgfpathmoveto{\pgfqpoint{-0.000000in}{0.000000in}}%
\pgfpathlineto{\pgfqpoint{-0.048611in}{0.000000in}}%
\pgfusepath{stroke,fill}%
}%
\begin{pgfscope}%
\pgfsys@transformshift{0.537394in}{2.911110in}%
\pgfsys@useobject{currentmarker}{}%
\end{pgfscope}%
\end{pgfscope}%
\begin{pgfscope}%
\definecolor{textcolor}{rgb}{0.000000,0.000000,0.000000}%
\pgfsetstrokecolor{textcolor}%
\pgfsetfillcolor{textcolor}%
\pgftext[x=0.263086in, y=2.868901in, left, base]{\color{textcolor}\sffamily\fontsize{8.000000}{9.600000}\selectfont \(\displaystyle {350}\)}%
\end{pgfscope}%
\begin{pgfscope}%
\pgfsetbuttcap%
\pgfsetroundjoin%
\definecolor{currentfill}{rgb}{0.000000,0.000000,0.000000}%
\pgfsetfillcolor{currentfill}%
\pgfsetlinewidth{0.803000pt}%
\definecolor{currentstroke}{rgb}{0.000000,0.000000,0.000000}%
\pgfsetstrokecolor{currentstroke}%
\pgfsetdash{}{0pt}%
\pgfsys@defobject{currentmarker}{\pgfqpoint{-0.048611in}{0.000000in}}{\pgfqpoint{-0.000000in}{0.000000in}}{%
\pgfpathmoveto{\pgfqpoint{-0.000000in}{0.000000in}}%
\pgfpathlineto{\pgfqpoint{-0.048611in}{0.000000in}}%
\pgfusepath{stroke,fill}%
}%
\begin{pgfscope}%
\pgfsys@transformshift{0.537394in}{3.260149in}%
\pgfsys@useobject{currentmarker}{}%
\end{pgfscope}%
\end{pgfscope}%
\begin{pgfscope}%
\definecolor{textcolor}{rgb}{0.000000,0.000000,0.000000}%
\pgfsetstrokecolor{textcolor}%
\pgfsetfillcolor{textcolor}%
\pgftext[x=0.263086in, y=3.217939in, left, base]{\color{textcolor}\sffamily\fontsize{8.000000}{9.600000}\selectfont \(\displaystyle {400}\)}%
\end{pgfscope}%
\begin{pgfscope}%
\definecolor{textcolor}{rgb}{0.000000,0.000000,0.000000}%
\pgfsetstrokecolor{textcolor}%
\pgfsetfillcolor{textcolor}%
\pgftext[x=0.207530in,y=1.863994in,,bottom,rotate=90.000000]{\color{textcolor}\sffamily\fontsize{8.000000}{9.600000}\selectfont Benchmarks solved}%
\end{pgfscope}%
\begin{pgfscope}%
\pgfpathrectangle{\pgfqpoint{0.537394in}{0.467838in}}{\pgfqpoint{4.094684in}{2.792310in}}%
\pgfusepath{clip}%
\pgfsetrectcap%
\pgfsetroundjoin%
\pgfsetlinewidth{1.003750pt}%
\definecolor{currentstroke}{rgb}{0.121569,0.466667,0.705882}%
\pgfsetstrokecolor{currentstroke}%
\pgfsetdash{}{0pt}%
\pgfpathmoveto{\pgfqpoint{0.537394in}{0.509723in}}%
\pgfpathlineto{\pgfqpoint{0.623180in}{0.516704in}}%
\pgfpathlineto{\pgfqpoint{0.652299in}{0.523685in}}%
\pgfpathlineto{\pgfqpoint{1.155535in}{0.530665in}}%
\pgfpathlineto{\pgfqpoint{1.161909in}{0.537646in}}%
\pgfpathlineto{\pgfqpoint{1.184532in}{0.544627in}}%
\pgfpathlineto{\pgfqpoint{1.193365in}{0.551608in}}%
\pgfpathlineto{\pgfqpoint{1.223585in}{0.558588in}}%
\pgfpathlineto{\pgfqpoint{1.256641in}{0.565569in}}%
\pgfpathlineto{\pgfqpoint{1.281347in}{0.572550in}}%
\pgfpathlineto{\pgfqpoint{1.292678in}{0.579531in}}%
\pgfpathlineto{\pgfqpoint{1.295722in}{0.586512in}}%
\pgfpathlineto{\pgfqpoint{1.301003in}{0.593492in}}%
\pgfpathlineto{\pgfqpoint{1.303519in}{0.600473in}}%
\pgfpathlineto{\pgfqpoint{1.320749in}{0.607454in}}%
\pgfpathlineto{\pgfqpoint{1.332065in}{0.614435in}}%
\pgfpathlineto{\pgfqpoint{1.342626in}{0.621415in}}%
\pgfpathlineto{\pgfqpoint{1.345267in}{0.628396in}}%
\pgfpathlineto{\pgfqpoint{1.355900in}{0.635377in}}%
\pgfpathlineto{\pgfqpoint{1.367340in}{0.642358in}}%
\pgfpathlineto{\pgfqpoint{1.368308in}{0.649339in}}%
\pgfpathlineto{\pgfqpoint{1.378260in}{0.656319in}}%
\pgfpathlineto{\pgfqpoint{1.387535in}{0.663300in}}%
\pgfpathlineto{\pgfqpoint{1.390917in}{0.670281in}}%
\pgfpathlineto{\pgfqpoint{1.397462in}{0.677262in}}%
\pgfpathlineto{\pgfqpoint{1.399096in}{0.684242in}}%
\pgfpathlineto{\pgfqpoint{1.407754in}{0.691223in}}%
\pgfpathlineto{\pgfqpoint{1.413951in}{0.698204in}}%
\pgfpathlineto{\pgfqpoint{1.421942in}{0.705185in}}%
\pgfpathlineto{\pgfqpoint{1.425962in}{0.712165in}}%
\pgfpathlineto{\pgfqpoint{1.430414in}{0.719146in}}%
\pgfpathlineto{\pgfqpoint{1.432837in}{0.726127in}}%
\pgfpathlineto{\pgfqpoint{1.435947in}{0.733108in}}%
\pgfpathlineto{\pgfqpoint{1.436441in}{0.740089in}}%
\pgfpathlineto{\pgfqpoint{1.436765in}{0.747069in}}%
\pgfpathlineto{\pgfqpoint{1.441966in}{0.754050in}}%
\pgfpathlineto{\pgfqpoint{1.443158in}{0.761031in}}%
\pgfpathlineto{\pgfqpoint{1.445457in}{0.768012in}}%
\pgfpathlineto{\pgfqpoint{1.446690in}{0.774992in}}%
\pgfpathlineto{\pgfqpoint{1.447142in}{0.781973in}}%
\pgfpathlineto{\pgfqpoint{1.450932in}{0.788954in}}%
\pgfpathlineto{\pgfqpoint{1.451534in}{0.795935in}}%
\pgfpathlineto{\pgfqpoint{1.453453in}{0.802916in}}%
\pgfpathlineto{\pgfqpoint{1.456447in}{0.809896in}}%
\pgfpathlineto{\pgfqpoint{1.456641in}{0.816877in}}%
\pgfpathlineto{\pgfqpoint{1.458359in}{0.823858in}}%
\pgfpathlineto{\pgfqpoint{1.458743in}{0.830839in}}%
\pgfpathlineto{\pgfqpoint{1.469429in}{0.837819in}}%
\pgfpathlineto{\pgfqpoint{1.471863in}{0.844800in}}%
\pgfpathlineto{\pgfqpoint{1.474679in}{0.851781in}}%
\pgfpathlineto{\pgfqpoint{1.474871in}{0.858762in}}%
\pgfpathlineto{\pgfqpoint{1.476853in}{0.865743in}}%
\pgfpathlineto{\pgfqpoint{1.478267in}{0.872723in}}%
\pgfpathlineto{\pgfqpoint{1.480719in}{0.879704in}}%
\pgfpathlineto{\pgfqpoint{1.483143in}{0.886685in}}%
\pgfpathlineto{\pgfqpoint{1.483941in}{0.893666in}}%
\pgfpathlineto{\pgfqpoint{1.488954in}{0.900646in}}%
\pgfpathlineto{\pgfqpoint{1.494438in}{0.907627in}}%
\pgfpathlineto{\pgfqpoint{1.499118in}{0.914608in}}%
\pgfpathlineto{\pgfqpoint{1.509457in}{0.921589in}}%
\pgfpathlineto{\pgfqpoint{1.517462in}{0.928570in}}%
\pgfpathlineto{\pgfqpoint{1.521267in}{0.935550in}}%
\pgfpathlineto{\pgfqpoint{1.526991in}{0.942531in}}%
\pgfpathlineto{\pgfqpoint{1.537478in}{0.949512in}}%
\pgfpathlineto{\pgfqpoint{1.544394in}{0.956493in}}%
\pgfpathlineto{\pgfqpoint{1.548674in}{0.963473in}}%
\pgfpathlineto{\pgfqpoint{1.549197in}{0.970454in}}%
\pgfpathlineto{\pgfqpoint{1.550089in}{0.977435in}}%
\pgfpathlineto{\pgfqpoint{1.550778in}{0.984416in}}%
\pgfpathlineto{\pgfqpoint{1.563149in}{0.991397in}}%
\pgfpathlineto{\pgfqpoint{1.569719in}{0.998377in}}%
\pgfpathlineto{\pgfqpoint{1.575970in}{1.005358in}}%
\pgfpathlineto{\pgfqpoint{1.581049in}{1.012339in}}%
\pgfpathlineto{\pgfqpoint{1.581427in}{1.019320in}}%
\pgfpathlineto{\pgfqpoint{1.585031in}{1.026300in}}%
\pgfpathlineto{\pgfqpoint{1.588977in}{1.033281in}}%
\pgfpathlineto{\pgfqpoint{1.590273in}{1.040262in}}%
\pgfpathlineto{\pgfqpoint{1.604962in}{1.047243in}}%
\pgfpathlineto{\pgfqpoint{1.607996in}{1.054224in}}%
\pgfpathlineto{\pgfqpoint{1.608017in}{1.061204in}}%
\pgfpathlineto{\pgfqpoint{1.609584in}{1.068185in}}%
\pgfpathlineto{\pgfqpoint{1.613808in}{1.075166in}}%
\pgfpathlineto{\pgfqpoint{1.637540in}{1.082147in}}%
\pgfpathlineto{\pgfqpoint{1.644115in}{1.089127in}}%
\pgfpathlineto{\pgfqpoint{1.648692in}{1.096108in}}%
\pgfpathlineto{\pgfqpoint{1.665201in}{1.103089in}}%
\pgfpathlineto{\pgfqpoint{1.679582in}{1.110070in}}%
\pgfpathlineto{\pgfqpoint{1.681824in}{1.117050in}}%
\pgfpathlineto{\pgfqpoint{1.691189in}{1.124031in}}%
\pgfpathlineto{\pgfqpoint{1.691711in}{1.131012in}}%
\pgfpathlineto{\pgfqpoint{1.724243in}{1.137993in}}%
\pgfpathlineto{\pgfqpoint{1.731394in}{1.144974in}}%
\pgfpathlineto{\pgfqpoint{1.733611in}{1.151954in}}%
\pgfpathlineto{\pgfqpoint{1.782049in}{1.158935in}}%
\pgfpathlineto{\pgfqpoint{1.793838in}{1.165916in}}%
\pgfpathlineto{\pgfqpoint{1.828569in}{1.172897in}}%
\pgfpathlineto{\pgfqpoint{1.833818in}{1.179877in}}%
\pgfpathlineto{\pgfqpoint{1.840259in}{1.186858in}}%
\pgfpathlineto{\pgfqpoint{1.845773in}{1.193839in}}%
\pgfpathlineto{\pgfqpoint{1.865641in}{1.200820in}}%
\pgfpathlineto{\pgfqpoint{1.908000in}{1.207801in}}%
\pgfpathlineto{\pgfqpoint{1.913583in}{1.214781in}}%
\pgfpathlineto{\pgfqpoint{2.350512in}{1.221762in}}%
\pgfpathlineto{\pgfqpoint{2.380719in}{1.228743in}}%
\pgfusepath{stroke}%
\end{pgfscope}%
\begin{pgfscope}%
\pgfpathrectangle{\pgfqpoint{0.537394in}{0.467838in}}{\pgfqpoint{4.094684in}{2.792310in}}%
\pgfusepath{clip}%
\pgfsetrectcap%
\pgfsetroundjoin%
\pgfsetlinewidth{1.003750pt}%
\definecolor{currentstroke}{rgb}{1.000000,0.498039,0.054902}%
\pgfsetstrokecolor{currentstroke}%
\pgfsetdash{}{0pt}%
\pgfpathmoveto{\pgfqpoint{1.587682in}{0.474819in}}%
\pgfpathlineto{\pgfqpoint{1.606153in}{0.481800in}}%
\pgfpathlineto{\pgfqpoint{1.610623in}{0.488781in}}%
\pgfpathlineto{\pgfqpoint{1.610786in}{0.495761in}}%
\pgfpathlineto{\pgfqpoint{1.614743in}{0.502742in}}%
\pgfpathlineto{\pgfqpoint{1.614896in}{0.509723in}}%
\pgfpathlineto{\pgfqpoint{1.618575in}{0.516704in}}%
\pgfpathlineto{\pgfqpoint{1.620140in}{0.523685in}}%
\pgfpathlineto{\pgfqpoint{1.624884in}{0.530665in}}%
\pgfpathlineto{\pgfqpoint{1.626175in}{0.537646in}}%
\pgfpathlineto{\pgfqpoint{1.631926in}{0.544627in}}%
\pgfpathlineto{\pgfqpoint{1.633380in}{0.551608in}}%
\pgfpathlineto{\pgfqpoint{1.633384in}{0.558588in}}%
\pgfpathlineto{\pgfqpoint{1.634676in}{0.565569in}}%
\pgfpathlineto{\pgfqpoint{1.635380in}{0.572550in}}%
\pgfpathlineto{\pgfqpoint{1.640551in}{0.579531in}}%
\pgfpathlineto{\pgfqpoint{1.643159in}{0.586512in}}%
\pgfpathlineto{\pgfqpoint{1.643374in}{0.593492in}}%
\pgfpathlineto{\pgfqpoint{1.644054in}{0.600473in}}%
\pgfpathlineto{\pgfqpoint{1.645141in}{0.607454in}}%
\pgfpathlineto{\pgfqpoint{1.646778in}{0.614435in}}%
\pgfpathlineto{\pgfqpoint{1.650265in}{0.621415in}}%
\pgfpathlineto{\pgfqpoint{1.653237in}{0.628396in}}%
\pgfpathlineto{\pgfqpoint{1.653662in}{0.635377in}}%
\pgfpathlineto{\pgfqpoint{1.655016in}{0.642358in}}%
\pgfpathlineto{\pgfqpoint{1.655956in}{0.649339in}}%
\pgfpathlineto{\pgfqpoint{1.656218in}{0.656319in}}%
\pgfpathlineto{\pgfqpoint{1.656255in}{0.663300in}}%
\pgfpathlineto{\pgfqpoint{1.656703in}{0.670281in}}%
\pgfpathlineto{\pgfqpoint{1.658222in}{0.677262in}}%
\pgfpathlineto{\pgfqpoint{1.660531in}{0.684242in}}%
\pgfpathlineto{\pgfqpoint{1.660700in}{0.691223in}}%
\pgfpathlineto{\pgfqpoint{1.660728in}{0.698204in}}%
\pgfpathlineto{\pgfqpoint{1.661823in}{0.705185in}}%
\pgfpathlineto{\pgfqpoint{1.662344in}{0.712165in}}%
\pgfpathlineto{\pgfqpoint{1.663169in}{0.719146in}}%
\pgfpathlineto{\pgfqpoint{1.663604in}{0.726127in}}%
\pgfpathlineto{\pgfqpoint{1.664296in}{0.733108in}}%
\pgfpathlineto{\pgfqpoint{1.664528in}{0.740089in}}%
\pgfpathlineto{\pgfqpoint{1.664813in}{0.747069in}}%
\pgfpathlineto{\pgfqpoint{1.666570in}{0.754050in}}%
\pgfpathlineto{\pgfqpoint{1.666994in}{0.761031in}}%
\pgfpathlineto{\pgfqpoint{1.667537in}{0.768012in}}%
\pgfpathlineto{\pgfqpoint{1.667734in}{0.774992in}}%
\pgfpathlineto{\pgfqpoint{1.668676in}{0.781973in}}%
\pgfpathlineto{\pgfqpoint{1.669083in}{0.788954in}}%
\pgfpathlineto{\pgfqpoint{1.670047in}{0.795935in}}%
\pgfpathlineto{\pgfqpoint{1.672522in}{0.802916in}}%
\pgfpathlineto{\pgfqpoint{1.673288in}{0.809896in}}%
\pgfpathlineto{\pgfqpoint{1.673325in}{0.816877in}}%
\pgfpathlineto{\pgfqpoint{1.673396in}{0.823858in}}%
\pgfpathlineto{\pgfqpoint{1.674456in}{0.830839in}}%
\pgfpathlineto{\pgfqpoint{1.675032in}{0.837819in}}%
\pgfpathlineto{\pgfqpoint{1.676712in}{0.844800in}}%
\pgfpathlineto{\pgfqpoint{1.676884in}{0.851781in}}%
\pgfpathlineto{\pgfqpoint{1.678001in}{0.858762in}}%
\pgfpathlineto{\pgfqpoint{1.681131in}{0.865743in}}%
\pgfpathlineto{\pgfqpoint{1.683950in}{0.872723in}}%
\pgfpathlineto{\pgfqpoint{1.684848in}{0.879704in}}%
\pgfpathlineto{\pgfqpoint{1.686190in}{0.886685in}}%
\pgfpathlineto{\pgfqpoint{1.687347in}{0.893666in}}%
\pgfpathlineto{\pgfqpoint{1.688737in}{0.900646in}}%
\pgfpathlineto{\pgfqpoint{1.689643in}{0.907627in}}%
\pgfpathlineto{\pgfqpoint{1.691249in}{0.914608in}}%
\pgfpathlineto{\pgfqpoint{1.692737in}{0.921589in}}%
\pgfpathlineto{\pgfqpoint{1.693043in}{0.928570in}}%
\pgfpathlineto{\pgfqpoint{1.695298in}{0.935550in}}%
\pgfpathlineto{\pgfqpoint{1.695580in}{0.942531in}}%
\pgfpathlineto{\pgfqpoint{1.695930in}{0.949512in}}%
\pgfpathlineto{\pgfqpoint{1.696780in}{0.956493in}}%
\pgfpathlineto{\pgfqpoint{1.700136in}{0.963473in}}%
\pgfpathlineto{\pgfqpoint{1.701227in}{0.970454in}}%
\pgfpathlineto{\pgfqpoint{1.701557in}{0.977435in}}%
\pgfpathlineto{\pgfqpoint{1.703074in}{0.984416in}}%
\pgfpathlineto{\pgfqpoint{1.705299in}{0.991397in}}%
\pgfpathlineto{\pgfqpoint{1.708703in}{0.998377in}}%
\pgfpathlineto{\pgfqpoint{1.709323in}{1.005358in}}%
\pgfpathlineto{\pgfqpoint{1.710241in}{1.012339in}}%
\pgfpathlineto{\pgfqpoint{1.710427in}{1.019320in}}%
\pgfpathlineto{\pgfqpoint{1.713202in}{1.026300in}}%
\pgfpathlineto{\pgfqpoint{1.715938in}{1.033281in}}%
\pgfpathlineto{\pgfqpoint{1.716185in}{1.040262in}}%
\pgfpathlineto{\pgfqpoint{1.720751in}{1.047243in}}%
\pgfpathlineto{\pgfqpoint{1.723054in}{1.054224in}}%
\pgfpathlineto{\pgfqpoint{1.723943in}{1.061204in}}%
\pgfpathlineto{\pgfqpoint{1.726644in}{1.068185in}}%
\pgfpathlineto{\pgfqpoint{1.728389in}{1.075166in}}%
\pgfpathlineto{\pgfqpoint{1.731767in}{1.082147in}}%
\pgfpathlineto{\pgfqpoint{1.732259in}{1.089127in}}%
\pgfpathlineto{\pgfqpoint{1.732387in}{1.096108in}}%
\pgfpathlineto{\pgfqpoint{1.733018in}{1.103089in}}%
\pgfpathlineto{\pgfqpoint{1.741615in}{1.110070in}}%
\pgfpathlineto{\pgfqpoint{1.746308in}{1.117050in}}%
\pgfpathlineto{\pgfqpoint{1.750463in}{1.124031in}}%
\pgfpathlineto{\pgfqpoint{1.751782in}{1.131012in}}%
\pgfpathlineto{\pgfqpoint{1.753319in}{1.137993in}}%
\pgfpathlineto{\pgfqpoint{1.758139in}{1.144974in}}%
\pgfpathlineto{\pgfqpoint{1.764432in}{1.151954in}}%
\pgfpathlineto{\pgfqpoint{1.768082in}{1.158935in}}%
\pgfpathlineto{\pgfqpoint{1.776508in}{1.165916in}}%
\pgfpathlineto{\pgfqpoint{1.783506in}{1.172897in}}%
\pgfpathlineto{\pgfqpoint{1.784600in}{1.179877in}}%
\pgfpathlineto{\pgfqpoint{1.785287in}{1.186858in}}%
\pgfpathlineto{\pgfqpoint{1.791005in}{1.193839in}}%
\pgfpathlineto{\pgfqpoint{1.799146in}{1.200820in}}%
\pgfpathlineto{\pgfqpoint{1.807066in}{1.207801in}}%
\pgfpathlineto{\pgfqpoint{1.808243in}{1.214781in}}%
\pgfpathlineto{\pgfqpoint{1.844278in}{1.221762in}}%
\pgfpathlineto{\pgfqpoint{1.862830in}{1.228743in}}%
\pgfpathlineto{\pgfqpoint{1.967921in}{1.235724in}}%
\pgfpathlineto{\pgfqpoint{1.973744in}{1.242704in}}%
\pgfpathlineto{\pgfqpoint{2.182101in}{1.249685in}}%
\pgfpathlineto{\pgfqpoint{2.259660in}{1.256666in}}%
\pgfpathlineto{\pgfqpoint{2.811787in}{1.263647in}}%
\pgfpathlineto{\pgfqpoint{3.025597in}{1.270628in}}%
\pgfpathlineto{\pgfqpoint{3.427804in}{1.277608in}}%
\pgfusepath{stroke}%
\end{pgfscope}%
\begin{pgfscope}%
\pgfpathrectangle{\pgfqpoint{0.537394in}{0.467838in}}{\pgfqpoint{4.094684in}{2.792310in}}%
\pgfusepath{clip}%
\pgfsetrectcap%
\pgfsetroundjoin%
\pgfsetlinewidth{1.003750pt}%
\definecolor{currentstroke}{rgb}{0.172549,0.627451,0.172549}%
\pgfsetstrokecolor{currentstroke}%
\pgfsetdash{}{0pt}%
\pgfpathmoveto{\pgfqpoint{2.629336in}{0.474819in}}%
\pgfpathlineto{\pgfqpoint{2.630777in}{0.481800in}}%
\pgfpathlineto{\pgfqpoint{2.632740in}{0.488781in}}%
\pgfpathlineto{\pgfqpoint{2.634423in}{0.495761in}}%
\pgfpathlineto{\pgfqpoint{2.634777in}{0.502742in}}%
\pgfpathlineto{\pgfqpoint{2.638446in}{0.509723in}}%
\pgfpathlineto{\pgfqpoint{2.639466in}{0.516704in}}%
\pgfpathlineto{\pgfqpoint{2.640442in}{0.523685in}}%
\pgfpathlineto{\pgfqpoint{2.645044in}{0.530665in}}%
\pgfpathlineto{\pgfqpoint{2.646381in}{0.537646in}}%
\pgfpathlineto{\pgfqpoint{2.648579in}{0.544627in}}%
\pgfpathlineto{\pgfqpoint{2.649997in}{0.551608in}}%
\pgfpathlineto{\pgfqpoint{2.650508in}{0.558588in}}%
\pgfpathlineto{\pgfqpoint{2.651662in}{0.565569in}}%
\pgfpathlineto{\pgfqpoint{2.652057in}{0.572550in}}%
\pgfpathlineto{\pgfqpoint{2.653167in}{0.579531in}}%
\pgfpathlineto{\pgfqpoint{2.659152in}{0.586512in}}%
\pgfpathlineto{\pgfqpoint{2.669236in}{0.593492in}}%
\pgfpathlineto{\pgfqpoint{2.672026in}{0.600473in}}%
\pgfpathlineto{\pgfqpoint{2.678716in}{0.607454in}}%
\pgfpathlineto{\pgfqpoint{2.682953in}{0.614435in}}%
\pgfpathlineto{\pgfqpoint{2.684181in}{0.621415in}}%
\pgfpathlineto{\pgfqpoint{2.686197in}{0.628396in}}%
\pgfpathlineto{\pgfqpoint{2.689015in}{0.635377in}}%
\pgfpathlineto{\pgfqpoint{2.689255in}{0.642358in}}%
\pgfpathlineto{\pgfqpoint{2.689564in}{0.649339in}}%
\pgfpathlineto{\pgfqpoint{2.692583in}{0.656319in}}%
\pgfpathlineto{\pgfqpoint{2.692692in}{0.663300in}}%
\pgfpathlineto{\pgfqpoint{2.697367in}{0.670281in}}%
\pgfpathlineto{\pgfqpoint{2.699098in}{0.677262in}}%
\pgfpathlineto{\pgfqpoint{2.701391in}{0.684242in}}%
\pgfpathlineto{\pgfqpoint{2.703416in}{0.691223in}}%
\pgfpathlineto{\pgfqpoint{2.703800in}{0.698204in}}%
\pgfpathlineto{\pgfqpoint{2.706197in}{0.705185in}}%
\pgfpathlineto{\pgfqpoint{2.710978in}{0.712165in}}%
\pgfpathlineto{\pgfqpoint{2.714609in}{0.719146in}}%
\pgfpathlineto{\pgfqpoint{2.728366in}{0.726127in}}%
\pgfpathlineto{\pgfqpoint{2.732094in}{0.733108in}}%
\pgfpathlineto{\pgfqpoint{2.733909in}{0.740089in}}%
\pgfpathlineto{\pgfqpoint{2.734624in}{0.747069in}}%
\pgfpathlineto{\pgfqpoint{2.735622in}{0.754050in}}%
\pgfpathlineto{\pgfqpoint{2.738080in}{0.761031in}}%
\pgfpathlineto{\pgfqpoint{2.742340in}{0.768012in}}%
\pgfpathlineto{\pgfqpoint{2.743171in}{0.774992in}}%
\pgfpathlineto{\pgfqpoint{2.768326in}{0.781973in}}%
\pgfpathlineto{\pgfqpoint{2.769193in}{0.788954in}}%
\pgfpathlineto{\pgfqpoint{2.771843in}{0.795935in}}%
\pgfpathlineto{\pgfqpoint{2.774546in}{0.802916in}}%
\pgfpathlineto{\pgfqpoint{2.777503in}{0.809896in}}%
\pgfpathlineto{\pgfqpoint{2.780019in}{0.816877in}}%
\pgfpathlineto{\pgfqpoint{2.780519in}{0.823858in}}%
\pgfpathlineto{\pgfqpoint{2.782335in}{0.830839in}}%
\pgfpathlineto{\pgfqpoint{2.791381in}{0.837819in}}%
\pgfpathlineto{\pgfqpoint{2.793183in}{0.844800in}}%
\pgfpathlineto{\pgfqpoint{2.797060in}{0.851781in}}%
\pgfpathlineto{\pgfqpoint{2.800837in}{0.858762in}}%
\pgfpathlineto{\pgfqpoint{2.806680in}{0.865743in}}%
\pgfpathlineto{\pgfqpoint{2.808580in}{0.872723in}}%
\pgfpathlineto{\pgfqpoint{2.808834in}{0.879704in}}%
\pgfpathlineto{\pgfqpoint{2.817504in}{0.886685in}}%
\pgfpathlineto{\pgfqpoint{2.817930in}{0.893666in}}%
\pgfpathlineto{\pgfqpoint{2.820964in}{0.900646in}}%
\pgfpathlineto{\pgfqpoint{2.821953in}{0.907627in}}%
\pgfpathlineto{\pgfqpoint{2.826339in}{0.914608in}}%
\pgfpathlineto{\pgfqpoint{2.827180in}{0.921589in}}%
\pgfpathlineto{\pgfqpoint{2.833731in}{0.928570in}}%
\pgfpathlineto{\pgfqpoint{2.833919in}{0.935550in}}%
\pgfpathlineto{\pgfqpoint{2.836498in}{0.942531in}}%
\pgfpathlineto{\pgfqpoint{2.841966in}{0.949512in}}%
\pgfpathlineto{\pgfqpoint{2.847268in}{0.956493in}}%
\pgfpathlineto{\pgfqpoint{2.849184in}{0.963473in}}%
\pgfpathlineto{\pgfqpoint{2.850101in}{0.970454in}}%
\pgfpathlineto{\pgfqpoint{2.850370in}{0.977435in}}%
\pgfpathlineto{\pgfqpoint{2.856100in}{0.984416in}}%
\pgfpathlineto{\pgfqpoint{2.861029in}{0.991397in}}%
\pgfpathlineto{\pgfqpoint{2.864223in}{0.998377in}}%
\pgfpathlineto{\pgfqpoint{2.870729in}{1.005358in}}%
\pgfpathlineto{\pgfqpoint{2.870856in}{1.012339in}}%
\pgfpathlineto{\pgfqpoint{2.872796in}{1.019320in}}%
\pgfpathlineto{\pgfqpoint{2.873479in}{1.026300in}}%
\pgfpathlineto{\pgfqpoint{2.884455in}{1.033281in}}%
\pgfpathlineto{\pgfqpoint{2.890655in}{1.040262in}}%
\pgfpathlineto{\pgfqpoint{2.896335in}{1.047243in}}%
\pgfpathlineto{\pgfqpoint{2.900271in}{1.054224in}}%
\pgfpathlineto{\pgfqpoint{2.900505in}{1.061204in}}%
\pgfpathlineto{\pgfqpoint{2.902393in}{1.068185in}}%
\pgfpathlineto{\pgfqpoint{2.904381in}{1.075166in}}%
\pgfpathlineto{\pgfqpoint{2.912657in}{1.082147in}}%
\pgfpathlineto{\pgfqpoint{2.946906in}{1.089127in}}%
\pgfpathlineto{\pgfqpoint{2.954005in}{1.096108in}}%
\pgfpathlineto{\pgfqpoint{2.955367in}{1.103089in}}%
\pgfpathlineto{\pgfqpoint{2.964918in}{1.110070in}}%
\pgfpathlineto{\pgfqpoint{2.969194in}{1.117050in}}%
\pgfpathlineto{\pgfqpoint{2.978289in}{1.124031in}}%
\pgfpathlineto{\pgfqpoint{2.978306in}{1.131012in}}%
\pgfpathlineto{\pgfqpoint{2.980802in}{1.137993in}}%
\pgfpathlineto{\pgfqpoint{2.985517in}{1.144974in}}%
\pgfpathlineto{\pgfqpoint{3.005528in}{1.151954in}}%
\pgfpathlineto{\pgfqpoint{3.009720in}{1.158935in}}%
\pgfpathlineto{\pgfqpoint{3.013024in}{1.165916in}}%
\pgfpathlineto{\pgfqpoint{3.021095in}{1.172897in}}%
\pgfpathlineto{\pgfqpoint{3.040409in}{1.179877in}}%
\pgfpathlineto{\pgfqpoint{3.115390in}{1.186858in}}%
\pgfpathlineto{\pgfqpoint{3.143721in}{1.193839in}}%
\pgfpathlineto{\pgfqpoint{3.146666in}{1.200820in}}%
\pgfpathlineto{\pgfqpoint{3.179080in}{1.207801in}}%
\pgfpathlineto{\pgfqpoint{3.194958in}{1.214781in}}%
\pgfpathlineto{\pgfqpoint{3.197795in}{1.221762in}}%
\pgfpathlineto{\pgfqpoint{3.199165in}{1.228743in}}%
\pgfpathlineto{\pgfqpoint{3.209094in}{1.235724in}}%
\pgfpathlineto{\pgfqpoint{3.213815in}{1.242704in}}%
\pgfpathlineto{\pgfqpoint{3.242407in}{1.249685in}}%
\pgfpathlineto{\pgfqpoint{3.402850in}{1.256666in}}%
\pgfpathlineto{\pgfqpoint{3.413546in}{1.263647in}}%
\pgfpathlineto{\pgfqpoint{3.468029in}{1.270628in}}%
\pgfpathlineto{\pgfqpoint{3.494898in}{1.277608in}}%
\pgfpathlineto{\pgfqpoint{3.737128in}{1.284589in}}%
\pgfpathlineto{\pgfqpoint{3.751816in}{1.291570in}}%
\pgfpathlineto{\pgfqpoint{4.033120in}{1.298551in}}%
\pgfusepath{stroke}%
\end{pgfscope}%
\begin{pgfscope}%
\pgfpathrectangle{\pgfqpoint{0.537394in}{0.467838in}}{\pgfqpoint{4.094684in}{2.792310in}}%
\pgfusepath{clip}%
\pgfsetbuttcap%
\pgfsetroundjoin%
\pgfsetlinewidth{1.003750pt}%
\definecolor{currentstroke}{rgb}{0.839216,0.152941,0.156863}%
\pgfsetstrokecolor{currentstroke}%
\pgfsetdash{{3.700000pt}{1.600000pt}}{0.000000pt}%
\pgfpathmoveto{\pgfqpoint{1.109806in}{0.502742in}}%
\pgfpathlineto{\pgfqpoint{1.174651in}{0.516704in}}%
\pgfpathlineto{\pgfqpoint{1.229476in}{0.530665in}}%
\pgfpathlineto{\pgfqpoint{1.318858in}{0.551608in}}%
\pgfpathlineto{\pgfqpoint{1.356331in}{0.558588in}}%
\pgfpathlineto{\pgfqpoint{1.390229in}{0.565569in}}%
\pgfpathlineto{\pgfqpoint{1.421175in}{0.572550in}}%
\pgfpathlineto{\pgfqpoint{1.476000in}{0.579531in}}%
\pgfpathlineto{\pgfqpoint{1.500538in}{0.593492in}}%
\pgfpathlineto{\pgfqpoint{1.565383in}{0.600473in}}%
\pgfpathlineto{\pgfqpoint{1.584612in}{0.607454in}}%
\pgfpathlineto{\pgfqpoint{1.636753in}{0.628396in}}%
\pgfpathlineto{\pgfqpoint{1.652563in}{0.642358in}}%
\pgfpathlineto{\pgfqpoint{1.667700in}{0.649339in}}%
\pgfpathlineto{\pgfqpoint{1.696168in}{0.656319in}}%
\pgfpathlineto{\pgfqpoint{1.709590in}{0.670281in}}%
\pgfpathlineto{\pgfqpoint{1.722525in}{0.684242in}}%
\pgfpathlineto{\pgfqpoint{1.849380in}{0.691223in}}%
\pgfusepath{stroke}%
\end{pgfscope}%
\begin{pgfscope}%
\pgfpathrectangle{\pgfqpoint{0.537394in}{0.467838in}}{\pgfqpoint{4.094684in}{2.792310in}}%
\pgfusepath{clip}%
\pgfsetbuttcap%
\pgfsetroundjoin%
\pgfsetlinewidth{1.003750pt}%
\definecolor{currentstroke}{rgb}{0.580392,0.403922,0.741176}%
\pgfsetstrokecolor{currentstroke}%
\pgfsetdash{{3.700000pt}{1.600000pt}}{0.000000pt}%
\pgfpathmoveto{\pgfqpoint{1.109806in}{0.495761in}}%
\pgfpathlineto{\pgfqpoint{1.174651in}{0.509723in}}%
\pgfpathlineto{\pgfqpoint{1.276968in}{0.516704in}}%
\pgfpathlineto{\pgfqpoint{1.318858in}{0.523685in}}%
\pgfpathlineto{\pgfqpoint{1.390229in}{0.530665in}}%
\pgfpathlineto{\pgfqpoint{1.421175in}{0.537646in}}%
\pgfpathlineto{\pgfqpoint{1.449643in}{0.551608in}}%
\pgfpathlineto{\pgfqpoint{1.476000in}{0.572550in}}%
\pgfpathlineto{\pgfqpoint{1.500538in}{0.586512in}}%
\pgfpathlineto{\pgfqpoint{1.523492in}{0.607454in}}%
\pgfpathlineto{\pgfqpoint{1.545054in}{0.614435in}}%
\pgfpathlineto{\pgfqpoint{1.565383in}{0.642358in}}%
\pgfpathlineto{\pgfqpoint{1.584612in}{0.656319in}}%
\pgfpathlineto{\pgfqpoint{1.620208in}{0.670281in}}%
\pgfpathlineto{\pgfqpoint{1.652563in}{0.684242in}}%
\pgfpathlineto{\pgfqpoint{1.709590in}{0.698204in}}%
\pgfpathlineto{\pgfqpoint{1.747063in}{0.705185in}}%
\pgfpathlineto{\pgfqpoint{1.770017in}{0.712165in}}%
\pgfpathlineto{\pgfqpoint{1.811907in}{0.719146in}}%
\pgfpathlineto{\pgfqpoint{1.831137in}{0.726127in}}%
\pgfpathlineto{\pgfqpoint{1.840375in}{0.733108in}}%
\pgfpathlineto{\pgfqpoint{1.866732in}{0.740089in}}%
\pgfpathlineto{\pgfqpoint{1.875101in}{0.747069in}}%
\pgfpathlineto{\pgfqpoint{1.899087in}{0.754050in}}%
\pgfpathlineto{\pgfqpoint{1.914224in}{0.761031in}}%
\pgfpathlineto{\pgfqpoint{1.921558in}{0.768012in}}%
\pgfpathlineto{\pgfqpoint{1.987610in}{0.774992in}}%
\pgfpathlineto{\pgfqpoint{1.993587in}{0.781973in}}%
\pgfpathlineto{\pgfqpoint{2.016541in}{0.788954in}}%
\pgfpathlineto{\pgfqpoint{2.038103in}{0.795935in}}%
\pgfpathlineto{\pgfqpoint{2.043295in}{0.802916in}}%
\pgfpathlineto{\pgfqpoint{2.095904in}{0.809896in}}%
\pgfpathlineto{\pgfqpoint{2.121626in}{0.816877in}}%
\pgfpathlineto{\pgfqpoint{2.164434in}{0.823858in}}%
\pgfusepath{stroke}%
\end{pgfscope}%
\begin{pgfscope}%
\pgfpathrectangle{\pgfqpoint{0.537394in}{0.467838in}}{\pgfqpoint{4.094684in}{2.792310in}}%
\pgfusepath{clip}%
\pgfsetbuttcap%
\pgfsetroundjoin%
\pgfsetlinewidth{1.003750pt}%
\definecolor{currentstroke}{rgb}{0.549020,0.337255,0.294118}%
\pgfsetstrokecolor{currentstroke}%
\pgfsetdash{{1.000000pt}{1.650000pt}}{0.000000pt}%
\pgfpathmoveto{\pgfqpoint{1.030443in}{0.481800in}}%
\pgfpathlineto{\pgfqpoint{1.109806in}{0.509723in}}%
\pgfpathlineto{\pgfqpoint{1.174651in}{0.551608in}}%
\pgfpathlineto{\pgfqpoint{1.276968in}{0.558588in}}%
\pgfpathlineto{\pgfqpoint{1.356331in}{0.572550in}}%
\pgfpathlineto{\pgfqpoint{1.449643in}{0.579531in}}%
\pgfpathlineto{\pgfqpoint{1.476000in}{0.600473in}}%
\pgfpathlineto{\pgfqpoint{1.500538in}{0.607454in}}%
\pgfpathlineto{\pgfqpoint{1.523492in}{0.614435in}}%
\pgfpathlineto{\pgfqpoint{1.545054in}{0.621415in}}%
\pgfpathlineto{\pgfqpoint{1.565383in}{0.642358in}}%
\pgfpathlineto{\pgfqpoint{1.584612in}{0.649339in}}%
\pgfpathlineto{\pgfqpoint{1.620208in}{0.656319in}}%
\pgfpathlineto{\pgfqpoint{1.636753in}{0.670281in}}%
\pgfpathlineto{\pgfqpoint{1.667700in}{0.684242in}}%
\pgfpathlineto{\pgfqpoint{1.696168in}{0.691223in}}%
\pgfpathlineto{\pgfqpoint{1.709590in}{0.698204in}}%
\pgfpathlineto{\pgfqpoint{1.722525in}{0.705185in}}%
\pgfpathlineto{\pgfqpoint{1.758725in}{0.712165in}}%
\pgfusepath{stroke}%
\end{pgfscope}%
\begin{pgfscope}%
\pgfpathrectangle{\pgfqpoint{0.537394in}{0.467838in}}{\pgfqpoint{4.094684in}{2.792310in}}%
\pgfusepath{clip}%
\pgfsetbuttcap%
\pgfsetroundjoin%
\pgfsetlinewidth{1.003750pt}%
\definecolor{currentstroke}{rgb}{0.890196,0.466667,0.760784}%
\pgfsetstrokecolor{currentstroke}%
\pgfsetdash{{1.000000pt}{1.650000pt}}{0.000000pt}%
\pgfpathmoveto{\pgfqpoint{1.030443in}{0.474819in}}%
\pgfpathlineto{\pgfqpoint{1.109806in}{0.502742in}}%
\pgfpathlineto{\pgfqpoint{1.229476in}{0.509723in}}%
\pgfpathlineto{\pgfqpoint{1.318858in}{0.523685in}}%
\pgfpathlineto{\pgfqpoint{1.356331in}{0.530665in}}%
\pgfpathlineto{\pgfqpoint{1.390229in}{0.544627in}}%
\pgfpathlineto{\pgfqpoint{1.421175in}{0.565569in}}%
\pgfpathlineto{\pgfqpoint{1.476000in}{0.572550in}}%
\pgfpathlineto{\pgfqpoint{1.500538in}{0.586512in}}%
\pgfpathlineto{\pgfqpoint{1.523492in}{0.593492in}}%
\pgfpathlineto{\pgfqpoint{1.545054in}{0.621415in}}%
\pgfpathlineto{\pgfqpoint{1.565383in}{0.628396in}}%
\pgfpathlineto{\pgfqpoint{1.602855in}{0.642358in}}%
\pgfpathlineto{\pgfqpoint{1.652563in}{0.649339in}}%
\pgfpathlineto{\pgfqpoint{1.682218in}{0.656319in}}%
\pgfpathlineto{\pgfqpoint{1.709590in}{0.670281in}}%
\pgfpathlineto{\pgfqpoint{1.735005in}{0.677262in}}%
\pgfpathlineto{\pgfqpoint{1.747063in}{0.684242in}}%
\pgfpathlineto{\pgfqpoint{1.758725in}{0.691223in}}%
\pgfpathlineto{\pgfqpoint{1.791578in}{0.698204in}}%
\pgfpathlineto{\pgfqpoint{1.821652in}{0.705185in}}%
\pgfpathlineto{\pgfqpoint{1.866732in}{0.712165in}}%
\pgfpathlineto{\pgfqpoint{2.058432in}{0.719146in}}%
\pgfpathlineto{\pgfqpoint{2.149457in}{0.726127in}}%
\pgfpathlineto{\pgfqpoint{2.164434in}{0.733108in}}%
\pgfpathlineto{\pgfqpoint{3.524930in}{0.740089in}}%
\pgfusepath{stroke}%
\end{pgfscope}%
\begin{pgfscope}%
\pgfpathrectangle{\pgfqpoint{0.537394in}{0.467838in}}{\pgfqpoint{4.094684in}{2.792310in}}%
\pgfusepath{clip}%
\pgfsetbuttcap%
\pgfsetroundjoin%
\pgfsetlinewidth{1.003750pt}%
\definecolor{currentstroke}{rgb}{0.498039,0.498039,0.498039}%
\pgfsetstrokecolor{currentstroke}%
\pgfsetdash{{6.400000pt}{1.600000pt}{1.000000pt}{1.600000pt}}{0.000000pt}%
\pgfpathmoveto{\pgfqpoint{1.109806in}{0.474819in}}%
\pgfpathlineto{\pgfqpoint{1.174651in}{0.516704in}}%
\pgfpathlineto{\pgfqpoint{1.318858in}{0.523685in}}%
\pgfpathlineto{\pgfqpoint{1.390229in}{0.530665in}}%
\pgfpathlineto{\pgfqpoint{1.421175in}{0.537646in}}%
\pgfpathlineto{\pgfqpoint{1.523492in}{0.544627in}}%
\pgfpathlineto{\pgfqpoint{1.636753in}{0.551608in}}%
\pgfpathlineto{\pgfqpoint{1.758725in}{0.558588in}}%
\pgfpathlineto{\pgfqpoint{1.770017in}{0.565569in}}%
\pgfpathlineto{\pgfqpoint{1.831137in}{0.572550in}}%
\pgfpathlineto{\pgfqpoint{1.849380in}{0.579531in}}%
\pgfpathlineto{\pgfqpoint{1.875101in}{0.586512in}}%
\pgfpathlineto{\pgfqpoint{1.883278in}{0.593492in}}%
\pgfpathlineto{\pgfqpoint{1.914224in}{0.600473in}}%
\pgfpathlineto{\pgfqpoint{1.928743in}{0.607454in}}%
\pgfpathlineto{\pgfqpoint{2.016541in}{0.614435in}}%
\pgfpathlineto{\pgfqpoint{2.022055in}{0.621415in}}%
\pgfpathlineto{\pgfqpoint{2.077661in}{0.628396in}}%
\pgfpathlineto{\pgfqpoint{2.082311in}{0.635377in}}%
\pgfpathlineto{\pgfqpoint{2.215574in}{0.642358in}}%
\pgfpathlineto{\pgfqpoint{2.307418in}{0.649339in}}%
\pgfpathlineto{\pgfqpoint{2.312290in}{0.656319in}}%
\pgfpathlineto{\pgfqpoint{2.335697in}{0.663300in}}%
\pgfpathlineto{\pgfqpoint{2.351211in}{0.670281in}}%
\pgfpathlineto{\pgfqpoint{2.357658in}{0.677262in}}%
\pgfpathlineto{\pgfqpoint{2.535053in}{0.684242in}}%
\pgfpathlineto{\pgfqpoint{2.760258in}{0.691223in}}%
\pgfusepath{stroke}%
\end{pgfscope}%
\begin{pgfscope}%
\pgfpathrectangle{\pgfqpoint{0.537394in}{0.467838in}}{\pgfqpoint{4.094684in}{2.792310in}}%
\pgfusepath{clip}%
\pgfsetbuttcap%
\pgfsetroundjoin%
\pgfsetlinewidth{1.003750pt}%
\definecolor{currentstroke}{rgb}{0.737255,0.741176,0.133333}%
\pgfsetstrokecolor{currentstroke}%
\pgfsetdash{{6.400000pt}{1.600000pt}{1.000000pt}{1.600000pt}}{0.000000pt}%
\pgfpathmoveto{\pgfqpoint{1.109806in}{0.488781in}}%
\pgfpathlineto{\pgfqpoint{1.174651in}{0.509723in}}%
\pgfpathlineto{\pgfqpoint{1.229476in}{0.516704in}}%
\pgfpathlineto{\pgfqpoint{1.318858in}{0.523685in}}%
\pgfpathlineto{\pgfqpoint{1.449643in}{0.537646in}}%
\pgfpathlineto{\pgfqpoint{1.523492in}{0.544627in}}%
\pgfpathlineto{\pgfqpoint{1.545054in}{0.551608in}}%
\pgfpathlineto{\pgfqpoint{1.602855in}{0.558588in}}%
\pgfpathlineto{\pgfqpoint{1.620208in}{0.565569in}}%
\pgfpathlineto{\pgfqpoint{1.840375in}{0.572550in}}%
\pgfpathlineto{\pgfqpoint{1.858162in}{0.579531in}}%
\pgfpathlineto{\pgfqpoint{1.875101in}{0.586512in}}%
\pgfpathlineto{\pgfqpoint{1.914224in}{0.593492in}}%
\pgfpathlineto{\pgfqpoint{1.928743in}{0.600473in}}%
\pgfpathlineto{\pgfqpoint{2.016541in}{0.607454in}}%
\pgfpathlineto{\pgfqpoint{2.032834in}{0.628396in}}%
\pgfpathlineto{\pgfqpoint{2.043295in}{0.642358in}}%
\pgfpathlineto{\pgfqpoint{2.063338in}{0.649339in}}%
\pgfpathlineto{\pgfqpoint{2.072951in}{0.656319in}}%
\pgfpathlineto{\pgfqpoint{2.153261in}{0.663300in}}%
\pgfpathlineto{\pgfqpoint{2.218735in}{0.670281in}}%
\pgfpathlineto{\pgfqpoint{2.307418in}{0.677262in}}%
\pgfpathlineto{\pgfqpoint{2.331137in}{0.684242in}}%
\pgfpathlineto{\pgfqpoint{2.340199in}{0.691223in}}%
\pgfpathlineto{\pgfqpoint{2.756809in}{0.698204in}}%
\pgfusepath{stroke}%
\end{pgfscope}%
\begin{pgfscope}%
\pgfpathrectangle{\pgfqpoint{0.537394in}{0.467838in}}{\pgfqpoint{4.094684in}{2.792310in}}%
\pgfusepath{clip}%
\pgfsetrectcap%
\pgfsetroundjoin%
\pgfsetlinewidth{1.003750pt}%
\definecolor{currentstroke}{rgb}{0.090196,0.745098,0.811765}%
\pgfsetstrokecolor{currentstroke}%
\pgfsetdash{}{0pt}%
\pgfpathmoveto{\pgfqpoint{1.109806in}{0.481800in}}%
\pgfpathlineto{\pgfqpoint{1.174651in}{0.502742in}}%
\pgfpathlineto{\pgfqpoint{1.229476in}{0.509723in}}%
\pgfpathlineto{\pgfqpoint{1.318858in}{0.516704in}}%
\pgfpathlineto{\pgfqpoint{1.356331in}{0.530665in}}%
\pgfpathlineto{\pgfqpoint{1.390229in}{0.537646in}}%
\pgfpathlineto{\pgfqpoint{1.449643in}{0.544627in}}%
\pgfpathlineto{\pgfqpoint{1.476000in}{0.551608in}}%
\pgfpathlineto{\pgfqpoint{1.500538in}{0.558588in}}%
\pgfpathlineto{\pgfqpoint{1.523492in}{0.572550in}}%
\pgfpathlineto{\pgfqpoint{1.545054in}{0.579531in}}%
\pgfpathlineto{\pgfqpoint{1.602855in}{0.586512in}}%
\pgfpathlineto{\pgfqpoint{1.652563in}{0.593492in}}%
\pgfpathlineto{\pgfqpoint{1.667700in}{0.600473in}}%
\pgfpathlineto{\pgfqpoint{1.722525in}{0.607454in}}%
\pgfpathlineto{\pgfqpoint{1.735005in}{0.621415in}}%
\pgfpathlineto{\pgfqpoint{1.758725in}{0.628396in}}%
\pgfpathlineto{\pgfqpoint{1.780961in}{0.635377in}}%
\pgfpathlineto{\pgfqpoint{1.801888in}{0.642358in}}%
\pgfpathlineto{\pgfqpoint{1.840375in}{0.649339in}}%
\pgfpathlineto{\pgfqpoint{1.849380in}{0.670281in}}%
\pgfpathlineto{\pgfqpoint{1.875101in}{0.677262in}}%
\pgfpathlineto{\pgfqpoint{1.883278in}{0.684242in}}%
\pgfpathlineto{\pgfqpoint{1.906736in}{0.691223in}}%
\pgfpathlineto{\pgfqpoint{1.928743in}{0.698204in}}%
\pgfpathlineto{\pgfqpoint{1.999466in}{0.705185in}}%
\pgfpathlineto{\pgfqpoint{2.022055in}{0.712165in}}%
\pgfpathlineto{\pgfqpoint{2.053458in}{0.719146in}}%
\pgfpathlineto{\pgfqpoint{2.072951in}{0.726127in}}%
\pgfpathlineto{\pgfqpoint{2.178806in}{0.733108in}}%
\pgfpathlineto{\pgfqpoint{2.189217in}{0.740089in}}%
\pgfpathlineto{\pgfqpoint{2.309862in}{0.747069in}}%
\pgfpathlineto{\pgfqpoint{2.357658in}{0.754050in}}%
\pgfpathlineto{\pgfqpoint{2.361892in}{0.761031in}}%
\pgfpathlineto{\pgfqpoint{2.409121in}{0.768012in}}%
\pgfpathlineto{\pgfqpoint{2.421792in}{0.774992in}}%
\pgfpathlineto{\pgfqpoint{2.474579in}{0.781973in}}%
\pgfpathlineto{\pgfqpoint{2.489588in}{0.788954in}}%
\pgfpathlineto{\pgfqpoint{2.496862in}{0.795935in}}%
\pgfpathlineto{\pgfqpoint{2.517830in}{0.802916in}}%
\pgfpathlineto{\pgfqpoint{2.519185in}{0.809896in}}%
\pgfpathlineto{\pgfqpoint{2.520534in}{0.816877in}}%
\pgfpathlineto{\pgfqpoint{2.535053in}{0.823858in}}%
\pgfusepath{stroke}%
\end{pgfscope}%
\begin{pgfscope}%
\pgfpathrectangle{\pgfqpoint{0.537394in}{0.467838in}}{\pgfqpoint{4.094684in}{2.792310in}}%
\pgfusepath{clip}%
\pgfsetrectcap%
\pgfsetroundjoin%
\pgfsetlinewidth{1.003750pt}%
\definecolor{currentstroke}{rgb}{0.121569,0.466667,0.705882}%
\pgfsetstrokecolor{currentstroke}%
\pgfsetdash{}{0pt}%
\pgfpathmoveto{\pgfqpoint{1.109806in}{0.481800in}}%
\pgfpathlineto{\pgfqpoint{1.174651in}{0.502742in}}%
\pgfpathlineto{\pgfqpoint{1.229476in}{0.509723in}}%
\pgfpathlineto{\pgfqpoint{1.318858in}{0.516704in}}%
\pgfpathlineto{\pgfqpoint{1.356331in}{0.523685in}}%
\pgfpathlineto{\pgfqpoint{1.421175in}{0.530665in}}%
\pgfpathlineto{\pgfqpoint{1.500538in}{0.537646in}}%
\pgfpathlineto{\pgfqpoint{1.545054in}{0.558588in}}%
\pgfpathlineto{\pgfqpoint{1.565383in}{0.579531in}}%
\pgfpathlineto{\pgfqpoint{1.652563in}{0.593492in}}%
\pgfpathlineto{\pgfqpoint{1.667700in}{0.607454in}}%
\pgfpathlineto{\pgfqpoint{1.682218in}{0.621415in}}%
\pgfpathlineto{\pgfqpoint{1.696168in}{0.628396in}}%
\pgfpathlineto{\pgfqpoint{1.735005in}{0.635377in}}%
\pgfpathlineto{\pgfqpoint{1.770017in}{0.642358in}}%
\pgfpathlineto{\pgfqpoint{1.811907in}{0.649339in}}%
\pgfpathlineto{\pgfqpoint{1.849380in}{0.656319in}}%
\pgfpathlineto{\pgfqpoint{1.858162in}{0.663300in}}%
\pgfpathlineto{\pgfqpoint{1.962641in}{0.670281in}}%
\pgfpathlineto{\pgfqpoint{3.966270in}{0.677262in}}%
\pgfusepath{stroke}%
\end{pgfscope}%
\begin{pgfscope}%
\pgfsetrectcap%
\pgfsetmiterjoin%
\pgfsetlinewidth{0.803000pt}%
\definecolor{currentstroke}{rgb}{0.000000,0.000000,0.000000}%
\pgfsetstrokecolor{currentstroke}%
\pgfsetdash{}{0pt}%
\pgfpathmoveto{\pgfqpoint{0.537394in}{0.467838in}}%
\pgfpathlineto{\pgfqpoint{0.537394in}{3.260149in}}%
\pgfusepath{stroke}%
\end{pgfscope}%
\begin{pgfscope}%
\pgfsetrectcap%
\pgfsetmiterjoin%
\pgfsetlinewidth{0.803000pt}%
\definecolor{currentstroke}{rgb}{0.000000,0.000000,0.000000}%
\pgfsetstrokecolor{currentstroke}%
\pgfsetdash{}{0pt}%
\pgfpathmoveto{\pgfqpoint{4.632078in}{0.467838in}}%
\pgfpathlineto{\pgfqpoint{4.632078in}{3.260149in}}%
\pgfusepath{stroke}%
\end{pgfscope}%
\begin{pgfscope}%
\pgfsetrectcap%
\pgfsetmiterjoin%
\pgfsetlinewidth{0.803000pt}%
\definecolor{currentstroke}{rgb}{0.000000,0.000000,0.000000}%
\pgfsetstrokecolor{currentstroke}%
\pgfsetdash{}{0pt}%
\pgfpathmoveto{\pgfqpoint{0.537394in}{0.467838in}}%
\pgfpathlineto{\pgfqpoint{4.632078in}{0.467838in}}%
\pgfusepath{stroke}%
\end{pgfscope}%
\begin{pgfscope}%
\pgfsetrectcap%
\pgfsetmiterjoin%
\pgfsetlinewidth{0.803000pt}%
\definecolor{currentstroke}{rgb}{0.000000,0.000000,0.000000}%
\pgfsetstrokecolor{currentstroke}%
\pgfsetdash{}{0pt}%
\pgfpathmoveto{\pgfqpoint{0.537394in}{3.260149in}}%
\pgfpathlineto{\pgfqpoint{4.632078in}{3.260149in}}%
\pgfusepath{stroke}%
\end{pgfscope}%
\begin{pgfscope}%
\pgfsetbuttcap%
\pgfsetmiterjoin%
\definecolor{currentfill}{rgb}{1.000000,1.000000,1.000000}%
\pgfsetfillcolor{currentfill}%
\pgfsetfillopacity{0.800000}%
\pgfsetlinewidth{1.003750pt}%
\definecolor{currentstroke}{rgb}{0.800000,0.800000,0.800000}%
\pgfsetstrokecolor{currentstroke}%
\pgfsetstrokeopacity{0.800000}%
\pgfsetdash{}{0pt}%
\pgfpathmoveto{\pgfqpoint{3.361234in}{1.377316in}}%
\pgfpathlineto{\pgfqpoint{4.554300in}{1.377316in}}%
\pgfpathquadraticcurveto{\pgfqpoint{4.576522in}{1.377316in}}{\pgfqpoint{4.576522in}{1.399538in}}%
\pgfpathlineto{\pgfqpoint{4.576522in}{3.182371in}}%
\pgfpathquadraticcurveto{\pgfqpoint{4.576522in}{3.204593in}}{\pgfqpoint{4.554300in}{3.204593in}}%
\pgfpathlineto{\pgfqpoint{3.361234in}{3.204593in}}%
\pgfpathquadraticcurveto{\pgfqpoint{3.339012in}{3.204593in}}{\pgfqpoint{3.339012in}{3.182371in}}%
\pgfpathlineto{\pgfqpoint{3.339012in}{1.399538in}}%
\pgfpathquadraticcurveto{\pgfqpoint{3.339012in}{1.377316in}}{\pgfqpoint{3.361234in}{1.377316in}}%
\pgfpathclose%
\pgfusepath{stroke,fill}%
\end{pgfscope}%
\begin{pgfscope}%
\pgfsetrectcap%
\pgfsetroundjoin%
\pgfsetlinewidth{1.003750pt}%
\definecolor{currentstroke}{rgb}{0.121569,0.466667,0.705882}%
\pgfsetstrokecolor{currentstroke}%
\pgfsetdash{}{0pt}%
\pgfpathmoveto{\pgfqpoint{3.383456in}{3.114619in}}%
\pgfpathlineto{\pgfqpoint{3.605678in}{3.114619in}}%
\pgfusepath{stroke}%
\end{pgfscope}%
\begin{pgfscope}%
\definecolor{textcolor}{rgb}{0.000000,0.000000,0.000000}%
\pgfsetstrokecolor{textcolor}%
\pgfsetfillcolor{textcolor}%
\pgftext[x=3.694567in,y=3.075730in,left,base]{\color{textcolor}\sffamily\fontsize{8.000000}{9.600000}\selectfont LG(FlowCutter)}%
\end{pgfscope}%
\begin{pgfscope}%
\pgfsetrectcap%
\pgfsetroundjoin%
\pgfsetlinewidth{1.003750pt}%
\definecolor{currentstroke}{rgb}{1.000000,0.498039,0.054902}%
\pgfsetstrokecolor{currentstroke}%
\pgfsetdash{}{0pt}%
\pgfpathmoveto{\pgfqpoint{3.383456in}{2.951533in}}%
\pgfpathlineto{\pgfqpoint{3.605678in}{2.951533in}}%
\pgfusepath{stroke}%
\end{pgfscope}%
\begin{pgfscope}%
\definecolor{textcolor}{rgb}{0.000000,0.000000,0.000000}%
\pgfsetstrokecolor{textcolor}%
\pgfsetfillcolor{textcolor}%
\pgftext[x=3.694567in,y=2.912644in,left,base]{\color{textcolor}\sffamily\fontsize{8.000000}{9.600000}\selectfont LG(htd)}%
\end{pgfscope}%
\begin{pgfscope}%
\pgfsetrectcap%
\pgfsetroundjoin%
\pgfsetlinewidth{1.003750pt}%
\definecolor{currentstroke}{rgb}{0.172549,0.627451,0.172549}%
\pgfsetstrokecolor{currentstroke}%
\pgfsetdash{}{0pt}%
\pgfpathmoveto{\pgfqpoint{3.383456in}{2.788447in}}%
\pgfpathlineto{\pgfqpoint{3.605678in}{2.788447in}}%
\pgfusepath{stroke}%
\end{pgfscope}%
\begin{pgfscope}%
\definecolor{textcolor}{rgb}{0.000000,0.000000,0.000000}%
\pgfsetstrokecolor{textcolor}%
\pgfsetfillcolor{textcolor}%
\pgftext[x=3.694567in,y=2.749559in,left,base]{\color{textcolor}\sffamily\fontsize{8.000000}{9.600000}\selectfont LG(Tamaki)}%
\end{pgfscope}%
\begin{pgfscope}%
\pgfsetbuttcap%
\pgfsetroundjoin%
\pgfsetlinewidth{1.003750pt}%
\definecolor{currentstroke}{rgb}{0.839216,0.152941,0.156863}%
\pgfsetstrokecolor{currentstroke}%
\pgfsetdash{{3.700000pt}{1.600000pt}}{0.000000pt}%
\pgfpathmoveto{\pgfqpoint{3.383456in}{2.625362in}}%
\pgfpathlineto{\pgfqpoint{3.605678in}{2.625362in}}%
\pgfusepath{stroke}%
\end{pgfscope}%
\begin{pgfscope}%
\definecolor{textcolor}{rgb}{0.000000,0.000000,0.000000}%
\pgfsetstrokecolor{textcolor}%
\pgfsetfillcolor{textcolor}%
\pgftext[x=3.694567in,y=2.586473in,left,base]{\color{textcolor}\sffamily\fontsize{8.000000}{9.600000}\selectfont HTB(MCS, BE)}%
\end{pgfscope}%
\begin{pgfscope}%
\pgfsetbuttcap%
\pgfsetroundjoin%
\pgfsetlinewidth{1.003750pt}%
\definecolor{currentstroke}{rgb}{0.580392,0.403922,0.741176}%
\pgfsetstrokecolor{currentstroke}%
\pgfsetdash{{3.700000pt}{1.600000pt}}{0.000000pt}%
\pgfpathmoveto{\pgfqpoint{3.383456in}{2.462276in}}%
\pgfpathlineto{\pgfqpoint{3.605678in}{2.462276in}}%
\pgfusepath{stroke}%
\end{pgfscope}%
\begin{pgfscope}%
\definecolor{textcolor}{rgb}{0.000000,0.000000,0.000000}%
\pgfsetstrokecolor{textcolor}%
\pgfsetfillcolor{textcolor}%
\pgftext[x=3.694567in,y=2.423387in,left,base]{\color{textcolor}\sffamily\fontsize{8.000000}{9.600000}\selectfont HTB(MCS, BM)}%
\end{pgfscope}%
\begin{pgfscope}%
\pgfsetbuttcap%
\pgfsetroundjoin%
\pgfsetlinewidth{1.003750pt}%
\definecolor{currentstroke}{rgb}{0.549020,0.337255,0.294118}%
\pgfsetstrokecolor{currentstroke}%
\pgfsetdash{{1.000000pt}{1.650000pt}}{0.000000pt}%
\pgfpathmoveto{\pgfqpoint{3.383456in}{2.299190in}}%
\pgfpathlineto{\pgfqpoint{3.605678in}{2.299190in}}%
\pgfusepath{stroke}%
\end{pgfscope}%
\begin{pgfscope}%
\definecolor{textcolor}{rgb}{0.000000,0.000000,0.000000}%
\pgfsetstrokecolor{textcolor}%
\pgfsetfillcolor{textcolor}%
\pgftext[x=3.694567in,y=2.260301in,left,base]{\color{textcolor}\sffamily\fontsize{8.000000}{9.600000}\selectfont HTB(LP, BE)}%
\end{pgfscope}%
\begin{pgfscope}%
\pgfsetbuttcap%
\pgfsetroundjoin%
\pgfsetlinewidth{1.003750pt}%
\definecolor{currentstroke}{rgb}{0.890196,0.466667,0.760784}%
\pgfsetstrokecolor{currentstroke}%
\pgfsetdash{{1.000000pt}{1.650000pt}}{0.000000pt}%
\pgfpathmoveto{\pgfqpoint{3.383456in}{2.136104in}}%
\pgfpathlineto{\pgfqpoint{3.605678in}{2.136104in}}%
\pgfusepath{stroke}%
\end{pgfscope}%
\begin{pgfscope}%
\definecolor{textcolor}{rgb}{0.000000,0.000000,0.000000}%
\pgfsetstrokecolor{textcolor}%
\pgfsetfillcolor{textcolor}%
\pgftext[x=3.694567in,y=2.097215in,left,base]{\color{textcolor}\sffamily\fontsize{8.000000}{9.600000}\selectfont HTB(LP, BM)}%
\end{pgfscope}%
\begin{pgfscope}%
\pgfsetbuttcap%
\pgfsetroundjoin%
\pgfsetlinewidth{1.003750pt}%
\definecolor{currentstroke}{rgb}{0.498039,0.498039,0.498039}%
\pgfsetstrokecolor{currentstroke}%
\pgfsetdash{{6.400000pt}{1.600000pt}{1.000000pt}{1.600000pt}}{0.000000pt}%
\pgfpathmoveto{\pgfqpoint{3.383456in}{1.973018in}}%
\pgfpathlineto{\pgfqpoint{3.605678in}{1.973018in}}%
\pgfusepath{stroke}%
\end{pgfscope}%
\begin{pgfscope}%
\definecolor{textcolor}{rgb}{0.000000,0.000000,0.000000}%
\pgfsetstrokecolor{textcolor}%
\pgfsetfillcolor{textcolor}%
\pgftext[x=3.694567in,y=1.934129in,left,base]{\color{textcolor}\sffamily\fontsize{8.000000}{9.600000}\selectfont HTB(LM, BE)}%
\end{pgfscope}%
\begin{pgfscope}%
\pgfsetbuttcap%
\pgfsetroundjoin%
\pgfsetlinewidth{1.003750pt}%
\definecolor{currentstroke}{rgb}{0.737255,0.741176,0.133333}%
\pgfsetstrokecolor{currentstroke}%
\pgfsetdash{{6.400000pt}{1.600000pt}{1.000000pt}{1.600000pt}}{0.000000pt}%
\pgfpathmoveto{\pgfqpoint{3.383456in}{1.809932in}}%
\pgfpathlineto{\pgfqpoint{3.605678in}{1.809932in}}%
\pgfusepath{stroke}%
\end{pgfscope}%
\begin{pgfscope}%
\definecolor{textcolor}{rgb}{0.000000,0.000000,0.000000}%
\pgfsetstrokecolor{textcolor}%
\pgfsetfillcolor{textcolor}%
\pgftext[x=3.694567in,y=1.771044in,left,base]{\color{textcolor}\sffamily\fontsize{8.000000}{9.600000}\selectfont HTB(LM, BM)}%
\end{pgfscope}%
\begin{pgfscope}%
\pgfsetrectcap%
\pgfsetroundjoin%
\pgfsetlinewidth{1.003750pt}%
\definecolor{currentstroke}{rgb}{0.090196,0.745098,0.811765}%
\pgfsetstrokecolor{currentstroke}%
\pgfsetdash{}{0pt}%
\pgfpathmoveto{\pgfqpoint{3.383456in}{1.646847in}}%
\pgfpathlineto{\pgfqpoint{3.605678in}{1.646847in}}%
\pgfusepath{stroke}%
\end{pgfscope}%
\begin{pgfscope}%
\definecolor{textcolor}{rgb}{0.000000,0.000000,0.000000}%
\pgfsetstrokecolor{textcolor}%
\pgfsetfillcolor{textcolor}%
\pgftext[x=3.694567in,y=1.607958in,left,base]{\color{textcolor}\sffamily\fontsize{8.000000}{9.600000}\selectfont HTB(MF, BE)}%
\end{pgfscope}%
\begin{pgfscope}%
\pgfsetrectcap%
\pgfsetroundjoin%
\pgfsetlinewidth{1.003750pt}%
\definecolor{currentstroke}{rgb}{0.121569,0.466667,0.705882}%
\pgfsetstrokecolor{currentstroke}%
\pgfsetdash{}{0pt}%
\pgfpathmoveto{\pgfqpoint{3.383456in}{1.483761in}}%
\pgfpathlineto{\pgfqpoint{3.605678in}{1.483761in}}%
\pgfusepath{stroke}%
\end{pgfscope}%
\begin{pgfscope}%
\definecolor{textcolor}{rgb}{0.000000,0.000000,0.000000}%
\pgfsetstrokecolor{textcolor}%
\pgfsetfillcolor{textcolor}%
\pgftext[x=3.694567in,y=1.444872in,left,base]{\color{textcolor}\sffamily\fontsize{8.000000}{9.600000}\selectfont HTB(MF, BM)}%
\end{pgfscope}%
\end{pgfpicture}%
\makeatother%
\endgroup%

    \caption{
        Experiment 1 compares the tree-decomposition-based planner \Lg{} to the constraint-satisfaction-based planner \htb{}. 
        \Lg{} can be used with a tree decomposer (\flowcutter{} \cite{strasser2017computing}, \htd{} \cite{AMW17}, or \tamaki{} \cite{Tamaki17}).
        \htb{} requires a variable-ordering heuristic (\mcs{} \cite{tarjan1984simple}, \lexp/\lexm{} \cite{koster2001treewidth}, or \minfill{} \cite{dechter03}) and a clause-ordering heuristic (\be{} \cite{dechter99} or \bm{} \cite{bouquet1999gestion}).
        A planner ``solves'' a benchmark when it eventually finds a project-join tree of width \maxWidth{} or lower.
        \Lg{} is an anytime tool that can output several trees (of decreasing widths) for each benchmark.
        On this plot, for each \Lg{} benchmark, we use the time of the first tree whose width is at most \maxWidth.
        In contrast, in Figure \ref{figPlanning}, we discard an \Lg{} benchmark when the first tree has width over \maxWidth, even if a later tree has width at most \maxWidth.
    }
    \label{figPlanningA}
\end{figure}

%%%%%%%%%%%%%%%%%%%%%%%%%%%%%%%%%%%%%%%%%%%%%%%%%%%%%%%%%%%%%%%%%%%%%%%%%%%%%%%%

\subsection{Experiment 2: Comparing Execution Heuristics}

Figure \ref{figExecutionA} shows the performance of four ADD variable-ordering heuristics with the executor \dmc{} in 100 seconds (execution time only, excluding planning time).
The graded project-join trees here are taken from Experiment 1.
Recall that \Lg{} is an anytime tool that may produce several project-join trees (of decreasing widths) for each benchmark.
We measure the execution time using the first tree and the last tree produced within 100 seconds for each benchmark.
\begin{figure}[H]
    \centering
    %% Creator: Matplotlib, PGF backend
%%
%% To include the figure in your LaTeX document, write
%%   \input{<filename>.pgf}
%%
%% Make sure the required packages are loaded in your preamble
%%   \usepackage{pgf}
%%
%% and, on pdftex
%%   \usepackage[utf8]{inputenc}\DeclareUnicodeCharacter{2212}{-}
%%
%% or, on luatex and xetex
%%   \usepackage{unicode-math}
%%
%% Figures using additional raster images can only be included by \input if
%% they are in the same directory as the main LaTeX file. For loading figures
%% from other directories you can use the `import` package
%%   \usepackage{import}
%%
%% and then include the figures with
%%   \import{<path to file>}{<filename>.pgf}
%%
%% Matplotlib used the following preamble
%%   \usepackage{fontspec}
%%   \setmainfont{DejaVuSerif.ttf}[Path=/home/vhp1/.local/lib/python3.8/site-packages/matplotlib/mpl-data/fonts/ttf/]
%%   \setsansfont{DejaVuSans.ttf}[Path=/home/vhp1/.local/lib/python3.8/site-packages/matplotlib/mpl-data/fonts/ttf/]
%%   \setmonofont{DejaVuSansMono.ttf}[Path=/home/vhp1/.local/lib/python3.8/site-packages/matplotlib/mpl-data/fonts/ttf/]
%%
\begingroup%
\makeatletter%
\begin{pgfpicture}%
\pgfpathrectangle{\pgfpointorigin}{\pgfqpoint{4.820041in}{2.804006in}}%
\pgfusepath{use as bounding box, clip}%
\begin{pgfscope}%
\pgfsetbuttcap%
\pgfsetmiterjoin%
\pgfsetlinewidth{0.000000pt}%
\definecolor{currentstroke}{rgb}{1.000000,1.000000,1.000000}%
\pgfsetstrokecolor{currentstroke}%
\pgfsetstrokeopacity{0.000000}%
\pgfsetdash{}{0pt}%
\pgfpathmoveto{\pgfqpoint{0.000000in}{0.000000in}}%
\pgfpathlineto{\pgfqpoint{4.820041in}{0.000000in}}%
\pgfpathlineto{\pgfqpoint{4.820041in}{2.804006in}}%
\pgfpathlineto{\pgfqpoint{0.000000in}{2.804006in}}%
\pgfpathclose%
\pgfusepath{}%
\end{pgfscope}%
\begin{pgfscope}%
\pgfsetbuttcap%
\pgfsetmiterjoin%
\definecolor{currentfill}{rgb}{1.000000,1.000000,1.000000}%
\pgfsetfillcolor{currentfill}%
\pgfsetlinewidth{0.000000pt}%
\definecolor{currentstroke}{rgb}{0.000000,0.000000,0.000000}%
\pgfsetstrokecolor{currentstroke}%
\pgfsetstrokeopacity{0.000000}%
\pgfsetdash{}{0pt}%
\pgfpathmoveto{\pgfqpoint{0.537394in}{0.467838in}}%
\pgfpathlineto{\pgfqpoint{4.632078in}{0.467838in}}%
\pgfpathlineto{\pgfqpoint{4.632078in}{2.661796in}}%
\pgfpathlineto{\pgfqpoint{0.537394in}{2.661796in}}%
\pgfpathclose%
\pgfusepath{fill}%
\end{pgfscope}%
\begin{pgfscope}%
\pgfsetbuttcap%
\pgfsetroundjoin%
\definecolor{currentfill}{rgb}{0.000000,0.000000,0.000000}%
\pgfsetfillcolor{currentfill}%
\pgfsetlinewidth{0.803000pt}%
\definecolor{currentstroke}{rgb}{0.000000,0.000000,0.000000}%
\pgfsetstrokecolor{currentstroke}%
\pgfsetdash{}{0pt}%
\pgfsys@defobject{currentmarker}{\pgfqpoint{0.000000in}{-0.048611in}}{\pgfqpoint{0.000000in}{0.000000in}}{%
\pgfpathmoveto{\pgfqpoint{0.000000in}{0.000000in}}%
\pgfpathlineto{\pgfqpoint{0.000000in}{-0.048611in}}%
\pgfusepath{stroke,fill}%
}%
\begin{pgfscope}%
\pgfsys@transformshift{0.537394in}{0.467838in}%
\pgfsys@useobject{currentmarker}{}%
\end{pgfscope}%
\end{pgfscope}%
\begin{pgfscope}%
\definecolor{textcolor}{rgb}{0.000000,0.000000,0.000000}%
\pgfsetstrokecolor{textcolor}%
\pgfsetfillcolor{textcolor}%
\pgftext[x=0.537394in,y=0.370616in,,top]{\color{textcolor}\sffamily\fontsize{8.000000}{9.600000}\selectfont \(\displaystyle {10^{-3}}\)}%
\end{pgfscope}%
\begin{pgfscope}%
\pgfsetbuttcap%
\pgfsetroundjoin%
\definecolor{currentfill}{rgb}{0.000000,0.000000,0.000000}%
\pgfsetfillcolor{currentfill}%
\pgfsetlinewidth{0.803000pt}%
\definecolor{currentstroke}{rgb}{0.000000,0.000000,0.000000}%
\pgfsetstrokecolor{currentstroke}%
\pgfsetdash{}{0pt}%
\pgfsys@defobject{currentmarker}{\pgfqpoint{0.000000in}{-0.048611in}}{\pgfqpoint{0.000000in}{0.000000in}}{%
\pgfpathmoveto{\pgfqpoint{0.000000in}{0.000000in}}%
\pgfpathlineto{\pgfqpoint{0.000000in}{-0.048611in}}%
\pgfusepath{stroke,fill}%
}%
\begin{pgfscope}%
\pgfsys@transformshift{1.356331in}{0.467838in}%
\pgfsys@useobject{currentmarker}{}%
\end{pgfscope}%
\end{pgfscope}%
\begin{pgfscope}%
\definecolor{textcolor}{rgb}{0.000000,0.000000,0.000000}%
\pgfsetstrokecolor{textcolor}%
\pgfsetfillcolor{textcolor}%
\pgftext[x=1.356331in,y=0.370616in,,top]{\color{textcolor}\sffamily\fontsize{8.000000}{9.600000}\selectfont \(\displaystyle {10^{-2}}\)}%
\end{pgfscope}%
\begin{pgfscope}%
\pgfsetbuttcap%
\pgfsetroundjoin%
\definecolor{currentfill}{rgb}{0.000000,0.000000,0.000000}%
\pgfsetfillcolor{currentfill}%
\pgfsetlinewidth{0.803000pt}%
\definecolor{currentstroke}{rgb}{0.000000,0.000000,0.000000}%
\pgfsetstrokecolor{currentstroke}%
\pgfsetdash{}{0pt}%
\pgfsys@defobject{currentmarker}{\pgfqpoint{0.000000in}{-0.048611in}}{\pgfqpoint{0.000000in}{0.000000in}}{%
\pgfpathmoveto{\pgfqpoint{0.000000in}{0.000000in}}%
\pgfpathlineto{\pgfqpoint{0.000000in}{-0.048611in}}%
\pgfusepath{stroke,fill}%
}%
\begin{pgfscope}%
\pgfsys@transformshift{2.175268in}{0.467838in}%
\pgfsys@useobject{currentmarker}{}%
\end{pgfscope}%
\end{pgfscope}%
\begin{pgfscope}%
\definecolor{textcolor}{rgb}{0.000000,0.000000,0.000000}%
\pgfsetstrokecolor{textcolor}%
\pgfsetfillcolor{textcolor}%
\pgftext[x=2.175268in,y=0.370616in,,top]{\color{textcolor}\sffamily\fontsize{8.000000}{9.600000}\selectfont \(\displaystyle {10^{-1}}\)}%
\end{pgfscope}%
\begin{pgfscope}%
\pgfsetbuttcap%
\pgfsetroundjoin%
\definecolor{currentfill}{rgb}{0.000000,0.000000,0.000000}%
\pgfsetfillcolor{currentfill}%
\pgfsetlinewidth{0.803000pt}%
\definecolor{currentstroke}{rgb}{0.000000,0.000000,0.000000}%
\pgfsetstrokecolor{currentstroke}%
\pgfsetdash{}{0pt}%
\pgfsys@defobject{currentmarker}{\pgfqpoint{0.000000in}{-0.048611in}}{\pgfqpoint{0.000000in}{0.000000in}}{%
\pgfpathmoveto{\pgfqpoint{0.000000in}{0.000000in}}%
\pgfpathlineto{\pgfqpoint{0.000000in}{-0.048611in}}%
\pgfusepath{stroke,fill}%
}%
\begin{pgfscope}%
\pgfsys@transformshift{2.994204in}{0.467838in}%
\pgfsys@useobject{currentmarker}{}%
\end{pgfscope}%
\end{pgfscope}%
\begin{pgfscope}%
\definecolor{textcolor}{rgb}{0.000000,0.000000,0.000000}%
\pgfsetstrokecolor{textcolor}%
\pgfsetfillcolor{textcolor}%
\pgftext[x=2.994204in,y=0.370616in,,top]{\color{textcolor}\sffamily\fontsize{8.000000}{9.600000}\selectfont \(\displaystyle {10^{0}}\)}%
\end{pgfscope}%
\begin{pgfscope}%
\pgfsetbuttcap%
\pgfsetroundjoin%
\definecolor{currentfill}{rgb}{0.000000,0.000000,0.000000}%
\pgfsetfillcolor{currentfill}%
\pgfsetlinewidth{0.803000pt}%
\definecolor{currentstroke}{rgb}{0.000000,0.000000,0.000000}%
\pgfsetstrokecolor{currentstroke}%
\pgfsetdash{}{0pt}%
\pgfsys@defobject{currentmarker}{\pgfqpoint{0.000000in}{-0.048611in}}{\pgfqpoint{0.000000in}{0.000000in}}{%
\pgfpathmoveto{\pgfqpoint{0.000000in}{0.000000in}}%
\pgfpathlineto{\pgfqpoint{0.000000in}{-0.048611in}}%
\pgfusepath{stroke,fill}%
}%
\begin{pgfscope}%
\pgfsys@transformshift{3.813141in}{0.467838in}%
\pgfsys@useobject{currentmarker}{}%
\end{pgfscope}%
\end{pgfscope}%
\begin{pgfscope}%
\definecolor{textcolor}{rgb}{0.000000,0.000000,0.000000}%
\pgfsetstrokecolor{textcolor}%
\pgfsetfillcolor{textcolor}%
\pgftext[x=3.813141in,y=0.370616in,,top]{\color{textcolor}\sffamily\fontsize{8.000000}{9.600000}\selectfont \(\displaystyle {10^{1}}\)}%
\end{pgfscope}%
\begin{pgfscope}%
\pgfsetbuttcap%
\pgfsetroundjoin%
\definecolor{currentfill}{rgb}{0.000000,0.000000,0.000000}%
\pgfsetfillcolor{currentfill}%
\pgfsetlinewidth{0.803000pt}%
\definecolor{currentstroke}{rgb}{0.000000,0.000000,0.000000}%
\pgfsetstrokecolor{currentstroke}%
\pgfsetdash{}{0pt}%
\pgfsys@defobject{currentmarker}{\pgfqpoint{0.000000in}{-0.048611in}}{\pgfqpoint{0.000000in}{0.000000in}}{%
\pgfpathmoveto{\pgfqpoint{0.000000in}{0.000000in}}%
\pgfpathlineto{\pgfqpoint{0.000000in}{-0.048611in}}%
\pgfusepath{stroke,fill}%
}%
\begin{pgfscope}%
\pgfsys@transformshift{4.632078in}{0.467838in}%
\pgfsys@useobject{currentmarker}{}%
\end{pgfscope}%
\end{pgfscope}%
\begin{pgfscope}%
\definecolor{textcolor}{rgb}{0.000000,0.000000,0.000000}%
\pgfsetstrokecolor{textcolor}%
\pgfsetfillcolor{textcolor}%
\pgftext[x=4.632078in,y=0.370616in,,top]{\color{textcolor}\sffamily\fontsize{8.000000}{9.600000}\selectfont \(\displaystyle {10^{2}}\)}%
\end{pgfscope}%
\begin{pgfscope}%
\pgfsetbuttcap%
\pgfsetroundjoin%
\definecolor{currentfill}{rgb}{0.000000,0.000000,0.000000}%
\pgfsetfillcolor{currentfill}%
\pgfsetlinewidth{0.602250pt}%
\definecolor{currentstroke}{rgb}{0.000000,0.000000,0.000000}%
\pgfsetstrokecolor{currentstroke}%
\pgfsetdash{}{0pt}%
\pgfsys@defobject{currentmarker}{\pgfqpoint{0.000000in}{-0.027778in}}{\pgfqpoint{0.000000in}{0.000000in}}{%
\pgfpathmoveto{\pgfqpoint{0.000000in}{0.000000in}}%
\pgfpathlineto{\pgfqpoint{0.000000in}{-0.027778in}}%
\pgfusepath{stroke,fill}%
}%
\begin{pgfscope}%
\pgfsys@transformshift{0.783918in}{0.467838in}%
\pgfsys@useobject{currentmarker}{}%
\end{pgfscope}%
\end{pgfscope}%
\begin{pgfscope}%
\pgfsetbuttcap%
\pgfsetroundjoin%
\definecolor{currentfill}{rgb}{0.000000,0.000000,0.000000}%
\pgfsetfillcolor{currentfill}%
\pgfsetlinewidth{0.602250pt}%
\definecolor{currentstroke}{rgb}{0.000000,0.000000,0.000000}%
\pgfsetstrokecolor{currentstroke}%
\pgfsetdash{}{0pt}%
\pgfsys@defobject{currentmarker}{\pgfqpoint{0.000000in}{-0.027778in}}{\pgfqpoint{0.000000in}{0.000000in}}{%
\pgfpathmoveto{\pgfqpoint{0.000000in}{0.000000in}}%
\pgfpathlineto{\pgfqpoint{0.000000in}{-0.027778in}}%
\pgfusepath{stroke,fill}%
}%
\begin{pgfscope}%
\pgfsys@transformshift{0.928126in}{0.467838in}%
\pgfsys@useobject{currentmarker}{}%
\end{pgfscope}%
\end{pgfscope}%
\begin{pgfscope}%
\pgfsetbuttcap%
\pgfsetroundjoin%
\definecolor{currentfill}{rgb}{0.000000,0.000000,0.000000}%
\pgfsetfillcolor{currentfill}%
\pgfsetlinewidth{0.602250pt}%
\definecolor{currentstroke}{rgb}{0.000000,0.000000,0.000000}%
\pgfsetstrokecolor{currentstroke}%
\pgfsetdash{}{0pt}%
\pgfsys@defobject{currentmarker}{\pgfqpoint{0.000000in}{-0.027778in}}{\pgfqpoint{0.000000in}{0.000000in}}{%
\pgfpathmoveto{\pgfqpoint{0.000000in}{0.000000in}}%
\pgfpathlineto{\pgfqpoint{0.000000in}{-0.027778in}}%
\pgfusepath{stroke,fill}%
}%
\begin{pgfscope}%
\pgfsys@transformshift{1.030443in}{0.467838in}%
\pgfsys@useobject{currentmarker}{}%
\end{pgfscope}%
\end{pgfscope}%
\begin{pgfscope}%
\pgfsetbuttcap%
\pgfsetroundjoin%
\definecolor{currentfill}{rgb}{0.000000,0.000000,0.000000}%
\pgfsetfillcolor{currentfill}%
\pgfsetlinewidth{0.602250pt}%
\definecolor{currentstroke}{rgb}{0.000000,0.000000,0.000000}%
\pgfsetstrokecolor{currentstroke}%
\pgfsetdash{}{0pt}%
\pgfsys@defobject{currentmarker}{\pgfqpoint{0.000000in}{-0.027778in}}{\pgfqpoint{0.000000in}{0.000000in}}{%
\pgfpathmoveto{\pgfqpoint{0.000000in}{0.000000in}}%
\pgfpathlineto{\pgfqpoint{0.000000in}{-0.027778in}}%
\pgfusepath{stroke,fill}%
}%
\begin{pgfscope}%
\pgfsys@transformshift{1.109806in}{0.467838in}%
\pgfsys@useobject{currentmarker}{}%
\end{pgfscope}%
\end{pgfscope}%
\begin{pgfscope}%
\pgfsetbuttcap%
\pgfsetroundjoin%
\definecolor{currentfill}{rgb}{0.000000,0.000000,0.000000}%
\pgfsetfillcolor{currentfill}%
\pgfsetlinewidth{0.602250pt}%
\definecolor{currentstroke}{rgb}{0.000000,0.000000,0.000000}%
\pgfsetstrokecolor{currentstroke}%
\pgfsetdash{}{0pt}%
\pgfsys@defobject{currentmarker}{\pgfqpoint{0.000000in}{-0.027778in}}{\pgfqpoint{0.000000in}{0.000000in}}{%
\pgfpathmoveto{\pgfqpoint{0.000000in}{0.000000in}}%
\pgfpathlineto{\pgfqpoint{0.000000in}{-0.027778in}}%
\pgfusepath{stroke,fill}%
}%
\begin{pgfscope}%
\pgfsys@transformshift{1.174651in}{0.467838in}%
\pgfsys@useobject{currentmarker}{}%
\end{pgfscope}%
\end{pgfscope}%
\begin{pgfscope}%
\pgfsetbuttcap%
\pgfsetroundjoin%
\definecolor{currentfill}{rgb}{0.000000,0.000000,0.000000}%
\pgfsetfillcolor{currentfill}%
\pgfsetlinewidth{0.602250pt}%
\definecolor{currentstroke}{rgb}{0.000000,0.000000,0.000000}%
\pgfsetstrokecolor{currentstroke}%
\pgfsetdash{}{0pt}%
\pgfsys@defobject{currentmarker}{\pgfqpoint{0.000000in}{-0.027778in}}{\pgfqpoint{0.000000in}{0.000000in}}{%
\pgfpathmoveto{\pgfqpoint{0.000000in}{0.000000in}}%
\pgfpathlineto{\pgfqpoint{0.000000in}{-0.027778in}}%
\pgfusepath{stroke,fill}%
}%
\begin{pgfscope}%
\pgfsys@transformshift{1.229476in}{0.467838in}%
\pgfsys@useobject{currentmarker}{}%
\end{pgfscope}%
\end{pgfscope}%
\begin{pgfscope}%
\pgfsetbuttcap%
\pgfsetroundjoin%
\definecolor{currentfill}{rgb}{0.000000,0.000000,0.000000}%
\pgfsetfillcolor{currentfill}%
\pgfsetlinewidth{0.602250pt}%
\definecolor{currentstroke}{rgb}{0.000000,0.000000,0.000000}%
\pgfsetstrokecolor{currentstroke}%
\pgfsetdash{}{0pt}%
\pgfsys@defobject{currentmarker}{\pgfqpoint{0.000000in}{-0.027778in}}{\pgfqpoint{0.000000in}{0.000000in}}{%
\pgfpathmoveto{\pgfqpoint{0.000000in}{0.000000in}}%
\pgfpathlineto{\pgfqpoint{0.000000in}{-0.027778in}}%
\pgfusepath{stroke,fill}%
}%
\begin{pgfscope}%
\pgfsys@transformshift{1.276968in}{0.467838in}%
\pgfsys@useobject{currentmarker}{}%
\end{pgfscope}%
\end{pgfscope}%
\begin{pgfscope}%
\pgfsetbuttcap%
\pgfsetroundjoin%
\definecolor{currentfill}{rgb}{0.000000,0.000000,0.000000}%
\pgfsetfillcolor{currentfill}%
\pgfsetlinewidth{0.602250pt}%
\definecolor{currentstroke}{rgb}{0.000000,0.000000,0.000000}%
\pgfsetstrokecolor{currentstroke}%
\pgfsetdash{}{0pt}%
\pgfsys@defobject{currentmarker}{\pgfqpoint{0.000000in}{-0.027778in}}{\pgfqpoint{0.000000in}{0.000000in}}{%
\pgfpathmoveto{\pgfqpoint{0.000000in}{0.000000in}}%
\pgfpathlineto{\pgfqpoint{0.000000in}{-0.027778in}}%
\pgfusepath{stroke,fill}%
}%
\begin{pgfscope}%
\pgfsys@transformshift{1.318858in}{0.467838in}%
\pgfsys@useobject{currentmarker}{}%
\end{pgfscope}%
\end{pgfscope}%
\begin{pgfscope}%
\pgfsetbuttcap%
\pgfsetroundjoin%
\definecolor{currentfill}{rgb}{0.000000,0.000000,0.000000}%
\pgfsetfillcolor{currentfill}%
\pgfsetlinewidth{0.602250pt}%
\definecolor{currentstroke}{rgb}{0.000000,0.000000,0.000000}%
\pgfsetstrokecolor{currentstroke}%
\pgfsetdash{}{0pt}%
\pgfsys@defobject{currentmarker}{\pgfqpoint{0.000000in}{-0.027778in}}{\pgfqpoint{0.000000in}{0.000000in}}{%
\pgfpathmoveto{\pgfqpoint{0.000000in}{0.000000in}}%
\pgfpathlineto{\pgfqpoint{0.000000in}{-0.027778in}}%
\pgfusepath{stroke,fill}%
}%
\begin{pgfscope}%
\pgfsys@transformshift{1.602855in}{0.467838in}%
\pgfsys@useobject{currentmarker}{}%
\end{pgfscope}%
\end{pgfscope}%
\begin{pgfscope}%
\pgfsetbuttcap%
\pgfsetroundjoin%
\definecolor{currentfill}{rgb}{0.000000,0.000000,0.000000}%
\pgfsetfillcolor{currentfill}%
\pgfsetlinewidth{0.602250pt}%
\definecolor{currentstroke}{rgb}{0.000000,0.000000,0.000000}%
\pgfsetstrokecolor{currentstroke}%
\pgfsetdash{}{0pt}%
\pgfsys@defobject{currentmarker}{\pgfqpoint{0.000000in}{-0.027778in}}{\pgfqpoint{0.000000in}{0.000000in}}{%
\pgfpathmoveto{\pgfqpoint{0.000000in}{0.000000in}}%
\pgfpathlineto{\pgfqpoint{0.000000in}{-0.027778in}}%
\pgfusepath{stroke,fill}%
}%
\begin{pgfscope}%
\pgfsys@transformshift{1.747063in}{0.467838in}%
\pgfsys@useobject{currentmarker}{}%
\end{pgfscope}%
\end{pgfscope}%
\begin{pgfscope}%
\pgfsetbuttcap%
\pgfsetroundjoin%
\definecolor{currentfill}{rgb}{0.000000,0.000000,0.000000}%
\pgfsetfillcolor{currentfill}%
\pgfsetlinewidth{0.602250pt}%
\definecolor{currentstroke}{rgb}{0.000000,0.000000,0.000000}%
\pgfsetstrokecolor{currentstroke}%
\pgfsetdash{}{0pt}%
\pgfsys@defobject{currentmarker}{\pgfqpoint{0.000000in}{-0.027778in}}{\pgfqpoint{0.000000in}{0.000000in}}{%
\pgfpathmoveto{\pgfqpoint{0.000000in}{0.000000in}}%
\pgfpathlineto{\pgfqpoint{0.000000in}{-0.027778in}}%
\pgfusepath{stroke,fill}%
}%
\begin{pgfscope}%
\pgfsys@transformshift{1.849380in}{0.467838in}%
\pgfsys@useobject{currentmarker}{}%
\end{pgfscope}%
\end{pgfscope}%
\begin{pgfscope}%
\pgfsetbuttcap%
\pgfsetroundjoin%
\definecolor{currentfill}{rgb}{0.000000,0.000000,0.000000}%
\pgfsetfillcolor{currentfill}%
\pgfsetlinewidth{0.602250pt}%
\definecolor{currentstroke}{rgb}{0.000000,0.000000,0.000000}%
\pgfsetstrokecolor{currentstroke}%
\pgfsetdash{}{0pt}%
\pgfsys@defobject{currentmarker}{\pgfqpoint{0.000000in}{-0.027778in}}{\pgfqpoint{0.000000in}{0.000000in}}{%
\pgfpathmoveto{\pgfqpoint{0.000000in}{0.000000in}}%
\pgfpathlineto{\pgfqpoint{0.000000in}{-0.027778in}}%
\pgfusepath{stroke,fill}%
}%
\begin{pgfscope}%
\pgfsys@transformshift{1.928743in}{0.467838in}%
\pgfsys@useobject{currentmarker}{}%
\end{pgfscope}%
\end{pgfscope}%
\begin{pgfscope}%
\pgfsetbuttcap%
\pgfsetroundjoin%
\definecolor{currentfill}{rgb}{0.000000,0.000000,0.000000}%
\pgfsetfillcolor{currentfill}%
\pgfsetlinewidth{0.602250pt}%
\definecolor{currentstroke}{rgb}{0.000000,0.000000,0.000000}%
\pgfsetstrokecolor{currentstroke}%
\pgfsetdash{}{0pt}%
\pgfsys@defobject{currentmarker}{\pgfqpoint{0.000000in}{-0.027778in}}{\pgfqpoint{0.000000in}{0.000000in}}{%
\pgfpathmoveto{\pgfqpoint{0.000000in}{0.000000in}}%
\pgfpathlineto{\pgfqpoint{0.000000in}{-0.027778in}}%
\pgfusepath{stroke,fill}%
}%
\begin{pgfscope}%
\pgfsys@transformshift{1.993587in}{0.467838in}%
\pgfsys@useobject{currentmarker}{}%
\end{pgfscope}%
\end{pgfscope}%
\begin{pgfscope}%
\pgfsetbuttcap%
\pgfsetroundjoin%
\definecolor{currentfill}{rgb}{0.000000,0.000000,0.000000}%
\pgfsetfillcolor{currentfill}%
\pgfsetlinewidth{0.602250pt}%
\definecolor{currentstroke}{rgb}{0.000000,0.000000,0.000000}%
\pgfsetstrokecolor{currentstroke}%
\pgfsetdash{}{0pt}%
\pgfsys@defobject{currentmarker}{\pgfqpoint{0.000000in}{-0.027778in}}{\pgfqpoint{0.000000in}{0.000000in}}{%
\pgfpathmoveto{\pgfqpoint{0.000000in}{0.000000in}}%
\pgfpathlineto{\pgfqpoint{0.000000in}{-0.027778in}}%
\pgfusepath{stroke,fill}%
}%
\begin{pgfscope}%
\pgfsys@transformshift{2.048413in}{0.467838in}%
\pgfsys@useobject{currentmarker}{}%
\end{pgfscope}%
\end{pgfscope}%
\begin{pgfscope}%
\pgfsetbuttcap%
\pgfsetroundjoin%
\definecolor{currentfill}{rgb}{0.000000,0.000000,0.000000}%
\pgfsetfillcolor{currentfill}%
\pgfsetlinewidth{0.602250pt}%
\definecolor{currentstroke}{rgb}{0.000000,0.000000,0.000000}%
\pgfsetstrokecolor{currentstroke}%
\pgfsetdash{}{0pt}%
\pgfsys@defobject{currentmarker}{\pgfqpoint{0.000000in}{-0.027778in}}{\pgfqpoint{0.000000in}{0.000000in}}{%
\pgfpathmoveto{\pgfqpoint{0.000000in}{0.000000in}}%
\pgfpathlineto{\pgfqpoint{0.000000in}{-0.027778in}}%
\pgfusepath{stroke,fill}%
}%
\begin{pgfscope}%
\pgfsys@transformshift{2.095904in}{0.467838in}%
\pgfsys@useobject{currentmarker}{}%
\end{pgfscope}%
\end{pgfscope}%
\begin{pgfscope}%
\pgfsetbuttcap%
\pgfsetroundjoin%
\definecolor{currentfill}{rgb}{0.000000,0.000000,0.000000}%
\pgfsetfillcolor{currentfill}%
\pgfsetlinewidth{0.602250pt}%
\definecolor{currentstroke}{rgb}{0.000000,0.000000,0.000000}%
\pgfsetstrokecolor{currentstroke}%
\pgfsetdash{}{0pt}%
\pgfsys@defobject{currentmarker}{\pgfqpoint{0.000000in}{-0.027778in}}{\pgfqpoint{0.000000in}{0.000000in}}{%
\pgfpathmoveto{\pgfqpoint{0.000000in}{0.000000in}}%
\pgfpathlineto{\pgfqpoint{0.000000in}{-0.027778in}}%
\pgfusepath{stroke,fill}%
}%
\begin{pgfscope}%
\pgfsys@transformshift{2.137795in}{0.467838in}%
\pgfsys@useobject{currentmarker}{}%
\end{pgfscope}%
\end{pgfscope}%
\begin{pgfscope}%
\pgfsetbuttcap%
\pgfsetroundjoin%
\definecolor{currentfill}{rgb}{0.000000,0.000000,0.000000}%
\pgfsetfillcolor{currentfill}%
\pgfsetlinewidth{0.602250pt}%
\definecolor{currentstroke}{rgb}{0.000000,0.000000,0.000000}%
\pgfsetstrokecolor{currentstroke}%
\pgfsetdash{}{0pt}%
\pgfsys@defobject{currentmarker}{\pgfqpoint{0.000000in}{-0.027778in}}{\pgfqpoint{0.000000in}{0.000000in}}{%
\pgfpathmoveto{\pgfqpoint{0.000000in}{0.000000in}}%
\pgfpathlineto{\pgfqpoint{0.000000in}{-0.027778in}}%
\pgfusepath{stroke,fill}%
}%
\begin{pgfscope}%
\pgfsys@transformshift{2.421792in}{0.467838in}%
\pgfsys@useobject{currentmarker}{}%
\end{pgfscope}%
\end{pgfscope}%
\begin{pgfscope}%
\pgfsetbuttcap%
\pgfsetroundjoin%
\definecolor{currentfill}{rgb}{0.000000,0.000000,0.000000}%
\pgfsetfillcolor{currentfill}%
\pgfsetlinewidth{0.602250pt}%
\definecolor{currentstroke}{rgb}{0.000000,0.000000,0.000000}%
\pgfsetstrokecolor{currentstroke}%
\pgfsetdash{}{0pt}%
\pgfsys@defobject{currentmarker}{\pgfqpoint{0.000000in}{-0.027778in}}{\pgfqpoint{0.000000in}{0.000000in}}{%
\pgfpathmoveto{\pgfqpoint{0.000000in}{0.000000in}}%
\pgfpathlineto{\pgfqpoint{0.000000in}{-0.027778in}}%
\pgfusepath{stroke,fill}%
}%
\begin{pgfscope}%
\pgfsys@transformshift{2.566000in}{0.467838in}%
\pgfsys@useobject{currentmarker}{}%
\end{pgfscope}%
\end{pgfscope}%
\begin{pgfscope}%
\pgfsetbuttcap%
\pgfsetroundjoin%
\definecolor{currentfill}{rgb}{0.000000,0.000000,0.000000}%
\pgfsetfillcolor{currentfill}%
\pgfsetlinewidth{0.602250pt}%
\definecolor{currentstroke}{rgb}{0.000000,0.000000,0.000000}%
\pgfsetstrokecolor{currentstroke}%
\pgfsetdash{}{0pt}%
\pgfsys@defobject{currentmarker}{\pgfqpoint{0.000000in}{-0.027778in}}{\pgfqpoint{0.000000in}{0.000000in}}{%
\pgfpathmoveto{\pgfqpoint{0.000000in}{0.000000in}}%
\pgfpathlineto{\pgfqpoint{0.000000in}{-0.027778in}}%
\pgfusepath{stroke,fill}%
}%
\begin{pgfscope}%
\pgfsys@transformshift{2.668317in}{0.467838in}%
\pgfsys@useobject{currentmarker}{}%
\end{pgfscope}%
\end{pgfscope}%
\begin{pgfscope}%
\pgfsetbuttcap%
\pgfsetroundjoin%
\definecolor{currentfill}{rgb}{0.000000,0.000000,0.000000}%
\pgfsetfillcolor{currentfill}%
\pgfsetlinewidth{0.602250pt}%
\definecolor{currentstroke}{rgb}{0.000000,0.000000,0.000000}%
\pgfsetstrokecolor{currentstroke}%
\pgfsetdash{}{0pt}%
\pgfsys@defobject{currentmarker}{\pgfqpoint{0.000000in}{-0.027778in}}{\pgfqpoint{0.000000in}{0.000000in}}{%
\pgfpathmoveto{\pgfqpoint{0.000000in}{0.000000in}}%
\pgfpathlineto{\pgfqpoint{0.000000in}{-0.027778in}}%
\pgfusepath{stroke,fill}%
}%
\begin{pgfscope}%
\pgfsys@transformshift{2.747680in}{0.467838in}%
\pgfsys@useobject{currentmarker}{}%
\end{pgfscope}%
\end{pgfscope}%
\begin{pgfscope}%
\pgfsetbuttcap%
\pgfsetroundjoin%
\definecolor{currentfill}{rgb}{0.000000,0.000000,0.000000}%
\pgfsetfillcolor{currentfill}%
\pgfsetlinewidth{0.602250pt}%
\definecolor{currentstroke}{rgb}{0.000000,0.000000,0.000000}%
\pgfsetstrokecolor{currentstroke}%
\pgfsetdash{}{0pt}%
\pgfsys@defobject{currentmarker}{\pgfqpoint{0.000000in}{-0.027778in}}{\pgfqpoint{0.000000in}{0.000000in}}{%
\pgfpathmoveto{\pgfqpoint{0.000000in}{0.000000in}}%
\pgfpathlineto{\pgfqpoint{0.000000in}{-0.027778in}}%
\pgfusepath{stroke,fill}%
}%
\begin{pgfscope}%
\pgfsys@transformshift{2.812524in}{0.467838in}%
\pgfsys@useobject{currentmarker}{}%
\end{pgfscope}%
\end{pgfscope}%
\begin{pgfscope}%
\pgfsetbuttcap%
\pgfsetroundjoin%
\definecolor{currentfill}{rgb}{0.000000,0.000000,0.000000}%
\pgfsetfillcolor{currentfill}%
\pgfsetlinewidth{0.602250pt}%
\definecolor{currentstroke}{rgb}{0.000000,0.000000,0.000000}%
\pgfsetstrokecolor{currentstroke}%
\pgfsetdash{}{0pt}%
\pgfsys@defobject{currentmarker}{\pgfqpoint{0.000000in}{-0.027778in}}{\pgfqpoint{0.000000in}{0.000000in}}{%
\pgfpathmoveto{\pgfqpoint{0.000000in}{0.000000in}}%
\pgfpathlineto{\pgfqpoint{0.000000in}{-0.027778in}}%
\pgfusepath{stroke,fill}%
}%
\begin{pgfscope}%
\pgfsys@transformshift{2.867349in}{0.467838in}%
\pgfsys@useobject{currentmarker}{}%
\end{pgfscope}%
\end{pgfscope}%
\begin{pgfscope}%
\pgfsetbuttcap%
\pgfsetroundjoin%
\definecolor{currentfill}{rgb}{0.000000,0.000000,0.000000}%
\pgfsetfillcolor{currentfill}%
\pgfsetlinewidth{0.602250pt}%
\definecolor{currentstroke}{rgb}{0.000000,0.000000,0.000000}%
\pgfsetstrokecolor{currentstroke}%
\pgfsetdash{}{0pt}%
\pgfsys@defobject{currentmarker}{\pgfqpoint{0.000000in}{-0.027778in}}{\pgfqpoint{0.000000in}{0.000000in}}{%
\pgfpathmoveto{\pgfqpoint{0.000000in}{0.000000in}}%
\pgfpathlineto{\pgfqpoint{0.000000in}{-0.027778in}}%
\pgfusepath{stroke,fill}%
}%
\begin{pgfscope}%
\pgfsys@transformshift{2.914841in}{0.467838in}%
\pgfsys@useobject{currentmarker}{}%
\end{pgfscope}%
\end{pgfscope}%
\begin{pgfscope}%
\pgfsetbuttcap%
\pgfsetroundjoin%
\definecolor{currentfill}{rgb}{0.000000,0.000000,0.000000}%
\pgfsetfillcolor{currentfill}%
\pgfsetlinewidth{0.602250pt}%
\definecolor{currentstroke}{rgb}{0.000000,0.000000,0.000000}%
\pgfsetstrokecolor{currentstroke}%
\pgfsetdash{}{0pt}%
\pgfsys@defobject{currentmarker}{\pgfqpoint{0.000000in}{-0.027778in}}{\pgfqpoint{0.000000in}{0.000000in}}{%
\pgfpathmoveto{\pgfqpoint{0.000000in}{0.000000in}}%
\pgfpathlineto{\pgfqpoint{0.000000in}{-0.027778in}}%
\pgfusepath{stroke,fill}%
}%
\begin{pgfscope}%
\pgfsys@transformshift{2.956732in}{0.467838in}%
\pgfsys@useobject{currentmarker}{}%
\end{pgfscope}%
\end{pgfscope}%
\begin{pgfscope}%
\pgfsetbuttcap%
\pgfsetroundjoin%
\definecolor{currentfill}{rgb}{0.000000,0.000000,0.000000}%
\pgfsetfillcolor{currentfill}%
\pgfsetlinewidth{0.602250pt}%
\definecolor{currentstroke}{rgb}{0.000000,0.000000,0.000000}%
\pgfsetstrokecolor{currentstroke}%
\pgfsetdash{}{0pt}%
\pgfsys@defobject{currentmarker}{\pgfqpoint{0.000000in}{-0.027778in}}{\pgfqpoint{0.000000in}{0.000000in}}{%
\pgfpathmoveto{\pgfqpoint{0.000000in}{0.000000in}}%
\pgfpathlineto{\pgfqpoint{0.000000in}{-0.027778in}}%
\pgfusepath{stroke,fill}%
}%
\begin{pgfscope}%
\pgfsys@transformshift{3.240729in}{0.467838in}%
\pgfsys@useobject{currentmarker}{}%
\end{pgfscope}%
\end{pgfscope}%
\begin{pgfscope}%
\pgfsetbuttcap%
\pgfsetroundjoin%
\definecolor{currentfill}{rgb}{0.000000,0.000000,0.000000}%
\pgfsetfillcolor{currentfill}%
\pgfsetlinewidth{0.602250pt}%
\definecolor{currentstroke}{rgb}{0.000000,0.000000,0.000000}%
\pgfsetstrokecolor{currentstroke}%
\pgfsetdash{}{0pt}%
\pgfsys@defobject{currentmarker}{\pgfqpoint{0.000000in}{-0.027778in}}{\pgfqpoint{0.000000in}{0.000000in}}{%
\pgfpathmoveto{\pgfqpoint{0.000000in}{0.000000in}}%
\pgfpathlineto{\pgfqpoint{0.000000in}{-0.027778in}}%
\pgfusepath{stroke,fill}%
}%
\begin{pgfscope}%
\pgfsys@transformshift{3.384936in}{0.467838in}%
\pgfsys@useobject{currentmarker}{}%
\end{pgfscope}%
\end{pgfscope}%
\begin{pgfscope}%
\pgfsetbuttcap%
\pgfsetroundjoin%
\definecolor{currentfill}{rgb}{0.000000,0.000000,0.000000}%
\pgfsetfillcolor{currentfill}%
\pgfsetlinewidth{0.602250pt}%
\definecolor{currentstroke}{rgb}{0.000000,0.000000,0.000000}%
\pgfsetstrokecolor{currentstroke}%
\pgfsetdash{}{0pt}%
\pgfsys@defobject{currentmarker}{\pgfqpoint{0.000000in}{-0.027778in}}{\pgfqpoint{0.000000in}{0.000000in}}{%
\pgfpathmoveto{\pgfqpoint{0.000000in}{0.000000in}}%
\pgfpathlineto{\pgfqpoint{0.000000in}{-0.027778in}}%
\pgfusepath{stroke,fill}%
}%
\begin{pgfscope}%
\pgfsys@transformshift{3.487253in}{0.467838in}%
\pgfsys@useobject{currentmarker}{}%
\end{pgfscope}%
\end{pgfscope}%
\begin{pgfscope}%
\pgfsetbuttcap%
\pgfsetroundjoin%
\definecolor{currentfill}{rgb}{0.000000,0.000000,0.000000}%
\pgfsetfillcolor{currentfill}%
\pgfsetlinewidth{0.602250pt}%
\definecolor{currentstroke}{rgb}{0.000000,0.000000,0.000000}%
\pgfsetstrokecolor{currentstroke}%
\pgfsetdash{}{0pt}%
\pgfsys@defobject{currentmarker}{\pgfqpoint{0.000000in}{-0.027778in}}{\pgfqpoint{0.000000in}{0.000000in}}{%
\pgfpathmoveto{\pgfqpoint{0.000000in}{0.000000in}}%
\pgfpathlineto{\pgfqpoint{0.000000in}{-0.027778in}}%
\pgfusepath{stroke,fill}%
}%
\begin{pgfscope}%
\pgfsys@transformshift{3.566617in}{0.467838in}%
\pgfsys@useobject{currentmarker}{}%
\end{pgfscope}%
\end{pgfscope}%
\begin{pgfscope}%
\pgfsetbuttcap%
\pgfsetroundjoin%
\definecolor{currentfill}{rgb}{0.000000,0.000000,0.000000}%
\pgfsetfillcolor{currentfill}%
\pgfsetlinewidth{0.602250pt}%
\definecolor{currentstroke}{rgb}{0.000000,0.000000,0.000000}%
\pgfsetstrokecolor{currentstroke}%
\pgfsetdash{}{0pt}%
\pgfsys@defobject{currentmarker}{\pgfqpoint{0.000000in}{-0.027778in}}{\pgfqpoint{0.000000in}{0.000000in}}{%
\pgfpathmoveto{\pgfqpoint{0.000000in}{0.000000in}}%
\pgfpathlineto{\pgfqpoint{0.000000in}{-0.027778in}}%
\pgfusepath{stroke,fill}%
}%
\begin{pgfscope}%
\pgfsys@transformshift{3.631461in}{0.467838in}%
\pgfsys@useobject{currentmarker}{}%
\end{pgfscope}%
\end{pgfscope}%
\begin{pgfscope}%
\pgfsetbuttcap%
\pgfsetroundjoin%
\definecolor{currentfill}{rgb}{0.000000,0.000000,0.000000}%
\pgfsetfillcolor{currentfill}%
\pgfsetlinewidth{0.602250pt}%
\definecolor{currentstroke}{rgb}{0.000000,0.000000,0.000000}%
\pgfsetstrokecolor{currentstroke}%
\pgfsetdash{}{0pt}%
\pgfsys@defobject{currentmarker}{\pgfqpoint{0.000000in}{-0.027778in}}{\pgfqpoint{0.000000in}{0.000000in}}{%
\pgfpathmoveto{\pgfqpoint{0.000000in}{0.000000in}}%
\pgfpathlineto{\pgfqpoint{0.000000in}{-0.027778in}}%
\pgfusepath{stroke,fill}%
}%
\begin{pgfscope}%
\pgfsys@transformshift{3.686286in}{0.467838in}%
\pgfsys@useobject{currentmarker}{}%
\end{pgfscope}%
\end{pgfscope}%
\begin{pgfscope}%
\pgfsetbuttcap%
\pgfsetroundjoin%
\definecolor{currentfill}{rgb}{0.000000,0.000000,0.000000}%
\pgfsetfillcolor{currentfill}%
\pgfsetlinewidth{0.602250pt}%
\definecolor{currentstroke}{rgb}{0.000000,0.000000,0.000000}%
\pgfsetstrokecolor{currentstroke}%
\pgfsetdash{}{0pt}%
\pgfsys@defobject{currentmarker}{\pgfqpoint{0.000000in}{-0.027778in}}{\pgfqpoint{0.000000in}{0.000000in}}{%
\pgfpathmoveto{\pgfqpoint{0.000000in}{0.000000in}}%
\pgfpathlineto{\pgfqpoint{0.000000in}{-0.027778in}}%
\pgfusepath{stroke,fill}%
}%
\begin{pgfscope}%
\pgfsys@transformshift{3.733778in}{0.467838in}%
\pgfsys@useobject{currentmarker}{}%
\end{pgfscope}%
\end{pgfscope}%
\begin{pgfscope}%
\pgfsetbuttcap%
\pgfsetroundjoin%
\definecolor{currentfill}{rgb}{0.000000,0.000000,0.000000}%
\pgfsetfillcolor{currentfill}%
\pgfsetlinewidth{0.602250pt}%
\definecolor{currentstroke}{rgb}{0.000000,0.000000,0.000000}%
\pgfsetstrokecolor{currentstroke}%
\pgfsetdash{}{0pt}%
\pgfsys@defobject{currentmarker}{\pgfqpoint{0.000000in}{-0.027778in}}{\pgfqpoint{0.000000in}{0.000000in}}{%
\pgfpathmoveto{\pgfqpoint{0.000000in}{0.000000in}}%
\pgfpathlineto{\pgfqpoint{0.000000in}{-0.027778in}}%
\pgfusepath{stroke,fill}%
}%
\begin{pgfscope}%
\pgfsys@transformshift{3.775669in}{0.467838in}%
\pgfsys@useobject{currentmarker}{}%
\end{pgfscope}%
\end{pgfscope}%
\begin{pgfscope}%
\pgfsetbuttcap%
\pgfsetroundjoin%
\definecolor{currentfill}{rgb}{0.000000,0.000000,0.000000}%
\pgfsetfillcolor{currentfill}%
\pgfsetlinewidth{0.602250pt}%
\definecolor{currentstroke}{rgb}{0.000000,0.000000,0.000000}%
\pgfsetstrokecolor{currentstroke}%
\pgfsetdash{}{0pt}%
\pgfsys@defobject{currentmarker}{\pgfqpoint{0.000000in}{-0.027778in}}{\pgfqpoint{0.000000in}{0.000000in}}{%
\pgfpathmoveto{\pgfqpoint{0.000000in}{0.000000in}}%
\pgfpathlineto{\pgfqpoint{0.000000in}{-0.027778in}}%
\pgfusepath{stroke,fill}%
}%
\begin{pgfscope}%
\pgfsys@transformshift{4.059666in}{0.467838in}%
\pgfsys@useobject{currentmarker}{}%
\end{pgfscope}%
\end{pgfscope}%
\begin{pgfscope}%
\pgfsetbuttcap%
\pgfsetroundjoin%
\definecolor{currentfill}{rgb}{0.000000,0.000000,0.000000}%
\pgfsetfillcolor{currentfill}%
\pgfsetlinewidth{0.602250pt}%
\definecolor{currentstroke}{rgb}{0.000000,0.000000,0.000000}%
\pgfsetstrokecolor{currentstroke}%
\pgfsetdash{}{0pt}%
\pgfsys@defobject{currentmarker}{\pgfqpoint{0.000000in}{-0.027778in}}{\pgfqpoint{0.000000in}{0.000000in}}{%
\pgfpathmoveto{\pgfqpoint{0.000000in}{0.000000in}}%
\pgfpathlineto{\pgfqpoint{0.000000in}{-0.027778in}}%
\pgfusepath{stroke,fill}%
}%
\begin{pgfscope}%
\pgfsys@transformshift{4.203873in}{0.467838in}%
\pgfsys@useobject{currentmarker}{}%
\end{pgfscope}%
\end{pgfscope}%
\begin{pgfscope}%
\pgfsetbuttcap%
\pgfsetroundjoin%
\definecolor{currentfill}{rgb}{0.000000,0.000000,0.000000}%
\pgfsetfillcolor{currentfill}%
\pgfsetlinewidth{0.602250pt}%
\definecolor{currentstroke}{rgb}{0.000000,0.000000,0.000000}%
\pgfsetstrokecolor{currentstroke}%
\pgfsetdash{}{0pt}%
\pgfsys@defobject{currentmarker}{\pgfqpoint{0.000000in}{-0.027778in}}{\pgfqpoint{0.000000in}{0.000000in}}{%
\pgfpathmoveto{\pgfqpoint{0.000000in}{0.000000in}}%
\pgfpathlineto{\pgfqpoint{0.000000in}{-0.027778in}}%
\pgfusepath{stroke,fill}%
}%
\begin{pgfscope}%
\pgfsys@transformshift{4.306190in}{0.467838in}%
\pgfsys@useobject{currentmarker}{}%
\end{pgfscope}%
\end{pgfscope}%
\begin{pgfscope}%
\pgfsetbuttcap%
\pgfsetroundjoin%
\definecolor{currentfill}{rgb}{0.000000,0.000000,0.000000}%
\pgfsetfillcolor{currentfill}%
\pgfsetlinewidth{0.602250pt}%
\definecolor{currentstroke}{rgb}{0.000000,0.000000,0.000000}%
\pgfsetstrokecolor{currentstroke}%
\pgfsetdash{}{0pt}%
\pgfsys@defobject{currentmarker}{\pgfqpoint{0.000000in}{-0.027778in}}{\pgfqpoint{0.000000in}{0.000000in}}{%
\pgfpathmoveto{\pgfqpoint{0.000000in}{0.000000in}}%
\pgfpathlineto{\pgfqpoint{0.000000in}{-0.027778in}}%
\pgfusepath{stroke,fill}%
}%
\begin{pgfscope}%
\pgfsys@transformshift{4.385553in}{0.467838in}%
\pgfsys@useobject{currentmarker}{}%
\end{pgfscope}%
\end{pgfscope}%
\begin{pgfscope}%
\pgfsetbuttcap%
\pgfsetroundjoin%
\definecolor{currentfill}{rgb}{0.000000,0.000000,0.000000}%
\pgfsetfillcolor{currentfill}%
\pgfsetlinewidth{0.602250pt}%
\definecolor{currentstroke}{rgb}{0.000000,0.000000,0.000000}%
\pgfsetstrokecolor{currentstroke}%
\pgfsetdash{}{0pt}%
\pgfsys@defobject{currentmarker}{\pgfqpoint{0.000000in}{-0.027778in}}{\pgfqpoint{0.000000in}{0.000000in}}{%
\pgfpathmoveto{\pgfqpoint{0.000000in}{0.000000in}}%
\pgfpathlineto{\pgfqpoint{0.000000in}{-0.027778in}}%
\pgfusepath{stroke,fill}%
}%
\begin{pgfscope}%
\pgfsys@transformshift{4.450398in}{0.467838in}%
\pgfsys@useobject{currentmarker}{}%
\end{pgfscope}%
\end{pgfscope}%
\begin{pgfscope}%
\pgfsetbuttcap%
\pgfsetroundjoin%
\definecolor{currentfill}{rgb}{0.000000,0.000000,0.000000}%
\pgfsetfillcolor{currentfill}%
\pgfsetlinewidth{0.602250pt}%
\definecolor{currentstroke}{rgb}{0.000000,0.000000,0.000000}%
\pgfsetstrokecolor{currentstroke}%
\pgfsetdash{}{0pt}%
\pgfsys@defobject{currentmarker}{\pgfqpoint{0.000000in}{-0.027778in}}{\pgfqpoint{0.000000in}{0.000000in}}{%
\pgfpathmoveto{\pgfqpoint{0.000000in}{0.000000in}}%
\pgfpathlineto{\pgfqpoint{0.000000in}{-0.027778in}}%
\pgfusepath{stroke,fill}%
}%
\begin{pgfscope}%
\pgfsys@transformshift{4.505223in}{0.467838in}%
\pgfsys@useobject{currentmarker}{}%
\end{pgfscope}%
\end{pgfscope}%
\begin{pgfscope}%
\pgfsetbuttcap%
\pgfsetroundjoin%
\definecolor{currentfill}{rgb}{0.000000,0.000000,0.000000}%
\pgfsetfillcolor{currentfill}%
\pgfsetlinewidth{0.602250pt}%
\definecolor{currentstroke}{rgb}{0.000000,0.000000,0.000000}%
\pgfsetstrokecolor{currentstroke}%
\pgfsetdash{}{0pt}%
\pgfsys@defobject{currentmarker}{\pgfqpoint{0.000000in}{-0.027778in}}{\pgfqpoint{0.000000in}{0.000000in}}{%
\pgfpathmoveto{\pgfqpoint{0.000000in}{0.000000in}}%
\pgfpathlineto{\pgfqpoint{0.000000in}{-0.027778in}}%
\pgfusepath{stroke,fill}%
}%
\begin{pgfscope}%
\pgfsys@transformshift{4.552715in}{0.467838in}%
\pgfsys@useobject{currentmarker}{}%
\end{pgfscope}%
\end{pgfscope}%
\begin{pgfscope}%
\pgfsetbuttcap%
\pgfsetroundjoin%
\definecolor{currentfill}{rgb}{0.000000,0.000000,0.000000}%
\pgfsetfillcolor{currentfill}%
\pgfsetlinewidth{0.602250pt}%
\definecolor{currentstroke}{rgb}{0.000000,0.000000,0.000000}%
\pgfsetstrokecolor{currentstroke}%
\pgfsetdash{}{0pt}%
\pgfsys@defobject{currentmarker}{\pgfqpoint{0.000000in}{-0.027778in}}{\pgfqpoint{0.000000in}{0.000000in}}{%
\pgfpathmoveto{\pgfqpoint{0.000000in}{0.000000in}}%
\pgfpathlineto{\pgfqpoint{0.000000in}{-0.027778in}}%
\pgfusepath{stroke,fill}%
}%
\begin{pgfscope}%
\pgfsys@transformshift{4.594605in}{0.467838in}%
\pgfsys@useobject{currentmarker}{}%
\end{pgfscope}%
\end{pgfscope}%
\begin{pgfscope}%
\definecolor{textcolor}{rgb}{0.000000,0.000000,0.000000}%
\pgfsetstrokecolor{textcolor}%
\pgfsetfillcolor{textcolor}%
\pgftext[x=2.584736in,y=0.207530in,,top]{\color{textcolor}\sffamily\fontsize{8.000000}{9.600000}\selectfont Longest solving time (seconds)}%
\end{pgfscope}%
\begin{pgfscope}%
\pgfsetbuttcap%
\pgfsetroundjoin%
\definecolor{currentfill}{rgb}{0.000000,0.000000,0.000000}%
\pgfsetfillcolor{currentfill}%
\pgfsetlinewidth{0.803000pt}%
\definecolor{currentstroke}{rgb}{0.000000,0.000000,0.000000}%
\pgfsetstrokecolor{currentstroke}%
\pgfsetdash{}{0pt}%
\pgfsys@defobject{currentmarker}{\pgfqpoint{-0.048611in}{0.000000in}}{\pgfqpoint{-0.000000in}{0.000000in}}{%
\pgfpathmoveto{\pgfqpoint{-0.000000in}{0.000000in}}%
\pgfpathlineto{\pgfqpoint{-0.048611in}{0.000000in}}%
\pgfusepath{stroke,fill}%
}%
\begin{pgfscope}%
\pgfsys@transformshift{0.537394in}{0.467838in}%
\pgfsys@useobject{currentmarker}{}%
\end{pgfscope}%
\end{pgfscope}%
\begin{pgfscope}%
\definecolor{textcolor}{rgb}{0.000000,0.000000,0.000000}%
\pgfsetstrokecolor{textcolor}%
\pgfsetfillcolor{textcolor}%
\pgftext[x=0.381143in, y=0.425629in, left, base]{\color{textcolor}\sffamily\fontsize{8.000000}{9.600000}\selectfont \(\displaystyle {0}\)}%
\end{pgfscope}%
\begin{pgfscope}%
\pgfsetbuttcap%
\pgfsetroundjoin%
\definecolor{currentfill}{rgb}{0.000000,0.000000,0.000000}%
\pgfsetfillcolor{currentfill}%
\pgfsetlinewidth{0.803000pt}%
\definecolor{currentstroke}{rgb}{0.000000,0.000000,0.000000}%
\pgfsetstrokecolor{currentstroke}%
\pgfsetdash{}{0pt}%
\pgfsys@defobject{currentmarker}{\pgfqpoint{-0.048611in}{0.000000in}}{\pgfqpoint{-0.000000in}{0.000000in}}{%
\pgfpathmoveto{\pgfqpoint{-0.000000in}{0.000000in}}%
\pgfpathlineto{\pgfqpoint{-0.048611in}{0.000000in}}%
\pgfusepath{stroke,fill}%
}%
\begin{pgfscope}%
\pgfsys@transformshift{0.537394in}{0.742083in}%
\pgfsys@useobject{currentmarker}{}%
\end{pgfscope}%
\end{pgfscope}%
\begin{pgfscope}%
\definecolor{textcolor}{rgb}{0.000000,0.000000,0.000000}%
\pgfsetstrokecolor{textcolor}%
\pgfsetfillcolor{textcolor}%
\pgftext[x=0.322114in, y=0.699874in, left, base]{\color{textcolor}\sffamily\fontsize{8.000000}{9.600000}\selectfont \(\displaystyle {50}\)}%
\end{pgfscope}%
\begin{pgfscope}%
\pgfsetbuttcap%
\pgfsetroundjoin%
\definecolor{currentfill}{rgb}{0.000000,0.000000,0.000000}%
\pgfsetfillcolor{currentfill}%
\pgfsetlinewidth{0.803000pt}%
\definecolor{currentstroke}{rgb}{0.000000,0.000000,0.000000}%
\pgfsetstrokecolor{currentstroke}%
\pgfsetdash{}{0pt}%
\pgfsys@defobject{currentmarker}{\pgfqpoint{-0.048611in}{0.000000in}}{\pgfqpoint{-0.000000in}{0.000000in}}{%
\pgfpathmoveto{\pgfqpoint{-0.000000in}{0.000000in}}%
\pgfpathlineto{\pgfqpoint{-0.048611in}{0.000000in}}%
\pgfusepath{stroke,fill}%
}%
\begin{pgfscope}%
\pgfsys@transformshift{0.537394in}{1.016328in}%
\pgfsys@useobject{currentmarker}{}%
\end{pgfscope}%
\end{pgfscope}%
\begin{pgfscope}%
\definecolor{textcolor}{rgb}{0.000000,0.000000,0.000000}%
\pgfsetstrokecolor{textcolor}%
\pgfsetfillcolor{textcolor}%
\pgftext[x=0.263086in, y=0.974119in, left, base]{\color{textcolor}\sffamily\fontsize{8.000000}{9.600000}\selectfont \(\displaystyle {100}\)}%
\end{pgfscope}%
\begin{pgfscope}%
\pgfsetbuttcap%
\pgfsetroundjoin%
\definecolor{currentfill}{rgb}{0.000000,0.000000,0.000000}%
\pgfsetfillcolor{currentfill}%
\pgfsetlinewidth{0.803000pt}%
\definecolor{currentstroke}{rgb}{0.000000,0.000000,0.000000}%
\pgfsetstrokecolor{currentstroke}%
\pgfsetdash{}{0pt}%
\pgfsys@defobject{currentmarker}{\pgfqpoint{-0.048611in}{0.000000in}}{\pgfqpoint{-0.000000in}{0.000000in}}{%
\pgfpathmoveto{\pgfqpoint{-0.000000in}{0.000000in}}%
\pgfpathlineto{\pgfqpoint{-0.048611in}{0.000000in}}%
\pgfusepath{stroke,fill}%
}%
\begin{pgfscope}%
\pgfsys@transformshift{0.537394in}{1.290573in}%
\pgfsys@useobject{currentmarker}{}%
\end{pgfscope}%
\end{pgfscope}%
\begin{pgfscope}%
\definecolor{textcolor}{rgb}{0.000000,0.000000,0.000000}%
\pgfsetstrokecolor{textcolor}%
\pgfsetfillcolor{textcolor}%
\pgftext[x=0.263086in, y=1.248363in, left, base]{\color{textcolor}\sffamily\fontsize{8.000000}{9.600000}\selectfont \(\displaystyle {150}\)}%
\end{pgfscope}%
\begin{pgfscope}%
\pgfsetbuttcap%
\pgfsetroundjoin%
\definecolor{currentfill}{rgb}{0.000000,0.000000,0.000000}%
\pgfsetfillcolor{currentfill}%
\pgfsetlinewidth{0.803000pt}%
\definecolor{currentstroke}{rgb}{0.000000,0.000000,0.000000}%
\pgfsetstrokecolor{currentstroke}%
\pgfsetdash{}{0pt}%
\pgfsys@defobject{currentmarker}{\pgfqpoint{-0.048611in}{0.000000in}}{\pgfqpoint{-0.000000in}{0.000000in}}{%
\pgfpathmoveto{\pgfqpoint{-0.000000in}{0.000000in}}%
\pgfpathlineto{\pgfqpoint{-0.048611in}{0.000000in}}%
\pgfusepath{stroke,fill}%
}%
\begin{pgfscope}%
\pgfsys@transformshift{0.537394in}{1.564817in}%
\pgfsys@useobject{currentmarker}{}%
\end{pgfscope}%
\end{pgfscope}%
\begin{pgfscope}%
\definecolor{textcolor}{rgb}{0.000000,0.000000,0.000000}%
\pgfsetstrokecolor{textcolor}%
\pgfsetfillcolor{textcolor}%
\pgftext[x=0.263086in, y=1.522608in, left, base]{\color{textcolor}\sffamily\fontsize{8.000000}{9.600000}\selectfont \(\displaystyle {200}\)}%
\end{pgfscope}%
\begin{pgfscope}%
\pgfsetbuttcap%
\pgfsetroundjoin%
\definecolor{currentfill}{rgb}{0.000000,0.000000,0.000000}%
\pgfsetfillcolor{currentfill}%
\pgfsetlinewidth{0.803000pt}%
\definecolor{currentstroke}{rgb}{0.000000,0.000000,0.000000}%
\pgfsetstrokecolor{currentstroke}%
\pgfsetdash{}{0pt}%
\pgfsys@defobject{currentmarker}{\pgfqpoint{-0.048611in}{0.000000in}}{\pgfqpoint{-0.000000in}{0.000000in}}{%
\pgfpathmoveto{\pgfqpoint{-0.000000in}{0.000000in}}%
\pgfpathlineto{\pgfqpoint{-0.048611in}{0.000000in}}%
\pgfusepath{stroke,fill}%
}%
\begin{pgfscope}%
\pgfsys@transformshift{0.537394in}{1.839062in}%
\pgfsys@useobject{currentmarker}{}%
\end{pgfscope}%
\end{pgfscope}%
\begin{pgfscope}%
\definecolor{textcolor}{rgb}{0.000000,0.000000,0.000000}%
\pgfsetstrokecolor{textcolor}%
\pgfsetfillcolor{textcolor}%
\pgftext[x=0.263086in, y=1.796853in, left, base]{\color{textcolor}\sffamily\fontsize{8.000000}{9.600000}\selectfont \(\displaystyle {250}\)}%
\end{pgfscope}%
\begin{pgfscope}%
\pgfsetbuttcap%
\pgfsetroundjoin%
\definecolor{currentfill}{rgb}{0.000000,0.000000,0.000000}%
\pgfsetfillcolor{currentfill}%
\pgfsetlinewidth{0.803000pt}%
\definecolor{currentstroke}{rgb}{0.000000,0.000000,0.000000}%
\pgfsetstrokecolor{currentstroke}%
\pgfsetdash{}{0pt}%
\pgfsys@defobject{currentmarker}{\pgfqpoint{-0.048611in}{0.000000in}}{\pgfqpoint{-0.000000in}{0.000000in}}{%
\pgfpathmoveto{\pgfqpoint{-0.000000in}{0.000000in}}%
\pgfpathlineto{\pgfqpoint{-0.048611in}{0.000000in}}%
\pgfusepath{stroke,fill}%
}%
\begin{pgfscope}%
\pgfsys@transformshift{0.537394in}{2.113307in}%
\pgfsys@useobject{currentmarker}{}%
\end{pgfscope}%
\end{pgfscope}%
\begin{pgfscope}%
\definecolor{textcolor}{rgb}{0.000000,0.000000,0.000000}%
\pgfsetstrokecolor{textcolor}%
\pgfsetfillcolor{textcolor}%
\pgftext[x=0.263086in, y=2.071098in, left, base]{\color{textcolor}\sffamily\fontsize{8.000000}{9.600000}\selectfont \(\displaystyle {300}\)}%
\end{pgfscope}%
\begin{pgfscope}%
\pgfsetbuttcap%
\pgfsetroundjoin%
\definecolor{currentfill}{rgb}{0.000000,0.000000,0.000000}%
\pgfsetfillcolor{currentfill}%
\pgfsetlinewidth{0.803000pt}%
\definecolor{currentstroke}{rgb}{0.000000,0.000000,0.000000}%
\pgfsetstrokecolor{currentstroke}%
\pgfsetdash{}{0pt}%
\pgfsys@defobject{currentmarker}{\pgfqpoint{-0.048611in}{0.000000in}}{\pgfqpoint{-0.000000in}{0.000000in}}{%
\pgfpathmoveto{\pgfqpoint{-0.000000in}{0.000000in}}%
\pgfpathlineto{\pgfqpoint{-0.048611in}{0.000000in}}%
\pgfusepath{stroke,fill}%
}%
\begin{pgfscope}%
\pgfsys@transformshift{0.537394in}{2.387552in}%
\pgfsys@useobject{currentmarker}{}%
\end{pgfscope}%
\end{pgfscope}%
\begin{pgfscope}%
\definecolor{textcolor}{rgb}{0.000000,0.000000,0.000000}%
\pgfsetstrokecolor{textcolor}%
\pgfsetfillcolor{textcolor}%
\pgftext[x=0.263086in, y=2.345342in, left, base]{\color{textcolor}\sffamily\fontsize{8.000000}{9.600000}\selectfont \(\displaystyle {350}\)}%
\end{pgfscope}%
\begin{pgfscope}%
\pgfsetbuttcap%
\pgfsetroundjoin%
\definecolor{currentfill}{rgb}{0.000000,0.000000,0.000000}%
\pgfsetfillcolor{currentfill}%
\pgfsetlinewidth{0.803000pt}%
\definecolor{currentstroke}{rgb}{0.000000,0.000000,0.000000}%
\pgfsetstrokecolor{currentstroke}%
\pgfsetdash{}{0pt}%
\pgfsys@defobject{currentmarker}{\pgfqpoint{-0.048611in}{0.000000in}}{\pgfqpoint{-0.000000in}{0.000000in}}{%
\pgfpathmoveto{\pgfqpoint{-0.000000in}{0.000000in}}%
\pgfpathlineto{\pgfqpoint{-0.048611in}{0.000000in}}%
\pgfusepath{stroke,fill}%
}%
\begin{pgfscope}%
\pgfsys@transformshift{0.537394in}{2.661796in}%
\pgfsys@useobject{currentmarker}{}%
\end{pgfscope}%
\end{pgfscope}%
\begin{pgfscope}%
\definecolor{textcolor}{rgb}{0.000000,0.000000,0.000000}%
\pgfsetstrokecolor{textcolor}%
\pgfsetfillcolor{textcolor}%
\pgftext[x=0.263086in, y=2.619587in, left, base]{\color{textcolor}\sffamily\fontsize{8.000000}{9.600000}\selectfont \(\displaystyle {400}\)}%
\end{pgfscope}%
\begin{pgfscope}%
\definecolor{textcolor}{rgb}{0.000000,0.000000,0.000000}%
\pgfsetstrokecolor{textcolor}%
\pgfsetfillcolor{textcolor}%
\pgftext[x=0.207530in,y=1.564817in,,bottom,rotate=90.000000]{\color{textcolor}\sffamily\fontsize{8.000000}{9.600000}\selectfont Benchmarks solved}%
\end{pgfscope}%
\begin{pgfscope}%
\pgfpathrectangle{\pgfqpoint{0.537394in}{0.467838in}}{\pgfqpoint{4.094684in}{2.193958in}}%
\pgfusepath{clip}%
\pgfsetrectcap%
\pgfsetroundjoin%
\pgfsetlinewidth{1.003750pt}%
\definecolor{currentstroke}{rgb}{0.121569,0.466667,0.705882}%
\pgfsetstrokecolor{currentstroke}%
\pgfsetdash{}{0pt}%
\pgfpathmoveto{\pgfqpoint{1.229476in}{0.484293in}}%
\pgfpathlineto{\pgfqpoint{1.318858in}{0.489778in}}%
\pgfpathlineto{\pgfqpoint{1.356331in}{0.506233in}}%
\pgfpathlineto{\pgfqpoint{1.390229in}{0.533657in}}%
\pgfpathlineto{\pgfqpoint{1.421175in}{0.550112in}}%
\pgfpathlineto{\pgfqpoint{1.449643in}{0.561082in}}%
\pgfpathlineto{\pgfqpoint{1.476000in}{0.593991in}}%
\pgfpathlineto{\pgfqpoint{1.500538in}{0.604961in}}%
\pgfpathlineto{\pgfqpoint{1.523492in}{0.626900in}}%
\pgfpathlineto{\pgfqpoint{1.565383in}{0.648840in}}%
\pgfpathlineto{\pgfqpoint{1.584612in}{0.659810in}}%
\pgfpathlineto{\pgfqpoint{1.602855in}{0.687234in}}%
\pgfpathlineto{\pgfqpoint{1.620208in}{0.698204in}}%
\pgfpathlineto{\pgfqpoint{1.636753in}{0.731113in}}%
\pgfpathlineto{\pgfqpoint{1.652563in}{0.747568in}}%
\pgfpathlineto{\pgfqpoint{1.667700in}{0.764023in}}%
\pgfpathlineto{\pgfqpoint{1.682218in}{0.791447in}}%
\pgfpathlineto{\pgfqpoint{1.696168in}{0.813387in}}%
\pgfpathlineto{\pgfqpoint{1.709590in}{0.846296in}}%
\pgfpathlineto{\pgfqpoint{1.722525in}{0.873721in}}%
\pgfpathlineto{\pgfqpoint{1.735005in}{0.901145in}}%
\pgfpathlineto{\pgfqpoint{1.747063in}{0.917600in}}%
\pgfpathlineto{\pgfqpoint{1.758725in}{0.934054in}}%
\pgfpathlineto{\pgfqpoint{1.770017in}{0.961479in}}%
\pgfpathlineto{\pgfqpoint{1.780961in}{0.972449in}}%
\pgfpathlineto{\pgfqpoint{1.791578in}{0.988903in}}%
\pgfpathlineto{\pgfqpoint{1.801888in}{1.027298in}}%
\pgfpathlineto{\pgfqpoint{1.811907in}{1.054722in}}%
\pgfpathlineto{\pgfqpoint{1.821652in}{1.087632in}}%
\pgfpathlineto{\pgfqpoint{1.840375in}{1.098601in}}%
\pgfpathlineto{\pgfqpoint{1.849380in}{1.109571in}}%
\pgfpathlineto{\pgfqpoint{1.866732in}{1.142480in}}%
\pgfpathlineto{\pgfqpoint{1.875101in}{1.164420in}}%
\pgfpathlineto{\pgfqpoint{1.883278in}{1.180875in}}%
\pgfpathlineto{\pgfqpoint{1.899087in}{1.191845in}}%
\pgfpathlineto{\pgfqpoint{1.906736in}{1.208299in}}%
\pgfpathlineto{\pgfqpoint{1.914224in}{1.213784in}}%
\pgfpathlineto{\pgfqpoint{1.921558in}{1.230239in}}%
\pgfpathlineto{\pgfqpoint{1.935786in}{1.241209in}}%
\pgfpathlineto{\pgfqpoint{1.949467in}{1.252178in}}%
\pgfpathlineto{\pgfqpoint{1.962641in}{1.263148in}}%
\pgfpathlineto{\pgfqpoint{1.969049in}{1.274118in}}%
\pgfpathlineto{\pgfqpoint{1.981530in}{1.279603in}}%
\pgfpathlineto{\pgfqpoint{1.987610in}{1.290573in}}%
\pgfpathlineto{\pgfqpoint{1.993587in}{1.296058in}}%
\pgfpathlineto{\pgfqpoint{2.005249in}{1.301542in}}%
\pgfpathlineto{\pgfqpoint{2.016541in}{1.323482in}}%
\pgfpathlineto{\pgfqpoint{2.022055in}{1.328967in}}%
\pgfpathlineto{\pgfqpoint{2.027485in}{1.339937in}}%
\pgfpathlineto{\pgfqpoint{2.032834in}{1.345422in}}%
\pgfpathlineto{\pgfqpoint{2.043295in}{1.367361in}}%
\pgfpathlineto{\pgfqpoint{2.058432in}{1.372846in}}%
\pgfpathlineto{\pgfqpoint{2.068177in}{1.389301in}}%
\pgfpathlineto{\pgfqpoint{2.072951in}{1.394786in}}%
\pgfpathlineto{\pgfqpoint{2.100323in}{1.411240in}}%
\pgfpathlineto{\pgfqpoint{2.113257in}{1.427695in}}%
\pgfpathlineto{\pgfqpoint{2.117466in}{1.438665in}}%
\pgfpathlineto{\pgfqpoint{2.121626in}{1.444150in}}%
\pgfpathlineto{\pgfqpoint{2.129802in}{1.449635in}}%
\pgfpathlineto{\pgfqpoint{2.133821in}{1.460604in}}%
\pgfpathlineto{\pgfqpoint{2.145612in}{1.477059in}}%
\pgfpathlineto{\pgfqpoint{2.175268in}{1.482544in}}%
\pgfpathlineto{\pgfqpoint{2.182311in}{1.488029in}}%
\pgfpathlineto{\pgfqpoint{2.189217in}{1.498999in}}%
\pgfpathlineto{\pgfqpoint{2.195991in}{1.504484in}}%
\pgfpathlineto{\pgfqpoint{2.209166in}{1.509968in}}%
\pgfpathlineto{\pgfqpoint{2.212384in}{1.515453in}}%
\pgfpathlineto{\pgfqpoint{2.221869in}{1.520938in}}%
\pgfpathlineto{\pgfqpoint{2.234134in}{1.542878in}}%
\pgfpathlineto{\pgfqpoint{2.240112in}{1.548363in}}%
\pgfpathlineto{\pgfqpoint{2.254631in}{1.559333in}}%
\pgfpathlineto{\pgfqpoint{2.263066in}{1.564817in}}%
\pgfpathlineto{\pgfqpoint{2.274010in}{1.570302in}}%
\pgfpathlineto{\pgfqpoint{2.289820in}{1.575787in}}%
\pgfpathlineto{\pgfqpoint{2.292388in}{1.581272in}}%
\pgfpathlineto{\pgfqpoint{2.333424in}{1.586757in}}%
\pgfpathlineto{\pgfqpoint{2.342429in}{1.597727in}}%
\pgfpathlineto{\pgfqpoint{2.349036in}{1.614181in}}%
\pgfpathlineto{\pgfqpoint{2.355522in}{1.619666in}}%
\pgfpathlineto{\pgfqpoint{2.359782in}{1.625151in}}%
\pgfpathlineto{\pgfqpoint{2.361892in}{1.630636in}}%
\pgfpathlineto{\pgfqpoint{2.372262in}{1.636121in}}%
\pgfpathlineto{\pgfqpoint{2.397889in}{1.641606in}}%
\pgfpathlineto{\pgfqpoint{2.414607in}{1.647091in}}%
\pgfpathlineto{\pgfqpoint{2.421792in}{1.652576in}}%
\pgfpathlineto{\pgfqpoint{2.440834in}{1.658061in}}%
\pgfpathlineto{\pgfqpoint{2.452442in}{1.663546in}}%
\pgfpathlineto{\pgfqpoint{2.469950in}{1.669030in}}%
\pgfpathlineto{\pgfqpoint{2.498299in}{1.674515in}}%
\pgfpathlineto{\pgfqpoint{2.523219in}{1.680000in}}%
\pgfpathlineto{\pgfqpoint{2.541462in}{1.685485in}}%
\pgfpathlineto{\pgfqpoint{2.552714in}{1.690970in}}%
\pgfpathlineto{\pgfqpoint{2.562425in}{1.696455in}}%
\pgfpathlineto{\pgfqpoint{2.573043in}{1.701940in}}%
\pgfpathlineto{\pgfqpoint{2.611560in}{1.707425in}}%
\pgfpathlineto{\pgfqpoint{2.615707in}{1.712910in}}%
\pgfpathlineto{\pgfqpoint{2.657483in}{1.718394in}}%
\pgfpathlineto{\pgfqpoint{2.675360in}{1.723879in}}%
\pgfpathlineto{\pgfqpoint{2.733161in}{1.729364in}}%
\pgfpathlineto{\pgfqpoint{2.758883in}{1.734849in}}%
\pgfpathlineto{\pgfqpoint{2.765032in}{1.740334in}}%
\pgfpathlineto{\pgfqpoint{2.789886in}{1.745819in}}%
\pgfpathlineto{\pgfqpoint{2.803520in}{1.751304in}}%
\pgfpathlineto{\pgfqpoint{2.842630in}{1.756789in}}%
\pgfpathlineto{\pgfqpoint{2.853888in}{1.762274in}}%
\pgfpathlineto{\pgfqpoint{2.982637in}{1.767759in}}%
\pgfpathlineto{\pgfqpoint{2.990989in}{1.773243in}}%
\pgfpathlineto{\pgfqpoint{2.992064in}{1.778728in}}%
\pgfpathlineto{\pgfqpoint{3.105910in}{1.784213in}}%
\pgfpathlineto{\pgfqpoint{3.165345in}{1.789698in}}%
\pgfpathlineto{\pgfqpoint{3.174245in}{1.795183in}}%
\pgfpathlineto{\pgfqpoint{3.181880in}{1.800668in}}%
\pgfpathlineto{\pgfqpoint{3.185221in}{1.806153in}}%
\pgfpathlineto{\pgfqpoint{3.349434in}{1.811638in}}%
\pgfpathlineto{\pgfqpoint{3.480793in}{1.817123in}}%
\pgfpathlineto{\pgfqpoint{3.507138in}{1.822607in}}%
\pgfpathlineto{\pgfqpoint{3.555930in}{1.828092in}}%
\pgfpathlineto{\pgfqpoint{3.558122in}{1.833577in}}%
\pgfpathlineto{\pgfqpoint{3.651569in}{1.839062in}}%
\pgfpathlineto{\pgfqpoint{3.669804in}{1.844547in}}%
\pgfpathlineto{\pgfqpoint{3.691181in}{1.850032in}}%
\pgfpathlineto{\pgfqpoint{3.696058in}{1.855517in}}%
\pgfpathlineto{\pgfqpoint{3.705039in}{1.861002in}}%
\pgfpathlineto{\pgfqpoint{3.833059in}{1.866487in}}%
\pgfpathlineto{\pgfqpoint{3.923858in}{1.871972in}}%
\pgfpathlineto{\pgfqpoint{4.024851in}{1.877456in}}%
\pgfpathlineto{\pgfqpoint{4.175414in}{1.882941in}}%
\pgfpathlineto{\pgfqpoint{4.311766in}{1.888426in}}%
\pgfpathlineto{\pgfqpoint{4.348586in}{1.893911in}}%
\pgfpathlineto{\pgfqpoint{4.391500in}{1.899396in}}%
\pgfpathlineto{\pgfqpoint{4.487557in}{1.904881in}}%
\pgfpathlineto{\pgfqpoint{4.543350in}{1.910366in}}%
\pgfpathlineto{\pgfqpoint{4.552968in}{1.915851in}}%
\pgfpathlineto{\pgfqpoint{4.552968in}{1.915851in}}%
\pgfusepath{stroke}%
\end{pgfscope}%
\begin{pgfscope}%
\pgfpathrectangle{\pgfqpoint{0.537394in}{0.467838in}}{\pgfqpoint{4.094684in}{2.193958in}}%
\pgfusepath{clip}%
\pgfsetrectcap%
\pgfsetroundjoin%
\pgfsetlinewidth{1.003750pt}%
\definecolor{currentstroke}{rgb}{1.000000,0.498039,0.054902}%
\pgfsetstrokecolor{currentstroke}%
\pgfsetdash{}{0pt}%
\pgfpathmoveto{\pgfqpoint{0.928126in}{0.495263in}}%
\pgfpathlineto{\pgfqpoint{1.030443in}{0.506233in}}%
\pgfpathlineto{\pgfqpoint{1.109806in}{0.517202in}}%
\pgfpathlineto{\pgfqpoint{1.229476in}{0.522687in}}%
\pgfpathlineto{\pgfqpoint{1.276968in}{0.539142in}}%
\pgfpathlineto{\pgfqpoint{1.318858in}{0.544627in}}%
\pgfpathlineto{\pgfqpoint{1.356331in}{0.555597in}}%
\pgfpathlineto{\pgfqpoint{1.390229in}{0.588506in}}%
\pgfpathlineto{\pgfqpoint{1.421175in}{0.599476in}}%
\pgfpathlineto{\pgfqpoint{1.449643in}{0.610446in}}%
\pgfpathlineto{\pgfqpoint{1.476000in}{0.648840in}}%
\pgfpathlineto{\pgfqpoint{1.500538in}{0.665295in}}%
\pgfpathlineto{\pgfqpoint{1.523492in}{0.698204in}}%
\pgfpathlineto{\pgfqpoint{1.545054in}{0.725628in}}%
\pgfpathlineto{\pgfqpoint{1.565383in}{0.742083in}}%
\pgfpathlineto{\pgfqpoint{1.584612in}{0.774992in}}%
\pgfpathlineto{\pgfqpoint{1.602855in}{0.780477in}}%
\pgfpathlineto{\pgfqpoint{1.620208in}{0.829841in}}%
\pgfpathlineto{\pgfqpoint{1.636753in}{0.840811in}}%
\pgfpathlineto{\pgfqpoint{1.652563in}{0.851781in}}%
\pgfpathlineto{\pgfqpoint{1.667700in}{0.884690in}}%
\pgfpathlineto{\pgfqpoint{1.682218in}{0.895660in}}%
\pgfpathlineto{\pgfqpoint{1.696168in}{0.906630in}}%
\pgfpathlineto{\pgfqpoint{1.709590in}{0.928570in}}%
\pgfpathlineto{\pgfqpoint{1.722525in}{0.955994in}}%
\pgfpathlineto{\pgfqpoint{1.735005in}{0.977934in}}%
\pgfpathlineto{\pgfqpoint{1.747063in}{0.994388in}}%
\pgfpathlineto{\pgfqpoint{1.758725in}{1.016328in}}%
\pgfpathlineto{\pgfqpoint{1.770017in}{1.043752in}}%
\pgfpathlineto{\pgfqpoint{1.780961in}{1.065692in}}%
\pgfpathlineto{\pgfqpoint{1.791578in}{1.076662in}}%
\pgfpathlineto{\pgfqpoint{1.821652in}{1.098601in}}%
\pgfpathlineto{\pgfqpoint{1.840375in}{1.109571in}}%
\pgfpathlineto{\pgfqpoint{1.858162in}{1.142480in}}%
\pgfpathlineto{\pgfqpoint{1.866732in}{1.153450in}}%
\pgfpathlineto{\pgfqpoint{1.883278in}{1.164420in}}%
\pgfpathlineto{\pgfqpoint{1.899087in}{1.175390in}}%
\pgfpathlineto{\pgfqpoint{1.906736in}{1.186360in}}%
\pgfpathlineto{\pgfqpoint{1.921558in}{1.191845in}}%
\pgfpathlineto{\pgfqpoint{1.935786in}{1.202814in}}%
\pgfpathlineto{\pgfqpoint{1.949467in}{1.213784in}}%
\pgfpathlineto{\pgfqpoint{1.962641in}{1.224754in}}%
\pgfpathlineto{\pgfqpoint{1.969049in}{1.230239in}}%
\pgfpathlineto{\pgfqpoint{1.981530in}{1.235724in}}%
\pgfpathlineto{\pgfqpoint{2.005249in}{1.241209in}}%
\pgfpathlineto{\pgfqpoint{2.010940in}{1.246693in}}%
\pgfpathlineto{\pgfqpoint{2.016541in}{1.257663in}}%
\pgfpathlineto{\pgfqpoint{2.022055in}{1.263148in}}%
\pgfpathlineto{\pgfqpoint{2.027485in}{1.279603in}}%
\pgfpathlineto{\pgfqpoint{2.043295in}{1.296058in}}%
\pgfpathlineto{\pgfqpoint{2.048413in}{1.301542in}}%
\pgfpathlineto{\pgfqpoint{2.058432in}{1.317997in}}%
\pgfpathlineto{\pgfqpoint{2.068177in}{1.323482in}}%
\pgfpathlineto{\pgfqpoint{2.082311in}{1.339937in}}%
\pgfpathlineto{\pgfqpoint{2.091431in}{1.345422in}}%
\pgfpathlineto{\pgfqpoint{2.095904in}{1.350906in}}%
\pgfpathlineto{\pgfqpoint{2.113257in}{1.372846in}}%
\pgfpathlineto{\pgfqpoint{2.121626in}{1.378331in}}%
\pgfpathlineto{\pgfqpoint{2.125738in}{1.389301in}}%
\pgfpathlineto{\pgfqpoint{2.137795in}{1.394786in}}%
\pgfpathlineto{\pgfqpoint{2.141725in}{1.400271in}}%
\pgfpathlineto{\pgfqpoint{2.145612in}{1.411240in}}%
\pgfpathlineto{\pgfqpoint{2.149457in}{1.427695in}}%
\pgfpathlineto{\pgfqpoint{2.157025in}{1.438665in}}%
\pgfpathlineto{\pgfqpoint{2.171693in}{1.444150in}}%
\pgfpathlineto{\pgfqpoint{2.175268in}{1.449635in}}%
\pgfpathlineto{\pgfqpoint{2.178806in}{1.466089in}}%
\pgfpathlineto{\pgfqpoint{2.182311in}{1.477059in}}%
\pgfpathlineto{\pgfqpoint{2.192620in}{1.482544in}}%
\pgfpathlineto{\pgfqpoint{2.209166in}{1.488029in}}%
\pgfpathlineto{\pgfqpoint{2.212384in}{1.493514in}}%
\pgfpathlineto{\pgfqpoint{2.240112in}{1.498999in}}%
\pgfpathlineto{\pgfqpoint{2.248894in}{1.504484in}}%
\pgfpathlineto{\pgfqpoint{2.254631in}{1.509968in}}%
\pgfpathlineto{\pgfqpoint{2.276694in}{1.515453in}}%
\pgfpathlineto{\pgfqpoint{2.279358in}{1.520938in}}%
\pgfpathlineto{\pgfqpoint{2.319475in}{1.531908in}}%
\pgfpathlineto{\pgfqpoint{2.321838in}{1.537393in}}%
\pgfpathlineto{\pgfqpoint{2.337955in}{1.542878in}}%
\pgfpathlineto{\pgfqpoint{2.342429in}{1.548363in}}%
\pgfpathlineto{\pgfqpoint{2.344645in}{1.553848in}}%
\pgfpathlineto{\pgfqpoint{2.386290in}{1.559333in}}%
\pgfpathlineto{\pgfqpoint{2.390198in}{1.570302in}}%
\pgfpathlineto{\pgfqpoint{2.403549in}{1.575787in}}%
\pgfpathlineto{\pgfqpoint{2.412788in}{1.581272in}}%
\pgfpathlineto{\pgfqpoint{2.420009in}{1.586757in}}%
\pgfpathlineto{\pgfqpoint{2.435741in}{1.592242in}}%
\pgfpathlineto{\pgfqpoint{2.462099in}{1.597727in}}%
\pgfpathlineto{\pgfqpoint{2.463683in}{1.603212in}}%
\pgfpathlineto{\pgfqpoint{2.469950in}{1.608697in}}%
\pgfpathlineto{\pgfqpoint{2.473043in}{1.614181in}}%
\pgfpathlineto{\pgfqpoint{2.482163in}{1.619666in}}%
\pgfpathlineto{\pgfqpoint{2.503989in}{1.630636in}}%
\pgfpathlineto{\pgfqpoint{2.510977in}{1.636121in}}%
\pgfpathlineto{\pgfqpoint{2.512358in}{1.647091in}}%
\pgfpathlineto{\pgfqpoint{2.515104in}{1.652576in}}%
\pgfpathlineto{\pgfqpoint{2.525883in}{1.658061in}}%
\pgfpathlineto{\pgfqpoint{2.531152in}{1.663546in}}%
\pgfpathlineto{\pgfqpoint{2.542730in}{1.669030in}}%
\pgfpathlineto{\pgfqpoint{2.547757in}{1.674515in}}%
\pgfpathlineto{\pgfqpoint{2.555167in}{1.680000in}}%
\pgfpathlineto{\pgfqpoint{2.613640in}{1.685485in}}%
\pgfpathlineto{\pgfqpoint{2.616737in}{1.690970in}}%
\pgfpathlineto{\pgfqpoint{2.657483in}{1.696455in}}%
\pgfpathlineto{\pgfqpoint{2.669205in}{1.701940in}}%
\pgfpathlineto{\pgfqpoint{2.673612in}{1.707425in}}%
\pgfpathlineto{\pgfqpoint{2.691548in}{1.712910in}}%
\pgfpathlineto{\pgfqpoint{2.715697in}{1.718394in}}%
\pgfpathlineto{\pgfqpoint{2.733161in}{1.723879in}}%
\pgfpathlineto{\pgfqpoint{2.817235in}{1.729364in}}%
\pgfpathlineto{\pgfqpoint{2.927934in}{1.734849in}}%
\pgfpathlineto{\pgfqpoint{2.938489in}{1.740334in}}%
\pgfpathlineto{\pgfqpoint{2.950754in}{1.745819in}}%
\pgfpathlineto{\pgfqpoint{3.059641in}{1.751304in}}%
\pgfpathlineto{\pgfqpoint{3.080889in}{1.756789in}}%
\pgfpathlineto{\pgfqpoint{3.250896in}{1.762274in}}%
\pgfpathlineto{\pgfqpoint{3.259940in}{1.767759in}}%
\pgfpathlineto{\pgfqpoint{3.261621in}{1.773243in}}%
\pgfpathlineto{\pgfqpoint{3.370788in}{1.778728in}}%
\pgfpathlineto{\pgfqpoint{3.420233in}{1.784213in}}%
\pgfpathlineto{\pgfqpoint{3.429243in}{1.789698in}}%
\pgfpathlineto{\pgfqpoint{3.448494in}{1.795183in}}%
\pgfpathlineto{\pgfqpoint{3.511483in}{1.800668in}}%
\pgfpathlineto{\pgfqpoint{3.597984in}{1.806153in}}%
\pgfpathlineto{\pgfqpoint{3.606287in}{1.811638in}}%
\pgfpathlineto{\pgfqpoint{3.744420in}{1.817123in}}%
\pgfpathlineto{\pgfqpoint{3.744679in}{1.822607in}}%
\pgfpathlineto{\pgfqpoint{3.745239in}{1.828092in}}%
\pgfpathlineto{\pgfqpoint{4.049382in}{1.833577in}}%
\pgfpathlineto{\pgfqpoint{4.091553in}{1.839062in}}%
\pgfpathlineto{\pgfqpoint{4.120306in}{1.844547in}}%
\pgfpathlineto{\pgfqpoint{4.140420in}{1.850032in}}%
\pgfpathlineto{\pgfqpoint{4.145346in}{1.855517in}}%
\pgfpathlineto{\pgfqpoint{4.158826in}{1.861002in}}%
\pgfpathlineto{\pgfqpoint{4.180755in}{1.866487in}}%
\pgfpathlineto{\pgfqpoint{4.236097in}{1.871972in}}%
\pgfpathlineto{\pgfqpoint{4.275770in}{1.877456in}}%
\pgfpathlineto{\pgfqpoint{4.291495in}{1.882941in}}%
\pgfpathlineto{\pgfqpoint{4.311407in}{1.888426in}}%
\pgfpathlineto{\pgfqpoint{4.322619in}{1.893911in}}%
\pgfpathlineto{\pgfqpoint{4.405195in}{1.899396in}}%
\pgfpathlineto{\pgfqpoint{4.415629in}{1.904881in}}%
\pgfpathlineto{\pgfqpoint{4.419393in}{1.910366in}}%
\pgfpathlineto{\pgfqpoint{4.419393in}{1.910366in}}%
\pgfusepath{stroke}%
\end{pgfscope}%
\begin{pgfscope}%
\pgfpathrectangle{\pgfqpoint{0.537394in}{0.467838in}}{\pgfqpoint{4.094684in}{2.193958in}}%
\pgfusepath{clip}%
\pgfsetbuttcap%
\pgfsetroundjoin%
\pgfsetlinewidth{1.003750pt}%
\definecolor{currentstroke}{rgb}{0.172549,0.627451,0.172549}%
\pgfsetstrokecolor{currentstroke}%
\pgfsetdash{{3.700000pt}{1.600000pt}}{0.000000pt}%
\pgfpathmoveto{\pgfqpoint{0.928126in}{0.506233in}}%
\pgfpathlineto{\pgfqpoint{1.030443in}{0.528172in}}%
\pgfpathlineto{\pgfqpoint{1.109806in}{0.533657in}}%
\pgfpathlineto{\pgfqpoint{1.174651in}{0.544627in}}%
\pgfpathlineto{\pgfqpoint{1.229476in}{0.550112in}}%
\pgfpathlineto{\pgfqpoint{1.276968in}{0.555597in}}%
\pgfpathlineto{\pgfqpoint{1.390229in}{0.572051in}}%
\pgfpathlineto{\pgfqpoint{1.421175in}{0.577536in}}%
\pgfpathlineto{\pgfqpoint{1.449643in}{0.588506in}}%
\pgfpathlineto{\pgfqpoint{1.476000in}{0.593991in}}%
\pgfpathlineto{\pgfqpoint{1.500538in}{0.599476in}}%
\pgfpathlineto{\pgfqpoint{1.584612in}{0.632385in}}%
\pgfpathlineto{\pgfqpoint{1.602855in}{0.637870in}}%
\pgfpathlineto{\pgfqpoint{1.620208in}{0.643355in}}%
\pgfpathlineto{\pgfqpoint{1.667700in}{0.648840in}}%
\pgfpathlineto{\pgfqpoint{1.682218in}{0.654325in}}%
\pgfpathlineto{\pgfqpoint{1.735005in}{0.659810in}}%
\pgfpathlineto{\pgfqpoint{1.801888in}{0.665295in}}%
\pgfpathlineto{\pgfqpoint{1.840375in}{0.676264in}}%
\pgfpathlineto{\pgfqpoint{1.858162in}{0.681749in}}%
\pgfpathlineto{\pgfqpoint{1.866732in}{0.687234in}}%
\pgfpathlineto{\pgfqpoint{1.928743in}{0.692719in}}%
\pgfpathlineto{\pgfqpoint{1.942692in}{0.698204in}}%
\pgfpathlineto{\pgfqpoint{1.981530in}{0.703689in}}%
\pgfpathlineto{\pgfqpoint{2.063338in}{0.709174in}}%
\pgfpathlineto{\pgfqpoint{2.068177in}{0.720144in}}%
\pgfpathlineto{\pgfqpoint{2.125738in}{0.725628in}}%
\pgfpathlineto{\pgfqpoint{2.129802in}{0.731113in}}%
\pgfpathlineto{\pgfqpoint{2.153261in}{0.736598in}}%
\pgfpathlineto{\pgfqpoint{2.157025in}{0.742083in}}%
\pgfpathlineto{\pgfqpoint{2.168082in}{0.753053in}}%
\pgfpathlineto{\pgfqpoint{2.205917in}{0.758538in}}%
\pgfpathlineto{\pgfqpoint{2.260276in}{0.764023in}}%
\pgfpathlineto{\pgfqpoint{2.276694in}{0.769508in}}%
\pgfpathlineto{\pgfqpoint{2.328835in}{0.774992in}}%
\pgfpathlineto{\pgfqpoint{2.346847in}{0.780477in}}%
\pgfpathlineto{\pgfqpoint{2.454070in}{0.785962in}}%
\pgfpathlineto{\pgfqpoint{2.468394in}{0.791447in}}%
\pgfpathlineto{\pgfqpoint{2.473043in}{0.796932in}}%
\pgfpathlineto{\pgfqpoint{2.502575in}{0.802417in}}%
\pgfpathlineto{\pgfqpoint{2.503989in}{0.813387in}}%
\pgfpathlineto{\pgfqpoint{2.523219in}{0.818872in}}%
\pgfpathlineto{\pgfqpoint{2.541462in}{0.829841in}}%
\pgfpathlineto{\pgfqpoint{2.547757in}{0.835326in}}%
\pgfpathlineto{\pgfqpoint{2.550244in}{0.840811in}}%
\pgfpathlineto{\pgfqpoint{2.586724in}{0.846296in}}%
\pgfpathlineto{\pgfqpoint{2.587840in}{0.851781in}}%
\pgfpathlineto{\pgfqpoint{2.591169in}{0.857266in}}%
\pgfpathlineto{\pgfqpoint{2.593372in}{0.862751in}}%
\pgfpathlineto{\pgfqpoint{2.594468in}{0.868236in}}%
\pgfpathlineto{\pgfqpoint{2.596650in}{0.873721in}}%
\pgfpathlineto{\pgfqpoint{2.598818in}{0.884690in}}%
\pgfpathlineto{\pgfqpoint{2.617763in}{0.890175in}}%
\pgfpathlineto{\pgfqpoint{2.622851in}{0.901145in}}%
\pgfpathlineto{\pgfqpoint{2.629855in}{0.912115in}}%
\pgfpathlineto{\pgfqpoint{2.639626in}{0.917600in}}%
\pgfpathlineto{\pgfqpoint{2.684821in}{0.923085in}}%
\pgfpathlineto{\pgfqpoint{2.749100in}{0.928570in}}%
\pgfpathlineto{\pgfqpoint{2.749807in}{0.934054in}}%
\pgfpathlineto{\pgfqpoint{2.760258in}{0.939539in}}%
\pgfpathlineto{\pgfqpoint{2.801079in}{0.945024in}}%
\pgfpathlineto{\pgfqpoint{2.808350in}{0.950509in}}%
\pgfpathlineto{\pgfqpoint{2.822461in}{0.955994in}}%
\pgfpathlineto{\pgfqpoint{2.850707in}{0.961479in}}%
\pgfpathlineto{\pgfqpoint{2.868364in}{0.966964in}}%
\pgfpathlineto{\pgfqpoint{2.872895in}{0.972449in}}%
\pgfpathlineto{\pgfqpoint{2.882274in}{0.977934in}}%
\pgfpathlineto{\pgfqpoint{2.882761in}{0.983418in}}%
\pgfpathlineto{\pgfqpoint{2.888073in}{0.988903in}}%
\pgfpathlineto{\pgfqpoint{2.894721in}{0.994388in}}%
\pgfpathlineto{\pgfqpoint{2.903090in}{0.999873in}}%
\pgfpathlineto{\pgfqpoint{2.907202in}{1.005358in}}%
\pgfpathlineto{\pgfqpoint{2.910367in}{1.010843in}}%
\pgfpathlineto{\pgfqpoint{2.910817in}{1.021813in}}%
\pgfpathlineto{\pgfqpoint{2.911267in}{1.027298in}}%
\pgfpathlineto{\pgfqpoint{2.912611in}{1.032783in}}%
\pgfpathlineto{\pgfqpoint{2.925354in}{1.038267in}}%
\pgfpathlineto{\pgfqpoint{2.926216in}{1.043752in}}%
\pgfpathlineto{\pgfqpoint{2.926647in}{1.049237in}}%
\pgfpathlineto{\pgfqpoint{2.929644in}{1.054722in}}%
\pgfpathlineto{\pgfqpoint{2.935145in}{1.060207in}}%
\pgfpathlineto{\pgfqpoint{2.937239in}{1.065692in}}%
\pgfpathlineto{\pgfqpoint{2.938905in}{1.071177in}}%
\pgfpathlineto{\pgfqpoint{2.945899in}{1.076662in}}%
\pgfpathlineto{\pgfqpoint{2.955544in}{1.082147in}}%
\pgfpathlineto{\pgfqpoint{2.958309in}{1.087632in}}%
\pgfpathlineto{\pgfqpoint{2.961443in}{1.093116in}}%
\pgfpathlineto{\pgfqpoint{2.962222in}{1.098601in}}%
\pgfpathlineto{\pgfqpoint{2.969920in}{1.104086in}}%
\pgfpathlineto{\pgfqpoint{2.973331in}{1.109571in}}%
\pgfpathlineto{\pgfqpoint{2.974085in}{1.120541in}}%
\pgfpathlineto{\pgfqpoint{2.975587in}{1.131511in}}%
\pgfpathlineto{\pgfqpoint{2.978201in}{1.136996in}}%
\pgfpathlineto{\pgfqpoint{2.984104in}{1.142480in}}%
\pgfpathlineto{\pgfqpoint{2.986292in}{1.147965in}}%
\pgfpathlineto{\pgfqpoint{2.988106in}{1.153450in}}%
\pgfpathlineto{\pgfqpoint{3.005752in}{1.158935in}}%
\pgfpathlineto{\pgfqpoint{3.006440in}{1.164420in}}%
\pgfpathlineto{\pgfqpoint{3.007811in}{1.169905in}}%
\pgfpathlineto{\pgfqpoint{3.010879in}{1.186360in}}%
\pgfpathlineto{\pgfqpoint{3.011218in}{1.191845in}}%
\pgfpathlineto{\pgfqpoint{3.015599in}{1.197329in}}%
\pgfpathlineto{\pgfqpoint{3.015933in}{1.202814in}}%
\pgfpathlineto{\pgfqpoint{3.017935in}{1.208299in}}%
\pgfpathlineto{\pgfqpoint{3.054875in}{1.213784in}}%
\pgfpathlineto{\pgfqpoint{3.076966in}{1.219269in}}%
\pgfpathlineto{\pgfqpoint{3.079213in}{1.224754in}}%
\pgfpathlineto{\pgfqpoint{3.101203in}{1.230239in}}%
\pgfpathlineto{\pgfqpoint{3.101466in}{1.235724in}}%
\pgfpathlineto{\pgfqpoint{3.104348in}{1.241209in}}%
\pgfpathlineto{\pgfqpoint{3.108756in}{1.246693in}}%
\pgfpathlineto{\pgfqpoint{3.136510in}{1.252178in}}%
\pgfpathlineto{\pgfqpoint{3.136748in}{1.257663in}}%
\pgfpathlineto{\pgfqpoint{3.139832in}{1.263148in}}%
\pgfpathlineto{\pgfqpoint{3.140775in}{1.285088in}}%
\pgfpathlineto{\pgfqpoint{3.142420in}{1.290573in}}%
\pgfpathlineto{\pgfqpoint{3.172956in}{1.296058in}}%
\pgfpathlineto{\pgfqpoint{3.184597in}{1.301542in}}%
\pgfpathlineto{\pgfqpoint{3.189149in}{1.307027in}}%
\pgfpathlineto{\pgfqpoint{3.189355in}{1.312512in}}%
\pgfpathlineto{\pgfqpoint{3.190995in}{1.317997in}}%
\pgfpathlineto{\pgfqpoint{3.193846in}{1.323482in}}%
\pgfpathlineto{\pgfqpoint{3.237693in}{1.328967in}}%
\pgfpathlineto{\pgfqpoint{3.258758in}{1.334452in}}%
\pgfpathlineto{\pgfqpoint{3.281353in}{1.339937in}}%
\pgfpathlineto{\pgfqpoint{3.322502in}{1.345422in}}%
\pgfpathlineto{\pgfqpoint{3.330605in}{1.350906in}}%
\pgfpathlineto{\pgfqpoint{3.331433in}{1.356391in}}%
\pgfpathlineto{\pgfqpoint{3.331570in}{1.361876in}}%
\pgfpathlineto{\pgfqpoint{3.333082in}{1.367361in}}%
\pgfpathlineto{\pgfqpoint{3.351264in}{1.372846in}}%
\pgfpathlineto{\pgfqpoint{3.352305in}{1.378331in}}%
\pgfpathlineto{\pgfqpoint{3.352694in}{1.389301in}}%
\pgfpathlineto{\pgfqpoint{3.352824in}{1.394786in}}%
\pgfpathlineto{\pgfqpoint{3.357337in}{1.400271in}}%
\pgfpathlineto{\pgfqpoint{3.382438in}{1.405755in}}%
\pgfpathlineto{\pgfqpoint{3.384343in}{1.411240in}}%
\pgfpathlineto{\pgfqpoint{3.395679in}{1.416725in}}%
\pgfpathlineto{\pgfqpoint{3.421198in}{1.422210in}}%
\pgfpathlineto{\pgfqpoint{3.422587in}{1.427695in}}%
\pgfpathlineto{\pgfqpoint{3.457114in}{1.433180in}}%
\pgfpathlineto{\pgfqpoint{3.458659in}{1.438665in}}%
\pgfpathlineto{\pgfqpoint{3.464963in}{1.444150in}}%
\pgfpathlineto{\pgfqpoint{3.465436in}{1.449635in}}%
\pgfpathlineto{\pgfqpoint{3.470784in}{1.455119in}}%
\pgfpathlineto{\pgfqpoint{3.471436in}{1.460604in}}%
\pgfpathlineto{\pgfqpoint{3.501032in}{1.466089in}}%
\pgfpathlineto{\pgfqpoint{3.508982in}{1.471574in}}%
\pgfpathlineto{\pgfqpoint{3.510068in}{1.477059in}}%
\pgfpathlineto{\pgfqpoint{3.512727in}{1.482544in}}%
\pgfpathlineto{\pgfqpoint{3.514543in}{1.488029in}}%
\pgfpathlineto{\pgfqpoint{3.516104in}{1.493514in}}%
\pgfpathlineto{\pgfqpoint{3.609454in}{1.498999in}}%
\pgfpathlineto{\pgfqpoint{3.616757in}{1.504484in}}%
\pgfpathlineto{\pgfqpoint{3.640879in}{1.509968in}}%
\pgfpathlineto{\pgfqpoint{3.642377in}{1.515453in}}%
\pgfpathlineto{\pgfqpoint{3.642951in}{1.520938in}}%
\pgfpathlineto{\pgfqpoint{3.645638in}{1.526423in}}%
\pgfpathlineto{\pgfqpoint{3.646776in}{1.531908in}}%
\pgfpathlineto{\pgfqpoint{3.647059in}{1.537393in}}%
\pgfpathlineto{\pgfqpoint{3.655303in}{1.542878in}}%
\pgfpathlineto{\pgfqpoint{3.655801in}{1.548363in}}%
\pgfpathlineto{\pgfqpoint{3.657127in}{1.553848in}}%
\pgfpathlineto{\pgfqpoint{3.659381in}{1.559333in}}%
\pgfpathlineto{\pgfqpoint{3.659436in}{1.564817in}}%
\pgfpathlineto{\pgfqpoint{3.672402in}{1.570302in}}%
\pgfpathlineto{\pgfqpoint{3.674614in}{1.575787in}}%
\pgfpathlineto{\pgfqpoint{3.683532in}{1.581272in}}%
\pgfpathlineto{\pgfqpoint{3.683737in}{1.586757in}}%
\pgfpathlineto{\pgfqpoint{3.691531in}{1.597727in}}%
\pgfpathlineto{\pgfqpoint{3.692781in}{1.603212in}}%
\pgfpathlineto{\pgfqpoint{3.707106in}{1.608697in}}%
\pgfpathlineto{\pgfqpoint{3.739248in}{1.614181in}}%
\pgfpathlineto{\pgfqpoint{3.740559in}{1.619666in}}%
\pgfpathlineto{\pgfqpoint{3.787751in}{1.625151in}}%
\pgfpathlineto{\pgfqpoint{3.788819in}{1.630636in}}%
\pgfpathlineto{\pgfqpoint{3.793435in}{1.636121in}}%
\pgfpathlineto{\pgfqpoint{3.794561in}{1.641606in}}%
\pgfpathlineto{\pgfqpoint{3.799695in}{1.647091in}}%
\pgfpathlineto{\pgfqpoint{3.800875in}{1.652576in}}%
\pgfpathlineto{\pgfqpoint{3.810428in}{1.658061in}}%
\pgfpathlineto{\pgfqpoint{3.811858in}{1.663546in}}%
\pgfpathlineto{\pgfqpoint{3.832386in}{1.669030in}}%
\pgfpathlineto{\pgfqpoint{3.836173in}{1.674515in}}%
\pgfpathlineto{\pgfqpoint{3.859743in}{1.680000in}}%
\pgfpathlineto{\pgfqpoint{3.924977in}{1.685485in}}%
\pgfpathlineto{\pgfqpoint{3.925936in}{1.690970in}}%
\pgfpathlineto{\pgfqpoint{3.929621in}{1.696455in}}%
\pgfpathlineto{\pgfqpoint{3.929826in}{1.701940in}}%
\pgfpathlineto{\pgfqpoint{3.980947in}{1.707425in}}%
\pgfpathlineto{\pgfqpoint{3.982098in}{1.712910in}}%
\pgfpathlineto{\pgfqpoint{4.012499in}{1.718394in}}%
\pgfpathlineto{\pgfqpoint{4.012702in}{1.723879in}}%
\pgfpathlineto{\pgfqpoint{4.013857in}{1.729364in}}%
\pgfpathlineto{\pgfqpoint{4.015914in}{1.734849in}}%
\pgfpathlineto{\pgfqpoint{4.017319in}{1.740334in}}%
\pgfpathlineto{\pgfqpoint{4.020510in}{1.745819in}}%
\pgfpathlineto{\pgfqpoint{4.025204in}{1.751304in}}%
\pgfpathlineto{\pgfqpoint{4.026416in}{1.756789in}}%
\pgfpathlineto{\pgfqpoint{4.045684in}{1.762274in}}%
\pgfpathlineto{\pgfqpoint{4.048007in}{1.767759in}}%
\pgfpathlineto{\pgfqpoint{4.050351in}{1.778728in}}%
\pgfpathlineto{\pgfqpoint{4.051190in}{1.784213in}}%
\pgfpathlineto{\pgfqpoint{4.056558in}{1.789698in}}%
\pgfpathlineto{\pgfqpoint{4.063557in}{1.795183in}}%
\pgfpathlineto{\pgfqpoint{4.089417in}{1.800668in}}%
\pgfpathlineto{\pgfqpoint{4.119691in}{1.806153in}}%
\pgfpathlineto{\pgfqpoint{4.130608in}{1.811638in}}%
\pgfpathlineto{\pgfqpoint{4.133697in}{1.817123in}}%
\pgfpathlineto{\pgfqpoint{4.184968in}{1.822607in}}%
\pgfpathlineto{\pgfqpoint{4.227660in}{1.828092in}}%
\pgfpathlineto{\pgfqpoint{4.234980in}{1.833577in}}%
\pgfpathlineto{\pgfqpoint{4.235392in}{1.839062in}}%
\pgfpathlineto{\pgfqpoint{4.239341in}{1.844547in}}%
\pgfpathlineto{\pgfqpoint{4.248169in}{1.850032in}}%
\pgfpathlineto{\pgfqpoint{4.278059in}{1.855517in}}%
\pgfpathlineto{\pgfqpoint{4.281115in}{1.861002in}}%
\pgfpathlineto{\pgfqpoint{4.344003in}{1.866487in}}%
\pgfpathlineto{\pgfqpoint{4.357710in}{1.871972in}}%
\pgfpathlineto{\pgfqpoint{4.382482in}{1.877456in}}%
\pgfpathlineto{\pgfqpoint{4.382847in}{1.882941in}}%
\pgfpathlineto{\pgfqpoint{4.386342in}{1.888426in}}%
\pgfpathlineto{\pgfqpoint{4.393279in}{1.893911in}}%
\pgfpathlineto{\pgfqpoint{4.394911in}{1.899396in}}%
\pgfpathlineto{\pgfqpoint{4.405451in}{1.904881in}}%
\pgfpathlineto{\pgfqpoint{4.469755in}{1.910366in}}%
\pgfpathlineto{\pgfqpoint{4.477034in}{1.915851in}}%
\pgfpathlineto{\pgfqpoint{4.477297in}{1.921336in}}%
\pgfpathlineto{\pgfqpoint{4.519182in}{1.926820in}}%
\pgfpathlineto{\pgfqpoint{4.519182in}{1.926820in}}%
\pgfusepath{stroke}%
\end{pgfscope}%
\begin{pgfscope}%
\pgfpathrectangle{\pgfqpoint{0.537394in}{0.467838in}}{\pgfqpoint{4.094684in}{2.193958in}}%
\pgfusepath{clip}%
\pgfsetbuttcap%
\pgfsetroundjoin%
\pgfsetlinewidth{1.003750pt}%
\definecolor{currentstroke}{rgb}{0.839216,0.152941,0.156863}%
\pgfsetstrokecolor{currentstroke}%
\pgfsetdash{{3.700000pt}{1.600000pt}}{0.000000pt}%
\pgfpathmoveto{\pgfqpoint{0.928126in}{0.528172in}}%
\pgfpathlineto{\pgfqpoint{1.030443in}{0.566566in}}%
\pgfpathlineto{\pgfqpoint{1.109806in}{0.593991in}}%
\pgfpathlineto{\pgfqpoint{1.174651in}{0.610446in}}%
\pgfpathlineto{\pgfqpoint{1.229476in}{0.637870in}}%
\pgfpathlineto{\pgfqpoint{1.276968in}{0.643355in}}%
\pgfpathlineto{\pgfqpoint{1.318858in}{0.665295in}}%
\pgfpathlineto{\pgfqpoint{1.421175in}{0.681749in}}%
\pgfpathlineto{\pgfqpoint{1.476000in}{0.692719in}}%
\pgfpathlineto{\pgfqpoint{1.500538in}{0.698204in}}%
\pgfpathlineto{\pgfqpoint{1.523492in}{0.703689in}}%
\pgfpathlineto{\pgfqpoint{1.565383in}{0.714659in}}%
\pgfpathlineto{\pgfqpoint{1.584612in}{0.731113in}}%
\pgfpathlineto{\pgfqpoint{1.620208in}{0.736598in}}%
\pgfpathlineto{\pgfqpoint{1.652563in}{0.758538in}}%
\pgfpathlineto{\pgfqpoint{1.667700in}{0.774992in}}%
\pgfpathlineto{\pgfqpoint{1.682218in}{0.785962in}}%
\pgfpathlineto{\pgfqpoint{1.709590in}{0.791447in}}%
\pgfpathlineto{\pgfqpoint{1.722525in}{0.796932in}}%
\pgfpathlineto{\pgfqpoint{1.758725in}{0.807902in}}%
\pgfpathlineto{\pgfqpoint{1.791578in}{0.813387in}}%
\pgfpathlineto{\pgfqpoint{1.811907in}{0.835326in}}%
\pgfpathlineto{\pgfqpoint{1.821652in}{0.840811in}}%
\pgfpathlineto{\pgfqpoint{1.831137in}{0.857266in}}%
\pgfpathlineto{\pgfqpoint{1.840375in}{0.862751in}}%
\pgfpathlineto{\pgfqpoint{1.849380in}{0.873721in}}%
\pgfpathlineto{\pgfqpoint{1.858162in}{0.895660in}}%
\pgfpathlineto{\pgfqpoint{1.875101in}{0.906630in}}%
\pgfpathlineto{\pgfqpoint{1.883278in}{0.928570in}}%
\pgfpathlineto{\pgfqpoint{1.899087in}{0.939539in}}%
\pgfpathlineto{\pgfqpoint{1.914224in}{0.945024in}}%
\pgfpathlineto{\pgfqpoint{1.935786in}{0.961479in}}%
\pgfpathlineto{\pgfqpoint{1.942692in}{0.972449in}}%
\pgfpathlineto{\pgfqpoint{1.949467in}{0.977934in}}%
\pgfpathlineto{\pgfqpoint{1.981530in}{0.988903in}}%
\pgfpathlineto{\pgfqpoint{1.987610in}{0.994388in}}%
\pgfpathlineto{\pgfqpoint{1.999466in}{0.999873in}}%
\pgfpathlineto{\pgfqpoint{2.005249in}{1.016328in}}%
\pgfpathlineto{\pgfqpoint{2.032834in}{1.021813in}}%
\pgfpathlineto{\pgfqpoint{2.058432in}{1.027298in}}%
\pgfpathlineto{\pgfqpoint{2.072951in}{1.038267in}}%
\pgfpathlineto{\pgfqpoint{2.091431in}{1.049237in}}%
\pgfpathlineto{\pgfqpoint{2.095904in}{1.054722in}}%
\pgfpathlineto{\pgfqpoint{2.104687in}{1.060207in}}%
\pgfpathlineto{\pgfqpoint{2.108998in}{1.071177in}}%
\pgfpathlineto{\pgfqpoint{2.121626in}{1.087632in}}%
\pgfpathlineto{\pgfqpoint{2.137795in}{1.093116in}}%
\pgfpathlineto{\pgfqpoint{2.145612in}{1.115056in}}%
\pgfpathlineto{\pgfqpoint{2.164434in}{1.126026in}}%
\pgfpathlineto{\pgfqpoint{2.178806in}{1.136996in}}%
\pgfpathlineto{\pgfqpoint{2.185780in}{1.153450in}}%
\pgfpathlineto{\pgfqpoint{2.189217in}{1.164420in}}%
\pgfpathlineto{\pgfqpoint{2.245991in}{1.169905in}}%
\pgfpathlineto{\pgfqpoint{2.248894in}{1.175390in}}%
\pgfpathlineto{\pgfqpoint{2.314701in}{1.180875in}}%
\pgfpathlineto{\pgfqpoint{2.324186in}{1.186360in}}%
\pgfpathlineto{\pgfqpoint{2.331137in}{1.191845in}}%
\pgfpathlineto{\pgfqpoint{2.337955in}{1.208299in}}%
\pgfpathlineto{\pgfqpoint{2.342429in}{1.213784in}}%
\pgfpathlineto{\pgfqpoint{2.351211in}{1.219269in}}%
\pgfpathlineto{\pgfqpoint{2.397889in}{1.224754in}}%
\pgfpathlineto{\pgfqpoint{2.414607in}{1.230239in}}%
\pgfpathlineto{\pgfqpoint{2.420009in}{1.235724in}}%
\pgfpathlineto{\pgfqpoint{2.423566in}{1.241209in}}%
\pgfpathlineto{\pgfqpoint{2.435741in}{1.246693in}}%
\pgfpathlineto{\pgfqpoint{2.437447in}{1.252178in}}%
\pgfpathlineto{\pgfqpoint{2.462099in}{1.257663in}}%
\pgfpathlineto{\pgfqpoint{2.474579in}{1.263148in}}%
\pgfpathlineto{\pgfqpoint{2.483660in}{1.268633in}}%
\pgfpathlineto{\pgfqpoint{2.517830in}{1.274118in}}%
\pgfpathlineto{\pgfqpoint{2.520534in}{1.285088in}}%
\pgfpathlineto{\pgfqpoint{2.546507in}{1.290573in}}%
\pgfpathlineto{\pgfqpoint{2.552714in}{1.296058in}}%
\pgfpathlineto{\pgfqpoint{2.562425in}{1.301542in}}%
\pgfpathlineto{\pgfqpoint{2.564812in}{1.307027in}}%
\pgfpathlineto{\pgfqpoint{2.566000in}{1.312512in}}%
\pgfpathlineto{\pgfqpoint{2.577662in}{1.317997in}}%
\pgfpathlineto{\pgfqpoint{2.584480in}{1.323482in}}%
\pgfpathlineto{\pgfqpoint{2.585603in}{1.328967in}}%
\pgfpathlineto{\pgfqpoint{2.595560in}{1.334452in}}%
\pgfpathlineto{\pgfqpoint{2.607363in}{1.339937in}}%
\pgfpathlineto{\pgfqpoint{2.611560in}{1.345422in}}%
\pgfpathlineto{\pgfqpoint{2.613640in}{1.356391in}}%
\pgfpathlineto{\pgfqpoint{2.615707in}{1.361876in}}%
\pgfpathlineto{\pgfqpoint{2.642506in}{1.367361in}}%
\pgfpathlineto{\pgfqpoint{2.650074in}{1.372846in}}%
\pgfpathlineto{\pgfqpoint{2.651008in}{1.378331in}}%
\pgfpathlineto{\pgfqpoint{2.654723in}{1.383816in}}%
\pgfpathlineto{\pgfqpoint{2.670090in}{1.389301in}}%
\pgfpathlineto{\pgfqpoint{2.691548in}{1.394786in}}%
\pgfpathlineto{\pgfqpoint{2.703022in}{1.400271in}}%
\pgfpathlineto{\pgfqpoint{2.727183in}{1.405755in}}%
\pgfpathlineto{\pgfqpoint{2.754723in}{1.411240in}}%
\pgfpathlineto{\pgfqpoint{2.762312in}{1.416725in}}%
\pgfpathlineto{\pgfqpoint{2.769074in}{1.422210in}}%
\pgfpathlineto{\pgfqpoint{2.779633in}{1.427695in}}%
\pgfpathlineto{\pgfqpoint{2.798005in}{1.433180in}}%
\pgfpathlineto{\pgfqpoint{2.803520in}{1.438665in}}%
\pgfpathlineto{\pgfqpoint{2.862232in}{1.444150in}}%
\pgfpathlineto{\pgfqpoint{2.883247in}{1.449635in}}%
\pgfpathlineto{\pgfqpoint{2.884218in}{1.455119in}}%
\pgfpathlineto{\pgfqpoint{2.886151in}{1.460604in}}%
\pgfpathlineto{\pgfqpoint{2.886632in}{1.466089in}}%
\pgfpathlineto{\pgfqpoint{2.891413in}{1.477059in}}%
\pgfpathlineto{\pgfqpoint{2.891887in}{1.482544in}}%
\pgfpathlineto{\pgfqpoint{2.901247in}{1.493514in}}%
\pgfpathlineto{\pgfqpoint{2.903090in}{1.498999in}}%
\pgfpathlineto{\pgfqpoint{2.903549in}{1.504484in}}%
\pgfpathlineto{\pgfqpoint{2.909014in}{1.515453in}}%
\pgfpathlineto{\pgfqpoint{2.917057in}{1.520938in}}%
\pgfpathlineto{\pgfqpoint{2.922755in}{1.526423in}}%
\pgfpathlineto{\pgfqpoint{2.927934in}{1.531908in}}%
\pgfpathlineto{\pgfqpoint{2.929644in}{1.537393in}}%
\pgfpathlineto{\pgfqpoint{2.946306in}{1.542878in}}%
\pgfpathlineto{\pgfqpoint{2.978944in}{1.548363in}}%
\pgfpathlineto{\pgfqpoint{2.991706in}{1.553848in}}%
\pgfpathlineto{\pgfqpoint{3.000200in}{1.559333in}}%
\pgfpathlineto{\pgfqpoint{3.017602in}{1.564817in}}%
\pgfpathlineto{\pgfqpoint{3.056073in}{1.570302in}}%
\pgfpathlineto{\pgfqpoint{3.085046in}{1.575787in}}%
\pgfpathlineto{\pgfqpoint{3.144291in}{1.581272in}}%
\pgfpathlineto{\pgfqpoint{3.144524in}{1.586757in}}%
\pgfpathlineto{\pgfqpoint{3.163140in}{1.592242in}}%
\pgfpathlineto{\pgfqpoint{3.187707in}{1.597727in}}%
\pgfpathlineto{\pgfqpoint{3.321228in}{1.603212in}}%
\pgfpathlineto{\pgfqpoint{3.327832in}{1.608697in}}%
\pgfpathlineto{\pgfqpoint{3.329914in}{1.614181in}}%
\pgfpathlineto{\pgfqpoint{3.339606in}{1.619666in}}%
\pgfpathlineto{\pgfqpoint{3.355539in}{1.625151in}}%
\pgfpathlineto{\pgfqpoint{3.388475in}{1.630636in}}%
\pgfpathlineto{\pgfqpoint{3.415042in}{1.636121in}}%
\pgfpathlineto{\pgfqpoint{3.424819in}{1.641606in}}%
\pgfpathlineto{\pgfqpoint{3.438948in}{1.647091in}}%
\pgfpathlineto{\pgfqpoint{3.440371in}{1.652576in}}%
\pgfpathlineto{\pgfqpoint{3.476237in}{1.658061in}}%
\pgfpathlineto{\pgfqpoint{3.477701in}{1.663546in}}%
\pgfpathlineto{\pgfqpoint{3.481878in}{1.669030in}}%
\pgfpathlineto{\pgfqpoint{3.493074in}{1.674515in}}%
\pgfpathlineto{\pgfqpoint{3.551950in}{1.680000in}}%
\pgfpathlineto{\pgfqpoint{3.578508in}{1.690970in}}%
\pgfpathlineto{\pgfqpoint{3.633824in}{1.696455in}}%
\pgfpathlineto{\pgfqpoint{3.647739in}{1.701940in}}%
\pgfpathlineto{\pgfqpoint{3.651905in}{1.707425in}}%
\pgfpathlineto{\pgfqpoint{3.659436in}{1.712910in}}%
\pgfpathlineto{\pgfqpoint{3.676290in}{1.718394in}}%
\pgfpathlineto{\pgfqpoint{3.743340in}{1.723879in}}%
\pgfpathlineto{\pgfqpoint{3.877926in}{1.729364in}}%
\pgfpathlineto{\pgfqpoint{3.880290in}{1.734849in}}%
\pgfpathlineto{\pgfqpoint{3.884127in}{1.740334in}}%
\pgfpathlineto{\pgfqpoint{3.989150in}{1.745819in}}%
\pgfpathlineto{\pgfqpoint{4.000481in}{1.751304in}}%
\pgfpathlineto{\pgfqpoint{4.009194in}{1.756789in}}%
\pgfpathlineto{\pgfqpoint{4.171761in}{1.762274in}}%
\pgfpathlineto{\pgfqpoint{4.191632in}{1.767759in}}%
\pgfpathlineto{\pgfqpoint{4.449615in}{1.773243in}}%
\pgfpathlineto{\pgfqpoint{4.455559in}{1.778728in}}%
\pgfpathlineto{\pgfqpoint{4.476103in}{1.784213in}}%
\pgfpathlineto{\pgfqpoint{4.580095in}{1.789698in}}%
\pgfpathlineto{\pgfqpoint{4.580103in}{1.795183in}}%
\pgfpathlineto{\pgfqpoint{4.590132in}{1.800668in}}%
\pgfpathlineto{\pgfqpoint{4.607935in}{1.806153in}}%
\pgfpathlineto{\pgfqpoint{4.627781in}{1.811638in}}%
\pgfpathlineto{\pgfqpoint{4.627781in}{1.811638in}}%
\pgfusepath{stroke}%
\end{pgfscope}%
\begin{pgfscope}%
\pgfpathrectangle{\pgfqpoint{0.537394in}{0.467838in}}{\pgfqpoint{4.094684in}{2.193958in}}%
\pgfusepath{clip}%
\pgfsetbuttcap%
\pgfsetroundjoin%
\pgfsetlinewidth{1.003750pt}%
\definecolor{currentstroke}{rgb}{0.580392,0.403922,0.741176}%
\pgfsetstrokecolor{currentstroke}%
\pgfsetdash{{1.000000pt}{1.650000pt}}{0.000000pt}%
\pgfpathmoveto{\pgfqpoint{0.783918in}{0.473323in}}%
\pgfpathlineto{\pgfqpoint{1.174651in}{0.484293in}}%
\pgfpathlineto{\pgfqpoint{1.276968in}{0.489778in}}%
\pgfpathlineto{\pgfqpoint{1.318858in}{0.495263in}}%
\pgfpathlineto{\pgfqpoint{1.356331in}{0.506233in}}%
\pgfpathlineto{\pgfqpoint{1.390229in}{0.511717in}}%
\pgfpathlineto{\pgfqpoint{1.421175in}{0.522687in}}%
\pgfpathlineto{\pgfqpoint{1.449643in}{0.533657in}}%
\pgfpathlineto{\pgfqpoint{1.476000in}{0.544627in}}%
\pgfpathlineto{\pgfqpoint{1.500538in}{0.561082in}}%
\pgfpathlineto{\pgfqpoint{1.523492in}{0.566566in}}%
\pgfpathlineto{\pgfqpoint{1.545054in}{0.577536in}}%
\pgfpathlineto{\pgfqpoint{1.565383in}{0.604961in}}%
\pgfpathlineto{\pgfqpoint{1.584612in}{0.626900in}}%
\pgfpathlineto{\pgfqpoint{1.602855in}{0.637870in}}%
\pgfpathlineto{\pgfqpoint{1.620208in}{0.643355in}}%
\pgfpathlineto{\pgfqpoint{1.636753in}{0.659810in}}%
\pgfpathlineto{\pgfqpoint{1.652563in}{0.681749in}}%
\pgfpathlineto{\pgfqpoint{1.667700in}{0.709174in}}%
\pgfpathlineto{\pgfqpoint{1.682218in}{0.747568in}}%
\pgfpathlineto{\pgfqpoint{1.696168in}{0.780477in}}%
\pgfpathlineto{\pgfqpoint{1.709590in}{0.796932in}}%
\pgfpathlineto{\pgfqpoint{1.722525in}{0.835326in}}%
\pgfpathlineto{\pgfqpoint{1.735005in}{0.884690in}}%
\pgfpathlineto{\pgfqpoint{1.747063in}{0.901145in}}%
\pgfpathlineto{\pgfqpoint{1.758725in}{0.928570in}}%
\pgfpathlineto{\pgfqpoint{1.770017in}{0.939539in}}%
\pgfpathlineto{\pgfqpoint{1.780961in}{0.966964in}}%
\pgfpathlineto{\pgfqpoint{1.791578in}{0.988903in}}%
\pgfpathlineto{\pgfqpoint{1.801888in}{1.010843in}}%
\pgfpathlineto{\pgfqpoint{1.811907in}{1.027298in}}%
\pgfpathlineto{\pgfqpoint{1.821652in}{1.043752in}}%
\pgfpathlineto{\pgfqpoint{1.831137in}{1.065692in}}%
\pgfpathlineto{\pgfqpoint{1.840375in}{1.076662in}}%
\pgfpathlineto{\pgfqpoint{1.849380in}{1.082147in}}%
\pgfpathlineto{\pgfqpoint{1.858162in}{1.104086in}}%
\pgfpathlineto{\pgfqpoint{1.866732in}{1.109571in}}%
\pgfpathlineto{\pgfqpoint{1.883278in}{1.153450in}}%
\pgfpathlineto{\pgfqpoint{1.899087in}{1.175390in}}%
\pgfpathlineto{\pgfqpoint{1.921558in}{1.180875in}}%
\pgfpathlineto{\pgfqpoint{1.935786in}{1.186360in}}%
\pgfpathlineto{\pgfqpoint{1.942692in}{1.213784in}}%
\pgfpathlineto{\pgfqpoint{1.949467in}{1.230239in}}%
\pgfpathlineto{\pgfqpoint{1.956115in}{1.235724in}}%
\pgfpathlineto{\pgfqpoint{1.969049in}{1.257663in}}%
\pgfpathlineto{\pgfqpoint{1.975344in}{1.268633in}}%
\pgfpathlineto{\pgfqpoint{1.987610in}{1.274118in}}%
\pgfpathlineto{\pgfqpoint{1.999466in}{1.290573in}}%
\pgfpathlineto{\pgfqpoint{2.016541in}{1.296058in}}%
\pgfpathlineto{\pgfqpoint{2.027485in}{1.317997in}}%
\pgfpathlineto{\pgfqpoint{2.038103in}{1.328967in}}%
\pgfpathlineto{\pgfqpoint{2.058432in}{1.339937in}}%
\pgfpathlineto{\pgfqpoint{2.063338in}{1.350906in}}%
\pgfpathlineto{\pgfqpoint{2.068177in}{1.356391in}}%
\pgfpathlineto{\pgfqpoint{2.077661in}{1.389301in}}%
\pgfpathlineto{\pgfqpoint{2.082311in}{1.394786in}}%
\pgfpathlineto{\pgfqpoint{2.091431in}{1.400271in}}%
\pgfpathlineto{\pgfqpoint{2.095904in}{1.416725in}}%
\pgfpathlineto{\pgfqpoint{2.104687in}{1.422210in}}%
\pgfpathlineto{\pgfqpoint{2.108998in}{1.433180in}}%
\pgfpathlineto{\pgfqpoint{2.117466in}{1.444150in}}%
\pgfpathlineto{\pgfqpoint{2.129802in}{1.449635in}}%
\pgfpathlineto{\pgfqpoint{2.145612in}{1.460604in}}%
\pgfpathlineto{\pgfqpoint{2.153261in}{1.466089in}}%
\pgfpathlineto{\pgfqpoint{2.157025in}{1.471574in}}%
\pgfpathlineto{\pgfqpoint{2.171693in}{1.477059in}}%
\pgfpathlineto{\pgfqpoint{2.185780in}{1.488029in}}%
\pgfpathlineto{\pgfqpoint{2.189217in}{1.493514in}}%
\pgfpathlineto{\pgfqpoint{2.202639in}{1.498999in}}%
\pgfpathlineto{\pgfqpoint{2.205917in}{1.504484in}}%
\pgfpathlineto{\pgfqpoint{2.215574in}{1.509968in}}%
\pgfpathlineto{\pgfqpoint{2.231107in}{1.515453in}}%
\pgfpathlineto{\pgfqpoint{2.240112in}{1.531908in}}%
\pgfpathlineto{\pgfqpoint{2.251774in}{1.542878in}}%
\pgfpathlineto{\pgfqpoint{2.254631in}{1.548363in}}%
\pgfpathlineto{\pgfqpoint{2.276694in}{1.553848in}}%
\pgfpathlineto{\pgfqpoint{2.294937in}{1.559333in}}%
\pgfpathlineto{\pgfqpoint{2.319475in}{1.564817in}}%
\pgfpathlineto{\pgfqpoint{2.328835in}{1.575787in}}%
\pgfpathlineto{\pgfqpoint{2.340199in}{1.581272in}}%
\pgfpathlineto{\pgfqpoint{2.344645in}{1.586757in}}%
\pgfpathlineto{\pgfqpoint{2.346847in}{1.592242in}}%
\pgfpathlineto{\pgfqpoint{2.378342in}{1.597727in}}%
\pgfpathlineto{\pgfqpoint{2.388250in}{1.603212in}}%
\pgfpathlineto{\pgfqpoint{2.394064in}{1.608697in}}%
\pgfpathlineto{\pgfqpoint{2.405416in}{1.614181in}}%
\pgfpathlineto{\pgfqpoint{2.407273in}{1.619666in}}%
\pgfpathlineto{\pgfqpoint{2.410959in}{1.625151in}}%
\pgfpathlineto{\pgfqpoint{2.418218in}{1.630636in}}%
\pgfpathlineto{\pgfqpoint{2.434027in}{1.636121in}}%
\pgfpathlineto{\pgfqpoint{2.450807in}{1.641606in}}%
\pgfpathlineto{\pgfqpoint{2.471500in}{1.647091in}}%
\pgfpathlineto{\pgfqpoint{2.473043in}{1.652576in}}%
\pgfpathlineto{\pgfqpoint{2.486637in}{1.658061in}}%
\pgfpathlineto{\pgfqpoint{2.505398in}{1.663546in}}%
\pgfpathlineto{\pgfqpoint{2.513734in}{1.669030in}}%
\pgfpathlineto{\pgfqpoint{2.527207in}{1.680000in}}%
\pgfpathlineto{\pgfqpoint{2.535053in}{1.685485in}}%
\pgfpathlineto{\pgfqpoint{2.566000in}{1.690970in}}%
\pgfpathlineto{\pgfqpoint{2.593372in}{1.696455in}}%
\pgfpathlineto{\pgfqpoint{2.625870in}{1.701940in}}%
\pgfpathlineto{\pgfqpoint{2.648197in}{1.707425in}}%
\pgfpathlineto{\pgfqpoint{2.661131in}{1.712910in}}%
\pgfpathlineto{\pgfqpoint{2.683120in}{1.718394in}}%
\pgfpathlineto{\pgfqpoint{2.755419in}{1.723879in}}%
\pgfpathlineto{\pgfqpoint{2.777022in}{1.729364in}}%
\pgfpathlineto{\pgfqpoint{2.797387in}{1.740334in}}%
\pgfpathlineto{\pgfqpoint{2.870888in}{1.745819in}}%
\pgfpathlineto{\pgfqpoint{2.982270in}{1.751304in}}%
\pgfpathlineto{\pgfqpoint{3.104348in}{1.756789in}}%
\pgfpathlineto{\pgfqpoint{3.129772in}{1.762274in}}%
\pgfpathlineto{\pgfqpoint{3.138175in}{1.767759in}}%
\pgfpathlineto{\pgfqpoint{3.172956in}{1.773243in}}%
\pgfpathlineto{\pgfqpoint{3.174887in}{1.778728in}}%
\pgfpathlineto{\pgfqpoint{3.175315in}{1.784213in}}%
\pgfpathlineto{\pgfqpoint{3.180197in}{1.789698in}}%
\pgfpathlineto{\pgfqpoint{3.337445in}{1.795183in}}%
\pgfpathlineto{\pgfqpoint{3.382557in}{1.800668in}}%
\pgfpathlineto{\pgfqpoint{3.421519in}{1.806153in}}%
\pgfpathlineto{\pgfqpoint{3.554682in}{1.811638in}}%
\pgfpathlineto{\pgfqpoint{3.574356in}{1.817123in}}%
\pgfpathlineto{\pgfqpoint{3.582272in}{1.822607in}}%
\pgfpathlineto{\pgfqpoint{3.606542in}{1.828092in}}%
\pgfpathlineto{\pgfqpoint{3.606987in}{1.833577in}}%
\pgfpathlineto{\pgfqpoint{3.613405in}{1.839062in}}%
\pgfpathlineto{\pgfqpoint{3.617929in}{1.844547in}}%
\pgfpathlineto{\pgfqpoint{3.632881in}{1.850032in}}%
\pgfpathlineto{\pgfqpoint{3.636464in}{1.855517in}}%
\pgfpathlineto{\pgfqpoint{3.716703in}{1.861002in}}%
\pgfpathlineto{\pgfqpoint{3.743167in}{1.866487in}}%
\pgfpathlineto{\pgfqpoint{3.912449in}{1.871972in}}%
\pgfpathlineto{\pgfqpoint{3.937455in}{1.877456in}}%
\pgfpathlineto{\pgfqpoint{4.066080in}{1.882941in}}%
\pgfpathlineto{\pgfqpoint{4.292116in}{1.893911in}}%
\pgfpathlineto{\pgfqpoint{4.303190in}{1.899396in}}%
\pgfpathlineto{\pgfqpoint{4.388098in}{1.904881in}}%
\pgfpathlineto{\pgfqpoint{4.482150in}{1.910366in}}%
\pgfpathlineto{\pgfqpoint{4.530368in}{1.915851in}}%
\pgfpathlineto{\pgfqpoint{4.563007in}{1.921336in}}%
\pgfpathlineto{\pgfqpoint{4.563007in}{1.921336in}}%
\pgfusepath{stroke}%
\end{pgfscope}%
\begin{pgfscope}%
\pgfpathrectangle{\pgfqpoint{0.537394in}{0.467838in}}{\pgfqpoint{4.094684in}{2.193958in}}%
\pgfusepath{clip}%
\pgfsetbuttcap%
\pgfsetroundjoin%
\pgfsetlinewidth{1.003750pt}%
\definecolor{currentstroke}{rgb}{0.549020,0.337255,0.294118}%
\pgfsetstrokecolor{currentstroke}%
\pgfsetdash{{1.000000pt}{1.650000pt}}{0.000000pt}%
\pgfpathmoveto{\pgfqpoint{0.928126in}{0.473323in}}%
\pgfpathlineto{\pgfqpoint{1.030443in}{0.478808in}}%
\pgfpathlineto{\pgfqpoint{1.109806in}{0.495263in}}%
\pgfpathlineto{\pgfqpoint{1.174651in}{0.506233in}}%
\pgfpathlineto{\pgfqpoint{1.229476in}{0.544627in}}%
\pgfpathlineto{\pgfqpoint{1.276968in}{0.583021in}}%
\pgfpathlineto{\pgfqpoint{1.318858in}{0.610446in}}%
\pgfpathlineto{\pgfqpoint{1.356331in}{0.615931in}}%
\pgfpathlineto{\pgfqpoint{1.390229in}{0.637870in}}%
\pgfpathlineto{\pgfqpoint{1.421175in}{0.654325in}}%
\pgfpathlineto{\pgfqpoint{1.449643in}{0.670779in}}%
\pgfpathlineto{\pgfqpoint{1.476000in}{0.692719in}}%
\pgfpathlineto{\pgfqpoint{1.500538in}{0.703689in}}%
\pgfpathlineto{\pgfqpoint{1.523492in}{0.725628in}}%
\pgfpathlineto{\pgfqpoint{1.545054in}{0.753053in}}%
\pgfpathlineto{\pgfqpoint{1.565383in}{0.758538in}}%
\pgfpathlineto{\pgfqpoint{1.584612in}{0.769508in}}%
\pgfpathlineto{\pgfqpoint{1.602855in}{0.791447in}}%
\pgfpathlineto{\pgfqpoint{1.620208in}{0.807902in}}%
\pgfpathlineto{\pgfqpoint{1.636753in}{0.829841in}}%
\pgfpathlineto{\pgfqpoint{1.652563in}{0.857266in}}%
\pgfpathlineto{\pgfqpoint{1.667700in}{0.895660in}}%
\pgfpathlineto{\pgfqpoint{1.682218in}{0.912115in}}%
\pgfpathlineto{\pgfqpoint{1.696168in}{0.939539in}}%
\pgfpathlineto{\pgfqpoint{1.709590in}{0.961479in}}%
\pgfpathlineto{\pgfqpoint{1.722525in}{0.983418in}}%
\pgfpathlineto{\pgfqpoint{1.735005in}{1.005358in}}%
\pgfpathlineto{\pgfqpoint{1.747063in}{1.021813in}}%
\pgfpathlineto{\pgfqpoint{1.770017in}{1.032783in}}%
\pgfpathlineto{\pgfqpoint{1.780961in}{1.054722in}}%
\pgfpathlineto{\pgfqpoint{1.791578in}{1.071177in}}%
\pgfpathlineto{\pgfqpoint{1.801888in}{1.093116in}}%
\pgfpathlineto{\pgfqpoint{1.811907in}{1.109571in}}%
\pgfpathlineto{\pgfqpoint{1.821652in}{1.131511in}}%
\pgfpathlineto{\pgfqpoint{1.840375in}{1.142480in}}%
\pgfpathlineto{\pgfqpoint{1.858162in}{1.147965in}}%
\pgfpathlineto{\pgfqpoint{1.866732in}{1.153450in}}%
\pgfpathlineto{\pgfqpoint{1.875101in}{1.169905in}}%
\pgfpathlineto{\pgfqpoint{1.883278in}{1.175390in}}%
\pgfpathlineto{\pgfqpoint{1.914224in}{1.186360in}}%
\pgfpathlineto{\pgfqpoint{1.921558in}{1.197329in}}%
\pgfpathlineto{\pgfqpoint{1.942692in}{1.213784in}}%
\pgfpathlineto{\pgfqpoint{1.949467in}{1.219269in}}%
\pgfpathlineto{\pgfqpoint{1.956115in}{1.230239in}}%
\pgfpathlineto{\pgfqpoint{1.969049in}{1.241209in}}%
\pgfpathlineto{\pgfqpoint{1.981530in}{1.263148in}}%
\pgfpathlineto{\pgfqpoint{1.999466in}{1.279603in}}%
\pgfpathlineto{\pgfqpoint{2.005249in}{1.285088in}}%
\pgfpathlineto{\pgfqpoint{2.016541in}{1.290573in}}%
\pgfpathlineto{\pgfqpoint{2.027485in}{1.301542in}}%
\pgfpathlineto{\pgfqpoint{2.032834in}{1.312512in}}%
\pgfpathlineto{\pgfqpoint{2.048413in}{1.328967in}}%
\pgfpathlineto{\pgfqpoint{2.063338in}{1.345422in}}%
\pgfpathlineto{\pgfqpoint{2.068177in}{1.361876in}}%
\pgfpathlineto{\pgfqpoint{2.082311in}{1.378331in}}%
\pgfpathlineto{\pgfqpoint{2.091431in}{1.389301in}}%
\pgfpathlineto{\pgfqpoint{2.095904in}{1.400271in}}%
\pgfpathlineto{\pgfqpoint{2.104687in}{1.405755in}}%
\pgfpathlineto{\pgfqpoint{2.108998in}{1.416725in}}%
\pgfpathlineto{\pgfqpoint{2.133821in}{1.427695in}}%
\pgfpathlineto{\pgfqpoint{2.141725in}{1.433180in}}%
\pgfpathlineto{\pgfqpoint{2.160749in}{1.438665in}}%
\pgfpathlineto{\pgfqpoint{2.171693in}{1.444150in}}%
\pgfpathlineto{\pgfqpoint{2.175268in}{1.455119in}}%
\pgfpathlineto{\pgfqpoint{2.182311in}{1.466089in}}%
\pgfpathlineto{\pgfqpoint{2.189217in}{1.471574in}}%
\pgfpathlineto{\pgfqpoint{2.195991in}{1.482544in}}%
\pgfpathlineto{\pgfqpoint{2.209166in}{1.493514in}}%
\pgfpathlineto{\pgfqpoint{2.231107in}{1.498999in}}%
\pgfpathlineto{\pgfqpoint{2.245991in}{1.504484in}}%
\pgfpathlineto{\pgfqpoint{2.251774in}{1.509968in}}%
\pgfpathlineto{\pgfqpoint{2.260276in}{1.515453in}}%
\pgfpathlineto{\pgfqpoint{2.282003in}{1.520938in}}%
\pgfpathlineto{\pgfqpoint{2.299982in}{1.531908in}}%
\pgfpathlineto{\pgfqpoint{2.302478in}{1.537393in}}%
\pgfpathlineto{\pgfqpoint{2.319475in}{1.548363in}}%
\pgfpathlineto{\pgfqpoint{2.324186in}{1.559333in}}%
\pgfpathlineto{\pgfqpoint{2.353373in}{1.564817in}}%
\pgfpathlineto{\pgfqpoint{2.355522in}{1.570302in}}%
\pgfpathlineto{\pgfqpoint{2.382338in}{1.575787in}}%
\pgfpathlineto{\pgfqpoint{2.388250in}{1.592242in}}%
\pgfpathlineto{\pgfqpoint{2.394064in}{1.597727in}}%
\pgfpathlineto{\pgfqpoint{2.416417in}{1.603212in}}%
\pgfpathlineto{\pgfqpoint{2.421792in}{1.608697in}}%
\pgfpathlineto{\pgfqpoint{2.447514in}{1.614181in}}%
\pgfpathlineto{\pgfqpoint{2.463683in}{1.625151in}}%
\pgfpathlineto{\pgfqpoint{2.465260in}{1.630636in}}%
\pgfpathlineto{\pgfqpoint{2.468394in}{1.636121in}}%
\pgfpathlineto{\pgfqpoint{2.486637in}{1.641606in}}%
\pgfpathlineto{\pgfqpoint{2.488115in}{1.647091in}}%
\pgfpathlineto{\pgfqpoint{2.499730in}{1.652576in}}%
\pgfpathlineto{\pgfqpoint{2.523219in}{1.658061in}}%
\pgfpathlineto{\pgfqpoint{2.547757in}{1.663546in}}%
\pgfpathlineto{\pgfqpoint{2.549002in}{1.669030in}}%
\pgfpathlineto{\pgfqpoint{2.551481in}{1.674515in}}%
\pgfpathlineto{\pgfqpoint{2.552714in}{1.680000in}}%
\pgfpathlineto{\pgfqpoint{2.615707in}{1.685485in}}%
\pgfpathlineto{\pgfqpoint{2.617763in}{1.690970in}}%
\pgfpathlineto{\pgfqpoint{2.622851in}{1.696455in}}%
\pgfpathlineto{\pgfqpoint{2.628863in}{1.707425in}}%
\pgfpathlineto{\pgfqpoint{2.678829in}{1.712910in}}%
\pgfpathlineto{\pgfqpoint{3.014257in}{1.718394in}}%
\pgfpathlineto{\pgfqpoint{3.016268in}{1.723879in}}%
\pgfpathlineto{\pgfqpoint{3.234631in}{1.729364in}}%
\pgfpathlineto{\pgfqpoint{3.244620in}{1.734849in}}%
\pgfpathlineto{\pgfqpoint{3.252276in}{1.740334in}}%
\pgfpathlineto{\pgfqpoint{3.254507in}{1.745819in}}%
\pgfpathlineto{\pgfqpoint{3.260445in}{1.751304in}}%
\pgfpathlineto{\pgfqpoint{3.383154in}{1.756789in}}%
\pgfpathlineto{\pgfqpoint{3.416130in}{1.767759in}}%
\pgfpathlineto{\pgfqpoint{3.427669in}{1.773243in}}%
\pgfpathlineto{\pgfqpoint{3.461539in}{1.778728in}}%
\pgfpathlineto{\pgfqpoint{3.501630in}{1.784213in}}%
\pgfpathlineto{\pgfqpoint{3.599608in}{1.789698in}}%
\pgfpathlineto{\pgfqpoint{3.710919in}{1.795183in}}%
\pgfpathlineto{\pgfqpoint{3.712244in}{1.800668in}}%
\pgfpathlineto{\pgfqpoint{3.726048in}{1.806153in}}%
\pgfpathlineto{\pgfqpoint{3.764795in}{1.811638in}}%
\pgfpathlineto{\pgfqpoint{3.766299in}{1.817123in}}%
\pgfpathlineto{\pgfqpoint{3.776537in}{1.822607in}}%
\pgfpathlineto{\pgfqpoint{3.784412in}{1.828092in}}%
\pgfpathlineto{\pgfqpoint{4.014463in}{1.833577in}}%
\pgfpathlineto{\pgfqpoint{4.035305in}{1.839062in}}%
\pgfpathlineto{\pgfqpoint{4.069332in}{1.844547in}}%
\pgfpathlineto{\pgfqpoint{4.070610in}{1.850032in}}%
\pgfpathlineto{\pgfqpoint{4.078455in}{1.855517in}}%
\pgfpathlineto{\pgfqpoint{4.082547in}{1.861002in}}%
\pgfpathlineto{\pgfqpoint{4.083713in}{1.866487in}}%
\pgfpathlineto{\pgfqpoint{4.102519in}{1.871972in}}%
\pgfpathlineto{\pgfqpoint{4.280542in}{1.877456in}}%
\pgfpathlineto{\pgfqpoint{4.285910in}{1.882941in}}%
\pgfpathlineto{\pgfqpoint{4.287188in}{1.888426in}}%
\pgfpathlineto{\pgfqpoint{4.324591in}{1.893911in}}%
\pgfpathlineto{\pgfqpoint{4.345638in}{1.899396in}}%
\pgfpathlineto{\pgfqpoint{4.447410in}{1.904881in}}%
\pgfpathlineto{\pgfqpoint{4.447410in}{1.904881in}}%
\pgfusepath{stroke}%
\end{pgfscope}%
\begin{pgfscope}%
\pgfpathrectangle{\pgfqpoint{0.537394in}{0.467838in}}{\pgfqpoint{4.094684in}{2.193958in}}%
\pgfusepath{clip}%
\pgfsetbuttcap%
\pgfsetroundjoin%
\pgfsetlinewidth{1.003750pt}%
\definecolor{currentstroke}{rgb}{0.890196,0.466667,0.760784}%
\pgfsetstrokecolor{currentstroke}%
\pgfsetdash{{6.400000pt}{1.600000pt}{1.000000pt}{1.600000pt}}{0.000000pt}%
\pgfpathmoveto{\pgfqpoint{1.030443in}{0.473323in}}%
\pgfpathlineto{\pgfqpoint{1.229476in}{0.478808in}}%
\pgfpathlineto{\pgfqpoint{1.476000in}{0.484293in}}%
\pgfpathlineto{\pgfqpoint{1.500538in}{0.489778in}}%
\pgfpathlineto{\pgfqpoint{1.565383in}{0.495263in}}%
\pgfpathlineto{\pgfqpoint{1.584612in}{0.500748in}}%
\pgfpathlineto{\pgfqpoint{1.602855in}{0.506233in}}%
\pgfpathlineto{\pgfqpoint{1.620208in}{0.511717in}}%
\pgfpathlineto{\pgfqpoint{1.636753in}{0.522687in}}%
\pgfpathlineto{\pgfqpoint{1.652563in}{0.528172in}}%
\pgfpathlineto{\pgfqpoint{1.667700in}{0.539142in}}%
\pgfpathlineto{\pgfqpoint{1.696168in}{0.544627in}}%
\pgfpathlineto{\pgfqpoint{1.709590in}{0.550112in}}%
\pgfpathlineto{\pgfqpoint{1.722525in}{0.561082in}}%
\pgfpathlineto{\pgfqpoint{1.747063in}{0.566566in}}%
\pgfpathlineto{\pgfqpoint{1.758725in}{0.572051in}}%
\pgfpathlineto{\pgfqpoint{1.780961in}{0.577536in}}%
\pgfpathlineto{\pgfqpoint{1.801888in}{0.588506in}}%
\pgfpathlineto{\pgfqpoint{1.821652in}{0.593991in}}%
\pgfpathlineto{\pgfqpoint{1.831137in}{0.599476in}}%
\pgfpathlineto{\pgfqpoint{1.849380in}{0.604961in}}%
\pgfpathlineto{\pgfqpoint{1.866732in}{0.621415in}}%
\pgfpathlineto{\pgfqpoint{1.875101in}{0.632385in}}%
\pgfpathlineto{\pgfqpoint{1.956115in}{0.637870in}}%
\pgfpathlineto{\pgfqpoint{1.975344in}{0.643355in}}%
\pgfpathlineto{\pgfqpoint{2.005249in}{0.648840in}}%
\pgfpathlineto{\pgfqpoint{2.016541in}{0.654325in}}%
\pgfpathlineto{\pgfqpoint{2.032834in}{0.659810in}}%
\pgfpathlineto{\pgfqpoint{2.053458in}{0.670779in}}%
\pgfpathlineto{\pgfqpoint{2.082311in}{0.676264in}}%
\pgfpathlineto{\pgfqpoint{2.091431in}{0.681749in}}%
\pgfpathlineto{\pgfqpoint{2.113257in}{0.698204in}}%
\pgfpathlineto{\pgfqpoint{2.121626in}{0.703689in}}%
\pgfpathlineto{\pgfqpoint{2.149457in}{0.709174in}}%
\pgfpathlineto{\pgfqpoint{2.157025in}{0.714659in}}%
\pgfpathlineto{\pgfqpoint{2.160749in}{0.720144in}}%
\pgfpathlineto{\pgfqpoint{2.175268in}{0.725628in}}%
\pgfpathlineto{\pgfqpoint{2.182311in}{0.731113in}}%
\pgfpathlineto{\pgfqpoint{2.215574in}{0.742083in}}%
\pgfpathlineto{\pgfqpoint{2.265834in}{0.747568in}}%
\pgfpathlineto{\pgfqpoint{2.307418in}{0.758538in}}%
\pgfpathlineto{\pgfqpoint{2.324186in}{0.764023in}}%
\pgfpathlineto{\pgfqpoint{2.337955in}{0.769508in}}%
\pgfpathlineto{\pgfqpoint{2.366077in}{0.774992in}}%
\pgfpathlineto{\pgfqpoint{2.374300in}{0.780477in}}%
\pgfpathlineto{\pgfqpoint{2.471500in}{0.785962in}}%
\pgfpathlineto{\pgfqpoint{2.502575in}{0.791447in}}%
\pgfpathlineto{\pgfqpoint{2.517830in}{0.796932in}}%
\pgfpathlineto{\pgfqpoint{2.520534in}{0.802417in}}%
\pgfpathlineto{\pgfqpoint{2.525883in}{0.807902in}}%
\pgfpathlineto{\pgfqpoint{2.538912in}{0.813387in}}%
\pgfpathlineto{\pgfqpoint{2.568363in}{0.818872in}}%
\pgfpathlineto{\pgfqpoint{2.569539in}{0.824357in}}%
\pgfpathlineto{\pgfqpoint{2.577662in}{0.829841in}}%
\pgfpathlineto{\pgfqpoint{2.599898in}{0.835326in}}%
\pgfpathlineto{\pgfqpoint{2.609468in}{0.846296in}}%
\pgfpathlineto{\pgfqpoint{2.611560in}{0.851781in}}%
\pgfpathlineto{\pgfqpoint{2.612601in}{0.857266in}}%
\pgfpathlineto{\pgfqpoint{2.615707in}{0.868236in}}%
\pgfpathlineto{\pgfqpoint{2.625870in}{0.873721in}}%
\pgfpathlineto{\pgfqpoint{2.639626in}{0.879205in}}%
\pgfpathlineto{\pgfqpoint{2.641549in}{0.884690in}}%
\pgfpathlineto{\pgfqpoint{2.647255in}{0.895660in}}%
\pgfpathlineto{\pgfqpoint{2.664742in}{0.901145in}}%
\pgfpathlineto{\pgfqpoint{2.673612in}{0.906630in}}%
\pgfpathlineto{\pgfqpoint{2.674487in}{0.912115in}}%
\pgfpathlineto{\pgfqpoint{2.698150in}{0.917600in}}%
\pgfpathlineto{\pgfqpoint{2.725673in}{0.923085in}}%
\pgfpathlineto{\pgfqpoint{2.744823in}{0.928570in}}%
\pgfpathlineto{\pgfqpoint{2.756809in}{0.934054in}}%
\pgfpathlineto{\pgfqpoint{2.778982in}{0.939539in}}%
\pgfpathlineto{\pgfqpoint{2.802302in}{0.945024in}}%
\pgfpathlineto{\pgfqpoint{2.816063in}{0.950509in}}%
\pgfpathlineto{\pgfqpoint{2.869881in}{0.955994in}}%
\pgfpathlineto{\pgfqpoint{2.873395in}{0.961479in}}%
\pgfpathlineto{\pgfqpoint{2.874890in}{0.966964in}}%
\pgfpathlineto{\pgfqpoint{2.887594in}{0.972449in}}%
\pgfpathlineto{\pgfqpoint{2.890462in}{0.983418in}}%
\pgfpathlineto{\pgfqpoint{2.894721in}{0.988903in}}%
\pgfpathlineto{\pgfqpoint{2.901709in}{0.994388in}}%
\pgfpathlineto{\pgfqpoint{2.903549in}{0.999873in}}%
\pgfpathlineto{\pgfqpoint{2.909917in}{1.005358in}}%
\pgfpathlineto{\pgfqpoint{2.911715in}{1.016328in}}%
\pgfpathlineto{\pgfqpoint{2.914396in}{1.021813in}}%
\pgfpathlineto{\pgfqpoint{2.920574in}{1.032783in}}%
\pgfpathlineto{\pgfqpoint{2.922755in}{1.038267in}}%
\pgfpathlineto{\pgfqpoint{2.927934in}{1.043752in}}%
\pgfpathlineto{\pgfqpoint{2.930071in}{1.049237in}}%
\pgfpathlineto{\pgfqpoint{2.931770in}{1.054722in}}%
\pgfpathlineto{\pgfqpoint{2.934725in}{1.060207in}}%
\pgfpathlineto{\pgfqpoint{2.938489in}{1.065692in}}%
\pgfpathlineto{\pgfqpoint{2.941389in}{1.071177in}}%
\pgfpathlineto{\pgfqpoint{2.947525in}{1.076662in}}%
\pgfpathlineto{\pgfqpoint{2.948335in}{1.082147in}}%
\pgfpathlineto{\pgfqpoint{2.958309in}{1.087632in}}%
\pgfpathlineto{\pgfqpoint{2.969920in}{1.093116in}}%
\pgfpathlineto{\pgfqpoint{2.972954in}{1.098601in}}%
\pgfpathlineto{\pgfqpoint{2.977456in}{1.104086in}}%
\pgfpathlineto{\pgfqpoint{2.980426in}{1.115056in}}%
\pgfpathlineto{\pgfqpoint{2.981164in}{1.120541in}}%
\pgfpathlineto{\pgfqpoint{2.988106in}{1.126026in}}%
\pgfpathlineto{\pgfqpoint{2.988829in}{1.131511in}}%
\pgfpathlineto{\pgfqpoint{2.993492in}{1.136996in}}%
\pgfpathlineto{\pgfqpoint{2.995270in}{1.147965in}}%
\pgfpathlineto{\pgfqpoint{2.997038in}{1.153450in}}%
\pgfpathlineto{\pgfqpoint{3.001247in}{1.158935in}}%
\pgfpathlineto{\pgfqpoint{3.012234in}{1.164420in}}%
\pgfpathlineto{\pgfqpoint{3.014593in}{1.169905in}}%
\pgfpathlineto{\pgfqpoint{3.015599in}{1.175390in}}%
\pgfpathlineto{\pgfqpoint{3.018268in}{1.180875in}}%
\pgfpathlineto{\pgfqpoint{3.019264in}{1.186360in}}%
\pgfpathlineto{\pgfqpoint{3.019595in}{1.191845in}}%
\pgfpathlineto{\pgfqpoint{3.020587in}{1.197329in}}%
\pgfpathlineto{\pgfqpoint{3.024201in}{1.202814in}}%
\pgfpathlineto{\pgfqpoint{3.037042in}{1.208299in}}%
\pgfpathlineto{\pgfqpoint{3.041429in}{1.213784in}}%
\pgfpathlineto{\pgfqpoint{3.087243in}{1.219269in}}%
\pgfpathlineto{\pgfqpoint{3.088063in}{1.224754in}}%
\pgfpathlineto{\pgfqpoint{3.096966in}{1.230239in}}%
\pgfpathlineto{\pgfqpoint{3.101729in}{1.235724in}}%
\pgfpathlineto{\pgfqpoint{3.105390in}{1.241209in}}%
\pgfpathlineto{\pgfqpoint{3.126845in}{1.246693in}}%
\pgfpathlineto{\pgfqpoint{3.139123in}{1.252178in}}%
\pgfpathlineto{\pgfqpoint{3.141246in}{1.257663in}}%
\pgfpathlineto{\pgfqpoint{3.143123in}{1.268633in}}%
\pgfpathlineto{\pgfqpoint{3.146847in}{1.274118in}}%
\pgfpathlineto{\pgfqpoint{3.148695in}{1.285088in}}%
\pgfpathlineto{\pgfqpoint{3.154860in}{1.290573in}}%
\pgfpathlineto{\pgfqpoint{3.174887in}{1.296058in}}%
\pgfpathlineto{\pgfqpoint{3.178083in}{1.301542in}}%
\pgfpathlineto{\pgfqpoint{3.195062in}{1.307027in}}%
\pgfpathlineto{\pgfqpoint{3.195869in}{1.317997in}}%
\pgfpathlineto{\pgfqpoint{3.198683in}{1.323482in}}%
\pgfpathlineto{\pgfqpoint{3.242680in}{1.328967in}}%
\pgfpathlineto{\pgfqpoint{3.282778in}{1.339937in}}%
\pgfpathlineto{\pgfqpoint{3.330605in}{1.345422in}}%
\pgfpathlineto{\pgfqpoint{3.333494in}{1.350906in}}%
\pgfpathlineto{\pgfqpoint{3.335951in}{1.356391in}}%
\pgfpathlineto{\pgfqpoint{3.338122in}{1.361876in}}%
\pgfpathlineto{\pgfqpoint{3.338527in}{1.367361in}}%
\pgfpathlineto{\pgfqpoint{3.343756in}{1.372846in}}%
\pgfpathlineto{\pgfqpoint{3.352694in}{1.378331in}}%
\pgfpathlineto{\pgfqpoint{3.354507in}{1.383816in}}%
\pgfpathlineto{\pgfqpoint{3.362299in}{1.389301in}}%
\pgfpathlineto{\pgfqpoint{3.363434in}{1.394786in}}%
\pgfpathlineto{\pgfqpoint{3.371897in}{1.400271in}}%
\pgfpathlineto{\pgfqpoint{3.379561in}{1.405755in}}%
\pgfpathlineto{\pgfqpoint{3.394527in}{1.411240in}}%
\pgfpathlineto{\pgfqpoint{3.397286in}{1.416725in}}%
\pgfpathlineto{\pgfqpoint{3.421519in}{1.422210in}}%
\pgfpathlineto{\pgfqpoint{3.425560in}{1.427695in}}%
\pgfpathlineto{\pgfqpoint{3.458467in}{1.433180in}}%
\pgfpathlineto{\pgfqpoint{3.466851in}{1.438665in}}%
\pgfpathlineto{\pgfqpoint{3.467698in}{1.444150in}}%
\pgfpathlineto{\pgfqpoint{3.471807in}{1.449635in}}%
\pgfpathlineto{\pgfqpoint{3.473105in}{1.455119in}}%
\pgfpathlineto{\pgfqpoint{3.477244in}{1.460604in}}%
\pgfpathlineto{\pgfqpoint{3.509150in}{1.466089in}}%
\pgfpathlineto{\pgfqpoint{3.515530in}{1.471574in}}%
\pgfpathlineto{\pgfqpoint{3.516186in}{1.482544in}}%
\pgfpathlineto{\pgfqpoint{3.519125in}{1.488029in}}%
\pgfpathlineto{\pgfqpoint{3.520990in}{1.493514in}}%
\pgfpathlineto{\pgfqpoint{3.609328in}{1.498999in}}%
\pgfpathlineto{\pgfqpoint{3.613031in}{1.504484in}}%
\pgfpathlineto{\pgfqpoint{3.638562in}{1.509968in}}%
\pgfpathlineto{\pgfqpoint{3.640301in}{1.515453in}}%
\pgfpathlineto{\pgfqpoint{3.647400in}{1.526423in}}%
\pgfpathlineto{\pgfqpoint{3.648870in}{1.531908in}}%
\pgfpathlineto{\pgfqpoint{3.655303in}{1.537393in}}%
\pgfpathlineto{\pgfqpoint{3.656907in}{1.542878in}}%
\pgfpathlineto{\pgfqpoint{3.658998in}{1.548363in}}%
\pgfpathlineto{\pgfqpoint{3.674457in}{1.553848in}}%
\pgfpathlineto{\pgfqpoint{3.677542in}{1.559333in}}%
\pgfpathlineto{\pgfqpoint{3.679515in}{1.564817in}}%
\pgfpathlineto{\pgfqpoint{3.679723in}{1.570302in}}%
\pgfpathlineto{\pgfqpoint{3.681787in}{1.575787in}}%
\pgfpathlineto{\pgfqpoint{3.689372in}{1.581272in}}%
\pgfpathlineto{\pgfqpoint{3.690880in}{1.586757in}}%
\pgfpathlineto{\pgfqpoint{3.693926in}{1.592242in}}%
\pgfpathlineto{\pgfqpoint{3.710492in}{1.597727in}}%
\pgfpathlineto{\pgfqpoint{3.745927in}{1.603212in}}%
\pgfpathlineto{\pgfqpoint{3.766299in}{1.608697in}}%
\pgfpathlineto{\pgfqpoint{3.788171in}{1.614181in}}%
\pgfpathlineto{\pgfqpoint{3.795310in}{1.619666in}}%
\pgfpathlineto{\pgfqpoint{3.796430in}{1.625151in}}%
\pgfpathlineto{\pgfqpoint{3.798363in}{1.630636in}}%
\pgfpathlineto{\pgfqpoint{3.800027in}{1.636121in}}%
\pgfpathlineto{\pgfqpoint{3.800617in}{1.641606in}}%
\pgfpathlineto{\pgfqpoint{3.809315in}{1.647091in}}%
\pgfpathlineto{\pgfqpoint{3.814348in}{1.652576in}}%
\pgfpathlineto{\pgfqpoint{3.824241in}{1.658061in}}%
\pgfpathlineto{\pgfqpoint{3.824344in}{1.663546in}}%
\pgfpathlineto{\pgfqpoint{3.832285in}{1.669030in}}%
\pgfpathlineto{\pgfqpoint{3.832756in}{1.674515in}}%
\pgfpathlineto{\pgfqpoint{3.852685in}{1.680000in}}%
\pgfpathlineto{\pgfqpoint{3.852812in}{1.685485in}}%
\pgfpathlineto{\pgfqpoint{3.871646in}{1.690970in}}%
\pgfpathlineto{\pgfqpoint{3.873362in}{1.696455in}}%
\pgfpathlineto{\pgfqpoint{3.882287in}{1.701940in}}%
\pgfpathlineto{\pgfqpoint{3.928414in}{1.707425in}}%
\pgfpathlineto{\pgfqpoint{3.928517in}{1.712910in}}%
\pgfpathlineto{\pgfqpoint{3.961052in}{1.718394in}}%
\pgfpathlineto{\pgfqpoint{3.961240in}{1.723879in}}%
\pgfpathlineto{\pgfqpoint{4.020013in}{1.729364in}}%
\pgfpathlineto{\pgfqpoint{4.020569in}{1.734849in}}%
\pgfpathlineto{\pgfqpoint{4.021541in}{1.740334in}}%
\pgfpathlineto{\pgfqpoint{4.021798in}{1.745819in}}%
\pgfpathlineto{\pgfqpoint{4.023337in}{1.751304in}}%
\pgfpathlineto{\pgfqpoint{4.026416in}{1.756789in}}%
\pgfpathlineto{\pgfqpoint{4.031110in}{1.762274in}}%
\pgfpathlineto{\pgfqpoint{4.032072in}{1.767759in}}%
\pgfpathlineto{\pgfqpoint{4.036958in}{1.773243in}}%
\pgfpathlineto{\pgfqpoint{4.045776in}{1.778728in}}%
\pgfpathlineto{\pgfqpoint{4.050260in}{1.784213in}}%
\pgfpathlineto{\pgfqpoint{4.052843in}{1.789698in}}%
\pgfpathlineto{\pgfqpoint{4.054741in}{1.795183in}}%
\pgfpathlineto{\pgfqpoint{4.054759in}{1.800668in}}%
\pgfpathlineto{\pgfqpoint{4.058811in}{1.806153in}}%
\pgfpathlineto{\pgfqpoint{4.089777in}{1.811638in}}%
\pgfpathlineto{\pgfqpoint{4.130870in}{1.817123in}}%
\pgfpathlineto{\pgfqpoint{4.132090in}{1.822607in}}%
\pgfpathlineto{\pgfqpoint{4.217298in}{1.828092in}}%
\pgfpathlineto{\pgfqpoint{4.237631in}{1.833577in}}%
\pgfpathlineto{\pgfqpoint{4.245374in}{1.839062in}}%
\pgfpathlineto{\pgfqpoint{4.279317in}{1.844547in}}%
\pgfpathlineto{\pgfqpoint{4.282819in}{1.850032in}}%
\pgfpathlineto{\pgfqpoint{4.286644in}{1.855517in}}%
\pgfpathlineto{\pgfqpoint{4.298378in}{1.861002in}}%
\pgfpathlineto{\pgfqpoint{4.302103in}{1.866487in}}%
\pgfpathlineto{\pgfqpoint{4.323314in}{1.871972in}}%
\pgfpathlineto{\pgfqpoint{4.323568in}{1.877456in}}%
\pgfpathlineto{\pgfqpoint{4.345654in}{1.882941in}}%
\pgfpathlineto{\pgfqpoint{4.351565in}{1.888426in}}%
\pgfpathlineto{\pgfqpoint{4.381461in}{1.893911in}}%
\pgfpathlineto{\pgfqpoint{4.383528in}{1.904881in}}%
\pgfpathlineto{\pgfqpoint{4.387695in}{1.910366in}}%
\pgfpathlineto{\pgfqpoint{4.392806in}{1.915851in}}%
\pgfpathlineto{\pgfqpoint{4.397795in}{1.926820in}}%
\pgfpathlineto{\pgfqpoint{4.403765in}{1.932305in}}%
\pgfpathlineto{\pgfqpoint{4.412589in}{1.937790in}}%
\pgfpathlineto{\pgfqpoint{4.412589in}{1.937790in}}%
\pgfusepath{stroke}%
\end{pgfscope}%
\begin{pgfscope}%
\pgfpathrectangle{\pgfqpoint{0.537394in}{0.467838in}}{\pgfqpoint{4.094684in}{2.193958in}}%
\pgfusepath{clip}%
\pgfsetbuttcap%
\pgfsetroundjoin%
\pgfsetlinewidth{1.003750pt}%
\definecolor{currentstroke}{rgb}{0.498039,0.498039,0.498039}%
\pgfsetstrokecolor{currentstroke}%
\pgfsetdash{{6.400000pt}{1.600000pt}{1.000000pt}{1.600000pt}}{0.000000pt}%
\pgfpathmoveto{\pgfqpoint{0.928126in}{0.517202in}}%
\pgfpathlineto{\pgfqpoint{1.030443in}{0.544627in}}%
\pgfpathlineto{\pgfqpoint{1.109806in}{0.550112in}}%
\pgfpathlineto{\pgfqpoint{1.174651in}{0.572051in}}%
\pgfpathlineto{\pgfqpoint{1.229476in}{0.583021in}}%
\pgfpathlineto{\pgfqpoint{1.276968in}{0.593991in}}%
\pgfpathlineto{\pgfqpoint{1.318858in}{0.615931in}}%
\pgfpathlineto{\pgfqpoint{1.356331in}{0.621415in}}%
\pgfpathlineto{\pgfqpoint{1.390229in}{0.632385in}}%
\pgfpathlineto{\pgfqpoint{1.421175in}{0.643355in}}%
\pgfpathlineto{\pgfqpoint{1.449643in}{0.648840in}}%
\pgfpathlineto{\pgfqpoint{1.476000in}{0.654325in}}%
\pgfpathlineto{\pgfqpoint{1.545054in}{0.659810in}}%
\pgfpathlineto{\pgfqpoint{1.584612in}{0.676264in}}%
\pgfpathlineto{\pgfqpoint{1.602855in}{0.687234in}}%
\pgfpathlineto{\pgfqpoint{1.620208in}{0.703689in}}%
\pgfpathlineto{\pgfqpoint{1.636753in}{0.709174in}}%
\pgfpathlineto{\pgfqpoint{1.652563in}{0.714659in}}%
\pgfpathlineto{\pgfqpoint{1.667700in}{0.725628in}}%
\pgfpathlineto{\pgfqpoint{1.682218in}{0.742083in}}%
\pgfpathlineto{\pgfqpoint{1.696168in}{0.753053in}}%
\pgfpathlineto{\pgfqpoint{1.709590in}{0.769508in}}%
\pgfpathlineto{\pgfqpoint{1.747063in}{0.796932in}}%
\pgfpathlineto{\pgfqpoint{1.780961in}{0.802417in}}%
\pgfpathlineto{\pgfqpoint{1.791578in}{0.813387in}}%
\pgfpathlineto{\pgfqpoint{1.801888in}{0.829841in}}%
\pgfpathlineto{\pgfqpoint{1.831137in}{0.840811in}}%
\pgfpathlineto{\pgfqpoint{1.840375in}{0.846296in}}%
\pgfpathlineto{\pgfqpoint{1.849380in}{0.857266in}}%
\pgfpathlineto{\pgfqpoint{1.858162in}{0.873721in}}%
\pgfpathlineto{\pgfqpoint{1.875101in}{0.879205in}}%
\pgfpathlineto{\pgfqpoint{1.883278in}{0.890175in}}%
\pgfpathlineto{\pgfqpoint{1.899087in}{0.901145in}}%
\pgfpathlineto{\pgfqpoint{1.906736in}{0.912115in}}%
\pgfpathlineto{\pgfqpoint{1.921558in}{0.923085in}}%
\pgfpathlineto{\pgfqpoint{1.942692in}{0.934054in}}%
\pgfpathlineto{\pgfqpoint{1.949467in}{0.939539in}}%
\pgfpathlineto{\pgfqpoint{1.962641in}{0.955994in}}%
\pgfpathlineto{\pgfqpoint{1.975344in}{0.961479in}}%
\pgfpathlineto{\pgfqpoint{1.987610in}{0.972449in}}%
\pgfpathlineto{\pgfqpoint{2.010940in}{0.983418in}}%
\pgfpathlineto{\pgfqpoint{2.016541in}{0.988903in}}%
\pgfpathlineto{\pgfqpoint{2.027485in}{0.994388in}}%
\pgfpathlineto{\pgfqpoint{2.038103in}{1.016328in}}%
\pgfpathlineto{\pgfqpoint{2.048413in}{1.027298in}}%
\pgfpathlineto{\pgfqpoint{2.053458in}{1.043752in}}%
\pgfpathlineto{\pgfqpoint{2.063338in}{1.065692in}}%
\pgfpathlineto{\pgfqpoint{2.072951in}{1.071177in}}%
\pgfpathlineto{\pgfqpoint{2.077661in}{1.082147in}}%
\pgfpathlineto{\pgfqpoint{2.104687in}{1.087632in}}%
\pgfpathlineto{\pgfqpoint{2.108998in}{1.093116in}}%
\pgfpathlineto{\pgfqpoint{2.129802in}{1.098601in}}%
\pgfpathlineto{\pgfqpoint{2.137795in}{1.109571in}}%
\pgfpathlineto{\pgfqpoint{2.141725in}{1.126026in}}%
\pgfpathlineto{\pgfqpoint{2.145612in}{1.131511in}}%
\pgfpathlineto{\pgfqpoint{2.153261in}{1.136996in}}%
\pgfpathlineto{\pgfqpoint{2.157025in}{1.147965in}}%
\pgfpathlineto{\pgfqpoint{2.178806in}{1.158935in}}%
\pgfpathlineto{\pgfqpoint{2.192620in}{1.169905in}}%
\pgfpathlineto{\pgfqpoint{2.199331in}{1.180875in}}%
\pgfpathlineto{\pgfqpoint{2.215574in}{1.186360in}}%
\pgfpathlineto{\pgfqpoint{2.221869in}{1.191845in}}%
\pgfpathlineto{\pgfqpoint{2.245991in}{1.197329in}}%
\pgfpathlineto{\pgfqpoint{2.263066in}{1.213784in}}%
\pgfpathlineto{\pgfqpoint{2.279358in}{1.219269in}}%
\pgfpathlineto{\pgfqpoint{2.337955in}{1.224754in}}%
\pgfpathlineto{\pgfqpoint{2.340199in}{1.230239in}}%
\pgfpathlineto{\pgfqpoint{2.344645in}{1.235724in}}%
\pgfpathlineto{\pgfqpoint{2.370212in}{1.246693in}}%
\pgfpathlineto{\pgfqpoint{2.378342in}{1.252178in}}%
\pgfpathlineto{\pgfqpoint{2.382338in}{1.257663in}}%
\pgfpathlineto{\pgfqpoint{2.414607in}{1.263148in}}%
\pgfpathlineto{\pgfqpoint{2.439145in}{1.268633in}}%
\pgfpathlineto{\pgfqpoint{2.444190in}{1.285088in}}%
\pgfpathlineto{\pgfqpoint{2.480659in}{1.290573in}}%
\pgfpathlineto{\pgfqpoint{2.499730in}{1.296058in}}%
\pgfpathlineto{\pgfqpoint{2.521879in}{1.301542in}}%
\pgfpathlineto{\pgfqpoint{2.527207in}{1.307027in}}%
\pgfpathlineto{\pgfqpoint{2.538912in}{1.312512in}}%
\pgfpathlineto{\pgfqpoint{2.549002in}{1.317997in}}%
\pgfpathlineto{\pgfqpoint{2.550244in}{1.323482in}}%
\pgfpathlineto{\pgfqpoint{2.553942in}{1.328967in}}%
\pgfpathlineto{\pgfqpoint{2.613640in}{1.339937in}}%
\pgfpathlineto{\pgfqpoint{2.633796in}{1.345422in}}%
\pgfpathlineto{\pgfqpoint{2.639626in}{1.350906in}}%
\pgfpathlineto{\pgfqpoint{2.651008in}{1.372846in}}%
\pgfpathlineto{\pgfqpoint{2.676230in}{1.378331in}}%
\pgfpathlineto{\pgfqpoint{2.714137in}{1.394786in}}%
\pgfpathlineto{\pgfqpoint{2.724916in}{1.405755in}}%
\pgfpathlineto{\pgfqpoint{2.757501in}{1.416725in}}%
\pgfpathlineto{\pgfqpoint{2.773070in}{1.422210in}}%
\pgfpathlineto{\pgfqpoint{2.801079in}{1.427695in}}%
\pgfpathlineto{\pgfqpoint{2.802911in}{1.433180in}}%
\pgfpathlineto{\pgfqpoint{2.846422in}{1.438665in}}%
\pgfpathlineto{\pgfqpoint{2.879339in}{1.444150in}}%
\pgfpathlineto{\pgfqpoint{2.890938in}{1.449635in}}%
\pgfpathlineto{\pgfqpoint{2.892835in}{1.460604in}}%
\pgfpathlineto{\pgfqpoint{2.896130in}{1.466089in}}%
\pgfpathlineto{\pgfqpoint{2.901709in}{1.488029in}}%
\pgfpathlineto{\pgfqpoint{2.903549in}{1.498999in}}%
\pgfpathlineto{\pgfqpoint{2.908562in}{1.504484in}}%
\pgfpathlineto{\pgfqpoint{2.909466in}{1.509968in}}%
\pgfpathlineto{\pgfqpoint{2.916172in}{1.515453in}}%
\pgfpathlineto{\pgfqpoint{2.917940in}{1.526423in}}%
\pgfpathlineto{\pgfqpoint{2.919698in}{1.531908in}}%
\pgfpathlineto{\pgfqpoint{2.921011in}{1.537393in}}%
\pgfpathlineto{\pgfqpoint{2.992064in}{1.542878in}}%
\pgfpathlineto{\pgfqpoint{2.992779in}{1.548363in}}%
\pgfpathlineto{\pgfqpoint{3.009519in}{1.553848in}}%
\pgfpathlineto{\pgfqpoint{3.011218in}{1.559333in}}%
\pgfpathlineto{\pgfqpoint{3.018600in}{1.564817in}}%
\pgfpathlineto{\pgfqpoint{3.019264in}{1.570302in}}%
\pgfpathlineto{\pgfqpoint{3.091867in}{1.575787in}}%
\pgfpathlineto{\pgfqpoint{3.129529in}{1.581272in}}%
\pgfpathlineto{\pgfqpoint{3.133878in}{1.586757in}}%
\pgfpathlineto{\pgfqpoint{3.144057in}{1.592242in}}%
\pgfpathlineto{\pgfqpoint{3.286393in}{1.597727in}}%
\pgfpathlineto{\pgfqpoint{3.288887in}{1.603212in}}%
\pgfpathlineto{\pgfqpoint{3.320661in}{1.608697in}}%
\pgfpathlineto{\pgfqpoint{3.348779in}{1.614181in}}%
\pgfpathlineto{\pgfqpoint{3.354894in}{1.619666in}}%
\pgfpathlineto{\pgfqpoint{3.367566in}{1.625151in}}%
\pgfpathlineto{\pgfqpoint{3.405437in}{1.630636in}}%
\pgfpathlineto{\pgfqpoint{3.420876in}{1.636121in}}%
\pgfpathlineto{\pgfqpoint{3.421839in}{1.641606in}}%
\pgfpathlineto{\pgfqpoint{3.433922in}{1.647091in}}%
\pgfpathlineto{\pgfqpoint{3.479523in}{1.652576in}}%
\pgfpathlineto{\pgfqpoint{3.482870in}{1.658061in}}%
\pgfpathlineto{\pgfqpoint{3.488673in}{1.663546in}}%
\pgfpathlineto{\pgfqpoint{3.495341in}{1.669030in}}%
\pgfpathlineto{\pgfqpoint{3.508145in}{1.674515in}}%
\pgfpathlineto{\pgfqpoint{3.508229in}{1.680000in}}%
\pgfpathlineto{\pgfqpoint{3.512064in}{1.685485in}}%
\pgfpathlineto{\pgfqpoint{3.537270in}{1.690970in}}%
\pgfpathlineto{\pgfqpoint{3.541036in}{1.696455in}}%
\pgfpathlineto{\pgfqpoint{3.560735in}{1.701940in}}%
\pgfpathlineto{\pgfqpoint{3.565762in}{1.707425in}}%
\pgfpathlineto{\pgfqpoint{3.582272in}{1.712910in}}%
\pgfpathlineto{\pgfqpoint{3.589948in}{1.718394in}}%
\pgfpathlineto{\pgfqpoint{3.626687in}{1.723879in}}%
\pgfpathlineto{\pgfqpoint{3.632763in}{1.729364in}}%
\pgfpathlineto{\pgfqpoint{3.656299in}{1.734849in}}%
\pgfpathlineto{\pgfqpoint{3.920613in}{1.740334in}}%
\pgfpathlineto{\pgfqpoint{3.960582in}{1.745819in}}%
\pgfpathlineto{\pgfqpoint{3.962060in}{1.751304in}}%
\pgfpathlineto{\pgfqpoint{4.048447in}{1.756789in}}%
\pgfpathlineto{\pgfqpoint{4.055282in}{1.762274in}}%
\pgfpathlineto{\pgfqpoint{4.237761in}{1.767759in}}%
\pgfpathlineto{\pgfqpoint{4.244508in}{1.773243in}}%
\pgfpathlineto{\pgfqpoint{4.279988in}{1.778728in}}%
\pgfpathlineto{\pgfqpoint{4.280122in}{1.784213in}}%
\pgfpathlineto{\pgfqpoint{4.352643in}{1.789698in}}%
\pgfpathlineto{\pgfqpoint{4.409949in}{1.795183in}}%
\pgfpathlineto{\pgfqpoint{4.578870in}{1.800668in}}%
\pgfpathlineto{\pgfqpoint{4.587980in}{1.806153in}}%
\pgfpathlineto{\pgfqpoint{4.592835in}{1.811638in}}%
\pgfpathlineto{\pgfqpoint{4.592835in}{1.811638in}}%
\pgfusepath{stroke}%
\end{pgfscope}%
\begin{pgfscope}%
\pgfsetrectcap%
\pgfsetmiterjoin%
\pgfsetlinewidth{0.803000pt}%
\definecolor{currentstroke}{rgb}{0.000000,0.000000,0.000000}%
\pgfsetstrokecolor{currentstroke}%
\pgfsetdash{}{0pt}%
\pgfpathmoveto{\pgfqpoint{0.537394in}{0.467838in}}%
\pgfpathlineto{\pgfqpoint{0.537394in}{2.661796in}}%
\pgfusepath{stroke}%
\end{pgfscope}%
\begin{pgfscope}%
\pgfsetrectcap%
\pgfsetmiterjoin%
\pgfsetlinewidth{0.803000pt}%
\definecolor{currentstroke}{rgb}{0.000000,0.000000,0.000000}%
\pgfsetstrokecolor{currentstroke}%
\pgfsetdash{}{0pt}%
\pgfpathmoveto{\pgfqpoint{4.632078in}{0.467838in}}%
\pgfpathlineto{\pgfqpoint{4.632078in}{2.661796in}}%
\pgfusepath{stroke}%
\end{pgfscope}%
\begin{pgfscope}%
\pgfsetrectcap%
\pgfsetmiterjoin%
\pgfsetlinewidth{0.803000pt}%
\definecolor{currentstroke}{rgb}{0.000000,0.000000,0.000000}%
\pgfsetstrokecolor{currentstroke}%
\pgfsetdash{}{0pt}%
\pgfpathmoveto{\pgfqpoint{0.537394in}{0.467838in}}%
\pgfpathlineto{\pgfqpoint{4.632078in}{0.467838in}}%
\pgfusepath{stroke}%
\end{pgfscope}%
\begin{pgfscope}%
\pgfsetrectcap%
\pgfsetmiterjoin%
\pgfsetlinewidth{0.803000pt}%
\definecolor{currentstroke}{rgb}{0.000000,0.000000,0.000000}%
\pgfsetstrokecolor{currentstroke}%
\pgfsetdash{}{0pt}%
\pgfpathmoveto{\pgfqpoint{0.537394in}{2.661796in}}%
\pgfpathlineto{\pgfqpoint{4.632078in}{2.661796in}}%
\pgfusepath{stroke}%
\end{pgfscope}%
\begin{pgfscope}%
\pgfsetbuttcap%
\pgfsetmiterjoin%
\definecolor{currentfill}{rgb}{1.000000,1.000000,1.000000}%
\pgfsetfillcolor{currentfill}%
\pgfsetfillopacity{0.800000}%
\pgfsetlinewidth{1.003750pt}%
\definecolor{currentstroke}{rgb}{0.800000,0.800000,0.800000}%
\pgfsetstrokecolor{currentstroke}%
\pgfsetstrokeopacity{0.800000}%
\pgfsetdash{}{0pt}%
\pgfpathmoveto{\pgfqpoint{0.615172in}{1.268221in}}%
\pgfpathlineto{\pgfqpoint{1.848169in}{1.268221in}}%
\pgfpathquadraticcurveto{\pgfqpoint{1.870391in}{1.268221in}}{\pgfqpoint{1.870391in}{1.290443in}}%
\pgfpathlineto{\pgfqpoint{1.870391in}{2.584019in}}%
\pgfpathquadraticcurveto{\pgfqpoint{1.870391in}{2.606241in}}{\pgfqpoint{1.848169in}{2.606241in}}%
\pgfpathlineto{\pgfqpoint{0.615172in}{2.606241in}}%
\pgfpathquadraticcurveto{\pgfqpoint{0.592949in}{2.606241in}}{\pgfqpoint{0.592949in}{2.584019in}}%
\pgfpathlineto{\pgfqpoint{0.592949in}{1.290443in}}%
\pgfpathquadraticcurveto{\pgfqpoint{0.592949in}{1.268221in}}{\pgfqpoint{0.615172in}{1.268221in}}%
\pgfpathclose%
\pgfusepath{stroke,fill}%
\end{pgfscope}%
\begin{pgfscope}%
\pgfsetrectcap%
\pgfsetroundjoin%
\pgfsetlinewidth{1.003750pt}%
\definecolor{currentstroke}{rgb}{0.121569,0.466667,0.705882}%
\pgfsetstrokecolor{currentstroke}%
\pgfsetdash{}{0pt}%
\pgfpathmoveto{\pgfqpoint{0.637394in}{2.516267in}}%
\pgfpathlineto{\pgfqpoint{0.859616in}{2.516267in}}%
\pgfusepath{stroke}%
\end{pgfscope}%
\begin{pgfscope}%
\definecolor{textcolor}{rgb}{0.000000,0.000000,0.000000}%
\pgfsetstrokecolor{textcolor}%
\pgfsetfillcolor{textcolor}%
\pgftext[x=0.948505in,y=2.477378in,left,base]{\color{textcolor}\sffamily\fontsize{8.000000}{9.600000}\selectfont DMC(MCS), first}%
\end{pgfscope}%
\begin{pgfscope}%
\pgfsetrectcap%
\pgfsetroundjoin%
\pgfsetlinewidth{1.003750pt}%
\definecolor{currentstroke}{rgb}{1.000000,0.498039,0.054902}%
\pgfsetstrokecolor{currentstroke}%
\pgfsetdash{}{0pt}%
\pgfpathmoveto{\pgfqpoint{0.637394in}{2.353181in}}%
\pgfpathlineto{\pgfqpoint{0.859616in}{2.353181in}}%
\pgfusepath{stroke}%
\end{pgfscope}%
\begin{pgfscope}%
\definecolor{textcolor}{rgb}{0.000000,0.000000,0.000000}%
\pgfsetstrokecolor{textcolor}%
\pgfsetfillcolor{textcolor}%
\pgftext[x=0.948505in,y=2.314292in,left,base]{\color{textcolor}\sffamily\fontsize{8.000000}{9.600000}\selectfont DMC(LP), first}%
\end{pgfscope}%
\begin{pgfscope}%
\pgfsetbuttcap%
\pgfsetroundjoin%
\pgfsetlinewidth{1.003750pt}%
\definecolor{currentstroke}{rgb}{0.172549,0.627451,0.172549}%
\pgfsetstrokecolor{currentstroke}%
\pgfsetdash{{3.700000pt}{1.600000pt}}{0.000000pt}%
\pgfpathmoveto{\pgfqpoint{0.637394in}{2.190095in}}%
\pgfpathlineto{\pgfqpoint{0.859616in}{2.190095in}}%
\pgfusepath{stroke}%
\end{pgfscope}%
\begin{pgfscope}%
\definecolor{textcolor}{rgb}{0.000000,0.000000,0.000000}%
\pgfsetstrokecolor{textcolor}%
\pgfsetfillcolor{textcolor}%
\pgftext[x=0.948505in,y=2.151206in,left,base]{\color{textcolor}\sffamily\fontsize{8.000000}{9.600000}\selectfont DMC(LM), first}%
\end{pgfscope}%
\begin{pgfscope}%
\pgfsetbuttcap%
\pgfsetroundjoin%
\pgfsetlinewidth{1.003750pt}%
\definecolor{currentstroke}{rgb}{0.839216,0.152941,0.156863}%
\pgfsetstrokecolor{currentstroke}%
\pgfsetdash{{3.700000pt}{1.600000pt}}{0.000000pt}%
\pgfpathmoveto{\pgfqpoint{0.637394in}{2.027009in}}%
\pgfpathlineto{\pgfqpoint{0.859616in}{2.027009in}}%
\pgfusepath{stroke}%
\end{pgfscope}%
\begin{pgfscope}%
\definecolor{textcolor}{rgb}{0.000000,0.000000,0.000000}%
\pgfsetstrokecolor{textcolor}%
\pgfsetfillcolor{textcolor}%
\pgftext[x=0.948505in,y=1.988121in,left,base]{\color{textcolor}\sffamily\fontsize{8.000000}{9.600000}\selectfont DMC(MF), first}%
\end{pgfscope}%
\begin{pgfscope}%
\pgfsetbuttcap%
\pgfsetroundjoin%
\pgfsetlinewidth{1.003750pt}%
\definecolor{currentstroke}{rgb}{0.580392,0.403922,0.741176}%
\pgfsetstrokecolor{currentstroke}%
\pgfsetdash{{1.000000pt}{1.650000pt}}{0.000000pt}%
\pgfpathmoveto{\pgfqpoint{0.637394in}{1.863924in}}%
\pgfpathlineto{\pgfqpoint{0.859616in}{1.863924in}}%
\pgfusepath{stroke}%
\end{pgfscope}%
\begin{pgfscope}%
\definecolor{textcolor}{rgb}{0.000000,0.000000,0.000000}%
\pgfsetstrokecolor{textcolor}%
\pgfsetfillcolor{textcolor}%
\pgftext[x=0.948505in,y=1.825035in,left,base]{\color{textcolor}\sffamily\fontsize{8.000000}{9.600000}\selectfont DMC(MCS), last}%
\end{pgfscope}%
\begin{pgfscope}%
\pgfsetbuttcap%
\pgfsetroundjoin%
\pgfsetlinewidth{1.003750pt}%
\definecolor{currentstroke}{rgb}{0.549020,0.337255,0.294118}%
\pgfsetstrokecolor{currentstroke}%
\pgfsetdash{{1.000000pt}{1.650000pt}}{0.000000pt}%
\pgfpathmoveto{\pgfqpoint{0.637394in}{1.700838in}}%
\pgfpathlineto{\pgfqpoint{0.859616in}{1.700838in}}%
\pgfusepath{stroke}%
\end{pgfscope}%
\begin{pgfscope}%
\definecolor{textcolor}{rgb}{0.000000,0.000000,0.000000}%
\pgfsetstrokecolor{textcolor}%
\pgfsetfillcolor{textcolor}%
\pgftext[x=0.948505in,y=1.661949in,left,base]{\color{textcolor}\sffamily\fontsize{8.000000}{9.600000}\selectfont DMC(LP), last}%
\end{pgfscope}%
\begin{pgfscope}%
\pgfsetbuttcap%
\pgfsetroundjoin%
\pgfsetlinewidth{1.003750pt}%
\definecolor{currentstroke}{rgb}{0.890196,0.466667,0.760784}%
\pgfsetstrokecolor{currentstroke}%
\pgfsetdash{{6.400000pt}{1.600000pt}{1.000000pt}{1.600000pt}}{0.000000pt}%
\pgfpathmoveto{\pgfqpoint{0.637394in}{1.537752in}}%
\pgfpathlineto{\pgfqpoint{0.859616in}{1.537752in}}%
\pgfusepath{stroke}%
\end{pgfscope}%
\begin{pgfscope}%
\definecolor{textcolor}{rgb}{0.000000,0.000000,0.000000}%
\pgfsetstrokecolor{textcolor}%
\pgfsetfillcolor{textcolor}%
\pgftext[x=0.948505in,y=1.498863in,left,base]{\color{textcolor}\sffamily\fontsize{8.000000}{9.600000}\selectfont DMC(LM), last}%
\end{pgfscope}%
\begin{pgfscope}%
\pgfsetbuttcap%
\pgfsetroundjoin%
\pgfsetlinewidth{1.003750pt}%
\definecolor{currentstroke}{rgb}{0.498039,0.498039,0.498039}%
\pgfsetstrokecolor{currentstroke}%
\pgfsetdash{{6.400000pt}{1.600000pt}{1.000000pt}{1.600000pt}}{0.000000pt}%
\pgfpathmoveto{\pgfqpoint{0.637394in}{1.374666in}}%
\pgfpathlineto{\pgfqpoint{0.859616in}{1.374666in}}%
\pgfusepath{stroke}%
\end{pgfscope}%
\begin{pgfscope}%
\definecolor{textcolor}{rgb}{0.000000,0.000000,0.000000}%
\pgfsetstrokecolor{textcolor}%
\pgfsetfillcolor{textcolor}%
\pgftext[x=0.948505in,y=1.335777in,left,base]{\color{textcolor}\sffamily\fontsize{8.000000}{9.600000}\selectfont DMC(MF), last}%
\end{pgfscope}%
\end{pgfpicture}%
\makeatother%
\endgroup%

    \caption{
        Experiment 2 compares various variable-ordering heuristics (\mcs{}, \lexp, \lexm, and \minfill{}) for the ADD-based executor \dmc.
        The graded project-join trees here were produced by the planner \Lg{} with the tree decomposer \flowcutter{} from Experiment 1.
        \Lg{} is an anytime tool that may produce several trees of decreasing widths per benchmark.
        Nevertheless, across all four variable-ordering heuristics, there is little difference in execution time between using the first tree and using the last tree (planning time is excluded).
        % This behavior for \dmc{} has been observed in prior work \cite{dudek2020dpmc}.
    }
    \label{figExecutionA}
\end{figure}

%%%%%%%%%%%%%%%%%%%%%%%%%%%%%%%%%%%%%%%%%%%%%%%%%%%%%%%%%%%%%%%%%%%%%%%%%%%%%%%%

\subsection{Experiment 3: Comparing Weighted Projected Model Counters}

Figure \ref{figSolvingA} illustrates how six combinations of three \Lg{} tree-decomposition tools (\flowcutter{}, \htd, and \tamaki) and two \dmc{} variable-ordering heuristics (\mcs{} and \lexp) compare in 1000 seconds.
We exclude the planner \htb{} because it is slower than \Lg{} in Experiment 1.
We also exclude the variable-ordering heuristics \lexm{} and \minfill{} because they are slower than \mcs{} and \lexp{} in Experiment 2.
\begin{figure}[H]
    \centering
    %% Creator: Matplotlib, PGF backend
%%
%% To include the figure in your LaTeX document, write
%%   \input{<filename>.pgf}
%%
%% Make sure the required packages are loaded in your preamble
%%   \usepackage{pgf}
%%
%% and, on pdftex
%%   \usepackage[utf8]{inputenc}\DeclareUnicodeCharacter{2212}{-}
%%
%% or, on luatex and xetex
%%   \usepackage{unicode-math}
%%
%% Figures using additional raster images can only be included by \input if
%% they are in the same directory as the main LaTeX file. For loading figures
%% from other directories you can use the `import` package
%%   \usepackage{import}
%%
%% and then include the figures with
%%   \import{<path to file>}{<filename>.pgf}
%%
%% Matplotlib used the following preamble
%%   \usepackage[utf8x]{inputenc}
%%   \usepackage[T1]{fontenc}
%%
\begingroup%
\makeatletter%
\begin{pgfpicture}%
\pgfpathrectangle{\pgfpointorigin}{\pgfqpoint{6.000000in}{2.500000in}}%
\pgfusepath{use as bounding box, clip}%
\begin{pgfscope}%
\pgfsetbuttcap%
\pgfsetmiterjoin%
\definecolor{currentfill}{rgb}{1.000000,1.000000,1.000000}%
\pgfsetfillcolor{currentfill}%
\pgfsetlinewidth{0.000000pt}%
\definecolor{currentstroke}{rgb}{1.000000,1.000000,1.000000}%
\pgfsetstrokecolor{currentstroke}%
\pgfsetdash{}{0pt}%
\pgfpathmoveto{\pgfqpoint{0.000000in}{0.000000in}}%
\pgfpathlineto{\pgfqpoint{6.000000in}{0.000000in}}%
\pgfpathlineto{\pgfqpoint{6.000000in}{2.500000in}}%
\pgfpathlineto{\pgfqpoint{0.000000in}{2.500000in}}%
\pgfpathclose%
\pgfusepath{fill}%
\end{pgfscope}%
\begin{pgfscope}%
\pgfsetbuttcap%
\pgfsetmiterjoin%
\definecolor{currentfill}{rgb}{1.000000,1.000000,1.000000}%
\pgfsetfillcolor{currentfill}%
\pgfsetlinewidth{0.000000pt}%
\definecolor{currentstroke}{rgb}{0.000000,0.000000,0.000000}%
\pgfsetstrokecolor{currentstroke}%
\pgfsetstrokeopacity{0.000000}%
\pgfsetdash{}{0pt}%
\pgfpathmoveto{\pgfqpoint{0.708220in}{0.535823in}}%
\pgfpathlineto{\pgfqpoint{5.753646in}{0.535823in}}%
\pgfpathlineto{\pgfqpoint{5.753646in}{2.305275in}}%
\pgfpathlineto{\pgfqpoint{0.708220in}{2.305275in}}%
\pgfpathclose%
\pgfusepath{fill}%
\end{pgfscope}%
\begin{pgfscope}%
\pgfsetbuttcap%
\pgfsetroundjoin%
\definecolor{currentfill}{rgb}{0.000000,0.000000,0.000000}%
\pgfsetfillcolor{currentfill}%
\pgfsetlinewidth{0.803000pt}%
\definecolor{currentstroke}{rgb}{0.000000,0.000000,0.000000}%
\pgfsetstrokecolor{currentstroke}%
\pgfsetdash{}{0pt}%
\pgfsys@defobject{currentmarker}{\pgfqpoint{0.000000in}{-0.048611in}}{\pgfqpoint{0.000000in}{0.000000in}}{%
\pgfpathmoveto{\pgfqpoint{0.000000in}{0.000000in}}%
\pgfpathlineto{\pgfqpoint{0.000000in}{-0.048611in}}%
\pgfusepath{stroke,fill}%
}%
\begin{pgfscope}%
\pgfsys@transformshift{0.708220in}{0.535823in}%
\pgfsys@useobject{currentmarker}{}%
\end{pgfscope}%
\end{pgfscope}%
\begin{pgfscope}%
\definecolor{textcolor}{rgb}{0.000000,0.000000,0.000000}%
\pgfsetstrokecolor{textcolor}%
\pgfsetfillcolor{textcolor}%
\pgftext[x=0.708220in,y=0.438600in,,top]{\color{textcolor}\rmfamily\fontsize{9.000000}{10.800000}\selectfont \(\displaystyle {0}\)}%
\end{pgfscope}%
\begin{pgfscope}%
\pgfsetbuttcap%
\pgfsetroundjoin%
\definecolor{currentfill}{rgb}{0.000000,0.000000,0.000000}%
\pgfsetfillcolor{currentfill}%
\pgfsetlinewidth{0.803000pt}%
\definecolor{currentstroke}{rgb}{0.000000,0.000000,0.000000}%
\pgfsetstrokecolor{currentstroke}%
\pgfsetdash{}{0pt}%
\pgfsys@defobject{currentmarker}{\pgfqpoint{0.000000in}{-0.048611in}}{\pgfqpoint{0.000000in}{0.000000in}}{%
\pgfpathmoveto{\pgfqpoint{0.000000in}{0.000000in}}%
\pgfpathlineto{\pgfqpoint{0.000000in}{-0.048611in}}%
\pgfusepath{stroke,fill}%
}%
\begin{pgfscope}%
\pgfsys@transformshift{1.338898in}{0.535823in}%
\pgfsys@useobject{currentmarker}{}%
\end{pgfscope}%
\end{pgfscope}%
\begin{pgfscope}%
\definecolor{textcolor}{rgb}{0.000000,0.000000,0.000000}%
\pgfsetstrokecolor{textcolor}%
\pgfsetfillcolor{textcolor}%
\pgftext[x=1.338898in,y=0.438600in,,top]{\color{textcolor}\rmfamily\fontsize{9.000000}{10.800000}\selectfont \(\displaystyle {50}\)}%
\end{pgfscope}%
\begin{pgfscope}%
\pgfsetbuttcap%
\pgfsetroundjoin%
\definecolor{currentfill}{rgb}{0.000000,0.000000,0.000000}%
\pgfsetfillcolor{currentfill}%
\pgfsetlinewidth{0.803000pt}%
\definecolor{currentstroke}{rgb}{0.000000,0.000000,0.000000}%
\pgfsetstrokecolor{currentstroke}%
\pgfsetdash{}{0pt}%
\pgfsys@defobject{currentmarker}{\pgfqpoint{0.000000in}{-0.048611in}}{\pgfqpoint{0.000000in}{0.000000in}}{%
\pgfpathmoveto{\pgfqpoint{0.000000in}{0.000000in}}%
\pgfpathlineto{\pgfqpoint{0.000000in}{-0.048611in}}%
\pgfusepath{stroke,fill}%
}%
\begin{pgfscope}%
\pgfsys@transformshift{1.969577in}{0.535823in}%
\pgfsys@useobject{currentmarker}{}%
\end{pgfscope}%
\end{pgfscope}%
\begin{pgfscope}%
\definecolor{textcolor}{rgb}{0.000000,0.000000,0.000000}%
\pgfsetstrokecolor{textcolor}%
\pgfsetfillcolor{textcolor}%
\pgftext[x=1.969577in,y=0.438600in,,top]{\color{textcolor}\rmfamily\fontsize{9.000000}{10.800000}\selectfont \(\displaystyle {100}\)}%
\end{pgfscope}%
\begin{pgfscope}%
\pgfsetbuttcap%
\pgfsetroundjoin%
\definecolor{currentfill}{rgb}{0.000000,0.000000,0.000000}%
\pgfsetfillcolor{currentfill}%
\pgfsetlinewidth{0.803000pt}%
\definecolor{currentstroke}{rgb}{0.000000,0.000000,0.000000}%
\pgfsetstrokecolor{currentstroke}%
\pgfsetdash{}{0pt}%
\pgfsys@defobject{currentmarker}{\pgfqpoint{0.000000in}{-0.048611in}}{\pgfqpoint{0.000000in}{0.000000in}}{%
\pgfpathmoveto{\pgfqpoint{0.000000in}{0.000000in}}%
\pgfpathlineto{\pgfqpoint{0.000000in}{-0.048611in}}%
\pgfusepath{stroke,fill}%
}%
\begin{pgfscope}%
\pgfsys@transformshift{2.600255in}{0.535823in}%
\pgfsys@useobject{currentmarker}{}%
\end{pgfscope}%
\end{pgfscope}%
\begin{pgfscope}%
\definecolor{textcolor}{rgb}{0.000000,0.000000,0.000000}%
\pgfsetstrokecolor{textcolor}%
\pgfsetfillcolor{textcolor}%
\pgftext[x=2.600255in,y=0.438600in,,top]{\color{textcolor}\rmfamily\fontsize{9.000000}{10.800000}\selectfont \(\displaystyle {150}\)}%
\end{pgfscope}%
\begin{pgfscope}%
\pgfsetbuttcap%
\pgfsetroundjoin%
\definecolor{currentfill}{rgb}{0.000000,0.000000,0.000000}%
\pgfsetfillcolor{currentfill}%
\pgfsetlinewidth{0.803000pt}%
\definecolor{currentstroke}{rgb}{0.000000,0.000000,0.000000}%
\pgfsetstrokecolor{currentstroke}%
\pgfsetdash{}{0pt}%
\pgfsys@defobject{currentmarker}{\pgfqpoint{0.000000in}{-0.048611in}}{\pgfqpoint{0.000000in}{0.000000in}}{%
\pgfpathmoveto{\pgfqpoint{0.000000in}{0.000000in}}%
\pgfpathlineto{\pgfqpoint{0.000000in}{-0.048611in}}%
\pgfusepath{stroke,fill}%
}%
\begin{pgfscope}%
\pgfsys@transformshift{3.230933in}{0.535823in}%
\pgfsys@useobject{currentmarker}{}%
\end{pgfscope}%
\end{pgfscope}%
\begin{pgfscope}%
\definecolor{textcolor}{rgb}{0.000000,0.000000,0.000000}%
\pgfsetstrokecolor{textcolor}%
\pgfsetfillcolor{textcolor}%
\pgftext[x=3.230933in,y=0.438600in,,top]{\color{textcolor}\rmfamily\fontsize{9.000000}{10.800000}\selectfont \(\displaystyle {200}\)}%
\end{pgfscope}%
\begin{pgfscope}%
\pgfsetbuttcap%
\pgfsetroundjoin%
\definecolor{currentfill}{rgb}{0.000000,0.000000,0.000000}%
\pgfsetfillcolor{currentfill}%
\pgfsetlinewidth{0.803000pt}%
\definecolor{currentstroke}{rgb}{0.000000,0.000000,0.000000}%
\pgfsetstrokecolor{currentstroke}%
\pgfsetdash{}{0pt}%
\pgfsys@defobject{currentmarker}{\pgfqpoint{0.000000in}{-0.048611in}}{\pgfqpoint{0.000000in}{0.000000in}}{%
\pgfpathmoveto{\pgfqpoint{0.000000in}{0.000000in}}%
\pgfpathlineto{\pgfqpoint{0.000000in}{-0.048611in}}%
\pgfusepath{stroke,fill}%
}%
\begin{pgfscope}%
\pgfsys@transformshift{3.861611in}{0.535823in}%
\pgfsys@useobject{currentmarker}{}%
\end{pgfscope}%
\end{pgfscope}%
\begin{pgfscope}%
\definecolor{textcolor}{rgb}{0.000000,0.000000,0.000000}%
\pgfsetstrokecolor{textcolor}%
\pgfsetfillcolor{textcolor}%
\pgftext[x=3.861611in,y=0.438600in,,top]{\color{textcolor}\rmfamily\fontsize{9.000000}{10.800000}\selectfont \(\displaystyle {250}\)}%
\end{pgfscope}%
\begin{pgfscope}%
\pgfsetbuttcap%
\pgfsetroundjoin%
\definecolor{currentfill}{rgb}{0.000000,0.000000,0.000000}%
\pgfsetfillcolor{currentfill}%
\pgfsetlinewidth{0.803000pt}%
\definecolor{currentstroke}{rgb}{0.000000,0.000000,0.000000}%
\pgfsetstrokecolor{currentstroke}%
\pgfsetdash{}{0pt}%
\pgfsys@defobject{currentmarker}{\pgfqpoint{0.000000in}{-0.048611in}}{\pgfqpoint{0.000000in}{0.000000in}}{%
\pgfpathmoveto{\pgfqpoint{0.000000in}{0.000000in}}%
\pgfpathlineto{\pgfqpoint{0.000000in}{-0.048611in}}%
\pgfusepath{stroke,fill}%
}%
\begin{pgfscope}%
\pgfsys@transformshift{4.492290in}{0.535823in}%
\pgfsys@useobject{currentmarker}{}%
\end{pgfscope}%
\end{pgfscope}%
\begin{pgfscope}%
\definecolor{textcolor}{rgb}{0.000000,0.000000,0.000000}%
\pgfsetstrokecolor{textcolor}%
\pgfsetfillcolor{textcolor}%
\pgftext[x=4.492290in,y=0.438600in,,top]{\color{textcolor}\rmfamily\fontsize{9.000000}{10.800000}\selectfont \(\displaystyle {300}\)}%
\end{pgfscope}%
\begin{pgfscope}%
\pgfsetbuttcap%
\pgfsetroundjoin%
\definecolor{currentfill}{rgb}{0.000000,0.000000,0.000000}%
\pgfsetfillcolor{currentfill}%
\pgfsetlinewidth{0.803000pt}%
\definecolor{currentstroke}{rgb}{0.000000,0.000000,0.000000}%
\pgfsetstrokecolor{currentstroke}%
\pgfsetdash{}{0pt}%
\pgfsys@defobject{currentmarker}{\pgfqpoint{0.000000in}{-0.048611in}}{\pgfqpoint{0.000000in}{0.000000in}}{%
\pgfpathmoveto{\pgfqpoint{0.000000in}{0.000000in}}%
\pgfpathlineto{\pgfqpoint{0.000000in}{-0.048611in}}%
\pgfusepath{stroke,fill}%
}%
\begin{pgfscope}%
\pgfsys@transformshift{5.122968in}{0.535823in}%
\pgfsys@useobject{currentmarker}{}%
\end{pgfscope}%
\end{pgfscope}%
\begin{pgfscope}%
\definecolor{textcolor}{rgb}{0.000000,0.000000,0.000000}%
\pgfsetstrokecolor{textcolor}%
\pgfsetfillcolor{textcolor}%
\pgftext[x=5.122968in,y=0.438600in,,top]{\color{textcolor}\rmfamily\fontsize{9.000000}{10.800000}\selectfont \(\displaystyle {350}\)}%
\end{pgfscope}%
\begin{pgfscope}%
\pgfsetbuttcap%
\pgfsetroundjoin%
\definecolor{currentfill}{rgb}{0.000000,0.000000,0.000000}%
\pgfsetfillcolor{currentfill}%
\pgfsetlinewidth{0.803000pt}%
\definecolor{currentstroke}{rgb}{0.000000,0.000000,0.000000}%
\pgfsetstrokecolor{currentstroke}%
\pgfsetdash{}{0pt}%
\pgfsys@defobject{currentmarker}{\pgfqpoint{0.000000in}{-0.048611in}}{\pgfqpoint{0.000000in}{0.000000in}}{%
\pgfpathmoveto{\pgfqpoint{0.000000in}{0.000000in}}%
\pgfpathlineto{\pgfqpoint{0.000000in}{-0.048611in}}%
\pgfusepath{stroke,fill}%
}%
\begin{pgfscope}%
\pgfsys@transformshift{5.753646in}{0.535823in}%
\pgfsys@useobject{currentmarker}{}%
\end{pgfscope}%
\end{pgfscope}%
\begin{pgfscope}%
\definecolor{textcolor}{rgb}{0.000000,0.000000,0.000000}%
\pgfsetstrokecolor{textcolor}%
\pgfsetfillcolor{textcolor}%
\pgftext[x=5.753646in,y=0.438600in,,top]{\color{textcolor}\rmfamily\fontsize{9.000000}{10.800000}\selectfont \(\displaystyle {400}\)}%
\end{pgfscope}%
\begin{pgfscope}%
\definecolor{textcolor}{rgb}{0.000000,0.000000,0.000000}%
\pgfsetstrokecolor{textcolor}%
\pgfsetfillcolor{textcolor}%
\pgftext[x=3.230933in,y=0.272655in,,top]{\color{textcolor}\rmfamily\fontsize{10.000000}{12.000000}\selectfont Number of benchmarks solved}%
\end{pgfscope}%
\begin{pgfscope}%
\pgfsetbuttcap%
\pgfsetroundjoin%
\definecolor{currentfill}{rgb}{0.000000,0.000000,0.000000}%
\pgfsetfillcolor{currentfill}%
\pgfsetlinewidth{0.803000pt}%
\definecolor{currentstroke}{rgb}{0.000000,0.000000,0.000000}%
\pgfsetstrokecolor{currentstroke}%
\pgfsetdash{}{0pt}%
\pgfsys@defobject{currentmarker}{\pgfqpoint{-0.048611in}{0.000000in}}{\pgfqpoint{-0.000000in}{0.000000in}}{%
\pgfpathmoveto{\pgfqpoint{-0.000000in}{0.000000in}}%
\pgfpathlineto{\pgfqpoint{-0.048611in}{0.000000in}}%
\pgfusepath{stroke,fill}%
}%
\begin{pgfscope}%
\pgfsys@transformshift{0.708220in}{0.620358in}%
\pgfsys@useobject{currentmarker}{}%
\end{pgfscope}%
\end{pgfscope}%
\begin{pgfscope}%
\definecolor{textcolor}{rgb}{0.000000,0.000000,0.000000}%
\pgfsetstrokecolor{textcolor}%
\pgfsetfillcolor{textcolor}%
\pgftext[x=0.344411in, y=0.575633in, left, base]{\color{textcolor}\rmfamily\fontsize{9.000000}{10.800000}\selectfont \(\displaystyle {10^{-3}}\)}%
\end{pgfscope}%
\begin{pgfscope}%
\pgfsetbuttcap%
\pgfsetroundjoin%
\definecolor{currentfill}{rgb}{0.000000,0.000000,0.000000}%
\pgfsetfillcolor{currentfill}%
\pgfsetlinewidth{0.803000pt}%
\definecolor{currentstroke}{rgb}{0.000000,0.000000,0.000000}%
\pgfsetstrokecolor{currentstroke}%
\pgfsetdash{}{0pt}%
\pgfsys@defobject{currentmarker}{\pgfqpoint{-0.048611in}{0.000000in}}{\pgfqpoint{-0.000000in}{0.000000in}}{%
\pgfpathmoveto{\pgfqpoint{-0.000000in}{0.000000in}}%
\pgfpathlineto{\pgfqpoint{-0.048611in}{0.000000in}}%
\pgfusepath{stroke,fill}%
}%
\begin{pgfscope}%
\pgfsys@transformshift{0.708220in}{0.901177in}%
\pgfsys@useobject{currentmarker}{}%
\end{pgfscope}%
\end{pgfscope}%
\begin{pgfscope}%
\definecolor{textcolor}{rgb}{0.000000,0.000000,0.000000}%
\pgfsetstrokecolor{textcolor}%
\pgfsetfillcolor{textcolor}%
\pgftext[x=0.344411in, y=0.856453in, left, base]{\color{textcolor}\rmfamily\fontsize{9.000000}{10.800000}\selectfont \(\displaystyle {10^{-2}}\)}%
\end{pgfscope}%
\begin{pgfscope}%
\pgfsetbuttcap%
\pgfsetroundjoin%
\definecolor{currentfill}{rgb}{0.000000,0.000000,0.000000}%
\pgfsetfillcolor{currentfill}%
\pgfsetlinewidth{0.803000pt}%
\definecolor{currentstroke}{rgb}{0.000000,0.000000,0.000000}%
\pgfsetstrokecolor{currentstroke}%
\pgfsetdash{}{0pt}%
\pgfsys@defobject{currentmarker}{\pgfqpoint{-0.048611in}{0.000000in}}{\pgfqpoint{-0.000000in}{0.000000in}}{%
\pgfpathmoveto{\pgfqpoint{-0.000000in}{0.000000in}}%
\pgfpathlineto{\pgfqpoint{-0.048611in}{0.000000in}}%
\pgfusepath{stroke,fill}%
}%
\begin{pgfscope}%
\pgfsys@transformshift{0.708220in}{1.181997in}%
\pgfsys@useobject{currentmarker}{}%
\end{pgfscope}%
\end{pgfscope}%
\begin{pgfscope}%
\definecolor{textcolor}{rgb}{0.000000,0.000000,0.000000}%
\pgfsetstrokecolor{textcolor}%
\pgfsetfillcolor{textcolor}%
\pgftext[x=0.344411in, y=1.137272in, left, base]{\color{textcolor}\rmfamily\fontsize{9.000000}{10.800000}\selectfont \(\displaystyle {10^{-1}}\)}%
\end{pgfscope}%
\begin{pgfscope}%
\pgfsetbuttcap%
\pgfsetroundjoin%
\definecolor{currentfill}{rgb}{0.000000,0.000000,0.000000}%
\pgfsetfillcolor{currentfill}%
\pgfsetlinewidth{0.803000pt}%
\definecolor{currentstroke}{rgb}{0.000000,0.000000,0.000000}%
\pgfsetstrokecolor{currentstroke}%
\pgfsetdash{}{0pt}%
\pgfsys@defobject{currentmarker}{\pgfqpoint{-0.048611in}{0.000000in}}{\pgfqpoint{-0.000000in}{0.000000in}}{%
\pgfpathmoveto{\pgfqpoint{-0.000000in}{0.000000in}}%
\pgfpathlineto{\pgfqpoint{-0.048611in}{0.000000in}}%
\pgfusepath{stroke,fill}%
}%
\begin{pgfscope}%
\pgfsys@transformshift{0.708220in}{1.462816in}%
\pgfsys@useobject{currentmarker}{}%
\end{pgfscope}%
\end{pgfscope}%
\begin{pgfscope}%
\definecolor{textcolor}{rgb}{0.000000,0.000000,0.000000}%
\pgfsetstrokecolor{textcolor}%
\pgfsetfillcolor{textcolor}%
\pgftext[x=0.424657in, y=1.418092in, left, base]{\color{textcolor}\rmfamily\fontsize{9.000000}{10.800000}\selectfont \(\displaystyle {10^{0}}\)}%
\end{pgfscope}%
\begin{pgfscope}%
\pgfsetbuttcap%
\pgfsetroundjoin%
\definecolor{currentfill}{rgb}{0.000000,0.000000,0.000000}%
\pgfsetfillcolor{currentfill}%
\pgfsetlinewidth{0.803000pt}%
\definecolor{currentstroke}{rgb}{0.000000,0.000000,0.000000}%
\pgfsetstrokecolor{currentstroke}%
\pgfsetdash{}{0pt}%
\pgfsys@defobject{currentmarker}{\pgfqpoint{-0.048611in}{0.000000in}}{\pgfqpoint{-0.000000in}{0.000000in}}{%
\pgfpathmoveto{\pgfqpoint{-0.000000in}{0.000000in}}%
\pgfpathlineto{\pgfqpoint{-0.048611in}{0.000000in}}%
\pgfusepath{stroke,fill}%
}%
\begin{pgfscope}%
\pgfsys@transformshift{0.708220in}{1.743636in}%
\pgfsys@useobject{currentmarker}{}%
\end{pgfscope}%
\end{pgfscope}%
\begin{pgfscope}%
\definecolor{textcolor}{rgb}{0.000000,0.000000,0.000000}%
\pgfsetstrokecolor{textcolor}%
\pgfsetfillcolor{textcolor}%
\pgftext[x=0.424657in, y=1.698911in, left, base]{\color{textcolor}\rmfamily\fontsize{9.000000}{10.800000}\selectfont \(\displaystyle {10^{1}}\)}%
\end{pgfscope}%
\begin{pgfscope}%
\pgfsetbuttcap%
\pgfsetroundjoin%
\definecolor{currentfill}{rgb}{0.000000,0.000000,0.000000}%
\pgfsetfillcolor{currentfill}%
\pgfsetlinewidth{0.803000pt}%
\definecolor{currentstroke}{rgb}{0.000000,0.000000,0.000000}%
\pgfsetstrokecolor{currentstroke}%
\pgfsetdash{}{0pt}%
\pgfsys@defobject{currentmarker}{\pgfqpoint{-0.048611in}{0.000000in}}{\pgfqpoint{-0.000000in}{0.000000in}}{%
\pgfpathmoveto{\pgfqpoint{-0.000000in}{0.000000in}}%
\pgfpathlineto{\pgfqpoint{-0.048611in}{0.000000in}}%
\pgfusepath{stroke,fill}%
}%
\begin{pgfscope}%
\pgfsys@transformshift{0.708220in}{2.024456in}%
\pgfsys@useobject{currentmarker}{}%
\end{pgfscope}%
\end{pgfscope}%
\begin{pgfscope}%
\definecolor{textcolor}{rgb}{0.000000,0.000000,0.000000}%
\pgfsetstrokecolor{textcolor}%
\pgfsetfillcolor{textcolor}%
\pgftext[x=0.424657in, y=1.979731in, left, base]{\color{textcolor}\rmfamily\fontsize{9.000000}{10.800000}\selectfont \(\displaystyle {10^{2}}\)}%
\end{pgfscope}%
\begin{pgfscope}%
\pgfsetbuttcap%
\pgfsetroundjoin%
\definecolor{currentfill}{rgb}{0.000000,0.000000,0.000000}%
\pgfsetfillcolor{currentfill}%
\pgfsetlinewidth{0.803000pt}%
\definecolor{currentstroke}{rgb}{0.000000,0.000000,0.000000}%
\pgfsetstrokecolor{currentstroke}%
\pgfsetdash{}{0pt}%
\pgfsys@defobject{currentmarker}{\pgfqpoint{-0.048611in}{0.000000in}}{\pgfqpoint{-0.000000in}{0.000000in}}{%
\pgfpathmoveto{\pgfqpoint{-0.000000in}{0.000000in}}%
\pgfpathlineto{\pgfqpoint{-0.048611in}{0.000000in}}%
\pgfusepath{stroke,fill}%
}%
\begin{pgfscope}%
\pgfsys@transformshift{0.708220in}{2.305275in}%
\pgfsys@useobject{currentmarker}{}%
\end{pgfscope}%
\end{pgfscope}%
\begin{pgfscope}%
\definecolor{textcolor}{rgb}{0.000000,0.000000,0.000000}%
\pgfsetstrokecolor{textcolor}%
\pgfsetfillcolor{textcolor}%
\pgftext[x=0.424657in, y=2.260550in, left, base]{\color{textcolor}\rmfamily\fontsize{9.000000}{10.800000}\selectfont \(\displaystyle {10^{3}}\)}%
\end{pgfscope}%
\begin{pgfscope}%
\pgfsetbuttcap%
\pgfsetroundjoin%
\definecolor{currentfill}{rgb}{0.000000,0.000000,0.000000}%
\pgfsetfillcolor{currentfill}%
\pgfsetlinewidth{0.602250pt}%
\definecolor{currentstroke}{rgb}{0.000000,0.000000,0.000000}%
\pgfsetstrokecolor{currentstroke}%
\pgfsetdash{}{0pt}%
\pgfsys@defobject{currentmarker}{\pgfqpoint{-0.027778in}{0.000000in}}{\pgfqpoint{-0.000000in}{0.000000in}}{%
\pgfpathmoveto{\pgfqpoint{-0.000000in}{0.000000in}}%
\pgfpathlineto{\pgfqpoint{-0.027778in}{0.000000in}}%
\pgfusepath{stroke,fill}%
}%
\begin{pgfscope}%
\pgfsys@transformshift{0.708220in}{0.535823in}%
\pgfsys@useobject{currentmarker}{}%
\end{pgfscope}%
\end{pgfscope}%
\begin{pgfscope}%
\pgfsetbuttcap%
\pgfsetroundjoin%
\definecolor{currentfill}{rgb}{0.000000,0.000000,0.000000}%
\pgfsetfillcolor{currentfill}%
\pgfsetlinewidth{0.602250pt}%
\definecolor{currentstroke}{rgb}{0.000000,0.000000,0.000000}%
\pgfsetstrokecolor{currentstroke}%
\pgfsetdash{}{0pt}%
\pgfsys@defobject{currentmarker}{\pgfqpoint{-0.027778in}{0.000000in}}{\pgfqpoint{-0.000000in}{0.000000in}}{%
\pgfpathmoveto{\pgfqpoint{-0.000000in}{0.000000in}}%
\pgfpathlineto{\pgfqpoint{-0.027778in}{0.000000in}}%
\pgfusepath{stroke,fill}%
}%
\begin{pgfscope}%
\pgfsys@transformshift{0.708220in}{0.558058in}%
\pgfsys@useobject{currentmarker}{}%
\end{pgfscope}%
\end{pgfscope}%
\begin{pgfscope}%
\pgfsetbuttcap%
\pgfsetroundjoin%
\definecolor{currentfill}{rgb}{0.000000,0.000000,0.000000}%
\pgfsetfillcolor{currentfill}%
\pgfsetlinewidth{0.602250pt}%
\definecolor{currentstroke}{rgb}{0.000000,0.000000,0.000000}%
\pgfsetstrokecolor{currentstroke}%
\pgfsetdash{}{0pt}%
\pgfsys@defobject{currentmarker}{\pgfqpoint{-0.027778in}{0.000000in}}{\pgfqpoint{-0.000000in}{0.000000in}}{%
\pgfpathmoveto{\pgfqpoint{-0.000000in}{0.000000in}}%
\pgfpathlineto{\pgfqpoint{-0.027778in}{0.000000in}}%
\pgfusepath{stroke,fill}%
}%
\begin{pgfscope}%
\pgfsys@transformshift{0.708220in}{0.576858in}%
\pgfsys@useobject{currentmarker}{}%
\end{pgfscope}%
\end{pgfscope}%
\begin{pgfscope}%
\pgfsetbuttcap%
\pgfsetroundjoin%
\definecolor{currentfill}{rgb}{0.000000,0.000000,0.000000}%
\pgfsetfillcolor{currentfill}%
\pgfsetlinewidth{0.602250pt}%
\definecolor{currentstroke}{rgb}{0.000000,0.000000,0.000000}%
\pgfsetstrokecolor{currentstroke}%
\pgfsetdash{}{0pt}%
\pgfsys@defobject{currentmarker}{\pgfqpoint{-0.027778in}{0.000000in}}{\pgfqpoint{-0.000000in}{0.000000in}}{%
\pgfpathmoveto{\pgfqpoint{-0.000000in}{0.000000in}}%
\pgfpathlineto{\pgfqpoint{-0.027778in}{0.000000in}}%
\pgfusepath{stroke,fill}%
}%
\begin{pgfscope}%
\pgfsys@transformshift{0.708220in}{0.593144in}%
\pgfsys@useobject{currentmarker}{}%
\end{pgfscope}%
\end{pgfscope}%
\begin{pgfscope}%
\pgfsetbuttcap%
\pgfsetroundjoin%
\definecolor{currentfill}{rgb}{0.000000,0.000000,0.000000}%
\pgfsetfillcolor{currentfill}%
\pgfsetlinewidth{0.602250pt}%
\definecolor{currentstroke}{rgb}{0.000000,0.000000,0.000000}%
\pgfsetstrokecolor{currentstroke}%
\pgfsetdash{}{0pt}%
\pgfsys@defobject{currentmarker}{\pgfqpoint{-0.027778in}{0.000000in}}{\pgfqpoint{-0.000000in}{0.000000in}}{%
\pgfpathmoveto{\pgfqpoint{-0.000000in}{0.000000in}}%
\pgfpathlineto{\pgfqpoint{-0.027778in}{0.000000in}}%
\pgfusepath{stroke,fill}%
}%
\begin{pgfscope}%
\pgfsys@transformshift{0.708220in}{0.607508in}%
\pgfsys@useobject{currentmarker}{}%
\end{pgfscope}%
\end{pgfscope}%
\begin{pgfscope}%
\pgfsetbuttcap%
\pgfsetroundjoin%
\definecolor{currentfill}{rgb}{0.000000,0.000000,0.000000}%
\pgfsetfillcolor{currentfill}%
\pgfsetlinewidth{0.602250pt}%
\definecolor{currentstroke}{rgb}{0.000000,0.000000,0.000000}%
\pgfsetstrokecolor{currentstroke}%
\pgfsetdash{}{0pt}%
\pgfsys@defobject{currentmarker}{\pgfqpoint{-0.027778in}{0.000000in}}{\pgfqpoint{-0.000000in}{0.000000in}}{%
\pgfpathmoveto{\pgfqpoint{-0.000000in}{0.000000in}}%
\pgfpathlineto{\pgfqpoint{-0.027778in}{0.000000in}}%
\pgfusepath{stroke,fill}%
}%
\begin{pgfscope}%
\pgfsys@transformshift{0.708220in}{0.704893in}%
\pgfsys@useobject{currentmarker}{}%
\end{pgfscope}%
\end{pgfscope}%
\begin{pgfscope}%
\pgfsetbuttcap%
\pgfsetroundjoin%
\definecolor{currentfill}{rgb}{0.000000,0.000000,0.000000}%
\pgfsetfillcolor{currentfill}%
\pgfsetlinewidth{0.602250pt}%
\definecolor{currentstroke}{rgb}{0.000000,0.000000,0.000000}%
\pgfsetstrokecolor{currentstroke}%
\pgfsetdash{}{0pt}%
\pgfsys@defobject{currentmarker}{\pgfqpoint{-0.027778in}{0.000000in}}{\pgfqpoint{-0.000000in}{0.000000in}}{%
\pgfpathmoveto{\pgfqpoint{-0.000000in}{0.000000in}}%
\pgfpathlineto{\pgfqpoint{-0.027778in}{0.000000in}}%
\pgfusepath{stroke,fill}%
}%
\begin{pgfscope}%
\pgfsys@transformshift{0.708220in}{0.754343in}%
\pgfsys@useobject{currentmarker}{}%
\end{pgfscope}%
\end{pgfscope}%
\begin{pgfscope}%
\pgfsetbuttcap%
\pgfsetroundjoin%
\definecolor{currentfill}{rgb}{0.000000,0.000000,0.000000}%
\pgfsetfillcolor{currentfill}%
\pgfsetlinewidth{0.602250pt}%
\definecolor{currentstroke}{rgb}{0.000000,0.000000,0.000000}%
\pgfsetstrokecolor{currentstroke}%
\pgfsetdash{}{0pt}%
\pgfsys@defobject{currentmarker}{\pgfqpoint{-0.027778in}{0.000000in}}{\pgfqpoint{-0.000000in}{0.000000in}}{%
\pgfpathmoveto{\pgfqpoint{-0.000000in}{0.000000in}}%
\pgfpathlineto{\pgfqpoint{-0.027778in}{0.000000in}}%
\pgfusepath{stroke,fill}%
}%
\begin{pgfscope}%
\pgfsys@transformshift{0.708220in}{0.789428in}%
\pgfsys@useobject{currentmarker}{}%
\end{pgfscope}%
\end{pgfscope}%
\begin{pgfscope}%
\pgfsetbuttcap%
\pgfsetroundjoin%
\definecolor{currentfill}{rgb}{0.000000,0.000000,0.000000}%
\pgfsetfillcolor{currentfill}%
\pgfsetlinewidth{0.602250pt}%
\definecolor{currentstroke}{rgb}{0.000000,0.000000,0.000000}%
\pgfsetstrokecolor{currentstroke}%
\pgfsetdash{}{0pt}%
\pgfsys@defobject{currentmarker}{\pgfqpoint{-0.027778in}{0.000000in}}{\pgfqpoint{-0.000000in}{0.000000in}}{%
\pgfpathmoveto{\pgfqpoint{-0.000000in}{0.000000in}}%
\pgfpathlineto{\pgfqpoint{-0.027778in}{0.000000in}}%
\pgfusepath{stroke,fill}%
}%
\begin{pgfscope}%
\pgfsys@transformshift{0.708220in}{0.816642in}%
\pgfsys@useobject{currentmarker}{}%
\end{pgfscope}%
\end{pgfscope}%
\begin{pgfscope}%
\pgfsetbuttcap%
\pgfsetroundjoin%
\definecolor{currentfill}{rgb}{0.000000,0.000000,0.000000}%
\pgfsetfillcolor{currentfill}%
\pgfsetlinewidth{0.602250pt}%
\definecolor{currentstroke}{rgb}{0.000000,0.000000,0.000000}%
\pgfsetstrokecolor{currentstroke}%
\pgfsetdash{}{0pt}%
\pgfsys@defobject{currentmarker}{\pgfqpoint{-0.027778in}{0.000000in}}{\pgfqpoint{-0.000000in}{0.000000in}}{%
\pgfpathmoveto{\pgfqpoint{-0.000000in}{0.000000in}}%
\pgfpathlineto{\pgfqpoint{-0.027778in}{0.000000in}}%
\pgfusepath{stroke,fill}%
}%
\begin{pgfscope}%
\pgfsys@transformshift{0.708220in}{0.838878in}%
\pgfsys@useobject{currentmarker}{}%
\end{pgfscope}%
\end{pgfscope}%
\begin{pgfscope}%
\pgfsetbuttcap%
\pgfsetroundjoin%
\definecolor{currentfill}{rgb}{0.000000,0.000000,0.000000}%
\pgfsetfillcolor{currentfill}%
\pgfsetlinewidth{0.602250pt}%
\definecolor{currentstroke}{rgb}{0.000000,0.000000,0.000000}%
\pgfsetstrokecolor{currentstroke}%
\pgfsetdash{}{0pt}%
\pgfsys@defobject{currentmarker}{\pgfqpoint{-0.027778in}{0.000000in}}{\pgfqpoint{-0.000000in}{0.000000in}}{%
\pgfpathmoveto{\pgfqpoint{-0.000000in}{0.000000in}}%
\pgfpathlineto{\pgfqpoint{-0.027778in}{0.000000in}}%
\pgfusepath{stroke,fill}%
}%
\begin{pgfscope}%
\pgfsys@transformshift{0.708220in}{0.857678in}%
\pgfsys@useobject{currentmarker}{}%
\end{pgfscope}%
\end{pgfscope}%
\begin{pgfscope}%
\pgfsetbuttcap%
\pgfsetroundjoin%
\definecolor{currentfill}{rgb}{0.000000,0.000000,0.000000}%
\pgfsetfillcolor{currentfill}%
\pgfsetlinewidth{0.602250pt}%
\definecolor{currentstroke}{rgb}{0.000000,0.000000,0.000000}%
\pgfsetstrokecolor{currentstroke}%
\pgfsetdash{}{0pt}%
\pgfsys@defobject{currentmarker}{\pgfqpoint{-0.027778in}{0.000000in}}{\pgfqpoint{-0.000000in}{0.000000in}}{%
\pgfpathmoveto{\pgfqpoint{-0.000000in}{0.000000in}}%
\pgfpathlineto{\pgfqpoint{-0.027778in}{0.000000in}}%
\pgfusepath{stroke,fill}%
}%
\begin{pgfscope}%
\pgfsys@transformshift{0.708220in}{0.873963in}%
\pgfsys@useobject{currentmarker}{}%
\end{pgfscope}%
\end{pgfscope}%
\begin{pgfscope}%
\pgfsetbuttcap%
\pgfsetroundjoin%
\definecolor{currentfill}{rgb}{0.000000,0.000000,0.000000}%
\pgfsetfillcolor{currentfill}%
\pgfsetlinewidth{0.602250pt}%
\definecolor{currentstroke}{rgb}{0.000000,0.000000,0.000000}%
\pgfsetstrokecolor{currentstroke}%
\pgfsetdash{}{0pt}%
\pgfsys@defobject{currentmarker}{\pgfqpoint{-0.027778in}{0.000000in}}{\pgfqpoint{-0.000000in}{0.000000in}}{%
\pgfpathmoveto{\pgfqpoint{-0.000000in}{0.000000in}}%
\pgfpathlineto{\pgfqpoint{-0.027778in}{0.000000in}}%
\pgfusepath{stroke,fill}%
}%
\begin{pgfscope}%
\pgfsys@transformshift{0.708220in}{0.888328in}%
\pgfsys@useobject{currentmarker}{}%
\end{pgfscope}%
\end{pgfscope}%
\begin{pgfscope}%
\pgfsetbuttcap%
\pgfsetroundjoin%
\definecolor{currentfill}{rgb}{0.000000,0.000000,0.000000}%
\pgfsetfillcolor{currentfill}%
\pgfsetlinewidth{0.602250pt}%
\definecolor{currentstroke}{rgb}{0.000000,0.000000,0.000000}%
\pgfsetstrokecolor{currentstroke}%
\pgfsetdash{}{0pt}%
\pgfsys@defobject{currentmarker}{\pgfqpoint{-0.027778in}{0.000000in}}{\pgfqpoint{-0.000000in}{0.000000in}}{%
\pgfpathmoveto{\pgfqpoint{-0.000000in}{0.000000in}}%
\pgfpathlineto{\pgfqpoint{-0.027778in}{0.000000in}}%
\pgfusepath{stroke,fill}%
}%
\begin{pgfscope}%
\pgfsys@transformshift{0.708220in}{0.985712in}%
\pgfsys@useobject{currentmarker}{}%
\end{pgfscope}%
\end{pgfscope}%
\begin{pgfscope}%
\pgfsetbuttcap%
\pgfsetroundjoin%
\definecolor{currentfill}{rgb}{0.000000,0.000000,0.000000}%
\pgfsetfillcolor{currentfill}%
\pgfsetlinewidth{0.602250pt}%
\definecolor{currentstroke}{rgb}{0.000000,0.000000,0.000000}%
\pgfsetstrokecolor{currentstroke}%
\pgfsetdash{}{0pt}%
\pgfsys@defobject{currentmarker}{\pgfqpoint{-0.027778in}{0.000000in}}{\pgfqpoint{-0.000000in}{0.000000in}}{%
\pgfpathmoveto{\pgfqpoint{-0.000000in}{0.000000in}}%
\pgfpathlineto{\pgfqpoint{-0.027778in}{0.000000in}}%
\pgfusepath{stroke,fill}%
}%
\begin{pgfscope}%
\pgfsys@transformshift{0.708220in}{1.035162in}%
\pgfsys@useobject{currentmarker}{}%
\end{pgfscope}%
\end{pgfscope}%
\begin{pgfscope}%
\pgfsetbuttcap%
\pgfsetroundjoin%
\definecolor{currentfill}{rgb}{0.000000,0.000000,0.000000}%
\pgfsetfillcolor{currentfill}%
\pgfsetlinewidth{0.602250pt}%
\definecolor{currentstroke}{rgb}{0.000000,0.000000,0.000000}%
\pgfsetstrokecolor{currentstroke}%
\pgfsetdash{}{0pt}%
\pgfsys@defobject{currentmarker}{\pgfqpoint{-0.027778in}{0.000000in}}{\pgfqpoint{-0.000000in}{0.000000in}}{%
\pgfpathmoveto{\pgfqpoint{-0.000000in}{0.000000in}}%
\pgfpathlineto{\pgfqpoint{-0.027778in}{0.000000in}}%
\pgfusepath{stroke,fill}%
}%
\begin{pgfscope}%
\pgfsys@transformshift{0.708220in}{1.070248in}%
\pgfsys@useobject{currentmarker}{}%
\end{pgfscope}%
\end{pgfscope}%
\begin{pgfscope}%
\pgfsetbuttcap%
\pgfsetroundjoin%
\definecolor{currentfill}{rgb}{0.000000,0.000000,0.000000}%
\pgfsetfillcolor{currentfill}%
\pgfsetlinewidth{0.602250pt}%
\definecolor{currentstroke}{rgb}{0.000000,0.000000,0.000000}%
\pgfsetstrokecolor{currentstroke}%
\pgfsetdash{}{0pt}%
\pgfsys@defobject{currentmarker}{\pgfqpoint{-0.027778in}{0.000000in}}{\pgfqpoint{-0.000000in}{0.000000in}}{%
\pgfpathmoveto{\pgfqpoint{-0.000000in}{0.000000in}}%
\pgfpathlineto{\pgfqpoint{-0.027778in}{0.000000in}}%
\pgfusepath{stroke,fill}%
}%
\begin{pgfscope}%
\pgfsys@transformshift{0.708220in}{1.097462in}%
\pgfsys@useobject{currentmarker}{}%
\end{pgfscope}%
\end{pgfscope}%
\begin{pgfscope}%
\pgfsetbuttcap%
\pgfsetroundjoin%
\definecolor{currentfill}{rgb}{0.000000,0.000000,0.000000}%
\pgfsetfillcolor{currentfill}%
\pgfsetlinewidth{0.602250pt}%
\definecolor{currentstroke}{rgb}{0.000000,0.000000,0.000000}%
\pgfsetstrokecolor{currentstroke}%
\pgfsetdash{}{0pt}%
\pgfsys@defobject{currentmarker}{\pgfqpoint{-0.027778in}{0.000000in}}{\pgfqpoint{-0.000000in}{0.000000in}}{%
\pgfpathmoveto{\pgfqpoint{-0.000000in}{0.000000in}}%
\pgfpathlineto{\pgfqpoint{-0.027778in}{0.000000in}}%
\pgfusepath{stroke,fill}%
}%
\begin{pgfscope}%
\pgfsys@transformshift{0.708220in}{1.119697in}%
\pgfsys@useobject{currentmarker}{}%
\end{pgfscope}%
\end{pgfscope}%
\begin{pgfscope}%
\pgfsetbuttcap%
\pgfsetroundjoin%
\definecolor{currentfill}{rgb}{0.000000,0.000000,0.000000}%
\pgfsetfillcolor{currentfill}%
\pgfsetlinewidth{0.602250pt}%
\definecolor{currentstroke}{rgb}{0.000000,0.000000,0.000000}%
\pgfsetstrokecolor{currentstroke}%
\pgfsetdash{}{0pt}%
\pgfsys@defobject{currentmarker}{\pgfqpoint{-0.027778in}{0.000000in}}{\pgfqpoint{-0.000000in}{0.000000in}}{%
\pgfpathmoveto{\pgfqpoint{-0.000000in}{0.000000in}}%
\pgfpathlineto{\pgfqpoint{-0.027778in}{0.000000in}}%
\pgfusepath{stroke,fill}%
}%
\begin{pgfscope}%
\pgfsys@transformshift{0.708220in}{1.138497in}%
\pgfsys@useobject{currentmarker}{}%
\end{pgfscope}%
\end{pgfscope}%
\begin{pgfscope}%
\pgfsetbuttcap%
\pgfsetroundjoin%
\definecolor{currentfill}{rgb}{0.000000,0.000000,0.000000}%
\pgfsetfillcolor{currentfill}%
\pgfsetlinewidth{0.602250pt}%
\definecolor{currentstroke}{rgb}{0.000000,0.000000,0.000000}%
\pgfsetstrokecolor{currentstroke}%
\pgfsetdash{}{0pt}%
\pgfsys@defobject{currentmarker}{\pgfqpoint{-0.027778in}{0.000000in}}{\pgfqpoint{-0.000000in}{0.000000in}}{%
\pgfpathmoveto{\pgfqpoint{-0.000000in}{0.000000in}}%
\pgfpathlineto{\pgfqpoint{-0.027778in}{0.000000in}}%
\pgfusepath{stroke,fill}%
}%
\begin{pgfscope}%
\pgfsys@transformshift{0.708220in}{1.154783in}%
\pgfsys@useobject{currentmarker}{}%
\end{pgfscope}%
\end{pgfscope}%
\begin{pgfscope}%
\pgfsetbuttcap%
\pgfsetroundjoin%
\definecolor{currentfill}{rgb}{0.000000,0.000000,0.000000}%
\pgfsetfillcolor{currentfill}%
\pgfsetlinewidth{0.602250pt}%
\definecolor{currentstroke}{rgb}{0.000000,0.000000,0.000000}%
\pgfsetstrokecolor{currentstroke}%
\pgfsetdash{}{0pt}%
\pgfsys@defobject{currentmarker}{\pgfqpoint{-0.027778in}{0.000000in}}{\pgfqpoint{-0.000000in}{0.000000in}}{%
\pgfpathmoveto{\pgfqpoint{-0.000000in}{0.000000in}}%
\pgfpathlineto{\pgfqpoint{-0.027778in}{0.000000in}}%
\pgfusepath{stroke,fill}%
}%
\begin{pgfscope}%
\pgfsys@transformshift{0.708220in}{1.169147in}%
\pgfsys@useobject{currentmarker}{}%
\end{pgfscope}%
\end{pgfscope}%
\begin{pgfscope}%
\pgfsetbuttcap%
\pgfsetroundjoin%
\definecolor{currentfill}{rgb}{0.000000,0.000000,0.000000}%
\pgfsetfillcolor{currentfill}%
\pgfsetlinewidth{0.602250pt}%
\definecolor{currentstroke}{rgb}{0.000000,0.000000,0.000000}%
\pgfsetstrokecolor{currentstroke}%
\pgfsetdash{}{0pt}%
\pgfsys@defobject{currentmarker}{\pgfqpoint{-0.027778in}{0.000000in}}{\pgfqpoint{-0.000000in}{0.000000in}}{%
\pgfpathmoveto{\pgfqpoint{-0.000000in}{0.000000in}}%
\pgfpathlineto{\pgfqpoint{-0.027778in}{0.000000in}}%
\pgfusepath{stroke,fill}%
}%
\begin{pgfscope}%
\pgfsys@transformshift{0.708220in}{1.266532in}%
\pgfsys@useobject{currentmarker}{}%
\end{pgfscope}%
\end{pgfscope}%
\begin{pgfscope}%
\pgfsetbuttcap%
\pgfsetroundjoin%
\definecolor{currentfill}{rgb}{0.000000,0.000000,0.000000}%
\pgfsetfillcolor{currentfill}%
\pgfsetlinewidth{0.602250pt}%
\definecolor{currentstroke}{rgb}{0.000000,0.000000,0.000000}%
\pgfsetstrokecolor{currentstroke}%
\pgfsetdash{}{0pt}%
\pgfsys@defobject{currentmarker}{\pgfqpoint{-0.027778in}{0.000000in}}{\pgfqpoint{-0.000000in}{0.000000in}}{%
\pgfpathmoveto{\pgfqpoint{-0.000000in}{0.000000in}}%
\pgfpathlineto{\pgfqpoint{-0.027778in}{0.000000in}}%
\pgfusepath{stroke,fill}%
}%
\begin{pgfscope}%
\pgfsys@transformshift{0.708220in}{1.315982in}%
\pgfsys@useobject{currentmarker}{}%
\end{pgfscope}%
\end{pgfscope}%
\begin{pgfscope}%
\pgfsetbuttcap%
\pgfsetroundjoin%
\definecolor{currentfill}{rgb}{0.000000,0.000000,0.000000}%
\pgfsetfillcolor{currentfill}%
\pgfsetlinewidth{0.602250pt}%
\definecolor{currentstroke}{rgb}{0.000000,0.000000,0.000000}%
\pgfsetstrokecolor{currentstroke}%
\pgfsetdash{}{0pt}%
\pgfsys@defobject{currentmarker}{\pgfqpoint{-0.027778in}{0.000000in}}{\pgfqpoint{-0.000000in}{0.000000in}}{%
\pgfpathmoveto{\pgfqpoint{-0.000000in}{0.000000in}}%
\pgfpathlineto{\pgfqpoint{-0.027778in}{0.000000in}}%
\pgfusepath{stroke,fill}%
}%
\begin{pgfscope}%
\pgfsys@transformshift{0.708220in}{1.351067in}%
\pgfsys@useobject{currentmarker}{}%
\end{pgfscope}%
\end{pgfscope}%
\begin{pgfscope}%
\pgfsetbuttcap%
\pgfsetroundjoin%
\definecolor{currentfill}{rgb}{0.000000,0.000000,0.000000}%
\pgfsetfillcolor{currentfill}%
\pgfsetlinewidth{0.602250pt}%
\definecolor{currentstroke}{rgb}{0.000000,0.000000,0.000000}%
\pgfsetstrokecolor{currentstroke}%
\pgfsetdash{}{0pt}%
\pgfsys@defobject{currentmarker}{\pgfqpoint{-0.027778in}{0.000000in}}{\pgfqpoint{-0.000000in}{0.000000in}}{%
\pgfpathmoveto{\pgfqpoint{-0.000000in}{0.000000in}}%
\pgfpathlineto{\pgfqpoint{-0.027778in}{0.000000in}}%
\pgfusepath{stroke,fill}%
}%
\begin{pgfscope}%
\pgfsys@transformshift{0.708220in}{1.378281in}%
\pgfsys@useobject{currentmarker}{}%
\end{pgfscope}%
\end{pgfscope}%
\begin{pgfscope}%
\pgfsetbuttcap%
\pgfsetroundjoin%
\definecolor{currentfill}{rgb}{0.000000,0.000000,0.000000}%
\pgfsetfillcolor{currentfill}%
\pgfsetlinewidth{0.602250pt}%
\definecolor{currentstroke}{rgb}{0.000000,0.000000,0.000000}%
\pgfsetstrokecolor{currentstroke}%
\pgfsetdash{}{0pt}%
\pgfsys@defobject{currentmarker}{\pgfqpoint{-0.027778in}{0.000000in}}{\pgfqpoint{-0.000000in}{0.000000in}}{%
\pgfpathmoveto{\pgfqpoint{-0.000000in}{0.000000in}}%
\pgfpathlineto{\pgfqpoint{-0.027778in}{0.000000in}}%
\pgfusepath{stroke,fill}%
}%
\begin{pgfscope}%
\pgfsys@transformshift{0.708220in}{1.400517in}%
\pgfsys@useobject{currentmarker}{}%
\end{pgfscope}%
\end{pgfscope}%
\begin{pgfscope}%
\pgfsetbuttcap%
\pgfsetroundjoin%
\definecolor{currentfill}{rgb}{0.000000,0.000000,0.000000}%
\pgfsetfillcolor{currentfill}%
\pgfsetlinewidth{0.602250pt}%
\definecolor{currentstroke}{rgb}{0.000000,0.000000,0.000000}%
\pgfsetstrokecolor{currentstroke}%
\pgfsetdash{}{0pt}%
\pgfsys@defobject{currentmarker}{\pgfqpoint{-0.027778in}{0.000000in}}{\pgfqpoint{-0.000000in}{0.000000in}}{%
\pgfpathmoveto{\pgfqpoint{-0.000000in}{0.000000in}}%
\pgfpathlineto{\pgfqpoint{-0.027778in}{0.000000in}}%
\pgfusepath{stroke,fill}%
}%
\begin{pgfscope}%
\pgfsys@transformshift{0.708220in}{1.419317in}%
\pgfsys@useobject{currentmarker}{}%
\end{pgfscope}%
\end{pgfscope}%
\begin{pgfscope}%
\pgfsetbuttcap%
\pgfsetroundjoin%
\definecolor{currentfill}{rgb}{0.000000,0.000000,0.000000}%
\pgfsetfillcolor{currentfill}%
\pgfsetlinewidth{0.602250pt}%
\definecolor{currentstroke}{rgb}{0.000000,0.000000,0.000000}%
\pgfsetstrokecolor{currentstroke}%
\pgfsetdash{}{0pt}%
\pgfsys@defobject{currentmarker}{\pgfqpoint{-0.027778in}{0.000000in}}{\pgfqpoint{-0.000000in}{0.000000in}}{%
\pgfpathmoveto{\pgfqpoint{-0.000000in}{0.000000in}}%
\pgfpathlineto{\pgfqpoint{-0.027778in}{0.000000in}}%
\pgfusepath{stroke,fill}%
}%
\begin{pgfscope}%
\pgfsys@transformshift{0.708220in}{1.435602in}%
\pgfsys@useobject{currentmarker}{}%
\end{pgfscope}%
\end{pgfscope}%
\begin{pgfscope}%
\pgfsetbuttcap%
\pgfsetroundjoin%
\definecolor{currentfill}{rgb}{0.000000,0.000000,0.000000}%
\pgfsetfillcolor{currentfill}%
\pgfsetlinewidth{0.602250pt}%
\definecolor{currentstroke}{rgb}{0.000000,0.000000,0.000000}%
\pgfsetstrokecolor{currentstroke}%
\pgfsetdash{}{0pt}%
\pgfsys@defobject{currentmarker}{\pgfqpoint{-0.027778in}{0.000000in}}{\pgfqpoint{-0.000000in}{0.000000in}}{%
\pgfpathmoveto{\pgfqpoint{-0.000000in}{0.000000in}}%
\pgfpathlineto{\pgfqpoint{-0.027778in}{0.000000in}}%
\pgfusepath{stroke,fill}%
}%
\begin{pgfscope}%
\pgfsys@transformshift{0.708220in}{1.449967in}%
\pgfsys@useobject{currentmarker}{}%
\end{pgfscope}%
\end{pgfscope}%
\begin{pgfscope}%
\pgfsetbuttcap%
\pgfsetroundjoin%
\definecolor{currentfill}{rgb}{0.000000,0.000000,0.000000}%
\pgfsetfillcolor{currentfill}%
\pgfsetlinewidth{0.602250pt}%
\definecolor{currentstroke}{rgb}{0.000000,0.000000,0.000000}%
\pgfsetstrokecolor{currentstroke}%
\pgfsetdash{}{0pt}%
\pgfsys@defobject{currentmarker}{\pgfqpoint{-0.027778in}{0.000000in}}{\pgfqpoint{-0.000000in}{0.000000in}}{%
\pgfpathmoveto{\pgfqpoint{-0.000000in}{0.000000in}}%
\pgfpathlineto{\pgfqpoint{-0.027778in}{0.000000in}}%
\pgfusepath{stroke,fill}%
}%
\begin{pgfscope}%
\pgfsys@transformshift{0.708220in}{1.547352in}%
\pgfsys@useobject{currentmarker}{}%
\end{pgfscope}%
\end{pgfscope}%
\begin{pgfscope}%
\pgfsetbuttcap%
\pgfsetroundjoin%
\definecolor{currentfill}{rgb}{0.000000,0.000000,0.000000}%
\pgfsetfillcolor{currentfill}%
\pgfsetlinewidth{0.602250pt}%
\definecolor{currentstroke}{rgb}{0.000000,0.000000,0.000000}%
\pgfsetstrokecolor{currentstroke}%
\pgfsetdash{}{0pt}%
\pgfsys@defobject{currentmarker}{\pgfqpoint{-0.027778in}{0.000000in}}{\pgfqpoint{-0.000000in}{0.000000in}}{%
\pgfpathmoveto{\pgfqpoint{-0.000000in}{0.000000in}}%
\pgfpathlineto{\pgfqpoint{-0.027778in}{0.000000in}}%
\pgfusepath{stroke,fill}%
}%
\begin{pgfscope}%
\pgfsys@transformshift{0.708220in}{1.596801in}%
\pgfsys@useobject{currentmarker}{}%
\end{pgfscope}%
\end{pgfscope}%
\begin{pgfscope}%
\pgfsetbuttcap%
\pgfsetroundjoin%
\definecolor{currentfill}{rgb}{0.000000,0.000000,0.000000}%
\pgfsetfillcolor{currentfill}%
\pgfsetlinewidth{0.602250pt}%
\definecolor{currentstroke}{rgb}{0.000000,0.000000,0.000000}%
\pgfsetstrokecolor{currentstroke}%
\pgfsetdash{}{0pt}%
\pgfsys@defobject{currentmarker}{\pgfqpoint{-0.027778in}{0.000000in}}{\pgfqpoint{-0.000000in}{0.000000in}}{%
\pgfpathmoveto{\pgfqpoint{-0.000000in}{0.000000in}}%
\pgfpathlineto{\pgfqpoint{-0.027778in}{0.000000in}}%
\pgfusepath{stroke,fill}%
}%
\begin{pgfscope}%
\pgfsys@transformshift{0.708220in}{1.631887in}%
\pgfsys@useobject{currentmarker}{}%
\end{pgfscope}%
\end{pgfscope}%
\begin{pgfscope}%
\pgfsetbuttcap%
\pgfsetroundjoin%
\definecolor{currentfill}{rgb}{0.000000,0.000000,0.000000}%
\pgfsetfillcolor{currentfill}%
\pgfsetlinewidth{0.602250pt}%
\definecolor{currentstroke}{rgb}{0.000000,0.000000,0.000000}%
\pgfsetstrokecolor{currentstroke}%
\pgfsetdash{}{0pt}%
\pgfsys@defobject{currentmarker}{\pgfqpoint{-0.027778in}{0.000000in}}{\pgfqpoint{-0.000000in}{0.000000in}}{%
\pgfpathmoveto{\pgfqpoint{-0.000000in}{0.000000in}}%
\pgfpathlineto{\pgfqpoint{-0.027778in}{0.000000in}}%
\pgfusepath{stroke,fill}%
}%
\begin{pgfscope}%
\pgfsys@transformshift{0.708220in}{1.659101in}%
\pgfsys@useobject{currentmarker}{}%
\end{pgfscope}%
\end{pgfscope}%
\begin{pgfscope}%
\pgfsetbuttcap%
\pgfsetroundjoin%
\definecolor{currentfill}{rgb}{0.000000,0.000000,0.000000}%
\pgfsetfillcolor{currentfill}%
\pgfsetlinewidth{0.602250pt}%
\definecolor{currentstroke}{rgb}{0.000000,0.000000,0.000000}%
\pgfsetstrokecolor{currentstroke}%
\pgfsetdash{}{0pt}%
\pgfsys@defobject{currentmarker}{\pgfqpoint{-0.027778in}{0.000000in}}{\pgfqpoint{-0.000000in}{0.000000in}}{%
\pgfpathmoveto{\pgfqpoint{-0.000000in}{0.000000in}}%
\pgfpathlineto{\pgfqpoint{-0.027778in}{0.000000in}}%
\pgfusepath{stroke,fill}%
}%
\begin{pgfscope}%
\pgfsys@transformshift{0.708220in}{1.681337in}%
\pgfsys@useobject{currentmarker}{}%
\end{pgfscope}%
\end{pgfscope}%
\begin{pgfscope}%
\pgfsetbuttcap%
\pgfsetroundjoin%
\definecolor{currentfill}{rgb}{0.000000,0.000000,0.000000}%
\pgfsetfillcolor{currentfill}%
\pgfsetlinewidth{0.602250pt}%
\definecolor{currentstroke}{rgb}{0.000000,0.000000,0.000000}%
\pgfsetstrokecolor{currentstroke}%
\pgfsetdash{}{0pt}%
\pgfsys@defobject{currentmarker}{\pgfqpoint{-0.027778in}{0.000000in}}{\pgfqpoint{-0.000000in}{0.000000in}}{%
\pgfpathmoveto{\pgfqpoint{-0.000000in}{0.000000in}}%
\pgfpathlineto{\pgfqpoint{-0.027778in}{0.000000in}}%
\pgfusepath{stroke,fill}%
}%
\begin{pgfscope}%
\pgfsys@transformshift{0.708220in}{1.700137in}%
\pgfsys@useobject{currentmarker}{}%
\end{pgfscope}%
\end{pgfscope}%
\begin{pgfscope}%
\pgfsetbuttcap%
\pgfsetroundjoin%
\definecolor{currentfill}{rgb}{0.000000,0.000000,0.000000}%
\pgfsetfillcolor{currentfill}%
\pgfsetlinewidth{0.602250pt}%
\definecolor{currentstroke}{rgb}{0.000000,0.000000,0.000000}%
\pgfsetstrokecolor{currentstroke}%
\pgfsetdash{}{0pt}%
\pgfsys@defobject{currentmarker}{\pgfqpoint{-0.027778in}{0.000000in}}{\pgfqpoint{-0.000000in}{0.000000in}}{%
\pgfpathmoveto{\pgfqpoint{-0.000000in}{0.000000in}}%
\pgfpathlineto{\pgfqpoint{-0.027778in}{0.000000in}}%
\pgfusepath{stroke,fill}%
}%
\begin{pgfscope}%
\pgfsys@transformshift{0.708220in}{1.716422in}%
\pgfsys@useobject{currentmarker}{}%
\end{pgfscope}%
\end{pgfscope}%
\begin{pgfscope}%
\pgfsetbuttcap%
\pgfsetroundjoin%
\definecolor{currentfill}{rgb}{0.000000,0.000000,0.000000}%
\pgfsetfillcolor{currentfill}%
\pgfsetlinewidth{0.602250pt}%
\definecolor{currentstroke}{rgb}{0.000000,0.000000,0.000000}%
\pgfsetstrokecolor{currentstroke}%
\pgfsetdash{}{0pt}%
\pgfsys@defobject{currentmarker}{\pgfqpoint{-0.027778in}{0.000000in}}{\pgfqpoint{-0.000000in}{0.000000in}}{%
\pgfpathmoveto{\pgfqpoint{-0.000000in}{0.000000in}}%
\pgfpathlineto{\pgfqpoint{-0.027778in}{0.000000in}}%
\pgfusepath{stroke,fill}%
}%
\begin{pgfscope}%
\pgfsys@transformshift{0.708220in}{1.730786in}%
\pgfsys@useobject{currentmarker}{}%
\end{pgfscope}%
\end{pgfscope}%
\begin{pgfscope}%
\pgfsetbuttcap%
\pgfsetroundjoin%
\definecolor{currentfill}{rgb}{0.000000,0.000000,0.000000}%
\pgfsetfillcolor{currentfill}%
\pgfsetlinewidth{0.602250pt}%
\definecolor{currentstroke}{rgb}{0.000000,0.000000,0.000000}%
\pgfsetstrokecolor{currentstroke}%
\pgfsetdash{}{0pt}%
\pgfsys@defobject{currentmarker}{\pgfqpoint{-0.027778in}{0.000000in}}{\pgfqpoint{-0.000000in}{0.000000in}}{%
\pgfpathmoveto{\pgfqpoint{-0.000000in}{0.000000in}}%
\pgfpathlineto{\pgfqpoint{-0.027778in}{0.000000in}}%
\pgfusepath{stroke,fill}%
}%
\begin{pgfscope}%
\pgfsys@transformshift{0.708220in}{1.828171in}%
\pgfsys@useobject{currentmarker}{}%
\end{pgfscope}%
\end{pgfscope}%
\begin{pgfscope}%
\pgfsetbuttcap%
\pgfsetroundjoin%
\definecolor{currentfill}{rgb}{0.000000,0.000000,0.000000}%
\pgfsetfillcolor{currentfill}%
\pgfsetlinewidth{0.602250pt}%
\definecolor{currentstroke}{rgb}{0.000000,0.000000,0.000000}%
\pgfsetstrokecolor{currentstroke}%
\pgfsetdash{}{0pt}%
\pgfsys@defobject{currentmarker}{\pgfqpoint{-0.027778in}{0.000000in}}{\pgfqpoint{-0.000000in}{0.000000in}}{%
\pgfpathmoveto{\pgfqpoint{-0.000000in}{0.000000in}}%
\pgfpathlineto{\pgfqpoint{-0.027778in}{0.000000in}}%
\pgfusepath{stroke,fill}%
}%
\begin{pgfscope}%
\pgfsys@transformshift{0.708220in}{1.877621in}%
\pgfsys@useobject{currentmarker}{}%
\end{pgfscope}%
\end{pgfscope}%
\begin{pgfscope}%
\pgfsetbuttcap%
\pgfsetroundjoin%
\definecolor{currentfill}{rgb}{0.000000,0.000000,0.000000}%
\pgfsetfillcolor{currentfill}%
\pgfsetlinewidth{0.602250pt}%
\definecolor{currentstroke}{rgb}{0.000000,0.000000,0.000000}%
\pgfsetstrokecolor{currentstroke}%
\pgfsetdash{}{0pt}%
\pgfsys@defobject{currentmarker}{\pgfqpoint{-0.027778in}{0.000000in}}{\pgfqpoint{-0.000000in}{0.000000in}}{%
\pgfpathmoveto{\pgfqpoint{-0.000000in}{0.000000in}}%
\pgfpathlineto{\pgfqpoint{-0.027778in}{0.000000in}}%
\pgfusepath{stroke,fill}%
}%
\begin{pgfscope}%
\pgfsys@transformshift{0.708220in}{1.912706in}%
\pgfsys@useobject{currentmarker}{}%
\end{pgfscope}%
\end{pgfscope}%
\begin{pgfscope}%
\pgfsetbuttcap%
\pgfsetroundjoin%
\definecolor{currentfill}{rgb}{0.000000,0.000000,0.000000}%
\pgfsetfillcolor{currentfill}%
\pgfsetlinewidth{0.602250pt}%
\definecolor{currentstroke}{rgb}{0.000000,0.000000,0.000000}%
\pgfsetstrokecolor{currentstroke}%
\pgfsetdash{}{0pt}%
\pgfsys@defobject{currentmarker}{\pgfqpoint{-0.027778in}{0.000000in}}{\pgfqpoint{-0.000000in}{0.000000in}}{%
\pgfpathmoveto{\pgfqpoint{-0.000000in}{0.000000in}}%
\pgfpathlineto{\pgfqpoint{-0.027778in}{0.000000in}}%
\pgfusepath{stroke,fill}%
}%
\begin{pgfscope}%
\pgfsys@transformshift{0.708220in}{1.939921in}%
\pgfsys@useobject{currentmarker}{}%
\end{pgfscope}%
\end{pgfscope}%
\begin{pgfscope}%
\pgfsetbuttcap%
\pgfsetroundjoin%
\definecolor{currentfill}{rgb}{0.000000,0.000000,0.000000}%
\pgfsetfillcolor{currentfill}%
\pgfsetlinewidth{0.602250pt}%
\definecolor{currentstroke}{rgb}{0.000000,0.000000,0.000000}%
\pgfsetstrokecolor{currentstroke}%
\pgfsetdash{}{0pt}%
\pgfsys@defobject{currentmarker}{\pgfqpoint{-0.027778in}{0.000000in}}{\pgfqpoint{-0.000000in}{0.000000in}}{%
\pgfpathmoveto{\pgfqpoint{-0.000000in}{0.000000in}}%
\pgfpathlineto{\pgfqpoint{-0.027778in}{0.000000in}}%
\pgfusepath{stroke,fill}%
}%
\begin{pgfscope}%
\pgfsys@transformshift{0.708220in}{1.962156in}%
\pgfsys@useobject{currentmarker}{}%
\end{pgfscope}%
\end{pgfscope}%
\begin{pgfscope}%
\pgfsetbuttcap%
\pgfsetroundjoin%
\definecolor{currentfill}{rgb}{0.000000,0.000000,0.000000}%
\pgfsetfillcolor{currentfill}%
\pgfsetlinewidth{0.602250pt}%
\definecolor{currentstroke}{rgb}{0.000000,0.000000,0.000000}%
\pgfsetstrokecolor{currentstroke}%
\pgfsetdash{}{0pt}%
\pgfsys@defobject{currentmarker}{\pgfqpoint{-0.027778in}{0.000000in}}{\pgfqpoint{-0.000000in}{0.000000in}}{%
\pgfpathmoveto{\pgfqpoint{-0.000000in}{0.000000in}}%
\pgfpathlineto{\pgfqpoint{-0.027778in}{0.000000in}}%
\pgfusepath{stroke,fill}%
}%
\begin{pgfscope}%
\pgfsys@transformshift{0.708220in}{1.980956in}%
\pgfsys@useobject{currentmarker}{}%
\end{pgfscope}%
\end{pgfscope}%
\begin{pgfscope}%
\pgfsetbuttcap%
\pgfsetroundjoin%
\definecolor{currentfill}{rgb}{0.000000,0.000000,0.000000}%
\pgfsetfillcolor{currentfill}%
\pgfsetlinewidth{0.602250pt}%
\definecolor{currentstroke}{rgb}{0.000000,0.000000,0.000000}%
\pgfsetstrokecolor{currentstroke}%
\pgfsetdash{}{0pt}%
\pgfsys@defobject{currentmarker}{\pgfqpoint{-0.027778in}{0.000000in}}{\pgfqpoint{-0.000000in}{0.000000in}}{%
\pgfpathmoveto{\pgfqpoint{-0.000000in}{0.000000in}}%
\pgfpathlineto{\pgfqpoint{-0.027778in}{0.000000in}}%
\pgfusepath{stroke,fill}%
}%
\begin{pgfscope}%
\pgfsys@transformshift{0.708220in}{1.997241in}%
\pgfsys@useobject{currentmarker}{}%
\end{pgfscope}%
\end{pgfscope}%
\begin{pgfscope}%
\pgfsetbuttcap%
\pgfsetroundjoin%
\definecolor{currentfill}{rgb}{0.000000,0.000000,0.000000}%
\pgfsetfillcolor{currentfill}%
\pgfsetlinewidth{0.602250pt}%
\definecolor{currentstroke}{rgb}{0.000000,0.000000,0.000000}%
\pgfsetstrokecolor{currentstroke}%
\pgfsetdash{}{0pt}%
\pgfsys@defobject{currentmarker}{\pgfqpoint{-0.027778in}{0.000000in}}{\pgfqpoint{-0.000000in}{0.000000in}}{%
\pgfpathmoveto{\pgfqpoint{-0.000000in}{0.000000in}}%
\pgfpathlineto{\pgfqpoint{-0.027778in}{0.000000in}}%
\pgfusepath{stroke,fill}%
}%
\begin{pgfscope}%
\pgfsys@transformshift{0.708220in}{2.011606in}%
\pgfsys@useobject{currentmarker}{}%
\end{pgfscope}%
\end{pgfscope}%
\begin{pgfscope}%
\pgfsetbuttcap%
\pgfsetroundjoin%
\definecolor{currentfill}{rgb}{0.000000,0.000000,0.000000}%
\pgfsetfillcolor{currentfill}%
\pgfsetlinewidth{0.602250pt}%
\definecolor{currentstroke}{rgb}{0.000000,0.000000,0.000000}%
\pgfsetstrokecolor{currentstroke}%
\pgfsetdash{}{0pt}%
\pgfsys@defobject{currentmarker}{\pgfqpoint{-0.027778in}{0.000000in}}{\pgfqpoint{-0.000000in}{0.000000in}}{%
\pgfpathmoveto{\pgfqpoint{-0.000000in}{0.000000in}}%
\pgfpathlineto{\pgfqpoint{-0.027778in}{0.000000in}}%
\pgfusepath{stroke,fill}%
}%
\begin{pgfscope}%
\pgfsys@transformshift{0.708220in}{2.108991in}%
\pgfsys@useobject{currentmarker}{}%
\end{pgfscope}%
\end{pgfscope}%
\begin{pgfscope}%
\pgfsetbuttcap%
\pgfsetroundjoin%
\definecolor{currentfill}{rgb}{0.000000,0.000000,0.000000}%
\pgfsetfillcolor{currentfill}%
\pgfsetlinewidth{0.602250pt}%
\definecolor{currentstroke}{rgb}{0.000000,0.000000,0.000000}%
\pgfsetstrokecolor{currentstroke}%
\pgfsetdash{}{0pt}%
\pgfsys@defobject{currentmarker}{\pgfqpoint{-0.027778in}{0.000000in}}{\pgfqpoint{-0.000000in}{0.000000in}}{%
\pgfpathmoveto{\pgfqpoint{-0.000000in}{0.000000in}}%
\pgfpathlineto{\pgfqpoint{-0.027778in}{0.000000in}}%
\pgfusepath{stroke,fill}%
}%
\begin{pgfscope}%
\pgfsys@transformshift{0.708220in}{2.158441in}%
\pgfsys@useobject{currentmarker}{}%
\end{pgfscope}%
\end{pgfscope}%
\begin{pgfscope}%
\pgfsetbuttcap%
\pgfsetroundjoin%
\definecolor{currentfill}{rgb}{0.000000,0.000000,0.000000}%
\pgfsetfillcolor{currentfill}%
\pgfsetlinewidth{0.602250pt}%
\definecolor{currentstroke}{rgb}{0.000000,0.000000,0.000000}%
\pgfsetstrokecolor{currentstroke}%
\pgfsetdash{}{0pt}%
\pgfsys@defobject{currentmarker}{\pgfqpoint{-0.027778in}{0.000000in}}{\pgfqpoint{-0.000000in}{0.000000in}}{%
\pgfpathmoveto{\pgfqpoint{-0.000000in}{0.000000in}}%
\pgfpathlineto{\pgfqpoint{-0.027778in}{0.000000in}}%
\pgfusepath{stroke,fill}%
}%
\begin{pgfscope}%
\pgfsys@transformshift{0.708220in}{2.193526in}%
\pgfsys@useobject{currentmarker}{}%
\end{pgfscope}%
\end{pgfscope}%
\begin{pgfscope}%
\pgfsetbuttcap%
\pgfsetroundjoin%
\definecolor{currentfill}{rgb}{0.000000,0.000000,0.000000}%
\pgfsetfillcolor{currentfill}%
\pgfsetlinewidth{0.602250pt}%
\definecolor{currentstroke}{rgb}{0.000000,0.000000,0.000000}%
\pgfsetstrokecolor{currentstroke}%
\pgfsetdash{}{0pt}%
\pgfsys@defobject{currentmarker}{\pgfqpoint{-0.027778in}{0.000000in}}{\pgfqpoint{-0.000000in}{0.000000in}}{%
\pgfpathmoveto{\pgfqpoint{-0.000000in}{0.000000in}}%
\pgfpathlineto{\pgfqpoint{-0.027778in}{0.000000in}}%
\pgfusepath{stroke,fill}%
}%
\begin{pgfscope}%
\pgfsys@transformshift{0.708220in}{2.220740in}%
\pgfsys@useobject{currentmarker}{}%
\end{pgfscope}%
\end{pgfscope}%
\begin{pgfscope}%
\pgfsetbuttcap%
\pgfsetroundjoin%
\definecolor{currentfill}{rgb}{0.000000,0.000000,0.000000}%
\pgfsetfillcolor{currentfill}%
\pgfsetlinewidth{0.602250pt}%
\definecolor{currentstroke}{rgb}{0.000000,0.000000,0.000000}%
\pgfsetstrokecolor{currentstroke}%
\pgfsetdash{}{0pt}%
\pgfsys@defobject{currentmarker}{\pgfqpoint{-0.027778in}{0.000000in}}{\pgfqpoint{-0.000000in}{0.000000in}}{%
\pgfpathmoveto{\pgfqpoint{-0.000000in}{0.000000in}}%
\pgfpathlineto{\pgfqpoint{-0.027778in}{0.000000in}}%
\pgfusepath{stroke,fill}%
}%
\begin{pgfscope}%
\pgfsys@transformshift{0.708220in}{2.242976in}%
\pgfsys@useobject{currentmarker}{}%
\end{pgfscope}%
\end{pgfscope}%
\begin{pgfscope}%
\pgfsetbuttcap%
\pgfsetroundjoin%
\definecolor{currentfill}{rgb}{0.000000,0.000000,0.000000}%
\pgfsetfillcolor{currentfill}%
\pgfsetlinewidth{0.602250pt}%
\definecolor{currentstroke}{rgb}{0.000000,0.000000,0.000000}%
\pgfsetstrokecolor{currentstroke}%
\pgfsetdash{}{0pt}%
\pgfsys@defobject{currentmarker}{\pgfqpoint{-0.027778in}{0.000000in}}{\pgfqpoint{-0.000000in}{0.000000in}}{%
\pgfpathmoveto{\pgfqpoint{-0.000000in}{0.000000in}}%
\pgfpathlineto{\pgfqpoint{-0.027778in}{0.000000in}}%
\pgfusepath{stroke,fill}%
}%
\begin{pgfscope}%
\pgfsys@transformshift{0.708220in}{2.261776in}%
\pgfsys@useobject{currentmarker}{}%
\end{pgfscope}%
\end{pgfscope}%
\begin{pgfscope}%
\pgfsetbuttcap%
\pgfsetroundjoin%
\definecolor{currentfill}{rgb}{0.000000,0.000000,0.000000}%
\pgfsetfillcolor{currentfill}%
\pgfsetlinewidth{0.602250pt}%
\definecolor{currentstroke}{rgb}{0.000000,0.000000,0.000000}%
\pgfsetstrokecolor{currentstroke}%
\pgfsetdash{}{0pt}%
\pgfsys@defobject{currentmarker}{\pgfqpoint{-0.027778in}{0.000000in}}{\pgfqpoint{-0.000000in}{0.000000in}}{%
\pgfpathmoveto{\pgfqpoint{-0.000000in}{0.000000in}}%
\pgfpathlineto{\pgfqpoint{-0.027778in}{0.000000in}}%
\pgfusepath{stroke,fill}%
}%
\begin{pgfscope}%
\pgfsys@transformshift{0.708220in}{2.278061in}%
\pgfsys@useobject{currentmarker}{}%
\end{pgfscope}%
\end{pgfscope}%
\begin{pgfscope}%
\pgfsetbuttcap%
\pgfsetroundjoin%
\definecolor{currentfill}{rgb}{0.000000,0.000000,0.000000}%
\pgfsetfillcolor{currentfill}%
\pgfsetlinewidth{0.602250pt}%
\definecolor{currentstroke}{rgb}{0.000000,0.000000,0.000000}%
\pgfsetstrokecolor{currentstroke}%
\pgfsetdash{}{0pt}%
\pgfsys@defobject{currentmarker}{\pgfqpoint{-0.027778in}{0.000000in}}{\pgfqpoint{-0.000000in}{0.000000in}}{%
\pgfpathmoveto{\pgfqpoint{-0.000000in}{0.000000in}}%
\pgfpathlineto{\pgfqpoint{-0.027778in}{0.000000in}}%
\pgfusepath{stroke,fill}%
}%
\begin{pgfscope}%
\pgfsys@transformshift{0.708220in}{2.292426in}%
\pgfsys@useobject{currentmarker}{}%
\end{pgfscope}%
\end{pgfscope}%
\begin{pgfscope}%
\definecolor{textcolor}{rgb}{0.000000,0.000000,0.000000}%
\pgfsetstrokecolor{textcolor}%
\pgfsetfillcolor{textcolor}%
\pgftext[x=0.288855in,y=1.420549in,,bottom,rotate=90.000000]{\color{textcolor}\rmfamily\fontsize{10.000000}{12.000000}\selectfont Longest solving time (s)}%
\end{pgfscope}%
\begin{pgfscope}%
\pgfpathrectangle{\pgfqpoint{0.708220in}{0.535823in}}{\pgfqpoint{5.045427in}{1.769453in}}%
\pgfusepath{clip}%
\pgfsetrectcap%
\pgfsetroundjoin%
\pgfsetlinewidth{1.003750pt}%
\definecolor{currentstroke}{rgb}{0.121569,0.466667,0.705882}%
\pgfsetstrokecolor{currentstroke}%
\pgfsetdash{}{0pt}%
\pgfpathmoveto{\pgfqpoint{0.708220in}{1.081871in}}%
\pgfpathlineto{\pgfqpoint{0.720833in}{1.119697in}}%
\pgfpathlineto{\pgfqpoint{0.733447in}{1.177018in}}%
\pgfpathlineto{\pgfqpoint{0.771288in}{1.180771in}}%
\pgfpathlineto{\pgfqpoint{0.796515in}{1.180771in}}%
\pgfpathlineto{\pgfqpoint{0.809128in}{1.181997in}}%
\pgfpathlineto{\pgfqpoint{0.846969in}{1.181997in}}%
\pgfpathlineto{\pgfqpoint{0.859583in}{1.183210in}}%
\pgfpathlineto{\pgfqpoint{0.897423in}{1.183210in}}%
\pgfpathlineto{\pgfqpoint{0.910037in}{1.184412in}}%
\pgfpathlineto{\pgfqpoint{0.947878in}{1.184412in}}%
\pgfpathlineto{\pgfqpoint{0.960491in}{1.185602in}}%
\pgfpathlineto{\pgfqpoint{0.985718in}{1.185602in}}%
\pgfpathlineto{\pgfqpoint{0.998332in}{1.186780in}}%
\pgfpathlineto{\pgfqpoint{1.010945in}{1.186780in}}%
\pgfpathlineto{\pgfqpoint{1.023559in}{1.187947in}}%
\pgfpathlineto{\pgfqpoint{1.036173in}{1.187947in}}%
\pgfpathlineto{\pgfqpoint{1.048786in}{1.189103in}}%
\pgfpathlineto{\pgfqpoint{1.074013in}{1.189103in}}%
\pgfpathlineto{\pgfqpoint{1.099240in}{1.191383in}}%
\pgfpathlineto{\pgfqpoint{1.124468in}{1.191383in}}%
\pgfpathlineto{\pgfqpoint{1.137081in}{1.192507in}}%
\pgfpathlineto{\pgfqpoint{1.162308in}{1.192507in}}%
\pgfpathlineto{\pgfqpoint{1.187535in}{1.193621in}}%
\pgfpathlineto{\pgfqpoint{1.263217in}{1.194724in}}%
\pgfpathlineto{\pgfqpoint{1.301057in}{1.196902in}}%
\pgfpathlineto{\pgfqpoint{1.439807in}{1.197977in}}%
\pgfpathlineto{\pgfqpoint{1.465034in}{1.199042in}}%
\pgfpathlineto{\pgfqpoint{1.528102in}{1.200098in}}%
\pgfpathlineto{\pgfqpoint{1.553329in}{1.201145in}}%
\pgfpathlineto{\pgfqpoint{1.578556in}{1.201145in}}%
\pgfpathlineto{\pgfqpoint{1.591170in}{1.203212in}}%
\pgfpathlineto{\pgfqpoint{1.629010in}{1.204233in}}%
\pgfpathlineto{\pgfqpoint{1.654237in}{1.205245in}}%
\pgfpathlineto{\pgfqpoint{1.729919in}{1.206248in}}%
\pgfpathlineto{\pgfqpoint{1.755146in}{1.207244in}}%
\pgfpathlineto{\pgfqpoint{1.805600in}{1.208232in}}%
\pgfpathlineto{\pgfqpoint{1.843441in}{1.210183in}}%
\pgfpathlineto{\pgfqpoint{1.906509in}{1.210183in}}%
\pgfpathlineto{\pgfqpoint{1.919122in}{1.212104in}}%
\pgfpathlineto{\pgfqpoint{1.944349in}{1.213053in}}%
\pgfpathlineto{\pgfqpoint{1.969577in}{1.213994in}}%
\pgfpathlineto{\pgfqpoint{2.045258in}{1.214929in}}%
\pgfpathlineto{\pgfqpoint{2.070485in}{1.215856in}}%
\pgfpathlineto{\pgfqpoint{2.158780in}{1.216777in}}%
\pgfpathlineto{\pgfqpoint{2.184007in}{1.217690in}}%
\pgfpathlineto{\pgfqpoint{2.234461in}{1.218597in}}%
\pgfpathlineto{\pgfqpoint{2.272302in}{1.220391in}}%
\pgfpathlineto{\pgfqpoint{2.297529in}{1.221278in}}%
\pgfpathlineto{\pgfqpoint{2.310143in}{1.224762in}}%
\pgfpathlineto{\pgfqpoint{2.322756in}{1.225618in}}%
\pgfpathlineto{\pgfqpoint{2.335370in}{1.228150in}}%
\pgfpathlineto{\pgfqpoint{2.360597in}{1.228983in}}%
\pgfpathlineto{\pgfqpoint{2.373211in}{1.231447in}}%
\pgfpathlineto{\pgfqpoint{2.385824in}{1.231447in}}%
\pgfpathlineto{\pgfqpoint{2.398438in}{1.233062in}}%
\pgfpathlineto{\pgfqpoint{2.461506in}{1.233862in}}%
\pgfpathlineto{\pgfqpoint{2.549801in}{1.238553in}}%
\pgfpathlineto{\pgfqpoint{2.562414in}{1.238553in}}%
\pgfpathlineto{\pgfqpoint{2.612868in}{1.244540in}}%
\pgfpathlineto{\pgfqpoint{2.650709in}{1.245268in}}%
\pgfpathlineto{\pgfqpoint{2.663323in}{1.246711in}}%
\pgfpathlineto{\pgfqpoint{2.675936in}{1.249548in}}%
\pgfpathlineto{\pgfqpoint{2.688550in}{1.250942in}}%
\pgfpathlineto{\pgfqpoint{2.713777in}{1.251633in}}%
\pgfpathlineto{\pgfqpoint{2.751618in}{1.255030in}}%
\pgfpathlineto{\pgfqpoint{2.991275in}{1.268348in}}%
\pgfpathlineto{\pgfqpoint{3.016503in}{1.270137in}}%
\pgfpathlineto{\pgfqpoint{3.041730in}{1.270728in}}%
\pgfpathlineto{\pgfqpoint{3.054343in}{1.271900in}}%
\pgfpathlineto{\pgfqpoint{3.079570in}{1.275918in}}%
\pgfpathlineto{\pgfqpoint{3.142638in}{1.279260in}}%
\pgfpathlineto{\pgfqpoint{3.167865in}{1.283046in}}%
\pgfpathlineto{\pgfqpoint{3.193092in}{1.283577in}}%
\pgfpathlineto{\pgfqpoint{3.205706in}{1.289780in}}%
\pgfpathlineto{\pgfqpoint{3.218320in}{1.293257in}}%
\pgfpathlineto{\pgfqpoint{3.268774in}{1.298060in}}%
\pgfpathlineto{\pgfqpoint{3.319228in}{1.304926in}}%
\pgfpathlineto{\pgfqpoint{3.331842in}{1.305370in}}%
\pgfpathlineto{\pgfqpoint{3.357069in}{1.319587in}}%
\pgfpathlineto{\pgfqpoint{3.369682in}{1.319981in}}%
\pgfpathlineto{\pgfqpoint{3.394910in}{1.323088in}}%
\pgfpathlineto{\pgfqpoint{3.407523in}{1.326118in}}%
\pgfpathlineto{\pgfqpoint{3.420137in}{1.336512in}}%
\pgfpathlineto{\pgfqpoint{3.457977in}{1.340566in}}%
\pgfpathlineto{\pgfqpoint{3.470591in}{1.344168in}}%
\pgfpathlineto{\pgfqpoint{3.495818in}{1.346089in}}%
\pgfpathlineto{\pgfqpoint{3.508432in}{1.353781in}}%
\pgfpathlineto{\pgfqpoint{3.546272in}{1.355850in}}%
\pgfpathlineto{\pgfqpoint{3.571499in}{1.365702in}}%
\pgfpathlineto{\pgfqpoint{3.584113in}{1.375319in}}%
\pgfpathlineto{\pgfqpoint{3.596727in}{1.375568in}}%
\pgfpathlineto{\pgfqpoint{3.609340in}{1.393187in}}%
\pgfpathlineto{\pgfqpoint{3.621954in}{1.403330in}}%
\pgfpathlineto{\pgfqpoint{3.634567in}{1.410466in}}%
\pgfpathlineto{\pgfqpoint{3.647181in}{1.415422in}}%
\pgfpathlineto{\pgfqpoint{3.659794in}{1.430624in}}%
\pgfpathlineto{\pgfqpoint{3.672408in}{1.438465in}}%
\pgfpathlineto{\pgfqpoint{3.685022in}{1.452780in}}%
\pgfpathlineto{\pgfqpoint{3.697635in}{1.471976in}}%
\pgfpathlineto{\pgfqpoint{3.710249in}{1.475324in}}%
\pgfpathlineto{\pgfqpoint{3.722862in}{1.508552in}}%
\pgfpathlineto{\pgfqpoint{3.735476in}{1.533550in}}%
\pgfpathlineto{\pgfqpoint{3.748089in}{1.535110in}}%
\pgfpathlineto{\pgfqpoint{3.760703in}{1.546311in}}%
\pgfpathlineto{\pgfqpoint{3.773317in}{1.585433in}}%
\pgfpathlineto{\pgfqpoint{3.785930in}{1.632130in}}%
\pgfpathlineto{\pgfqpoint{3.798544in}{1.640820in}}%
\pgfpathlineto{\pgfqpoint{3.811157in}{1.664352in}}%
\pgfpathlineto{\pgfqpoint{3.823771in}{1.675124in}}%
\pgfpathlineto{\pgfqpoint{3.836384in}{1.689341in}}%
\pgfpathlineto{\pgfqpoint{3.848998in}{1.691679in}}%
\pgfpathlineto{\pgfqpoint{3.874225in}{1.708535in}}%
\pgfpathlineto{\pgfqpoint{3.886839in}{1.716772in}}%
\pgfpathlineto{\pgfqpoint{3.912066in}{1.786847in}}%
\pgfpathlineto{\pgfqpoint{3.924679in}{1.788251in}}%
\pgfpathlineto{\pgfqpoint{3.937293in}{1.842763in}}%
\pgfpathlineto{\pgfqpoint{3.949906in}{1.848558in}}%
\pgfpathlineto{\pgfqpoint{3.962520in}{1.869329in}}%
\pgfpathlineto{\pgfqpoint{3.975134in}{1.910590in}}%
\pgfpathlineto{\pgfqpoint{3.987747in}{1.946298in}}%
\pgfpathlineto{\pgfqpoint{4.012974in}{1.993094in}}%
\pgfpathlineto{\pgfqpoint{4.025588in}{2.002672in}}%
\pgfpathlineto{\pgfqpoint{4.038201in}{2.006356in}}%
\pgfpathlineto{\pgfqpoint{4.050815in}{2.021025in}}%
\pgfpathlineto{\pgfqpoint{4.063429in}{2.098041in}}%
\pgfpathlineto{\pgfqpoint{4.076042in}{2.106049in}}%
\pgfpathlineto{\pgfqpoint{4.088656in}{2.142626in}}%
\pgfpathlineto{\pgfqpoint{4.101269in}{2.152286in}}%
\pgfpathlineto{\pgfqpoint{4.113883in}{2.154419in}}%
\pgfpathlineto{\pgfqpoint{4.139110in}{2.167796in}}%
\pgfpathlineto{\pgfqpoint{4.151724in}{2.168054in}}%
\pgfpathlineto{\pgfqpoint{4.164337in}{2.174326in}}%
\pgfpathlineto{\pgfqpoint{4.176951in}{2.175430in}}%
\pgfpathlineto{\pgfqpoint{4.189564in}{2.184958in}}%
\pgfpathlineto{\pgfqpoint{4.202178in}{2.206324in}}%
\pgfpathlineto{\pgfqpoint{4.214791in}{2.206670in}}%
\pgfpathlineto{\pgfqpoint{4.227405in}{2.210631in}}%
\pgfpathlineto{\pgfqpoint{4.240018in}{2.212381in}}%
\pgfpathlineto{\pgfqpoint{4.265246in}{2.281125in}}%
\pgfpathlineto{\pgfqpoint{4.277859in}{2.305275in}}%
\pgfpathlineto{\pgfqpoint{4.277859in}{2.305275in}}%
\pgfusepath{stroke}%
\end{pgfscope}%
\begin{pgfscope}%
\pgfpathrectangle{\pgfqpoint{0.708220in}{0.535823in}}{\pgfqpoint{5.045427in}{1.769453in}}%
\pgfusepath{clip}%
\pgfsetbuttcap%
\pgfsetroundjoin%
\pgfsetlinewidth{1.003750pt}%
\definecolor{currentstroke}{rgb}{1.000000,0.498039,0.054902}%
\pgfsetstrokecolor{currentstroke}%
\pgfsetdash{{6.400000pt}{1.600000pt}{1.000000pt}{1.600000pt}}{0.000000pt}%
\pgfpathmoveto{\pgfqpoint{0.708220in}{1.179533in}}%
\pgfpathlineto{\pgfqpoint{0.733447in}{1.179533in}}%
\pgfpathlineto{\pgfqpoint{0.746061in}{1.180771in}}%
\pgfpathlineto{\pgfqpoint{0.771288in}{1.180771in}}%
\pgfpathlineto{\pgfqpoint{0.796515in}{1.183210in}}%
\pgfpathlineto{\pgfqpoint{0.834356in}{1.183210in}}%
\pgfpathlineto{\pgfqpoint{0.846969in}{1.184412in}}%
\pgfpathlineto{\pgfqpoint{0.859583in}{1.184412in}}%
\pgfpathlineto{\pgfqpoint{0.897423in}{1.187947in}}%
\pgfpathlineto{\pgfqpoint{0.910037in}{1.187947in}}%
\pgfpathlineto{\pgfqpoint{0.922650in}{1.189103in}}%
\pgfpathlineto{\pgfqpoint{0.973105in}{1.189103in}}%
\pgfpathlineto{\pgfqpoint{0.985718in}{1.190248in}}%
\pgfpathlineto{\pgfqpoint{1.010945in}{1.190248in}}%
\pgfpathlineto{\pgfqpoint{1.023559in}{1.191383in}}%
\pgfpathlineto{\pgfqpoint{1.048786in}{1.191383in}}%
\pgfpathlineto{\pgfqpoint{1.061400in}{1.192507in}}%
\pgfpathlineto{\pgfqpoint{1.099240in}{1.192507in}}%
\pgfpathlineto{\pgfqpoint{1.124468in}{1.193621in}}%
\pgfpathlineto{\pgfqpoint{1.200149in}{1.194724in}}%
\pgfpathlineto{\pgfqpoint{1.237990in}{1.196902in}}%
\pgfpathlineto{\pgfqpoint{1.351512in}{1.197977in}}%
\pgfpathlineto{\pgfqpoint{1.376739in}{1.199042in}}%
\pgfpathlineto{\pgfqpoint{1.439807in}{1.200098in}}%
\pgfpathlineto{\pgfqpoint{1.490261in}{1.203212in}}%
\pgfpathlineto{\pgfqpoint{1.528102in}{1.204233in}}%
\pgfpathlineto{\pgfqpoint{1.553329in}{1.205245in}}%
\pgfpathlineto{\pgfqpoint{1.603783in}{1.206248in}}%
\pgfpathlineto{\pgfqpoint{1.629010in}{1.207244in}}%
\pgfpathlineto{\pgfqpoint{1.692078in}{1.208232in}}%
\pgfpathlineto{\pgfqpoint{1.717305in}{1.209211in}}%
\pgfpathlineto{\pgfqpoint{1.767759in}{1.210183in}}%
\pgfpathlineto{\pgfqpoint{1.792987in}{1.211147in}}%
\pgfpathlineto{\pgfqpoint{1.856054in}{1.212104in}}%
\pgfpathlineto{\pgfqpoint{1.881282in}{1.213053in}}%
\pgfpathlineto{\pgfqpoint{1.919122in}{1.213994in}}%
\pgfpathlineto{\pgfqpoint{1.931736in}{1.215856in}}%
\pgfpathlineto{\pgfqpoint{1.982190in}{1.216777in}}%
\pgfpathlineto{\pgfqpoint{2.020031in}{1.218597in}}%
\pgfpathlineto{\pgfqpoint{2.032644in}{1.218597in}}%
\pgfpathlineto{\pgfqpoint{2.045258in}{1.220391in}}%
\pgfpathlineto{\pgfqpoint{2.133553in}{1.221278in}}%
\pgfpathlineto{\pgfqpoint{2.158780in}{1.222158in}}%
\pgfpathlineto{\pgfqpoint{2.209234in}{1.223033in}}%
\pgfpathlineto{\pgfqpoint{2.234461in}{1.223901in}}%
\pgfpathlineto{\pgfqpoint{2.259689in}{1.224762in}}%
\pgfpathlineto{\pgfqpoint{2.347984in}{1.231447in}}%
\pgfpathlineto{\pgfqpoint{2.373211in}{1.233862in}}%
\pgfpathlineto{\pgfqpoint{2.537187in}{1.245268in}}%
\pgfpathlineto{\pgfqpoint{2.549801in}{1.250942in}}%
\pgfpathlineto{\pgfqpoint{2.575028in}{1.251633in}}%
\pgfpathlineto{\pgfqpoint{2.600255in}{1.253682in}}%
\pgfpathlineto{\pgfqpoint{2.625482in}{1.254358in}}%
\pgfpathlineto{\pgfqpoint{2.663323in}{1.257024in}}%
\pgfpathlineto{\pgfqpoint{2.675936in}{1.258335in}}%
\pgfpathlineto{\pgfqpoint{2.688550in}{1.258335in}}%
\pgfpathlineto{\pgfqpoint{2.701163in}{1.260276in}}%
\pgfpathlineto{\pgfqpoint{2.726390in}{1.260917in}}%
\pgfpathlineto{\pgfqpoint{2.739004in}{1.263444in}}%
\pgfpathlineto{\pgfqpoint{2.877753in}{1.269543in}}%
\pgfpathlineto{\pgfqpoint{2.915594in}{1.270137in}}%
\pgfpathlineto{\pgfqpoint{2.966048in}{1.273062in}}%
\pgfpathlineto{\pgfqpoint{2.978662in}{1.277600in}}%
\pgfpathlineto{\pgfqpoint{2.991275in}{1.278156in}}%
\pgfpathlineto{\pgfqpoint{3.003889in}{1.280353in}}%
\pgfpathlineto{\pgfqpoint{3.029116in}{1.280897in}}%
\pgfpathlineto{\pgfqpoint{3.066957in}{1.283577in}}%
\pgfpathlineto{\pgfqpoint{3.079570in}{1.287234in}}%
\pgfpathlineto{\pgfqpoint{3.104797in}{1.288258in}}%
\pgfpathlineto{\pgfqpoint{3.117411in}{1.291779in}}%
\pgfpathlineto{\pgfqpoint{3.130025in}{1.298530in}}%
\pgfpathlineto{\pgfqpoint{3.142638in}{1.299464in}}%
\pgfpathlineto{\pgfqpoint{3.167865in}{1.303132in}}%
\pgfpathlineto{\pgfqpoint{3.180479in}{1.303132in}}%
\pgfpathlineto{\pgfqpoint{3.205706in}{1.308867in}}%
\pgfpathlineto{\pgfqpoint{3.218320in}{1.318795in}}%
\pgfpathlineto{\pgfqpoint{3.256160in}{1.320374in}}%
\pgfpathlineto{\pgfqpoint{3.268774in}{1.323471in}}%
\pgfpathlineto{\pgfqpoint{3.281387in}{1.329803in}}%
\pgfpathlineto{\pgfqpoint{3.306615in}{1.330887in}}%
\pgfpathlineto{\pgfqpoint{3.319228in}{1.332318in}}%
\pgfpathlineto{\pgfqpoint{3.331842in}{1.335477in}}%
\pgfpathlineto{\pgfqpoint{3.369682in}{1.337878in}}%
\pgfpathlineto{\pgfqpoint{3.382296in}{1.340898in}}%
\pgfpathlineto{\pgfqpoint{3.394910in}{1.341888in}}%
\pgfpathlineto{\pgfqpoint{3.407523in}{1.345771in}}%
\pgfpathlineto{\pgfqpoint{3.445364in}{1.348603in}}%
\pgfpathlineto{\pgfqpoint{3.457977in}{1.363520in}}%
\pgfpathlineto{\pgfqpoint{3.470591in}{1.368377in}}%
\pgfpathlineto{\pgfqpoint{3.483204in}{1.377547in}}%
\pgfpathlineto{\pgfqpoint{3.495818in}{1.378768in}}%
\pgfpathlineto{\pgfqpoint{3.508432in}{1.383299in}}%
\pgfpathlineto{\pgfqpoint{3.521045in}{1.386305in}}%
\pgfpathlineto{\pgfqpoint{3.533659in}{1.391228in}}%
\pgfpathlineto{\pgfqpoint{3.546272in}{1.392538in}}%
\pgfpathlineto{\pgfqpoint{3.558886in}{1.407431in}}%
\pgfpathlineto{\pgfqpoint{3.571499in}{1.460849in}}%
\pgfpathlineto{\pgfqpoint{3.584113in}{1.462938in}}%
\pgfpathlineto{\pgfqpoint{3.609340in}{1.464151in}}%
\pgfpathlineto{\pgfqpoint{3.621954in}{1.495282in}}%
\pgfpathlineto{\pgfqpoint{3.634567in}{1.506182in}}%
\pgfpathlineto{\pgfqpoint{3.647181in}{1.556058in}}%
\pgfpathlineto{\pgfqpoint{3.672408in}{1.562580in}}%
\pgfpathlineto{\pgfqpoint{3.685022in}{1.591823in}}%
\pgfpathlineto{\pgfqpoint{3.697635in}{1.614446in}}%
\pgfpathlineto{\pgfqpoint{3.710249in}{1.615183in}}%
\pgfpathlineto{\pgfqpoint{3.722862in}{1.638705in}}%
\pgfpathlineto{\pgfqpoint{3.735476in}{1.640877in}}%
\pgfpathlineto{\pgfqpoint{3.748089in}{1.640962in}}%
\pgfpathlineto{\pgfqpoint{3.760703in}{1.645108in}}%
\pgfpathlineto{\pgfqpoint{3.773317in}{1.670924in}}%
\pgfpathlineto{\pgfqpoint{3.785930in}{1.676316in}}%
\pgfpathlineto{\pgfqpoint{3.798544in}{1.685237in}}%
\pgfpathlineto{\pgfqpoint{3.811157in}{1.717892in}}%
\pgfpathlineto{\pgfqpoint{3.823771in}{1.720322in}}%
\pgfpathlineto{\pgfqpoint{3.836384in}{1.721571in}}%
\pgfpathlineto{\pgfqpoint{3.848998in}{1.812664in}}%
\pgfpathlineto{\pgfqpoint{3.861611in}{1.849606in}}%
\pgfpathlineto{\pgfqpoint{3.874225in}{1.850447in}}%
\pgfpathlineto{\pgfqpoint{3.886839in}{1.852617in}}%
\pgfpathlineto{\pgfqpoint{3.899452in}{1.863208in}}%
\pgfpathlineto{\pgfqpoint{3.912066in}{1.871656in}}%
\pgfpathlineto{\pgfqpoint{3.924679in}{1.874721in}}%
\pgfpathlineto{\pgfqpoint{3.937293in}{1.888700in}}%
\pgfpathlineto{\pgfqpoint{3.949906in}{1.898099in}}%
\pgfpathlineto{\pgfqpoint{3.962520in}{1.905720in}}%
\pgfpathlineto{\pgfqpoint{3.975134in}{1.918784in}}%
\pgfpathlineto{\pgfqpoint{3.987747in}{1.919455in}}%
\pgfpathlineto{\pgfqpoint{4.000361in}{1.945102in}}%
\pgfpathlineto{\pgfqpoint{4.012974in}{1.949119in}}%
\pgfpathlineto{\pgfqpoint{4.025588in}{1.954828in}}%
\pgfpathlineto{\pgfqpoint{4.038201in}{2.065928in}}%
\pgfpathlineto{\pgfqpoint{4.050815in}{2.069409in}}%
\pgfpathlineto{\pgfqpoint{4.063429in}{2.108985in}}%
\pgfpathlineto{\pgfqpoint{4.076042in}{2.134900in}}%
\pgfpathlineto{\pgfqpoint{4.088656in}{2.139175in}}%
\pgfpathlineto{\pgfqpoint{4.101269in}{2.182566in}}%
\pgfpathlineto{\pgfqpoint{4.113883in}{2.185487in}}%
\pgfpathlineto{\pgfqpoint{4.126496in}{2.185693in}}%
\pgfpathlineto{\pgfqpoint{4.151724in}{2.187505in}}%
\pgfpathlineto{\pgfqpoint{4.164337in}{2.212091in}}%
\pgfpathlineto{\pgfqpoint{4.176951in}{2.216907in}}%
\pgfpathlineto{\pgfqpoint{4.189564in}{2.237440in}}%
\pgfpathlineto{\pgfqpoint{4.202178in}{2.247054in}}%
\pgfpathlineto{\pgfqpoint{4.214791in}{2.247219in}}%
\pgfpathlineto{\pgfqpoint{4.227405in}{2.249812in}}%
\pgfpathlineto{\pgfqpoint{4.240018in}{2.269432in}}%
\pgfpathlineto{\pgfqpoint{4.252632in}{2.272099in}}%
\pgfpathlineto{\pgfqpoint{4.265246in}{2.283600in}}%
\pgfpathlineto{\pgfqpoint{4.277859in}{2.293428in}}%
\pgfpathlineto{\pgfqpoint{4.290473in}{2.305275in}}%
\pgfpathlineto{\pgfqpoint{4.290473in}{2.305275in}}%
\pgfusepath{stroke}%
\end{pgfscope}%
\begin{pgfscope}%
\pgfpathrectangle{\pgfqpoint{0.708220in}{0.535823in}}{\pgfqpoint{5.045427in}{1.769453in}}%
\pgfusepath{clip}%
\pgfsetbuttcap%
\pgfsetroundjoin%
\pgfsetlinewidth{1.003750pt}%
\definecolor{currentstroke}{rgb}{0.172549,0.627451,0.172549}%
\pgfsetstrokecolor{currentstroke}%
\pgfsetdash{{6.400000pt}{1.600000pt}{1.000000pt}{1.600000pt}}{0.000000pt}%
\pgfpathmoveto{\pgfqpoint{0.708220in}{1.210183in}}%
\pgfpathlineto{\pgfqpoint{0.758674in}{1.211147in}}%
\pgfpathlineto{\pgfqpoint{0.771288in}{1.213053in}}%
\pgfpathlineto{\pgfqpoint{0.809128in}{1.213994in}}%
\pgfpathlineto{\pgfqpoint{0.859583in}{1.216777in}}%
\pgfpathlineto{\pgfqpoint{0.960491in}{1.217690in}}%
\pgfpathlineto{\pgfqpoint{0.985718in}{1.220391in}}%
\pgfpathlineto{\pgfqpoint{1.010945in}{1.221278in}}%
\pgfpathlineto{\pgfqpoint{1.061400in}{1.222158in}}%
\pgfpathlineto{\pgfqpoint{1.086627in}{1.223033in}}%
\pgfpathlineto{\pgfqpoint{1.162308in}{1.223901in}}%
\pgfpathlineto{\pgfqpoint{1.200149in}{1.225618in}}%
\pgfpathlineto{\pgfqpoint{1.263217in}{1.226468in}}%
\pgfpathlineto{\pgfqpoint{1.313671in}{1.228983in}}%
\pgfpathlineto{\pgfqpoint{1.364125in}{1.229810in}}%
\pgfpathlineto{\pgfqpoint{1.389352in}{1.230631in}}%
\pgfpathlineto{\pgfqpoint{1.490261in}{1.231447in}}%
\pgfpathlineto{\pgfqpoint{1.502875in}{1.233062in}}%
\pgfpathlineto{\pgfqpoint{1.603783in}{1.233862in}}%
\pgfpathlineto{\pgfqpoint{1.629010in}{1.234656in}}%
\pgfpathlineto{\pgfqpoint{1.692078in}{1.235446in}}%
\pgfpathlineto{\pgfqpoint{1.704692in}{1.237009in}}%
\pgfpathlineto{\pgfqpoint{1.767759in}{1.237784in}}%
\pgfpathlineto{\pgfqpoint{1.792987in}{1.238553in}}%
\pgfpathlineto{\pgfqpoint{1.868668in}{1.239318in}}%
\pgfpathlineto{\pgfqpoint{1.893895in}{1.240078in}}%
\pgfpathlineto{\pgfqpoint{1.994804in}{1.240833in}}%
\pgfpathlineto{\pgfqpoint{2.095712in}{1.245992in}}%
\pgfpathlineto{\pgfqpoint{2.146166in}{1.246711in}}%
\pgfpathlineto{\pgfqpoint{2.221848in}{1.250247in}}%
\pgfpathlineto{\pgfqpoint{2.234461in}{1.250247in}}%
\pgfpathlineto{\pgfqpoint{2.272302in}{1.254358in}}%
\pgfpathlineto{\pgfqpoint{2.297529in}{1.255030in}}%
\pgfpathlineto{\pgfqpoint{2.347984in}{1.257024in}}%
\pgfpathlineto{\pgfqpoint{2.360597in}{1.257024in}}%
\pgfpathlineto{\pgfqpoint{2.373211in}{1.258986in}}%
\pgfpathlineto{\pgfqpoint{2.398438in}{1.258986in}}%
\pgfpathlineto{\pgfqpoint{2.411051in}{1.260276in}}%
\pgfpathlineto{\pgfqpoint{2.461506in}{1.260917in}}%
\pgfpathlineto{\pgfqpoint{2.486733in}{1.261553in}}%
\pgfpathlineto{\pgfqpoint{2.524573in}{1.262187in}}%
\pgfpathlineto{\pgfqpoint{2.562414in}{1.263444in}}%
\pgfpathlineto{\pgfqpoint{2.587641in}{1.264068in}}%
\pgfpathlineto{\pgfqpoint{2.638096in}{1.265921in}}%
\pgfpathlineto{\pgfqpoint{2.650709in}{1.265921in}}%
\pgfpathlineto{\pgfqpoint{2.675936in}{1.267746in}}%
\pgfpathlineto{\pgfqpoint{2.713777in}{1.268348in}}%
\pgfpathlineto{\pgfqpoint{2.726390in}{1.269543in}}%
\pgfpathlineto{\pgfqpoint{2.739004in}{1.269543in}}%
\pgfpathlineto{\pgfqpoint{2.776845in}{1.273062in}}%
\pgfpathlineto{\pgfqpoint{2.789458in}{1.273062in}}%
\pgfpathlineto{\pgfqpoint{2.802072in}{1.277042in}}%
\pgfpathlineto{\pgfqpoint{2.827299in}{1.277600in}}%
\pgfpathlineto{\pgfqpoint{2.865140in}{1.278709in}}%
\pgfpathlineto{\pgfqpoint{2.877753in}{1.281437in}}%
\pgfpathlineto{\pgfqpoint{2.915594in}{1.282512in}}%
\pgfpathlineto{\pgfqpoint{2.978662in}{1.287234in}}%
\pgfpathlineto{\pgfqpoint{3.003889in}{1.288258in}}%
\pgfpathlineto{\pgfqpoint{3.029116in}{1.291779in}}%
\pgfpathlineto{\pgfqpoint{3.054343in}{1.293257in}}%
\pgfpathlineto{\pgfqpoint{3.066957in}{1.295201in}}%
\pgfpathlineto{\pgfqpoint{3.092184in}{1.295682in}}%
\pgfpathlineto{\pgfqpoint{3.104797in}{1.303583in}}%
\pgfpathlineto{\pgfqpoint{3.117411in}{1.309298in}}%
\pgfpathlineto{\pgfqpoint{3.130025in}{1.313518in}}%
\pgfpathlineto{\pgfqpoint{3.142638in}{1.313518in}}%
\pgfpathlineto{\pgfqpoint{3.193092in}{1.319587in}}%
\pgfpathlineto{\pgfqpoint{3.205706in}{1.320374in}}%
\pgfpathlineto{\pgfqpoint{3.218320in}{1.326118in}}%
\pgfpathlineto{\pgfqpoint{3.281387in}{1.331962in}}%
\pgfpathlineto{\pgfqpoint{3.294001in}{1.334433in}}%
\pgfpathlineto{\pgfqpoint{3.306615in}{1.340898in}}%
\pgfpathlineto{\pgfqpoint{3.344455in}{1.343196in}}%
\pgfpathlineto{\pgfqpoint{3.357069in}{1.352281in}}%
\pgfpathlineto{\pgfqpoint{3.382296in}{1.354376in}}%
\pgfpathlineto{\pgfqpoint{3.407523in}{1.355850in}}%
\pgfpathlineto{\pgfqpoint{3.420137in}{1.363520in}}%
\pgfpathlineto{\pgfqpoint{3.445364in}{1.365160in}}%
\pgfpathlineto{\pgfqpoint{3.470591in}{1.368112in}}%
\pgfpathlineto{\pgfqpoint{3.495818in}{1.373048in}}%
\pgfpathlineto{\pgfqpoint{3.508432in}{1.385618in}}%
\pgfpathlineto{\pgfqpoint{3.521045in}{1.387441in}}%
\pgfpathlineto{\pgfqpoint{3.533659in}{1.388118in}}%
\pgfpathlineto{\pgfqpoint{3.546272in}{1.396593in}}%
\pgfpathlineto{\pgfqpoint{3.558886in}{1.397846in}}%
\pgfpathlineto{\pgfqpoint{3.571499in}{1.408578in}}%
\pgfpathlineto{\pgfqpoint{3.584113in}{1.417562in}}%
\pgfpathlineto{\pgfqpoint{3.596727in}{1.417739in}}%
\pgfpathlineto{\pgfqpoint{3.609340in}{1.432045in}}%
\pgfpathlineto{\pgfqpoint{3.621954in}{1.434222in}}%
\pgfpathlineto{\pgfqpoint{3.647181in}{1.443139in}}%
\pgfpathlineto{\pgfqpoint{3.672408in}{1.443996in}}%
\pgfpathlineto{\pgfqpoint{3.685022in}{1.449696in}}%
\pgfpathlineto{\pgfqpoint{3.697635in}{1.461344in}}%
\pgfpathlineto{\pgfqpoint{3.710249in}{1.468185in}}%
\pgfpathlineto{\pgfqpoint{3.722862in}{1.515238in}}%
\pgfpathlineto{\pgfqpoint{3.735476in}{1.530437in}}%
\pgfpathlineto{\pgfqpoint{3.748089in}{1.532934in}}%
\pgfpathlineto{\pgfqpoint{3.760703in}{1.579302in}}%
\pgfpathlineto{\pgfqpoint{3.773317in}{1.584672in}}%
\pgfpathlineto{\pgfqpoint{3.785930in}{1.619645in}}%
\pgfpathlineto{\pgfqpoint{3.798544in}{1.624664in}}%
\pgfpathlineto{\pgfqpoint{3.811157in}{1.632525in}}%
\pgfpathlineto{\pgfqpoint{3.823771in}{1.638330in}}%
\pgfpathlineto{\pgfqpoint{3.836384in}{1.638474in}}%
\pgfpathlineto{\pgfqpoint{3.848998in}{1.658441in}}%
\pgfpathlineto{\pgfqpoint{3.861611in}{1.663013in}}%
\pgfpathlineto{\pgfqpoint{3.874225in}{1.673292in}}%
\pgfpathlineto{\pgfqpoint{3.886839in}{1.675870in}}%
\pgfpathlineto{\pgfqpoint{3.899452in}{1.686569in}}%
\pgfpathlineto{\pgfqpoint{3.912066in}{1.686957in}}%
\pgfpathlineto{\pgfqpoint{3.924679in}{1.697744in}}%
\pgfpathlineto{\pgfqpoint{3.937293in}{1.706153in}}%
\pgfpathlineto{\pgfqpoint{3.949906in}{1.779684in}}%
\pgfpathlineto{\pgfqpoint{3.962520in}{1.785592in}}%
\pgfpathlineto{\pgfqpoint{3.975134in}{1.865537in}}%
\pgfpathlineto{\pgfqpoint{3.987747in}{1.901472in}}%
\pgfpathlineto{\pgfqpoint{4.000361in}{1.913208in}}%
\pgfpathlineto{\pgfqpoint{4.012974in}{1.922872in}}%
\pgfpathlineto{\pgfqpoint{4.025588in}{1.952685in}}%
\pgfpathlineto{\pgfqpoint{4.038201in}{1.969594in}}%
\pgfpathlineto{\pgfqpoint{4.050815in}{1.971014in}}%
\pgfpathlineto{\pgfqpoint{4.063429in}{1.995957in}}%
\pgfpathlineto{\pgfqpoint{4.076042in}{2.067370in}}%
\pgfpathlineto{\pgfqpoint{4.088656in}{2.068699in}}%
\pgfpathlineto{\pgfqpoint{4.101269in}{2.099816in}}%
\pgfpathlineto{\pgfqpoint{4.113883in}{2.104935in}}%
\pgfpathlineto{\pgfqpoint{4.126496in}{2.116029in}}%
\pgfpathlineto{\pgfqpoint{4.139110in}{2.156021in}}%
\pgfpathlineto{\pgfqpoint{4.151724in}{2.168623in}}%
\pgfpathlineto{\pgfqpoint{4.176951in}{2.175082in}}%
\pgfpathlineto{\pgfqpoint{4.189564in}{2.197260in}}%
\pgfpathlineto{\pgfqpoint{4.202178in}{2.208581in}}%
\pgfpathlineto{\pgfqpoint{4.214791in}{2.269455in}}%
\pgfpathlineto{\pgfqpoint{4.227405in}{2.274202in}}%
\pgfpathlineto{\pgfqpoint{4.240018in}{2.284535in}}%
\pgfpathlineto{\pgfqpoint{4.252632in}{2.287747in}}%
\pgfpathlineto{\pgfqpoint{4.265246in}{2.303369in}}%
\pgfpathlineto{\pgfqpoint{4.277859in}{2.305275in}}%
\pgfpathlineto{\pgfqpoint{4.277859in}{2.305275in}}%
\pgfusepath{stroke}%
\end{pgfscope}%
\begin{pgfscope}%
\pgfpathrectangle{\pgfqpoint{0.708220in}{0.535823in}}{\pgfqpoint{5.045427in}{1.769453in}}%
\pgfusepath{clip}%
\pgfsetbuttcap%
\pgfsetroundjoin%
\pgfsetlinewidth{1.003750pt}%
\definecolor{currentstroke}{rgb}{0.839216,0.152941,0.156863}%
\pgfsetstrokecolor{currentstroke}%
\pgfsetdash{{6.400000pt}{1.600000pt}{1.000000pt}{1.600000pt}}{0.000000pt}%
\pgfpathmoveto{\pgfqpoint{0.708220in}{1.208232in}}%
\pgfpathlineto{\pgfqpoint{0.758674in}{1.211147in}}%
\pgfpathlineto{\pgfqpoint{0.809128in}{1.212104in}}%
\pgfpathlineto{\pgfqpoint{0.821742in}{1.214929in}}%
\pgfpathlineto{\pgfqpoint{0.846969in}{1.215856in}}%
\pgfpathlineto{\pgfqpoint{0.872196in}{1.219497in}}%
\pgfpathlineto{\pgfqpoint{0.910037in}{1.220391in}}%
\pgfpathlineto{\pgfqpoint{0.960491in}{1.223033in}}%
\pgfpathlineto{\pgfqpoint{1.036173in}{1.223901in}}%
\pgfpathlineto{\pgfqpoint{1.074013in}{1.225618in}}%
\pgfpathlineto{\pgfqpoint{1.162308in}{1.226468in}}%
\pgfpathlineto{\pgfqpoint{1.187535in}{1.227312in}}%
\pgfpathlineto{\pgfqpoint{1.212763in}{1.228150in}}%
\pgfpathlineto{\pgfqpoint{1.237990in}{1.228983in}}%
\pgfpathlineto{\pgfqpoint{1.275830in}{1.229810in}}%
\pgfpathlineto{\pgfqpoint{1.301057in}{1.230631in}}%
\pgfpathlineto{\pgfqpoint{1.326285in}{1.231447in}}%
\pgfpathlineto{\pgfqpoint{1.351512in}{1.232257in}}%
\pgfpathlineto{\pgfqpoint{1.427193in}{1.233062in}}%
\pgfpathlineto{\pgfqpoint{1.452420in}{1.233862in}}%
\pgfpathlineto{\pgfqpoint{1.490261in}{1.234656in}}%
\pgfpathlineto{\pgfqpoint{1.515488in}{1.235446in}}%
\pgfpathlineto{\pgfqpoint{1.578556in}{1.236230in}}%
\pgfpathlineto{\pgfqpoint{1.616397in}{1.237784in}}%
\pgfpathlineto{\pgfqpoint{1.704692in}{1.238553in}}%
\pgfpathlineto{\pgfqpoint{1.742532in}{1.240078in}}%
\pgfpathlineto{\pgfqpoint{1.755146in}{1.240078in}}%
\pgfpathlineto{\pgfqpoint{1.767759in}{1.241583in}}%
\pgfpathlineto{\pgfqpoint{1.805600in}{1.242329in}}%
\pgfpathlineto{\pgfqpoint{1.818214in}{1.244540in}}%
\pgfpathlineto{\pgfqpoint{1.830827in}{1.245268in}}%
\pgfpathlineto{\pgfqpoint{1.843441in}{1.247427in}}%
\pgfpathlineto{\pgfqpoint{1.931736in}{1.251633in}}%
\pgfpathlineto{\pgfqpoint{1.956963in}{1.252320in}}%
\pgfpathlineto{\pgfqpoint{1.982190in}{1.253003in}}%
\pgfpathlineto{\pgfqpoint{2.007417in}{1.253682in}}%
\pgfpathlineto{\pgfqpoint{2.209234in}{1.263444in}}%
\pgfpathlineto{\pgfqpoint{2.247075in}{1.264068in}}%
\pgfpathlineto{\pgfqpoint{2.259689in}{1.265921in}}%
\pgfpathlineto{\pgfqpoint{2.297529in}{1.266532in}}%
\pgfpathlineto{\pgfqpoint{2.360597in}{1.268947in}}%
\pgfpathlineto{\pgfqpoint{2.411051in}{1.269543in}}%
\pgfpathlineto{\pgfqpoint{2.486733in}{1.272482in}}%
\pgfpathlineto{\pgfqpoint{2.575028in}{1.273062in}}%
\pgfpathlineto{\pgfqpoint{2.600255in}{1.274212in}}%
\pgfpathlineto{\pgfqpoint{2.612868in}{1.275918in}}%
\pgfpathlineto{\pgfqpoint{2.663323in}{1.279260in}}%
\pgfpathlineto{\pgfqpoint{2.739004in}{1.285158in}}%
\pgfpathlineto{\pgfqpoint{2.814685in}{1.288258in}}%
\pgfpathlineto{\pgfqpoint{2.852526in}{1.293257in}}%
\pgfpathlineto{\pgfqpoint{2.890367in}{1.298060in}}%
\pgfpathlineto{\pgfqpoint{2.915594in}{1.298998in}}%
\pgfpathlineto{\pgfqpoint{2.928208in}{1.304480in}}%
\pgfpathlineto{\pgfqpoint{2.940821in}{1.306254in}}%
\pgfpathlineto{\pgfqpoint{2.953435in}{1.310579in}}%
\pgfpathlineto{\pgfqpoint{3.003889in}{1.312686in}}%
\pgfpathlineto{\pgfqpoint{3.016503in}{1.314345in}}%
\pgfpathlineto{\pgfqpoint{3.079570in}{1.317195in}}%
\pgfpathlineto{\pgfqpoint{3.092184in}{1.318795in}}%
\pgfpathlineto{\pgfqpoint{3.104797in}{1.318795in}}%
\pgfpathlineto{\pgfqpoint{3.117411in}{1.320765in}}%
\pgfpathlineto{\pgfqpoint{3.167865in}{1.324991in}}%
\pgfpathlineto{\pgfqpoint{3.180479in}{1.326492in}}%
\pgfpathlineto{\pgfqpoint{3.193092in}{1.332318in}}%
\pgfpathlineto{\pgfqpoint{3.205706in}{1.348914in}}%
\pgfpathlineto{\pgfqpoint{3.243547in}{1.354376in}}%
\pgfpathlineto{\pgfqpoint{3.256160in}{1.361017in}}%
\pgfpathlineto{\pgfqpoint{3.281387in}{1.361577in}}%
\pgfpathlineto{\pgfqpoint{3.294001in}{1.368112in}}%
\pgfpathlineto{\pgfqpoint{3.306615in}{1.377302in}}%
\pgfpathlineto{\pgfqpoint{3.319228in}{1.378281in}}%
\pgfpathlineto{\pgfqpoint{3.344455in}{1.388567in}}%
\pgfpathlineto{\pgfqpoint{3.357069in}{1.388791in}}%
\pgfpathlineto{\pgfqpoint{3.369682in}{1.391009in}}%
\pgfpathlineto{\pgfqpoint{3.382296in}{1.416140in}}%
\pgfpathlineto{\pgfqpoint{3.394910in}{1.425599in}}%
\pgfpathlineto{\pgfqpoint{3.407523in}{1.441843in}}%
\pgfpathlineto{\pgfqpoint{3.420137in}{1.447917in}}%
\pgfpathlineto{\pgfqpoint{3.445364in}{1.468068in}}%
\pgfpathlineto{\pgfqpoint{3.457977in}{1.497505in}}%
\pgfpathlineto{\pgfqpoint{3.470591in}{1.503678in}}%
\pgfpathlineto{\pgfqpoint{3.483204in}{1.524921in}}%
\pgfpathlineto{\pgfqpoint{3.495818in}{1.525797in}}%
\pgfpathlineto{\pgfqpoint{3.521045in}{1.530647in}}%
\pgfpathlineto{\pgfqpoint{3.546272in}{1.548565in}}%
\pgfpathlineto{\pgfqpoint{3.571499in}{1.551075in}}%
\pgfpathlineto{\pgfqpoint{3.584113in}{1.555660in}}%
\pgfpathlineto{\pgfqpoint{3.609340in}{1.557526in}}%
\pgfpathlineto{\pgfqpoint{3.621954in}{1.571052in}}%
\pgfpathlineto{\pgfqpoint{3.634567in}{1.574175in}}%
\pgfpathlineto{\pgfqpoint{3.647181in}{1.604291in}}%
\pgfpathlineto{\pgfqpoint{3.659794in}{1.610004in}}%
\pgfpathlineto{\pgfqpoint{3.672408in}{1.619173in}}%
\pgfpathlineto{\pgfqpoint{3.685022in}{1.623690in}}%
\pgfpathlineto{\pgfqpoint{3.710249in}{1.627099in}}%
\pgfpathlineto{\pgfqpoint{3.722862in}{1.735635in}}%
\pgfpathlineto{\pgfqpoint{3.735476in}{1.759209in}}%
\pgfpathlineto{\pgfqpoint{3.748089in}{1.767337in}}%
\pgfpathlineto{\pgfqpoint{3.773317in}{1.798415in}}%
\pgfpathlineto{\pgfqpoint{3.785930in}{1.801398in}}%
\pgfpathlineto{\pgfqpoint{3.798544in}{1.816240in}}%
\pgfpathlineto{\pgfqpoint{3.823771in}{1.854632in}}%
\pgfpathlineto{\pgfqpoint{3.836384in}{1.858387in}}%
\pgfpathlineto{\pgfqpoint{3.848998in}{1.879597in}}%
\pgfpathlineto{\pgfqpoint{3.861611in}{1.883222in}}%
\pgfpathlineto{\pgfqpoint{3.874225in}{1.916059in}}%
\pgfpathlineto{\pgfqpoint{3.886839in}{1.916766in}}%
\pgfpathlineto{\pgfqpoint{3.912066in}{1.932424in}}%
\pgfpathlineto{\pgfqpoint{3.924679in}{1.933837in}}%
\pgfpathlineto{\pgfqpoint{3.937293in}{2.009996in}}%
\pgfpathlineto{\pgfqpoint{3.949906in}{2.012147in}}%
\pgfpathlineto{\pgfqpoint{3.975134in}{2.013790in}}%
\pgfpathlineto{\pgfqpoint{3.987747in}{2.026688in}}%
\pgfpathlineto{\pgfqpoint{4.000361in}{2.055701in}}%
\pgfpathlineto{\pgfqpoint{4.025588in}{2.058238in}}%
\pgfpathlineto{\pgfqpoint{4.038201in}{2.059673in}}%
\pgfpathlineto{\pgfqpoint{4.063429in}{2.065637in}}%
\pgfpathlineto{\pgfqpoint{4.076042in}{2.086302in}}%
\pgfpathlineto{\pgfqpoint{4.088656in}{2.099460in}}%
\pgfpathlineto{\pgfqpoint{4.139110in}{2.102731in}}%
\pgfpathlineto{\pgfqpoint{4.151724in}{2.107888in}}%
\pgfpathlineto{\pgfqpoint{4.164337in}{2.116061in}}%
\pgfpathlineto{\pgfqpoint{4.202178in}{2.180818in}}%
\pgfpathlineto{\pgfqpoint{4.214791in}{2.186078in}}%
\pgfpathlineto{\pgfqpoint{4.227405in}{2.208213in}}%
\pgfpathlineto{\pgfqpoint{4.240018in}{2.223616in}}%
\pgfpathlineto{\pgfqpoint{4.252632in}{2.253959in}}%
\pgfpathlineto{\pgfqpoint{4.265246in}{2.258077in}}%
\pgfpathlineto{\pgfqpoint{4.277859in}{2.268515in}}%
\pgfpathlineto{\pgfqpoint{4.290473in}{2.305275in}}%
\pgfpathlineto{\pgfqpoint{4.290473in}{2.305275in}}%
\pgfusepath{stroke}%
\end{pgfscope}%
\begin{pgfscope}%
\pgfpathrectangle{\pgfqpoint{0.708220in}{0.535823in}}{\pgfqpoint{5.045427in}{1.769453in}}%
\pgfusepath{clip}%
\pgfsetbuttcap%
\pgfsetroundjoin%
\pgfsetlinewidth{1.003750pt}%
\definecolor{currentstroke}{rgb}{0.580392,0.403922,0.741176}%
\pgfsetstrokecolor{currentstroke}%
\pgfsetdash{{6.400000pt}{1.600000pt}{1.000000pt}{1.600000pt}}{0.000000pt}%
\pgfpathmoveto{\pgfqpoint{0.708220in}{1.370215in}}%
\pgfpathlineto{\pgfqpoint{0.783901in}{1.373810in}}%
\pgfpathlineto{\pgfqpoint{0.809128in}{1.375817in}}%
\pgfpathlineto{\pgfqpoint{0.859583in}{1.379736in}}%
\pgfpathlineto{\pgfqpoint{0.897423in}{1.381174in}}%
\pgfpathlineto{\pgfqpoint{0.922650in}{1.381412in}}%
\pgfpathlineto{\pgfqpoint{0.947878in}{1.385157in}}%
\pgfpathlineto{\pgfqpoint{0.960491in}{1.385388in}}%
\pgfpathlineto{\pgfqpoint{0.973105in}{1.387667in}}%
\pgfpathlineto{\pgfqpoint{0.998332in}{1.388118in}}%
\pgfpathlineto{\pgfqpoint{1.036173in}{1.392103in}}%
\pgfpathlineto{\pgfqpoint{1.061400in}{1.409148in}}%
\pgfpathlineto{\pgfqpoint{1.086627in}{1.409715in}}%
\pgfpathlineto{\pgfqpoint{1.237990in}{1.422243in}}%
\pgfpathlineto{\pgfqpoint{1.275830in}{1.424100in}}%
\pgfpathlineto{\pgfqpoint{1.288444in}{1.427569in}}%
\pgfpathlineto{\pgfqpoint{1.364125in}{1.432358in}}%
\pgfpathlineto{\pgfqpoint{1.414580in}{1.433604in}}%
\pgfpathlineto{\pgfqpoint{1.427193in}{1.435602in}}%
\pgfpathlineto{\pgfqpoint{1.515488in}{1.440532in}}%
\pgfpathlineto{\pgfqpoint{1.528102in}{1.442276in}}%
\pgfpathlineto{\pgfqpoint{1.565942in}{1.443426in}}%
\pgfpathlineto{\pgfqpoint{1.578556in}{1.445972in}}%
\pgfpathlineto{\pgfqpoint{1.591170in}{1.446531in}}%
\pgfpathlineto{\pgfqpoint{1.603783in}{1.448878in}}%
\pgfpathlineto{\pgfqpoint{1.641624in}{1.450777in}}%
\pgfpathlineto{\pgfqpoint{1.666851in}{1.453176in}}%
\pgfpathlineto{\pgfqpoint{1.679464in}{1.456046in}}%
\pgfpathlineto{\pgfqpoint{1.717305in}{1.458471in}}%
\pgfpathlineto{\pgfqpoint{1.729919in}{1.461221in}}%
\pgfpathlineto{\pgfqpoint{1.742532in}{1.462816in}}%
\pgfpathlineto{\pgfqpoint{1.780373in}{1.464392in}}%
\pgfpathlineto{\pgfqpoint{1.792987in}{1.465828in}}%
\pgfpathlineto{\pgfqpoint{1.881282in}{1.468418in}}%
\pgfpathlineto{\pgfqpoint{1.906509in}{1.470954in}}%
\pgfpathlineto{\pgfqpoint{1.969577in}{1.471863in}}%
\pgfpathlineto{\pgfqpoint{1.994804in}{1.473215in}}%
\pgfpathlineto{\pgfqpoint{2.057871in}{1.476747in}}%
\pgfpathlineto{\pgfqpoint{2.070485in}{1.480496in}}%
\pgfpathlineto{\pgfqpoint{2.120939in}{1.483002in}}%
\pgfpathlineto{\pgfqpoint{2.184007in}{1.489346in}}%
\pgfpathlineto{\pgfqpoint{2.360597in}{1.495469in}}%
\pgfpathlineto{\pgfqpoint{2.385824in}{1.502626in}}%
\pgfpathlineto{\pgfqpoint{2.411051in}{1.504979in}}%
\pgfpathlineto{\pgfqpoint{2.423665in}{1.510134in}}%
\pgfpathlineto{\pgfqpoint{2.511960in}{1.514841in}}%
\pgfpathlineto{\pgfqpoint{2.537187in}{1.519066in}}%
\pgfpathlineto{\pgfqpoint{2.549801in}{1.520746in}}%
\pgfpathlineto{\pgfqpoint{2.562414in}{1.520821in}}%
\pgfpathlineto{\pgfqpoint{2.612868in}{1.525870in}}%
\pgfpathlineto{\pgfqpoint{2.625482in}{1.528957in}}%
\pgfpathlineto{\pgfqpoint{2.638096in}{1.530157in}}%
\pgfpathlineto{\pgfqpoint{2.650709in}{1.530227in}}%
\pgfpathlineto{\pgfqpoint{2.663323in}{1.532176in}}%
\pgfpathlineto{\pgfqpoint{2.688550in}{1.532728in}}%
\pgfpathlineto{\pgfqpoint{2.701163in}{1.534908in}}%
\pgfpathlineto{\pgfqpoint{2.739004in}{1.535581in}}%
\pgfpathlineto{\pgfqpoint{2.751618in}{1.541800in}}%
\pgfpathlineto{\pgfqpoint{2.764231in}{1.552194in}}%
\pgfpathlineto{\pgfqpoint{2.776845in}{1.553302in}}%
\pgfpathlineto{\pgfqpoint{2.802072in}{1.556907in}}%
\pgfpathlineto{\pgfqpoint{2.852526in}{1.562634in}}%
\pgfpathlineto{\pgfqpoint{2.865140in}{1.566656in}}%
\pgfpathlineto{\pgfqpoint{2.902980in}{1.568156in}}%
\pgfpathlineto{\pgfqpoint{2.915594in}{1.572201in}}%
\pgfpathlineto{\pgfqpoint{2.928208in}{1.572747in}}%
\pgfpathlineto{\pgfqpoint{2.940821in}{1.574566in}}%
\pgfpathlineto{\pgfqpoint{2.978662in}{1.576742in}}%
\pgfpathlineto{\pgfqpoint{3.029116in}{1.578218in}}%
\pgfpathlineto{\pgfqpoint{3.054343in}{1.579957in}}%
\pgfpathlineto{\pgfqpoint{3.066957in}{1.580702in}}%
\pgfpathlineto{\pgfqpoint{3.079570in}{1.584178in}}%
\pgfpathlineto{\pgfqpoint{3.130025in}{1.588431in}}%
\pgfpathlineto{\pgfqpoint{3.142638in}{1.591441in}}%
\pgfpathlineto{\pgfqpoint{3.167865in}{1.592161in}}%
\pgfpathlineto{\pgfqpoint{3.193092in}{1.595576in}}%
\pgfpathlineto{\pgfqpoint{3.205706in}{1.600840in}}%
\pgfpathlineto{\pgfqpoint{3.230933in}{1.601428in}}%
\pgfpathlineto{\pgfqpoint{3.243547in}{1.603908in}}%
\pgfpathlineto{\pgfqpoint{3.268774in}{1.610659in}}%
\pgfpathlineto{\pgfqpoint{3.306615in}{1.613705in}}%
\pgfpathlineto{\pgfqpoint{3.319228in}{1.623657in}}%
\pgfpathlineto{\pgfqpoint{3.331842in}{1.624761in}}%
\pgfpathlineto{\pgfqpoint{3.369682in}{1.642871in}}%
\pgfpathlineto{\pgfqpoint{3.382296in}{1.643870in}}%
\pgfpathlineto{\pgfqpoint{3.394910in}{1.649408in}}%
\pgfpathlineto{\pgfqpoint{3.407523in}{1.656363in}}%
\pgfpathlineto{\pgfqpoint{3.457977in}{1.660580in}}%
\pgfpathlineto{\pgfqpoint{3.470591in}{1.660748in}}%
\pgfpathlineto{\pgfqpoint{3.495818in}{1.663908in}}%
\pgfpathlineto{\pgfqpoint{3.508432in}{1.666345in}}%
\pgfpathlineto{\pgfqpoint{3.521045in}{1.667056in}}%
\pgfpathlineto{\pgfqpoint{3.533659in}{1.670102in}}%
\pgfpathlineto{\pgfqpoint{3.558886in}{1.671587in}}%
\pgfpathlineto{\pgfqpoint{3.571499in}{1.674760in}}%
\pgfpathlineto{\pgfqpoint{3.584113in}{1.676527in}}%
\pgfpathlineto{\pgfqpoint{3.596727in}{1.680848in}}%
\pgfpathlineto{\pgfqpoint{3.609340in}{1.702893in}}%
\pgfpathlineto{\pgfqpoint{3.621954in}{1.716742in}}%
\pgfpathlineto{\pgfqpoint{3.634567in}{1.716772in}}%
\pgfpathlineto{\pgfqpoint{3.659794in}{1.722314in}}%
\pgfpathlineto{\pgfqpoint{3.672408in}{1.727615in}}%
\pgfpathlineto{\pgfqpoint{3.685022in}{1.751796in}}%
\pgfpathlineto{\pgfqpoint{3.697635in}{1.773026in}}%
\pgfpathlineto{\pgfqpoint{3.710249in}{1.775887in}}%
\pgfpathlineto{\pgfqpoint{3.722862in}{1.775990in}}%
\pgfpathlineto{\pgfqpoint{3.735476in}{1.778389in}}%
\pgfpathlineto{\pgfqpoint{3.748089in}{1.788378in}}%
\pgfpathlineto{\pgfqpoint{3.760703in}{1.817444in}}%
\pgfpathlineto{\pgfqpoint{3.773317in}{1.821369in}}%
\pgfpathlineto{\pgfqpoint{3.785930in}{1.832042in}}%
\pgfpathlineto{\pgfqpoint{3.798544in}{1.876494in}}%
\pgfpathlineto{\pgfqpoint{3.823771in}{1.897012in}}%
\pgfpathlineto{\pgfqpoint{3.836384in}{1.902265in}}%
\pgfpathlineto{\pgfqpoint{3.861611in}{1.925456in}}%
\pgfpathlineto{\pgfqpoint{3.874225in}{1.959041in}}%
\pgfpathlineto{\pgfqpoint{3.886839in}{1.965096in}}%
\pgfpathlineto{\pgfqpoint{3.899452in}{1.997922in}}%
\pgfpathlineto{\pgfqpoint{3.912066in}{2.079696in}}%
\pgfpathlineto{\pgfqpoint{3.924679in}{2.089409in}}%
\pgfpathlineto{\pgfqpoint{3.937293in}{2.097534in}}%
\pgfpathlineto{\pgfqpoint{3.949906in}{2.112552in}}%
\pgfpathlineto{\pgfqpoint{3.962520in}{2.124502in}}%
\pgfpathlineto{\pgfqpoint{3.975134in}{2.128606in}}%
\pgfpathlineto{\pgfqpoint{3.987747in}{2.161637in}}%
\pgfpathlineto{\pgfqpoint{4.000361in}{2.166166in}}%
\pgfpathlineto{\pgfqpoint{4.025588in}{2.170200in}}%
\pgfpathlineto{\pgfqpoint{4.038201in}{2.173836in}}%
\pgfpathlineto{\pgfqpoint{4.050815in}{2.191333in}}%
\pgfpathlineto{\pgfqpoint{4.076042in}{2.278083in}}%
\pgfpathlineto{\pgfqpoint{4.088656in}{2.278171in}}%
\pgfpathlineto{\pgfqpoint{4.101269in}{2.285507in}}%
\pgfpathlineto{\pgfqpoint{4.113883in}{2.305275in}}%
\pgfpathlineto{\pgfqpoint{4.113883in}{2.305275in}}%
\pgfusepath{stroke}%
\end{pgfscope}%
\begin{pgfscope}%
\pgfpathrectangle{\pgfqpoint{0.708220in}{0.535823in}}{\pgfqpoint{5.045427in}{1.769453in}}%
\pgfusepath{clip}%
\pgfsetbuttcap%
\pgfsetroundjoin%
\pgfsetlinewidth{1.003750pt}%
\definecolor{currentstroke}{rgb}{0.549020,0.337255,0.294118}%
\pgfsetstrokecolor{currentstroke}%
\pgfsetdash{{6.400000pt}{1.600000pt}{1.000000pt}{1.600000pt}}{0.000000pt}%
\pgfpathmoveto{\pgfqpoint{0.708220in}{1.372282in}}%
\pgfpathlineto{\pgfqpoint{0.771288in}{1.373303in}}%
\pgfpathlineto{\pgfqpoint{0.834356in}{1.379011in}}%
\pgfpathlineto{\pgfqpoint{0.897423in}{1.381886in}}%
\pgfpathlineto{\pgfqpoint{0.910037in}{1.383065in}}%
\pgfpathlineto{\pgfqpoint{0.935264in}{1.383299in}}%
\pgfpathlineto{\pgfqpoint{0.947878in}{1.385388in}}%
\pgfpathlineto{\pgfqpoint{0.985718in}{1.389015in}}%
\pgfpathlineto{\pgfqpoint{1.023559in}{1.394902in}}%
\pgfpathlineto{\pgfqpoint{1.036173in}{1.394902in}}%
\pgfpathlineto{\pgfqpoint{1.048786in}{1.398880in}}%
\pgfpathlineto{\pgfqpoint{1.086627in}{1.402932in}}%
\pgfpathlineto{\pgfqpoint{1.099240in}{1.411956in}}%
\pgfpathlineto{\pgfqpoint{1.162308in}{1.413061in}}%
\pgfpathlineto{\pgfqpoint{1.263217in}{1.422243in}}%
\pgfpathlineto{\pgfqpoint{1.439807in}{1.429987in}}%
\pgfpathlineto{\pgfqpoint{1.452420in}{1.431257in}}%
\pgfpathlineto{\pgfqpoint{1.465034in}{1.434530in}}%
\pgfpathlineto{\pgfqpoint{1.490261in}{1.435450in}}%
\pgfpathlineto{\pgfqpoint{1.528102in}{1.435907in}}%
\pgfpathlineto{\pgfqpoint{1.540715in}{1.437418in}}%
\pgfpathlineto{\pgfqpoint{1.578556in}{1.438167in}}%
\pgfpathlineto{\pgfqpoint{1.629010in}{1.439355in}}%
\pgfpathlineto{\pgfqpoint{1.641624in}{1.442565in}}%
\pgfpathlineto{\pgfqpoint{1.679464in}{1.444138in}}%
\pgfpathlineto{\pgfqpoint{1.692078in}{1.449015in}}%
\pgfpathlineto{\pgfqpoint{1.704692in}{1.449015in}}%
\pgfpathlineto{\pgfqpoint{1.717305in}{1.452116in}}%
\pgfpathlineto{\pgfqpoint{1.742532in}{1.454097in}}%
\pgfpathlineto{\pgfqpoint{1.755146in}{1.455788in}}%
\pgfpathlineto{\pgfqpoint{1.767759in}{1.459227in}}%
\pgfpathlineto{\pgfqpoint{1.780373in}{1.460849in}}%
\pgfpathlineto{\pgfqpoint{1.792987in}{1.465828in}}%
\pgfpathlineto{\pgfqpoint{1.982190in}{1.468418in}}%
\pgfpathlineto{\pgfqpoint{1.994804in}{1.470038in}}%
\pgfpathlineto{\pgfqpoint{2.032644in}{1.470840in}}%
\pgfpathlineto{\pgfqpoint{2.057871in}{1.473327in}}%
\pgfpathlineto{\pgfqpoint{2.095712in}{1.474662in}}%
\pgfpathlineto{\pgfqpoint{2.120939in}{1.477722in}}%
\pgfpathlineto{\pgfqpoint{2.133553in}{1.482173in}}%
\pgfpathlineto{\pgfqpoint{2.184007in}{1.483518in}}%
\pgfpathlineto{\pgfqpoint{2.221848in}{1.485963in}}%
\pgfpathlineto{\pgfqpoint{2.310143in}{1.491389in}}%
\pgfpathlineto{\pgfqpoint{2.373211in}{1.493114in}}%
\pgfpathlineto{\pgfqpoint{2.461506in}{1.502802in}}%
\pgfpathlineto{\pgfqpoint{2.486733in}{1.503852in}}%
\pgfpathlineto{\pgfqpoint{2.499346in}{1.504807in}}%
\pgfpathlineto{\pgfqpoint{2.511960in}{1.510217in}}%
\pgfpathlineto{\pgfqpoint{2.575028in}{1.515476in}}%
\pgfpathlineto{\pgfqpoint{2.587641in}{1.515950in}}%
\pgfpathlineto{\pgfqpoint{2.650709in}{1.525360in}}%
\pgfpathlineto{\pgfqpoint{2.688550in}{1.526667in}}%
\pgfpathlineto{\pgfqpoint{2.713777in}{1.530087in}}%
\pgfpathlineto{\pgfqpoint{2.739004in}{1.530717in}}%
\pgfpathlineto{\pgfqpoint{2.751618in}{1.533481in}}%
\pgfpathlineto{\pgfqpoint{2.802072in}{1.534975in}}%
\pgfpathlineto{\pgfqpoint{2.814685in}{1.536917in}}%
\pgfpathlineto{\pgfqpoint{2.839913in}{1.539220in}}%
\pgfpathlineto{\pgfqpoint{2.852526in}{1.550957in}}%
\pgfpathlineto{\pgfqpoint{2.865140in}{1.557357in}}%
\pgfpathlineto{\pgfqpoint{2.877753in}{1.559694in}}%
\pgfpathlineto{\pgfqpoint{2.890367in}{1.559969in}}%
\pgfpathlineto{\pgfqpoint{2.915594in}{1.562149in}}%
\pgfpathlineto{\pgfqpoint{2.928208in}{1.566082in}}%
\pgfpathlineto{\pgfqpoint{2.966048in}{1.566812in}}%
\pgfpathlineto{\pgfqpoint{2.978662in}{1.567641in}}%
\pgfpathlineto{\pgfqpoint{2.991275in}{1.570246in}}%
\pgfpathlineto{\pgfqpoint{3.003889in}{1.570246in}}%
\pgfpathlineto{\pgfqpoint{3.029116in}{1.574566in}}%
\pgfpathlineto{\pgfqpoint{3.079570in}{1.578076in}}%
\pgfpathlineto{\pgfqpoint{3.092184in}{1.579817in}}%
\pgfpathlineto{\pgfqpoint{3.130025in}{1.581350in}}%
\pgfpathlineto{\pgfqpoint{3.142638in}{1.585031in}}%
\pgfpathlineto{\pgfqpoint{3.180479in}{1.588561in}}%
\pgfpathlineto{\pgfqpoint{3.218320in}{1.600879in}}%
\pgfpathlineto{\pgfqpoint{3.230933in}{1.601428in}}%
\pgfpathlineto{\pgfqpoint{3.256160in}{1.605659in}}%
\pgfpathlineto{\pgfqpoint{3.268774in}{1.609346in}}%
\pgfpathlineto{\pgfqpoint{3.281387in}{1.609822in}}%
\pgfpathlineto{\pgfqpoint{3.306615in}{1.617056in}}%
\pgfpathlineto{\pgfqpoint{3.331842in}{1.620016in}}%
\pgfpathlineto{\pgfqpoint{3.344455in}{1.623494in}}%
\pgfpathlineto{\pgfqpoint{3.382296in}{1.626239in}}%
\pgfpathlineto{\pgfqpoint{3.394910in}{1.629485in}}%
\pgfpathlineto{\pgfqpoint{3.407523in}{1.629578in}}%
\pgfpathlineto{\pgfqpoint{3.420137in}{1.630907in}}%
\pgfpathlineto{\pgfqpoint{3.432750in}{1.638532in}}%
\pgfpathlineto{\pgfqpoint{3.445364in}{1.655034in}}%
\pgfpathlineto{\pgfqpoint{3.457977in}{1.655210in}}%
\pgfpathlineto{\pgfqpoint{3.470591in}{1.657555in}}%
\pgfpathlineto{\pgfqpoint{3.495818in}{1.658072in}}%
\pgfpathlineto{\pgfqpoint{3.584113in}{1.664772in}}%
\pgfpathlineto{\pgfqpoint{3.596727in}{1.675295in}}%
\pgfpathlineto{\pgfqpoint{3.609340in}{1.675636in}}%
\pgfpathlineto{\pgfqpoint{3.647181in}{1.683851in}}%
\pgfpathlineto{\pgfqpoint{3.659794in}{1.694466in}}%
\pgfpathlineto{\pgfqpoint{3.672408in}{1.696637in}}%
\pgfpathlineto{\pgfqpoint{3.697635in}{1.697513in}}%
\pgfpathlineto{\pgfqpoint{3.710249in}{1.697566in}}%
\pgfpathlineto{\pgfqpoint{3.722862in}{1.704568in}}%
\pgfpathlineto{\pgfqpoint{3.748089in}{1.710439in}}%
\pgfpathlineto{\pgfqpoint{3.760703in}{1.732054in}}%
\pgfpathlineto{\pgfqpoint{3.773317in}{1.743746in}}%
\pgfpathlineto{\pgfqpoint{3.811157in}{1.752320in}}%
\pgfpathlineto{\pgfqpoint{3.823771in}{1.805167in}}%
\pgfpathlineto{\pgfqpoint{3.836384in}{1.812629in}}%
\pgfpathlineto{\pgfqpoint{3.848998in}{1.864210in}}%
\pgfpathlineto{\pgfqpoint{3.861611in}{1.890356in}}%
\pgfpathlineto{\pgfqpoint{3.899452in}{1.894613in}}%
\pgfpathlineto{\pgfqpoint{3.912066in}{1.894694in}}%
\pgfpathlineto{\pgfqpoint{3.924679in}{1.899010in}}%
\pgfpathlineto{\pgfqpoint{3.937293in}{1.906399in}}%
\pgfpathlineto{\pgfqpoint{3.949906in}{1.931520in}}%
\pgfpathlineto{\pgfqpoint{3.962520in}{1.967074in}}%
\pgfpathlineto{\pgfqpoint{3.975134in}{2.010297in}}%
\pgfpathlineto{\pgfqpoint{4.012974in}{2.012598in}}%
\pgfpathlineto{\pgfqpoint{4.025588in}{2.017048in}}%
\pgfpathlineto{\pgfqpoint{4.038201in}{2.042673in}}%
\pgfpathlineto{\pgfqpoint{4.050815in}{2.061181in}}%
\pgfpathlineto{\pgfqpoint{4.063429in}{2.066738in}}%
\pgfpathlineto{\pgfqpoint{4.088656in}{2.068735in}}%
\pgfpathlineto{\pgfqpoint{4.101269in}{2.072767in}}%
\pgfpathlineto{\pgfqpoint{4.113883in}{2.073537in}}%
\pgfpathlineto{\pgfqpoint{4.126496in}{2.080522in}}%
\pgfpathlineto{\pgfqpoint{4.139110in}{2.090198in}}%
\pgfpathlineto{\pgfqpoint{4.151724in}{2.093140in}}%
\pgfpathlineto{\pgfqpoint{4.164337in}{2.093666in}}%
\pgfpathlineto{\pgfqpoint{4.176951in}{2.115130in}}%
\pgfpathlineto{\pgfqpoint{4.189564in}{2.156864in}}%
\pgfpathlineto{\pgfqpoint{4.202178in}{2.164009in}}%
\pgfpathlineto{\pgfqpoint{4.214791in}{2.169487in}}%
\pgfpathlineto{\pgfqpoint{4.227405in}{2.186298in}}%
\pgfpathlineto{\pgfqpoint{4.240018in}{2.196415in}}%
\pgfpathlineto{\pgfqpoint{4.252632in}{2.220210in}}%
\pgfpathlineto{\pgfqpoint{4.265246in}{2.224771in}}%
\pgfpathlineto{\pgfqpoint{4.277859in}{2.230741in}}%
\pgfpathlineto{\pgfqpoint{4.290473in}{2.245595in}}%
\pgfpathlineto{\pgfqpoint{4.303086in}{2.246548in}}%
\pgfpathlineto{\pgfqpoint{4.315700in}{2.252430in}}%
\pgfpathlineto{\pgfqpoint{4.328313in}{2.288875in}}%
\pgfpathlineto{\pgfqpoint{4.353541in}{2.305275in}}%
\pgfpathlineto{\pgfqpoint{4.353541in}{2.305275in}}%
\pgfusepath{stroke}%
\end{pgfscope}%
\begin{pgfscope}%
\pgfsetrectcap%
\pgfsetmiterjoin%
\pgfsetlinewidth{0.803000pt}%
\definecolor{currentstroke}{rgb}{0.000000,0.000000,0.000000}%
\pgfsetstrokecolor{currentstroke}%
\pgfsetdash{}{0pt}%
\pgfpathmoveto{\pgfqpoint{0.708220in}{0.535823in}}%
\pgfpathlineto{\pgfqpoint{0.708220in}{2.305275in}}%
\pgfusepath{stroke}%
\end{pgfscope}%
\begin{pgfscope}%
\pgfsetrectcap%
\pgfsetmiterjoin%
\pgfsetlinewidth{0.803000pt}%
\definecolor{currentstroke}{rgb}{0.000000,0.000000,0.000000}%
\pgfsetstrokecolor{currentstroke}%
\pgfsetdash{}{0pt}%
\pgfpathmoveto{\pgfqpoint{5.753646in}{0.535823in}}%
\pgfpathlineto{\pgfqpoint{5.753646in}{2.305275in}}%
\pgfusepath{stroke}%
\end{pgfscope}%
\begin{pgfscope}%
\pgfsetrectcap%
\pgfsetmiterjoin%
\pgfsetlinewidth{0.803000pt}%
\definecolor{currentstroke}{rgb}{0.000000,0.000000,0.000000}%
\pgfsetstrokecolor{currentstroke}%
\pgfsetdash{}{0pt}%
\pgfpathmoveto{\pgfqpoint{0.708220in}{0.535823in}}%
\pgfpathlineto{\pgfqpoint{5.753646in}{0.535823in}}%
\pgfusepath{stroke}%
\end{pgfscope}%
\begin{pgfscope}%
\pgfsetrectcap%
\pgfsetmiterjoin%
\pgfsetlinewidth{0.803000pt}%
\definecolor{currentstroke}{rgb}{0.000000,0.000000,0.000000}%
\pgfsetstrokecolor{currentstroke}%
\pgfsetdash{}{0pt}%
\pgfpathmoveto{\pgfqpoint{0.708220in}{2.305275in}}%
\pgfpathlineto{\pgfqpoint{5.753646in}{2.305275in}}%
\pgfusepath{stroke}%
\end{pgfscope}%
\begin{pgfscope}%
\pgfsetrectcap%
\pgfsetroundjoin%
\pgfsetlinewidth{1.003750pt}%
\definecolor{currentstroke}{rgb}{0.121569,0.466667,0.705882}%
\pgfsetstrokecolor{currentstroke}%
\pgfsetdash{}{0pt}%
\pgfpathmoveto{\pgfqpoint{3.776169in}{1.535662in}}%
\pgfpathlineto{\pgfqpoint{4.026169in}{1.535662in}}%
\pgfusepath{stroke}%
\end{pgfscope}%
\begin{pgfscope}%
\definecolor{textcolor}{rgb}{0.000000,0.000000,0.000000}%
\pgfsetstrokecolor{textcolor}%
\pgfsetfillcolor{textcolor}%
\pgftext[x=4.051169in,y=1.491912in,left,base]{\color{textcolor}\rmfamily\fontsize{9.000000}{10.800000}\selectfont ProCount(FlowCutter, MCS)}%
\end{pgfscope}%
\begin{pgfscope}%
\pgfsetbuttcap%
\pgfsetroundjoin%
\pgfsetlinewidth{1.003750pt}%
\definecolor{currentstroke}{rgb}{1.000000,0.498039,0.054902}%
\pgfsetstrokecolor{currentstroke}%
\pgfsetdash{{6.400000pt}{1.600000pt}{1.000000pt}{1.600000pt}}{0.000000pt}%
\pgfpathmoveto{\pgfqpoint{3.776169in}{1.360693in}}%
\pgfpathlineto{\pgfqpoint{4.026169in}{1.360693in}}%
\pgfusepath{stroke}%
\end{pgfscope}%
\begin{pgfscope}%
\definecolor{textcolor}{rgb}{0.000000,0.000000,0.000000}%
\pgfsetstrokecolor{textcolor}%
\pgfsetfillcolor{textcolor}%
\pgftext[x=4.051169in,y=1.316943in,left,base]{\color{textcolor}\rmfamily\fontsize{9.000000}{10.800000}\selectfont ProCount(FlowCutter, LP)}%
\end{pgfscope}%
\begin{pgfscope}%
\pgfsetbuttcap%
\pgfsetroundjoin%
\pgfsetlinewidth{1.003750pt}%
\definecolor{currentstroke}{rgb}{0.172549,0.627451,0.172549}%
\pgfsetstrokecolor{currentstroke}%
\pgfsetdash{{6.400000pt}{1.600000pt}{1.000000pt}{1.600000pt}}{0.000000pt}%
\pgfpathmoveto{\pgfqpoint{3.776169in}{1.185723in}}%
\pgfpathlineto{\pgfqpoint{4.026169in}{1.185723in}}%
\pgfusepath{stroke}%
\end{pgfscope}%
\begin{pgfscope}%
\definecolor{textcolor}{rgb}{0.000000,0.000000,0.000000}%
\pgfsetstrokecolor{textcolor}%
\pgfsetfillcolor{textcolor}%
\pgftext[x=4.051169in,y=1.141973in,left,base]{\color{textcolor}\rmfamily\fontsize{9.000000}{10.800000}\selectfont ProCount(htd, MCS)}%
\end{pgfscope}%
\begin{pgfscope}%
\pgfsetbuttcap%
\pgfsetroundjoin%
\pgfsetlinewidth{1.003750pt}%
\definecolor{currentstroke}{rgb}{0.839216,0.152941,0.156863}%
\pgfsetstrokecolor{currentstroke}%
\pgfsetdash{{6.400000pt}{1.600000pt}{1.000000pt}{1.600000pt}}{0.000000pt}%
\pgfpathmoveto{\pgfqpoint{3.776169in}{1.010754in}}%
\pgfpathlineto{\pgfqpoint{4.026169in}{1.010754in}}%
\pgfusepath{stroke}%
\end{pgfscope}%
\begin{pgfscope}%
\definecolor{textcolor}{rgb}{0.000000,0.000000,0.000000}%
\pgfsetstrokecolor{textcolor}%
\pgfsetfillcolor{textcolor}%
\pgftext[x=4.051169in,y=0.967004in,left,base]{\color{textcolor}\rmfamily\fontsize{9.000000}{10.800000}\selectfont ProCount(htd, LP)}%
\end{pgfscope}%
\begin{pgfscope}%
\pgfsetbuttcap%
\pgfsetroundjoin%
\pgfsetlinewidth{1.003750pt}%
\definecolor{currentstroke}{rgb}{0.580392,0.403922,0.741176}%
\pgfsetstrokecolor{currentstroke}%
\pgfsetdash{{6.400000pt}{1.600000pt}{1.000000pt}{1.600000pt}}{0.000000pt}%
\pgfpathmoveto{\pgfqpoint{3.776169in}{0.835784in}}%
\pgfpathlineto{\pgfqpoint{4.026169in}{0.835784in}}%
\pgfusepath{stroke}%
\end{pgfscope}%
\begin{pgfscope}%
\definecolor{textcolor}{rgb}{0.000000,0.000000,0.000000}%
\pgfsetstrokecolor{textcolor}%
\pgfsetfillcolor{textcolor}%
\pgftext[x=4.051169in,y=0.792034in,left,base]{\color{textcolor}\rmfamily\fontsize{9.000000}{10.800000}\selectfont ProCount(Tamaki, MCS)}%
\end{pgfscope}%
\begin{pgfscope}%
\pgfsetbuttcap%
\pgfsetroundjoin%
\pgfsetlinewidth{1.003750pt}%
\definecolor{currentstroke}{rgb}{0.549020,0.337255,0.294118}%
\pgfsetstrokecolor{currentstroke}%
\pgfsetdash{{6.400000pt}{1.600000pt}{1.000000pt}{1.600000pt}}{0.000000pt}%
\pgfpathmoveto{\pgfqpoint{3.776169in}{0.660815in}}%
\pgfpathlineto{\pgfqpoint{4.026169in}{0.660815in}}%
\pgfusepath{stroke}%
\end{pgfscope}%
\begin{pgfscope}%
\definecolor{textcolor}{rgb}{0.000000,0.000000,0.000000}%
\pgfsetstrokecolor{textcolor}%
\pgfsetfillcolor{textcolor}%
\pgftext[x=4.051169in,y=0.617065in,left,base]{\color{textcolor}\rmfamily\fontsize{9.000000}{10.800000}\selectfont ProCount(Tamaki, LP)}%
\end{pgfscope}%
\end{pgfpicture}%
\makeatother%
\endgroup%

    \caption{
        This plot compares different combinations of an \Lg{} tree decomposer (\flowcutter{}, \htd{}, or \tamaki{}) and a \dmc{} variable-ordering heuristic (\mcs{} or \lexp) for our framework \procount.
        We choose \Lg{} with \flowcutter{} and \dmc{} width \mcs{} as the representative setting of \procount{} to compete with existing projected model counters.
    }
    \label{figSolvingA}
\end{figure}
