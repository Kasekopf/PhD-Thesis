\section{Chapter Summary}
\label{sec_discussion}

In this work, we introduced the concept of project-join trees for weighted model counting.
These trees are at the center of a dynamic-programming framework that unifies and generalizes several model counting algorithms, including those based on ADDs \cite{DPV20}, tensors \cite{DDV19}, and database management systems \cite{fichte2020exploiting}.
This framework performs model counting in two phases.
First, the planning phase produces a project-join tree from a CNF formula.
Second, the execution phase uses the project-join tree to guide the dynamic-programming computation of the model count of the formula \wrt{} a literal-weight function.
The current implementation of our dynamic-programming model-counting framework \Dpmc{} includes two planners (\Htb{} and \Lg) and two executors (\Dmc{} and \Tensor{}).

For the planning phase, we implemented \Htb{} based on constraint-satisfaction heuristics
\cite{tarjan1984simple,koster2001treewidth,dechter03,dechter99,bouquet1999gestion}
and \Lg{} based on tree-decomposition tools \cite{strasser2017computing,Tamaki17,AMW17}.
Our empirical work indicates that tree-decomposition tools tend to produce project-join trees of lower widths in shorter times.
This is a significant finding with applications beyond model counting, \eg, in Boolean functional synthesis \cite{tabajara2017factored}.

For the execution phase, we implemented \Dmc{} based on ADDs \cite{DPV20,somenzi2015cudd} and \Tensor{} based on tensors \cite{DDV19,numpy}.
Empirically, we observed that (sparse) ADDs outperform (dense) tensors on single CPU cores.
Whether this holds for richer architectures as well is a subject for future work.
We will also consider adding to our framework an executor based on databases (\eg, \cite{fichte2020exploiting}).

We showed that our dynamic-programming model-counting framework \Dpmc{} is competitive with state-of-the-art tools (\cachet{} \cite{sang2004combining}, \ctd{} \cite{darwiche2004new}, \df{} \cite{LM17}, and \minictd{} \cite{OD15}).
Although no single model counter dominates,
\Dpmc{} considerably improves the virtual best solver and thus is valuable as part of the portfolio.