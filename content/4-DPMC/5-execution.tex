\section{Execution Phase: Performing the Valuation}
\label{sec_execution}

The execution phase involves a CNF formula $\phi$ over variables $X$, a project-join tree $(T, r, \gamma, \pi)$ of $\phi$, and a literal-weight function $W$ over $X$.
The goal is to compute the valuation $f^W_r$ using Equation \eqref{eq_valuation}.
Several data structures can be used for the pseudo-Boolean functions that occur while using Equation \eqref{eq_valuation}.
In this work, we consider two data structures that have been applied to weighted model counting: ADDs (as in \cite{dudek2020addmc}) and tensors (as in \cite{dudek2019efficient}).

% Algorithm \ref{alg_jt_wmc} computes the model count of a CNF formula, employing dynamic programming as guided by a project-join tree.
% \begin{algorithm*}[H]
% \label{alg_jt_wmc}
% \caption{Model counting with a project-join tree of a CNF formula}
%     \DontPrintSemicolon
%     \KwIn{$X$: set of Boolean variables}
%     \KwIn{$\phi$: CNF formula over $X$}
%     \KwIn{$(T, r, \pi, \gamma)$: project-join tree of $\phi$}
%     \KwIn{$W$: literal-weight function over $X$}
%     \KwOut{$W(\phi)$: literal-weighted model count of $\phi$ \wrt{} $W$}
%     % \Begin{
%         \Function{\upshape $\nodeValuation(n)$}{
%             \KwIn{$n \in \V{T}$: node of project-join tree}
%             \KwOut{$f^W_n$: $W$-valuation of $n$}
%             % \Begin{
%                 \If{$n \in \Lv{T}$}{
%                     \Return{\upshape corresponding clause $\gamma(n) \in \phi$}
%                 }
%                 \Else{
%                     $P \gets \pi(n)$
%                         \tcc*{variables to project}
%                     \Return{$\sum_{P} \pars{ \prod_{o \in \C(n)} f^W_o \mult \prod_{x \in P} W_x }$}\;
%                 }
%             % }
%         }
%         \Return{\upshape $\nodeValuation(r)(\emptyset)$}
%             \tcc*{$\nodeValuation(r) : 2^\emptyset \to \R$}
%     % }
% \end{algorithm*}
% Note that Algorithm \ref{alg_jt_wmc} can be implemented with different data structures.
% For example, the pseudo-Boolean functions $\gamma(n)$, $f^W_n$, and $W_x$ can be represented by algebraic decision diagrams or tensors.

%%%%%%%%%%%%%%%%%%%%%%%%%%%%%%%%%%%%%%%%%%%%%%%%%%%%%%%%%%%%%%%%%%%%%%%%%%%%%%%%







\subsection{Algebraic Decision Diagrams}

An \emph{algebraic decision diagram (ADD)} is a compact representation of a pseudo-Boolean function as a directed acyclic graph \cite{bahar1997algebraic}.
For functions with logical structure, an ADD representation can be exponentially smaller than the explicit representation.
Originally designed for matrix multiplication and shortest path algorithms, ADDs have also been used for Bayesian inference \cite{chavira2007compiling,gogate2011approximation}, stochastic planning \cite{hoey1999spudd}, model checking \cite{kwiatkowska2007stochastic}, and model counting \cite{fargier2014knowledge,dudek2020addmc}.

Formally, an ADD is a tuple $(X, S, \sigma, G)$, where $X$ is a set of Boolean variables, $S$ is an arbitrary set (called the \textdef{carrier set}), $\sigma: X \to \N$ is an injection (called the \textdef{diagram variable order}), and $G$ is a rooted directed acyclic graph satisfying the following three properties.
First, every leaf node of $G$ is labeled with an element of $S$.
Second, every internal node of $G$ is labeled with an element of $X$ and has two outgoing edges, labeled 0 and 1.
Finally, for every path in $G$, the labels of internal nodes must occur in increasing order under $\sigma$.
In this work, we only need to consider ADDs with the carrier set $S = \mathbb{R}$.

An ADD $(X, S, \sigma, G)$ is a compact representation of a function $f: 2^X \to S$.
Although there are many ADDs representing $f$, for each injection $\sigma: X \to \N$, there is a unique minimal ADD that represents $f$ with $\sigma$ as the diagram variable order, called the \textdef{canonical ADD}.
ADDs can be minimized in polynomial time, so it is typical to only work with canonical ADDs.

Several packages exist for efficiently manipulating ADDs.
For example, \cudd{} \cite{somenzi2015cudd} implements both product and projection on ADDs in polynomial time (in the size of the ADD representation).
\cudd{} was used as the primary data structure for weighted model counting in \cite{dudek2020addmc}.
In this work, we also use ADDs with \cudd{} to compute $W$-valuations.

\Mcs{} was the best diagram variable order on a set of standard weighted model counting benchmarks in \cite{dudek2020addmc}.
So we use \Mcs{} as the diagram variable order in this work.
Note that all other heuristics discussed in Section \ref{sec_csp} for cluster variable order could also be used as heuristics for diagram variable order.
% JD: We only consider MCS in Section 6, so this paragraph is not really needed
% One challenge in using ADDs is choosing the diagram variable order.
% The choice of diagram variable order can have a dramatic impact on the size of the ADD; some variable orders may produce ADDs that are exponentially smaller than others for the same real-valued function.
% A variety of techniques exist in prior work to heuristically find diagram variable orders \cite{tarjan1984simple,koster2001treewidth,dechter2003constraint}.
% Moreover, since binary decision diagrams (BDDs) \cite{bryant1986graph} can be seen as ADDs with carrier set $S = \set{0, 1}$, there is significant overlap with the techniques to find variable orders for BDDs.
% All cluster-variable-order heuristics discussed in Section \ref{sec_csp} (\Random, \Mcs, \Invmcs, \Lexp, \Invlexp, \Lexm, \Invlexm, \Minfill, and \Invminfill) can also be used as diagram-variable-order heuristics.
% In \cite{dudek2020addmc}, \Mcs{} was the best diagram variable order on a set of standard weighted model counting benchmarks.

% ADDs were explored for weighted model counting in a dynamic-programming framework in \cite{dudek2020addmc}.
% In detail, their model-counting algorithm essentially combines Algorithm \ref{alg_csp_jt} and Algorithm \ref{alg_jt_wmc}, using ADDs as the primary data structure.
% Their two best heuristic configurations have \Mcs{} as the diagram variable order.
% The other diagram variable orders they investigated were \Invmcs, \Lexp, \Invlexp, \Lexm, \Invlexm, and \Random.


%%%%%%%%%%%%%%%%%%%%%%%%%%%%%%%%%%%%%%%%%%%%%%%%%%%%%%%%%%%%%%%%%%%%%%%%%%%%%%%%

\subsection{Tensors}
% A \emph{tensor} is a multi-dimensional generalization of a matrix.
% Tensor are widely used in data analysis \cite{Cichocki14}, signal and image processing \cite{cichocki2015tensor}, quantum physics \cite{arad2010quantum}, quantum chemistry \cite{smilde2005multi}, and many other areas of science.
% Given the diverse applications of tensors and tensor networks, a variety of tools \cite{baumgartner2005synthesis,KKCLA17} exist to manipulate them efficiently on a variety of hardware architectures, including multi-core and GPU-enhanced architectures.

A \emph{tensor} is a multi-dimensional generalization of a matrix and can be used to represent pseudo-Boolean functions in a dense way.
Tensors are particularly efficient at computing the contraction of two pseudo-Boolean functions: given two functions $f: 2^X \to \mathbb{R}$ and $g: 2^Y \to \mathbb{R}$, their \emph{contraction} $f \contract g$ is the pseudo-Boolean function $\proj_{X \cap Y} f \cdot g$.
The contraction of two tensors can be implemented as matrix multiplication and so leverage significant work in high-performance computing on matrix multiplication on CPUs \cite{LHKK77} and GPUs \cite{FSH04}.
To efficiently use tensors to compute $W$-valuations, we implement projection and product using tensor contraction.

First, we must compute the weighted projection of a function $f: 2^X \to \mathbb{R}$, \ie, we must compute $\proj_x f \cdot W_x$ for some $x \in X$.
This is exactly equivalent to $f \contract W_x$.
Second, we must compute the product of two functions $f: 2^X \to \mathbb{R}$ and $g: 2^Y \to \mathbb{R}$.
The central challenge is that tensor contraction implicitly projects all variables in $X \cap Y$, but we often need to maintain some shared variables in the result of $f \cdot g$.
In Chapter \ref{ch:tensors}, this problem was solved using a reduction from weighted model counting to tensor networks.
After the reduction, all indices appear exactly twice, so one never needs to perform a product without also projecting all shared indices.
Moreover, the additional overhead incurred by this reduction was lessened through the \textbf{FT} planner (see Section \ref{sec:tensors:preprocessing}).

In order to incorporate tensors in our project-join-tree-based framework, we take a different strategy that uses copy tensors.
The \emph{copy tensor} for a set $X$ represents the pseudo-Boolean function $\blacksquare_X: 2^X \to \mathbb{R}$ \st{} $\blacksquare_X(\tau)$ is $1$ if $\tau \in \{ \emptyset, X \}$ and $0$ otherwise.
We can simulate product using contraction by including additional copy tensors.
In detail, for each $z \in X \cap Y$ make two fresh variables $z'$ and $z''$.
Replace each $z$ in $f$ with $z'$ to produce $f'$, and replace each $z$ in $g$ with $z''$ to produce $g'$.
Then one can check that $f \cdot g = f' \contract g' \contract \bigcontract_{z \in X \cap Y} \blacksquare_{\{z, z', z''\}}$.

When a product is immediately followed by the projection of shared variables (\ie, we are computing $\proj_Z f \cdot g$ for some $Z \subseteq X \cap Y$), we can optimize this procedure.
In particular, we skip creating copy tensors for the variables in $Z$ and instead eliminate them directly as we perform $f' \contract g'$.
In this case, we do not ever fully compute $f \cdot g$, so the maximum number of variables needed in each intermediate tensor may be lower than the width of the project-join tree.
In the context of tensor networks and contraction trees, the maximum number of variables needed after accounting for this optimization is exactly the \emph{max-rank} of the contraction tree \cite{KCMR18}.
The max-rank is often lower than the width of the corresponding project-joint tree.
On the other hand, the intermediate terms in the computation of $f \cdot g$ with contractions may have more variables than either $f$, $g$, or $f \cdot g$.
Thus the number of variables in each intermediate tensor may be higher than the width of the project-join tree (by at most a factor of 1.5).

