\section{Discrete Integration}
\label{sec:wmc}

In discrete integration (also called constrained counting) the task is to count the total weight, subject to a given weight function, of the set of solutions of input constraints \cite{GSS08}. We represent weight functions as pseudo-Boolean functions. As is standard, we focus on literal-weight functions: 
\begin{definition}[Literal-Weight Function]
\label{def:literal-weight}
A pseudo-Boolean function $W: 2^X \rightarrow \mathbb{R}$ is \emph{literal-weight} function if there exist pseudo-Boolean functions $W_x: 2^{\{x\}} \rightarrow \mathbb{R}$ for all $x \in X$ such that $W = \prod_{x \in X} W_x$. 
\end{definition}
Literal-weight functions are also called \emph{log-linear} functions and can be employed to capture a wide variety of probabilistic distributions that arise from graphical models, conditional random fields, skip-gram models, and the like~\cite{KF09}.
$W_x(\{x\})$ indicates the weight of the positive literal $x$ in $W$ and $W_x(\emptyset)$ indicates the weight of the negative literal $\neg x$ in $W$.

This thesis primarily focuses on two subclasses of discrete integration: model counting and projected model counting.

\subsection{Model Counting}
In (weighted) model counting, the input constraints are given as a propositional formula, typically in CNF. Formally:
\begin{definition}[Weighted Model Count]
  Let $\varphi$ be a Boolean formula over Boolean variables $X$ and let $W: 2^X \rightarrow \mathbb{R}$ be a pseudo-Boolean function. We say that $(X, \phi, W)$ is an instance of \emph{weighted model counting}. The \emph{$W$-weighted model count} of $\varphi$ is
  $$W(\varphi) \equiv \sum_{\tau \in 2^X} \varphi(\tau) \cdot W(\tau).$$
\end{definition}

The \emph{unweighted model count} of a Boolean formula $\varphi$ is the number of solutions of $\varphi$, i.e. the $W$-weighted model count of $\varphi$ where $W(\tau) = 1$ for all $\tau \in 2^X$.

Existing approaches to weighted model counting can be split broadly into three categories: \emph{search}, \emph{knowledge compilation}, and \emph{dynamic programming}. In counters based on search (e.g., \tool{cachet}  \cite{SBK05} and \tool{SharpSAT} \cite{Thurley2006}), the idea is to take an algorithm for exploring the entire solution space of $\varphi$ (e.g. DPLL \cite{davis1960computing,davis1962machine}, and later CDCL \cite{biere2009conflict}) and augment it to enumerate the number of partial solutions encountered. In counters based on knowledge compilation (e.g. \tool{miniC2D} \cite{OD15} and \tool{d4} \cite{LM17}), the idea is to compile $\varphi$ into an alternative representation on which counting is easy. In counters based on dynamic programming (e.g. \tool{ADDMC} \cite{DPV20} and \tool{gpuSAT2} \cite{FHWZ18,FHZ19}), the idea is to traverse the constraint structure of $\varphi$ and reason about subsets of constraints. 
The techniques developed in this thesis are all based on dynamic programming.

\subsection{Projected Model Counting}
In (weighted) projected model counting, the input constraints are given as a propositional formula, typically in CNF, together with a set of irrelevant, existential variables.
Formally:
\begin{definition}[Weighted Projected Model Count]
	Let $\phi$ be a Boolean formula, $\{X, Y\}$ be a partition of $\vars(\phi)$, and $W: 2^X \to \R$ be a pseudo-Boolean function. We say that $(X, Y, \phi, W)$ is an instance of \emph{weighted projected model counting}.
	The \emph{$W$-weighted $Y$-projected model count} of $\phi$ is
	$$\func{WPMC}(\phi, W, Y) \equiv \sum_{\tau \in 2^X} \pars{ W(\tau) \mult \max_{\sigma \in 2^Y} \phi(\tau \cup \sigma) }.$$
\end{definition}

Variables in $X$ are called \emph{relevant} or \emph{additive}, and variables in $Y$ are called \emph{irrelevant} or \emph{disjunctive}. 
Notice that model counting is a special case of projected model counting where all variables are relevant. Thus $W(\varphi) = \func{WPMC}(\phi, W, \emptyset)$ for all Boolean formulas $\varphi$ and weight functions $W: 2^{\vars(\varphi)} \rightarrow \mathbb{R}$. 

There are several recent tools for weighted projected model counting.
For example, \dfp{} is based on compilation to decision decomposable negation normal form \cite{lagniez2019recursive}, \projmc{} is based on search (to build disjunctive decompositions) \cite{lagniez2019recursive}, and \ssat{} is based on search \cite{lee2017solving}. 
While our focus in Chapter \ref{ch:procount} is on deterministic exact weighted projected model counting, it is worth mentioning that various relaxations have also been studied, e.g., probabilistic \cite{sharma2019ganak}, approximate \cite{ermon2013taming,fremont2017maximum,soos2019bird}, or unweighted \cite{zawadzki2013generalization,aziz2015projected,mohle2018dualizing,hecher2020taming} projected model counting.