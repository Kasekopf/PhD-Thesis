\section{Implementation and Evaluation} \label{sec:tensors:experiments}
We aim to answer the following experimental research questions:
\begin{enumerate}
    \item[(RQ1)] Are tensor-network-based approaches competitive with existing state-of-the-art unweighted model counters?
    \item[(RQ2)] Are tensor-network-based approaches competitive with existing state-of-the-art weighted model counters?
    \item[(RQ3)] What are the structural properties of benchmarks for which the tensor-network-based approaches perform well?
\end{enumerate}

To answer these questions, we implement Algorithm 2 in \tool{TensorOrder}, a new tool for weighted model counting using tensor networks. \tool{TensorOrder} can be configured to find contraction trees using existing planning methods -- \textbf{greedy} (using a greedy algorithm), \textbf{metis} (using graph partitioning), and \textbf{GN} (using community structure detection) \cite{KCMR18}-- or planning methods presented in this paper-- \textbf{LG} (Section \ref{sec:tensors:contraction-theory}) and \textbf{FT} (Section \ref{sec:tensors:preprocessing}). Implementation details appear in Section \ref{sec:tensors:experiments:implementation}.

To answer RQ1, in Section \ref{sec:tensors:experiments:cubic} we compare \tool{TensorOrder} with existing state-of-the-art tools for unweighted model counting (\tool{dynQBF} \cite{CW16}, \tool{dynasp} \cite{FHMW17}, \tool{SharpSAT} \cite{Thurley2006}, \tool{cachet} \cite{SBK05}, \tool{miniC2D} \cite{OD15} and \tool{d4} \cite{LM17}) on formulas that count the number of vertex covers of randomly-generated cubic graphs \cite{KCMR18}. Note \tool{dynQBF} and \tool{dynasp} are solvers from related domains (that can be used as tools for unweighted model counting) that also use tree decompositions.

To answer RQ2, in Section \ref{sec:tensors:experiments:cachet} we compare \tool{TensorOrder} with existing state-of-the-art tools for weighted model counting (\tool{cachet} \cite{SBK05}, \tool{miniC2D} \cite{OD15} and \tool{d4} \cite{LM17}) on formulas whose weighted count corresponds to exact inference on Bayesian networks \cite{SBK05}. Note that the other tools (\tool{dynQBF}, \tool{dynasp}, and \tool{SharpSAT}) cannot perform weighted model counting.

% We use \tool{TensorOrder} to compare tensor-based methods with existing state-of-the-art tools for weighted model counting: \tool{cachet} \cite{SBK05}, \tool{miniC2D} \cite{OD15} and \tool{d4} \cite{LM17}. %Note that \tool{d4} requires a d-DNNF reasoner to perform weighted model counting; we use \cite{CDLL18}. 
% We also compare with \tool{dynQBF} \cite{CW16}, \tool{dynasp} \cite{FHMW17} and \tool{SharpSAT} \cite{Thurley2006} when the benchmarks are unweighted. Note \tool{dynQBF} and \tool{dynasp} are solvers from related domains (that can be used as model counters) that also use tree decompositions.

% We then compare our tensor-based algorithms for weighted model counting against state-of-the-art weighted model counters and against existing tensor-based algorithms. We demonstrate that our tensor-based algorithms are useful as part of a portfolio of weighted model counters. 

% We compare \tool{TensorOrder} on two sets of existing benchmarks. First, in Section \ref{sec:tensors:experiments:cubic} we compare on formulas that count the number of vertex covers of randomly-generated cubic graphs \cite{KCMR18}. Second, in Section \ref{sec:tensors:experiments:cachet} we compare 
To answer RQ3, we compute upper bounds on the treewidth and carving width of these benchmarks. We run each of three heuristic tree decomposition solvers (\pkg{Tamaki} \cite{Tamaki17}, \pkg{FlowCutter} \cite{HS18}, and \pkg{htd} \cite{AMW17}) on each benchmark with a timeout of 1000 seconds: once on the structure graph $G$ corresponding to the benchmark, and once on $\Line{G}$. The minimal width of all produced tree decompositions for $G$ (resp. $\Line{G}$) is an upper bound for the treewidth of $G$ (resp. $\Line{G}$). The minimal max rank of the contraction trees produced by running \textbf{LG} (resp. \textbf{FT}) on each tree decomposition of $\Line{G}$ (resp. $G$) is an upper bound for the carving width of $G$ (resp. $G$ after \text{FT}-preprocessing).

Each experiment was run in a high-performance cluster (Linux kernel 2.6.32) using a single 2.80 GHz core of an Intel Xeon X5660 CPU and 48 GB RAM. We provide all code, benchmarks, and detailed data of benchmark runs at \url{https://github.com/vardigroup/TensorOrder}.

% \paragraph{Experimental Methodology}
% To evaluate runtime performance, we run each tool once on each benchmark with a timeout of 1000 seconds and record the wall-clock time taken. For \tool{TensorOrder}, recorded times include all of Algorithm \ref{alg:wmc} (and, specifically, include the time of the underlying tree-decomposition solver).



% We then use these tree decompositions to compute upper bounds on the treewidth of $G$ (the minimal width of all found tree decompositions for $G$), the treewidth of $\Line{G}$ (the minimal width of all found tree decompositions for $\Line{G}$), the carving width of $G$ (the minimal max rank of all contraction trees produced by running \textbf{LG} on tree decompositions of $\Line{G}$), and the carving width of $G$ after \textbf{FT}-preprocessing (the minimal max rank of all contraction trees produced by running \textbf{FT} on tree decompositions of $\Line{G}$).

% \begin{enumerate}
%     \item the treewidth of $G$ by taking the minimum width amongst all discovered tree decompositions for $G$,
%     \item the treewidth of $\Line{G}$ by taking the minimum width amongst all discovered tree decompositions for $\Line{G}$,
%     \item the carving width of $G$ by taking the minimum width amongst all contraction trees produced by running \textbf{LG} on each tree decomposition of $\Line{G}$, and
%     \item the carving width of $G$ after \textbf{FT}-preprocessing by taking the minimum width amongst all contraction trees produced by running \textbf{FT} on each tree decomposition of $G$.
% \end{enumerate}

% compute an upper bound for the treewidth of $G$ (resp. $\Line{G}$) by taking the minimum width amongst all discovered tree decompositions for $G$ (resp. $\Line{G}$). Similarly, we compute an upper bound for the carving width of $G$ by taking the minimum width amongst all contraction trees produced by running \textbf{LG} on each tree decomposition of $\Line{G}$.


% We then record the width of the best tree decomposition found amongst all tree-decomposition solvers. We also 

% We also evaluate the structural properties of these benchmarks by performing an experimental comparison of treewidth and carving width of the incidence. For each benchmark with corresponding incidence graph $G$, we run each of the three heuristic tree decomposition solvers \tool{Tamaki}, \tool{FlowCutter}, and \tool{htd} for 1000 seconds on $G$ and $\Line{G}$. We then record the width of the best tree decomposition found amongst all tree-decomposition solvers. On each tree decomposition for $\Line{G}$ found by the solvers, we use \textbf{LG} to compute the corresponding carving decomposition of $G$ and recorded the smallest width found amongst all decompositions. Similarly, on each tree decomposition for $G$ found by the solvers, we use \textbf{FT} to compute the corresponding carving decomposition of the preprocessed graph and recorded the smallest width found amongst all decompositions. 

\begin{figure}
	\centering
	%% Creator: Matplotlib, PGF backend
%%
%% To include the figure in your LaTeX document, write
%%   \input{<filename>.pgf}
%%
%% Make sure the required packages are loaded in your preamble
%%   \usepackage{pgf}
%%
%% and, on pdftex
%%   \usepackage[utf8]{inputenc}\DeclareUnicodeCharacter{2212}{-}
%%
%% or, on luatex and xetex
%%   \usepackage{unicode-math}
%%
%% Figures using additional raster images can only be included by \input if
%% they are in the same directory as the main LaTeX file. For loading figures
%% from other directories you can use the `import` package
%%   \usepackage{import}
%%
%% and then include the figures with
%%   \import{<path to file>}{<filename>.pgf}
%%
%% Matplotlib used the following preamble
%%   \usepackage[utf8x]{inputenc}
%%   \usepackage[T1]{fontenc}
%%
\begingroup%
\makeatletter%
\begin{pgfpicture}%
\pgfpathrectangle{\pgfpointorigin}{\pgfqpoint{6.000000in}{3.400000in}}%
\pgfusepath{use as bounding box, clip}%
\begin{pgfscope}%
\pgfsetbuttcap%
\pgfsetmiterjoin%
\definecolor{currentfill}{rgb}{1.000000,1.000000,1.000000}%
\pgfsetfillcolor{currentfill}%
\pgfsetlinewidth{0.000000pt}%
\definecolor{currentstroke}{rgb}{1.000000,1.000000,1.000000}%
\pgfsetstrokecolor{currentstroke}%
\pgfsetdash{}{0pt}%
\pgfpathmoveto{\pgfqpoint{0.000000in}{0.000000in}}%
\pgfpathlineto{\pgfqpoint{6.000000in}{0.000000in}}%
\pgfpathlineto{\pgfqpoint{6.000000in}{3.400000in}}%
\pgfpathlineto{\pgfqpoint{0.000000in}{3.400000in}}%
\pgfpathclose%
\pgfusepath{fill}%
\end{pgfscope}%
\begin{pgfscope}%
\pgfsetbuttcap%
\pgfsetmiterjoin%
\definecolor{currentfill}{rgb}{1.000000,1.000000,1.000000}%
\pgfsetfillcolor{currentfill}%
\pgfsetlinewidth{0.000000pt}%
\definecolor{currentstroke}{rgb}{0.000000,0.000000,0.000000}%
\pgfsetstrokecolor{currentstroke}%
\pgfsetstrokeopacity{0.000000}%
\pgfsetdash{}{0pt}%
\pgfpathmoveto{\pgfqpoint{0.708220in}{0.535823in}}%
\pgfpathlineto{\pgfqpoint{5.850000in}{0.535823in}}%
\pgfpathlineto{\pgfqpoint{5.850000in}{3.205275in}}%
\pgfpathlineto{\pgfqpoint{0.708220in}{3.205275in}}%
\pgfpathclose%
\pgfusepath{fill}%
\end{pgfscope}%
\begin{pgfscope}%
\pgfsetbuttcap%
\pgfsetroundjoin%
\definecolor{currentfill}{rgb}{0.000000,0.000000,0.000000}%
\pgfsetfillcolor{currentfill}%
\pgfsetlinewidth{0.803000pt}%
\definecolor{currentstroke}{rgb}{0.000000,0.000000,0.000000}%
\pgfsetstrokecolor{currentstroke}%
\pgfsetdash{}{0pt}%
\pgfsys@defobject{currentmarker}{\pgfqpoint{0.000000in}{-0.048611in}}{\pgfqpoint{0.000000in}{0.000000in}}{%
\pgfpathmoveto{\pgfqpoint{0.000000in}{0.000000in}}%
\pgfpathlineto{\pgfqpoint{0.000000in}{-0.048611in}}%
\pgfusepath{stroke,fill}%
}%
\begin{pgfscope}%
\pgfsys@transformshift{0.708220in}{0.535823in}%
\pgfsys@useobject{currentmarker}{}%
\end{pgfscope}%
\end{pgfscope}%
\begin{pgfscope}%
\definecolor{textcolor}{rgb}{0.000000,0.000000,0.000000}%
\pgfsetstrokecolor{textcolor}%
\pgfsetfillcolor{textcolor}%
\pgftext[x=0.708220in,y=0.438600in,,top]{\color{textcolor}\rmfamily\fontsize{9.000000}{10.800000}\selectfont \(\displaystyle {0}\)}%
\end{pgfscope}%
\begin{pgfscope}%
\pgfsetbuttcap%
\pgfsetroundjoin%
\definecolor{currentfill}{rgb}{0.000000,0.000000,0.000000}%
\pgfsetfillcolor{currentfill}%
\pgfsetlinewidth{0.803000pt}%
\definecolor{currentstroke}{rgb}{0.000000,0.000000,0.000000}%
\pgfsetstrokecolor{currentstroke}%
\pgfsetdash{}{0pt}%
\pgfsys@defobject{currentmarker}{\pgfqpoint{0.000000in}{-0.048611in}}{\pgfqpoint{0.000000in}{0.000000in}}{%
\pgfpathmoveto{\pgfqpoint{0.000000in}{0.000000in}}%
\pgfpathlineto{\pgfqpoint{0.000000in}{-0.048611in}}%
\pgfusepath{stroke,fill}%
}%
\begin{pgfscope}%
\pgfsys@transformshift{1.833336in}{0.535823in}%
\pgfsys@useobject{currentmarker}{}%
\end{pgfscope}%
\end{pgfscope}%
\begin{pgfscope}%
\definecolor{textcolor}{rgb}{0.000000,0.000000,0.000000}%
\pgfsetstrokecolor{textcolor}%
\pgfsetfillcolor{textcolor}%
\pgftext[x=1.833336in,y=0.438600in,,top]{\color{textcolor}\rmfamily\fontsize{9.000000}{10.800000}\selectfont \(\displaystyle {50}\)}%
\end{pgfscope}%
\begin{pgfscope}%
\pgfsetbuttcap%
\pgfsetroundjoin%
\definecolor{currentfill}{rgb}{0.000000,0.000000,0.000000}%
\pgfsetfillcolor{currentfill}%
\pgfsetlinewidth{0.803000pt}%
\definecolor{currentstroke}{rgb}{0.000000,0.000000,0.000000}%
\pgfsetstrokecolor{currentstroke}%
\pgfsetdash{}{0pt}%
\pgfsys@defobject{currentmarker}{\pgfqpoint{0.000000in}{-0.048611in}}{\pgfqpoint{0.000000in}{0.000000in}}{%
\pgfpathmoveto{\pgfqpoint{0.000000in}{0.000000in}}%
\pgfpathlineto{\pgfqpoint{0.000000in}{-0.048611in}}%
\pgfusepath{stroke,fill}%
}%
\begin{pgfscope}%
\pgfsys@transformshift{2.958452in}{0.535823in}%
\pgfsys@useobject{currentmarker}{}%
\end{pgfscope}%
\end{pgfscope}%
\begin{pgfscope}%
\definecolor{textcolor}{rgb}{0.000000,0.000000,0.000000}%
\pgfsetstrokecolor{textcolor}%
\pgfsetfillcolor{textcolor}%
\pgftext[x=2.958452in,y=0.438600in,,top]{\color{textcolor}\rmfamily\fontsize{9.000000}{10.800000}\selectfont \(\displaystyle {100}\)}%
\end{pgfscope}%
\begin{pgfscope}%
\pgfsetbuttcap%
\pgfsetroundjoin%
\definecolor{currentfill}{rgb}{0.000000,0.000000,0.000000}%
\pgfsetfillcolor{currentfill}%
\pgfsetlinewidth{0.803000pt}%
\definecolor{currentstroke}{rgb}{0.000000,0.000000,0.000000}%
\pgfsetstrokecolor{currentstroke}%
\pgfsetdash{}{0pt}%
\pgfsys@defobject{currentmarker}{\pgfqpoint{0.000000in}{-0.048611in}}{\pgfqpoint{0.000000in}{0.000000in}}{%
\pgfpathmoveto{\pgfqpoint{0.000000in}{0.000000in}}%
\pgfpathlineto{\pgfqpoint{0.000000in}{-0.048611in}}%
\pgfusepath{stroke,fill}%
}%
\begin{pgfscope}%
\pgfsys@transformshift{4.083568in}{0.535823in}%
\pgfsys@useobject{currentmarker}{}%
\end{pgfscope}%
\end{pgfscope}%
\begin{pgfscope}%
\definecolor{textcolor}{rgb}{0.000000,0.000000,0.000000}%
\pgfsetstrokecolor{textcolor}%
\pgfsetfillcolor{textcolor}%
\pgftext[x=4.083568in,y=0.438600in,,top]{\color{textcolor}\rmfamily\fontsize{9.000000}{10.800000}\selectfont \(\displaystyle {150}\)}%
\end{pgfscope}%
\begin{pgfscope}%
\pgfsetbuttcap%
\pgfsetroundjoin%
\definecolor{currentfill}{rgb}{0.000000,0.000000,0.000000}%
\pgfsetfillcolor{currentfill}%
\pgfsetlinewidth{0.803000pt}%
\definecolor{currentstroke}{rgb}{0.000000,0.000000,0.000000}%
\pgfsetstrokecolor{currentstroke}%
\pgfsetdash{}{0pt}%
\pgfsys@defobject{currentmarker}{\pgfqpoint{0.000000in}{-0.048611in}}{\pgfqpoint{0.000000in}{0.000000in}}{%
\pgfpathmoveto{\pgfqpoint{0.000000in}{0.000000in}}%
\pgfpathlineto{\pgfqpoint{0.000000in}{-0.048611in}}%
\pgfusepath{stroke,fill}%
}%
\begin{pgfscope}%
\pgfsys@transformshift{5.208684in}{0.535823in}%
\pgfsys@useobject{currentmarker}{}%
\end{pgfscope}%
\end{pgfscope}%
\begin{pgfscope}%
\definecolor{textcolor}{rgb}{0.000000,0.000000,0.000000}%
\pgfsetstrokecolor{textcolor}%
\pgfsetfillcolor{textcolor}%
\pgftext[x=5.208684in,y=0.438600in,,top]{\color{textcolor}\rmfamily\fontsize{9.000000}{10.800000}\selectfont \(\displaystyle {200}\)}%
\end{pgfscope}%
\begin{pgfscope}%
\definecolor{textcolor}{rgb}{0.000000,0.000000,0.000000}%
\pgfsetstrokecolor{textcolor}%
\pgfsetfillcolor{textcolor}%
\pgftext[x=3.279110in,y=0.272655in,,top]{\color{textcolor}\rmfamily\fontsize{10.000000}{12.000000}\selectfont \(\displaystyle n\): Number of vertices}%
\end{pgfscope}%
\begin{pgfscope}%
\pgfsetbuttcap%
\pgfsetroundjoin%
\definecolor{currentfill}{rgb}{0.000000,0.000000,0.000000}%
\pgfsetfillcolor{currentfill}%
\pgfsetlinewidth{0.803000pt}%
\definecolor{currentstroke}{rgb}{0.000000,0.000000,0.000000}%
\pgfsetstrokecolor{currentstroke}%
\pgfsetdash{}{0pt}%
\pgfsys@defobject{currentmarker}{\pgfqpoint{-0.048611in}{0.000000in}}{\pgfqpoint{-0.000000in}{0.000000in}}{%
\pgfpathmoveto{\pgfqpoint{-0.000000in}{0.000000in}}%
\pgfpathlineto{\pgfqpoint{-0.048611in}{0.000000in}}%
\pgfusepath{stroke,fill}%
}%
\begin{pgfscope}%
\pgfsys@transformshift{0.708220in}{0.535823in}%
\pgfsys@useobject{currentmarker}{}%
\end{pgfscope}%
\end{pgfscope}%
\begin{pgfscope}%
\definecolor{textcolor}{rgb}{0.000000,0.000000,0.000000}%
\pgfsetstrokecolor{textcolor}%
\pgfsetfillcolor{textcolor}%
\pgftext[x=0.344411in, y=0.491098in, left, base]{\color{textcolor}\rmfamily\fontsize{9.000000}{10.800000}\selectfont \(\displaystyle {10^{-1}}\)}%
\end{pgfscope}%
\begin{pgfscope}%
\pgfsetbuttcap%
\pgfsetroundjoin%
\definecolor{currentfill}{rgb}{0.000000,0.000000,0.000000}%
\pgfsetfillcolor{currentfill}%
\pgfsetlinewidth{0.803000pt}%
\definecolor{currentstroke}{rgb}{0.000000,0.000000,0.000000}%
\pgfsetstrokecolor{currentstroke}%
\pgfsetdash{}{0pt}%
\pgfsys@defobject{currentmarker}{\pgfqpoint{-0.048611in}{0.000000in}}{\pgfqpoint{-0.000000in}{0.000000in}}{%
\pgfpathmoveto{\pgfqpoint{-0.000000in}{0.000000in}}%
\pgfpathlineto{\pgfqpoint{-0.048611in}{0.000000in}}%
\pgfusepath{stroke,fill}%
}%
\begin{pgfscope}%
\pgfsys@transformshift{0.708220in}{1.203186in}%
\pgfsys@useobject{currentmarker}{}%
\end{pgfscope}%
\end{pgfscope}%
\begin{pgfscope}%
\definecolor{textcolor}{rgb}{0.000000,0.000000,0.000000}%
\pgfsetstrokecolor{textcolor}%
\pgfsetfillcolor{textcolor}%
\pgftext[x=0.424657in, y=1.158461in, left, base]{\color{textcolor}\rmfamily\fontsize{9.000000}{10.800000}\selectfont \(\displaystyle {10^{0}}\)}%
\end{pgfscope}%
\begin{pgfscope}%
\pgfsetbuttcap%
\pgfsetroundjoin%
\definecolor{currentfill}{rgb}{0.000000,0.000000,0.000000}%
\pgfsetfillcolor{currentfill}%
\pgfsetlinewidth{0.803000pt}%
\definecolor{currentstroke}{rgb}{0.000000,0.000000,0.000000}%
\pgfsetstrokecolor{currentstroke}%
\pgfsetdash{}{0pt}%
\pgfsys@defobject{currentmarker}{\pgfqpoint{-0.048611in}{0.000000in}}{\pgfqpoint{-0.000000in}{0.000000in}}{%
\pgfpathmoveto{\pgfqpoint{-0.000000in}{0.000000in}}%
\pgfpathlineto{\pgfqpoint{-0.048611in}{0.000000in}}%
\pgfusepath{stroke,fill}%
}%
\begin{pgfscope}%
\pgfsys@transformshift{0.708220in}{1.870549in}%
\pgfsys@useobject{currentmarker}{}%
\end{pgfscope}%
\end{pgfscope}%
\begin{pgfscope}%
\definecolor{textcolor}{rgb}{0.000000,0.000000,0.000000}%
\pgfsetstrokecolor{textcolor}%
\pgfsetfillcolor{textcolor}%
\pgftext[x=0.424657in, y=1.825824in, left, base]{\color{textcolor}\rmfamily\fontsize{9.000000}{10.800000}\selectfont \(\displaystyle {10^{1}}\)}%
\end{pgfscope}%
\begin{pgfscope}%
\pgfsetbuttcap%
\pgfsetroundjoin%
\definecolor{currentfill}{rgb}{0.000000,0.000000,0.000000}%
\pgfsetfillcolor{currentfill}%
\pgfsetlinewidth{0.803000pt}%
\definecolor{currentstroke}{rgb}{0.000000,0.000000,0.000000}%
\pgfsetstrokecolor{currentstroke}%
\pgfsetdash{}{0pt}%
\pgfsys@defobject{currentmarker}{\pgfqpoint{-0.048611in}{0.000000in}}{\pgfqpoint{-0.000000in}{0.000000in}}{%
\pgfpathmoveto{\pgfqpoint{-0.000000in}{0.000000in}}%
\pgfpathlineto{\pgfqpoint{-0.048611in}{0.000000in}}%
\pgfusepath{stroke,fill}%
}%
\begin{pgfscope}%
\pgfsys@transformshift{0.708220in}{2.537912in}%
\pgfsys@useobject{currentmarker}{}%
\end{pgfscope}%
\end{pgfscope}%
\begin{pgfscope}%
\definecolor{textcolor}{rgb}{0.000000,0.000000,0.000000}%
\pgfsetstrokecolor{textcolor}%
\pgfsetfillcolor{textcolor}%
\pgftext[x=0.424657in, y=2.493187in, left, base]{\color{textcolor}\rmfamily\fontsize{9.000000}{10.800000}\selectfont \(\displaystyle {10^{2}}\)}%
\end{pgfscope}%
\begin{pgfscope}%
\pgfsetbuttcap%
\pgfsetroundjoin%
\definecolor{currentfill}{rgb}{0.000000,0.000000,0.000000}%
\pgfsetfillcolor{currentfill}%
\pgfsetlinewidth{0.803000pt}%
\definecolor{currentstroke}{rgb}{0.000000,0.000000,0.000000}%
\pgfsetstrokecolor{currentstroke}%
\pgfsetdash{}{0pt}%
\pgfsys@defobject{currentmarker}{\pgfqpoint{-0.048611in}{0.000000in}}{\pgfqpoint{-0.000000in}{0.000000in}}{%
\pgfpathmoveto{\pgfqpoint{-0.000000in}{0.000000in}}%
\pgfpathlineto{\pgfqpoint{-0.048611in}{0.000000in}}%
\pgfusepath{stroke,fill}%
}%
\begin{pgfscope}%
\pgfsys@transformshift{0.708220in}{3.205275in}%
\pgfsys@useobject{currentmarker}{}%
\end{pgfscope}%
\end{pgfscope}%
\begin{pgfscope}%
\definecolor{textcolor}{rgb}{0.000000,0.000000,0.000000}%
\pgfsetstrokecolor{textcolor}%
\pgfsetfillcolor{textcolor}%
\pgftext[x=0.424657in, y=3.160550in, left, base]{\color{textcolor}\rmfamily\fontsize{9.000000}{10.800000}\selectfont \(\displaystyle {10^{3}}\)}%
\end{pgfscope}%
\begin{pgfscope}%
\pgfsetbuttcap%
\pgfsetroundjoin%
\definecolor{currentfill}{rgb}{0.000000,0.000000,0.000000}%
\pgfsetfillcolor{currentfill}%
\pgfsetlinewidth{0.602250pt}%
\definecolor{currentstroke}{rgb}{0.000000,0.000000,0.000000}%
\pgfsetstrokecolor{currentstroke}%
\pgfsetdash{}{0pt}%
\pgfsys@defobject{currentmarker}{\pgfqpoint{-0.027778in}{0.000000in}}{\pgfqpoint{-0.000000in}{0.000000in}}{%
\pgfpathmoveto{\pgfqpoint{-0.000000in}{0.000000in}}%
\pgfpathlineto{\pgfqpoint{-0.027778in}{0.000000in}}%
\pgfusepath{stroke,fill}%
}%
\begin{pgfscope}%
\pgfsys@transformshift{0.708220in}{0.736719in}%
\pgfsys@useobject{currentmarker}{}%
\end{pgfscope}%
\end{pgfscope}%
\begin{pgfscope}%
\pgfsetbuttcap%
\pgfsetroundjoin%
\definecolor{currentfill}{rgb}{0.000000,0.000000,0.000000}%
\pgfsetfillcolor{currentfill}%
\pgfsetlinewidth{0.602250pt}%
\definecolor{currentstroke}{rgb}{0.000000,0.000000,0.000000}%
\pgfsetstrokecolor{currentstroke}%
\pgfsetdash{}{0pt}%
\pgfsys@defobject{currentmarker}{\pgfqpoint{-0.027778in}{0.000000in}}{\pgfqpoint{-0.000000in}{0.000000in}}{%
\pgfpathmoveto{\pgfqpoint{-0.000000in}{0.000000in}}%
\pgfpathlineto{\pgfqpoint{-0.027778in}{0.000000in}}%
\pgfusepath{stroke,fill}%
}%
\begin{pgfscope}%
\pgfsys@transformshift{0.708220in}{0.854236in}%
\pgfsys@useobject{currentmarker}{}%
\end{pgfscope}%
\end{pgfscope}%
\begin{pgfscope}%
\pgfsetbuttcap%
\pgfsetroundjoin%
\definecolor{currentfill}{rgb}{0.000000,0.000000,0.000000}%
\pgfsetfillcolor{currentfill}%
\pgfsetlinewidth{0.602250pt}%
\definecolor{currentstroke}{rgb}{0.000000,0.000000,0.000000}%
\pgfsetstrokecolor{currentstroke}%
\pgfsetdash{}{0pt}%
\pgfsys@defobject{currentmarker}{\pgfqpoint{-0.027778in}{0.000000in}}{\pgfqpoint{-0.000000in}{0.000000in}}{%
\pgfpathmoveto{\pgfqpoint{-0.000000in}{0.000000in}}%
\pgfpathlineto{\pgfqpoint{-0.027778in}{0.000000in}}%
\pgfusepath{stroke,fill}%
}%
\begin{pgfscope}%
\pgfsys@transformshift{0.708220in}{0.937615in}%
\pgfsys@useobject{currentmarker}{}%
\end{pgfscope}%
\end{pgfscope}%
\begin{pgfscope}%
\pgfsetbuttcap%
\pgfsetroundjoin%
\definecolor{currentfill}{rgb}{0.000000,0.000000,0.000000}%
\pgfsetfillcolor{currentfill}%
\pgfsetlinewidth{0.602250pt}%
\definecolor{currentstroke}{rgb}{0.000000,0.000000,0.000000}%
\pgfsetstrokecolor{currentstroke}%
\pgfsetdash{}{0pt}%
\pgfsys@defobject{currentmarker}{\pgfqpoint{-0.027778in}{0.000000in}}{\pgfqpoint{-0.000000in}{0.000000in}}{%
\pgfpathmoveto{\pgfqpoint{-0.000000in}{0.000000in}}%
\pgfpathlineto{\pgfqpoint{-0.027778in}{0.000000in}}%
\pgfusepath{stroke,fill}%
}%
\begin{pgfscope}%
\pgfsys@transformshift{0.708220in}{1.002289in}%
\pgfsys@useobject{currentmarker}{}%
\end{pgfscope}%
\end{pgfscope}%
\begin{pgfscope}%
\pgfsetbuttcap%
\pgfsetroundjoin%
\definecolor{currentfill}{rgb}{0.000000,0.000000,0.000000}%
\pgfsetfillcolor{currentfill}%
\pgfsetlinewidth{0.602250pt}%
\definecolor{currentstroke}{rgb}{0.000000,0.000000,0.000000}%
\pgfsetstrokecolor{currentstroke}%
\pgfsetdash{}{0pt}%
\pgfsys@defobject{currentmarker}{\pgfqpoint{-0.027778in}{0.000000in}}{\pgfqpoint{-0.000000in}{0.000000in}}{%
\pgfpathmoveto{\pgfqpoint{-0.000000in}{0.000000in}}%
\pgfpathlineto{\pgfqpoint{-0.027778in}{0.000000in}}%
\pgfusepath{stroke,fill}%
}%
\begin{pgfscope}%
\pgfsys@transformshift{0.708220in}{1.055132in}%
\pgfsys@useobject{currentmarker}{}%
\end{pgfscope}%
\end{pgfscope}%
\begin{pgfscope}%
\pgfsetbuttcap%
\pgfsetroundjoin%
\definecolor{currentfill}{rgb}{0.000000,0.000000,0.000000}%
\pgfsetfillcolor{currentfill}%
\pgfsetlinewidth{0.602250pt}%
\definecolor{currentstroke}{rgb}{0.000000,0.000000,0.000000}%
\pgfsetstrokecolor{currentstroke}%
\pgfsetdash{}{0pt}%
\pgfsys@defobject{currentmarker}{\pgfqpoint{-0.027778in}{0.000000in}}{\pgfqpoint{-0.000000in}{0.000000in}}{%
\pgfpathmoveto{\pgfqpoint{-0.000000in}{0.000000in}}%
\pgfpathlineto{\pgfqpoint{-0.027778in}{0.000000in}}%
\pgfusepath{stroke,fill}%
}%
\begin{pgfscope}%
\pgfsys@transformshift{0.708220in}{1.099810in}%
\pgfsys@useobject{currentmarker}{}%
\end{pgfscope}%
\end{pgfscope}%
\begin{pgfscope}%
\pgfsetbuttcap%
\pgfsetroundjoin%
\definecolor{currentfill}{rgb}{0.000000,0.000000,0.000000}%
\pgfsetfillcolor{currentfill}%
\pgfsetlinewidth{0.602250pt}%
\definecolor{currentstroke}{rgb}{0.000000,0.000000,0.000000}%
\pgfsetstrokecolor{currentstroke}%
\pgfsetdash{}{0pt}%
\pgfsys@defobject{currentmarker}{\pgfqpoint{-0.027778in}{0.000000in}}{\pgfqpoint{-0.000000in}{0.000000in}}{%
\pgfpathmoveto{\pgfqpoint{-0.000000in}{0.000000in}}%
\pgfpathlineto{\pgfqpoint{-0.027778in}{0.000000in}}%
\pgfusepath{stroke,fill}%
}%
\begin{pgfscope}%
\pgfsys@transformshift{0.708220in}{1.138512in}%
\pgfsys@useobject{currentmarker}{}%
\end{pgfscope}%
\end{pgfscope}%
\begin{pgfscope}%
\pgfsetbuttcap%
\pgfsetroundjoin%
\definecolor{currentfill}{rgb}{0.000000,0.000000,0.000000}%
\pgfsetfillcolor{currentfill}%
\pgfsetlinewidth{0.602250pt}%
\definecolor{currentstroke}{rgb}{0.000000,0.000000,0.000000}%
\pgfsetstrokecolor{currentstroke}%
\pgfsetdash{}{0pt}%
\pgfsys@defobject{currentmarker}{\pgfqpoint{-0.027778in}{0.000000in}}{\pgfqpoint{-0.000000in}{0.000000in}}{%
\pgfpathmoveto{\pgfqpoint{-0.000000in}{0.000000in}}%
\pgfpathlineto{\pgfqpoint{-0.027778in}{0.000000in}}%
\pgfusepath{stroke,fill}%
}%
\begin{pgfscope}%
\pgfsys@transformshift{0.708220in}{1.172649in}%
\pgfsys@useobject{currentmarker}{}%
\end{pgfscope}%
\end{pgfscope}%
\begin{pgfscope}%
\pgfsetbuttcap%
\pgfsetroundjoin%
\definecolor{currentfill}{rgb}{0.000000,0.000000,0.000000}%
\pgfsetfillcolor{currentfill}%
\pgfsetlinewidth{0.602250pt}%
\definecolor{currentstroke}{rgb}{0.000000,0.000000,0.000000}%
\pgfsetstrokecolor{currentstroke}%
\pgfsetdash{}{0pt}%
\pgfsys@defobject{currentmarker}{\pgfqpoint{-0.027778in}{0.000000in}}{\pgfqpoint{-0.000000in}{0.000000in}}{%
\pgfpathmoveto{\pgfqpoint{-0.000000in}{0.000000in}}%
\pgfpathlineto{\pgfqpoint{-0.027778in}{0.000000in}}%
\pgfusepath{stroke,fill}%
}%
\begin{pgfscope}%
\pgfsys@transformshift{0.708220in}{1.404082in}%
\pgfsys@useobject{currentmarker}{}%
\end{pgfscope}%
\end{pgfscope}%
\begin{pgfscope}%
\pgfsetbuttcap%
\pgfsetroundjoin%
\definecolor{currentfill}{rgb}{0.000000,0.000000,0.000000}%
\pgfsetfillcolor{currentfill}%
\pgfsetlinewidth{0.602250pt}%
\definecolor{currentstroke}{rgb}{0.000000,0.000000,0.000000}%
\pgfsetstrokecolor{currentstroke}%
\pgfsetdash{}{0pt}%
\pgfsys@defobject{currentmarker}{\pgfqpoint{-0.027778in}{0.000000in}}{\pgfqpoint{-0.000000in}{0.000000in}}{%
\pgfpathmoveto{\pgfqpoint{-0.000000in}{0.000000in}}%
\pgfpathlineto{\pgfqpoint{-0.027778in}{0.000000in}}%
\pgfusepath{stroke,fill}%
}%
\begin{pgfscope}%
\pgfsys@transformshift{0.708220in}{1.521599in}%
\pgfsys@useobject{currentmarker}{}%
\end{pgfscope}%
\end{pgfscope}%
\begin{pgfscope}%
\pgfsetbuttcap%
\pgfsetroundjoin%
\definecolor{currentfill}{rgb}{0.000000,0.000000,0.000000}%
\pgfsetfillcolor{currentfill}%
\pgfsetlinewidth{0.602250pt}%
\definecolor{currentstroke}{rgb}{0.000000,0.000000,0.000000}%
\pgfsetstrokecolor{currentstroke}%
\pgfsetdash{}{0pt}%
\pgfsys@defobject{currentmarker}{\pgfqpoint{-0.027778in}{0.000000in}}{\pgfqpoint{-0.000000in}{0.000000in}}{%
\pgfpathmoveto{\pgfqpoint{-0.000000in}{0.000000in}}%
\pgfpathlineto{\pgfqpoint{-0.027778in}{0.000000in}}%
\pgfusepath{stroke,fill}%
}%
\begin{pgfscope}%
\pgfsys@transformshift{0.708220in}{1.604978in}%
\pgfsys@useobject{currentmarker}{}%
\end{pgfscope}%
\end{pgfscope}%
\begin{pgfscope}%
\pgfsetbuttcap%
\pgfsetroundjoin%
\definecolor{currentfill}{rgb}{0.000000,0.000000,0.000000}%
\pgfsetfillcolor{currentfill}%
\pgfsetlinewidth{0.602250pt}%
\definecolor{currentstroke}{rgb}{0.000000,0.000000,0.000000}%
\pgfsetstrokecolor{currentstroke}%
\pgfsetdash{}{0pt}%
\pgfsys@defobject{currentmarker}{\pgfqpoint{-0.027778in}{0.000000in}}{\pgfqpoint{-0.000000in}{0.000000in}}{%
\pgfpathmoveto{\pgfqpoint{-0.000000in}{0.000000in}}%
\pgfpathlineto{\pgfqpoint{-0.027778in}{0.000000in}}%
\pgfusepath{stroke,fill}%
}%
\begin{pgfscope}%
\pgfsys@transformshift{0.708220in}{1.669653in}%
\pgfsys@useobject{currentmarker}{}%
\end{pgfscope}%
\end{pgfscope}%
\begin{pgfscope}%
\pgfsetbuttcap%
\pgfsetroundjoin%
\definecolor{currentfill}{rgb}{0.000000,0.000000,0.000000}%
\pgfsetfillcolor{currentfill}%
\pgfsetlinewidth{0.602250pt}%
\definecolor{currentstroke}{rgb}{0.000000,0.000000,0.000000}%
\pgfsetstrokecolor{currentstroke}%
\pgfsetdash{}{0pt}%
\pgfsys@defobject{currentmarker}{\pgfqpoint{-0.027778in}{0.000000in}}{\pgfqpoint{-0.000000in}{0.000000in}}{%
\pgfpathmoveto{\pgfqpoint{-0.000000in}{0.000000in}}%
\pgfpathlineto{\pgfqpoint{-0.027778in}{0.000000in}}%
\pgfusepath{stroke,fill}%
}%
\begin{pgfscope}%
\pgfsys@transformshift{0.708220in}{1.722495in}%
\pgfsys@useobject{currentmarker}{}%
\end{pgfscope}%
\end{pgfscope}%
\begin{pgfscope}%
\pgfsetbuttcap%
\pgfsetroundjoin%
\definecolor{currentfill}{rgb}{0.000000,0.000000,0.000000}%
\pgfsetfillcolor{currentfill}%
\pgfsetlinewidth{0.602250pt}%
\definecolor{currentstroke}{rgb}{0.000000,0.000000,0.000000}%
\pgfsetstrokecolor{currentstroke}%
\pgfsetdash{}{0pt}%
\pgfsys@defobject{currentmarker}{\pgfqpoint{-0.027778in}{0.000000in}}{\pgfqpoint{-0.000000in}{0.000000in}}{%
\pgfpathmoveto{\pgfqpoint{-0.000000in}{0.000000in}}%
\pgfpathlineto{\pgfqpoint{-0.027778in}{0.000000in}}%
\pgfusepath{stroke,fill}%
}%
\begin{pgfscope}%
\pgfsys@transformshift{0.708220in}{1.767173in}%
\pgfsys@useobject{currentmarker}{}%
\end{pgfscope}%
\end{pgfscope}%
\begin{pgfscope}%
\pgfsetbuttcap%
\pgfsetroundjoin%
\definecolor{currentfill}{rgb}{0.000000,0.000000,0.000000}%
\pgfsetfillcolor{currentfill}%
\pgfsetlinewidth{0.602250pt}%
\definecolor{currentstroke}{rgb}{0.000000,0.000000,0.000000}%
\pgfsetstrokecolor{currentstroke}%
\pgfsetdash{}{0pt}%
\pgfsys@defobject{currentmarker}{\pgfqpoint{-0.027778in}{0.000000in}}{\pgfqpoint{-0.000000in}{0.000000in}}{%
\pgfpathmoveto{\pgfqpoint{-0.000000in}{0.000000in}}%
\pgfpathlineto{\pgfqpoint{-0.027778in}{0.000000in}}%
\pgfusepath{stroke,fill}%
}%
\begin{pgfscope}%
\pgfsys@transformshift{0.708220in}{1.805875in}%
\pgfsys@useobject{currentmarker}{}%
\end{pgfscope}%
\end{pgfscope}%
\begin{pgfscope}%
\pgfsetbuttcap%
\pgfsetroundjoin%
\definecolor{currentfill}{rgb}{0.000000,0.000000,0.000000}%
\pgfsetfillcolor{currentfill}%
\pgfsetlinewidth{0.602250pt}%
\definecolor{currentstroke}{rgb}{0.000000,0.000000,0.000000}%
\pgfsetstrokecolor{currentstroke}%
\pgfsetdash{}{0pt}%
\pgfsys@defobject{currentmarker}{\pgfqpoint{-0.027778in}{0.000000in}}{\pgfqpoint{-0.000000in}{0.000000in}}{%
\pgfpathmoveto{\pgfqpoint{-0.000000in}{0.000000in}}%
\pgfpathlineto{\pgfqpoint{-0.027778in}{0.000000in}}%
\pgfusepath{stroke,fill}%
}%
\begin{pgfscope}%
\pgfsys@transformshift{0.708220in}{1.840012in}%
\pgfsys@useobject{currentmarker}{}%
\end{pgfscope}%
\end{pgfscope}%
\begin{pgfscope}%
\pgfsetbuttcap%
\pgfsetroundjoin%
\definecolor{currentfill}{rgb}{0.000000,0.000000,0.000000}%
\pgfsetfillcolor{currentfill}%
\pgfsetlinewidth{0.602250pt}%
\definecolor{currentstroke}{rgb}{0.000000,0.000000,0.000000}%
\pgfsetstrokecolor{currentstroke}%
\pgfsetdash{}{0pt}%
\pgfsys@defobject{currentmarker}{\pgfqpoint{-0.027778in}{0.000000in}}{\pgfqpoint{-0.000000in}{0.000000in}}{%
\pgfpathmoveto{\pgfqpoint{-0.000000in}{0.000000in}}%
\pgfpathlineto{\pgfqpoint{-0.027778in}{0.000000in}}%
\pgfusepath{stroke,fill}%
}%
\begin{pgfscope}%
\pgfsys@transformshift{0.708220in}{2.071445in}%
\pgfsys@useobject{currentmarker}{}%
\end{pgfscope}%
\end{pgfscope}%
\begin{pgfscope}%
\pgfsetbuttcap%
\pgfsetroundjoin%
\definecolor{currentfill}{rgb}{0.000000,0.000000,0.000000}%
\pgfsetfillcolor{currentfill}%
\pgfsetlinewidth{0.602250pt}%
\definecolor{currentstroke}{rgb}{0.000000,0.000000,0.000000}%
\pgfsetstrokecolor{currentstroke}%
\pgfsetdash{}{0pt}%
\pgfsys@defobject{currentmarker}{\pgfqpoint{-0.027778in}{0.000000in}}{\pgfqpoint{-0.000000in}{0.000000in}}{%
\pgfpathmoveto{\pgfqpoint{-0.000000in}{0.000000in}}%
\pgfpathlineto{\pgfqpoint{-0.027778in}{0.000000in}}%
\pgfusepath{stroke,fill}%
}%
\begin{pgfscope}%
\pgfsys@transformshift{0.708220in}{2.188962in}%
\pgfsys@useobject{currentmarker}{}%
\end{pgfscope}%
\end{pgfscope}%
\begin{pgfscope}%
\pgfsetbuttcap%
\pgfsetroundjoin%
\definecolor{currentfill}{rgb}{0.000000,0.000000,0.000000}%
\pgfsetfillcolor{currentfill}%
\pgfsetlinewidth{0.602250pt}%
\definecolor{currentstroke}{rgb}{0.000000,0.000000,0.000000}%
\pgfsetstrokecolor{currentstroke}%
\pgfsetdash{}{0pt}%
\pgfsys@defobject{currentmarker}{\pgfqpoint{-0.027778in}{0.000000in}}{\pgfqpoint{-0.000000in}{0.000000in}}{%
\pgfpathmoveto{\pgfqpoint{-0.000000in}{0.000000in}}%
\pgfpathlineto{\pgfqpoint{-0.027778in}{0.000000in}}%
\pgfusepath{stroke,fill}%
}%
\begin{pgfscope}%
\pgfsys@transformshift{0.708220in}{2.272342in}%
\pgfsys@useobject{currentmarker}{}%
\end{pgfscope}%
\end{pgfscope}%
\begin{pgfscope}%
\pgfsetbuttcap%
\pgfsetroundjoin%
\definecolor{currentfill}{rgb}{0.000000,0.000000,0.000000}%
\pgfsetfillcolor{currentfill}%
\pgfsetlinewidth{0.602250pt}%
\definecolor{currentstroke}{rgb}{0.000000,0.000000,0.000000}%
\pgfsetstrokecolor{currentstroke}%
\pgfsetdash{}{0pt}%
\pgfsys@defobject{currentmarker}{\pgfqpoint{-0.027778in}{0.000000in}}{\pgfqpoint{-0.000000in}{0.000000in}}{%
\pgfpathmoveto{\pgfqpoint{-0.000000in}{0.000000in}}%
\pgfpathlineto{\pgfqpoint{-0.027778in}{0.000000in}}%
\pgfusepath{stroke,fill}%
}%
\begin{pgfscope}%
\pgfsys@transformshift{0.708220in}{2.337016in}%
\pgfsys@useobject{currentmarker}{}%
\end{pgfscope}%
\end{pgfscope}%
\begin{pgfscope}%
\pgfsetbuttcap%
\pgfsetroundjoin%
\definecolor{currentfill}{rgb}{0.000000,0.000000,0.000000}%
\pgfsetfillcolor{currentfill}%
\pgfsetlinewidth{0.602250pt}%
\definecolor{currentstroke}{rgb}{0.000000,0.000000,0.000000}%
\pgfsetstrokecolor{currentstroke}%
\pgfsetdash{}{0pt}%
\pgfsys@defobject{currentmarker}{\pgfqpoint{-0.027778in}{0.000000in}}{\pgfqpoint{-0.000000in}{0.000000in}}{%
\pgfpathmoveto{\pgfqpoint{-0.000000in}{0.000000in}}%
\pgfpathlineto{\pgfqpoint{-0.027778in}{0.000000in}}%
\pgfusepath{stroke,fill}%
}%
\begin{pgfscope}%
\pgfsys@transformshift{0.708220in}{2.389858in}%
\pgfsys@useobject{currentmarker}{}%
\end{pgfscope}%
\end{pgfscope}%
\begin{pgfscope}%
\pgfsetbuttcap%
\pgfsetroundjoin%
\definecolor{currentfill}{rgb}{0.000000,0.000000,0.000000}%
\pgfsetfillcolor{currentfill}%
\pgfsetlinewidth{0.602250pt}%
\definecolor{currentstroke}{rgb}{0.000000,0.000000,0.000000}%
\pgfsetstrokecolor{currentstroke}%
\pgfsetdash{}{0pt}%
\pgfsys@defobject{currentmarker}{\pgfqpoint{-0.027778in}{0.000000in}}{\pgfqpoint{-0.000000in}{0.000000in}}{%
\pgfpathmoveto{\pgfqpoint{-0.000000in}{0.000000in}}%
\pgfpathlineto{\pgfqpoint{-0.027778in}{0.000000in}}%
\pgfusepath{stroke,fill}%
}%
\begin{pgfscope}%
\pgfsys@transformshift{0.708220in}{2.434536in}%
\pgfsys@useobject{currentmarker}{}%
\end{pgfscope}%
\end{pgfscope}%
\begin{pgfscope}%
\pgfsetbuttcap%
\pgfsetroundjoin%
\definecolor{currentfill}{rgb}{0.000000,0.000000,0.000000}%
\pgfsetfillcolor{currentfill}%
\pgfsetlinewidth{0.602250pt}%
\definecolor{currentstroke}{rgb}{0.000000,0.000000,0.000000}%
\pgfsetstrokecolor{currentstroke}%
\pgfsetdash{}{0pt}%
\pgfsys@defobject{currentmarker}{\pgfqpoint{-0.027778in}{0.000000in}}{\pgfqpoint{-0.000000in}{0.000000in}}{%
\pgfpathmoveto{\pgfqpoint{-0.000000in}{0.000000in}}%
\pgfpathlineto{\pgfqpoint{-0.027778in}{0.000000in}}%
\pgfusepath{stroke,fill}%
}%
\begin{pgfscope}%
\pgfsys@transformshift{0.708220in}{2.473238in}%
\pgfsys@useobject{currentmarker}{}%
\end{pgfscope}%
\end{pgfscope}%
\begin{pgfscope}%
\pgfsetbuttcap%
\pgfsetroundjoin%
\definecolor{currentfill}{rgb}{0.000000,0.000000,0.000000}%
\pgfsetfillcolor{currentfill}%
\pgfsetlinewidth{0.602250pt}%
\definecolor{currentstroke}{rgb}{0.000000,0.000000,0.000000}%
\pgfsetstrokecolor{currentstroke}%
\pgfsetdash{}{0pt}%
\pgfsys@defobject{currentmarker}{\pgfqpoint{-0.027778in}{0.000000in}}{\pgfqpoint{-0.000000in}{0.000000in}}{%
\pgfpathmoveto{\pgfqpoint{-0.000000in}{0.000000in}}%
\pgfpathlineto{\pgfqpoint{-0.027778in}{0.000000in}}%
\pgfusepath{stroke,fill}%
}%
\begin{pgfscope}%
\pgfsys@transformshift{0.708220in}{2.507375in}%
\pgfsys@useobject{currentmarker}{}%
\end{pgfscope}%
\end{pgfscope}%
\begin{pgfscope}%
\pgfsetbuttcap%
\pgfsetroundjoin%
\definecolor{currentfill}{rgb}{0.000000,0.000000,0.000000}%
\pgfsetfillcolor{currentfill}%
\pgfsetlinewidth{0.602250pt}%
\definecolor{currentstroke}{rgb}{0.000000,0.000000,0.000000}%
\pgfsetstrokecolor{currentstroke}%
\pgfsetdash{}{0pt}%
\pgfsys@defobject{currentmarker}{\pgfqpoint{-0.027778in}{0.000000in}}{\pgfqpoint{-0.000000in}{0.000000in}}{%
\pgfpathmoveto{\pgfqpoint{-0.000000in}{0.000000in}}%
\pgfpathlineto{\pgfqpoint{-0.027778in}{0.000000in}}%
\pgfusepath{stroke,fill}%
}%
\begin{pgfscope}%
\pgfsys@transformshift{0.708220in}{2.738808in}%
\pgfsys@useobject{currentmarker}{}%
\end{pgfscope}%
\end{pgfscope}%
\begin{pgfscope}%
\pgfsetbuttcap%
\pgfsetroundjoin%
\definecolor{currentfill}{rgb}{0.000000,0.000000,0.000000}%
\pgfsetfillcolor{currentfill}%
\pgfsetlinewidth{0.602250pt}%
\definecolor{currentstroke}{rgb}{0.000000,0.000000,0.000000}%
\pgfsetstrokecolor{currentstroke}%
\pgfsetdash{}{0pt}%
\pgfsys@defobject{currentmarker}{\pgfqpoint{-0.027778in}{0.000000in}}{\pgfqpoint{-0.000000in}{0.000000in}}{%
\pgfpathmoveto{\pgfqpoint{-0.000000in}{0.000000in}}%
\pgfpathlineto{\pgfqpoint{-0.027778in}{0.000000in}}%
\pgfusepath{stroke,fill}%
}%
\begin{pgfscope}%
\pgfsys@transformshift{0.708220in}{2.856325in}%
\pgfsys@useobject{currentmarker}{}%
\end{pgfscope}%
\end{pgfscope}%
\begin{pgfscope}%
\pgfsetbuttcap%
\pgfsetroundjoin%
\definecolor{currentfill}{rgb}{0.000000,0.000000,0.000000}%
\pgfsetfillcolor{currentfill}%
\pgfsetlinewidth{0.602250pt}%
\definecolor{currentstroke}{rgb}{0.000000,0.000000,0.000000}%
\pgfsetstrokecolor{currentstroke}%
\pgfsetdash{}{0pt}%
\pgfsys@defobject{currentmarker}{\pgfqpoint{-0.027778in}{0.000000in}}{\pgfqpoint{-0.000000in}{0.000000in}}{%
\pgfpathmoveto{\pgfqpoint{-0.000000in}{0.000000in}}%
\pgfpathlineto{\pgfqpoint{-0.027778in}{0.000000in}}%
\pgfusepath{stroke,fill}%
}%
\begin{pgfscope}%
\pgfsys@transformshift{0.708220in}{2.939705in}%
\pgfsys@useobject{currentmarker}{}%
\end{pgfscope}%
\end{pgfscope}%
\begin{pgfscope}%
\pgfsetbuttcap%
\pgfsetroundjoin%
\definecolor{currentfill}{rgb}{0.000000,0.000000,0.000000}%
\pgfsetfillcolor{currentfill}%
\pgfsetlinewidth{0.602250pt}%
\definecolor{currentstroke}{rgb}{0.000000,0.000000,0.000000}%
\pgfsetstrokecolor{currentstroke}%
\pgfsetdash{}{0pt}%
\pgfsys@defobject{currentmarker}{\pgfqpoint{-0.027778in}{0.000000in}}{\pgfqpoint{-0.000000in}{0.000000in}}{%
\pgfpathmoveto{\pgfqpoint{-0.000000in}{0.000000in}}%
\pgfpathlineto{\pgfqpoint{-0.027778in}{0.000000in}}%
\pgfusepath{stroke,fill}%
}%
\begin{pgfscope}%
\pgfsys@transformshift{0.708220in}{3.004379in}%
\pgfsys@useobject{currentmarker}{}%
\end{pgfscope}%
\end{pgfscope}%
\begin{pgfscope}%
\pgfsetbuttcap%
\pgfsetroundjoin%
\definecolor{currentfill}{rgb}{0.000000,0.000000,0.000000}%
\pgfsetfillcolor{currentfill}%
\pgfsetlinewidth{0.602250pt}%
\definecolor{currentstroke}{rgb}{0.000000,0.000000,0.000000}%
\pgfsetstrokecolor{currentstroke}%
\pgfsetdash{}{0pt}%
\pgfsys@defobject{currentmarker}{\pgfqpoint{-0.027778in}{0.000000in}}{\pgfqpoint{-0.000000in}{0.000000in}}{%
\pgfpathmoveto{\pgfqpoint{-0.000000in}{0.000000in}}%
\pgfpathlineto{\pgfqpoint{-0.027778in}{0.000000in}}%
\pgfusepath{stroke,fill}%
}%
\begin{pgfscope}%
\pgfsys@transformshift{0.708220in}{3.057222in}%
\pgfsys@useobject{currentmarker}{}%
\end{pgfscope}%
\end{pgfscope}%
\begin{pgfscope}%
\pgfsetbuttcap%
\pgfsetroundjoin%
\definecolor{currentfill}{rgb}{0.000000,0.000000,0.000000}%
\pgfsetfillcolor{currentfill}%
\pgfsetlinewidth{0.602250pt}%
\definecolor{currentstroke}{rgb}{0.000000,0.000000,0.000000}%
\pgfsetstrokecolor{currentstroke}%
\pgfsetdash{}{0pt}%
\pgfsys@defobject{currentmarker}{\pgfqpoint{-0.027778in}{0.000000in}}{\pgfqpoint{-0.000000in}{0.000000in}}{%
\pgfpathmoveto{\pgfqpoint{-0.000000in}{0.000000in}}%
\pgfpathlineto{\pgfqpoint{-0.027778in}{0.000000in}}%
\pgfusepath{stroke,fill}%
}%
\begin{pgfscope}%
\pgfsys@transformshift{0.708220in}{3.101899in}%
\pgfsys@useobject{currentmarker}{}%
\end{pgfscope}%
\end{pgfscope}%
\begin{pgfscope}%
\pgfsetbuttcap%
\pgfsetroundjoin%
\definecolor{currentfill}{rgb}{0.000000,0.000000,0.000000}%
\pgfsetfillcolor{currentfill}%
\pgfsetlinewidth{0.602250pt}%
\definecolor{currentstroke}{rgb}{0.000000,0.000000,0.000000}%
\pgfsetstrokecolor{currentstroke}%
\pgfsetdash{}{0pt}%
\pgfsys@defobject{currentmarker}{\pgfqpoint{-0.027778in}{0.000000in}}{\pgfqpoint{-0.000000in}{0.000000in}}{%
\pgfpathmoveto{\pgfqpoint{-0.000000in}{0.000000in}}%
\pgfpathlineto{\pgfqpoint{-0.027778in}{0.000000in}}%
\pgfusepath{stroke,fill}%
}%
\begin{pgfscope}%
\pgfsys@transformshift{0.708220in}{3.140601in}%
\pgfsys@useobject{currentmarker}{}%
\end{pgfscope}%
\end{pgfscope}%
\begin{pgfscope}%
\pgfsetbuttcap%
\pgfsetroundjoin%
\definecolor{currentfill}{rgb}{0.000000,0.000000,0.000000}%
\pgfsetfillcolor{currentfill}%
\pgfsetlinewidth{0.602250pt}%
\definecolor{currentstroke}{rgb}{0.000000,0.000000,0.000000}%
\pgfsetstrokecolor{currentstroke}%
\pgfsetdash{}{0pt}%
\pgfsys@defobject{currentmarker}{\pgfqpoint{-0.027778in}{0.000000in}}{\pgfqpoint{-0.000000in}{0.000000in}}{%
\pgfpathmoveto{\pgfqpoint{-0.000000in}{0.000000in}}%
\pgfpathlineto{\pgfqpoint{-0.027778in}{0.000000in}}%
\pgfusepath{stroke,fill}%
}%
\begin{pgfscope}%
\pgfsys@transformshift{0.708220in}{3.174738in}%
\pgfsys@useobject{currentmarker}{}%
\end{pgfscope}%
\end{pgfscope}%
\begin{pgfscope}%
\definecolor{textcolor}{rgb}{0.000000,0.000000,0.000000}%
\pgfsetstrokecolor{textcolor}%
\pgfsetfillcolor{textcolor}%
\pgftext[x=0.288855in,y=1.870549in,,bottom,rotate=90.000000]{\color{textcolor}\rmfamily\fontsize{10.000000}{12.000000}\selectfont Median solving time (s)}%
\end{pgfscope}%
\begin{pgfscope}%
\pgfpathrectangle{\pgfqpoint{0.708220in}{0.535823in}}{\pgfqpoint{5.141780in}{2.669453in}}%
\pgfusepath{clip}%
\pgfsetrectcap%
\pgfsetroundjoin%
\pgfsetlinewidth{1.003750pt}%
\definecolor{currentstroke}{rgb}{0.866667,0.058824,0.058824}%
\pgfsetstrokecolor{currentstroke}%
\pgfsetdash{}{0pt}%
\pgfpathmoveto{\pgfqpoint{1.928026in}{0.525823in}}%
\pgfpathlineto{\pgfqpoint{2.058359in}{0.761651in}}%
\pgfpathlineto{\pgfqpoint{2.283382in}{1.194612in}}%
\pgfpathlineto{\pgfqpoint{2.508405in}{1.631965in}}%
\pgfpathlineto{\pgfqpoint{2.733429in}{2.079842in}}%
\pgfpathlineto{\pgfqpoint{2.958452in}{2.526097in}}%
\pgfpathlineto{\pgfqpoint{3.183475in}{3.000213in}}%
\pgfusepath{stroke}%
\end{pgfscope}%
\begin{pgfscope}%
\pgfpathrectangle{\pgfqpoint{0.708220in}{0.535823in}}{\pgfqpoint{5.141780in}{2.669453in}}%
\pgfusepath{clip}%
\pgfsetbuttcap%
\pgfsetmiterjoin%
\definecolor{currentfill}{rgb}{0.866667,0.058824,0.058824}%
\pgfsetfillcolor{currentfill}%
\pgfsetlinewidth{0.501875pt}%
\definecolor{currentstroke}{rgb}{0.000000,0.000000,0.000000}%
\pgfsetstrokecolor{currentstroke}%
\pgfsetdash{}{0pt}%
\pgfsys@defobject{currentmarker}{\pgfqpoint{-0.033023in}{-0.028091in}}{\pgfqpoint{0.033023in}{0.034722in}}{%
\pgfpathmoveto{\pgfqpoint{0.000000in}{0.034722in}}%
\pgfpathlineto{\pgfqpoint{-0.033023in}{0.010730in}}%
\pgfpathlineto{\pgfqpoint{-0.020409in}{-0.028091in}}%
\pgfpathlineto{\pgfqpoint{0.020409in}{-0.028091in}}%
\pgfpathlineto{\pgfqpoint{0.033023in}{0.010730in}}%
\pgfpathclose%
\pgfusepath{stroke,fill}%
}%
\begin{pgfscope}%
\pgfsys@transformshift{1.833336in}{0.354487in}%
\pgfsys@useobject{currentmarker}{}%
\end{pgfscope}%
\begin{pgfscope}%
\pgfsys@transformshift{2.058359in}{0.761651in}%
\pgfsys@useobject{currentmarker}{}%
\end{pgfscope}%
\begin{pgfscope}%
\pgfsys@transformshift{2.283382in}{1.194612in}%
\pgfsys@useobject{currentmarker}{}%
\end{pgfscope}%
\begin{pgfscope}%
\pgfsys@transformshift{2.508405in}{1.631965in}%
\pgfsys@useobject{currentmarker}{}%
\end{pgfscope}%
\begin{pgfscope}%
\pgfsys@transformshift{2.733429in}{2.079842in}%
\pgfsys@useobject{currentmarker}{}%
\end{pgfscope}%
\begin{pgfscope}%
\pgfsys@transformshift{2.958452in}{2.526097in}%
\pgfsys@useobject{currentmarker}{}%
\end{pgfscope}%
\begin{pgfscope}%
\pgfsys@transformshift{3.183475in}{3.000213in}%
\pgfsys@useobject{currentmarker}{}%
\end{pgfscope}%
\end{pgfscope}%
\begin{pgfscope}%
\pgfpathrectangle{\pgfqpoint{0.708220in}{0.535823in}}{\pgfqpoint{5.141780in}{2.669453in}}%
\pgfusepath{clip}%
\pgfsetrectcap%
\pgfsetroundjoin%
\pgfsetlinewidth{1.003750pt}%
\definecolor{currentstroke}{rgb}{0.000000,0.000000,0.200000}%
\pgfsetstrokecolor{currentstroke}%
\pgfsetdash{}{0pt}%
\pgfpathmoveto{\pgfqpoint{1.833336in}{0.676718in}}%
\pgfpathlineto{\pgfqpoint{2.058359in}{0.906415in}}%
\pgfpathlineto{\pgfqpoint{2.283382in}{1.295336in}}%
\pgfpathlineto{\pgfqpoint{2.508405in}{1.701898in}}%
\pgfpathlineto{\pgfqpoint{2.733429in}{1.970334in}}%
\pgfpathlineto{\pgfqpoint{2.958452in}{2.394586in}}%
\pgfpathlineto{\pgfqpoint{3.183475in}{2.793972in}}%
\pgfusepath{stroke}%
\end{pgfscope}%
\begin{pgfscope}%
\pgfpathrectangle{\pgfqpoint{0.708220in}{0.535823in}}{\pgfqpoint{5.141780in}{2.669453in}}%
\pgfusepath{clip}%
\pgfsetbuttcap%
\pgfsetmiterjoin%
\definecolor{currentfill}{rgb}{0.000000,0.000000,0.200000}%
\pgfsetfillcolor{currentfill}%
\pgfsetlinewidth{0.501875pt}%
\definecolor{currentstroke}{rgb}{0.000000,0.000000,0.000000}%
\pgfsetstrokecolor{currentstroke}%
\pgfsetdash{}{0pt}%
\pgfsys@defobject{currentmarker}{\pgfqpoint{-0.034722in}{-0.034722in}}{\pgfqpoint{0.034722in}{0.034722in}}{%
\pgfpathmoveto{\pgfqpoint{-0.011574in}{-0.034722in}}%
\pgfpathlineto{\pgfqpoint{0.011574in}{-0.034722in}}%
\pgfpathlineto{\pgfqpoint{0.011574in}{-0.011574in}}%
\pgfpathlineto{\pgfqpoint{0.034722in}{-0.011574in}}%
\pgfpathlineto{\pgfqpoint{0.034722in}{0.011574in}}%
\pgfpathlineto{\pgfqpoint{0.011574in}{0.011574in}}%
\pgfpathlineto{\pgfqpoint{0.011574in}{0.034722in}}%
\pgfpathlineto{\pgfqpoint{-0.011574in}{0.034722in}}%
\pgfpathlineto{\pgfqpoint{-0.011574in}{0.011574in}}%
\pgfpathlineto{\pgfqpoint{-0.034722in}{0.011574in}}%
\pgfpathlineto{\pgfqpoint{-0.034722in}{-0.011574in}}%
\pgfpathlineto{\pgfqpoint{-0.011574in}{-0.011574in}}%
\pgfpathclose%
\pgfusepath{stroke,fill}%
}%
\begin{pgfscope}%
\pgfsys@transformshift{1.833336in}{0.676718in}%
\pgfsys@useobject{currentmarker}{}%
\end{pgfscope}%
\begin{pgfscope}%
\pgfsys@transformshift{2.058359in}{0.906415in}%
\pgfsys@useobject{currentmarker}{}%
\end{pgfscope}%
\begin{pgfscope}%
\pgfsys@transformshift{2.283382in}{1.295336in}%
\pgfsys@useobject{currentmarker}{}%
\end{pgfscope}%
\begin{pgfscope}%
\pgfsys@transformshift{2.508405in}{1.701898in}%
\pgfsys@useobject{currentmarker}{}%
\end{pgfscope}%
\begin{pgfscope}%
\pgfsys@transformshift{2.733429in}{1.970334in}%
\pgfsys@useobject{currentmarker}{}%
\end{pgfscope}%
\begin{pgfscope}%
\pgfsys@transformshift{2.958452in}{2.394586in}%
\pgfsys@useobject{currentmarker}{}%
\end{pgfscope}%
\begin{pgfscope}%
\pgfsys@transformshift{3.183475in}{2.793972in}%
\pgfsys@useobject{currentmarker}{}%
\end{pgfscope}%
\end{pgfscope}%
\begin{pgfscope}%
\pgfpathrectangle{\pgfqpoint{0.708220in}{0.535823in}}{\pgfqpoint{5.141780in}{2.669453in}}%
\pgfusepath{clip}%
\pgfsetrectcap%
\pgfsetroundjoin%
\pgfsetlinewidth{1.003750pt}%
\definecolor{currentstroke}{rgb}{0.000000,0.000000,0.866667}%
\pgfsetstrokecolor{currentstroke}%
\pgfsetdash{}{0pt}%
\pgfpathmoveto{\pgfqpoint{1.833336in}{0.610115in}}%
\pgfpathlineto{\pgfqpoint{2.058359in}{0.767859in}}%
\pgfpathlineto{\pgfqpoint{2.283382in}{1.115023in}}%
\pgfpathlineto{\pgfqpoint{2.508405in}{1.546157in}}%
\pgfpathlineto{\pgfqpoint{2.733429in}{1.940281in}}%
\pgfpathlineto{\pgfqpoint{2.958452in}{2.294916in}}%
\pgfpathlineto{\pgfqpoint{3.183475in}{2.632491in}}%
\pgfpathlineto{\pgfqpoint{3.408498in}{3.043909in}}%
\pgfusepath{stroke}%
\end{pgfscope}%
\begin{pgfscope}%
\pgfpathrectangle{\pgfqpoint{0.708220in}{0.535823in}}{\pgfqpoint{5.141780in}{2.669453in}}%
\pgfusepath{clip}%
\pgfsetbuttcap%
\pgfsetmiterjoin%
\definecolor{currentfill}{rgb}{0.000000,0.000000,0.866667}%
\pgfsetfillcolor{currentfill}%
\pgfsetlinewidth{0.501875pt}%
\definecolor{currentstroke}{rgb}{0.000000,0.000000,0.000000}%
\pgfsetstrokecolor{currentstroke}%
\pgfsetdash{}{0pt}%
\pgfsys@defobject{currentmarker}{\pgfqpoint{-0.029463in}{-0.049105in}}{\pgfqpoint{0.029463in}{0.049105in}}{%
\pgfpathmoveto{\pgfqpoint{0.000000in}{-0.049105in}}%
\pgfpathlineto{\pgfqpoint{0.029463in}{0.000000in}}%
\pgfpathlineto{\pgfqpoint{0.000000in}{0.049105in}}%
\pgfpathlineto{\pgfqpoint{-0.029463in}{0.000000in}}%
\pgfpathclose%
\pgfusepath{stroke,fill}%
}%
\begin{pgfscope}%
\pgfsys@transformshift{1.833336in}{0.610115in}%
\pgfsys@useobject{currentmarker}{}%
\end{pgfscope}%
\begin{pgfscope}%
\pgfsys@transformshift{2.058359in}{0.767859in}%
\pgfsys@useobject{currentmarker}{}%
\end{pgfscope}%
\begin{pgfscope}%
\pgfsys@transformshift{2.283382in}{1.115023in}%
\pgfsys@useobject{currentmarker}{}%
\end{pgfscope}%
\begin{pgfscope}%
\pgfsys@transformshift{2.508405in}{1.546157in}%
\pgfsys@useobject{currentmarker}{}%
\end{pgfscope}%
\begin{pgfscope}%
\pgfsys@transformshift{2.733429in}{1.940281in}%
\pgfsys@useobject{currentmarker}{}%
\end{pgfscope}%
\begin{pgfscope}%
\pgfsys@transformshift{2.958452in}{2.294916in}%
\pgfsys@useobject{currentmarker}{}%
\end{pgfscope}%
\begin{pgfscope}%
\pgfsys@transformshift{3.183475in}{2.632491in}%
\pgfsys@useobject{currentmarker}{}%
\end{pgfscope}%
\begin{pgfscope}%
\pgfsys@transformshift{3.408498in}{3.043909in}%
\pgfsys@useobject{currentmarker}{}%
\end{pgfscope}%
\end{pgfscope}%
\begin{pgfscope}%
\pgfpathrectangle{\pgfqpoint{0.708220in}{0.535823in}}{\pgfqpoint{5.141780in}{2.669453in}}%
\pgfusepath{clip}%
\pgfsetrectcap%
\pgfsetroundjoin%
\pgfsetlinewidth{1.003750pt}%
\definecolor{currentstroke}{rgb}{0.250980,0.231373,0.796078}%
\pgfsetstrokecolor{currentstroke}%
\pgfsetdash{}{0pt}%
\pgfpathmoveto{\pgfqpoint{2.095649in}{0.525823in}}%
\pgfpathlineto{\pgfqpoint{2.283382in}{0.846558in}}%
\pgfpathlineto{\pgfqpoint{2.508405in}{1.257717in}}%
\pgfpathlineto{\pgfqpoint{2.733429in}{1.657480in}}%
\pgfpathlineto{\pgfqpoint{2.958452in}{2.073652in}}%
\pgfpathlineto{\pgfqpoint{3.183475in}{2.469956in}}%
\pgfpathlineto{\pgfqpoint{3.408498in}{2.885012in}}%
\pgfusepath{stroke}%
\end{pgfscope}%
\begin{pgfscope}%
\pgfpathrectangle{\pgfqpoint{0.708220in}{0.535823in}}{\pgfqpoint{5.141780in}{2.669453in}}%
\pgfusepath{clip}%
\pgfsetbuttcap%
\pgfsetmiterjoin%
\definecolor{currentfill}{rgb}{0.250980,0.231373,0.796078}%
\pgfsetfillcolor{currentfill}%
\pgfsetlinewidth{0.501875pt}%
\definecolor{currentstroke}{rgb}{0.000000,0.000000,0.000000}%
\pgfsetstrokecolor{currentstroke}%
\pgfsetdash{}{0pt}%
\pgfsys@defobject{currentmarker}{\pgfqpoint{-0.034722in}{-0.034722in}}{\pgfqpoint{0.034722in}{0.034722in}}{%
\pgfpathmoveto{\pgfqpoint{-0.034722in}{-0.034722in}}%
\pgfpathlineto{\pgfqpoint{0.034722in}{-0.034722in}}%
\pgfpathlineto{\pgfqpoint{0.034722in}{0.034722in}}%
\pgfpathlineto{\pgfqpoint{-0.034722in}{0.034722in}}%
\pgfpathclose%
\pgfusepath{stroke,fill}%
}%
\begin{pgfscope}%
\pgfsys@transformshift{1.833336in}{0.132041in}%
\pgfsys@useobject{currentmarker}{}%
\end{pgfscope}%
\begin{pgfscope}%
\pgfsys@transformshift{2.058359in}{0.462115in}%
\pgfsys@useobject{currentmarker}{}%
\end{pgfscope}%
\begin{pgfscope}%
\pgfsys@transformshift{2.283382in}{0.846558in}%
\pgfsys@useobject{currentmarker}{}%
\end{pgfscope}%
\begin{pgfscope}%
\pgfsys@transformshift{2.508405in}{1.257717in}%
\pgfsys@useobject{currentmarker}{}%
\end{pgfscope}%
\begin{pgfscope}%
\pgfsys@transformshift{2.733429in}{1.657480in}%
\pgfsys@useobject{currentmarker}{}%
\end{pgfscope}%
\begin{pgfscope}%
\pgfsys@transformshift{2.958452in}{2.073652in}%
\pgfsys@useobject{currentmarker}{}%
\end{pgfscope}%
\begin{pgfscope}%
\pgfsys@transformshift{3.183475in}{2.469956in}%
\pgfsys@useobject{currentmarker}{}%
\end{pgfscope}%
\begin{pgfscope}%
\pgfsys@transformshift{3.408498in}{2.885012in}%
\pgfsys@useobject{currentmarker}{}%
\end{pgfscope}%
\end{pgfscope}%
\begin{pgfscope}%
\pgfpathrectangle{\pgfqpoint{0.708220in}{0.535823in}}{\pgfqpoint{5.141780in}{2.669453in}}%
\pgfusepath{clip}%
\pgfsetrectcap%
\pgfsetroundjoin%
\pgfsetlinewidth{1.003750pt}%
\definecolor{currentstroke}{rgb}{0.615686,0.007843,0.843137}%
\pgfsetstrokecolor{currentstroke}%
\pgfsetdash{}{0pt}%
\pgfpathmoveto{\pgfqpoint{2.027184in}{0.525823in}}%
\pgfpathlineto{\pgfqpoint{2.058359in}{0.571199in}}%
\pgfpathlineto{\pgfqpoint{2.283382in}{0.866486in}}%
\pgfpathlineto{\pgfqpoint{2.508405in}{1.219071in}}%
\pgfpathlineto{\pgfqpoint{2.733429in}{1.535650in}}%
\pgfpathlineto{\pgfqpoint{2.958452in}{1.852695in}}%
\pgfpathlineto{\pgfqpoint{3.183475in}{2.103243in}}%
\pgfpathlineto{\pgfqpoint{3.408498in}{2.465875in}}%
\pgfpathlineto{\pgfqpoint{3.633521in}{2.755438in}}%
\pgfpathlineto{\pgfqpoint{3.858545in}{3.018596in}}%
\pgfusepath{stroke}%
\end{pgfscope}%
\begin{pgfscope}%
\pgfpathrectangle{\pgfqpoint{0.708220in}{0.535823in}}{\pgfqpoint{5.141780in}{2.669453in}}%
\pgfusepath{clip}%
\pgfsetbuttcap%
\pgfsetroundjoin%
\definecolor{currentfill}{rgb}{0.615686,0.007843,0.843137}%
\pgfsetfillcolor{currentfill}%
\pgfsetlinewidth{0.501875pt}%
\definecolor{currentstroke}{rgb}{0.000000,0.000000,0.000000}%
\pgfsetstrokecolor{currentstroke}%
\pgfsetdash{}{0pt}%
\pgfsys@defobject{currentmarker}{\pgfqpoint{-0.034722in}{-0.034722in}}{\pgfqpoint{0.034722in}{0.034722in}}{%
\pgfpathmoveto{\pgfqpoint{0.000000in}{-0.034722in}}%
\pgfpathcurveto{\pgfqpoint{0.009208in}{-0.034722in}}{\pgfqpoint{0.018041in}{-0.031064in}}{\pgfqpoint{0.024552in}{-0.024552in}}%
\pgfpathcurveto{\pgfqpoint{0.031064in}{-0.018041in}}{\pgfqpoint{0.034722in}{-0.009208in}}{\pgfqpoint{0.034722in}{0.000000in}}%
\pgfpathcurveto{\pgfqpoint{0.034722in}{0.009208in}}{\pgfqpoint{0.031064in}{0.018041in}}{\pgfqpoint{0.024552in}{0.024552in}}%
\pgfpathcurveto{\pgfqpoint{0.018041in}{0.031064in}}{\pgfqpoint{0.009208in}{0.034722in}}{\pgfqpoint{0.000000in}{0.034722in}}%
\pgfpathcurveto{\pgfqpoint{-0.009208in}{0.034722in}}{\pgfqpoint{-0.018041in}{0.031064in}}{\pgfqpoint{-0.024552in}{0.024552in}}%
\pgfpathcurveto{\pgfqpoint{-0.031064in}{0.018041in}}{\pgfqpoint{-0.034722in}{0.009208in}}{\pgfqpoint{-0.034722in}{0.000000in}}%
\pgfpathcurveto{\pgfqpoint{-0.034722in}{-0.009208in}}{\pgfqpoint{-0.031064in}{-0.018041in}}{\pgfqpoint{-0.024552in}{-0.024552in}}%
\pgfpathcurveto{\pgfqpoint{-0.018041in}{-0.031064in}}{\pgfqpoint{-0.009208in}{-0.034722in}}{\pgfqpoint{0.000000in}{-0.034722in}}%
\pgfpathclose%
\pgfusepath{stroke,fill}%
}%
\begin{pgfscope}%
\pgfsys@transformshift{1.833336in}{0.243665in}%
\pgfsys@useobject{currentmarker}{}%
\end{pgfscope}%
\begin{pgfscope}%
\pgfsys@transformshift{2.058359in}{0.571199in}%
\pgfsys@useobject{currentmarker}{}%
\end{pgfscope}%
\begin{pgfscope}%
\pgfsys@transformshift{2.283382in}{0.866486in}%
\pgfsys@useobject{currentmarker}{}%
\end{pgfscope}%
\begin{pgfscope}%
\pgfsys@transformshift{2.508405in}{1.219071in}%
\pgfsys@useobject{currentmarker}{}%
\end{pgfscope}%
\begin{pgfscope}%
\pgfsys@transformshift{2.733429in}{1.535650in}%
\pgfsys@useobject{currentmarker}{}%
\end{pgfscope}%
\begin{pgfscope}%
\pgfsys@transformshift{2.958452in}{1.852695in}%
\pgfsys@useobject{currentmarker}{}%
\end{pgfscope}%
\begin{pgfscope}%
\pgfsys@transformshift{3.183475in}{2.103243in}%
\pgfsys@useobject{currentmarker}{}%
\end{pgfscope}%
\begin{pgfscope}%
\pgfsys@transformshift{3.408498in}{2.465875in}%
\pgfsys@useobject{currentmarker}{}%
\end{pgfscope}%
\begin{pgfscope}%
\pgfsys@transformshift{3.633521in}{2.755438in}%
\pgfsys@useobject{currentmarker}{}%
\end{pgfscope}%
\begin{pgfscope}%
\pgfsys@transformshift{3.858545in}{3.018596in}%
\pgfsys@useobject{currentmarker}{}%
\end{pgfscope}%
\end{pgfscope}%
\begin{pgfscope}%
\pgfpathrectangle{\pgfqpoint{0.708220in}{0.535823in}}{\pgfqpoint{5.141780in}{2.669453in}}%
\pgfusepath{clip}%
\pgfsetrectcap%
\pgfsetroundjoin%
\pgfsetlinewidth{1.003750pt}%
\definecolor{currentstroke}{rgb}{0.917647,0.372549,0.580392}%
\pgfsetstrokecolor{currentstroke}%
\pgfsetdash{}{0pt}%
\pgfpathmoveto{\pgfqpoint{2.685237in}{0.525823in}}%
\pgfpathlineto{\pgfqpoint{2.733429in}{0.563447in}}%
\pgfpathlineto{\pgfqpoint{2.958452in}{0.736719in}}%
\pgfpathlineto{\pgfqpoint{3.183475in}{0.978123in}}%
\pgfpathlineto{\pgfqpoint{3.408498in}{1.305836in}}%
\pgfpathlineto{\pgfqpoint{3.633521in}{1.559575in}}%
\pgfpathlineto{\pgfqpoint{3.858545in}{1.866610in}}%
\pgfpathlineto{\pgfqpoint{4.083568in}{2.126813in}}%
\pgfpathlineto{\pgfqpoint{4.308591in}{2.390895in}}%
\pgfpathlineto{\pgfqpoint{4.533614in}{2.643174in}}%
\pgfpathlineto{\pgfqpoint{4.758637in}{3.041851in}}%
\pgfusepath{stroke}%
\end{pgfscope}%
\begin{pgfscope}%
\pgfpathrectangle{\pgfqpoint{0.708220in}{0.535823in}}{\pgfqpoint{5.141780in}{2.669453in}}%
\pgfusepath{clip}%
\pgfsetbuttcap%
\pgfsetmiterjoin%
\definecolor{currentfill}{rgb}{0.917647,0.372549,0.580392}%
\pgfsetfillcolor{currentfill}%
\pgfsetlinewidth{0.501875pt}%
\definecolor{currentstroke}{rgb}{0.000000,0.000000,0.000000}%
\pgfsetstrokecolor{currentstroke}%
\pgfsetdash{}{0pt}%
\pgfsys@defobject{currentmarker}{\pgfqpoint{-0.049105in}{-0.049105in}}{\pgfqpoint{0.049105in}{0.049105in}}{%
\pgfpathmoveto{\pgfqpoint{0.000000in}{-0.049105in}}%
\pgfpathlineto{\pgfqpoint{0.049105in}{0.000000in}}%
\pgfpathlineto{\pgfqpoint{0.000000in}{0.049105in}}%
\pgfpathlineto{\pgfqpoint{-0.049105in}{0.000000in}}%
\pgfpathclose%
\pgfusepath{stroke,fill}%
}%
\begin{pgfscope}%
\pgfsys@transformshift{1.833336in}{0.270252in}%
\pgfsys@useobject{currentmarker}{}%
\end{pgfscope}%
\begin{pgfscope}%
\pgfsys@transformshift{2.058359in}{0.270252in}%
\pgfsys@useobject{currentmarker}{}%
\end{pgfscope}%
\begin{pgfscope}%
\pgfsys@transformshift{2.283382in}{0.270252in}%
\pgfsys@useobject{currentmarker}{}%
\end{pgfscope}%
\begin{pgfscope}%
\pgfsys@transformshift{2.508405in}{0.387769in}%
\pgfsys@useobject{currentmarker}{}%
\end{pgfscope}%
\begin{pgfscope}%
\pgfsys@transformshift{2.733429in}{0.563447in}%
\pgfsys@useobject{currentmarker}{}%
\end{pgfscope}%
\begin{pgfscope}%
\pgfsys@transformshift{2.958452in}{0.736719in}%
\pgfsys@useobject{currentmarker}{}%
\end{pgfscope}%
\begin{pgfscope}%
\pgfsys@transformshift{3.183475in}{0.978123in}%
\pgfsys@useobject{currentmarker}{}%
\end{pgfscope}%
\begin{pgfscope}%
\pgfsys@transformshift{3.408498in}{1.305836in}%
\pgfsys@useobject{currentmarker}{}%
\end{pgfscope}%
\begin{pgfscope}%
\pgfsys@transformshift{3.633521in}{1.559575in}%
\pgfsys@useobject{currentmarker}{}%
\end{pgfscope}%
\begin{pgfscope}%
\pgfsys@transformshift{3.858545in}{1.866610in}%
\pgfsys@useobject{currentmarker}{}%
\end{pgfscope}%
\begin{pgfscope}%
\pgfsys@transformshift{4.083568in}{2.126813in}%
\pgfsys@useobject{currentmarker}{}%
\end{pgfscope}%
\begin{pgfscope}%
\pgfsys@transformshift{4.308591in}{2.390895in}%
\pgfsys@useobject{currentmarker}{}%
\end{pgfscope}%
\begin{pgfscope}%
\pgfsys@transformshift{4.533614in}{2.643174in}%
\pgfsys@useobject{currentmarker}{}%
\end{pgfscope}%
\begin{pgfscope}%
\pgfsys@transformshift{4.758637in}{3.041851in}%
\pgfsys@useobject{currentmarker}{}%
\end{pgfscope}%
\end{pgfscope}%
\begin{pgfscope}%
\pgfpathrectangle{\pgfqpoint{0.708220in}{0.535823in}}{\pgfqpoint{5.141780in}{2.669453in}}%
\pgfusepath{clip}%
\pgfsetrectcap%
\pgfsetroundjoin%
\pgfsetlinewidth{1.003750pt}%
\definecolor{currentstroke}{rgb}{0.529412,0.462745,0.384314}%
\pgfsetstrokecolor{currentstroke}%
\pgfsetdash{}{0pt}%
\pgfpathmoveto{\pgfqpoint{1.920185in}{0.525823in}}%
\pgfpathlineto{\pgfqpoint{2.058359in}{0.573473in}}%
\pgfpathlineto{\pgfqpoint{2.283382in}{0.647919in}}%
\pgfpathlineto{\pgfqpoint{2.508405in}{0.725868in}}%
\pgfpathlineto{\pgfqpoint{2.733429in}{0.798513in}}%
\pgfpathlineto{\pgfqpoint{2.958452in}{0.899271in}}%
\pgfpathlineto{\pgfqpoint{3.183475in}{1.039475in}}%
\pgfpathlineto{\pgfqpoint{3.408498in}{1.399329in}}%
\pgfpathlineto{\pgfqpoint{3.633521in}{1.819793in}}%
\pgfpathlineto{\pgfqpoint{3.858545in}{2.283030in}}%
\pgfpathlineto{\pgfqpoint{4.083568in}{2.802781in}}%
\pgfusepath{stroke}%
\end{pgfscope}%
\begin{pgfscope}%
\pgfpathrectangle{\pgfqpoint{0.708220in}{0.535823in}}{\pgfqpoint{5.141780in}{2.669453in}}%
\pgfusepath{clip}%
\pgfsetbuttcap%
\pgfsetmiterjoin%
\definecolor{currentfill}{rgb}{0.529412,0.462745,0.384314}%
\pgfsetfillcolor{currentfill}%
\pgfsetlinewidth{0.501875pt}%
\definecolor{currentstroke}{rgb}{0.000000,0.000000,0.000000}%
\pgfsetstrokecolor{currentstroke}%
\pgfsetdash{}{0pt}%
\pgfsys@defobject{currentmarker}{\pgfqpoint{-0.034722in}{-0.034722in}}{\pgfqpoint{0.034722in}{0.034722in}}{%
\pgfpathmoveto{\pgfqpoint{-0.000000in}{-0.034722in}}%
\pgfpathlineto{\pgfqpoint{0.034722in}{0.034722in}}%
\pgfpathlineto{\pgfqpoint{-0.034722in}{0.034722in}}%
\pgfpathclose%
\pgfusepath{stroke,fill}%
}%
\begin{pgfscope}%
\pgfsys@transformshift{1.833336in}{0.495872in}%
\pgfsys@useobject{currentmarker}{}%
\end{pgfscope}%
\begin{pgfscope}%
\pgfsys@transformshift{2.058359in}{0.573473in}%
\pgfsys@useobject{currentmarker}{}%
\end{pgfscope}%
\begin{pgfscope}%
\pgfsys@transformshift{2.283382in}{0.647919in}%
\pgfsys@useobject{currentmarker}{}%
\end{pgfscope}%
\begin{pgfscope}%
\pgfsys@transformshift{2.508405in}{0.725868in}%
\pgfsys@useobject{currentmarker}{}%
\end{pgfscope}%
\begin{pgfscope}%
\pgfsys@transformshift{2.733429in}{0.798513in}%
\pgfsys@useobject{currentmarker}{}%
\end{pgfscope}%
\begin{pgfscope}%
\pgfsys@transformshift{2.958452in}{0.899271in}%
\pgfsys@useobject{currentmarker}{}%
\end{pgfscope}%
\begin{pgfscope}%
\pgfsys@transformshift{3.183475in}{1.039475in}%
\pgfsys@useobject{currentmarker}{}%
\end{pgfscope}%
\begin{pgfscope}%
\pgfsys@transformshift{3.408498in}{1.399329in}%
\pgfsys@useobject{currentmarker}{}%
\end{pgfscope}%
\begin{pgfscope}%
\pgfsys@transformshift{3.633521in}{1.819793in}%
\pgfsys@useobject{currentmarker}{}%
\end{pgfscope}%
\begin{pgfscope}%
\pgfsys@transformshift{3.858545in}{2.283030in}%
\pgfsys@useobject{currentmarker}{}%
\end{pgfscope}%
\begin{pgfscope}%
\pgfsys@transformshift{4.083568in}{2.802781in}%
\pgfsys@useobject{currentmarker}{}%
\end{pgfscope}%
\end{pgfscope}%
\begin{pgfscope}%
\pgfpathrectangle{\pgfqpoint{0.708220in}{0.535823in}}{\pgfqpoint{5.141780in}{2.669453in}}%
\pgfusepath{clip}%
\pgfsetrectcap%
\pgfsetroundjoin%
\pgfsetlinewidth{1.003750pt}%
\definecolor{currentstroke}{rgb}{0.611765,0.568627,0.274510}%
\pgfsetstrokecolor{currentstroke}%
\pgfsetdash{}{0pt}%
\pgfpathmoveto{\pgfqpoint{1.833336in}{0.533110in}}%
\pgfpathlineto{\pgfqpoint{2.058359in}{0.600165in}}%
\pgfpathlineto{\pgfqpoint{2.283382in}{0.656378in}}%
\pgfpathlineto{\pgfqpoint{2.508405in}{0.708555in}}%
\pgfpathlineto{\pgfqpoint{2.733429in}{0.755478in}}%
\pgfpathlineto{\pgfqpoint{2.958452in}{0.800865in}}%
\pgfpathlineto{\pgfqpoint{3.183475in}{0.845169in}}%
\pgfpathlineto{\pgfqpoint{3.408498in}{0.901495in}}%
\pgfpathlineto{\pgfqpoint{3.633521in}{0.983805in}}%
\pgfpathlineto{\pgfqpoint{3.858545in}{1.134783in}}%
\pgfpathlineto{\pgfqpoint{4.083568in}{1.319608in}}%
\pgfpathlineto{\pgfqpoint{4.308591in}{1.587480in}}%
\pgfpathlineto{\pgfqpoint{4.533614in}{1.922967in}}%
\pgfpathlineto{\pgfqpoint{4.758637in}{2.377229in}}%
\pgfpathlineto{\pgfqpoint{4.983661in}{2.606679in}}%
\pgfpathlineto{\pgfqpoint{5.208684in}{3.081033in}}%
\pgfusepath{stroke}%
\end{pgfscope}%
\begin{pgfscope}%
\pgfpathrectangle{\pgfqpoint{0.708220in}{0.535823in}}{\pgfqpoint{5.141780in}{2.669453in}}%
\pgfusepath{clip}%
\pgfsetbuttcap%
\pgfsetmiterjoin%
\definecolor{currentfill}{rgb}{0.611765,0.568627,0.274510}%
\pgfsetfillcolor{currentfill}%
\pgfsetlinewidth{0.501875pt}%
\definecolor{currentstroke}{rgb}{0.000000,0.000000,0.000000}%
\pgfsetstrokecolor{currentstroke}%
\pgfsetdash{}{0pt}%
\pgfsys@defobject{currentmarker}{\pgfqpoint{-0.034722in}{-0.034722in}}{\pgfqpoint{0.034722in}{0.034722in}}{%
\pgfpathmoveto{\pgfqpoint{-0.034722in}{0.000000in}}%
\pgfpathlineto{\pgfqpoint{0.034722in}{-0.034722in}}%
\pgfpathlineto{\pgfqpoint{0.034722in}{0.034722in}}%
\pgfpathclose%
\pgfusepath{stroke,fill}%
}%
\begin{pgfscope}%
\pgfsys@transformshift{1.833336in}{0.533110in}%
\pgfsys@useobject{currentmarker}{}%
\end{pgfscope}%
\begin{pgfscope}%
\pgfsys@transformshift{2.058359in}{0.600165in}%
\pgfsys@useobject{currentmarker}{}%
\end{pgfscope}%
\begin{pgfscope}%
\pgfsys@transformshift{2.283382in}{0.656378in}%
\pgfsys@useobject{currentmarker}{}%
\end{pgfscope}%
\begin{pgfscope}%
\pgfsys@transformshift{2.508405in}{0.708555in}%
\pgfsys@useobject{currentmarker}{}%
\end{pgfscope}%
\begin{pgfscope}%
\pgfsys@transformshift{2.733429in}{0.755478in}%
\pgfsys@useobject{currentmarker}{}%
\end{pgfscope}%
\begin{pgfscope}%
\pgfsys@transformshift{2.958452in}{0.800865in}%
\pgfsys@useobject{currentmarker}{}%
\end{pgfscope}%
\begin{pgfscope}%
\pgfsys@transformshift{3.183475in}{0.845169in}%
\pgfsys@useobject{currentmarker}{}%
\end{pgfscope}%
\begin{pgfscope}%
\pgfsys@transformshift{3.408498in}{0.901495in}%
\pgfsys@useobject{currentmarker}{}%
\end{pgfscope}%
\begin{pgfscope}%
\pgfsys@transformshift{3.633521in}{0.983805in}%
\pgfsys@useobject{currentmarker}{}%
\end{pgfscope}%
\begin{pgfscope}%
\pgfsys@transformshift{3.858545in}{1.134783in}%
\pgfsys@useobject{currentmarker}{}%
\end{pgfscope}%
\begin{pgfscope}%
\pgfsys@transformshift{4.083568in}{1.319608in}%
\pgfsys@useobject{currentmarker}{}%
\end{pgfscope}%
\begin{pgfscope}%
\pgfsys@transformshift{4.308591in}{1.587480in}%
\pgfsys@useobject{currentmarker}{}%
\end{pgfscope}%
\begin{pgfscope}%
\pgfsys@transformshift{4.533614in}{1.922967in}%
\pgfsys@useobject{currentmarker}{}%
\end{pgfscope}%
\begin{pgfscope}%
\pgfsys@transformshift{4.758637in}{2.377229in}%
\pgfsys@useobject{currentmarker}{}%
\end{pgfscope}%
\begin{pgfscope}%
\pgfsys@transformshift{4.983661in}{2.606679in}%
\pgfsys@useobject{currentmarker}{}%
\end{pgfscope}%
\begin{pgfscope}%
\pgfsys@transformshift{5.208684in}{3.081033in}%
\pgfsys@useobject{currentmarker}{}%
\end{pgfscope}%
\end{pgfscope}%
\begin{pgfscope}%
\pgfpathrectangle{\pgfqpoint{0.708220in}{0.535823in}}{\pgfqpoint{5.141780in}{2.669453in}}%
\pgfusepath{clip}%
\pgfsetrectcap%
\pgfsetroundjoin%
\pgfsetlinewidth{1.003750pt}%
\definecolor{currentstroke}{rgb}{0.780392,0.643137,0.254902}%
\pgfsetstrokecolor{currentstroke}%
\pgfsetdash{}{0pt}%
\pgfpathmoveto{\pgfqpoint{1.833336in}{0.607768in}}%
\pgfpathlineto{\pgfqpoint{2.058359in}{0.714594in}}%
\pgfpathlineto{\pgfqpoint{2.283382in}{0.810268in}}%
\pgfpathlineto{\pgfqpoint{2.508405in}{0.899028in}}%
\pgfpathlineto{\pgfqpoint{2.733429in}{0.976962in}}%
\pgfpathlineto{\pgfqpoint{2.958452in}{1.048705in}}%
\pgfpathlineto{\pgfqpoint{3.183475in}{1.117476in}}%
\pgfpathlineto{\pgfqpoint{3.408498in}{1.184622in}}%
\pgfpathlineto{\pgfqpoint{3.633521in}{1.247489in}}%
\pgfpathlineto{\pgfqpoint{3.858545in}{1.333602in}}%
\pgfpathlineto{\pgfqpoint{4.083568in}{1.407839in}}%
\pgfpathlineto{\pgfqpoint{4.308591in}{1.554250in}}%
\pgfpathlineto{\pgfqpoint{4.533614in}{1.754038in}}%
\pgfpathlineto{\pgfqpoint{4.758637in}{2.105422in}}%
\pgfpathlineto{\pgfqpoint{4.983661in}{2.186292in}}%
\pgfpathlineto{\pgfqpoint{5.208684in}{2.855359in}}%
\pgfusepath{stroke}%
\end{pgfscope}%
\begin{pgfscope}%
\pgfpathrectangle{\pgfqpoint{0.708220in}{0.535823in}}{\pgfqpoint{5.141780in}{2.669453in}}%
\pgfusepath{clip}%
\pgfsetbuttcap%
\pgfsetmiterjoin%
\definecolor{currentfill}{rgb}{0.780392,0.643137,0.254902}%
\pgfsetfillcolor{currentfill}%
\pgfsetlinewidth{0.501875pt}%
\definecolor{currentstroke}{rgb}{0.000000,0.000000,0.000000}%
\pgfsetstrokecolor{currentstroke}%
\pgfsetdash{}{0pt}%
\pgfsys@defobject{currentmarker}{\pgfqpoint{-0.034722in}{-0.034722in}}{\pgfqpoint{0.034722in}{0.034722in}}{%
\pgfpathmoveto{\pgfqpoint{0.034722in}{-0.000000in}}%
\pgfpathlineto{\pgfqpoint{-0.034722in}{0.034722in}}%
\pgfpathlineto{\pgfqpoint{-0.034722in}{-0.034722in}}%
\pgfpathclose%
\pgfusepath{stroke,fill}%
}%
\begin{pgfscope}%
\pgfsys@transformshift{1.833336in}{0.607768in}%
\pgfsys@useobject{currentmarker}{}%
\end{pgfscope}%
\begin{pgfscope}%
\pgfsys@transformshift{2.058359in}{0.714594in}%
\pgfsys@useobject{currentmarker}{}%
\end{pgfscope}%
\begin{pgfscope}%
\pgfsys@transformshift{2.283382in}{0.810268in}%
\pgfsys@useobject{currentmarker}{}%
\end{pgfscope}%
\begin{pgfscope}%
\pgfsys@transformshift{2.508405in}{0.899028in}%
\pgfsys@useobject{currentmarker}{}%
\end{pgfscope}%
\begin{pgfscope}%
\pgfsys@transformshift{2.733429in}{0.976962in}%
\pgfsys@useobject{currentmarker}{}%
\end{pgfscope}%
\begin{pgfscope}%
\pgfsys@transformshift{2.958452in}{1.048705in}%
\pgfsys@useobject{currentmarker}{}%
\end{pgfscope}%
\begin{pgfscope}%
\pgfsys@transformshift{3.183475in}{1.117476in}%
\pgfsys@useobject{currentmarker}{}%
\end{pgfscope}%
\begin{pgfscope}%
\pgfsys@transformshift{3.408498in}{1.184622in}%
\pgfsys@useobject{currentmarker}{}%
\end{pgfscope}%
\begin{pgfscope}%
\pgfsys@transformshift{3.633521in}{1.247489in}%
\pgfsys@useobject{currentmarker}{}%
\end{pgfscope}%
\begin{pgfscope}%
\pgfsys@transformshift{3.858545in}{1.333602in}%
\pgfsys@useobject{currentmarker}{}%
\end{pgfscope}%
\begin{pgfscope}%
\pgfsys@transformshift{4.083568in}{1.407839in}%
\pgfsys@useobject{currentmarker}{}%
\end{pgfscope}%
\begin{pgfscope}%
\pgfsys@transformshift{4.308591in}{1.554250in}%
\pgfsys@useobject{currentmarker}{}%
\end{pgfscope}%
\begin{pgfscope}%
\pgfsys@transformshift{4.533614in}{1.754038in}%
\pgfsys@useobject{currentmarker}{}%
\end{pgfscope}%
\begin{pgfscope}%
\pgfsys@transformshift{4.758637in}{2.105422in}%
\pgfsys@useobject{currentmarker}{}%
\end{pgfscope}%
\begin{pgfscope}%
\pgfsys@transformshift{4.983661in}{2.186292in}%
\pgfsys@useobject{currentmarker}{}%
\end{pgfscope}%
\begin{pgfscope}%
\pgfsys@transformshift{5.208684in}{2.855359in}%
\pgfsys@useobject{currentmarker}{}%
\end{pgfscope}%
\end{pgfscope}%
\begin{pgfscope}%
\pgfpathrectangle{\pgfqpoint{0.708220in}{0.535823in}}{\pgfqpoint{5.141780in}{2.669453in}}%
\pgfusepath{clip}%
\pgfsetrectcap%
\pgfsetroundjoin%
\pgfsetlinewidth{1.003750pt}%
\definecolor{currentstroke}{rgb}{1.000000,0.694118,0.305882}%
\pgfsetstrokecolor{currentstroke}%
\pgfsetdash{}{0pt}%
\pgfpathmoveto{\pgfqpoint{2.423238in}{0.525823in}}%
\pgfpathlineto{\pgfqpoint{2.508405in}{0.555088in}}%
\pgfpathlineto{\pgfqpoint{2.733429in}{0.713199in}}%
\pgfpathlineto{\pgfqpoint{2.958452in}{0.872472in}}%
\pgfpathlineto{\pgfqpoint{3.183475in}{1.125431in}}%
\pgfpathlineto{\pgfqpoint{3.408498in}{1.272615in}}%
\pgfpathlineto{\pgfqpoint{3.633521in}{1.392305in}}%
\pgfpathlineto{\pgfqpoint{3.858545in}{1.478523in}}%
\pgfpathlineto{\pgfqpoint{4.083568in}{1.557487in}}%
\pgfpathlineto{\pgfqpoint{4.308591in}{1.662572in}}%
\pgfpathlineto{\pgfqpoint{4.533614in}{1.734776in}}%
\pgfpathlineto{\pgfqpoint{4.758637in}{1.924610in}}%
\pgfpathlineto{\pgfqpoint{4.983661in}{2.079869in}}%
\pgfpathlineto{\pgfqpoint{5.208684in}{2.525204in}}%
\pgfpathlineto{\pgfqpoint{5.433707in}{2.672837in}}%
\pgfpathlineto{\pgfqpoint{5.658730in}{3.144706in}}%
\pgfusepath{stroke}%
\end{pgfscope}%
\begin{pgfscope}%
\pgfpathrectangle{\pgfqpoint{0.708220in}{0.535823in}}{\pgfqpoint{5.141780in}{2.669453in}}%
\pgfusepath{clip}%
\pgfsetbuttcap%
\pgfsetbeveljoin%
\definecolor{currentfill}{rgb}{1.000000,0.694118,0.305882}%
\pgfsetfillcolor{currentfill}%
\pgfsetlinewidth{0.501875pt}%
\definecolor{currentstroke}{rgb}{0.000000,0.000000,0.000000}%
\pgfsetstrokecolor{currentstroke}%
\pgfsetdash{}{0pt}%
\pgfsys@defobject{currentmarker}{\pgfqpoint{-0.033023in}{-0.028091in}}{\pgfqpoint{0.033023in}{0.034722in}}{%
\pgfpathmoveto{\pgfqpoint{0.000000in}{0.034722in}}%
\pgfpathlineto{\pgfqpoint{-0.007796in}{0.010730in}}%
\pgfpathlineto{\pgfqpoint{-0.033023in}{0.010730in}}%
\pgfpathlineto{\pgfqpoint{-0.012614in}{-0.004098in}}%
\pgfpathlineto{\pgfqpoint{-0.020409in}{-0.028091in}}%
\pgfpathlineto{\pgfqpoint{-0.000000in}{-0.013263in}}%
\pgfpathlineto{\pgfqpoint{0.020409in}{-0.028091in}}%
\pgfpathlineto{\pgfqpoint{0.012614in}{-0.004098in}}%
\pgfpathlineto{\pgfqpoint{0.033023in}{0.010730in}}%
\pgfpathlineto{\pgfqpoint{0.007796in}{0.010730in}}%
\pgfpathclose%
\pgfusepath{stroke,fill}%
}%
\begin{pgfscope}%
\pgfsys@transformshift{1.833336in}{0.359278in}%
\pgfsys@useobject{currentmarker}{}%
\end{pgfscope}%
\begin{pgfscope}%
\pgfsys@transformshift{2.058359in}{0.431153in}%
\pgfsys@useobject{currentmarker}{}%
\end{pgfscope}%
\begin{pgfscope}%
\pgfsys@transformshift{2.283382in}{0.477766in}%
\pgfsys@useobject{currentmarker}{}%
\end{pgfscope}%
\begin{pgfscope}%
\pgfsys@transformshift{2.508405in}{0.555088in}%
\pgfsys@useobject{currentmarker}{}%
\end{pgfscope}%
\begin{pgfscope}%
\pgfsys@transformshift{2.733429in}{0.713199in}%
\pgfsys@useobject{currentmarker}{}%
\end{pgfscope}%
\begin{pgfscope}%
\pgfsys@transformshift{2.958452in}{0.872472in}%
\pgfsys@useobject{currentmarker}{}%
\end{pgfscope}%
\begin{pgfscope}%
\pgfsys@transformshift{3.183475in}{1.125431in}%
\pgfsys@useobject{currentmarker}{}%
\end{pgfscope}%
\begin{pgfscope}%
\pgfsys@transformshift{3.408498in}{1.272615in}%
\pgfsys@useobject{currentmarker}{}%
\end{pgfscope}%
\begin{pgfscope}%
\pgfsys@transformshift{3.633521in}{1.392305in}%
\pgfsys@useobject{currentmarker}{}%
\end{pgfscope}%
\begin{pgfscope}%
\pgfsys@transformshift{3.858545in}{1.478523in}%
\pgfsys@useobject{currentmarker}{}%
\end{pgfscope}%
\begin{pgfscope}%
\pgfsys@transformshift{4.083568in}{1.557487in}%
\pgfsys@useobject{currentmarker}{}%
\end{pgfscope}%
\begin{pgfscope}%
\pgfsys@transformshift{4.308591in}{1.662572in}%
\pgfsys@useobject{currentmarker}{}%
\end{pgfscope}%
\begin{pgfscope}%
\pgfsys@transformshift{4.533614in}{1.734776in}%
\pgfsys@useobject{currentmarker}{}%
\end{pgfscope}%
\begin{pgfscope}%
\pgfsys@transformshift{4.758637in}{1.924610in}%
\pgfsys@useobject{currentmarker}{}%
\end{pgfscope}%
\begin{pgfscope}%
\pgfsys@transformshift{4.983661in}{2.079869in}%
\pgfsys@useobject{currentmarker}{}%
\end{pgfscope}%
\begin{pgfscope}%
\pgfsys@transformshift{5.208684in}{2.525204in}%
\pgfsys@useobject{currentmarker}{}%
\end{pgfscope}%
\begin{pgfscope}%
\pgfsys@transformshift{5.433707in}{2.672837in}%
\pgfsys@useobject{currentmarker}{}%
\end{pgfscope}%
\begin{pgfscope}%
\pgfsys@transformshift{5.658730in}{3.144706in}%
\pgfsys@useobject{currentmarker}{}%
\end{pgfscope}%
\end{pgfscope}%
\begin{pgfscope}%
\pgfsetrectcap%
\pgfsetmiterjoin%
\pgfsetlinewidth{0.803000pt}%
\definecolor{currentstroke}{rgb}{0.000000,0.000000,0.000000}%
\pgfsetstrokecolor{currentstroke}%
\pgfsetdash{}{0pt}%
\pgfpathmoveto{\pgfqpoint{0.708220in}{0.535823in}}%
\pgfpathlineto{\pgfqpoint{0.708220in}{3.205275in}}%
\pgfusepath{stroke}%
\end{pgfscope}%
\begin{pgfscope}%
\pgfsetrectcap%
\pgfsetmiterjoin%
\pgfsetlinewidth{0.803000pt}%
\definecolor{currentstroke}{rgb}{0.000000,0.000000,0.000000}%
\pgfsetstrokecolor{currentstroke}%
\pgfsetdash{}{0pt}%
\pgfpathmoveto{\pgfqpoint{5.850000in}{0.535823in}}%
\pgfpathlineto{\pgfqpoint{5.850000in}{3.205275in}}%
\pgfusepath{stroke}%
\end{pgfscope}%
\begin{pgfscope}%
\pgfsetrectcap%
\pgfsetmiterjoin%
\pgfsetlinewidth{0.803000pt}%
\definecolor{currentstroke}{rgb}{0.000000,0.000000,0.000000}%
\pgfsetstrokecolor{currentstroke}%
\pgfsetdash{}{0pt}%
\pgfpathmoveto{\pgfqpoint{0.708220in}{0.535823in}}%
\pgfpathlineto{\pgfqpoint{5.850000in}{0.535823in}}%
\pgfusepath{stroke}%
\end{pgfscope}%
\begin{pgfscope}%
\pgfsetrectcap%
\pgfsetmiterjoin%
\pgfsetlinewidth{0.803000pt}%
\definecolor{currentstroke}{rgb}{0.000000,0.000000,0.000000}%
\pgfsetstrokecolor{currentstroke}%
\pgfsetdash{}{0pt}%
\pgfpathmoveto{\pgfqpoint{0.708220in}{3.205275in}}%
\pgfpathlineto{\pgfqpoint{5.850000in}{3.205275in}}%
\pgfusepath{stroke}%
\end{pgfscope}%
\begin{pgfscope}%
\pgfsetrectcap%
\pgfsetroundjoin%
\pgfsetlinewidth{1.003750pt}%
\definecolor{currentstroke}{rgb}{0.866667,0.058824,0.058824}%
\pgfsetstrokecolor{currentstroke}%
\pgfsetdash{}{0pt}%
\pgfpathmoveto{\pgfqpoint{0.758220in}{3.111525in}}%
\pgfpathlineto{\pgfqpoint{1.008220in}{3.111525in}}%
\pgfusepath{stroke}%
\end{pgfscope}%
\begin{pgfscope}%
\pgfsetbuttcap%
\pgfsetmiterjoin%
\definecolor{currentfill}{rgb}{0.866667,0.058824,0.058824}%
\pgfsetfillcolor{currentfill}%
\pgfsetlinewidth{0.501875pt}%
\definecolor{currentstroke}{rgb}{0.000000,0.000000,0.000000}%
\pgfsetstrokecolor{currentstroke}%
\pgfsetdash{}{0pt}%
\pgfsys@defobject{currentmarker}{\pgfqpoint{-0.033023in}{-0.028091in}}{\pgfqpoint{0.033023in}{0.034722in}}{%
\pgfpathmoveto{\pgfqpoint{0.000000in}{0.034722in}}%
\pgfpathlineto{\pgfqpoint{-0.033023in}{0.010730in}}%
\pgfpathlineto{\pgfqpoint{-0.020409in}{-0.028091in}}%
\pgfpathlineto{\pgfqpoint{0.020409in}{-0.028091in}}%
\pgfpathlineto{\pgfqpoint{0.033023in}{0.010730in}}%
\pgfpathclose%
\pgfusepath{stroke,fill}%
}%
\begin{pgfscope}%
\pgfsys@transformshift{0.883220in}{3.111525in}%
\pgfsys@useobject{currentmarker}{}%
\end{pgfscope}%
\end{pgfscope}%
\begin{pgfscope}%
\definecolor{textcolor}{rgb}{0.000000,0.000000,0.000000}%
\pgfsetstrokecolor{textcolor}%
\pgfsetfillcolor{textcolor}%
\pgftext[x=1.033220in,y=3.067775in,left,base]{\color{textcolor}\rmfamily\fontsize{9.000000}{10.800000}\selectfont cachet}%
\end{pgfscope}%
\begin{pgfscope}%
\pgfsetrectcap%
\pgfsetroundjoin%
\pgfsetlinewidth{1.003750pt}%
\definecolor{currentstroke}{rgb}{0.000000,0.000000,0.200000}%
\pgfsetstrokecolor{currentstroke}%
\pgfsetdash{}{0pt}%
\pgfpathmoveto{\pgfqpoint{0.758220in}{2.949726in}}%
\pgfpathlineto{\pgfqpoint{1.008220in}{2.949726in}}%
\pgfusepath{stroke}%
\end{pgfscope}%
\begin{pgfscope}%
\pgfsetbuttcap%
\pgfsetmiterjoin%
\definecolor{currentfill}{rgb}{0.000000,0.000000,0.200000}%
\pgfsetfillcolor{currentfill}%
\pgfsetlinewidth{0.501875pt}%
\definecolor{currentstroke}{rgb}{0.000000,0.000000,0.000000}%
\pgfsetstrokecolor{currentstroke}%
\pgfsetdash{}{0pt}%
\pgfsys@defobject{currentmarker}{\pgfqpoint{-0.034722in}{-0.034722in}}{\pgfqpoint{0.034722in}{0.034722in}}{%
\pgfpathmoveto{\pgfqpoint{-0.011574in}{-0.034722in}}%
\pgfpathlineto{\pgfqpoint{0.011574in}{-0.034722in}}%
\pgfpathlineto{\pgfqpoint{0.011574in}{-0.011574in}}%
\pgfpathlineto{\pgfqpoint{0.034722in}{-0.011574in}}%
\pgfpathlineto{\pgfqpoint{0.034722in}{0.011574in}}%
\pgfpathlineto{\pgfqpoint{0.011574in}{0.011574in}}%
\pgfpathlineto{\pgfqpoint{0.011574in}{0.034722in}}%
\pgfpathlineto{\pgfqpoint{-0.011574in}{0.034722in}}%
\pgfpathlineto{\pgfqpoint{-0.011574in}{0.011574in}}%
\pgfpathlineto{\pgfqpoint{-0.034722in}{0.011574in}}%
\pgfpathlineto{\pgfqpoint{-0.034722in}{-0.011574in}}%
\pgfpathlineto{\pgfqpoint{-0.011574in}{-0.011574in}}%
\pgfpathclose%
\pgfusepath{stroke,fill}%
}%
\begin{pgfscope}%
\pgfsys@transformshift{0.883220in}{2.949726in}%
\pgfsys@useobject{currentmarker}{}%
\end{pgfscope}%
\end{pgfscope}%
\begin{pgfscope}%
\definecolor{textcolor}{rgb}{0.000000,0.000000,0.000000}%
\pgfsetstrokecolor{textcolor}%
\pgfsetfillcolor{textcolor}%
\pgftext[x=1.033220in,y=2.905976in,left,base]{\color{textcolor}\rmfamily\fontsize{9.000000}{10.800000}\selectfont dynQBF}%
\end{pgfscope}%
\begin{pgfscope}%
\pgfsetrectcap%
\pgfsetroundjoin%
\pgfsetlinewidth{1.003750pt}%
\definecolor{currentstroke}{rgb}{0.000000,0.000000,0.866667}%
\pgfsetstrokecolor{currentstroke}%
\pgfsetdash{}{0pt}%
\pgfpathmoveto{\pgfqpoint{0.758220in}{2.787926in}}%
\pgfpathlineto{\pgfqpoint{1.008220in}{2.787926in}}%
\pgfusepath{stroke}%
\end{pgfscope}%
\begin{pgfscope}%
\pgfsetbuttcap%
\pgfsetmiterjoin%
\definecolor{currentfill}{rgb}{0.000000,0.000000,0.866667}%
\pgfsetfillcolor{currentfill}%
\pgfsetlinewidth{0.501875pt}%
\definecolor{currentstroke}{rgb}{0.000000,0.000000,0.000000}%
\pgfsetstrokecolor{currentstroke}%
\pgfsetdash{}{0pt}%
\pgfsys@defobject{currentmarker}{\pgfqpoint{-0.029463in}{-0.049105in}}{\pgfqpoint{0.029463in}{0.049105in}}{%
\pgfpathmoveto{\pgfqpoint{0.000000in}{-0.049105in}}%
\pgfpathlineto{\pgfqpoint{0.029463in}{0.000000in}}%
\pgfpathlineto{\pgfqpoint{0.000000in}{0.049105in}}%
\pgfpathlineto{\pgfqpoint{-0.029463in}{0.000000in}}%
\pgfpathclose%
\pgfusepath{stroke,fill}%
}%
\begin{pgfscope}%
\pgfsys@transformshift{0.883220in}{2.787926in}%
\pgfsys@useobject{currentmarker}{}%
\end{pgfscope}%
\end{pgfscope}%
\begin{pgfscope}%
\definecolor{textcolor}{rgb}{0.000000,0.000000,0.000000}%
\pgfsetstrokecolor{textcolor}%
\pgfsetfillcolor{textcolor}%
\pgftext[x=1.033220in,y=2.744176in,left,base]{\color{textcolor}\rmfamily\fontsize{9.000000}{10.800000}\selectfont dynasp}%
\end{pgfscope}%
\begin{pgfscope}%
\pgfsetrectcap%
\pgfsetroundjoin%
\pgfsetlinewidth{1.003750pt}%
\definecolor{currentstroke}{rgb}{0.250980,0.231373,0.796078}%
\pgfsetstrokecolor{currentstroke}%
\pgfsetdash{}{0pt}%
\pgfpathmoveto{\pgfqpoint{0.758220in}{2.626126in}}%
\pgfpathlineto{\pgfqpoint{1.008220in}{2.626126in}}%
\pgfusepath{stroke}%
\end{pgfscope}%
\begin{pgfscope}%
\pgfsetbuttcap%
\pgfsetmiterjoin%
\definecolor{currentfill}{rgb}{0.250980,0.231373,0.796078}%
\pgfsetfillcolor{currentfill}%
\pgfsetlinewidth{0.501875pt}%
\definecolor{currentstroke}{rgb}{0.000000,0.000000,0.000000}%
\pgfsetstrokecolor{currentstroke}%
\pgfsetdash{}{0pt}%
\pgfsys@defobject{currentmarker}{\pgfqpoint{-0.034722in}{-0.034722in}}{\pgfqpoint{0.034722in}{0.034722in}}{%
\pgfpathmoveto{\pgfqpoint{-0.034722in}{-0.034722in}}%
\pgfpathlineto{\pgfqpoint{0.034722in}{-0.034722in}}%
\pgfpathlineto{\pgfqpoint{0.034722in}{0.034722in}}%
\pgfpathlineto{\pgfqpoint{-0.034722in}{0.034722in}}%
\pgfpathclose%
\pgfusepath{stroke,fill}%
}%
\begin{pgfscope}%
\pgfsys@transformshift{0.883220in}{2.626126in}%
\pgfsys@useobject{currentmarker}{}%
\end{pgfscope}%
\end{pgfscope}%
\begin{pgfscope}%
\definecolor{textcolor}{rgb}{0.000000,0.000000,0.000000}%
\pgfsetstrokecolor{textcolor}%
\pgfsetfillcolor{textcolor}%
\pgftext[x=1.033220in,y=2.582376in,left,base]{\color{textcolor}\rmfamily\fontsize{9.000000}{10.800000}\selectfont sharpSAT}%
\end{pgfscope}%
\begin{pgfscope}%
\pgfsetrectcap%
\pgfsetroundjoin%
\pgfsetlinewidth{1.003750pt}%
\definecolor{currentstroke}{rgb}{0.615686,0.007843,0.843137}%
\pgfsetstrokecolor{currentstroke}%
\pgfsetdash{}{0pt}%
\pgfpathmoveto{\pgfqpoint{0.758220in}{2.464327in}}%
\pgfpathlineto{\pgfqpoint{1.008220in}{2.464327in}}%
\pgfusepath{stroke}%
\end{pgfscope}%
\begin{pgfscope}%
\pgfsetbuttcap%
\pgfsetroundjoin%
\definecolor{currentfill}{rgb}{0.615686,0.007843,0.843137}%
\pgfsetfillcolor{currentfill}%
\pgfsetlinewidth{0.501875pt}%
\definecolor{currentstroke}{rgb}{0.000000,0.000000,0.000000}%
\pgfsetstrokecolor{currentstroke}%
\pgfsetdash{}{0pt}%
\pgfsys@defobject{currentmarker}{\pgfqpoint{-0.034722in}{-0.034722in}}{\pgfqpoint{0.034722in}{0.034722in}}{%
\pgfpathmoveto{\pgfqpoint{0.000000in}{-0.034722in}}%
\pgfpathcurveto{\pgfqpoint{0.009208in}{-0.034722in}}{\pgfqpoint{0.018041in}{-0.031064in}}{\pgfqpoint{0.024552in}{-0.024552in}}%
\pgfpathcurveto{\pgfqpoint{0.031064in}{-0.018041in}}{\pgfqpoint{0.034722in}{-0.009208in}}{\pgfqpoint{0.034722in}{0.000000in}}%
\pgfpathcurveto{\pgfqpoint{0.034722in}{0.009208in}}{\pgfqpoint{0.031064in}{0.018041in}}{\pgfqpoint{0.024552in}{0.024552in}}%
\pgfpathcurveto{\pgfqpoint{0.018041in}{0.031064in}}{\pgfqpoint{0.009208in}{0.034722in}}{\pgfqpoint{0.000000in}{0.034722in}}%
\pgfpathcurveto{\pgfqpoint{-0.009208in}{0.034722in}}{\pgfqpoint{-0.018041in}{0.031064in}}{\pgfqpoint{-0.024552in}{0.024552in}}%
\pgfpathcurveto{\pgfqpoint{-0.031064in}{0.018041in}}{\pgfqpoint{-0.034722in}{0.009208in}}{\pgfqpoint{-0.034722in}{0.000000in}}%
\pgfpathcurveto{\pgfqpoint{-0.034722in}{-0.009208in}}{\pgfqpoint{-0.031064in}{-0.018041in}}{\pgfqpoint{-0.024552in}{-0.024552in}}%
\pgfpathcurveto{\pgfqpoint{-0.018041in}{-0.031064in}}{\pgfqpoint{-0.009208in}{-0.034722in}}{\pgfqpoint{0.000000in}{-0.034722in}}%
\pgfpathclose%
\pgfusepath{stroke,fill}%
}%
\begin{pgfscope}%
\pgfsys@transformshift{0.883220in}{2.464327in}%
\pgfsys@useobject{currentmarker}{}%
\end{pgfscope}%
\end{pgfscope}%
\begin{pgfscope}%
\definecolor{textcolor}{rgb}{0.000000,0.000000,0.000000}%
\pgfsetstrokecolor{textcolor}%
\pgfsetfillcolor{textcolor}%
\pgftext[x=1.033220in,y=2.420577in,left,base]{\color{textcolor}\rmfamily\fontsize{9.000000}{10.800000}\selectfont d4}%
\end{pgfscope}%
\begin{pgfscope}%
\pgfsetrectcap%
\pgfsetroundjoin%
\pgfsetlinewidth{1.003750pt}%
\definecolor{currentstroke}{rgb}{0.917647,0.372549,0.580392}%
\pgfsetstrokecolor{currentstroke}%
\pgfsetdash{}{0pt}%
\pgfpathmoveto{\pgfqpoint{0.758220in}{2.302527in}}%
\pgfpathlineto{\pgfqpoint{1.008220in}{2.302527in}}%
\pgfusepath{stroke}%
\end{pgfscope}%
\begin{pgfscope}%
\pgfsetbuttcap%
\pgfsetmiterjoin%
\definecolor{currentfill}{rgb}{0.917647,0.372549,0.580392}%
\pgfsetfillcolor{currentfill}%
\pgfsetlinewidth{0.501875pt}%
\definecolor{currentstroke}{rgb}{0.000000,0.000000,0.000000}%
\pgfsetstrokecolor{currentstroke}%
\pgfsetdash{}{0pt}%
\pgfsys@defobject{currentmarker}{\pgfqpoint{-0.049105in}{-0.049105in}}{\pgfqpoint{0.049105in}{0.049105in}}{%
\pgfpathmoveto{\pgfqpoint{0.000000in}{-0.049105in}}%
\pgfpathlineto{\pgfqpoint{0.049105in}{0.000000in}}%
\pgfpathlineto{\pgfqpoint{0.000000in}{0.049105in}}%
\pgfpathlineto{\pgfqpoint{-0.049105in}{0.000000in}}%
\pgfpathclose%
\pgfusepath{stroke,fill}%
}%
\begin{pgfscope}%
\pgfsys@transformshift{0.883220in}{2.302527in}%
\pgfsys@useobject{currentmarker}{}%
\end{pgfscope}%
\end{pgfscope}%
\begin{pgfscope}%
\definecolor{textcolor}{rgb}{0.000000,0.000000,0.000000}%
\pgfsetstrokecolor{textcolor}%
\pgfsetfillcolor{textcolor}%
\pgftext[x=1.033220in,y=2.258777in,left,base]{\color{textcolor}\rmfamily\fontsize{9.000000}{10.800000}\selectfont miniC2D}%
\end{pgfscope}%
\begin{pgfscope}%
\pgfsetrectcap%
\pgfsetroundjoin%
\pgfsetlinewidth{1.003750pt}%
\definecolor{currentstroke}{rgb}{0.529412,0.462745,0.384314}%
\pgfsetstrokecolor{currentstroke}%
\pgfsetdash{}{0pt}%
\pgfpathmoveto{\pgfqpoint{0.758220in}{2.140728in}}%
\pgfpathlineto{\pgfqpoint{1.008220in}{2.140728in}}%
\pgfusepath{stroke}%
\end{pgfscope}%
\begin{pgfscope}%
\pgfsetbuttcap%
\pgfsetmiterjoin%
\definecolor{currentfill}{rgb}{0.529412,0.462745,0.384314}%
\pgfsetfillcolor{currentfill}%
\pgfsetlinewidth{0.501875pt}%
\definecolor{currentstroke}{rgb}{0.000000,0.000000,0.000000}%
\pgfsetstrokecolor{currentstroke}%
\pgfsetdash{}{0pt}%
\pgfsys@defobject{currentmarker}{\pgfqpoint{-0.034722in}{-0.034722in}}{\pgfqpoint{0.034722in}{0.034722in}}{%
\pgfpathmoveto{\pgfqpoint{-0.000000in}{-0.034722in}}%
\pgfpathlineto{\pgfqpoint{0.034722in}{0.034722in}}%
\pgfpathlineto{\pgfqpoint{-0.034722in}{0.034722in}}%
\pgfpathclose%
\pgfusepath{stroke,fill}%
}%
\begin{pgfscope}%
\pgfsys@transformshift{0.883220in}{2.140728in}%
\pgfsys@useobject{currentmarker}{}%
\end{pgfscope}%
\end{pgfscope}%
\begin{pgfscope}%
\definecolor{textcolor}{rgb}{0.000000,0.000000,0.000000}%
\pgfsetstrokecolor{textcolor}%
\pgfsetfillcolor{textcolor}%
\pgftext[x=1.033220in,y=2.096978in,left,base]{\color{textcolor}\rmfamily\fontsize{9.000000}{10.800000}\selectfont greedy}%
\end{pgfscope}%
\begin{pgfscope}%
\pgfsetrectcap%
\pgfsetroundjoin%
\pgfsetlinewidth{1.003750pt}%
\definecolor{currentstroke}{rgb}{0.611765,0.568627,0.274510}%
\pgfsetstrokecolor{currentstroke}%
\pgfsetdash{}{0pt}%
\pgfpathmoveto{\pgfqpoint{0.758220in}{1.978928in}}%
\pgfpathlineto{\pgfqpoint{1.008220in}{1.978928in}}%
\pgfusepath{stroke}%
\end{pgfscope}%
\begin{pgfscope}%
\pgfsetbuttcap%
\pgfsetmiterjoin%
\definecolor{currentfill}{rgb}{0.611765,0.568627,0.274510}%
\pgfsetfillcolor{currentfill}%
\pgfsetlinewidth{0.501875pt}%
\definecolor{currentstroke}{rgb}{0.000000,0.000000,0.000000}%
\pgfsetstrokecolor{currentstroke}%
\pgfsetdash{}{0pt}%
\pgfsys@defobject{currentmarker}{\pgfqpoint{-0.034722in}{-0.034722in}}{\pgfqpoint{0.034722in}{0.034722in}}{%
\pgfpathmoveto{\pgfqpoint{-0.034722in}{0.000000in}}%
\pgfpathlineto{\pgfqpoint{0.034722in}{-0.034722in}}%
\pgfpathlineto{\pgfqpoint{0.034722in}{0.034722in}}%
\pgfpathclose%
\pgfusepath{stroke,fill}%
}%
\begin{pgfscope}%
\pgfsys@transformshift{0.883220in}{1.978928in}%
\pgfsys@useobject{currentmarker}{}%
\end{pgfscope}%
\end{pgfscope}%
\begin{pgfscope}%
\definecolor{textcolor}{rgb}{0.000000,0.000000,0.000000}%
\pgfsetstrokecolor{textcolor}%
\pgfsetfillcolor{textcolor}%
\pgftext[x=1.033220in,y=1.935178in,left,base]{\color{textcolor}\rmfamily\fontsize{9.000000}{10.800000}\selectfont metis}%
\end{pgfscope}%
\begin{pgfscope}%
\pgfsetrectcap%
\pgfsetroundjoin%
\pgfsetlinewidth{1.003750pt}%
\definecolor{currentstroke}{rgb}{0.780392,0.643137,0.254902}%
\pgfsetstrokecolor{currentstroke}%
\pgfsetdash{}{0pt}%
\pgfpathmoveto{\pgfqpoint{0.758220in}{1.817129in}}%
\pgfpathlineto{\pgfqpoint{1.008220in}{1.817129in}}%
\pgfusepath{stroke}%
\end{pgfscope}%
\begin{pgfscope}%
\pgfsetbuttcap%
\pgfsetmiterjoin%
\definecolor{currentfill}{rgb}{0.780392,0.643137,0.254902}%
\pgfsetfillcolor{currentfill}%
\pgfsetlinewidth{0.501875pt}%
\definecolor{currentstroke}{rgb}{0.000000,0.000000,0.000000}%
\pgfsetstrokecolor{currentstroke}%
\pgfsetdash{}{0pt}%
\pgfsys@defobject{currentmarker}{\pgfqpoint{-0.034722in}{-0.034722in}}{\pgfqpoint{0.034722in}{0.034722in}}{%
\pgfpathmoveto{\pgfqpoint{0.034722in}{-0.000000in}}%
\pgfpathlineto{\pgfqpoint{-0.034722in}{0.034722in}}%
\pgfpathlineto{\pgfqpoint{-0.034722in}{-0.034722in}}%
\pgfpathclose%
\pgfusepath{stroke,fill}%
}%
\begin{pgfscope}%
\pgfsys@transformshift{0.883220in}{1.817129in}%
\pgfsys@useobject{currentmarker}{}%
\end{pgfscope}%
\end{pgfscope}%
\begin{pgfscope}%
\definecolor{textcolor}{rgb}{0.000000,0.000000,0.000000}%
\pgfsetstrokecolor{textcolor}%
\pgfsetfillcolor{textcolor}%
\pgftext[x=1.033220in,y=1.773379in,left,base]{\color{textcolor}\rmfamily\fontsize{9.000000}{10.800000}\selectfont GN}%
\end{pgfscope}%
\begin{pgfscope}%
\pgfsetrectcap%
\pgfsetroundjoin%
\pgfsetlinewidth{1.003750pt}%
\definecolor{currentstroke}{rgb}{1.000000,0.694118,0.305882}%
\pgfsetstrokecolor{currentstroke}%
\pgfsetdash{}{0pt}%
\pgfpathmoveto{\pgfqpoint{0.758220in}{1.655329in}}%
\pgfpathlineto{\pgfqpoint{1.008220in}{1.655329in}}%
\pgfusepath{stroke}%
\end{pgfscope}%
\begin{pgfscope}%
\pgfsetbuttcap%
\pgfsetbeveljoin%
\definecolor{currentfill}{rgb}{1.000000,0.694118,0.305882}%
\pgfsetfillcolor{currentfill}%
\pgfsetlinewidth{0.501875pt}%
\definecolor{currentstroke}{rgb}{0.000000,0.000000,0.000000}%
\pgfsetstrokecolor{currentstroke}%
\pgfsetdash{}{0pt}%
\pgfsys@defobject{currentmarker}{\pgfqpoint{-0.033023in}{-0.028091in}}{\pgfqpoint{0.033023in}{0.034722in}}{%
\pgfpathmoveto{\pgfqpoint{0.000000in}{0.034722in}}%
\pgfpathlineto{\pgfqpoint{-0.007796in}{0.010730in}}%
\pgfpathlineto{\pgfqpoint{-0.033023in}{0.010730in}}%
\pgfpathlineto{\pgfqpoint{-0.012614in}{-0.004098in}}%
\pgfpathlineto{\pgfqpoint{-0.020409in}{-0.028091in}}%
\pgfpathlineto{\pgfqpoint{-0.000000in}{-0.013263in}}%
\pgfpathlineto{\pgfqpoint{0.020409in}{-0.028091in}}%
\pgfpathlineto{\pgfqpoint{0.012614in}{-0.004098in}}%
\pgfpathlineto{\pgfqpoint{0.033023in}{0.010730in}}%
\pgfpathlineto{\pgfqpoint{0.007796in}{0.010730in}}%
\pgfpathclose%
\pgfusepath{stroke,fill}%
}%
\begin{pgfscope}%
\pgfsys@transformshift{0.883220in}{1.655329in}%
\pgfsys@useobject{currentmarker}{}%
\end{pgfscope}%
\end{pgfscope}%
\begin{pgfscope}%
\definecolor{textcolor}{rgb}{0.000000,0.000000,0.000000}%
\pgfsetstrokecolor{textcolor}%
\pgfsetfillcolor{textcolor}%
\pgftext[x=1.033220in,y=1.611579in,left,base]{\color{textcolor}\rmfamily\fontsize{9.000000}{10.800000}\selectfont LG+Flow}%
\end{pgfscope}%
\end{pgfpicture}%
\makeatother%
\endgroup%

	%% Creator: Matplotlib, PGF backend
%%
%% To include the figure in your LaTeX document, write
%%   \input{<filename>.pgf}
%%
%% Make sure the required packages are loaded in your preamble
%%   \usepackage{pgf}
%%
%% and, on pdftex
%%   \usepackage[utf8]{inputenc}\DeclareUnicodeCharacter{2212}{-}
%%
%% or, on luatex and xetex
%%   \usepackage{unicode-math}
%%
%% Figures using additional raster images can only be included by \input if
%% they are in the same directory as the main LaTeX file. For loading figures
%% from other directories you can use the `import` package
%%   \usepackage{import}
%%
%% and then include the figures with
%%   \import{<path to file>}{<filename>.pgf}
%%
%% Matplotlib used the following preamble
%%   \usepackage[utf8x]{inputenc}
%%   \usepackage[T1]{fontenc}
%%
\begingroup%
\makeatletter%
\begin{pgfpicture}%
\pgfpathrectangle{\pgfpointorigin}{\pgfqpoint{6.000000in}{3.400000in}}%
\pgfusepath{use as bounding box, clip}%
\begin{pgfscope}%
\pgfsetbuttcap%
\pgfsetmiterjoin%
\definecolor{currentfill}{rgb}{1.000000,1.000000,1.000000}%
\pgfsetfillcolor{currentfill}%
\pgfsetlinewidth{0.000000pt}%
\definecolor{currentstroke}{rgb}{1.000000,1.000000,1.000000}%
\pgfsetstrokecolor{currentstroke}%
\pgfsetdash{}{0pt}%
\pgfpathmoveto{\pgfqpoint{0.000000in}{0.000000in}}%
\pgfpathlineto{\pgfqpoint{6.000000in}{0.000000in}}%
\pgfpathlineto{\pgfqpoint{6.000000in}{3.400000in}}%
\pgfpathlineto{\pgfqpoint{0.000000in}{3.400000in}}%
\pgfpathclose%
\pgfusepath{fill}%
\end{pgfscope}%
\begin{pgfscope}%
\pgfsetbuttcap%
\pgfsetmiterjoin%
\definecolor{currentfill}{rgb}{1.000000,1.000000,1.000000}%
\pgfsetfillcolor{currentfill}%
\pgfsetlinewidth{0.000000pt}%
\definecolor{currentstroke}{rgb}{0.000000,0.000000,0.000000}%
\pgfsetstrokecolor{currentstroke}%
\pgfsetstrokeopacity{0.000000}%
\pgfsetdash{}{0pt}%
\pgfpathmoveto{\pgfqpoint{0.553904in}{0.535823in}}%
\pgfpathlineto{\pgfqpoint{5.850000in}{0.535823in}}%
\pgfpathlineto{\pgfqpoint{5.850000in}{3.250000in}}%
\pgfpathlineto{\pgfqpoint{0.553904in}{3.250000in}}%
\pgfpathclose%
\pgfusepath{fill}%
\end{pgfscope}%
\begin{pgfscope}%
\pgfsetbuttcap%
\pgfsetroundjoin%
\definecolor{currentfill}{rgb}{0.000000,0.000000,0.000000}%
\pgfsetfillcolor{currentfill}%
\pgfsetlinewidth{0.803000pt}%
\definecolor{currentstroke}{rgb}{0.000000,0.000000,0.000000}%
\pgfsetstrokecolor{currentstroke}%
\pgfsetdash{}{0pt}%
\pgfsys@defobject{currentmarker}{\pgfqpoint{0.000000in}{-0.048611in}}{\pgfqpoint{0.000000in}{0.000000in}}{%
\pgfpathmoveto{\pgfqpoint{0.000000in}{0.000000in}}%
\pgfpathlineto{\pgfqpoint{0.000000in}{-0.048611in}}%
\pgfusepath{stroke,fill}%
}%
\begin{pgfscope}%
\pgfsys@transformshift{0.794636in}{0.535823in}%
\pgfsys@useobject{currentmarker}{}%
\end{pgfscope}%
\end{pgfscope}%
\begin{pgfscope}%
\definecolor{textcolor}{rgb}{0.000000,0.000000,0.000000}%
\pgfsetstrokecolor{textcolor}%
\pgfsetfillcolor{textcolor}%
\pgftext[x=0.794636in,y=0.438600in,,top]{\color{textcolor}\rmfamily\fontsize{9.000000}{10.800000}\selectfont \(\displaystyle {50}\)}%
\end{pgfscope}%
\begin{pgfscope}%
\pgfsetbuttcap%
\pgfsetroundjoin%
\definecolor{currentfill}{rgb}{0.000000,0.000000,0.000000}%
\pgfsetfillcolor{currentfill}%
\pgfsetlinewidth{0.803000pt}%
\definecolor{currentstroke}{rgb}{0.000000,0.000000,0.000000}%
\pgfsetstrokecolor{currentstroke}%
\pgfsetdash{}{0pt}%
\pgfsys@defobject{currentmarker}{\pgfqpoint{0.000000in}{-0.048611in}}{\pgfqpoint{0.000000in}{0.000000in}}{%
\pgfpathmoveto{\pgfqpoint{0.000000in}{0.000000in}}%
\pgfpathlineto{\pgfqpoint{0.000000in}{-0.048611in}}%
\pgfusepath{stroke,fill}%
}%
\begin{pgfscope}%
\pgfsys@transformshift{1.998294in}{0.535823in}%
\pgfsys@useobject{currentmarker}{}%
\end{pgfscope}%
\end{pgfscope}%
\begin{pgfscope}%
\definecolor{textcolor}{rgb}{0.000000,0.000000,0.000000}%
\pgfsetstrokecolor{textcolor}%
\pgfsetfillcolor{textcolor}%
\pgftext[x=1.998294in,y=0.438600in,,top]{\color{textcolor}\rmfamily\fontsize{9.000000}{10.800000}\selectfont \(\displaystyle {100}\)}%
\end{pgfscope}%
\begin{pgfscope}%
\pgfsetbuttcap%
\pgfsetroundjoin%
\definecolor{currentfill}{rgb}{0.000000,0.000000,0.000000}%
\pgfsetfillcolor{currentfill}%
\pgfsetlinewidth{0.803000pt}%
\definecolor{currentstroke}{rgb}{0.000000,0.000000,0.000000}%
\pgfsetstrokecolor{currentstroke}%
\pgfsetdash{}{0pt}%
\pgfsys@defobject{currentmarker}{\pgfqpoint{0.000000in}{-0.048611in}}{\pgfqpoint{0.000000in}{0.000000in}}{%
\pgfpathmoveto{\pgfqpoint{0.000000in}{0.000000in}}%
\pgfpathlineto{\pgfqpoint{0.000000in}{-0.048611in}}%
\pgfusepath{stroke,fill}%
}%
\begin{pgfscope}%
\pgfsys@transformshift{3.201952in}{0.535823in}%
\pgfsys@useobject{currentmarker}{}%
\end{pgfscope}%
\end{pgfscope}%
\begin{pgfscope}%
\definecolor{textcolor}{rgb}{0.000000,0.000000,0.000000}%
\pgfsetstrokecolor{textcolor}%
\pgfsetfillcolor{textcolor}%
\pgftext[x=3.201952in,y=0.438600in,,top]{\color{textcolor}\rmfamily\fontsize{9.000000}{10.800000}\selectfont \(\displaystyle {150}\)}%
\end{pgfscope}%
\begin{pgfscope}%
\pgfsetbuttcap%
\pgfsetroundjoin%
\definecolor{currentfill}{rgb}{0.000000,0.000000,0.000000}%
\pgfsetfillcolor{currentfill}%
\pgfsetlinewidth{0.803000pt}%
\definecolor{currentstroke}{rgb}{0.000000,0.000000,0.000000}%
\pgfsetstrokecolor{currentstroke}%
\pgfsetdash{}{0pt}%
\pgfsys@defobject{currentmarker}{\pgfqpoint{0.000000in}{-0.048611in}}{\pgfqpoint{0.000000in}{0.000000in}}{%
\pgfpathmoveto{\pgfqpoint{0.000000in}{0.000000in}}%
\pgfpathlineto{\pgfqpoint{0.000000in}{-0.048611in}}%
\pgfusepath{stroke,fill}%
}%
\begin{pgfscope}%
\pgfsys@transformshift{4.405610in}{0.535823in}%
\pgfsys@useobject{currentmarker}{}%
\end{pgfscope}%
\end{pgfscope}%
\begin{pgfscope}%
\definecolor{textcolor}{rgb}{0.000000,0.000000,0.000000}%
\pgfsetstrokecolor{textcolor}%
\pgfsetfillcolor{textcolor}%
\pgftext[x=4.405610in,y=0.438600in,,top]{\color{textcolor}\rmfamily\fontsize{9.000000}{10.800000}\selectfont \(\displaystyle {200}\)}%
\end{pgfscope}%
\begin{pgfscope}%
\pgfsetbuttcap%
\pgfsetroundjoin%
\definecolor{currentfill}{rgb}{0.000000,0.000000,0.000000}%
\pgfsetfillcolor{currentfill}%
\pgfsetlinewidth{0.803000pt}%
\definecolor{currentstroke}{rgb}{0.000000,0.000000,0.000000}%
\pgfsetstrokecolor{currentstroke}%
\pgfsetdash{}{0pt}%
\pgfsys@defobject{currentmarker}{\pgfqpoint{0.000000in}{-0.048611in}}{\pgfqpoint{0.000000in}{0.000000in}}{%
\pgfpathmoveto{\pgfqpoint{0.000000in}{0.000000in}}%
\pgfpathlineto{\pgfqpoint{0.000000in}{-0.048611in}}%
\pgfusepath{stroke,fill}%
}%
\begin{pgfscope}%
\pgfsys@transformshift{5.609268in}{0.535823in}%
\pgfsys@useobject{currentmarker}{}%
\end{pgfscope}%
\end{pgfscope}%
\begin{pgfscope}%
\definecolor{textcolor}{rgb}{0.000000,0.000000,0.000000}%
\pgfsetstrokecolor{textcolor}%
\pgfsetfillcolor{textcolor}%
\pgftext[x=5.609268in,y=0.438600in,,top]{\color{textcolor}\rmfamily\fontsize{9.000000}{10.800000}\selectfont \(\displaystyle {250}\)}%
\end{pgfscope}%
\begin{pgfscope}%
\definecolor{textcolor}{rgb}{0.000000,0.000000,0.000000}%
\pgfsetstrokecolor{textcolor}%
\pgfsetfillcolor{textcolor}%
\pgftext[x=3.201952in,y=0.272655in,,top]{\color{textcolor}\rmfamily\fontsize{10.000000}{12.000000}\selectfont \(\displaystyle n\): Number of vertices}%
\end{pgfscope}%
\begin{pgfscope}%
\pgfsetbuttcap%
\pgfsetroundjoin%
\definecolor{currentfill}{rgb}{0.000000,0.000000,0.000000}%
\pgfsetfillcolor{currentfill}%
\pgfsetlinewidth{0.803000pt}%
\definecolor{currentstroke}{rgb}{0.000000,0.000000,0.000000}%
\pgfsetstrokecolor{currentstroke}%
\pgfsetdash{}{0pt}%
\pgfsys@defobject{currentmarker}{\pgfqpoint{-0.048611in}{0.000000in}}{\pgfqpoint{-0.000000in}{0.000000in}}{%
\pgfpathmoveto{\pgfqpoint{-0.000000in}{0.000000in}}%
\pgfpathlineto{\pgfqpoint{-0.048611in}{0.000000in}}%
\pgfusepath{stroke,fill}%
}%
\begin{pgfscope}%
\pgfsys@transformshift{0.553904in}{0.719376in}%
\pgfsys@useobject{currentmarker}{}%
\end{pgfscope}%
\end{pgfscope}%
\begin{pgfscope}%
\definecolor{textcolor}{rgb}{0.000000,0.000000,0.000000}%
\pgfsetstrokecolor{textcolor}%
\pgfsetfillcolor{textcolor}%
\pgftext[x=0.328211in, y=0.676331in, left, base]{\color{textcolor}\rmfamily\fontsize{9.000000}{10.800000}\selectfont \(\displaystyle {10}\)}%
\end{pgfscope}%
\begin{pgfscope}%
\pgfsetbuttcap%
\pgfsetroundjoin%
\definecolor{currentfill}{rgb}{0.000000,0.000000,0.000000}%
\pgfsetfillcolor{currentfill}%
\pgfsetlinewidth{0.803000pt}%
\definecolor{currentstroke}{rgb}{0.000000,0.000000,0.000000}%
\pgfsetstrokecolor{currentstroke}%
\pgfsetdash{}{0pt}%
\pgfsys@defobject{currentmarker}{\pgfqpoint{-0.048611in}{0.000000in}}{\pgfqpoint{-0.000000in}{0.000000in}}{%
\pgfpathmoveto{\pgfqpoint{-0.000000in}{0.000000in}}%
\pgfpathlineto{\pgfqpoint{-0.048611in}{0.000000in}}%
\pgfusepath{stroke,fill}%
}%
\begin{pgfscope}%
\pgfsys@transformshift{0.553904in}{1.321189in}%
\pgfsys@useobject{currentmarker}{}%
\end{pgfscope}%
\end{pgfscope}%
\begin{pgfscope}%
\definecolor{textcolor}{rgb}{0.000000,0.000000,0.000000}%
\pgfsetstrokecolor{textcolor}%
\pgfsetfillcolor{textcolor}%
\pgftext[x=0.328211in, y=1.278144in, left, base]{\color{textcolor}\rmfamily\fontsize{9.000000}{10.800000}\selectfont \(\displaystyle {20}\)}%
\end{pgfscope}%
\begin{pgfscope}%
\pgfsetbuttcap%
\pgfsetroundjoin%
\definecolor{currentfill}{rgb}{0.000000,0.000000,0.000000}%
\pgfsetfillcolor{currentfill}%
\pgfsetlinewidth{0.803000pt}%
\definecolor{currentstroke}{rgb}{0.000000,0.000000,0.000000}%
\pgfsetstrokecolor{currentstroke}%
\pgfsetdash{}{0pt}%
\pgfsys@defobject{currentmarker}{\pgfqpoint{-0.048611in}{0.000000in}}{\pgfqpoint{-0.000000in}{0.000000in}}{%
\pgfpathmoveto{\pgfqpoint{-0.000000in}{0.000000in}}%
\pgfpathlineto{\pgfqpoint{-0.048611in}{0.000000in}}%
\pgfusepath{stroke,fill}%
}%
\begin{pgfscope}%
\pgfsys@transformshift{0.553904in}{1.923002in}%
\pgfsys@useobject{currentmarker}{}%
\end{pgfscope}%
\end{pgfscope}%
\begin{pgfscope}%
\definecolor{textcolor}{rgb}{0.000000,0.000000,0.000000}%
\pgfsetstrokecolor{textcolor}%
\pgfsetfillcolor{textcolor}%
\pgftext[x=0.328211in, y=1.879957in, left, base]{\color{textcolor}\rmfamily\fontsize{9.000000}{10.800000}\selectfont \(\displaystyle {30}\)}%
\end{pgfscope}%
\begin{pgfscope}%
\pgfsetbuttcap%
\pgfsetroundjoin%
\definecolor{currentfill}{rgb}{0.000000,0.000000,0.000000}%
\pgfsetfillcolor{currentfill}%
\pgfsetlinewidth{0.803000pt}%
\definecolor{currentstroke}{rgb}{0.000000,0.000000,0.000000}%
\pgfsetstrokecolor{currentstroke}%
\pgfsetdash{}{0pt}%
\pgfsys@defobject{currentmarker}{\pgfqpoint{-0.048611in}{0.000000in}}{\pgfqpoint{-0.000000in}{0.000000in}}{%
\pgfpathmoveto{\pgfqpoint{-0.000000in}{0.000000in}}%
\pgfpathlineto{\pgfqpoint{-0.048611in}{0.000000in}}%
\pgfusepath{stroke,fill}%
}%
\begin{pgfscope}%
\pgfsys@transformshift{0.553904in}{2.524815in}%
\pgfsys@useobject{currentmarker}{}%
\end{pgfscope}%
\end{pgfscope}%
\begin{pgfscope}%
\definecolor{textcolor}{rgb}{0.000000,0.000000,0.000000}%
\pgfsetstrokecolor{textcolor}%
\pgfsetfillcolor{textcolor}%
\pgftext[x=0.328211in, y=2.481770in, left, base]{\color{textcolor}\rmfamily\fontsize{9.000000}{10.800000}\selectfont \(\displaystyle {40}\)}%
\end{pgfscope}%
\begin{pgfscope}%
\pgfsetbuttcap%
\pgfsetroundjoin%
\definecolor{currentfill}{rgb}{0.000000,0.000000,0.000000}%
\pgfsetfillcolor{currentfill}%
\pgfsetlinewidth{0.803000pt}%
\definecolor{currentstroke}{rgb}{0.000000,0.000000,0.000000}%
\pgfsetstrokecolor{currentstroke}%
\pgfsetdash{}{0pt}%
\pgfsys@defobject{currentmarker}{\pgfqpoint{-0.048611in}{0.000000in}}{\pgfqpoint{-0.000000in}{0.000000in}}{%
\pgfpathmoveto{\pgfqpoint{-0.000000in}{0.000000in}}%
\pgfpathlineto{\pgfqpoint{-0.048611in}{0.000000in}}%
\pgfusepath{stroke,fill}%
}%
\begin{pgfscope}%
\pgfsys@transformshift{0.553904in}{3.126628in}%
\pgfsys@useobject{currentmarker}{}%
\end{pgfscope}%
\end{pgfscope}%
\begin{pgfscope}%
\definecolor{textcolor}{rgb}{0.000000,0.000000,0.000000}%
\pgfsetstrokecolor{textcolor}%
\pgfsetfillcolor{textcolor}%
\pgftext[x=0.328211in, y=3.083583in, left, base]{\color{textcolor}\rmfamily\fontsize{9.000000}{10.800000}\selectfont \(\displaystyle {50}\)}%
\end{pgfscope}%
\begin{pgfscope}%
\definecolor{textcolor}{rgb}{0.000000,0.000000,0.000000}%
\pgfsetstrokecolor{textcolor}%
\pgfsetfillcolor{textcolor}%
\pgftext[x=0.272655in,y=1.892911in,,bottom,rotate=90.000000]{\color{textcolor}\rmfamily\fontsize{10.000000}{12.000000}\selectfont Median max rank}%
\end{pgfscope}%
\begin{pgfscope}%
\pgfpathrectangle{\pgfqpoint{0.553904in}{0.535823in}}{\pgfqpoint{5.296096in}{2.714177in}}%
\pgfusepath{clip}%
\pgfsetrectcap%
\pgfsetroundjoin%
\pgfsetlinewidth{1.003750pt}%
\definecolor{currentstroke}{rgb}{0.529412,0.462745,0.384314}%
\pgfsetstrokecolor{currentstroke}%
\pgfsetdash{}{0pt}%
\pgfpathmoveto{\pgfqpoint{0.794636in}{0.779557in}}%
\pgfpathlineto{\pgfqpoint{1.035368in}{0.899920in}}%
\pgfpathlineto{\pgfqpoint{1.276099in}{0.960101in}}%
\pgfpathlineto{\pgfqpoint{1.516831in}{1.080464in}}%
\pgfpathlineto{\pgfqpoint{1.757562in}{1.200826in}}%
\pgfpathlineto{\pgfqpoint{1.998294in}{1.321189in}}%
\pgfpathlineto{\pgfqpoint{2.239026in}{1.441551in}}%
\pgfpathlineto{\pgfqpoint{2.479757in}{1.561914in}}%
\pgfpathlineto{\pgfqpoint{2.720489in}{1.682277in}}%
\pgfpathlineto{\pgfqpoint{2.961221in}{1.772549in}}%
\pgfpathlineto{\pgfqpoint{3.201952in}{1.923002in}}%
\pgfpathlineto{\pgfqpoint{3.442684in}{2.013274in}}%
\pgfpathlineto{\pgfqpoint{3.683415in}{2.103546in}}%
\pgfpathlineto{\pgfqpoint{3.924147in}{2.223909in}}%
\pgfpathlineto{\pgfqpoint{4.164879in}{2.344271in}}%
\pgfpathlineto{\pgfqpoint{4.405610in}{2.464634in}}%
\pgfpathlineto{\pgfqpoint{4.646342in}{2.584996in}}%
\pgfpathlineto{\pgfqpoint{4.887074in}{2.705359in}}%
\pgfpathlineto{\pgfqpoint{5.127805in}{2.885903in}}%
\pgfpathlineto{\pgfqpoint{5.368537in}{2.946084in}}%
\pgfpathlineto{\pgfqpoint{5.609268in}{3.126628in}}%
\pgfusepath{stroke}%
\end{pgfscope}%
\begin{pgfscope}%
\pgfpathrectangle{\pgfqpoint{0.553904in}{0.535823in}}{\pgfqpoint{5.296096in}{2.714177in}}%
\pgfusepath{clip}%
\pgfsetbuttcap%
\pgfsetmiterjoin%
\definecolor{currentfill}{rgb}{0.529412,0.462745,0.384314}%
\pgfsetfillcolor{currentfill}%
\pgfsetlinewidth{0.501875pt}%
\definecolor{currentstroke}{rgb}{0.000000,0.000000,0.000000}%
\pgfsetstrokecolor{currentstroke}%
\pgfsetdash{}{0pt}%
\pgfsys@defobject{currentmarker}{\pgfqpoint{-0.034722in}{-0.034722in}}{\pgfqpoint{0.034722in}{0.034722in}}{%
\pgfpathmoveto{\pgfqpoint{-0.000000in}{-0.034722in}}%
\pgfpathlineto{\pgfqpoint{0.034722in}{0.034722in}}%
\pgfpathlineto{\pgfqpoint{-0.034722in}{0.034722in}}%
\pgfpathclose%
\pgfusepath{stroke,fill}%
}%
\begin{pgfscope}%
\pgfsys@transformshift{0.794636in}{0.779557in}%
\pgfsys@useobject{currentmarker}{}%
\end{pgfscope}%
\begin{pgfscope}%
\pgfsys@transformshift{1.035368in}{0.899920in}%
\pgfsys@useobject{currentmarker}{}%
\end{pgfscope}%
\begin{pgfscope}%
\pgfsys@transformshift{1.276099in}{0.960101in}%
\pgfsys@useobject{currentmarker}{}%
\end{pgfscope}%
\begin{pgfscope}%
\pgfsys@transformshift{1.516831in}{1.080464in}%
\pgfsys@useobject{currentmarker}{}%
\end{pgfscope}%
\begin{pgfscope}%
\pgfsys@transformshift{1.757562in}{1.200826in}%
\pgfsys@useobject{currentmarker}{}%
\end{pgfscope}%
\begin{pgfscope}%
\pgfsys@transformshift{1.998294in}{1.321189in}%
\pgfsys@useobject{currentmarker}{}%
\end{pgfscope}%
\begin{pgfscope}%
\pgfsys@transformshift{2.239026in}{1.441551in}%
\pgfsys@useobject{currentmarker}{}%
\end{pgfscope}%
\begin{pgfscope}%
\pgfsys@transformshift{2.479757in}{1.561914in}%
\pgfsys@useobject{currentmarker}{}%
\end{pgfscope}%
\begin{pgfscope}%
\pgfsys@transformshift{2.720489in}{1.682277in}%
\pgfsys@useobject{currentmarker}{}%
\end{pgfscope}%
\begin{pgfscope}%
\pgfsys@transformshift{2.961221in}{1.772549in}%
\pgfsys@useobject{currentmarker}{}%
\end{pgfscope}%
\begin{pgfscope}%
\pgfsys@transformshift{3.201952in}{1.923002in}%
\pgfsys@useobject{currentmarker}{}%
\end{pgfscope}%
\begin{pgfscope}%
\pgfsys@transformshift{3.442684in}{2.013274in}%
\pgfsys@useobject{currentmarker}{}%
\end{pgfscope}%
\begin{pgfscope}%
\pgfsys@transformshift{3.683415in}{2.103546in}%
\pgfsys@useobject{currentmarker}{}%
\end{pgfscope}%
\begin{pgfscope}%
\pgfsys@transformshift{3.924147in}{2.223909in}%
\pgfsys@useobject{currentmarker}{}%
\end{pgfscope}%
\begin{pgfscope}%
\pgfsys@transformshift{4.164879in}{2.344271in}%
\pgfsys@useobject{currentmarker}{}%
\end{pgfscope}%
\begin{pgfscope}%
\pgfsys@transformshift{4.405610in}{2.464634in}%
\pgfsys@useobject{currentmarker}{}%
\end{pgfscope}%
\begin{pgfscope}%
\pgfsys@transformshift{4.646342in}{2.584996in}%
\pgfsys@useobject{currentmarker}{}%
\end{pgfscope}%
\begin{pgfscope}%
\pgfsys@transformshift{4.887074in}{2.705359in}%
\pgfsys@useobject{currentmarker}{}%
\end{pgfscope}%
\begin{pgfscope}%
\pgfsys@transformshift{5.127805in}{2.885903in}%
\pgfsys@useobject{currentmarker}{}%
\end{pgfscope}%
\begin{pgfscope}%
\pgfsys@transformshift{5.368537in}{2.946084in}%
\pgfsys@useobject{currentmarker}{}%
\end{pgfscope}%
\begin{pgfscope}%
\pgfsys@transformshift{5.609268in}{3.126628in}%
\pgfsys@useobject{currentmarker}{}%
\end{pgfscope}%
\end{pgfscope}%
\begin{pgfscope}%
\pgfpathrectangle{\pgfqpoint{0.553904in}{0.535823in}}{\pgfqpoint{5.296096in}{2.714177in}}%
\pgfusepath{clip}%
\pgfsetrectcap%
\pgfsetroundjoin%
\pgfsetlinewidth{1.003750pt}%
\definecolor{currentstroke}{rgb}{0.611765,0.568627,0.274510}%
\pgfsetstrokecolor{currentstroke}%
\pgfsetdash{}{0pt}%
\pgfpathmoveto{\pgfqpoint{0.794636in}{0.719376in}}%
\pgfpathlineto{\pgfqpoint{1.035368in}{0.779557in}}%
\pgfpathlineto{\pgfqpoint{1.276099in}{0.839738in}}%
\pgfpathlineto{\pgfqpoint{1.516831in}{0.899920in}}%
\pgfpathlineto{\pgfqpoint{1.757562in}{1.020282in}}%
\pgfpathlineto{\pgfqpoint{1.998294in}{1.080464in}}%
\pgfpathlineto{\pgfqpoint{2.239026in}{1.140645in}}%
\pgfpathlineto{\pgfqpoint{2.479757in}{1.200826in}}%
\pgfpathlineto{\pgfqpoint{2.720489in}{1.321189in}}%
\pgfpathlineto{\pgfqpoint{2.961221in}{1.411461in}}%
\pgfpathlineto{\pgfqpoint{3.201952in}{1.471642in}}%
\pgfpathlineto{\pgfqpoint{3.442684in}{1.561914in}}%
\pgfpathlineto{\pgfqpoint{3.683415in}{1.622095in}}%
\pgfpathlineto{\pgfqpoint{3.924147in}{1.742458in}}%
\pgfpathlineto{\pgfqpoint{4.164879in}{1.802639in}}%
\pgfpathlineto{\pgfqpoint{4.405610in}{1.862821in}}%
\pgfpathlineto{\pgfqpoint{4.646342in}{1.923002in}}%
\pgfpathlineto{\pgfqpoint{4.887074in}{2.043365in}}%
\pgfpathlineto{\pgfqpoint{5.127805in}{2.163727in}}%
\pgfpathlineto{\pgfqpoint{5.368537in}{2.223909in}}%
\pgfpathlineto{\pgfqpoint{5.609268in}{2.284090in}}%
\pgfusepath{stroke}%
\end{pgfscope}%
\begin{pgfscope}%
\pgfpathrectangle{\pgfqpoint{0.553904in}{0.535823in}}{\pgfqpoint{5.296096in}{2.714177in}}%
\pgfusepath{clip}%
\pgfsetbuttcap%
\pgfsetmiterjoin%
\definecolor{currentfill}{rgb}{0.611765,0.568627,0.274510}%
\pgfsetfillcolor{currentfill}%
\pgfsetlinewidth{0.501875pt}%
\definecolor{currentstroke}{rgb}{0.000000,0.000000,0.000000}%
\pgfsetstrokecolor{currentstroke}%
\pgfsetdash{}{0pt}%
\pgfsys@defobject{currentmarker}{\pgfqpoint{-0.034722in}{-0.034722in}}{\pgfqpoint{0.034722in}{0.034722in}}{%
\pgfpathmoveto{\pgfqpoint{-0.034722in}{0.000000in}}%
\pgfpathlineto{\pgfqpoint{0.034722in}{-0.034722in}}%
\pgfpathlineto{\pgfqpoint{0.034722in}{0.034722in}}%
\pgfpathclose%
\pgfusepath{stroke,fill}%
}%
\begin{pgfscope}%
\pgfsys@transformshift{0.794636in}{0.719376in}%
\pgfsys@useobject{currentmarker}{}%
\end{pgfscope}%
\begin{pgfscope}%
\pgfsys@transformshift{1.035368in}{0.779557in}%
\pgfsys@useobject{currentmarker}{}%
\end{pgfscope}%
\begin{pgfscope}%
\pgfsys@transformshift{1.276099in}{0.839738in}%
\pgfsys@useobject{currentmarker}{}%
\end{pgfscope}%
\begin{pgfscope}%
\pgfsys@transformshift{1.516831in}{0.899920in}%
\pgfsys@useobject{currentmarker}{}%
\end{pgfscope}%
\begin{pgfscope}%
\pgfsys@transformshift{1.757562in}{1.020282in}%
\pgfsys@useobject{currentmarker}{}%
\end{pgfscope}%
\begin{pgfscope}%
\pgfsys@transformshift{1.998294in}{1.080464in}%
\pgfsys@useobject{currentmarker}{}%
\end{pgfscope}%
\begin{pgfscope}%
\pgfsys@transformshift{2.239026in}{1.140645in}%
\pgfsys@useobject{currentmarker}{}%
\end{pgfscope}%
\begin{pgfscope}%
\pgfsys@transformshift{2.479757in}{1.200826in}%
\pgfsys@useobject{currentmarker}{}%
\end{pgfscope}%
\begin{pgfscope}%
\pgfsys@transformshift{2.720489in}{1.321189in}%
\pgfsys@useobject{currentmarker}{}%
\end{pgfscope}%
\begin{pgfscope}%
\pgfsys@transformshift{2.961221in}{1.411461in}%
\pgfsys@useobject{currentmarker}{}%
\end{pgfscope}%
\begin{pgfscope}%
\pgfsys@transformshift{3.201952in}{1.471642in}%
\pgfsys@useobject{currentmarker}{}%
\end{pgfscope}%
\begin{pgfscope}%
\pgfsys@transformshift{3.442684in}{1.561914in}%
\pgfsys@useobject{currentmarker}{}%
\end{pgfscope}%
\begin{pgfscope}%
\pgfsys@transformshift{3.683415in}{1.622095in}%
\pgfsys@useobject{currentmarker}{}%
\end{pgfscope}%
\begin{pgfscope}%
\pgfsys@transformshift{3.924147in}{1.742458in}%
\pgfsys@useobject{currentmarker}{}%
\end{pgfscope}%
\begin{pgfscope}%
\pgfsys@transformshift{4.164879in}{1.802639in}%
\pgfsys@useobject{currentmarker}{}%
\end{pgfscope}%
\begin{pgfscope}%
\pgfsys@transformshift{4.405610in}{1.862821in}%
\pgfsys@useobject{currentmarker}{}%
\end{pgfscope}%
\begin{pgfscope}%
\pgfsys@transformshift{4.646342in}{1.923002in}%
\pgfsys@useobject{currentmarker}{}%
\end{pgfscope}%
\begin{pgfscope}%
\pgfsys@transformshift{4.887074in}{2.043365in}%
\pgfsys@useobject{currentmarker}{}%
\end{pgfscope}%
\begin{pgfscope}%
\pgfsys@transformshift{5.127805in}{2.163727in}%
\pgfsys@useobject{currentmarker}{}%
\end{pgfscope}%
\begin{pgfscope}%
\pgfsys@transformshift{5.368537in}{2.223909in}%
\pgfsys@useobject{currentmarker}{}%
\end{pgfscope}%
\begin{pgfscope}%
\pgfsys@transformshift{5.609268in}{2.284090in}%
\pgfsys@useobject{currentmarker}{}%
\end{pgfscope}%
\end{pgfscope}%
\begin{pgfscope}%
\pgfpathrectangle{\pgfqpoint{0.553904in}{0.535823in}}{\pgfqpoint{5.296096in}{2.714177in}}%
\pgfusepath{clip}%
\pgfsetrectcap%
\pgfsetroundjoin%
\pgfsetlinewidth{1.003750pt}%
\definecolor{currentstroke}{rgb}{0.780392,0.643137,0.254902}%
\pgfsetstrokecolor{currentstroke}%
\pgfsetdash{}{0pt}%
\pgfpathmoveto{\pgfqpoint{0.794636in}{0.659194in}}%
\pgfpathlineto{\pgfqpoint{1.035368in}{0.779557in}}%
\pgfpathlineto{\pgfqpoint{1.276099in}{0.839738in}}%
\pgfpathlineto{\pgfqpoint{1.516831in}{0.899920in}}%
\pgfpathlineto{\pgfqpoint{1.757562in}{1.020282in}}%
\pgfpathlineto{\pgfqpoint{1.998294in}{1.080464in}}%
\pgfpathlineto{\pgfqpoint{2.239026in}{1.140645in}}%
\pgfpathlineto{\pgfqpoint{2.479757in}{1.261007in}}%
\pgfpathlineto{\pgfqpoint{2.720489in}{1.321189in}}%
\pgfpathlineto{\pgfqpoint{2.961221in}{1.441551in}}%
\pgfpathlineto{\pgfqpoint{3.201952in}{1.501733in}}%
\pgfpathlineto{\pgfqpoint{3.442684in}{1.561914in}}%
\pgfpathlineto{\pgfqpoint{3.683415in}{1.622095in}}%
\pgfpathlineto{\pgfqpoint{3.924147in}{1.742458in}}%
\pgfpathlineto{\pgfqpoint{4.164879in}{1.802639in}}%
\pgfpathlineto{\pgfqpoint{4.405610in}{1.923002in}}%
\pgfpathlineto{\pgfqpoint{4.646342in}{1.923002in}}%
\pgfpathlineto{\pgfqpoint{4.887074in}{2.043365in}}%
\pgfpathlineto{\pgfqpoint{5.127805in}{2.103546in}}%
\pgfpathlineto{\pgfqpoint{5.368537in}{2.223909in}}%
\pgfpathlineto{\pgfqpoint{5.609268in}{2.344271in}}%
\pgfusepath{stroke}%
\end{pgfscope}%
\begin{pgfscope}%
\pgfpathrectangle{\pgfqpoint{0.553904in}{0.535823in}}{\pgfqpoint{5.296096in}{2.714177in}}%
\pgfusepath{clip}%
\pgfsetbuttcap%
\pgfsetmiterjoin%
\definecolor{currentfill}{rgb}{0.780392,0.643137,0.254902}%
\pgfsetfillcolor{currentfill}%
\pgfsetlinewidth{0.501875pt}%
\definecolor{currentstroke}{rgb}{0.000000,0.000000,0.000000}%
\pgfsetstrokecolor{currentstroke}%
\pgfsetdash{}{0pt}%
\pgfsys@defobject{currentmarker}{\pgfqpoint{-0.034722in}{-0.034722in}}{\pgfqpoint{0.034722in}{0.034722in}}{%
\pgfpathmoveto{\pgfqpoint{0.034722in}{-0.000000in}}%
\pgfpathlineto{\pgfqpoint{-0.034722in}{0.034722in}}%
\pgfpathlineto{\pgfqpoint{-0.034722in}{-0.034722in}}%
\pgfpathclose%
\pgfusepath{stroke,fill}%
}%
\begin{pgfscope}%
\pgfsys@transformshift{0.794636in}{0.659194in}%
\pgfsys@useobject{currentmarker}{}%
\end{pgfscope}%
\begin{pgfscope}%
\pgfsys@transformshift{1.035368in}{0.779557in}%
\pgfsys@useobject{currentmarker}{}%
\end{pgfscope}%
\begin{pgfscope}%
\pgfsys@transformshift{1.276099in}{0.839738in}%
\pgfsys@useobject{currentmarker}{}%
\end{pgfscope}%
\begin{pgfscope}%
\pgfsys@transformshift{1.516831in}{0.899920in}%
\pgfsys@useobject{currentmarker}{}%
\end{pgfscope}%
\begin{pgfscope}%
\pgfsys@transformshift{1.757562in}{1.020282in}%
\pgfsys@useobject{currentmarker}{}%
\end{pgfscope}%
\begin{pgfscope}%
\pgfsys@transformshift{1.998294in}{1.080464in}%
\pgfsys@useobject{currentmarker}{}%
\end{pgfscope}%
\begin{pgfscope}%
\pgfsys@transformshift{2.239026in}{1.140645in}%
\pgfsys@useobject{currentmarker}{}%
\end{pgfscope}%
\begin{pgfscope}%
\pgfsys@transformshift{2.479757in}{1.261007in}%
\pgfsys@useobject{currentmarker}{}%
\end{pgfscope}%
\begin{pgfscope}%
\pgfsys@transformshift{2.720489in}{1.321189in}%
\pgfsys@useobject{currentmarker}{}%
\end{pgfscope}%
\begin{pgfscope}%
\pgfsys@transformshift{2.961221in}{1.441551in}%
\pgfsys@useobject{currentmarker}{}%
\end{pgfscope}%
\begin{pgfscope}%
\pgfsys@transformshift{3.201952in}{1.501733in}%
\pgfsys@useobject{currentmarker}{}%
\end{pgfscope}%
\begin{pgfscope}%
\pgfsys@transformshift{3.442684in}{1.561914in}%
\pgfsys@useobject{currentmarker}{}%
\end{pgfscope}%
\begin{pgfscope}%
\pgfsys@transformshift{3.683415in}{1.622095in}%
\pgfsys@useobject{currentmarker}{}%
\end{pgfscope}%
\begin{pgfscope}%
\pgfsys@transformshift{3.924147in}{1.742458in}%
\pgfsys@useobject{currentmarker}{}%
\end{pgfscope}%
\begin{pgfscope}%
\pgfsys@transformshift{4.164879in}{1.802639in}%
\pgfsys@useobject{currentmarker}{}%
\end{pgfscope}%
\begin{pgfscope}%
\pgfsys@transformshift{4.405610in}{1.923002in}%
\pgfsys@useobject{currentmarker}{}%
\end{pgfscope}%
\begin{pgfscope}%
\pgfsys@transformshift{4.646342in}{1.923002in}%
\pgfsys@useobject{currentmarker}{}%
\end{pgfscope}%
\begin{pgfscope}%
\pgfsys@transformshift{4.887074in}{2.043365in}%
\pgfsys@useobject{currentmarker}{}%
\end{pgfscope}%
\begin{pgfscope}%
\pgfsys@transformshift{5.127805in}{2.103546in}%
\pgfsys@useobject{currentmarker}{}%
\end{pgfscope}%
\begin{pgfscope}%
\pgfsys@transformshift{5.368537in}{2.223909in}%
\pgfsys@useobject{currentmarker}{}%
\end{pgfscope}%
\begin{pgfscope}%
\pgfsys@transformshift{5.609268in}{2.344271in}%
\pgfsys@useobject{currentmarker}{}%
\end{pgfscope}%
\end{pgfscope}%
\begin{pgfscope}%
\pgfpathrectangle{\pgfqpoint{0.553904in}{0.535823in}}{\pgfqpoint{5.296096in}{2.714177in}}%
\pgfusepath{clip}%
\pgfsetrectcap%
\pgfsetroundjoin%
\pgfsetlinewidth{1.003750pt}%
\definecolor{currentstroke}{rgb}{1.000000,0.694118,0.305882}%
\pgfsetstrokecolor{currentstroke}%
\pgfsetdash{}{0pt}%
\pgfpathmoveto{\pgfqpoint{0.794636in}{0.839738in}}%
\pgfpathlineto{\pgfqpoint{1.035368in}{0.960101in}}%
\pgfpathlineto{\pgfqpoint{1.276099in}{1.080464in}}%
\pgfpathlineto{\pgfqpoint{1.516831in}{1.200826in}}%
\pgfpathlineto{\pgfqpoint{1.757562in}{1.321189in}}%
\pgfpathlineto{\pgfqpoint{1.998294in}{1.381370in}}%
\pgfpathlineto{\pgfqpoint{2.239026in}{1.471642in}}%
\pgfpathlineto{\pgfqpoint{2.479757in}{1.501733in}}%
\pgfpathlineto{\pgfqpoint{2.720489in}{1.561914in}}%
\pgfpathlineto{\pgfqpoint{2.961221in}{1.561914in}}%
\pgfpathlineto{\pgfqpoint{3.201952in}{1.561914in}}%
\pgfpathlineto{\pgfqpoint{3.442684in}{1.561914in}}%
\pgfpathlineto{\pgfqpoint{3.683415in}{1.561914in}}%
\pgfpathlineto{\pgfqpoint{3.924147in}{1.622095in}}%
\pgfpathlineto{\pgfqpoint{4.164879in}{1.622095in}}%
\pgfpathlineto{\pgfqpoint{4.405610in}{1.682277in}}%
\pgfpathlineto{\pgfqpoint{4.646342in}{1.742458in}}%
\pgfpathlineto{\pgfqpoint{4.887074in}{1.802639in}}%
\pgfpathlineto{\pgfqpoint{5.127805in}{1.923002in}}%
\pgfpathlineto{\pgfqpoint{5.368537in}{1.953093in}}%
\pgfpathlineto{\pgfqpoint{5.609268in}{2.043365in}}%
\pgfusepath{stroke}%
\end{pgfscope}%
\begin{pgfscope}%
\pgfpathrectangle{\pgfqpoint{0.553904in}{0.535823in}}{\pgfqpoint{5.296096in}{2.714177in}}%
\pgfusepath{clip}%
\pgfsetbuttcap%
\pgfsetbeveljoin%
\definecolor{currentfill}{rgb}{1.000000,0.694118,0.305882}%
\pgfsetfillcolor{currentfill}%
\pgfsetlinewidth{0.501875pt}%
\definecolor{currentstroke}{rgb}{0.000000,0.000000,0.000000}%
\pgfsetstrokecolor{currentstroke}%
\pgfsetdash{}{0pt}%
\pgfsys@defobject{currentmarker}{\pgfqpoint{-0.033023in}{-0.028091in}}{\pgfqpoint{0.033023in}{0.034722in}}{%
\pgfpathmoveto{\pgfqpoint{0.000000in}{0.034722in}}%
\pgfpathlineto{\pgfqpoint{-0.007796in}{0.010730in}}%
\pgfpathlineto{\pgfqpoint{-0.033023in}{0.010730in}}%
\pgfpathlineto{\pgfqpoint{-0.012614in}{-0.004098in}}%
\pgfpathlineto{\pgfqpoint{-0.020409in}{-0.028091in}}%
\pgfpathlineto{\pgfqpoint{-0.000000in}{-0.013263in}}%
\pgfpathlineto{\pgfqpoint{0.020409in}{-0.028091in}}%
\pgfpathlineto{\pgfqpoint{0.012614in}{-0.004098in}}%
\pgfpathlineto{\pgfqpoint{0.033023in}{0.010730in}}%
\pgfpathlineto{\pgfqpoint{0.007796in}{0.010730in}}%
\pgfpathclose%
\pgfusepath{stroke,fill}%
}%
\begin{pgfscope}%
\pgfsys@transformshift{0.794636in}{0.839738in}%
\pgfsys@useobject{currentmarker}{}%
\end{pgfscope}%
\begin{pgfscope}%
\pgfsys@transformshift{1.035368in}{0.960101in}%
\pgfsys@useobject{currentmarker}{}%
\end{pgfscope}%
\begin{pgfscope}%
\pgfsys@transformshift{1.276099in}{1.080464in}%
\pgfsys@useobject{currentmarker}{}%
\end{pgfscope}%
\begin{pgfscope}%
\pgfsys@transformshift{1.516831in}{1.200826in}%
\pgfsys@useobject{currentmarker}{}%
\end{pgfscope}%
\begin{pgfscope}%
\pgfsys@transformshift{1.757562in}{1.321189in}%
\pgfsys@useobject{currentmarker}{}%
\end{pgfscope}%
\begin{pgfscope}%
\pgfsys@transformshift{1.998294in}{1.381370in}%
\pgfsys@useobject{currentmarker}{}%
\end{pgfscope}%
\begin{pgfscope}%
\pgfsys@transformshift{2.239026in}{1.471642in}%
\pgfsys@useobject{currentmarker}{}%
\end{pgfscope}%
\begin{pgfscope}%
\pgfsys@transformshift{2.479757in}{1.501733in}%
\pgfsys@useobject{currentmarker}{}%
\end{pgfscope}%
\begin{pgfscope}%
\pgfsys@transformshift{2.720489in}{1.561914in}%
\pgfsys@useobject{currentmarker}{}%
\end{pgfscope}%
\begin{pgfscope}%
\pgfsys@transformshift{2.961221in}{1.561914in}%
\pgfsys@useobject{currentmarker}{}%
\end{pgfscope}%
\begin{pgfscope}%
\pgfsys@transformshift{3.201952in}{1.561914in}%
\pgfsys@useobject{currentmarker}{}%
\end{pgfscope}%
\begin{pgfscope}%
\pgfsys@transformshift{3.442684in}{1.561914in}%
\pgfsys@useobject{currentmarker}{}%
\end{pgfscope}%
\begin{pgfscope}%
\pgfsys@transformshift{3.683415in}{1.561914in}%
\pgfsys@useobject{currentmarker}{}%
\end{pgfscope}%
\begin{pgfscope}%
\pgfsys@transformshift{3.924147in}{1.622095in}%
\pgfsys@useobject{currentmarker}{}%
\end{pgfscope}%
\begin{pgfscope}%
\pgfsys@transformshift{4.164879in}{1.622095in}%
\pgfsys@useobject{currentmarker}{}%
\end{pgfscope}%
\begin{pgfscope}%
\pgfsys@transformshift{4.405610in}{1.682277in}%
\pgfsys@useobject{currentmarker}{}%
\end{pgfscope}%
\begin{pgfscope}%
\pgfsys@transformshift{4.646342in}{1.742458in}%
\pgfsys@useobject{currentmarker}{}%
\end{pgfscope}%
\begin{pgfscope}%
\pgfsys@transformshift{4.887074in}{1.802639in}%
\pgfsys@useobject{currentmarker}{}%
\end{pgfscope}%
\begin{pgfscope}%
\pgfsys@transformshift{5.127805in}{1.923002in}%
\pgfsys@useobject{currentmarker}{}%
\end{pgfscope}%
\begin{pgfscope}%
\pgfsys@transformshift{5.368537in}{1.953093in}%
\pgfsys@useobject{currentmarker}{}%
\end{pgfscope}%
\begin{pgfscope}%
\pgfsys@transformshift{5.609268in}{2.043365in}%
\pgfsys@useobject{currentmarker}{}%
\end{pgfscope}%
\end{pgfscope}%
\begin{pgfscope}%
\pgfsetrectcap%
\pgfsetmiterjoin%
\pgfsetlinewidth{0.803000pt}%
\definecolor{currentstroke}{rgb}{0.000000,0.000000,0.000000}%
\pgfsetstrokecolor{currentstroke}%
\pgfsetdash{}{0pt}%
\pgfpathmoveto{\pgfqpoint{0.553904in}{0.535823in}}%
\pgfpathlineto{\pgfqpoint{0.553904in}{3.250000in}}%
\pgfusepath{stroke}%
\end{pgfscope}%
\begin{pgfscope}%
\pgfsetrectcap%
\pgfsetmiterjoin%
\pgfsetlinewidth{0.803000pt}%
\definecolor{currentstroke}{rgb}{0.000000,0.000000,0.000000}%
\pgfsetstrokecolor{currentstroke}%
\pgfsetdash{}{0pt}%
\pgfpathmoveto{\pgfqpoint{5.850000in}{0.535823in}}%
\pgfpathlineto{\pgfqpoint{5.850000in}{3.250000in}}%
\pgfusepath{stroke}%
\end{pgfscope}%
\begin{pgfscope}%
\pgfsetrectcap%
\pgfsetmiterjoin%
\pgfsetlinewidth{0.803000pt}%
\definecolor{currentstroke}{rgb}{0.000000,0.000000,0.000000}%
\pgfsetstrokecolor{currentstroke}%
\pgfsetdash{}{0pt}%
\pgfpathmoveto{\pgfqpoint{0.553904in}{0.535823in}}%
\pgfpathlineto{\pgfqpoint{5.850000in}{0.535823in}}%
\pgfusepath{stroke}%
\end{pgfscope}%
\begin{pgfscope}%
\pgfsetrectcap%
\pgfsetmiterjoin%
\pgfsetlinewidth{0.803000pt}%
\definecolor{currentstroke}{rgb}{0.000000,0.000000,0.000000}%
\pgfsetstrokecolor{currentstroke}%
\pgfsetdash{}{0pt}%
\pgfpathmoveto{\pgfqpoint{0.553904in}{3.250000in}}%
\pgfpathlineto{\pgfqpoint{5.850000in}{3.250000in}}%
\pgfusepath{stroke}%
\end{pgfscope}%
\begin{pgfscope}%
\pgfsetrectcap%
\pgfsetroundjoin%
\pgfsetlinewidth{1.003750pt}%
\definecolor{currentstroke}{rgb}{0.529412,0.462745,0.384314}%
\pgfsetstrokecolor{currentstroke}%
\pgfsetdash{}{0pt}%
\pgfpathmoveto{\pgfqpoint{0.603904in}{3.156250in}}%
\pgfpathlineto{\pgfqpoint{0.853904in}{3.156250in}}%
\pgfusepath{stroke}%
\end{pgfscope}%
\begin{pgfscope}%
\pgfsetbuttcap%
\pgfsetmiterjoin%
\definecolor{currentfill}{rgb}{0.529412,0.462745,0.384314}%
\pgfsetfillcolor{currentfill}%
\pgfsetlinewidth{0.501875pt}%
\definecolor{currentstroke}{rgb}{0.000000,0.000000,0.000000}%
\pgfsetstrokecolor{currentstroke}%
\pgfsetdash{}{0pt}%
\pgfsys@defobject{currentmarker}{\pgfqpoint{-0.034722in}{-0.034722in}}{\pgfqpoint{0.034722in}{0.034722in}}{%
\pgfpathmoveto{\pgfqpoint{-0.000000in}{-0.034722in}}%
\pgfpathlineto{\pgfqpoint{0.034722in}{0.034722in}}%
\pgfpathlineto{\pgfqpoint{-0.034722in}{0.034722in}}%
\pgfpathclose%
\pgfusepath{stroke,fill}%
}%
\begin{pgfscope}%
\pgfsys@transformshift{0.728904in}{3.156250in}%
\pgfsys@useobject{currentmarker}{}%
\end{pgfscope}%
\end{pgfscope}%
\begin{pgfscope}%
\definecolor{textcolor}{rgb}{0.000000,0.000000,0.000000}%
\pgfsetstrokecolor{textcolor}%
\pgfsetfillcolor{textcolor}%
\pgftext[x=0.878904in,y=3.112500in,left,base]{\color{textcolor}\rmfamily\fontsize{9.000000}{10.800000}\selectfont greedy}%
\end{pgfscope}%
\begin{pgfscope}%
\pgfsetrectcap%
\pgfsetroundjoin%
\pgfsetlinewidth{1.003750pt}%
\definecolor{currentstroke}{rgb}{0.611765,0.568627,0.274510}%
\pgfsetstrokecolor{currentstroke}%
\pgfsetdash{}{0pt}%
\pgfpathmoveto{\pgfqpoint{0.603904in}{2.994450in}}%
\pgfpathlineto{\pgfqpoint{0.853904in}{2.994450in}}%
\pgfusepath{stroke}%
\end{pgfscope}%
\begin{pgfscope}%
\pgfsetbuttcap%
\pgfsetmiterjoin%
\definecolor{currentfill}{rgb}{0.611765,0.568627,0.274510}%
\pgfsetfillcolor{currentfill}%
\pgfsetlinewidth{0.501875pt}%
\definecolor{currentstroke}{rgb}{0.000000,0.000000,0.000000}%
\pgfsetstrokecolor{currentstroke}%
\pgfsetdash{}{0pt}%
\pgfsys@defobject{currentmarker}{\pgfqpoint{-0.034722in}{-0.034722in}}{\pgfqpoint{0.034722in}{0.034722in}}{%
\pgfpathmoveto{\pgfqpoint{-0.034722in}{0.000000in}}%
\pgfpathlineto{\pgfqpoint{0.034722in}{-0.034722in}}%
\pgfpathlineto{\pgfqpoint{0.034722in}{0.034722in}}%
\pgfpathclose%
\pgfusepath{stroke,fill}%
}%
\begin{pgfscope}%
\pgfsys@transformshift{0.728904in}{2.994450in}%
\pgfsys@useobject{currentmarker}{}%
\end{pgfscope}%
\end{pgfscope}%
\begin{pgfscope}%
\definecolor{textcolor}{rgb}{0.000000,0.000000,0.000000}%
\pgfsetstrokecolor{textcolor}%
\pgfsetfillcolor{textcolor}%
\pgftext[x=0.878904in,y=2.950700in,left,base]{\color{textcolor}\rmfamily\fontsize{9.000000}{10.800000}\selectfont metis}%
\end{pgfscope}%
\begin{pgfscope}%
\pgfsetrectcap%
\pgfsetroundjoin%
\pgfsetlinewidth{1.003750pt}%
\definecolor{currentstroke}{rgb}{0.780392,0.643137,0.254902}%
\pgfsetstrokecolor{currentstroke}%
\pgfsetdash{}{0pt}%
\pgfpathmoveto{\pgfqpoint{0.603904in}{2.832651in}}%
\pgfpathlineto{\pgfqpoint{0.853904in}{2.832651in}}%
\pgfusepath{stroke}%
\end{pgfscope}%
\begin{pgfscope}%
\pgfsetbuttcap%
\pgfsetmiterjoin%
\definecolor{currentfill}{rgb}{0.780392,0.643137,0.254902}%
\pgfsetfillcolor{currentfill}%
\pgfsetlinewidth{0.501875pt}%
\definecolor{currentstroke}{rgb}{0.000000,0.000000,0.000000}%
\pgfsetstrokecolor{currentstroke}%
\pgfsetdash{}{0pt}%
\pgfsys@defobject{currentmarker}{\pgfqpoint{-0.034722in}{-0.034722in}}{\pgfqpoint{0.034722in}{0.034722in}}{%
\pgfpathmoveto{\pgfqpoint{0.034722in}{-0.000000in}}%
\pgfpathlineto{\pgfqpoint{-0.034722in}{0.034722in}}%
\pgfpathlineto{\pgfqpoint{-0.034722in}{-0.034722in}}%
\pgfpathclose%
\pgfusepath{stroke,fill}%
}%
\begin{pgfscope}%
\pgfsys@transformshift{0.728904in}{2.832651in}%
\pgfsys@useobject{currentmarker}{}%
\end{pgfscope}%
\end{pgfscope}%
\begin{pgfscope}%
\definecolor{textcolor}{rgb}{0.000000,0.000000,0.000000}%
\pgfsetstrokecolor{textcolor}%
\pgfsetfillcolor{textcolor}%
\pgftext[x=0.878904in,y=2.788901in,left,base]{\color{textcolor}\rmfamily\fontsize{9.000000}{10.800000}\selectfont GN}%
\end{pgfscope}%
\begin{pgfscope}%
\pgfsetrectcap%
\pgfsetroundjoin%
\pgfsetlinewidth{1.003750pt}%
\definecolor{currentstroke}{rgb}{1.000000,0.694118,0.305882}%
\pgfsetstrokecolor{currentstroke}%
\pgfsetdash{}{0pt}%
\pgfpathmoveto{\pgfqpoint{0.603904in}{2.670851in}}%
\pgfpathlineto{\pgfqpoint{0.853904in}{2.670851in}}%
\pgfusepath{stroke}%
\end{pgfscope}%
\begin{pgfscope}%
\pgfsetbuttcap%
\pgfsetbeveljoin%
\definecolor{currentfill}{rgb}{1.000000,0.694118,0.305882}%
\pgfsetfillcolor{currentfill}%
\pgfsetlinewidth{0.501875pt}%
\definecolor{currentstroke}{rgb}{0.000000,0.000000,0.000000}%
\pgfsetstrokecolor{currentstroke}%
\pgfsetdash{}{0pt}%
\pgfsys@defobject{currentmarker}{\pgfqpoint{-0.033023in}{-0.028091in}}{\pgfqpoint{0.033023in}{0.034722in}}{%
\pgfpathmoveto{\pgfqpoint{0.000000in}{0.034722in}}%
\pgfpathlineto{\pgfqpoint{-0.007796in}{0.010730in}}%
\pgfpathlineto{\pgfqpoint{-0.033023in}{0.010730in}}%
\pgfpathlineto{\pgfqpoint{-0.012614in}{-0.004098in}}%
\pgfpathlineto{\pgfqpoint{-0.020409in}{-0.028091in}}%
\pgfpathlineto{\pgfqpoint{-0.000000in}{-0.013263in}}%
\pgfpathlineto{\pgfqpoint{0.020409in}{-0.028091in}}%
\pgfpathlineto{\pgfqpoint{0.012614in}{-0.004098in}}%
\pgfpathlineto{\pgfqpoint{0.033023in}{0.010730in}}%
\pgfpathlineto{\pgfqpoint{0.007796in}{0.010730in}}%
\pgfpathclose%
\pgfusepath{stroke,fill}%
}%
\begin{pgfscope}%
\pgfsys@transformshift{0.728904in}{2.670851in}%
\pgfsys@useobject{currentmarker}{}%
\end{pgfscope}%
\end{pgfscope}%
\begin{pgfscope}%
\definecolor{textcolor}{rgb}{0.000000,0.000000,0.000000}%
\pgfsetstrokecolor{textcolor}%
\pgfsetfillcolor{textcolor}%
\pgftext[x=0.878904in,y=2.627101in,left,base]{\color{textcolor}\rmfamily\fontsize{9.000000}{10.800000}\selectfont LG+Flow}%
\end{pgfscope}%
\end{pgfpicture}%
\makeatother%
\endgroup%

	\caption{\label{fig:cubic-time} Median solving time (top) and max rank of the computed contraction tree (bottom) of various counters and tensor-based methods, run on benchmarks counting the number of vertex covers of 100 cubic graphs with $n$ vertices. Solving time of datapoints that ran out of time ($1000$ seconds) or memory (48 GB) are not shown. When $n \geq 170$, our contribution \textbf{LG+Flow} is faster than all other methods and finds contraction trees of lower max rank than all other tensor-based methods.}
\end{figure}

\subsection{Implementation Details of \tool{TensorOrder}}
\label{sec:tensors:experiments:implementation}
\tool{TensorOrder} is implemented in Python 3.6. All tensor contractions are performed using \pkg{numpy} 1.15 and 64-bit double precision floats. \tool{TensorOrder} also supports infinite-precision integer arithmetic, but the performance is significantly degraded by limited \pkg{numpy} support. Note that \pkg{numpy} is able to leverage SIMD parallelism for tensor contraction.

Both \textbf{LG} and \textbf{FT} require first finding a tree decomposition. To do this, we leverage three heuristic tree-decomposition solvers: \pkg{Tamaki} \cite{Tamaki17}, \pkg{FlowCutter} \cite{HS18}, and \pkg{htd} \cite{AMW17}. \tool{TensorOrder} therefore has three implementations of \textbf{LG} (using \textbf{LG+Tamaki}, \textbf{LG+Flow}, and \textbf{LG+htd}) and three implementations of \textbf{FT} (using \textbf{FT+Tamaki}, \textbf{FT+Flow}, and \textbf{FT+htd}) for different choices of solver.

All the tree-decomposition solvers we consider are anytime solvers and so each implementation must decide how long to run the solver (this time is included in the measured running time). 
In Algorithm \ref{alg:wmc}, this is governed by the parameter $\alpha$.
\tool{TensorOrder} estimates the time to contract each potential contraction tree (using techniques from the \pkg{einsum} package of \pkg{numpy}) and configures $\alpha$ so that it continues to look for better tree decompositions until it expects to have spent more than half of the running time on finding a tree decomposition.  This strikes a balance between improving and using the contraction trees.
% Note that we determine $\alpha$ empirically in Chapter \ref{ch:parallel}.

\begin{figure}
	\centering
	%% Creator: Matplotlib, PGF backend
%%
%% To include the figure in your LaTeX document, write
%%   \input{<filename>.pgf}
%%
%% Make sure the required packages are loaded in your preamble
%%   \usepackage{pgf}
%%
%% Figures using additional raster images can only be included by \input if
%% they are in the same directory as the main LaTeX file. For loading figures
%% from other directories you can use the `import` package
%%   \usepackage{import}
%% and then include the figures with
%%   \import{<path to file>}{<filename>.pgf}
%%
%% Matplotlib used the following preamble
%%   \usepackage[utf8x]{inputenc}
%%   \usepackage[T1]{fontenc}
%%
\begingroup%
\makeatletter%
\begin{pgfpicture}%
\pgfpathrectangle{\pgfpointorigin}{\pgfqpoint{4.497025in}{2.798215in}}%
\pgfusepath{use as bounding box, clip}%
\begin{pgfscope}%
\pgfsetbuttcap%
\pgfsetmiterjoin%
\definecolor{currentfill}{rgb}{1.000000,1.000000,1.000000}%
\pgfsetfillcolor{currentfill}%
\pgfsetlinewidth{0.000000pt}%
\definecolor{currentstroke}{rgb}{1.000000,1.000000,1.000000}%
\pgfsetstrokecolor{currentstroke}%
\pgfsetdash{}{0pt}%
\pgfpathmoveto{\pgfqpoint{0.000000in}{0.000000in}}%
\pgfpathlineto{\pgfqpoint{4.497025in}{0.000000in}}%
\pgfpathlineto{\pgfqpoint{4.497025in}{2.798215in}}%
\pgfpathlineto{\pgfqpoint{0.000000in}{2.798215in}}%
\pgfpathclose%
\pgfusepath{fill}%
\end{pgfscope}%
\begin{pgfscope}%
\pgfsetbuttcap%
\pgfsetmiterjoin%
\definecolor{currentfill}{rgb}{1.000000,1.000000,1.000000}%
\pgfsetfillcolor{currentfill}%
\pgfsetlinewidth{0.000000pt}%
\definecolor{currentstroke}{rgb}{0.000000,0.000000,0.000000}%
\pgfsetstrokecolor{currentstroke}%
\pgfsetstrokeopacity{0.000000}%
\pgfsetdash{}{0pt}%
\pgfpathmoveto{\pgfqpoint{0.553904in}{0.535823in}}%
\pgfpathlineto{\pgfqpoint{4.312025in}{0.535823in}}%
\pgfpathlineto{\pgfqpoint{4.312025in}{2.613215in}}%
\pgfpathlineto{\pgfqpoint{0.553904in}{2.613215in}}%
\pgfpathclose%
\pgfusepath{fill}%
\end{pgfscope}%
\begin{pgfscope}%
\pgfsetbuttcap%
\pgfsetroundjoin%
\definecolor{currentfill}{rgb}{0.000000,0.000000,0.000000}%
\pgfsetfillcolor{currentfill}%
\pgfsetlinewidth{0.803000pt}%
\definecolor{currentstroke}{rgb}{0.000000,0.000000,0.000000}%
\pgfsetstrokecolor{currentstroke}%
\pgfsetdash{}{0pt}%
\pgfsys@defobject{currentmarker}{\pgfqpoint{0.000000in}{-0.048611in}}{\pgfqpoint{0.000000in}{0.000000in}}{%
\pgfpathmoveto{\pgfqpoint{0.000000in}{0.000000in}}%
\pgfpathlineto{\pgfqpoint{0.000000in}{-0.048611in}}%
\pgfusepath{stroke,fill}%
}%
\begin{pgfscope}%
\pgfsys@transformshift{0.724728in}{0.535823in}%
\pgfsys@useobject{currentmarker}{}%
\end{pgfscope}%
\end{pgfscope}%
\begin{pgfscope}%
\definecolor{textcolor}{rgb}{0.000000,0.000000,0.000000}%
\pgfsetstrokecolor{textcolor}%
\pgfsetfillcolor{textcolor}%
\pgftext[x=0.724728in,y=0.438600in,,top]{\color{textcolor}\rmfamily\fontsize{9.000000}{10.800000}\selectfont \(\displaystyle 50\)}%
\end{pgfscope}%
\begin{pgfscope}%
\pgfsetbuttcap%
\pgfsetroundjoin%
\definecolor{currentfill}{rgb}{0.000000,0.000000,0.000000}%
\pgfsetfillcolor{currentfill}%
\pgfsetlinewidth{0.803000pt}%
\definecolor{currentstroke}{rgb}{0.000000,0.000000,0.000000}%
\pgfsetstrokecolor{currentstroke}%
\pgfsetdash{}{0pt}%
\pgfsys@defobject{currentmarker}{\pgfqpoint{0.000000in}{-0.048611in}}{\pgfqpoint{0.000000in}{0.000000in}}{%
\pgfpathmoveto{\pgfqpoint{0.000000in}{0.000000in}}%
\pgfpathlineto{\pgfqpoint{0.000000in}{-0.048611in}}%
\pgfusepath{stroke,fill}%
}%
\begin{pgfscope}%
\pgfsys@transformshift{1.578846in}{0.535823in}%
\pgfsys@useobject{currentmarker}{}%
\end{pgfscope}%
\end{pgfscope}%
\begin{pgfscope}%
\definecolor{textcolor}{rgb}{0.000000,0.000000,0.000000}%
\pgfsetstrokecolor{textcolor}%
\pgfsetfillcolor{textcolor}%
\pgftext[x=1.578846in,y=0.438600in,,top]{\color{textcolor}\rmfamily\fontsize{9.000000}{10.800000}\selectfont \(\displaystyle 100\)}%
\end{pgfscope}%
\begin{pgfscope}%
\pgfsetbuttcap%
\pgfsetroundjoin%
\definecolor{currentfill}{rgb}{0.000000,0.000000,0.000000}%
\pgfsetfillcolor{currentfill}%
\pgfsetlinewidth{0.803000pt}%
\definecolor{currentstroke}{rgb}{0.000000,0.000000,0.000000}%
\pgfsetstrokecolor{currentstroke}%
\pgfsetdash{}{0pt}%
\pgfsys@defobject{currentmarker}{\pgfqpoint{0.000000in}{-0.048611in}}{\pgfqpoint{0.000000in}{0.000000in}}{%
\pgfpathmoveto{\pgfqpoint{0.000000in}{0.000000in}}%
\pgfpathlineto{\pgfqpoint{0.000000in}{-0.048611in}}%
\pgfusepath{stroke,fill}%
}%
\begin{pgfscope}%
\pgfsys@transformshift{2.432965in}{0.535823in}%
\pgfsys@useobject{currentmarker}{}%
\end{pgfscope}%
\end{pgfscope}%
\begin{pgfscope}%
\definecolor{textcolor}{rgb}{0.000000,0.000000,0.000000}%
\pgfsetstrokecolor{textcolor}%
\pgfsetfillcolor{textcolor}%
\pgftext[x=2.432965in,y=0.438600in,,top]{\color{textcolor}\rmfamily\fontsize{9.000000}{10.800000}\selectfont \(\displaystyle 150\)}%
\end{pgfscope}%
\begin{pgfscope}%
\pgfsetbuttcap%
\pgfsetroundjoin%
\definecolor{currentfill}{rgb}{0.000000,0.000000,0.000000}%
\pgfsetfillcolor{currentfill}%
\pgfsetlinewidth{0.803000pt}%
\definecolor{currentstroke}{rgb}{0.000000,0.000000,0.000000}%
\pgfsetstrokecolor{currentstroke}%
\pgfsetdash{}{0pt}%
\pgfsys@defobject{currentmarker}{\pgfqpoint{0.000000in}{-0.048611in}}{\pgfqpoint{0.000000in}{0.000000in}}{%
\pgfpathmoveto{\pgfqpoint{0.000000in}{0.000000in}}%
\pgfpathlineto{\pgfqpoint{0.000000in}{-0.048611in}}%
\pgfusepath{stroke,fill}%
}%
\begin{pgfscope}%
\pgfsys@transformshift{3.287083in}{0.535823in}%
\pgfsys@useobject{currentmarker}{}%
\end{pgfscope}%
\end{pgfscope}%
\begin{pgfscope}%
\definecolor{textcolor}{rgb}{0.000000,0.000000,0.000000}%
\pgfsetstrokecolor{textcolor}%
\pgfsetfillcolor{textcolor}%
\pgftext[x=3.287083in,y=0.438600in,,top]{\color{textcolor}\rmfamily\fontsize{9.000000}{10.800000}\selectfont \(\displaystyle 200\)}%
\end{pgfscope}%
\begin{pgfscope}%
\pgfsetbuttcap%
\pgfsetroundjoin%
\definecolor{currentfill}{rgb}{0.000000,0.000000,0.000000}%
\pgfsetfillcolor{currentfill}%
\pgfsetlinewidth{0.803000pt}%
\definecolor{currentstroke}{rgb}{0.000000,0.000000,0.000000}%
\pgfsetstrokecolor{currentstroke}%
\pgfsetdash{}{0pt}%
\pgfsys@defobject{currentmarker}{\pgfqpoint{0.000000in}{-0.048611in}}{\pgfqpoint{0.000000in}{0.000000in}}{%
\pgfpathmoveto{\pgfqpoint{0.000000in}{0.000000in}}%
\pgfpathlineto{\pgfqpoint{0.000000in}{-0.048611in}}%
\pgfusepath{stroke,fill}%
}%
\begin{pgfscope}%
\pgfsys@transformshift{4.141201in}{0.535823in}%
\pgfsys@useobject{currentmarker}{}%
\end{pgfscope}%
\end{pgfscope}%
\begin{pgfscope}%
\definecolor{textcolor}{rgb}{0.000000,0.000000,0.000000}%
\pgfsetstrokecolor{textcolor}%
\pgfsetfillcolor{textcolor}%
\pgftext[x=4.141201in,y=0.438600in,,top]{\color{textcolor}\rmfamily\fontsize{9.000000}{10.800000}\selectfont \(\displaystyle 250\)}%
\end{pgfscope}%
\begin{pgfscope}%
\definecolor{textcolor}{rgb}{0.000000,0.000000,0.000000}%
\pgfsetstrokecolor{textcolor}%
\pgfsetfillcolor{textcolor}%
\pgftext[x=2.432965in,y=0.272655in,,top]{\color{textcolor}\rmfamily\fontsize{10.000000}{12.000000}\selectfont \(\displaystyle n\): Number of vertices}%
\end{pgfscope}%
\begin{pgfscope}%
\pgfsetbuttcap%
\pgfsetroundjoin%
\definecolor{currentfill}{rgb}{0.000000,0.000000,0.000000}%
\pgfsetfillcolor{currentfill}%
\pgfsetlinewidth{0.803000pt}%
\definecolor{currentstroke}{rgb}{0.000000,0.000000,0.000000}%
\pgfsetstrokecolor{currentstroke}%
\pgfsetdash{}{0pt}%
\pgfsys@defobject{currentmarker}{\pgfqpoint{-0.048611in}{0.000000in}}{\pgfqpoint{0.000000in}{0.000000in}}{%
\pgfpathmoveto{\pgfqpoint{0.000000in}{0.000000in}}%
\pgfpathlineto{\pgfqpoint{-0.048611in}{0.000000in}}%
\pgfusepath{stroke,fill}%
}%
\begin{pgfscope}%
\pgfsys@transformshift{0.553904in}{0.762779in}%
\pgfsys@useobject{currentmarker}{}%
\end{pgfscope}%
\end{pgfscope}%
\begin{pgfscope}%
\definecolor{textcolor}{rgb}{0.000000,0.000000,0.000000}%
\pgfsetstrokecolor{textcolor}%
\pgfsetfillcolor{textcolor}%
\pgftext[x=0.328211in,y=0.719734in,left,base]{\color{textcolor}\rmfamily\fontsize{9.000000}{10.800000}\selectfont \(\displaystyle 10\)}%
\end{pgfscope}%
\begin{pgfscope}%
\pgfsetbuttcap%
\pgfsetroundjoin%
\definecolor{currentfill}{rgb}{0.000000,0.000000,0.000000}%
\pgfsetfillcolor{currentfill}%
\pgfsetlinewidth{0.803000pt}%
\definecolor{currentstroke}{rgb}{0.000000,0.000000,0.000000}%
\pgfsetstrokecolor{currentstroke}%
\pgfsetdash{}{0pt}%
\pgfsys@defobject{currentmarker}{\pgfqpoint{-0.048611in}{0.000000in}}{\pgfqpoint{0.000000in}{0.000000in}}{%
\pgfpathmoveto{\pgfqpoint{0.000000in}{0.000000in}}%
\pgfpathlineto{\pgfqpoint{-0.048611in}{0.000000in}}%
\pgfusepath{stroke,fill}%
}%
\begin{pgfscope}%
\pgfsys@transformshift{0.553904in}{1.094101in}%
\pgfsys@useobject{currentmarker}{}%
\end{pgfscope}%
\end{pgfscope}%
\begin{pgfscope}%
\definecolor{textcolor}{rgb}{0.000000,0.000000,0.000000}%
\pgfsetstrokecolor{textcolor}%
\pgfsetfillcolor{textcolor}%
\pgftext[x=0.328211in,y=1.051056in,left,base]{\color{textcolor}\rmfamily\fontsize{9.000000}{10.800000}\selectfont \(\displaystyle 15\)}%
\end{pgfscope}%
\begin{pgfscope}%
\pgfsetbuttcap%
\pgfsetroundjoin%
\definecolor{currentfill}{rgb}{0.000000,0.000000,0.000000}%
\pgfsetfillcolor{currentfill}%
\pgfsetlinewidth{0.803000pt}%
\definecolor{currentstroke}{rgb}{0.000000,0.000000,0.000000}%
\pgfsetstrokecolor{currentstroke}%
\pgfsetdash{}{0pt}%
\pgfsys@defobject{currentmarker}{\pgfqpoint{-0.048611in}{0.000000in}}{\pgfqpoint{0.000000in}{0.000000in}}{%
\pgfpathmoveto{\pgfqpoint{0.000000in}{0.000000in}}%
\pgfpathlineto{\pgfqpoint{-0.048611in}{0.000000in}}%
\pgfusepath{stroke,fill}%
}%
\begin{pgfscope}%
\pgfsys@transformshift{0.553904in}{1.425424in}%
\pgfsys@useobject{currentmarker}{}%
\end{pgfscope}%
\end{pgfscope}%
\begin{pgfscope}%
\definecolor{textcolor}{rgb}{0.000000,0.000000,0.000000}%
\pgfsetstrokecolor{textcolor}%
\pgfsetfillcolor{textcolor}%
\pgftext[x=0.328211in,y=1.382379in,left,base]{\color{textcolor}\rmfamily\fontsize{9.000000}{10.800000}\selectfont \(\displaystyle 20\)}%
\end{pgfscope}%
\begin{pgfscope}%
\pgfsetbuttcap%
\pgfsetroundjoin%
\definecolor{currentfill}{rgb}{0.000000,0.000000,0.000000}%
\pgfsetfillcolor{currentfill}%
\pgfsetlinewidth{0.803000pt}%
\definecolor{currentstroke}{rgb}{0.000000,0.000000,0.000000}%
\pgfsetstrokecolor{currentstroke}%
\pgfsetdash{}{0pt}%
\pgfsys@defobject{currentmarker}{\pgfqpoint{-0.048611in}{0.000000in}}{\pgfqpoint{0.000000in}{0.000000in}}{%
\pgfpathmoveto{\pgfqpoint{0.000000in}{0.000000in}}%
\pgfpathlineto{\pgfqpoint{-0.048611in}{0.000000in}}%
\pgfusepath{stroke,fill}%
}%
\begin{pgfscope}%
\pgfsys@transformshift{0.553904in}{1.756746in}%
\pgfsys@useobject{currentmarker}{}%
\end{pgfscope}%
\end{pgfscope}%
\begin{pgfscope}%
\definecolor{textcolor}{rgb}{0.000000,0.000000,0.000000}%
\pgfsetstrokecolor{textcolor}%
\pgfsetfillcolor{textcolor}%
\pgftext[x=0.328211in,y=1.713701in,left,base]{\color{textcolor}\rmfamily\fontsize{9.000000}{10.800000}\selectfont \(\displaystyle 25\)}%
\end{pgfscope}%
\begin{pgfscope}%
\pgfsetbuttcap%
\pgfsetroundjoin%
\definecolor{currentfill}{rgb}{0.000000,0.000000,0.000000}%
\pgfsetfillcolor{currentfill}%
\pgfsetlinewidth{0.803000pt}%
\definecolor{currentstroke}{rgb}{0.000000,0.000000,0.000000}%
\pgfsetstrokecolor{currentstroke}%
\pgfsetdash{}{0pt}%
\pgfsys@defobject{currentmarker}{\pgfqpoint{-0.048611in}{0.000000in}}{\pgfqpoint{0.000000in}{0.000000in}}{%
\pgfpathmoveto{\pgfqpoint{0.000000in}{0.000000in}}%
\pgfpathlineto{\pgfqpoint{-0.048611in}{0.000000in}}%
\pgfusepath{stroke,fill}%
}%
\begin{pgfscope}%
\pgfsys@transformshift{0.553904in}{2.088069in}%
\pgfsys@useobject{currentmarker}{}%
\end{pgfscope}%
\end{pgfscope}%
\begin{pgfscope}%
\definecolor{textcolor}{rgb}{0.000000,0.000000,0.000000}%
\pgfsetstrokecolor{textcolor}%
\pgfsetfillcolor{textcolor}%
\pgftext[x=0.328211in,y=2.045024in,left,base]{\color{textcolor}\rmfamily\fontsize{9.000000}{10.800000}\selectfont \(\displaystyle 30\)}%
\end{pgfscope}%
\begin{pgfscope}%
\pgfsetbuttcap%
\pgfsetroundjoin%
\definecolor{currentfill}{rgb}{0.000000,0.000000,0.000000}%
\pgfsetfillcolor{currentfill}%
\pgfsetlinewidth{0.803000pt}%
\definecolor{currentstroke}{rgb}{0.000000,0.000000,0.000000}%
\pgfsetstrokecolor{currentstroke}%
\pgfsetdash{}{0pt}%
\pgfsys@defobject{currentmarker}{\pgfqpoint{-0.048611in}{0.000000in}}{\pgfqpoint{0.000000in}{0.000000in}}{%
\pgfpathmoveto{\pgfqpoint{0.000000in}{0.000000in}}%
\pgfpathlineto{\pgfqpoint{-0.048611in}{0.000000in}}%
\pgfusepath{stroke,fill}%
}%
\begin{pgfscope}%
\pgfsys@transformshift{0.553904in}{2.419391in}%
\pgfsys@useobject{currentmarker}{}%
\end{pgfscope}%
\end{pgfscope}%
\begin{pgfscope}%
\definecolor{textcolor}{rgb}{0.000000,0.000000,0.000000}%
\pgfsetstrokecolor{textcolor}%
\pgfsetfillcolor{textcolor}%
\pgftext[x=0.328211in,y=2.376346in,left,base]{\color{textcolor}\rmfamily\fontsize{9.000000}{10.800000}\selectfont \(\displaystyle 35\)}%
\end{pgfscope}%
\begin{pgfscope}%
\definecolor{textcolor}{rgb}{0.000000,0.000000,0.000000}%
\pgfsetstrokecolor{textcolor}%
\pgfsetfillcolor{textcolor}%
\pgftext[x=0.272655in,y=1.574519in,,bottom,rotate=90.000000]{\color{textcolor}\rmfamily\fontsize{10.000000}{12.000000}\selectfont Width of decomposition}%
\end{pgfscope}%
\begin{pgfscope}%
\pgfpathrectangle{\pgfqpoint{0.553904in}{0.535823in}}{\pgfqpoint{3.758121in}{2.077392in}}%
\pgfusepath{clip}%
\pgfsetrectcap%
\pgfsetroundjoin%
\pgfsetlinewidth{1.003750pt}%
\definecolor{currentstroke}{rgb}{0.756863,0.117647,0.588235}%
\pgfsetstrokecolor{currentstroke}%
\pgfsetdash{}{0pt}%
\pgfpathmoveto{\pgfqpoint{0.724728in}{0.762779in}}%
\pgfpathlineto{\pgfqpoint{0.895552in}{0.829043in}}%
\pgfpathlineto{\pgfqpoint{1.066375in}{0.895308in}}%
\pgfpathlineto{\pgfqpoint{1.237199in}{0.961572in}}%
\pgfpathlineto{\pgfqpoint{1.408023in}{1.094101in}}%
\pgfpathlineto{\pgfqpoint{1.578846in}{1.160366in}}%
\pgfpathlineto{\pgfqpoint{1.749670in}{1.226630in}}%
\pgfpathlineto{\pgfqpoint{1.920494in}{1.359159in}}%
\pgfpathlineto{\pgfqpoint{2.091317in}{1.425424in}}%
\pgfpathlineto{\pgfqpoint{2.262141in}{1.557953in}}%
\pgfpathlineto{\pgfqpoint{2.432965in}{1.624217in}}%
\pgfpathlineto{\pgfqpoint{2.603788in}{1.690482in}}%
\pgfpathlineto{\pgfqpoint{2.774612in}{1.756746in}}%
\pgfpathlineto{\pgfqpoint{2.945436in}{1.889275in}}%
\pgfpathlineto{\pgfqpoint{3.116259in}{1.955540in}}%
\pgfpathlineto{\pgfqpoint{3.287083in}{2.088069in}}%
\pgfpathlineto{\pgfqpoint{3.457907in}{2.154333in}}%
\pgfpathlineto{\pgfqpoint{3.628730in}{2.220598in}}%
\pgfpathlineto{\pgfqpoint{3.799554in}{2.353127in}}%
\pgfpathlineto{\pgfqpoint{3.970378in}{2.419391in}}%
\pgfpathlineto{\pgfqpoint{4.141201in}{2.518788in}}%
\pgfusepath{stroke}%
\end{pgfscope}%
\begin{pgfscope}%
\pgfpathrectangle{\pgfqpoint{0.553904in}{0.535823in}}{\pgfqpoint{3.758121in}{2.077392in}}%
\pgfusepath{clip}%
\pgfsetbuttcap%
\pgfsetroundjoin%
\definecolor{currentfill}{rgb}{0.756863,0.117647,0.588235}%
\pgfsetfillcolor{currentfill}%
\pgfsetlinewidth{0.501875pt}%
\definecolor{currentstroke}{rgb}{0.000000,0.000000,0.000000}%
\pgfsetstrokecolor{currentstroke}%
\pgfsetdash{}{0pt}%
\pgfsys@defobject{currentmarker}{\pgfqpoint{-0.034722in}{-0.034722in}}{\pgfqpoint{0.034722in}{0.034722in}}{%
\pgfpathmoveto{\pgfqpoint{0.000000in}{-0.034722in}}%
\pgfpathcurveto{\pgfqpoint{0.009208in}{-0.034722in}}{\pgfqpoint{0.018041in}{-0.031064in}}{\pgfqpoint{0.024552in}{-0.024552in}}%
\pgfpathcurveto{\pgfqpoint{0.031064in}{-0.018041in}}{\pgfqpoint{0.034722in}{-0.009208in}}{\pgfqpoint{0.034722in}{0.000000in}}%
\pgfpathcurveto{\pgfqpoint{0.034722in}{0.009208in}}{\pgfqpoint{0.031064in}{0.018041in}}{\pgfqpoint{0.024552in}{0.024552in}}%
\pgfpathcurveto{\pgfqpoint{0.018041in}{0.031064in}}{\pgfqpoint{0.009208in}{0.034722in}}{\pgfqpoint{0.000000in}{0.034722in}}%
\pgfpathcurveto{\pgfqpoint{-0.009208in}{0.034722in}}{\pgfqpoint{-0.018041in}{0.031064in}}{\pgfqpoint{-0.024552in}{0.024552in}}%
\pgfpathcurveto{\pgfqpoint{-0.031064in}{0.018041in}}{\pgfqpoint{-0.034722in}{0.009208in}}{\pgfqpoint{-0.034722in}{0.000000in}}%
\pgfpathcurveto{\pgfqpoint{-0.034722in}{-0.009208in}}{\pgfqpoint{-0.031064in}{-0.018041in}}{\pgfqpoint{-0.024552in}{-0.024552in}}%
\pgfpathcurveto{\pgfqpoint{-0.018041in}{-0.031064in}}{\pgfqpoint{-0.009208in}{-0.034722in}}{\pgfqpoint{0.000000in}{-0.034722in}}%
\pgfpathclose%
\pgfusepath{stroke,fill}%
}%
\begin{pgfscope}%
\pgfsys@transformshift{0.724728in}{0.762779in}%
\pgfsys@useobject{currentmarker}{}%
\end{pgfscope}%
\begin{pgfscope}%
\pgfsys@transformshift{0.895552in}{0.829043in}%
\pgfsys@useobject{currentmarker}{}%
\end{pgfscope}%
\begin{pgfscope}%
\pgfsys@transformshift{1.066375in}{0.895308in}%
\pgfsys@useobject{currentmarker}{}%
\end{pgfscope}%
\begin{pgfscope}%
\pgfsys@transformshift{1.237199in}{0.961572in}%
\pgfsys@useobject{currentmarker}{}%
\end{pgfscope}%
\begin{pgfscope}%
\pgfsys@transformshift{1.408023in}{1.094101in}%
\pgfsys@useobject{currentmarker}{}%
\end{pgfscope}%
\begin{pgfscope}%
\pgfsys@transformshift{1.578846in}{1.160366in}%
\pgfsys@useobject{currentmarker}{}%
\end{pgfscope}%
\begin{pgfscope}%
\pgfsys@transformshift{1.749670in}{1.226630in}%
\pgfsys@useobject{currentmarker}{}%
\end{pgfscope}%
\begin{pgfscope}%
\pgfsys@transformshift{1.920494in}{1.359159in}%
\pgfsys@useobject{currentmarker}{}%
\end{pgfscope}%
\begin{pgfscope}%
\pgfsys@transformshift{2.091317in}{1.425424in}%
\pgfsys@useobject{currentmarker}{}%
\end{pgfscope}%
\begin{pgfscope}%
\pgfsys@transformshift{2.262141in}{1.557953in}%
\pgfsys@useobject{currentmarker}{}%
\end{pgfscope}%
\begin{pgfscope}%
\pgfsys@transformshift{2.432965in}{1.624217in}%
\pgfsys@useobject{currentmarker}{}%
\end{pgfscope}%
\begin{pgfscope}%
\pgfsys@transformshift{2.603788in}{1.690482in}%
\pgfsys@useobject{currentmarker}{}%
\end{pgfscope}%
\begin{pgfscope}%
\pgfsys@transformshift{2.774612in}{1.756746in}%
\pgfsys@useobject{currentmarker}{}%
\end{pgfscope}%
\begin{pgfscope}%
\pgfsys@transformshift{2.945436in}{1.889275in}%
\pgfsys@useobject{currentmarker}{}%
\end{pgfscope}%
\begin{pgfscope}%
\pgfsys@transformshift{3.116259in}{1.955540in}%
\pgfsys@useobject{currentmarker}{}%
\end{pgfscope}%
\begin{pgfscope}%
\pgfsys@transformshift{3.287083in}{2.088069in}%
\pgfsys@useobject{currentmarker}{}%
\end{pgfscope}%
\begin{pgfscope}%
\pgfsys@transformshift{3.457907in}{2.154333in}%
\pgfsys@useobject{currentmarker}{}%
\end{pgfscope}%
\begin{pgfscope}%
\pgfsys@transformshift{3.628730in}{2.220598in}%
\pgfsys@useobject{currentmarker}{}%
\end{pgfscope}%
\begin{pgfscope}%
\pgfsys@transformshift{3.799554in}{2.353127in}%
\pgfsys@useobject{currentmarker}{}%
\end{pgfscope}%
\begin{pgfscope}%
\pgfsys@transformshift{3.970378in}{2.419391in}%
\pgfsys@useobject{currentmarker}{}%
\end{pgfscope}%
\begin{pgfscope}%
\pgfsys@transformshift{4.141201in}{2.518788in}%
\pgfsys@useobject{currentmarker}{}%
\end{pgfscope}%
\end{pgfscope}%
\begin{pgfscope}%
\pgfpathrectangle{\pgfqpoint{0.553904in}{0.535823in}}{\pgfqpoint{3.758121in}{2.077392in}}%
\pgfusepath{clip}%
\pgfsetrectcap%
\pgfsetroundjoin%
\pgfsetlinewidth{1.003750pt}%
\definecolor{currentstroke}{rgb}{0.007843,0.219608,0.501961}%
\pgfsetstrokecolor{currentstroke}%
\pgfsetdash{}{0pt}%
\pgfpathmoveto{\pgfqpoint{0.724728in}{0.630250in}}%
\pgfpathlineto{\pgfqpoint{0.895552in}{0.696514in}}%
\pgfpathlineto{\pgfqpoint{1.066375in}{0.829043in}}%
\pgfpathlineto{\pgfqpoint{1.237199in}{0.895308in}}%
\pgfpathlineto{\pgfqpoint{1.408023in}{0.961572in}}%
\pgfpathlineto{\pgfqpoint{1.578846in}{1.027837in}}%
\pgfpathlineto{\pgfqpoint{1.749670in}{1.094101in}}%
\pgfpathlineto{\pgfqpoint{1.920494in}{1.226630in}}%
\pgfpathlineto{\pgfqpoint{2.091317in}{1.292895in}}%
\pgfpathlineto{\pgfqpoint{2.262141in}{1.425424in}}%
\pgfpathlineto{\pgfqpoint{2.432965in}{1.491688in}}%
\pgfpathlineto{\pgfqpoint{2.603788in}{1.557953in}}%
\pgfpathlineto{\pgfqpoint{2.774612in}{1.624217in}}%
\pgfpathlineto{\pgfqpoint{2.945436in}{1.756746in}}%
\pgfpathlineto{\pgfqpoint{3.116259in}{1.823011in}}%
\pgfpathlineto{\pgfqpoint{3.287083in}{1.955540in}}%
\pgfpathlineto{\pgfqpoint{3.457907in}{1.955540in}}%
\pgfpathlineto{\pgfqpoint{3.628730in}{2.088069in}}%
\pgfpathlineto{\pgfqpoint{3.799554in}{2.154333in}}%
\pgfpathlineto{\pgfqpoint{3.970378in}{2.220598in}}%
\pgfpathlineto{\pgfqpoint{4.141201in}{2.353127in}}%
\pgfusepath{stroke}%
\end{pgfscope}%
\begin{pgfscope}%
\pgfpathrectangle{\pgfqpoint{0.553904in}{0.535823in}}{\pgfqpoint{3.758121in}{2.077392in}}%
\pgfusepath{clip}%
\pgfsetbuttcap%
\pgfsetmiterjoin%
\definecolor{currentfill}{rgb}{0.007843,0.219608,0.501961}%
\pgfsetfillcolor{currentfill}%
\pgfsetlinewidth{0.501875pt}%
\definecolor{currentstroke}{rgb}{0.000000,0.000000,0.000000}%
\pgfsetstrokecolor{currentstroke}%
\pgfsetdash{}{0pt}%
\pgfsys@defobject{currentmarker}{\pgfqpoint{-0.034722in}{-0.034722in}}{\pgfqpoint{0.034722in}{0.034722in}}{%
\pgfpathmoveto{\pgfqpoint{-0.000000in}{-0.034722in}}%
\pgfpathlineto{\pgfqpoint{0.034722in}{0.034722in}}%
\pgfpathlineto{\pgfqpoint{-0.034722in}{0.034722in}}%
\pgfpathclose%
\pgfusepath{stroke,fill}%
}%
\begin{pgfscope}%
\pgfsys@transformshift{0.724728in}{0.630250in}%
\pgfsys@useobject{currentmarker}{}%
\end{pgfscope}%
\begin{pgfscope}%
\pgfsys@transformshift{0.895552in}{0.696514in}%
\pgfsys@useobject{currentmarker}{}%
\end{pgfscope}%
\begin{pgfscope}%
\pgfsys@transformshift{1.066375in}{0.829043in}%
\pgfsys@useobject{currentmarker}{}%
\end{pgfscope}%
\begin{pgfscope}%
\pgfsys@transformshift{1.237199in}{0.895308in}%
\pgfsys@useobject{currentmarker}{}%
\end{pgfscope}%
\begin{pgfscope}%
\pgfsys@transformshift{1.408023in}{0.961572in}%
\pgfsys@useobject{currentmarker}{}%
\end{pgfscope}%
\begin{pgfscope}%
\pgfsys@transformshift{1.578846in}{1.027837in}%
\pgfsys@useobject{currentmarker}{}%
\end{pgfscope}%
\begin{pgfscope}%
\pgfsys@transformshift{1.749670in}{1.094101in}%
\pgfsys@useobject{currentmarker}{}%
\end{pgfscope}%
\begin{pgfscope}%
\pgfsys@transformshift{1.920494in}{1.226630in}%
\pgfsys@useobject{currentmarker}{}%
\end{pgfscope}%
\begin{pgfscope}%
\pgfsys@transformshift{2.091317in}{1.292895in}%
\pgfsys@useobject{currentmarker}{}%
\end{pgfscope}%
\begin{pgfscope}%
\pgfsys@transformshift{2.262141in}{1.425424in}%
\pgfsys@useobject{currentmarker}{}%
\end{pgfscope}%
\begin{pgfscope}%
\pgfsys@transformshift{2.432965in}{1.491688in}%
\pgfsys@useobject{currentmarker}{}%
\end{pgfscope}%
\begin{pgfscope}%
\pgfsys@transformshift{2.603788in}{1.557953in}%
\pgfsys@useobject{currentmarker}{}%
\end{pgfscope}%
\begin{pgfscope}%
\pgfsys@transformshift{2.774612in}{1.624217in}%
\pgfsys@useobject{currentmarker}{}%
\end{pgfscope}%
\begin{pgfscope}%
\pgfsys@transformshift{2.945436in}{1.756746in}%
\pgfsys@useobject{currentmarker}{}%
\end{pgfscope}%
\begin{pgfscope}%
\pgfsys@transformshift{3.116259in}{1.823011in}%
\pgfsys@useobject{currentmarker}{}%
\end{pgfscope}%
\begin{pgfscope}%
\pgfsys@transformshift{3.287083in}{1.955540in}%
\pgfsys@useobject{currentmarker}{}%
\end{pgfscope}%
\begin{pgfscope}%
\pgfsys@transformshift{3.457907in}{1.955540in}%
\pgfsys@useobject{currentmarker}{}%
\end{pgfscope}%
\begin{pgfscope}%
\pgfsys@transformshift{3.628730in}{2.088069in}%
\pgfsys@useobject{currentmarker}{}%
\end{pgfscope}%
\begin{pgfscope}%
\pgfsys@transformshift{3.799554in}{2.154333in}%
\pgfsys@useobject{currentmarker}{}%
\end{pgfscope}%
\begin{pgfscope}%
\pgfsys@transformshift{3.970378in}{2.220598in}%
\pgfsys@useobject{currentmarker}{}%
\end{pgfscope}%
\begin{pgfscope}%
\pgfsys@transformshift{4.141201in}{2.353127in}%
\pgfsys@useobject{currentmarker}{}%
\end{pgfscope}%
\end{pgfscope}%
\begin{pgfscope}%
\pgfpathrectangle{\pgfqpoint{0.553904in}{0.535823in}}{\pgfqpoint{3.758121in}{2.077392in}}%
\pgfusepath{clip}%
\pgfsetrectcap%
\pgfsetroundjoin%
\pgfsetlinewidth{1.003750pt}%
\definecolor{currentstroke}{rgb}{0.654902,0.909804,0.192157}%
\pgfsetstrokecolor{currentstroke}%
\pgfsetdash{}{0pt}%
\pgfpathmoveto{\pgfqpoint{0.724728in}{0.630250in}}%
\pgfpathlineto{\pgfqpoint{0.895552in}{0.762779in}}%
\pgfpathlineto{\pgfqpoint{1.066375in}{0.829043in}}%
\pgfpathlineto{\pgfqpoint{1.237199in}{0.895308in}}%
\pgfpathlineto{\pgfqpoint{1.408023in}{0.961572in}}%
\pgfpathlineto{\pgfqpoint{1.578846in}{1.027837in}}%
\pgfpathlineto{\pgfqpoint{1.749670in}{1.160366in}}%
\pgfpathlineto{\pgfqpoint{1.920494in}{1.226630in}}%
\pgfpathlineto{\pgfqpoint{2.091317in}{1.292895in}}%
\pgfpathlineto{\pgfqpoint{2.262141in}{1.425424in}}%
\pgfpathlineto{\pgfqpoint{2.432965in}{1.491688in}}%
\pgfpathlineto{\pgfqpoint{2.603788in}{1.557953in}}%
\pgfpathlineto{\pgfqpoint{2.774612in}{1.624217in}}%
\pgfpathlineto{\pgfqpoint{2.945436in}{1.723614in}}%
\pgfpathlineto{\pgfqpoint{3.116259in}{1.823011in}}%
\pgfpathlineto{\pgfqpoint{3.287083in}{1.889275in}}%
\pgfpathlineto{\pgfqpoint{3.457907in}{1.955540in}}%
\pgfpathlineto{\pgfqpoint{3.628730in}{2.021804in}}%
\pgfpathlineto{\pgfqpoint{3.799554in}{2.154333in}}%
\pgfpathlineto{\pgfqpoint{3.970378in}{2.220598in}}%
\pgfpathlineto{\pgfqpoint{4.141201in}{2.319995in}}%
\pgfusepath{stroke}%
\end{pgfscope}%
\begin{pgfscope}%
\pgfpathrectangle{\pgfqpoint{0.553904in}{0.535823in}}{\pgfqpoint{3.758121in}{2.077392in}}%
\pgfusepath{clip}%
\pgfsetbuttcap%
\pgfsetmiterjoin%
\definecolor{currentfill}{rgb}{0.654902,0.909804,0.192157}%
\pgfsetfillcolor{currentfill}%
\pgfsetlinewidth{0.501875pt}%
\definecolor{currentstroke}{rgb}{0.000000,0.000000,0.000000}%
\pgfsetstrokecolor{currentstroke}%
\pgfsetdash{}{0pt}%
\pgfsys@defobject{currentmarker}{\pgfqpoint{-0.034722in}{-0.034722in}}{\pgfqpoint{0.034722in}{0.034722in}}{%
\pgfpathmoveto{\pgfqpoint{-0.034722in}{-0.034722in}}%
\pgfpathlineto{\pgfqpoint{0.034722in}{-0.034722in}}%
\pgfpathlineto{\pgfqpoint{0.034722in}{0.034722in}}%
\pgfpathlineto{\pgfqpoint{-0.034722in}{0.034722in}}%
\pgfpathclose%
\pgfusepath{stroke,fill}%
}%
\begin{pgfscope}%
\pgfsys@transformshift{0.724728in}{0.630250in}%
\pgfsys@useobject{currentmarker}{}%
\end{pgfscope}%
\begin{pgfscope}%
\pgfsys@transformshift{0.895552in}{0.762779in}%
\pgfsys@useobject{currentmarker}{}%
\end{pgfscope}%
\begin{pgfscope}%
\pgfsys@transformshift{1.066375in}{0.829043in}%
\pgfsys@useobject{currentmarker}{}%
\end{pgfscope}%
\begin{pgfscope}%
\pgfsys@transformshift{1.237199in}{0.895308in}%
\pgfsys@useobject{currentmarker}{}%
\end{pgfscope}%
\begin{pgfscope}%
\pgfsys@transformshift{1.408023in}{0.961572in}%
\pgfsys@useobject{currentmarker}{}%
\end{pgfscope}%
\begin{pgfscope}%
\pgfsys@transformshift{1.578846in}{1.027837in}%
\pgfsys@useobject{currentmarker}{}%
\end{pgfscope}%
\begin{pgfscope}%
\pgfsys@transformshift{1.749670in}{1.160366in}%
\pgfsys@useobject{currentmarker}{}%
\end{pgfscope}%
\begin{pgfscope}%
\pgfsys@transformshift{1.920494in}{1.226630in}%
\pgfsys@useobject{currentmarker}{}%
\end{pgfscope}%
\begin{pgfscope}%
\pgfsys@transformshift{2.091317in}{1.292895in}%
\pgfsys@useobject{currentmarker}{}%
\end{pgfscope}%
\begin{pgfscope}%
\pgfsys@transformshift{2.262141in}{1.425424in}%
\pgfsys@useobject{currentmarker}{}%
\end{pgfscope}%
\begin{pgfscope}%
\pgfsys@transformshift{2.432965in}{1.491688in}%
\pgfsys@useobject{currentmarker}{}%
\end{pgfscope}%
\begin{pgfscope}%
\pgfsys@transformshift{2.603788in}{1.557953in}%
\pgfsys@useobject{currentmarker}{}%
\end{pgfscope}%
\begin{pgfscope}%
\pgfsys@transformshift{2.774612in}{1.624217in}%
\pgfsys@useobject{currentmarker}{}%
\end{pgfscope}%
\begin{pgfscope}%
\pgfsys@transformshift{2.945436in}{1.723614in}%
\pgfsys@useobject{currentmarker}{}%
\end{pgfscope}%
\begin{pgfscope}%
\pgfsys@transformshift{3.116259in}{1.823011in}%
\pgfsys@useobject{currentmarker}{}%
\end{pgfscope}%
\begin{pgfscope}%
\pgfsys@transformshift{3.287083in}{1.889275in}%
\pgfsys@useobject{currentmarker}{}%
\end{pgfscope}%
\begin{pgfscope}%
\pgfsys@transformshift{3.457907in}{1.955540in}%
\pgfsys@useobject{currentmarker}{}%
\end{pgfscope}%
\begin{pgfscope}%
\pgfsys@transformshift{3.628730in}{2.021804in}%
\pgfsys@useobject{currentmarker}{}%
\end{pgfscope}%
\begin{pgfscope}%
\pgfsys@transformshift{3.799554in}{2.154333in}%
\pgfsys@useobject{currentmarker}{}%
\end{pgfscope}%
\begin{pgfscope}%
\pgfsys@transformshift{3.970378in}{2.220598in}%
\pgfsys@useobject{currentmarker}{}%
\end{pgfscope}%
\begin{pgfscope}%
\pgfsys@transformshift{4.141201in}{2.319995in}%
\pgfsys@useobject{currentmarker}{}%
\end{pgfscope}%
\end{pgfscope}%
\begin{pgfscope}%
\pgfpathrectangle{\pgfqpoint{0.553904in}{0.535823in}}{\pgfqpoint{3.758121in}{2.077392in}}%
\pgfusepath{clip}%
\pgfsetrectcap%
\pgfsetroundjoin%
\pgfsetlinewidth{1.003750pt}%
\definecolor{currentstroke}{rgb}{0.525490,0.843137,0.662745}%
\pgfsetstrokecolor{currentstroke}%
\pgfsetdash{}{0pt}%
\pgfpathmoveto{\pgfqpoint{0.724728in}{0.696514in}}%
\pgfpathlineto{\pgfqpoint{0.895552in}{0.762779in}}%
\pgfpathlineto{\pgfqpoint{1.066375in}{0.829043in}}%
\pgfpathlineto{\pgfqpoint{1.237199in}{0.895308in}}%
\pgfpathlineto{\pgfqpoint{1.408023in}{1.027837in}}%
\pgfpathlineto{\pgfqpoint{1.578846in}{1.094101in}}%
\pgfpathlineto{\pgfqpoint{1.749670in}{1.160366in}}%
\pgfpathlineto{\pgfqpoint{1.920494in}{1.226630in}}%
\pgfpathlineto{\pgfqpoint{2.091317in}{1.292895in}}%
\pgfpathlineto{\pgfqpoint{2.262141in}{1.359159in}}%
\pgfpathlineto{\pgfqpoint{2.432965in}{1.425424in}}%
\pgfpathlineto{\pgfqpoint{2.603788in}{1.524820in}}%
\pgfpathlineto{\pgfqpoint{2.774612in}{1.624217in}}%
\pgfpathlineto{\pgfqpoint{2.945436in}{1.690482in}}%
\pgfpathlineto{\pgfqpoint{3.116259in}{1.756746in}}%
\pgfpathlineto{\pgfqpoint{3.287083in}{1.823011in}}%
\pgfpathlineto{\pgfqpoint{3.457907in}{1.889275in}}%
\pgfpathlineto{\pgfqpoint{3.628730in}{1.955540in}}%
\pgfpathlineto{\pgfqpoint{3.799554in}{2.088069in}}%
\pgfpathlineto{\pgfqpoint{3.970378in}{2.121201in}}%
\pgfpathlineto{\pgfqpoint{4.141201in}{2.220598in}}%
\pgfusepath{stroke}%
\end{pgfscope}%
\begin{pgfscope}%
\pgfpathrectangle{\pgfqpoint{0.553904in}{0.535823in}}{\pgfqpoint{3.758121in}{2.077392in}}%
\pgfusepath{clip}%
\pgfsetbuttcap%
\pgfsetbeveljoin%
\definecolor{currentfill}{rgb}{0.525490,0.843137,0.662745}%
\pgfsetfillcolor{currentfill}%
\pgfsetlinewidth{0.501875pt}%
\definecolor{currentstroke}{rgb}{0.000000,0.000000,0.000000}%
\pgfsetstrokecolor{currentstroke}%
\pgfsetdash{}{0pt}%
\pgfsys@defobject{currentmarker}{\pgfqpoint{-0.033023in}{-0.028091in}}{\pgfqpoint{0.033023in}{0.034722in}}{%
\pgfpathmoveto{\pgfqpoint{0.000000in}{0.034722in}}%
\pgfpathlineto{\pgfqpoint{-0.007796in}{0.010730in}}%
\pgfpathlineto{\pgfqpoint{-0.033023in}{0.010730in}}%
\pgfpathlineto{\pgfqpoint{-0.012614in}{-0.004098in}}%
\pgfpathlineto{\pgfqpoint{-0.020409in}{-0.028091in}}%
\pgfpathlineto{\pgfqpoint{-0.000000in}{-0.013263in}}%
\pgfpathlineto{\pgfqpoint{0.020409in}{-0.028091in}}%
\pgfpathlineto{\pgfqpoint{0.012614in}{-0.004098in}}%
\pgfpathlineto{\pgfqpoint{0.033023in}{0.010730in}}%
\pgfpathlineto{\pgfqpoint{0.007796in}{0.010730in}}%
\pgfpathclose%
\pgfusepath{stroke,fill}%
}%
\begin{pgfscope}%
\pgfsys@transformshift{0.724728in}{0.696514in}%
\pgfsys@useobject{currentmarker}{}%
\end{pgfscope}%
\begin{pgfscope}%
\pgfsys@transformshift{0.895552in}{0.762779in}%
\pgfsys@useobject{currentmarker}{}%
\end{pgfscope}%
\begin{pgfscope}%
\pgfsys@transformshift{1.066375in}{0.829043in}%
\pgfsys@useobject{currentmarker}{}%
\end{pgfscope}%
\begin{pgfscope}%
\pgfsys@transformshift{1.237199in}{0.895308in}%
\pgfsys@useobject{currentmarker}{}%
\end{pgfscope}%
\begin{pgfscope}%
\pgfsys@transformshift{1.408023in}{1.027837in}%
\pgfsys@useobject{currentmarker}{}%
\end{pgfscope}%
\begin{pgfscope}%
\pgfsys@transformshift{1.578846in}{1.094101in}%
\pgfsys@useobject{currentmarker}{}%
\end{pgfscope}%
\begin{pgfscope}%
\pgfsys@transformshift{1.749670in}{1.160366in}%
\pgfsys@useobject{currentmarker}{}%
\end{pgfscope}%
\begin{pgfscope}%
\pgfsys@transformshift{1.920494in}{1.226630in}%
\pgfsys@useobject{currentmarker}{}%
\end{pgfscope}%
\begin{pgfscope}%
\pgfsys@transformshift{2.091317in}{1.292895in}%
\pgfsys@useobject{currentmarker}{}%
\end{pgfscope}%
\begin{pgfscope}%
\pgfsys@transformshift{2.262141in}{1.359159in}%
\pgfsys@useobject{currentmarker}{}%
\end{pgfscope}%
\begin{pgfscope}%
\pgfsys@transformshift{2.432965in}{1.425424in}%
\pgfsys@useobject{currentmarker}{}%
\end{pgfscope}%
\begin{pgfscope}%
\pgfsys@transformshift{2.603788in}{1.524820in}%
\pgfsys@useobject{currentmarker}{}%
\end{pgfscope}%
\begin{pgfscope}%
\pgfsys@transformshift{2.774612in}{1.624217in}%
\pgfsys@useobject{currentmarker}{}%
\end{pgfscope}%
\begin{pgfscope}%
\pgfsys@transformshift{2.945436in}{1.690482in}%
\pgfsys@useobject{currentmarker}{}%
\end{pgfscope}%
\begin{pgfscope}%
\pgfsys@transformshift{3.116259in}{1.756746in}%
\pgfsys@useobject{currentmarker}{}%
\end{pgfscope}%
\begin{pgfscope}%
\pgfsys@transformshift{3.287083in}{1.823011in}%
\pgfsys@useobject{currentmarker}{}%
\end{pgfscope}%
\begin{pgfscope}%
\pgfsys@transformshift{3.457907in}{1.889275in}%
\pgfsys@useobject{currentmarker}{}%
\end{pgfscope}%
\begin{pgfscope}%
\pgfsys@transformshift{3.628730in}{1.955540in}%
\pgfsys@useobject{currentmarker}{}%
\end{pgfscope}%
\begin{pgfscope}%
\pgfsys@transformshift{3.799554in}{2.088069in}%
\pgfsys@useobject{currentmarker}{}%
\end{pgfscope}%
\begin{pgfscope}%
\pgfsys@transformshift{3.970378in}{2.121201in}%
\pgfsys@useobject{currentmarker}{}%
\end{pgfscope}%
\begin{pgfscope}%
\pgfsys@transformshift{4.141201in}{2.220598in}%
\pgfsys@useobject{currentmarker}{}%
\end{pgfscope}%
\end{pgfscope}%
\begin{pgfscope}%
\pgfsetrectcap%
\pgfsetmiterjoin%
\pgfsetlinewidth{0.803000pt}%
\definecolor{currentstroke}{rgb}{0.000000,0.000000,0.000000}%
\pgfsetstrokecolor{currentstroke}%
\pgfsetdash{}{0pt}%
\pgfpathmoveto{\pgfqpoint{0.553904in}{0.535823in}}%
\pgfpathlineto{\pgfqpoint{0.553904in}{2.613215in}}%
\pgfusepath{stroke}%
\end{pgfscope}%
\begin{pgfscope}%
\pgfsetrectcap%
\pgfsetmiterjoin%
\pgfsetlinewidth{0.803000pt}%
\definecolor{currentstroke}{rgb}{0.000000,0.000000,0.000000}%
\pgfsetstrokecolor{currentstroke}%
\pgfsetdash{}{0pt}%
\pgfpathmoveto{\pgfqpoint{4.312025in}{0.535823in}}%
\pgfpathlineto{\pgfqpoint{4.312025in}{2.613215in}}%
\pgfusepath{stroke}%
\end{pgfscope}%
\begin{pgfscope}%
\pgfsetrectcap%
\pgfsetmiterjoin%
\pgfsetlinewidth{0.803000pt}%
\definecolor{currentstroke}{rgb}{0.000000,0.000000,0.000000}%
\pgfsetstrokecolor{currentstroke}%
\pgfsetdash{}{0pt}%
\pgfpathmoveto{\pgfqpoint{0.553904in}{0.535823in}}%
\pgfpathlineto{\pgfqpoint{4.312025in}{0.535823in}}%
\pgfusepath{stroke}%
\end{pgfscope}%
\begin{pgfscope}%
\pgfsetrectcap%
\pgfsetmiterjoin%
\pgfsetlinewidth{0.803000pt}%
\definecolor{currentstroke}{rgb}{0.000000,0.000000,0.000000}%
\pgfsetstrokecolor{currentstroke}%
\pgfsetdash{}{0pt}%
\pgfpathmoveto{\pgfqpoint{0.553904in}{2.613215in}}%
\pgfpathlineto{\pgfqpoint{4.312025in}{2.613215in}}%
\pgfusepath{stroke}%
\end{pgfscope}%
\begin{pgfscope}%
\pgfsetrectcap%
\pgfsetroundjoin%
\pgfsetlinewidth{1.003750pt}%
\definecolor{currentstroke}{rgb}{0.756863,0.117647,0.588235}%
\pgfsetstrokecolor{currentstroke}%
\pgfsetdash{}{0pt}%
\pgfpathmoveto{\pgfqpoint{0.603904in}{2.513215in}}%
\pgfpathlineto{\pgfqpoint{0.853904in}{2.513215in}}%
\pgfusepath{stroke}%
\end{pgfscope}%
\begin{pgfscope}%
\pgfsetbuttcap%
\pgfsetroundjoin%
\definecolor{currentfill}{rgb}{0.756863,0.117647,0.588235}%
\pgfsetfillcolor{currentfill}%
\pgfsetlinewidth{0.501875pt}%
\definecolor{currentstroke}{rgb}{0.000000,0.000000,0.000000}%
\pgfsetstrokecolor{currentstroke}%
\pgfsetdash{}{0pt}%
\pgfsys@defobject{currentmarker}{\pgfqpoint{-0.034722in}{-0.034722in}}{\pgfqpoint{0.034722in}{0.034722in}}{%
\pgfpathmoveto{\pgfqpoint{0.000000in}{-0.034722in}}%
\pgfpathcurveto{\pgfqpoint{0.009208in}{-0.034722in}}{\pgfqpoint{0.018041in}{-0.031064in}}{\pgfqpoint{0.024552in}{-0.024552in}}%
\pgfpathcurveto{\pgfqpoint{0.031064in}{-0.018041in}}{\pgfqpoint{0.034722in}{-0.009208in}}{\pgfqpoint{0.034722in}{0.000000in}}%
\pgfpathcurveto{\pgfqpoint{0.034722in}{0.009208in}}{\pgfqpoint{0.031064in}{0.018041in}}{\pgfqpoint{0.024552in}{0.024552in}}%
\pgfpathcurveto{\pgfqpoint{0.018041in}{0.031064in}}{\pgfqpoint{0.009208in}{0.034722in}}{\pgfqpoint{0.000000in}{0.034722in}}%
\pgfpathcurveto{\pgfqpoint{-0.009208in}{0.034722in}}{\pgfqpoint{-0.018041in}{0.031064in}}{\pgfqpoint{-0.024552in}{0.024552in}}%
\pgfpathcurveto{\pgfqpoint{-0.031064in}{0.018041in}}{\pgfqpoint{-0.034722in}{0.009208in}}{\pgfqpoint{-0.034722in}{0.000000in}}%
\pgfpathcurveto{\pgfqpoint{-0.034722in}{-0.009208in}}{\pgfqpoint{-0.031064in}{-0.018041in}}{\pgfqpoint{-0.024552in}{-0.024552in}}%
\pgfpathcurveto{\pgfqpoint{-0.018041in}{-0.031064in}}{\pgfqpoint{-0.009208in}{-0.034722in}}{\pgfqpoint{0.000000in}{-0.034722in}}%
\pgfpathclose%
\pgfusepath{stroke,fill}%
}%
\begin{pgfscope}%
\pgfsys@transformshift{0.728904in}{2.513215in}%
\pgfsys@useobject{currentmarker}{}%
\end{pgfscope}%
\end{pgfscope}%
\begin{pgfscope}%
\definecolor{textcolor}{rgb}{0.000000,0.000000,0.000000}%
\pgfsetstrokecolor{textcolor}%
\pgfsetfillcolor{textcolor}%
\pgftext[x=0.878904in,y=2.469465in,left,base]{\color{textcolor}\rmfamily\fontsize{9.000000}{10.800000}\selectfont Treewidth of \(\displaystyle Line(G)\)}%
\end{pgfscope}%
\begin{pgfscope}%
\pgfsetrectcap%
\pgfsetroundjoin%
\pgfsetlinewidth{1.003750pt}%
\definecolor{currentstroke}{rgb}{0.007843,0.219608,0.501961}%
\pgfsetstrokecolor{currentstroke}%
\pgfsetdash{}{0pt}%
\pgfpathmoveto{\pgfqpoint{0.603904in}{2.344465in}}%
\pgfpathlineto{\pgfqpoint{0.853904in}{2.344465in}}%
\pgfusepath{stroke}%
\end{pgfscope}%
\begin{pgfscope}%
\pgfsetbuttcap%
\pgfsetmiterjoin%
\definecolor{currentfill}{rgb}{0.007843,0.219608,0.501961}%
\pgfsetfillcolor{currentfill}%
\pgfsetlinewidth{0.501875pt}%
\definecolor{currentstroke}{rgb}{0.000000,0.000000,0.000000}%
\pgfsetstrokecolor{currentstroke}%
\pgfsetdash{}{0pt}%
\pgfsys@defobject{currentmarker}{\pgfqpoint{-0.034722in}{-0.034722in}}{\pgfqpoint{0.034722in}{0.034722in}}{%
\pgfpathmoveto{\pgfqpoint{-0.000000in}{-0.034722in}}%
\pgfpathlineto{\pgfqpoint{0.034722in}{0.034722in}}%
\pgfpathlineto{\pgfqpoint{-0.034722in}{0.034722in}}%
\pgfpathclose%
\pgfusepath{stroke,fill}%
}%
\begin{pgfscope}%
\pgfsys@transformshift{0.728904in}{2.344465in}%
\pgfsys@useobject{currentmarker}{}%
\end{pgfscope}%
\end{pgfscope}%
\begin{pgfscope}%
\definecolor{textcolor}{rgb}{0.000000,0.000000,0.000000}%
\pgfsetstrokecolor{textcolor}%
\pgfsetfillcolor{textcolor}%
\pgftext[x=0.878904in,y=2.300715in,left,base]{\color{textcolor}\rmfamily\fontsize{9.000000}{10.800000}\selectfont Treewidth of \(\displaystyle G\)}%
\end{pgfscope}%
\begin{pgfscope}%
\pgfsetrectcap%
\pgfsetroundjoin%
\pgfsetlinewidth{1.003750pt}%
\definecolor{currentstroke}{rgb}{0.654902,0.909804,0.192157}%
\pgfsetstrokecolor{currentstroke}%
\pgfsetdash{}{0pt}%
\pgfpathmoveto{\pgfqpoint{0.603904in}{2.182665in}}%
\pgfpathlineto{\pgfqpoint{0.853904in}{2.182665in}}%
\pgfusepath{stroke}%
\end{pgfscope}%
\begin{pgfscope}%
\pgfsetbuttcap%
\pgfsetmiterjoin%
\definecolor{currentfill}{rgb}{0.654902,0.909804,0.192157}%
\pgfsetfillcolor{currentfill}%
\pgfsetlinewidth{0.501875pt}%
\definecolor{currentstroke}{rgb}{0.000000,0.000000,0.000000}%
\pgfsetstrokecolor{currentstroke}%
\pgfsetdash{}{0pt}%
\pgfsys@defobject{currentmarker}{\pgfqpoint{-0.034722in}{-0.034722in}}{\pgfqpoint{0.034722in}{0.034722in}}{%
\pgfpathmoveto{\pgfqpoint{-0.034722in}{-0.034722in}}%
\pgfpathlineto{\pgfqpoint{0.034722in}{-0.034722in}}%
\pgfpathlineto{\pgfqpoint{0.034722in}{0.034722in}}%
\pgfpathlineto{\pgfqpoint{-0.034722in}{0.034722in}}%
\pgfpathclose%
\pgfusepath{stroke,fill}%
}%
\begin{pgfscope}%
\pgfsys@transformshift{0.728904in}{2.182665in}%
\pgfsys@useobject{currentmarker}{}%
\end{pgfscope}%
\end{pgfscope}%
\begin{pgfscope}%
\definecolor{textcolor}{rgb}{0.000000,0.000000,0.000000}%
\pgfsetstrokecolor{textcolor}%
\pgfsetfillcolor{textcolor}%
\pgftext[x=0.878904in,y=2.138915in,left,base]{\color{textcolor}\rmfamily\fontsize{9.000000}{10.800000}\selectfont Carving width of \(\displaystyle G\) using \textbf{FT}}%
\end{pgfscope}%
\begin{pgfscope}%
\pgfsetrectcap%
\pgfsetroundjoin%
\pgfsetlinewidth{1.003750pt}%
\definecolor{currentstroke}{rgb}{0.525490,0.843137,0.662745}%
\pgfsetstrokecolor{currentstroke}%
\pgfsetdash{}{0pt}%
\pgfpathmoveto{\pgfqpoint{0.603904in}{2.020866in}}%
\pgfpathlineto{\pgfqpoint{0.853904in}{2.020866in}}%
\pgfusepath{stroke}%
\end{pgfscope}%
\begin{pgfscope}%
\pgfsetbuttcap%
\pgfsetbeveljoin%
\definecolor{currentfill}{rgb}{0.525490,0.843137,0.662745}%
\pgfsetfillcolor{currentfill}%
\pgfsetlinewidth{0.501875pt}%
\definecolor{currentstroke}{rgb}{0.000000,0.000000,0.000000}%
\pgfsetstrokecolor{currentstroke}%
\pgfsetdash{}{0pt}%
\pgfsys@defobject{currentmarker}{\pgfqpoint{-0.033023in}{-0.028091in}}{\pgfqpoint{0.033023in}{0.034722in}}{%
\pgfpathmoveto{\pgfqpoint{0.000000in}{0.034722in}}%
\pgfpathlineto{\pgfqpoint{-0.007796in}{0.010730in}}%
\pgfpathlineto{\pgfqpoint{-0.033023in}{0.010730in}}%
\pgfpathlineto{\pgfqpoint{-0.012614in}{-0.004098in}}%
\pgfpathlineto{\pgfqpoint{-0.020409in}{-0.028091in}}%
\pgfpathlineto{\pgfqpoint{-0.000000in}{-0.013263in}}%
\pgfpathlineto{\pgfqpoint{0.020409in}{-0.028091in}}%
\pgfpathlineto{\pgfqpoint{0.012614in}{-0.004098in}}%
\pgfpathlineto{\pgfqpoint{0.033023in}{0.010730in}}%
\pgfpathlineto{\pgfqpoint{0.007796in}{0.010730in}}%
\pgfpathclose%
\pgfusepath{stroke,fill}%
}%
\begin{pgfscope}%
\pgfsys@transformshift{0.728904in}{2.020866in}%
\pgfsys@useobject{currentmarker}{}%
\end{pgfscope}%
\end{pgfscope}%
\begin{pgfscope}%
\definecolor{textcolor}{rgb}{0.000000,0.000000,0.000000}%
\pgfsetstrokecolor{textcolor}%
\pgfsetfillcolor{textcolor}%
\pgftext[x=0.878904in,y=1.977116in,left,base]{\color{textcolor}\rmfamily\fontsize{9.000000}{10.800000}\selectfont Carving width of \(\displaystyle G\) using \textbf{LG}}%
\end{pgfscope}%
\end{pgfpicture}%
\makeatother%
\endgroup%

	\caption{\label{fig:vertex-cover-width} Median of the best upper bound found for treewidth and carving width of 100 cubic graphs with $n$ vertices. For most large graphs, the carving width of $G$ is smaller than the treewidth of $G$, which is smaller that the treewidth of $\Line{G}$.}
\end{figure}

%\begin{figure}
%	\centering
%	%% Creator: Matplotlib, PGF backend
%%
%% To include the figure in your LaTeX document, write
%%   \input{<filename>.pgf}
%%
%% Make sure the required packages are loaded in your preamble
%%   \usepackage{pgf}
%%
%% and, on pdftex
%%   \usepackage[utf8]{inputenc}\DeclareUnicodeCharacter{2212}{-}
%%
%% or, on luatex and xetex
%%   \usepackage{unicode-math}
%%
%% Figures using additional raster images can only be included by \input if
%% they are in the same directory as the main LaTeX file. For loading figures
%% from other directories you can use the `import` package
%%   \usepackage{import}
%%
%% and then include the figures with
%%   \import{<path to file>}{<filename>.pgf}
%%
%% Matplotlib used the following preamble
%%   \usepackage[utf8x]{inputenc}
%%   \usepackage[T1]{fontenc}
%%
\begingroup%
\makeatletter%
\begin{pgfpicture}%
\pgfpathrectangle{\pgfpointorigin}{\pgfqpoint{6.000000in}{3.400000in}}%
\pgfusepath{use as bounding box, clip}%
\begin{pgfscope}%
\pgfsetbuttcap%
\pgfsetmiterjoin%
\definecolor{currentfill}{rgb}{1.000000,1.000000,1.000000}%
\pgfsetfillcolor{currentfill}%
\pgfsetlinewidth{0.000000pt}%
\definecolor{currentstroke}{rgb}{1.000000,1.000000,1.000000}%
\pgfsetstrokecolor{currentstroke}%
\pgfsetdash{}{0pt}%
\pgfpathmoveto{\pgfqpoint{0.000000in}{0.000000in}}%
\pgfpathlineto{\pgfqpoint{6.000000in}{0.000000in}}%
\pgfpathlineto{\pgfqpoint{6.000000in}{3.400000in}}%
\pgfpathlineto{\pgfqpoint{0.000000in}{3.400000in}}%
\pgfpathclose%
\pgfusepath{fill}%
\end{pgfscope}%
\begin{pgfscope}%
\pgfsetbuttcap%
\pgfsetmiterjoin%
\definecolor{currentfill}{rgb}{1.000000,1.000000,1.000000}%
\pgfsetfillcolor{currentfill}%
\pgfsetlinewidth{0.000000pt}%
\definecolor{currentstroke}{rgb}{0.000000,0.000000,0.000000}%
\pgfsetstrokecolor{currentstroke}%
\pgfsetstrokeopacity{0.000000}%
\pgfsetdash{}{0pt}%
\pgfpathmoveto{\pgfqpoint{0.708220in}{0.535823in}}%
\pgfpathlineto{\pgfqpoint{5.850000in}{0.535823in}}%
\pgfpathlineto{\pgfqpoint{5.850000in}{3.205275in}}%
\pgfpathlineto{\pgfqpoint{0.708220in}{3.205275in}}%
\pgfpathclose%
\pgfusepath{fill}%
\end{pgfscope}%
\begin{pgfscope}%
\pgfsetbuttcap%
\pgfsetroundjoin%
\definecolor{currentfill}{rgb}{0.000000,0.000000,0.000000}%
\pgfsetfillcolor{currentfill}%
\pgfsetlinewidth{0.803000pt}%
\definecolor{currentstroke}{rgb}{0.000000,0.000000,0.000000}%
\pgfsetstrokecolor{currentstroke}%
\pgfsetdash{}{0pt}%
\pgfsys@defobject{currentmarker}{\pgfqpoint{0.000000in}{-0.048611in}}{\pgfqpoint{0.000000in}{0.000000in}}{%
\pgfpathmoveto{\pgfqpoint{0.000000in}{0.000000in}}%
\pgfpathlineto{\pgfqpoint{0.000000in}{-0.048611in}}%
\pgfusepath{stroke,fill}%
}%
\begin{pgfscope}%
\pgfsys@transformshift{0.708220in}{0.535823in}%
\pgfsys@useobject{currentmarker}{}%
\end{pgfscope}%
\end{pgfscope}%
\begin{pgfscope}%
\definecolor{textcolor}{rgb}{0.000000,0.000000,0.000000}%
\pgfsetstrokecolor{textcolor}%
\pgfsetfillcolor{textcolor}%
\pgftext[x=0.708220in,y=0.438600in,,top]{\color{textcolor}\rmfamily\fontsize{9.000000}{10.800000}\selectfont \(\displaystyle {0}\)}%
\end{pgfscope}%
\begin{pgfscope}%
\pgfsetbuttcap%
\pgfsetroundjoin%
\definecolor{currentfill}{rgb}{0.000000,0.000000,0.000000}%
\pgfsetfillcolor{currentfill}%
\pgfsetlinewidth{0.803000pt}%
\definecolor{currentstroke}{rgb}{0.000000,0.000000,0.000000}%
\pgfsetstrokecolor{currentstroke}%
\pgfsetdash{}{0pt}%
\pgfsys@defobject{currentmarker}{\pgfqpoint{0.000000in}{-0.048611in}}{\pgfqpoint{0.000000in}{0.000000in}}{%
\pgfpathmoveto{\pgfqpoint{0.000000in}{0.000000in}}%
\pgfpathlineto{\pgfqpoint{0.000000in}{-0.048611in}}%
\pgfusepath{stroke,fill}%
}%
\begin{pgfscope}%
\pgfsys@transformshift{1.833336in}{0.535823in}%
\pgfsys@useobject{currentmarker}{}%
\end{pgfscope}%
\end{pgfscope}%
\begin{pgfscope}%
\definecolor{textcolor}{rgb}{0.000000,0.000000,0.000000}%
\pgfsetstrokecolor{textcolor}%
\pgfsetfillcolor{textcolor}%
\pgftext[x=1.833336in,y=0.438600in,,top]{\color{textcolor}\rmfamily\fontsize{9.000000}{10.800000}\selectfont \(\displaystyle {50}\)}%
\end{pgfscope}%
\begin{pgfscope}%
\pgfsetbuttcap%
\pgfsetroundjoin%
\definecolor{currentfill}{rgb}{0.000000,0.000000,0.000000}%
\pgfsetfillcolor{currentfill}%
\pgfsetlinewidth{0.803000pt}%
\definecolor{currentstroke}{rgb}{0.000000,0.000000,0.000000}%
\pgfsetstrokecolor{currentstroke}%
\pgfsetdash{}{0pt}%
\pgfsys@defobject{currentmarker}{\pgfqpoint{0.000000in}{-0.048611in}}{\pgfqpoint{0.000000in}{0.000000in}}{%
\pgfpathmoveto{\pgfqpoint{0.000000in}{0.000000in}}%
\pgfpathlineto{\pgfqpoint{0.000000in}{-0.048611in}}%
\pgfusepath{stroke,fill}%
}%
\begin{pgfscope}%
\pgfsys@transformshift{2.958452in}{0.535823in}%
\pgfsys@useobject{currentmarker}{}%
\end{pgfscope}%
\end{pgfscope}%
\begin{pgfscope}%
\definecolor{textcolor}{rgb}{0.000000,0.000000,0.000000}%
\pgfsetstrokecolor{textcolor}%
\pgfsetfillcolor{textcolor}%
\pgftext[x=2.958452in,y=0.438600in,,top]{\color{textcolor}\rmfamily\fontsize{9.000000}{10.800000}\selectfont \(\displaystyle {100}\)}%
\end{pgfscope}%
\begin{pgfscope}%
\pgfsetbuttcap%
\pgfsetroundjoin%
\definecolor{currentfill}{rgb}{0.000000,0.000000,0.000000}%
\pgfsetfillcolor{currentfill}%
\pgfsetlinewidth{0.803000pt}%
\definecolor{currentstroke}{rgb}{0.000000,0.000000,0.000000}%
\pgfsetstrokecolor{currentstroke}%
\pgfsetdash{}{0pt}%
\pgfsys@defobject{currentmarker}{\pgfqpoint{0.000000in}{-0.048611in}}{\pgfqpoint{0.000000in}{0.000000in}}{%
\pgfpathmoveto{\pgfqpoint{0.000000in}{0.000000in}}%
\pgfpathlineto{\pgfqpoint{0.000000in}{-0.048611in}}%
\pgfusepath{stroke,fill}%
}%
\begin{pgfscope}%
\pgfsys@transformshift{4.083568in}{0.535823in}%
\pgfsys@useobject{currentmarker}{}%
\end{pgfscope}%
\end{pgfscope}%
\begin{pgfscope}%
\definecolor{textcolor}{rgb}{0.000000,0.000000,0.000000}%
\pgfsetstrokecolor{textcolor}%
\pgfsetfillcolor{textcolor}%
\pgftext[x=4.083568in,y=0.438600in,,top]{\color{textcolor}\rmfamily\fontsize{9.000000}{10.800000}\selectfont \(\displaystyle {150}\)}%
\end{pgfscope}%
\begin{pgfscope}%
\pgfsetbuttcap%
\pgfsetroundjoin%
\definecolor{currentfill}{rgb}{0.000000,0.000000,0.000000}%
\pgfsetfillcolor{currentfill}%
\pgfsetlinewidth{0.803000pt}%
\definecolor{currentstroke}{rgb}{0.000000,0.000000,0.000000}%
\pgfsetstrokecolor{currentstroke}%
\pgfsetdash{}{0pt}%
\pgfsys@defobject{currentmarker}{\pgfqpoint{0.000000in}{-0.048611in}}{\pgfqpoint{0.000000in}{0.000000in}}{%
\pgfpathmoveto{\pgfqpoint{0.000000in}{0.000000in}}%
\pgfpathlineto{\pgfqpoint{0.000000in}{-0.048611in}}%
\pgfusepath{stroke,fill}%
}%
\begin{pgfscope}%
\pgfsys@transformshift{5.208684in}{0.535823in}%
\pgfsys@useobject{currentmarker}{}%
\end{pgfscope}%
\end{pgfscope}%
\begin{pgfscope}%
\definecolor{textcolor}{rgb}{0.000000,0.000000,0.000000}%
\pgfsetstrokecolor{textcolor}%
\pgfsetfillcolor{textcolor}%
\pgftext[x=5.208684in,y=0.438600in,,top]{\color{textcolor}\rmfamily\fontsize{9.000000}{10.800000}\selectfont \(\displaystyle {200}\)}%
\end{pgfscope}%
\begin{pgfscope}%
\definecolor{textcolor}{rgb}{0.000000,0.000000,0.000000}%
\pgfsetstrokecolor{textcolor}%
\pgfsetfillcolor{textcolor}%
\pgftext[x=3.279110in,y=0.272655in,,top]{\color{textcolor}\rmfamily\fontsize{10.000000}{12.000000}\selectfont \(\displaystyle n\): Number of vertices}%
\end{pgfscope}%
\begin{pgfscope}%
\pgfsetbuttcap%
\pgfsetroundjoin%
\definecolor{currentfill}{rgb}{0.000000,0.000000,0.000000}%
\pgfsetfillcolor{currentfill}%
\pgfsetlinewidth{0.803000pt}%
\definecolor{currentstroke}{rgb}{0.000000,0.000000,0.000000}%
\pgfsetstrokecolor{currentstroke}%
\pgfsetdash{}{0pt}%
\pgfsys@defobject{currentmarker}{\pgfqpoint{-0.048611in}{0.000000in}}{\pgfqpoint{-0.000000in}{0.000000in}}{%
\pgfpathmoveto{\pgfqpoint{-0.000000in}{0.000000in}}%
\pgfpathlineto{\pgfqpoint{-0.048611in}{0.000000in}}%
\pgfusepath{stroke,fill}%
}%
\begin{pgfscope}%
\pgfsys@transformshift{0.708220in}{0.535823in}%
\pgfsys@useobject{currentmarker}{}%
\end{pgfscope}%
\end{pgfscope}%
\begin{pgfscope}%
\definecolor{textcolor}{rgb}{0.000000,0.000000,0.000000}%
\pgfsetstrokecolor{textcolor}%
\pgfsetfillcolor{textcolor}%
\pgftext[x=0.344411in, y=0.491098in, left, base]{\color{textcolor}\rmfamily\fontsize{9.000000}{10.800000}\selectfont \(\displaystyle {10^{-1}}\)}%
\end{pgfscope}%
\begin{pgfscope}%
\pgfsetbuttcap%
\pgfsetroundjoin%
\definecolor{currentfill}{rgb}{0.000000,0.000000,0.000000}%
\pgfsetfillcolor{currentfill}%
\pgfsetlinewidth{0.803000pt}%
\definecolor{currentstroke}{rgb}{0.000000,0.000000,0.000000}%
\pgfsetstrokecolor{currentstroke}%
\pgfsetdash{}{0pt}%
\pgfsys@defobject{currentmarker}{\pgfqpoint{-0.048611in}{0.000000in}}{\pgfqpoint{-0.000000in}{0.000000in}}{%
\pgfpathmoveto{\pgfqpoint{-0.000000in}{0.000000in}}%
\pgfpathlineto{\pgfqpoint{-0.048611in}{0.000000in}}%
\pgfusepath{stroke,fill}%
}%
\begin{pgfscope}%
\pgfsys@transformshift{0.708220in}{1.203186in}%
\pgfsys@useobject{currentmarker}{}%
\end{pgfscope}%
\end{pgfscope}%
\begin{pgfscope}%
\definecolor{textcolor}{rgb}{0.000000,0.000000,0.000000}%
\pgfsetstrokecolor{textcolor}%
\pgfsetfillcolor{textcolor}%
\pgftext[x=0.424657in, y=1.158461in, left, base]{\color{textcolor}\rmfamily\fontsize{9.000000}{10.800000}\selectfont \(\displaystyle {10^{0}}\)}%
\end{pgfscope}%
\begin{pgfscope}%
\pgfsetbuttcap%
\pgfsetroundjoin%
\definecolor{currentfill}{rgb}{0.000000,0.000000,0.000000}%
\pgfsetfillcolor{currentfill}%
\pgfsetlinewidth{0.803000pt}%
\definecolor{currentstroke}{rgb}{0.000000,0.000000,0.000000}%
\pgfsetstrokecolor{currentstroke}%
\pgfsetdash{}{0pt}%
\pgfsys@defobject{currentmarker}{\pgfqpoint{-0.048611in}{0.000000in}}{\pgfqpoint{-0.000000in}{0.000000in}}{%
\pgfpathmoveto{\pgfqpoint{-0.000000in}{0.000000in}}%
\pgfpathlineto{\pgfqpoint{-0.048611in}{0.000000in}}%
\pgfusepath{stroke,fill}%
}%
\begin{pgfscope}%
\pgfsys@transformshift{0.708220in}{1.870549in}%
\pgfsys@useobject{currentmarker}{}%
\end{pgfscope}%
\end{pgfscope}%
\begin{pgfscope}%
\definecolor{textcolor}{rgb}{0.000000,0.000000,0.000000}%
\pgfsetstrokecolor{textcolor}%
\pgfsetfillcolor{textcolor}%
\pgftext[x=0.424657in, y=1.825824in, left, base]{\color{textcolor}\rmfamily\fontsize{9.000000}{10.800000}\selectfont \(\displaystyle {10^{1}}\)}%
\end{pgfscope}%
\begin{pgfscope}%
\pgfsetbuttcap%
\pgfsetroundjoin%
\definecolor{currentfill}{rgb}{0.000000,0.000000,0.000000}%
\pgfsetfillcolor{currentfill}%
\pgfsetlinewidth{0.803000pt}%
\definecolor{currentstroke}{rgb}{0.000000,0.000000,0.000000}%
\pgfsetstrokecolor{currentstroke}%
\pgfsetdash{}{0pt}%
\pgfsys@defobject{currentmarker}{\pgfqpoint{-0.048611in}{0.000000in}}{\pgfqpoint{-0.000000in}{0.000000in}}{%
\pgfpathmoveto{\pgfqpoint{-0.000000in}{0.000000in}}%
\pgfpathlineto{\pgfqpoint{-0.048611in}{0.000000in}}%
\pgfusepath{stroke,fill}%
}%
\begin{pgfscope}%
\pgfsys@transformshift{0.708220in}{2.537912in}%
\pgfsys@useobject{currentmarker}{}%
\end{pgfscope}%
\end{pgfscope}%
\begin{pgfscope}%
\definecolor{textcolor}{rgb}{0.000000,0.000000,0.000000}%
\pgfsetstrokecolor{textcolor}%
\pgfsetfillcolor{textcolor}%
\pgftext[x=0.424657in, y=2.493187in, left, base]{\color{textcolor}\rmfamily\fontsize{9.000000}{10.800000}\selectfont \(\displaystyle {10^{2}}\)}%
\end{pgfscope}%
\begin{pgfscope}%
\pgfsetbuttcap%
\pgfsetroundjoin%
\definecolor{currentfill}{rgb}{0.000000,0.000000,0.000000}%
\pgfsetfillcolor{currentfill}%
\pgfsetlinewidth{0.803000pt}%
\definecolor{currentstroke}{rgb}{0.000000,0.000000,0.000000}%
\pgfsetstrokecolor{currentstroke}%
\pgfsetdash{}{0pt}%
\pgfsys@defobject{currentmarker}{\pgfqpoint{-0.048611in}{0.000000in}}{\pgfqpoint{-0.000000in}{0.000000in}}{%
\pgfpathmoveto{\pgfqpoint{-0.000000in}{0.000000in}}%
\pgfpathlineto{\pgfqpoint{-0.048611in}{0.000000in}}%
\pgfusepath{stroke,fill}%
}%
\begin{pgfscope}%
\pgfsys@transformshift{0.708220in}{3.205275in}%
\pgfsys@useobject{currentmarker}{}%
\end{pgfscope}%
\end{pgfscope}%
\begin{pgfscope}%
\definecolor{textcolor}{rgb}{0.000000,0.000000,0.000000}%
\pgfsetstrokecolor{textcolor}%
\pgfsetfillcolor{textcolor}%
\pgftext[x=0.424657in, y=3.160550in, left, base]{\color{textcolor}\rmfamily\fontsize{9.000000}{10.800000}\selectfont \(\displaystyle {10^{3}}\)}%
\end{pgfscope}%
\begin{pgfscope}%
\pgfsetbuttcap%
\pgfsetroundjoin%
\definecolor{currentfill}{rgb}{0.000000,0.000000,0.000000}%
\pgfsetfillcolor{currentfill}%
\pgfsetlinewidth{0.602250pt}%
\definecolor{currentstroke}{rgb}{0.000000,0.000000,0.000000}%
\pgfsetstrokecolor{currentstroke}%
\pgfsetdash{}{0pt}%
\pgfsys@defobject{currentmarker}{\pgfqpoint{-0.027778in}{0.000000in}}{\pgfqpoint{-0.000000in}{0.000000in}}{%
\pgfpathmoveto{\pgfqpoint{-0.000000in}{0.000000in}}%
\pgfpathlineto{\pgfqpoint{-0.027778in}{0.000000in}}%
\pgfusepath{stroke,fill}%
}%
\begin{pgfscope}%
\pgfsys@transformshift{0.708220in}{0.736719in}%
\pgfsys@useobject{currentmarker}{}%
\end{pgfscope}%
\end{pgfscope}%
\begin{pgfscope}%
\pgfsetbuttcap%
\pgfsetroundjoin%
\definecolor{currentfill}{rgb}{0.000000,0.000000,0.000000}%
\pgfsetfillcolor{currentfill}%
\pgfsetlinewidth{0.602250pt}%
\definecolor{currentstroke}{rgb}{0.000000,0.000000,0.000000}%
\pgfsetstrokecolor{currentstroke}%
\pgfsetdash{}{0pt}%
\pgfsys@defobject{currentmarker}{\pgfqpoint{-0.027778in}{0.000000in}}{\pgfqpoint{-0.000000in}{0.000000in}}{%
\pgfpathmoveto{\pgfqpoint{-0.000000in}{0.000000in}}%
\pgfpathlineto{\pgfqpoint{-0.027778in}{0.000000in}}%
\pgfusepath{stroke,fill}%
}%
\begin{pgfscope}%
\pgfsys@transformshift{0.708220in}{0.854236in}%
\pgfsys@useobject{currentmarker}{}%
\end{pgfscope}%
\end{pgfscope}%
\begin{pgfscope}%
\pgfsetbuttcap%
\pgfsetroundjoin%
\definecolor{currentfill}{rgb}{0.000000,0.000000,0.000000}%
\pgfsetfillcolor{currentfill}%
\pgfsetlinewidth{0.602250pt}%
\definecolor{currentstroke}{rgb}{0.000000,0.000000,0.000000}%
\pgfsetstrokecolor{currentstroke}%
\pgfsetdash{}{0pt}%
\pgfsys@defobject{currentmarker}{\pgfqpoint{-0.027778in}{0.000000in}}{\pgfqpoint{-0.000000in}{0.000000in}}{%
\pgfpathmoveto{\pgfqpoint{-0.000000in}{0.000000in}}%
\pgfpathlineto{\pgfqpoint{-0.027778in}{0.000000in}}%
\pgfusepath{stroke,fill}%
}%
\begin{pgfscope}%
\pgfsys@transformshift{0.708220in}{0.937615in}%
\pgfsys@useobject{currentmarker}{}%
\end{pgfscope}%
\end{pgfscope}%
\begin{pgfscope}%
\pgfsetbuttcap%
\pgfsetroundjoin%
\definecolor{currentfill}{rgb}{0.000000,0.000000,0.000000}%
\pgfsetfillcolor{currentfill}%
\pgfsetlinewidth{0.602250pt}%
\definecolor{currentstroke}{rgb}{0.000000,0.000000,0.000000}%
\pgfsetstrokecolor{currentstroke}%
\pgfsetdash{}{0pt}%
\pgfsys@defobject{currentmarker}{\pgfqpoint{-0.027778in}{0.000000in}}{\pgfqpoint{-0.000000in}{0.000000in}}{%
\pgfpathmoveto{\pgfqpoint{-0.000000in}{0.000000in}}%
\pgfpathlineto{\pgfqpoint{-0.027778in}{0.000000in}}%
\pgfusepath{stroke,fill}%
}%
\begin{pgfscope}%
\pgfsys@transformshift{0.708220in}{1.002289in}%
\pgfsys@useobject{currentmarker}{}%
\end{pgfscope}%
\end{pgfscope}%
\begin{pgfscope}%
\pgfsetbuttcap%
\pgfsetroundjoin%
\definecolor{currentfill}{rgb}{0.000000,0.000000,0.000000}%
\pgfsetfillcolor{currentfill}%
\pgfsetlinewidth{0.602250pt}%
\definecolor{currentstroke}{rgb}{0.000000,0.000000,0.000000}%
\pgfsetstrokecolor{currentstroke}%
\pgfsetdash{}{0pt}%
\pgfsys@defobject{currentmarker}{\pgfqpoint{-0.027778in}{0.000000in}}{\pgfqpoint{-0.000000in}{0.000000in}}{%
\pgfpathmoveto{\pgfqpoint{-0.000000in}{0.000000in}}%
\pgfpathlineto{\pgfqpoint{-0.027778in}{0.000000in}}%
\pgfusepath{stroke,fill}%
}%
\begin{pgfscope}%
\pgfsys@transformshift{0.708220in}{1.055132in}%
\pgfsys@useobject{currentmarker}{}%
\end{pgfscope}%
\end{pgfscope}%
\begin{pgfscope}%
\pgfsetbuttcap%
\pgfsetroundjoin%
\definecolor{currentfill}{rgb}{0.000000,0.000000,0.000000}%
\pgfsetfillcolor{currentfill}%
\pgfsetlinewidth{0.602250pt}%
\definecolor{currentstroke}{rgb}{0.000000,0.000000,0.000000}%
\pgfsetstrokecolor{currentstroke}%
\pgfsetdash{}{0pt}%
\pgfsys@defobject{currentmarker}{\pgfqpoint{-0.027778in}{0.000000in}}{\pgfqpoint{-0.000000in}{0.000000in}}{%
\pgfpathmoveto{\pgfqpoint{-0.000000in}{0.000000in}}%
\pgfpathlineto{\pgfqpoint{-0.027778in}{0.000000in}}%
\pgfusepath{stroke,fill}%
}%
\begin{pgfscope}%
\pgfsys@transformshift{0.708220in}{1.099810in}%
\pgfsys@useobject{currentmarker}{}%
\end{pgfscope}%
\end{pgfscope}%
\begin{pgfscope}%
\pgfsetbuttcap%
\pgfsetroundjoin%
\definecolor{currentfill}{rgb}{0.000000,0.000000,0.000000}%
\pgfsetfillcolor{currentfill}%
\pgfsetlinewidth{0.602250pt}%
\definecolor{currentstroke}{rgb}{0.000000,0.000000,0.000000}%
\pgfsetstrokecolor{currentstroke}%
\pgfsetdash{}{0pt}%
\pgfsys@defobject{currentmarker}{\pgfqpoint{-0.027778in}{0.000000in}}{\pgfqpoint{-0.000000in}{0.000000in}}{%
\pgfpathmoveto{\pgfqpoint{-0.000000in}{0.000000in}}%
\pgfpathlineto{\pgfqpoint{-0.027778in}{0.000000in}}%
\pgfusepath{stroke,fill}%
}%
\begin{pgfscope}%
\pgfsys@transformshift{0.708220in}{1.138512in}%
\pgfsys@useobject{currentmarker}{}%
\end{pgfscope}%
\end{pgfscope}%
\begin{pgfscope}%
\pgfsetbuttcap%
\pgfsetroundjoin%
\definecolor{currentfill}{rgb}{0.000000,0.000000,0.000000}%
\pgfsetfillcolor{currentfill}%
\pgfsetlinewidth{0.602250pt}%
\definecolor{currentstroke}{rgb}{0.000000,0.000000,0.000000}%
\pgfsetstrokecolor{currentstroke}%
\pgfsetdash{}{0pt}%
\pgfsys@defobject{currentmarker}{\pgfqpoint{-0.027778in}{0.000000in}}{\pgfqpoint{-0.000000in}{0.000000in}}{%
\pgfpathmoveto{\pgfqpoint{-0.000000in}{0.000000in}}%
\pgfpathlineto{\pgfqpoint{-0.027778in}{0.000000in}}%
\pgfusepath{stroke,fill}%
}%
\begin{pgfscope}%
\pgfsys@transformshift{0.708220in}{1.172649in}%
\pgfsys@useobject{currentmarker}{}%
\end{pgfscope}%
\end{pgfscope}%
\begin{pgfscope}%
\pgfsetbuttcap%
\pgfsetroundjoin%
\definecolor{currentfill}{rgb}{0.000000,0.000000,0.000000}%
\pgfsetfillcolor{currentfill}%
\pgfsetlinewidth{0.602250pt}%
\definecolor{currentstroke}{rgb}{0.000000,0.000000,0.000000}%
\pgfsetstrokecolor{currentstroke}%
\pgfsetdash{}{0pt}%
\pgfsys@defobject{currentmarker}{\pgfqpoint{-0.027778in}{0.000000in}}{\pgfqpoint{-0.000000in}{0.000000in}}{%
\pgfpathmoveto{\pgfqpoint{-0.000000in}{0.000000in}}%
\pgfpathlineto{\pgfqpoint{-0.027778in}{0.000000in}}%
\pgfusepath{stroke,fill}%
}%
\begin{pgfscope}%
\pgfsys@transformshift{0.708220in}{1.404082in}%
\pgfsys@useobject{currentmarker}{}%
\end{pgfscope}%
\end{pgfscope}%
\begin{pgfscope}%
\pgfsetbuttcap%
\pgfsetroundjoin%
\definecolor{currentfill}{rgb}{0.000000,0.000000,0.000000}%
\pgfsetfillcolor{currentfill}%
\pgfsetlinewidth{0.602250pt}%
\definecolor{currentstroke}{rgb}{0.000000,0.000000,0.000000}%
\pgfsetstrokecolor{currentstroke}%
\pgfsetdash{}{0pt}%
\pgfsys@defobject{currentmarker}{\pgfqpoint{-0.027778in}{0.000000in}}{\pgfqpoint{-0.000000in}{0.000000in}}{%
\pgfpathmoveto{\pgfqpoint{-0.000000in}{0.000000in}}%
\pgfpathlineto{\pgfqpoint{-0.027778in}{0.000000in}}%
\pgfusepath{stroke,fill}%
}%
\begin{pgfscope}%
\pgfsys@transformshift{0.708220in}{1.521599in}%
\pgfsys@useobject{currentmarker}{}%
\end{pgfscope}%
\end{pgfscope}%
\begin{pgfscope}%
\pgfsetbuttcap%
\pgfsetroundjoin%
\definecolor{currentfill}{rgb}{0.000000,0.000000,0.000000}%
\pgfsetfillcolor{currentfill}%
\pgfsetlinewidth{0.602250pt}%
\definecolor{currentstroke}{rgb}{0.000000,0.000000,0.000000}%
\pgfsetstrokecolor{currentstroke}%
\pgfsetdash{}{0pt}%
\pgfsys@defobject{currentmarker}{\pgfqpoint{-0.027778in}{0.000000in}}{\pgfqpoint{-0.000000in}{0.000000in}}{%
\pgfpathmoveto{\pgfqpoint{-0.000000in}{0.000000in}}%
\pgfpathlineto{\pgfqpoint{-0.027778in}{0.000000in}}%
\pgfusepath{stroke,fill}%
}%
\begin{pgfscope}%
\pgfsys@transformshift{0.708220in}{1.604978in}%
\pgfsys@useobject{currentmarker}{}%
\end{pgfscope}%
\end{pgfscope}%
\begin{pgfscope}%
\pgfsetbuttcap%
\pgfsetroundjoin%
\definecolor{currentfill}{rgb}{0.000000,0.000000,0.000000}%
\pgfsetfillcolor{currentfill}%
\pgfsetlinewidth{0.602250pt}%
\definecolor{currentstroke}{rgb}{0.000000,0.000000,0.000000}%
\pgfsetstrokecolor{currentstroke}%
\pgfsetdash{}{0pt}%
\pgfsys@defobject{currentmarker}{\pgfqpoint{-0.027778in}{0.000000in}}{\pgfqpoint{-0.000000in}{0.000000in}}{%
\pgfpathmoveto{\pgfqpoint{-0.000000in}{0.000000in}}%
\pgfpathlineto{\pgfqpoint{-0.027778in}{0.000000in}}%
\pgfusepath{stroke,fill}%
}%
\begin{pgfscope}%
\pgfsys@transformshift{0.708220in}{1.669653in}%
\pgfsys@useobject{currentmarker}{}%
\end{pgfscope}%
\end{pgfscope}%
\begin{pgfscope}%
\pgfsetbuttcap%
\pgfsetroundjoin%
\definecolor{currentfill}{rgb}{0.000000,0.000000,0.000000}%
\pgfsetfillcolor{currentfill}%
\pgfsetlinewidth{0.602250pt}%
\definecolor{currentstroke}{rgb}{0.000000,0.000000,0.000000}%
\pgfsetstrokecolor{currentstroke}%
\pgfsetdash{}{0pt}%
\pgfsys@defobject{currentmarker}{\pgfqpoint{-0.027778in}{0.000000in}}{\pgfqpoint{-0.000000in}{0.000000in}}{%
\pgfpathmoveto{\pgfqpoint{-0.000000in}{0.000000in}}%
\pgfpathlineto{\pgfqpoint{-0.027778in}{0.000000in}}%
\pgfusepath{stroke,fill}%
}%
\begin{pgfscope}%
\pgfsys@transformshift{0.708220in}{1.722495in}%
\pgfsys@useobject{currentmarker}{}%
\end{pgfscope}%
\end{pgfscope}%
\begin{pgfscope}%
\pgfsetbuttcap%
\pgfsetroundjoin%
\definecolor{currentfill}{rgb}{0.000000,0.000000,0.000000}%
\pgfsetfillcolor{currentfill}%
\pgfsetlinewidth{0.602250pt}%
\definecolor{currentstroke}{rgb}{0.000000,0.000000,0.000000}%
\pgfsetstrokecolor{currentstroke}%
\pgfsetdash{}{0pt}%
\pgfsys@defobject{currentmarker}{\pgfqpoint{-0.027778in}{0.000000in}}{\pgfqpoint{-0.000000in}{0.000000in}}{%
\pgfpathmoveto{\pgfqpoint{-0.000000in}{0.000000in}}%
\pgfpathlineto{\pgfqpoint{-0.027778in}{0.000000in}}%
\pgfusepath{stroke,fill}%
}%
\begin{pgfscope}%
\pgfsys@transformshift{0.708220in}{1.767173in}%
\pgfsys@useobject{currentmarker}{}%
\end{pgfscope}%
\end{pgfscope}%
\begin{pgfscope}%
\pgfsetbuttcap%
\pgfsetroundjoin%
\definecolor{currentfill}{rgb}{0.000000,0.000000,0.000000}%
\pgfsetfillcolor{currentfill}%
\pgfsetlinewidth{0.602250pt}%
\definecolor{currentstroke}{rgb}{0.000000,0.000000,0.000000}%
\pgfsetstrokecolor{currentstroke}%
\pgfsetdash{}{0pt}%
\pgfsys@defobject{currentmarker}{\pgfqpoint{-0.027778in}{0.000000in}}{\pgfqpoint{-0.000000in}{0.000000in}}{%
\pgfpathmoveto{\pgfqpoint{-0.000000in}{0.000000in}}%
\pgfpathlineto{\pgfqpoint{-0.027778in}{0.000000in}}%
\pgfusepath{stroke,fill}%
}%
\begin{pgfscope}%
\pgfsys@transformshift{0.708220in}{1.805875in}%
\pgfsys@useobject{currentmarker}{}%
\end{pgfscope}%
\end{pgfscope}%
\begin{pgfscope}%
\pgfsetbuttcap%
\pgfsetroundjoin%
\definecolor{currentfill}{rgb}{0.000000,0.000000,0.000000}%
\pgfsetfillcolor{currentfill}%
\pgfsetlinewidth{0.602250pt}%
\definecolor{currentstroke}{rgb}{0.000000,0.000000,0.000000}%
\pgfsetstrokecolor{currentstroke}%
\pgfsetdash{}{0pt}%
\pgfsys@defobject{currentmarker}{\pgfqpoint{-0.027778in}{0.000000in}}{\pgfqpoint{-0.000000in}{0.000000in}}{%
\pgfpathmoveto{\pgfqpoint{-0.000000in}{0.000000in}}%
\pgfpathlineto{\pgfqpoint{-0.027778in}{0.000000in}}%
\pgfusepath{stroke,fill}%
}%
\begin{pgfscope}%
\pgfsys@transformshift{0.708220in}{1.840012in}%
\pgfsys@useobject{currentmarker}{}%
\end{pgfscope}%
\end{pgfscope}%
\begin{pgfscope}%
\pgfsetbuttcap%
\pgfsetroundjoin%
\definecolor{currentfill}{rgb}{0.000000,0.000000,0.000000}%
\pgfsetfillcolor{currentfill}%
\pgfsetlinewidth{0.602250pt}%
\definecolor{currentstroke}{rgb}{0.000000,0.000000,0.000000}%
\pgfsetstrokecolor{currentstroke}%
\pgfsetdash{}{0pt}%
\pgfsys@defobject{currentmarker}{\pgfqpoint{-0.027778in}{0.000000in}}{\pgfqpoint{-0.000000in}{0.000000in}}{%
\pgfpathmoveto{\pgfqpoint{-0.000000in}{0.000000in}}%
\pgfpathlineto{\pgfqpoint{-0.027778in}{0.000000in}}%
\pgfusepath{stroke,fill}%
}%
\begin{pgfscope}%
\pgfsys@transformshift{0.708220in}{2.071445in}%
\pgfsys@useobject{currentmarker}{}%
\end{pgfscope}%
\end{pgfscope}%
\begin{pgfscope}%
\pgfsetbuttcap%
\pgfsetroundjoin%
\definecolor{currentfill}{rgb}{0.000000,0.000000,0.000000}%
\pgfsetfillcolor{currentfill}%
\pgfsetlinewidth{0.602250pt}%
\definecolor{currentstroke}{rgb}{0.000000,0.000000,0.000000}%
\pgfsetstrokecolor{currentstroke}%
\pgfsetdash{}{0pt}%
\pgfsys@defobject{currentmarker}{\pgfqpoint{-0.027778in}{0.000000in}}{\pgfqpoint{-0.000000in}{0.000000in}}{%
\pgfpathmoveto{\pgfqpoint{-0.000000in}{0.000000in}}%
\pgfpathlineto{\pgfqpoint{-0.027778in}{0.000000in}}%
\pgfusepath{stroke,fill}%
}%
\begin{pgfscope}%
\pgfsys@transformshift{0.708220in}{2.188962in}%
\pgfsys@useobject{currentmarker}{}%
\end{pgfscope}%
\end{pgfscope}%
\begin{pgfscope}%
\pgfsetbuttcap%
\pgfsetroundjoin%
\definecolor{currentfill}{rgb}{0.000000,0.000000,0.000000}%
\pgfsetfillcolor{currentfill}%
\pgfsetlinewidth{0.602250pt}%
\definecolor{currentstroke}{rgb}{0.000000,0.000000,0.000000}%
\pgfsetstrokecolor{currentstroke}%
\pgfsetdash{}{0pt}%
\pgfsys@defobject{currentmarker}{\pgfqpoint{-0.027778in}{0.000000in}}{\pgfqpoint{-0.000000in}{0.000000in}}{%
\pgfpathmoveto{\pgfqpoint{-0.000000in}{0.000000in}}%
\pgfpathlineto{\pgfqpoint{-0.027778in}{0.000000in}}%
\pgfusepath{stroke,fill}%
}%
\begin{pgfscope}%
\pgfsys@transformshift{0.708220in}{2.272342in}%
\pgfsys@useobject{currentmarker}{}%
\end{pgfscope}%
\end{pgfscope}%
\begin{pgfscope}%
\pgfsetbuttcap%
\pgfsetroundjoin%
\definecolor{currentfill}{rgb}{0.000000,0.000000,0.000000}%
\pgfsetfillcolor{currentfill}%
\pgfsetlinewidth{0.602250pt}%
\definecolor{currentstroke}{rgb}{0.000000,0.000000,0.000000}%
\pgfsetstrokecolor{currentstroke}%
\pgfsetdash{}{0pt}%
\pgfsys@defobject{currentmarker}{\pgfqpoint{-0.027778in}{0.000000in}}{\pgfqpoint{-0.000000in}{0.000000in}}{%
\pgfpathmoveto{\pgfqpoint{-0.000000in}{0.000000in}}%
\pgfpathlineto{\pgfqpoint{-0.027778in}{0.000000in}}%
\pgfusepath{stroke,fill}%
}%
\begin{pgfscope}%
\pgfsys@transformshift{0.708220in}{2.337016in}%
\pgfsys@useobject{currentmarker}{}%
\end{pgfscope}%
\end{pgfscope}%
\begin{pgfscope}%
\pgfsetbuttcap%
\pgfsetroundjoin%
\definecolor{currentfill}{rgb}{0.000000,0.000000,0.000000}%
\pgfsetfillcolor{currentfill}%
\pgfsetlinewidth{0.602250pt}%
\definecolor{currentstroke}{rgb}{0.000000,0.000000,0.000000}%
\pgfsetstrokecolor{currentstroke}%
\pgfsetdash{}{0pt}%
\pgfsys@defobject{currentmarker}{\pgfqpoint{-0.027778in}{0.000000in}}{\pgfqpoint{-0.000000in}{0.000000in}}{%
\pgfpathmoveto{\pgfqpoint{-0.000000in}{0.000000in}}%
\pgfpathlineto{\pgfqpoint{-0.027778in}{0.000000in}}%
\pgfusepath{stroke,fill}%
}%
\begin{pgfscope}%
\pgfsys@transformshift{0.708220in}{2.389858in}%
\pgfsys@useobject{currentmarker}{}%
\end{pgfscope}%
\end{pgfscope}%
\begin{pgfscope}%
\pgfsetbuttcap%
\pgfsetroundjoin%
\definecolor{currentfill}{rgb}{0.000000,0.000000,0.000000}%
\pgfsetfillcolor{currentfill}%
\pgfsetlinewidth{0.602250pt}%
\definecolor{currentstroke}{rgb}{0.000000,0.000000,0.000000}%
\pgfsetstrokecolor{currentstroke}%
\pgfsetdash{}{0pt}%
\pgfsys@defobject{currentmarker}{\pgfqpoint{-0.027778in}{0.000000in}}{\pgfqpoint{-0.000000in}{0.000000in}}{%
\pgfpathmoveto{\pgfqpoint{-0.000000in}{0.000000in}}%
\pgfpathlineto{\pgfqpoint{-0.027778in}{0.000000in}}%
\pgfusepath{stroke,fill}%
}%
\begin{pgfscope}%
\pgfsys@transformshift{0.708220in}{2.434536in}%
\pgfsys@useobject{currentmarker}{}%
\end{pgfscope}%
\end{pgfscope}%
\begin{pgfscope}%
\pgfsetbuttcap%
\pgfsetroundjoin%
\definecolor{currentfill}{rgb}{0.000000,0.000000,0.000000}%
\pgfsetfillcolor{currentfill}%
\pgfsetlinewidth{0.602250pt}%
\definecolor{currentstroke}{rgb}{0.000000,0.000000,0.000000}%
\pgfsetstrokecolor{currentstroke}%
\pgfsetdash{}{0pt}%
\pgfsys@defobject{currentmarker}{\pgfqpoint{-0.027778in}{0.000000in}}{\pgfqpoint{-0.000000in}{0.000000in}}{%
\pgfpathmoveto{\pgfqpoint{-0.000000in}{0.000000in}}%
\pgfpathlineto{\pgfqpoint{-0.027778in}{0.000000in}}%
\pgfusepath{stroke,fill}%
}%
\begin{pgfscope}%
\pgfsys@transformshift{0.708220in}{2.473238in}%
\pgfsys@useobject{currentmarker}{}%
\end{pgfscope}%
\end{pgfscope}%
\begin{pgfscope}%
\pgfsetbuttcap%
\pgfsetroundjoin%
\definecolor{currentfill}{rgb}{0.000000,0.000000,0.000000}%
\pgfsetfillcolor{currentfill}%
\pgfsetlinewidth{0.602250pt}%
\definecolor{currentstroke}{rgb}{0.000000,0.000000,0.000000}%
\pgfsetstrokecolor{currentstroke}%
\pgfsetdash{}{0pt}%
\pgfsys@defobject{currentmarker}{\pgfqpoint{-0.027778in}{0.000000in}}{\pgfqpoint{-0.000000in}{0.000000in}}{%
\pgfpathmoveto{\pgfqpoint{-0.000000in}{0.000000in}}%
\pgfpathlineto{\pgfqpoint{-0.027778in}{0.000000in}}%
\pgfusepath{stroke,fill}%
}%
\begin{pgfscope}%
\pgfsys@transformshift{0.708220in}{2.507375in}%
\pgfsys@useobject{currentmarker}{}%
\end{pgfscope}%
\end{pgfscope}%
\begin{pgfscope}%
\pgfsetbuttcap%
\pgfsetroundjoin%
\definecolor{currentfill}{rgb}{0.000000,0.000000,0.000000}%
\pgfsetfillcolor{currentfill}%
\pgfsetlinewidth{0.602250pt}%
\definecolor{currentstroke}{rgb}{0.000000,0.000000,0.000000}%
\pgfsetstrokecolor{currentstroke}%
\pgfsetdash{}{0pt}%
\pgfsys@defobject{currentmarker}{\pgfqpoint{-0.027778in}{0.000000in}}{\pgfqpoint{-0.000000in}{0.000000in}}{%
\pgfpathmoveto{\pgfqpoint{-0.000000in}{0.000000in}}%
\pgfpathlineto{\pgfqpoint{-0.027778in}{0.000000in}}%
\pgfusepath{stroke,fill}%
}%
\begin{pgfscope}%
\pgfsys@transformshift{0.708220in}{2.738808in}%
\pgfsys@useobject{currentmarker}{}%
\end{pgfscope}%
\end{pgfscope}%
\begin{pgfscope}%
\pgfsetbuttcap%
\pgfsetroundjoin%
\definecolor{currentfill}{rgb}{0.000000,0.000000,0.000000}%
\pgfsetfillcolor{currentfill}%
\pgfsetlinewidth{0.602250pt}%
\definecolor{currentstroke}{rgb}{0.000000,0.000000,0.000000}%
\pgfsetstrokecolor{currentstroke}%
\pgfsetdash{}{0pt}%
\pgfsys@defobject{currentmarker}{\pgfqpoint{-0.027778in}{0.000000in}}{\pgfqpoint{-0.000000in}{0.000000in}}{%
\pgfpathmoveto{\pgfqpoint{-0.000000in}{0.000000in}}%
\pgfpathlineto{\pgfqpoint{-0.027778in}{0.000000in}}%
\pgfusepath{stroke,fill}%
}%
\begin{pgfscope}%
\pgfsys@transformshift{0.708220in}{2.856325in}%
\pgfsys@useobject{currentmarker}{}%
\end{pgfscope}%
\end{pgfscope}%
\begin{pgfscope}%
\pgfsetbuttcap%
\pgfsetroundjoin%
\definecolor{currentfill}{rgb}{0.000000,0.000000,0.000000}%
\pgfsetfillcolor{currentfill}%
\pgfsetlinewidth{0.602250pt}%
\definecolor{currentstroke}{rgb}{0.000000,0.000000,0.000000}%
\pgfsetstrokecolor{currentstroke}%
\pgfsetdash{}{0pt}%
\pgfsys@defobject{currentmarker}{\pgfqpoint{-0.027778in}{0.000000in}}{\pgfqpoint{-0.000000in}{0.000000in}}{%
\pgfpathmoveto{\pgfqpoint{-0.000000in}{0.000000in}}%
\pgfpathlineto{\pgfqpoint{-0.027778in}{0.000000in}}%
\pgfusepath{stroke,fill}%
}%
\begin{pgfscope}%
\pgfsys@transformshift{0.708220in}{2.939705in}%
\pgfsys@useobject{currentmarker}{}%
\end{pgfscope}%
\end{pgfscope}%
\begin{pgfscope}%
\pgfsetbuttcap%
\pgfsetroundjoin%
\definecolor{currentfill}{rgb}{0.000000,0.000000,0.000000}%
\pgfsetfillcolor{currentfill}%
\pgfsetlinewidth{0.602250pt}%
\definecolor{currentstroke}{rgb}{0.000000,0.000000,0.000000}%
\pgfsetstrokecolor{currentstroke}%
\pgfsetdash{}{0pt}%
\pgfsys@defobject{currentmarker}{\pgfqpoint{-0.027778in}{0.000000in}}{\pgfqpoint{-0.000000in}{0.000000in}}{%
\pgfpathmoveto{\pgfqpoint{-0.000000in}{0.000000in}}%
\pgfpathlineto{\pgfqpoint{-0.027778in}{0.000000in}}%
\pgfusepath{stroke,fill}%
}%
\begin{pgfscope}%
\pgfsys@transformshift{0.708220in}{3.004379in}%
\pgfsys@useobject{currentmarker}{}%
\end{pgfscope}%
\end{pgfscope}%
\begin{pgfscope}%
\pgfsetbuttcap%
\pgfsetroundjoin%
\definecolor{currentfill}{rgb}{0.000000,0.000000,0.000000}%
\pgfsetfillcolor{currentfill}%
\pgfsetlinewidth{0.602250pt}%
\definecolor{currentstroke}{rgb}{0.000000,0.000000,0.000000}%
\pgfsetstrokecolor{currentstroke}%
\pgfsetdash{}{0pt}%
\pgfsys@defobject{currentmarker}{\pgfqpoint{-0.027778in}{0.000000in}}{\pgfqpoint{-0.000000in}{0.000000in}}{%
\pgfpathmoveto{\pgfqpoint{-0.000000in}{0.000000in}}%
\pgfpathlineto{\pgfqpoint{-0.027778in}{0.000000in}}%
\pgfusepath{stroke,fill}%
}%
\begin{pgfscope}%
\pgfsys@transformshift{0.708220in}{3.057222in}%
\pgfsys@useobject{currentmarker}{}%
\end{pgfscope}%
\end{pgfscope}%
\begin{pgfscope}%
\pgfsetbuttcap%
\pgfsetroundjoin%
\definecolor{currentfill}{rgb}{0.000000,0.000000,0.000000}%
\pgfsetfillcolor{currentfill}%
\pgfsetlinewidth{0.602250pt}%
\definecolor{currentstroke}{rgb}{0.000000,0.000000,0.000000}%
\pgfsetstrokecolor{currentstroke}%
\pgfsetdash{}{0pt}%
\pgfsys@defobject{currentmarker}{\pgfqpoint{-0.027778in}{0.000000in}}{\pgfqpoint{-0.000000in}{0.000000in}}{%
\pgfpathmoveto{\pgfqpoint{-0.000000in}{0.000000in}}%
\pgfpathlineto{\pgfqpoint{-0.027778in}{0.000000in}}%
\pgfusepath{stroke,fill}%
}%
\begin{pgfscope}%
\pgfsys@transformshift{0.708220in}{3.101899in}%
\pgfsys@useobject{currentmarker}{}%
\end{pgfscope}%
\end{pgfscope}%
\begin{pgfscope}%
\pgfsetbuttcap%
\pgfsetroundjoin%
\definecolor{currentfill}{rgb}{0.000000,0.000000,0.000000}%
\pgfsetfillcolor{currentfill}%
\pgfsetlinewidth{0.602250pt}%
\definecolor{currentstroke}{rgb}{0.000000,0.000000,0.000000}%
\pgfsetstrokecolor{currentstroke}%
\pgfsetdash{}{0pt}%
\pgfsys@defobject{currentmarker}{\pgfqpoint{-0.027778in}{0.000000in}}{\pgfqpoint{-0.000000in}{0.000000in}}{%
\pgfpathmoveto{\pgfqpoint{-0.000000in}{0.000000in}}%
\pgfpathlineto{\pgfqpoint{-0.027778in}{0.000000in}}%
\pgfusepath{stroke,fill}%
}%
\begin{pgfscope}%
\pgfsys@transformshift{0.708220in}{3.140601in}%
\pgfsys@useobject{currentmarker}{}%
\end{pgfscope}%
\end{pgfscope}%
\begin{pgfscope}%
\pgfsetbuttcap%
\pgfsetroundjoin%
\definecolor{currentfill}{rgb}{0.000000,0.000000,0.000000}%
\pgfsetfillcolor{currentfill}%
\pgfsetlinewidth{0.602250pt}%
\definecolor{currentstroke}{rgb}{0.000000,0.000000,0.000000}%
\pgfsetstrokecolor{currentstroke}%
\pgfsetdash{}{0pt}%
\pgfsys@defobject{currentmarker}{\pgfqpoint{-0.027778in}{0.000000in}}{\pgfqpoint{-0.000000in}{0.000000in}}{%
\pgfpathmoveto{\pgfqpoint{-0.000000in}{0.000000in}}%
\pgfpathlineto{\pgfqpoint{-0.027778in}{0.000000in}}%
\pgfusepath{stroke,fill}%
}%
\begin{pgfscope}%
\pgfsys@transformshift{0.708220in}{3.174738in}%
\pgfsys@useobject{currentmarker}{}%
\end{pgfscope}%
\end{pgfscope}%
\begin{pgfscope}%
\definecolor{textcolor}{rgb}{0.000000,0.000000,0.000000}%
\pgfsetstrokecolor{textcolor}%
\pgfsetfillcolor{textcolor}%
\pgftext[x=0.288855in,y=1.870549in,,bottom,rotate=90.000000]{\color{textcolor}\rmfamily\fontsize{10.000000}{12.000000}\selectfont Median solving time (s)}%
\end{pgfscope}%
\begin{pgfscope}%
\pgfpathrectangle{\pgfqpoint{0.708220in}{0.535823in}}{\pgfqpoint{5.141780in}{2.669453in}}%
\pgfusepath{clip}%
\pgfsetrectcap%
\pgfsetroundjoin%
\pgfsetlinewidth{1.003750pt}%
\definecolor{currentstroke}{rgb}{0.866667,0.058824,0.058824}%
\pgfsetstrokecolor{currentstroke}%
\pgfsetdash{}{0pt}%
\pgfpathmoveto{\pgfqpoint{1.928026in}{0.525823in}}%
\pgfpathlineto{\pgfqpoint{2.058359in}{0.761651in}}%
\pgfpathlineto{\pgfqpoint{2.283382in}{1.194612in}}%
\pgfpathlineto{\pgfqpoint{2.508405in}{1.631965in}}%
\pgfpathlineto{\pgfqpoint{2.733429in}{2.079842in}}%
\pgfpathlineto{\pgfqpoint{2.958452in}{2.526097in}}%
\pgfpathlineto{\pgfqpoint{3.183475in}{3.000213in}}%
\pgfusepath{stroke}%
\end{pgfscope}%
\begin{pgfscope}%
\pgfpathrectangle{\pgfqpoint{0.708220in}{0.535823in}}{\pgfqpoint{5.141780in}{2.669453in}}%
\pgfusepath{clip}%
\pgfsetbuttcap%
\pgfsetmiterjoin%
\definecolor{currentfill}{rgb}{0.866667,0.058824,0.058824}%
\pgfsetfillcolor{currentfill}%
\pgfsetlinewidth{0.501875pt}%
\definecolor{currentstroke}{rgb}{0.000000,0.000000,0.000000}%
\pgfsetstrokecolor{currentstroke}%
\pgfsetdash{}{0pt}%
\pgfsys@defobject{currentmarker}{\pgfqpoint{-0.033023in}{-0.028091in}}{\pgfqpoint{0.033023in}{0.034722in}}{%
\pgfpathmoveto{\pgfqpoint{0.000000in}{0.034722in}}%
\pgfpathlineto{\pgfqpoint{-0.033023in}{0.010730in}}%
\pgfpathlineto{\pgfqpoint{-0.020409in}{-0.028091in}}%
\pgfpathlineto{\pgfqpoint{0.020409in}{-0.028091in}}%
\pgfpathlineto{\pgfqpoint{0.033023in}{0.010730in}}%
\pgfpathclose%
\pgfusepath{stroke,fill}%
}%
\begin{pgfscope}%
\pgfsys@transformshift{1.833336in}{0.354487in}%
\pgfsys@useobject{currentmarker}{}%
\end{pgfscope}%
\begin{pgfscope}%
\pgfsys@transformshift{2.058359in}{0.761651in}%
\pgfsys@useobject{currentmarker}{}%
\end{pgfscope}%
\begin{pgfscope}%
\pgfsys@transformshift{2.283382in}{1.194612in}%
\pgfsys@useobject{currentmarker}{}%
\end{pgfscope}%
\begin{pgfscope}%
\pgfsys@transformshift{2.508405in}{1.631965in}%
\pgfsys@useobject{currentmarker}{}%
\end{pgfscope}%
\begin{pgfscope}%
\pgfsys@transformshift{2.733429in}{2.079842in}%
\pgfsys@useobject{currentmarker}{}%
\end{pgfscope}%
\begin{pgfscope}%
\pgfsys@transformshift{2.958452in}{2.526097in}%
\pgfsys@useobject{currentmarker}{}%
\end{pgfscope}%
\begin{pgfscope}%
\pgfsys@transformshift{3.183475in}{3.000213in}%
\pgfsys@useobject{currentmarker}{}%
\end{pgfscope}%
\end{pgfscope}%
\begin{pgfscope}%
\pgfpathrectangle{\pgfqpoint{0.708220in}{0.535823in}}{\pgfqpoint{5.141780in}{2.669453in}}%
\pgfusepath{clip}%
\pgfsetrectcap%
\pgfsetroundjoin%
\pgfsetlinewidth{1.003750pt}%
\definecolor{currentstroke}{rgb}{0.000000,0.000000,0.200000}%
\pgfsetstrokecolor{currentstroke}%
\pgfsetdash{}{0pt}%
\pgfpathmoveto{\pgfqpoint{1.833336in}{0.676718in}}%
\pgfpathlineto{\pgfqpoint{2.058359in}{0.906415in}}%
\pgfpathlineto{\pgfqpoint{2.283382in}{1.295336in}}%
\pgfpathlineto{\pgfqpoint{2.508405in}{1.701898in}}%
\pgfpathlineto{\pgfqpoint{2.733429in}{1.970334in}}%
\pgfpathlineto{\pgfqpoint{2.958452in}{2.394586in}}%
\pgfpathlineto{\pgfqpoint{3.183475in}{2.793972in}}%
\pgfusepath{stroke}%
\end{pgfscope}%
\begin{pgfscope}%
\pgfpathrectangle{\pgfqpoint{0.708220in}{0.535823in}}{\pgfqpoint{5.141780in}{2.669453in}}%
\pgfusepath{clip}%
\pgfsetbuttcap%
\pgfsetmiterjoin%
\definecolor{currentfill}{rgb}{0.000000,0.000000,0.200000}%
\pgfsetfillcolor{currentfill}%
\pgfsetlinewidth{0.501875pt}%
\definecolor{currentstroke}{rgb}{0.000000,0.000000,0.000000}%
\pgfsetstrokecolor{currentstroke}%
\pgfsetdash{}{0pt}%
\pgfsys@defobject{currentmarker}{\pgfqpoint{-0.034722in}{-0.034722in}}{\pgfqpoint{0.034722in}{0.034722in}}{%
\pgfpathmoveto{\pgfqpoint{-0.011574in}{-0.034722in}}%
\pgfpathlineto{\pgfqpoint{0.011574in}{-0.034722in}}%
\pgfpathlineto{\pgfqpoint{0.011574in}{-0.011574in}}%
\pgfpathlineto{\pgfqpoint{0.034722in}{-0.011574in}}%
\pgfpathlineto{\pgfqpoint{0.034722in}{0.011574in}}%
\pgfpathlineto{\pgfqpoint{0.011574in}{0.011574in}}%
\pgfpathlineto{\pgfqpoint{0.011574in}{0.034722in}}%
\pgfpathlineto{\pgfqpoint{-0.011574in}{0.034722in}}%
\pgfpathlineto{\pgfqpoint{-0.011574in}{0.011574in}}%
\pgfpathlineto{\pgfqpoint{-0.034722in}{0.011574in}}%
\pgfpathlineto{\pgfqpoint{-0.034722in}{-0.011574in}}%
\pgfpathlineto{\pgfqpoint{-0.011574in}{-0.011574in}}%
\pgfpathclose%
\pgfusepath{stroke,fill}%
}%
\begin{pgfscope}%
\pgfsys@transformshift{1.833336in}{0.676718in}%
\pgfsys@useobject{currentmarker}{}%
\end{pgfscope}%
\begin{pgfscope}%
\pgfsys@transformshift{2.058359in}{0.906415in}%
\pgfsys@useobject{currentmarker}{}%
\end{pgfscope}%
\begin{pgfscope}%
\pgfsys@transformshift{2.283382in}{1.295336in}%
\pgfsys@useobject{currentmarker}{}%
\end{pgfscope}%
\begin{pgfscope}%
\pgfsys@transformshift{2.508405in}{1.701898in}%
\pgfsys@useobject{currentmarker}{}%
\end{pgfscope}%
\begin{pgfscope}%
\pgfsys@transformshift{2.733429in}{1.970334in}%
\pgfsys@useobject{currentmarker}{}%
\end{pgfscope}%
\begin{pgfscope}%
\pgfsys@transformshift{2.958452in}{2.394586in}%
\pgfsys@useobject{currentmarker}{}%
\end{pgfscope}%
\begin{pgfscope}%
\pgfsys@transformshift{3.183475in}{2.793972in}%
\pgfsys@useobject{currentmarker}{}%
\end{pgfscope}%
\end{pgfscope}%
\begin{pgfscope}%
\pgfpathrectangle{\pgfqpoint{0.708220in}{0.535823in}}{\pgfqpoint{5.141780in}{2.669453in}}%
\pgfusepath{clip}%
\pgfsetrectcap%
\pgfsetroundjoin%
\pgfsetlinewidth{1.003750pt}%
\definecolor{currentstroke}{rgb}{0.000000,0.000000,0.866667}%
\pgfsetstrokecolor{currentstroke}%
\pgfsetdash{}{0pt}%
\pgfpathmoveto{\pgfqpoint{1.833336in}{0.610115in}}%
\pgfpathlineto{\pgfqpoint{2.058359in}{0.767859in}}%
\pgfpathlineto{\pgfqpoint{2.283382in}{1.115023in}}%
\pgfpathlineto{\pgfqpoint{2.508405in}{1.546157in}}%
\pgfpathlineto{\pgfqpoint{2.733429in}{1.940281in}}%
\pgfpathlineto{\pgfqpoint{2.958452in}{2.294916in}}%
\pgfpathlineto{\pgfqpoint{3.183475in}{2.632491in}}%
\pgfpathlineto{\pgfqpoint{3.408498in}{3.043909in}}%
\pgfusepath{stroke}%
\end{pgfscope}%
\begin{pgfscope}%
\pgfpathrectangle{\pgfqpoint{0.708220in}{0.535823in}}{\pgfqpoint{5.141780in}{2.669453in}}%
\pgfusepath{clip}%
\pgfsetbuttcap%
\pgfsetmiterjoin%
\definecolor{currentfill}{rgb}{0.000000,0.000000,0.866667}%
\pgfsetfillcolor{currentfill}%
\pgfsetlinewidth{0.501875pt}%
\definecolor{currentstroke}{rgb}{0.000000,0.000000,0.000000}%
\pgfsetstrokecolor{currentstroke}%
\pgfsetdash{}{0pt}%
\pgfsys@defobject{currentmarker}{\pgfqpoint{-0.029463in}{-0.049105in}}{\pgfqpoint{0.029463in}{0.049105in}}{%
\pgfpathmoveto{\pgfqpoint{0.000000in}{-0.049105in}}%
\pgfpathlineto{\pgfqpoint{0.029463in}{0.000000in}}%
\pgfpathlineto{\pgfqpoint{0.000000in}{0.049105in}}%
\pgfpathlineto{\pgfqpoint{-0.029463in}{0.000000in}}%
\pgfpathclose%
\pgfusepath{stroke,fill}%
}%
\begin{pgfscope}%
\pgfsys@transformshift{1.833336in}{0.610115in}%
\pgfsys@useobject{currentmarker}{}%
\end{pgfscope}%
\begin{pgfscope}%
\pgfsys@transformshift{2.058359in}{0.767859in}%
\pgfsys@useobject{currentmarker}{}%
\end{pgfscope}%
\begin{pgfscope}%
\pgfsys@transformshift{2.283382in}{1.115023in}%
\pgfsys@useobject{currentmarker}{}%
\end{pgfscope}%
\begin{pgfscope}%
\pgfsys@transformshift{2.508405in}{1.546157in}%
\pgfsys@useobject{currentmarker}{}%
\end{pgfscope}%
\begin{pgfscope}%
\pgfsys@transformshift{2.733429in}{1.940281in}%
\pgfsys@useobject{currentmarker}{}%
\end{pgfscope}%
\begin{pgfscope}%
\pgfsys@transformshift{2.958452in}{2.294916in}%
\pgfsys@useobject{currentmarker}{}%
\end{pgfscope}%
\begin{pgfscope}%
\pgfsys@transformshift{3.183475in}{2.632491in}%
\pgfsys@useobject{currentmarker}{}%
\end{pgfscope}%
\begin{pgfscope}%
\pgfsys@transformshift{3.408498in}{3.043909in}%
\pgfsys@useobject{currentmarker}{}%
\end{pgfscope}%
\end{pgfscope}%
\begin{pgfscope}%
\pgfpathrectangle{\pgfqpoint{0.708220in}{0.535823in}}{\pgfqpoint{5.141780in}{2.669453in}}%
\pgfusepath{clip}%
\pgfsetrectcap%
\pgfsetroundjoin%
\pgfsetlinewidth{1.003750pt}%
\definecolor{currentstroke}{rgb}{0.250980,0.231373,0.796078}%
\pgfsetstrokecolor{currentstroke}%
\pgfsetdash{}{0pt}%
\pgfpathmoveto{\pgfqpoint{2.095649in}{0.525823in}}%
\pgfpathlineto{\pgfqpoint{2.283382in}{0.846558in}}%
\pgfpathlineto{\pgfqpoint{2.508405in}{1.257717in}}%
\pgfpathlineto{\pgfqpoint{2.733429in}{1.657480in}}%
\pgfpathlineto{\pgfqpoint{2.958452in}{2.073652in}}%
\pgfpathlineto{\pgfqpoint{3.183475in}{2.469956in}}%
\pgfpathlineto{\pgfqpoint{3.408498in}{2.885012in}}%
\pgfusepath{stroke}%
\end{pgfscope}%
\begin{pgfscope}%
\pgfpathrectangle{\pgfqpoint{0.708220in}{0.535823in}}{\pgfqpoint{5.141780in}{2.669453in}}%
\pgfusepath{clip}%
\pgfsetbuttcap%
\pgfsetmiterjoin%
\definecolor{currentfill}{rgb}{0.250980,0.231373,0.796078}%
\pgfsetfillcolor{currentfill}%
\pgfsetlinewidth{0.501875pt}%
\definecolor{currentstroke}{rgb}{0.000000,0.000000,0.000000}%
\pgfsetstrokecolor{currentstroke}%
\pgfsetdash{}{0pt}%
\pgfsys@defobject{currentmarker}{\pgfqpoint{-0.034722in}{-0.034722in}}{\pgfqpoint{0.034722in}{0.034722in}}{%
\pgfpathmoveto{\pgfqpoint{-0.034722in}{-0.034722in}}%
\pgfpathlineto{\pgfqpoint{0.034722in}{-0.034722in}}%
\pgfpathlineto{\pgfqpoint{0.034722in}{0.034722in}}%
\pgfpathlineto{\pgfqpoint{-0.034722in}{0.034722in}}%
\pgfpathclose%
\pgfusepath{stroke,fill}%
}%
\begin{pgfscope}%
\pgfsys@transformshift{1.833336in}{0.132041in}%
\pgfsys@useobject{currentmarker}{}%
\end{pgfscope}%
\begin{pgfscope}%
\pgfsys@transformshift{2.058359in}{0.462115in}%
\pgfsys@useobject{currentmarker}{}%
\end{pgfscope}%
\begin{pgfscope}%
\pgfsys@transformshift{2.283382in}{0.846558in}%
\pgfsys@useobject{currentmarker}{}%
\end{pgfscope}%
\begin{pgfscope}%
\pgfsys@transformshift{2.508405in}{1.257717in}%
\pgfsys@useobject{currentmarker}{}%
\end{pgfscope}%
\begin{pgfscope}%
\pgfsys@transformshift{2.733429in}{1.657480in}%
\pgfsys@useobject{currentmarker}{}%
\end{pgfscope}%
\begin{pgfscope}%
\pgfsys@transformshift{2.958452in}{2.073652in}%
\pgfsys@useobject{currentmarker}{}%
\end{pgfscope}%
\begin{pgfscope}%
\pgfsys@transformshift{3.183475in}{2.469956in}%
\pgfsys@useobject{currentmarker}{}%
\end{pgfscope}%
\begin{pgfscope}%
\pgfsys@transformshift{3.408498in}{2.885012in}%
\pgfsys@useobject{currentmarker}{}%
\end{pgfscope}%
\end{pgfscope}%
\begin{pgfscope}%
\pgfpathrectangle{\pgfqpoint{0.708220in}{0.535823in}}{\pgfqpoint{5.141780in}{2.669453in}}%
\pgfusepath{clip}%
\pgfsetrectcap%
\pgfsetroundjoin%
\pgfsetlinewidth{1.003750pt}%
\definecolor{currentstroke}{rgb}{0.615686,0.007843,0.843137}%
\pgfsetstrokecolor{currentstroke}%
\pgfsetdash{}{0pt}%
\pgfpathmoveto{\pgfqpoint{2.027184in}{0.525823in}}%
\pgfpathlineto{\pgfqpoint{2.058359in}{0.571199in}}%
\pgfpathlineto{\pgfqpoint{2.283382in}{0.866486in}}%
\pgfpathlineto{\pgfqpoint{2.508405in}{1.219071in}}%
\pgfpathlineto{\pgfqpoint{2.733429in}{1.535650in}}%
\pgfpathlineto{\pgfqpoint{2.958452in}{1.852695in}}%
\pgfpathlineto{\pgfqpoint{3.183475in}{2.103243in}}%
\pgfpathlineto{\pgfqpoint{3.408498in}{2.465875in}}%
\pgfpathlineto{\pgfqpoint{3.633521in}{2.755438in}}%
\pgfpathlineto{\pgfqpoint{3.858545in}{3.018596in}}%
\pgfusepath{stroke}%
\end{pgfscope}%
\begin{pgfscope}%
\pgfpathrectangle{\pgfqpoint{0.708220in}{0.535823in}}{\pgfqpoint{5.141780in}{2.669453in}}%
\pgfusepath{clip}%
\pgfsetbuttcap%
\pgfsetroundjoin%
\definecolor{currentfill}{rgb}{0.615686,0.007843,0.843137}%
\pgfsetfillcolor{currentfill}%
\pgfsetlinewidth{0.501875pt}%
\definecolor{currentstroke}{rgb}{0.000000,0.000000,0.000000}%
\pgfsetstrokecolor{currentstroke}%
\pgfsetdash{}{0pt}%
\pgfsys@defobject{currentmarker}{\pgfqpoint{-0.034722in}{-0.034722in}}{\pgfqpoint{0.034722in}{0.034722in}}{%
\pgfpathmoveto{\pgfqpoint{0.000000in}{-0.034722in}}%
\pgfpathcurveto{\pgfqpoint{0.009208in}{-0.034722in}}{\pgfqpoint{0.018041in}{-0.031064in}}{\pgfqpoint{0.024552in}{-0.024552in}}%
\pgfpathcurveto{\pgfqpoint{0.031064in}{-0.018041in}}{\pgfqpoint{0.034722in}{-0.009208in}}{\pgfqpoint{0.034722in}{0.000000in}}%
\pgfpathcurveto{\pgfqpoint{0.034722in}{0.009208in}}{\pgfqpoint{0.031064in}{0.018041in}}{\pgfqpoint{0.024552in}{0.024552in}}%
\pgfpathcurveto{\pgfqpoint{0.018041in}{0.031064in}}{\pgfqpoint{0.009208in}{0.034722in}}{\pgfqpoint{0.000000in}{0.034722in}}%
\pgfpathcurveto{\pgfqpoint{-0.009208in}{0.034722in}}{\pgfqpoint{-0.018041in}{0.031064in}}{\pgfqpoint{-0.024552in}{0.024552in}}%
\pgfpathcurveto{\pgfqpoint{-0.031064in}{0.018041in}}{\pgfqpoint{-0.034722in}{0.009208in}}{\pgfqpoint{-0.034722in}{0.000000in}}%
\pgfpathcurveto{\pgfqpoint{-0.034722in}{-0.009208in}}{\pgfqpoint{-0.031064in}{-0.018041in}}{\pgfqpoint{-0.024552in}{-0.024552in}}%
\pgfpathcurveto{\pgfqpoint{-0.018041in}{-0.031064in}}{\pgfqpoint{-0.009208in}{-0.034722in}}{\pgfqpoint{0.000000in}{-0.034722in}}%
\pgfpathclose%
\pgfusepath{stroke,fill}%
}%
\begin{pgfscope}%
\pgfsys@transformshift{1.833336in}{0.243665in}%
\pgfsys@useobject{currentmarker}{}%
\end{pgfscope}%
\begin{pgfscope}%
\pgfsys@transformshift{2.058359in}{0.571199in}%
\pgfsys@useobject{currentmarker}{}%
\end{pgfscope}%
\begin{pgfscope}%
\pgfsys@transformshift{2.283382in}{0.866486in}%
\pgfsys@useobject{currentmarker}{}%
\end{pgfscope}%
\begin{pgfscope}%
\pgfsys@transformshift{2.508405in}{1.219071in}%
\pgfsys@useobject{currentmarker}{}%
\end{pgfscope}%
\begin{pgfscope}%
\pgfsys@transformshift{2.733429in}{1.535650in}%
\pgfsys@useobject{currentmarker}{}%
\end{pgfscope}%
\begin{pgfscope}%
\pgfsys@transformshift{2.958452in}{1.852695in}%
\pgfsys@useobject{currentmarker}{}%
\end{pgfscope}%
\begin{pgfscope}%
\pgfsys@transformshift{3.183475in}{2.103243in}%
\pgfsys@useobject{currentmarker}{}%
\end{pgfscope}%
\begin{pgfscope}%
\pgfsys@transformshift{3.408498in}{2.465875in}%
\pgfsys@useobject{currentmarker}{}%
\end{pgfscope}%
\begin{pgfscope}%
\pgfsys@transformshift{3.633521in}{2.755438in}%
\pgfsys@useobject{currentmarker}{}%
\end{pgfscope}%
\begin{pgfscope}%
\pgfsys@transformshift{3.858545in}{3.018596in}%
\pgfsys@useobject{currentmarker}{}%
\end{pgfscope}%
\end{pgfscope}%
\begin{pgfscope}%
\pgfpathrectangle{\pgfqpoint{0.708220in}{0.535823in}}{\pgfqpoint{5.141780in}{2.669453in}}%
\pgfusepath{clip}%
\pgfsetrectcap%
\pgfsetroundjoin%
\pgfsetlinewidth{1.003750pt}%
\definecolor{currentstroke}{rgb}{0.917647,0.372549,0.580392}%
\pgfsetstrokecolor{currentstroke}%
\pgfsetdash{}{0pt}%
\pgfpathmoveto{\pgfqpoint{2.685237in}{0.525823in}}%
\pgfpathlineto{\pgfqpoint{2.733429in}{0.563447in}}%
\pgfpathlineto{\pgfqpoint{2.958452in}{0.736719in}}%
\pgfpathlineto{\pgfqpoint{3.183475in}{0.978123in}}%
\pgfpathlineto{\pgfqpoint{3.408498in}{1.305836in}}%
\pgfpathlineto{\pgfqpoint{3.633521in}{1.559575in}}%
\pgfpathlineto{\pgfqpoint{3.858545in}{1.866610in}}%
\pgfpathlineto{\pgfqpoint{4.083568in}{2.126813in}}%
\pgfpathlineto{\pgfqpoint{4.308591in}{2.390895in}}%
\pgfpathlineto{\pgfqpoint{4.533614in}{2.643174in}}%
\pgfpathlineto{\pgfqpoint{4.758637in}{3.041851in}}%
\pgfusepath{stroke}%
\end{pgfscope}%
\begin{pgfscope}%
\pgfpathrectangle{\pgfqpoint{0.708220in}{0.535823in}}{\pgfqpoint{5.141780in}{2.669453in}}%
\pgfusepath{clip}%
\pgfsetbuttcap%
\pgfsetmiterjoin%
\definecolor{currentfill}{rgb}{0.917647,0.372549,0.580392}%
\pgfsetfillcolor{currentfill}%
\pgfsetlinewidth{0.501875pt}%
\definecolor{currentstroke}{rgb}{0.000000,0.000000,0.000000}%
\pgfsetstrokecolor{currentstroke}%
\pgfsetdash{}{0pt}%
\pgfsys@defobject{currentmarker}{\pgfqpoint{-0.049105in}{-0.049105in}}{\pgfqpoint{0.049105in}{0.049105in}}{%
\pgfpathmoveto{\pgfqpoint{0.000000in}{-0.049105in}}%
\pgfpathlineto{\pgfqpoint{0.049105in}{0.000000in}}%
\pgfpathlineto{\pgfqpoint{0.000000in}{0.049105in}}%
\pgfpathlineto{\pgfqpoint{-0.049105in}{0.000000in}}%
\pgfpathclose%
\pgfusepath{stroke,fill}%
}%
\begin{pgfscope}%
\pgfsys@transformshift{1.833336in}{0.270252in}%
\pgfsys@useobject{currentmarker}{}%
\end{pgfscope}%
\begin{pgfscope}%
\pgfsys@transformshift{2.058359in}{0.270252in}%
\pgfsys@useobject{currentmarker}{}%
\end{pgfscope}%
\begin{pgfscope}%
\pgfsys@transformshift{2.283382in}{0.270252in}%
\pgfsys@useobject{currentmarker}{}%
\end{pgfscope}%
\begin{pgfscope}%
\pgfsys@transformshift{2.508405in}{0.387769in}%
\pgfsys@useobject{currentmarker}{}%
\end{pgfscope}%
\begin{pgfscope}%
\pgfsys@transformshift{2.733429in}{0.563447in}%
\pgfsys@useobject{currentmarker}{}%
\end{pgfscope}%
\begin{pgfscope}%
\pgfsys@transformshift{2.958452in}{0.736719in}%
\pgfsys@useobject{currentmarker}{}%
\end{pgfscope}%
\begin{pgfscope}%
\pgfsys@transformshift{3.183475in}{0.978123in}%
\pgfsys@useobject{currentmarker}{}%
\end{pgfscope}%
\begin{pgfscope}%
\pgfsys@transformshift{3.408498in}{1.305836in}%
\pgfsys@useobject{currentmarker}{}%
\end{pgfscope}%
\begin{pgfscope}%
\pgfsys@transformshift{3.633521in}{1.559575in}%
\pgfsys@useobject{currentmarker}{}%
\end{pgfscope}%
\begin{pgfscope}%
\pgfsys@transformshift{3.858545in}{1.866610in}%
\pgfsys@useobject{currentmarker}{}%
\end{pgfscope}%
\begin{pgfscope}%
\pgfsys@transformshift{4.083568in}{2.126813in}%
\pgfsys@useobject{currentmarker}{}%
\end{pgfscope}%
\begin{pgfscope}%
\pgfsys@transformshift{4.308591in}{2.390895in}%
\pgfsys@useobject{currentmarker}{}%
\end{pgfscope}%
\begin{pgfscope}%
\pgfsys@transformshift{4.533614in}{2.643174in}%
\pgfsys@useobject{currentmarker}{}%
\end{pgfscope}%
\begin{pgfscope}%
\pgfsys@transformshift{4.758637in}{3.041851in}%
\pgfsys@useobject{currentmarker}{}%
\end{pgfscope}%
\end{pgfscope}%
\begin{pgfscope}%
\pgfpathrectangle{\pgfqpoint{0.708220in}{0.535823in}}{\pgfqpoint{5.141780in}{2.669453in}}%
\pgfusepath{clip}%
\pgfsetrectcap%
\pgfsetroundjoin%
\pgfsetlinewidth{1.003750pt}%
\definecolor{currentstroke}{rgb}{0.529412,0.462745,0.384314}%
\pgfsetstrokecolor{currentstroke}%
\pgfsetdash{}{0pt}%
\pgfpathmoveto{\pgfqpoint{1.920185in}{0.525823in}}%
\pgfpathlineto{\pgfqpoint{2.058359in}{0.573473in}}%
\pgfpathlineto{\pgfqpoint{2.283382in}{0.647919in}}%
\pgfpathlineto{\pgfqpoint{2.508405in}{0.725868in}}%
\pgfpathlineto{\pgfqpoint{2.733429in}{0.798513in}}%
\pgfpathlineto{\pgfqpoint{2.958452in}{0.899271in}}%
\pgfpathlineto{\pgfqpoint{3.183475in}{1.039475in}}%
\pgfpathlineto{\pgfqpoint{3.408498in}{1.399329in}}%
\pgfpathlineto{\pgfqpoint{3.633521in}{1.819793in}}%
\pgfpathlineto{\pgfqpoint{3.858545in}{2.283030in}}%
\pgfpathlineto{\pgfqpoint{4.083568in}{2.802781in}}%
\pgfusepath{stroke}%
\end{pgfscope}%
\begin{pgfscope}%
\pgfpathrectangle{\pgfqpoint{0.708220in}{0.535823in}}{\pgfqpoint{5.141780in}{2.669453in}}%
\pgfusepath{clip}%
\pgfsetbuttcap%
\pgfsetmiterjoin%
\definecolor{currentfill}{rgb}{0.529412,0.462745,0.384314}%
\pgfsetfillcolor{currentfill}%
\pgfsetlinewidth{0.501875pt}%
\definecolor{currentstroke}{rgb}{0.000000,0.000000,0.000000}%
\pgfsetstrokecolor{currentstroke}%
\pgfsetdash{}{0pt}%
\pgfsys@defobject{currentmarker}{\pgfqpoint{-0.034722in}{-0.034722in}}{\pgfqpoint{0.034722in}{0.034722in}}{%
\pgfpathmoveto{\pgfqpoint{-0.000000in}{-0.034722in}}%
\pgfpathlineto{\pgfqpoint{0.034722in}{0.034722in}}%
\pgfpathlineto{\pgfqpoint{-0.034722in}{0.034722in}}%
\pgfpathclose%
\pgfusepath{stroke,fill}%
}%
\begin{pgfscope}%
\pgfsys@transformshift{1.833336in}{0.495872in}%
\pgfsys@useobject{currentmarker}{}%
\end{pgfscope}%
\begin{pgfscope}%
\pgfsys@transformshift{2.058359in}{0.573473in}%
\pgfsys@useobject{currentmarker}{}%
\end{pgfscope}%
\begin{pgfscope}%
\pgfsys@transformshift{2.283382in}{0.647919in}%
\pgfsys@useobject{currentmarker}{}%
\end{pgfscope}%
\begin{pgfscope}%
\pgfsys@transformshift{2.508405in}{0.725868in}%
\pgfsys@useobject{currentmarker}{}%
\end{pgfscope}%
\begin{pgfscope}%
\pgfsys@transformshift{2.733429in}{0.798513in}%
\pgfsys@useobject{currentmarker}{}%
\end{pgfscope}%
\begin{pgfscope}%
\pgfsys@transformshift{2.958452in}{0.899271in}%
\pgfsys@useobject{currentmarker}{}%
\end{pgfscope}%
\begin{pgfscope}%
\pgfsys@transformshift{3.183475in}{1.039475in}%
\pgfsys@useobject{currentmarker}{}%
\end{pgfscope}%
\begin{pgfscope}%
\pgfsys@transformshift{3.408498in}{1.399329in}%
\pgfsys@useobject{currentmarker}{}%
\end{pgfscope}%
\begin{pgfscope}%
\pgfsys@transformshift{3.633521in}{1.819793in}%
\pgfsys@useobject{currentmarker}{}%
\end{pgfscope}%
\begin{pgfscope}%
\pgfsys@transformshift{3.858545in}{2.283030in}%
\pgfsys@useobject{currentmarker}{}%
\end{pgfscope}%
\begin{pgfscope}%
\pgfsys@transformshift{4.083568in}{2.802781in}%
\pgfsys@useobject{currentmarker}{}%
\end{pgfscope}%
\end{pgfscope}%
\begin{pgfscope}%
\pgfpathrectangle{\pgfqpoint{0.708220in}{0.535823in}}{\pgfqpoint{5.141780in}{2.669453in}}%
\pgfusepath{clip}%
\pgfsetrectcap%
\pgfsetroundjoin%
\pgfsetlinewidth{1.003750pt}%
\definecolor{currentstroke}{rgb}{0.611765,0.568627,0.274510}%
\pgfsetstrokecolor{currentstroke}%
\pgfsetdash{}{0pt}%
\pgfpathmoveto{\pgfqpoint{1.833336in}{0.533110in}}%
\pgfpathlineto{\pgfqpoint{2.058359in}{0.600165in}}%
\pgfpathlineto{\pgfqpoint{2.283382in}{0.656378in}}%
\pgfpathlineto{\pgfqpoint{2.508405in}{0.708555in}}%
\pgfpathlineto{\pgfqpoint{2.733429in}{0.755478in}}%
\pgfpathlineto{\pgfqpoint{2.958452in}{0.800865in}}%
\pgfpathlineto{\pgfqpoint{3.183475in}{0.845169in}}%
\pgfpathlineto{\pgfqpoint{3.408498in}{0.901495in}}%
\pgfpathlineto{\pgfqpoint{3.633521in}{0.983805in}}%
\pgfpathlineto{\pgfqpoint{3.858545in}{1.134783in}}%
\pgfpathlineto{\pgfqpoint{4.083568in}{1.319608in}}%
\pgfpathlineto{\pgfqpoint{4.308591in}{1.587480in}}%
\pgfpathlineto{\pgfqpoint{4.533614in}{1.922967in}}%
\pgfpathlineto{\pgfqpoint{4.758637in}{2.377229in}}%
\pgfpathlineto{\pgfqpoint{4.983661in}{2.606679in}}%
\pgfpathlineto{\pgfqpoint{5.208684in}{3.081033in}}%
\pgfusepath{stroke}%
\end{pgfscope}%
\begin{pgfscope}%
\pgfpathrectangle{\pgfqpoint{0.708220in}{0.535823in}}{\pgfqpoint{5.141780in}{2.669453in}}%
\pgfusepath{clip}%
\pgfsetbuttcap%
\pgfsetmiterjoin%
\definecolor{currentfill}{rgb}{0.611765,0.568627,0.274510}%
\pgfsetfillcolor{currentfill}%
\pgfsetlinewidth{0.501875pt}%
\definecolor{currentstroke}{rgb}{0.000000,0.000000,0.000000}%
\pgfsetstrokecolor{currentstroke}%
\pgfsetdash{}{0pt}%
\pgfsys@defobject{currentmarker}{\pgfqpoint{-0.034722in}{-0.034722in}}{\pgfqpoint{0.034722in}{0.034722in}}{%
\pgfpathmoveto{\pgfqpoint{-0.034722in}{0.000000in}}%
\pgfpathlineto{\pgfqpoint{0.034722in}{-0.034722in}}%
\pgfpathlineto{\pgfqpoint{0.034722in}{0.034722in}}%
\pgfpathclose%
\pgfusepath{stroke,fill}%
}%
\begin{pgfscope}%
\pgfsys@transformshift{1.833336in}{0.533110in}%
\pgfsys@useobject{currentmarker}{}%
\end{pgfscope}%
\begin{pgfscope}%
\pgfsys@transformshift{2.058359in}{0.600165in}%
\pgfsys@useobject{currentmarker}{}%
\end{pgfscope}%
\begin{pgfscope}%
\pgfsys@transformshift{2.283382in}{0.656378in}%
\pgfsys@useobject{currentmarker}{}%
\end{pgfscope}%
\begin{pgfscope}%
\pgfsys@transformshift{2.508405in}{0.708555in}%
\pgfsys@useobject{currentmarker}{}%
\end{pgfscope}%
\begin{pgfscope}%
\pgfsys@transformshift{2.733429in}{0.755478in}%
\pgfsys@useobject{currentmarker}{}%
\end{pgfscope}%
\begin{pgfscope}%
\pgfsys@transformshift{2.958452in}{0.800865in}%
\pgfsys@useobject{currentmarker}{}%
\end{pgfscope}%
\begin{pgfscope}%
\pgfsys@transformshift{3.183475in}{0.845169in}%
\pgfsys@useobject{currentmarker}{}%
\end{pgfscope}%
\begin{pgfscope}%
\pgfsys@transformshift{3.408498in}{0.901495in}%
\pgfsys@useobject{currentmarker}{}%
\end{pgfscope}%
\begin{pgfscope}%
\pgfsys@transformshift{3.633521in}{0.983805in}%
\pgfsys@useobject{currentmarker}{}%
\end{pgfscope}%
\begin{pgfscope}%
\pgfsys@transformshift{3.858545in}{1.134783in}%
\pgfsys@useobject{currentmarker}{}%
\end{pgfscope}%
\begin{pgfscope}%
\pgfsys@transformshift{4.083568in}{1.319608in}%
\pgfsys@useobject{currentmarker}{}%
\end{pgfscope}%
\begin{pgfscope}%
\pgfsys@transformshift{4.308591in}{1.587480in}%
\pgfsys@useobject{currentmarker}{}%
\end{pgfscope}%
\begin{pgfscope}%
\pgfsys@transformshift{4.533614in}{1.922967in}%
\pgfsys@useobject{currentmarker}{}%
\end{pgfscope}%
\begin{pgfscope}%
\pgfsys@transformshift{4.758637in}{2.377229in}%
\pgfsys@useobject{currentmarker}{}%
\end{pgfscope}%
\begin{pgfscope}%
\pgfsys@transformshift{4.983661in}{2.606679in}%
\pgfsys@useobject{currentmarker}{}%
\end{pgfscope}%
\begin{pgfscope}%
\pgfsys@transformshift{5.208684in}{3.081033in}%
\pgfsys@useobject{currentmarker}{}%
\end{pgfscope}%
\end{pgfscope}%
\begin{pgfscope}%
\pgfpathrectangle{\pgfqpoint{0.708220in}{0.535823in}}{\pgfqpoint{5.141780in}{2.669453in}}%
\pgfusepath{clip}%
\pgfsetrectcap%
\pgfsetroundjoin%
\pgfsetlinewidth{1.003750pt}%
\definecolor{currentstroke}{rgb}{0.780392,0.643137,0.254902}%
\pgfsetstrokecolor{currentstroke}%
\pgfsetdash{}{0pt}%
\pgfpathmoveto{\pgfqpoint{1.833336in}{0.607768in}}%
\pgfpathlineto{\pgfqpoint{2.058359in}{0.714594in}}%
\pgfpathlineto{\pgfqpoint{2.283382in}{0.810268in}}%
\pgfpathlineto{\pgfqpoint{2.508405in}{0.899028in}}%
\pgfpathlineto{\pgfqpoint{2.733429in}{0.976962in}}%
\pgfpathlineto{\pgfqpoint{2.958452in}{1.048705in}}%
\pgfpathlineto{\pgfqpoint{3.183475in}{1.117476in}}%
\pgfpathlineto{\pgfqpoint{3.408498in}{1.184622in}}%
\pgfpathlineto{\pgfqpoint{3.633521in}{1.247489in}}%
\pgfpathlineto{\pgfqpoint{3.858545in}{1.333602in}}%
\pgfpathlineto{\pgfqpoint{4.083568in}{1.407839in}}%
\pgfpathlineto{\pgfqpoint{4.308591in}{1.554250in}}%
\pgfpathlineto{\pgfqpoint{4.533614in}{1.754038in}}%
\pgfpathlineto{\pgfqpoint{4.758637in}{2.105422in}}%
\pgfpathlineto{\pgfqpoint{4.983661in}{2.186292in}}%
\pgfpathlineto{\pgfqpoint{5.208684in}{2.855359in}}%
\pgfusepath{stroke}%
\end{pgfscope}%
\begin{pgfscope}%
\pgfpathrectangle{\pgfqpoint{0.708220in}{0.535823in}}{\pgfqpoint{5.141780in}{2.669453in}}%
\pgfusepath{clip}%
\pgfsetbuttcap%
\pgfsetmiterjoin%
\definecolor{currentfill}{rgb}{0.780392,0.643137,0.254902}%
\pgfsetfillcolor{currentfill}%
\pgfsetlinewidth{0.501875pt}%
\definecolor{currentstroke}{rgb}{0.000000,0.000000,0.000000}%
\pgfsetstrokecolor{currentstroke}%
\pgfsetdash{}{0pt}%
\pgfsys@defobject{currentmarker}{\pgfqpoint{-0.034722in}{-0.034722in}}{\pgfqpoint{0.034722in}{0.034722in}}{%
\pgfpathmoveto{\pgfqpoint{0.034722in}{-0.000000in}}%
\pgfpathlineto{\pgfqpoint{-0.034722in}{0.034722in}}%
\pgfpathlineto{\pgfqpoint{-0.034722in}{-0.034722in}}%
\pgfpathclose%
\pgfusepath{stroke,fill}%
}%
\begin{pgfscope}%
\pgfsys@transformshift{1.833336in}{0.607768in}%
\pgfsys@useobject{currentmarker}{}%
\end{pgfscope}%
\begin{pgfscope}%
\pgfsys@transformshift{2.058359in}{0.714594in}%
\pgfsys@useobject{currentmarker}{}%
\end{pgfscope}%
\begin{pgfscope}%
\pgfsys@transformshift{2.283382in}{0.810268in}%
\pgfsys@useobject{currentmarker}{}%
\end{pgfscope}%
\begin{pgfscope}%
\pgfsys@transformshift{2.508405in}{0.899028in}%
\pgfsys@useobject{currentmarker}{}%
\end{pgfscope}%
\begin{pgfscope}%
\pgfsys@transformshift{2.733429in}{0.976962in}%
\pgfsys@useobject{currentmarker}{}%
\end{pgfscope}%
\begin{pgfscope}%
\pgfsys@transformshift{2.958452in}{1.048705in}%
\pgfsys@useobject{currentmarker}{}%
\end{pgfscope}%
\begin{pgfscope}%
\pgfsys@transformshift{3.183475in}{1.117476in}%
\pgfsys@useobject{currentmarker}{}%
\end{pgfscope}%
\begin{pgfscope}%
\pgfsys@transformshift{3.408498in}{1.184622in}%
\pgfsys@useobject{currentmarker}{}%
\end{pgfscope}%
\begin{pgfscope}%
\pgfsys@transformshift{3.633521in}{1.247489in}%
\pgfsys@useobject{currentmarker}{}%
\end{pgfscope}%
\begin{pgfscope}%
\pgfsys@transformshift{3.858545in}{1.333602in}%
\pgfsys@useobject{currentmarker}{}%
\end{pgfscope}%
\begin{pgfscope}%
\pgfsys@transformshift{4.083568in}{1.407839in}%
\pgfsys@useobject{currentmarker}{}%
\end{pgfscope}%
\begin{pgfscope}%
\pgfsys@transformshift{4.308591in}{1.554250in}%
\pgfsys@useobject{currentmarker}{}%
\end{pgfscope}%
\begin{pgfscope}%
\pgfsys@transformshift{4.533614in}{1.754038in}%
\pgfsys@useobject{currentmarker}{}%
\end{pgfscope}%
\begin{pgfscope}%
\pgfsys@transformshift{4.758637in}{2.105422in}%
\pgfsys@useobject{currentmarker}{}%
\end{pgfscope}%
\begin{pgfscope}%
\pgfsys@transformshift{4.983661in}{2.186292in}%
\pgfsys@useobject{currentmarker}{}%
\end{pgfscope}%
\begin{pgfscope}%
\pgfsys@transformshift{5.208684in}{2.855359in}%
\pgfsys@useobject{currentmarker}{}%
\end{pgfscope}%
\end{pgfscope}%
\begin{pgfscope}%
\pgfpathrectangle{\pgfqpoint{0.708220in}{0.535823in}}{\pgfqpoint{5.141780in}{2.669453in}}%
\pgfusepath{clip}%
\pgfsetrectcap%
\pgfsetroundjoin%
\pgfsetlinewidth{1.003750pt}%
\definecolor{currentstroke}{rgb}{1.000000,0.694118,0.305882}%
\pgfsetstrokecolor{currentstroke}%
\pgfsetdash{}{0pt}%
\pgfpathmoveto{\pgfqpoint{2.423238in}{0.525823in}}%
\pgfpathlineto{\pgfqpoint{2.508405in}{0.555088in}}%
\pgfpathlineto{\pgfqpoint{2.733429in}{0.713199in}}%
\pgfpathlineto{\pgfqpoint{2.958452in}{0.872472in}}%
\pgfpathlineto{\pgfqpoint{3.183475in}{1.125431in}}%
\pgfpathlineto{\pgfqpoint{3.408498in}{1.272615in}}%
\pgfpathlineto{\pgfqpoint{3.633521in}{1.392305in}}%
\pgfpathlineto{\pgfqpoint{3.858545in}{1.478523in}}%
\pgfpathlineto{\pgfqpoint{4.083568in}{1.557487in}}%
\pgfpathlineto{\pgfqpoint{4.308591in}{1.662572in}}%
\pgfpathlineto{\pgfqpoint{4.533614in}{1.734776in}}%
\pgfpathlineto{\pgfqpoint{4.758637in}{1.924610in}}%
\pgfpathlineto{\pgfqpoint{4.983661in}{2.079869in}}%
\pgfpathlineto{\pgfqpoint{5.208684in}{2.525204in}}%
\pgfpathlineto{\pgfqpoint{5.433707in}{2.672837in}}%
\pgfpathlineto{\pgfqpoint{5.658730in}{3.144706in}}%
\pgfusepath{stroke}%
\end{pgfscope}%
\begin{pgfscope}%
\pgfpathrectangle{\pgfqpoint{0.708220in}{0.535823in}}{\pgfqpoint{5.141780in}{2.669453in}}%
\pgfusepath{clip}%
\pgfsetbuttcap%
\pgfsetbeveljoin%
\definecolor{currentfill}{rgb}{1.000000,0.694118,0.305882}%
\pgfsetfillcolor{currentfill}%
\pgfsetlinewidth{0.501875pt}%
\definecolor{currentstroke}{rgb}{0.000000,0.000000,0.000000}%
\pgfsetstrokecolor{currentstroke}%
\pgfsetdash{}{0pt}%
\pgfsys@defobject{currentmarker}{\pgfqpoint{-0.033023in}{-0.028091in}}{\pgfqpoint{0.033023in}{0.034722in}}{%
\pgfpathmoveto{\pgfqpoint{0.000000in}{0.034722in}}%
\pgfpathlineto{\pgfqpoint{-0.007796in}{0.010730in}}%
\pgfpathlineto{\pgfqpoint{-0.033023in}{0.010730in}}%
\pgfpathlineto{\pgfqpoint{-0.012614in}{-0.004098in}}%
\pgfpathlineto{\pgfqpoint{-0.020409in}{-0.028091in}}%
\pgfpathlineto{\pgfqpoint{-0.000000in}{-0.013263in}}%
\pgfpathlineto{\pgfqpoint{0.020409in}{-0.028091in}}%
\pgfpathlineto{\pgfqpoint{0.012614in}{-0.004098in}}%
\pgfpathlineto{\pgfqpoint{0.033023in}{0.010730in}}%
\pgfpathlineto{\pgfqpoint{0.007796in}{0.010730in}}%
\pgfpathclose%
\pgfusepath{stroke,fill}%
}%
\begin{pgfscope}%
\pgfsys@transformshift{1.833336in}{0.359278in}%
\pgfsys@useobject{currentmarker}{}%
\end{pgfscope}%
\begin{pgfscope}%
\pgfsys@transformshift{2.058359in}{0.431153in}%
\pgfsys@useobject{currentmarker}{}%
\end{pgfscope}%
\begin{pgfscope}%
\pgfsys@transformshift{2.283382in}{0.477766in}%
\pgfsys@useobject{currentmarker}{}%
\end{pgfscope}%
\begin{pgfscope}%
\pgfsys@transformshift{2.508405in}{0.555088in}%
\pgfsys@useobject{currentmarker}{}%
\end{pgfscope}%
\begin{pgfscope}%
\pgfsys@transformshift{2.733429in}{0.713199in}%
\pgfsys@useobject{currentmarker}{}%
\end{pgfscope}%
\begin{pgfscope}%
\pgfsys@transformshift{2.958452in}{0.872472in}%
\pgfsys@useobject{currentmarker}{}%
\end{pgfscope}%
\begin{pgfscope}%
\pgfsys@transformshift{3.183475in}{1.125431in}%
\pgfsys@useobject{currentmarker}{}%
\end{pgfscope}%
\begin{pgfscope}%
\pgfsys@transformshift{3.408498in}{1.272615in}%
\pgfsys@useobject{currentmarker}{}%
\end{pgfscope}%
\begin{pgfscope}%
\pgfsys@transformshift{3.633521in}{1.392305in}%
\pgfsys@useobject{currentmarker}{}%
\end{pgfscope}%
\begin{pgfscope}%
\pgfsys@transformshift{3.858545in}{1.478523in}%
\pgfsys@useobject{currentmarker}{}%
\end{pgfscope}%
\begin{pgfscope}%
\pgfsys@transformshift{4.083568in}{1.557487in}%
\pgfsys@useobject{currentmarker}{}%
\end{pgfscope}%
\begin{pgfscope}%
\pgfsys@transformshift{4.308591in}{1.662572in}%
\pgfsys@useobject{currentmarker}{}%
\end{pgfscope}%
\begin{pgfscope}%
\pgfsys@transformshift{4.533614in}{1.734776in}%
\pgfsys@useobject{currentmarker}{}%
\end{pgfscope}%
\begin{pgfscope}%
\pgfsys@transformshift{4.758637in}{1.924610in}%
\pgfsys@useobject{currentmarker}{}%
\end{pgfscope}%
\begin{pgfscope}%
\pgfsys@transformshift{4.983661in}{2.079869in}%
\pgfsys@useobject{currentmarker}{}%
\end{pgfscope}%
\begin{pgfscope}%
\pgfsys@transformshift{5.208684in}{2.525204in}%
\pgfsys@useobject{currentmarker}{}%
\end{pgfscope}%
\begin{pgfscope}%
\pgfsys@transformshift{5.433707in}{2.672837in}%
\pgfsys@useobject{currentmarker}{}%
\end{pgfscope}%
\begin{pgfscope}%
\pgfsys@transformshift{5.658730in}{3.144706in}%
\pgfsys@useobject{currentmarker}{}%
\end{pgfscope}%
\end{pgfscope}%
\begin{pgfscope}%
\pgfsetrectcap%
\pgfsetmiterjoin%
\pgfsetlinewidth{0.803000pt}%
\definecolor{currentstroke}{rgb}{0.000000,0.000000,0.000000}%
\pgfsetstrokecolor{currentstroke}%
\pgfsetdash{}{0pt}%
\pgfpathmoveto{\pgfqpoint{0.708220in}{0.535823in}}%
\pgfpathlineto{\pgfqpoint{0.708220in}{3.205275in}}%
\pgfusepath{stroke}%
\end{pgfscope}%
\begin{pgfscope}%
\pgfsetrectcap%
\pgfsetmiterjoin%
\pgfsetlinewidth{0.803000pt}%
\definecolor{currentstroke}{rgb}{0.000000,0.000000,0.000000}%
\pgfsetstrokecolor{currentstroke}%
\pgfsetdash{}{0pt}%
\pgfpathmoveto{\pgfqpoint{5.850000in}{0.535823in}}%
\pgfpathlineto{\pgfqpoint{5.850000in}{3.205275in}}%
\pgfusepath{stroke}%
\end{pgfscope}%
\begin{pgfscope}%
\pgfsetrectcap%
\pgfsetmiterjoin%
\pgfsetlinewidth{0.803000pt}%
\definecolor{currentstroke}{rgb}{0.000000,0.000000,0.000000}%
\pgfsetstrokecolor{currentstroke}%
\pgfsetdash{}{0pt}%
\pgfpathmoveto{\pgfqpoint{0.708220in}{0.535823in}}%
\pgfpathlineto{\pgfqpoint{5.850000in}{0.535823in}}%
\pgfusepath{stroke}%
\end{pgfscope}%
\begin{pgfscope}%
\pgfsetrectcap%
\pgfsetmiterjoin%
\pgfsetlinewidth{0.803000pt}%
\definecolor{currentstroke}{rgb}{0.000000,0.000000,0.000000}%
\pgfsetstrokecolor{currentstroke}%
\pgfsetdash{}{0pt}%
\pgfpathmoveto{\pgfqpoint{0.708220in}{3.205275in}}%
\pgfpathlineto{\pgfqpoint{5.850000in}{3.205275in}}%
\pgfusepath{stroke}%
\end{pgfscope}%
\begin{pgfscope}%
\pgfsetrectcap%
\pgfsetroundjoin%
\pgfsetlinewidth{1.003750pt}%
\definecolor{currentstroke}{rgb}{0.866667,0.058824,0.058824}%
\pgfsetstrokecolor{currentstroke}%
\pgfsetdash{}{0pt}%
\pgfpathmoveto{\pgfqpoint{0.758220in}{3.111525in}}%
\pgfpathlineto{\pgfqpoint{1.008220in}{3.111525in}}%
\pgfusepath{stroke}%
\end{pgfscope}%
\begin{pgfscope}%
\pgfsetbuttcap%
\pgfsetmiterjoin%
\definecolor{currentfill}{rgb}{0.866667,0.058824,0.058824}%
\pgfsetfillcolor{currentfill}%
\pgfsetlinewidth{0.501875pt}%
\definecolor{currentstroke}{rgb}{0.000000,0.000000,0.000000}%
\pgfsetstrokecolor{currentstroke}%
\pgfsetdash{}{0pt}%
\pgfsys@defobject{currentmarker}{\pgfqpoint{-0.033023in}{-0.028091in}}{\pgfqpoint{0.033023in}{0.034722in}}{%
\pgfpathmoveto{\pgfqpoint{0.000000in}{0.034722in}}%
\pgfpathlineto{\pgfqpoint{-0.033023in}{0.010730in}}%
\pgfpathlineto{\pgfqpoint{-0.020409in}{-0.028091in}}%
\pgfpathlineto{\pgfqpoint{0.020409in}{-0.028091in}}%
\pgfpathlineto{\pgfqpoint{0.033023in}{0.010730in}}%
\pgfpathclose%
\pgfusepath{stroke,fill}%
}%
\begin{pgfscope}%
\pgfsys@transformshift{0.883220in}{3.111525in}%
\pgfsys@useobject{currentmarker}{}%
\end{pgfscope}%
\end{pgfscope}%
\begin{pgfscope}%
\definecolor{textcolor}{rgb}{0.000000,0.000000,0.000000}%
\pgfsetstrokecolor{textcolor}%
\pgfsetfillcolor{textcolor}%
\pgftext[x=1.033220in,y=3.067775in,left,base]{\color{textcolor}\rmfamily\fontsize{9.000000}{10.800000}\selectfont cachet}%
\end{pgfscope}%
\begin{pgfscope}%
\pgfsetrectcap%
\pgfsetroundjoin%
\pgfsetlinewidth{1.003750pt}%
\definecolor{currentstroke}{rgb}{0.000000,0.000000,0.200000}%
\pgfsetstrokecolor{currentstroke}%
\pgfsetdash{}{0pt}%
\pgfpathmoveto{\pgfqpoint{0.758220in}{2.949726in}}%
\pgfpathlineto{\pgfqpoint{1.008220in}{2.949726in}}%
\pgfusepath{stroke}%
\end{pgfscope}%
\begin{pgfscope}%
\pgfsetbuttcap%
\pgfsetmiterjoin%
\definecolor{currentfill}{rgb}{0.000000,0.000000,0.200000}%
\pgfsetfillcolor{currentfill}%
\pgfsetlinewidth{0.501875pt}%
\definecolor{currentstroke}{rgb}{0.000000,0.000000,0.000000}%
\pgfsetstrokecolor{currentstroke}%
\pgfsetdash{}{0pt}%
\pgfsys@defobject{currentmarker}{\pgfqpoint{-0.034722in}{-0.034722in}}{\pgfqpoint{0.034722in}{0.034722in}}{%
\pgfpathmoveto{\pgfqpoint{-0.011574in}{-0.034722in}}%
\pgfpathlineto{\pgfqpoint{0.011574in}{-0.034722in}}%
\pgfpathlineto{\pgfqpoint{0.011574in}{-0.011574in}}%
\pgfpathlineto{\pgfqpoint{0.034722in}{-0.011574in}}%
\pgfpathlineto{\pgfqpoint{0.034722in}{0.011574in}}%
\pgfpathlineto{\pgfqpoint{0.011574in}{0.011574in}}%
\pgfpathlineto{\pgfqpoint{0.011574in}{0.034722in}}%
\pgfpathlineto{\pgfqpoint{-0.011574in}{0.034722in}}%
\pgfpathlineto{\pgfqpoint{-0.011574in}{0.011574in}}%
\pgfpathlineto{\pgfqpoint{-0.034722in}{0.011574in}}%
\pgfpathlineto{\pgfqpoint{-0.034722in}{-0.011574in}}%
\pgfpathlineto{\pgfqpoint{-0.011574in}{-0.011574in}}%
\pgfpathclose%
\pgfusepath{stroke,fill}%
}%
\begin{pgfscope}%
\pgfsys@transformshift{0.883220in}{2.949726in}%
\pgfsys@useobject{currentmarker}{}%
\end{pgfscope}%
\end{pgfscope}%
\begin{pgfscope}%
\definecolor{textcolor}{rgb}{0.000000,0.000000,0.000000}%
\pgfsetstrokecolor{textcolor}%
\pgfsetfillcolor{textcolor}%
\pgftext[x=1.033220in,y=2.905976in,left,base]{\color{textcolor}\rmfamily\fontsize{9.000000}{10.800000}\selectfont dynQBF}%
\end{pgfscope}%
\begin{pgfscope}%
\pgfsetrectcap%
\pgfsetroundjoin%
\pgfsetlinewidth{1.003750pt}%
\definecolor{currentstroke}{rgb}{0.000000,0.000000,0.866667}%
\pgfsetstrokecolor{currentstroke}%
\pgfsetdash{}{0pt}%
\pgfpathmoveto{\pgfqpoint{0.758220in}{2.787926in}}%
\pgfpathlineto{\pgfqpoint{1.008220in}{2.787926in}}%
\pgfusepath{stroke}%
\end{pgfscope}%
\begin{pgfscope}%
\pgfsetbuttcap%
\pgfsetmiterjoin%
\definecolor{currentfill}{rgb}{0.000000,0.000000,0.866667}%
\pgfsetfillcolor{currentfill}%
\pgfsetlinewidth{0.501875pt}%
\definecolor{currentstroke}{rgb}{0.000000,0.000000,0.000000}%
\pgfsetstrokecolor{currentstroke}%
\pgfsetdash{}{0pt}%
\pgfsys@defobject{currentmarker}{\pgfqpoint{-0.029463in}{-0.049105in}}{\pgfqpoint{0.029463in}{0.049105in}}{%
\pgfpathmoveto{\pgfqpoint{0.000000in}{-0.049105in}}%
\pgfpathlineto{\pgfqpoint{0.029463in}{0.000000in}}%
\pgfpathlineto{\pgfqpoint{0.000000in}{0.049105in}}%
\pgfpathlineto{\pgfqpoint{-0.029463in}{0.000000in}}%
\pgfpathclose%
\pgfusepath{stroke,fill}%
}%
\begin{pgfscope}%
\pgfsys@transformshift{0.883220in}{2.787926in}%
\pgfsys@useobject{currentmarker}{}%
\end{pgfscope}%
\end{pgfscope}%
\begin{pgfscope}%
\definecolor{textcolor}{rgb}{0.000000,0.000000,0.000000}%
\pgfsetstrokecolor{textcolor}%
\pgfsetfillcolor{textcolor}%
\pgftext[x=1.033220in,y=2.744176in,left,base]{\color{textcolor}\rmfamily\fontsize{9.000000}{10.800000}\selectfont dynasp}%
\end{pgfscope}%
\begin{pgfscope}%
\pgfsetrectcap%
\pgfsetroundjoin%
\pgfsetlinewidth{1.003750pt}%
\definecolor{currentstroke}{rgb}{0.250980,0.231373,0.796078}%
\pgfsetstrokecolor{currentstroke}%
\pgfsetdash{}{0pt}%
\pgfpathmoveto{\pgfqpoint{0.758220in}{2.626126in}}%
\pgfpathlineto{\pgfqpoint{1.008220in}{2.626126in}}%
\pgfusepath{stroke}%
\end{pgfscope}%
\begin{pgfscope}%
\pgfsetbuttcap%
\pgfsetmiterjoin%
\definecolor{currentfill}{rgb}{0.250980,0.231373,0.796078}%
\pgfsetfillcolor{currentfill}%
\pgfsetlinewidth{0.501875pt}%
\definecolor{currentstroke}{rgb}{0.000000,0.000000,0.000000}%
\pgfsetstrokecolor{currentstroke}%
\pgfsetdash{}{0pt}%
\pgfsys@defobject{currentmarker}{\pgfqpoint{-0.034722in}{-0.034722in}}{\pgfqpoint{0.034722in}{0.034722in}}{%
\pgfpathmoveto{\pgfqpoint{-0.034722in}{-0.034722in}}%
\pgfpathlineto{\pgfqpoint{0.034722in}{-0.034722in}}%
\pgfpathlineto{\pgfqpoint{0.034722in}{0.034722in}}%
\pgfpathlineto{\pgfqpoint{-0.034722in}{0.034722in}}%
\pgfpathclose%
\pgfusepath{stroke,fill}%
}%
\begin{pgfscope}%
\pgfsys@transformshift{0.883220in}{2.626126in}%
\pgfsys@useobject{currentmarker}{}%
\end{pgfscope}%
\end{pgfscope}%
\begin{pgfscope}%
\definecolor{textcolor}{rgb}{0.000000,0.000000,0.000000}%
\pgfsetstrokecolor{textcolor}%
\pgfsetfillcolor{textcolor}%
\pgftext[x=1.033220in,y=2.582376in,left,base]{\color{textcolor}\rmfamily\fontsize{9.000000}{10.800000}\selectfont sharpSAT}%
\end{pgfscope}%
\begin{pgfscope}%
\pgfsetrectcap%
\pgfsetroundjoin%
\pgfsetlinewidth{1.003750pt}%
\definecolor{currentstroke}{rgb}{0.615686,0.007843,0.843137}%
\pgfsetstrokecolor{currentstroke}%
\pgfsetdash{}{0pt}%
\pgfpathmoveto{\pgfqpoint{0.758220in}{2.464327in}}%
\pgfpathlineto{\pgfqpoint{1.008220in}{2.464327in}}%
\pgfusepath{stroke}%
\end{pgfscope}%
\begin{pgfscope}%
\pgfsetbuttcap%
\pgfsetroundjoin%
\definecolor{currentfill}{rgb}{0.615686,0.007843,0.843137}%
\pgfsetfillcolor{currentfill}%
\pgfsetlinewidth{0.501875pt}%
\definecolor{currentstroke}{rgb}{0.000000,0.000000,0.000000}%
\pgfsetstrokecolor{currentstroke}%
\pgfsetdash{}{0pt}%
\pgfsys@defobject{currentmarker}{\pgfqpoint{-0.034722in}{-0.034722in}}{\pgfqpoint{0.034722in}{0.034722in}}{%
\pgfpathmoveto{\pgfqpoint{0.000000in}{-0.034722in}}%
\pgfpathcurveto{\pgfqpoint{0.009208in}{-0.034722in}}{\pgfqpoint{0.018041in}{-0.031064in}}{\pgfqpoint{0.024552in}{-0.024552in}}%
\pgfpathcurveto{\pgfqpoint{0.031064in}{-0.018041in}}{\pgfqpoint{0.034722in}{-0.009208in}}{\pgfqpoint{0.034722in}{0.000000in}}%
\pgfpathcurveto{\pgfqpoint{0.034722in}{0.009208in}}{\pgfqpoint{0.031064in}{0.018041in}}{\pgfqpoint{0.024552in}{0.024552in}}%
\pgfpathcurveto{\pgfqpoint{0.018041in}{0.031064in}}{\pgfqpoint{0.009208in}{0.034722in}}{\pgfqpoint{0.000000in}{0.034722in}}%
\pgfpathcurveto{\pgfqpoint{-0.009208in}{0.034722in}}{\pgfqpoint{-0.018041in}{0.031064in}}{\pgfqpoint{-0.024552in}{0.024552in}}%
\pgfpathcurveto{\pgfqpoint{-0.031064in}{0.018041in}}{\pgfqpoint{-0.034722in}{0.009208in}}{\pgfqpoint{-0.034722in}{0.000000in}}%
\pgfpathcurveto{\pgfqpoint{-0.034722in}{-0.009208in}}{\pgfqpoint{-0.031064in}{-0.018041in}}{\pgfqpoint{-0.024552in}{-0.024552in}}%
\pgfpathcurveto{\pgfqpoint{-0.018041in}{-0.031064in}}{\pgfqpoint{-0.009208in}{-0.034722in}}{\pgfqpoint{0.000000in}{-0.034722in}}%
\pgfpathclose%
\pgfusepath{stroke,fill}%
}%
\begin{pgfscope}%
\pgfsys@transformshift{0.883220in}{2.464327in}%
\pgfsys@useobject{currentmarker}{}%
\end{pgfscope}%
\end{pgfscope}%
\begin{pgfscope}%
\definecolor{textcolor}{rgb}{0.000000,0.000000,0.000000}%
\pgfsetstrokecolor{textcolor}%
\pgfsetfillcolor{textcolor}%
\pgftext[x=1.033220in,y=2.420577in,left,base]{\color{textcolor}\rmfamily\fontsize{9.000000}{10.800000}\selectfont d4}%
\end{pgfscope}%
\begin{pgfscope}%
\pgfsetrectcap%
\pgfsetroundjoin%
\pgfsetlinewidth{1.003750pt}%
\definecolor{currentstroke}{rgb}{0.917647,0.372549,0.580392}%
\pgfsetstrokecolor{currentstroke}%
\pgfsetdash{}{0pt}%
\pgfpathmoveto{\pgfqpoint{0.758220in}{2.302527in}}%
\pgfpathlineto{\pgfqpoint{1.008220in}{2.302527in}}%
\pgfusepath{stroke}%
\end{pgfscope}%
\begin{pgfscope}%
\pgfsetbuttcap%
\pgfsetmiterjoin%
\definecolor{currentfill}{rgb}{0.917647,0.372549,0.580392}%
\pgfsetfillcolor{currentfill}%
\pgfsetlinewidth{0.501875pt}%
\definecolor{currentstroke}{rgb}{0.000000,0.000000,0.000000}%
\pgfsetstrokecolor{currentstroke}%
\pgfsetdash{}{0pt}%
\pgfsys@defobject{currentmarker}{\pgfqpoint{-0.049105in}{-0.049105in}}{\pgfqpoint{0.049105in}{0.049105in}}{%
\pgfpathmoveto{\pgfqpoint{0.000000in}{-0.049105in}}%
\pgfpathlineto{\pgfqpoint{0.049105in}{0.000000in}}%
\pgfpathlineto{\pgfqpoint{0.000000in}{0.049105in}}%
\pgfpathlineto{\pgfqpoint{-0.049105in}{0.000000in}}%
\pgfpathclose%
\pgfusepath{stroke,fill}%
}%
\begin{pgfscope}%
\pgfsys@transformshift{0.883220in}{2.302527in}%
\pgfsys@useobject{currentmarker}{}%
\end{pgfscope}%
\end{pgfscope}%
\begin{pgfscope}%
\definecolor{textcolor}{rgb}{0.000000,0.000000,0.000000}%
\pgfsetstrokecolor{textcolor}%
\pgfsetfillcolor{textcolor}%
\pgftext[x=1.033220in,y=2.258777in,left,base]{\color{textcolor}\rmfamily\fontsize{9.000000}{10.800000}\selectfont miniC2D}%
\end{pgfscope}%
\begin{pgfscope}%
\pgfsetrectcap%
\pgfsetroundjoin%
\pgfsetlinewidth{1.003750pt}%
\definecolor{currentstroke}{rgb}{0.529412,0.462745,0.384314}%
\pgfsetstrokecolor{currentstroke}%
\pgfsetdash{}{0pt}%
\pgfpathmoveto{\pgfqpoint{0.758220in}{2.140728in}}%
\pgfpathlineto{\pgfqpoint{1.008220in}{2.140728in}}%
\pgfusepath{stroke}%
\end{pgfscope}%
\begin{pgfscope}%
\pgfsetbuttcap%
\pgfsetmiterjoin%
\definecolor{currentfill}{rgb}{0.529412,0.462745,0.384314}%
\pgfsetfillcolor{currentfill}%
\pgfsetlinewidth{0.501875pt}%
\definecolor{currentstroke}{rgb}{0.000000,0.000000,0.000000}%
\pgfsetstrokecolor{currentstroke}%
\pgfsetdash{}{0pt}%
\pgfsys@defobject{currentmarker}{\pgfqpoint{-0.034722in}{-0.034722in}}{\pgfqpoint{0.034722in}{0.034722in}}{%
\pgfpathmoveto{\pgfqpoint{-0.000000in}{-0.034722in}}%
\pgfpathlineto{\pgfqpoint{0.034722in}{0.034722in}}%
\pgfpathlineto{\pgfqpoint{-0.034722in}{0.034722in}}%
\pgfpathclose%
\pgfusepath{stroke,fill}%
}%
\begin{pgfscope}%
\pgfsys@transformshift{0.883220in}{2.140728in}%
\pgfsys@useobject{currentmarker}{}%
\end{pgfscope}%
\end{pgfscope}%
\begin{pgfscope}%
\definecolor{textcolor}{rgb}{0.000000,0.000000,0.000000}%
\pgfsetstrokecolor{textcolor}%
\pgfsetfillcolor{textcolor}%
\pgftext[x=1.033220in,y=2.096978in,left,base]{\color{textcolor}\rmfamily\fontsize{9.000000}{10.800000}\selectfont greedy}%
\end{pgfscope}%
\begin{pgfscope}%
\pgfsetrectcap%
\pgfsetroundjoin%
\pgfsetlinewidth{1.003750pt}%
\definecolor{currentstroke}{rgb}{0.611765,0.568627,0.274510}%
\pgfsetstrokecolor{currentstroke}%
\pgfsetdash{}{0pt}%
\pgfpathmoveto{\pgfqpoint{0.758220in}{1.978928in}}%
\pgfpathlineto{\pgfqpoint{1.008220in}{1.978928in}}%
\pgfusepath{stroke}%
\end{pgfscope}%
\begin{pgfscope}%
\pgfsetbuttcap%
\pgfsetmiterjoin%
\definecolor{currentfill}{rgb}{0.611765,0.568627,0.274510}%
\pgfsetfillcolor{currentfill}%
\pgfsetlinewidth{0.501875pt}%
\definecolor{currentstroke}{rgb}{0.000000,0.000000,0.000000}%
\pgfsetstrokecolor{currentstroke}%
\pgfsetdash{}{0pt}%
\pgfsys@defobject{currentmarker}{\pgfqpoint{-0.034722in}{-0.034722in}}{\pgfqpoint{0.034722in}{0.034722in}}{%
\pgfpathmoveto{\pgfqpoint{-0.034722in}{0.000000in}}%
\pgfpathlineto{\pgfqpoint{0.034722in}{-0.034722in}}%
\pgfpathlineto{\pgfqpoint{0.034722in}{0.034722in}}%
\pgfpathclose%
\pgfusepath{stroke,fill}%
}%
\begin{pgfscope}%
\pgfsys@transformshift{0.883220in}{1.978928in}%
\pgfsys@useobject{currentmarker}{}%
\end{pgfscope}%
\end{pgfscope}%
\begin{pgfscope}%
\definecolor{textcolor}{rgb}{0.000000,0.000000,0.000000}%
\pgfsetstrokecolor{textcolor}%
\pgfsetfillcolor{textcolor}%
\pgftext[x=1.033220in,y=1.935178in,left,base]{\color{textcolor}\rmfamily\fontsize{9.000000}{10.800000}\selectfont metis}%
\end{pgfscope}%
\begin{pgfscope}%
\pgfsetrectcap%
\pgfsetroundjoin%
\pgfsetlinewidth{1.003750pt}%
\definecolor{currentstroke}{rgb}{0.780392,0.643137,0.254902}%
\pgfsetstrokecolor{currentstroke}%
\pgfsetdash{}{0pt}%
\pgfpathmoveto{\pgfqpoint{0.758220in}{1.817129in}}%
\pgfpathlineto{\pgfqpoint{1.008220in}{1.817129in}}%
\pgfusepath{stroke}%
\end{pgfscope}%
\begin{pgfscope}%
\pgfsetbuttcap%
\pgfsetmiterjoin%
\definecolor{currentfill}{rgb}{0.780392,0.643137,0.254902}%
\pgfsetfillcolor{currentfill}%
\pgfsetlinewidth{0.501875pt}%
\definecolor{currentstroke}{rgb}{0.000000,0.000000,0.000000}%
\pgfsetstrokecolor{currentstroke}%
\pgfsetdash{}{0pt}%
\pgfsys@defobject{currentmarker}{\pgfqpoint{-0.034722in}{-0.034722in}}{\pgfqpoint{0.034722in}{0.034722in}}{%
\pgfpathmoveto{\pgfqpoint{0.034722in}{-0.000000in}}%
\pgfpathlineto{\pgfqpoint{-0.034722in}{0.034722in}}%
\pgfpathlineto{\pgfqpoint{-0.034722in}{-0.034722in}}%
\pgfpathclose%
\pgfusepath{stroke,fill}%
}%
\begin{pgfscope}%
\pgfsys@transformshift{0.883220in}{1.817129in}%
\pgfsys@useobject{currentmarker}{}%
\end{pgfscope}%
\end{pgfscope}%
\begin{pgfscope}%
\definecolor{textcolor}{rgb}{0.000000,0.000000,0.000000}%
\pgfsetstrokecolor{textcolor}%
\pgfsetfillcolor{textcolor}%
\pgftext[x=1.033220in,y=1.773379in,left,base]{\color{textcolor}\rmfamily\fontsize{9.000000}{10.800000}\selectfont GN}%
\end{pgfscope}%
\begin{pgfscope}%
\pgfsetrectcap%
\pgfsetroundjoin%
\pgfsetlinewidth{1.003750pt}%
\definecolor{currentstroke}{rgb}{1.000000,0.694118,0.305882}%
\pgfsetstrokecolor{currentstroke}%
\pgfsetdash{}{0pt}%
\pgfpathmoveto{\pgfqpoint{0.758220in}{1.655329in}}%
\pgfpathlineto{\pgfqpoint{1.008220in}{1.655329in}}%
\pgfusepath{stroke}%
\end{pgfscope}%
\begin{pgfscope}%
\pgfsetbuttcap%
\pgfsetbeveljoin%
\definecolor{currentfill}{rgb}{1.000000,0.694118,0.305882}%
\pgfsetfillcolor{currentfill}%
\pgfsetlinewidth{0.501875pt}%
\definecolor{currentstroke}{rgb}{0.000000,0.000000,0.000000}%
\pgfsetstrokecolor{currentstroke}%
\pgfsetdash{}{0pt}%
\pgfsys@defobject{currentmarker}{\pgfqpoint{-0.033023in}{-0.028091in}}{\pgfqpoint{0.033023in}{0.034722in}}{%
\pgfpathmoveto{\pgfqpoint{0.000000in}{0.034722in}}%
\pgfpathlineto{\pgfqpoint{-0.007796in}{0.010730in}}%
\pgfpathlineto{\pgfqpoint{-0.033023in}{0.010730in}}%
\pgfpathlineto{\pgfqpoint{-0.012614in}{-0.004098in}}%
\pgfpathlineto{\pgfqpoint{-0.020409in}{-0.028091in}}%
\pgfpathlineto{\pgfqpoint{-0.000000in}{-0.013263in}}%
\pgfpathlineto{\pgfqpoint{0.020409in}{-0.028091in}}%
\pgfpathlineto{\pgfqpoint{0.012614in}{-0.004098in}}%
\pgfpathlineto{\pgfqpoint{0.033023in}{0.010730in}}%
\pgfpathlineto{\pgfqpoint{0.007796in}{0.010730in}}%
\pgfpathclose%
\pgfusepath{stroke,fill}%
}%
\begin{pgfscope}%
\pgfsys@transformshift{0.883220in}{1.655329in}%
\pgfsys@useobject{currentmarker}{}%
\end{pgfscope}%
\end{pgfscope}%
\begin{pgfscope}%
\definecolor{textcolor}{rgb}{0.000000,0.000000,0.000000}%
\pgfsetstrokecolor{textcolor}%
\pgfsetfillcolor{textcolor}%
\pgftext[x=1.033220in,y=1.611579in,left,base]{\color{textcolor}\rmfamily\fontsize{9.000000}{10.800000}\selectfont LG+Flow}%
\end{pgfscope}%
\end{pgfpicture}%
\makeatother%
\endgroup%

%	\caption{\label{fig:cubic-time} Median runtime of various methods on counting the number of vertex covers of 100 randomly-sampled cubic graphs with $n$ vertices. Datapoints that ran out of time ($1000$ seconds) or memory (48 GB) are not shown. Our contribution \textbf{Line+Flow} is faster than the other methods on formulas counting vertex covers of large graphs.}
%\end{figure}

%\begin{figure}
%	\centering
%	%% Creator: Matplotlib, PGF backend
%%
%% To include the figure in your LaTeX document, write
%%   \input{<filename>.pgf}
%%
%% Make sure the required packages are loaded in your preamble
%%   \usepackage{pgf}
%%
%% and, on pdftex
%%   \usepackage[utf8]{inputenc}\DeclareUnicodeCharacter{2212}{-}
%%
%% or, on luatex and xetex
%%   \usepackage{unicode-math}
%%
%% Figures using additional raster images can only be included by \input if
%% they are in the same directory as the main LaTeX file. For loading figures
%% from other directories you can use the `import` package
%%   \usepackage{import}
%%
%% and then include the figures with
%%   \import{<path to file>}{<filename>.pgf}
%%
%% Matplotlib used the following preamble
%%   \usepackage[utf8x]{inputenc}
%%   \usepackage[T1]{fontenc}
%%
\begingroup%
\makeatletter%
\begin{pgfpicture}%
\pgfpathrectangle{\pgfpointorigin}{\pgfqpoint{6.000000in}{3.400000in}}%
\pgfusepath{use as bounding box, clip}%
\begin{pgfscope}%
\pgfsetbuttcap%
\pgfsetmiterjoin%
\definecolor{currentfill}{rgb}{1.000000,1.000000,1.000000}%
\pgfsetfillcolor{currentfill}%
\pgfsetlinewidth{0.000000pt}%
\definecolor{currentstroke}{rgb}{1.000000,1.000000,1.000000}%
\pgfsetstrokecolor{currentstroke}%
\pgfsetdash{}{0pt}%
\pgfpathmoveto{\pgfqpoint{0.000000in}{0.000000in}}%
\pgfpathlineto{\pgfqpoint{6.000000in}{0.000000in}}%
\pgfpathlineto{\pgfqpoint{6.000000in}{3.400000in}}%
\pgfpathlineto{\pgfqpoint{0.000000in}{3.400000in}}%
\pgfpathclose%
\pgfusepath{fill}%
\end{pgfscope}%
\begin{pgfscope}%
\pgfsetbuttcap%
\pgfsetmiterjoin%
\definecolor{currentfill}{rgb}{1.000000,1.000000,1.000000}%
\pgfsetfillcolor{currentfill}%
\pgfsetlinewidth{0.000000pt}%
\definecolor{currentstroke}{rgb}{0.000000,0.000000,0.000000}%
\pgfsetstrokecolor{currentstroke}%
\pgfsetstrokeopacity{0.000000}%
\pgfsetdash{}{0pt}%
\pgfpathmoveto{\pgfqpoint{0.553904in}{0.535823in}}%
\pgfpathlineto{\pgfqpoint{5.850000in}{0.535823in}}%
\pgfpathlineto{\pgfqpoint{5.850000in}{3.250000in}}%
\pgfpathlineto{\pgfqpoint{0.553904in}{3.250000in}}%
\pgfpathclose%
\pgfusepath{fill}%
\end{pgfscope}%
\begin{pgfscope}%
\pgfsetbuttcap%
\pgfsetroundjoin%
\definecolor{currentfill}{rgb}{0.000000,0.000000,0.000000}%
\pgfsetfillcolor{currentfill}%
\pgfsetlinewidth{0.803000pt}%
\definecolor{currentstroke}{rgb}{0.000000,0.000000,0.000000}%
\pgfsetstrokecolor{currentstroke}%
\pgfsetdash{}{0pt}%
\pgfsys@defobject{currentmarker}{\pgfqpoint{0.000000in}{-0.048611in}}{\pgfqpoint{0.000000in}{0.000000in}}{%
\pgfpathmoveto{\pgfqpoint{0.000000in}{0.000000in}}%
\pgfpathlineto{\pgfqpoint{0.000000in}{-0.048611in}}%
\pgfusepath{stroke,fill}%
}%
\begin{pgfscope}%
\pgfsys@transformshift{0.794636in}{0.535823in}%
\pgfsys@useobject{currentmarker}{}%
\end{pgfscope}%
\end{pgfscope}%
\begin{pgfscope}%
\definecolor{textcolor}{rgb}{0.000000,0.000000,0.000000}%
\pgfsetstrokecolor{textcolor}%
\pgfsetfillcolor{textcolor}%
\pgftext[x=0.794636in,y=0.438600in,,top]{\color{textcolor}\rmfamily\fontsize{9.000000}{10.800000}\selectfont \(\displaystyle {50}\)}%
\end{pgfscope}%
\begin{pgfscope}%
\pgfsetbuttcap%
\pgfsetroundjoin%
\definecolor{currentfill}{rgb}{0.000000,0.000000,0.000000}%
\pgfsetfillcolor{currentfill}%
\pgfsetlinewidth{0.803000pt}%
\definecolor{currentstroke}{rgb}{0.000000,0.000000,0.000000}%
\pgfsetstrokecolor{currentstroke}%
\pgfsetdash{}{0pt}%
\pgfsys@defobject{currentmarker}{\pgfqpoint{0.000000in}{-0.048611in}}{\pgfqpoint{0.000000in}{0.000000in}}{%
\pgfpathmoveto{\pgfqpoint{0.000000in}{0.000000in}}%
\pgfpathlineto{\pgfqpoint{0.000000in}{-0.048611in}}%
\pgfusepath{stroke,fill}%
}%
\begin{pgfscope}%
\pgfsys@transformshift{1.998294in}{0.535823in}%
\pgfsys@useobject{currentmarker}{}%
\end{pgfscope}%
\end{pgfscope}%
\begin{pgfscope}%
\definecolor{textcolor}{rgb}{0.000000,0.000000,0.000000}%
\pgfsetstrokecolor{textcolor}%
\pgfsetfillcolor{textcolor}%
\pgftext[x=1.998294in,y=0.438600in,,top]{\color{textcolor}\rmfamily\fontsize{9.000000}{10.800000}\selectfont \(\displaystyle {100}\)}%
\end{pgfscope}%
\begin{pgfscope}%
\pgfsetbuttcap%
\pgfsetroundjoin%
\definecolor{currentfill}{rgb}{0.000000,0.000000,0.000000}%
\pgfsetfillcolor{currentfill}%
\pgfsetlinewidth{0.803000pt}%
\definecolor{currentstroke}{rgb}{0.000000,0.000000,0.000000}%
\pgfsetstrokecolor{currentstroke}%
\pgfsetdash{}{0pt}%
\pgfsys@defobject{currentmarker}{\pgfqpoint{0.000000in}{-0.048611in}}{\pgfqpoint{0.000000in}{0.000000in}}{%
\pgfpathmoveto{\pgfqpoint{0.000000in}{0.000000in}}%
\pgfpathlineto{\pgfqpoint{0.000000in}{-0.048611in}}%
\pgfusepath{stroke,fill}%
}%
\begin{pgfscope}%
\pgfsys@transformshift{3.201952in}{0.535823in}%
\pgfsys@useobject{currentmarker}{}%
\end{pgfscope}%
\end{pgfscope}%
\begin{pgfscope}%
\definecolor{textcolor}{rgb}{0.000000,0.000000,0.000000}%
\pgfsetstrokecolor{textcolor}%
\pgfsetfillcolor{textcolor}%
\pgftext[x=3.201952in,y=0.438600in,,top]{\color{textcolor}\rmfamily\fontsize{9.000000}{10.800000}\selectfont \(\displaystyle {150}\)}%
\end{pgfscope}%
\begin{pgfscope}%
\pgfsetbuttcap%
\pgfsetroundjoin%
\definecolor{currentfill}{rgb}{0.000000,0.000000,0.000000}%
\pgfsetfillcolor{currentfill}%
\pgfsetlinewidth{0.803000pt}%
\definecolor{currentstroke}{rgb}{0.000000,0.000000,0.000000}%
\pgfsetstrokecolor{currentstroke}%
\pgfsetdash{}{0pt}%
\pgfsys@defobject{currentmarker}{\pgfqpoint{0.000000in}{-0.048611in}}{\pgfqpoint{0.000000in}{0.000000in}}{%
\pgfpathmoveto{\pgfqpoint{0.000000in}{0.000000in}}%
\pgfpathlineto{\pgfqpoint{0.000000in}{-0.048611in}}%
\pgfusepath{stroke,fill}%
}%
\begin{pgfscope}%
\pgfsys@transformshift{4.405610in}{0.535823in}%
\pgfsys@useobject{currentmarker}{}%
\end{pgfscope}%
\end{pgfscope}%
\begin{pgfscope}%
\definecolor{textcolor}{rgb}{0.000000,0.000000,0.000000}%
\pgfsetstrokecolor{textcolor}%
\pgfsetfillcolor{textcolor}%
\pgftext[x=4.405610in,y=0.438600in,,top]{\color{textcolor}\rmfamily\fontsize{9.000000}{10.800000}\selectfont \(\displaystyle {200}\)}%
\end{pgfscope}%
\begin{pgfscope}%
\pgfsetbuttcap%
\pgfsetroundjoin%
\definecolor{currentfill}{rgb}{0.000000,0.000000,0.000000}%
\pgfsetfillcolor{currentfill}%
\pgfsetlinewidth{0.803000pt}%
\definecolor{currentstroke}{rgb}{0.000000,0.000000,0.000000}%
\pgfsetstrokecolor{currentstroke}%
\pgfsetdash{}{0pt}%
\pgfsys@defobject{currentmarker}{\pgfqpoint{0.000000in}{-0.048611in}}{\pgfqpoint{0.000000in}{0.000000in}}{%
\pgfpathmoveto{\pgfqpoint{0.000000in}{0.000000in}}%
\pgfpathlineto{\pgfqpoint{0.000000in}{-0.048611in}}%
\pgfusepath{stroke,fill}%
}%
\begin{pgfscope}%
\pgfsys@transformshift{5.609268in}{0.535823in}%
\pgfsys@useobject{currentmarker}{}%
\end{pgfscope}%
\end{pgfscope}%
\begin{pgfscope}%
\definecolor{textcolor}{rgb}{0.000000,0.000000,0.000000}%
\pgfsetstrokecolor{textcolor}%
\pgfsetfillcolor{textcolor}%
\pgftext[x=5.609268in,y=0.438600in,,top]{\color{textcolor}\rmfamily\fontsize{9.000000}{10.800000}\selectfont \(\displaystyle {250}\)}%
\end{pgfscope}%
\begin{pgfscope}%
\definecolor{textcolor}{rgb}{0.000000,0.000000,0.000000}%
\pgfsetstrokecolor{textcolor}%
\pgfsetfillcolor{textcolor}%
\pgftext[x=3.201952in,y=0.272655in,,top]{\color{textcolor}\rmfamily\fontsize{10.000000}{12.000000}\selectfont \(\displaystyle n\): Number of vertices}%
\end{pgfscope}%
\begin{pgfscope}%
\pgfsetbuttcap%
\pgfsetroundjoin%
\definecolor{currentfill}{rgb}{0.000000,0.000000,0.000000}%
\pgfsetfillcolor{currentfill}%
\pgfsetlinewidth{0.803000pt}%
\definecolor{currentstroke}{rgb}{0.000000,0.000000,0.000000}%
\pgfsetstrokecolor{currentstroke}%
\pgfsetdash{}{0pt}%
\pgfsys@defobject{currentmarker}{\pgfqpoint{-0.048611in}{0.000000in}}{\pgfqpoint{-0.000000in}{0.000000in}}{%
\pgfpathmoveto{\pgfqpoint{-0.000000in}{0.000000in}}%
\pgfpathlineto{\pgfqpoint{-0.048611in}{0.000000in}}%
\pgfusepath{stroke,fill}%
}%
\begin{pgfscope}%
\pgfsys@transformshift{0.553904in}{0.719376in}%
\pgfsys@useobject{currentmarker}{}%
\end{pgfscope}%
\end{pgfscope}%
\begin{pgfscope}%
\definecolor{textcolor}{rgb}{0.000000,0.000000,0.000000}%
\pgfsetstrokecolor{textcolor}%
\pgfsetfillcolor{textcolor}%
\pgftext[x=0.328211in, y=0.676331in, left, base]{\color{textcolor}\rmfamily\fontsize{9.000000}{10.800000}\selectfont \(\displaystyle {10}\)}%
\end{pgfscope}%
\begin{pgfscope}%
\pgfsetbuttcap%
\pgfsetroundjoin%
\definecolor{currentfill}{rgb}{0.000000,0.000000,0.000000}%
\pgfsetfillcolor{currentfill}%
\pgfsetlinewidth{0.803000pt}%
\definecolor{currentstroke}{rgb}{0.000000,0.000000,0.000000}%
\pgfsetstrokecolor{currentstroke}%
\pgfsetdash{}{0pt}%
\pgfsys@defobject{currentmarker}{\pgfqpoint{-0.048611in}{0.000000in}}{\pgfqpoint{-0.000000in}{0.000000in}}{%
\pgfpathmoveto{\pgfqpoint{-0.000000in}{0.000000in}}%
\pgfpathlineto{\pgfqpoint{-0.048611in}{0.000000in}}%
\pgfusepath{stroke,fill}%
}%
\begin{pgfscope}%
\pgfsys@transformshift{0.553904in}{1.321189in}%
\pgfsys@useobject{currentmarker}{}%
\end{pgfscope}%
\end{pgfscope}%
\begin{pgfscope}%
\definecolor{textcolor}{rgb}{0.000000,0.000000,0.000000}%
\pgfsetstrokecolor{textcolor}%
\pgfsetfillcolor{textcolor}%
\pgftext[x=0.328211in, y=1.278144in, left, base]{\color{textcolor}\rmfamily\fontsize{9.000000}{10.800000}\selectfont \(\displaystyle {20}\)}%
\end{pgfscope}%
\begin{pgfscope}%
\pgfsetbuttcap%
\pgfsetroundjoin%
\definecolor{currentfill}{rgb}{0.000000,0.000000,0.000000}%
\pgfsetfillcolor{currentfill}%
\pgfsetlinewidth{0.803000pt}%
\definecolor{currentstroke}{rgb}{0.000000,0.000000,0.000000}%
\pgfsetstrokecolor{currentstroke}%
\pgfsetdash{}{0pt}%
\pgfsys@defobject{currentmarker}{\pgfqpoint{-0.048611in}{0.000000in}}{\pgfqpoint{-0.000000in}{0.000000in}}{%
\pgfpathmoveto{\pgfqpoint{-0.000000in}{0.000000in}}%
\pgfpathlineto{\pgfqpoint{-0.048611in}{0.000000in}}%
\pgfusepath{stroke,fill}%
}%
\begin{pgfscope}%
\pgfsys@transformshift{0.553904in}{1.923002in}%
\pgfsys@useobject{currentmarker}{}%
\end{pgfscope}%
\end{pgfscope}%
\begin{pgfscope}%
\definecolor{textcolor}{rgb}{0.000000,0.000000,0.000000}%
\pgfsetstrokecolor{textcolor}%
\pgfsetfillcolor{textcolor}%
\pgftext[x=0.328211in, y=1.879957in, left, base]{\color{textcolor}\rmfamily\fontsize{9.000000}{10.800000}\selectfont \(\displaystyle {30}\)}%
\end{pgfscope}%
\begin{pgfscope}%
\pgfsetbuttcap%
\pgfsetroundjoin%
\definecolor{currentfill}{rgb}{0.000000,0.000000,0.000000}%
\pgfsetfillcolor{currentfill}%
\pgfsetlinewidth{0.803000pt}%
\definecolor{currentstroke}{rgb}{0.000000,0.000000,0.000000}%
\pgfsetstrokecolor{currentstroke}%
\pgfsetdash{}{0pt}%
\pgfsys@defobject{currentmarker}{\pgfqpoint{-0.048611in}{0.000000in}}{\pgfqpoint{-0.000000in}{0.000000in}}{%
\pgfpathmoveto{\pgfqpoint{-0.000000in}{0.000000in}}%
\pgfpathlineto{\pgfqpoint{-0.048611in}{0.000000in}}%
\pgfusepath{stroke,fill}%
}%
\begin{pgfscope}%
\pgfsys@transformshift{0.553904in}{2.524815in}%
\pgfsys@useobject{currentmarker}{}%
\end{pgfscope}%
\end{pgfscope}%
\begin{pgfscope}%
\definecolor{textcolor}{rgb}{0.000000,0.000000,0.000000}%
\pgfsetstrokecolor{textcolor}%
\pgfsetfillcolor{textcolor}%
\pgftext[x=0.328211in, y=2.481770in, left, base]{\color{textcolor}\rmfamily\fontsize{9.000000}{10.800000}\selectfont \(\displaystyle {40}\)}%
\end{pgfscope}%
\begin{pgfscope}%
\pgfsetbuttcap%
\pgfsetroundjoin%
\definecolor{currentfill}{rgb}{0.000000,0.000000,0.000000}%
\pgfsetfillcolor{currentfill}%
\pgfsetlinewidth{0.803000pt}%
\definecolor{currentstroke}{rgb}{0.000000,0.000000,0.000000}%
\pgfsetstrokecolor{currentstroke}%
\pgfsetdash{}{0pt}%
\pgfsys@defobject{currentmarker}{\pgfqpoint{-0.048611in}{0.000000in}}{\pgfqpoint{-0.000000in}{0.000000in}}{%
\pgfpathmoveto{\pgfqpoint{-0.000000in}{0.000000in}}%
\pgfpathlineto{\pgfqpoint{-0.048611in}{0.000000in}}%
\pgfusepath{stroke,fill}%
}%
\begin{pgfscope}%
\pgfsys@transformshift{0.553904in}{3.126628in}%
\pgfsys@useobject{currentmarker}{}%
\end{pgfscope}%
\end{pgfscope}%
\begin{pgfscope}%
\definecolor{textcolor}{rgb}{0.000000,0.000000,0.000000}%
\pgfsetstrokecolor{textcolor}%
\pgfsetfillcolor{textcolor}%
\pgftext[x=0.328211in, y=3.083583in, left, base]{\color{textcolor}\rmfamily\fontsize{9.000000}{10.800000}\selectfont \(\displaystyle {50}\)}%
\end{pgfscope}%
\begin{pgfscope}%
\definecolor{textcolor}{rgb}{0.000000,0.000000,0.000000}%
\pgfsetstrokecolor{textcolor}%
\pgfsetfillcolor{textcolor}%
\pgftext[x=0.272655in,y=1.892911in,,bottom,rotate=90.000000]{\color{textcolor}\rmfamily\fontsize{10.000000}{12.000000}\selectfont Median max rank}%
\end{pgfscope}%
\begin{pgfscope}%
\pgfpathrectangle{\pgfqpoint{0.553904in}{0.535823in}}{\pgfqpoint{5.296096in}{2.714177in}}%
\pgfusepath{clip}%
\pgfsetrectcap%
\pgfsetroundjoin%
\pgfsetlinewidth{1.003750pt}%
\definecolor{currentstroke}{rgb}{0.529412,0.462745,0.384314}%
\pgfsetstrokecolor{currentstroke}%
\pgfsetdash{}{0pt}%
\pgfpathmoveto{\pgfqpoint{0.794636in}{0.779557in}}%
\pgfpathlineto{\pgfqpoint{1.035368in}{0.899920in}}%
\pgfpathlineto{\pgfqpoint{1.276099in}{0.960101in}}%
\pgfpathlineto{\pgfqpoint{1.516831in}{1.080464in}}%
\pgfpathlineto{\pgfqpoint{1.757562in}{1.200826in}}%
\pgfpathlineto{\pgfqpoint{1.998294in}{1.321189in}}%
\pgfpathlineto{\pgfqpoint{2.239026in}{1.441551in}}%
\pgfpathlineto{\pgfqpoint{2.479757in}{1.561914in}}%
\pgfpathlineto{\pgfqpoint{2.720489in}{1.682277in}}%
\pgfpathlineto{\pgfqpoint{2.961221in}{1.772549in}}%
\pgfpathlineto{\pgfqpoint{3.201952in}{1.923002in}}%
\pgfpathlineto{\pgfqpoint{3.442684in}{2.013274in}}%
\pgfpathlineto{\pgfqpoint{3.683415in}{2.103546in}}%
\pgfpathlineto{\pgfqpoint{3.924147in}{2.223909in}}%
\pgfpathlineto{\pgfqpoint{4.164879in}{2.344271in}}%
\pgfpathlineto{\pgfqpoint{4.405610in}{2.464634in}}%
\pgfpathlineto{\pgfqpoint{4.646342in}{2.584996in}}%
\pgfpathlineto{\pgfqpoint{4.887074in}{2.705359in}}%
\pgfpathlineto{\pgfqpoint{5.127805in}{2.885903in}}%
\pgfpathlineto{\pgfqpoint{5.368537in}{2.946084in}}%
\pgfpathlineto{\pgfqpoint{5.609268in}{3.126628in}}%
\pgfusepath{stroke}%
\end{pgfscope}%
\begin{pgfscope}%
\pgfpathrectangle{\pgfqpoint{0.553904in}{0.535823in}}{\pgfqpoint{5.296096in}{2.714177in}}%
\pgfusepath{clip}%
\pgfsetbuttcap%
\pgfsetmiterjoin%
\definecolor{currentfill}{rgb}{0.529412,0.462745,0.384314}%
\pgfsetfillcolor{currentfill}%
\pgfsetlinewidth{0.501875pt}%
\definecolor{currentstroke}{rgb}{0.000000,0.000000,0.000000}%
\pgfsetstrokecolor{currentstroke}%
\pgfsetdash{}{0pt}%
\pgfsys@defobject{currentmarker}{\pgfqpoint{-0.034722in}{-0.034722in}}{\pgfqpoint{0.034722in}{0.034722in}}{%
\pgfpathmoveto{\pgfqpoint{-0.000000in}{-0.034722in}}%
\pgfpathlineto{\pgfqpoint{0.034722in}{0.034722in}}%
\pgfpathlineto{\pgfqpoint{-0.034722in}{0.034722in}}%
\pgfpathclose%
\pgfusepath{stroke,fill}%
}%
\begin{pgfscope}%
\pgfsys@transformshift{0.794636in}{0.779557in}%
\pgfsys@useobject{currentmarker}{}%
\end{pgfscope}%
\begin{pgfscope}%
\pgfsys@transformshift{1.035368in}{0.899920in}%
\pgfsys@useobject{currentmarker}{}%
\end{pgfscope}%
\begin{pgfscope}%
\pgfsys@transformshift{1.276099in}{0.960101in}%
\pgfsys@useobject{currentmarker}{}%
\end{pgfscope}%
\begin{pgfscope}%
\pgfsys@transformshift{1.516831in}{1.080464in}%
\pgfsys@useobject{currentmarker}{}%
\end{pgfscope}%
\begin{pgfscope}%
\pgfsys@transformshift{1.757562in}{1.200826in}%
\pgfsys@useobject{currentmarker}{}%
\end{pgfscope}%
\begin{pgfscope}%
\pgfsys@transformshift{1.998294in}{1.321189in}%
\pgfsys@useobject{currentmarker}{}%
\end{pgfscope}%
\begin{pgfscope}%
\pgfsys@transformshift{2.239026in}{1.441551in}%
\pgfsys@useobject{currentmarker}{}%
\end{pgfscope}%
\begin{pgfscope}%
\pgfsys@transformshift{2.479757in}{1.561914in}%
\pgfsys@useobject{currentmarker}{}%
\end{pgfscope}%
\begin{pgfscope}%
\pgfsys@transformshift{2.720489in}{1.682277in}%
\pgfsys@useobject{currentmarker}{}%
\end{pgfscope}%
\begin{pgfscope}%
\pgfsys@transformshift{2.961221in}{1.772549in}%
\pgfsys@useobject{currentmarker}{}%
\end{pgfscope}%
\begin{pgfscope}%
\pgfsys@transformshift{3.201952in}{1.923002in}%
\pgfsys@useobject{currentmarker}{}%
\end{pgfscope}%
\begin{pgfscope}%
\pgfsys@transformshift{3.442684in}{2.013274in}%
\pgfsys@useobject{currentmarker}{}%
\end{pgfscope}%
\begin{pgfscope}%
\pgfsys@transformshift{3.683415in}{2.103546in}%
\pgfsys@useobject{currentmarker}{}%
\end{pgfscope}%
\begin{pgfscope}%
\pgfsys@transformshift{3.924147in}{2.223909in}%
\pgfsys@useobject{currentmarker}{}%
\end{pgfscope}%
\begin{pgfscope}%
\pgfsys@transformshift{4.164879in}{2.344271in}%
\pgfsys@useobject{currentmarker}{}%
\end{pgfscope}%
\begin{pgfscope}%
\pgfsys@transformshift{4.405610in}{2.464634in}%
\pgfsys@useobject{currentmarker}{}%
\end{pgfscope}%
\begin{pgfscope}%
\pgfsys@transformshift{4.646342in}{2.584996in}%
\pgfsys@useobject{currentmarker}{}%
\end{pgfscope}%
\begin{pgfscope}%
\pgfsys@transformshift{4.887074in}{2.705359in}%
\pgfsys@useobject{currentmarker}{}%
\end{pgfscope}%
\begin{pgfscope}%
\pgfsys@transformshift{5.127805in}{2.885903in}%
\pgfsys@useobject{currentmarker}{}%
\end{pgfscope}%
\begin{pgfscope}%
\pgfsys@transformshift{5.368537in}{2.946084in}%
\pgfsys@useobject{currentmarker}{}%
\end{pgfscope}%
\begin{pgfscope}%
\pgfsys@transformshift{5.609268in}{3.126628in}%
\pgfsys@useobject{currentmarker}{}%
\end{pgfscope}%
\end{pgfscope}%
\begin{pgfscope}%
\pgfpathrectangle{\pgfqpoint{0.553904in}{0.535823in}}{\pgfqpoint{5.296096in}{2.714177in}}%
\pgfusepath{clip}%
\pgfsetrectcap%
\pgfsetroundjoin%
\pgfsetlinewidth{1.003750pt}%
\definecolor{currentstroke}{rgb}{0.611765,0.568627,0.274510}%
\pgfsetstrokecolor{currentstroke}%
\pgfsetdash{}{0pt}%
\pgfpathmoveto{\pgfqpoint{0.794636in}{0.719376in}}%
\pgfpathlineto{\pgfqpoint{1.035368in}{0.779557in}}%
\pgfpathlineto{\pgfqpoint{1.276099in}{0.839738in}}%
\pgfpathlineto{\pgfqpoint{1.516831in}{0.899920in}}%
\pgfpathlineto{\pgfqpoint{1.757562in}{1.020282in}}%
\pgfpathlineto{\pgfqpoint{1.998294in}{1.080464in}}%
\pgfpathlineto{\pgfqpoint{2.239026in}{1.140645in}}%
\pgfpathlineto{\pgfqpoint{2.479757in}{1.200826in}}%
\pgfpathlineto{\pgfqpoint{2.720489in}{1.321189in}}%
\pgfpathlineto{\pgfqpoint{2.961221in}{1.411461in}}%
\pgfpathlineto{\pgfqpoint{3.201952in}{1.471642in}}%
\pgfpathlineto{\pgfqpoint{3.442684in}{1.561914in}}%
\pgfpathlineto{\pgfqpoint{3.683415in}{1.622095in}}%
\pgfpathlineto{\pgfqpoint{3.924147in}{1.742458in}}%
\pgfpathlineto{\pgfqpoint{4.164879in}{1.802639in}}%
\pgfpathlineto{\pgfqpoint{4.405610in}{1.862821in}}%
\pgfpathlineto{\pgfqpoint{4.646342in}{1.923002in}}%
\pgfpathlineto{\pgfqpoint{4.887074in}{2.043365in}}%
\pgfpathlineto{\pgfqpoint{5.127805in}{2.163727in}}%
\pgfpathlineto{\pgfqpoint{5.368537in}{2.223909in}}%
\pgfpathlineto{\pgfqpoint{5.609268in}{2.284090in}}%
\pgfusepath{stroke}%
\end{pgfscope}%
\begin{pgfscope}%
\pgfpathrectangle{\pgfqpoint{0.553904in}{0.535823in}}{\pgfqpoint{5.296096in}{2.714177in}}%
\pgfusepath{clip}%
\pgfsetbuttcap%
\pgfsetmiterjoin%
\definecolor{currentfill}{rgb}{0.611765,0.568627,0.274510}%
\pgfsetfillcolor{currentfill}%
\pgfsetlinewidth{0.501875pt}%
\definecolor{currentstroke}{rgb}{0.000000,0.000000,0.000000}%
\pgfsetstrokecolor{currentstroke}%
\pgfsetdash{}{0pt}%
\pgfsys@defobject{currentmarker}{\pgfqpoint{-0.034722in}{-0.034722in}}{\pgfqpoint{0.034722in}{0.034722in}}{%
\pgfpathmoveto{\pgfqpoint{-0.034722in}{0.000000in}}%
\pgfpathlineto{\pgfqpoint{0.034722in}{-0.034722in}}%
\pgfpathlineto{\pgfqpoint{0.034722in}{0.034722in}}%
\pgfpathclose%
\pgfusepath{stroke,fill}%
}%
\begin{pgfscope}%
\pgfsys@transformshift{0.794636in}{0.719376in}%
\pgfsys@useobject{currentmarker}{}%
\end{pgfscope}%
\begin{pgfscope}%
\pgfsys@transformshift{1.035368in}{0.779557in}%
\pgfsys@useobject{currentmarker}{}%
\end{pgfscope}%
\begin{pgfscope}%
\pgfsys@transformshift{1.276099in}{0.839738in}%
\pgfsys@useobject{currentmarker}{}%
\end{pgfscope}%
\begin{pgfscope}%
\pgfsys@transformshift{1.516831in}{0.899920in}%
\pgfsys@useobject{currentmarker}{}%
\end{pgfscope}%
\begin{pgfscope}%
\pgfsys@transformshift{1.757562in}{1.020282in}%
\pgfsys@useobject{currentmarker}{}%
\end{pgfscope}%
\begin{pgfscope}%
\pgfsys@transformshift{1.998294in}{1.080464in}%
\pgfsys@useobject{currentmarker}{}%
\end{pgfscope}%
\begin{pgfscope}%
\pgfsys@transformshift{2.239026in}{1.140645in}%
\pgfsys@useobject{currentmarker}{}%
\end{pgfscope}%
\begin{pgfscope}%
\pgfsys@transformshift{2.479757in}{1.200826in}%
\pgfsys@useobject{currentmarker}{}%
\end{pgfscope}%
\begin{pgfscope}%
\pgfsys@transformshift{2.720489in}{1.321189in}%
\pgfsys@useobject{currentmarker}{}%
\end{pgfscope}%
\begin{pgfscope}%
\pgfsys@transformshift{2.961221in}{1.411461in}%
\pgfsys@useobject{currentmarker}{}%
\end{pgfscope}%
\begin{pgfscope}%
\pgfsys@transformshift{3.201952in}{1.471642in}%
\pgfsys@useobject{currentmarker}{}%
\end{pgfscope}%
\begin{pgfscope}%
\pgfsys@transformshift{3.442684in}{1.561914in}%
\pgfsys@useobject{currentmarker}{}%
\end{pgfscope}%
\begin{pgfscope}%
\pgfsys@transformshift{3.683415in}{1.622095in}%
\pgfsys@useobject{currentmarker}{}%
\end{pgfscope}%
\begin{pgfscope}%
\pgfsys@transformshift{3.924147in}{1.742458in}%
\pgfsys@useobject{currentmarker}{}%
\end{pgfscope}%
\begin{pgfscope}%
\pgfsys@transformshift{4.164879in}{1.802639in}%
\pgfsys@useobject{currentmarker}{}%
\end{pgfscope}%
\begin{pgfscope}%
\pgfsys@transformshift{4.405610in}{1.862821in}%
\pgfsys@useobject{currentmarker}{}%
\end{pgfscope}%
\begin{pgfscope}%
\pgfsys@transformshift{4.646342in}{1.923002in}%
\pgfsys@useobject{currentmarker}{}%
\end{pgfscope}%
\begin{pgfscope}%
\pgfsys@transformshift{4.887074in}{2.043365in}%
\pgfsys@useobject{currentmarker}{}%
\end{pgfscope}%
\begin{pgfscope}%
\pgfsys@transformshift{5.127805in}{2.163727in}%
\pgfsys@useobject{currentmarker}{}%
\end{pgfscope}%
\begin{pgfscope}%
\pgfsys@transformshift{5.368537in}{2.223909in}%
\pgfsys@useobject{currentmarker}{}%
\end{pgfscope}%
\begin{pgfscope}%
\pgfsys@transformshift{5.609268in}{2.284090in}%
\pgfsys@useobject{currentmarker}{}%
\end{pgfscope}%
\end{pgfscope}%
\begin{pgfscope}%
\pgfpathrectangle{\pgfqpoint{0.553904in}{0.535823in}}{\pgfqpoint{5.296096in}{2.714177in}}%
\pgfusepath{clip}%
\pgfsetrectcap%
\pgfsetroundjoin%
\pgfsetlinewidth{1.003750pt}%
\definecolor{currentstroke}{rgb}{0.780392,0.643137,0.254902}%
\pgfsetstrokecolor{currentstroke}%
\pgfsetdash{}{0pt}%
\pgfpathmoveto{\pgfqpoint{0.794636in}{0.659194in}}%
\pgfpathlineto{\pgfqpoint{1.035368in}{0.779557in}}%
\pgfpathlineto{\pgfqpoint{1.276099in}{0.839738in}}%
\pgfpathlineto{\pgfqpoint{1.516831in}{0.899920in}}%
\pgfpathlineto{\pgfqpoint{1.757562in}{1.020282in}}%
\pgfpathlineto{\pgfqpoint{1.998294in}{1.080464in}}%
\pgfpathlineto{\pgfqpoint{2.239026in}{1.140645in}}%
\pgfpathlineto{\pgfqpoint{2.479757in}{1.261007in}}%
\pgfpathlineto{\pgfqpoint{2.720489in}{1.321189in}}%
\pgfpathlineto{\pgfqpoint{2.961221in}{1.441551in}}%
\pgfpathlineto{\pgfqpoint{3.201952in}{1.501733in}}%
\pgfpathlineto{\pgfqpoint{3.442684in}{1.561914in}}%
\pgfpathlineto{\pgfqpoint{3.683415in}{1.622095in}}%
\pgfpathlineto{\pgfqpoint{3.924147in}{1.742458in}}%
\pgfpathlineto{\pgfqpoint{4.164879in}{1.802639in}}%
\pgfpathlineto{\pgfqpoint{4.405610in}{1.923002in}}%
\pgfpathlineto{\pgfqpoint{4.646342in}{1.923002in}}%
\pgfpathlineto{\pgfqpoint{4.887074in}{2.043365in}}%
\pgfpathlineto{\pgfqpoint{5.127805in}{2.103546in}}%
\pgfpathlineto{\pgfqpoint{5.368537in}{2.223909in}}%
\pgfpathlineto{\pgfqpoint{5.609268in}{2.344271in}}%
\pgfusepath{stroke}%
\end{pgfscope}%
\begin{pgfscope}%
\pgfpathrectangle{\pgfqpoint{0.553904in}{0.535823in}}{\pgfqpoint{5.296096in}{2.714177in}}%
\pgfusepath{clip}%
\pgfsetbuttcap%
\pgfsetmiterjoin%
\definecolor{currentfill}{rgb}{0.780392,0.643137,0.254902}%
\pgfsetfillcolor{currentfill}%
\pgfsetlinewidth{0.501875pt}%
\definecolor{currentstroke}{rgb}{0.000000,0.000000,0.000000}%
\pgfsetstrokecolor{currentstroke}%
\pgfsetdash{}{0pt}%
\pgfsys@defobject{currentmarker}{\pgfqpoint{-0.034722in}{-0.034722in}}{\pgfqpoint{0.034722in}{0.034722in}}{%
\pgfpathmoveto{\pgfqpoint{0.034722in}{-0.000000in}}%
\pgfpathlineto{\pgfqpoint{-0.034722in}{0.034722in}}%
\pgfpathlineto{\pgfqpoint{-0.034722in}{-0.034722in}}%
\pgfpathclose%
\pgfusepath{stroke,fill}%
}%
\begin{pgfscope}%
\pgfsys@transformshift{0.794636in}{0.659194in}%
\pgfsys@useobject{currentmarker}{}%
\end{pgfscope}%
\begin{pgfscope}%
\pgfsys@transformshift{1.035368in}{0.779557in}%
\pgfsys@useobject{currentmarker}{}%
\end{pgfscope}%
\begin{pgfscope}%
\pgfsys@transformshift{1.276099in}{0.839738in}%
\pgfsys@useobject{currentmarker}{}%
\end{pgfscope}%
\begin{pgfscope}%
\pgfsys@transformshift{1.516831in}{0.899920in}%
\pgfsys@useobject{currentmarker}{}%
\end{pgfscope}%
\begin{pgfscope}%
\pgfsys@transformshift{1.757562in}{1.020282in}%
\pgfsys@useobject{currentmarker}{}%
\end{pgfscope}%
\begin{pgfscope}%
\pgfsys@transformshift{1.998294in}{1.080464in}%
\pgfsys@useobject{currentmarker}{}%
\end{pgfscope}%
\begin{pgfscope}%
\pgfsys@transformshift{2.239026in}{1.140645in}%
\pgfsys@useobject{currentmarker}{}%
\end{pgfscope}%
\begin{pgfscope}%
\pgfsys@transformshift{2.479757in}{1.261007in}%
\pgfsys@useobject{currentmarker}{}%
\end{pgfscope}%
\begin{pgfscope}%
\pgfsys@transformshift{2.720489in}{1.321189in}%
\pgfsys@useobject{currentmarker}{}%
\end{pgfscope}%
\begin{pgfscope}%
\pgfsys@transformshift{2.961221in}{1.441551in}%
\pgfsys@useobject{currentmarker}{}%
\end{pgfscope}%
\begin{pgfscope}%
\pgfsys@transformshift{3.201952in}{1.501733in}%
\pgfsys@useobject{currentmarker}{}%
\end{pgfscope}%
\begin{pgfscope}%
\pgfsys@transformshift{3.442684in}{1.561914in}%
\pgfsys@useobject{currentmarker}{}%
\end{pgfscope}%
\begin{pgfscope}%
\pgfsys@transformshift{3.683415in}{1.622095in}%
\pgfsys@useobject{currentmarker}{}%
\end{pgfscope}%
\begin{pgfscope}%
\pgfsys@transformshift{3.924147in}{1.742458in}%
\pgfsys@useobject{currentmarker}{}%
\end{pgfscope}%
\begin{pgfscope}%
\pgfsys@transformshift{4.164879in}{1.802639in}%
\pgfsys@useobject{currentmarker}{}%
\end{pgfscope}%
\begin{pgfscope}%
\pgfsys@transformshift{4.405610in}{1.923002in}%
\pgfsys@useobject{currentmarker}{}%
\end{pgfscope}%
\begin{pgfscope}%
\pgfsys@transformshift{4.646342in}{1.923002in}%
\pgfsys@useobject{currentmarker}{}%
\end{pgfscope}%
\begin{pgfscope}%
\pgfsys@transformshift{4.887074in}{2.043365in}%
\pgfsys@useobject{currentmarker}{}%
\end{pgfscope}%
\begin{pgfscope}%
\pgfsys@transformshift{5.127805in}{2.103546in}%
\pgfsys@useobject{currentmarker}{}%
\end{pgfscope}%
\begin{pgfscope}%
\pgfsys@transformshift{5.368537in}{2.223909in}%
\pgfsys@useobject{currentmarker}{}%
\end{pgfscope}%
\begin{pgfscope}%
\pgfsys@transformshift{5.609268in}{2.344271in}%
\pgfsys@useobject{currentmarker}{}%
\end{pgfscope}%
\end{pgfscope}%
\begin{pgfscope}%
\pgfpathrectangle{\pgfqpoint{0.553904in}{0.535823in}}{\pgfqpoint{5.296096in}{2.714177in}}%
\pgfusepath{clip}%
\pgfsetrectcap%
\pgfsetroundjoin%
\pgfsetlinewidth{1.003750pt}%
\definecolor{currentstroke}{rgb}{1.000000,0.694118,0.305882}%
\pgfsetstrokecolor{currentstroke}%
\pgfsetdash{}{0pt}%
\pgfpathmoveto{\pgfqpoint{0.794636in}{0.839738in}}%
\pgfpathlineto{\pgfqpoint{1.035368in}{0.960101in}}%
\pgfpathlineto{\pgfqpoint{1.276099in}{1.080464in}}%
\pgfpathlineto{\pgfqpoint{1.516831in}{1.200826in}}%
\pgfpathlineto{\pgfqpoint{1.757562in}{1.321189in}}%
\pgfpathlineto{\pgfqpoint{1.998294in}{1.381370in}}%
\pgfpathlineto{\pgfqpoint{2.239026in}{1.471642in}}%
\pgfpathlineto{\pgfqpoint{2.479757in}{1.501733in}}%
\pgfpathlineto{\pgfqpoint{2.720489in}{1.561914in}}%
\pgfpathlineto{\pgfqpoint{2.961221in}{1.561914in}}%
\pgfpathlineto{\pgfqpoint{3.201952in}{1.561914in}}%
\pgfpathlineto{\pgfqpoint{3.442684in}{1.561914in}}%
\pgfpathlineto{\pgfqpoint{3.683415in}{1.561914in}}%
\pgfpathlineto{\pgfqpoint{3.924147in}{1.622095in}}%
\pgfpathlineto{\pgfqpoint{4.164879in}{1.622095in}}%
\pgfpathlineto{\pgfqpoint{4.405610in}{1.682277in}}%
\pgfpathlineto{\pgfqpoint{4.646342in}{1.742458in}}%
\pgfpathlineto{\pgfqpoint{4.887074in}{1.802639in}}%
\pgfpathlineto{\pgfqpoint{5.127805in}{1.923002in}}%
\pgfpathlineto{\pgfqpoint{5.368537in}{1.953093in}}%
\pgfpathlineto{\pgfqpoint{5.609268in}{2.043365in}}%
\pgfusepath{stroke}%
\end{pgfscope}%
\begin{pgfscope}%
\pgfpathrectangle{\pgfqpoint{0.553904in}{0.535823in}}{\pgfqpoint{5.296096in}{2.714177in}}%
\pgfusepath{clip}%
\pgfsetbuttcap%
\pgfsetbeveljoin%
\definecolor{currentfill}{rgb}{1.000000,0.694118,0.305882}%
\pgfsetfillcolor{currentfill}%
\pgfsetlinewidth{0.501875pt}%
\definecolor{currentstroke}{rgb}{0.000000,0.000000,0.000000}%
\pgfsetstrokecolor{currentstroke}%
\pgfsetdash{}{0pt}%
\pgfsys@defobject{currentmarker}{\pgfqpoint{-0.033023in}{-0.028091in}}{\pgfqpoint{0.033023in}{0.034722in}}{%
\pgfpathmoveto{\pgfqpoint{0.000000in}{0.034722in}}%
\pgfpathlineto{\pgfqpoint{-0.007796in}{0.010730in}}%
\pgfpathlineto{\pgfqpoint{-0.033023in}{0.010730in}}%
\pgfpathlineto{\pgfqpoint{-0.012614in}{-0.004098in}}%
\pgfpathlineto{\pgfqpoint{-0.020409in}{-0.028091in}}%
\pgfpathlineto{\pgfqpoint{-0.000000in}{-0.013263in}}%
\pgfpathlineto{\pgfqpoint{0.020409in}{-0.028091in}}%
\pgfpathlineto{\pgfqpoint{0.012614in}{-0.004098in}}%
\pgfpathlineto{\pgfqpoint{0.033023in}{0.010730in}}%
\pgfpathlineto{\pgfqpoint{0.007796in}{0.010730in}}%
\pgfpathclose%
\pgfusepath{stroke,fill}%
}%
\begin{pgfscope}%
\pgfsys@transformshift{0.794636in}{0.839738in}%
\pgfsys@useobject{currentmarker}{}%
\end{pgfscope}%
\begin{pgfscope}%
\pgfsys@transformshift{1.035368in}{0.960101in}%
\pgfsys@useobject{currentmarker}{}%
\end{pgfscope}%
\begin{pgfscope}%
\pgfsys@transformshift{1.276099in}{1.080464in}%
\pgfsys@useobject{currentmarker}{}%
\end{pgfscope}%
\begin{pgfscope}%
\pgfsys@transformshift{1.516831in}{1.200826in}%
\pgfsys@useobject{currentmarker}{}%
\end{pgfscope}%
\begin{pgfscope}%
\pgfsys@transformshift{1.757562in}{1.321189in}%
\pgfsys@useobject{currentmarker}{}%
\end{pgfscope}%
\begin{pgfscope}%
\pgfsys@transformshift{1.998294in}{1.381370in}%
\pgfsys@useobject{currentmarker}{}%
\end{pgfscope}%
\begin{pgfscope}%
\pgfsys@transformshift{2.239026in}{1.471642in}%
\pgfsys@useobject{currentmarker}{}%
\end{pgfscope}%
\begin{pgfscope}%
\pgfsys@transformshift{2.479757in}{1.501733in}%
\pgfsys@useobject{currentmarker}{}%
\end{pgfscope}%
\begin{pgfscope}%
\pgfsys@transformshift{2.720489in}{1.561914in}%
\pgfsys@useobject{currentmarker}{}%
\end{pgfscope}%
\begin{pgfscope}%
\pgfsys@transformshift{2.961221in}{1.561914in}%
\pgfsys@useobject{currentmarker}{}%
\end{pgfscope}%
\begin{pgfscope}%
\pgfsys@transformshift{3.201952in}{1.561914in}%
\pgfsys@useobject{currentmarker}{}%
\end{pgfscope}%
\begin{pgfscope}%
\pgfsys@transformshift{3.442684in}{1.561914in}%
\pgfsys@useobject{currentmarker}{}%
\end{pgfscope}%
\begin{pgfscope}%
\pgfsys@transformshift{3.683415in}{1.561914in}%
\pgfsys@useobject{currentmarker}{}%
\end{pgfscope}%
\begin{pgfscope}%
\pgfsys@transformshift{3.924147in}{1.622095in}%
\pgfsys@useobject{currentmarker}{}%
\end{pgfscope}%
\begin{pgfscope}%
\pgfsys@transformshift{4.164879in}{1.622095in}%
\pgfsys@useobject{currentmarker}{}%
\end{pgfscope}%
\begin{pgfscope}%
\pgfsys@transformshift{4.405610in}{1.682277in}%
\pgfsys@useobject{currentmarker}{}%
\end{pgfscope}%
\begin{pgfscope}%
\pgfsys@transformshift{4.646342in}{1.742458in}%
\pgfsys@useobject{currentmarker}{}%
\end{pgfscope}%
\begin{pgfscope}%
\pgfsys@transformshift{4.887074in}{1.802639in}%
\pgfsys@useobject{currentmarker}{}%
\end{pgfscope}%
\begin{pgfscope}%
\pgfsys@transformshift{5.127805in}{1.923002in}%
\pgfsys@useobject{currentmarker}{}%
\end{pgfscope}%
\begin{pgfscope}%
\pgfsys@transformshift{5.368537in}{1.953093in}%
\pgfsys@useobject{currentmarker}{}%
\end{pgfscope}%
\begin{pgfscope}%
\pgfsys@transformshift{5.609268in}{2.043365in}%
\pgfsys@useobject{currentmarker}{}%
\end{pgfscope}%
\end{pgfscope}%
\begin{pgfscope}%
\pgfsetrectcap%
\pgfsetmiterjoin%
\pgfsetlinewidth{0.803000pt}%
\definecolor{currentstroke}{rgb}{0.000000,0.000000,0.000000}%
\pgfsetstrokecolor{currentstroke}%
\pgfsetdash{}{0pt}%
\pgfpathmoveto{\pgfqpoint{0.553904in}{0.535823in}}%
\pgfpathlineto{\pgfqpoint{0.553904in}{3.250000in}}%
\pgfusepath{stroke}%
\end{pgfscope}%
\begin{pgfscope}%
\pgfsetrectcap%
\pgfsetmiterjoin%
\pgfsetlinewidth{0.803000pt}%
\definecolor{currentstroke}{rgb}{0.000000,0.000000,0.000000}%
\pgfsetstrokecolor{currentstroke}%
\pgfsetdash{}{0pt}%
\pgfpathmoveto{\pgfqpoint{5.850000in}{0.535823in}}%
\pgfpathlineto{\pgfqpoint{5.850000in}{3.250000in}}%
\pgfusepath{stroke}%
\end{pgfscope}%
\begin{pgfscope}%
\pgfsetrectcap%
\pgfsetmiterjoin%
\pgfsetlinewidth{0.803000pt}%
\definecolor{currentstroke}{rgb}{0.000000,0.000000,0.000000}%
\pgfsetstrokecolor{currentstroke}%
\pgfsetdash{}{0pt}%
\pgfpathmoveto{\pgfqpoint{0.553904in}{0.535823in}}%
\pgfpathlineto{\pgfqpoint{5.850000in}{0.535823in}}%
\pgfusepath{stroke}%
\end{pgfscope}%
\begin{pgfscope}%
\pgfsetrectcap%
\pgfsetmiterjoin%
\pgfsetlinewidth{0.803000pt}%
\definecolor{currentstroke}{rgb}{0.000000,0.000000,0.000000}%
\pgfsetstrokecolor{currentstroke}%
\pgfsetdash{}{0pt}%
\pgfpathmoveto{\pgfqpoint{0.553904in}{3.250000in}}%
\pgfpathlineto{\pgfqpoint{5.850000in}{3.250000in}}%
\pgfusepath{stroke}%
\end{pgfscope}%
\begin{pgfscope}%
\pgfsetrectcap%
\pgfsetroundjoin%
\pgfsetlinewidth{1.003750pt}%
\definecolor{currentstroke}{rgb}{0.529412,0.462745,0.384314}%
\pgfsetstrokecolor{currentstroke}%
\pgfsetdash{}{0pt}%
\pgfpathmoveto{\pgfqpoint{0.603904in}{3.156250in}}%
\pgfpathlineto{\pgfqpoint{0.853904in}{3.156250in}}%
\pgfusepath{stroke}%
\end{pgfscope}%
\begin{pgfscope}%
\pgfsetbuttcap%
\pgfsetmiterjoin%
\definecolor{currentfill}{rgb}{0.529412,0.462745,0.384314}%
\pgfsetfillcolor{currentfill}%
\pgfsetlinewidth{0.501875pt}%
\definecolor{currentstroke}{rgb}{0.000000,0.000000,0.000000}%
\pgfsetstrokecolor{currentstroke}%
\pgfsetdash{}{0pt}%
\pgfsys@defobject{currentmarker}{\pgfqpoint{-0.034722in}{-0.034722in}}{\pgfqpoint{0.034722in}{0.034722in}}{%
\pgfpathmoveto{\pgfqpoint{-0.000000in}{-0.034722in}}%
\pgfpathlineto{\pgfqpoint{0.034722in}{0.034722in}}%
\pgfpathlineto{\pgfqpoint{-0.034722in}{0.034722in}}%
\pgfpathclose%
\pgfusepath{stroke,fill}%
}%
\begin{pgfscope}%
\pgfsys@transformshift{0.728904in}{3.156250in}%
\pgfsys@useobject{currentmarker}{}%
\end{pgfscope}%
\end{pgfscope}%
\begin{pgfscope}%
\definecolor{textcolor}{rgb}{0.000000,0.000000,0.000000}%
\pgfsetstrokecolor{textcolor}%
\pgfsetfillcolor{textcolor}%
\pgftext[x=0.878904in,y=3.112500in,left,base]{\color{textcolor}\rmfamily\fontsize{9.000000}{10.800000}\selectfont greedy}%
\end{pgfscope}%
\begin{pgfscope}%
\pgfsetrectcap%
\pgfsetroundjoin%
\pgfsetlinewidth{1.003750pt}%
\definecolor{currentstroke}{rgb}{0.611765,0.568627,0.274510}%
\pgfsetstrokecolor{currentstroke}%
\pgfsetdash{}{0pt}%
\pgfpathmoveto{\pgfqpoint{0.603904in}{2.994450in}}%
\pgfpathlineto{\pgfqpoint{0.853904in}{2.994450in}}%
\pgfusepath{stroke}%
\end{pgfscope}%
\begin{pgfscope}%
\pgfsetbuttcap%
\pgfsetmiterjoin%
\definecolor{currentfill}{rgb}{0.611765,0.568627,0.274510}%
\pgfsetfillcolor{currentfill}%
\pgfsetlinewidth{0.501875pt}%
\definecolor{currentstroke}{rgb}{0.000000,0.000000,0.000000}%
\pgfsetstrokecolor{currentstroke}%
\pgfsetdash{}{0pt}%
\pgfsys@defobject{currentmarker}{\pgfqpoint{-0.034722in}{-0.034722in}}{\pgfqpoint{0.034722in}{0.034722in}}{%
\pgfpathmoveto{\pgfqpoint{-0.034722in}{0.000000in}}%
\pgfpathlineto{\pgfqpoint{0.034722in}{-0.034722in}}%
\pgfpathlineto{\pgfqpoint{0.034722in}{0.034722in}}%
\pgfpathclose%
\pgfusepath{stroke,fill}%
}%
\begin{pgfscope}%
\pgfsys@transformshift{0.728904in}{2.994450in}%
\pgfsys@useobject{currentmarker}{}%
\end{pgfscope}%
\end{pgfscope}%
\begin{pgfscope}%
\definecolor{textcolor}{rgb}{0.000000,0.000000,0.000000}%
\pgfsetstrokecolor{textcolor}%
\pgfsetfillcolor{textcolor}%
\pgftext[x=0.878904in,y=2.950700in,left,base]{\color{textcolor}\rmfamily\fontsize{9.000000}{10.800000}\selectfont metis}%
\end{pgfscope}%
\begin{pgfscope}%
\pgfsetrectcap%
\pgfsetroundjoin%
\pgfsetlinewidth{1.003750pt}%
\definecolor{currentstroke}{rgb}{0.780392,0.643137,0.254902}%
\pgfsetstrokecolor{currentstroke}%
\pgfsetdash{}{0pt}%
\pgfpathmoveto{\pgfqpoint{0.603904in}{2.832651in}}%
\pgfpathlineto{\pgfqpoint{0.853904in}{2.832651in}}%
\pgfusepath{stroke}%
\end{pgfscope}%
\begin{pgfscope}%
\pgfsetbuttcap%
\pgfsetmiterjoin%
\definecolor{currentfill}{rgb}{0.780392,0.643137,0.254902}%
\pgfsetfillcolor{currentfill}%
\pgfsetlinewidth{0.501875pt}%
\definecolor{currentstroke}{rgb}{0.000000,0.000000,0.000000}%
\pgfsetstrokecolor{currentstroke}%
\pgfsetdash{}{0pt}%
\pgfsys@defobject{currentmarker}{\pgfqpoint{-0.034722in}{-0.034722in}}{\pgfqpoint{0.034722in}{0.034722in}}{%
\pgfpathmoveto{\pgfqpoint{0.034722in}{-0.000000in}}%
\pgfpathlineto{\pgfqpoint{-0.034722in}{0.034722in}}%
\pgfpathlineto{\pgfqpoint{-0.034722in}{-0.034722in}}%
\pgfpathclose%
\pgfusepath{stroke,fill}%
}%
\begin{pgfscope}%
\pgfsys@transformshift{0.728904in}{2.832651in}%
\pgfsys@useobject{currentmarker}{}%
\end{pgfscope}%
\end{pgfscope}%
\begin{pgfscope}%
\definecolor{textcolor}{rgb}{0.000000,0.000000,0.000000}%
\pgfsetstrokecolor{textcolor}%
\pgfsetfillcolor{textcolor}%
\pgftext[x=0.878904in,y=2.788901in,left,base]{\color{textcolor}\rmfamily\fontsize{9.000000}{10.800000}\selectfont GN}%
\end{pgfscope}%
\begin{pgfscope}%
\pgfsetrectcap%
\pgfsetroundjoin%
\pgfsetlinewidth{1.003750pt}%
\definecolor{currentstroke}{rgb}{1.000000,0.694118,0.305882}%
\pgfsetstrokecolor{currentstroke}%
\pgfsetdash{}{0pt}%
\pgfpathmoveto{\pgfqpoint{0.603904in}{2.670851in}}%
\pgfpathlineto{\pgfqpoint{0.853904in}{2.670851in}}%
\pgfusepath{stroke}%
\end{pgfscope}%
\begin{pgfscope}%
\pgfsetbuttcap%
\pgfsetbeveljoin%
\definecolor{currentfill}{rgb}{1.000000,0.694118,0.305882}%
\pgfsetfillcolor{currentfill}%
\pgfsetlinewidth{0.501875pt}%
\definecolor{currentstroke}{rgb}{0.000000,0.000000,0.000000}%
\pgfsetstrokecolor{currentstroke}%
\pgfsetdash{}{0pt}%
\pgfsys@defobject{currentmarker}{\pgfqpoint{-0.033023in}{-0.028091in}}{\pgfqpoint{0.033023in}{0.034722in}}{%
\pgfpathmoveto{\pgfqpoint{0.000000in}{0.034722in}}%
\pgfpathlineto{\pgfqpoint{-0.007796in}{0.010730in}}%
\pgfpathlineto{\pgfqpoint{-0.033023in}{0.010730in}}%
\pgfpathlineto{\pgfqpoint{-0.012614in}{-0.004098in}}%
\pgfpathlineto{\pgfqpoint{-0.020409in}{-0.028091in}}%
\pgfpathlineto{\pgfqpoint{-0.000000in}{-0.013263in}}%
\pgfpathlineto{\pgfqpoint{0.020409in}{-0.028091in}}%
\pgfpathlineto{\pgfqpoint{0.012614in}{-0.004098in}}%
\pgfpathlineto{\pgfqpoint{0.033023in}{0.010730in}}%
\pgfpathlineto{\pgfqpoint{0.007796in}{0.010730in}}%
\pgfpathclose%
\pgfusepath{stroke,fill}%
}%
\begin{pgfscope}%
\pgfsys@transformshift{0.728904in}{2.670851in}%
\pgfsys@useobject{currentmarker}{}%
\end{pgfscope}%
\end{pgfscope}%
\begin{pgfscope}%
\definecolor{textcolor}{rgb}{0.000000,0.000000,0.000000}%
\pgfsetstrokecolor{textcolor}%
\pgfsetfillcolor{textcolor}%
\pgftext[x=0.878904in,y=2.627101in,left,base]{\color{textcolor}\rmfamily\fontsize{9.000000}{10.800000}\selectfont LG+Flow}%
\end{pgfscope}%
\end{pgfpicture}%
\makeatother%
\endgroup%

%	\caption{\label{fig:cubic-rank} Median max rank of the contraction tree found by various tensor-based methods, on tensor networks that count the number of vertex covers of 100 randomly-sampled cubic graphs with $n$ vertices. Our contribution \textbf{Line+Flow} finds lower max rank contraction trees than other methods when $n \geq 170$.}
%\end{figure}

\subsection{Unweighted Model Counting: Vertex Covers of Cubic Graphs}
\label{sec:tensors:experiments:cubic}

% (i.e., the number of sets of vertices where every edge of the graph is incident to at least one vertex) 
We first compare on benchmarks that count the number of vertex covers of randomly-generated cubic graphs \cite{KCMR18}. For each number of vertices $n \in \{50, 60, 70, \cdots, 250\}$ we randomly sample 100 connected cubic graphs using a Monte Carlo procedure \cite{VL05}. These benchmarks are monotone 2-CNF formulas in which every variable appears 3 times. We run each tool once on each benchmark with a timeout of 1000 seconds and record the wall-clock time taken.

Results on the runtime performance for these benchmarks are summarized in Figure \ref{fig:cubic-time}. For ease of presentation, we display only the best-performing of the \textbf{LG} and \textbf{FT} implementations: \textbf{LG+Flow}. We observe that tensor-based methods are fastest when $n \geq 110$. On large graphs ($n \geq 170$) our contribution \textbf{LG+Flow} is fastest and able to find the lowest max rank contraction trees. \textbf{LG+Flow} is the only implementation able to solve at least 50 benchmarks within 1000 seconds when $n$ is $220$. We conclude that tensor-network-based approaches outperform state-of-the-art unweighted model counters on these benchmarks.

Results on the structural properties of these benchmarks are summarized in Figure \ref{fig:vertex-cover-width}.
% Note that both \textbf{LG} and \textbf{FT} can be used to find carving decompositions on these benchmarks; although \textbf{FT} requires preprocessing to factor all tensors of order 4 or higher, each vertex in each cubic graph has exactly three incident edges and so there is no factoring required. 
For most large graphs $G$, we observe that the carving width of $G$ is smaller than the treewidth of $G$ which is smaller that the treewidth of $\Line{G}$. 
% Note that the carving width of $G$ is indeed smaller than the upper bound guaranteed by Theorem \ref{thm:carving-equiv-tree} of $width_t(\Line{G})+1$. In fact, the carving width of $G$ is smaller than the treewidth of $G$.

% width of the carving decompositions of $G$ found by \textbf{LG} are indeed smaller than the upper bound guaranteed by Theorem \ref{thm:carving-equiv-tree} of the tree decomposition width plus 1.

% In Figure \ref{fig:cubic-rank}, we present the median max rank of the contraction tree ultimately returned by each tensor-based method when counting each benchmark. We observe that \textbf{line-Flow} is consistently able to find better contraction trees than the other methods when $n \geq 180$. The flat performance of \textbf{line-Flow} when $n$ is between $100$ and $200$ is a result of the algorithm halting the online search for a contraction tree to immediately perform the contraction. % If \textbf{line-Flow} continues to improve a contraction tree for the full 5 minutes, we see (from the points labeled \textbf{line-Flow} [5 min]) that it consistently finds better contraction trees.

\begin{figure}[t]
	\centering
	%% Creator: Matplotlib, PGF backend
%%
%% To include the figure in your LaTeX document, write
%%   \input{<filename>.pgf}
%%
%% Make sure the required packages are loaded in your preamble
%%   \usepackage{pgf}
%%
%% Figures using additional raster images can only be included by \input if
%% they are in the same directory as the main LaTeX file. For loading figures
%% from other directories you can use the `import` package
%%   \usepackage{import}
%% and then include the figures with
%%   \import{<path to file>}{<filename>.pgf}
%%
%% Matplotlib used the following preamble
%%   \usepackage[utf8x]{inputenc}
%%   \usepackage[T1]{fontenc}
%%
\begingroup%
\makeatletter%
\begin{pgfpicture}%
\pgfpathrectangle{\pgfpointorigin}{\pgfqpoint{4.497025in}{2.798215in}}%
\pgfusepath{use as bounding box, clip}%
\begin{pgfscope}%
\pgfsetbuttcap%
\pgfsetmiterjoin%
\definecolor{currentfill}{rgb}{1.000000,1.000000,1.000000}%
\pgfsetfillcolor{currentfill}%
\pgfsetlinewidth{0.000000pt}%
\definecolor{currentstroke}{rgb}{1.000000,1.000000,1.000000}%
\pgfsetstrokecolor{currentstroke}%
\pgfsetdash{}{0pt}%
\pgfpathmoveto{\pgfqpoint{0.000000in}{0.000000in}}%
\pgfpathlineto{\pgfqpoint{4.497025in}{0.000000in}}%
\pgfpathlineto{\pgfqpoint{4.497025in}{2.798215in}}%
\pgfpathlineto{\pgfqpoint{0.000000in}{2.798215in}}%
\pgfpathclose%
\pgfusepath{fill}%
\end{pgfscope}%
\begin{pgfscope}%
\pgfsetbuttcap%
\pgfsetmiterjoin%
\definecolor{currentfill}{rgb}{1.000000,1.000000,1.000000}%
\pgfsetfillcolor{currentfill}%
\pgfsetlinewidth{0.000000pt}%
\definecolor{currentstroke}{rgb}{0.000000,0.000000,0.000000}%
\pgfsetstrokecolor{currentstroke}%
\pgfsetstrokeopacity{0.000000}%
\pgfsetdash{}{0pt}%
\pgfpathmoveto{\pgfqpoint{0.698576in}{0.535823in}}%
\pgfpathlineto{\pgfqpoint{4.312025in}{0.535823in}}%
\pgfpathlineto{\pgfqpoint{4.312025in}{2.605170in}}%
\pgfpathlineto{\pgfqpoint{0.698576in}{2.605170in}}%
\pgfpathclose%
\pgfusepath{fill}%
\end{pgfscope}%
\begin{pgfscope}%
\pgfsetbuttcap%
\pgfsetroundjoin%
\definecolor{currentfill}{rgb}{0.000000,0.000000,0.000000}%
\pgfsetfillcolor{currentfill}%
\pgfsetlinewidth{0.803000pt}%
\definecolor{currentstroke}{rgb}{0.000000,0.000000,0.000000}%
\pgfsetstrokecolor{currentstroke}%
\pgfsetdash{}{0pt}%
\pgfsys@defobject{currentmarker}{\pgfqpoint{0.000000in}{-0.048611in}}{\pgfqpoint{0.000000in}{0.000000in}}{%
\pgfpathmoveto{\pgfqpoint{0.000000in}{0.000000in}}%
\pgfpathlineto{\pgfqpoint{0.000000in}{-0.048611in}}%
\pgfusepath{stroke,fill}%
}%
\begin{pgfscope}%
\pgfsys@transformshift{0.698576in}{0.535823in}%
\pgfsys@useobject{currentmarker}{}%
\end{pgfscope}%
\end{pgfscope}%
\begin{pgfscope}%
\definecolor{textcolor}{rgb}{0.000000,0.000000,0.000000}%
\pgfsetstrokecolor{textcolor}%
\pgfsetfillcolor{textcolor}%
\pgftext[x=0.698576in,y=0.438600in,,top]{\color{textcolor}\rmfamily\fontsize{9.000000}{10.800000}\selectfont \(\displaystyle 0\)}%
\end{pgfscope}%
\begin{pgfscope}%
\pgfsetbuttcap%
\pgfsetroundjoin%
\definecolor{currentfill}{rgb}{0.000000,0.000000,0.000000}%
\pgfsetfillcolor{currentfill}%
\pgfsetlinewidth{0.803000pt}%
\definecolor{currentstroke}{rgb}{0.000000,0.000000,0.000000}%
\pgfsetstrokecolor{currentstroke}%
\pgfsetdash{}{0pt}%
\pgfsys@defobject{currentmarker}{\pgfqpoint{0.000000in}{-0.048611in}}{\pgfqpoint{0.000000in}{0.000000in}}{%
\pgfpathmoveto{\pgfqpoint{0.000000in}{0.000000in}}%
\pgfpathlineto{\pgfqpoint{0.000000in}{-0.048611in}}%
\pgfusepath{stroke,fill}%
}%
\begin{pgfscope}%
\pgfsys@transformshift{1.355566in}{0.535823in}%
\pgfsys@useobject{currentmarker}{}%
\end{pgfscope}%
\end{pgfscope}%
\begin{pgfscope}%
\definecolor{textcolor}{rgb}{0.000000,0.000000,0.000000}%
\pgfsetstrokecolor{textcolor}%
\pgfsetfillcolor{textcolor}%
\pgftext[x=1.355566in,y=0.438600in,,top]{\color{textcolor}\rmfamily\fontsize{9.000000}{10.800000}\selectfont \(\displaystyle 200\)}%
\end{pgfscope}%
\begin{pgfscope}%
\pgfsetbuttcap%
\pgfsetroundjoin%
\definecolor{currentfill}{rgb}{0.000000,0.000000,0.000000}%
\pgfsetfillcolor{currentfill}%
\pgfsetlinewidth{0.803000pt}%
\definecolor{currentstroke}{rgb}{0.000000,0.000000,0.000000}%
\pgfsetstrokecolor{currentstroke}%
\pgfsetdash{}{0pt}%
\pgfsys@defobject{currentmarker}{\pgfqpoint{0.000000in}{-0.048611in}}{\pgfqpoint{0.000000in}{0.000000in}}{%
\pgfpathmoveto{\pgfqpoint{0.000000in}{0.000000in}}%
\pgfpathlineto{\pgfqpoint{0.000000in}{-0.048611in}}%
\pgfusepath{stroke,fill}%
}%
\begin{pgfscope}%
\pgfsys@transformshift{2.012557in}{0.535823in}%
\pgfsys@useobject{currentmarker}{}%
\end{pgfscope}%
\end{pgfscope}%
\begin{pgfscope}%
\definecolor{textcolor}{rgb}{0.000000,0.000000,0.000000}%
\pgfsetstrokecolor{textcolor}%
\pgfsetfillcolor{textcolor}%
\pgftext[x=2.012557in,y=0.438600in,,top]{\color{textcolor}\rmfamily\fontsize{9.000000}{10.800000}\selectfont \(\displaystyle 400\)}%
\end{pgfscope}%
\begin{pgfscope}%
\pgfsetbuttcap%
\pgfsetroundjoin%
\definecolor{currentfill}{rgb}{0.000000,0.000000,0.000000}%
\pgfsetfillcolor{currentfill}%
\pgfsetlinewidth{0.803000pt}%
\definecolor{currentstroke}{rgb}{0.000000,0.000000,0.000000}%
\pgfsetstrokecolor{currentstroke}%
\pgfsetdash{}{0pt}%
\pgfsys@defobject{currentmarker}{\pgfqpoint{0.000000in}{-0.048611in}}{\pgfqpoint{0.000000in}{0.000000in}}{%
\pgfpathmoveto{\pgfqpoint{0.000000in}{0.000000in}}%
\pgfpathlineto{\pgfqpoint{0.000000in}{-0.048611in}}%
\pgfusepath{stroke,fill}%
}%
\begin{pgfscope}%
\pgfsys@transformshift{2.669548in}{0.535823in}%
\pgfsys@useobject{currentmarker}{}%
\end{pgfscope}%
\end{pgfscope}%
\begin{pgfscope}%
\definecolor{textcolor}{rgb}{0.000000,0.000000,0.000000}%
\pgfsetstrokecolor{textcolor}%
\pgfsetfillcolor{textcolor}%
\pgftext[x=2.669548in,y=0.438600in,,top]{\color{textcolor}\rmfamily\fontsize{9.000000}{10.800000}\selectfont \(\displaystyle 600\)}%
\end{pgfscope}%
\begin{pgfscope}%
\pgfsetbuttcap%
\pgfsetroundjoin%
\definecolor{currentfill}{rgb}{0.000000,0.000000,0.000000}%
\pgfsetfillcolor{currentfill}%
\pgfsetlinewidth{0.803000pt}%
\definecolor{currentstroke}{rgb}{0.000000,0.000000,0.000000}%
\pgfsetstrokecolor{currentstroke}%
\pgfsetdash{}{0pt}%
\pgfsys@defobject{currentmarker}{\pgfqpoint{0.000000in}{-0.048611in}}{\pgfqpoint{0.000000in}{0.000000in}}{%
\pgfpathmoveto{\pgfqpoint{0.000000in}{0.000000in}}%
\pgfpathlineto{\pgfqpoint{0.000000in}{-0.048611in}}%
\pgfusepath{stroke,fill}%
}%
\begin{pgfscope}%
\pgfsys@transformshift{3.326539in}{0.535823in}%
\pgfsys@useobject{currentmarker}{}%
\end{pgfscope}%
\end{pgfscope}%
\begin{pgfscope}%
\definecolor{textcolor}{rgb}{0.000000,0.000000,0.000000}%
\pgfsetstrokecolor{textcolor}%
\pgfsetfillcolor{textcolor}%
\pgftext[x=3.326539in,y=0.438600in,,top]{\color{textcolor}\rmfamily\fontsize{9.000000}{10.800000}\selectfont \(\displaystyle 800\)}%
\end{pgfscope}%
\begin{pgfscope}%
\pgfsetbuttcap%
\pgfsetroundjoin%
\definecolor{currentfill}{rgb}{0.000000,0.000000,0.000000}%
\pgfsetfillcolor{currentfill}%
\pgfsetlinewidth{0.803000pt}%
\definecolor{currentstroke}{rgb}{0.000000,0.000000,0.000000}%
\pgfsetstrokecolor{currentstroke}%
\pgfsetdash{}{0pt}%
\pgfsys@defobject{currentmarker}{\pgfqpoint{0.000000in}{-0.048611in}}{\pgfqpoint{0.000000in}{0.000000in}}{%
\pgfpathmoveto{\pgfqpoint{0.000000in}{0.000000in}}%
\pgfpathlineto{\pgfqpoint{0.000000in}{-0.048611in}}%
\pgfusepath{stroke,fill}%
}%
\begin{pgfscope}%
\pgfsys@transformshift{3.983530in}{0.535823in}%
\pgfsys@useobject{currentmarker}{}%
\end{pgfscope}%
\end{pgfscope}%
\begin{pgfscope}%
\definecolor{textcolor}{rgb}{0.000000,0.000000,0.000000}%
\pgfsetstrokecolor{textcolor}%
\pgfsetfillcolor{textcolor}%
\pgftext[x=3.983530in,y=0.438600in,,top]{\color{textcolor}\rmfamily\fontsize{9.000000}{10.800000}\selectfont \(\displaystyle 1000\)}%
\end{pgfscope}%
\begin{pgfscope}%
\definecolor{textcolor}{rgb}{0.000000,0.000000,0.000000}%
\pgfsetstrokecolor{textcolor}%
\pgfsetfillcolor{textcolor}%
\pgftext[x=2.505300in,y=0.272655in,,top]{\color{textcolor}\rmfamily\fontsize{10.000000}{12.000000}\selectfont Solved instance}%
\end{pgfscope}%
\begin{pgfscope}%
\pgfsetbuttcap%
\pgfsetroundjoin%
\definecolor{currentfill}{rgb}{0.000000,0.000000,0.000000}%
\pgfsetfillcolor{currentfill}%
\pgfsetlinewidth{0.803000pt}%
\definecolor{currentstroke}{rgb}{0.000000,0.000000,0.000000}%
\pgfsetstrokecolor{currentstroke}%
\pgfsetdash{}{0pt}%
\pgfsys@defobject{currentmarker}{\pgfqpoint{-0.048611in}{0.000000in}}{\pgfqpoint{0.000000in}{0.000000in}}{%
\pgfpathmoveto{\pgfqpoint{0.000000in}{0.000000in}}%
\pgfpathlineto{\pgfqpoint{-0.048611in}{0.000000in}}%
\pgfusepath{stroke,fill}%
}%
\begin{pgfscope}%
\pgfsys@transformshift{0.698576in}{0.535823in}%
\pgfsys@useobject{currentmarker}{}%
\end{pgfscope}%
\end{pgfscope}%
\begin{pgfscope}%
\definecolor{textcolor}{rgb}{0.000000,0.000000,0.000000}%
\pgfsetstrokecolor{textcolor}%
\pgfsetfillcolor{textcolor}%
\pgftext[x=0.537118in,y=0.492778in,left,base]{\color{textcolor}\rmfamily\fontsize{9.000000}{10.800000}\selectfont \(\displaystyle 0\)}%
\end{pgfscope}%
\begin{pgfscope}%
\pgfsetbuttcap%
\pgfsetroundjoin%
\definecolor{currentfill}{rgb}{0.000000,0.000000,0.000000}%
\pgfsetfillcolor{currentfill}%
\pgfsetlinewidth{0.803000pt}%
\definecolor{currentstroke}{rgb}{0.000000,0.000000,0.000000}%
\pgfsetstrokecolor{currentstroke}%
\pgfsetdash{}{0pt}%
\pgfsys@defobject{currentmarker}{\pgfqpoint{-0.048611in}{0.000000in}}{\pgfqpoint{0.000000in}{0.000000in}}{%
\pgfpathmoveto{\pgfqpoint{0.000000in}{0.000000in}}%
\pgfpathlineto{\pgfqpoint{-0.048611in}{0.000000in}}%
\pgfusepath{stroke,fill}%
}%
\begin{pgfscope}%
\pgfsys@transformshift{0.698576in}{0.949692in}%
\pgfsys@useobject{currentmarker}{}%
\end{pgfscope}%
\end{pgfscope}%
\begin{pgfscope}%
\definecolor{textcolor}{rgb}{0.000000,0.000000,0.000000}%
\pgfsetstrokecolor{textcolor}%
\pgfsetfillcolor{textcolor}%
\pgftext[x=0.408646in,y=0.906647in,left,base]{\color{textcolor}\rmfamily\fontsize{9.000000}{10.800000}\selectfont \(\displaystyle 200\)}%
\end{pgfscope}%
\begin{pgfscope}%
\pgfsetbuttcap%
\pgfsetroundjoin%
\definecolor{currentfill}{rgb}{0.000000,0.000000,0.000000}%
\pgfsetfillcolor{currentfill}%
\pgfsetlinewidth{0.803000pt}%
\definecolor{currentstroke}{rgb}{0.000000,0.000000,0.000000}%
\pgfsetstrokecolor{currentstroke}%
\pgfsetdash{}{0pt}%
\pgfsys@defobject{currentmarker}{\pgfqpoint{-0.048611in}{0.000000in}}{\pgfqpoint{0.000000in}{0.000000in}}{%
\pgfpathmoveto{\pgfqpoint{0.000000in}{0.000000in}}%
\pgfpathlineto{\pgfqpoint{-0.048611in}{0.000000in}}%
\pgfusepath{stroke,fill}%
}%
\begin{pgfscope}%
\pgfsys@transformshift{0.698576in}{1.363562in}%
\pgfsys@useobject{currentmarker}{}%
\end{pgfscope}%
\end{pgfscope}%
\begin{pgfscope}%
\definecolor{textcolor}{rgb}{0.000000,0.000000,0.000000}%
\pgfsetstrokecolor{textcolor}%
\pgfsetfillcolor{textcolor}%
\pgftext[x=0.408646in,y=1.320516in,left,base]{\color{textcolor}\rmfamily\fontsize{9.000000}{10.800000}\selectfont \(\displaystyle 400\)}%
\end{pgfscope}%
\begin{pgfscope}%
\pgfsetbuttcap%
\pgfsetroundjoin%
\definecolor{currentfill}{rgb}{0.000000,0.000000,0.000000}%
\pgfsetfillcolor{currentfill}%
\pgfsetlinewidth{0.803000pt}%
\definecolor{currentstroke}{rgb}{0.000000,0.000000,0.000000}%
\pgfsetstrokecolor{currentstroke}%
\pgfsetdash{}{0pt}%
\pgfsys@defobject{currentmarker}{\pgfqpoint{-0.048611in}{0.000000in}}{\pgfqpoint{0.000000in}{0.000000in}}{%
\pgfpathmoveto{\pgfqpoint{0.000000in}{0.000000in}}%
\pgfpathlineto{\pgfqpoint{-0.048611in}{0.000000in}}%
\pgfusepath{stroke,fill}%
}%
\begin{pgfscope}%
\pgfsys@transformshift{0.698576in}{1.777431in}%
\pgfsys@useobject{currentmarker}{}%
\end{pgfscope}%
\end{pgfscope}%
\begin{pgfscope}%
\definecolor{textcolor}{rgb}{0.000000,0.000000,0.000000}%
\pgfsetstrokecolor{textcolor}%
\pgfsetfillcolor{textcolor}%
\pgftext[x=0.408646in,y=1.734386in,left,base]{\color{textcolor}\rmfamily\fontsize{9.000000}{10.800000}\selectfont \(\displaystyle 600\)}%
\end{pgfscope}%
\begin{pgfscope}%
\pgfsetbuttcap%
\pgfsetroundjoin%
\definecolor{currentfill}{rgb}{0.000000,0.000000,0.000000}%
\pgfsetfillcolor{currentfill}%
\pgfsetlinewidth{0.803000pt}%
\definecolor{currentstroke}{rgb}{0.000000,0.000000,0.000000}%
\pgfsetstrokecolor{currentstroke}%
\pgfsetdash{}{0pt}%
\pgfsys@defobject{currentmarker}{\pgfqpoint{-0.048611in}{0.000000in}}{\pgfqpoint{0.000000in}{0.000000in}}{%
\pgfpathmoveto{\pgfqpoint{0.000000in}{0.000000in}}%
\pgfpathlineto{\pgfqpoint{-0.048611in}{0.000000in}}%
\pgfusepath{stroke,fill}%
}%
\begin{pgfscope}%
\pgfsys@transformshift{0.698576in}{2.191300in}%
\pgfsys@useobject{currentmarker}{}%
\end{pgfscope}%
\end{pgfscope}%
\begin{pgfscope}%
\definecolor{textcolor}{rgb}{0.000000,0.000000,0.000000}%
\pgfsetstrokecolor{textcolor}%
\pgfsetfillcolor{textcolor}%
\pgftext[x=0.408646in,y=2.148255in,left,base]{\color{textcolor}\rmfamily\fontsize{9.000000}{10.800000}\selectfont \(\displaystyle 800\)}%
\end{pgfscope}%
\begin{pgfscope}%
\pgfsetbuttcap%
\pgfsetroundjoin%
\definecolor{currentfill}{rgb}{0.000000,0.000000,0.000000}%
\pgfsetfillcolor{currentfill}%
\pgfsetlinewidth{0.803000pt}%
\definecolor{currentstroke}{rgb}{0.000000,0.000000,0.000000}%
\pgfsetstrokecolor{currentstroke}%
\pgfsetdash{}{0pt}%
\pgfsys@defobject{currentmarker}{\pgfqpoint{-0.048611in}{0.000000in}}{\pgfqpoint{0.000000in}{0.000000in}}{%
\pgfpathmoveto{\pgfqpoint{0.000000in}{0.000000in}}%
\pgfpathlineto{\pgfqpoint{-0.048611in}{0.000000in}}%
\pgfusepath{stroke,fill}%
}%
\begin{pgfscope}%
\pgfsys@transformshift{0.698576in}{2.605170in}%
\pgfsys@useobject{currentmarker}{}%
\end{pgfscope}%
\end{pgfscope}%
\begin{pgfscope}%
\definecolor{textcolor}{rgb}{0.000000,0.000000,0.000000}%
\pgfsetstrokecolor{textcolor}%
\pgfsetfillcolor{textcolor}%
\pgftext[x=0.344411in,y=2.562125in,left,base]{\color{textcolor}\rmfamily\fontsize{9.000000}{10.800000}\selectfont \(\displaystyle 1000\)}%
\end{pgfscope}%
\begin{pgfscope}%
\definecolor{textcolor}{rgb}{0.000000,0.000000,0.000000}%
\pgfsetstrokecolor{textcolor}%
\pgfsetfillcolor{textcolor}%
\pgftext[x=0.288855in,y=1.570496in,,bottom,rotate=90.000000]{\color{textcolor}\rmfamily\fontsize{10.000000}{12.000000}\selectfont Solving time (s)}%
\end{pgfscope}%
\begin{pgfscope}%
\pgfpathrectangle{\pgfqpoint{0.698576in}{0.535823in}}{\pgfqpoint{3.613449in}{2.069347in}}%
\pgfusepath{clip}%
\pgfsetbuttcap%
\pgfsetroundjoin%
\pgfsetlinewidth{2.007500pt}%
\definecolor{currentstroke}{rgb}{1.000000,0.843137,0.000000}%
\pgfsetstrokecolor{currentstroke}%
\pgfsetdash{{7.400000pt}{3.200000pt}}{0.000000pt}%
\pgfpathmoveto{\pgfqpoint{0.698576in}{0.537668in}}%
\pgfpathlineto{\pgfqpoint{0.961372in}{0.541093in}}%
\pgfpathlineto{\pgfqpoint{1.158469in}{0.546604in}}%
\pgfpathlineto{\pgfqpoint{1.181464in}{0.547412in}}%
\pgfpathlineto{\pgfqpoint{1.197889in}{0.548113in}}%
\pgfpathlineto{\pgfqpoint{1.211028in}{0.549234in}}%
\pgfpathlineto{\pgfqpoint{1.240593in}{0.550602in}}%
\pgfpathlineto{\pgfqpoint{1.250448in}{0.552595in}}%
\pgfpathlineto{\pgfqpoint{1.312862in}{0.554870in}}%
\pgfpathlineto{\pgfqpoint{1.358851in}{0.555992in}}%
\pgfpathlineto{\pgfqpoint{1.378561in}{0.557016in}}%
\pgfpathlineto{\pgfqpoint{1.414696in}{0.561978in}}%
\pgfpathlineto{\pgfqpoint{1.421265in}{0.562413in}}%
\pgfpathlineto{\pgfqpoint{1.431120in}{0.565305in}}%
\pgfpathlineto{\pgfqpoint{1.434405in}{0.569289in}}%
\pgfpathlineto{\pgfqpoint{1.450830in}{0.570332in}}%
\pgfpathlineto{\pgfqpoint{1.467255in}{0.573733in}}%
\pgfpathlineto{\pgfqpoint{1.473825in}{0.573821in}}%
\pgfpathlineto{\pgfqpoint{1.480395in}{0.577167in}}%
\pgfpathlineto{\pgfqpoint{1.486965in}{0.578404in}}%
\pgfpathlineto{\pgfqpoint{1.509959in}{0.584111in}}%
\pgfpathlineto{\pgfqpoint{1.539524in}{0.587548in}}%
\pgfpathlineto{\pgfqpoint{1.546094in}{0.588365in}}%
\pgfpathlineto{\pgfqpoint{1.552664in}{0.589781in}}%
\pgfpathlineto{\pgfqpoint{1.569088in}{0.591374in}}%
\pgfpathlineto{\pgfqpoint{1.572373in}{0.592769in}}%
\pgfpathlineto{\pgfqpoint{1.575658in}{0.595435in}}%
\pgfpathlineto{\pgfqpoint{1.592083in}{0.598511in}}%
\pgfpathlineto{\pgfqpoint{1.595368in}{0.600374in}}%
\pgfpathlineto{\pgfqpoint{1.601938in}{0.600532in}}%
\pgfpathlineto{\pgfqpoint{1.608508in}{0.603538in}}%
\pgfpathlineto{\pgfqpoint{1.624933in}{0.609140in}}%
\pgfpathlineto{\pgfqpoint{1.631503in}{0.610770in}}%
\pgfpathlineto{\pgfqpoint{1.634787in}{0.620640in}}%
\pgfpathlineto{\pgfqpoint{1.644642in}{0.622524in}}%
\pgfpathlineto{\pgfqpoint{1.654497in}{0.624170in}}%
\pgfpathlineto{\pgfqpoint{1.661067in}{0.625610in}}%
\pgfpathlineto{\pgfqpoint{1.674207in}{0.628844in}}%
\pgfpathlineto{\pgfqpoint{1.693917in}{0.631040in}}%
\pgfpathlineto{\pgfqpoint{1.700487in}{0.632886in}}%
\pgfpathlineto{\pgfqpoint{1.703772in}{0.633185in}}%
\pgfpathlineto{\pgfqpoint{1.710341in}{0.636314in}}%
\pgfpathlineto{\pgfqpoint{1.720196in}{0.638145in}}%
\pgfpathlineto{\pgfqpoint{1.723481in}{0.640078in}}%
\pgfpathlineto{\pgfqpoint{1.726766in}{0.645391in}}%
\pgfpathlineto{\pgfqpoint{1.730051in}{0.654089in}}%
\pgfpathlineto{\pgfqpoint{1.733336in}{0.657001in}}%
\pgfpathlineto{\pgfqpoint{1.736621in}{0.661647in}}%
\pgfpathlineto{\pgfqpoint{1.749761in}{0.665369in}}%
\pgfpathlineto{\pgfqpoint{1.753046in}{0.667048in}}%
\pgfpathlineto{\pgfqpoint{1.759616in}{0.667491in}}%
\pgfpathlineto{\pgfqpoint{1.762901in}{0.672354in}}%
\pgfpathlineto{\pgfqpoint{1.766186in}{0.672416in}}%
\pgfpathlineto{\pgfqpoint{1.769471in}{0.673697in}}%
\pgfpathlineto{\pgfqpoint{1.772756in}{0.673740in}}%
\pgfpathlineto{\pgfqpoint{1.779325in}{0.683259in}}%
\pgfpathlineto{\pgfqpoint{1.782610in}{0.694339in}}%
\pgfpathlineto{\pgfqpoint{1.785895in}{0.701125in}}%
\pgfpathlineto{\pgfqpoint{1.812175in}{0.703918in}}%
\pgfpathlineto{\pgfqpoint{1.822030in}{0.705995in}}%
\pgfpathlineto{\pgfqpoint{1.828600in}{0.706830in}}%
\pgfpathlineto{\pgfqpoint{1.831885in}{0.709619in}}%
\pgfpathlineto{\pgfqpoint{1.848309in}{0.714912in}}%
\pgfpathlineto{\pgfqpoint{1.851594in}{0.725435in}}%
\pgfpathlineto{\pgfqpoint{1.858164in}{0.728313in}}%
\pgfpathlineto{\pgfqpoint{1.864734in}{0.729527in}}%
\pgfpathlineto{\pgfqpoint{1.868019in}{0.732725in}}%
\pgfpathlineto{\pgfqpoint{1.871304in}{0.733397in}}%
\pgfpathlineto{\pgfqpoint{1.874589in}{0.735462in}}%
\pgfpathlineto{\pgfqpoint{1.881159in}{0.736830in}}%
\pgfpathlineto{\pgfqpoint{1.887729in}{0.739861in}}%
\pgfpathlineto{\pgfqpoint{1.894299in}{0.740549in}}%
\pgfpathlineto{\pgfqpoint{1.897584in}{0.745866in}}%
\pgfpathlineto{\pgfqpoint{1.907439in}{0.749599in}}%
\pgfpathlineto{\pgfqpoint{1.917294in}{0.750613in}}%
\pgfpathlineto{\pgfqpoint{1.920578in}{0.753845in}}%
\pgfpathlineto{\pgfqpoint{1.923863in}{0.755195in}}%
\pgfpathlineto{\pgfqpoint{1.927148in}{0.759242in}}%
\pgfpathlineto{\pgfqpoint{1.937003in}{0.761365in}}%
\pgfpathlineto{\pgfqpoint{1.940288in}{0.782237in}}%
\pgfpathlineto{\pgfqpoint{1.943573in}{0.784533in}}%
\pgfpathlineto{\pgfqpoint{1.953428in}{0.786286in}}%
\pgfpathlineto{\pgfqpoint{1.956713in}{0.786936in}}%
\pgfpathlineto{\pgfqpoint{1.959998in}{0.790188in}}%
\pgfpathlineto{\pgfqpoint{1.966568in}{0.800289in}}%
\pgfpathlineto{\pgfqpoint{1.969853in}{0.838644in}}%
\pgfpathlineto{\pgfqpoint{1.973138in}{0.843170in}}%
\pgfpathlineto{\pgfqpoint{1.976423in}{0.845466in}}%
\pgfpathlineto{\pgfqpoint{1.982993in}{0.846335in}}%
\pgfpathlineto{\pgfqpoint{1.986278in}{0.854333in}}%
\pgfpathlineto{\pgfqpoint{1.992847in}{0.854731in}}%
\pgfpathlineto{\pgfqpoint{1.996132in}{0.855630in}}%
\pgfpathlineto{\pgfqpoint{1.999417in}{0.872223in}}%
\pgfpathlineto{\pgfqpoint{2.002702in}{0.873516in}}%
\pgfpathlineto{\pgfqpoint{2.005987in}{0.890213in}}%
\pgfpathlineto{\pgfqpoint{2.009272in}{0.942044in}}%
\pgfpathlineto{\pgfqpoint{2.015842in}{0.960102in}}%
\pgfpathlineto{\pgfqpoint{2.022412in}{0.964577in}}%
\pgfpathlineto{\pgfqpoint{2.025697in}{0.981003in}}%
\pgfpathlineto{\pgfqpoint{2.035552in}{0.985860in}}%
\pgfpathlineto{\pgfqpoint{2.038837in}{0.987535in}}%
\pgfpathlineto{\pgfqpoint{2.045407in}{0.994918in}}%
\pgfpathlineto{\pgfqpoint{2.051977in}{0.995693in}}%
\pgfpathlineto{\pgfqpoint{2.058547in}{1.010195in}}%
\pgfpathlineto{\pgfqpoint{2.061831in}{1.010860in}}%
\pgfpathlineto{\pgfqpoint{2.065116in}{1.016058in}}%
\pgfpathlineto{\pgfqpoint{2.068401in}{1.023383in}}%
\pgfpathlineto{\pgfqpoint{2.071686in}{1.024985in}}%
\pgfpathlineto{\pgfqpoint{2.074971in}{1.032116in}}%
\pgfpathlineto{\pgfqpoint{2.078256in}{1.036102in}}%
\pgfpathlineto{\pgfqpoint{2.081541in}{1.037372in}}%
\pgfpathlineto{\pgfqpoint{2.084826in}{1.042979in}}%
\pgfpathlineto{\pgfqpoint{2.094681in}{1.067544in}}%
\pgfpathlineto{\pgfqpoint{2.097966in}{1.106364in}}%
\pgfpathlineto{\pgfqpoint{2.101251in}{1.123468in}}%
\pgfpathlineto{\pgfqpoint{2.104536in}{1.130259in}}%
\pgfpathlineto{\pgfqpoint{2.107821in}{1.132940in}}%
\pgfpathlineto{\pgfqpoint{2.111106in}{1.198645in}}%
\pgfpathlineto{\pgfqpoint{2.117676in}{1.231660in}}%
\pgfpathlineto{\pgfqpoint{2.120961in}{1.239423in}}%
\pgfpathlineto{\pgfqpoint{2.124246in}{1.250831in}}%
\pgfpathlineto{\pgfqpoint{2.127531in}{1.251808in}}%
\pgfpathlineto{\pgfqpoint{2.130816in}{1.258906in}}%
\pgfpathlineto{\pgfqpoint{2.134100in}{1.284129in}}%
\pgfpathlineto{\pgfqpoint{2.137385in}{1.291674in}}%
\pgfpathlineto{\pgfqpoint{2.140670in}{1.310900in}}%
\pgfpathlineto{\pgfqpoint{2.143955in}{1.517067in}}%
\pgfpathlineto{\pgfqpoint{2.150525in}{1.548318in}}%
\pgfpathlineto{\pgfqpoint{2.153810in}{1.550571in}}%
\pgfpathlineto{\pgfqpoint{2.160380in}{1.573741in}}%
\pgfpathlineto{\pgfqpoint{2.163665in}{1.578423in}}%
\pgfpathlineto{\pgfqpoint{2.166950in}{1.579869in}}%
\pgfpathlineto{\pgfqpoint{2.173520in}{1.606732in}}%
\pgfpathlineto{\pgfqpoint{2.180090in}{1.618736in}}%
\pgfpathlineto{\pgfqpoint{2.183375in}{1.618961in}}%
\pgfpathlineto{\pgfqpoint{2.186660in}{1.622537in}}%
\pgfpathlineto{\pgfqpoint{2.189945in}{1.623284in}}%
\pgfpathlineto{\pgfqpoint{2.193230in}{1.627727in}}%
\pgfpathlineto{\pgfqpoint{2.199800in}{1.641141in}}%
\pgfpathlineto{\pgfqpoint{2.203085in}{1.642772in}}%
\pgfpathlineto{\pgfqpoint{2.206369in}{1.666024in}}%
\pgfpathlineto{\pgfqpoint{2.209654in}{1.700584in}}%
\pgfpathlineto{\pgfqpoint{2.212939in}{1.701754in}}%
\pgfpathlineto{\pgfqpoint{2.216224in}{1.771561in}}%
\pgfpathlineto{\pgfqpoint{2.222794in}{1.777309in}}%
\pgfpathlineto{\pgfqpoint{2.226079in}{1.778834in}}%
\pgfpathlineto{\pgfqpoint{2.235934in}{1.808568in}}%
\pgfpathlineto{\pgfqpoint{2.239219in}{1.828596in}}%
\pgfpathlineto{\pgfqpoint{2.245789in}{1.936226in}}%
\pgfpathlineto{\pgfqpoint{2.249074in}{1.944218in}}%
\pgfpathlineto{\pgfqpoint{2.252359in}{1.944657in}}%
\pgfpathlineto{\pgfqpoint{2.258929in}{1.998212in}}%
\pgfpathlineto{\pgfqpoint{2.262214in}{2.002732in}}%
\pgfpathlineto{\pgfqpoint{2.265499in}{2.014670in}}%
\pgfpathlineto{\pgfqpoint{2.268784in}{2.047186in}}%
\pgfpathlineto{\pgfqpoint{2.272069in}{2.048333in}}%
\pgfpathlineto{\pgfqpoint{2.275354in}{2.062019in}}%
\pgfpathlineto{\pgfqpoint{2.278638in}{2.067812in}}%
\pgfpathlineto{\pgfqpoint{2.281923in}{2.100283in}}%
\pgfpathlineto{\pgfqpoint{2.285208in}{2.100403in}}%
\pgfpathlineto{\pgfqpoint{2.288493in}{2.107847in}}%
\pgfpathlineto{\pgfqpoint{2.291778in}{2.133257in}}%
\pgfpathlineto{\pgfqpoint{2.295063in}{2.169319in}}%
\pgfpathlineto{\pgfqpoint{2.301633in}{2.193763in}}%
\pgfpathlineto{\pgfqpoint{2.311488in}{2.198494in}}%
\pgfpathlineto{\pgfqpoint{2.318058in}{2.333175in}}%
\pgfpathlineto{\pgfqpoint{2.318058in}{2.333175in}}%
\pgfusepath{stroke}%
\end{pgfscope}%
\begin{pgfscope}%
\pgfpathrectangle{\pgfqpoint{0.698576in}{0.535823in}}{\pgfqpoint{3.613449in}{2.069347in}}%
\pgfusepath{clip}%
\pgfsetbuttcap%
\pgfsetroundjoin%
\pgfsetlinewidth{2.007500pt}%
\definecolor{currentstroke}{rgb}{1.000000,0.694118,0.305882}%
\pgfsetstrokecolor{currentstroke}%
\pgfsetdash{{2.000000pt}{3.300000pt}}{0.000000pt}%
\pgfpathmoveto{\pgfqpoint{0.698576in}{0.536449in}}%
\pgfpathlineto{\pgfqpoint{0.718285in}{0.536718in}}%
\pgfpathlineto{\pgfqpoint{0.977797in}{0.537895in}}%
\pgfpathlineto{\pgfqpoint{1.326002in}{0.540476in}}%
\pgfpathlineto{\pgfqpoint{1.670922in}{0.550637in}}%
\pgfpathlineto{\pgfqpoint{1.891014in}{0.562822in}}%
\pgfpathlineto{\pgfqpoint{1.914009in}{0.567840in}}%
\pgfpathlineto{\pgfqpoint{1.930433in}{0.572481in}}%
\pgfpathlineto{\pgfqpoint{1.953428in}{0.574805in}}%
\pgfpathlineto{\pgfqpoint{1.966568in}{0.576111in}}%
\pgfpathlineto{\pgfqpoint{1.979708in}{0.577320in}}%
\pgfpathlineto{\pgfqpoint{1.986278in}{0.580278in}}%
\pgfpathlineto{\pgfqpoint{2.002702in}{0.581488in}}%
\pgfpathlineto{\pgfqpoint{2.005987in}{0.582960in}}%
\pgfpathlineto{\pgfqpoint{2.012557in}{0.584260in}}%
\pgfpathlineto{\pgfqpoint{2.022412in}{0.585966in}}%
\pgfpathlineto{\pgfqpoint{2.042122in}{0.593808in}}%
\pgfpathlineto{\pgfqpoint{2.048692in}{0.594381in}}%
\pgfpathlineto{\pgfqpoint{2.051977in}{0.598390in}}%
\pgfpathlineto{\pgfqpoint{2.068401in}{0.601344in}}%
\pgfpathlineto{\pgfqpoint{2.071686in}{0.607281in}}%
\pgfpathlineto{\pgfqpoint{2.074971in}{0.608048in}}%
\pgfpathlineto{\pgfqpoint{2.078256in}{0.613117in}}%
\pgfpathlineto{\pgfqpoint{2.084826in}{0.616768in}}%
\pgfpathlineto{\pgfqpoint{2.088111in}{0.617470in}}%
\pgfpathlineto{\pgfqpoint{2.091396in}{0.619854in}}%
\pgfpathlineto{\pgfqpoint{2.107821in}{0.622523in}}%
\pgfpathlineto{\pgfqpoint{2.114391in}{0.623408in}}%
\pgfpathlineto{\pgfqpoint{2.120961in}{0.624392in}}%
\pgfpathlineto{\pgfqpoint{2.124246in}{0.624871in}}%
\pgfpathlineto{\pgfqpoint{2.127531in}{0.629152in}}%
\pgfpathlineto{\pgfqpoint{2.140670in}{0.632681in}}%
\pgfpathlineto{\pgfqpoint{2.143955in}{0.635738in}}%
\pgfpathlineto{\pgfqpoint{2.147240in}{0.635975in}}%
\pgfpathlineto{\pgfqpoint{2.157095in}{0.644389in}}%
\pgfpathlineto{\pgfqpoint{2.170235in}{0.648346in}}%
\pgfpathlineto{\pgfqpoint{2.173520in}{0.655432in}}%
\pgfpathlineto{\pgfqpoint{2.180090in}{0.658702in}}%
\pgfpathlineto{\pgfqpoint{2.183375in}{0.663350in}}%
\pgfpathlineto{\pgfqpoint{2.199800in}{0.666617in}}%
\pgfpathlineto{\pgfqpoint{2.206369in}{0.674241in}}%
\pgfpathlineto{\pgfqpoint{2.209654in}{0.677225in}}%
\pgfpathlineto{\pgfqpoint{2.212939in}{0.682064in}}%
\pgfpathlineto{\pgfqpoint{2.216224in}{0.684737in}}%
\pgfpathlineto{\pgfqpoint{2.219509in}{0.689188in}}%
\pgfpathlineto{\pgfqpoint{2.226079in}{0.691266in}}%
\pgfpathlineto{\pgfqpoint{2.229364in}{0.702365in}}%
\pgfpathlineto{\pgfqpoint{2.235934in}{0.705779in}}%
\pgfpathlineto{\pgfqpoint{2.242504in}{0.706498in}}%
\pgfpathlineto{\pgfqpoint{2.249074in}{0.709163in}}%
\pgfpathlineto{\pgfqpoint{2.255644in}{0.710037in}}%
\pgfpathlineto{\pgfqpoint{2.262214in}{0.711042in}}%
\pgfpathlineto{\pgfqpoint{2.265499in}{0.711466in}}%
\pgfpathlineto{\pgfqpoint{2.272069in}{0.715540in}}%
\pgfpathlineto{\pgfqpoint{2.285208in}{0.719892in}}%
\pgfpathlineto{\pgfqpoint{2.288493in}{0.720245in}}%
\pgfpathlineto{\pgfqpoint{2.291778in}{0.730862in}}%
\pgfpathlineto{\pgfqpoint{2.295063in}{0.731234in}}%
\pgfpathlineto{\pgfqpoint{2.298348in}{0.736519in}}%
\pgfpathlineto{\pgfqpoint{2.301633in}{0.745724in}}%
\pgfpathlineto{\pgfqpoint{2.304918in}{0.746382in}}%
\pgfpathlineto{\pgfqpoint{2.308203in}{0.782099in}}%
\pgfpathlineto{\pgfqpoint{2.314773in}{0.792602in}}%
\pgfpathlineto{\pgfqpoint{2.321343in}{0.798999in}}%
\pgfpathlineto{\pgfqpoint{2.327913in}{0.807457in}}%
\pgfpathlineto{\pgfqpoint{2.334483in}{0.808930in}}%
\pgfpathlineto{\pgfqpoint{2.344338in}{0.809933in}}%
\pgfpathlineto{\pgfqpoint{2.347623in}{0.815846in}}%
\pgfpathlineto{\pgfqpoint{2.350907in}{0.818785in}}%
\pgfpathlineto{\pgfqpoint{2.360762in}{0.821125in}}%
\pgfpathlineto{\pgfqpoint{2.373902in}{0.843090in}}%
\pgfpathlineto{\pgfqpoint{2.380472in}{0.860347in}}%
\pgfpathlineto{\pgfqpoint{2.383757in}{0.860529in}}%
\pgfpathlineto{\pgfqpoint{2.387042in}{0.867381in}}%
\pgfpathlineto{\pgfqpoint{2.390327in}{0.885598in}}%
\pgfpathlineto{\pgfqpoint{2.393612in}{0.892111in}}%
\pgfpathlineto{\pgfqpoint{2.396897in}{0.893559in}}%
\pgfpathlineto{\pgfqpoint{2.403467in}{0.908990in}}%
\pgfpathlineto{\pgfqpoint{2.406752in}{0.914968in}}%
\pgfpathlineto{\pgfqpoint{2.410037in}{0.946837in}}%
\pgfpathlineto{\pgfqpoint{2.413322in}{0.957997in}}%
\pgfpathlineto{\pgfqpoint{2.423176in}{0.970101in}}%
\pgfpathlineto{\pgfqpoint{2.429746in}{0.971469in}}%
\pgfpathlineto{\pgfqpoint{2.433031in}{0.978079in}}%
\pgfpathlineto{\pgfqpoint{2.439601in}{0.980144in}}%
\pgfpathlineto{\pgfqpoint{2.449456in}{0.997991in}}%
\pgfpathlineto{\pgfqpoint{2.456026in}{0.999071in}}%
\pgfpathlineto{\pgfqpoint{2.462596in}{1.005399in}}%
\pgfpathlineto{\pgfqpoint{2.465881in}{1.005671in}}%
\pgfpathlineto{\pgfqpoint{2.469166in}{1.018185in}}%
\pgfpathlineto{\pgfqpoint{2.472451in}{1.037607in}}%
\pgfpathlineto{\pgfqpoint{2.475736in}{1.044467in}}%
\pgfpathlineto{\pgfqpoint{2.479021in}{1.047447in}}%
\pgfpathlineto{\pgfqpoint{2.482306in}{1.047677in}}%
\pgfpathlineto{\pgfqpoint{2.485591in}{1.062233in}}%
\pgfpathlineto{\pgfqpoint{2.488876in}{1.067161in}}%
\pgfpathlineto{\pgfqpoint{2.492160in}{1.067820in}}%
\pgfpathlineto{\pgfqpoint{2.495445in}{1.078831in}}%
\pgfpathlineto{\pgfqpoint{2.498730in}{1.118738in}}%
\pgfpathlineto{\pgfqpoint{2.502015in}{1.126418in}}%
\pgfpathlineto{\pgfqpoint{2.505300in}{1.130361in}}%
\pgfpathlineto{\pgfqpoint{2.511870in}{1.146294in}}%
\pgfpathlineto{\pgfqpoint{2.515155in}{1.150935in}}%
\pgfpathlineto{\pgfqpoint{2.518440in}{1.160982in}}%
\pgfpathlineto{\pgfqpoint{2.521725in}{1.164391in}}%
\pgfpathlineto{\pgfqpoint{2.528295in}{1.196190in}}%
\pgfpathlineto{\pgfqpoint{2.531580in}{1.203752in}}%
\pgfpathlineto{\pgfqpoint{2.534865in}{1.204284in}}%
\pgfpathlineto{\pgfqpoint{2.541435in}{1.283925in}}%
\pgfpathlineto{\pgfqpoint{2.544720in}{1.302630in}}%
\pgfpathlineto{\pgfqpoint{2.548005in}{1.310875in}}%
\pgfpathlineto{\pgfqpoint{2.551290in}{1.314157in}}%
\pgfpathlineto{\pgfqpoint{2.557860in}{1.387648in}}%
\pgfpathlineto{\pgfqpoint{2.561145in}{1.390921in}}%
\pgfpathlineto{\pgfqpoint{2.567714in}{1.551511in}}%
\pgfpathlineto{\pgfqpoint{2.577569in}{1.600592in}}%
\pgfpathlineto{\pgfqpoint{2.584139in}{1.706682in}}%
\pgfpathlineto{\pgfqpoint{2.587424in}{1.731460in}}%
\pgfpathlineto{\pgfqpoint{2.590709in}{1.772042in}}%
\pgfpathlineto{\pgfqpoint{2.593994in}{1.777819in}}%
\pgfpathlineto{\pgfqpoint{2.600564in}{1.782945in}}%
\pgfpathlineto{\pgfqpoint{2.603849in}{1.789794in}}%
\pgfpathlineto{\pgfqpoint{2.607134in}{1.793022in}}%
\pgfpathlineto{\pgfqpoint{2.616989in}{1.808792in}}%
\pgfpathlineto{\pgfqpoint{2.623559in}{1.811453in}}%
\pgfpathlineto{\pgfqpoint{2.630129in}{1.813616in}}%
\pgfpathlineto{\pgfqpoint{2.633414in}{1.815089in}}%
\pgfpathlineto{\pgfqpoint{2.636698in}{1.840129in}}%
\pgfpathlineto{\pgfqpoint{2.639983in}{1.841764in}}%
\pgfpathlineto{\pgfqpoint{2.643268in}{1.847311in}}%
\pgfpathlineto{\pgfqpoint{2.646553in}{1.859225in}}%
\pgfpathlineto{\pgfqpoint{2.649838in}{1.864223in}}%
\pgfpathlineto{\pgfqpoint{2.653123in}{1.872150in}}%
\pgfpathlineto{\pgfqpoint{2.656408in}{1.873488in}}%
\pgfpathlineto{\pgfqpoint{2.659693in}{1.877744in}}%
\pgfpathlineto{\pgfqpoint{2.666263in}{1.942907in}}%
\pgfpathlineto{\pgfqpoint{2.669548in}{1.943714in}}%
\pgfpathlineto{\pgfqpoint{2.679403in}{2.194692in}}%
\pgfpathlineto{\pgfqpoint{2.685973in}{2.208995in}}%
\pgfpathlineto{\pgfqpoint{2.689258in}{2.209287in}}%
\pgfpathlineto{\pgfqpoint{2.692543in}{2.228239in}}%
\pgfpathlineto{\pgfqpoint{2.695828in}{2.274790in}}%
\pgfpathlineto{\pgfqpoint{2.699113in}{2.429676in}}%
\pgfpathlineto{\pgfqpoint{2.702398in}{2.434256in}}%
\pgfpathlineto{\pgfqpoint{2.705682in}{2.450839in}}%
\pgfpathlineto{\pgfqpoint{2.705682in}{2.450839in}}%
\pgfusepath{stroke}%
\end{pgfscope}%
\begin{pgfscope}%
\pgfpathrectangle{\pgfqpoint{0.698576in}{0.535823in}}{\pgfqpoint{3.613449in}{2.069347in}}%
\pgfusepath{clip}%
\pgfsetrectcap%
\pgfsetroundjoin%
\pgfsetlinewidth{2.007500pt}%
\definecolor{currentstroke}{rgb}{0.980392,0.529412,0.458824}%
\pgfsetstrokecolor{currentstroke}%
\pgfsetdash{}{0pt}%
\pgfpathmoveto{\pgfqpoint{0.698576in}{0.537625in}}%
\pgfpathlineto{\pgfqpoint{0.708430in}{0.538095in}}%
\pgfpathlineto{\pgfqpoint{0.879248in}{0.540368in}}%
\pgfpathlineto{\pgfqpoint{0.885818in}{0.541159in}}%
\pgfpathlineto{\pgfqpoint{0.981082in}{0.543666in}}%
\pgfpathlineto{\pgfqpoint{0.990936in}{0.544310in}}%
\pgfpathlineto{\pgfqpoint{1.066490in}{0.546376in}}%
\pgfpathlineto{\pgfqpoint{1.089485in}{0.547195in}}%
\pgfpathlineto{\pgfqpoint{1.105910in}{0.548422in}}%
\pgfpathlineto{\pgfqpoint{1.424550in}{0.559813in}}%
\pgfpathlineto{\pgfqpoint{1.431120in}{0.560749in}}%
\pgfpathlineto{\pgfqpoint{1.483680in}{0.562979in}}%
\pgfpathlineto{\pgfqpoint{1.493534in}{0.563669in}}%
\pgfpathlineto{\pgfqpoint{1.500104in}{0.564593in}}%
\pgfpathlineto{\pgfqpoint{1.552664in}{0.566769in}}%
\pgfpathlineto{\pgfqpoint{1.569088in}{0.569162in}}%
\pgfpathlineto{\pgfqpoint{1.572373in}{0.572184in}}%
\pgfpathlineto{\pgfqpoint{1.624933in}{0.576708in}}%
\pgfpathlineto{\pgfqpoint{1.638072in}{0.578968in}}%
\pgfpathlineto{\pgfqpoint{1.667637in}{0.580000in}}%
\pgfpathlineto{\pgfqpoint{1.677492in}{0.581639in}}%
\pgfpathlineto{\pgfqpoint{1.700487in}{0.584648in}}%
\pgfpathlineto{\pgfqpoint{1.703772in}{0.586208in}}%
\pgfpathlineto{\pgfqpoint{1.713626in}{0.587678in}}%
\pgfpathlineto{\pgfqpoint{1.730051in}{0.589066in}}%
\pgfpathlineto{\pgfqpoint{1.749761in}{0.591542in}}%
\pgfpathlineto{\pgfqpoint{1.762901in}{0.593376in}}%
\pgfpathlineto{\pgfqpoint{1.782610in}{0.594723in}}%
\pgfpathlineto{\pgfqpoint{1.789180in}{0.595562in}}%
\pgfpathlineto{\pgfqpoint{1.792465in}{0.595704in}}%
\pgfpathlineto{\pgfqpoint{1.795750in}{0.597029in}}%
\pgfpathlineto{\pgfqpoint{1.825315in}{0.598398in}}%
\pgfpathlineto{\pgfqpoint{1.838455in}{0.600332in}}%
\pgfpathlineto{\pgfqpoint{1.854879in}{0.603204in}}%
\pgfpathlineto{\pgfqpoint{1.864734in}{0.604140in}}%
\pgfpathlineto{\pgfqpoint{1.871304in}{0.605421in}}%
\pgfpathlineto{\pgfqpoint{1.884444in}{0.606467in}}%
\pgfpathlineto{\pgfqpoint{1.894299in}{0.608001in}}%
\pgfpathlineto{\pgfqpoint{1.907439in}{0.609555in}}%
\pgfpathlineto{\pgfqpoint{1.930433in}{0.612152in}}%
\pgfpathlineto{\pgfqpoint{1.937003in}{0.616814in}}%
\pgfpathlineto{\pgfqpoint{1.943573in}{0.618137in}}%
\pgfpathlineto{\pgfqpoint{1.953428in}{0.618865in}}%
\pgfpathlineto{\pgfqpoint{1.966568in}{0.620170in}}%
\pgfpathlineto{\pgfqpoint{1.982993in}{0.621176in}}%
\pgfpathlineto{\pgfqpoint{1.992847in}{0.623344in}}%
\pgfpathlineto{\pgfqpoint{2.032267in}{0.625955in}}%
\pgfpathlineto{\pgfqpoint{2.042122in}{0.631006in}}%
\pgfpathlineto{\pgfqpoint{2.071686in}{0.634161in}}%
\pgfpathlineto{\pgfqpoint{2.074971in}{0.634244in}}%
\pgfpathlineto{\pgfqpoint{2.084826in}{0.640993in}}%
\pgfpathlineto{\pgfqpoint{2.091396in}{0.641637in}}%
\pgfpathlineto{\pgfqpoint{2.097966in}{0.643803in}}%
\pgfpathlineto{\pgfqpoint{2.101251in}{0.644184in}}%
\pgfpathlineto{\pgfqpoint{2.104536in}{0.646623in}}%
\pgfpathlineto{\pgfqpoint{2.130816in}{0.653901in}}%
\pgfpathlineto{\pgfqpoint{2.147240in}{0.655056in}}%
\pgfpathlineto{\pgfqpoint{2.173520in}{0.657063in}}%
\pgfpathlineto{\pgfqpoint{2.176805in}{0.659324in}}%
\pgfpathlineto{\pgfqpoint{2.222794in}{0.668854in}}%
\pgfpathlineto{\pgfqpoint{2.226079in}{0.672242in}}%
\pgfpathlineto{\pgfqpoint{2.229364in}{0.684663in}}%
\pgfpathlineto{\pgfqpoint{2.232649in}{0.685408in}}%
\pgfpathlineto{\pgfqpoint{2.235934in}{0.687731in}}%
\pgfpathlineto{\pgfqpoint{2.239219in}{0.688367in}}%
\pgfpathlineto{\pgfqpoint{2.242504in}{0.691989in}}%
\pgfpathlineto{\pgfqpoint{2.245789in}{0.693097in}}%
\pgfpathlineto{\pgfqpoint{2.249074in}{0.698168in}}%
\pgfpathlineto{\pgfqpoint{2.252359in}{0.698976in}}%
\pgfpathlineto{\pgfqpoint{2.262214in}{0.705639in}}%
\pgfpathlineto{\pgfqpoint{2.265499in}{0.710814in}}%
\pgfpathlineto{\pgfqpoint{2.285208in}{0.713336in}}%
\pgfpathlineto{\pgfqpoint{2.288493in}{0.720650in}}%
\pgfpathlineto{\pgfqpoint{2.291778in}{0.725059in}}%
\pgfpathlineto{\pgfqpoint{2.301633in}{0.726344in}}%
\pgfpathlineto{\pgfqpoint{2.308203in}{0.730190in}}%
\pgfpathlineto{\pgfqpoint{2.314773in}{0.733191in}}%
\pgfpathlineto{\pgfqpoint{2.318058in}{0.738124in}}%
\pgfpathlineto{\pgfqpoint{2.334483in}{0.742449in}}%
\pgfpathlineto{\pgfqpoint{2.347623in}{0.747460in}}%
\pgfpathlineto{\pgfqpoint{2.350907in}{0.747587in}}%
\pgfpathlineto{\pgfqpoint{2.354192in}{0.753157in}}%
\pgfpathlineto{\pgfqpoint{2.357477in}{0.756017in}}%
\pgfpathlineto{\pgfqpoint{2.380472in}{0.762029in}}%
\pgfpathlineto{\pgfqpoint{2.393612in}{0.763601in}}%
\pgfpathlineto{\pgfqpoint{2.396897in}{0.765324in}}%
\pgfpathlineto{\pgfqpoint{2.400182in}{0.768797in}}%
\pgfpathlineto{\pgfqpoint{2.410037in}{0.773636in}}%
\pgfpathlineto{\pgfqpoint{2.416607in}{0.774661in}}%
\pgfpathlineto{\pgfqpoint{2.419891in}{0.776361in}}%
\pgfpathlineto{\pgfqpoint{2.426461in}{0.777469in}}%
\pgfpathlineto{\pgfqpoint{2.436316in}{0.785416in}}%
\pgfpathlineto{\pgfqpoint{2.439601in}{0.791437in}}%
\pgfpathlineto{\pgfqpoint{2.442886in}{0.793857in}}%
\pgfpathlineto{\pgfqpoint{2.446171in}{0.802212in}}%
\pgfpathlineto{\pgfqpoint{2.456026in}{0.806748in}}%
\pgfpathlineto{\pgfqpoint{2.459311in}{0.811290in}}%
\pgfpathlineto{\pgfqpoint{2.465881in}{0.812333in}}%
\pgfpathlineto{\pgfqpoint{2.469166in}{0.815855in}}%
\pgfpathlineto{\pgfqpoint{2.472451in}{0.816890in}}%
\pgfpathlineto{\pgfqpoint{2.475736in}{0.819994in}}%
\pgfpathlineto{\pgfqpoint{2.482306in}{0.822099in}}%
\pgfpathlineto{\pgfqpoint{2.492160in}{0.824302in}}%
\pgfpathlineto{\pgfqpoint{2.495445in}{0.827346in}}%
\pgfpathlineto{\pgfqpoint{2.498730in}{0.828134in}}%
\pgfpathlineto{\pgfqpoint{2.505300in}{0.831431in}}%
\pgfpathlineto{\pgfqpoint{2.508585in}{0.836515in}}%
\pgfpathlineto{\pgfqpoint{2.511870in}{0.837372in}}%
\pgfpathlineto{\pgfqpoint{2.515155in}{0.842755in}}%
\pgfpathlineto{\pgfqpoint{2.521725in}{0.843833in}}%
\pgfpathlineto{\pgfqpoint{2.525010in}{0.848617in}}%
\pgfpathlineto{\pgfqpoint{2.528295in}{0.857819in}}%
\pgfpathlineto{\pgfqpoint{2.534865in}{0.865070in}}%
\pgfpathlineto{\pgfqpoint{2.538150in}{0.865767in}}%
\pgfpathlineto{\pgfqpoint{2.541435in}{0.870430in}}%
\pgfpathlineto{\pgfqpoint{2.548005in}{0.871473in}}%
\pgfpathlineto{\pgfqpoint{2.551290in}{0.872249in}}%
\pgfpathlineto{\pgfqpoint{2.557860in}{0.886898in}}%
\pgfpathlineto{\pgfqpoint{2.561145in}{0.888867in}}%
\pgfpathlineto{\pgfqpoint{2.564429in}{0.892279in}}%
\pgfpathlineto{\pgfqpoint{2.567714in}{0.899827in}}%
\pgfpathlineto{\pgfqpoint{2.570999in}{0.903824in}}%
\pgfpathlineto{\pgfqpoint{2.574284in}{0.911072in}}%
\pgfpathlineto{\pgfqpoint{2.577569in}{0.911476in}}%
\pgfpathlineto{\pgfqpoint{2.580854in}{0.925661in}}%
\pgfpathlineto{\pgfqpoint{2.587424in}{0.940947in}}%
\pgfpathlineto{\pgfqpoint{2.590709in}{0.944411in}}%
\pgfpathlineto{\pgfqpoint{2.593994in}{0.944434in}}%
\pgfpathlineto{\pgfqpoint{2.597279in}{0.951646in}}%
\pgfpathlineto{\pgfqpoint{2.600564in}{0.952301in}}%
\pgfpathlineto{\pgfqpoint{2.603849in}{0.956311in}}%
\pgfpathlineto{\pgfqpoint{2.607134in}{0.957020in}}%
\pgfpathlineto{\pgfqpoint{2.613704in}{0.969227in}}%
\pgfpathlineto{\pgfqpoint{2.616989in}{0.969894in}}%
\pgfpathlineto{\pgfqpoint{2.620274in}{0.974683in}}%
\pgfpathlineto{\pgfqpoint{2.623559in}{0.983207in}}%
\pgfpathlineto{\pgfqpoint{2.626844in}{0.987006in}}%
\pgfpathlineto{\pgfqpoint{2.633414in}{0.991765in}}%
\pgfpathlineto{\pgfqpoint{2.636698in}{1.001397in}}%
\pgfpathlineto{\pgfqpoint{2.639983in}{1.003745in}}%
\pgfpathlineto{\pgfqpoint{2.643268in}{1.024940in}}%
\pgfpathlineto{\pgfqpoint{2.646553in}{1.038071in}}%
\pgfpathlineto{\pgfqpoint{2.653123in}{1.047453in}}%
\pgfpathlineto{\pgfqpoint{2.656408in}{1.050870in}}%
\pgfpathlineto{\pgfqpoint{2.659693in}{1.058912in}}%
\pgfpathlineto{\pgfqpoint{2.662978in}{1.060055in}}%
\pgfpathlineto{\pgfqpoint{2.669548in}{1.070668in}}%
\pgfpathlineto{\pgfqpoint{2.672833in}{1.072605in}}%
\pgfpathlineto{\pgfqpoint{2.676118in}{1.077382in}}%
\pgfpathlineto{\pgfqpoint{2.682688in}{1.082799in}}%
\pgfpathlineto{\pgfqpoint{2.685973in}{1.085619in}}%
\pgfpathlineto{\pgfqpoint{2.689258in}{1.094169in}}%
\pgfpathlineto{\pgfqpoint{2.692543in}{1.118106in}}%
\pgfpathlineto{\pgfqpoint{2.695828in}{1.129139in}}%
\pgfpathlineto{\pgfqpoint{2.702398in}{1.134331in}}%
\pgfpathlineto{\pgfqpoint{2.705682in}{1.142287in}}%
\pgfpathlineto{\pgfqpoint{2.708967in}{1.154111in}}%
\pgfpathlineto{\pgfqpoint{2.712252in}{1.158011in}}%
\pgfpathlineto{\pgfqpoint{2.715537in}{1.167565in}}%
\pgfpathlineto{\pgfqpoint{2.718822in}{1.169129in}}%
\pgfpathlineto{\pgfqpoint{2.722107in}{1.172913in}}%
\pgfpathlineto{\pgfqpoint{2.725392in}{1.173347in}}%
\pgfpathlineto{\pgfqpoint{2.731962in}{1.176081in}}%
\pgfpathlineto{\pgfqpoint{2.735247in}{1.187216in}}%
\pgfpathlineto{\pgfqpoint{2.738532in}{1.190593in}}%
\pgfpathlineto{\pgfqpoint{2.741817in}{1.203527in}}%
\pgfpathlineto{\pgfqpoint{2.745102in}{1.204059in}}%
\pgfpathlineto{\pgfqpoint{2.748387in}{1.207083in}}%
\pgfpathlineto{\pgfqpoint{2.751672in}{1.213227in}}%
\pgfpathlineto{\pgfqpoint{2.754957in}{1.248742in}}%
\pgfpathlineto{\pgfqpoint{2.761527in}{1.355785in}}%
\pgfpathlineto{\pgfqpoint{2.768097in}{1.419768in}}%
\pgfpathlineto{\pgfqpoint{2.771382in}{1.427202in}}%
\pgfpathlineto{\pgfqpoint{2.777951in}{1.436947in}}%
\pgfpathlineto{\pgfqpoint{2.781236in}{1.504367in}}%
\pgfpathlineto{\pgfqpoint{2.787806in}{1.506168in}}%
\pgfpathlineto{\pgfqpoint{2.794376in}{1.517841in}}%
\pgfpathlineto{\pgfqpoint{2.797661in}{1.519529in}}%
\pgfpathlineto{\pgfqpoint{2.800946in}{1.543544in}}%
\pgfpathlineto{\pgfqpoint{2.804231in}{1.543628in}}%
\pgfpathlineto{\pgfqpoint{2.807516in}{1.547703in}}%
\pgfpathlineto{\pgfqpoint{2.810801in}{1.564421in}}%
\pgfpathlineto{\pgfqpoint{2.814086in}{1.619422in}}%
\pgfpathlineto{\pgfqpoint{2.823941in}{1.644559in}}%
\pgfpathlineto{\pgfqpoint{2.830511in}{1.667041in}}%
\pgfpathlineto{\pgfqpoint{2.833796in}{1.669741in}}%
\pgfpathlineto{\pgfqpoint{2.837081in}{1.670183in}}%
\pgfpathlineto{\pgfqpoint{2.843651in}{1.681558in}}%
\pgfpathlineto{\pgfqpoint{2.846936in}{1.695107in}}%
\pgfpathlineto{\pgfqpoint{2.850220in}{1.699693in}}%
\pgfpathlineto{\pgfqpoint{2.853505in}{1.702162in}}%
\pgfpathlineto{\pgfqpoint{2.856790in}{1.776321in}}%
\pgfpathlineto{\pgfqpoint{2.860075in}{1.784962in}}%
\pgfpathlineto{\pgfqpoint{2.863360in}{1.787074in}}%
\pgfpathlineto{\pgfqpoint{2.866645in}{1.787140in}}%
\pgfpathlineto{\pgfqpoint{2.869930in}{1.797272in}}%
\pgfpathlineto{\pgfqpoint{2.873215in}{1.816856in}}%
\pgfpathlineto{\pgfqpoint{2.879785in}{1.836330in}}%
\pgfpathlineto{\pgfqpoint{2.883070in}{1.839881in}}%
\pgfpathlineto{\pgfqpoint{2.886355in}{1.863472in}}%
\pgfpathlineto{\pgfqpoint{2.892925in}{1.893260in}}%
\pgfpathlineto{\pgfqpoint{2.896210in}{1.919636in}}%
\pgfpathlineto{\pgfqpoint{2.902780in}{1.940708in}}%
\pgfpathlineto{\pgfqpoint{2.906065in}{1.941169in}}%
\pgfpathlineto{\pgfqpoint{2.909350in}{1.947121in}}%
\pgfpathlineto{\pgfqpoint{2.912635in}{1.950242in}}%
\pgfpathlineto{\pgfqpoint{2.915920in}{1.962222in}}%
\pgfpathlineto{\pgfqpoint{2.919205in}{1.969566in}}%
\pgfpathlineto{\pgfqpoint{2.922489in}{1.981871in}}%
\pgfpathlineto{\pgfqpoint{2.925774in}{2.010711in}}%
\pgfpathlineto{\pgfqpoint{2.929059in}{2.014768in}}%
\pgfpathlineto{\pgfqpoint{2.935629in}{2.054192in}}%
\pgfpathlineto{\pgfqpoint{2.942199in}{2.060096in}}%
\pgfpathlineto{\pgfqpoint{2.945484in}{2.098312in}}%
\pgfpathlineto{\pgfqpoint{2.948769in}{2.116269in}}%
\pgfpathlineto{\pgfqpoint{2.952054in}{2.143161in}}%
\pgfpathlineto{\pgfqpoint{2.955339in}{2.193863in}}%
\pgfpathlineto{\pgfqpoint{2.958624in}{2.205337in}}%
\pgfpathlineto{\pgfqpoint{2.961909in}{2.211471in}}%
\pgfpathlineto{\pgfqpoint{2.965194in}{2.228683in}}%
\pgfpathlineto{\pgfqpoint{2.968479in}{2.273661in}}%
\pgfpathlineto{\pgfqpoint{2.971764in}{2.285526in}}%
\pgfpathlineto{\pgfqpoint{2.975049in}{2.285715in}}%
\pgfpathlineto{\pgfqpoint{2.978334in}{2.292146in}}%
\pgfpathlineto{\pgfqpoint{2.981619in}{2.292389in}}%
\pgfpathlineto{\pgfqpoint{2.984904in}{2.304290in}}%
\pgfpathlineto{\pgfqpoint{2.988189in}{2.345824in}}%
\pgfpathlineto{\pgfqpoint{2.991474in}{2.505452in}}%
\pgfpathlineto{\pgfqpoint{2.991474in}{2.505452in}}%
\pgfusepath{stroke}%
\end{pgfscope}%
\begin{pgfscope}%
\pgfpathrectangle{\pgfqpoint{0.698576in}{0.535823in}}{\pgfqpoint{3.613449in}{2.069347in}}%
\pgfusepath{clip}%
\pgfsetbuttcap%
\pgfsetroundjoin%
\pgfsetlinewidth{2.007500pt}%
\definecolor{currentstroke}{rgb}{0.866667,0.058824,0.058824}%
\pgfsetstrokecolor{currentstroke}%
\pgfsetdash{{7.400000pt}{3.200000pt}}{0.000000pt}%
\pgfpathmoveto{\pgfqpoint{0.698576in}{0.535841in}}%
\pgfpathlineto{\pgfqpoint{2.242504in}{0.536950in}}%
\pgfpathlineto{\pgfqpoint{2.390327in}{0.538581in}}%
\pgfpathlineto{\pgfqpoint{2.492160in}{0.542092in}}%
\pgfpathlineto{\pgfqpoint{2.597279in}{0.549304in}}%
\pgfpathlineto{\pgfqpoint{2.616989in}{0.550848in}}%
\pgfpathlineto{\pgfqpoint{2.679403in}{0.557016in}}%
\pgfpathlineto{\pgfqpoint{2.689258in}{0.558385in}}%
\pgfpathlineto{\pgfqpoint{2.722107in}{0.561971in}}%
\pgfpathlineto{\pgfqpoint{2.738532in}{0.563606in}}%
\pgfpathlineto{\pgfqpoint{2.754957in}{0.566333in}}%
\pgfpathlineto{\pgfqpoint{2.774667in}{0.570787in}}%
\pgfpathlineto{\pgfqpoint{2.781236in}{0.571793in}}%
\pgfpathlineto{\pgfqpoint{2.784521in}{0.574454in}}%
\pgfpathlineto{\pgfqpoint{2.794376in}{0.576409in}}%
\pgfpathlineto{\pgfqpoint{2.800946in}{0.583973in}}%
\pgfpathlineto{\pgfqpoint{2.817371in}{0.587975in}}%
\pgfpathlineto{\pgfqpoint{2.820656in}{0.592758in}}%
\pgfpathlineto{\pgfqpoint{2.823941in}{0.595326in}}%
\pgfpathlineto{\pgfqpoint{2.833796in}{0.596251in}}%
\pgfpathlineto{\pgfqpoint{2.840366in}{0.604544in}}%
\pgfpathlineto{\pgfqpoint{2.843651in}{0.607953in}}%
\pgfpathlineto{\pgfqpoint{2.850220in}{0.611127in}}%
\pgfpathlineto{\pgfqpoint{2.853505in}{0.612636in}}%
\pgfpathlineto{\pgfqpoint{2.856790in}{0.624551in}}%
\pgfpathlineto{\pgfqpoint{2.860075in}{0.627336in}}%
\pgfpathlineto{\pgfqpoint{2.866645in}{0.628348in}}%
\pgfpathlineto{\pgfqpoint{2.869930in}{0.633479in}}%
\pgfpathlineto{\pgfqpoint{2.873215in}{0.634307in}}%
\pgfpathlineto{\pgfqpoint{2.876500in}{0.638209in}}%
\pgfpathlineto{\pgfqpoint{2.879785in}{0.638507in}}%
\pgfpathlineto{\pgfqpoint{2.883070in}{0.651184in}}%
\pgfpathlineto{\pgfqpoint{2.896210in}{0.662472in}}%
\pgfpathlineto{\pgfqpoint{2.902780in}{0.664719in}}%
\pgfpathlineto{\pgfqpoint{2.906065in}{0.667260in}}%
\pgfpathlineto{\pgfqpoint{2.909350in}{0.667322in}}%
\pgfpathlineto{\pgfqpoint{2.912635in}{0.676703in}}%
\pgfpathlineto{\pgfqpoint{2.915920in}{0.689945in}}%
\pgfpathlineto{\pgfqpoint{2.919205in}{0.695169in}}%
\pgfpathlineto{\pgfqpoint{2.929059in}{0.702926in}}%
\pgfpathlineto{\pgfqpoint{2.942199in}{0.706698in}}%
\pgfpathlineto{\pgfqpoint{2.945484in}{0.722596in}}%
\pgfpathlineto{\pgfqpoint{2.948769in}{0.729780in}}%
\pgfpathlineto{\pgfqpoint{2.952054in}{0.730221in}}%
\pgfpathlineto{\pgfqpoint{2.955339in}{0.736641in}}%
\pgfpathlineto{\pgfqpoint{2.958624in}{0.748432in}}%
\pgfpathlineto{\pgfqpoint{2.961909in}{0.749245in}}%
\pgfpathlineto{\pgfqpoint{2.965194in}{0.759832in}}%
\pgfpathlineto{\pgfqpoint{2.968479in}{0.764177in}}%
\pgfpathlineto{\pgfqpoint{2.971764in}{0.776150in}}%
\pgfpathlineto{\pgfqpoint{2.978334in}{0.780049in}}%
\pgfpathlineto{\pgfqpoint{2.981619in}{0.789212in}}%
\pgfpathlineto{\pgfqpoint{2.984904in}{0.790063in}}%
\pgfpathlineto{\pgfqpoint{2.988189in}{0.796817in}}%
\pgfpathlineto{\pgfqpoint{2.991474in}{0.799033in}}%
\pgfpathlineto{\pgfqpoint{2.998043in}{0.817455in}}%
\pgfpathlineto{\pgfqpoint{3.001328in}{0.818580in}}%
\pgfpathlineto{\pgfqpoint{3.004613in}{0.821583in}}%
\pgfpathlineto{\pgfqpoint{3.007898in}{0.837569in}}%
\pgfpathlineto{\pgfqpoint{3.011183in}{0.839833in}}%
\pgfpathlineto{\pgfqpoint{3.014468in}{0.843806in}}%
\pgfpathlineto{\pgfqpoint{3.024323in}{0.846372in}}%
\pgfpathlineto{\pgfqpoint{3.027608in}{0.852656in}}%
\pgfpathlineto{\pgfqpoint{3.030893in}{0.856251in}}%
\pgfpathlineto{\pgfqpoint{3.034178in}{0.880102in}}%
\pgfpathlineto{\pgfqpoint{3.037463in}{0.890745in}}%
\pgfpathlineto{\pgfqpoint{3.040748in}{0.896690in}}%
\pgfpathlineto{\pgfqpoint{3.044033in}{0.908094in}}%
\pgfpathlineto{\pgfqpoint{3.047318in}{0.915316in}}%
\pgfpathlineto{\pgfqpoint{3.060458in}{0.962874in}}%
\pgfpathlineto{\pgfqpoint{3.063742in}{1.003044in}}%
\pgfpathlineto{\pgfqpoint{3.067027in}{1.017981in}}%
\pgfpathlineto{\pgfqpoint{3.070312in}{1.045865in}}%
\pgfpathlineto{\pgfqpoint{3.073597in}{1.102991in}}%
\pgfpathlineto{\pgfqpoint{3.076882in}{1.128084in}}%
\pgfpathlineto{\pgfqpoint{3.080167in}{1.130118in}}%
\pgfpathlineto{\pgfqpoint{3.083452in}{1.138290in}}%
\pgfpathlineto{\pgfqpoint{3.090022in}{1.244435in}}%
\pgfpathlineto{\pgfqpoint{3.093307in}{1.248998in}}%
\pgfpathlineto{\pgfqpoint{3.096592in}{1.261955in}}%
\pgfpathlineto{\pgfqpoint{3.099877in}{1.353409in}}%
\pgfpathlineto{\pgfqpoint{3.103162in}{1.371301in}}%
\pgfpathlineto{\pgfqpoint{3.106447in}{1.379876in}}%
\pgfpathlineto{\pgfqpoint{3.109732in}{1.428392in}}%
\pgfpathlineto{\pgfqpoint{3.113017in}{1.569383in}}%
\pgfpathlineto{\pgfqpoint{3.116302in}{1.583529in}}%
\pgfpathlineto{\pgfqpoint{3.119587in}{1.609332in}}%
\pgfpathlineto{\pgfqpoint{3.122872in}{1.685438in}}%
\pgfpathlineto{\pgfqpoint{3.129442in}{1.782919in}}%
\pgfpathlineto{\pgfqpoint{3.132727in}{1.860611in}}%
\pgfpathlineto{\pgfqpoint{3.136011in}{1.878767in}}%
\pgfpathlineto{\pgfqpoint{3.142581in}{2.019543in}}%
\pgfpathlineto{\pgfqpoint{3.145866in}{2.044809in}}%
\pgfpathlineto{\pgfqpoint{3.152436in}{2.065652in}}%
\pgfpathlineto{\pgfqpoint{3.155721in}{2.069577in}}%
\pgfpathlineto{\pgfqpoint{3.159006in}{2.183253in}}%
\pgfpathlineto{\pgfqpoint{3.162291in}{2.198924in}}%
\pgfpathlineto{\pgfqpoint{3.165576in}{2.324519in}}%
\pgfpathlineto{\pgfqpoint{3.168861in}{2.332347in}}%
\pgfpathlineto{\pgfqpoint{3.172146in}{2.433836in}}%
\pgfpathlineto{\pgfqpoint{3.178716in}{2.474896in}}%
\pgfpathlineto{\pgfqpoint{3.182001in}{2.565176in}}%
\pgfpathlineto{\pgfqpoint{3.182001in}{2.565176in}}%
\pgfusepath{stroke}%
\end{pgfscope}%
\begin{pgfscope}%
\pgfpathrectangle{\pgfqpoint{0.698576in}{0.535823in}}{\pgfqpoint{3.613449in}{2.069347in}}%
\pgfusepath{clip}%
\pgfsetbuttcap%
\pgfsetroundjoin%
\pgfsetlinewidth{2.007500pt}%
\definecolor{currentstroke}{rgb}{0.917647,0.372549,0.580392}%
\pgfsetstrokecolor{currentstroke}%
\pgfsetdash{{2.000000pt}{3.300000pt}}{0.000000pt}%
\pgfpathmoveto{\pgfqpoint{0.698576in}{0.535885in}}%
\pgfpathlineto{\pgfqpoint{2.275354in}{0.536961in}}%
\pgfpathlineto{\pgfqpoint{2.656408in}{0.542714in}}%
\pgfpathlineto{\pgfqpoint{2.676118in}{0.543914in}}%
\pgfpathlineto{\pgfqpoint{2.899495in}{0.555937in}}%
\pgfpathlineto{\pgfqpoint{2.915920in}{0.559496in}}%
\pgfpathlineto{\pgfqpoint{2.925774in}{0.560469in}}%
\pgfpathlineto{\pgfqpoint{2.942199in}{0.565063in}}%
\pgfpathlineto{\pgfqpoint{2.955339in}{0.566366in}}%
\pgfpathlineto{\pgfqpoint{2.958624in}{0.566470in}}%
\pgfpathlineto{\pgfqpoint{2.961909in}{0.567815in}}%
\pgfpathlineto{\pgfqpoint{2.975049in}{0.568456in}}%
\pgfpathlineto{\pgfqpoint{2.994758in}{0.574043in}}%
\pgfpathlineto{\pgfqpoint{3.001328in}{0.580003in}}%
\pgfpathlineto{\pgfqpoint{3.037463in}{0.587349in}}%
\pgfpathlineto{\pgfqpoint{3.047318in}{0.587991in}}%
\pgfpathlineto{\pgfqpoint{3.057173in}{0.596724in}}%
\pgfpathlineto{\pgfqpoint{3.070312in}{0.598917in}}%
\pgfpathlineto{\pgfqpoint{3.073597in}{0.600614in}}%
\pgfpathlineto{\pgfqpoint{3.076882in}{0.600676in}}%
\pgfpathlineto{\pgfqpoint{3.080167in}{0.605353in}}%
\pgfpathlineto{\pgfqpoint{3.083452in}{0.606077in}}%
\pgfpathlineto{\pgfqpoint{3.086737in}{0.607960in}}%
\pgfpathlineto{\pgfqpoint{3.096592in}{0.608664in}}%
\pgfpathlineto{\pgfqpoint{3.099877in}{0.611105in}}%
\pgfpathlineto{\pgfqpoint{3.139296in}{0.621018in}}%
\pgfpathlineto{\pgfqpoint{3.145866in}{0.622073in}}%
\pgfpathlineto{\pgfqpoint{3.168861in}{0.626460in}}%
\pgfpathlineto{\pgfqpoint{3.178716in}{0.632585in}}%
\pgfpathlineto{\pgfqpoint{3.185286in}{0.634551in}}%
\pgfpathlineto{\pgfqpoint{3.188571in}{0.637159in}}%
\pgfpathlineto{\pgfqpoint{3.191856in}{0.643925in}}%
\pgfpathlineto{\pgfqpoint{3.195141in}{0.646222in}}%
\pgfpathlineto{\pgfqpoint{3.208280in}{0.649016in}}%
\pgfpathlineto{\pgfqpoint{3.221420in}{0.657666in}}%
\pgfpathlineto{\pgfqpoint{3.224705in}{0.658349in}}%
\pgfpathlineto{\pgfqpoint{3.227990in}{0.661535in}}%
\pgfpathlineto{\pgfqpoint{3.234560in}{0.664019in}}%
\pgfpathlineto{\pgfqpoint{3.237845in}{0.667909in}}%
\pgfpathlineto{\pgfqpoint{3.241130in}{0.677552in}}%
\pgfpathlineto{\pgfqpoint{3.244415in}{0.681091in}}%
\pgfpathlineto{\pgfqpoint{3.247700in}{0.681670in}}%
\pgfpathlineto{\pgfqpoint{3.250985in}{0.684050in}}%
\pgfpathlineto{\pgfqpoint{3.257555in}{0.685188in}}%
\pgfpathlineto{\pgfqpoint{3.260840in}{0.691334in}}%
\pgfpathlineto{\pgfqpoint{3.264125in}{0.692369in}}%
\pgfpathlineto{\pgfqpoint{3.267410in}{0.695473in}}%
\pgfpathlineto{\pgfqpoint{3.270695in}{0.695825in}}%
\pgfpathlineto{\pgfqpoint{3.273980in}{0.697439in}}%
\pgfpathlineto{\pgfqpoint{3.277265in}{0.712379in}}%
\pgfpathlineto{\pgfqpoint{3.280549in}{0.715670in}}%
\pgfpathlineto{\pgfqpoint{3.283834in}{0.716083in}}%
\pgfpathlineto{\pgfqpoint{3.290404in}{0.724651in}}%
\pgfpathlineto{\pgfqpoint{3.293689in}{0.724733in}}%
\pgfpathlineto{\pgfqpoint{3.296974in}{0.735184in}}%
\pgfpathlineto{\pgfqpoint{3.300259in}{0.735949in}}%
\pgfpathlineto{\pgfqpoint{3.303544in}{0.739612in}}%
\pgfpathlineto{\pgfqpoint{3.306829in}{0.740274in}}%
\pgfpathlineto{\pgfqpoint{3.310114in}{0.753352in}}%
\pgfpathlineto{\pgfqpoint{3.316684in}{0.757905in}}%
\pgfpathlineto{\pgfqpoint{3.319969in}{0.779178in}}%
\pgfpathlineto{\pgfqpoint{3.323254in}{0.789856in}}%
\pgfpathlineto{\pgfqpoint{3.326539in}{0.789938in}}%
\pgfpathlineto{\pgfqpoint{3.329824in}{0.793187in}}%
\pgfpathlineto{\pgfqpoint{3.333109in}{0.798588in}}%
\pgfpathlineto{\pgfqpoint{3.336394in}{0.799251in}}%
\pgfpathlineto{\pgfqpoint{3.339679in}{0.825821in}}%
\pgfpathlineto{\pgfqpoint{3.346249in}{0.845211in}}%
\pgfpathlineto{\pgfqpoint{3.349533in}{0.851170in}}%
\pgfpathlineto{\pgfqpoint{3.352818in}{0.874968in}}%
\pgfpathlineto{\pgfqpoint{3.356103in}{0.921197in}}%
\pgfpathlineto{\pgfqpoint{3.359388in}{0.921570in}}%
\pgfpathlineto{\pgfqpoint{3.362673in}{0.940649in}}%
\pgfpathlineto{\pgfqpoint{3.372528in}{0.973345in}}%
\pgfpathlineto{\pgfqpoint{3.375813in}{0.974152in}}%
\pgfpathlineto{\pgfqpoint{3.379098in}{0.978766in}}%
\pgfpathlineto{\pgfqpoint{3.382383in}{0.979801in}}%
\pgfpathlineto{\pgfqpoint{3.385668in}{0.986071in}}%
\pgfpathlineto{\pgfqpoint{3.392238in}{1.003123in}}%
\pgfpathlineto{\pgfqpoint{3.395523in}{1.023237in}}%
\pgfpathlineto{\pgfqpoint{3.398808in}{1.023837in}}%
\pgfpathlineto{\pgfqpoint{3.402093in}{1.031928in}}%
\pgfpathlineto{\pgfqpoint{3.405378in}{1.081510in}}%
\pgfpathlineto{\pgfqpoint{3.408663in}{1.088980in}}%
\pgfpathlineto{\pgfqpoint{3.411948in}{1.092974in}}%
\pgfpathlineto{\pgfqpoint{3.415233in}{1.107707in}}%
\pgfpathlineto{\pgfqpoint{3.418518in}{1.110439in}}%
\pgfpathlineto{\pgfqpoint{3.421802in}{1.120351in}}%
\pgfpathlineto{\pgfqpoint{3.425087in}{1.149922in}}%
\pgfpathlineto{\pgfqpoint{3.428372in}{1.160559in}}%
\pgfpathlineto{\pgfqpoint{3.431657in}{1.162152in}}%
\pgfpathlineto{\pgfqpoint{3.434942in}{1.204677in}}%
\pgfpathlineto{\pgfqpoint{3.441512in}{1.237704in}}%
\pgfpathlineto{\pgfqpoint{3.444797in}{1.242712in}}%
\pgfpathlineto{\pgfqpoint{3.448082in}{1.244553in}}%
\pgfpathlineto{\pgfqpoint{3.461222in}{1.344772in}}%
\pgfpathlineto{\pgfqpoint{3.464507in}{1.402403in}}%
\pgfpathlineto{\pgfqpoint{3.467792in}{1.429036in}}%
\pgfpathlineto{\pgfqpoint{3.471077in}{1.432533in}}%
\pgfpathlineto{\pgfqpoint{3.474362in}{1.569648in}}%
\pgfpathlineto{\pgfqpoint{3.477647in}{1.585892in}}%
\pgfpathlineto{\pgfqpoint{3.480932in}{1.628893in}}%
\pgfpathlineto{\pgfqpoint{3.484217in}{1.635908in}}%
\pgfpathlineto{\pgfqpoint{3.487502in}{1.692629in}}%
\pgfpathlineto{\pgfqpoint{3.490787in}{1.722759in}}%
\pgfpathlineto{\pgfqpoint{3.494071in}{1.727311in}}%
\pgfpathlineto{\pgfqpoint{3.497356in}{1.728491in}}%
\pgfpathlineto{\pgfqpoint{3.500641in}{1.763649in}}%
\pgfpathlineto{\pgfqpoint{3.503926in}{1.829061in}}%
\pgfpathlineto{\pgfqpoint{3.507211in}{1.840877in}}%
\pgfpathlineto{\pgfqpoint{3.510496in}{1.842781in}}%
\pgfpathlineto{\pgfqpoint{3.513781in}{1.901033in}}%
\pgfpathlineto{\pgfqpoint{3.517066in}{1.931246in}}%
\pgfpathlineto{\pgfqpoint{3.520351in}{1.935074in}}%
\pgfpathlineto{\pgfqpoint{3.523636in}{1.951173in}}%
\pgfpathlineto{\pgfqpoint{3.526921in}{1.974102in}}%
\pgfpathlineto{\pgfqpoint{3.530206in}{2.046529in}}%
\pgfpathlineto{\pgfqpoint{3.533491in}{2.083053in}}%
\pgfpathlineto{\pgfqpoint{3.536776in}{2.106623in}}%
\pgfpathlineto{\pgfqpoint{3.540061in}{2.115418in}}%
\pgfpathlineto{\pgfqpoint{3.543346in}{2.129820in}}%
\pgfpathlineto{\pgfqpoint{3.546631in}{2.174994in}}%
\pgfpathlineto{\pgfqpoint{3.549916in}{2.187100in}}%
\pgfpathlineto{\pgfqpoint{3.553201in}{2.274447in}}%
\pgfpathlineto{\pgfqpoint{3.556486in}{2.298948in}}%
\pgfpathlineto{\pgfqpoint{3.559771in}{2.305632in}}%
\pgfpathlineto{\pgfqpoint{3.563056in}{2.409141in}}%
\pgfpathlineto{\pgfqpoint{3.569625in}{2.494667in}}%
\pgfpathlineto{\pgfqpoint{3.569625in}{2.494667in}}%
\pgfusepath{stroke}%
\end{pgfscope}%
\begin{pgfscope}%
\pgfpathrectangle{\pgfqpoint{0.698576in}{0.535823in}}{\pgfqpoint{3.613449in}{2.069347in}}%
\pgfusepath{clip}%
\pgfsetrectcap%
\pgfsetroundjoin%
\pgfsetlinewidth{2.007500pt}%
\definecolor{currentstroke}{rgb}{0.615686,0.007843,0.843137}%
\pgfsetstrokecolor{currentstroke}%
\pgfsetdash{}{0pt}%
\pgfpathmoveto{\pgfqpoint{0.698576in}{0.535826in}}%
\pgfpathlineto{\pgfqpoint{2.344338in}{0.537080in}}%
\pgfpathlineto{\pgfqpoint{2.564429in}{0.539205in}}%
\pgfpathlineto{\pgfqpoint{2.653123in}{0.541082in}}%
\pgfpathlineto{\pgfqpoint{2.669548in}{0.541805in}}%
\pgfpathlineto{\pgfqpoint{2.708967in}{0.543063in}}%
\pgfpathlineto{\pgfqpoint{2.754957in}{0.545322in}}%
\pgfpathlineto{\pgfqpoint{2.817371in}{0.549678in}}%
\pgfpathlineto{\pgfqpoint{2.840366in}{0.551293in}}%
\pgfpathlineto{\pgfqpoint{2.850220in}{0.552071in}}%
\pgfpathlineto{\pgfqpoint{2.892925in}{0.554776in}}%
\pgfpathlineto{\pgfqpoint{2.932344in}{0.559057in}}%
\pgfpathlineto{\pgfqpoint{2.955339in}{0.561851in}}%
\pgfpathlineto{\pgfqpoint{2.965194in}{0.562975in}}%
\pgfpathlineto{\pgfqpoint{2.978334in}{0.564277in}}%
\pgfpathlineto{\pgfqpoint{2.994758in}{0.565420in}}%
\pgfpathlineto{\pgfqpoint{3.004613in}{0.566084in}}%
\pgfpathlineto{\pgfqpoint{3.021038in}{0.568748in}}%
\pgfpathlineto{\pgfqpoint{3.034178in}{0.569670in}}%
\pgfpathlineto{\pgfqpoint{3.047318in}{0.571765in}}%
\pgfpathlineto{\pgfqpoint{3.067027in}{0.574719in}}%
\pgfpathlineto{\pgfqpoint{3.070312in}{0.577102in}}%
\pgfpathlineto{\pgfqpoint{3.073597in}{0.577153in}}%
\pgfpathlineto{\pgfqpoint{3.080167in}{0.578396in}}%
\pgfpathlineto{\pgfqpoint{3.093307in}{0.579409in}}%
\pgfpathlineto{\pgfqpoint{3.109732in}{0.583263in}}%
\pgfpathlineto{\pgfqpoint{3.119587in}{0.584477in}}%
\pgfpathlineto{\pgfqpoint{3.122872in}{0.586690in}}%
\pgfpathlineto{\pgfqpoint{3.172146in}{0.593093in}}%
\pgfpathlineto{\pgfqpoint{3.175431in}{0.596758in}}%
\pgfpathlineto{\pgfqpoint{3.178716in}{0.597999in}}%
\pgfpathlineto{\pgfqpoint{3.185286in}{0.606234in}}%
\pgfpathlineto{\pgfqpoint{3.191856in}{0.607299in}}%
\pgfpathlineto{\pgfqpoint{3.195141in}{0.611843in}}%
\pgfpathlineto{\pgfqpoint{3.204996in}{0.614301in}}%
\pgfpathlineto{\pgfqpoint{3.208280in}{0.617132in}}%
\pgfpathlineto{\pgfqpoint{3.218135in}{0.619636in}}%
\pgfpathlineto{\pgfqpoint{3.221420in}{0.623148in}}%
\pgfpathlineto{\pgfqpoint{3.234560in}{0.626738in}}%
\pgfpathlineto{\pgfqpoint{3.237845in}{0.636430in}}%
\pgfpathlineto{\pgfqpoint{3.241130in}{0.637740in}}%
\pgfpathlineto{\pgfqpoint{3.244415in}{0.641845in}}%
\pgfpathlineto{\pgfqpoint{3.247700in}{0.641941in}}%
\pgfpathlineto{\pgfqpoint{3.254270in}{0.645556in}}%
\pgfpathlineto{\pgfqpoint{3.257555in}{0.649057in}}%
\pgfpathlineto{\pgfqpoint{3.264125in}{0.649551in}}%
\pgfpathlineto{\pgfqpoint{3.267410in}{0.650514in}}%
\pgfpathlineto{\pgfqpoint{3.270695in}{0.653604in}}%
\pgfpathlineto{\pgfqpoint{3.273980in}{0.659248in}}%
\pgfpathlineto{\pgfqpoint{3.280549in}{0.662390in}}%
\pgfpathlineto{\pgfqpoint{3.283834in}{0.665260in}}%
\pgfpathlineto{\pgfqpoint{3.287119in}{0.665873in}}%
\pgfpathlineto{\pgfqpoint{3.290404in}{0.670349in}}%
\pgfpathlineto{\pgfqpoint{3.300259in}{0.677169in}}%
\pgfpathlineto{\pgfqpoint{3.310114in}{0.679666in}}%
\pgfpathlineto{\pgfqpoint{3.316684in}{0.680459in}}%
\pgfpathlineto{\pgfqpoint{3.319969in}{0.685938in}}%
\pgfpathlineto{\pgfqpoint{3.326539in}{0.688587in}}%
\pgfpathlineto{\pgfqpoint{3.333109in}{0.691256in}}%
\pgfpathlineto{\pgfqpoint{3.336394in}{0.694579in}}%
\pgfpathlineto{\pgfqpoint{3.339679in}{0.700559in}}%
\pgfpathlineto{\pgfqpoint{3.342964in}{0.702466in}}%
\pgfpathlineto{\pgfqpoint{3.346249in}{0.707267in}}%
\pgfpathlineto{\pgfqpoint{3.349533in}{0.707462in}}%
\pgfpathlineto{\pgfqpoint{3.352818in}{0.714938in}}%
\pgfpathlineto{\pgfqpoint{3.362673in}{0.720589in}}%
\pgfpathlineto{\pgfqpoint{3.365958in}{0.722371in}}%
\pgfpathlineto{\pgfqpoint{3.372528in}{0.729731in}}%
\pgfpathlineto{\pgfqpoint{3.375813in}{0.735575in}}%
\pgfpathlineto{\pgfqpoint{3.379098in}{0.737251in}}%
\pgfpathlineto{\pgfqpoint{3.385668in}{0.738608in}}%
\pgfpathlineto{\pgfqpoint{3.388953in}{0.743557in}}%
\pgfpathlineto{\pgfqpoint{3.392238in}{0.744410in}}%
\pgfpathlineto{\pgfqpoint{3.405378in}{0.772584in}}%
\pgfpathlineto{\pgfqpoint{3.408663in}{0.773916in}}%
\pgfpathlineto{\pgfqpoint{3.411948in}{0.778355in}}%
\pgfpathlineto{\pgfqpoint{3.415233in}{0.786845in}}%
\pgfpathlineto{\pgfqpoint{3.421802in}{0.790671in}}%
\pgfpathlineto{\pgfqpoint{3.425087in}{0.795894in}}%
\pgfpathlineto{\pgfqpoint{3.428372in}{0.805204in}}%
\pgfpathlineto{\pgfqpoint{3.431657in}{0.809559in}}%
\pgfpathlineto{\pgfqpoint{3.434942in}{0.811133in}}%
\pgfpathlineto{\pgfqpoint{3.438227in}{0.813957in}}%
\pgfpathlineto{\pgfqpoint{3.444797in}{0.841339in}}%
\pgfpathlineto{\pgfqpoint{3.451367in}{0.860986in}}%
\pgfpathlineto{\pgfqpoint{3.457937in}{0.869042in}}%
\pgfpathlineto{\pgfqpoint{3.461222in}{0.875986in}}%
\pgfpathlineto{\pgfqpoint{3.464507in}{0.878302in}}%
\pgfpathlineto{\pgfqpoint{3.471077in}{0.902915in}}%
\pgfpathlineto{\pgfqpoint{3.480932in}{0.907670in}}%
\pgfpathlineto{\pgfqpoint{3.484217in}{0.917350in}}%
\pgfpathlineto{\pgfqpoint{3.487502in}{0.958467in}}%
\pgfpathlineto{\pgfqpoint{3.494071in}{0.964859in}}%
\pgfpathlineto{\pgfqpoint{3.500641in}{0.973964in}}%
\pgfpathlineto{\pgfqpoint{3.507211in}{0.978802in}}%
\pgfpathlineto{\pgfqpoint{3.510496in}{0.979622in}}%
\pgfpathlineto{\pgfqpoint{3.513781in}{0.984170in}}%
\pgfpathlineto{\pgfqpoint{3.517066in}{0.985281in}}%
\pgfpathlineto{\pgfqpoint{3.520351in}{0.998791in}}%
\pgfpathlineto{\pgfqpoint{3.523636in}{1.033901in}}%
\pgfpathlineto{\pgfqpoint{3.526921in}{1.035664in}}%
\pgfpathlineto{\pgfqpoint{3.533491in}{1.062250in}}%
\pgfpathlineto{\pgfqpoint{3.536776in}{1.085170in}}%
\pgfpathlineto{\pgfqpoint{3.540061in}{1.086249in}}%
\pgfpathlineto{\pgfqpoint{3.543346in}{1.090699in}}%
\pgfpathlineto{\pgfqpoint{3.549916in}{1.106916in}}%
\pgfpathlineto{\pgfqpoint{3.553201in}{1.107489in}}%
\pgfpathlineto{\pgfqpoint{3.556486in}{1.147273in}}%
\pgfpathlineto{\pgfqpoint{3.563056in}{1.158713in}}%
\pgfpathlineto{\pgfqpoint{3.566340in}{1.200687in}}%
\pgfpathlineto{\pgfqpoint{3.569625in}{1.210826in}}%
\pgfpathlineto{\pgfqpoint{3.572910in}{1.246004in}}%
\pgfpathlineto{\pgfqpoint{3.576195in}{1.330685in}}%
\pgfpathlineto{\pgfqpoint{3.579480in}{1.331973in}}%
\pgfpathlineto{\pgfqpoint{3.582765in}{1.335408in}}%
\pgfpathlineto{\pgfqpoint{3.586050in}{1.344731in}}%
\pgfpathlineto{\pgfqpoint{3.589335in}{1.365997in}}%
\pgfpathlineto{\pgfqpoint{3.595905in}{1.377610in}}%
\pgfpathlineto{\pgfqpoint{3.599190in}{1.377861in}}%
\pgfpathlineto{\pgfqpoint{3.602475in}{1.387144in}}%
\pgfpathlineto{\pgfqpoint{3.605760in}{1.407874in}}%
\pgfpathlineto{\pgfqpoint{3.609045in}{1.418263in}}%
\pgfpathlineto{\pgfqpoint{3.618900in}{1.468032in}}%
\pgfpathlineto{\pgfqpoint{3.622185in}{1.469002in}}%
\pgfpathlineto{\pgfqpoint{3.625470in}{1.484242in}}%
\pgfpathlineto{\pgfqpoint{3.638609in}{1.590656in}}%
\pgfpathlineto{\pgfqpoint{3.641894in}{1.591417in}}%
\pgfpathlineto{\pgfqpoint{3.645179in}{1.598402in}}%
\pgfpathlineto{\pgfqpoint{3.648464in}{1.609385in}}%
\pgfpathlineto{\pgfqpoint{3.651749in}{1.641089in}}%
\pgfpathlineto{\pgfqpoint{3.658319in}{1.643741in}}%
\pgfpathlineto{\pgfqpoint{3.661604in}{1.686677in}}%
\pgfpathlineto{\pgfqpoint{3.664889in}{1.698656in}}%
\pgfpathlineto{\pgfqpoint{3.674744in}{1.747674in}}%
\pgfpathlineto{\pgfqpoint{3.678029in}{1.784623in}}%
\pgfpathlineto{\pgfqpoint{3.684599in}{1.813609in}}%
\pgfpathlineto{\pgfqpoint{3.687884in}{1.823837in}}%
\pgfpathlineto{\pgfqpoint{3.691169in}{1.861177in}}%
\pgfpathlineto{\pgfqpoint{3.694454in}{1.964044in}}%
\pgfpathlineto{\pgfqpoint{3.697739in}{1.978098in}}%
\pgfpathlineto{\pgfqpoint{3.704309in}{1.996612in}}%
\pgfpathlineto{\pgfqpoint{3.707593in}{2.084430in}}%
\pgfpathlineto{\pgfqpoint{3.710878in}{2.086161in}}%
\pgfpathlineto{\pgfqpoint{3.714163in}{2.161393in}}%
\pgfpathlineto{\pgfqpoint{3.727303in}{2.348341in}}%
\pgfpathlineto{\pgfqpoint{3.730588in}{2.356909in}}%
\pgfpathlineto{\pgfqpoint{3.733873in}{2.370484in}}%
\pgfpathlineto{\pgfqpoint{3.737158in}{2.390364in}}%
\pgfpathlineto{\pgfqpoint{3.743728in}{2.397431in}}%
\pgfpathlineto{\pgfqpoint{3.747013in}{2.454427in}}%
\pgfpathlineto{\pgfqpoint{3.750298in}{2.461339in}}%
\pgfpathlineto{\pgfqpoint{3.753583in}{2.500572in}}%
\pgfpathlineto{\pgfqpoint{3.756868in}{2.568464in}}%
\pgfpathlineto{\pgfqpoint{3.760153in}{2.570521in}}%
\pgfpathlineto{\pgfqpoint{3.763438in}{2.594678in}}%
\pgfpathlineto{\pgfqpoint{3.763438in}{2.594678in}}%
\pgfusepath{stroke}%
\end{pgfscope}%
\begin{pgfscope}%
\pgfpathrectangle{\pgfqpoint{0.698576in}{0.535823in}}{\pgfqpoint{3.613449in}{2.069347in}}%
\pgfusepath{clip}%
\pgfsetbuttcap%
\pgfsetroundjoin%
\pgfsetlinewidth{2.007500pt}%
\definecolor{currentstroke}{rgb}{0.000000,0.000000,1.000000}%
\pgfsetstrokecolor{currentstroke}%
\pgfsetdash{{7.400000pt}{3.200000pt}}{0.000000pt}%
\pgfpathmoveto{\pgfqpoint{0.698576in}{0.535826in}}%
\pgfpathlineto{\pgfqpoint{2.462596in}{0.537085in}}%
\pgfpathlineto{\pgfqpoint{3.011183in}{0.541658in}}%
\pgfpathlineto{\pgfqpoint{3.053888in}{0.542593in}}%
\pgfpathlineto{\pgfqpoint{3.076882in}{0.543686in}}%
\pgfpathlineto{\pgfqpoint{3.132727in}{0.545308in}}%
\pgfpathlineto{\pgfqpoint{3.142581in}{0.546045in}}%
\pgfpathlineto{\pgfqpoint{3.185286in}{0.547289in}}%
\pgfpathlineto{\pgfqpoint{3.221420in}{0.549189in}}%
\pgfpathlineto{\pgfqpoint{3.313399in}{0.554259in}}%
\pgfpathlineto{\pgfqpoint{3.339679in}{0.555491in}}%
\pgfpathlineto{\pgfqpoint{3.359388in}{0.556869in}}%
\pgfpathlineto{\pgfqpoint{3.375813in}{0.557834in}}%
\pgfpathlineto{\pgfqpoint{3.382383in}{0.559574in}}%
\pgfpathlineto{\pgfqpoint{3.405378in}{0.560469in}}%
\pgfpathlineto{\pgfqpoint{3.411948in}{0.562547in}}%
\pgfpathlineto{\pgfqpoint{3.418518in}{0.563409in}}%
\pgfpathlineto{\pgfqpoint{3.441512in}{0.568920in}}%
\pgfpathlineto{\pgfqpoint{3.444797in}{0.568925in}}%
\pgfpathlineto{\pgfqpoint{3.454652in}{0.572671in}}%
\pgfpathlineto{\pgfqpoint{3.464507in}{0.574454in}}%
\pgfpathlineto{\pgfqpoint{3.467792in}{0.576217in}}%
\pgfpathlineto{\pgfqpoint{3.480932in}{0.577102in}}%
\pgfpathlineto{\pgfqpoint{3.484217in}{0.579409in}}%
\pgfpathlineto{\pgfqpoint{3.543346in}{0.589175in}}%
\pgfpathlineto{\pgfqpoint{3.546631in}{0.594381in}}%
\pgfpathlineto{\pgfqpoint{3.549916in}{0.596724in}}%
\pgfpathlineto{\pgfqpoint{3.559771in}{0.597800in}}%
\pgfpathlineto{\pgfqpoint{3.566340in}{0.598871in}}%
\pgfpathlineto{\pgfqpoint{3.569625in}{0.603204in}}%
\pgfpathlineto{\pgfqpoint{3.572910in}{0.605353in}}%
\pgfpathlineto{\pgfqpoint{3.576195in}{0.608998in}}%
\pgfpathlineto{\pgfqpoint{3.589335in}{0.612636in}}%
\pgfpathlineto{\pgfqpoint{3.602475in}{0.615408in}}%
\pgfpathlineto{\pgfqpoint{3.605760in}{0.616814in}}%
\pgfpathlineto{\pgfqpoint{3.609045in}{0.621121in}}%
\pgfpathlineto{\pgfqpoint{3.612330in}{0.621179in}}%
\pgfpathlineto{\pgfqpoint{3.615615in}{0.625361in}}%
\pgfpathlineto{\pgfqpoint{3.632040in}{0.631670in}}%
\pgfpathlineto{\pgfqpoint{3.638609in}{0.643222in}}%
\pgfpathlineto{\pgfqpoint{3.661604in}{0.651313in}}%
\pgfpathlineto{\pgfqpoint{3.668174in}{0.655684in}}%
\pgfpathlineto{\pgfqpoint{3.681314in}{0.658702in}}%
\pgfpathlineto{\pgfqpoint{3.684599in}{0.663550in}}%
\pgfpathlineto{\pgfqpoint{3.687884in}{0.677552in}}%
\pgfpathlineto{\pgfqpoint{3.691169in}{0.678150in}}%
\pgfpathlineto{\pgfqpoint{3.694454in}{0.681670in}}%
\pgfpathlineto{\pgfqpoint{3.697739in}{0.682064in}}%
\pgfpathlineto{\pgfqpoint{3.701024in}{0.691256in}}%
\pgfpathlineto{\pgfqpoint{3.704309in}{0.692369in}}%
\pgfpathlineto{\pgfqpoint{3.707593in}{0.695169in}}%
\pgfpathlineto{\pgfqpoint{3.710878in}{0.695473in}}%
\pgfpathlineto{\pgfqpoint{3.714163in}{0.704275in}}%
\pgfpathlineto{\pgfqpoint{3.717448in}{0.707462in}}%
\pgfpathlineto{\pgfqpoint{3.720733in}{0.717132in}}%
\pgfpathlineto{\pgfqpoint{3.730588in}{0.724733in}}%
\pgfpathlineto{\pgfqpoint{3.733873in}{0.749245in}}%
\pgfpathlineto{\pgfqpoint{3.737158in}{0.759832in}}%
\pgfpathlineto{\pgfqpoint{3.740443in}{0.778355in}}%
\pgfpathlineto{\pgfqpoint{3.743728in}{0.782099in}}%
\pgfpathlineto{\pgfqpoint{3.747013in}{0.788101in}}%
\pgfpathlineto{\pgfqpoint{3.750298in}{0.798588in}}%
\pgfpathlineto{\pgfqpoint{3.756868in}{0.806102in}}%
\pgfpathlineto{\pgfqpoint{3.766723in}{0.809559in}}%
\pgfpathlineto{\pgfqpoint{3.770008in}{0.829923in}}%
\pgfpathlineto{\pgfqpoint{3.773293in}{0.837372in}}%
\pgfpathlineto{\pgfqpoint{3.779862in}{0.843090in}}%
\pgfpathlineto{\pgfqpoint{3.786432in}{0.880102in}}%
\pgfpathlineto{\pgfqpoint{3.789717in}{0.885598in}}%
\pgfpathlineto{\pgfqpoint{3.793002in}{0.902915in}}%
\pgfpathlineto{\pgfqpoint{3.796287in}{0.907670in}}%
\pgfpathlineto{\pgfqpoint{3.802857in}{0.911476in}}%
\pgfpathlineto{\pgfqpoint{3.806142in}{0.933604in}}%
\pgfpathlineto{\pgfqpoint{3.809427in}{0.944411in}}%
\pgfpathlineto{\pgfqpoint{3.812712in}{0.964859in}}%
\pgfpathlineto{\pgfqpoint{3.815997in}{0.969562in}}%
\pgfpathlineto{\pgfqpoint{3.819282in}{0.984170in}}%
\pgfpathlineto{\pgfqpoint{3.825852in}{0.988860in}}%
\pgfpathlineto{\pgfqpoint{3.832422in}{0.996128in}}%
\pgfpathlineto{\pgfqpoint{3.835707in}{0.999071in}}%
\pgfpathlineto{\pgfqpoint{3.842277in}{1.081510in}}%
\pgfpathlineto{\pgfqpoint{3.848847in}{1.088980in}}%
\pgfpathlineto{\pgfqpoint{3.852131in}{1.102991in}}%
\pgfpathlineto{\pgfqpoint{3.855416in}{1.131839in}}%
\pgfpathlineto{\pgfqpoint{3.861986in}{1.154111in}}%
\pgfpathlineto{\pgfqpoint{3.865271in}{1.158011in}}%
\pgfpathlineto{\pgfqpoint{3.868556in}{1.169129in}}%
\pgfpathlineto{\pgfqpoint{3.871841in}{1.173347in}}%
\pgfpathlineto{\pgfqpoint{3.875126in}{1.203527in}}%
\pgfpathlineto{\pgfqpoint{3.878411in}{1.204677in}}%
\pgfpathlineto{\pgfqpoint{3.881696in}{1.314157in}}%
\pgfpathlineto{\pgfqpoint{3.884981in}{1.330685in}}%
\pgfpathlineto{\pgfqpoint{3.888266in}{1.331973in}}%
\pgfpathlineto{\pgfqpoint{3.891551in}{1.335408in}}%
\pgfpathlineto{\pgfqpoint{3.898121in}{1.372864in}}%
\pgfpathlineto{\pgfqpoint{3.901406in}{1.432533in}}%
\pgfpathlineto{\pgfqpoint{3.904691in}{1.468032in}}%
\pgfpathlineto{\pgfqpoint{3.907976in}{1.547703in}}%
\pgfpathlineto{\pgfqpoint{3.911261in}{1.551511in}}%
\pgfpathlineto{\pgfqpoint{3.917831in}{1.590656in}}%
\pgfpathlineto{\pgfqpoint{3.921116in}{1.591417in}}%
\pgfpathlineto{\pgfqpoint{3.924400in}{1.598402in}}%
\pgfpathlineto{\pgfqpoint{3.927685in}{1.656159in}}%
\pgfpathlineto{\pgfqpoint{3.930970in}{1.667041in}}%
\pgfpathlineto{\pgfqpoint{3.934255in}{1.669741in}}%
\pgfpathlineto{\pgfqpoint{3.937540in}{1.670183in}}%
\pgfpathlineto{\pgfqpoint{3.944110in}{1.681558in}}%
\pgfpathlineto{\pgfqpoint{3.953965in}{1.695107in}}%
\pgfpathlineto{\pgfqpoint{3.957250in}{1.699693in}}%
\pgfpathlineto{\pgfqpoint{3.967105in}{1.782919in}}%
\pgfpathlineto{\pgfqpoint{3.970390in}{1.823837in}}%
\pgfpathlineto{\pgfqpoint{3.973675in}{1.840877in}}%
\pgfpathlineto{\pgfqpoint{3.976960in}{1.878767in}}%
\pgfpathlineto{\pgfqpoint{3.993384in}{2.193863in}}%
\pgfpathlineto{\pgfqpoint{3.999954in}{2.228683in}}%
\pgfpathlineto{\pgfqpoint{4.003239in}{2.274447in}}%
\pgfpathlineto{\pgfqpoint{4.006524in}{2.285526in}}%
\pgfpathlineto{\pgfqpoint{4.009809in}{2.285715in}}%
\pgfpathlineto{\pgfqpoint{4.013094in}{2.292146in}}%
\pgfpathlineto{\pgfqpoint{4.016379in}{2.292389in}}%
\pgfpathlineto{\pgfqpoint{4.019664in}{2.304290in}}%
\pgfpathlineto{\pgfqpoint{4.022949in}{2.305092in}}%
\pgfpathlineto{\pgfqpoint{4.026234in}{2.324519in}}%
\pgfpathlineto{\pgfqpoint{4.029519in}{2.332347in}}%
\pgfpathlineto{\pgfqpoint{4.032804in}{2.345824in}}%
\pgfpathlineto{\pgfqpoint{4.036089in}{2.390364in}}%
\pgfpathlineto{\pgfqpoint{4.039374in}{2.461339in}}%
\pgfpathlineto{\pgfqpoint{4.042659in}{2.570521in}}%
\pgfpathlineto{\pgfqpoint{4.042659in}{2.570521in}}%
\pgfusepath{stroke}%
\end{pgfscope}%
\begin{pgfscope}%
\pgfsetrectcap%
\pgfsetmiterjoin%
\pgfsetlinewidth{0.803000pt}%
\definecolor{currentstroke}{rgb}{0.000000,0.000000,0.000000}%
\pgfsetstrokecolor{currentstroke}%
\pgfsetdash{}{0pt}%
\pgfpathmoveto{\pgfqpoint{0.698576in}{0.535823in}}%
\pgfpathlineto{\pgfqpoint{0.698576in}{2.605170in}}%
\pgfusepath{stroke}%
\end{pgfscope}%
\begin{pgfscope}%
\pgfsetrectcap%
\pgfsetmiterjoin%
\pgfsetlinewidth{0.803000pt}%
\definecolor{currentstroke}{rgb}{0.000000,0.000000,0.000000}%
\pgfsetstrokecolor{currentstroke}%
\pgfsetdash{}{0pt}%
\pgfpathmoveto{\pgfqpoint{4.312025in}{0.535823in}}%
\pgfpathlineto{\pgfqpoint{4.312025in}{2.605170in}}%
\pgfusepath{stroke}%
\end{pgfscope}%
\begin{pgfscope}%
\pgfsetrectcap%
\pgfsetmiterjoin%
\pgfsetlinewidth{0.803000pt}%
\definecolor{currentstroke}{rgb}{0.000000,0.000000,0.000000}%
\pgfsetstrokecolor{currentstroke}%
\pgfsetdash{}{0pt}%
\pgfpathmoveto{\pgfqpoint{0.698576in}{0.535823in}}%
\pgfpathlineto{\pgfqpoint{4.312025in}{0.535823in}}%
\pgfusepath{stroke}%
\end{pgfscope}%
\begin{pgfscope}%
\pgfsetrectcap%
\pgfsetmiterjoin%
\pgfsetlinewidth{0.803000pt}%
\definecolor{currentstroke}{rgb}{0.000000,0.000000,0.000000}%
\pgfsetstrokecolor{currentstroke}%
\pgfsetdash{}{0pt}%
\pgfpathmoveto{\pgfqpoint{0.698576in}{2.605170in}}%
\pgfpathlineto{\pgfqpoint{4.312025in}{2.605170in}}%
\pgfusepath{stroke}%
\end{pgfscope}%
\begin{pgfscope}%
\pgfsetbuttcap%
\pgfsetroundjoin%
\pgfsetlinewidth{2.007500pt}%
\definecolor{currentstroke}{rgb}{1.000000,0.843137,0.000000}%
\pgfsetstrokecolor{currentstroke}%
\pgfsetdash{{7.400000pt}{3.200000pt}}{0.000000pt}%
\pgfpathmoveto{\pgfqpoint{0.748576in}{2.511420in}}%
\pgfpathlineto{\pgfqpoint{0.998576in}{2.511420in}}%
\pgfusepath{stroke}%
\end{pgfscope}%
\begin{pgfscope}%
\definecolor{textcolor}{rgb}{0.000000,0.000000,0.000000}%
\pgfsetstrokecolor{textcolor}%
\pgfsetfillcolor{textcolor}%
\pgftext[x=1.023576in,y=2.467670in,left,base]{\color{textcolor}\rmfamily\fontsize{9.000000}{10.800000}\selectfont FT+htd}%
\end{pgfscope}%
\begin{pgfscope}%
\pgfsetbuttcap%
\pgfsetroundjoin%
\pgfsetlinewidth{2.007500pt}%
\definecolor{currentstroke}{rgb}{1.000000,0.694118,0.305882}%
\pgfsetstrokecolor{currentstroke}%
\pgfsetdash{{2.000000pt}{3.300000pt}}{0.000000pt}%
\pgfpathmoveto{\pgfqpoint{0.748576in}{2.349620in}}%
\pgfpathlineto{\pgfqpoint{0.998576in}{2.349620in}}%
\pgfusepath{stroke}%
\end{pgfscope}%
\begin{pgfscope}%
\definecolor{textcolor}{rgb}{0.000000,0.000000,0.000000}%
\pgfsetstrokecolor{textcolor}%
\pgfsetfillcolor{textcolor}%
\pgftext[x=1.023576in,y=2.305870in,left,base]{\color{textcolor}\rmfamily\fontsize{9.000000}{10.800000}\selectfont FT+Flow}%
\end{pgfscope}%
\begin{pgfscope}%
\pgfsetrectcap%
\pgfsetroundjoin%
\pgfsetlinewidth{2.007500pt}%
\definecolor{currentstroke}{rgb}{0.980392,0.529412,0.458824}%
\pgfsetstrokecolor{currentstroke}%
\pgfsetdash{}{0pt}%
\pgfpathmoveto{\pgfqpoint{0.748576in}{2.187821in}}%
\pgfpathlineto{\pgfqpoint{0.998576in}{2.187821in}}%
\pgfusepath{stroke}%
\end{pgfscope}%
\begin{pgfscope}%
\definecolor{textcolor}{rgb}{0.000000,0.000000,0.000000}%
\pgfsetstrokecolor{textcolor}%
\pgfsetfillcolor{textcolor}%
\pgftext[x=1.023576in,y=2.144071in,left,base]{\color{textcolor}\rmfamily\fontsize{9.000000}{10.800000}\selectfont FT+Tamaki}%
\end{pgfscope}%
\begin{pgfscope}%
\pgfsetbuttcap%
\pgfsetroundjoin%
\pgfsetlinewidth{2.007500pt}%
\definecolor{currentstroke}{rgb}{0.866667,0.058824,0.058824}%
\pgfsetstrokecolor{currentstroke}%
\pgfsetdash{{7.400000pt}{3.200000pt}}{0.000000pt}%
\pgfpathmoveto{\pgfqpoint{0.748576in}{2.026021in}}%
\pgfpathlineto{\pgfqpoint{0.998576in}{2.026021in}}%
\pgfusepath{stroke}%
\end{pgfscope}%
\begin{pgfscope}%
\definecolor{textcolor}{rgb}{0.000000,0.000000,0.000000}%
\pgfsetstrokecolor{textcolor}%
\pgfsetfillcolor{textcolor}%
\pgftext[x=1.023576in,y=1.982271in,left,base]{\color{textcolor}\rmfamily\fontsize{9.000000}{10.800000}\selectfont cachet}%
\end{pgfscope}%
\begin{pgfscope}%
\pgfsetbuttcap%
\pgfsetroundjoin%
\pgfsetlinewidth{2.007500pt}%
\definecolor{currentstroke}{rgb}{0.917647,0.372549,0.580392}%
\pgfsetstrokecolor{currentstroke}%
\pgfsetdash{{2.000000pt}{3.300000pt}}{0.000000pt}%
\pgfpathmoveto{\pgfqpoint{0.748576in}{1.864222in}}%
\pgfpathlineto{\pgfqpoint{0.998576in}{1.864222in}}%
\pgfusepath{stroke}%
\end{pgfscope}%
\begin{pgfscope}%
\definecolor{textcolor}{rgb}{0.000000,0.000000,0.000000}%
\pgfsetstrokecolor{textcolor}%
\pgfsetfillcolor{textcolor}%
\pgftext[x=1.023576in,y=1.820472in,left,base]{\color{textcolor}\rmfamily\fontsize{9.000000}{10.800000}\selectfont miniC2D}%
\end{pgfscope}%
\begin{pgfscope}%
\pgfsetrectcap%
\pgfsetroundjoin%
\pgfsetlinewidth{2.007500pt}%
\definecolor{currentstroke}{rgb}{0.615686,0.007843,0.843137}%
\pgfsetstrokecolor{currentstroke}%
\pgfsetdash{}{0pt}%
\pgfpathmoveto{\pgfqpoint{0.748576in}{1.702422in}}%
\pgfpathlineto{\pgfqpoint{0.998576in}{1.702422in}}%
\pgfusepath{stroke}%
\end{pgfscope}%
\begin{pgfscope}%
\definecolor{textcolor}{rgb}{0.000000,0.000000,0.000000}%
\pgfsetstrokecolor{textcolor}%
\pgfsetfillcolor{textcolor}%
\pgftext[x=1.023576in,y=1.658672in,left,base]{\color{textcolor}\rmfamily\fontsize{9.000000}{10.800000}\selectfont d4}%
\end{pgfscope}%
\begin{pgfscope}%
\pgfsetbuttcap%
\pgfsetroundjoin%
\pgfsetlinewidth{2.007500pt}%
\definecolor{currentstroke}{rgb}{0.000000,0.000000,1.000000}%
\pgfsetstrokecolor{currentstroke}%
\pgfsetdash{{7.400000pt}{3.200000pt}}{0.000000pt}%
\pgfpathmoveto{\pgfqpoint{0.748576in}{1.540622in}}%
\pgfpathlineto{\pgfqpoint{0.998576in}{1.540622in}}%
\pgfusepath{stroke}%
\end{pgfscope}%
\begin{pgfscope}%
\definecolor{textcolor}{rgb}{0.000000,0.000000,0.000000}%
\pgfsetstrokecolor{textcolor}%
\pgfsetfillcolor{textcolor}%
\pgftext[x=1.023576in,y=1.496872in,left,base]{\color{textcolor}\rmfamily\fontsize{9.000000}{10.800000}\selectfont VBS}%
\end{pgfscope}%
\end{pgfpicture}%
\makeatother%
\endgroup%

	\caption{\label{fig:cachet-cactus} A cactus plot of the number of benchmarks solved by various methods out of 1091 probabilistic inference benchmarks. Although our contributions \textbf{FT+*} solve fewer benchmarks than the existing weighted model counters \tool{cachet}, \tool{miniC2D}, and \tool{d4}, they improve the virtual best solver on 231 benchmarks. Note that \tool{dynQBF}, \tool{dynasp}, and \tool{SharpSAT} are unweighted model counters and so cannot solve these weighted benchmarks.}
\end{figure}

\begin{figure}[t]
	\centering
	%% Creator: Matplotlib, PGF backend
%%
%% To include the figure in your LaTeX document, write
%%   \input{<filename>.pgf}
%%
%% Make sure the required packages are loaded in your preamble
%%   \usepackage{pgf}
%%
%% and, on pdftex
%%   \usepackage[utf8]{inputenc}\DeclareUnicodeCharacter{2212}{-}
%%
%% or, on luatex and xetex
%%   \usepackage{unicode-math}
%%
%% Figures using additional raster images can only be included by \input if
%% they are in the same directory as the main LaTeX file. For loading figures
%% from other directories you can use the `import` package
%%   \usepackage{import}
%%
%% and then include the figures with
%%   \import{<path to file>}{<filename>.pgf}
%%
%% Matplotlib used the following preamble
%%   \usepackage[utf8x]{inputenc}
%%   \usepackage[T1]{fontenc}
%%
\begingroup%
\makeatletter%
\begin{pgfpicture}%
\pgfpathrectangle{\pgfpointorigin}{\pgfqpoint{6.000000in}{3.400000in}}%
\pgfusepath{use as bounding box, clip}%
\begin{pgfscope}%
\pgfsetbuttcap%
\pgfsetmiterjoin%
\definecolor{currentfill}{rgb}{1.000000,1.000000,1.000000}%
\pgfsetfillcolor{currentfill}%
\pgfsetlinewidth{0.000000pt}%
\definecolor{currentstroke}{rgb}{1.000000,1.000000,1.000000}%
\pgfsetstrokecolor{currentstroke}%
\pgfsetdash{}{0pt}%
\pgfpathmoveto{\pgfqpoint{0.000000in}{0.000000in}}%
\pgfpathlineto{\pgfqpoint{6.000000in}{0.000000in}}%
\pgfpathlineto{\pgfqpoint{6.000000in}{3.400000in}}%
\pgfpathlineto{\pgfqpoint{0.000000in}{3.400000in}}%
\pgfpathclose%
\pgfusepath{fill}%
\end{pgfscope}%
\begin{pgfscope}%
\pgfsetbuttcap%
\pgfsetmiterjoin%
\definecolor{currentfill}{rgb}{1.000000,1.000000,1.000000}%
\pgfsetfillcolor{currentfill}%
\pgfsetlinewidth{0.000000pt}%
\definecolor{currentstroke}{rgb}{0.000000,0.000000,0.000000}%
\pgfsetstrokecolor{currentstroke}%
\pgfsetstrokeopacity{0.000000}%
\pgfsetdash{}{0pt}%
\pgfpathmoveto{\pgfqpoint{0.682376in}{0.535823in}}%
\pgfpathlineto{\pgfqpoint{5.785764in}{0.535823in}}%
\pgfpathlineto{\pgfqpoint{5.785764in}{3.250000in}}%
\pgfpathlineto{\pgfqpoint{0.682376in}{3.250000in}}%
\pgfpathclose%
\pgfusepath{fill}%
\end{pgfscope}%
\begin{pgfscope}%
\pgfsetbuttcap%
\pgfsetroundjoin%
\definecolor{currentfill}{rgb}{0.000000,0.000000,0.000000}%
\pgfsetfillcolor{currentfill}%
\pgfsetlinewidth{0.803000pt}%
\definecolor{currentstroke}{rgb}{0.000000,0.000000,0.000000}%
\pgfsetstrokecolor{currentstroke}%
\pgfsetdash{}{0pt}%
\pgfsys@defobject{currentmarker}{\pgfqpoint{0.000000in}{-0.048611in}}{\pgfqpoint{0.000000in}{0.000000in}}{%
\pgfpathmoveto{\pgfqpoint{0.000000in}{0.000000in}}%
\pgfpathlineto{\pgfqpoint{0.000000in}{-0.048611in}}%
\pgfusepath{stroke,fill}%
}%
\begin{pgfscope}%
\pgfsys@transformshift{0.806849in}{0.535823in}%
\pgfsys@useobject{currentmarker}{}%
\end{pgfscope}%
\end{pgfscope}%
\begin{pgfscope}%
\definecolor{textcolor}{rgb}{0.000000,0.000000,0.000000}%
\pgfsetstrokecolor{textcolor}%
\pgfsetfillcolor{textcolor}%
\pgftext[x=0.806849in,y=0.438600in,,top]{\color{textcolor}\rmfamily\fontsize{9.000000}{10.800000}\selectfont \(\displaystyle {10}\)}%
\end{pgfscope}%
\begin{pgfscope}%
\pgfsetbuttcap%
\pgfsetroundjoin%
\definecolor{currentfill}{rgb}{0.000000,0.000000,0.000000}%
\pgfsetfillcolor{currentfill}%
\pgfsetlinewidth{0.803000pt}%
\definecolor{currentstroke}{rgb}{0.000000,0.000000,0.000000}%
\pgfsetstrokecolor{currentstroke}%
\pgfsetdash{}{0pt}%
\pgfsys@defobject{currentmarker}{\pgfqpoint{0.000000in}{-0.048611in}}{\pgfqpoint{0.000000in}{0.000000in}}{%
\pgfpathmoveto{\pgfqpoint{0.000000in}{0.000000in}}%
\pgfpathlineto{\pgfqpoint{0.000000in}{-0.048611in}}%
\pgfusepath{stroke,fill}%
}%
\begin{pgfscope}%
\pgfsys@transformshift{1.429213in}{0.535823in}%
\pgfsys@useobject{currentmarker}{}%
\end{pgfscope}%
\end{pgfscope}%
\begin{pgfscope}%
\definecolor{textcolor}{rgb}{0.000000,0.000000,0.000000}%
\pgfsetstrokecolor{textcolor}%
\pgfsetfillcolor{textcolor}%
\pgftext[x=1.429213in,y=0.438600in,,top]{\color{textcolor}\rmfamily\fontsize{9.000000}{10.800000}\selectfont \(\displaystyle {15}\)}%
\end{pgfscope}%
\begin{pgfscope}%
\pgfsetbuttcap%
\pgfsetroundjoin%
\definecolor{currentfill}{rgb}{0.000000,0.000000,0.000000}%
\pgfsetfillcolor{currentfill}%
\pgfsetlinewidth{0.803000pt}%
\definecolor{currentstroke}{rgb}{0.000000,0.000000,0.000000}%
\pgfsetstrokecolor{currentstroke}%
\pgfsetdash{}{0pt}%
\pgfsys@defobject{currentmarker}{\pgfqpoint{0.000000in}{-0.048611in}}{\pgfqpoint{0.000000in}{0.000000in}}{%
\pgfpathmoveto{\pgfqpoint{0.000000in}{0.000000in}}%
\pgfpathlineto{\pgfqpoint{0.000000in}{-0.048611in}}%
\pgfusepath{stroke,fill}%
}%
\begin{pgfscope}%
\pgfsys@transformshift{2.051578in}{0.535823in}%
\pgfsys@useobject{currentmarker}{}%
\end{pgfscope}%
\end{pgfscope}%
\begin{pgfscope}%
\definecolor{textcolor}{rgb}{0.000000,0.000000,0.000000}%
\pgfsetstrokecolor{textcolor}%
\pgfsetfillcolor{textcolor}%
\pgftext[x=2.051578in,y=0.438600in,,top]{\color{textcolor}\rmfamily\fontsize{9.000000}{10.800000}\selectfont \(\displaystyle {20}\)}%
\end{pgfscope}%
\begin{pgfscope}%
\pgfsetbuttcap%
\pgfsetroundjoin%
\definecolor{currentfill}{rgb}{0.000000,0.000000,0.000000}%
\pgfsetfillcolor{currentfill}%
\pgfsetlinewidth{0.803000pt}%
\definecolor{currentstroke}{rgb}{0.000000,0.000000,0.000000}%
\pgfsetstrokecolor{currentstroke}%
\pgfsetdash{}{0pt}%
\pgfsys@defobject{currentmarker}{\pgfqpoint{0.000000in}{-0.048611in}}{\pgfqpoint{0.000000in}{0.000000in}}{%
\pgfpathmoveto{\pgfqpoint{0.000000in}{0.000000in}}%
\pgfpathlineto{\pgfqpoint{0.000000in}{-0.048611in}}%
\pgfusepath{stroke,fill}%
}%
\begin{pgfscope}%
\pgfsys@transformshift{2.673942in}{0.535823in}%
\pgfsys@useobject{currentmarker}{}%
\end{pgfscope}%
\end{pgfscope}%
\begin{pgfscope}%
\definecolor{textcolor}{rgb}{0.000000,0.000000,0.000000}%
\pgfsetstrokecolor{textcolor}%
\pgfsetfillcolor{textcolor}%
\pgftext[x=2.673942in,y=0.438600in,,top]{\color{textcolor}\rmfamily\fontsize{9.000000}{10.800000}\selectfont \(\displaystyle {25}\)}%
\end{pgfscope}%
\begin{pgfscope}%
\pgfsetbuttcap%
\pgfsetroundjoin%
\definecolor{currentfill}{rgb}{0.000000,0.000000,0.000000}%
\pgfsetfillcolor{currentfill}%
\pgfsetlinewidth{0.803000pt}%
\definecolor{currentstroke}{rgb}{0.000000,0.000000,0.000000}%
\pgfsetstrokecolor{currentstroke}%
\pgfsetdash{}{0pt}%
\pgfsys@defobject{currentmarker}{\pgfqpoint{0.000000in}{-0.048611in}}{\pgfqpoint{0.000000in}{0.000000in}}{%
\pgfpathmoveto{\pgfqpoint{0.000000in}{0.000000in}}%
\pgfpathlineto{\pgfqpoint{0.000000in}{-0.048611in}}%
\pgfusepath{stroke,fill}%
}%
\begin{pgfscope}%
\pgfsys@transformshift{3.296306in}{0.535823in}%
\pgfsys@useobject{currentmarker}{}%
\end{pgfscope}%
\end{pgfscope}%
\begin{pgfscope}%
\definecolor{textcolor}{rgb}{0.000000,0.000000,0.000000}%
\pgfsetstrokecolor{textcolor}%
\pgfsetfillcolor{textcolor}%
\pgftext[x=3.296306in,y=0.438600in,,top]{\color{textcolor}\rmfamily\fontsize{9.000000}{10.800000}\selectfont \(\displaystyle {30}\)}%
\end{pgfscope}%
\begin{pgfscope}%
\pgfsetbuttcap%
\pgfsetroundjoin%
\definecolor{currentfill}{rgb}{0.000000,0.000000,0.000000}%
\pgfsetfillcolor{currentfill}%
\pgfsetlinewidth{0.803000pt}%
\definecolor{currentstroke}{rgb}{0.000000,0.000000,0.000000}%
\pgfsetstrokecolor{currentstroke}%
\pgfsetdash{}{0pt}%
\pgfsys@defobject{currentmarker}{\pgfqpoint{0.000000in}{-0.048611in}}{\pgfqpoint{0.000000in}{0.000000in}}{%
\pgfpathmoveto{\pgfqpoint{0.000000in}{0.000000in}}%
\pgfpathlineto{\pgfqpoint{0.000000in}{-0.048611in}}%
\pgfusepath{stroke,fill}%
}%
\begin{pgfscope}%
\pgfsys@transformshift{3.918671in}{0.535823in}%
\pgfsys@useobject{currentmarker}{}%
\end{pgfscope}%
\end{pgfscope}%
\begin{pgfscope}%
\definecolor{textcolor}{rgb}{0.000000,0.000000,0.000000}%
\pgfsetstrokecolor{textcolor}%
\pgfsetfillcolor{textcolor}%
\pgftext[x=3.918671in,y=0.438600in,,top]{\color{textcolor}\rmfamily\fontsize{9.000000}{10.800000}\selectfont \(\displaystyle {35}\)}%
\end{pgfscope}%
\begin{pgfscope}%
\pgfsetbuttcap%
\pgfsetroundjoin%
\definecolor{currentfill}{rgb}{0.000000,0.000000,0.000000}%
\pgfsetfillcolor{currentfill}%
\pgfsetlinewidth{0.803000pt}%
\definecolor{currentstroke}{rgb}{0.000000,0.000000,0.000000}%
\pgfsetstrokecolor{currentstroke}%
\pgfsetdash{}{0pt}%
\pgfsys@defobject{currentmarker}{\pgfqpoint{0.000000in}{-0.048611in}}{\pgfqpoint{0.000000in}{0.000000in}}{%
\pgfpathmoveto{\pgfqpoint{0.000000in}{0.000000in}}%
\pgfpathlineto{\pgfqpoint{0.000000in}{-0.048611in}}%
\pgfusepath{stroke,fill}%
}%
\begin{pgfscope}%
\pgfsys@transformshift{4.541035in}{0.535823in}%
\pgfsys@useobject{currentmarker}{}%
\end{pgfscope}%
\end{pgfscope}%
\begin{pgfscope}%
\definecolor{textcolor}{rgb}{0.000000,0.000000,0.000000}%
\pgfsetstrokecolor{textcolor}%
\pgfsetfillcolor{textcolor}%
\pgftext[x=4.541035in,y=0.438600in,,top]{\color{textcolor}\rmfamily\fontsize{9.000000}{10.800000}\selectfont \(\displaystyle {40}\)}%
\end{pgfscope}%
\begin{pgfscope}%
\pgfsetbuttcap%
\pgfsetroundjoin%
\definecolor{currentfill}{rgb}{0.000000,0.000000,0.000000}%
\pgfsetfillcolor{currentfill}%
\pgfsetlinewidth{0.803000pt}%
\definecolor{currentstroke}{rgb}{0.000000,0.000000,0.000000}%
\pgfsetstrokecolor{currentstroke}%
\pgfsetdash{}{0pt}%
\pgfsys@defobject{currentmarker}{\pgfqpoint{0.000000in}{-0.048611in}}{\pgfqpoint{0.000000in}{0.000000in}}{%
\pgfpathmoveto{\pgfqpoint{0.000000in}{0.000000in}}%
\pgfpathlineto{\pgfqpoint{0.000000in}{-0.048611in}}%
\pgfusepath{stroke,fill}%
}%
\begin{pgfscope}%
\pgfsys@transformshift{5.163400in}{0.535823in}%
\pgfsys@useobject{currentmarker}{}%
\end{pgfscope}%
\end{pgfscope}%
\begin{pgfscope}%
\definecolor{textcolor}{rgb}{0.000000,0.000000,0.000000}%
\pgfsetstrokecolor{textcolor}%
\pgfsetfillcolor{textcolor}%
\pgftext[x=5.163400in,y=0.438600in,,top]{\color{textcolor}\rmfamily\fontsize{9.000000}{10.800000}\selectfont \(\displaystyle {45}\)}%
\end{pgfscope}%
\begin{pgfscope}%
\pgfsetbuttcap%
\pgfsetroundjoin%
\definecolor{currentfill}{rgb}{0.000000,0.000000,0.000000}%
\pgfsetfillcolor{currentfill}%
\pgfsetlinewidth{0.803000pt}%
\definecolor{currentstroke}{rgb}{0.000000,0.000000,0.000000}%
\pgfsetstrokecolor{currentstroke}%
\pgfsetdash{}{0pt}%
\pgfsys@defobject{currentmarker}{\pgfqpoint{0.000000in}{-0.048611in}}{\pgfqpoint{0.000000in}{0.000000in}}{%
\pgfpathmoveto{\pgfqpoint{0.000000in}{0.000000in}}%
\pgfpathlineto{\pgfqpoint{0.000000in}{-0.048611in}}%
\pgfusepath{stroke,fill}%
}%
\begin{pgfscope}%
\pgfsys@transformshift{5.785764in}{0.535823in}%
\pgfsys@useobject{currentmarker}{}%
\end{pgfscope}%
\end{pgfscope}%
\begin{pgfscope}%
\definecolor{textcolor}{rgb}{0.000000,0.000000,0.000000}%
\pgfsetstrokecolor{textcolor}%
\pgfsetfillcolor{textcolor}%
\pgftext[x=5.785764in,y=0.438600in,,top]{\color{textcolor}\rmfamily\fontsize{9.000000}{10.800000}\selectfont \(\displaystyle {50}\)}%
\end{pgfscope}%
\begin{pgfscope}%
\definecolor{textcolor}{rgb}{0.000000,0.000000,0.000000}%
\pgfsetstrokecolor{textcolor}%
\pgfsetfillcolor{textcolor}%
\pgftext[x=3.234070in,y=0.272655in,,top]{\color{textcolor}\rmfamily\fontsize{10.000000}{12.000000}\selectfont Upper bound on carving width}%
\end{pgfscope}%
\begin{pgfscope}%
\pgfsetbuttcap%
\pgfsetroundjoin%
\definecolor{currentfill}{rgb}{0.000000,0.000000,0.000000}%
\pgfsetfillcolor{currentfill}%
\pgfsetlinewidth{0.803000pt}%
\definecolor{currentstroke}{rgb}{0.000000,0.000000,0.000000}%
\pgfsetstrokecolor{currentstroke}%
\pgfsetdash{}{0pt}%
\pgfsys@defobject{currentmarker}{\pgfqpoint{-0.048611in}{0.000000in}}{\pgfqpoint{-0.000000in}{0.000000in}}{%
\pgfpathmoveto{\pgfqpoint{-0.000000in}{0.000000in}}%
\pgfpathlineto{\pgfqpoint{-0.048611in}{0.000000in}}%
\pgfusepath{stroke,fill}%
}%
\begin{pgfscope}%
\pgfsys@transformshift{0.682376in}{0.535823in}%
\pgfsys@useobject{currentmarker}{}%
\end{pgfscope}%
\end{pgfscope}%
\begin{pgfscope}%
\definecolor{textcolor}{rgb}{0.000000,0.000000,0.000000}%
\pgfsetstrokecolor{textcolor}%
\pgfsetfillcolor{textcolor}%
\pgftext[x=0.520918in, y=0.492778in, left, base]{\color{textcolor}\rmfamily\fontsize{9.000000}{10.800000}\selectfont \(\displaystyle {0}\)}%
\end{pgfscope}%
\begin{pgfscope}%
\pgfsetbuttcap%
\pgfsetroundjoin%
\definecolor{currentfill}{rgb}{0.000000,0.000000,0.000000}%
\pgfsetfillcolor{currentfill}%
\pgfsetlinewidth{0.803000pt}%
\definecolor{currentstroke}{rgb}{0.000000,0.000000,0.000000}%
\pgfsetstrokecolor{currentstroke}%
\pgfsetdash{}{0pt}%
\pgfsys@defobject{currentmarker}{\pgfqpoint{-0.048611in}{0.000000in}}{\pgfqpoint{-0.000000in}{0.000000in}}{%
\pgfpathmoveto{\pgfqpoint{-0.000000in}{0.000000in}}%
\pgfpathlineto{\pgfqpoint{-0.048611in}{0.000000in}}%
\pgfusepath{stroke,fill}%
}%
\begin{pgfscope}%
\pgfsys@transformshift{0.682376in}{1.029309in}%
\pgfsys@useobject{currentmarker}{}%
\end{pgfscope}%
\end{pgfscope}%
\begin{pgfscope}%
\definecolor{textcolor}{rgb}{0.000000,0.000000,0.000000}%
\pgfsetstrokecolor{textcolor}%
\pgfsetfillcolor{textcolor}%
\pgftext[x=0.392446in, y=0.986264in, left, base]{\color{textcolor}\rmfamily\fontsize{9.000000}{10.800000}\selectfont \(\displaystyle {200}\)}%
\end{pgfscope}%
\begin{pgfscope}%
\pgfsetbuttcap%
\pgfsetroundjoin%
\definecolor{currentfill}{rgb}{0.000000,0.000000,0.000000}%
\pgfsetfillcolor{currentfill}%
\pgfsetlinewidth{0.803000pt}%
\definecolor{currentstroke}{rgb}{0.000000,0.000000,0.000000}%
\pgfsetstrokecolor{currentstroke}%
\pgfsetdash{}{0pt}%
\pgfsys@defobject{currentmarker}{\pgfqpoint{-0.048611in}{0.000000in}}{\pgfqpoint{-0.000000in}{0.000000in}}{%
\pgfpathmoveto{\pgfqpoint{-0.000000in}{0.000000in}}%
\pgfpathlineto{\pgfqpoint{-0.048611in}{0.000000in}}%
\pgfusepath{stroke,fill}%
}%
\begin{pgfscope}%
\pgfsys@transformshift{0.682376in}{1.522796in}%
\pgfsys@useobject{currentmarker}{}%
\end{pgfscope}%
\end{pgfscope}%
\begin{pgfscope}%
\definecolor{textcolor}{rgb}{0.000000,0.000000,0.000000}%
\pgfsetstrokecolor{textcolor}%
\pgfsetfillcolor{textcolor}%
\pgftext[x=0.392446in, y=1.479751in, left, base]{\color{textcolor}\rmfamily\fontsize{9.000000}{10.800000}\selectfont \(\displaystyle {400}\)}%
\end{pgfscope}%
\begin{pgfscope}%
\pgfsetbuttcap%
\pgfsetroundjoin%
\definecolor{currentfill}{rgb}{0.000000,0.000000,0.000000}%
\pgfsetfillcolor{currentfill}%
\pgfsetlinewidth{0.803000pt}%
\definecolor{currentstroke}{rgb}{0.000000,0.000000,0.000000}%
\pgfsetstrokecolor{currentstroke}%
\pgfsetdash{}{0pt}%
\pgfsys@defobject{currentmarker}{\pgfqpoint{-0.048611in}{0.000000in}}{\pgfqpoint{-0.000000in}{0.000000in}}{%
\pgfpathmoveto{\pgfqpoint{-0.000000in}{0.000000in}}%
\pgfpathlineto{\pgfqpoint{-0.048611in}{0.000000in}}%
\pgfusepath{stroke,fill}%
}%
\begin{pgfscope}%
\pgfsys@transformshift{0.682376in}{2.016283in}%
\pgfsys@useobject{currentmarker}{}%
\end{pgfscope}%
\end{pgfscope}%
\begin{pgfscope}%
\definecolor{textcolor}{rgb}{0.000000,0.000000,0.000000}%
\pgfsetstrokecolor{textcolor}%
\pgfsetfillcolor{textcolor}%
\pgftext[x=0.392446in, y=1.973238in, left, base]{\color{textcolor}\rmfamily\fontsize{9.000000}{10.800000}\selectfont \(\displaystyle {600}\)}%
\end{pgfscope}%
\begin{pgfscope}%
\pgfsetbuttcap%
\pgfsetroundjoin%
\definecolor{currentfill}{rgb}{0.000000,0.000000,0.000000}%
\pgfsetfillcolor{currentfill}%
\pgfsetlinewidth{0.803000pt}%
\definecolor{currentstroke}{rgb}{0.000000,0.000000,0.000000}%
\pgfsetstrokecolor{currentstroke}%
\pgfsetdash{}{0pt}%
\pgfsys@defobject{currentmarker}{\pgfqpoint{-0.048611in}{0.000000in}}{\pgfqpoint{-0.000000in}{0.000000in}}{%
\pgfpathmoveto{\pgfqpoint{-0.000000in}{0.000000in}}%
\pgfpathlineto{\pgfqpoint{-0.048611in}{0.000000in}}%
\pgfusepath{stroke,fill}%
}%
\begin{pgfscope}%
\pgfsys@transformshift{0.682376in}{2.509770in}%
\pgfsys@useobject{currentmarker}{}%
\end{pgfscope}%
\end{pgfscope}%
\begin{pgfscope}%
\definecolor{textcolor}{rgb}{0.000000,0.000000,0.000000}%
\pgfsetstrokecolor{textcolor}%
\pgfsetfillcolor{textcolor}%
\pgftext[x=0.392446in, y=2.466725in, left, base]{\color{textcolor}\rmfamily\fontsize{9.000000}{10.800000}\selectfont \(\displaystyle {800}\)}%
\end{pgfscope}%
\begin{pgfscope}%
\pgfsetbuttcap%
\pgfsetroundjoin%
\definecolor{currentfill}{rgb}{0.000000,0.000000,0.000000}%
\pgfsetfillcolor{currentfill}%
\pgfsetlinewidth{0.803000pt}%
\definecolor{currentstroke}{rgb}{0.000000,0.000000,0.000000}%
\pgfsetstrokecolor{currentstroke}%
\pgfsetdash{}{0pt}%
\pgfsys@defobject{currentmarker}{\pgfqpoint{-0.048611in}{0.000000in}}{\pgfqpoint{-0.000000in}{0.000000in}}{%
\pgfpathmoveto{\pgfqpoint{-0.000000in}{0.000000in}}%
\pgfpathlineto{\pgfqpoint{-0.048611in}{0.000000in}}%
\pgfusepath{stroke,fill}%
}%
\begin{pgfscope}%
\pgfsys@transformshift{0.682376in}{3.003257in}%
\pgfsys@useobject{currentmarker}{}%
\end{pgfscope}%
\end{pgfscope}%
\begin{pgfscope}%
\definecolor{textcolor}{rgb}{0.000000,0.000000,0.000000}%
\pgfsetstrokecolor{textcolor}%
\pgfsetfillcolor{textcolor}%
\pgftext[x=0.328211in, y=2.960212in, left, base]{\color{textcolor}\rmfamily\fontsize{9.000000}{10.800000}\selectfont \(\displaystyle {1000}\)}%
\end{pgfscope}%
\begin{pgfscope}%
\definecolor{textcolor}{rgb}{0.000000,0.000000,0.000000}%
\pgfsetstrokecolor{textcolor}%
\pgfsetfillcolor{textcolor}%
\pgftext[x=0.272655in,y=1.892911in,,bottom,rotate=90.000000]{\color{textcolor}\rmfamily\fontsize{10.000000}{12.000000}\selectfont Number of solved benchmarks}%
\end{pgfscope}%
\begin{pgfscope}%
\pgfpathrectangle{\pgfqpoint{0.682376in}{0.535823in}}{\pgfqpoint{5.103389in}{2.714177in}}%
\pgfusepath{clip}%
\pgfsetbuttcap%
\pgfsetroundjoin%
\pgfsetlinewidth{2.007500pt}%
\definecolor{currentstroke}{rgb}{0.000000,0.000000,0.000000}%
\pgfsetstrokecolor{currentstroke}%
\pgfsetdash{{2.000000pt}{3.300000pt}}{0.000000pt}%
\pgfpathmoveto{\pgfqpoint{0.806849in}{0.595041in}}%
\pgfpathlineto{\pgfqpoint{0.931321in}{0.607378in}}%
\pgfpathlineto{\pgfqpoint{1.055794in}{0.632053in}}%
\pgfpathlineto{\pgfqpoint{1.180267in}{0.681401in}}%
\pgfpathlineto{\pgfqpoint{1.304740in}{0.698673in}}%
\pgfpathlineto{\pgfqpoint{1.429213in}{0.757892in}}%
\pgfpathlineto{\pgfqpoint{1.553686in}{0.829447in}}%
\pgfpathlineto{\pgfqpoint{1.678159in}{0.881263in}}%
\pgfpathlineto{\pgfqpoint{1.802632in}{0.972558in}}%
\pgfpathlineto{\pgfqpoint{1.927105in}{1.049049in}}%
\pgfpathlineto{\pgfqpoint{2.051578in}{1.199562in}}%
\pgfpathlineto{\pgfqpoint{2.176050in}{1.342674in}}%
\pgfpathlineto{\pgfqpoint{2.300523in}{1.505524in}}%
\pgfpathlineto{\pgfqpoint{2.424996in}{1.643700in}}%
\pgfpathlineto{\pgfqpoint{2.549469in}{1.813953in}}%
\pgfpathlineto{\pgfqpoint{2.673942in}{1.971869in}}%
\pgfpathlineto{\pgfqpoint{2.798415in}{2.147057in}}%
\pgfpathlineto{\pgfqpoint{2.922888in}{2.235885in}}%
\pgfpathlineto{\pgfqpoint{3.047361in}{2.381463in}}%
\pgfpathlineto{\pgfqpoint{3.171834in}{2.445617in}}%
\pgfpathlineto{\pgfqpoint{3.296306in}{2.494965in}}%
\pgfpathlineto{\pgfqpoint{3.420779in}{2.531977in}}%
\pgfpathlineto{\pgfqpoint{3.545252in}{2.546781in}}%
\pgfpathlineto{\pgfqpoint{3.669725in}{2.564053in}}%
\pgfpathlineto{\pgfqpoint{4.167617in}{2.633142in}}%
\pgfpathlineto{\pgfqpoint{4.292090in}{2.729371in}}%
\pgfpathlineto{\pgfqpoint{4.416562in}{2.833004in}}%
\pgfpathlineto{\pgfqpoint{4.541035in}{2.973647in}}%
\pgfpathlineto{\pgfqpoint{4.665508in}{3.060008in}}%
\pgfpathlineto{\pgfqpoint{4.789981in}{3.114291in}}%
\pgfpathlineto{\pgfqpoint{4.914454in}{3.134031in}}%
\pgfpathlineto{\pgfqpoint{5.038927in}{3.138965in}}%
\pgfpathlineto{\pgfqpoint{5.163400in}{3.163640in}}%
\pgfpathlineto{\pgfqpoint{5.795764in}{3.169338in}}%
\pgfusepath{stroke}%
\end{pgfscope}%
\begin{pgfscope}%
\pgfpathrectangle{\pgfqpoint{0.682376in}{0.535823in}}{\pgfqpoint{5.103389in}{2.714177in}}%
\pgfusepath{clip}%
\pgfsetbuttcap%
\pgfsetroundjoin%
\pgfsetlinewidth{2.007500pt}%
\definecolor{currentstroke}{rgb}{1.000000,0.843137,0.000000}%
\pgfsetstrokecolor{currentstroke}%
\pgfsetdash{{7.400000pt}{3.200000pt}}{0.000000pt}%
\pgfpathmoveto{\pgfqpoint{0.806849in}{0.595041in}}%
\pgfpathlineto{\pgfqpoint{0.931321in}{0.607378in}}%
\pgfpathlineto{\pgfqpoint{1.055794in}{0.632053in}}%
\pgfpathlineto{\pgfqpoint{1.180267in}{0.681401in}}%
\pgfpathlineto{\pgfqpoint{1.304740in}{0.698673in}}%
\pgfpathlineto{\pgfqpoint{1.429213in}{0.757892in}}%
\pgfpathlineto{\pgfqpoint{1.553686in}{0.829447in}}%
\pgfpathlineto{\pgfqpoint{1.678159in}{0.881263in}}%
\pgfpathlineto{\pgfqpoint{1.802632in}{0.972558in}}%
\pgfpathlineto{\pgfqpoint{1.927105in}{1.049049in}}%
\pgfpathlineto{\pgfqpoint{2.051578in}{1.199562in}}%
\pgfpathlineto{\pgfqpoint{2.176050in}{1.342674in}}%
\pgfpathlineto{\pgfqpoint{2.300523in}{1.463578in}}%
\pgfpathlineto{\pgfqpoint{2.424996in}{1.535133in}}%
\pgfpathlineto{\pgfqpoint{2.549469in}{1.638766in}}%
\pgfpathlineto{\pgfqpoint{2.673942in}{1.705386in}}%
\pgfpathlineto{\pgfqpoint{2.798415in}{1.734996in}}%
\pgfpathlineto{\pgfqpoint{2.922888in}{1.752268in}}%
\pgfusepath{stroke}%
\end{pgfscope}%
\begin{pgfscope}%
\pgfpathrectangle{\pgfqpoint{0.682376in}{0.535823in}}{\pgfqpoint{5.103389in}{2.714177in}}%
\pgfusepath{clip}%
\pgfsetbuttcap%
\pgfsetroundjoin%
\pgfsetlinewidth{2.007500pt}%
\definecolor{currentstroke}{rgb}{1.000000,0.694118,0.305882}%
\pgfsetstrokecolor{currentstroke}%
\pgfsetdash{{2.000000pt}{3.300000pt}}{0.000000pt}%
\pgfpathmoveto{\pgfqpoint{0.806849in}{0.595041in}}%
\pgfpathlineto{\pgfqpoint{0.931321in}{0.607378in}}%
\pgfpathlineto{\pgfqpoint{1.055794in}{0.632053in}}%
\pgfpathlineto{\pgfqpoint{1.180267in}{0.681401in}}%
\pgfpathlineto{\pgfqpoint{1.304740in}{0.698673in}}%
\pgfpathlineto{\pgfqpoint{1.429213in}{0.757892in}}%
\pgfpathlineto{\pgfqpoint{1.553686in}{0.829447in}}%
\pgfpathlineto{\pgfqpoint{1.678159in}{0.881263in}}%
\pgfpathlineto{\pgfqpoint{1.802632in}{0.972558in}}%
\pgfpathlineto{\pgfqpoint{1.927105in}{1.049049in}}%
\pgfpathlineto{\pgfqpoint{2.051578in}{1.199562in}}%
\pgfpathlineto{\pgfqpoint{2.176050in}{1.342674in}}%
\pgfpathlineto{\pgfqpoint{2.300523in}{1.505524in}}%
\pgfpathlineto{\pgfqpoint{2.424996in}{1.638766in}}%
\pgfpathlineto{\pgfqpoint{2.549469in}{1.796681in}}%
\pgfpathlineto{\pgfqpoint{2.673942in}{1.905248in}}%
\pgfpathlineto{\pgfqpoint{2.798415in}{1.976804in}}%
\pgfpathlineto{\pgfqpoint{2.922888in}{2.003946in}}%
\pgfpathlineto{\pgfqpoint{3.047361in}{2.031088in}}%
\pgfpathlineto{\pgfqpoint{3.171834in}{2.043425in}}%
\pgfusepath{stroke}%
\end{pgfscope}%
\begin{pgfscope}%
\pgfpathrectangle{\pgfqpoint{0.682376in}{0.535823in}}{\pgfqpoint{5.103389in}{2.714177in}}%
\pgfusepath{clip}%
\pgfsetrectcap%
\pgfsetroundjoin%
\pgfsetlinewidth{2.007500pt}%
\definecolor{currentstroke}{rgb}{0.980392,0.529412,0.458824}%
\pgfsetstrokecolor{currentstroke}%
\pgfsetdash{}{0pt}%
\pgfpathmoveto{\pgfqpoint{0.806849in}{0.595041in}}%
\pgfpathlineto{\pgfqpoint{0.931321in}{0.607378in}}%
\pgfpathlineto{\pgfqpoint{1.055794in}{0.632053in}}%
\pgfpathlineto{\pgfqpoint{1.180267in}{0.681401in}}%
\pgfpathlineto{\pgfqpoint{1.304740in}{0.698673in}}%
\pgfpathlineto{\pgfqpoint{1.429213in}{0.757892in}}%
\pgfpathlineto{\pgfqpoint{1.553686in}{0.829447in}}%
\pgfpathlineto{\pgfqpoint{1.678159in}{0.881263in}}%
\pgfpathlineto{\pgfqpoint{1.802632in}{0.972558in}}%
\pgfpathlineto{\pgfqpoint{1.927105in}{1.049049in}}%
\pgfpathlineto{\pgfqpoint{2.051578in}{1.199562in}}%
\pgfpathlineto{\pgfqpoint{2.176050in}{1.342674in}}%
\pgfpathlineto{\pgfqpoint{2.300523in}{1.505524in}}%
\pgfpathlineto{\pgfqpoint{2.424996in}{1.643700in}}%
\pgfpathlineto{\pgfqpoint{2.549469in}{1.809019in}}%
\pgfpathlineto{\pgfqpoint{2.673942in}{1.961999in}}%
\pgfpathlineto{\pgfqpoint{2.798415in}{2.129785in}}%
\pgfpathlineto{\pgfqpoint{2.922888in}{2.213678in}}%
\pgfpathlineto{\pgfqpoint{3.047361in}{2.243287in}}%
\pgfpathlineto{\pgfqpoint{3.171834in}{2.255624in}}%
\pgfpathlineto{\pgfqpoint{3.296306in}{2.258092in}}%
\pgfusepath{stroke}%
\end{pgfscope}%
\begin{pgfscope}%
\pgfpathrectangle{\pgfqpoint{0.682376in}{0.535823in}}{\pgfqpoint{5.103389in}{2.714177in}}%
\pgfusepath{clip}%
\pgfsetbuttcap%
\pgfsetroundjoin%
\pgfsetlinewidth{2.007500pt}%
\definecolor{currentstroke}{rgb}{0.866667,0.058824,0.058824}%
\pgfsetstrokecolor{currentstroke}%
\pgfsetdash{{7.400000pt}{3.200000pt}}{0.000000pt}%
\pgfpathmoveto{\pgfqpoint{0.806849in}{0.577769in}}%
\pgfpathlineto{\pgfqpoint{0.931321in}{0.585171in}}%
\pgfpathlineto{\pgfqpoint{1.055794in}{0.599976in}}%
\pgfpathlineto{\pgfqpoint{1.180267in}{0.634520in}}%
\pgfpathlineto{\pgfqpoint{1.304740in}{0.649325in}}%
\pgfpathlineto{\pgfqpoint{1.429213in}{0.671532in}}%
\pgfpathlineto{\pgfqpoint{1.553686in}{0.703608in}}%
\pgfpathlineto{\pgfqpoint{1.678159in}{0.728282in}}%
\pgfpathlineto{\pgfqpoint{1.802632in}{0.755424in}}%
\pgfpathlineto{\pgfqpoint{1.927105in}{0.799838in}}%
\pgfpathlineto{\pgfqpoint{2.051578in}{0.918275in}}%
\pgfpathlineto{\pgfqpoint{2.176050in}{1.024375in}}%
\pgfpathlineto{\pgfqpoint{2.300523in}{1.150214in}}%
\pgfpathlineto{\pgfqpoint{2.424996in}{1.253846in}}%
\pgfpathlineto{\pgfqpoint{2.549469in}{1.394490in}}%
\pgfpathlineto{\pgfqpoint{2.673942in}{1.532666in}}%
\pgfpathlineto{\pgfqpoint{2.798415in}{1.636298in}}%
\pgfpathlineto{\pgfqpoint{2.922888in}{1.717724in}}%
\pgfpathlineto{\pgfqpoint{3.047361in}{1.850965in}}%
\pgfpathlineto{\pgfqpoint{3.171834in}{1.910183in}}%
\pgfpathlineto{\pgfqpoint{3.296306in}{1.954597in}}%
\pgfpathlineto{\pgfqpoint{3.420779in}{1.986674in}}%
\pgfpathlineto{\pgfqpoint{3.545252in}{1.991609in}}%
\pgfpathlineto{\pgfqpoint{4.167617in}{2.028620in}}%
\pgfpathlineto{\pgfqpoint{4.292090in}{2.102643in}}%
\pgfpathlineto{\pgfqpoint{4.416562in}{2.184069in}}%
\pgfpathlineto{\pgfqpoint{4.541035in}{2.280298in}}%
\pgfpathlineto{\pgfqpoint{4.665508in}{2.329647in}}%
\pgfpathlineto{\pgfqpoint{4.789981in}{2.371594in}}%
\pgfpathlineto{\pgfqpoint{4.914454in}{2.388866in}}%
\pgfpathlineto{\pgfqpoint{5.038927in}{2.391333in}}%
\pgfpathlineto{\pgfqpoint{5.795764in}{2.392583in}}%
\pgfusepath{stroke}%
\end{pgfscope}%
\begin{pgfscope}%
\pgfpathrectangle{\pgfqpoint{0.682376in}{0.535823in}}{\pgfqpoint{5.103389in}{2.714177in}}%
\pgfusepath{clip}%
\pgfsetbuttcap%
\pgfsetroundjoin%
\pgfsetlinewidth{2.007500pt}%
\definecolor{currentstroke}{rgb}{0.917647,0.372549,0.580392}%
\pgfsetstrokecolor{currentstroke}%
\pgfsetdash{{2.000000pt}{3.300000pt}}{0.000000pt}%
\pgfpathmoveto{\pgfqpoint{0.806849in}{0.595041in}}%
\pgfpathlineto{\pgfqpoint{0.931321in}{0.607378in}}%
\pgfpathlineto{\pgfqpoint{1.055794in}{0.632053in}}%
\pgfpathlineto{\pgfqpoint{1.180267in}{0.681401in}}%
\pgfpathlineto{\pgfqpoint{1.304740in}{0.698673in}}%
\pgfpathlineto{\pgfqpoint{1.429213in}{0.755424in}}%
\pgfpathlineto{\pgfqpoint{1.553686in}{0.817110in}}%
\pgfpathlineto{\pgfqpoint{1.678159in}{0.866459in}}%
\pgfpathlineto{\pgfqpoint{1.802632in}{0.938014in}}%
\pgfpathlineto{\pgfqpoint{1.927105in}{0.997233in}}%
\pgfpathlineto{\pgfqpoint{2.051578in}{1.125539in}}%
\pgfpathlineto{\pgfqpoint{2.176050in}{1.251378in}}%
\pgfpathlineto{\pgfqpoint{2.300523in}{1.392022in}}%
\pgfpathlineto{\pgfqpoint{2.424996in}{1.505524in}}%
\pgfpathlineto{\pgfqpoint{2.549469in}{1.660973in}}%
\pgfpathlineto{\pgfqpoint{2.673942in}{1.801616in}}%
\pgfpathlineto{\pgfqpoint{2.798415in}{1.929923in}}%
\pgfpathlineto{\pgfqpoint{2.922888in}{2.013816in}}%
\pgfpathlineto{\pgfqpoint{3.047361in}{2.154459in}}%
\pgfpathlineto{\pgfqpoint{3.171834in}{2.211210in}}%
\pgfpathlineto{\pgfqpoint{3.296306in}{2.250689in}}%
\pgfpathlineto{\pgfqpoint{3.420779in}{2.282766in}}%
\pgfpathlineto{\pgfqpoint{3.545252in}{2.287701in}}%
\pgfpathlineto{\pgfqpoint{3.669725in}{2.295103in}}%
\pgfpathlineto{\pgfqpoint{4.167617in}{2.334582in}}%
\pgfpathlineto{\pgfqpoint{4.292090in}{2.408605in}}%
\pgfpathlineto{\pgfqpoint{4.416562in}{2.487563in}}%
\pgfpathlineto{\pgfqpoint{4.541035in}{2.581325in}}%
\pgfpathlineto{\pgfqpoint{4.665508in}{2.625739in}}%
\pgfpathlineto{\pgfqpoint{4.789981in}{2.662751in}}%
\pgfpathlineto{\pgfqpoint{4.914454in}{2.680023in}}%
\pgfpathlineto{\pgfqpoint{5.038927in}{2.682490in}}%
\pgfpathlineto{\pgfqpoint{5.795764in}{2.683740in}}%
\pgfusepath{stroke}%
\end{pgfscope}%
\begin{pgfscope}%
\pgfpathrectangle{\pgfqpoint{0.682376in}{0.535823in}}{\pgfqpoint{5.103389in}{2.714177in}}%
\pgfusepath{clip}%
\pgfsetrectcap%
\pgfsetroundjoin%
\pgfsetlinewidth{2.007500pt}%
\definecolor{currentstroke}{rgb}{0.615686,0.007843,0.843137}%
\pgfsetstrokecolor{currentstroke}%
\pgfsetdash{}{0pt}%
\pgfpathmoveto{\pgfqpoint{0.806849in}{0.595041in}}%
\pgfpathlineto{\pgfqpoint{0.931321in}{0.607378in}}%
\pgfpathlineto{\pgfqpoint{1.055794in}{0.632053in}}%
\pgfpathlineto{\pgfqpoint{1.180267in}{0.681401in}}%
\pgfpathlineto{\pgfqpoint{1.304740in}{0.698673in}}%
\pgfpathlineto{\pgfqpoint{1.429213in}{0.755424in}}%
\pgfpathlineto{\pgfqpoint{1.553686in}{0.812175in}}%
\pgfpathlineto{\pgfqpoint{1.678159in}{0.856589in}}%
\pgfpathlineto{\pgfqpoint{1.802632in}{0.928145in}}%
\pgfpathlineto{\pgfqpoint{1.927105in}{0.997233in}}%
\pgfpathlineto{\pgfqpoint{2.051578in}{1.130474in}}%
\pgfpathlineto{\pgfqpoint{2.176050in}{1.258781in}}%
\pgfpathlineto{\pgfqpoint{2.300523in}{1.404359in}}%
\pgfpathlineto{\pgfqpoint{2.424996in}{1.530199in}}%
\pgfpathlineto{\pgfqpoint{2.549469in}{1.693049in}}%
\pgfpathlineto{\pgfqpoint{2.673942in}{1.838628in}}%
\pgfpathlineto{\pgfqpoint{2.798415in}{1.974337in}}%
\pgfpathlineto{\pgfqpoint{2.922888in}{2.055762in}}%
\pgfpathlineto{\pgfqpoint{3.047361in}{2.191471in}}%
\pgfpathlineto{\pgfqpoint{3.171834in}{2.245754in}}%
\pgfpathlineto{\pgfqpoint{3.296306in}{2.287701in}}%
\pgfpathlineto{\pgfqpoint{3.420779in}{2.319777in}}%
\pgfpathlineto{\pgfqpoint{3.545252in}{2.332115in}}%
\pgfpathlineto{\pgfqpoint{3.669725in}{2.349387in}}%
\pgfpathlineto{\pgfqpoint{4.167617in}{2.408605in}}%
\pgfpathlineto{\pgfqpoint{4.292090in}{2.485095in}}%
\pgfpathlineto{\pgfqpoint{4.416562in}{2.564053in}}%
\pgfpathlineto{\pgfqpoint{4.541035in}{2.660283in}}%
\pgfpathlineto{\pgfqpoint{4.665508in}{2.729371in}}%
\pgfpathlineto{\pgfqpoint{4.789981in}{2.768850in}}%
\pgfpathlineto{\pgfqpoint{4.914454in}{2.786122in}}%
\pgfpathlineto{\pgfqpoint{5.038927in}{2.788590in}}%
\pgfpathlineto{\pgfqpoint{5.163400in}{2.805862in}}%
\pgfpathlineto{\pgfqpoint{5.795764in}{2.808141in}}%
\pgfusepath{stroke}%
\end{pgfscope}%
\begin{pgfscope}%
\pgfsetrectcap%
\pgfsetmiterjoin%
\pgfsetlinewidth{0.803000pt}%
\definecolor{currentstroke}{rgb}{0.000000,0.000000,0.000000}%
\pgfsetstrokecolor{currentstroke}%
\pgfsetdash{}{0pt}%
\pgfpathmoveto{\pgfqpoint{0.682376in}{0.535823in}}%
\pgfpathlineto{\pgfqpoint{0.682376in}{3.250000in}}%
\pgfusepath{stroke}%
\end{pgfscope}%
\begin{pgfscope}%
\pgfsetrectcap%
\pgfsetmiterjoin%
\pgfsetlinewidth{0.803000pt}%
\definecolor{currentstroke}{rgb}{0.000000,0.000000,0.000000}%
\pgfsetstrokecolor{currentstroke}%
\pgfsetdash{}{0pt}%
\pgfpathmoveto{\pgfqpoint{5.785764in}{0.535823in}}%
\pgfpathlineto{\pgfqpoint{5.785764in}{3.250000in}}%
\pgfusepath{stroke}%
\end{pgfscope}%
\begin{pgfscope}%
\pgfsetrectcap%
\pgfsetmiterjoin%
\pgfsetlinewidth{0.803000pt}%
\definecolor{currentstroke}{rgb}{0.000000,0.000000,0.000000}%
\pgfsetstrokecolor{currentstroke}%
\pgfsetdash{}{0pt}%
\pgfpathmoveto{\pgfqpoint{0.682376in}{0.535823in}}%
\pgfpathlineto{\pgfqpoint{5.785764in}{0.535823in}}%
\pgfusepath{stroke}%
\end{pgfscope}%
\begin{pgfscope}%
\pgfsetrectcap%
\pgfsetmiterjoin%
\pgfsetlinewidth{0.803000pt}%
\definecolor{currentstroke}{rgb}{0.000000,0.000000,0.000000}%
\pgfsetstrokecolor{currentstroke}%
\pgfsetdash{}{0pt}%
\pgfpathmoveto{\pgfqpoint{0.682376in}{3.250000in}}%
\pgfpathlineto{\pgfqpoint{5.785764in}{3.250000in}}%
\pgfusepath{stroke}%
\end{pgfscope}%
\begin{pgfscope}%
\pgfsetbuttcap%
\pgfsetroundjoin%
\pgfsetlinewidth{2.007500pt}%
\definecolor{currentstroke}{rgb}{0.000000,0.000000,0.000000}%
\pgfsetstrokecolor{currentstroke}%
\pgfsetdash{{2.000000pt}{3.300000pt}}{0.000000pt}%
\pgfpathmoveto{\pgfqpoint{4.582054in}{1.624670in}}%
\pgfpathlineto{\pgfqpoint{4.832054in}{1.624670in}}%
\pgfusepath{stroke}%
\end{pgfscope}%
\begin{pgfscope}%
\definecolor{textcolor}{rgb}{0.000000,0.000000,0.000000}%
\pgfsetstrokecolor{textcolor}%
\pgfsetfillcolor{textcolor}%
\pgftext[x=4.857054in,y=1.580920in,left,base]{\color{textcolor}\rmfamily\fontsize{9.000000}{10.800000}\selectfont All benchmarks}%
\end{pgfscope}%
\begin{pgfscope}%
\pgfsetbuttcap%
\pgfsetroundjoin%
\pgfsetlinewidth{2.007500pt}%
\definecolor{currentstroke}{rgb}{1.000000,0.843137,0.000000}%
\pgfsetstrokecolor{currentstroke}%
\pgfsetdash{{7.400000pt}{3.200000pt}}{0.000000pt}%
\pgfpathmoveto{\pgfqpoint{4.582054in}{1.462870in}}%
\pgfpathlineto{\pgfqpoint{4.832054in}{1.462870in}}%
\pgfusepath{stroke}%
\end{pgfscope}%
\begin{pgfscope}%
\definecolor{textcolor}{rgb}{0.000000,0.000000,0.000000}%
\pgfsetstrokecolor{textcolor}%
\pgfsetfillcolor{textcolor}%
\pgftext[x=4.857054in,y=1.419120in,left,base]{\color{textcolor}\rmfamily\fontsize{9.000000}{10.800000}\selectfont FT+htd}%
\end{pgfscope}%
\begin{pgfscope}%
\pgfsetbuttcap%
\pgfsetroundjoin%
\pgfsetlinewidth{2.007500pt}%
\definecolor{currentstroke}{rgb}{1.000000,0.694118,0.305882}%
\pgfsetstrokecolor{currentstroke}%
\pgfsetdash{{2.000000pt}{3.300000pt}}{0.000000pt}%
\pgfpathmoveto{\pgfqpoint{4.582054in}{1.301071in}}%
\pgfpathlineto{\pgfqpoint{4.832054in}{1.301071in}}%
\pgfusepath{stroke}%
\end{pgfscope}%
\begin{pgfscope}%
\definecolor{textcolor}{rgb}{0.000000,0.000000,0.000000}%
\pgfsetstrokecolor{textcolor}%
\pgfsetfillcolor{textcolor}%
\pgftext[x=4.857054in,y=1.257321in,left,base]{\color{textcolor}\rmfamily\fontsize{9.000000}{10.800000}\selectfont FT+Flow}%
\end{pgfscope}%
\begin{pgfscope}%
\pgfsetrectcap%
\pgfsetroundjoin%
\pgfsetlinewidth{2.007500pt}%
\definecolor{currentstroke}{rgb}{0.980392,0.529412,0.458824}%
\pgfsetstrokecolor{currentstroke}%
\pgfsetdash{}{0pt}%
\pgfpathmoveto{\pgfqpoint{4.582054in}{1.139271in}}%
\pgfpathlineto{\pgfqpoint{4.832054in}{1.139271in}}%
\pgfusepath{stroke}%
\end{pgfscope}%
\begin{pgfscope}%
\definecolor{textcolor}{rgb}{0.000000,0.000000,0.000000}%
\pgfsetstrokecolor{textcolor}%
\pgfsetfillcolor{textcolor}%
\pgftext[x=4.857054in,y=1.095521in,left,base]{\color{textcolor}\rmfamily\fontsize{9.000000}{10.800000}\selectfont FT+Tamaki}%
\end{pgfscope}%
\begin{pgfscope}%
\pgfsetbuttcap%
\pgfsetroundjoin%
\pgfsetlinewidth{2.007500pt}%
\definecolor{currentstroke}{rgb}{0.866667,0.058824,0.058824}%
\pgfsetstrokecolor{currentstroke}%
\pgfsetdash{{7.400000pt}{3.200000pt}}{0.000000pt}%
\pgfpathmoveto{\pgfqpoint{4.582054in}{0.977471in}}%
\pgfpathlineto{\pgfqpoint{4.832054in}{0.977471in}}%
\pgfusepath{stroke}%
\end{pgfscope}%
\begin{pgfscope}%
\definecolor{textcolor}{rgb}{0.000000,0.000000,0.000000}%
\pgfsetstrokecolor{textcolor}%
\pgfsetfillcolor{textcolor}%
\pgftext[x=4.857054in,y=0.933721in,left,base]{\color{textcolor}\rmfamily\fontsize{9.000000}{10.800000}\selectfont cachet}%
\end{pgfscope}%
\begin{pgfscope}%
\pgfsetbuttcap%
\pgfsetroundjoin%
\pgfsetlinewidth{2.007500pt}%
\definecolor{currentstroke}{rgb}{0.917647,0.372549,0.580392}%
\pgfsetstrokecolor{currentstroke}%
\pgfsetdash{{2.000000pt}{3.300000pt}}{0.000000pt}%
\pgfpathmoveto{\pgfqpoint{4.582054in}{0.815672in}}%
\pgfpathlineto{\pgfqpoint{4.832054in}{0.815672in}}%
\pgfusepath{stroke}%
\end{pgfscope}%
\begin{pgfscope}%
\definecolor{textcolor}{rgb}{0.000000,0.000000,0.000000}%
\pgfsetstrokecolor{textcolor}%
\pgfsetfillcolor{textcolor}%
\pgftext[x=4.857054in,y=0.771922in,left,base]{\color{textcolor}\rmfamily\fontsize{9.000000}{10.800000}\selectfont miniC2D}%
\end{pgfscope}%
\begin{pgfscope}%
\pgfsetrectcap%
\pgfsetroundjoin%
\pgfsetlinewidth{2.007500pt}%
\definecolor{currentstroke}{rgb}{0.615686,0.007843,0.843137}%
\pgfsetstrokecolor{currentstroke}%
\pgfsetdash{}{0pt}%
\pgfpathmoveto{\pgfqpoint{4.582054in}{0.653872in}}%
\pgfpathlineto{\pgfqpoint{4.832054in}{0.653872in}}%
\pgfusepath{stroke}%
\end{pgfscope}%
\begin{pgfscope}%
\definecolor{textcolor}{rgb}{0.000000,0.000000,0.000000}%
\pgfsetstrokecolor{textcolor}%
\pgfsetfillcolor{textcolor}%
\pgftext[x=4.857054in,y=0.610122in,left,base]{\color{textcolor}\rmfamily\fontsize{9.000000}{10.800000}\selectfont d4}%
\end{pgfscope}%
\end{pgfpicture}%
\makeatother%
\endgroup%

	\caption{\label{fig:cachet-carving-cactus} A plot of the number of benchmarks solved by various methods organized by carving width. Each $(x,y)$ data point indicates that the corresponding tool was able to solve $y$ benchmarks whose carving width (after \textbf{FT}-preprocessing) was at most $x$. Our approach \textbf{FT+Tamaki} can solve almost all benchmarks with carving width below 27 (unlike existing model counters, which fail on many benchmarks with small carving width) and no benchmarks with carving width above 30.}
\end{figure}

\subsection{Weighted Model Counting: Exact Inference}
\label{sec:tensors:experiments:cachet}
We next compare on a set of weighted model counting benchmarks from Sang, Beame, and Kautz \shortcite{SBK05}. These 1091 benchmarks are formulas whose weighted model count corresponds to exact inference on Bayesian networks. We compare against the weighted model counters \tool{cachet} \cite{SBK05}, \tool{miniC2D} \cite{OD15} and \tool{d4} \cite{LM17}. Since these benchmarks are weighted, we cannot compare against tools that can only perform unweighted model counting (\tool{dynQBF} \cite{CW16}, \tool{dynasp} \cite{FHMW17} and \tool{SharpSAT} \cite{Thurley2006}). We run each tool once on each benchmark with a timeout of 1000 seconds and record the wall-clock time taken.

We first evaluate numerical accuracy, since our approach uses 64-bit double precision floats: on all benchmarks that \tool{miniC2D} also finishes, the weighted model count returned by our approaches agrees within $10^{-3}$.

We next evaluate runtime performance. Results on these benchmarks are summarized in Figure \ref{fig:cachet-cactus}. \textbf{FT+Tamaki} is able to solve the most benchmarks of all tensor-based methods. Our implementations of \textbf{FT} each solve fewer benchmarks than \tool{cachet}, \tool{miniC2D}, and \tool{d4}. Nevertheless, \textbf{FT+*} are together able to solve 231 benchmarks faster than existing counters (\textbf{FT+Tamaki} is fastest on 50, \textbf{FT+Flow} is fastest on 175, and \textbf{FT+htd} is fastest on 6), including 62 benchmarks on which \tool{cachet}, \tool{miniC2D}, and \tool{d4} all time out. This significantly improves the virtual best solver (VBS) when \textbf{FT+*} are included. We conclude that \textbf{FT} is useful as part of a portfolio of weighted model counters.

% A more detailed analysis is available in the appendix.

\begin{figure}
	\centering
	%% Creator: Matplotlib, PGF backend
%%
%% To include the figure in your LaTeX document, write
%%   \input{<filename>.pgf}
%%
%% Make sure the required packages are loaded in your preamble
%%   \usepackage{pgf}
%%
%% and, on pdftex
%%   \usepackage[utf8]{inputenc}\DeclareUnicodeCharacter{2212}{-}
%%
%% or, on luatex and xetex
%%   \usepackage{unicode-math}
%%
%% Figures using additional raster images can only be included by \input if
%% they are in the same directory as the main LaTeX file. For loading figures
%% from other directories you can use the `import` package
%%   \usepackage{import}
%%
%% and then include the figures with
%%   \import{<path to file>}{<filename>.pgf}
%%
%% Matplotlib used the following preamble
%%   \usepackage[utf8x]{inputenc}
%%   \usepackage[T1]{fontenc}
%%
\begingroup%
\makeatletter%
\begin{pgfpicture}%
\pgfpathrectangle{\pgfpointorigin}{\pgfqpoint{6.000000in}{6.100000in}}%
\pgfusepath{use as bounding box, clip}%
\begin{pgfscope}%
\pgfsetbuttcap%
\pgfsetmiterjoin%
\definecolor{currentfill}{rgb}{1.000000,1.000000,1.000000}%
\pgfsetfillcolor{currentfill}%
\pgfsetlinewidth{0.000000pt}%
\definecolor{currentstroke}{rgb}{1.000000,1.000000,1.000000}%
\pgfsetstrokecolor{currentstroke}%
\pgfsetdash{}{0pt}%
\pgfpathmoveto{\pgfqpoint{0.000000in}{0.000000in}}%
\pgfpathlineto{\pgfqpoint{6.000000in}{0.000000in}}%
\pgfpathlineto{\pgfqpoint{6.000000in}{6.100000in}}%
\pgfpathlineto{\pgfqpoint{0.000000in}{6.100000in}}%
\pgfpathclose%
\pgfusepath{fill}%
\end{pgfscope}%
\begin{pgfscope}%
\pgfsetbuttcap%
\pgfsetmiterjoin%
\definecolor{currentfill}{rgb}{1.000000,1.000000,1.000000}%
\pgfsetfillcolor{currentfill}%
\pgfsetlinewidth{0.000000pt}%
\definecolor{currentstroke}{rgb}{0.000000,0.000000,0.000000}%
\pgfsetstrokecolor{currentstroke}%
\pgfsetstrokeopacity{0.000000}%
\pgfsetdash{}{0pt}%
\pgfpathmoveto{\pgfqpoint{0.708220in}{4.502489in}}%
\pgfpathlineto{\pgfqpoint{5.850000in}{4.502489in}}%
\pgfpathlineto{\pgfqpoint{5.850000in}{5.905275in}}%
\pgfpathlineto{\pgfqpoint{0.708220in}{5.905275in}}%
\pgfpathclose%
\pgfusepath{fill}%
\end{pgfscope}%
\begin{pgfscope}%
\pgfsetbuttcap%
\pgfsetroundjoin%
\definecolor{currentfill}{rgb}{0.000000,0.000000,0.000000}%
\pgfsetfillcolor{currentfill}%
\pgfsetlinewidth{0.803000pt}%
\definecolor{currentstroke}{rgb}{0.000000,0.000000,0.000000}%
\pgfsetstrokecolor{currentstroke}%
\pgfsetdash{}{0pt}%
\pgfsys@defobject{currentmarker}{\pgfqpoint{0.000000in}{-0.048611in}}{\pgfqpoint{0.000000in}{0.000000in}}{%
\pgfpathmoveto{\pgfqpoint{0.000000in}{0.000000in}}%
\pgfpathlineto{\pgfqpoint{0.000000in}{-0.048611in}}%
\pgfusepath{stroke,fill}%
}%
\begin{pgfscope}%
\pgfsys@transformshift{0.708220in}{4.502489in}%
\pgfsys@useobject{currentmarker}{}%
\end{pgfscope}%
\end{pgfscope}%
\begin{pgfscope}%
\definecolor{textcolor}{rgb}{0.000000,0.000000,0.000000}%
\pgfsetstrokecolor{textcolor}%
\pgfsetfillcolor{textcolor}%
\pgftext[x=0.708220in,y=4.405267in,,top]{\color{textcolor}\rmfamily\fontsize{9.000000}{10.800000}\selectfont \(\displaystyle {0}\)}%
\end{pgfscope}%
\begin{pgfscope}%
\pgfsetbuttcap%
\pgfsetroundjoin%
\definecolor{currentfill}{rgb}{0.000000,0.000000,0.000000}%
\pgfsetfillcolor{currentfill}%
\pgfsetlinewidth{0.803000pt}%
\definecolor{currentstroke}{rgb}{0.000000,0.000000,0.000000}%
\pgfsetstrokecolor{currentstroke}%
\pgfsetdash{}{0pt}%
\pgfsys@defobject{currentmarker}{\pgfqpoint{0.000000in}{-0.048611in}}{\pgfqpoint{0.000000in}{0.000000in}}{%
\pgfpathmoveto{\pgfqpoint{0.000000in}{0.000000in}}%
\pgfpathlineto{\pgfqpoint{0.000000in}{-0.048611in}}%
\pgfusepath{stroke,fill}%
}%
\begin{pgfscope}%
\pgfsys@transformshift{1.650801in}{4.502489in}%
\pgfsys@useobject{currentmarker}{}%
\end{pgfscope}%
\end{pgfscope}%
\begin{pgfscope}%
\definecolor{textcolor}{rgb}{0.000000,0.000000,0.000000}%
\pgfsetstrokecolor{textcolor}%
\pgfsetfillcolor{textcolor}%
\pgftext[x=1.650801in,y=4.405267in,,top]{\color{textcolor}\rmfamily\fontsize{9.000000}{10.800000}\selectfont \(\displaystyle {200}\)}%
\end{pgfscope}%
\begin{pgfscope}%
\pgfsetbuttcap%
\pgfsetroundjoin%
\definecolor{currentfill}{rgb}{0.000000,0.000000,0.000000}%
\pgfsetfillcolor{currentfill}%
\pgfsetlinewidth{0.803000pt}%
\definecolor{currentstroke}{rgb}{0.000000,0.000000,0.000000}%
\pgfsetstrokecolor{currentstroke}%
\pgfsetdash{}{0pt}%
\pgfsys@defobject{currentmarker}{\pgfqpoint{0.000000in}{-0.048611in}}{\pgfqpoint{0.000000in}{0.000000in}}{%
\pgfpathmoveto{\pgfqpoint{0.000000in}{0.000000in}}%
\pgfpathlineto{\pgfqpoint{0.000000in}{-0.048611in}}%
\pgfusepath{stroke,fill}%
}%
\begin{pgfscope}%
\pgfsys@transformshift{2.593382in}{4.502489in}%
\pgfsys@useobject{currentmarker}{}%
\end{pgfscope}%
\end{pgfscope}%
\begin{pgfscope}%
\definecolor{textcolor}{rgb}{0.000000,0.000000,0.000000}%
\pgfsetstrokecolor{textcolor}%
\pgfsetfillcolor{textcolor}%
\pgftext[x=2.593382in,y=4.405267in,,top]{\color{textcolor}\rmfamily\fontsize{9.000000}{10.800000}\selectfont \(\displaystyle {400}\)}%
\end{pgfscope}%
\begin{pgfscope}%
\pgfsetbuttcap%
\pgfsetroundjoin%
\definecolor{currentfill}{rgb}{0.000000,0.000000,0.000000}%
\pgfsetfillcolor{currentfill}%
\pgfsetlinewidth{0.803000pt}%
\definecolor{currentstroke}{rgb}{0.000000,0.000000,0.000000}%
\pgfsetstrokecolor{currentstroke}%
\pgfsetdash{}{0pt}%
\pgfsys@defobject{currentmarker}{\pgfqpoint{0.000000in}{-0.048611in}}{\pgfqpoint{0.000000in}{0.000000in}}{%
\pgfpathmoveto{\pgfqpoint{0.000000in}{0.000000in}}%
\pgfpathlineto{\pgfqpoint{0.000000in}{-0.048611in}}%
\pgfusepath{stroke,fill}%
}%
\begin{pgfscope}%
\pgfsys@transformshift{3.535963in}{4.502489in}%
\pgfsys@useobject{currentmarker}{}%
\end{pgfscope}%
\end{pgfscope}%
\begin{pgfscope}%
\definecolor{textcolor}{rgb}{0.000000,0.000000,0.000000}%
\pgfsetstrokecolor{textcolor}%
\pgfsetfillcolor{textcolor}%
\pgftext[x=3.535963in,y=4.405267in,,top]{\color{textcolor}\rmfamily\fontsize{9.000000}{10.800000}\selectfont \(\displaystyle {600}\)}%
\end{pgfscope}%
\begin{pgfscope}%
\pgfsetbuttcap%
\pgfsetroundjoin%
\definecolor{currentfill}{rgb}{0.000000,0.000000,0.000000}%
\pgfsetfillcolor{currentfill}%
\pgfsetlinewidth{0.803000pt}%
\definecolor{currentstroke}{rgb}{0.000000,0.000000,0.000000}%
\pgfsetstrokecolor{currentstroke}%
\pgfsetdash{}{0pt}%
\pgfsys@defobject{currentmarker}{\pgfqpoint{0.000000in}{-0.048611in}}{\pgfqpoint{0.000000in}{0.000000in}}{%
\pgfpathmoveto{\pgfqpoint{0.000000in}{0.000000in}}%
\pgfpathlineto{\pgfqpoint{0.000000in}{-0.048611in}}%
\pgfusepath{stroke,fill}%
}%
\begin{pgfscope}%
\pgfsys@transformshift{4.478544in}{4.502489in}%
\pgfsys@useobject{currentmarker}{}%
\end{pgfscope}%
\end{pgfscope}%
\begin{pgfscope}%
\definecolor{textcolor}{rgb}{0.000000,0.000000,0.000000}%
\pgfsetstrokecolor{textcolor}%
\pgfsetfillcolor{textcolor}%
\pgftext[x=4.478544in,y=4.405267in,,top]{\color{textcolor}\rmfamily\fontsize{9.000000}{10.800000}\selectfont \(\displaystyle {800}\)}%
\end{pgfscope}%
\begin{pgfscope}%
\pgfsetbuttcap%
\pgfsetroundjoin%
\definecolor{currentfill}{rgb}{0.000000,0.000000,0.000000}%
\pgfsetfillcolor{currentfill}%
\pgfsetlinewidth{0.803000pt}%
\definecolor{currentstroke}{rgb}{0.000000,0.000000,0.000000}%
\pgfsetstrokecolor{currentstroke}%
\pgfsetdash{}{0pt}%
\pgfsys@defobject{currentmarker}{\pgfqpoint{0.000000in}{-0.048611in}}{\pgfqpoint{0.000000in}{0.000000in}}{%
\pgfpathmoveto{\pgfqpoint{0.000000in}{0.000000in}}%
\pgfpathlineto{\pgfqpoint{0.000000in}{-0.048611in}}%
\pgfusepath{stroke,fill}%
}%
\begin{pgfscope}%
\pgfsys@transformshift{5.421126in}{4.502489in}%
\pgfsys@useobject{currentmarker}{}%
\end{pgfscope}%
\end{pgfscope}%
\begin{pgfscope}%
\definecolor{textcolor}{rgb}{0.000000,0.000000,0.000000}%
\pgfsetstrokecolor{textcolor}%
\pgfsetfillcolor{textcolor}%
\pgftext[x=5.421126in,y=4.405267in,,top]{\color{textcolor}\rmfamily\fontsize{9.000000}{10.800000}\selectfont \(\displaystyle {1000}\)}%
\end{pgfscope}%
\begin{pgfscope}%
\definecolor{textcolor}{rgb}{0.000000,0.000000,0.000000}%
\pgfsetstrokecolor{textcolor}%
\pgfsetfillcolor{textcolor}%
\pgftext[x=3.279110in,y=4.239322in,,top]{\color{textcolor}\rmfamily\fontsize{10.000000}{12.000000}\selectfont Number of benchmarks solved}%
\end{pgfscope}%
\begin{pgfscope}%
\pgfsetbuttcap%
\pgfsetroundjoin%
\definecolor{currentfill}{rgb}{0.000000,0.000000,0.000000}%
\pgfsetfillcolor{currentfill}%
\pgfsetlinewidth{0.803000pt}%
\definecolor{currentstroke}{rgb}{0.000000,0.000000,0.000000}%
\pgfsetstrokecolor{currentstroke}%
\pgfsetdash{}{0pt}%
\pgfsys@defobject{currentmarker}{\pgfqpoint{-0.048611in}{0.000000in}}{\pgfqpoint{-0.000000in}{0.000000in}}{%
\pgfpathmoveto{\pgfqpoint{-0.000000in}{0.000000in}}%
\pgfpathlineto{\pgfqpoint{-0.048611in}{0.000000in}}%
\pgfusepath{stroke,fill}%
}%
\begin{pgfscope}%
\pgfsys@transformshift{0.708220in}{4.502489in}%
\pgfsys@useobject{currentmarker}{}%
\end{pgfscope}%
\end{pgfscope}%
\begin{pgfscope}%
\definecolor{textcolor}{rgb}{0.000000,0.000000,0.000000}%
\pgfsetstrokecolor{textcolor}%
\pgfsetfillcolor{textcolor}%
\pgftext[x=0.344411in, y=4.457765in, left, base]{\color{textcolor}\rmfamily\fontsize{9.000000}{10.800000}\selectfont \(\displaystyle {10^{-1}}\)}%
\end{pgfscope}%
\begin{pgfscope}%
\pgfsetbuttcap%
\pgfsetroundjoin%
\definecolor{currentfill}{rgb}{0.000000,0.000000,0.000000}%
\pgfsetfillcolor{currentfill}%
\pgfsetlinewidth{0.803000pt}%
\definecolor{currentstroke}{rgb}{0.000000,0.000000,0.000000}%
\pgfsetstrokecolor{currentstroke}%
\pgfsetdash{}{0pt}%
\pgfsys@defobject{currentmarker}{\pgfqpoint{-0.048611in}{0.000000in}}{\pgfqpoint{-0.000000in}{0.000000in}}{%
\pgfpathmoveto{\pgfqpoint{-0.000000in}{0.000000in}}%
\pgfpathlineto{\pgfqpoint{-0.048611in}{0.000000in}}%
\pgfusepath{stroke,fill}%
}%
\begin{pgfscope}%
\pgfsys@transformshift{0.708220in}{4.853186in}%
\pgfsys@useobject{currentmarker}{}%
\end{pgfscope}%
\end{pgfscope}%
\begin{pgfscope}%
\definecolor{textcolor}{rgb}{0.000000,0.000000,0.000000}%
\pgfsetstrokecolor{textcolor}%
\pgfsetfillcolor{textcolor}%
\pgftext[x=0.424657in, y=4.808461in, left, base]{\color{textcolor}\rmfamily\fontsize{9.000000}{10.800000}\selectfont \(\displaystyle {10^{0}}\)}%
\end{pgfscope}%
\begin{pgfscope}%
\pgfsetbuttcap%
\pgfsetroundjoin%
\definecolor{currentfill}{rgb}{0.000000,0.000000,0.000000}%
\pgfsetfillcolor{currentfill}%
\pgfsetlinewidth{0.803000pt}%
\definecolor{currentstroke}{rgb}{0.000000,0.000000,0.000000}%
\pgfsetstrokecolor{currentstroke}%
\pgfsetdash{}{0pt}%
\pgfsys@defobject{currentmarker}{\pgfqpoint{-0.048611in}{0.000000in}}{\pgfqpoint{-0.000000in}{0.000000in}}{%
\pgfpathmoveto{\pgfqpoint{-0.000000in}{0.000000in}}%
\pgfpathlineto{\pgfqpoint{-0.048611in}{0.000000in}}%
\pgfusepath{stroke,fill}%
}%
\begin{pgfscope}%
\pgfsys@transformshift{0.708220in}{5.203882in}%
\pgfsys@useobject{currentmarker}{}%
\end{pgfscope}%
\end{pgfscope}%
\begin{pgfscope}%
\definecolor{textcolor}{rgb}{0.000000,0.000000,0.000000}%
\pgfsetstrokecolor{textcolor}%
\pgfsetfillcolor{textcolor}%
\pgftext[x=0.424657in, y=5.159157in, left, base]{\color{textcolor}\rmfamily\fontsize{9.000000}{10.800000}\selectfont \(\displaystyle {10^{1}}\)}%
\end{pgfscope}%
\begin{pgfscope}%
\pgfsetbuttcap%
\pgfsetroundjoin%
\definecolor{currentfill}{rgb}{0.000000,0.000000,0.000000}%
\pgfsetfillcolor{currentfill}%
\pgfsetlinewidth{0.803000pt}%
\definecolor{currentstroke}{rgb}{0.000000,0.000000,0.000000}%
\pgfsetstrokecolor{currentstroke}%
\pgfsetdash{}{0pt}%
\pgfsys@defobject{currentmarker}{\pgfqpoint{-0.048611in}{0.000000in}}{\pgfqpoint{-0.000000in}{0.000000in}}{%
\pgfpathmoveto{\pgfqpoint{-0.000000in}{0.000000in}}%
\pgfpathlineto{\pgfqpoint{-0.048611in}{0.000000in}}%
\pgfusepath{stroke,fill}%
}%
\begin{pgfscope}%
\pgfsys@transformshift{0.708220in}{5.554579in}%
\pgfsys@useobject{currentmarker}{}%
\end{pgfscope}%
\end{pgfscope}%
\begin{pgfscope}%
\definecolor{textcolor}{rgb}{0.000000,0.000000,0.000000}%
\pgfsetstrokecolor{textcolor}%
\pgfsetfillcolor{textcolor}%
\pgftext[x=0.424657in, y=5.509854in, left, base]{\color{textcolor}\rmfamily\fontsize{9.000000}{10.800000}\selectfont \(\displaystyle {10^{2}}\)}%
\end{pgfscope}%
\begin{pgfscope}%
\pgfsetbuttcap%
\pgfsetroundjoin%
\definecolor{currentfill}{rgb}{0.000000,0.000000,0.000000}%
\pgfsetfillcolor{currentfill}%
\pgfsetlinewidth{0.803000pt}%
\definecolor{currentstroke}{rgb}{0.000000,0.000000,0.000000}%
\pgfsetstrokecolor{currentstroke}%
\pgfsetdash{}{0pt}%
\pgfsys@defobject{currentmarker}{\pgfqpoint{-0.048611in}{0.000000in}}{\pgfqpoint{-0.000000in}{0.000000in}}{%
\pgfpathmoveto{\pgfqpoint{-0.000000in}{0.000000in}}%
\pgfpathlineto{\pgfqpoint{-0.048611in}{0.000000in}}%
\pgfusepath{stroke,fill}%
}%
\begin{pgfscope}%
\pgfsys@transformshift{0.708220in}{5.905275in}%
\pgfsys@useobject{currentmarker}{}%
\end{pgfscope}%
\end{pgfscope}%
\begin{pgfscope}%
\definecolor{textcolor}{rgb}{0.000000,0.000000,0.000000}%
\pgfsetstrokecolor{textcolor}%
\pgfsetfillcolor{textcolor}%
\pgftext[x=0.424657in, y=5.860550in, left, base]{\color{textcolor}\rmfamily\fontsize{9.000000}{10.800000}\selectfont \(\displaystyle {10^{3}}\)}%
\end{pgfscope}%
\begin{pgfscope}%
\pgfsetbuttcap%
\pgfsetroundjoin%
\definecolor{currentfill}{rgb}{0.000000,0.000000,0.000000}%
\pgfsetfillcolor{currentfill}%
\pgfsetlinewidth{0.602250pt}%
\definecolor{currentstroke}{rgb}{0.000000,0.000000,0.000000}%
\pgfsetstrokecolor{currentstroke}%
\pgfsetdash{}{0pt}%
\pgfsys@defobject{currentmarker}{\pgfqpoint{-0.027778in}{0.000000in}}{\pgfqpoint{-0.000000in}{0.000000in}}{%
\pgfpathmoveto{\pgfqpoint{-0.000000in}{0.000000in}}%
\pgfpathlineto{\pgfqpoint{-0.027778in}{0.000000in}}%
\pgfusepath{stroke,fill}%
}%
\begin{pgfscope}%
\pgfsys@transformshift{0.708220in}{4.608059in}%
\pgfsys@useobject{currentmarker}{}%
\end{pgfscope}%
\end{pgfscope}%
\begin{pgfscope}%
\pgfsetbuttcap%
\pgfsetroundjoin%
\definecolor{currentfill}{rgb}{0.000000,0.000000,0.000000}%
\pgfsetfillcolor{currentfill}%
\pgfsetlinewidth{0.602250pt}%
\definecolor{currentstroke}{rgb}{0.000000,0.000000,0.000000}%
\pgfsetstrokecolor{currentstroke}%
\pgfsetdash{}{0pt}%
\pgfsys@defobject{currentmarker}{\pgfqpoint{-0.027778in}{0.000000in}}{\pgfqpoint{-0.000000in}{0.000000in}}{%
\pgfpathmoveto{\pgfqpoint{-0.000000in}{0.000000in}}%
\pgfpathlineto{\pgfqpoint{-0.027778in}{0.000000in}}%
\pgfusepath{stroke,fill}%
}%
\begin{pgfscope}%
\pgfsys@transformshift{0.708220in}{4.669814in}%
\pgfsys@useobject{currentmarker}{}%
\end{pgfscope}%
\end{pgfscope}%
\begin{pgfscope}%
\pgfsetbuttcap%
\pgfsetroundjoin%
\definecolor{currentfill}{rgb}{0.000000,0.000000,0.000000}%
\pgfsetfillcolor{currentfill}%
\pgfsetlinewidth{0.602250pt}%
\definecolor{currentstroke}{rgb}{0.000000,0.000000,0.000000}%
\pgfsetstrokecolor{currentstroke}%
\pgfsetdash{}{0pt}%
\pgfsys@defobject{currentmarker}{\pgfqpoint{-0.027778in}{0.000000in}}{\pgfqpoint{-0.000000in}{0.000000in}}{%
\pgfpathmoveto{\pgfqpoint{-0.000000in}{0.000000in}}%
\pgfpathlineto{\pgfqpoint{-0.027778in}{0.000000in}}%
\pgfusepath{stroke,fill}%
}%
\begin{pgfscope}%
\pgfsys@transformshift{0.708220in}{4.713630in}%
\pgfsys@useobject{currentmarker}{}%
\end{pgfscope}%
\end{pgfscope}%
\begin{pgfscope}%
\pgfsetbuttcap%
\pgfsetroundjoin%
\definecolor{currentfill}{rgb}{0.000000,0.000000,0.000000}%
\pgfsetfillcolor{currentfill}%
\pgfsetlinewidth{0.602250pt}%
\definecolor{currentstroke}{rgb}{0.000000,0.000000,0.000000}%
\pgfsetstrokecolor{currentstroke}%
\pgfsetdash{}{0pt}%
\pgfsys@defobject{currentmarker}{\pgfqpoint{-0.027778in}{0.000000in}}{\pgfqpoint{-0.000000in}{0.000000in}}{%
\pgfpathmoveto{\pgfqpoint{-0.000000in}{0.000000in}}%
\pgfpathlineto{\pgfqpoint{-0.027778in}{0.000000in}}%
\pgfusepath{stroke,fill}%
}%
\begin{pgfscope}%
\pgfsys@transformshift{0.708220in}{4.747616in}%
\pgfsys@useobject{currentmarker}{}%
\end{pgfscope}%
\end{pgfscope}%
\begin{pgfscope}%
\pgfsetbuttcap%
\pgfsetroundjoin%
\definecolor{currentfill}{rgb}{0.000000,0.000000,0.000000}%
\pgfsetfillcolor{currentfill}%
\pgfsetlinewidth{0.602250pt}%
\definecolor{currentstroke}{rgb}{0.000000,0.000000,0.000000}%
\pgfsetstrokecolor{currentstroke}%
\pgfsetdash{}{0pt}%
\pgfsys@defobject{currentmarker}{\pgfqpoint{-0.027778in}{0.000000in}}{\pgfqpoint{-0.000000in}{0.000000in}}{%
\pgfpathmoveto{\pgfqpoint{-0.000000in}{0.000000in}}%
\pgfpathlineto{\pgfqpoint{-0.027778in}{0.000000in}}%
\pgfusepath{stroke,fill}%
}%
\begin{pgfscope}%
\pgfsys@transformshift{0.708220in}{4.775384in}%
\pgfsys@useobject{currentmarker}{}%
\end{pgfscope}%
\end{pgfscope}%
\begin{pgfscope}%
\pgfsetbuttcap%
\pgfsetroundjoin%
\definecolor{currentfill}{rgb}{0.000000,0.000000,0.000000}%
\pgfsetfillcolor{currentfill}%
\pgfsetlinewidth{0.602250pt}%
\definecolor{currentstroke}{rgb}{0.000000,0.000000,0.000000}%
\pgfsetstrokecolor{currentstroke}%
\pgfsetdash{}{0pt}%
\pgfsys@defobject{currentmarker}{\pgfqpoint{-0.027778in}{0.000000in}}{\pgfqpoint{-0.000000in}{0.000000in}}{%
\pgfpathmoveto{\pgfqpoint{-0.000000in}{0.000000in}}%
\pgfpathlineto{\pgfqpoint{-0.027778in}{0.000000in}}%
\pgfusepath{stroke,fill}%
}%
\begin{pgfscope}%
\pgfsys@transformshift{0.708220in}{4.798862in}%
\pgfsys@useobject{currentmarker}{}%
\end{pgfscope}%
\end{pgfscope}%
\begin{pgfscope}%
\pgfsetbuttcap%
\pgfsetroundjoin%
\definecolor{currentfill}{rgb}{0.000000,0.000000,0.000000}%
\pgfsetfillcolor{currentfill}%
\pgfsetlinewidth{0.602250pt}%
\definecolor{currentstroke}{rgb}{0.000000,0.000000,0.000000}%
\pgfsetstrokecolor{currentstroke}%
\pgfsetdash{}{0pt}%
\pgfsys@defobject{currentmarker}{\pgfqpoint{-0.027778in}{0.000000in}}{\pgfqpoint{-0.000000in}{0.000000in}}{%
\pgfpathmoveto{\pgfqpoint{-0.000000in}{0.000000in}}%
\pgfpathlineto{\pgfqpoint{-0.027778in}{0.000000in}}%
\pgfusepath{stroke,fill}%
}%
\begin{pgfscope}%
\pgfsys@transformshift{0.708220in}{4.819200in}%
\pgfsys@useobject{currentmarker}{}%
\end{pgfscope}%
\end{pgfscope}%
\begin{pgfscope}%
\pgfsetbuttcap%
\pgfsetroundjoin%
\definecolor{currentfill}{rgb}{0.000000,0.000000,0.000000}%
\pgfsetfillcolor{currentfill}%
\pgfsetlinewidth{0.602250pt}%
\definecolor{currentstroke}{rgb}{0.000000,0.000000,0.000000}%
\pgfsetstrokecolor{currentstroke}%
\pgfsetdash{}{0pt}%
\pgfsys@defobject{currentmarker}{\pgfqpoint{-0.027778in}{0.000000in}}{\pgfqpoint{-0.000000in}{0.000000in}}{%
\pgfpathmoveto{\pgfqpoint{-0.000000in}{0.000000in}}%
\pgfpathlineto{\pgfqpoint{-0.027778in}{0.000000in}}%
\pgfusepath{stroke,fill}%
}%
\begin{pgfscope}%
\pgfsys@transformshift{0.708220in}{4.837139in}%
\pgfsys@useobject{currentmarker}{}%
\end{pgfscope}%
\end{pgfscope}%
\begin{pgfscope}%
\pgfsetbuttcap%
\pgfsetroundjoin%
\definecolor{currentfill}{rgb}{0.000000,0.000000,0.000000}%
\pgfsetfillcolor{currentfill}%
\pgfsetlinewidth{0.602250pt}%
\definecolor{currentstroke}{rgb}{0.000000,0.000000,0.000000}%
\pgfsetstrokecolor{currentstroke}%
\pgfsetdash{}{0pt}%
\pgfsys@defobject{currentmarker}{\pgfqpoint{-0.027778in}{0.000000in}}{\pgfqpoint{-0.000000in}{0.000000in}}{%
\pgfpathmoveto{\pgfqpoint{-0.000000in}{0.000000in}}%
\pgfpathlineto{\pgfqpoint{-0.027778in}{0.000000in}}%
\pgfusepath{stroke,fill}%
}%
\begin{pgfscope}%
\pgfsys@transformshift{0.708220in}{4.958756in}%
\pgfsys@useobject{currentmarker}{}%
\end{pgfscope}%
\end{pgfscope}%
\begin{pgfscope}%
\pgfsetbuttcap%
\pgfsetroundjoin%
\definecolor{currentfill}{rgb}{0.000000,0.000000,0.000000}%
\pgfsetfillcolor{currentfill}%
\pgfsetlinewidth{0.602250pt}%
\definecolor{currentstroke}{rgb}{0.000000,0.000000,0.000000}%
\pgfsetstrokecolor{currentstroke}%
\pgfsetdash{}{0pt}%
\pgfsys@defobject{currentmarker}{\pgfqpoint{-0.027778in}{0.000000in}}{\pgfqpoint{-0.000000in}{0.000000in}}{%
\pgfpathmoveto{\pgfqpoint{-0.000000in}{0.000000in}}%
\pgfpathlineto{\pgfqpoint{-0.027778in}{0.000000in}}%
\pgfusepath{stroke,fill}%
}%
\begin{pgfscope}%
\pgfsys@transformshift{0.708220in}{5.020511in}%
\pgfsys@useobject{currentmarker}{}%
\end{pgfscope}%
\end{pgfscope}%
\begin{pgfscope}%
\pgfsetbuttcap%
\pgfsetroundjoin%
\definecolor{currentfill}{rgb}{0.000000,0.000000,0.000000}%
\pgfsetfillcolor{currentfill}%
\pgfsetlinewidth{0.602250pt}%
\definecolor{currentstroke}{rgb}{0.000000,0.000000,0.000000}%
\pgfsetstrokecolor{currentstroke}%
\pgfsetdash{}{0pt}%
\pgfsys@defobject{currentmarker}{\pgfqpoint{-0.027778in}{0.000000in}}{\pgfqpoint{-0.000000in}{0.000000in}}{%
\pgfpathmoveto{\pgfqpoint{-0.000000in}{0.000000in}}%
\pgfpathlineto{\pgfqpoint{-0.027778in}{0.000000in}}%
\pgfusepath{stroke,fill}%
}%
\begin{pgfscope}%
\pgfsys@transformshift{0.708220in}{5.064326in}%
\pgfsys@useobject{currentmarker}{}%
\end{pgfscope}%
\end{pgfscope}%
\begin{pgfscope}%
\pgfsetbuttcap%
\pgfsetroundjoin%
\definecolor{currentfill}{rgb}{0.000000,0.000000,0.000000}%
\pgfsetfillcolor{currentfill}%
\pgfsetlinewidth{0.602250pt}%
\definecolor{currentstroke}{rgb}{0.000000,0.000000,0.000000}%
\pgfsetstrokecolor{currentstroke}%
\pgfsetdash{}{0pt}%
\pgfsys@defobject{currentmarker}{\pgfqpoint{-0.027778in}{0.000000in}}{\pgfqpoint{-0.000000in}{0.000000in}}{%
\pgfpathmoveto{\pgfqpoint{-0.000000in}{0.000000in}}%
\pgfpathlineto{\pgfqpoint{-0.027778in}{0.000000in}}%
\pgfusepath{stroke,fill}%
}%
\begin{pgfscope}%
\pgfsys@transformshift{0.708220in}{5.098312in}%
\pgfsys@useobject{currentmarker}{}%
\end{pgfscope}%
\end{pgfscope}%
\begin{pgfscope}%
\pgfsetbuttcap%
\pgfsetroundjoin%
\definecolor{currentfill}{rgb}{0.000000,0.000000,0.000000}%
\pgfsetfillcolor{currentfill}%
\pgfsetlinewidth{0.602250pt}%
\definecolor{currentstroke}{rgb}{0.000000,0.000000,0.000000}%
\pgfsetstrokecolor{currentstroke}%
\pgfsetdash{}{0pt}%
\pgfsys@defobject{currentmarker}{\pgfqpoint{-0.027778in}{0.000000in}}{\pgfqpoint{-0.000000in}{0.000000in}}{%
\pgfpathmoveto{\pgfqpoint{-0.000000in}{0.000000in}}%
\pgfpathlineto{\pgfqpoint{-0.027778in}{0.000000in}}%
\pgfusepath{stroke,fill}%
}%
\begin{pgfscope}%
\pgfsys@transformshift{0.708220in}{5.126081in}%
\pgfsys@useobject{currentmarker}{}%
\end{pgfscope}%
\end{pgfscope}%
\begin{pgfscope}%
\pgfsetbuttcap%
\pgfsetroundjoin%
\definecolor{currentfill}{rgb}{0.000000,0.000000,0.000000}%
\pgfsetfillcolor{currentfill}%
\pgfsetlinewidth{0.602250pt}%
\definecolor{currentstroke}{rgb}{0.000000,0.000000,0.000000}%
\pgfsetstrokecolor{currentstroke}%
\pgfsetdash{}{0pt}%
\pgfsys@defobject{currentmarker}{\pgfqpoint{-0.027778in}{0.000000in}}{\pgfqpoint{-0.000000in}{0.000000in}}{%
\pgfpathmoveto{\pgfqpoint{-0.000000in}{0.000000in}}%
\pgfpathlineto{\pgfqpoint{-0.027778in}{0.000000in}}%
\pgfusepath{stroke,fill}%
}%
\begin{pgfscope}%
\pgfsys@transformshift{0.708220in}{5.149559in}%
\pgfsys@useobject{currentmarker}{}%
\end{pgfscope}%
\end{pgfscope}%
\begin{pgfscope}%
\pgfsetbuttcap%
\pgfsetroundjoin%
\definecolor{currentfill}{rgb}{0.000000,0.000000,0.000000}%
\pgfsetfillcolor{currentfill}%
\pgfsetlinewidth{0.602250pt}%
\definecolor{currentstroke}{rgb}{0.000000,0.000000,0.000000}%
\pgfsetstrokecolor{currentstroke}%
\pgfsetdash{}{0pt}%
\pgfsys@defobject{currentmarker}{\pgfqpoint{-0.027778in}{0.000000in}}{\pgfqpoint{-0.000000in}{0.000000in}}{%
\pgfpathmoveto{\pgfqpoint{-0.000000in}{0.000000in}}%
\pgfpathlineto{\pgfqpoint{-0.027778in}{0.000000in}}%
\pgfusepath{stroke,fill}%
}%
\begin{pgfscope}%
\pgfsys@transformshift{0.708220in}{5.169896in}%
\pgfsys@useobject{currentmarker}{}%
\end{pgfscope}%
\end{pgfscope}%
\begin{pgfscope}%
\pgfsetbuttcap%
\pgfsetroundjoin%
\definecolor{currentfill}{rgb}{0.000000,0.000000,0.000000}%
\pgfsetfillcolor{currentfill}%
\pgfsetlinewidth{0.602250pt}%
\definecolor{currentstroke}{rgb}{0.000000,0.000000,0.000000}%
\pgfsetstrokecolor{currentstroke}%
\pgfsetdash{}{0pt}%
\pgfsys@defobject{currentmarker}{\pgfqpoint{-0.027778in}{0.000000in}}{\pgfqpoint{-0.000000in}{0.000000in}}{%
\pgfpathmoveto{\pgfqpoint{-0.000000in}{0.000000in}}%
\pgfpathlineto{\pgfqpoint{-0.027778in}{0.000000in}}%
\pgfusepath{stroke,fill}%
}%
\begin{pgfscope}%
\pgfsys@transformshift{0.708220in}{5.187835in}%
\pgfsys@useobject{currentmarker}{}%
\end{pgfscope}%
\end{pgfscope}%
\begin{pgfscope}%
\pgfsetbuttcap%
\pgfsetroundjoin%
\definecolor{currentfill}{rgb}{0.000000,0.000000,0.000000}%
\pgfsetfillcolor{currentfill}%
\pgfsetlinewidth{0.602250pt}%
\definecolor{currentstroke}{rgb}{0.000000,0.000000,0.000000}%
\pgfsetstrokecolor{currentstroke}%
\pgfsetdash{}{0pt}%
\pgfsys@defobject{currentmarker}{\pgfqpoint{-0.027778in}{0.000000in}}{\pgfqpoint{-0.000000in}{0.000000in}}{%
\pgfpathmoveto{\pgfqpoint{-0.000000in}{0.000000in}}%
\pgfpathlineto{\pgfqpoint{-0.027778in}{0.000000in}}%
\pgfusepath{stroke,fill}%
}%
\begin{pgfscope}%
\pgfsys@transformshift{0.708220in}{5.309452in}%
\pgfsys@useobject{currentmarker}{}%
\end{pgfscope}%
\end{pgfscope}%
\begin{pgfscope}%
\pgfsetbuttcap%
\pgfsetroundjoin%
\definecolor{currentfill}{rgb}{0.000000,0.000000,0.000000}%
\pgfsetfillcolor{currentfill}%
\pgfsetlinewidth{0.602250pt}%
\definecolor{currentstroke}{rgb}{0.000000,0.000000,0.000000}%
\pgfsetstrokecolor{currentstroke}%
\pgfsetdash{}{0pt}%
\pgfsys@defobject{currentmarker}{\pgfqpoint{-0.027778in}{0.000000in}}{\pgfqpoint{-0.000000in}{0.000000in}}{%
\pgfpathmoveto{\pgfqpoint{-0.000000in}{0.000000in}}%
\pgfpathlineto{\pgfqpoint{-0.027778in}{0.000000in}}%
\pgfusepath{stroke,fill}%
}%
\begin{pgfscope}%
\pgfsys@transformshift{0.708220in}{5.371207in}%
\pgfsys@useobject{currentmarker}{}%
\end{pgfscope}%
\end{pgfscope}%
\begin{pgfscope}%
\pgfsetbuttcap%
\pgfsetroundjoin%
\definecolor{currentfill}{rgb}{0.000000,0.000000,0.000000}%
\pgfsetfillcolor{currentfill}%
\pgfsetlinewidth{0.602250pt}%
\definecolor{currentstroke}{rgb}{0.000000,0.000000,0.000000}%
\pgfsetstrokecolor{currentstroke}%
\pgfsetdash{}{0pt}%
\pgfsys@defobject{currentmarker}{\pgfqpoint{-0.027778in}{0.000000in}}{\pgfqpoint{-0.000000in}{0.000000in}}{%
\pgfpathmoveto{\pgfqpoint{-0.000000in}{0.000000in}}%
\pgfpathlineto{\pgfqpoint{-0.027778in}{0.000000in}}%
\pgfusepath{stroke,fill}%
}%
\begin{pgfscope}%
\pgfsys@transformshift{0.708220in}{5.415023in}%
\pgfsys@useobject{currentmarker}{}%
\end{pgfscope}%
\end{pgfscope}%
\begin{pgfscope}%
\pgfsetbuttcap%
\pgfsetroundjoin%
\definecolor{currentfill}{rgb}{0.000000,0.000000,0.000000}%
\pgfsetfillcolor{currentfill}%
\pgfsetlinewidth{0.602250pt}%
\definecolor{currentstroke}{rgb}{0.000000,0.000000,0.000000}%
\pgfsetstrokecolor{currentstroke}%
\pgfsetdash{}{0pt}%
\pgfsys@defobject{currentmarker}{\pgfqpoint{-0.027778in}{0.000000in}}{\pgfqpoint{-0.000000in}{0.000000in}}{%
\pgfpathmoveto{\pgfqpoint{-0.000000in}{0.000000in}}%
\pgfpathlineto{\pgfqpoint{-0.027778in}{0.000000in}}%
\pgfusepath{stroke,fill}%
}%
\begin{pgfscope}%
\pgfsys@transformshift{0.708220in}{5.449009in}%
\pgfsys@useobject{currentmarker}{}%
\end{pgfscope}%
\end{pgfscope}%
\begin{pgfscope}%
\pgfsetbuttcap%
\pgfsetroundjoin%
\definecolor{currentfill}{rgb}{0.000000,0.000000,0.000000}%
\pgfsetfillcolor{currentfill}%
\pgfsetlinewidth{0.602250pt}%
\definecolor{currentstroke}{rgb}{0.000000,0.000000,0.000000}%
\pgfsetstrokecolor{currentstroke}%
\pgfsetdash{}{0pt}%
\pgfsys@defobject{currentmarker}{\pgfqpoint{-0.027778in}{0.000000in}}{\pgfqpoint{-0.000000in}{0.000000in}}{%
\pgfpathmoveto{\pgfqpoint{-0.000000in}{0.000000in}}%
\pgfpathlineto{\pgfqpoint{-0.027778in}{0.000000in}}%
\pgfusepath{stroke,fill}%
}%
\begin{pgfscope}%
\pgfsys@transformshift{0.708220in}{5.476777in}%
\pgfsys@useobject{currentmarker}{}%
\end{pgfscope}%
\end{pgfscope}%
\begin{pgfscope}%
\pgfsetbuttcap%
\pgfsetroundjoin%
\definecolor{currentfill}{rgb}{0.000000,0.000000,0.000000}%
\pgfsetfillcolor{currentfill}%
\pgfsetlinewidth{0.602250pt}%
\definecolor{currentstroke}{rgb}{0.000000,0.000000,0.000000}%
\pgfsetstrokecolor{currentstroke}%
\pgfsetdash{}{0pt}%
\pgfsys@defobject{currentmarker}{\pgfqpoint{-0.027778in}{0.000000in}}{\pgfqpoint{-0.000000in}{0.000000in}}{%
\pgfpathmoveto{\pgfqpoint{-0.000000in}{0.000000in}}%
\pgfpathlineto{\pgfqpoint{-0.027778in}{0.000000in}}%
\pgfusepath{stroke,fill}%
}%
\begin{pgfscope}%
\pgfsys@transformshift{0.708220in}{5.500255in}%
\pgfsys@useobject{currentmarker}{}%
\end{pgfscope}%
\end{pgfscope}%
\begin{pgfscope}%
\pgfsetbuttcap%
\pgfsetroundjoin%
\definecolor{currentfill}{rgb}{0.000000,0.000000,0.000000}%
\pgfsetfillcolor{currentfill}%
\pgfsetlinewidth{0.602250pt}%
\definecolor{currentstroke}{rgb}{0.000000,0.000000,0.000000}%
\pgfsetstrokecolor{currentstroke}%
\pgfsetdash{}{0pt}%
\pgfsys@defobject{currentmarker}{\pgfqpoint{-0.027778in}{0.000000in}}{\pgfqpoint{-0.000000in}{0.000000in}}{%
\pgfpathmoveto{\pgfqpoint{-0.000000in}{0.000000in}}%
\pgfpathlineto{\pgfqpoint{-0.027778in}{0.000000in}}%
\pgfusepath{stroke,fill}%
}%
\begin{pgfscope}%
\pgfsys@transformshift{0.708220in}{5.520593in}%
\pgfsys@useobject{currentmarker}{}%
\end{pgfscope}%
\end{pgfscope}%
\begin{pgfscope}%
\pgfsetbuttcap%
\pgfsetroundjoin%
\definecolor{currentfill}{rgb}{0.000000,0.000000,0.000000}%
\pgfsetfillcolor{currentfill}%
\pgfsetlinewidth{0.602250pt}%
\definecolor{currentstroke}{rgb}{0.000000,0.000000,0.000000}%
\pgfsetstrokecolor{currentstroke}%
\pgfsetdash{}{0pt}%
\pgfsys@defobject{currentmarker}{\pgfqpoint{-0.027778in}{0.000000in}}{\pgfqpoint{-0.000000in}{0.000000in}}{%
\pgfpathmoveto{\pgfqpoint{-0.000000in}{0.000000in}}%
\pgfpathlineto{\pgfqpoint{-0.027778in}{0.000000in}}%
\pgfusepath{stroke,fill}%
}%
\begin{pgfscope}%
\pgfsys@transformshift{0.708220in}{5.538532in}%
\pgfsys@useobject{currentmarker}{}%
\end{pgfscope}%
\end{pgfscope}%
\begin{pgfscope}%
\pgfsetbuttcap%
\pgfsetroundjoin%
\definecolor{currentfill}{rgb}{0.000000,0.000000,0.000000}%
\pgfsetfillcolor{currentfill}%
\pgfsetlinewidth{0.602250pt}%
\definecolor{currentstroke}{rgb}{0.000000,0.000000,0.000000}%
\pgfsetstrokecolor{currentstroke}%
\pgfsetdash{}{0pt}%
\pgfsys@defobject{currentmarker}{\pgfqpoint{-0.027778in}{0.000000in}}{\pgfqpoint{-0.000000in}{0.000000in}}{%
\pgfpathmoveto{\pgfqpoint{-0.000000in}{0.000000in}}%
\pgfpathlineto{\pgfqpoint{-0.027778in}{0.000000in}}%
\pgfusepath{stroke,fill}%
}%
\begin{pgfscope}%
\pgfsys@transformshift{0.708220in}{5.660149in}%
\pgfsys@useobject{currentmarker}{}%
\end{pgfscope}%
\end{pgfscope}%
\begin{pgfscope}%
\pgfsetbuttcap%
\pgfsetroundjoin%
\definecolor{currentfill}{rgb}{0.000000,0.000000,0.000000}%
\pgfsetfillcolor{currentfill}%
\pgfsetlinewidth{0.602250pt}%
\definecolor{currentstroke}{rgb}{0.000000,0.000000,0.000000}%
\pgfsetstrokecolor{currentstroke}%
\pgfsetdash{}{0pt}%
\pgfsys@defobject{currentmarker}{\pgfqpoint{-0.027778in}{0.000000in}}{\pgfqpoint{-0.000000in}{0.000000in}}{%
\pgfpathmoveto{\pgfqpoint{-0.000000in}{0.000000in}}%
\pgfpathlineto{\pgfqpoint{-0.027778in}{0.000000in}}%
\pgfusepath{stroke,fill}%
}%
\begin{pgfscope}%
\pgfsys@transformshift{0.708220in}{5.721903in}%
\pgfsys@useobject{currentmarker}{}%
\end{pgfscope}%
\end{pgfscope}%
\begin{pgfscope}%
\pgfsetbuttcap%
\pgfsetroundjoin%
\definecolor{currentfill}{rgb}{0.000000,0.000000,0.000000}%
\pgfsetfillcolor{currentfill}%
\pgfsetlinewidth{0.602250pt}%
\definecolor{currentstroke}{rgb}{0.000000,0.000000,0.000000}%
\pgfsetstrokecolor{currentstroke}%
\pgfsetdash{}{0pt}%
\pgfsys@defobject{currentmarker}{\pgfqpoint{-0.027778in}{0.000000in}}{\pgfqpoint{-0.000000in}{0.000000in}}{%
\pgfpathmoveto{\pgfqpoint{-0.000000in}{0.000000in}}%
\pgfpathlineto{\pgfqpoint{-0.027778in}{0.000000in}}%
\pgfusepath{stroke,fill}%
}%
\begin{pgfscope}%
\pgfsys@transformshift{0.708220in}{5.765719in}%
\pgfsys@useobject{currentmarker}{}%
\end{pgfscope}%
\end{pgfscope}%
\begin{pgfscope}%
\pgfsetbuttcap%
\pgfsetroundjoin%
\definecolor{currentfill}{rgb}{0.000000,0.000000,0.000000}%
\pgfsetfillcolor{currentfill}%
\pgfsetlinewidth{0.602250pt}%
\definecolor{currentstroke}{rgb}{0.000000,0.000000,0.000000}%
\pgfsetstrokecolor{currentstroke}%
\pgfsetdash{}{0pt}%
\pgfsys@defobject{currentmarker}{\pgfqpoint{-0.027778in}{0.000000in}}{\pgfqpoint{-0.000000in}{0.000000in}}{%
\pgfpathmoveto{\pgfqpoint{-0.000000in}{0.000000in}}%
\pgfpathlineto{\pgfqpoint{-0.027778in}{0.000000in}}%
\pgfusepath{stroke,fill}%
}%
\begin{pgfscope}%
\pgfsys@transformshift{0.708220in}{5.799705in}%
\pgfsys@useobject{currentmarker}{}%
\end{pgfscope}%
\end{pgfscope}%
\begin{pgfscope}%
\pgfsetbuttcap%
\pgfsetroundjoin%
\definecolor{currentfill}{rgb}{0.000000,0.000000,0.000000}%
\pgfsetfillcolor{currentfill}%
\pgfsetlinewidth{0.602250pt}%
\definecolor{currentstroke}{rgb}{0.000000,0.000000,0.000000}%
\pgfsetstrokecolor{currentstroke}%
\pgfsetdash{}{0pt}%
\pgfsys@defobject{currentmarker}{\pgfqpoint{-0.027778in}{0.000000in}}{\pgfqpoint{-0.000000in}{0.000000in}}{%
\pgfpathmoveto{\pgfqpoint{-0.000000in}{0.000000in}}%
\pgfpathlineto{\pgfqpoint{-0.027778in}{0.000000in}}%
\pgfusepath{stroke,fill}%
}%
\begin{pgfscope}%
\pgfsys@transformshift{0.708220in}{5.827474in}%
\pgfsys@useobject{currentmarker}{}%
\end{pgfscope}%
\end{pgfscope}%
\begin{pgfscope}%
\pgfsetbuttcap%
\pgfsetroundjoin%
\definecolor{currentfill}{rgb}{0.000000,0.000000,0.000000}%
\pgfsetfillcolor{currentfill}%
\pgfsetlinewidth{0.602250pt}%
\definecolor{currentstroke}{rgb}{0.000000,0.000000,0.000000}%
\pgfsetstrokecolor{currentstroke}%
\pgfsetdash{}{0pt}%
\pgfsys@defobject{currentmarker}{\pgfqpoint{-0.027778in}{0.000000in}}{\pgfqpoint{-0.000000in}{0.000000in}}{%
\pgfpathmoveto{\pgfqpoint{-0.000000in}{0.000000in}}%
\pgfpathlineto{\pgfqpoint{-0.027778in}{0.000000in}}%
\pgfusepath{stroke,fill}%
}%
\begin{pgfscope}%
\pgfsys@transformshift{0.708220in}{5.850952in}%
\pgfsys@useobject{currentmarker}{}%
\end{pgfscope}%
\end{pgfscope}%
\begin{pgfscope}%
\pgfsetbuttcap%
\pgfsetroundjoin%
\definecolor{currentfill}{rgb}{0.000000,0.000000,0.000000}%
\pgfsetfillcolor{currentfill}%
\pgfsetlinewidth{0.602250pt}%
\definecolor{currentstroke}{rgb}{0.000000,0.000000,0.000000}%
\pgfsetstrokecolor{currentstroke}%
\pgfsetdash{}{0pt}%
\pgfsys@defobject{currentmarker}{\pgfqpoint{-0.027778in}{0.000000in}}{\pgfqpoint{-0.000000in}{0.000000in}}{%
\pgfpathmoveto{\pgfqpoint{-0.000000in}{0.000000in}}%
\pgfpathlineto{\pgfqpoint{-0.027778in}{0.000000in}}%
\pgfusepath{stroke,fill}%
}%
\begin{pgfscope}%
\pgfsys@transformshift{0.708220in}{5.871289in}%
\pgfsys@useobject{currentmarker}{}%
\end{pgfscope}%
\end{pgfscope}%
\begin{pgfscope}%
\pgfsetbuttcap%
\pgfsetroundjoin%
\definecolor{currentfill}{rgb}{0.000000,0.000000,0.000000}%
\pgfsetfillcolor{currentfill}%
\pgfsetlinewidth{0.602250pt}%
\definecolor{currentstroke}{rgb}{0.000000,0.000000,0.000000}%
\pgfsetstrokecolor{currentstroke}%
\pgfsetdash{}{0pt}%
\pgfsys@defobject{currentmarker}{\pgfqpoint{-0.027778in}{0.000000in}}{\pgfqpoint{-0.000000in}{0.000000in}}{%
\pgfpathmoveto{\pgfqpoint{-0.000000in}{0.000000in}}%
\pgfpathlineto{\pgfqpoint{-0.027778in}{0.000000in}}%
\pgfusepath{stroke,fill}%
}%
\begin{pgfscope}%
\pgfsys@transformshift{0.708220in}{5.889228in}%
\pgfsys@useobject{currentmarker}{}%
\end{pgfscope}%
\end{pgfscope}%
\begin{pgfscope}%
\definecolor{textcolor}{rgb}{0.000000,0.000000,0.000000}%
\pgfsetstrokecolor{textcolor}%
\pgfsetfillcolor{textcolor}%
\pgftext[x=0.288855in,y=5.203882in,,bottom,rotate=90.000000]{\color{textcolor}\rmfamily\fontsize{10.000000}{12.000000}\selectfont Longest solving time (s)}%
\end{pgfscope}%
\begin{pgfscope}%
\pgfpathrectangle{\pgfqpoint{0.708220in}{4.502489in}}{\pgfqpoint{5.141780in}{1.402786in}}%
\pgfusepath{clip}%
\pgfsetbuttcap%
\pgfsetroundjoin%
\pgfsetlinewidth{2.007500pt}%
\definecolor{currentstroke}{rgb}{1.000000,0.843137,0.000000}%
\pgfsetstrokecolor{currentstroke}%
\pgfsetdash{{7.400000pt}{3.200000pt}}{0.000000pt}%
\pgfpathmoveto{\pgfqpoint{0.708220in}{4.728093in}}%
\pgfpathlineto{\pgfqpoint{0.840181in}{4.730743in}}%
\pgfpathlineto{\pgfqpoint{1.028697in}{4.733603in}}%
\pgfpathlineto{\pgfqpoint{1.047549in}{4.734714in}}%
\pgfpathlineto{\pgfqpoint{1.052262in}{4.736246in}}%
\pgfpathlineto{\pgfqpoint{1.056975in}{4.741986in}}%
\pgfpathlineto{\pgfqpoint{1.075827in}{4.743023in}}%
\pgfpathlineto{\pgfqpoint{1.080539in}{4.744282in}}%
\pgfpathlineto{\pgfqpoint{1.287907in}{4.746791in}}%
\pgfpathlineto{\pgfqpoint{1.391591in}{4.749779in}}%
\pgfpathlineto{\pgfqpoint{1.396304in}{4.750497in}}%
\pgfpathlineto{\pgfqpoint{1.401017in}{4.752957in}}%
\pgfpathlineto{\pgfqpoint{1.410443in}{4.753751in}}%
\pgfpathlineto{\pgfqpoint{1.415156in}{4.779399in}}%
\pgfpathlineto{\pgfqpoint{1.825179in}{4.787107in}}%
\pgfpathlineto{\pgfqpoint{1.829891in}{4.789497in}}%
\pgfpathlineto{\pgfqpoint{2.008982in}{4.794406in}}%
\pgfpathlineto{\pgfqpoint{2.013695in}{4.798304in}}%
\pgfpathlineto{\pgfqpoint{2.023121in}{4.800954in}}%
\pgfpathlineto{\pgfqpoint{2.027833in}{4.803733in}}%
\pgfpathlineto{\pgfqpoint{2.032546in}{4.803797in}}%
\pgfpathlineto{\pgfqpoint{2.037259in}{4.806875in}}%
\pgfpathlineto{\pgfqpoint{2.056111in}{4.810233in}}%
\pgfpathlineto{\pgfqpoint{2.093814in}{4.812058in}}%
\pgfpathlineto{\pgfqpoint{2.112666in}{4.815764in}}%
\pgfpathlineto{\pgfqpoint{2.131517in}{4.817141in}}%
\pgfpathlineto{\pgfqpoint{2.155082in}{4.818294in}}%
\pgfpathlineto{\pgfqpoint{2.183359in}{4.819316in}}%
\pgfpathlineto{\pgfqpoint{2.197498in}{4.819959in}}%
\pgfpathlineto{\pgfqpoint{2.202211in}{4.823680in}}%
\pgfpathlineto{\pgfqpoint{2.216350in}{4.828305in}}%
\pgfpathlineto{\pgfqpoint{2.230488in}{4.832002in}}%
\pgfpathlineto{\pgfqpoint{2.235201in}{4.841129in}}%
\pgfpathlineto{\pgfqpoint{2.244627in}{4.842570in}}%
\pgfpathlineto{\pgfqpoint{2.287043in}{4.844250in}}%
\pgfpathlineto{\pgfqpoint{2.315321in}{4.845354in}}%
\pgfpathlineto{\pgfqpoint{2.348311in}{4.846367in}}%
\pgfpathlineto{\pgfqpoint{2.362450in}{4.847399in}}%
\pgfpathlineto{\pgfqpoint{2.381301in}{4.850163in}}%
\pgfpathlineto{\pgfqpoint{2.442569in}{4.854016in}}%
\pgfpathlineto{\pgfqpoint{2.480272in}{4.855327in}}%
\pgfpathlineto{\pgfqpoint{2.489698in}{4.856251in}}%
\pgfpathlineto{\pgfqpoint{2.499124in}{4.857776in}}%
\pgfpathlineto{\pgfqpoint{2.503837in}{4.865914in}}%
\pgfpathlineto{\pgfqpoint{2.508550in}{4.866521in}}%
\pgfpathlineto{\pgfqpoint{2.522689in}{4.872495in}}%
\pgfpathlineto{\pgfqpoint{2.536827in}{4.884964in}}%
\pgfpathlineto{\pgfqpoint{2.541540in}{4.887568in}}%
\pgfpathlineto{\pgfqpoint{2.546253in}{4.899982in}}%
\pgfpathlineto{\pgfqpoint{2.555679in}{4.905137in}}%
\pgfpathlineto{\pgfqpoint{2.579243in}{4.909524in}}%
\pgfpathlineto{\pgfqpoint{2.583956in}{4.910189in}}%
\pgfpathlineto{\pgfqpoint{2.598095in}{4.918885in}}%
\pgfpathlineto{\pgfqpoint{2.602808in}{4.924026in}}%
\pgfpathlineto{\pgfqpoint{2.607521in}{4.924348in}}%
\pgfpathlineto{\pgfqpoint{2.612234in}{4.930718in}}%
\pgfpathlineto{\pgfqpoint{2.616947in}{4.934489in}}%
\pgfpathlineto{\pgfqpoint{2.621660in}{4.935886in}}%
\pgfpathlineto{\pgfqpoint{2.626372in}{4.941875in}}%
\pgfpathlineto{\pgfqpoint{2.631085in}{4.944295in}}%
\pgfpathlineto{\pgfqpoint{2.649937in}{4.948507in}}%
\pgfpathlineto{\pgfqpoint{2.659363in}{4.954463in}}%
\pgfpathlineto{\pgfqpoint{2.664076in}{4.966451in}}%
\pgfpathlineto{\pgfqpoint{2.668789in}{4.969696in}}%
\pgfpathlineto{\pgfqpoint{2.673502in}{4.970547in}}%
\pgfpathlineto{\pgfqpoint{2.678214in}{4.977131in}}%
\pgfpathlineto{\pgfqpoint{2.692353in}{4.978196in}}%
\pgfpathlineto{\pgfqpoint{2.697066in}{4.979358in}}%
\pgfpathlineto{\pgfqpoint{2.701779in}{4.982506in}}%
\pgfpathlineto{\pgfqpoint{2.711205in}{4.996799in}}%
\pgfpathlineto{\pgfqpoint{2.734769in}{5.003732in}}%
\pgfpathlineto{\pgfqpoint{2.739482in}{5.010823in}}%
\pgfpathlineto{\pgfqpoint{2.744195in}{5.012004in}}%
\pgfpathlineto{\pgfqpoint{2.748908in}{5.014470in}}%
\pgfpathlineto{\pgfqpoint{2.758334in}{5.016505in}}%
\pgfpathlineto{\pgfqpoint{2.772473in}{5.017695in}}%
\pgfpathlineto{\pgfqpoint{2.777185in}{5.018364in}}%
\pgfpathlineto{\pgfqpoint{2.781898in}{5.030951in}}%
\pgfpathlineto{\pgfqpoint{2.791324in}{5.031892in}}%
\pgfpathlineto{\pgfqpoint{2.800750in}{5.032096in}}%
\pgfpathlineto{\pgfqpoint{2.810176in}{5.033993in}}%
\pgfpathlineto{\pgfqpoint{2.824315in}{5.035463in}}%
\pgfpathlineto{\pgfqpoint{2.833740in}{5.037040in}}%
\pgfpathlineto{\pgfqpoint{2.862018in}{5.047038in}}%
\pgfpathlineto{\pgfqpoint{2.866731in}{5.047976in}}%
\pgfpathlineto{\pgfqpoint{2.871444in}{5.061184in}}%
\pgfpathlineto{\pgfqpoint{2.899721in}{5.063773in}}%
\pgfpathlineto{\pgfqpoint{2.909147in}{5.065832in}}%
\pgfpathlineto{\pgfqpoint{2.913860in}{5.068138in}}%
\pgfpathlineto{\pgfqpoint{2.927998in}{5.070539in}}%
\pgfpathlineto{\pgfqpoint{2.932711in}{5.072630in}}%
\pgfpathlineto{\pgfqpoint{2.956276in}{5.075468in}}%
\pgfpathlineto{\pgfqpoint{2.960989in}{5.077471in}}%
\pgfpathlineto{\pgfqpoint{2.965702in}{5.078108in}}%
\pgfpathlineto{\pgfqpoint{2.970415in}{5.087551in}}%
\pgfpathlineto{\pgfqpoint{2.975128in}{5.089559in}}%
\pgfpathlineto{\pgfqpoint{2.979840in}{5.095771in}}%
\pgfpathlineto{\pgfqpoint{2.989266in}{5.096192in}}%
\pgfpathlineto{\pgfqpoint{2.998692in}{5.097585in}}%
\pgfpathlineto{\pgfqpoint{3.003405in}{5.097884in}}%
\pgfpathlineto{\pgfqpoint{3.008118in}{5.100699in}}%
\pgfpathlineto{\pgfqpoint{3.022257in}{5.103300in}}%
\pgfpathlineto{\pgfqpoint{3.026969in}{5.103714in}}%
\pgfpathlineto{\pgfqpoint{3.031682in}{5.109335in}}%
\pgfpathlineto{\pgfqpoint{3.055247in}{5.111412in}}%
\pgfpathlineto{\pgfqpoint{3.064673in}{5.112052in}}%
\pgfpathlineto{\pgfqpoint{3.078811in}{5.112871in}}%
\pgfpathlineto{\pgfqpoint{3.083524in}{5.113355in}}%
\pgfpathlineto{\pgfqpoint{3.088237in}{5.127018in}}%
\pgfpathlineto{\pgfqpoint{3.102376in}{5.128712in}}%
\pgfpathlineto{\pgfqpoint{3.130653in}{5.130606in}}%
\pgfpathlineto{\pgfqpoint{3.149505in}{5.133991in}}%
\pgfpathlineto{\pgfqpoint{3.154218in}{5.134624in}}%
\pgfpathlineto{\pgfqpoint{3.168357in}{5.139708in}}%
\pgfpathlineto{\pgfqpoint{3.173070in}{5.140310in}}%
\pgfpathlineto{\pgfqpoint{3.182495in}{5.144101in}}%
\pgfpathlineto{\pgfqpoint{3.187208in}{5.153376in}}%
\pgfpathlineto{\pgfqpoint{3.215486in}{5.158673in}}%
\pgfpathlineto{\pgfqpoint{3.220199in}{5.158929in}}%
\pgfpathlineto{\pgfqpoint{3.224912in}{5.161465in}}%
\pgfpathlineto{\pgfqpoint{3.239050in}{5.163633in}}%
\pgfpathlineto{\pgfqpoint{3.243763in}{5.167473in}}%
\pgfpathlineto{\pgfqpoint{3.272041in}{5.170108in}}%
\pgfpathlineto{\pgfqpoint{3.276753in}{5.172136in}}%
\pgfpathlineto{\pgfqpoint{3.295605in}{5.174024in}}%
\pgfpathlineto{\pgfqpoint{3.305031in}{5.176195in}}%
\pgfpathlineto{\pgfqpoint{3.309744in}{5.185076in}}%
\pgfpathlineto{\pgfqpoint{3.323883in}{5.187135in}}%
\pgfpathlineto{\pgfqpoint{3.375724in}{5.192222in}}%
\pgfpathlineto{\pgfqpoint{3.380437in}{5.196896in}}%
\pgfpathlineto{\pgfqpoint{3.399289in}{5.205656in}}%
\pgfpathlineto{\pgfqpoint{3.418141in}{5.207271in}}%
\pgfpathlineto{\pgfqpoint{3.427566in}{5.209342in}}%
\pgfpathlineto{\pgfqpoint{3.432279in}{5.220026in}}%
\pgfpathlineto{\pgfqpoint{3.436992in}{5.221898in}}%
\pgfpathlineto{\pgfqpoint{3.451131in}{5.222209in}}%
\pgfpathlineto{\pgfqpoint{3.465270in}{5.225683in}}%
\pgfpathlineto{\pgfqpoint{3.484121in}{5.229624in}}%
\pgfpathlineto{\pgfqpoint{3.488834in}{5.233686in}}%
\pgfpathlineto{\pgfqpoint{3.512399in}{5.234915in}}%
\pgfpathlineto{\pgfqpoint{3.545389in}{5.236403in}}%
\pgfpathlineto{\pgfqpoint{3.568954in}{5.237799in}}%
\pgfpathlineto{\pgfqpoint{3.578379in}{5.251450in}}%
\pgfpathlineto{\pgfqpoint{3.592518in}{5.252691in}}%
\pgfpathlineto{\pgfqpoint{3.597231in}{5.255469in}}%
\pgfpathlineto{\pgfqpoint{3.700915in}{5.267026in}}%
\pgfpathlineto{\pgfqpoint{3.705628in}{5.281272in}}%
\pgfpathlineto{\pgfqpoint{3.710341in}{5.281523in}}%
\pgfpathlineto{\pgfqpoint{3.724480in}{5.287589in}}%
\pgfpathlineto{\pgfqpoint{3.743331in}{5.289402in}}%
\pgfpathlineto{\pgfqpoint{3.762183in}{5.291591in}}%
\pgfpathlineto{\pgfqpoint{3.790460in}{5.294110in}}%
\pgfpathlineto{\pgfqpoint{3.795173in}{5.294187in}}%
\pgfpathlineto{\pgfqpoint{3.799886in}{5.302941in}}%
\pgfpathlineto{\pgfqpoint{3.818738in}{5.304320in}}%
\pgfpathlineto{\pgfqpoint{3.828163in}{5.316469in}}%
\pgfpathlineto{\pgfqpoint{3.837589in}{5.319308in}}%
\pgfpathlineto{\pgfqpoint{3.936560in}{5.326424in}}%
\pgfpathlineto{\pgfqpoint{3.945986in}{5.331410in}}%
\pgfpathlineto{\pgfqpoint{3.960125in}{5.331856in}}%
\pgfpathlineto{\pgfqpoint{3.964838in}{5.334060in}}%
\pgfpathlineto{\pgfqpoint{3.978976in}{5.335049in}}%
\pgfpathlineto{\pgfqpoint{3.993115in}{5.338146in}}%
\pgfpathlineto{\pgfqpoint{3.997828in}{5.341903in}}%
\pgfpathlineto{\pgfqpoint{4.002541in}{5.348261in}}%
\pgfpathlineto{\pgfqpoint{4.035531in}{5.352773in}}%
\pgfpathlineto{\pgfqpoint{4.040244in}{5.355705in}}%
\pgfpathlineto{\pgfqpoint{4.049670in}{5.365359in}}%
\pgfpathlineto{\pgfqpoint{4.054383in}{5.374613in}}%
\pgfpathlineto{\pgfqpoint{4.063809in}{5.376621in}}%
\pgfpathlineto{\pgfqpoint{4.096799in}{5.383236in}}%
\pgfpathlineto{\pgfqpoint{4.101512in}{5.386951in}}%
\pgfpathlineto{\pgfqpoint{4.110938in}{5.387581in}}%
\pgfpathlineto{\pgfqpoint{4.120364in}{5.391786in}}%
\pgfpathlineto{\pgfqpoint{4.129789in}{5.392886in}}%
\pgfpathlineto{\pgfqpoint{4.134502in}{5.407725in}}%
\pgfpathlineto{\pgfqpoint{4.143928in}{5.905275in}}%
\pgfpathlineto{\pgfqpoint{5.845287in}{5.905275in}}%
\pgfpathlineto{\pgfqpoint{5.845287in}{5.905275in}}%
\pgfusepath{stroke}%
\end{pgfscope}%
\begin{pgfscope}%
\pgfpathrectangle{\pgfqpoint{0.708220in}{4.502489in}}{\pgfqpoint{5.141780in}{1.402786in}}%
\pgfusepath{clip}%
\pgfsetbuttcap%
\pgfsetroundjoin%
\pgfsetlinewidth{2.007500pt}%
\definecolor{currentstroke}{rgb}{1.000000,0.694118,0.305882}%
\pgfsetstrokecolor{currentstroke}%
\pgfsetdash{{2.000000pt}{3.300000pt}}{0.000000pt}%
\pgfpathmoveto{\pgfqpoint{0.708220in}{4.575686in}}%
\pgfpathlineto{\pgfqpoint{0.722359in}{4.577027in}}%
\pgfpathlineto{\pgfqpoint{1.372740in}{4.583679in}}%
\pgfpathlineto{\pgfqpoint{1.386878in}{4.584619in}}%
\pgfpathlineto{\pgfqpoint{1.391591in}{4.584630in}}%
\pgfpathlineto{\pgfqpoint{1.396304in}{4.586758in}}%
\pgfpathlineto{\pgfqpoint{1.401017in}{4.594530in}}%
\pgfpathlineto{\pgfqpoint{1.405730in}{4.596867in}}%
\pgfpathlineto{\pgfqpoint{1.419869in}{4.599099in}}%
\pgfpathlineto{\pgfqpoint{1.443433in}{4.599687in}}%
\pgfpathlineto{\pgfqpoint{1.716782in}{4.602148in}}%
\pgfpathlineto{\pgfqpoint{1.839317in}{4.603322in}}%
\pgfpathlineto{\pgfqpoint{1.886446in}{4.603649in}}%
\pgfpathlineto{\pgfqpoint{1.975991in}{4.604887in}}%
\pgfpathlineto{\pgfqpoint{1.990130in}{4.605564in}}%
\pgfpathlineto{\pgfqpoint{1.999556in}{4.608550in}}%
\pgfpathlineto{\pgfqpoint{2.013695in}{4.618676in}}%
\pgfpathlineto{\pgfqpoint{2.070250in}{4.620492in}}%
\pgfpathlineto{\pgfqpoint{2.183359in}{4.623892in}}%
\pgfpathlineto{\pgfqpoint{2.206924in}{4.626010in}}%
\pgfpathlineto{\pgfqpoint{2.211637in}{4.629424in}}%
\pgfpathlineto{\pgfqpoint{2.216350in}{4.647832in}}%
\pgfpathlineto{\pgfqpoint{2.230488in}{4.650708in}}%
\pgfpathlineto{\pgfqpoint{2.239914in}{4.653906in}}%
\pgfpathlineto{\pgfqpoint{2.258766in}{4.658518in}}%
\pgfpathlineto{\pgfqpoint{2.263479in}{4.671626in}}%
\pgfpathlineto{\pgfqpoint{2.268192in}{4.674134in}}%
\pgfpathlineto{\pgfqpoint{2.277617in}{4.674964in}}%
\pgfpathlineto{\pgfqpoint{2.291756in}{4.680857in}}%
\pgfpathlineto{\pgfqpoint{2.310608in}{4.681963in}}%
\pgfpathlineto{\pgfqpoint{2.315321in}{4.683356in}}%
\pgfpathlineto{\pgfqpoint{2.348311in}{4.685825in}}%
\pgfpathlineto{\pgfqpoint{2.357737in}{4.688192in}}%
\pgfpathlineto{\pgfqpoint{2.362450in}{4.696546in}}%
\pgfpathlineto{\pgfqpoint{2.371876in}{4.701279in}}%
\pgfpathlineto{\pgfqpoint{2.381301in}{4.702312in}}%
\pgfpathlineto{\pgfqpoint{2.489698in}{4.705440in}}%
\pgfpathlineto{\pgfqpoint{2.565105in}{4.707508in}}%
\pgfpathlineto{\pgfqpoint{2.593382in}{4.708585in}}%
\pgfpathlineto{\pgfqpoint{2.635798in}{4.710876in}}%
\pgfpathlineto{\pgfqpoint{2.659363in}{4.715253in}}%
\pgfpathlineto{\pgfqpoint{2.664076in}{4.716954in}}%
\pgfpathlineto{\pgfqpoint{2.668789in}{4.717077in}}%
\pgfpathlineto{\pgfqpoint{2.682927in}{4.720175in}}%
\pgfpathlineto{\pgfqpoint{2.687640in}{4.726657in}}%
\pgfpathlineto{\pgfqpoint{2.697066in}{4.727461in}}%
\pgfpathlineto{\pgfqpoint{2.706492in}{4.729110in}}%
\pgfpathlineto{\pgfqpoint{2.711205in}{4.729173in}}%
\pgfpathlineto{\pgfqpoint{2.715918in}{4.731567in}}%
\pgfpathlineto{\pgfqpoint{2.725344in}{4.733241in}}%
\pgfpathlineto{\pgfqpoint{2.730056in}{4.738342in}}%
\pgfpathlineto{\pgfqpoint{2.739482in}{4.739897in}}%
\pgfpathlineto{\pgfqpoint{2.744195in}{4.758295in}}%
\pgfpathlineto{\pgfqpoint{2.753621in}{4.761653in}}%
\pgfpathlineto{\pgfqpoint{2.772473in}{4.762463in}}%
\pgfpathlineto{\pgfqpoint{2.777185in}{4.764980in}}%
\pgfpathlineto{\pgfqpoint{2.781898in}{4.765126in}}%
\pgfpathlineto{\pgfqpoint{2.786611in}{4.767501in}}%
\pgfpathlineto{\pgfqpoint{2.796037in}{4.767627in}}%
\pgfpathlineto{\pgfqpoint{2.810176in}{4.772374in}}%
\pgfpathlineto{\pgfqpoint{2.819602in}{4.773276in}}%
\pgfpathlineto{\pgfqpoint{2.843166in}{4.775744in}}%
\pgfpathlineto{\pgfqpoint{2.847879in}{4.777443in}}%
\pgfpathlineto{\pgfqpoint{2.857305in}{4.778405in}}%
\pgfpathlineto{\pgfqpoint{2.862018in}{4.780090in}}%
\pgfpathlineto{\pgfqpoint{2.904434in}{4.783203in}}%
\pgfpathlineto{\pgfqpoint{2.909147in}{4.786076in}}%
\pgfpathlineto{\pgfqpoint{2.923286in}{4.787254in}}%
\pgfpathlineto{\pgfqpoint{2.927998in}{4.788671in}}%
\pgfpathlineto{\pgfqpoint{2.932711in}{4.791749in}}%
\pgfpathlineto{\pgfqpoint{2.942137in}{4.793884in}}%
\pgfpathlineto{\pgfqpoint{2.951563in}{4.794105in}}%
\pgfpathlineto{\pgfqpoint{2.960989in}{4.796525in}}%
\pgfpathlineto{\pgfqpoint{2.975128in}{4.797634in}}%
\pgfpathlineto{\pgfqpoint{2.979840in}{4.799422in}}%
\pgfpathlineto{\pgfqpoint{2.993979in}{4.801065in}}%
\pgfpathlineto{\pgfqpoint{2.998692in}{4.803094in}}%
\pgfpathlineto{\pgfqpoint{3.003405in}{4.803538in}}%
\pgfpathlineto{\pgfqpoint{3.008118in}{4.805437in}}%
\pgfpathlineto{\pgfqpoint{3.036395in}{4.809773in}}%
\pgfpathlineto{\pgfqpoint{3.045821in}{4.812528in}}%
\pgfpathlineto{\pgfqpoint{3.055247in}{4.814006in}}%
\pgfpathlineto{\pgfqpoint{3.064673in}{4.814702in}}%
\pgfpathlineto{\pgfqpoint{3.097663in}{4.820512in}}%
\pgfpathlineto{\pgfqpoint{3.130653in}{4.822504in}}%
\pgfpathlineto{\pgfqpoint{3.140079in}{4.824887in}}%
\pgfpathlineto{\pgfqpoint{3.163644in}{4.826436in}}%
\pgfpathlineto{\pgfqpoint{3.177782in}{4.827325in}}%
\pgfpathlineto{\pgfqpoint{3.182495in}{4.829725in}}%
\pgfpathlineto{\pgfqpoint{3.187208in}{4.839898in}}%
\pgfpathlineto{\pgfqpoint{3.201347in}{4.840718in}}%
\pgfpathlineto{\pgfqpoint{3.210773in}{4.842516in}}%
\pgfpathlineto{\pgfqpoint{3.215486in}{4.842704in}}%
\pgfpathlineto{\pgfqpoint{3.220199in}{4.844139in}}%
\pgfpathlineto{\pgfqpoint{3.229624in}{4.845618in}}%
\pgfpathlineto{\pgfqpoint{3.239050in}{4.847691in}}%
\pgfpathlineto{\pgfqpoint{3.243763in}{4.849798in}}%
\pgfpathlineto{\pgfqpoint{3.267328in}{4.851319in}}%
\pgfpathlineto{\pgfqpoint{3.276753in}{4.852407in}}%
\pgfpathlineto{\pgfqpoint{3.281466in}{4.854128in}}%
\pgfpathlineto{\pgfqpoint{3.286179in}{4.854661in}}%
\pgfpathlineto{\pgfqpoint{3.290892in}{4.856700in}}%
\pgfpathlineto{\pgfqpoint{3.323883in}{4.861520in}}%
\pgfpathlineto{\pgfqpoint{3.333308in}{4.863837in}}%
\pgfpathlineto{\pgfqpoint{3.352160in}{4.864811in}}%
\pgfpathlineto{\pgfqpoint{3.380437in}{4.866933in}}%
\pgfpathlineto{\pgfqpoint{3.436992in}{4.871169in}}%
\pgfpathlineto{\pgfqpoint{3.441705in}{4.872127in}}%
\pgfpathlineto{\pgfqpoint{3.446418in}{4.875000in}}%
\pgfpathlineto{\pgfqpoint{3.469983in}{4.880187in}}%
\pgfpathlineto{\pgfqpoint{3.498260in}{4.882437in}}%
\pgfpathlineto{\pgfqpoint{3.512399in}{4.883546in}}%
\pgfpathlineto{\pgfqpoint{3.526537in}{4.885272in}}%
\pgfpathlineto{\pgfqpoint{3.531250in}{4.897268in}}%
\pgfpathlineto{\pgfqpoint{3.568954in}{4.903702in}}%
\pgfpathlineto{\pgfqpoint{3.583092in}{4.904901in}}%
\pgfpathlineto{\pgfqpoint{3.611370in}{4.906567in}}%
\pgfpathlineto{\pgfqpoint{3.634934in}{4.907689in}}%
\pgfpathlineto{\pgfqpoint{3.649073in}{4.909217in}}%
\pgfpathlineto{\pgfqpoint{3.653786in}{4.909617in}}%
\pgfpathlineto{\pgfqpoint{3.658499in}{4.913111in}}%
\pgfpathlineto{\pgfqpoint{3.667925in}{4.914023in}}%
\pgfpathlineto{\pgfqpoint{3.677350in}{4.915221in}}%
\pgfpathlineto{\pgfqpoint{3.700915in}{4.917629in}}%
\pgfpathlineto{\pgfqpoint{3.719767in}{4.920669in}}%
\pgfpathlineto{\pgfqpoint{3.724480in}{4.926393in}}%
\pgfpathlineto{\pgfqpoint{3.729192in}{4.927383in}}%
\pgfpathlineto{\pgfqpoint{3.733905in}{4.930561in}}%
\pgfpathlineto{\pgfqpoint{3.738618in}{4.940078in}}%
\pgfpathlineto{\pgfqpoint{3.762183in}{4.941663in}}%
\pgfpathlineto{\pgfqpoint{3.781034in}{4.943376in}}%
\pgfpathlineto{\pgfqpoint{3.790460in}{4.944487in}}%
\pgfpathlineto{\pgfqpoint{3.799886in}{4.945672in}}%
\pgfpathlineto{\pgfqpoint{3.823451in}{4.947287in}}%
\pgfpathlineto{\pgfqpoint{3.828163in}{4.949276in}}%
\pgfpathlineto{\pgfqpoint{3.832876in}{4.955403in}}%
\pgfpathlineto{\pgfqpoint{3.837589in}{4.959151in}}%
\pgfpathlineto{\pgfqpoint{3.842302in}{4.960205in}}%
\pgfpathlineto{\pgfqpoint{3.847015in}{4.974420in}}%
\pgfpathlineto{\pgfqpoint{3.851728in}{4.982712in}}%
\pgfpathlineto{\pgfqpoint{3.856441in}{4.984178in}}%
\pgfpathlineto{\pgfqpoint{3.861154in}{4.991753in}}%
\pgfpathlineto{\pgfqpoint{3.865867in}{4.993084in}}%
\pgfpathlineto{\pgfqpoint{3.870580in}{5.005690in}}%
\pgfpathlineto{\pgfqpoint{3.880005in}{5.008592in}}%
\pgfpathlineto{\pgfqpoint{3.884718in}{5.021462in}}%
\pgfpathlineto{\pgfqpoint{3.889431in}{5.026788in}}%
\pgfpathlineto{\pgfqpoint{3.912996in}{5.029357in}}%
\pgfpathlineto{\pgfqpoint{3.922422in}{5.032609in}}%
\pgfpathlineto{\pgfqpoint{3.927134in}{5.036632in}}%
\pgfpathlineto{\pgfqpoint{3.941273in}{5.037816in}}%
\pgfpathlineto{\pgfqpoint{3.945986in}{5.039755in}}%
\pgfpathlineto{\pgfqpoint{3.955412in}{5.040048in}}%
\pgfpathlineto{\pgfqpoint{3.960125in}{5.046514in}}%
\pgfpathlineto{\pgfqpoint{3.964838in}{5.048731in}}%
\pgfpathlineto{\pgfqpoint{3.978976in}{5.048916in}}%
\pgfpathlineto{\pgfqpoint{3.988402in}{5.050454in}}%
\pgfpathlineto{\pgfqpoint{3.997828in}{5.058586in}}%
\pgfpathlineto{\pgfqpoint{4.002541in}{5.059113in}}%
\pgfpathlineto{\pgfqpoint{4.007254in}{5.061944in}}%
\pgfpathlineto{\pgfqpoint{4.030818in}{5.066156in}}%
\pgfpathlineto{\pgfqpoint{4.044957in}{5.072083in}}%
\pgfpathlineto{\pgfqpoint{4.049670in}{5.076490in}}%
\pgfpathlineto{\pgfqpoint{4.054383in}{5.076807in}}%
\pgfpathlineto{\pgfqpoint{4.059096in}{5.078407in}}%
\pgfpathlineto{\pgfqpoint{4.073235in}{5.089313in}}%
\pgfpathlineto{\pgfqpoint{4.077947in}{5.091116in}}%
\pgfpathlineto{\pgfqpoint{4.082660in}{5.144930in}}%
\pgfpathlineto{\pgfqpoint{4.087373in}{5.154558in}}%
\pgfpathlineto{\pgfqpoint{4.092086in}{5.157608in}}%
\pgfpathlineto{\pgfqpoint{4.096799in}{5.165826in}}%
\pgfpathlineto{\pgfqpoint{4.106225in}{5.203822in}}%
\pgfpathlineto{\pgfqpoint{4.110938in}{5.203848in}}%
\pgfpathlineto{\pgfqpoint{4.115651in}{5.209878in}}%
\pgfpathlineto{\pgfqpoint{4.120364in}{5.228144in}}%
\pgfpathlineto{\pgfqpoint{4.125077in}{5.230046in}}%
\pgfpathlineto{\pgfqpoint{4.129789in}{5.244814in}}%
\pgfpathlineto{\pgfqpoint{4.134502in}{5.306157in}}%
\pgfpathlineto{\pgfqpoint{4.139215in}{5.451029in}}%
\pgfpathlineto{\pgfqpoint{4.143928in}{5.467669in}}%
\pgfpathlineto{\pgfqpoint{4.148641in}{5.514647in}}%
\pgfpathlineto{\pgfqpoint{4.153354in}{5.770257in}}%
\pgfpathlineto{\pgfqpoint{4.158067in}{5.779182in}}%
\pgfpathlineto{\pgfqpoint{4.167493in}{5.905275in}}%
\pgfpathlineto{\pgfqpoint{5.845287in}{5.905275in}}%
\pgfpathlineto{\pgfqpoint{5.845287in}{5.905275in}}%
\pgfusepath{stroke}%
\end{pgfscope}%
\begin{pgfscope}%
\pgfpathrectangle{\pgfqpoint{0.708220in}{4.502489in}}{\pgfqpoint{5.141780in}{1.402786in}}%
\pgfusepath{clip}%
\pgfsetrectcap%
\pgfsetroundjoin%
\pgfsetlinewidth{2.007500pt}%
\definecolor{currentstroke}{rgb}{0.980392,0.529412,0.458824}%
\pgfsetstrokecolor{currentstroke}%
\pgfsetdash{}{0pt}%
\pgfpathmoveto{\pgfqpoint{0.708220in}{4.782864in}}%
\pgfpathlineto{\pgfqpoint{0.717646in}{4.784557in}}%
\pgfpathlineto{\pgfqpoint{0.727071in}{4.785703in}}%
\pgfpathlineto{\pgfqpoint{0.821330in}{4.788240in}}%
\pgfpathlineto{\pgfqpoint{0.849607in}{4.789212in}}%
\pgfpathlineto{\pgfqpoint{0.906162in}{4.790401in}}%
\pgfpathlineto{\pgfqpoint{0.939152in}{4.791266in}}%
\pgfpathlineto{\pgfqpoint{0.990994in}{4.792296in}}%
\pgfpathlineto{\pgfqpoint{1.000420in}{4.793180in}}%
\pgfpathlineto{\pgfqpoint{1.221927in}{4.797417in}}%
\pgfpathlineto{\pgfqpoint{1.302046in}{4.799714in}}%
\pgfpathlineto{\pgfqpoint{1.320898in}{4.800566in}}%
\pgfpathlineto{\pgfqpoint{1.358601in}{4.801713in}}%
\pgfpathlineto{\pgfqpoint{1.405730in}{4.805834in}}%
\pgfpathlineto{\pgfqpoint{1.410443in}{4.808929in}}%
\pgfpathlineto{\pgfqpoint{1.495275in}{4.813067in}}%
\pgfpathlineto{\pgfqpoint{1.740346in}{4.816720in}}%
\pgfpathlineto{\pgfqpoint{1.811040in}{4.817949in}}%
\pgfpathlineto{\pgfqpoint{1.858169in}{4.818976in}}%
\pgfpathlineto{\pgfqpoint{1.924150in}{4.820525in}}%
\pgfpathlineto{\pgfqpoint{1.975991in}{4.822050in}}%
\pgfpathlineto{\pgfqpoint{2.018408in}{4.823531in}}%
\pgfpathlineto{\pgfqpoint{2.023121in}{4.823635in}}%
\pgfpathlineto{\pgfqpoint{2.032546in}{4.825882in}}%
\pgfpathlineto{\pgfqpoint{2.065537in}{4.827575in}}%
\pgfpathlineto{\pgfqpoint{2.070250in}{4.838260in}}%
\pgfpathlineto{\pgfqpoint{2.098527in}{4.840163in}}%
\pgfpathlineto{\pgfqpoint{2.206924in}{4.841781in}}%
\pgfpathlineto{\pgfqpoint{2.239914in}{4.842815in}}%
\pgfpathlineto{\pgfqpoint{2.348311in}{4.844648in}}%
\pgfpathlineto{\pgfqpoint{2.357737in}{4.846244in}}%
\pgfpathlineto{\pgfqpoint{2.390727in}{4.849211in}}%
\pgfpathlineto{\pgfqpoint{2.395440in}{4.868461in}}%
\pgfpathlineto{\pgfqpoint{2.400153in}{4.878142in}}%
\pgfpathlineto{\pgfqpoint{2.409579in}{4.880398in}}%
\pgfpathlineto{\pgfqpoint{2.414292in}{4.883980in}}%
\pgfpathlineto{\pgfqpoint{2.423718in}{4.885284in}}%
\pgfpathlineto{\pgfqpoint{2.428430in}{4.885538in}}%
\pgfpathlineto{\pgfqpoint{2.433143in}{4.891260in}}%
\pgfpathlineto{\pgfqpoint{2.437856in}{4.891946in}}%
\pgfpathlineto{\pgfqpoint{2.442569in}{4.902909in}}%
\pgfpathlineto{\pgfqpoint{2.461421in}{4.907644in}}%
\pgfpathlineto{\pgfqpoint{2.466134in}{4.915293in}}%
\pgfpathlineto{\pgfqpoint{2.470847in}{4.919742in}}%
\pgfpathlineto{\pgfqpoint{2.489698in}{4.925709in}}%
\pgfpathlineto{\pgfqpoint{2.499124in}{4.934684in}}%
\pgfpathlineto{\pgfqpoint{2.508550in}{4.937375in}}%
\pgfpathlineto{\pgfqpoint{2.522689in}{4.952833in}}%
\pgfpathlineto{\pgfqpoint{2.527401in}{4.953445in}}%
\pgfpathlineto{\pgfqpoint{2.532114in}{4.959200in}}%
\pgfpathlineto{\pgfqpoint{2.536827in}{4.959723in}}%
\pgfpathlineto{\pgfqpoint{2.541540in}{4.978508in}}%
\pgfpathlineto{\pgfqpoint{2.546253in}{4.984614in}}%
\pgfpathlineto{\pgfqpoint{2.560392in}{4.985964in}}%
\pgfpathlineto{\pgfqpoint{2.565105in}{4.988989in}}%
\pgfpathlineto{\pgfqpoint{2.569818in}{5.001477in}}%
\pgfpathlineto{\pgfqpoint{2.574531in}{5.006797in}}%
\pgfpathlineto{\pgfqpoint{2.579243in}{5.007296in}}%
\pgfpathlineto{\pgfqpoint{2.583956in}{5.011990in}}%
\pgfpathlineto{\pgfqpoint{2.593382in}{5.014442in}}%
\pgfpathlineto{\pgfqpoint{2.602808in}{5.018340in}}%
\pgfpathlineto{\pgfqpoint{2.607521in}{5.018908in}}%
\pgfpathlineto{\pgfqpoint{2.616947in}{5.022120in}}%
\pgfpathlineto{\pgfqpoint{2.626372in}{5.022806in}}%
\pgfpathlineto{\pgfqpoint{2.640511in}{5.027057in}}%
\pgfpathlineto{\pgfqpoint{2.645224in}{5.027171in}}%
\pgfpathlineto{\pgfqpoint{2.654650in}{5.033352in}}%
\pgfpathlineto{\pgfqpoint{2.659363in}{5.034056in}}%
\pgfpathlineto{\pgfqpoint{2.664076in}{5.041378in}}%
\pgfpathlineto{\pgfqpoint{2.673502in}{5.042804in}}%
\pgfpathlineto{\pgfqpoint{2.687640in}{5.044519in}}%
\pgfpathlineto{\pgfqpoint{2.697066in}{5.046568in}}%
\pgfpathlineto{\pgfqpoint{2.701779in}{5.052607in}}%
\pgfpathlineto{\pgfqpoint{2.715918in}{5.055986in}}%
\pgfpathlineto{\pgfqpoint{2.725344in}{5.058585in}}%
\pgfpathlineto{\pgfqpoint{2.753621in}{5.060806in}}%
\pgfpathlineto{\pgfqpoint{2.772473in}{5.064228in}}%
\pgfpathlineto{\pgfqpoint{2.781898in}{5.065799in}}%
\pgfpathlineto{\pgfqpoint{2.786611in}{5.069207in}}%
\pgfpathlineto{\pgfqpoint{2.800750in}{5.071568in}}%
\pgfpathlineto{\pgfqpoint{2.805463in}{5.080531in}}%
\pgfpathlineto{\pgfqpoint{2.810176in}{5.081175in}}%
\pgfpathlineto{\pgfqpoint{2.814889in}{5.087024in}}%
\pgfpathlineto{\pgfqpoint{2.829027in}{5.089740in}}%
\pgfpathlineto{\pgfqpoint{2.833740in}{5.094182in}}%
\pgfpathlineto{\pgfqpoint{2.847879in}{5.101269in}}%
\pgfpathlineto{\pgfqpoint{2.852592in}{5.108880in}}%
\pgfpathlineto{\pgfqpoint{2.857305in}{5.112251in}}%
\pgfpathlineto{\pgfqpoint{2.866731in}{5.114486in}}%
\pgfpathlineto{\pgfqpoint{2.871444in}{5.118678in}}%
\pgfpathlineto{\pgfqpoint{2.880869in}{5.121465in}}%
\pgfpathlineto{\pgfqpoint{2.885582in}{5.127137in}}%
\pgfpathlineto{\pgfqpoint{2.895008in}{5.127963in}}%
\pgfpathlineto{\pgfqpoint{2.904434in}{5.137468in}}%
\pgfpathlineto{\pgfqpoint{2.909147in}{5.138078in}}%
\pgfpathlineto{\pgfqpoint{2.913860in}{5.140653in}}%
\pgfpathlineto{\pgfqpoint{2.918573in}{5.144521in}}%
\pgfpathlineto{\pgfqpoint{2.937424in}{5.150046in}}%
\pgfpathlineto{\pgfqpoint{2.951563in}{5.160917in}}%
\pgfpathlineto{\pgfqpoint{2.956276in}{5.160931in}}%
\pgfpathlineto{\pgfqpoint{2.960989in}{5.169021in}}%
\pgfpathlineto{\pgfqpoint{2.965702in}{5.169775in}}%
\pgfpathlineto{\pgfqpoint{2.970415in}{5.173472in}}%
\pgfpathlineto{\pgfqpoint{2.975128in}{5.175751in}}%
\pgfpathlineto{\pgfqpoint{2.984553in}{5.176842in}}%
\pgfpathlineto{\pgfqpoint{2.989266in}{5.181123in}}%
\pgfpathlineto{\pgfqpoint{2.993979in}{5.181536in}}%
\pgfpathlineto{\pgfqpoint{3.003405in}{5.190928in}}%
\pgfpathlineto{\pgfqpoint{3.012831in}{5.193292in}}%
\pgfpathlineto{\pgfqpoint{3.022257in}{5.199180in}}%
\pgfpathlineto{\pgfqpoint{3.026969in}{5.206216in}}%
\pgfpathlineto{\pgfqpoint{3.045821in}{5.208564in}}%
\pgfpathlineto{\pgfqpoint{3.055247in}{5.209872in}}%
\pgfpathlineto{\pgfqpoint{3.059960in}{5.215031in}}%
\pgfpathlineto{\pgfqpoint{3.069386in}{5.229936in}}%
\pgfpathlineto{\pgfqpoint{3.078811in}{5.231748in}}%
\pgfpathlineto{\pgfqpoint{3.083524in}{5.234361in}}%
\pgfpathlineto{\pgfqpoint{3.088237in}{5.235454in}}%
\pgfpathlineto{\pgfqpoint{3.092950in}{5.239846in}}%
\pgfpathlineto{\pgfqpoint{3.111802in}{5.241679in}}%
\pgfpathlineto{\pgfqpoint{3.121228in}{5.243354in}}%
\pgfpathlineto{\pgfqpoint{3.125940in}{5.247499in}}%
\pgfpathlineto{\pgfqpoint{3.182495in}{5.255927in}}%
\pgfpathlineto{\pgfqpoint{3.196634in}{5.256969in}}%
\pgfpathlineto{\pgfqpoint{3.201347in}{5.265553in}}%
\pgfpathlineto{\pgfqpoint{3.206060in}{5.270108in}}%
\pgfpathlineto{\pgfqpoint{3.210773in}{5.285113in}}%
\pgfpathlineto{\pgfqpoint{3.220199in}{5.289499in}}%
\pgfpathlineto{\pgfqpoint{3.224912in}{5.289920in}}%
\pgfpathlineto{\pgfqpoint{3.229624in}{5.292237in}}%
\pgfpathlineto{\pgfqpoint{3.234337in}{5.298023in}}%
\pgfpathlineto{\pgfqpoint{3.239050in}{5.299223in}}%
\pgfpathlineto{\pgfqpoint{3.243763in}{5.299246in}}%
\pgfpathlineto{\pgfqpoint{3.253189in}{5.304047in}}%
\pgfpathlineto{\pgfqpoint{3.257902in}{5.304406in}}%
\pgfpathlineto{\pgfqpoint{3.262615in}{5.305881in}}%
\pgfpathlineto{\pgfqpoint{3.267328in}{5.310073in}}%
\pgfpathlineto{\pgfqpoint{3.290892in}{5.312052in}}%
\pgfpathlineto{\pgfqpoint{3.305031in}{5.317707in}}%
\pgfpathlineto{\pgfqpoint{3.309744in}{5.318149in}}%
\pgfpathlineto{\pgfqpoint{3.314457in}{5.321515in}}%
\pgfpathlineto{\pgfqpoint{3.323883in}{5.322919in}}%
\pgfpathlineto{\pgfqpoint{3.328595in}{5.332500in}}%
\pgfpathlineto{\pgfqpoint{3.333308in}{5.338982in}}%
\pgfpathlineto{\pgfqpoint{3.352160in}{5.343563in}}%
\pgfpathlineto{\pgfqpoint{3.356873in}{5.343773in}}%
\pgfpathlineto{\pgfqpoint{3.371012in}{5.350171in}}%
\pgfpathlineto{\pgfqpoint{3.375724in}{5.350292in}}%
\pgfpathlineto{\pgfqpoint{3.385150in}{5.352359in}}%
\pgfpathlineto{\pgfqpoint{3.418141in}{5.356872in}}%
\pgfpathlineto{\pgfqpoint{3.427566in}{5.357624in}}%
\pgfpathlineto{\pgfqpoint{3.432279in}{5.359442in}}%
\pgfpathlineto{\pgfqpoint{3.441705in}{5.360211in}}%
\pgfpathlineto{\pgfqpoint{3.451131in}{5.364862in}}%
\pgfpathlineto{\pgfqpoint{3.465270in}{5.367127in}}%
\pgfpathlineto{\pgfqpoint{3.469983in}{5.378033in}}%
\pgfpathlineto{\pgfqpoint{3.484121in}{5.390601in}}%
\pgfpathlineto{\pgfqpoint{3.488834in}{5.390962in}}%
\pgfpathlineto{\pgfqpoint{3.493547in}{5.394073in}}%
\pgfpathlineto{\pgfqpoint{3.498260in}{5.394826in}}%
\pgfpathlineto{\pgfqpoint{3.502973in}{5.397158in}}%
\pgfpathlineto{\pgfqpoint{3.507686in}{5.397300in}}%
\pgfpathlineto{\pgfqpoint{3.517112in}{5.407208in}}%
\pgfpathlineto{\pgfqpoint{3.521825in}{5.407661in}}%
\pgfpathlineto{\pgfqpoint{3.526537in}{5.410083in}}%
\pgfpathlineto{\pgfqpoint{3.535963in}{5.411620in}}%
\pgfpathlineto{\pgfqpoint{3.545389in}{5.413550in}}%
\pgfpathlineto{\pgfqpoint{3.550102in}{5.417132in}}%
\pgfpathlineto{\pgfqpoint{3.554815in}{5.418925in}}%
\pgfpathlineto{\pgfqpoint{3.568954in}{5.420351in}}%
\pgfpathlineto{\pgfqpoint{3.583092in}{5.427734in}}%
\pgfpathlineto{\pgfqpoint{3.587805in}{5.442225in}}%
\pgfpathlineto{\pgfqpoint{3.592518in}{5.447128in}}%
\pgfpathlineto{\pgfqpoint{3.630221in}{5.455121in}}%
\pgfpathlineto{\pgfqpoint{3.634934in}{5.458362in}}%
\pgfpathlineto{\pgfqpoint{3.677350in}{5.462222in}}%
\pgfpathlineto{\pgfqpoint{3.682063in}{5.463075in}}%
\pgfpathlineto{\pgfqpoint{3.686776in}{5.468746in}}%
\pgfpathlineto{\pgfqpoint{3.691489in}{5.469064in}}%
\pgfpathlineto{\pgfqpoint{3.696202in}{5.471304in}}%
\pgfpathlineto{\pgfqpoint{3.719767in}{5.472976in}}%
\pgfpathlineto{\pgfqpoint{3.724480in}{5.478201in}}%
\pgfpathlineto{\pgfqpoint{3.729192in}{5.479731in}}%
\pgfpathlineto{\pgfqpoint{3.733905in}{5.487667in}}%
\pgfpathlineto{\pgfqpoint{3.738618in}{5.492955in}}%
\pgfpathlineto{\pgfqpoint{3.743331in}{5.494997in}}%
\pgfpathlineto{\pgfqpoint{3.752757in}{5.500871in}}%
\pgfpathlineto{\pgfqpoint{3.757470in}{5.501279in}}%
\pgfpathlineto{\pgfqpoint{3.762183in}{5.503357in}}%
\pgfpathlineto{\pgfqpoint{3.766896in}{5.503841in}}%
\pgfpathlineto{\pgfqpoint{3.771609in}{5.505590in}}%
\pgfpathlineto{\pgfqpoint{3.776321in}{5.505598in}}%
\pgfpathlineto{\pgfqpoint{3.781034in}{5.510552in}}%
\pgfpathlineto{\pgfqpoint{3.795173in}{5.512266in}}%
\pgfpathlineto{\pgfqpoint{3.799886in}{5.514675in}}%
\pgfpathlineto{\pgfqpoint{3.809312in}{5.515917in}}%
\pgfpathlineto{\pgfqpoint{3.814025in}{5.519111in}}%
\pgfpathlineto{\pgfqpoint{3.818738in}{5.520602in}}%
\pgfpathlineto{\pgfqpoint{3.823451in}{5.528692in}}%
\pgfpathlineto{\pgfqpoint{3.828163in}{5.529032in}}%
\pgfpathlineto{\pgfqpoint{3.832876in}{5.537438in}}%
\pgfpathlineto{\pgfqpoint{3.842302in}{5.545872in}}%
\pgfpathlineto{\pgfqpoint{3.870580in}{5.549601in}}%
\pgfpathlineto{\pgfqpoint{3.875293in}{5.552425in}}%
\pgfpathlineto{\pgfqpoint{3.903570in}{5.556148in}}%
\pgfpathlineto{\pgfqpoint{3.908283in}{5.558193in}}%
\pgfpathlineto{\pgfqpoint{3.917709in}{5.559687in}}%
\pgfpathlineto{\pgfqpoint{3.922422in}{5.561107in}}%
\pgfpathlineto{\pgfqpoint{3.931847in}{5.565982in}}%
\pgfpathlineto{\pgfqpoint{3.945986in}{5.568091in}}%
\pgfpathlineto{\pgfqpoint{3.950699in}{5.569725in}}%
\pgfpathlineto{\pgfqpoint{3.955412in}{5.569938in}}%
\pgfpathlineto{\pgfqpoint{3.964838in}{5.572827in}}%
\pgfpathlineto{\pgfqpoint{3.969551in}{5.574072in}}%
\pgfpathlineto{\pgfqpoint{3.978976in}{5.583038in}}%
\pgfpathlineto{\pgfqpoint{3.993115in}{5.585451in}}%
\pgfpathlineto{\pgfqpoint{3.997828in}{5.588475in}}%
\pgfpathlineto{\pgfqpoint{4.040244in}{5.593912in}}%
\pgfpathlineto{\pgfqpoint{4.044957in}{5.598248in}}%
\pgfpathlineto{\pgfqpoint{4.054383in}{5.599291in}}%
\pgfpathlineto{\pgfqpoint{4.059096in}{5.599574in}}%
\pgfpathlineto{\pgfqpoint{4.068522in}{5.603776in}}%
\pgfpathlineto{\pgfqpoint{4.077947in}{5.605228in}}%
\pgfpathlineto{\pgfqpoint{4.082660in}{5.616673in}}%
\pgfpathlineto{\pgfqpoint{4.096799in}{5.623299in}}%
\pgfpathlineto{\pgfqpoint{4.101512in}{5.623587in}}%
\pgfpathlineto{\pgfqpoint{4.110938in}{5.627149in}}%
\pgfpathlineto{\pgfqpoint{4.125077in}{5.629305in}}%
\pgfpathlineto{\pgfqpoint{4.129789in}{5.632533in}}%
\pgfpathlineto{\pgfqpoint{4.139215in}{5.634912in}}%
\pgfpathlineto{\pgfqpoint{4.143928in}{5.638689in}}%
\pgfpathlineto{\pgfqpoint{4.153354in}{5.640046in}}%
\pgfpathlineto{\pgfqpoint{4.162780in}{5.642504in}}%
\pgfpathlineto{\pgfqpoint{4.167493in}{5.642527in}}%
\pgfpathlineto{\pgfqpoint{4.172206in}{5.654851in}}%
\pgfpathlineto{\pgfqpoint{4.186344in}{5.659429in}}%
\pgfpathlineto{\pgfqpoint{4.195770in}{5.661319in}}%
\pgfpathlineto{\pgfqpoint{4.209909in}{5.662819in}}%
\pgfpathlineto{\pgfqpoint{4.224048in}{5.663800in}}%
\pgfpathlineto{\pgfqpoint{4.228760in}{5.664012in}}%
\pgfpathlineto{\pgfqpoint{4.233473in}{5.667883in}}%
\pgfpathlineto{\pgfqpoint{4.247612in}{5.670380in}}%
\pgfpathlineto{\pgfqpoint{4.252325in}{5.676552in}}%
\pgfpathlineto{\pgfqpoint{4.261751in}{5.679536in}}%
\pgfpathlineto{\pgfqpoint{4.266464in}{5.683464in}}%
\pgfpathlineto{\pgfqpoint{4.271177in}{5.691831in}}%
\pgfpathlineto{\pgfqpoint{4.275889in}{5.693992in}}%
\pgfpathlineto{\pgfqpoint{4.280602in}{5.694141in}}%
\pgfpathlineto{\pgfqpoint{4.285315in}{5.698921in}}%
\pgfpathlineto{\pgfqpoint{4.299454in}{5.700812in}}%
\pgfpathlineto{\pgfqpoint{4.304167in}{5.700978in}}%
\pgfpathlineto{\pgfqpoint{4.308880in}{5.702832in}}%
\pgfpathlineto{\pgfqpoint{4.313593in}{5.703061in}}%
\pgfpathlineto{\pgfqpoint{4.318306in}{5.708475in}}%
\pgfpathlineto{\pgfqpoint{4.327731in}{5.710915in}}%
\pgfpathlineto{\pgfqpoint{4.332444in}{5.710991in}}%
\pgfpathlineto{\pgfqpoint{4.337157in}{5.712733in}}%
\pgfpathlineto{\pgfqpoint{4.356009in}{5.715359in}}%
\pgfpathlineto{\pgfqpoint{4.370148in}{5.730409in}}%
\pgfpathlineto{\pgfqpoint{4.379573in}{5.736077in}}%
\pgfpathlineto{\pgfqpoint{4.384286in}{5.742667in}}%
\pgfpathlineto{\pgfqpoint{4.388999in}{5.742709in}}%
\pgfpathlineto{\pgfqpoint{4.393712in}{5.744583in}}%
\pgfpathlineto{\pgfqpoint{4.398425in}{5.744848in}}%
\pgfpathlineto{\pgfqpoint{4.403138in}{5.752797in}}%
\pgfpathlineto{\pgfqpoint{4.407851in}{5.753141in}}%
\pgfpathlineto{\pgfqpoint{4.412564in}{5.757406in}}%
\pgfpathlineto{\pgfqpoint{4.417277in}{5.770061in}}%
\pgfpathlineto{\pgfqpoint{4.421990in}{5.870479in}}%
\pgfpathlineto{\pgfqpoint{4.426702in}{5.905275in}}%
\pgfpathlineto{\pgfqpoint{5.845287in}{5.905275in}}%
\pgfpathlineto{\pgfqpoint{5.845287in}{5.905275in}}%
\pgfusepath{stroke}%
\end{pgfscope}%
\begin{pgfscope}%
\pgfsetrectcap%
\pgfsetmiterjoin%
\pgfsetlinewidth{0.803000pt}%
\definecolor{currentstroke}{rgb}{0.000000,0.000000,0.000000}%
\pgfsetstrokecolor{currentstroke}%
\pgfsetdash{}{0pt}%
\pgfpathmoveto{\pgfqpoint{0.708220in}{4.502489in}}%
\pgfpathlineto{\pgfqpoint{0.708220in}{5.905275in}}%
\pgfusepath{stroke}%
\end{pgfscope}%
\begin{pgfscope}%
\pgfsetrectcap%
\pgfsetmiterjoin%
\pgfsetlinewidth{0.803000pt}%
\definecolor{currentstroke}{rgb}{0.000000,0.000000,0.000000}%
\pgfsetstrokecolor{currentstroke}%
\pgfsetdash{}{0pt}%
\pgfpathmoveto{\pgfqpoint{5.850000in}{4.502489in}}%
\pgfpathlineto{\pgfqpoint{5.850000in}{5.905275in}}%
\pgfusepath{stroke}%
\end{pgfscope}%
\begin{pgfscope}%
\pgfsetrectcap%
\pgfsetmiterjoin%
\pgfsetlinewidth{0.803000pt}%
\definecolor{currentstroke}{rgb}{0.000000,0.000000,0.000000}%
\pgfsetstrokecolor{currentstroke}%
\pgfsetdash{}{0pt}%
\pgfpathmoveto{\pgfqpoint{0.708220in}{4.502489in}}%
\pgfpathlineto{\pgfqpoint{5.850000in}{4.502489in}}%
\pgfusepath{stroke}%
\end{pgfscope}%
\begin{pgfscope}%
\pgfsetrectcap%
\pgfsetmiterjoin%
\pgfsetlinewidth{0.803000pt}%
\definecolor{currentstroke}{rgb}{0.000000,0.000000,0.000000}%
\pgfsetstrokecolor{currentstroke}%
\pgfsetdash{}{0pt}%
\pgfpathmoveto{\pgfqpoint{0.708220in}{5.905275in}}%
\pgfpathlineto{\pgfqpoint{5.850000in}{5.905275in}}%
\pgfusepath{stroke}%
\end{pgfscope}%
\begin{pgfscope}%
\pgfsetbuttcap%
\pgfsetroundjoin%
\pgfsetlinewidth{2.007500pt}%
\definecolor{currentstroke}{rgb}{1.000000,0.843137,0.000000}%
\pgfsetstrokecolor{currentstroke}%
\pgfsetdash{{7.400000pt}{3.200000pt}}{0.000000pt}%
\pgfpathmoveto{\pgfqpoint{4.827505in}{4.944138in}}%
\pgfpathlineto{\pgfqpoint{5.077505in}{4.944138in}}%
\pgfusepath{stroke}%
\end{pgfscope}%
\begin{pgfscope}%
\definecolor{textcolor}{rgb}{0.000000,0.000000,0.000000}%
\pgfsetstrokecolor{textcolor}%
\pgfsetfillcolor{textcolor}%
\pgftext[x=5.102505in,y=4.900388in,left,base]{\color{textcolor}\rmfamily\fontsize{9.000000}{10.800000}\selectfont FT+htd}%
\end{pgfscope}%
\begin{pgfscope}%
\pgfsetbuttcap%
\pgfsetroundjoin%
\pgfsetlinewidth{2.007500pt}%
\definecolor{currentstroke}{rgb}{1.000000,0.694118,0.305882}%
\pgfsetstrokecolor{currentstroke}%
\pgfsetdash{{2.000000pt}{3.300000pt}}{0.000000pt}%
\pgfpathmoveto{\pgfqpoint{4.827505in}{4.782338in}}%
\pgfpathlineto{\pgfqpoint{5.077505in}{4.782338in}}%
\pgfusepath{stroke}%
\end{pgfscope}%
\begin{pgfscope}%
\definecolor{textcolor}{rgb}{0.000000,0.000000,0.000000}%
\pgfsetstrokecolor{textcolor}%
\pgfsetfillcolor{textcolor}%
\pgftext[x=5.102505in,y=4.738588in,left,base]{\color{textcolor}\rmfamily\fontsize{9.000000}{10.800000}\selectfont FT+Flow}%
\end{pgfscope}%
\begin{pgfscope}%
\pgfsetrectcap%
\pgfsetroundjoin%
\pgfsetlinewidth{2.007500pt}%
\definecolor{currentstroke}{rgb}{0.980392,0.529412,0.458824}%
\pgfsetstrokecolor{currentstroke}%
\pgfsetdash{}{0pt}%
\pgfpathmoveto{\pgfqpoint{4.827505in}{4.620539in}}%
\pgfpathlineto{\pgfqpoint{5.077505in}{4.620539in}}%
\pgfusepath{stroke}%
\end{pgfscope}%
\begin{pgfscope}%
\definecolor{textcolor}{rgb}{0.000000,0.000000,0.000000}%
\pgfsetstrokecolor{textcolor}%
\pgfsetfillcolor{textcolor}%
\pgftext[x=5.102505in,y=4.576789in,left,base]{\color{textcolor}\rmfamily\fontsize{9.000000}{10.800000}\selectfont FT+Tamaki}%
\end{pgfscope}%
\begin{pgfscope}%
\pgfsetbuttcap%
\pgfsetmiterjoin%
\definecolor{currentfill}{rgb}{1.000000,1.000000,1.000000}%
\pgfsetfillcolor{currentfill}%
\pgfsetlinewidth{0.000000pt}%
\definecolor{currentstroke}{rgb}{0.000000,0.000000,0.000000}%
\pgfsetstrokecolor{currentstroke}%
\pgfsetstrokeopacity{0.000000}%
\pgfsetdash{}{0pt}%
\pgfpathmoveto{\pgfqpoint{0.708220in}{2.519156in}}%
\pgfpathlineto{\pgfqpoint{5.850000in}{2.519156in}}%
\pgfpathlineto{\pgfqpoint{5.850000in}{3.921942in}}%
\pgfpathlineto{\pgfqpoint{0.708220in}{3.921942in}}%
\pgfpathclose%
\pgfusepath{fill}%
\end{pgfscope}%
\begin{pgfscope}%
\pgfsetbuttcap%
\pgfsetroundjoin%
\definecolor{currentfill}{rgb}{0.000000,0.000000,0.000000}%
\pgfsetfillcolor{currentfill}%
\pgfsetlinewidth{0.803000pt}%
\definecolor{currentstroke}{rgb}{0.000000,0.000000,0.000000}%
\pgfsetstrokecolor{currentstroke}%
\pgfsetdash{}{0pt}%
\pgfsys@defobject{currentmarker}{\pgfqpoint{0.000000in}{-0.048611in}}{\pgfqpoint{0.000000in}{0.000000in}}{%
\pgfpathmoveto{\pgfqpoint{0.000000in}{0.000000in}}%
\pgfpathlineto{\pgfqpoint{0.000000in}{-0.048611in}}%
\pgfusepath{stroke,fill}%
}%
\begin{pgfscope}%
\pgfsys@transformshift{0.708220in}{2.519156in}%
\pgfsys@useobject{currentmarker}{}%
\end{pgfscope}%
\end{pgfscope}%
\begin{pgfscope}%
\definecolor{textcolor}{rgb}{0.000000,0.000000,0.000000}%
\pgfsetstrokecolor{textcolor}%
\pgfsetfillcolor{textcolor}%
\pgftext[x=0.708220in,y=2.421934in,,top]{\color{textcolor}\rmfamily\fontsize{9.000000}{10.800000}\selectfont \(\displaystyle {0}\)}%
\end{pgfscope}%
\begin{pgfscope}%
\pgfsetbuttcap%
\pgfsetroundjoin%
\definecolor{currentfill}{rgb}{0.000000,0.000000,0.000000}%
\pgfsetfillcolor{currentfill}%
\pgfsetlinewidth{0.803000pt}%
\definecolor{currentstroke}{rgb}{0.000000,0.000000,0.000000}%
\pgfsetstrokecolor{currentstroke}%
\pgfsetdash{}{0pt}%
\pgfsys@defobject{currentmarker}{\pgfqpoint{0.000000in}{-0.048611in}}{\pgfqpoint{0.000000in}{0.000000in}}{%
\pgfpathmoveto{\pgfqpoint{0.000000in}{0.000000in}}%
\pgfpathlineto{\pgfqpoint{0.000000in}{-0.048611in}}%
\pgfusepath{stroke,fill}%
}%
\begin{pgfscope}%
\pgfsys@transformshift{1.650801in}{2.519156in}%
\pgfsys@useobject{currentmarker}{}%
\end{pgfscope}%
\end{pgfscope}%
\begin{pgfscope}%
\definecolor{textcolor}{rgb}{0.000000,0.000000,0.000000}%
\pgfsetstrokecolor{textcolor}%
\pgfsetfillcolor{textcolor}%
\pgftext[x=1.650801in,y=2.421934in,,top]{\color{textcolor}\rmfamily\fontsize{9.000000}{10.800000}\selectfont \(\displaystyle {200}\)}%
\end{pgfscope}%
\begin{pgfscope}%
\pgfsetbuttcap%
\pgfsetroundjoin%
\definecolor{currentfill}{rgb}{0.000000,0.000000,0.000000}%
\pgfsetfillcolor{currentfill}%
\pgfsetlinewidth{0.803000pt}%
\definecolor{currentstroke}{rgb}{0.000000,0.000000,0.000000}%
\pgfsetstrokecolor{currentstroke}%
\pgfsetdash{}{0pt}%
\pgfsys@defobject{currentmarker}{\pgfqpoint{0.000000in}{-0.048611in}}{\pgfqpoint{0.000000in}{0.000000in}}{%
\pgfpathmoveto{\pgfqpoint{0.000000in}{0.000000in}}%
\pgfpathlineto{\pgfqpoint{0.000000in}{-0.048611in}}%
\pgfusepath{stroke,fill}%
}%
\begin{pgfscope}%
\pgfsys@transformshift{2.593382in}{2.519156in}%
\pgfsys@useobject{currentmarker}{}%
\end{pgfscope}%
\end{pgfscope}%
\begin{pgfscope}%
\definecolor{textcolor}{rgb}{0.000000,0.000000,0.000000}%
\pgfsetstrokecolor{textcolor}%
\pgfsetfillcolor{textcolor}%
\pgftext[x=2.593382in,y=2.421934in,,top]{\color{textcolor}\rmfamily\fontsize{9.000000}{10.800000}\selectfont \(\displaystyle {400}\)}%
\end{pgfscope}%
\begin{pgfscope}%
\pgfsetbuttcap%
\pgfsetroundjoin%
\definecolor{currentfill}{rgb}{0.000000,0.000000,0.000000}%
\pgfsetfillcolor{currentfill}%
\pgfsetlinewidth{0.803000pt}%
\definecolor{currentstroke}{rgb}{0.000000,0.000000,0.000000}%
\pgfsetstrokecolor{currentstroke}%
\pgfsetdash{}{0pt}%
\pgfsys@defobject{currentmarker}{\pgfqpoint{0.000000in}{-0.048611in}}{\pgfqpoint{0.000000in}{0.000000in}}{%
\pgfpathmoveto{\pgfqpoint{0.000000in}{0.000000in}}%
\pgfpathlineto{\pgfqpoint{0.000000in}{-0.048611in}}%
\pgfusepath{stroke,fill}%
}%
\begin{pgfscope}%
\pgfsys@transformshift{3.535963in}{2.519156in}%
\pgfsys@useobject{currentmarker}{}%
\end{pgfscope}%
\end{pgfscope}%
\begin{pgfscope}%
\definecolor{textcolor}{rgb}{0.000000,0.000000,0.000000}%
\pgfsetstrokecolor{textcolor}%
\pgfsetfillcolor{textcolor}%
\pgftext[x=3.535963in,y=2.421934in,,top]{\color{textcolor}\rmfamily\fontsize{9.000000}{10.800000}\selectfont \(\displaystyle {600}\)}%
\end{pgfscope}%
\begin{pgfscope}%
\pgfsetbuttcap%
\pgfsetroundjoin%
\definecolor{currentfill}{rgb}{0.000000,0.000000,0.000000}%
\pgfsetfillcolor{currentfill}%
\pgfsetlinewidth{0.803000pt}%
\definecolor{currentstroke}{rgb}{0.000000,0.000000,0.000000}%
\pgfsetstrokecolor{currentstroke}%
\pgfsetdash{}{0pt}%
\pgfsys@defobject{currentmarker}{\pgfqpoint{0.000000in}{-0.048611in}}{\pgfqpoint{0.000000in}{0.000000in}}{%
\pgfpathmoveto{\pgfqpoint{0.000000in}{0.000000in}}%
\pgfpathlineto{\pgfqpoint{0.000000in}{-0.048611in}}%
\pgfusepath{stroke,fill}%
}%
\begin{pgfscope}%
\pgfsys@transformshift{4.478544in}{2.519156in}%
\pgfsys@useobject{currentmarker}{}%
\end{pgfscope}%
\end{pgfscope}%
\begin{pgfscope}%
\definecolor{textcolor}{rgb}{0.000000,0.000000,0.000000}%
\pgfsetstrokecolor{textcolor}%
\pgfsetfillcolor{textcolor}%
\pgftext[x=4.478544in,y=2.421934in,,top]{\color{textcolor}\rmfamily\fontsize{9.000000}{10.800000}\selectfont \(\displaystyle {800}\)}%
\end{pgfscope}%
\begin{pgfscope}%
\pgfsetbuttcap%
\pgfsetroundjoin%
\definecolor{currentfill}{rgb}{0.000000,0.000000,0.000000}%
\pgfsetfillcolor{currentfill}%
\pgfsetlinewidth{0.803000pt}%
\definecolor{currentstroke}{rgb}{0.000000,0.000000,0.000000}%
\pgfsetstrokecolor{currentstroke}%
\pgfsetdash{}{0pt}%
\pgfsys@defobject{currentmarker}{\pgfqpoint{0.000000in}{-0.048611in}}{\pgfqpoint{0.000000in}{0.000000in}}{%
\pgfpathmoveto{\pgfqpoint{0.000000in}{0.000000in}}%
\pgfpathlineto{\pgfqpoint{0.000000in}{-0.048611in}}%
\pgfusepath{stroke,fill}%
}%
\begin{pgfscope}%
\pgfsys@transformshift{5.421126in}{2.519156in}%
\pgfsys@useobject{currentmarker}{}%
\end{pgfscope}%
\end{pgfscope}%
\begin{pgfscope}%
\definecolor{textcolor}{rgb}{0.000000,0.000000,0.000000}%
\pgfsetstrokecolor{textcolor}%
\pgfsetfillcolor{textcolor}%
\pgftext[x=5.421126in,y=2.421934in,,top]{\color{textcolor}\rmfamily\fontsize{9.000000}{10.800000}\selectfont \(\displaystyle {1000}\)}%
\end{pgfscope}%
\begin{pgfscope}%
\definecolor{textcolor}{rgb}{0.000000,0.000000,0.000000}%
\pgfsetstrokecolor{textcolor}%
\pgfsetfillcolor{textcolor}%
\pgftext[x=3.279110in,y=2.255988in,,top]{\color{textcolor}\rmfamily\fontsize{10.000000}{12.000000}\selectfont Number of benchmarks solved}%
\end{pgfscope}%
\begin{pgfscope}%
\pgfsetbuttcap%
\pgfsetroundjoin%
\definecolor{currentfill}{rgb}{0.000000,0.000000,0.000000}%
\pgfsetfillcolor{currentfill}%
\pgfsetlinewidth{0.803000pt}%
\definecolor{currentstroke}{rgb}{0.000000,0.000000,0.000000}%
\pgfsetstrokecolor{currentstroke}%
\pgfsetdash{}{0pt}%
\pgfsys@defobject{currentmarker}{\pgfqpoint{-0.048611in}{0.000000in}}{\pgfqpoint{-0.000000in}{0.000000in}}{%
\pgfpathmoveto{\pgfqpoint{-0.000000in}{0.000000in}}%
\pgfpathlineto{\pgfqpoint{-0.048611in}{0.000000in}}%
\pgfusepath{stroke,fill}%
}%
\begin{pgfscope}%
\pgfsys@transformshift{0.708220in}{2.519156in}%
\pgfsys@useobject{currentmarker}{}%
\end{pgfscope}%
\end{pgfscope}%
\begin{pgfscope}%
\definecolor{textcolor}{rgb}{0.000000,0.000000,0.000000}%
\pgfsetstrokecolor{textcolor}%
\pgfsetfillcolor{textcolor}%
\pgftext[x=0.344411in, y=2.474431in, left, base]{\color{textcolor}\rmfamily\fontsize{9.000000}{10.800000}\selectfont \(\displaystyle {10^{-1}}\)}%
\end{pgfscope}%
\begin{pgfscope}%
\pgfsetbuttcap%
\pgfsetroundjoin%
\definecolor{currentfill}{rgb}{0.000000,0.000000,0.000000}%
\pgfsetfillcolor{currentfill}%
\pgfsetlinewidth{0.803000pt}%
\definecolor{currentstroke}{rgb}{0.000000,0.000000,0.000000}%
\pgfsetstrokecolor{currentstroke}%
\pgfsetdash{}{0pt}%
\pgfsys@defobject{currentmarker}{\pgfqpoint{-0.048611in}{0.000000in}}{\pgfqpoint{-0.000000in}{0.000000in}}{%
\pgfpathmoveto{\pgfqpoint{-0.000000in}{0.000000in}}%
\pgfpathlineto{\pgfqpoint{-0.048611in}{0.000000in}}%
\pgfusepath{stroke,fill}%
}%
\begin{pgfscope}%
\pgfsys@transformshift{0.708220in}{2.869852in}%
\pgfsys@useobject{currentmarker}{}%
\end{pgfscope}%
\end{pgfscope}%
\begin{pgfscope}%
\definecolor{textcolor}{rgb}{0.000000,0.000000,0.000000}%
\pgfsetstrokecolor{textcolor}%
\pgfsetfillcolor{textcolor}%
\pgftext[x=0.424657in, y=2.825128in, left, base]{\color{textcolor}\rmfamily\fontsize{9.000000}{10.800000}\selectfont \(\displaystyle {10^{0}}\)}%
\end{pgfscope}%
\begin{pgfscope}%
\pgfsetbuttcap%
\pgfsetroundjoin%
\definecolor{currentfill}{rgb}{0.000000,0.000000,0.000000}%
\pgfsetfillcolor{currentfill}%
\pgfsetlinewidth{0.803000pt}%
\definecolor{currentstroke}{rgb}{0.000000,0.000000,0.000000}%
\pgfsetstrokecolor{currentstroke}%
\pgfsetdash{}{0pt}%
\pgfsys@defobject{currentmarker}{\pgfqpoint{-0.048611in}{0.000000in}}{\pgfqpoint{-0.000000in}{0.000000in}}{%
\pgfpathmoveto{\pgfqpoint{-0.000000in}{0.000000in}}%
\pgfpathlineto{\pgfqpoint{-0.048611in}{0.000000in}}%
\pgfusepath{stroke,fill}%
}%
\begin{pgfscope}%
\pgfsys@transformshift{0.708220in}{3.220549in}%
\pgfsys@useobject{currentmarker}{}%
\end{pgfscope}%
\end{pgfscope}%
\begin{pgfscope}%
\definecolor{textcolor}{rgb}{0.000000,0.000000,0.000000}%
\pgfsetstrokecolor{textcolor}%
\pgfsetfillcolor{textcolor}%
\pgftext[x=0.424657in, y=3.175824in, left, base]{\color{textcolor}\rmfamily\fontsize{9.000000}{10.800000}\selectfont \(\displaystyle {10^{1}}\)}%
\end{pgfscope}%
\begin{pgfscope}%
\pgfsetbuttcap%
\pgfsetroundjoin%
\definecolor{currentfill}{rgb}{0.000000,0.000000,0.000000}%
\pgfsetfillcolor{currentfill}%
\pgfsetlinewidth{0.803000pt}%
\definecolor{currentstroke}{rgb}{0.000000,0.000000,0.000000}%
\pgfsetstrokecolor{currentstroke}%
\pgfsetdash{}{0pt}%
\pgfsys@defobject{currentmarker}{\pgfqpoint{-0.048611in}{0.000000in}}{\pgfqpoint{-0.000000in}{0.000000in}}{%
\pgfpathmoveto{\pgfqpoint{-0.000000in}{0.000000in}}%
\pgfpathlineto{\pgfqpoint{-0.048611in}{0.000000in}}%
\pgfusepath{stroke,fill}%
}%
\begin{pgfscope}%
\pgfsys@transformshift{0.708220in}{3.571245in}%
\pgfsys@useobject{currentmarker}{}%
\end{pgfscope}%
\end{pgfscope}%
\begin{pgfscope}%
\definecolor{textcolor}{rgb}{0.000000,0.000000,0.000000}%
\pgfsetstrokecolor{textcolor}%
\pgfsetfillcolor{textcolor}%
\pgftext[x=0.424657in, y=3.526521in, left, base]{\color{textcolor}\rmfamily\fontsize{9.000000}{10.800000}\selectfont \(\displaystyle {10^{2}}\)}%
\end{pgfscope}%
\begin{pgfscope}%
\pgfsetbuttcap%
\pgfsetroundjoin%
\definecolor{currentfill}{rgb}{0.000000,0.000000,0.000000}%
\pgfsetfillcolor{currentfill}%
\pgfsetlinewidth{0.803000pt}%
\definecolor{currentstroke}{rgb}{0.000000,0.000000,0.000000}%
\pgfsetstrokecolor{currentstroke}%
\pgfsetdash{}{0pt}%
\pgfsys@defobject{currentmarker}{\pgfqpoint{-0.048611in}{0.000000in}}{\pgfqpoint{-0.000000in}{0.000000in}}{%
\pgfpathmoveto{\pgfqpoint{-0.000000in}{0.000000in}}%
\pgfpathlineto{\pgfqpoint{-0.048611in}{0.000000in}}%
\pgfusepath{stroke,fill}%
}%
\begin{pgfscope}%
\pgfsys@transformshift{0.708220in}{3.921942in}%
\pgfsys@useobject{currentmarker}{}%
\end{pgfscope}%
\end{pgfscope}%
\begin{pgfscope}%
\definecolor{textcolor}{rgb}{0.000000,0.000000,0.000000}%
\pgfsetstrokecolor{textcolor}%
\pgfsetfillcolor{textcolor}%
\pgftext[x=0.424657in, y=3.877217in, left, base]{\color{textcolor}\rmfamily\fontsize{9.000000}{10.800000}\selectfont \(\displaystyle {10^{3}}\)}%
\end{pgfscope}%
\begin{pgfscope}%
\pgfsetbuttcap%
\pgfsetroundjoin%
\definecolor{currentfill}{rgb}{0.000000,0.000000,0.000000}%
\pgfsetfillcolor{currentfill}%
\pgfsetlinewidth{0.602250pt}%
\definecolor{currentstroke}{rgb}{0.000000,0.000000,0.000000}%
\pgfsetstrokecolor{currentstroke}%
\pgfsetdash{}{0pt}%
\pgfsys@defobject{currentmarker}{\pgfqpoint{-0.027778in}{0.000000in}}{\pgfqpoint{-0.000000in}{0.000000in}}{%
\pgfpathmoveto{\pgfqpoint{-0.000000in}{0.000000in}}%
\pgfpathlineto{\pgfqpoint{-0.027778in}{0.000000in}}%
\pgfusepath{stroke,fill}%
}%
\begin{pgfscope}%
\pgfsys@transformshift{0.708220in}{2.624726in}%
\pgfsys@useobject{currentmarker}{}%
\end{pgfscope}%
\end{pgfscope}%
\begin{pgfscope}%
\pgfsetbuttcap%
\pgfsetroundjoin%
\definecolor{currentfill}{rgb}{0.000000,0.000000,0.000000}%
\pgfsetfillcolor{currentfill}%
\pgfsetlinewidth{0.602250pt}%
\definecolor{currentstroke}{rgb}{0.000000,0.000000,0.000000}%
\pgfsetstrokecolor{currentstroke}%
\pgfsetdash{}{0pt}%
\pgfsys@defobject{currentmarker}{\pgfqpoint{-0.027778in}{0.000000in}}{\pgfqpoint{-0.000000in}{0.000000in}}{%
\pgfpathmoveto{\pgfqpoint{-0.000000in}{0.000000in}}%
\pgfpathlineto{\pgfqpoint{-0.027778in}{0.000000in}}%
\pgfusepath{stroke,fill}%
}%
\begin{pgfscope}%
\pgfsys@transformshift{0.708220in}{2.686481in}%
\pgfsys@useobject{currentmarker}{}%
\end{pgfscope}%
\end{pgfscope}%
\begin{pgfscope}%
\pgfsetbuttcap%
\pgfsetroundjoin%
\definecolor{currentfill}{rgb}{0.000000,0.000000,0.000000}%
\pgfsetfillcolor{currentfill}%
\pgfsetlinewidth{0.602250pt}%
\definecolor{currentstroke}{rgb}{0.000000,0.000000,0.000000}%
\pgfsetstrokecolor{currentstroke}%
\pgfsetdash{}{0pt}%
\pgfsys@defobject{currentmarker}{\pgfqpoint{-0.027778in}{0.000000in}}{\pgfqpoint{-0.000000in}{0.000000in}}{%
\pgfpathmoveto{\pgfqpoint{-0.000000in}{0.000000in}}%
\pgfpathlineto{\pgfqpoint{-0.027778in}{0.000000in}}%
\pgfusepath{stroke,fill}%
}%
\begin{pgfscope}%
\pgfsys@transformshift{0.708220in}{2.730296in}%
\pgfsys@useobject{currentmarker}{}%
\end{pgfscope}%
\end{pgfscope}%
\begin{pgfscope}%
\pgfsetbuttcap%
\pgfsetroundjoin%
\definecolor{currentfill}{rgb}{0.000000,0.000000,0.000000}%
\pgfsetfillcolor{currentfill}%
\pgfsetlinewidth{0.602250pt}%
\definecolor{currentstroke}{rgb}{0.000000,0.000000,0.000000}%
\pgfsetstrokecolor{currentstroke}%
\pgfsetdash{}{0pt}%
\pgfsys@defobject{currentmarker}{\pgfqpoint{-0.027778in}{0.000000in}}{\pgfqpoint{-0.000000in}{0.000000in}}{%
\pgfpathmoveto{\pgfqpoint{-0.000000in}{0.000000in}}%
\pgfpathlineto{\pgfqpoint{-0.027778in}{0.000000in}}%
\pgfusepath{stroke,fill}%
}%
\begin{pgfscope}%
\pgfsys@transformshift{0.708220in}{2.764282in}%
\pgfsys@useobject{currentmarker}{}%
\end{pgfscope}%
\end{pgfscope}%
\begin{pgfscope}%
\pgfsetbuttcap%
\pgfsetroundjoin%
\definecolor{currentfill}{rgb}{0.000000,0.000000,0.000000}%
\pgfsetfillcolor{currentfill}%
\pgfsetlinewidth{0.602250pt}%
\definecolor{currentstroke}{rgb}{0.000000,0.000000,0.000000}%
\pgfsetstrokecolor{currentstroke}%
\pgfsetdash{}{0pt}%
\pgfsys@defobject{currentmarker}{\pgfqpoint{-0.027778in}{0.000000in}}{\pgfqpoint{-0.000000in}{0.000000in}}{%
\pgfpathmoveto{\pgfqpoint{-0.000000in}{0.000000in}}%
\pgfpathlineto{\pgfqpoint{-0.027778in}{0.000000in}}%
\pgfusepath{stroke,fill}%
}%
\begin{pgfscope}%
\pgfsys@transformshift{0.708220in}{2.792051in}%
\pgfsys@useobject{currentmarker}{}%
\end{pgfscope}%
\end{pgfscope}%
\begin{pgfscope}%
\pgfsetbuttcap%
\pgfsetroundjoin%
\definecolor{currentfill}{rgb}{0.000000,0.000000,0.000000}%
\pgfsetfillcolor{currentfill}%
\pgfsetlinewidth{0.602250pt}%
\definecolor{currentstroke}{rgb}{0.000000,0.000000,0.000000}%
\pgfsetstrokecolor{currentstroke}%
\pgfsetdash{}{0pt}%
\pgfsys@defobject{currentmarker}{\pgfqpoint{-0.027778in}{0.000000in}}{\pgfqpoint{-0.000000in}{0.000000in}}{%
\pgfpathmoveto{\pgfqpoint{-0.000000in}{0.000000in}}%
\pgfpathlineto{\pgfqpoint{-0.027778in}{0.000000in}}%
\pgfusepath{stroke,fill}%
}%
\begin{pgfscope}%
\pgfsys@transformshift{0.708220in}{2.815529in}%
\pgfsys@useobject{currentmarker}{}%
\end{pgfscope}%
\end{pgfscope}%
\begin{pgfscope}%
\pgfsetbuttcap%
\pgfsetroundjoin%
\definecolor{currentfill}{rgb}{0.000000,0.000000,0.000000}%
\pgfsetfillcolor{currentfill}%
\pgfsetlinewidth{0.602250pt}%
\definecolor{currentstroke}{rgb}{0.000000,0.000000,0.000000}%
\pgfsetstrokecolor{currentstroke}%
\pgfsetdash{}{0pt}%
\pgfsys@defobject{currentmarker}{\pgfqpoint{-0.027778in}{0.000000in}}{\pgfqpoint{-0.000000in}{0.000000in}}{%
\pgfpathmoveto{\pgfqpoint{-0.000000in}{0.000000in}}%
\pgfpathlineto{\pgfqpoint{-0.027778in}{0.000000in}}%
\pgfusepath{stroke,fill}%
}%
\begin{pgfscope}%
\pgfsys@transformshift{0.708220in}{2.835866in}%
\pgfsys@useobject{currentmarker}{}%
\end{pgfscope}%
\end{pgfscope}%
\begin{pgfscope}%
\pgfsetbuttcap%
\pgfsetroundjoin%
\definecolor{currentfill}{rgb}{0.000000,0.000000,0.000000}%
\pgfsetfillcolor{currentfill}%
\pgfsetlinewidth{0.602250pt}%
\definecolor{currentstroke}{rgb}{0.000000,0.000000,0.000000}%
\pgfsetstrokecolor{currentstroke}%
\pgfsetdash{}{0pt}%
\pgfsys@defobject{currentmarker}{\pgfqpoint{-0.027778in}{0.000000in}}{\pgfqpoint{-0.000000in}{0.000000in}}{%
\pgfpathmoveto{\pgfqpoint{-0.000000in}{0.000000in}}%
\pgfpathlineto{\pgfqpoint{-0.027778in}{0.000000in}}%
\pgfusepath{stroke,fill}%
}%
\begin{pgfscope}%
\pgfsys@transformshift{0.708220in}{2.853805in}%
\pgfsys@useobject{currentmarker}{}%
\end{pgfscope}%
\end{pgfscope}%
\begin{pgfscope}%
\pgfsetbuttcap%
\pgfsetroundjoin%
\definecolor{currentfill}{rgb}{0.000000,0.000000,0.000000}%
\pgfsetfillcolor{currentfill}%
\pgfsetlinewidth{0.602250pt}%
\definecolor{currentstroke}{rgb}{0.000000,0.000000,0.000000}%
\pgfsetstrokecolor{currentstroke}%
\pgfsetdash{}{0pt}%
\pgfsys@defobject{currentmarker}{\pgfqpoint{-0.027778in}{0.000000in}}{\pgfqpoint{-0.000000in}{0.000000in}}{%
\pgfpathmoveto{\pgfqpoint{-0.000000in}{0.000000in}}%
\pgfpathlineto{\pgfqpoint{-0.027778in}{0.000000in}}%
\pgfusepath{stroke,fill}%
}%
\begin{pgfscope}%
\pgfsys@transformshift{0.708220in}{2.975423in}%
\pgfsys@useobject{currentmarker}{}%
\end{pgfscope}%
\end{pgfscope}%
\begin{pgfscope}%
\pgfsetbuttcap%
\pgfsetroundjoin%
\definecolor{currentfill}{rgb}{0.000000,0.000000,0.000000}%
\pgfsetfillcolor{currentfill}%
\pgfsetlinewidth{0.602250pt}%
\definecolor{currentstroke}{rgb}{0.000000,0.000000,0.000000}%
\pgfsetstrokecolor{currentstroke}%
\pgfsetdash{}{0pt}%
\pgfsys@defobject{currentmarker}{\pgfqpoint{-0.027778in}{0.000000in}}{\pgfqpoint{-0.000000in}{0.000000in}}{%
\pgfpathmoveto{\pgfqpoint{-0.000000in}{0.000000in}}%
\pgfpathlineto{\pgfqpoint{-0.027778in}{0.000000in}}%
\pgfusepath{stroke,fill}%
}%
\begin{pgfscope}%
\pgfsys@transformshift{0.708220in}{3.037177in}%
\pgfsys@useobject{currentmarker}{}%
\end{pgfscope}%
\end{pgfscope}%
\begin{pgfscope}%
\pgfsetbuttcap%
\pgfsetroundjoin%
\definecolor{currentfill}{rgb}{0.000000,0.000000,0.000000}%
\pgfsetfillcolor{currentfill}%
\pgfsetlinewidth{0.602250pt}%
\definecolor{currentstroke}{rgb}{0.000000,0.000000,0.000000}%
\pgfsetstrokecolor{currentstroke}%
\pgfsetdash{}{0pt}%
\pgfsys@defobject{currentmarker}{\pgfqpoint{-0.027778in}{0.000000in}}{\pgfqpoint{-0.000000in}{0.000000in}}{%
\pgfpathmoveto{\pgfqpoint{-0.000000in}{0.000000in}}%
\pgfpathlineto{\pgfqpoint{-0.027778in}{0.000000in}}%
\pgfusepath{stroke,fill}%
}%
\begin{pgfscope}%
\pgfsys@transformshift{0.708220in}{3.080993in}%
\pgfsys@useobject{currentmarker}{}%
\end{pgfscope}%
\end{pgfscope}%
\begin{pgfscope}%
\pgfsetbuttcap%
\pgfsetroundjoin%
\definecolor{currentfill}{rgb}{0.000000,0.000000,0.000000}%
\pgfsetfillcolor{currentfill}%
\pgfsetlinewidth{0.602250pt}%
\definecolor{currentstroke}{rgb}{0.000000,0.000000,0.000000}%
\pgfsetstrokecolor{currentstroke}%
\pgfsetdash{}{0pt}%
\pgfsys@defobject{currentmarker}{\pgfqpoint{-0.027778in}{0.000000in}}{\pgfqpoint{-0.000000in}{0.000000in}}{%
\pgfpathmoveto{\pgfqpoint{-0.000000in}{0.000000in}}%
\pgfpathlineto{\pgfqpoint{-0.027778in}{0.000000in}}%
\pgfusepath{stroke,fill}%
}%
\begin{pgfscope}%
\pgfsys@transformshift{0.708220in}{3.114979in}%
\pgfsys@useobject{currentmarker}{}%
\end{pgfscope}%
\end{pgfscope}%
\begin{pgfscope}%
\pgfsetbuttcap%
\pgfsetroundjoin%
\definecolor{currentfill}{rgb}{0.000000,0.000000,0.000000}%
\pgfsetfillcolor{currentfill}%
\pgfsetlinewidth{0.602250pt}%
\definecolor{currentstroke}{rgb}{0.000000,0.000000,0.000000}%
\pgfsetstrokecolor{currentstroke}%
\pgfsetdash{}{0pt}%
\pgfsys@defobject{currentmarker}{\pgfqpoint{-0.027778in}{0.000000in}}{\pgfqpoint{-0.000000in}{0.000000in}}{%
\pgfpathmoveto{\pgfqpoint{-0.000000in}{0.000000in}}%
\pgfpathlineto{\pgfqpoint{-0.027778in}{0.000000in}}%
\pgfusepath{stroke,fill}%
}%
\begin{pgfscope}%
\pgfsys@transformshift{0.708220in}{3.142747in}%
\pgfsys@useobject{currentmarker}{}%
\end{pgfscope}%
\end{pgfscope}%
\begin{pgfscope}%
\pgfsetbuttcap%
\pgfsetroundjoin%
\definecolor{currentfill}{rgb}{0.000000,0.000000,0.000000}%
\pgfsetfillcolor{currentfill}%
\pgfsetlinewidth{0.602250pt}%
\definecolor{currentstroke}{rgb}{0.000000,0.000000,0.000000}%
\pgfsetstrokecolor{currentstroke}%
\pgfsetdash{}{0pt}%
\pgfsys@defobject{currentmarker}{\pgfqpoint{-0.027778in}{0.000000in}}{\pgfqpoint{-0.000000in}{0.000000in}}{%
\pgfpathmoveto{\pgfqpoint{-0.000000in}{0.000000in}}%
\pgfpathlineto{\pgfqpoint{-0.027778in}{0.000000in}}%
\pgfusepath{stroke,fill}%
}%
\begin{pgfscope}%
\pgfsys@transformshift{0.708220in}{3.166225in}%
\pgfsys@useobject{currentmarker}{}%
\end{pgfscope}%
\end{pgfscope}%
\begin{pgfscope}%
\pgfsetbuttcap%
\pgfsetroundjoin%
\definecolor{currentfill}{rgb}{0.000000,0.000000,0.000000}%
\pgfsetfillcolor{currentfill}%
\pgfsetlinewidth{0.602250pt}%
\definecolor{currentstroke}{rgb}{0.000000,0.000000,0.000000}%
\pgfsetstrokecolor{currentstroke}%
\pgfsetdash{}{0pt}%
\pgfsys@defobject{currentmarker}{\pgfqpoint{-0.027778in}{0.000000in}}{\pgfqpoint{-0.000000in}{0.000000in}}{%
\pgfpathmoveto{\pgfqpoint{-0.000000in}{0.000000in}}%
\pgfpathlineto{\pgfqpoint{-0.027778in}{0.000000in}}%
\pgfusepath{stroke,fill}%
}%
\begin{pgfscope}%
\pgfsys@transformshift{0.708220in}{3.186563in}%
\pgfsys@useobject{currentmarker}{}%
\end{pgfscope}%
\end{pgfscope}%
\begin{pgfscope}%
\pgfsetbuttcap%
\pgfsetroundjoin%
\definecolor{currentfill}{rgb}{0.000000,0.000000,0.000000}%
\pgfsetfillcolor{currentfill}%
\pgfsetlinewidth{0.602250pt}%
\definecolor{currentstroke}{rgb}{0.000000,0.000000,0.000000}%
\pgfsetstrokecolor{currentstroke}%
\pgfsetdash{}{0pt}%
\pgfsys@defobject{currentmarker}{\pgfqpoint{-0.027778in}{0.000000in}}{\pgfqpoint{-0.000000in}{0.000000in}}{%
\pgfpathmoveto{\pgfqpoint{-0.000000in}{0.000000in}}%
\pgfpathlineto{\pgfqpoint{-0.027778in}{0.000000in}}%
\pgfusepath{stroke,fill}%
}%
\begin{pgfscope}%
\pgfsys@transformshift{0.708220in}{3.204502in}%
\pgfsys@useobject{currentmarker}{}%
\end{pgfscope}%
\end{pgfscope}%
\begin{pgfscope}%
\pgfsetbuttcap%
\pgfsetroundjoin%
\definecolor{currentfill}{rgb}{0.000000,0.000000,0.000000}%
\pgfsetfillcolor{currentfill}%
\pgfsetlinewidth{0.602250pt}%
\definecolor{currentstroke}{rgb}{0.000000,0.000000,0.000000}%
\pgfsetstrokecolor{currentstroke}%
\pgfsetdash{}{0pt}%
\pgfsys@defobject{currentmarker}{\pgfqpoint{-0.027778in}{0.000000in}}{\pgfqpoint{-0.000000in}{0.000000in}}{%
\pgfpathmoveto{\pgfqpoint{-0.000000in}{0.000000in}}%
\pgfpathlineto{\pgfqpoint{-0.027778in}{0.000000in}}%
\pgfusepath{stroke,fill}%
}%
\begin{pgfscope}%
\pgfsys@transformshift{0.708220in}{3.326119in}%
\pgfsys@useobject{currentmarker}{}%
\end{pgfscope}%
\end{pgfscope}%
\begin{pgfscope}%
\pgfsetbuttcap%
\pgfsetroundjoin%
\definecolor{currentfill}{rgb}{0.000000,0.000000,0.000000}%
\pgfsetfillcolor{currentfill}%
\pgfsetlinewidth{0.602250pt}%
\definecolor{currentstroke}{rgb}{0.000000,0.000000,0.000000}%
\pgfsetstrokecolor{currentstroke}%
\pgfsetdash{}{0pt}%
\pgfsys@defobject{currentmarker}{\pgfqpoint{-0.027778in}{0.000000in}}{\pgfqpoint{-0.000000in}{0.000000in}}{%
\pgfpathmoveto{\pgfqpoint{-0.000000in}{0.000000in}}%
\pgfpathlineto{\pgfqpoint{-0.027778in}{0.000000in}}%
\pgfusepath{stroke,fill}%
}%
\begin{pgfscope}%
\pgfsys@transformshift{0.708220in}{3.387874in}%
\pgfsys@useobject{currentmarker}{}%
\end{pgfscope}%
\end{pgfscope}%
\begin{pgfscope}%
\pgfsetbuttcap%
\pgfsetroundjoin%
\definecolor{currentfill}{rgb}{0.000000,0.000000,0.000000}%
\pgfsetfillcolor{currentfill}%
\pgfsetlinewidth{0.602250pt}%
\definecolor{currentstroke}{rgb}{0.000000,0.000000,0.000000}%
\pgfsetstrokecolor{currentstroke}%
\pgfsetdash{}{0pt}%
\pgfsys@defobject{currentmarker}{\pgfqpoint{-0.027778in}{0.000000in}}{\pgfqpoint{-0.000000in}{0.000000in}}{%
\pgfpathmoveto{\pgfqpoint{-0.000000in}{0.000000in}}%
\pgfpathlineto{\pgfqpoint{-0.027778in}{0.000000in}}%
\pgfusepath{stroke,fill}%
}%
\begin{pgfscope}%
\pgfsys@transformshift{0.708220in}{3.431689in}%
\pgfsys@useobject{currentmarker}{}%
\end{pgfscope}%
\end{pgfscope}%
\begin{pgfscope}%
\pgfsetbuttcap%
\pgfsetroundjoin%
\definecolor{currentfill}{rgb}{0.000000,0.000000,0.000000}%
\pgfsetfillcolor{currentfill}%
\pgfsetlinewidth{0.602250pt}%
\definecolor{currentstroke}{rgb}{0.000000,0.000000,0.000000}%
\pgfsetstrokecolor{currentstroke}%
\pgfsetdash{}{0pt}%
\pgfsys@defobject{currentmarker}{\pgfqpoint{-0.027778in}{0.000000in}}{\pgfqpoint{-0.000000in}{0.000000in}}{%
\pgfpathmoveto{\pgfqpoint{-0.000000in}{0.000000in}}%
\pgfpathlineto{\pgfqpoint{-0.027778in}{0.000000in}}%
\pgfusepath{stroke,fill}%
}%
\begin{pgfscope}%
\pgfsys@transformshift{0.708220in}{3.465675in}%
\pgfsys@useobject{currentmarker}{}%
\end{pgfscope}%
\end{pgfscope}%
\begin{pgfscope}%
\pgfsetbuttcap%
\pgfsetroundjoin%
\definecolor{currentfill}{rgb}{0.000000,0.000000,0.000000}%
\pgfsetfillcolor{currentfill}%
\pgfsetlinewidth{0.602250pt}%
\definecolor{currentstroke}{rgb}{0.000000,0.000000,0.000000}%
\pgfsetstrokecolor{currentstroke}%
\pgfsetdash{}{0pt}%
\pgfsys@defobject{currentmarker}{\pgfqpoint{-0.027778in}{0.000000in}}{\pgfqpoint{-0.000000in}{0.000000in}}{%
\pgfpathmoveto{\pgfqpoint{-0.000000in}{0.000000in}}%
\pgfpathlineto{\pgfqpoint{-0.027778in}{0.000000in}}%
\pgfusepath{stroke,fill}%
}%
\begin{pgfscope}%
\pgfsys@transformshift{0.708220in}{3.493444in}%
\pgfsys@useobject{currentmarker}{}%
\end{pgfscope}%
\end{pgfscope}%
\begin{pgfscope}%
\pgfsetbuttcap%
\pgfsetroundjoin%
\definecolor{currentfill}{rgb}{0.000000,0.000000,0.000000}%
\pgfsetfillcolor{currentfill}%
\pgfsetlinewidth{0.602250pt}%
\definecolor{currentstroke}{rgb}{0.000000,0.000000,0.000000}%
\pgfsetstrokecolor{currentstroke}%
\pgfsetdash{}{0pt}%
\pgfsys@defobject{currentmarker}{\pgfqpoint{-0.027778in}{0.000000in}}{\pgfqpoint{-0.000000in}{0.000000in}}{%
\pgfpathmoveto{\pgfqpoint{-0.000000in}{0.000000in}}%
\pgfpathlineto{\pgfqpoint{-0.027778in}{0.000000in}}%
\pgfusepath{stroke,fill}%
}%
\begin{pgfscope}%
\pgfsys@transformshift{0.708220in}{3.516922in}%
\pgfsys@useobject{currentmarker}{}%
\end{pgfscope}%
\end{pgfscope}%
\begin{pgfscope}%
\pgfsetbuttcap%
\pgfsetroundjoin%
\definecolor{currentfill}{rgb}{0.000000,0.000000,0.000000}%
\pgfsetfillcolor{currentfill}%
\pgfsetlinewidth{0.602250pt}%
\definecolor{currentstroke}{rgb}{0.000000,0.000000,0.000000}%
\pgfsetstrokecolor{currentstroke}%
\pgfsetdash{}{0pt}%
\pgfsys@defobject{currentmarker}{\pgfqpoint{-0.027778in}{0.000000in}}{\pgfqpoint{-0.000000in}{0.000000in}}{%
\pgfpathmoveto{\pgfqpoint{-0.000000in}{0.000000in}}%
\pgfpathlineto{\pgfqpoint{-0.027778in}{0.000000in}}%
\pgfusepath{stroke,fill}%
}%
\begin{pgfscope}%
\pgfsys@transformshift{0.708220in}{3.537259in}%
\pgfsys@useobject{currentmarker}{}%
\end{pgfscope}%
\end{pgfscope}%
\begin{pgfscope}%
\pgfsetbuttcap%
\pgfsetroundjoin%
\definecolor{currentfill}{rgb}{0.000000,0.000000,0.000000}%
\pgfsetfillcolor{currentfill}%
\pgfsetlinewidth{0.602250pt}%
\definecolor{currentstroke}{rgb}{0.000000,0.000000,0.000000}%
\pgfsetstrokecolor{currentstroke}%
\pgfsetdash{}{0pt}%
\pgfsys@defobject{currentmarker}{\pgfqpoint{-0.027778in}{0.000000in}}{\pgfqpoint{-0.000000in}{0.000000in}}{%
\pgfpathmoveto{\pgfqpoint{-0.000000in}{0.000000in}}%
\pgfpathlineto{\pgfqpoint{-0.027778in}{0.000000in}}%
\pgfusepath{stroke,fill}%
}%
\begin{pgfscope}%
\pgfsys@transformshift{0.708220in}{3.555198in}%
\pgfsys@useobject{currentmarker}{}%
\end{pgfscope}%
\end{pgfscope}%
\begin{pgfscope}%
\pgfsetbuttcap%
\pgfsetroundjoin%
\definecolor{currentfill}{rgb}{0.000000,0.000000,0.000000}%
\pgfsetfillcolor{currentfill}%
\pgfsetlinewidth{0.602250pt}%
\definecolor{currentstroke}{rgb}{0.000000,0.000000,0.000000}%
\pgfsetstrokecolor{currentstroke}%
\pgfsetdash{}{0pt}%
\pgfsys@defobject{currentmarker}{\pgfqpoint{-0.027778in}{0.000000in}}{\pgfqpoint{-0.000000in}{0.000000in}}{%
\pgfpathmoveto{\pgfqpoint{-0.000000in}{0.000000in}}%
\pgfpathlineto{\pgfqpoint{-0.027778in}{0.000000in}}%
\pgfusepath{stroke,fill}%
}%
\begin{pgfscope}%
\pgfsys@transformshift{0.708220in}{3.676816in}%
\pgfsys@useobject{currentmarker}{}%
\end{pgfscope}%
\end{pgfscope}%
\begin{pgfscope}%
\pgfsetbuttcap%
\pgfsetroundjoin%
\definecolor{currentfill}{rgb}{0.000000,0.000000,0.000000}%
\pgfsetfillcolor{currentfill}%
\pgfsetlinewidth{0.602250pt}%
\definecolor{currentstroke}{rgb}{0.000000,0.000000,0.000000}%
\pgfsetstrokecolor{currentstroke}%
\pgfsetdash{}{0pt}%
\pgfsys@defobject{currentmarker}{\pgfqpoint{-0.027778in}{0.000000in}}{\pgfqpoint{-0.000000in}{0.000000in}}{%
\pgfpathmoveto{\pgfqpoint{-0.000000in}{0.000000in}}%
\pgfpathlineto{\pgfqpoint{-0.027778in}{0.000000in}}%
\pgfusepath{stroke,fill}%
}%
\begin{pgfscope}%
\pgfsys@transformshift{0.708220in}{3.738570in}%
\pgfsys@useobject{currentmarker}{}%
\end{pgfscope}%
\end{pgfscope}%
\begin{pgfscope}%
\pgfsetbuttcap%
\pgfsetroundjoin%
\definecolor{currentfill}{rgb}{0.000000,0.000000,0.000000}%
\pgfsetfillcolor{currentfill}%
\pgfsetlinewidth{0.602250pt}%
\definecolor{currentstroke}{rgb}{0.000000,0.000000,0.000000}%
\pgfsetstrokecolor{currentstroke}%
\pgfsetdash{}{0pt}%
\pgfsys@defobject{currentmarker}{\pgfqpoint{-0.027778in}{0.000000in}}{\pgfqpoint{-0.000000in}{0.000000in}}{%
\pgfpathmoveto{\pgfqpoint{-0.000000in}{0.000000in}}%
\pgfpathlineto{\pgfqpoint{-0.027778in}{0.000000in}}%
\pgfusepath{stroke,fill}%
}%
\begin{pgfscope}%
\pgfsys@transformshift{0.708220in}{3.782386in}%
\pgfsys@useobject{currentmarker}{}%
\end{pgfscope}%
\end{pgfscope}%
\begin{pgfscope}%
\pgfsetbuttcap%
\pgfsetroundjoin%
\definecolor{currentfill}{rgb}{0.000000,0.000000,0.000000}%
\pgfsetfillcolor{currentfill}%
\pgfsetlinewidth{0.602250pt}%
\definecolor{currentstroke}{rgb}{0.000000,0.000000,0.000000}%
\pgfsetstrokecolor{currentstroke}%
\pgfsetdash{}{0pt}%
\pgfsys@defobject{currentmarker}{\pgfqpoint{-0.027778in}{0.000000in}}{\pgfqpoint{-0.000000in}{0.000000in}}{%
\pgfpathmoveto{\pgfqpoint{-0.000000in}{0.000000in}}%
\pgfpathlineto{\pgfqpoint{-0.027778in}{0.000000in}}%
\pgfusepath{stroke,fill}%
}%
\begin{pgfscope}%
\pgfsys@transformshift{0.708220in}{3.816372in}%
\pgfsys@useobject{currentmarker}{}%
\end{pgfscope}%
\end{pgfscope}%
\begin{pgfscope}%
\pgfsetbuttcap%
\pgfsetroundjoin%
\definecolor{currentfill}{rgb}{0.000000,0.000000,0.000000}%
\pgfsetfillcolor{currentfill}%
\pgfsetlinewidth{0.602250pt}%
\definecolor{currentstroke}{rgb}{0.000000,0.000000,0.000000}%
\pgfsetstrokecolor{currentstroke}%
\pgfsetdash{}{0pt}%
\pgfsys@defobject{currentmarker}{\pgfqpoint{-0.027778in}{0.000000in}}{\pgfqpoint{-0.000000in}{0.000000in}}{%
\pgfpathmoveto{\pgfqpoint{-0.000000in}{0.000000in}}%
\pgfpathlineto{\pgfqpoint{-0.027778in}{0.000000in}}%
\pgfusepath{stroke,fill}%
}%
\begin{pgfscope}%
\pgfsys@transformshift{0.708220in}{3.844140in}%
\pgfsys@useobject{currentmarker}{}%
\end{pgfscope}%
\end{pgfscope}%
\begin{pgfscope}%
\pgfsetbuttcap%
\pgfsetroundjoin%
\definecolor{currentfill}{rgb}{0.000000,0.000000,0.000000}%
\pgfsetfillcolor{currentfill}%
\pgfsetlinewidth{0.602250pt}%
\definecolor{currentstroke}{rgb}{0.000000,0.000000,0.000000}%
\pgfsetstrokecolor{currentstroke}%
\pgfsetdash{}{0pt}%
\pgfsys@defobject{currentmarker}{\pgfqpoint{-0.027778in}{0.000000in}}{\pgfqpoint{-0.000000in}{0.000000in}}{%
\pgfpathmoveto{\pgfqpoint{-0.000000in}{0.000000in}}%
\pgfpathlineto{\pgfqpoint{-0.027778in}{0.000000in}}%
\pgfusepath{stroke,fill}%
}%
\begin{pgfscope}%
\pgfsys@transformshift{0.708220in}{3.867618in}%
\pgfsys@useobject{currentmarker}{}%
\end{pgfscope}%
\end{pgfscope}%
\begin{pgfscope}%
\pgfsetbuttcap%
\pgfsetroundjoin%
\definecolor{currentfill}{rgb}{0.000000,0.000000,0.000000}%
\pgfsetfillcolor{currentfill}%
\pgfsetlinewidth{0.602250pt}%
\definecolor{currentstroke}{rgb}{0.000000,0.000000,0.000000}%
\pgfsetstrokecolor{currentstroke}%
\pgfsetdash{}{0pt}%
\pgfsys@defobject{currentmarker}{\pgfqpoint{-0.027778in}{0.000000in}}{\pgfqpoint{-0.000000in}{0.000000in}}{%
\pgfpathmoveto{\pgfqpoint{-0.000000in}{0.000000in}}%
\pgfpathlineto{\pgfqpoint{-0.027778in}{0.000000in}}%
\pgfusepath{stroke,fill}%
}%
\begin{pgfscope}%
\pgfsys@transformshift{0.708220in}{3.887956in}%
\pgfsys@useobject{currentmarker}{}%
\end{pgfscope}%
\end{pgfscope}%
\begin{pgfscope}%
\pgfsetbuttcap%
\pgfsetroundjoin%
\definecolor{currentfill}{rgb}{0.000000,0.000000,0.000000}%
\pgfsetfillcolor{currentfill}%
\pgfsetlinewidth{0.602250pt}%
\definecolor{currentstroke}{rgb}{0.000000,0.000000,0.000000}%
\pgfsetstrokecolor{currentstroke}%
\pgfsetdash{}{0pt}%
\pgfsys@defobject{currentmarker}{\pgfqpoint{-0.027778in}{0.000000in}}{\pgfqpoint{-0.000000in}{0.000000in}}{%
\pgfpathmoveto{\pgfqpoint{-0.000000in}{0.000000in}}%
\pgfpathlineto{\pgfqpoint{-0.027778in}{0.000000in}}%
\pgfusepath{stroke,fill}%
}%
\begin{pgfscope}%
\pgfsys@transformshift{0.708220in}{3.905895in}%
\pgfsys@useobject{currentmarker}{}%
\end{pgfscope}%
\end{pgfscope}%
\begin{pgfscope}%
\definecolor{textcolor}{rgb}{0.000000,0.000000,0.000000}%
\pgfsetstrokecolor{textcolor}%
\pgfsetfillcolor{textcolor}%
\pgftext[x=0.288855in,y=3.220549in,,bottom,rotate=90.000000]{\color{textcolor}\rmfamily\fontsize{10.000000}{12.000000}\selectfont Longest solving time (s)}%
\end{pgfscope}%
\begin{pgfscope}%
\pgfpathrectangle{\pgfqpoint{0.708220in}{2.519156in}}{\pgfqpoint{5.141780in}{1.402786in}}%
\pgfusepath{clip}%
\pgfsetbuttcap%
\pgfsetroundjoin%
\pgfsetlinewidth{2.007500pt}%
\definecolor{currentstroke}{rgb}{1.000000,0.843137,0.000000}%
\pgfsetstrokecolor{currentstroke}%
\pgfsetdash{{7.400000pt}{3.200000pt}}{0.000000pt}%
\pgfpathmoveto{\pgfqpoint{0.708220in}{2.744893in}}%
\pgfpathlineto{\pgfqpoint{0.712933in}{2.746168in}}%
\pgfpathlineto{\pgfqpoint{0.783626in}{2.747478in}}%
\pgfpathlineto{\pgfqpoint{0.826043in}{2.748628in}}%
\pgfpathlineto{\pgfqpoint{0.877884in}{2.750430in}}%
\pgfpathlineto{\pgfqpoint{0.882597in}{2.752913in}}%
\pgfpathlineto{\pgfqpoint{0.887310in}{2.758652in}}%
\pgfpathlineto{\pgfqpoint{0.892023in}{2.760981in}}%
\pgfpathlineto{\pgfqpoint{0.934439in}{2.762072in}}%
\pgfpathlineto{\pgfqpoint{1.023985in}{2.765622in}}%
\pgfpathlineto{\pgfqpoint{1.038123in}{2.770291in}}%
\pgfpathlineto{\pgfqpoint{1.042836in}{2.770418in}}%
\pgfpathlineto{\pgfqpoint{1.047549in}{2.787417in}}%
\pgfpathlineto{\pgfqpoint{1.127668in}{2.790471in}}%
\pgfpathlineto{\pgfqpoint{1.207788in}{2.798015in}}%
\pgfpathlineto{\pgfqpoint{1.212501in}{2.800565in}}%
\pgfpathlineto{\pgfqpoint{1.217214in}{2.810008in}}%
\pgfpathlineto{\pgfqpoint{1.221927in}{2.814623in}}%
\pgfpathlineto{\pgfqpoint{1.236065in}{2.816190in}}%
\pgfpathlineto{\pgfqpoint{1.240778in}{2.817620in}}%
\pgfpathlineto{\pgfqpoint{1.245491in}{2.820400in}}%
\pgfpathlineto{\pgfqpoint{1.283194in}{2.826859in}}%
\pgfpathlineto{\pgfqpoint{1.386878in}{2.832008in}}%
\pgfpathlineto{\pgfqpoint{1.405730in}{2.835870in}}%
\pgfpathlineto{\pgfqpoint{1.438720in}{2.838138in}}%
\pgfpathlineto{\pgfqpoint{1.443433in}{2.839645in}}%
\pgfpathlineto{\pgfqpoint{1.452859in}{2.857796in}}%
\pgfpathlineto{\pgfqpoint{1.457572in}{2.858965in}}%
\pgfpathlineto{\pgfqpoint{1.466998in}{2.862806in}}%
\pgfpathlineto{\pgfqpoint{1.471711in}{2.868488in}}%
\pgfpathlineto{\pgfqpoint{1.485849in}{2.869178in}}%
\pgfpathlineto{\pgfqpoint{1.490562in}{2.873739in}}%
\pgfpathlineto{\pgfqpoint{1.499988in}{2.899054in}}%
\pgfpathlineto{\pgfqpoint{1.509414in}{2.904235in}}%
\pgfpathlineto{\pgfqpoint{1.514127in}{2.916648in}}%
\pgfpathlineto{\pgfqpoint{1.523553in}{2.921804in}}%
\pgfpathlineto{\pgfqpoint{1.547117in}{2.926191in}}%
\pgfpathlineto{\pgfqpoint{1.551830in}{2.926855in}}%
\pgfpathlineto{\pgfqpoint{1.565969in}{2.935551in}}%
\pgfpathlineto{\pgfqpoint{1.570682in}{2.940693in}}%
\pgfpathlineto{\pgfqpoint{1.575395in}{2.941014in}}%
\pgfpathlineto{\pgfqpoint{1.580107in}{2.947385in}}%
\pgfpathlineto{\pgfqpoint{1.584820in}{2.951156in}}%
\pgfpathlineto{\pgfqpoint{1.589533in}{2.952553in}}%
\pgfpathlineto{\pgfqpoint{1.594246in}{2.958542in}}%
\pgfpathlineto{\pgfqpoint{1.598959in}{2.960962in}}%
\pgfpathlineto{\pgfqpoint{1.617811in}{2.965174in}}%
\pgfpathlineto{\pgfqpoint{1.627236in}{2.971130in}}%
\pgfpathlineto{\pgfqpoint{1.631949in}{2.983117in}}%
\pgfpathlineto{\pgfqpoint{1.636662in}{2.986363in}}%
\pgfpathlineto{\pgfqpoint{1.641375in}{2.987213in}}%
\pgfpathlineto{\pgfqpoint{1.646088in}{2.993798in}}%
\pgfpathlineto{\pgfqpoint{1.660227in}{2.994863in}}%
\pgfpathlineto{\pgfqpoint{1.664940in}{2.996024in}}%
\pgfpathlineto{\pgfqpoint{1.669653in}{2.999173in}}%
\pgfpathlineto{\pgfqpoint{1.679078in}{3.013466in}}%
\pgfpathlineto{\pgfqpoint{1.702643in}{3.020399in}}%
\pgfpathlineto{\pgfqpoint{1.707356in}{3.027490in}}%
\pgfpathlineto{\pgfqpoint{1.712069in}{3.028671in}}%
\pgfpathlineto{\pgfqpoint{1.716782in}{3.031137in}}%
\pgfpathlineto{\pgfqpoint{1.726207in}{3.033171in}}%
\pgfpathlineto{\pgfqpoint{1.740346in}{3.034362in}}%
\pgfpathlineto{\pgfqpoint{1.745059in}{3.035031in}}%
\pgfpathlineto{\pgfqpoint{1.749772in}{3.047618in}}%
\pgfpathlineto{\pgfqpoint{1.759198in}{3.048559in}}%
\pgfpathlineto{\pgfqpoint{1.768624in}{3.048763in}}%
\pgfpathlineto{\pgfqpoint{1.778049in}{3.050660in}}%
\pgfpathlineto{\pgfqpoint{1.792188in}{3.052130in}}%
\pgfpathlineto{\pgfqpoint{1.801614in}{3.053706in}}%
\pgfpathlineto{\pgfqpoint{1.829891in}{3.063705in}}%
\pgfpathlineto{\pgfqpoint{1.834604in}{3.064642in}}%
\pgfpathlineto{\pgfqpoint{1.839317in}{3.077851in}}%
\pgfpathlineto{\pgfqpoint{1.867595in}{3.080439in}}%
\pgfpathlineto{\pgfqpoint{1.877020in}{3.082499in}}%
\pgfpathlineto{\pgfqpoint{1.881733in}{3.084805in}}%
\pgfpathlineto{\pgfqpoint{1.895872in}{3.087206in}}%
\pgfpathlineto{\pgfqpoint{1.900585in}{3.089297in}}%
\pgfpathlineto{\pgfqpoint{1.924150in}{3.092135in}}%
\pgfpathlineto{\pgfqpoint{1.928862in}{3.094138in}}%
\pgfpathlineto{\pgfqpoint{1.933575in}{3.094774in}}%
\pgfpathlineto{\pgfqpoint{1.938288in}{3.104218in}}%
\pgfpathlineto{\pgfqpoint{1.943001in}{3.106226in}}%
\pgfpathlineto{\pgfqpoint{1.947714in}{3.112438in}}%
\pgfpathlineto{\pgfqpoint{1.957140in}{3.112858in}}%
\pgfpathlineto{\pgfqpoint{1.966566in}{3.114252in}}%
\pgfpathlineto{\pgfqpoint{1.971279in}{3.114551in}}%
\pgfpathlineto{\pgfqpoint{1.975991in}{3.117366in}}%
\pgfpathlineto{\pgfqpoint{1.990130in}{3.119966in}}%
\pgfpathlineto{\pgfqpoint{1.994843in}{3.120380in}}%
\pgfpathlineto{\pgfqpoint{1.999556in}{3.126002in}}%
\pgfpathlineto{\pgfqpoint{2.023121in}{3.128078in}}%
\pgfpathlineto{\pgfqpoint{2.032546in}{3.128718in}}%
\pgfpathlineto{\pgfqpoint{2.046685in}{3.129538in}}%
\pgfpathlineto{\pgfqpoint{2.051398in}{3.130022in}}%
\pgfpathlineto{\pgfqpoint{2.056111in}{3.143684in}}%
\pgfpathlineto{\pgfqpoint{2.070250in}{3.145379in}}%
\pgfpathlineto{\pgfqpoint{2.098527in}{3.147273in}}%
\pgfpathlineto{\pgfqpoint{2.117379in}{3.150657in}}%
\pgfpathlineto{\pgfqpoint{2.122092in}{3.151291in}}%
\pgfpathlineto{\pgfqpoint{2.136230in}{3.156374in}}%
\pgfpathlineto{\pgfqpoint{2.140943in}{3.156976in}}%
\pgfpathlineto{\pgfqpoint{2.150369in}{3.160767in}}%
\pgfpathlineto{\pgfqpoint{2.155082in}{3.170042in}}%
\pgfpathlineto{\pgfqpoint{2.183359in}{3.175340in}}%
\pgfpathlineto{\pgfqpoint{2.188072in}{3.175595in}}%
\pgfpathlineto{\pgfqpoint{2.192785in}{3.178132in}}%
\pgfpathlineto{\pgfqpoint{2.206924in}{3.180300in}}%
\pgfpathlineto{\pgfqpoint{2.211637in}{3.184140in}}%
\pgfpathlineto{\pgfqpoint{2.239914in}{3.186775in}}%
\pgfpathlineto{\pgfqpoint{2.244627in}{3.188803in}}%
\pgfpathlineto{\pgfqpoint{2.263479in}{3.190690in}}%
\pgfpathlineto{\pgfqpoint{2.272905in}{3.192862in}}%
\pgfpathlineto{\pgfqpoint{2.277617in}{3.201743in}}%
\pgfpathlineto{\pgfqpoint{2.291756in}{3.203801in}}%
\pgfpathlineto{\pgfqpoint{2.343598in}{3.208889in}}%
\pgfpathlineto{\pgfqpoint{2.348311in}{3.213562in}}%
\pgfpathlineto{\pgfqpoint{2.367163in}{3.222322in}}%
\pgfpathlineto{\pgfqpoint{2.386014in}{3.223938in}}%
\pgfpathlineto{\pgfqpoint{2.395440in}{3.226009in}}%
\pgfpathlineto{\pgfqpoint{2.400153in}{3.236693in}}%
\pgfpathlineto{\pgfqpoint{2.404866in}{3.238565in}}%
\pgfpathlineto{\pgfqpoint{2.419005in}{3.238876in}}%
\pgfpathlineto{\pgfqpoint{2.433143in}{3.242349in}}%
\pgfpathlineto{\pgfqpoint{2.451995in}{3.246291in}}%
\pgfpathlineto{\pgfqpoint{2.456708in}{3.250352in}}%
\pgfpathlineto{\pgfqpoint{2.480272in}{3.251581in}}%
\pgfpathlineto{\pgfqpoint{2.513263in}{3.253070in}}%
\pgfpathlineto{\pgfqpoint{2.536827in}{3.254465in}}%
\pgfpathlineto{\pgfqpoint{2.546253in}{3.268117in}}%
\pgfpathlineto{\pgfqpoint{2.560392in}{3.269357in}}%
\pgfpathlineto{\pgfqpoint{2.565105in}{3.272136in}}%
\pgfpathlineto{\pgfqpoint{2.579243in}{3.274752in}}%
\pgfpathlineto{\pgfqpoint{2.583956in}{3.277687in}}%
\pgfpathlineto{\pgfqpoint{2.593382in}{3.278920in}}%
\pgfpathlineto{\pgfqpoint{2.626372in}{3.283692in}}%
\pgfpathlineto{\pgfqpoint{2.631085in}{3.297938in}}%
\pgfpathlineto{\pgfqpoint{2.635798in}{3.305261in}}%
\pgfpathlineto{\pgfqpoint{2.640511in}{3.307958in}}%
\pgfpathlineto{\pgfqpoint{2.659363in}{3.310469in}}%
\pgfpathlineto{\pgfqpoint{2.668789in}{3.310853in}}%
\pgfpathlineto{\pgfqpoint{2.673502in}{3.319608in}}%
\pgfpathlineto{\pgfqpoint{2.687640in}{3.320966in}}%
\pgfpathlineto{\pgfqpoint{2.692353in}{3.320987in}}%
\pgfpathlineto{\pgfqpoint{2.701779in}{3.326503in}}%
\pgfpathlineto{\pgfqpoint{2.706492in}{3.336797in}}%
\pgfpathlineto{\pgfqpoint{2.711205in}{3.338097in}}%
\pgfpathlineto{\pgfqpoint{2.715918in}{3.338177in}}%
\pgfpathlineto{\pgfqpoint{2.720631in}{3.339825in}}%
\pgfpathlineto{\pgfqpoint{2.730056in}{3.341323in}}%
\pgfpathlineto{\pgfqpoint{2.734769in}{3.342830in}}%
\pgfpathlineto{\pgfqpoint{2.739482in}{3.372605in}}%
\pgfpathlineto{\pgfqpoint{2.744195in}{3.390170in}}%
\pgfpathlineto{\pgfqpoint{2.748908in}{3.392738in}}%
\pgfpathlineto{\pgfqpoint{2.758334in}{3.404249in}}%
\pgfpathlineto{\pgfqpoint{2.763047in}{3.420772in}}%
\pgfpathlineto{\pgfqpoint{2.767760in}{3.424416in}}%
\pgfpathlineto{\pgfqpoint{2.772473in}{3.433537in}}%
\pgfpathlineto{\pgfqpoint{2.777185in}{3.438334in}}%
\pgfpathlineto{\pgfqpoint{2.786611in}{3.686124in}}%
\pgfpathlineto{\pgfqpoint{2.791324in}{3.921942in}}%
\pgfpathlineto{\pgfqpoint{5.845287in}{3.921942in}}%
\pgfpathlineto{\pgfqpoint{5.845287in}{3.921942in}}%
\pgfusepath{stroke}%
\end{pgfscope}%
\begin{pgfscope}%
\pgfpathrectangle{\pgfqpoint{0.708220in}{2.519156in}}{\pgfqpoint{5.141780in}{1.402786in}}%
\pgfusepath{clip}%
\pgfsetbuttcap%
\pgfsetroundjoin%
\pgfsetlinewidth{2.007500pt}%
\definecolor{currentstroke}{rgb}{1.000000,0.694118,0.305882}%
\pgfsetstrokecolor{currentstroke}%
\pgfsetdash{{2.000000pt}{3.300000pt}}{0.000000pt}%
\pgfpathmoveto{\pgfqpoint{0.708220in}{2.592353in}}%
\pgfpathlineto{\pgfqpoint{0.727071in}{2.593926in}}%
\pgfpathlineto{\pgfqpoint{0.920301in}{2.596598in}}%
\pgfpathlineto{\pgfqpoint{1.052262in}{2.598148in}}%
\pgfpathlineto{\pgfqpoint{1.085252in}{2.598719in}}%
\pgfpathlineto{\pgfqpoint{1.165372in}{2.601297in}}%
\pgfpathlineto{\pgfqpoint{1.170085in}{2.613533in}}%
\pgfpathlineto{\pgfqpoint{1.184223in}{2.616267in}}%
\pgfpathlineto{\pgfqpoint{1.198362in}{2.617419in}}%
\pgfpathlineto{\pgfqpoint{1.203075in}{2.619027in}}%
\pgfpathlineto{\pgfqpoint{1.207788in}{2.619161in}}%
\pgfpathlineto{\pgfqpoint{1.212501in}{2.621495in}}%
\pgfpathlineto{\pgfqpoint{1.217214in}{2.621554in}}%
\pgfpathlineto{\pgfqpoint{1.221927in}{2.635583in}}%
\pgfpathlineto{\pgfqpoint{1.226639in}{2.646090in}}%
\pgfpathlineto{\pgfqpoint{1.231352in}{2.664498in}}%
\pgfpathlineto{\pgfqpoint{1.245491in}{2.667375in}}%
\pgfpathlineto{\pgfqpoint{1.254917in}{2.670572in}}%
\pgfpathlineto{\pgfqpoint{1.264343in}{2.672651in}}%
\pgfpathlineto{\pgfqpoint{1.283194in}{2.673548in}}%
\pgfpathlineto{\pgfqpoint{1.306759in}{2.674710in}}%
\pgfpathlineto{\pgfqpoint{1.358601in}{2.675758in}}%
\pgfpathlineto{\pgfqpoint{1.372740in}{2.676974in}}%
\pgfpathlineto{\pgfqpoint{1.401017in}{2.678175in}}%
\pgfpathlineto{\pgfqpoint{1.410443in}{2.680894in}}%
\pgfpathlineto{\pgfqpoint{1.415156in}{2.688293in}}%
\pgfpathlineto{\pgfqpoint{1.419869in}{2.690800in}}%
\pgfpathlineto{\pgfqpoint{1.429294in}{2.691630in}}%
\pgfpathlineto{\pgfqpoint{1.443433in}{2.697524in}}%
\pgfpathlineto{\pgfqpoint{1.528265in}{2.700470in}}%
\pgfpathlineto{\pgfqpoint{1.570682in}{2.702026in}}%
\pgfpathlineto{\pgfqpoint{1.580107in}{2.702830in}}%
\pgfpathlineto{\pgfqpoint{1.598959in}{2.703133in}}%
\pgfpathlineto{\pgfqpoint{1.603672in}{2.704829in}}%
\pgfpathlineto{\pgfqpoint{1.608385in}{2.704859in}}%
\pgfpathlineto{\pgfqpoint{1.613098in}{2.712685in}}%
\pgfpathlineto{\pgfqpoint{1.617811in}{2.713213in}}%
\pgfpathlineto{\pgfqpoint{1.622524in}{2.715653in}}%
\pgfpathlineto{\pgfqpoint{1.627236in}{2.720339in}}%
\pgfpathlineto{\pgfqpoint{1.646088in}{2.722003in}}%
\pgfpathlineto{\pgfqpoint{1.660227in}{2.725044in}}%
\pgfpathlineto{\pgfqpoint{1.669653in}{2.726384in}}%
\pgfpathlineto{\pgfqpoint{1.679078in}{2.728261in}}%
\pgfpathlineto{\pgfqpoint{1.688504in}{2.729377in}}%
\pgfpathlineto{\pgfqpoint{1.693217in}{2.730087in}}%
\pgfpathlineto{\pgfqpoint{1.697930in}{2.733621in}}%
\pgfpathlineto{\pgfqpoint{1.702643in}{2.733744in}}%
\pgfpathlineto{\pgfqpoint{1.712069in}{2.735734in}}%
\pgfpathlineto{\pgfqpoint{1.716782in}{2.743323in}}%
\pgfpathlineto{\pgfqpoint{1.726207in}{2.744128in}}%
\pgfpathlineto{\pgfqpoint{1.735633in}{2.745777in}}%
\pgfpathlineto{\pgfqpoint{1.740346in}{2.745839in}}%
\pgfpathlineto{\pgfqpoint{1.745059in}{2.748233in}}%
\pgfpathlineto{\pgfqpoint{1.754485in}{2.749908in}}%
\pgfpathlineto{\pgfqpoint{1.759198in}{2.755009in}}%
\pgfpathlineto{\pgfqpoint{1.768624in}{2.756564in}}%
\pgfpathlineto{\pgfqpoint{1.773337in}{2.774962in}}%
\pgfpathlineto{\pgfqpoint{1.782762in}{2.778320in}}%
\pgfpathlineto{\pgfqpoint{1.801614in}{2.779130in}}%
\pgfpathlineto{\pgfqpoint{1.806327in}{2.781647in}}%
\pgfpathlineto{\pgfqpoint{1.811040in}{2.781793in}}%
\pgfpathlineto{\pgfqpoint{1.815753in}{2.784167in}}%
\pgfpathlineto{\pgfqpoint{1.825179in}{2.784294in}}%
\pgfpathlineto{\pgfqpoint{1.839317in}{2.789040in}}%
\pgfpathlineto{\pgfqpoint{1.848743in}{2.789943in}}%
\pgfpathlineto{\pgfqpoint{1.872308in}{2.792411in}}%
\pgfpathlineto{\pgfqpoint{1.877020in}{2.794109in}}%
\pgfpathlineto{\pgfqpoint{1.886446in}{2.795072in}}%
\pgfpathlineto{\pgfqpoint{1.891159in}{2.796757in}}%
\pgfpathlineto{\pgfqpoint{1.933575in}{2.799870in}}%
\pgfpathlineto{\pgfqpoint{1.938288in}{2.802743in}}%
\pgfpathlineto{\pgfqpoint{1.952427in}{2.803920in}}%
\pgfpathlineto{\pgfqpoint{1.957140in}{2.805337in}}%
\pgfpathlineto{\pgfqpoint{1.961853in}{2.808416in}}%
\pgfpathlineto{\pgfqpoint{1.971279in}{2.810551in}}%
\pgfpathlineto{\pgfqpoint{1.980704in}{2.810772in}}%
\pgfpathlineto{\pgfqpoint{1.990130in}{2.813192in}}%
\pgfpathlineto{\pgfqpoint{2.004269in}{2.814301in}}%
\pgfpathlineto{\pgfqpoint{2.008982in}{2.816089in}}%
\pgfpathlineto{\pgfqpoint{2.023121in}{2.817732in}}%
\pgfpathlineto{\pgfqpoint{2.027833in}{2.819761in}}%
\pgfpathlineto{\pgfqpoint{2.032546in}{2.820204in}}%
\pgfpathlineto{\pgfqpoint{2.037259in}{2.822104in}}%
\pgfpathlineto{\pgfqpoint{2.065537in}{2.826440in}}%
\pgfpathlineto{\pgfqpoint{2.074963in}{2.829195in}}%
\pgfpathlineto{\pgfqpoint{2.084388in}{2.830672in}}%
\pgfpathlineto{\pgfqpoint{2.093814in}{2.831368in}}%
\pgfpathlineto{\pgfqpoint{2.126804in}{2.837179in}}%
\pgfpathlineto{\pgfqpoint{2.159795in}{2.839170in}}%
\pgfpathlineto{\pgfqpoint{2.169221in}{2.841554in}}%
\pgfpathlineto{\pgfqpoint{2.183359in}{2.842902in}}%
\pgfpathlineto{\pgfqpoint{2.202211in}{2.843992in}}%
\pgfpathlineto{\pgfqpoint{2.206924in}{2.846392in}}%
\pgfpathlineto{\pgfqpoint{2.211637in}{2.856565in}}%
\pgfpathlineto{\pgfqpoint{2.225776in}{2.857385in}}%
\pgfpathlineto{\pgfqpoint{2.239914in}{2.859183in}}%
\pgfpathlineto{\pgfqpoint{2.244627in}{2.859371in}}%
\pgfpathlineto{\pgfqpoint{2.249340in}{2.860806in}}%
\pgfpathlineto{\pgfqpoint{2.254053in}{2.861018in}}%
\pgfpathlineto{\pgfqpoint{2.268192in}{2.866464in}}%
\pgfpathlineto{\pgfqpoint{2.291756in}{2.869074in}}%
\pgfpathlineto{\pgfqpoint{2.301182in}{2.873367in}}%
\pgfpathlineto{\pgfqpoint{2.310608in}{2.877176in}}%
\pgfpathlineto{\pgfqpoint{2.338885in}{2.881297in}}%
\pgfpathlineto{\pgfqpoint{2.353024in}{2.882213in}}%
\pgfpathlineto{\pgfqpoint{2.367163in}{2.884807in}}%
\pgfpathlineto{\pgfqpoint{2.386014in}{2.885776in}}%
\pgfpathlineto{\pgfqpoint{2.414292in}{2.897052in}}%
\pgfpathlineto{\pgfqpoint{2.433143in}{2.900213in}}%
\pgfpathlineto{\pgfqpoint{2.437856in}{2.917345in}}%
\pgfpathlineto{\pgfqpoint{2.447282in}{2.920368in}}%
\pgfpathlineto{\pgfqpoint{2.456708in}{2.921568in}}%
\pgfpathlineto{\pgfqpoint{2.480272in}{2.924356in}}%
\pgfpathlineto{\pgfqpoint{2.484985in}{2.925368in}}%
\pgfpathlineto{\pgfqpoint{2.489698in}{2.929984in}}%
\pgfpathlineto{\pgfqpoint{2.499124in}{2.931888in}}%
\pgfpathlineto{\pgfqpoint{2.503837in}{2.936804in}}%
\pgfpathlineto{\pgfqpoint{2.508550in}{2.958373in}}%
\pgfpathlineto{\pgfqpoint{2.513263in}{2.958633in}}%
\pgfpathlineto{\pgfqpoint{2.517976in}{2.961500in}}%
\pgfpathlineto{\pgfqpoint{2.527401in}{2.962629in}}%
\pgfpathlineto{\pgfqpoint{2.532114in}{2.972866in}}%
\pgfpathlineto{\pgfqpoint{2.546253in}{2.979112in}}%
\pgfpathlineto{\pgfqpoint{2.550966in}{2.984915in}}%
\pgfpathlineto{\pgfqpoint{2.569818in}{2.987540in}}%
\pgfpathlineto{\pgfqpoint{2.579243in}{2.990192in}}%
\pgfpathlineto{\pgfqpoint{2.588669in}{2.997912in}}%
\pgfpathlineto{\pgfqpoint{2.598095in}{3.002249in}}%
\pgfpathlineto{\pgfqpoint{2.602808in}{3.003749in}}%
\pgfpathlineto{\pgfqpoint{2.607521in}{3.009751in}}%
\pgfpathlineto{\pgfqpoint{2.612234in}{3.009805in}}%
\pgfpathlineto{\pgfqpoint{2.616947in}{3.015238in}}%
\pgfpathlineto{\pgfqpoint{2.626372in}{3.016769in}}%
\pgfpathlineto{\pgfqpoint{2.631085in}{3.019931in}}%
\pgfpathlineto{\pgfqpoint{2.635798in}{3.019979in}}%
\pgfpathlineto{\pgfqpoint{2.649937in}{3.026986in}}%
\pgfpathlineto{\pgfqpoint{2.654650in}{3.027014in}}%
\pgfpathlineto{\pgfqpoint{2.659363in}{3.040612in}}%
\pgfpathlineto{\pgfqpoint{2.668789in}{3.041631in}}%
\pgfpathlineto{\pgfqpoint{2.673502in}{3.043602in}}%
\pgfpathlineto{\pgfqpoint{2.678214in}{3.056398in}}%
\pgfpathlineto{\pgfqpoint{2.687640in}{3.058187in}}%
\pgfpathlineto{\pgfqpoint{2.697066in}{3.061157in}}%
\pgfpathlineto{\pgfqpoint{2.706492in}{3.062147in}}%
\pgfpathlineto{\pgfqpoint{2.715918in}{3.065397in}}%
\pgfpathlineto{\pgfqpoint{2.720631in}{3.065586in}}%
\pgfpathlineto{\pgfqpoint{2.725344in}{3.107786in}}%
\pgfpathlineto{\pgfqpoint{2.730056in}{3.120491in}}%
\pgfpathlineto{\pgfqpoint{2.734769in}{3.171049in}}%
\pgfpathlineto{\pgfqpoint{2.739482in}{3.180100in}}%
\pgfpathlineto{\pgfqpoint{2.744195in}{3.194991in}}%
\pgfpathlineto{\pgfqpoint{2.748908in}{3.250082in}}%
\pgfpathlineto{\pgfqpoint{2.753621in}{3.256391in}}%
\pgfpathlineto{\pgfqpoint{2.758334in}{3.257889in}}%
\pgfpathlineto{\pgfqpoint{2.763047in}{3.842887in}}%
\pgfpathlineto{\pgfqpoint{2.767760in}{3.843284in}}%
\pgfpathlineto{\pgfqpoint{2.772473in}{3.921942in}}%
\pgfpathlineto{\pgfqpoint{5.845287in}{3.921942in}}%
\pgfpathlineto{\pgfqpoint{5.845287in}{3.921942in}}%
\pgfusepath{stroke}%
\end{pgfscope}%
\begin{pgfscope}%
\pgfpathrectangle{\pgfqpoint{0.708220in}{2.519156in}}{\pgfqpoint{5.141780in}{1.402786in}}%
\pgfusepath{clip}%
\pgfsetrectcap%
\pgfsetroundjoin%
\pgfsetlinewidth{2.007500pt}%
\definecolor{currentstroke}{rgb}{0.980392,0.529412,0.458824}%
\pgfsetstrokecolor{currentstroke}%
\pgfsetdash{}{0pt}%
\pgfpathmoveto{\pgfqpoint{0.708220in}{2.799531in}}%
\pgfpathlineto{\pgfqpoint{0.717646in}{2.801224in}}%
\pgfpathlineto{\pgfqpoint{0.727071in}{2.802450in}}%
\pgfpathlineto{\pgfqpoint{0.764775in}{2.804251in}}%
\pgfpathlineto{\pgfqpoint{0.802478in}{2.805358in}}%
\pgfpathlineto{\pgfqpoint{0.830755in}{2.806539in}}%
\pgfpathlineto{\pgfqpoint{0.901449in}{2.808963in}}%
\pgfpathlineto{\pgfqpoint{0.910875in}{2.810009in}}%
\pgfpathlineto{\pgfqpoint{0.929726in}{2.810729in}}%
\pgfpathlineto{\pgfqpoint{1.198362in}{2.817891in}}%
\pgfpathlineto{\pgfqpoint{1.250204in}{2.821982in}}%
\pgfpathlineto{\pgfqpoint{1.254917in}{2.825596in}}%
\pgfpathlineto{\pgfqpoint{1.264343in}{2.826258in}}%
\pgfpathlineto{\pgfqpoint{1.269056in}{2.828461in}}%
\pgfpathlineto{\pgfqpoint{1.278481in}{2.829481in}}%
\pgfpathlineto{\pgfqpoint{1.306759in}{2.830881in}}%
\pgfpathlineto{\pgfqpoint{1.316185in}{2.832100in}}%
\pgfpathlineto{\pgfqpoint{1.349175in}{2.833966in}}%
\pgfpathlineto{\pgfqpoint{1.443433in}{2.840302in}}%
\pgfpathlineto{\pgfqpoint{1.448146in}{2.842888in}}%
\pgfpathlineto{\pgfqpoint{1.452859in}{2.858564in}}%
\pgfpathlineto{\pgfqpoint{1.462285in}{2.860840in}}%
\pgfpathlineto{\pgfqpoint{1.466998in}{2.861199in}}%
\pgfpathlineto{\pgfqpoint{1.471711in}{2.865041in}}%
\pgfpathlineto{\pgfqpoint{1.485849in}{2.869370in}}%
\pgfpathlineto{\pgfqpoint{1.490562in}{2.869656in}}%
\pgfpathlineto{\pgfqpoint{1.495275in}{2.873604in}}%
\pgfpathlineto{\pgfqpoint{1.499988in}{2.880314in}}%
\pgfpathlineto{\pgfqpoint{1.504701in}{2.889409in}}%
\pgfpathlineto{\pgfqpoint{1.509414in}{2.890450in}}%
\pgfpathlineto{\pgfqpoint{1.514127in}{2.896020in}}%
\pgfpathlineto{\pgfqpoint{1.523553in}{2.898600in}}%
\pgfpathlineto{\pgfqpoint{1.528265in}{2.900707in}}%
\pgfpathlineto{\pgfqpoint{1.532978in}{2.915405in}}%
\pgfpathlineto{\pgfqpoint{1.537691in}{2.919575in}}%
\pgfpathlineto{\pgfqpoint{1.556543in}{2.924310in}}%
\pgfpathlineto{\pgfqpoint{1.561256in}{2.931959in}}%
\pgfpathlineto{\pgfqpoint{1.565969in}{2.936408in}}%
\pgfpathlineto{\pgfqpoint{1.570682in}{2.937871in}}%
\pgfpathlineto{\pgfqpoint{1.575395in}{2.938003in}}%
\pgfpathlineto{\pgfqpoint{1.589533in}{2.944198in}}%
\pgfpathlineto{\pgfqpoint{1.598959in}{2.951351in}}%
\pgfpathlineto{\pgfqpoint{1.608385in}{2.954042in}}%
\pgfpathlineto{\pgfqpoint{1.613098in}{2.959569in}}%
\pgfpathlineto{\pgfqpoint{1.617811in}{2.959801in}}%
\pgfpathlineto{\pgfqpoint{1.627236in}{2.969500in}}%
\pgfpathlineto{\pgfqpoint{1.631949in}{2.970112in}}%
\pgfpathlineto{\pgfqpoint{1.636662in}{2.975867in}}%
\pgfpathlineto{\pgfqpoint{1.641375in}{2.976390in}}%
\pgfpathlineto{\pgfqpoint{1.646088in}{2.995175in}}%
\pgfpathlineto{\pgfqpoint{1.650801in}{2.999802in}}%
\pgfpathlineto{\pgfqpoint{1.664940in}{3.001818in}}%
\pgfpathlineto{\pgfqpoint{1.721495in}{3.007781in}}%
\pgfpathlineto{\pgfqpoint{1.726207in}{3.009975in}}%
\pgfpathlineto{\pgfqpoint{1.735633in}{3.018126in}}%
\pgfpathlineto{\pgfqpoint{1.745059in}{3.019043in}}%
\pgfpathlineto{\pgfqpoint{1.749772in}{3.022071in}}%
\pgfpathlineto{\pgfqpoint{1.759198in}{3.023963in}}%
\pgfpathlineto{\pgfqpoint{1.763911in}{3.028657in}}%
\pgfpathlineto{\pgfqpoint{1.782762in}{3.032311in}}%
\pgfpathlineto{\pgfqpoint{1.792188in}{3.036649in}}%
\pgfpathlineto{\pgfqpoint{1.829891in}{3.043838in}}%
\pgfpathlineto{\pgfqpoint{1.834604in}{3.045982in}}%
\pgfpathlineto{\pgfqpoint{1.839317in}{3.046280in}}%
\pgfpathlineto{\pgfqpoint{1.844030in}{3.049752in}}%
\pgfpathlineto{\pgfqpoint{1.853456in}{3.050723in}}%
\pgfpathlineto{\pgfqpoint{1.862882in}{3.055020in}}%
\pgfpathlineto{\pgfqpoint{1.867595in}{3.060232in}}%
\pgfpathlineto{\pgfqpoint{1.872308in}{3.062762in}}%
\pgfpathlineto{\pgfqpoint{1.886446in}{3.064460in}}%
\pgfpathlineto{\pgfqpoint{1.891159in}{3.071220in}}%
\pgfpathlineto{\pgfqpoint{1.895872in}{3.074025in}}%
\pgfpathlineto{\pgfqpoint{1.900585in}{3.080070in}}%
\pgfpathlineto{\pgfqpoint{1.910011in}{3.083853in}}%
\pgfpathlineto{\pgfqpoint{1.919437in}{3.086819in}}%
\pgfpathlineto{\pgfqpoint{1.924150in}{3.097841in}}%
\pgfpathlineto{\pgfqpoint{1.928862in}{3.102963in}}%
\pgfpathlineto{\pgfqpoint{1.933575in}{3.104762in}}%
\pgfpathlineto{\pgfqpoint{1.938288in}{3.110849in}}%
\pgfpathlineto{\pgfqpoint{1.943001in}{3.125379in}}%
\pgfpathlineto{\pgfqpoint{1.947714in}{3.130487in}}%
\pgfpathlineto{\pgfqpoint{1.952427in}{3.131153in}}%
\pgfpathlineto{\pgfqpoint{1.957140in}{3.135345in}}%
\pgfpathlineto{\pgfqpoint{1.966566in}{3.138132in}}%
\pgfpathlineto{\pgfqpoint{1.971279in}{3.143803in}}%
\pgfpathlineto{\pgfqpoint{1.980704in}{3.144630in}}%
\pgfpathlineto{\pgfqpoint{1.990130in}{3.149243in}}%
\pgfpathlineto{\pgfqpoint{1.994843in}{3.154135in}}%
\pgfpathlineto{\pgfqpoint{2.008982in}{3.162396in}}%
\pgfpathlineto{\pgfqpoint{2.013695in}{3.164475in}}%
\pgfpathlineto{\pgfqpoint{2.018408in}{3.165215in}}%
\pgfpathlineto{\pgfqpoint{2.023121in}{3.170262in}}%
\pgfpathlineto{\pgfqpoint{2.027833in}{3.173116in}}%
\pgfpathlineto{\pgfqpoint{2.032546in}{3.177584in}}%
\pgfpathlineto{\pgfqpoint{2.037259in}{3.177598in}}%
\pgfpathlineto{\pgfqpoint{2.041972in}{3.185688in}}%
\pgfpathlineto{\pgfqpoint{2.046685in}{3.186442in}}%
\pgfpathlineto{\pgfqpoint{2.060824in}{3.197790in}}%
\pgfpathlineto{\pgfqpoint{2.070250in}{3.207595in}}%
\pgfpathlineto{\pgfqpoint{2.079675in}{3.209959in}}%
\pgfpathlineto{\pgfqpoint{2.089101in}{3.215847in}}%
\pgfpathlineto{\pgfqpoint{2.093814in}{3.222882in}}%
\pgfpathlineto{\pgfqpoint{2.112666in}{3.225230in}}%
\pgfpathlineto{\pgfqpoint{2.122092in}{3.226539in}}%
\pgfpathlineto{\pgfqpoint{2.126804in}{3.231698in}}%
\pgfpathlineto{\pgfqpoint{2.136230in}{3.246602in}}%
\pgfpathlineto{\pgfqpoint{2.145656in}{3.248415in}}%
\pgfpathlineto{\pgfqpoint{2.150369in}{3.251028in}}%
\pgfpathlineto{\pgfqpoint{2.155082in}{3.252121in}}%
\pgfpathlineto{\pgfqpoint{2.159795in}{3.256512in}}%
\pgfpathlineto{\pgfqpoint{2.178646in}{3.258346in}}%
\pgfpathlineto{\pgfqpoint{2.188072in}{3.260021in}}%
\pgfpathlineto{\pgfqpoint{2.192785in}{3.264165in}}%
\pgfpathlineto{\pgfqpoint{2.249340in}{3.272445in}}%
\pgfpathlineto{\pgfqpoint{2.268192in}{3.273636in}}%
\pgfpathlineto{\pgfqpoint{2.277617in}{3.275265in}}%
\pgfpathlineto{\pgfqpoint{2.287043in}{3.282220in}}%
\pgfpathlineto{\pgfqpoint{2.291756in}{3.287656in}}%
\pgfpathlineto{\pgfqpoint{2.305895in}{3.289127in}}%
\pgfpathlineto{\pgfqpoint{2.310608in}{3.291384in}}%
\pgfpathlineto{\pgfqpoint{2.315321in}{3.292029in}}%
\pgfpathlineto{\pgfqpoint{2.320034in}{3.304041in}}%
\pgfpathlineto{\pgfqpoint{2.324747in}{3.306165in}}%
\pgfpathlineto{\pgfqpoint{2.329459in}{3.306587in}}%
\pgfpathlineto{\pgfqpoint{2.348311in}{3.315890in}}%
\pgfpathlineto{\pgfqpoint{2.353024in}{3.315913in}}%
\pgfpathlineto{\pgfqpoint{2.357737in}{3.317831in}}%
\pgfpathlineto{\pgfqpoint{2.362450in}{3.318341in}}%
\pgfpathlineto{\pgfqpoint{2.371876in}{3.322547in}}%
\pgfpathlineto{\pgfqpoint{2.376588in}{3.322557in}}%
\pgfpathlineto{\pgfqpoint{2.381301in}{3.326739in}}%
\pgfpathlineto{\pgfqpoint{2.404866in}{3.328719in}}%
\pgfpathlineto{\pgfqpoint{2.414292in}{3.332071in}}%
\pgfpathlineto{\pgfqpoint{2.419005in}{3.338182in}}%
\pgfpathlineto{\pgfqpoint{2.428430in}{3.341350in}}%
\pgfpathlineto{\pgfqpoint{2.437856in}{3.355648in}}%
\pgfpathlineto{\pgfqpoint{2.456708in}{3.360229in}}%
\pgfpathlineto{\pgfqpoint{2.466134in}{3.361444in}}%
\pgfpathlineto{\pgfqpoint{2.470847in}{3.362900in}}%
\pgfpathlineto{\pgfqpoint{2.475560in}{3.365754in}}%
\pgfpathlineto{\pgfqpoint{2.494411in}{3.369790in}}%
\pgfpathlineto{\pgfqpoint{2.499124in}{3.369912in}}%
\pgfpathlineto{\pgfqpoint{2.503837in}{3.371507in}}%
\pgfpathlineto{\pgfqpoint{2.532114in}{3.374291in}}%
\pgfpathlineto{\pgfqpoint{2.536827in}{3.376108in}}%
\pgfpathlineto{\pgfqpoint{2.550966in}{3.376878in}}%
\pgfpathlineto{\pgfqpoint{2.560392in}{3.381529in}}%
\pgfpathlineto{\pgfqpoint{2.574531in}{3.383794in}}%
\pgfpathlineto{\pgfqpoint{2.579243in}{3.386895in}}%
\pgfpathlineto{\pgfqpoint{2.583956in}{3.394700in}}%
\pgfpathlineto{\pgfqpoint{2.588669in}{3.398430in}}%
\pgfpathlineto{\pgfqpoint{2.593382in}{3.407268in}}%
\pgfpathlineto{\pgfqpoint{2.598095in}{3.407628in}}%
\pgfpathlineto{\pgfqpoint{2.602808in}{3.410740in}}%
\pgfpathlineto{\pgfqpoint{2.607521in}{3.411492in}}%
\pgfpathlineto{\pgfqpoint{2.612234in}{3.413824in}}%
\pgfpathlineto{\pgfqpoint{2.616947in}{3.418434in}}%
\pgfpathlineto{\pgfqpoint{2.626372in}{3.423874in}}%
\pgfpathlineto{\pgfqpoint{2.635798in}{3.424719in}}%
\pgfpathlineto{\pgfqpoint{2.640511in}{3.426749in}}%
\pgfpathlineto{\pgfqpoint{2.649937in}{3.428287in}}%
\pgfpathlineto{\pgfqpoint{2.668789in}{3.433799in}}%
\pgfpathlineto{\pgfqpoint{2.673502in}{3.436128in}}%
\pgfpathlineto{\pgfqpoint{2.682927in}{3.437017in}}%
\pgfpathlineto{\pgfqpoint{2.687640in}{3.443821in}}%
\pgfpathlineto{\pgfqpoint{2.697066in}{3.445924in}}%
\pgfpathlineto{\pgfqpoint{2.701779in}{3.455072in}}%
\pgfpathlineto{\pgfqpoint{2.706492in}{3.458872in}}%
\pgfpathlineto{\pgfqpoint{2.715918in}{3.459825in}}%
\pgfpathlineto{\pgfqpoint{2.720631in}{3.463794in}}%
\pgfpathlineto{\pgfqpoint{2.734769in}{3.465608in}}%
\pgfpathlineto{\pgfqpoint{2.744195in}{3.466921in}}%
\pgfpathlineto{\pgfqpoint{2.767760in}{3.471788in}}%
\pgfpathlineto{\pgfqpoint{2.772473in}{3.475029in}}%
\pgfpathlineto{\pgfqpoint{2.814889in}{3.478888in}}%
\pgfpathlineto{\pgfqpoint{2.819602in}{3.479742in}}%
\pgfpathlineto{\pgfqpoint{2.824315in}{3.485412in}}%
\pgfpathlineto{\pgfqpoint{2.829027in}{3.485731in}}%
\pgfpathlineto{\pgfqpoint{2.833740in}{3.487971in}}%
\pgfpathlineto{\pgfqpoint{2.857305in}{3.489643in}}%
\pgfpathlineto{\pgfqpoint{2.862018in}{3.494868in}}%
\pgfpathlineto{\pgfqpoint{2.866731in}{3.496397in}}%
\pgfpathlineto{\pgfqpoint{2.876156in}{3.512173in}}%
\pgfpathlineto{\pgfqpoint{2.880869in}{3.514594in}}%
\pgfpathlineto{\pgfqpoint{2.885582in}{3.514997in}}%
\pgfpathlineto{\pgfqpoint{2.895008in}{3.520024in}}%
\pgfpathlineto{\pgfqpoint{2.899721in}{3.520036in}}%
\pgfpathlineto{\pgfqpoint{2.913860in}{3.524886in}}%
\pgfpathlineto{\pgfqpoint{2.918573in}{3.524888in}}%
\pgfpathlineto{\pgfqpoint{2.923286in}{3.527219in}}%
\pgfpathlineto{\pgfqpoint{2.927998in}{3.531342in}}%
\pgfpathlineto{\pgfqpoint{2.951563in}{3.533182in}}%
\pgfpathlineto{\pgfqpoint{2.956276in}{3.537269in}}%
\pgfpathlineto{\pgfqpoint{2.960989in}{3.538000in}}%
\pgfpathlineto{\pgfqpoint{2.965702in}{3.545359in}}%
\pgfpathlineto{\pgfqpoint{2.970415in}{3.545699in}}%
\pgfpathlineto{\pgfqpoint{2.975128in}{3.554105in}}%
\pgfpathlineto{\pgfqpoint{2.984553in}{3.562539in}}%
\pgfpathlineto{\pgfqpoint{2.989266in}{3.563111in}}%
\pgfpathlineto{\pgfqpoint{2.993979in}{3.565085in}}%
\pgfpathlineto{\pgfqpoint{3.017544in}{3.568362in}}%
\pgfpathlineto{\pgfqpoint{3.022257in}{3.572322in}}%
\pgfpathlineto{\pgfqpoint{3.064673in}{3.577774in}}%
\pgfpathlineto{\pgfqpoint{3.078811in}{3.582911in}}%
\pgfpathlineto{\pgfqpoint{3.083524in}{3.583073in}}%
\pgfpathlineto{\pgfqpoint{3.107089in}{3.589494in}}%
\pgfpathlineto{\pgfqpoint{3.111802in}{3.599705in}}%
\pgfpathlineto{\pgfqpoint{3.125940in}{3.602118in}}%
\pgfpathlineto{\pgfqpoint{3.130653in}{3.607615in}}%
\pgfpathlineto{\pgfqpoint{3.140079in}{3.608068in}}%
\pgfpathlineto{\pgfqpoint{3.144792in}{3.610528in}}%
\pgfpathlineto{\pgfqpoint{3.149505in}{3.615913in}}%
\pgfpathlineto{\pgfqpoint{3.154218in}{3.616241in}}%
\pgfpathlineto{\pgfqpoint{3.158931in}{3.620774in}}%
\pgfpathlineto{\pgfqpoint{3.163644in}{3.621894in}}%
\pgfpathlineto{\pgfqpoint{3.168357in}{3.698200in}}%
\pgfpathlineto{\pgfqpoint{3.173070in}{3.706196in}}%
\pgfpathlineto{\pgfqpoint{3.177782in}{3.710821in}}%
\pgfpathlineto{\pgfqpoint{3.182495in}{3.713305in}}%
\pgfpathlineto{\pgfqpoint{3.187208in}{3.714091in}}%
\pgfpathlineto{\pgfqpoint{3.191921in}{3.718637in}}%
\pgfpathlineto{\pgfqpoint{3.201347in}{3.719025in}}%
\pgfpathlineto{\pgfqpoint{3.210773in}{3.723886in}}%
\pgfpathlineto{\pgfqpoint{3.215486in}{3.728542in}}%
\pgfpathlineto{\pgfqpoint{3.224912in}{3.730545in}}%
\pgfpathlineto{\pgfqpoint{3.248476in}{3.735665in}}%
\pgfpathlineto{\pgfqpoint{3.253189in}{3.738164in}}%
\pgfpathlineto{\pgfqpoint{3.262615in}{3.750773in}}%
\pgfpathlineto{\pgfqpoint{3.267328in}{3.752714in}}%
\pgfpathlineto{\pgfqpoint{3.272041in}{3.758843in}}%
\pgfpathlineto{\pgfqpoint{3.276753in}{3.760407in}}%
\pgfpathlineto{\pgfqpoint{3.281466in}{3.767709in}}%
\pgfpathlineto{\pgfqpoint{3.295605in}{3.773576in}}%
\pgfpathlineto{\pgfqpoint{3.300318in}{3.780746in}}%
\pgfpathlineto{\pgfqpoint{3.309744in}{3.785079in}}%
\pgfpathlineto{\pgfqpoint{3.319170in}{3.791712in}}%
\pgfpathlineto{\pgfqpoint{3.323883in}{3.793397in}}%
\pgfpathlineto{\pgfqpoint{3.338021in}{3.794941in}}%
\pgfpathlineto{\pgfqpoint{3.347447in}{3.799615in}}%
\pgfpathlineto{\pgfqpoint{3.352160in}{3.799873in}}%
\pgfpathlineto{\pgfqpoint{3.361586in}{3.803581in}}%
\pgfpathlineto{\pgfqpoint{3.366299in}{3.814802in}}%
\pgfpathlineto{\pgfqpoint{3.371012in}{3.849600in}}%
\pgfpathlineto{\pgfqpoint{3.375724in}{3.849904in}}%
\pgfpathlineto{\pgfqpoint{3.380437in}{3.852897in}}%
\pgfpathlineto{\pgfqpoint{3.385150in}{3.862535in}}%
\pgfpathlineto{\pgfqpoint{3.389863in}{3.876768in}}%
\pgfpathlineto{\pgfqpoint{3.394576in}{3.878420in}}%
\pgfpathlineto{\pgfqpoint{3.399289in}{3.921942in}}%
\pgfpathlineto{\pgfqpoint{5.845287in}{3.921942in}}%
\pgfpathlineto{\pgfqpoint{5.845287in}{3.921942in}}%
\pgfusepath{stroke}%
\end{pgfscope}%
\begin{pgfscope}%
\pgfsetrectcap%
\pgfsetmiterjoin%
\pgfsetlinewidth{0.803000pt}%
\definecolor{currentstroke}{rgb}{0.000000,0.000000,0.000000}%
\pgfsetstrokecolor{currentstroke}%
\pgfsetdash{}{0pt}%
\pgfpathmoveto{\pgfqpoint{0.708220in}{2.519156in}}%
\pgfpathlineto{\pgfqpoint{0.708220in}{3.921942in}}%
\pgfusepath{stroke}%
\end{pgfscope}%
\begin{pgfscope}%
\pgfsetrectcap%
\pgfsetmiterjoin%
\pgfsetlinewidth{0.803000pt}%
\definecolor{currentstroke}{rgb}{0.000000,0.000000,0.000000}%
\pgfsetstrokecolor{currentstroke}%
\pgfsetdash{}{0pt}%
\pgfpathmoveto{\pgfqpoint{5.850000in}{2.519156in}}%
\pgfpathlineto{\pgfqpoint{5.850000in}{3.921942in}}%
\pgfusepath{stroke}%
\end{pgfscope}%
\begin{pgfscope}%
\pgfsetrectcap%
\pgfsetmiterjoin%
\pgfsetlinewidth{0.803000pt}%
\definecolor{currentstroke}{rgb}{0.000000,0.000000,0.000000}%
\pgfsetstrokecolor{currentstroke}%
\pgfsetdash{}{0pt}%
\pgfpathmoveto{\pgfqpoint{0.708220in}{2.519156in}}%
\pgfpathlineto{\pgfqpoint{5.850000in}{2.519156in}}%
\pgfusepath{stroke}%
\end{pgfscope}%
\begin{pgfscope}%
\pgfsetrectcap%
\pgfsetmiterjoin%
\pgfsetlinewidth{0.803000pt}%
\definecolor{currentstroke}{rgb}{0.000000,0.000000,0.000000}%
\pgfsetstrokecolor{currentstroke}%
\pgfsetdash{}{0pt}%
\pgfpathmoveto{\pgfqpoint{0.708220in}{3.921942in}}%
\pgfpathlineto{\pgfqpoint{5.850000in}{3.921942in}}%
\pgfusepath{stroke}%
\end{pgfscope}%
\begin{pgfscope}%
\pgfsetbuttcap%
\pgfsetroundjoin%
\pgfsetlinewidth{2.007500pt}%
\definecolor{currentstroke}{rgb}{1.000000,0.843137,0.000000}%
\pgfsetstrokecolor{currentstroke}%
\pgfsetdash{{7.400000pt}{3.200000pt}}{0.000000pt}%
\pgfpathmoveto{\pgfqpoint{4.827505in}{2.960805in}}%
\pgfpathlineto{\pgfqpoint{5.077505in}{2.960805in}}%
\pgfusepath{stroke}%
\end{pgfscope}%
\begin{pgfscope}%
\definecolor{textcolor}{rgb}{0.000000,0.000000,0.000000}%
\pgfsetstrokecolor{textcolor}%
\pgfsetfillcolor{textcolor}%
\pgftext[x=5.102505in,y=2.917055in,left,base]{\color{textcolor}\rmfamily\fontsize{9.000000}{10.800000}\selectfont FT+htd}%
\end{pgfscope}%
\begin{pgfscope}%
\pgfsetbuttcap%
\pgfsetroundjoin%
\pgfsetlinewidth{2.007500pt}%
\definecolor{currentstroke}{rgb}{1.000000,0.694118,0.305882}%
\pgfsetstrokecolor{currentstroke}%
\pgfsetdash{{2.000000pt}{3.300000pt}}{0.000000pt}%
\pgfpathmoveto{\pgfqpoint{4.827505in}{2.799005in}}%
\pgfpathlineto{\pgfqpoint{5.077505in}{2.799005in}}%
\pgfusepath{stroke}%
\end{pgfscope}%
\begin{pgfscope}%
\definecolor{textcolor}{rgb}{0.000000,0.000000,0.000000}%
\pgfsetstrokecolor{textcolor}%
\pgfsetfillcolor{textcolor}%
\pgftext[x=5.102505in,y=2.755255in,left,base]{\color{textcolor}\rmfamily\fontsize{9.000000}{10.800000}\selectfont FT+Flow}%
\end{pgfscope}%
\begin{pgfscope}%
\pgfsetrectcap%
\pgfsetroundjoin%
\pgfsetlinewidth{2.007500pt}%
\definecolor{currentstroke}{rgb}{0.980392,0.529412,0.458824}%
\pgfsetstrokecolor{currentstroke}%
\pgfsetdash{}{0pt}%
\pgfpathmoveto{\pgfqpoint{4.827505in}{2.637206in}}%
\pgfpathlineto{\pgfqpoint{5.077505in}{2.637206in}}%
\pgfusepath{stroke}%
\end{pgfscope}%
\begin{pgfscope}%
\definecolor{textcolor}{rgb}{0.000000,0.000000,0.000000}%
\pgfsetstrokecolor{textcolor}%
\pgfsetfillcolor{textcolor}%
\pgftext[x=5.102505in,y=2.593456in,left,base]{\color{textcolor}\rmfamily\fontsize{9.000000}{10.800000}\selectfont FT+Tamaki}%
\end{pgfscope}%
\begin{pgfscope}%
\pgfsetbuttcap%
\pgfsetmiterjoin%
\definecolor{currentfill}{rgb}{1.000000,1.000000,1.000000}%
\pgfsetfillcolor{currentfill}%
\pgfsetlinewidth{0.000000pt}%
\definecolor{currentstroke}{rgb}{0.000000,0.000000,0.000000}%
\pgfsetstrokecolor{currentstroke}%
\pgfsetstrokeopacity{0.000000}%
\pgfsetdash{}{0pt}%
\pgfpathmoveto{\pgfqpoint{0.708220in}{0.535823in}}%
\pgfpathlineto{\pgfqpoint{5.850000in}{0.535823in}}%
\pgfpathlineto{\pgfqpoint{5.850000in}{1.938609in}}%
\pgfpathlineto{\pgfqpoint{0.708220in}{1.938609in}}%
\pgfpathclose%
\pgfusepath{fill}%
\end{pgfscope}%
\begin{pgfscope}%
\pgfsetbuttcap%
\pgfsetroundjoin%
\definecolor{currentfill}{rgb}{0.000000,0.000000,0.000000}%
\pgfsetfillcolor{currentfill}%
\pgfsetlinewidth{0.803000pt}%
\definecolor{currentstroke}{rgb}{0.000000,0.000000,0.000000}%
\pgfsetstrokecolor{currentstroke}%
\pgfsetdash{}{0pt}%
\pgfsys@defobject{currentmarker}{\pgfqpoint{0.000000in}{-0.048611in}}{\pgfqpoint{0.000000in}{0.000000in}}{%
\pgfpathmoveto{\pgfqpoint{0.000000in}{0.000000in}}%
\pgfpathlineto{\pgfqpoint{0.000000in}{-0.048611in}}%
\pgfusepath{stroke,fill}%
}%
\begin{pgfscope}%
\pgfsys@transformshift{0.708220in}{0.535823in}%
\pgfsys@useobject{currentmarker}{}%
\end{pgfscope}%
\end{pgfscope}%
\begin{pgfscope}%
\definecolor{textcolor}{rgb}{0.000000,0.000000,0.000000}%
\pgfsetstrokecolor{textcolor}%
\pgfsetfillcolor{textcolor}%
\pgftext[x=0.708220in,y=0.438600in,,top]{\color{textcolor}\rmfamily\fontsize{9.000000}{10.800000}\selectfont \(\displaystyle {0}\)}%
\end{pgfscope}%
\begin{pgfscope}%
\pgfsetbuttcap%
\pgfsetroundjoin%
\definecolor{currentfill}{rgb}{0.000000,0.000000,0.000000}%
\pgfsetfillcolor{currentfill}%
\pgfsetlinewidth{0.803000pt}%
\definecolor{currentstroke}{rgb}{0.000000,0.000000,0.000000}%
\pgfsetstrokecolor{currentstroke}%
\pgfsetdash{}{0pt}%
\pgfsys@defobject{currentmarker}{\pgfqpoint{0.000000in}{-0.048611in}}{\pgfqpoint{0.000000in}{0.000000in}}{%
\pgfpathmoveto{\pgfqpoint{0.000000in}{0.000000in}}%
\pgfpathlineto{\pgfqpoint{0.000000in}{-0.048611in}}%
\pgfusepath{stroke,fill}%
}%
\begin{pgfscope}%
\pgfsys@transformshift{1.650801in}{0.535823in}%
\pgfsys@useobject{currentmarker}{}%
\end{pgfscope}%
\end{pgfscope}%
\begin{pgfscope}%
\definecolor{textcolor}{rgb}{0.000000,0.000000,0.000000}%
\pgfsetstrokecolor{textcolor}%
\pgfsetfillcolor{textcolor}%
\pgftext[x=1.650801in,y=0.438600in,,top]{\color{textcolor}\rmfamily\fontsize{9.000000}{10.800000}\selectfont \(\displaystyle {200}\)}%
\end{pgfscope}%
\begin{pgfscope}%
\pgfsetbuttcap%
\pgfsetroundjoin%
\definecolor{currentfill}{rgb}{0.000000,0.000000,0.000000}%
\pgfsetfillcolor{currentfill}%
\pgfsetlinewidth{0.803000pt}%
\definecolor{currentstroke}{rgb}{0.000000,0.000000,0.000000}%
\pgfsetstrokecolor{currentstroke}%
\pgfsetdash{}{0pt}%
\pgfsys@defobject{currentmarker}{\pgfqpoint{0.000000in}{-0.048611in}}{\pgfqpoint{0.000000in}{0.000000in}}{%
\pgfpathmoveto{\pgfqpoint{0.000000in}{0.000000in}}%
\pgfpathlineto{\pgfqpoint{0.000000in}{-0.048611in}}%
\pgfusepath{stroke,fill}%
}%
\begin{pgfscope}%
\pgfsys@transformshift{2.593382in}{0.535823in}%
\pgfsys@useobject{currentmarker}{}%
\end{pgfscope}%
\end{pgfscope}%
\begin{pgfscope}%
\definecolor{textcolor}{rgb}{0.000000,0.000000,0.000000}%
\pgfsetstrokecolor{textcolor}%
\pgfsetfillcolor{textcolor}%
\pgftext[x=2.593382in,y=0.438600in,,top]{\color{textcolor}\rmfamily\fontsize{9.000000}{10.800000}\selectfont \(\displaystyle {400}\)}%
\end{pgfscope}%
\begin{pgfscope}%
\pgfsetbuttcap%
\pgfsetroundjoin%
\definecolor{currentfill}{rgb}{0.000000,0.000000,0.000000}%
\pgfsetfillcolor{currentfill}%
\pgfsetlinewidth{0.803000pt}%
\definecolor{currentstroke}{rgb}{0.000000,0.000000,0.000000}%
\pgfsetstrokecolor{currentstroke}%
\pgfsetdash{}{0pt}%
\pgfsys@defobject{currentmarker}{\pgfqpoint{0.000000in}{-0.048611in}}{\pgfqpoint{0.000000in}{0.000000in}}{%
\pgfpathmoveto{\pgfqpoint{0.000000in}{0.000000in}}%
\pgfpathlineto{\pgfqpoint{0.000000in}{-0.048611in}}%
\pgfusepath{stroke,fill}%
}%
\begin{pgfscope}%
\pgfsys@transformshift{3.535963in}{0.535823in}%
\pgfsys@useobject{currentmarker}{}%
\end{pgfscope}%
\end{pgfscope}%
\begin{pgfscope}%
\definecolor{textcolor}{rgb}{0.000000,0.000000,0.000000}%
\pgfsetstrokecolor{textcolor}%
\pgfsetfillcolor{textcolor}%
\pgftext[x=3.535963in,y=0.438600in,,top]{\color{textcolor}\rmfamily\fontsize{9.000000}{10.800000}\selectfont \(\displaystyle {600}\)}%
\end{pgfscope}%
\begin{pgfscope}%
\pgfsetbuttcap%
\pgfsetroundjoin%
\definecolor{currentfill}{rgb}{0.000000,0.000000,0.000000}%
\pgfsetfillcolor{currentfill}%
\pgfsetlinewidth{0.803000pt}%
\definecolor{currentstroke}{rgb}{0.000000,0.000000,0.000000}%
\pgfsetstrokecolor{currentstroke}%
\pgfsetdash{}{0pt}%
\pgfsys@defobject{currentmarker}{\pgfqpoint{0.000000in}{-0.048611in}}{\pgfqpoint{0.000000in}{0.000000in}}{%
\pgfpathmoveto{\pgfqpoint{0.000000in}{0.000000in}}%
\pgfpathlineto{\pgfqpoint{0.000000in}{-0.048611in}}%
\pgfusepath{stroke,fill}%
}%
\begin{pgfscope}%
\pgfsys@transformshift{4.478544in}{0.535823in}%
\pgfsys@useobject{currentmarker}{}%
\end{pgfscope}%
\end{pgfscope}%
\begin{pgfscope}%
\definecolor{textcolor}{rgb}{0.000000,0.000000,0.000000}%
\pgfsetstrokecolor{textcolor}%
\pgfsetfillcolor{textcolor}%
\pgftext[x=4.478544in,y=0.438600in,,top]{\color{textcolor}\rmfamily\fontsize{9.000000}{10.800000}\selectfont \(\displaystyle {800}\)}%
\end{pgfscope}%
\begin{pgfscope}%
\pgfsetbuttcap%
\pgfsetroundjoin%
\definecolor{currentfill}{rgb}{0.000000,0.000000,0.000000}%
\pgfsetfillcolor{currentfill}%
\pgfsetlinewidth{0.803000pt}%
\definecolor{currentstroke}{rgb}{0.000000,0.000000,0.000000}%
\pgfsetstrokecolor{currentstroke}%
\pgfsetdash{}{0pt}%
\pgfsys@defobject{currentmarker}{\pgfqpoint{0.000000in}{-0.048611in}}{\pgfqpoint{0.000000in}{0.000000in}}{%
\pgfpathmoveto{\pgfqpoint{0.000000in}{0.000000in}}%
\pgfpathlineto{\pgfqpoint{0.000000in}{-0.048611in}}%
\pgfusepath{stroke,fill}%
}%
\begin{pgfscope}%
\pgfsys@transformshift{5.421126in}{0.535823in}%
\pgfsys@useobject{currentmarker}{}%
\end{pgfscope}%
\end{pgfscope}%
\begin{pgfscope}%
\definecolor{textcolor}{rgb}{0.000000,0.000000,0.000000}%
\pgfsetstrokecolor{textcolor}%
\pgfsetfillcolor{textcolor}%
\pgftext[x=5.421126in,y=0.438600in,,top]{\color{textcolor}\rmfamily\fontsize{9.000000}{10.800000}\selectfont \(\displaystyle {1000}\)}%
\end{pgfscope}%
\begin{pgfscope}%
\definecolor{textcolor}{rgb}{0.000000,0.000000,0.000000}%
\pgfsetstrokecolor{textcolor}%
\pgfsetfillcolor{textcolor}%
\pgftext[x=3.279110in,y=0.272655in,,top]{\color{textcolor}\rmfamily\fontsize{10.000000}{12.000000}\selectfont Number of benchmarks solved}%
\end{pgfscope}%
\begin{pgfscope}%
\pgfsetbuttcap%
\pgfsetroundjoin%
\definecolor{currentfill}{rgb}{0.000000,0.000000,0.000000}%
\pgfsetfillcolor{currentfill}%
\pgfsetlinewidth{0.803000pt}%
\definecolor{currentstroke}{rgb}{0.000000,0.000000,0.000000}%
\pgfsetstrokecolor{currentstroke}%
\pgfsetdash{}{0pt}%
\pgfsys@defobject{currentmarker}{\pgfqpoint{-0.048611in}{0.000000in}}{\pgfqpoint{-0.000000in}{0.000000in}}{%
\pgfpathmoveto{\pgfqpoint{-0.000000in}{0.000000in}}%
\pgfpathlineto{\pgfqpoint{-0.048611in}{0.000000in}}%
\pgfusepath{stroke,fill}%
}%
\begin{pgfscope}%
\pgfsys@transformshift{0.708220in}{0.535823in}%
\pgfsys@useobject{currentmarker}{}%
\end{pgfscope}%
\end{pgfscope}%
\begin{pgfscope}%
\definecolor{textcolor}{rgb}{0.000000,0.000000,0.000000}%
\pgfsetstrokecolor{textcolor}%
\pgfsetfillcolor{textcolor}%
\pgftext[x=0.344411in, y=0.491098in, left, base]{\color{textcolor}\rmfamily\fontsize{9.000000}{10.800000}\selectfont \(\displaystyle {10^{-1}}\)}%
\end{pgfscope}%
\begin{pgfscope}%
\pgfsetbuttcap%
\pgfsetroundjoin%
\definecolor{currentfill}{rgb}{0.000000,0.000000,0.000000}%
\pgfsetfillcolor{currentfill}%
\pgfsetlinewidth{0.803000pt}%
\definecolor{currentstroke}{rgb}{0.000000,0.000000,0.000000}%
\pgfsetstrokecolor{currentstroke}%
\pgfsetdash{}{0pt}%
\pgfsys@defobject{currentmarker}{\pgfqpoint{-0.048611in}{0.000000in}}{\pgfqpoint{-0.000000in}{0.000000in}}{%
\pgfpathmoveto{\pgfqpoint{-0.000000in}{0.000000in}}%
\pgfpathlineto{\pgfqpoint{-0.048611in}{0.000000in}}%
\pgfusepath{stroke,fill}%
}%
\begin{pgfscope}%
\pgfsys@transformshift{0.708220in}{0.886519in}%
\pgfsys@useobject{currentmarker}{}%
\end{pgfscope}%
\end{pgfscope}%
\begin{pgfscope}%
\definecolor{textcolor}{rgb}{0.000000,0.000000,0.000000}%
\pgfsetstrokecolor{textcolor}%
\pgfsetfillcolor{textcolor}%
\pgftext[x=0.424657in, y=0.841794in, left, base]{\color{textcolor}\rmfamily\fontsize{9.000000}{10.800000}\selectfont \(\displaystyle {10^{0}}\)}%
\end{pgfscope}%
\begin{pgfscope}%
\pgfsetbuttcap%
\pgfsetroundjoin%
\definecolor{currentfill}{rgb}{0.000000,0.000000,0.000000}%
\pgfsetfillcolor{currentfill}%
\pgfsetlinewidth{0.803000pt}%
\definecolor{currentstroke}{rgb}{0.000000,0.000000,0.000000}%
\pgfsetstrokecolor{currentstroke}%
\pgfsetdash{}{0pt}%
\pgfsys@defobject{currentmarker}{\pgfqpoint{-0.048611in}{0.000000in}}{\pgfqpoint{-0.000000in}{0.000000in}}{%
\pgfpathmoveto{\pgfqpoint{-0.000000in}{0.000000in}}%
\pgfpathlineto{\pgfqpoint{-0.048611in}{0.000000in}}%
\pgfusepath{stroke,fill}%
}%
\begin{pgfscope}%
\pgfsys@transformshift{0.708220in}{1.237216in}%
\pgfsys@useobject{currentmarker}{}%
\end{pgfscope}%
\end{pgfscope}%
\begin{pgfscope}%
\definecolor{textcolor}{rgb}{0.000000,0.000000,0.000000}%
\pgfsetstrokecolor{textcolor}%
\pgfsetfillcolor{textcolor}%
\pgftext[x=0.424657in, y=1.192491in, left, base]{\color{textcolor}\rmfamily\fontsize{9.000000}{10.800000}\selectfont \(\displaystyle {10^{1}}\)}%
\end{pgfscope}%
\begin{pgfscope}%
\pgfsetbuttcap%
\pgfsetroundjoin%
\definecolor{currentfill}{rgb}{0.000000,0.000000,0.000000}%
\pgfsetfillcolor{currentfill}%
\pgfsetlinewidth{0.803000pt}%
\definecolor{currentstroke}{rgb}{0.000000,0.000000,0.000000}%
\pgfsetstrokecolor{currentstroke}%
\pgfsetdash{}{0pt}%
\pgfsys@defobject{currentmarker}{\pgfqpoint{-0.048611in}{0.000000in}}{\pgfqpoint{-0.000000in}{0.000000in}}{%
\pgfpathmoveto{\pgfqpoint{-0.000000in}{0.000000in}}%
\pgfpathlineto{\pgfqpoint{-0.048611in}{0.000000in}}%
\pgfusepath{stroke,fill}%
}%
\begin{pgfscope}%
\pgfsys@transformshift{0.708220in}{1.587912in}%
\pgfsys@useobject{currentmarker}{}%
\end{pgfscope}%
\end{pgfscope}%
\begin{pgfscope}%
\definecolor{textcolor}{rgb}{0.000000,0.000000,0.000000}%
\pgfsetstrokecolor{textcolor}%
\pgfsetfillcolor{textcolor}%
\pgftext[x=0.424657in, y=1.543187in, left, base]{\color{textcolor}\rmfamily\fontsize{9.000000}{10.800000}\selectfont \(\displaystyle {10^{2}}\)}%
\end{pgfscope}%
\begin{pgfscope}%
\pgfsetbuttcap%
\pgfsetroundjoin%
\definecolor{currentfill}{rgb}{0.000000,0.000000,0.000000}%
\pgfsetfillcolor{currentfill}%
\pgfsetlinewidth{0.803000pt}%
\definecolor{currentstroke}{rgb}{0.000000,0.000000,0.000000}%
\pgfsetstrokecolor{currentstroke}%
\pgfsetdash{}{0pt}%
\pgfsys@defobject{currentmarker}{\pgfqpoint{-0.048611in}{0.000000in}}{\pgfqpoint{-0.000000in}{0.000000in}}{%
\pgfpathmoveto{\pgfqpoint{-0.000000in}{0.000000in}}%
\pgfpathlineto{\pgfqpoint{-0.048611in}{0.000000in}}%
\pgfusepath{stroke,fill}%
}%
\begin{pgfscope}%
\pgfsys@transformshift{0.708220in}{1.938609in}%
\pgfsys@useobject{currentmarker}{}%
\end{pgfscope}%
\end{pgfscope}%
\begin{pgfscope}%
\definecolor{textcolor}{rgb}{0.000000,0.000000,0.000000}%
\pgfsetstrokecolor{textcolor}%
\pgfsetfillcolor{textcolor}%
\pgftext[x=0.424657in, y=1.893884in, left, base]{\color{textcolor}\rmfamily\fontsize{9.000000}{10.800000}\selectfont \(\displaystyle {10^{3}}\)}%
\end{pgfscope}%
\begin{pgfscope}%
\pgfsetbuttcap%
\pgfsetroundjoin%
\definecolor{currentfill}{rgb}{0.000000,0.000000,0.000000}%
\pgfsetfillcolor{currentfill}%
\pgfsetlinewidth{0.602250pt}%
\definecolor{currentstroke}{rgb}{0.000000,0.000000,0.000000}%
\pgfsetstrokecolor{currentstroke}%
\pgfsetdash{}{0pt}%
\pgfsys@defobject{currentmarker}{\pgfqpoint{-0.027778in}{0.000000in}}{\pgfqpoint{-0.000000in}{0.000000in}}{%
\pgfpathmoveto{\pgfqpoint{-0.000000in}{0.000000in}}%
\pgfpathlineto{\pgfqpoint{-0.027778in}{0.000000in}}%
\pgfusepath{stroke,fill}%
}%
\begin{pgfscope}%
\pgfsys@transformshift{0.708220in}{0.641393in}%
\pgfsys@useobject{currentmarker}{}%
\end{pgfscope}%
\end{pgfscope}%
\begin{pgfscope}%
\pgfsetbuttcap%
\pgfsetroundjoin%
\definecolor{currentfill}{rgb}{0.000000,0.000000,0.000000}%
\pgfsetfillcolor{currentfill}%
\pgfsetlinewidth{0.602250pt}%
\definecolor{currentstroke}{rgb}{0.000000,0.000000,0.000000}%
\pgfsetstrokecolor{currentstroke}%
\pgfsetdash{}{0pt}%
\pgfsys@defobject{currentmarker}{\pgfqpoint{-0.027778in}{0.000000in}}{\pgfqpoint{-0.000000in}{0.000000in}}{%
\pgfpathmoveto{\pgfqpoint{-0.000000in}{0.000000in}}%
\pgfpathlineto{\pgfqpoint{-0.027778in}{0.000000in}}%
\pgfusepath{stroke,fill}%
}%
\begin{pgfscope}%
\pgfsys@transformshift{0.708220in}{0.703147in}%
\pgfsys@useobject{currentmarker}{}%
\end{pgfscope}%
\end{pgfscope}%
\begin{pgfscope}%
\pgfsetbuttcap%
\pgfsetroundjoin%
\definecolor{currentfill}{rgb}{0.000000,0.000000,0.000000}%
\pgfsetfillcolor{currentfill}%
\pgfsetlinewidth{0.602250pt}%
\definecolor{currentstroke}{rgb}{0.000000,0.000000,0.000000}%
\pgfsetstrokecolor{currentstroke}%
\pgfsetdash{}{0pt}%
\pgfsys@defobject{currentmarker}{\pgfqpoint{-0.027778in}{0.000000in}}{\pgfqpoint{-0.000000in}{0.000000in}}{%
\pgfpathmoveto{\pgfqpoint{-0.000000in}{0.000000in}}%
\pgfpathlineto{\pgfqpoint{-0.027778in}{0.000000in}}%
\pgfusepath{stroke,fill}%
}%
\begin{pgfscope}%
\pgfsys@transformshift{0.708220in}{0.746963in}%
\pgfsys@useobject{currentmarker}{}%
\end{pgfscope}%
\end{pgfscope}%
\begin{pgfscope}%
\pgfsetbuttcap%
\pgfsetroundjoin%
\definecolor{currentfill}{rgb}{0.000000,0.000000,0.000000}%
\pgfsetfillcolor{currentfill}%
\pgfsetlinewidth{0.602250pt}%
\definecolor{currentstroke}{rgb}{0.000000,0.000000,0.000000}%
\pgfsetstrokecolor{currentstroke}%
\pgfsetdash{}{0pt}%
\pgfsys@defobject{currentmarker}{\pgfqpoint{-0.027778in}{0.000000in}}{\pgfqpoint{-0.000000in}{0.000000in}}{%
\pgfpathmoveto{\pgfqpoint{-0.000000in}{0.000000in}}%
\pgfpathlineto{\pgfqpoint{-0.027778in}{0.000000in}}%
\pgfusepath{stroke,fill}%
}%
\begin{pgfscope}%
\pgfsys@transformshift{0.708220in}{0.780949in}%
\pgfsys@useobject{currentmarker}{}%
\end{pgfscope}%
\end{pgfscope}%
\begin{pgfscope}%
\pgfsetbuttcap%
\pgfsetroundjoin%
\definecolor{currentfill}{rgb}{0.000000,0.000000,0.000000}%
\pgfsetfillcolor{currentfill}%
\pgfsetlinewidth{0.602250pt}%
\definecolor{currentstroke}{rgb}{0.000000,0.000000,0.000000}%
\pgfsetstrokecolor{currentstroke}%
\pgfsetdash{}{0pt}%
\pgfsys@defobject{currentmarker}{\pgfqpoint{-0.027778in}{0.000000in}}{\pgfqpoint{-0.000000in}{0.000000in}}{%
\pgfpathmoveto{\pgfqpoint{-0.000000in}{0.000000in}}%
\pgfpathlineto{\pgfqpoint{-0.027778in}{0.000000in}}%
\pgfusepath{stroke,fill}%
}%
\begin{pgfscope}%
\pgfsys@transformshift{0.708220in}{0.808718in}%
\pgfsys@useobject{currentmarker}{}%
\end{pgfscope}%
\end{pgfscope}%
\begin{pgfscope}%
\pgfsetbuttcap%
\pgfsetroundjoin%
\definecolor{currentfill}{rgb}{0.000000,0.000000,0.000000}%
\pgfsetfillcolor{currentfill}%
\pgfsetlinewidth{0.602250pt}%
\definecolor{currentstroke}{rgb}{0.000000,0.000000,0.000000}%
\pgfsetstrokecolor{currentstroke}%
\pgfsetdash{}{0pt}%
\pgfsys@defobject{currentmarker}{\pgfqpoint{-0.027778in}{0.000000in}}{\pgfqpoint{-0.000000in}{0.000000in}}{%
\pgfpathmoveto{\pgfqpoint{-0.000000in}{0.000000in}}%
\pgfpathlineto{\pgfqpoint{-0.027778in}{0.000000in}}%
\pgfusepath{stroke,fill}%
}%
\begin{pgfscope}%
\pgfsys@transformshift{0.708220in}{0.832196in}%
\pgfsys@useobject{currentmarker}{}%
\end{pgfscope}%
\end{pgfscope}%
\begin{pgfscope}%
\pgfsetbuttcap%
\pgfsetroundjoin%
\definecolor{currentfill}{rgb}{0.000000,0.000000,0.000000}%
\pgfsetfillcolor{currentfill}%
\pgfsetlinewidth{0.602250pt}%
\definecolor{currentstroke}{rgb}{0.000000,0.000000,0.000000}%
\pgfsetstrokecolor{currentstroke}%
\pgfsetdash{}{0pt}%
\pgfsys@defobject{currentmarker}{\pgfqpoint{-0.027778in}{0.000000in}}{\pgfqpoint{-0.000000in}{0.000000in}}{%
\pgfpathmoveto{\pgfqpoint{-0.000000in}{0.000000in}}%
\pgfpathlineto{\pgfqpoint{-0.027778in}{0.000000in}}%
\pgfusepath{stroke,fill}%
}%
\begin{pgfscope}%
\pgfsys@transformshift{0.708220in}{0.852533in}%
\pgfsys@useobject{currentmarker}{}%
\end{pgfscope}%
\end{pgfscope}%
\begin{pgfscope}%
\pgfsetbuttcap%
\pgfsetroundjoin%
\definecolor{currentfill}{rgb}{0.000000,0.000000,0.000000}%
\pgfsetfillcolor{currentfill}%
\pgfsetlinewidth{0.602250pt}%
\definecolor{currentstroke}{rgb}{0.000000,0.000000,0.000000}%
\pgfsetstrokecolor{currentstroke}%
\pgfsetdash{}{0pt}%
\pgfsys@defobject{currentmarker}{\pgfqpoint{-0.027778in}{0.000000in}}{\pgfqpoint{-0.000000in}{0.000000in}}{%
\pgfpathmoveto{\pgfqpoint{-0.000000in}{0.000000in}}%
\pgfpathlineto{\pgfqpoint{-0.027778in}{0.000000in}}%
\pgfusepath{stroke,fill}%
}%
\begin{pgfscope}%
\pgfsys@transformshift{0.708220in}{0.870472in}%
\pgfsys@useobject{currentmarker}{}%
\end{pgfscope}%
\end{pgfscope}%
\begin{pgfscope}%
\pgfsetbuttcap%
\pgfsetroundjoin%
\definecolor{currentfill}{rgb}{0.000000,0.000000,0.000000}%
\pgfsetfillcolor{currentfill}%
\pgfsetlinewidth{0.602250pt}%
\definecolor{currentstroke}{rgb}{0.000000,0.000000,0.000000}%
\pgfsetstrokecolor{currentstroke}%
\pgfsetdash{}{0pt}%
\pgfsys@defobject{currentmarker}{\pgfqpoint{-0.027778in}{0.000000in}}{\pgfqpoint{-0.000000in}{0.000000in}}{%
\pgfpathmoveto{\pgfqpoint{-0.000000in}{0.000000in}}%
\pgfpathlineto{\pgfqpoint{-0.027778in}{0.000000in}}%
\pgfusepath{stroke,fill}%
}%
\begin{pgfscope}%
\pgfsys@transformshift{0.708220in}{0.992089in}%
\pgfsys@useobject{currentmarker}{}%
\end{pgfscope}%
\end{pgfscope}%
\begin{pgfscope}%
\pgfsetbuttcap%
\pgfsetroundjoin%
\definecolor{currentfill}{rgb}{0.000000,0.000000,0.000000}%
\pgfsetfillcolor{currentfill}%
\pgfsetlinewidth{0.602250pt}%
\definecolor{currentstroke}{rgb}{0.000000,0.000000,0.000000}%
\pgfsetstrokecolor{currentstroke}%
\pgfsetdash{}{0pt}%
\pgfsys@defobject{currentmarker}{\pgfqpoint{-0.027778in}{0.000000in}}{\pgfqpoint{-0.000000in}{0.000000in}}{%
\pgfpathmoveto{\pgfqpoint{-0.000000in}{0.000000in}}%
\pgfpathlineto{\pgfqpoint{-0.027778in}{0.000000in}}%
\pgfusepath{stroke,fill}%
}%
\begin{pgfscope}%
\pgfsys@transformshift{0.708220in}{1.053844in}%
\pgfsys@useobject{currentmarker}{}%
\end{pgfscope}%
\end{pgfscope}%
\begin{pgfscope}%
\pgfsetbuttcap%
\pgfsetroundjoin%
\definecolor{currentfill}{rgb}{0.000000,0.000000,0.000000}%
\pgfsetfillcolor{currentfill}%
\pgfsetlinewidth{0.602250pt}%
\definecolor{currentstroke}{rgb}{0.000000,0.000000,0.000000}%
\pgfsetstrokecolor{currentstroke}%
\pgfsetdash{}{0pt}%
\pgfsys@defobject{currentmarker}{\pgfqpoint{-0.027778in}{0.000000in}}{\pgfqpoint{-0.000000in}{0.000000in}}{%
\pgfpathmoveto{\pgfqpoint{-0.000000in}{0.000000in}}%
\pgfpathlineto{\pgfqpoint{-0.027778in}{0.000000in}}%
\pgfusepath{stroke,fill}%
}%
\begin{pgfscope}%
\pgfsys@transformshift{0.708220in}{1.097659in}%
\pgfsys@useobject{currentmarker}{}%
\end{pgfscope}%
\end{pgfscope}%
\begin{pgfscope}%
\pgfsetbuttcap%
\pgfsetroundjoin%
\definecolor{currentfill}{rgb}{0.000000,0.000000,0.000000}%
\pgfsetfillcolor{currentfill}%
\pgfsetlinewidth{0.602250pt}%
\definecolor{currentstroke}{rgb}{0.000000,0.000000,0.000000}%
\pgfsetstrokecolor{currentstroke}%
\pgfsetdash{}{0pt}%
\pgfsys@defobject{currentmarker}{\pgfqpoint{-0.027778in}{0.000000in}}{\pgfqpoint{-0.000000in}{0.000000in}}{%
\pgfpathmoveto{\pgfqpoint{-0.000000in}{0.000000in}}%
\pgfpathlineto{\pgfqpoint{-0.027778in}{0.000000in}}%
\pgfusepath{stroke,fill}%
}%
\begin{pgfscope}%
\pgfsys@transformshift{0.708220in}{1.131645in}%
\pgfsys@useobject{currentmarker}{}%
\end{pgfscope}%
\end{pgfscope}%
\begin{pgfscope}%
\pgfsetbuttcap%
\pgfsetroundjoin%
\definecolor{currentfill}{rgb}{0.000000,0.000000,0.000000}%
\pgfsetfillcolor{currentfill}%
\pgfsetlinewidth{0.602250pt}%
\definecolor{currentstroke}{rgb}{0.000000,0.000000,0.000000}%
\pgfsetstrokecolor{currentstroke}%
\pgfsetdash{}{0pt}%
\pgfsys@defobject{currentmarker}{\pgfqpoint{-0.027778in}{0.000000in}}{\pgfqpoint{-0.000000in}{0.000000in}}{%
\pgfpathmoveto{\pgfqpoint{-0.000000in}{0.000000in}}%
\pgfpathlineto{\pgfqpoint{-0.027778in}{0.000000in}}%
\pgfusepath{stroke,fill}%
}%
\begin{pgfscope}%
\pgfsys@transformshift{0.708220in}{1.159414in}%
\pgfsys@useobject{currentmarker}{}%
\end{pgfscope}%
\end{pgfscope}%
\begin{pgfscope}%
\pgfsetbuttcap%
\pgfsetroundjoin%
\definecolor{currentfill}{rgb}{0.000000,0.000000,0.000000}%
\pgfsetfillcolor{currentfill}%
\pgfsetlinewidth{0.602250pt}%
\definecolor{currentstroke}{rgb}{0.000000,0.000000,0.000000}%
\pgfsetstrokecolor{currentstroke}%
\pgfsetdash{}{0pt}%
\pgfsys@defobject{currentmarker}{\pgfqpoint{-0.027778in}{0.000000in}}{\pgfqpoint{-0.000000in}{0.000000in}}{%
\pgfpathmoveto{\pgfqpoint{-0.000000in}{0.000000in}}%
\pgfpathlineto{\pgfqpoint{-0.027778in}{0.000000in}}%
\pgfusepath{stroke,fill}%
}%
\begin{pgfscope}%
\pgfsys@transformshift{0.708220in}{1.182892in}%
\pgfsys@useobject{currentmarker}{}%
\end{pgfscope}%
\end{pgfscope}%
\begin{pgfscope}%
\pgfsetbuttcap%
\pgfsetroundjoin%
\definecolor{currentfill}{rgb}{0.000000,0.000000,0.000000}%
\pgfsetfillcolor{currentfill}%
\pgfsetlinewidth{0.602250pt}%
\definecolor{currentstroke}{rgb}{0.000000,0.000000,0.000000}%
\pgfsetstrokecolor{currentstroke}%
\pgfsetdash{}{0pt}%
\pgfsys@defobject{currentmarker}{\pgfqpoint{-0.027778in}{0.000000in}}{\pgfqpoint{-0.000000in}{0.000000in}}{%
\pgfpathmoveto{\pgfqpoint{-0.000000in}{0.000000in}}%
\pgfpathlineto{\pgfqpoint{-0.027778in}{0.000000in}}%
\pgfusepath{stroke,fill}%
}%
\begin{pgfscope}%
\pgfsys@transformshift{0.708220in}{1.203230in}%
\pgfsys@useobject{currentmarker}{}%
\end{pgfscope}%
\end{pgfscope}%
\begin{pgfscope}%
\pgfsetbuttcap%
\pgfsetroundjoin%
\definecolor{currentfill}{rgb}{0.000000,0.000000,0.000000}%
\pgfsetfillcolor{currentfill}%
\pgfsetlinewidth{0.602250pt}%
\definecolor{currentstroke}{rgb}{0.000000,0.000000,0.000000}%
\pgfsetstrokecolor{currentstroke}%
\pgfsetdash{}{0pt}%
\pgfsys@defobject{currentmarker}{\pgfqpoint{-0.027778in}{0.000000in}}{\pgfqpoint{-0.000000in}{0.000000in}}{%
\pgfpathmoveto{\pgfqpoint{-0.000000in}{0.000000in}}%
\pgfpathlineto{\pgfqpoint{-0.027778in}{0.000000in}}%
\pgfusepath{stroke,fill}%
}%
\begin{pgfscope}%
\pgfsys@transformshift{0.708220in}{1.221169in}%
\pgfsys@useobject{currentmarker}{}%
\end{pgfscope}%
\end{pgfscope}%
\begin{pgfscope}%
\pgfsetbuttcap%
\pgfsetroundjoin%
\definecolor{currentfill}{rgb}{0.000000,0.000000,0.000000}%
\pgfsetfillcolor{currentfill}%
\pgfsetlinewidth{0.602250pt}%
\definecolor{currentstroke}{rgb}{0.000000,0.000000,0.000000}%
\pgfsetstrokecolor{currentstroke}%
\pgfsetdash{}{0pt}%
\pgfsys@defobject{currentmarker}{\pgfqpoint{-0.027778in}{0.000000in}}{\pgfqpoint{-0.000000in}{0.000000in}}{%
\pgfpathmoveto{\pgfqpoint{-0.000000in}{0.000000in}}%
\pgfpathlineto{\pgfqpoint{-0.027778in}{0.000000in}}%
\pgfusepath{stroke,fill}%
}%
\begin{pgfscope}%
\pgfsys@transformshift{0.708220in}{1.342786in}%
\pgfsys@useobject{currentmarker}{}%
\end{pgfscope}%
\end{pgfscope}%
\begin{pgfscope}%
\pgfsetbuttcap%
\pgfsetroundjoin%
\definecolor{currentfill}{rgb}{0.000000,0.000000,0.000000}%
\pgfsetfillcolor{currentfill}%
\pgfsetlinewidth{0.602250pt}%
\definecolor{currentstroke}{rgb}{0.000000,0.000000,0.000000}%
\pgfsetstrokecolor{currentstroke}%
\pgfsetdash{}{0pt}%
\pgfsys@defobject{currentmarker}{\pgfqpoint{-0.027778in}{0.000000in}}{\pgfqpoint{-0.000000in}{0.000000in}}{%
\pgfpathmoveto{\pgfqpoint{-0.000000in}{0.000000in}}%
\pgfpathlineto{\pgfqpoint{-0.027778in}{0.000000in}}%
\pgfusepath{stroke,fill}%
}%
\begin{pgfscope}%
\pgfsys@transformshift{0.708220in}{1.404540in}%
\pgfsys@useobject{currentmarker}{}%
\end{pgfscope}%
\end{pgfscope}%
\begin{pgfscope}%
\pgfsetbuttcap%
\pgfsetroundjoin%
\definecolor{currentfill}{rgb}{0.000000,0.000000,0.000000}%
\pgfsetfillcolor{currentfill}%
\pgfsetlinewidth{0.602250pt}%
\definecolor{currentstroke}{rgb}{0.000000,0.000000,0.000000}%
\pgfsetstrokecolor{currentstroke}%
\pgfsetdash{}{0pt}%
\pgfsys@defobject{currentmarker}{\pgfqpoint{-0.027778in}{0.000000in}}{\pgfqpoint{-0.000000in}{0.000000in}}{%
\pgfpathmoveto{\pgfqpoint{-0.000000in}{0.000000in}}%
\pgfpathlineto{\pgfqpoint{-0.027778in}{0.000000in}}%
\pgfusepath{stroke,fill}%
}%
\begin{pgfscope}%
\pgfsys@transformshift{0.708220in}{1.448356in}%
\pgfsys@useobject{currentmarker}{}%
\end{pgfscope}%
\end{pgfscope}%
\begin{pgfscope}%
\pgfsetbuttcap%
\pgfsetroundjoin%
\definecolor{currentfill}{rgb}{0.000000,0.000000,0.000000}%
\pgfsetfillcolor{currentfill}%
\pgfsetlinewidth{0.602250pt}%
\definecolor{currentstroke}{rgb}{0.000000,0.000000,0.000000}%
\pgfsetstrokecolor{currentstroke}%
\pgfsetdash{}{0pt}%
\pgfsys@defobject{currentmarker}{\pgfqpoint{-0.027778in}{0.000000in}}{\pgfqpoint{-0.000000in}{0.000000in}}{%
\pgfpathmoveto{\pgfqpoint{-0.000000in}{0.000000in}}%
\pgfpathlineto{\pgfqpoint{-0.027778in}{0.000000in}}%
\pgfusepath{stroke,fill}%
}%
\begin{pgfscope}%
\pgfsys@transformshift{0.708220in}{1.482342in}%
\pgfsys@useobject{currentmarker}{}%
\end{pgfscope}%
\end{pgfscope}%
\begin{pgfscope}%
\pgfsetbuttcap%
\pgfsetroundjoin%
\definecolor{currentfill}{rgb}{0.000000,0.000000,0.000000}%
\pgfsetfillcolor{currentfill}%
\pgfsetlinewidth{0.602250pt}%
\definecolor{currentstroke}{rgb}{0.000000,0.000000,0.000000}%
\pgfsetstrokecolor{currentstroke}%
\pgfsetdash{}{0pt}%
\pgfsys@defobject{currentmarker}{\pgfqpoint{-0.027778in}{0.000000in}}{\pgfqpoint{-0.000000in}{0.000000in}}{%
\pgfpathmoveto{\pgfqpoint{-0.000000in}{0.000000in}}%
\pgfpathlineto{\pgfqpoint{-0.027778in}{0.000000in}}%
\pgfusepath{stroke,fill}%
}%
\begin{pgfscope}%
\pgfsys@transformshift{0.708220in}{1.510110in}%
\pgfsys@useobject{currentmarker}{}%
\end{pgfscope}%
\end{pgfscope}%
\begin{pgfscope}%
\pgfsetbuttcap%
\pgfsetroundjoin%
\definecolor{currentfill}{rgb}{0.000000,0.000000,0.000000}%
\pgfsetfillcolor{currentfill}%
\pgfsetlinewidth{0.602250pt}%
\definecolor{currentstroke}{rgb}{0.000000,0.000000,0.000000}%
\pgfsetstrokecolor{currentstroke}%
\pgfsetdash{}{0pt}%
\pgfsys@defobject{currentmarker}{\pgfqpoint{-0.027778in}{0.000000in}}{\pgfqpoint{-0.000000in}{0.000000in}}{%
\pgfpathmoveto{\pgfqpoint{-0.000000in}{0.000000in}}%
\pgfpathlineto{\pgfqpoint{-0.027778in}{0.000000in}}%
\pgfusepath{stroke,fill}%
}%
\begin{pgfscope}%
\pgfsys@transformshift{0.708220in}{1.533588in}%
\pgfsys@useobject{currentmarker}{}%
\end{pgfscope}%
\end{pgfscope}%
\begin{pgfscope}%
\pgfsetbuttcap%
\pgfsetroundjoin%
\definecolor{currentfill}{rgb}{0.000000,0.000000,0.000000}%
\pgfsetfillcolor{currentfill}%
\pgfsetlinewidth{0.602250pt}%
\definecolor{currentstroke}{rgb}{0.000000,0.000000,0.000000}%
\pgfsetstrokecolor{currentstroke}%
\pgfsetdash{}{0pt}%
\pgfsys@defobject{currentmarker}{\pgfqpoint{-0.027778in}{0.000000in}}{\pgfqpoint{-0.000000in}{0.000000in}}{%
\pgfpathmoveto{\pgfqpoint{-0.000000in}{0.000000in}}%
\pgfpathlineto{\pgfqpoint{-0.027778in}{0.000000in}}%
\pgfusepath{stroke,fill}%
}%
\begin{pgfscope}%
\pgfsys@transformshift{0.708220in}{1.553926in}%
\pgfsys@useobject{currentmarker}{}%
\end{pgfscope}%
\end{pgfscope}%
\begin{pgfscope}%
\pgfsetbuttcap%
\pgfsetroundjoin%
\definecolor{currentfill}{rgb}{0.000000,0.000000,0.000000}%
\pgfsetfillcolor{currentfill}%
\pgfsetlinewidth{0.602250pt}%
\definecolor{currentstroke}{rgb}{0.000000,0.000000,0.000000}%
\pgfsetstrokecolor{currentstroke}%
\pgfsetdash{}{0pt}%
\pgfsys@defobject{currentmarker}{\pgfqpoint{-0.027778in}{0.000000in}}{\pgfqpoint{-0.000000in}{0.000000in}}{%
\pgfpathmoveto{\pgfqpoint{-0.000000in}{0.000000in}}%
\pgfpathlineto{\pgfqpoint{-0.027778in}{0.000000in}}%
\pgfusepath{stroke,fill}%
}%
\begin{pgfscope}%
\pgfsys@transformshift{0.708220in}{1.571865in}%
\pgfsys@useobject{currentmarker}{}%
\end{pgfscope}%
\end{pgfscope}%
\begin{pgfscope}%
\pgfsetbuttcap%
\pgfsetroundjoin%
\definecolor{currentfill}{rgb}{0.000000,0.000000,0.000000}%
\pgfsetfillcolor{currentfill}%
\pgfsetlinewidth{0.602250pt}%
\definecolor{currentstroke}{rgb}{0.000000,0.000000,0.000000}%
\pgfsetstrokecolor{currentstroke}%
\pgfsetdash{}{0pt}%
\pgfsys@defobject{currentmarker}{\pgfqpoint{-0.027778in}{0.000000in}}{\pgfqpoint{-0.000000in}{0.000000in}}{%
\pgfpathmoveto{\pgfqpoint{-0.000000in}{0.000000in}}%
\pgfpathlineto{\pgfqpoint{-0.027778in}{0.000000in}}%
\pgfusepath{stroke,fill}%
}%
\begin{pgfscope}%
\pgfsys@transformshift{0.708220in}{1.693482in}%
\pgfsys@useobject{currentmarker}{}%
\end{pgfscope}%
\end{pgfscope}%
\begin{pgfscope}%
\pgfsetbuttcap%
\pgfsetroundjoin%
\definecolor{currentfill}{rgb}{0.000000,0.000000,0.000000}%
\pgfsetfillcolor{currentfill}%
\pgfsetlinewidth{0.602250pt}%
\definecolor{currentstroke}{rgb}{0.000000,0.000000,0.000000}%
\pgfsetstrokecolor{currentstroke}%
\pgfsetdash{}{0pt}%
\pgfsys@defobject{currentmarker}{\pgfqpoint{-0.027778in}{0.000000in}}{\pgfqpoint{-0.000000in}{0.000000in}}{%
\pgfpathmoveto{\pgfqpoint{-0.000000in}{0.000000in}}%
\pgfpathlineto{\pgfqpoint{-0.027778in}{0.000000in}}%
\pgfusepath{stroke,fill}%
}%
\begin{pgfscope}%
\pgfsys@transformshift{0.708220in}{1.755237in}%
\pgfsys@useobject{currentmarker}{}%
\end{pgfscope}%
\end{pgfscope}%
\begin{pgfscope}%
\pgfsetbuttcap%
\pgfsetroundjoin%
\definecolor{currentfill}{rgb}{0.000000,0.000000,0.000000}%
\pgfsetfillcolor{currentfill}%
\pgfsetlinewidth{0.602250pt}%
\definecolor{currentstroke}{rgb}{0.000000,0.000000,0.000000}%
\pgfsetstrokecolor{currentstroke}%
\pgfsetdash{}{0pt}%
\pgfsys@defobject{currentmarker}{\pgfqpoint{-0.027778in}{0.000000in}}{\pgfqpoint{-0.000000in}{0.000000in}}{%
\pgfpathmoveto{\pgfqpoint{-0.000000in}{0.000000in}}%
\pgfpathlineto{\pgfqpoint{-0.027778in}{0.000000in}}%
\pgfusepath{stroke,fill}%
}%
\begin{pgfscope}%
\pgfsys@transformshift{0.708220in}{1.799052in}%
\pgfsys@useobject{currentmarker}{}%
\end{pgfscope}%
\end{pgfscope}%
\begin{pgfscope}%
\pgfsetbuttcap%
\pgfsetroundjoin%
\definecolor{currentfill}{rgb}{0.000000,0.000000,0.000000}%
\pgfsetfillcolor{currentfill}%
\pgfsetlinewidth{0.602250pt}%
\definecolor{currentstroke}{rgb}{0.000000,0.000000,0.000000}%
\pgfsetstrokecolor{currentstroke}%
\pgfsetdash{}{0pt}%
\pgfsys@defobject{currentmarker}{\pgfqpoint{-0.027778in}{0.000000in}}{\pgfqpoint{-0.000000in}{0.000000in}}{%
\pgfpathmoveto{\pgfqpoint{-0.000000in}{0.000000in}}%
\pgfpathlineto{\pgfqpoint{-0.027778in}{0.000000in}}%
\pgfusepath{stroke,fill}%
}%
\begin{pgfscope}%
\pgfsys@transformshift{0.708220in}{1.833038in}%
\pgfsys@useobject{currentmarker}{}%
\end{pgfscope}%
\end{pgfscope}%
\begin{pgfscope}%
\pgfsetbuttcap%
\pgfsetroundjoin%
\definecolor{currentfill}{rgb}{0.000000,0.000000,0.000000}%
\pgfsetfillcolor{currentfill}%
\pgfsetlinewidth{0.602250pt}%
\definecolor{currentstroke}{rgb}{0.000000,0.000000,0.000000}%
\pgfsetstrokecolor{currentstroke}%
\pgfsetdash{}{0pt}%
\pgfsys@defobject{currentmarker}{\pgfqpoint{-0.027778in}{0.000000in}}{\pgfqpoint{-0.000000in}{0.000000in}}{%
\pgfpathmoveto{\pgfqpoint{-0.000000in}{0.000000in}}%
\pgfpathlineto{\pgfqpoint{-0.027778in}{0.000000in}}%
\pgfusepath{stroke,fill}%
}%
\begin{pgfscope}%
\pgfsys@transformshift{0.708220in}{1.860807in}%
\pgfsys@useobject{currentmarker}{}%
\end{pgfscope}%
\end{pgfscope}%
\begin{pgfscope}%
\pgfsetbuttcap%
\pgfsetroundjoin%
\definecolor{currentfill}{rgb}{0.000000,0.000000,0.000000}%
\pgfsetfillcolor{currentfill}%
\pgfsetlinewidth{0.602250pt}%
\definecolor{currentstroke}{rgb}{0.000000,0.000000,0.000000}%
\pgfsetstrokecolor{currentstroke}%
\pgfsetdash{}{0pt}%
\pgfsys@defobject{currentmarker}{\pgfqpoint{-0.027778in}{0.000000in}}{\pgfqpoint{-0.000000in}{0.000000in}}{%
\pgfpathmoveto{\pgfqpoint{-0.000000in}{0.000000in}}%
\pgfpathlineto{\pgfqpoint{-0.027778in}{0.000000in}}%
\pgfusepath{stroke,fill}%
}%
\begin{pgfscope}%
\pgfsys@transformshift{0.708220in}{1.884285in}%
\pgfsys@useobject{currentmarker}{}%
\end{pgfscope}%
\end{pgfscope}%
\begin{pgfscope}%
\pgfsetbuttcap%
\pgfsetroundjoin%
\definecolor{currentfill}{rgb}{0.000000,0.000000,0.000000}%
\pgfsetfillcolor{currentfill}%
\pgfsetlinewidth{0.602250pt}%
\definecolor{currentstroke}{rgb}{0.000000,0.000000,0.000000}%
\pgfsetstrokecolor{currentstroke}%
\pgfsetdash{}{0pt}%
\pgfsys@defobject{currentmarker}{\pgfqpoint{-0.027778in}{0.000000in}}{\pgfqpoint{-0.000000in}{0.000000in}}{%
\pgfpathmoveto{\pgfqpoint{-0.000000in}{0.000000in}}%
\pgfpathlineto{\pgfqpoint{-0.027778in}{0.000000in}}%
\pgfusepath{stroke,fill}%
}%
\begin{pgfscope}%
\pgfsys@transformshift{0.708220in}{1.904623in}%
\pgfsys@useobject{currentmarker}{}%
\end{pgfscope}%
\end{pgfscope}%
\begin{pgfscope}%
\pgfsetbuttcap%
\pgfsetroundjoin%
\definecolor{currentfill}{rgb}{0.000000,0.000000,0.000000}%
\pgfsetfillcolor{currentfill}%
\pgfsetlinewidth{0.602250pt}%
\definecolor{currentstroke}{rgb}{0.000000,0.000000,0.000000}%
\pgfsetstrokecolor{currentstroke}%
\pgfsetdash{}{0pt}%
\pgfsys@defobject{currentmarker}{\pgfqpoint{-0.027778in}{0.000000in}}{\pgfqpoint{-0.000000in}{0.000000in}}{%
\pgfpathmoveto{\pgfqpoint{-0.000000in}{0.000000in}}%
\pgfpathlineto{\pgfqpoint{-0.027778in}{0.000000in}}%
\pgfusepath{stroke,fill}%
}%
\begin{pgfscope}%
\pgfsys@transformshift{0.708220in}{1.922562in}%
\pgfsys@useobject{currentmarker}{}%
\end{pgfscope}%
\end{pgfscope}%
\begin{pgfscope}%
\definecolor{textcolor}{rgb}{0.000000,0.000000,0.000000}%
\pgfsetstrokecolor{textcolor}%
\pgfsetfillcolor{textcolor}%
\pgftext[x=0.288855in,y=1.237216in,,bottom,rotate=90.000000]{\color{textcolor}\rmfamily\fontsize{10.000000}{12.000000}\selectfont Longest solving time (s)}%
\end{pgfscope}%
\begin{pgfscope}%
\pgfpathrectangle{\pgfqpoint{0.708220in}{0.535823in}}{\pgfqpoint{5.141780in}{1.402786in}}%
\pgfusepath{clip}%
\pgfsetbuttcap%
\pgfsetroundjoin%
\pgfsetlinewidth{2.007500pt}%
\definecolor{currentstroke}{rgb}{1.000000,0.843137,0.000000}%
\pgfsetstrokecolor{currentstroke}%
\pgfsetdash{{7.400000pt}{3.200000pt}}{0.000000pt}%
\pgfpathmoveto{\pgfqpoint{0.708220in}{0.804867in}}%
\pgfpathlineto{\pgfqpoint{0.712933in}{0.811364in}}%
\pgfpathlineto{\pgfqpoint{0.717646in}{0.831637in}}%
\pgfpathlineto{\pgfqpoint{0.727071in}{0.834287in}}%
\pgfpathlineto{\pgfqpoint{0.736497in}{0.840208in}}%
\pgfpathlineto{\pgfqpoint{0.745923in}{0.841367in}}%
\pgfpathlineto{\pgfqpoint{0.750636in}{0.844515in}}%
\pgfpathlineto{\pgfqpoint{0.760062in}{0.845068in}}%
\pgfpathlineto{\pgfqpoint{0.764775in}{0.865336in}}%
\pgfpathlineto{\pgfqpoint{0.769488in}{0.874462in}}%
\pgfpathlineto{\pgfqpoint{0.774201in}{0.875631in}}%
\pgfpathlineto{\pgfqpoint{0.783626in}{0.879473in}}%
\pgfpathlineto{\pgfqpoint{0.788339in}{0.885155in}}%
\pgfpathlineto{\pgfqpoint{0.802478in}{0.885844in}}%
\pgfpathlineto{\pgfqpoint{0.807191in}{0.890406in}}%
\pgfpathlineto{\pgfqpoint{0.816617in}{0.915721in}}%
\pgfpathlineto{\pgfqpoint{0.821330in}{0.918298in}}%
\pgfpathlineto{\pgfqpoint{0.826043in}{0.933315in}}%
\pgfpathlineto{\pgfqpoint{0.835468in}{0.938470in}}%
\pgfpathlineto{\pgfqpoint{0.859033in}{0.942858in}}%
\pgfpathlineto{\pgfqpoint{0.863746in}{0.943522in}}%
\pgfpathlineto{\pgfqpoint{0.877884in}{0.952218in}}%
\pgfpathlineto{\pgfqpoint{0.882597in}{0.957360in}}%
\pgfpathlineto{\pgfqpoint{0.887310in}{0.957681in}}%
\pgfpathlineto{\pgfqpoint{0.892023in}{0.964052in}}%
\pgfpathlineto{\pgfqpoint{0.896736in}{0.967822in}}%
\pgfpathlineto{\pgfqpoint{0.901449in}{0.969220in}}%
\pgfpathlineto{\pgfqpoint{0.906162in}{0.975209in}}%
\pgfpathlineto{\pgfqpoint{0.910875in}{0.977628in}}%
\pgfpathlineto{\pgfqpoint{0.929726in}{0.981840in}}%
\pgfpathlineto{\pgfqpoint{0.939152in}{0.987797in}}%
\pgfpathlineto{\pgfqpoint{0.943865in}{0.999784in}}%
\pgfpathlineto{\pgfqpoint{0.948578in}{1.003030in}}%
\pgfpathlineto{\pgfqpoint{0.953291in}{1.003880in}}%
\pgfpathlineto{\pgfqpoint{0.958004in}{1.010464in}}%
\pgfpathlineto{\pgfqpoint{0.972143in}{1.011530in}}%
\pgfpathlineto{\pgfqpoint{0.976855in}{1.012691in}}%
\pgfpathlineto{\pgfqpoint{0.981568in}{1.015839in}}%
\pgfpathlineto{\pgfqpoint{0.990994in}{1.030132in}}%
\pgfpathlineto{\pgfqpoint{1.014559in}{1.037065in}}%
\pgfpathlineto{\pgfqpoint{1.019272in}{1.044156in}}%
\pgfpathlineto{\pgfqpoint{1.023985in}{1.045337in}}%
\pgfpathlineto{\pgfqpoint{1.028697in}{1.047803in}}%
\pgfpathlineto{\pgfqpoint{1.038123in}{1.049838in}}%
\pgfpathlineto{\pgfqpoint{1.052262in}{1.051029in}}%
\pgfpathlineto{\pgfqpoint{1.056975in}{1.051698in}}%
\pgfpathlineto{\pgfqpoint{1.061688in}{1.064284in}}%
\pgfpathlineto{\pgfqpoint{1.071114in}{1.065226in}}%
\pgfpathlineto{\pgfqpoint{1.080539in}{1.065429in}}%
\pgfpathlineto{\pgfqpoint{1.089965in}{1.067327in}}%
\pgfpathlineto{\pgfqpoint{1.104104in}{1.068796in}}%
\pgfpathlineto{\pgfqpoint{1.113530in}{1.070373in}}%
\pgfpathlineto{\pgfqpoint{1.141807in}{1.080372in}}%
\pgfpathlineto{\pgfqpoint{1.146520in}{1.081309in}}%
\pgfpathlineto{\pgfqpoint{1.151233in}{1.094518in}}%
\pgfpathlineto{\pgfqpoint{1.179510in}{1.097106in}}%
\pgfpathlineto{\pgfqpoint{1.188936in}{1.099165in}}%
\pgfpathlineto{\pgfqpoint{1.193649in}{1.101471in}}%
\pgfpathlineto{\pgfqpoint{1.207788in}{1.103872in}}%
\pgfpathlineto{\pgfqpoint{1.212501in}{1.105964in}}%
\pgfpathlineto{\pgfqpoint{1.236065in}{1.108802in}}%
\pgfpathlineto{\pgfqpoint{1.240778in}{1.110804in}}%
\pgfpathlineto{\pgfqpoint{1.245491in}{1.111441in}}%
\pgfpathlineto{\pgfqpoint{1.250204in}{1.120884in}}%
\pgfpathlineto{\pgfqpoint{1.254917in}{1.122892in}}%
\pgfpathlineto{\pgfqpoint{1.259630in}{1.129104in}}%
\pgfpathlineto{\pgfqpoint{1.269056in}{1.129525in}}%
\pgfpathlineto{\pgfqpoint{1.278481in}{1.130919in}}%
\pgfpathlineto{\pgfqpoint{1.283194in}{1.131218in}}%
\pgfpathlineto{\pgfqpoint{1.287907in}{1.134033in}}%
\pgfpathlineto{\pgfqpoint{1.302046in}{1.136633in}}%
\pgfpathlineto{\pgfqpoint{1.306759in}{1.137047in}}%
\pgfpathlineto{\pgfqpoint{1.311472in}{1.142669in}}%
\pgfpathlineto{\pgfqpoint{1.335036in}{1.144745in}}%
\pgfpathlineto{\pgfqpoint{1.344462in}{1.145385in}}%
\pgfpathlineto{\pgfqpoint{1.353888in}{1.146205in}}%
\pgfpathlineto{\pgfqpoint{1.358601in}{1.146689in}}%
\pgfpathlineto{\pgfqpoint{1.363314in}{1.163240in}}%
\pgfpathlineto{\pgfqpoint{1.368027in}{1.163521in}}%
\pgfpathlineto{\pgfqpoint{1.382165in}{1.167324in}}%
\pgfpathlineto{\pgfqpoint{1.386878in}{1.167958in}}%
\pgfpathlineto{\pgfqpoint{1.410443in}{1.177434in}}%
\pgfpathlineto{\pgfqpoint{1.415156in}{1.185222in}}%
\pgfpathlineto{\pgfqpoint{1.419869in}{1.189278in}}%
\pgfpathlineto{\pgfqpoint{1.424582in}{1.189326in}}%
\pgfpathlineto{\pgfqpoint{1.438720in}{1.195905in}}%
\pgfpathlineto{\pgfqpoint{1.443433in}{1.201637in}}%
\pgfpathlineto{\pgfqpoint{1.457572in}{1.206980in}}%
\pgfpathlineto{\pgfqpoint{1.471711in}{1.209000in}}%
\pgfpathlineto{\pgfqpoint{1.476423in}{1.210578in}}%
\pgfpathlineto{\pgfqpoint{1.481136in}{1.222256in}}%
\pgfpathlineto{\pgfqpoint{1.490562in}{1.225555in}}%
\pgfpathlineto{\pgfqpoint{1.495275in}{1.227079in}}%
\pgfpathlineto{\pgfqpoint{1.499988in}{1.232578in}}%
\pgfpathlineto{\pgfqpoint{1.504701in}{1.234960in}}%
\pgfpathlineto{\pgfqpoint{1.509414in}{1.244768in}}%
\pgfpathlineto{\pgfqpoint{1.514127in}{1.247995in}}%
\pgfpathlineto{\pgfqpoint{1.518840in}{1.248179in}}%
\pgfpathlineto{\pgfqpoint{1.528265in}{1.253981in}}%
\pgfpathlineto{\pgfqpoint{1.532978in}{1.254664in}}%
\pgfpathlineto{\pgfqpoint{1.537691in}{1.266334in}}%
\pgfpathlineto{\pgfqpoint{1.542404in}{1.270138in}}%
\pgfpathlineto{\pgfqpoint{1.547117in}{1.279001in}}%
\pgfpathlineto{\pgfqpoint{1.551830in}{1.305786in}}%
\pgfpathlineto{\pgfqpoint{1.556543in}{1.310741in}}%
\pgfpathlineto{\pgfqpoint{1.565969in}{1.316493in}}%
\pgfpathlineto{\pgfqpoint{1.570682in}{1.321949in}}%
\pgfpathlineto{\pgfqpoint{1.575395in}{1.343219in}}%
\pgfpathlineto{\pgfqpoint{1.580107in}{1.346888in}}%
\pgfpathlineto{\pgfqpoint{1.584820in}{1.353518in}}%
\pgfpathlineto{\pgfqpoint{1.594246in}{1.377254in}}%
\pgfpathlineto{\pgfqpoint{1.603672in}{1.481793in}}%
\pgfpathlineto{\pgfqpoint{1.608385in}{1.926740in}}%
\pgfpathlineto{\pgfqpoint{1.613098in}{1.938609in}}%
\pgfpathlineto{\pgfqpoint{5.845287in}{1.938609in}}%
\pgfpathlineto{\pgfqpoint{5.845287in}{1.938609in}}%
\pgfusepath{stroke}%
\end{pgfscope}%
\begin{pgfscope}%
\pgfpathrectangle{\pgfqpoint{0.708220in}{0.535823in}}{\pgfqpoint{5.141780in}{1.402786in}}%
\pgfusepath{clip}%
\pgfsetbuttcap%
\pgfsetroundjoin%
\pgfsetlinewidth{2.007500pt}%
\definecolor{currentstroke}{rgb}{1.000000,0.694118,0.305882}%
\pgfsetstrokecolor{currentstroke}%
\pgfsetdash{{2.000000pt}{3.300000pt}}{0.000000pt}%
\pgfpathmoveto{\pgfqpoint{0.708220in}{0.638220in}}%
\pgfpathlineto{\pgfqpoint{0.712933in}{0.681165in}}%
\pgfpathlineto{\pgfqpoint{0.727071in}{0.684042in}}%
\pgfpathlineto{\pgfqpoint{0.736497in}{0.687239in}}%
\pgfpathlineto{\pgfqpoint{0.745923in}{0.688772in}}%
\pgfpathlineto{\pgfqpoint{0.755349in}{0.689831in}}%
\pgfpathlineto{\pgfqpoint{0.807191in}{0.691885in}}%
\pgfpathlineto{\pgfqpoint{0.826043in}{0.695171in}}%
\pgfpathlineto{\pgfqpoint{0.830755in}{0.695494in}}%
\pgfpathlineto{\pgfqpoint{0.835468in}{0.704959in}}%
\pgfpathlineto{\pgfqpoint{0.840181in}{0.707467in}}%
\pgfpathlineto{\pgfqpoint{0.849607in}{0.708297in}}%
\pgfpathlineto{\pgfqpoint{0.863746in}{0.714191in}}%
\pgfpathlineto{\pgfqpoint{0.868459in}{0.714922in}}%
\pgfpathlineto{\pgfqpoint{0.873172in}{0.716896in}}%
\pgfpathlineto{\pgfqpoint{0.877884in}{0.721525in}}%
\pgfpathlineto{\pgfqpoint{0.882597in}{0.729879in}}%
\pgfpathlineto{\pgfqpoint{0.887310in}{0.732319in}}%
\pgfpathlineto{\pgfqpoint{0.892023in}{0.737005in}}%
\pgfpathlineto{\pgfqpoint{0.906162in}{0.738670in}}%
\pgfpathlineto{\pgfqpoint{0.910875in}{0.741710in}}%
\pgfpathlineto{\pgfqpoint{0.920301in}{0.743051in}}%
\pgfpathlineto{\pgfqpoint{0.929726in}{0.744927in}}%
\pgfpathlineto{\pgfqpoint{0.939152in}{0.746044in}}%
\pgfpathlineto{\pgfqpoint{0.943865in}{0.746754in}}%
\pgfpathlineto{\pgfqpoint{0.948578in}{0.750288in}}%
\pgfpathlineto{\pgfqpoint{0.953291in}{0.750410in}}%
\pgfpathlineto{\pgfqpoint{0.962717in}{0.752401in}}%
\pgfpathlineto{\pgfqpoint{0.967430in}{0.759990in}}%
\pgfpathlineto{\pgfqpoint{0.976855in}{0.760794in}}%
\pgfpathlineto{\pgfqpoint{0.986281in}{0.762443in}}%
\pgfpathlineto{\pgfqpoint{0.990994in}{0.762506in}}%
\pgfpathlineto{\pgfqpoint{0.995707in}{0.764900in}}%
\pgfpathlineto{\pgfqpoint{1.005133in}{0.766574in}}%
\pgfpathlineto{\pgfqpoint{1.009846in}{0.771675in}}%
\pgfpathlineto{\pgfqpoint{1.019272in}{0.773230in}}%
\pgfpathlineto{\pgfqpoint{1.023985in}{0.791628in}}%
\pgfpathlineto{\pgfqpoint{1.033410in}{0.794987in}}%
\pgfpathlineto{\pgfqpoint{1.047549in}{0.795796in}}%
\pgfpathlineto{\pgfqpoint{1.052262in}{0.798314in}}%
\pgfpathlineto{\pgfqpoint{1.056975in}{0.798459in}}%
\pgfpathlineto{\pgfqpoint{1.061688in}{0.800834in}}%
\pgfpathlineto{\pgfqpoint{1.071114in}{0.800961in}}%
\pgfpathlineto{\pgfqpoint{1.085252in}{0.805707in}}%
\pgfpathlineto{\pgfqpoint{1.094678in}{0.806609in}}%
\pgfpathlineto{\pgfqpoint{1.099391in}{0.808268in}}%
\pgfpathlineto{\pgfqpoint{1.113530in}{0.809077in}}%
\pgfpathlineto{\pgfqpoint{1.127668in}{0.813423in}}%
\pgfpathlineto{\pgfqpoint{1.170085in}{0.816536in}}%
\pgfpathlineto{\pgfqpoint{1.174798in}{0.819410in}}%
\pgfpathlineto{\pgfqpoint{1.188936in}{0.820587in}}%
\pgfpathlineto{\pgfqpoint{1.193649in}{0.822004in}}%
\pgfpathlineto{\pgfqpoint{1.203075in}{0.829858in}}%
\pgfpathlineto{\pgfqpoint{1.217214in}{0.830967in}}%
\pgfpathlineto{\pgfqpoint{1.221927in}{0.832756in}}%
\pgfpathlineto{\pgfqpoint{1.236065in}{0.834398in}}%
\pgfpathlineto{\pgfqpoint{1.240778in}{0.839396in}}%
\pgfpathlineto{\pgfqpoint{1.250204in}{0.840337in}}%
\pgfpathlineto{\pgfqpoint{1.259630in}{0.844791in}}%
\pgfpathlineto{\pgfqpoint{1.283194in}{0.850383in}}%
\pgfpathlineto{\pgfqpoint{1.287907in}{0.852429in}}%
\pgfpathlineto{\pgfqpoint{1.292620in}{0.852559in}}%
\pgfpathlineto{\pgfqpoint{1.302046in}{0.854419in}}%
\pgfpathlineto{\pgfqpoint{1.311472in}{0.854720in}}%
\pgfpathlineto{\pgfqpoint{1.320898in}{0.858220in}}%
\pgfpathlineto{\pgfqpoint{1.325611in}{0.858309in}}%
\pgfpathlineto{\pgfqpoint{1.330323in}{0.860659in}}%
\pgfpathlineto{\pgfqpoint{1.335036in}{0.882956in}}%
\pgfpathlineto{\pgfqpoint{1.339749in}{0.883520in}}%
\pgfpathlineto{\pgfqpoint{1.344462in}{0.897238in}}%
\pgfpathlineto{\pgfqpoint{1.349175in}{0.901563in}}%
\pgfpathlineto{\pgfqpoint{1.353888in}{0.902090in}}%
\pgfpathlineto{\pgfqpoint{1.368027in}{0.908334in}}%
\pgfpathlineto{\pgfqpoint{1.372740in}{0.910264in}}%
\pgfpathlineto{\pgfqpoint{1.377452in}{0.923929in}}%
\pgfpathlineto{\pgfqpoint{1.396304in}{0.928247in}}%
\pgfpathlineto{\pgfqpoint{1.401017in}{0.935495in}}%
\pgfpathlineto{\pgfqpoint{1.405730in}{0.935501in}}%
\pgfpathlineto{\pgfqpoint{1.410443in}{0.937769in}}%
\pgfpathlineto{\pgfqpoint{1.415156in}{0.938478in}}%
\pgfpathlineto{\pgfqpoint{1.419869in}{0.956348in}}%
\pgfpathlineto{\pgfqpoint{1.429294in}{0.961320in}}%
\pgfpathlineto{\pgfqpoint{1.438720in}{0.962580in}}%
\pgfpathlineto{\pgfqpoint{1.443433in}{0.962834in}}%
\pgfpathlineto{\pgfqpoint{1.452859in}{0.968993in}}%
\pgfpathlineto{\pgfqpoint{1.457572in}{0.969268in}}%
\pgfpathlineto{\pgfqpoint{1.462285in}{0.972491in}}%
\pgfpathlineto{\pgfqpoint{1.466998in}{0.979480in}}%
\pgfpathlineto{\pgfqpoint{1.476423in}{0.981802in}}%
\pgfpathlineto{\pgfqpoint{1.481136in}{0.999351in}}%
\pgfpathlineto{\pgfqpoint{1.485849in}{1.001244in}}%
\pgfpathlineto{\pgfqpoint{1.499988in}{1.002736in}}%
\pgfpathlineto{\pgfqpoint{1.504701in}{1.005996in}}%
\pgfpathlineto{\pgfqpoint{1.509414in}{1.013311in}}%
\pgfpathlineto{\pgfqpoint{1.514127in}{1.041239in}}%
\pgfpathlineto{\pgfqpoint{1.518840in}{1.049146in}}%
\pgfpathlineto{\pgfqpoint{1.523553in}{1.060231in}}%
\pgfpathlineto{\pgfqpoint{1.528265in}{1.938609in}}%
\pgfpathlineto{\pgfqpoint{5.845287in}{1.938609in}}%
\pgfpathlineto{\pgfqpoint{5.845287in}{1.938609in}}%
\pgfusepath{stroke}%
\end{pgfscope}%
\begin{pgfscope}%
\pgfpathrectangle{\pgfqpoint{0.708220in}{0.535823in}}{\pgfqpoint{5.141780in}{1.402786in}}%
\pgfusepath{clip}%
\pgfsetrectcap%
\pgfsetroundjoin%
\pgfsetlinewidth{2.007500pt}%
\definecolor{currentstroke}{rgb}{0.980392,0.529412,0.458824}%
\pgfsetstrokecolor{currentstroke}%
\pgfsetdash{}{0pt}%
\pgfpathmoveto{\pgfqpoint{0.708220in}{0.830704in}}%
\pgfpathlineto{\pgfqpoint{0.712933in}{0.832616in}}%
\pgfpathlineto{\pgfqpoint{0.717646in}{0.832715in}}%
\pgfpathlineto{\pgfqpoint{0.727071in}{0.842924in}}%
\pgfpathlineto{\pgfqpoint{0.731784in}{0.913732in}}%
\pgfpathlineto{\pgfqpoint{0.736497in}{0.936242in}}%
\pgfpathlineto{\pgfqpoint{0.755349in}{0.940977in}}%
\pgfpathlineto{\pgfqpoint{0.760062in}{0.948626in}}%
\pgfpathlineto{\pgfqpoint{0.764775in}{0.953075in}}%
\pgfpathlineto{\pgfqpoint{0.774201in}{0.956554in}}%
\pgfpathlineto{\pgfqpoint{0.778913in}{0.959043in}}%
\pgfpathlineto{\pgfqpoint{0.788339in}{0.968018in}}%
\pgfpathlineto{\pgfqpoint{0.802478in}{0.971514in}}%
\pgfpathlineto{\pgfqpoint{0.816617in}{0.986167in}}%
\pgfpathlineto{\pgfqpoint{0.821330in}{0.986778in}}%
\pgfpathlineto{\pgfqpoint{0.826043in}{0.990653in}}%
\pgfpathlineto{\pgfqpoint{0.830755in}{0.992533in}}%
\pgfpathlineto{\pgfqpoint{0.849607in}{0.995729in}}%
\pgfpathlineto{\pgfqpoint{0.854320in}{0.998778in}}%
\pgfpathlineto{\pgfqpoint{0.859033in}{1.010459in}}%
\pgfpathlineto{\pgfqpoint{0.868459in}{1.011842in}}%
\pgfpathlineto{\pgfqpoint{0.873172in}{1.015972in}}%
\pgfpathlineto{\pgfqpoint{0.877884in}{1.017948in}}%
\pgfpathlineto{\pgfqpoint{0.892023in}{1.019297in}}%
\pgfpathlineto{\pgfqpoint{0.896736in}{1.022322in}}%
\pgfpathlineto{\pgfqpoint{0.901449in}{1.034811in}}%
\pgfpathlineto{\pgfqpoint{0.906162in}{1.040130in}}%
\pgfpathlineto{\pgfqpoint{0.910875in}{1.040630in}}%
\pgfpathlineto{\pgfqpoint{0.915588in}{1.045324in}}%
\pgfpathlineto{\pgfqpoint{0.925014in}{1.047775in}}%
\pgfpathlineto{\pgfqpoint{0.929726in}{1.051674in}}%
\pgfpathlineto{\pgfqpoint{0.939152in}{1.055453in}}%
\pgfpathlineto{\pgfqpoint{0.948578in}{1.055821in}}%
\pgfpathlineto{\pgfqpoint{0.958004in}{1.060504in}}%
\pgfpathlineto{\pgfqpoint{0.962717in}{1.066685in}}%
\pgfpathlineto{\pgfqpoint{0.967430in}{1.067390in}}%
\pgfpathlineto{\pgfqpoint{0.972143in}{1.076898in}}%
\pgfpathlineto{\pgfqpoint{0.976855in}{1.079429in}}%
\pgfpathlineto{\pgfqpoint{0.981568in}{1.079901in}}%
\pgfpathlineto{\pgfqpoint{0.986281in}{1.087887in}}%
\pgfpathlineto{\pgfqpoint{0.990994in}{1.090691in}}%
\pgfpathlineto{\pgfqpoint{0.995707in}{1.096737in}}%
\pgfpathlineto{\pgfqpoint{1.009846in}{1.103486in}}%
\pgfpathlineto{\pgfqpoint{1.014559in}{1.114508in}}%
\pgfpathlineto{\pgfqpoint{1.023985in}{1.127516in}}%
\pgfpathlineto{\pgfqpoint{1.028697in}{1.147153in}}%
\pgfpathlineto{\pgfqpoint{1.033410in}{1.147820in}}%
\pgfpathlineto{\pgfqpoint{1.038123in}{1.152011in}}%
\pgfpathlineto{\pgfqpoint{1.047549in}{1.154799in}}%
\pgfpathlineto{\pgfqpoint{1.052262in}{1.160470in}}%
\pgfpathlineto{\pgfqpoint{1.061688in}{1.161297in}}%
\pgfpathlineto{\pgfqpoint{1.075827in}{1.173987in}}%
\pgfpathlineto{\pgfqpoint{1.080539in}{1.177854in}}%
\pgfpathlineto{\pgfqpoint{1.094678in}{1.181882in}}%
\pgfpathlineto{\pgfqpoint{1.099391in}{1.186928in}}%
\pgfpathlineto{\pgfqpoint{1.104104in}{1.189783in}}%
\pgfpathlineto{\pgfqpoint{1.108817in}{1.194251in}}%
\pgfpathlineto{\pgfqpoint{1.113530in}{1.194264in}}%
\pgfpathlineto{\pgfqpoint{1.118243in}{1.202355in}}%
\pgfpathlineto{\pgfqpoint{1.122956in}{1.203109in}}%
\pgfpathlineto{\pgfqpoint{1.137094in}{1.214456in}}%
\pgfpathlineto{\pgfqpoint{1.146520in}{1.224262in}}%
\pgfpathlineto{\pgfqpoint{1.155946in}{1.226626in}}%
\pgfpathlineto{\pgfqpoint{1.165372in}{1.232513in}}%
\pgfpathlineto{\pgfqpoint{1.170085in}{1.239549in}}%
\pgfpathlineto{\pgfqpoint{1.188936in}{1.241897in}}%
\pgfpathlineto{\pgfqpoint{1.198362in}{1.243205in}}%
\pgfpathlineto{\pgfqpoint{1.203075in}{1.248364in}}%
\pgfpathlineto{\pgfqpoint{1.212501in}{1.263269in}}%
\pgfpathlineto{\pgfqpoint{1.221927in}{1.265081in}}%
\pgfpathlineto{\pgfqpoint{1.226639in}{1.267695in}}%
\pgfpathlineto{\pgfqpoint{1.231352in}{1.268787in}}%
\pgfpathlineto{\pgfqpoint{1.236065in}{1.272479in}}%
\pgfpathlineto{\pgfqpoint{1.245491in}{1.273370in}}%
\pgfpathlineto{\pgfqpoint{1.254917in}{1.274016in}}%
\pgfpathlineto{\pgfqpoint{1.269056in}{1.276687in}}%
\pgfpathlineto{\pgfqpoint{1.273769in}{1.280832in}}%
\pgfpathlineto{\pgfqpoint{1.297333in}{1.283768in}}%
\pgfpathlineto{\pgfqpoint{1.311472in}{1.285198in}}%
\pgfpathlineto{\pgfqpoint{1.316185in}{1.286576in}}%
\pgfpathlineto{\pgfqpoint{1.330323in}{1.287702in}}%
\pgfpathlineto{\pgfqpoint{1.344462in}{1.289261in}}%
\pgfpathlineto{\pgfqpoint{1.358601in}{1.290302in}}%
\pgfpathlineto{\pgfqpoint{1.363314in}{1.296241in}}%
\pgfpathlineto{\pgfqpoint{1.368027in}{1.298886in}}%
\pgfpathlineto{\pgfqpoint{1.372740in}{1.307492in}}%
\pgfpathlineto{\pgfqpoint{1.377452in}{1.309365in}}%
\pgfpathlineto{\pgfqpoint{1.382165in}{1.320708in}}%
\pgfpathlineto{\pgfqpoint{1.386878in}{1.322832in}}%
\pgfpathlineto{\pgfqpoint{1.391591in}{1.323253in}}%
\pgfpathlineto{\pgfqpoint{1.396304in}{1.332556in}}%
\pgfpathlineto{\pgfqpoint{1.401017in}{1.332580in}}%
\pgfpathlineto{\pgfqpoint{1.415156in}{1.339214in}}%
\pgfpathlineto{\pgfqpoint{1.419869in}{1.339438in}}%
\pgfpathlineto{\pgfqpoint{1.424582in}{1.343874in}}%
\pgfpathlineto{\pgfqpoint{1.429294in}{1.345386in}}%
\pgfpathlineto{\pgfqpoint{1.434007in}{1.345489in}}%
\pgfpathlineto{\pgfqpoint{1.438720in}{1.351082in}}%
\pgfpathlineto{\pgfqpoint{1.443433in}{1.351107in}}%
\pgfpathlineto{\pgfqpoint{1.452859in}{1.354848in}}%
\pgfpathlineto{\pgfqpoint{1.462285in}{1.364014in}}%
\pgfpathlineto{\pgfqpoint{1.466998in}{1.372170in}}%
\pgfpathlineto{\pgfqpoint{1.471711in}{1.372315in}}%
\pgfpathlineto{\pgfqpoint{1.476423in}{1.373680in}}%
\pgfpathlineto{\pgfqpoint{1.485849in}{1.380603in}}%
\pgfpathlineto{\pgfqpoint{1.499988in}{1.383626in}}%
\pgfpathlineto{\pgfqpoint{1.528265in}{1.387852in}}%
\pgfpathlineto{\pgfqpoint{1.532978in}{1.388478in}}%
\pgfpathlineto{\pgfqpoint{1.537691in}{1.392775in}}%
\pgfpathlineto{\pgfqpoint{1.542404in}{1.393263in}}%
\pgfpathlineto{\pgfqpoint{1.556543in}{1.398844in}}%
\pgfpathlineto{\pgfqpoint{1.575395in}{1.400460in}}%
\pgfpathlineto{\pgfqpoint{1.580107in}{1.407076in}}%
\pgfpathlineto{\pgfqpoint{1.589533in}{1.415097in}}%
\pgfpathlineto{\pgfqpoint{1.594246in}{1.442562in}}%
\pgfpathlineto{\pgfqpoint{1.598959in}{1.443416in}}%
\pgfpathlineto{\pgfqpoint{1.603672in}{1.462719in}}%
\pgfpathlineto{\pgfqpoint{1.608385in}{1.471789in}}%
\pgfpathlineto{\pgfqpoint{1.613098in}{1.472395in}}%
\pgfpathlineto{\pgfqpoint{1.617811in}{1.474125in}}%
\pgfpathlineto{\pgfqpoint{1.622524in}{1.480461in}}%
\pgfpathlineto{\pgfqpoint{1.627236in}{1.481424in}}%
\pgfpathlineto{\pgfqpoint{1.631949in}{1.485127in}}%
\pgfpathlineto{\pgfqpoint{1.636662in}{1.486268in}}%
\pgfpathlineto{\pgfqpoint{1.641375in}{1.488800in}}%
\pgfpathlineto{\pgfqpoint{1.650801in}{1.490896in}}%
\pgfpathlineto{\pgfqpoint{1.655514in}{1.495220in}}%
\pgfpathlineto{\pgfqpoint{1.660227in}{1.504927in}}%
\pgfpathlineto{\pgfqpoint{1.664940in}{1.511535in}}%
\pgfpathlineto{\pgfqpoint{1.669653in}{1.520159in}}%
\pgfpathlineto{\pgfqpoint{1.679078in}{1.522618in}}%
\pgfpathlineto{\pgfqpoint{1.683791in}{1.526075in}}%
\pgfpathlineto{\pgfqpoint{1.688504in}{1.526682in}}%
\pgfpathlineto{\pgfqpoint{1.693217in}{1.528945in}}%
\pgfpathlineto{\pgfqpoint{1.702643in}{1.529871in}}%
\pgfpathlineto{\pgfqpoint{1.707356in}{1.534511in}}%
\pgfpathlineto{\pgfqpoint{1.712069in}{1.546864in}}%
\pgfpathlineto{\pgfqpoint{1.716782in}{1.549193in}}%
\pgfpathlineto{\pgfqpoint{1.721495in}{1.575044in}}%
\pgfpathlineto{\pgfqpoint{1.726207in}{1.584319in}}%
\pgfpathlineto{\pgfqpoint{1.730920in}{1.589049in}}%
\pgfpathlineto{\pgfqpoint{1.735633in}{1.597536in}}%
\pgfpathlineto{\pgfqpoint{1.740346in}{1.599955in}}%
\pgfpathlineto{\pgfqpoint{1.745059in}{1.606194in}}%
\pgfpathlineto{\pgfqpoint{1.754485in}{1.606866in}}%
\pgfpathlineto{\pgfqpoint{1.763911in}{1.610345in}}%
\pgfpathlineto{\pgfqpoint{1.773337in}{1.610999in}}%
\pgfpathlineto{\pgfqpoint{1.782762in}{1.613770in}}%
\pgfpathlineto{\pgfqpoint{1.787475in}{1.616015in}}%
\pgfpathlineto{\pgfqpoint{1.811040in}{1.619074in}}%
\pgfpathlineto{\pgfqpoint{1.825179in}{1.631734in}}%
\pgfpathlineto{\pgfqpoint{1.829891in}{1.640946in}}%
\pgfpathlineto{\pgfqpoint{1.839317in}{1.642891in}}%
\pgfpathlineto{\pgfqpoint{1.844030in}{1.658513in}}%
\pgfpathlineto{\pgfqpoint{1.848743in}{1.662369in}}%
\pgfpathlineto{\pgfqpoint{1.853456in}{1.671113in}}%
\pgfpathlineto{\pgfqpoint{1.858169in}{1.673373in}}%
\pgfpathlineto{\pgfqpoint{1.862882in}{1.673390in}}%
\pgfpathlineto{\pgfqpoint{1.867595in}{1.677757in}}%
\pgfpathlineto{\pgfqpoint{1.891159in}{1.681358in}}%
\pgfpathlineto{\pgfqpoint{1.895872in}{1.685101in}}%
\pgfpathlineto{\pgfqpoint{1.900585in}{1.696887in}}%
\pgfpathlineto{\pgfqpoint{1.905298in}{1.698460in}}%
\pgfpathlineto{\pgfqpoint{1.914724in}{1.703558in}}%
\pgfpathlineto{\pgfqpoint{1.919437in}{1.706482in}}%
\pgfpathlineto{\pgfqpoint{1.924150in}{1.724729in}}%
\pgfpathlineto{\pgfqpoint{1.928862in}{1.799534in}}%
\pgfpathlineto{\pgfqpoint{1.933575in}{1.938609in}}%
\pgfpathlineto{\pgfqpoint{5.845287in}{1.938609in}}%
\pgfpathlineto{\pgfqpoint{5.845287in}{1.938609in}}%
\pgfusepath{stroke}%
\end{pgfscope}%
\begin{pgfscope}%
\pgfsetrectcap%
\pgfsetmiterjoin%
\pgfsetlinewidth{0.803000pt}%
\definecolor{currentstroke}{rgb}{0.000000,0.000000,0.000000}%
\pgfsetstrokecolor{currentstroke}%
\pgfsetdash{}{0pt}%
\pgfpathmoveto{\pgfqpoint{0.708220in}{0.535823in}}%
\pgfpathlineto{\pgfqpoint{0.708220in}{1.938609in}}%
\pgfusepath{stroke}%
\end{pgfscope}%
\begin{pgfscope}%
\pgfsetrectcap%
\pgfsetmiterjoin%
\pgfsetlinewidth{0.803000pt}%
\definecolor{currentstroke}{rgb}{0.000000,0.000000,0.000000}%
\pgfsetstrokecolor{currentstroke}%
\pgfsetdash{}{0pt}%
\pgfpathmoveto{\pgfqpoint{5.850000in}{0.535823in}}%
\pgfpathlineto{\pgfqpoint{5.850000in}{1.938609in}}%
\pgfusepath{stroke}%
\end{pgfscope}%
\begin{pgfscope}%
\pgfsetrectcap%
\pgfsetmiterjoin%
\pgfsetlinewidth{0.803000pt}%
\definecolor{currentstroke}{rgb}{0.000000,0.000000,0.000000}%
\pgfsetstrokecolor{currentstroke}%
\pgfsetdash{}{0pt}%
\pgfpathmoveto{\pgfqpoint{0.708220in}{0.535823in}}%
\pgfpathlineto{\pgfqpoint{5.850000in}{0.535823in}}%
\pgfusepath{stroke}%
\end{pgfscope}%
\begin{pgfscope}%
\pgfsetrectcap%
\pgfsetmiterjoin%
\pgfsetlinewidth{0.803000pt}%
\definecolor{currentstroke}{rgb}{0.000000,0.000000,0.000000}%
\pgfsetstrokecolor{currentstroke}%
\pgfsetdash{}{0pt}%
\pgfpathmoveto{\pgfqpoint{0.708220in}{1.938609in}}%
\pgfpathlineto{\pgfqpoint{5.850000in}{1.938609in}}%
\pgfusepath{stroke}%
\end{pgfscope}%
\begin{pgfscope}%
\pgfsetbuttcap%
\pgfsetroundjoin%
\pgfsetlinewidth{2.007500pt}%
\definecolor{currentstroke}{rgb}{1.000000,0.843137,0.000000}%
\pgfsetstrokecolor{currentstroke}%
\pgfsetdash{{7.400000pt}{3.200000pt}}{0.000000pt}%
\pgfpathmoveto{\pgfqpoint{4.827505in}{0.977471in}}%
\pgfpathlineto{\pgfqpoint{5.077505in}{0.977471in}}%
\pgfusepath{stroke}%
\end{pgfscope}%
\begin{pgfscope}%
\definecolor{textcolor}{rgb}{0.000000,0.000000,0.000000}%
\pgfsetstrokecolor{textcolor}%
\pgfsetfillcolor{textcolor}%
\pgftext[x=5.102505in,y=0.933721in,left,base]{\color{textcolor}\rmfamily\fontsize{9.000000}{10.800000}\selectfont FT+htd}%
\end{pgfscope}%
\begin{pgfscope}%
\pgfsetbuttcap%
\pgfsetroundjoin%
\pgfsetlinewidth{2.007500pt}%
\definecolor{currentstroke}{rgb}{1.000000,0.694118,0.305882}%
\pgfsetstrokecolor{currentstroke}%
\pgfsetdash{{2.000000pt}{3.300000pt}}{0.000000pt}%
\pgfpathmoveto{\pgfqpoint{4.827505in}{0.815672in}}%
\pgfpathlineto{\pgfqpoint{5.077505in}{0.815672in}}%
\pgfusepath{stroke}%
\end{pgfscope}%
\begin{pgfscope}%
\definecolor{textcolor}{rgb}{0.000000,0.000000,0.000000}%
\pgfsetstrokecolor{textcolor}%
\pgfsetfillcolor{textcolor}%
\pgftext[x=5.102505in,y=0.771922in,left,base]{\color{textcolor}\rmfamily\fontsize{9.000000}{10.800000}\selectfont FT+Flow}%
\end{pgfscope}%
\begin{pgfscope}%
\pgfsetrectcap%
\pgfsetroundjoin%
\pgfsetlinewidth{2.007500pt}%
\definecolor{currentstroke}{rgb}{0.980392,0.529412,0.458824}%
\pgfsetstrokecolor{currentstroke}%
\pgfsetdash{}{0pt}%
\pgfpathmoveto{\pgfqpoint{4.827505in}{0.653872in}}%
\pgfpathlineto{\pgfqpoint{5.077505in}{0.653872in}}%
\pgfusepath{stroke}%
\end{pgfscope}%
\begin{pgfscope}%
\definecolor{textcolor}{rgb}{0.000000,0.000000,0.000000}%
\pgfsetstrokecolor{textcolor}%
\pgfsetfillcolor{textcolor}%
\pgftext[x=5.102505in,y=0.610122in,left,base]{\color{textcolor}\rmfamily\fontsize{9.000000}{10.800000}\selectfont FT+Tamaki}%
\end{pgfscope}%
\end{pgfpicture}%
\makeatother%
\endgroup%

	\caption{\label{fig:solver-analysis} The number of probabilistic-inference benchmarks (out of 1091) for which \textbf{FT+Tamaki}, \textbf{FT+Flow}, and \textbf{FT+htd} were able to find a contraction tree whose max rank was no larger than (top) 30, (middle) 25, or (bottom) 20 within the indicated time.}
\end{figure}

The existing tensor-based methods (\textbf{LG}, \textbf{greedy}, \textbf{metis}, and \textbf{GN}) that do not perform factoring were only able to count a single benchmark from this set within 1000 seconds. We observe that most of these benchmarks have a variable that appears many times, which significantly hinders tensor-based methods that do not perform factoring. This justifies our motivation for \textbf{FT} in Section \ref{sec:tensors:preprocessing}.

We next evaluate the structural properties of benchmarks for which \textbf{FT} outperforms other approaches. In Figure \ref{fig:cachet-carving-cactus}, we organize the number of benchmarks completed for each tool by carving width after \textbf{FT}-preprocessing. We observe that all tensor-based methods perform best on benchmarks with small carving width. In particular, \textbf{FT+Tamaki} was able to solve almost all benchmarks whose width is below 27. On the other hand, existing tools do not heavily rely on structural properties and so solve fewer low-width benchmarks than \textbf{FT+Tamaki} but significantly more high-width benchmarks. We conclude that tensor-network-based approaches perform well on benchmark instances of low carving width (after  \textbf{FT}-preprocessing).

Finally, we are interested in explaining the relative performance of the tensor-based methods \textbf{FT+Tamaki}, \textbf{FT+Flow}, and \textbf{FT+htd} on these benchmarks. To do this, we analyze the quality of the contraction trees they produce over time. Specifically, we rerun each implementation of \textbf{FT} for 1000 seconds with the contraction step disabled (i.e. remove step 3 of Algorithm \ref{alg:wmc}). Each implementation of \textbf{FT} is an online solver and so produces a sequence of contraction trees over time. For each contraction tree produced on each benchmark, we record the max rank and time of production.

Results of this experiment are summarized in Figure \ref{fig:solver-analysis}. 
\textbf{FT+Flow} is able to find more contraction trees of small max rank within 10 seconds than the other methods, while \textbf{FT+Tamaki} is able to find more contraction trees of small max rank within 1000 seconds than the other methods. This matches our previous observations that, among the tensor-based methods, \textbf{FT+Flow} was the fastest method on the most benchmarks while \textbf{FT+Tamaki} was able to solve the most benchmarks after 1000 seconds.

% Moreover, we observe that the quality of the discovered contraction trees does not significantly improve on most benchmarks after 4 seconds for \textbf{FT+Flow}, 20 seconds for \textbf{FT+htd}, and 500 seconds for \textbf{FT+Tamaki}. This suggests that these methods are not likely to discover significantly better contract

We conclude from the experiments in Section \ref{sec:tensors:experiments:cubic} and Section \ref{sec:tensors:experiments:cachet} that both \textbf{LG} and \textbf{FT} are useful as part of a portfolio of model counters.

% Although \pkg{Tamaki} placed above \pkg{FlowCutter} in the PACE 2017 competition, our implementations based on \pkg{FlowCutter} outperformed our implementations based on \pkg{Tamaki} on both sets of benchmarks. This suggests that tensor-based methods might be improved by developing specialized decomposition solvers.