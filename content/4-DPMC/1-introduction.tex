\section{Introduction}
\label{sec_intro}
\textdef{Model counting} is a fundamental problem in artificial intelligence, with applications in machine learning, probabilistic reasoning, and verification \cite{domshlak2007probabilistic,gomes2009model,naveh2007constraint}.
Given an input set of constraints, with the focus in this paper on Boolean constraints, the model-counting problem is to count the number of satisfying assignments.
Although this problem is \#P-Complete \cite{valiant1979complexity}, a variety of tools exist that can handle industrial sets of constraints, \eg, \cite{sang2004combining,oztok2015top,darwiche2004new,lagniez2017improved}.

Dynamic programming is a powerful technique that has been applied across computer science \cite{bellman1966dynamic}, including to model counting \cite{bacchus2009solving,samer2010algorithms,jegou2016improving}.
The key idea is to solve a large problem by solving a sequence of smaller subproblems and then incrementally combining these solutions into the final result.
Dynamic programming provides a natural framework to solve a variety of problems defined on sets of constraints: subproblems can be formed by partitioning the constraints.
This framework has been instantiated into algorithms for database-query optimization \cite{mcmahan2004projection}, satisfiability solving \cite{uribe1994ordered,aguirre2001random,pan2005symbolic}, and QBF evaluation \cite{charwat2016bdd}.

Dynamic programming has also been the basis of several tools for model counting \cite{dudek2020addmc,dudek2019efficient,dudek2020parallel,fichte2020exploiting}.
Although each tool uses a different data structure--algebraic decision diagrams (ADDs) \cite{dudek2020addmc}, tensors \cite{dudek2019efficient,dudek2020parallel}, or database tables \cite{fichte2020exploiting}--the overall algorithms have similar structure.
% Also, decomposition techniques have seen many applications in model counting, cf.~\cite{jegou2016improving} and the reference therein. %JD: Moved the citation to paragraph 2 instead
The goal of this work is to unify these approaches into a single conceptual framework: \emph{project-join trees}.
Project-join trees are not an entirely new idea.
Similar concepts have been used in constraint programming (as join trees \cite{dechter1989tree}), probabilistic inference (as cluster trees \cite{shachter1994global}), and database-query optimization (as join-expression trees \cite{mcmahan2004projection}).
Our original contributions include the unification of these concepts into project-join trees and the application of this unifying framework to model counting.

We argue that project-join trees provide the natural formalism to describe execution plans for dynamic-programming algorithms for model counting.
In particular, considering project-join trees as \emph{execution plans} enables us to decompose dynamic-programming algorithms such as the one in \cite{dudek2020addmc} into two phases, following the breakdown in \cite{dudek2020parallel}: a \emph{planning} phase and an \emph{execution} phase.
This enables us to study and compare different planning algorithms, different execution environments, and the interplay between planning and execution.
Such a study is the main focus of this work.
While the focus here is on model counting, our framework is of broader interest.
For example, in \cite{tabajara2017factored}, Tabajara and Vardi described a dynamic-programming, binary-decision-diagram-based framework for functional Boolean synthesis.
Refactoring the algorithm into a planning phase followed by an execution phase is also of interest in that context.

The primary contribution of the work here is a dynamic-programming framework for weighted model counting based on project-join trees.
In particular:
\begin{enumerate}
    \item We show that several recent algorithms for weighted model counting \cite{dudek2020addmc,dudek2019efficient,fichte2020exploiting} can be unified into a single framework using project-join trees.
    \item We compare the one-shot%
    \footnote{A \emph{one-shot} algorithm outputs exactly one solution and then terminates immediately.}
    constraint-satisfaction heuristics used in \cite{dudek2020addmc} with the anytime%
    \footnote{An \emph{anytime} algorithm outputs better and better solutions the longer it runs.} tree-decomposition tools used in \cite{dudek2019efficient} and find     tree-decomposition tools to outperform
    constraint-satisfaction heuristics.
    \item We compare (sparse) ADDs \cite{bahar1997algebraic} with (dense) tensors \cite{kjolstad2017tensor} and find that ADDs outperform tensors on single CPU cores.
    \item We find that project-join-tree-based algorithms contribute to a portfolio of model counters containing \cachet{} \cite{sang2004combining}, \ctd{} \cite{darwiche2004new}, \df{} \cite{lagniez2017improved}, and \minictd{} \cite{oztok2015top}.
\end{enumerate}
These conclusions have significance beyond model counting.
The superiority of anytime tree-decomposition tools over classical one-shot constraint-satisfaction heuristics can have broad applicability.
Similarly, the advantage of compact data structures for dynamic programming may apply to other optimization problems.

%%MYV: Cut paragraph to save space
% In Section \ref{sec_jointree}, we show how project-join trees can be used for weighted model counting.
% In Section \ref{sec_planning}, we present a variety of existing heuristics that can be rephrased to generate project-join trees.
% In Section \ref{sec_experiments_planning}, we compare various methods for generating project-join trees.
% In Section \ref{sec_experiments_execution}, we compare various execution environments for utilizing project-join trees.
% In Section \ref{sec_experiments_wmc}, we compare our project-join tree framework against state-of-the-art weighted model counters.
% Finally, we conclude in Section \ref{sec_discussion} with a discussion of the broader implications of our results.
