% \textdef{Model counting} is a fundamental problem in artificial intelligence, with applications in machine learning, probabilistic reasoning, and verification \cite{DH07,GSS08,naveh2007constraint}.
% Given an input set of constraints, with the focus in this paper on Boolean constraints, the model-counting problem is to count the number of satisfying assignments.
% Although this problem is \#P-Complete \cite{Valiant79}, a variety of tools exist that can handle industrial sets of constraints, \eg, \cite{sang2004combining,OD15,darwiche2004new,LM17}.

% Dynamic programming is a powerful technique that has been applied across computer science \cite{bellman1966dynamic}, including to model counting \cite{BDP09,SS10,jegou2016improving}.
% The key idea is to solve a large problem by solving a sequence of smaller subproblems and then incrementally combining these solutions into the final result.
% Dynamic programming provides a natural framework to solve a variety of problems defined on sets of constraints: subproblems can be formed by partitioning the constraints.
% This framework has been instantiated into algorithms for database-query optimization \cite{MPPV04}, satisfiability solving \cite{uribe1994ordered,aguirre2001random,pan2005symbolic}, and QBF evaluation \cite{charwat2016bdd}.

Dynamic programming has been the basis of several tools for model counting \cite{DPV20,DDV19,dudek2020parallel,fichte2020exploiting}.
Although each tool uses a different data structure--algebraic decision diagrams (ADDs) \cite{DPV20}, tensors (see Chapter \ref{ch:tensors}), or database tables \cite{fichte2020exploiting}--the overall algorithms have similar structure.
% Also, decomposition techniques have seen many applications in model counting, cf.~\cite{jegou2016improving} and the reference therein. %JD: Moved the citation to paragraph 2 instead
The goal of this chapter is to unify these approaches into a single conceptual framework: \emph{project-join trees}.

Project-join trees are not an entirely new idea.
Similar concepts have been used in constraint programming (as join trees \cite{dechter1989tree}), probabilistic inference (as cluster trees \cite{SAS94}), and database-query optimization (as join-expression trees \cite{MPPV04}).
Our original contributions include the unification of these concepts into project-join trees and the application of this unifying framework to model counting.

We argue that project-join trees provide a natural formalism to describe \emph{execution plans} for dynamic-programming algorithms for model counting.
In particular, considering project-join trees as execution plans enables us to decompose dynamic-programming algorithms for model counting such as the one in \cite{DPV20} into a 2-phase framework: a \emph{planning} phase, where a project-join tree is determined, and an \emph{execution} phase, where the the determined project-join tree is used to compute the model count.
This framework mirrors the 3-phase framework for model counting through tensor-network contraction in Chapter \ref{ch:tensors}, although since we now attack model counting directly there is no need for a reduction phase.

This breakdown enables us to study and compare different planning algorithms, different execution libraries, and the interplay between planning and execution.
Such a study is the main focus of this chapter. 
For example, we replace the one-shot%
\footnote{A \emph{one-shot} algorithm outputs exactly one solution and then terminates immediately.}
planner based on constraint-satisfaction heuristics \cite{dechter03} used in \cite{DPV20} with an anytime%
\footnote{An \emph{anytime} algorithm outputs better and better solutions the longer it runs.} planner based on tree-decomposition tools \cite{MPPV04,AMW17,HS18,Tamaki17}.
We also replace the (sparse) ADD-based executor \cite{bahar1997algebraic} used in \cite{DPV20} with an executor based on (dense) tensors \cite{KKCLA17}.

The primary contribution of this chapter is a dynamic-programming framework for weighted model counting based on project-join trees.
In particular:
\begin{enumerate}
    \item We show that several recent algorithms for weighted model counting \cite{DPV20,fichte2020exploiting} can be unified into a single framework using project-join trees.
    \item We compare the planner based on constraint-satisfaction heuristics \cite{dechter03} used in \cite{DPV20} with a planner based on tree-decomposition tools \cite{robertson1984graph,AMW17,HS18,Tamaki17} and find tree-decomposition tools to outperform constraint-satisfaction heuristics.
    \item We compare the ADD-based executor \cite{bahar1997algebraic} used in \cite{DPV20} with an executor based on tensors \cite{KKCLA17} and find that ADDs outperform tensors on single CPU cores.
    \item We find that project-join-tree-based algorithms contribute to a portfolio of model counters containing \cachet{} \cite{sang2004combining}, \ctd{} \cite{darwiche2004new}, \df{} \cite{LM17}, and \minictd{} \cite{OD15}.
\end{enumerate}
% These conclusions have significance beyond model counting.
% The superiority of anytime tree-decomposition tools over classical one-shot constraint-satisfaction heuristics can have broad applicability.
% Similarly, the advantage of compact data structures for dynamic programming may apply to other optimization problems.

%%MYV: Cut paragraph to save space
% In Section \ref{sec_jointree}, we show how project-join trees can be used for weighted model counting.
% In Section \ref{sec_planning}, we present a variety of existing heuristics that can be rephrased to generate project-join trees.
% In Section \ref{sec_experiments_planning}, we compare various methods for generating project-join trees.
% In Section \ref{sec_experiments_execution}, we compare various execution environments for utilizing project-join trees.
% In Section \ref{sec_experiments_wmc}, we compare our project-join tree framework against state-of-the-art weighted model counters.
% Finally, we conclude in Section \ref{sec_discussion} with a discussion of the broader implications of our results.
