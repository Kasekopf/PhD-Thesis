\section{Implementation and Evaluation}
\label{sec:procount:experiments}

To implement our projected model counter \procount, we modify the unprojected model counter \dpmc{}, which is based on ungraded project-join trees \cite{dudek2020dpmc}.
The \dpmc{} framework includes: 
(1) the \Lg{} planner that uses tree-decomposition techniques, 
(2) the \htb{} planner that uses constraint-satisfaction heuristics, and
(3) the \dmc{} executor that uses \emph{algebraic decision diagrams (ADDs)}.
We generalize these three components to support graded project-join trees and projected model counting.

We aim to answer the following experimental research questions:
\begin{enumerate}
    \item[(RQ1)] In the planning phase, how do tree-decomposition techniques compare with constraint-satisfaction heuristics?
    \item[(RQ2)] In the execution phase, how do different ADD variable orders compare?
    \item[(RQ3)] How does \procount{} compare to other exact weighted projected tools?
\end{enumerate}

To answer RQ1, in Experiment 1 we compare the planner \Lg{} (which uses tree decompositions) to \htb{} (which uses constraint-satisfaction heuristics).
\Lg{} uses the tree decomposers \flowcutter{} \cite{strasser2017computing}, \htd{} \cite{AMW17}, and \tamaki{} \cite{Tamaki17}.
\htb{} implements four heuristics for variable ordering: maximal-cardinality search (\mcs{}) \cite{tarjan1984simple}, lexicographic search for perfect/minimal orders (\lexp/\lexm{}) \cite{koster2001treewidth}, and min-fill (\minfill{}) \cite{dechter03}.
\htb{} also implements two clause-ordering heuristics: bucket elimination (\be) \cite{dechter99} and Bouquet's Method (\bm) \cite{bouquet1999gestion}%
% as well as two clause clustering heuristics \List{} and \tree{} \cite{DPV20}
.

To answer RQ2, in Experiment 2 we compare variable-ordering heuristics for the ADD-based executor \dmc.
An ADD \cite{bahar1997algebraic} is a directed acyclic graph that compactly represents a pseudo-Boolean function.
% Each internal node of an ADD corresponds to an input variable of the function.
An ADD requires a variable order, which strongly influences the compactness of the ADD.
\dmc{} implements four variable-ordering heuristics (see above): \mcs, \lexp, \lexm, and \minfill{}.

To answer RQ3, in Experiment 3 we compare \procount{} to state-of-the-art exact weighted projected model counters \dfp{} \cite{lagniez2019recursive}, \projmc{} \cite{lagniez2019recursive}, and \ssat{} \cite{lee2017solving}.
% We do not consider tools that are probabilistic, approximate, or unweighted.

%\dfp{} compiles Boolean formulas into decision decomposable negation normal form,
%\projmc{} uses disjunctive decomposition,
%and \ssat{} solves random-exist stochastic SAT by combining counting with SAT techniques.

We use \benchmarks{} CNF benchmarks gathered from two families.
The first family contains \wapsBenchmarks{} formulas and was used for weighted projected sampling \cite{gupta2019waps}.
For each benchmark in this family, a positive literal $x$ has weight $0 < W_x(\set{x}) < 1$, and a negative literal $\neg x$ has weight $W(\emptyset) = 1 - W_x(\set{x})$.
The second family contains \birdBenchmarks{} formulas and was used for unweighted projected model counting \cite{soos2019bird}.
We add weights to this family by randomly assigning $W_x(\set{x}) = 0.4$ and $W_x(\emptyset) = 0.6$ or vice versa to each variable $x$.
All \benchmarks{} benchmarks are satisfiable, as verified by the SAT solver \sat{} \cite{soos2009extending}.

We run all experiments on single CPU cores of a Linux cluster with Intel Xeon E5-2650v2 processors (2.60-GHz) and 30 GB of RAM.
All code and data are available (\url{https://github.com/vardigroup/DPMC/tree/v2.0.0}).

\noindent
% \the\columnwidth    \\ % 347.12354pt
% \the\textwidth      \\ % 347.12354pt
% \the\linewidth      \\ % 347.12354pt
% \the\hsize          \\ % 347.12354pt

%%%%%%%%%%%%%%%%%%%%%%%%%%%%%%%%%%%%%%%%%%%%%%%%%%%%%%%%%%%%%%%%%%%%%%%%%%%%%%%%

\subsection{Experiment 1: Comparing Planners}

In this experiment, we run all configurations of the planners \Lg{} and \htb{} once on each CNF benchmark with a timeout of 100 seconds.
We present results in Figure \ref{figPlanning}.
Each point $(x, y)$ on a plotted curve indicates that, within $x$ seconds on each of $y$ benchmarks, the first graded project-join tree produced by the corresponding planner has width at most \maxWidth{}.
We choose \maxWidth{} because previous work shows that executors do not handle larger project-join trees well \cite{DDV19,dudek2020dpmc}. % JD: This is more true for tensors than for ADDs, let's just leave it out
% Figure \ref{figPlanning} is qualitatively similar for other width cutoffs.

While \Lg{} is an \emph{anytime} tool that produces several trees (of decreasing widths) for each benchmark, we use only the first tree produced on each benchmark.
We find that this does not significantly affect the performance of \procount{}.

The tree-decomposition-based planner \Lg{} outputs more low-width trees than the constraint-satisfaction-based planner \htb{}.
Moreover, for \Lg{}, the tree decomposer \flowcutter{} is faster than \htd{} and \tamaki{}.
Thus we use \Lg{} with \flowcutter{} in \procount{} for later experiments.
\begin{figure}[t]
    \centering
    %% Creator: Matplotlib, PGF backend
%%
%% To include the figure in your LaTeX document, write
%%   \input{<filename>.pgf}
%%
%% Make sure the required packages are loaded in your preamble
%%   \usepackage{pgf}
%%
%% and, on pdftex
%%   \usepackage[utf8]{inputenc}\DeclareUnicodeCharacter{2212}{-}
%%
%% or, on luatex and xetex
%%   \usepackage{unicode-math}
%%
%% Figures using additional raster images can only be included by \input if
%% they are in the same directory as the main LaTeX file. For loading figures
%% from other directories you can use the `import` package
%%   \usepackage{import}
%%
%% and then include the figures with
%%   \import{<path to file>}{<filename>.pgf}
%%
%% Matplotlib used the following preamble
%%   \usepackage{fontspec}
%%   \setmainfont{DejaVuSerif.ttf}[Path=/home/vhp1/.local/lib/python3.8/site-packages/matplotlib/mpl-data/fonts/ttf/]
%%   \setsansfont{DejaVuSans.ttf}[Path=/home/vhp1/.local/lib/python3.8/site-packages/matplotlib/mpl-data/fonts/ttf/]
%%   \setmonofont{DejaVuSansMono.ttf}[Path=/home/vhp1/.local/lib/python3.8/site-packages/matplotlib/mpl-data/fonts/ttf/]
%%
\begingroup%
\makeatletter%
\begin{pgfpicture}%
\pgfpathrectangle{\pgfpointorigin}{\pgfqpoint{4.820041in}{1.610194in}}%
\pgfusepath{use as bounding box, clip}%
\begin{pgfscope}%
\pgfsetbuttcap%
\pgfsetmiterjoin%
\pgfsetlinewidth{0.000000pt}%
\definecolor{currentstroke}{rgb}{1.000000,1.000000,1.000000}%
\pgfsetstrokecolor{currentstroke}%
\pgfsetstrokeopacity{0.000000}%
\pgfsetdash{}{0pt}%
\pgfpathmoveto{\pgfqpoint{0.000000in}{0.000000in}}%
\pgfpathlineto{\pgfqpoint{4.820041in}{0.000000in}}%
\pgfpathlineto{\pgfqpoint{4.820041in}{1.610194in}}%
\pgfpathlineto{\pgfqpoint{0.000000in}{1.610194in}}%
\pgfpathclose%
\pgfusepath{}%
\end{pgfscope}%
\begin{pgfscope}%
\pgfsetbuttcap%
\pgfsetmiterjoin%
\definecolor{currentfill}{rgb}{1.000000,1.000000,1.000000}%
\pgfsetfillcolor{currentfill}%
\pgfsetlinewidth{0.000000pt}%
\definecolor{currentstroke}{rgb}{0.000000,0.000000,0.000000}%
\pgfsetstrokecolor{currentstroke}%
\pgfsetstrokeopacity{0.000000}%
\pgfsetdash{}{0pt}%
\pgfpathmoveto{\pgfqpoint{0.537394in}{0.467838in}}%
\pgfpathlineto{\pgfqpoint{4.632078in}{0.467838in}}%
\pgfpathlineto{\pgfqpoint{4.632078in}{1.465092in}}%
\pgfpathlineto{\pgfqpoint{0.537394in}{1.465092in}}%
\pgfpathclose%
\pgfusepath{fill}%
\end{pgfscope}%
\begin{pgfscope}%
\pgfsetbuttcap%
\pgfsetroundjoin%
\definecolor{currentfill}{rgb}{0.000000,0.000000,0.000000}%
\pgfsetfillcolor{currentfill}%
\pgfsetlinewidth{0.803000pt}%
\definecolor{currentstroke}{rgb}{0.000000,0.000000,0.000000}%
\pgfsetstrokecolor{currentstroke}%
\pgfsetdash{}{0pt}%
\pgfsys@defobject{currentmarker}{\pgfqpoint{0.000000in}{-0.048611in}}{\pgfqpoint{0.000000in}{0.000000in}}{%
\pgfpathmoveto{\pgfqpoint{0.000000in}{0.000000in}}%
\pgfpathlineto{\pgfqpoint{0.000000in}{-0.048611in}}%
\pgfusepath{stroke,fill}%
}%
\begin{pgfscope}%
\pgfsys@transformshift{0.537394in}{0.467838in}%
\pgfsys@useobject{currentmarker}{}%
\end{pgfscope}%
\end{pgfscope}%
\begin{pgfscope}%
\definecolor{textcolor}{rgb}{0.000000,0.000000,0.000000}%
\pgfsetstrokecolor{textcolor}%
\pgfsetfillcolor{textcolor}%
\pgftext[x=0.537394in,y=0.370616in,,top]{\color{textcolor}\sffamily\fontsize{8.000000}{9.600000}\selectfont \(\displaystyle {10^{-3}}\)}%
\end{pgfscope}%
\begin{pgfscope}%
\pgfsetbuttcap%
\pgfsetroundjoin%
\definecolor{currentfill}{rgb}{0.000000,0.000000,0.000000}%
\pgfsetfillcolor{currentfill}%
\pgfsetlinewidth{0.803000pt}%
\definecolor{currentstroke}{rgb}{0.000000,0.000000,0.000000}%
\pgfsetstrokecolor{currentstroke}%
\pgfsetdash{}{0pt}%
\pgfsys@defobject{currentmarker}{\pgfqpoint{0.000000in}{-0.048611in}}{\pgfqpoint{0.000000in}{0.000000in}}{%
\pgfpathmoveto{\pgfqpoint{0.000000in}{0.000000in}}%
\pgfpathlineto{\pgfqpoint{0.000000in}{-0.048611in}}%
\pgfusepath{stroke,fill}%
}%
\begin{pgfscope}%
\pgfsys@transformshift{1.356331in}{0.467838in}%
\pgfsys@useobject{currentmarker}{}%
\end{pgfscope}%
\end{pgfscope}%
\begin{pgfscope}%
\definecolor{textcolor}{rgb}{0.000000,0.000000,0.000000}%
\pgfsetstrokecolor{textcolor}%
\pgfsetfillcolor{textcolor}%
\pgftext[x=1.356331in,y=0.370616in,,top]{\color{textcolor}\sffamily\fontsize{8.000000}{9.600000}\selectfont \(\displaystyle {10^{-2}}\)}%
\end{pgfscope}%
\begin{pgfscope}%
\pgfsetbuttcap%
\pgfsetroundjoin%
\definecolor{currentfill}{rgb}{0.000000,0.000000,0.000000}%
\pgfsetfillcolor{currentfill}%
\pgfsetlinewidth{0.803000pt}%
\definecolor{currentstroke}{rgb}{0.000000,0.000000,0.000000}%
\pgfsetstrokecolor{currentstroke}%
\pgfsetdash{}{0pt}%
\pgfsys@defobject{currentmarker}{\pgfqpoint{0.000000in}{-0.048611in}}{\pgfqpoint{0.000000in}{0.000000in}}{%
\pgfpathmoveto{\pgfqpoint{0.000000in}{0.000000in}}%
\pgfpathlineto{\pgfqpoint{0.000000in}{-0.048611in}}%
\pgfusepath{stroke,fill}%
}%
\begin{pgfscope}%
\pgfsys@transformshift{2.175268in}{0.467838in}%
\pgfsys@useobject{currentmarker}{}%
\end{pgfscope}%
\end{pgfscope}%
\begin{pgfscope}%
\definecolor{textcolor}{rgb}{0.000000,0.000000,0.000000}%
\pgfsetstrokecolor{textcolor}%
\pgfsetfillcolor{textcolor}%
\pgftext[x=2.175268in,y=0.370616in,,top]{\color{textcolor}\sffamily\fontsize{8.000000}{9.600000}\selectfont \(\displaystyle {10^{-1}}\)}%
\end{pgfscope}%
\begin{pgfscope}%
\pgfsetbuttcap%
\pgfsetroundjoin%
\definecolor{currentfill}{rgb}{0.000000,0.000000,0.000000}%
\pgfsetfillcolor{currentfill}%
\pgfsetlinewidth{0.803000pt}%
\definecolor{currentstroke}{rgb}{0.000000,0.000000,0.000000}%
\pgfsetstrokecolor{currentstroke}%
\pgfsetdash{}{0pt}%
\pgfsys@defobject{currentmarker}{\pgfqpoint{0.000000in}{-0.048611in}}{\pgfqpoint{0.000000in}{0.000000in}}{%
\pgfpathmoveto{\pgfqpoint{0.000000in}{0.000000in}}%
\pgfpathlineto{\pgfqpoint{0.000000in}{-0.048611in}}%
\pgfusepath{stroke,fill}%
}%
\begin{pgfscope}%
\pgfsys@transformshift{2.994204in}{0.467838in}%
\pgfsys@useobject{currentmarker}{}%
\end{pgfscope}%
\end{pgfscope}%
\begin{pgfscope}%
\definecolor{textcolor}{rgb}{0.000000,0.000000,0.000000}%
\pgfsetstrokecolor{textcolor}%
\pgfsetfillcolor{textcolor}%
\pgftext[x=2.994204in,y=0.370616in,,top]{\color{textcolor}\sffamily\fontsize{8.000000}{9.600000}\selectfont \(\displaystyle {10^{0}}\)}%
\end{pgfscope}%
\begin{pgfscope}%
\pgfsetbuttcap%
\pgfsetroundjoin%
\definecolor{currentfill}{rgb}{0.000000,0.000000,0.000000}%
\pgfsetfillcolor{currentfill}%
\pgfsetlinewidth{0.803000pt}%
\definecolor{currentstroke}{rgb}{0.000000,0.000000,0.000000}%
\pgfsetstrokecolor{currentstroke}%
\pgfsetdash{}{0pt}%
\pgfsys@defobject{currentmarker}{\pgfqpoint{0.000000in}{-0.048611in}}{\pgfqpoint{0.000000in}{0.000000in}}{%
\pgfpathmoveto{\pgfqpoint{0.000000in}{0.000000in}}%
\pgfpathlineto{\pgfqpoint{0.000000in}{-0.048611in}}%
\pgfusepath{stroke,fill}%
}%
\begin{pgfscope}%
\pgfsys@transformshift{3.813141in}{0.467838in}%
\pgfsys@useobject{currentmarker}{}%
\end{pgfscope}%
\end{pgfscope}%
\begin{pgfscope}%
\definecolor{textcolor}{rgb}{0.000000,0.000000,0.000000}%
\pgfsetstrokecolor{textcolor}%
\pgfsetfillcolor{textcolor}%
\pgftext[x=3.813141in,y=0.370616in,,top]{\color{textcolor}\sffamily\fontsize{8.000000}{9.600000}\selectfont \(\displaystyle {10^{1}}\)}%
\end{pgfscope}%
\begin{pgfscope}%
\pgfsetbuttcap%
\pgfsetroundjoin%
\definecolor{currentfill}{rgb}{0.000000,0.000000,0.000000}%
\pgfsetfillcolor{currentfill}%
\pgfsetlinewidth{0.803000pt}%
\definecolor{currentstroke}{rgb}{0.000000,0.000000,0.000000}%
\pgfsetstrokecolor{currentstroke}%
\pgfsetdash{}{0pt}%
\pgfsys@defobject{currentmarker}{\pgfqpoint{0.000000in}{-0.048611in}}{\pgfqpoint{0.000000in}{0.000000in}}{%
\pgfpathmoveto{\pgfqpoint{0.000000in}{0.000000in}}%
\pgfpathlineto{\pgfqpoint{0.000000in}{-0.048611in}}%
\pgfusepath{stroke,fill}%
}%
\begin{pgfscope}%
\pgfsys@transformshift{4.632078in}{0.467838in}%
\pgfsys@useobject{currentmarker}{}%
\end{pgfscope}%
\end{pgfscope}%
\begin{pgfscope}%
\definecolor{textcolor}{rgb}{0.000000,0.000000,0.000000}%
\pgfsetstrokecolor{textcolor}%
\pgfsetfillcolor{textcolor}%
\pgftext[x=4.632078in,y=0.370616in,,top]{\color{textcolor}\sffamily\fontsize{8.000000}{9.600000}\selectfont \(\displaystyle {10^{2}}\)}%
\end{pgfscope}%
\begin{pgfscope}%
\pgfsetbuttcap%
\pgfsetroundjoin%
\definecolor{currentfill}{rgb}{0.000000,0.000000,0.000000}%
\pgfsetfillcolor{currentfill}%
\pgfsetlinewidth{0.602250pt}%
\definecolor{currentstroke}{rgb}{0.000000,0.000000,0.000000}%
\pgfsetstrokecolor{currentstroke}%
\pgfsetdash{}{0pt}%
\pgfsys@defobject{currentmarker}{\pgfqpoint{0.000000in}{-0.027778in}}{\pgfqpoint{0.000000in}{0.000000in}}{%
\pgfpathmoveto{\pgfqpoint{0.000000in}{0.000000in}}%
\pgfpathlineto{\pgfqpoint{0.000000in}{-0.027778in}}%
\pgfusepath{stroke,fill}%
}%
\begin{pgfscope}%
\pgfsys@transformshift{0.783918in}{0.467838in}%
\pgfsys@useobject{currentmarker}{}%
\end{pgfscope}%
\end{pgfscope}%
\begin{pgfscope}%
\pgfsetbuttcap%
\pgfsetroundjoin%
\definecolor{currentfill}{rgb}{0.000000,0.000000,0.000000}%
\pgfsetfillcolor{currentfill}%
\pgfsetlinewidth{0.602250pt}%
\definecolor{currentstroke}{rgb}{0.000000,0.000000,0.000000}%
\pgfsetstrokecolor{currentstroke}%
\pgfsetdash{}{0pt}%
\pgfsys@defobject{currentmarker}{\pgfqpoint{0.000000in}{-0.027778in}}{\pgfqpoint{0.000000in}{0.000000in}}{%
\pgfpathmoveto{\pgfqpoint{0.000000in}{0.000000in}}%
\pgfpathlineto{\pgfqpoint{0.000000in}{-0.027778in}}%
\pgfusepath{stroke,fill}%
}%
\begin{pgfscope}%
\pgfsys@transformshift{0.928126in}{0.467838in}%
\pgfsys@useobject{currentmarker}{}%
\end{pgfscope}%
\end{pgfscope}%
\begin{pgfscope}%
\pgfsetbuttcap%
\pgfsetroundjoin%
\definecolor{currentfill}{rgb}{0.000000,0.000000,0.000000}%
\pgfsetfillcolor{currentfill}%
\pgfsetlinewidth{0.602250pt}%
\definecolor{currentstroke}{rgb}{0.000000,0.000000,0.000000}%
\pgfsetstrokecolor{currentstroke}%
\pgfsetdash{}{0pt}%
\pgfsys@defobject{currentmarker}{\pgfqpoint{0.000000in}{-0.027778in}}{\pgfqpoint{0.000000in}{0.000000in}}{%
\pgfpathmoveto{\pgfqpoint{0.000000in}{0.000000in}}%
\pgfpathlineto{\pgfqpoint{0.000000in}{-0.027778in}}%
\pgfusepath{stroke,fill}%
}%
\begin{pgfscope}%
\pgfsys@transformshift{1.030443in}{0.467838in}%
\pgfsys@useobject{currentmarker}{}%
\end{pgfscope}%
\end{pgfscope}%
\begin{pgfscope}%
\pgfsetbuttcap%
\pgfsetroundjoin%
\definecolor{currentfill}{rgb}{0.000000,0.000000,0.000000}%
\pgfsetfillcolor{currentfill}%
\pgfsetlinewidth{0.602250pt}%
\definecolor{currentstroke}{rgb}{0.000000,0.000000,0.000000}%
\pgfsetstrokecolor{currentstroke}%
\pgfsetdash{}{0pt}%
\pgfsys@defobject{currentmarker}{\pgfqpoint{0.000000in}{-0.027778in}}{\pgfqpoint{0.000000in}{0.000000in}}{%
\pgfpathmoveto{\pgfqpoint{0.000000in}{0.000000in}}%
\pgfpathlineto{\pgfqpoint{0.000000in}{-0.027778in}}%
\pgfusepath{stroke,fill}%
}%
\begin{pgfscope}%
\pgfsys@transformshift{1.109806in}{0.467838in}%
\pgfsys@useobject{currentmarker}{}%
\end{pgfscope}%
\end{pgfscope}%
\begin{pgfscope}%
\pgfsetbuttcap%
\pgfsetroundjoin%
\definecolor{currentfill}{rgb}{0.000000,0.000000,0.000000}%
\pgfsetfillcolor{currentfill}%
\pgfsetlinewidth{0.602250pt}%
\definecolor{currentstroke}{rgb}{0.000000,0.000000,0.000000}%
\pgfsetstrokecolor{currentstroke}%
\pgfsetdash{}{0pt}%
\pgfsys@defobject{currentmarker}{\pgfqpoint{0.000000in}{-0.027778in}}{\pgfqpoint{0.000000in}{0.000000in}}{%
\pgfpathmoveto{\pgfqpoint{0.000000in}{0.000000in}}%
\pgfpathlineto{\pgfqpoint{0.000000in}{-0.027778in}}%
\pgfusepath{stroke,fill}%
}%
\begin{pgfscope}%
\pgfsys@transformshift{1.174651in}{0.467838in}%
\pgfsys@useobject{currentmarker}{}%
\end{pgfscope}%
\end{pgfscope}%
\begin{pgfscope}%
\pgfsetbuttcap%
\pgfsetroundjoin%
\definecolor{currentfill}{rgb}{0.000000,0.000000,0.000000}%
\pgfsetfillcolor{currentfill}%
\pgfsetlinewidth{0.602250pt}%
\definecolor{currentstroke}{rgb}{0.000000,0.000000,0.000000}%
\pgfsetstrokecolor{currentstroke}%
\pgfsetdash{}{0pt}%
\pgfsys@defobject{currentmarker}{\pgfqpoint{0.000000in}{-0.027778in}}{\pgfqpoint{0.000000in}{0.000000in}}{%
\pgfpathmoveto{\pgfqpoint{0.000000in}{0.000000in}}%
\pgfpathlineto{\pgfqpoint{0.000000in}{-0.027778in}}%
\pgfusepath{stroke,fill}%
}%
\begin{pgfscope}%
\pgfsys@transformshift{1.229476in}{0.467838in}%
\pgfsys@useobject{currentmarker}{}%
\end{pgfscope}%
\end{pgfscope}%
\begin{pgfscope}%
\pgfsetbuttcap%
\pgfsetroundjoin%
\definecolor{currentfill}{rgb}{0.000000,0.000000,0.000000}%
\pgfsetfillcolor{currentfill}%
\pgfsetlinewidth{0.602250pt}%
\definecolor{currentstroke}{rgb}{0.000000,0.000000,0.000000}%
\pgfsetstrokecolor{currentstroke}%
\pgfsetdash{}{0pt}%
\pgfsys@defobject{currentmarker}{\pgfqpoint{0.000000in}{-0.027778in}}{\pgfqpoint{0.000000in}{0.000000in}}{%
\pgfpathmoveto{\pgfqpoint{0.000000in}{0.000000in}}%
\pgfpathlineto{\pgfqpoint{0.000000in}{-0.027778in}}%
\pgfusepath{stroke,fill}%
}%
\begin{pgfscope}%
\pgfsys@transformshift{1.276968in}{0.467838in}%
\pgfsys@useobject{currentmarker}{}%
\end{pgfscope}%
\end{pgfscope}%
\begin{pgfscope}%
\pgfsetbuttcap%
\pgfsetroundjoin%
\definecolor{currentfill}{rgb}{0.000000,0.000000,0.000000}%
\pgfsetfillcolor{currentfill}%
\pgfsetlinewidth{0.602250pt}%
\definecolor{currentstroke}{rgb}{0.000000,0.000000,0.000000}%
\pgfsetstrokecolor{currentstroke}%
\pgfsetdash{}{0pt}%
\pgfsys@defobject{currentmarker}{\pgfqpoint{0.000000in}{-0.027778in}}{\pgfqpoint{0.000000in}{0.000000in}}{%
\pgfpathmoveto{\pgfqpoint{0.000000in}{0.000000in}}%
\pgfpathlineto{\pgfqpoint{0.000000in}{-0.027778in}}%
\pgfusepath{stroke,fill}%
}%
\begin{pgfscope}%
\pgfsys@transformshift{1.318858in}{0.467838in}%
\pgfsys@useobject{currentmarker}{}%
\end{pgfscope}%
\end{pgfscope}%
\begin{pgfscope}%
\pgfsetbuttcap%
\pgfsetroundjoin%
\definecolor{currentfill}{rgb}{0.000000,0.000000,0.000000}%
\pgfsetfillcolor{currentfill}%
\pgfsetlinewidth{0.602250pt}%
\definecolor{currentstroke}{rgb}{0.000000,0.000000,0.000000}%
\pgfsetstrokecolor{currentstroke}%
\pgfsetdash{}{0pt}%
\pgfsys@defobject{currentmarker}{\pgfqpoint{0.000000in}{-0.027778in}}{\pgfqpoint{0.000000in}{0.000000in}}{%
\pgfpathmoveto{\pgfqpoint{0.000000in}{0.000000in}}%
\pgfpathlineto{\pgfqpoint{0.000000in}{-0.027778in}}%
\pgfusepath{stroke,fill}%
}%
\begin{pgfscope}%
\pgfsys@transformshift{1.602855in}{0.467838in}%
\pgfsys@useobject{currentmarker}{}%
\end{pgfscope}%
\end{pgfscope}%
\begin{pgfscope}%
\pgfsetbuttcap%
\pgfsetroundjoin%
\definecolor{currentfill}{rgb}{0.000000,0.000000,0.000000}%
\pgfsetfillcolor{currentfill}%
\pgfsetlinewidth{0.602250pt}%
\definecolor{currentstroke}{rgb}{0.000000,0.000000,0.000000}%
\pgfsetstrokecolor{currentstroke}%
\pgfsetdash{}{0pt}%
\pgfsys@defobject{currentmarker}{\pgfqpoint{0.000000in}{-0.027778in}}{\pgfqpoint{0.000000in}{0.000000in}}{%
\pgfpathmoveto{\pgfqpoint{0.000000in}{0.000000in}}%
\pgfpathlineto{\pgfqpoint{0.000000in}{-0.027778in}}%
\pgfusepath{stroke,fill}%
}%
\begin{pgfscope}%
\pgfsys@transformshift{1.747063in}{0.467838in}%
\pgfsys@useobject{currentmarker}{}%
\end{pgfscope}%
\end{pgfscope}%
\begin{pgfscope}%
\pgfsetbuttcap%
\pgfsetroundjoin%
\definecolor{currentfill}{rgb}{0.000000,0.000000,0.000000}%
\pgfsetfillcolor{currentfill}%
\pgfsetlinewidth{0.602250pt}%
\definecolor{currentstroke}{rgb}{0.000000,0.000000,0.000000}%
\pgfsetstrokecolor{currentstroke}%
\pgfsetdash{}{0pt}%
\pgfsys@defobject{currentmarker}{\pgfqpoint{0.000000in}{-0.027778in}}{\pgfqpoint{0.000000in}{0.000000in}}{%
\pgfpathmoveto{\pgfqpoint{0.000000in}{0.000000in}}%
\pgfpathlineto{\pgfqpoint{0.000000in}{-0.027778in}}%
\pgfusepath{stroke,fill}%
}%
\begin{pgfscope}%
\pgfsys@transformshift{1.849380in}{0.467838in}%
\pgfsys@useobject{currentmarker}{}%
\end{pgfscope}%
\end{pgfscope}%
\begin{pgfscope}%
\pgfsetbuttcap%
\pgfsetroundjoin%
\definecolor{currentfill}{rgb}{0.000000,0.000000,0.000000}%
\pgfsetfillcolor{currentfill}%
\pgfsetlinewidth{0.602250pt}%
\definecolor{currentstroke}{rgb}{0.000000,0.000000,0.000000}%
\pgfsetstrokecolor{currentstroke}%
\pgfsetdash{}{0pt}%
\pgfsys@defobject{currentmarker}{\pgfqpoint{0.000000in}{-0.027778in}}{\pgfqpoint{0.000000in}{0.000000in}}{%
\pgfpathmoveto{\pgfqpoint{0.000000in}{0.000000in}}%
\pgfpathlineto{\pgfqpoint{0.000000in}{-0.027778in}}%
\pgfusepath{stroke,fill}%
}%
\begin{pgfscope}%
\pgfsys@transformshift{1.928743in}{0.467838in}%
\pgfsys@useobject{currentmarker}{}%
\end{pgfscope}%
\end{pgfscope}%
\begin{pgfscope}%
\pgfsetbuttcap%
\pgfsetroundjoin%
\definecolor{currentfill}{rgb}{0.000000,0.000000,0.000000}%
\pgfsetfillcolor{currentfill}%
\pgfsetlinewidth{0.602250pt}%
\definecolor{currentstroke}{rgb}{0.000000,0.000000,0.000000}%
\pgfsetstrokecolor{currentstroke}%
\pgfsetdash{}{0pt}%
\pgfsys@defobject{currentmarker}{\pgfqpoint{0.000000in}{-0.027778in}}{\pgfqpoint{0.000000in}{0.000000in}}{%
\pgfpathmoveto{\pgfqpoint{0.000000in}{0.000000in}}%
\pgfpathlineto{\pgfqpoint{0.000000in}{-0.027778in}}%
\pgfusepath{stroke,fill}%
}%
\begin{pgfscope}%
\pgfsys@transformshift{1.993587in}{0.467838in}%
\pgfsys@useobject{currentmarker}{}%
\end{pgfscope}%
\end{pgfscope}%
\begin{pgfscope}%
\pgfsetbuttcap%
\pgfsetroundjoin%
\definecolor{currentfill}{rgb}{0.000000,0.000000,0.000000}%
\pgfsetfillcolor{currentfill}%
\pgfsetlinewidth{0.602250pt}%
\definecolor{currentstroke}{rgb}{0.000000,0.000000,0.000000}%
\pgfsetstrokecolor{currentstroke}%
\pgfsetdash{}{0pt}%
\pgfsys@defobject{currentmarker}{\pgfqpoint{0.000000in}{-0.027778in}}{\pgfqpoint{0.000000in}{0.000000in}}{%
\pgfpathmoveto{\pgfqpoint{0.000000in}{0.000000in}}%
\pgfpathlineto{\pgfqpoint{0.000000in}{-0.027778in}}%
\pgfusepath{stroke,fill}%
}%
\begin{pgfscope}%
\pgfsys@transformshift{2.048413in}{0.467838in}%
\pgfsys@useobject{currentmarker}{}%
\end{pgfscope}%
\end{pgfscope}%
\begin{pgfscope}%
\pgfsetbuttcap%
\pgfsetroundjoin%
\definecolor{currentfill}{rgb}{0.000000,0.000000,0.000000}%
\pgfsetfillcolor{currentfill}%
\pgfsetlinewidth{0.602250pt}%
\definecolor{currentstroke}{rgb}{0.000000,0.000000,0.000000}%
\pgfsetstrokecolor{currentstroke}%
\pgfsetdash{}{0pt}%
\pgfsys@defobject{currentmarker}{\pgfqpoint{0.000000in}{-0.027778in}}{\pgfqpoint{0.000000in}{0.000000in}}{%
\pgfpathmoveto{\pgfqpoint{0.000000in}{0.000000in}}%
\pgfpathlineto{\pgfqpoint{0.000000in}{-0.027778in}}%
\pgfusepath{stroke,fill}%
}%
\begin{pgfscope}%
\pgfsys@transformshift{2.095904in}{0.467838in}%
\pgfsys@useobject{currentmarker}{}%
\end{pgfscope}%
\end{pgfscope}%
\begin{pgfscope}%
\pgfsetbuttcap%
\pgfsetroundjoin%
\definecolor{currentfill}{rgb}{0.000000,0.000000,0.000000}%
\pgfsetfillcolor{currentfill}%
\pgfsetlinewidth{0.602250pt}%
\definecolor{currentstroke}{rgb}{0.000000,0.000000,0.000000}%
\pgfsetstrokecolor{currentstroke}%
\pgfsetdash{}{0pt}%
\pgfsys@defobject{currentmarker}{\pgfqpoint{0.000000in}{-0.027778in}}{\pgfqpoint{0.000000in}{0.000000in}}{%
\pgfpathmoveto{\pgfqpoint{0.000000in}{0.000000in}}%
\pgfpathlineto{\pgfqpoint{0.000000in}{-0.027778in}}%
\pgfusepath{stroke,fill}%
}%
\begin{pgfscope}%
\pgfsys@transformshift{2.137795in}{0.467838in}%
\pgfsys@useobject{currentmarker}{}%
\end{pgfscope}%
\end{pgfscope}%
\begin{pgfscope}%
\pgfsetbuttcap%
\pgfsetroundjoin%
\definecolor{currentfill}{rgb}{0.000000,0.000000,0.000000}%
\pgfsetfillcolor{currentfill}%
\pgfsetlinewidth{0.602250pt}%
\definecolor{currentstroke}{rgb}{0.000000,0.000000,0.000000}%
\pgfsetstrokecolor{currentstroke}%
\pgfsetdash{}{0pt}%
\pgfsys@defobject{currentmarker}{\pgfqpoint{0.000000in}{-0.027778in}}{\pgfqpoint{0.000000in}{0.000000in}}{%
\pgfpathmoveto{\pgfqpoint{0.000000in}{0.000000in}}%
\pgfpathlineto{\pgfqpoint{0.000000in}{-0.027778in}}%
\pgfusepath{stroke,fill}%
}%
\begin{pgfscope}%
\pgfsys@transformshift{2.421792in}{0.467838in}%
\pgfsys@useobject{currentmarker}{}%
\end{pgfscope}%
\end{pgfscope}%
\begin{pgfscope}%
\pgfsetbuttcap%
\pgfsetroundjoin%
\definecolor{currentfill}{rgb}{0.000000,0.000000,0.000000}%
\pgfsetfillcolor{currentfill}%
\pgfsetlinewidth{0.602250pt}%
\definecolor{currentstroke}{rgb}{0.000000,0.000000,0.000000}%
\pgfsetstrokecolor{currentstroke}%
\pgfsetdash{}{0pt}%
\pgfsys@defobject{currentmarker}{\pgfqpoint{0.000000in}{-0.027778in}}{\pgfqpoint{0.000000in}{0.000000in}}{%
\pgfpathmoveto{\pgfqpoint{0.000000in}{0.000000in}}%
\pgfpathlineto{\pgfqpoint{0.000000in}{-0.027778in}}%
\pgfusepath{stroke,fill}%
}%
\begin{pgfscope}%
\pgfsys@transformshift{2.566000in}{0.467838in}%
\pgfsys@useobject{currentmarker}{}%
\end{pgfscope}%
\end{pgfscope}%
\begin{pgfscope}%
\pgfsetbuttcap%
\pgfsetroundjoin%
\definecolor{currentfill}{rgb}{0.000000,0.000000,0.000000}%
\pgfsetfillcolor{currentfill}%
\pgfsetlinewidth{0.602250pt}%
\definecolor{currentstroke}{rgb}{0.000000,0.000000,0.000000}%
\pgfsetstrokecolor{currentstroke}%
\pgfsetdash{}{0pt}%
\pgfsys@defobject{currentmarker}{\pgfqpoint{0.000000in}{-0.027778in}}{\pgfqpoint{0.000000in}{0.000000in}}{%
\pgfpathmoveto{\pgfqpoint{0.000000in}{0.000000in}}%
\pgfpathlineto{\pgfqpoint{0.000000in}{-0.027778in}}%
\pgfusepath{stroke,fill}%
}%
\begin{pgfscope}%
\pgfsys@transformshift{2.668317in}{0.467838in}%
\pgfsys@useobject{currentmarker}{}%
\end{pgfscope}%
\end{pgfscope}%
\begin{pgfscope}%
\pgfsetbuttcap%
\pgfsetroundjoin%
\definecolor{currentfill}{rgb}{0.000000,0.000000,0.000000}%
\pgfsetfillcolor{currentfill}%
\pgfsetlinewidth{0.602250pt}%
\definecolor{currentstroke}{rgb}{0.000000,0.000000,0.000000}%
\pgfsetstrokecolor{currentstroke}%
\pgfsetdash{}{0pt}%
\pgfsys@defobject{currentmarker}{\pgfqpoint{0.000000in}{-0.027778in}}{\pgfqpoint{0.000000in}{0.000000in}}{%
\pgfpathmoveto{\pgfqpoint{0.000000in}{0.000000in}}%
\pgfpathlineto{\pgfqpoint{0.000000in}{-0.027778in}}%
\pgfusepath{stroke,fill}%
}%
\begin{pgfscope}%
\pgfsys@transformshift{2.747680in}{0.467838in}%
\pgfsys@useobject{currentmarker}{}%
\end{pgfscope}%
\end{pgfscope}%
\begin{pgfscope}%
\pgfsetbuttcap%
\pgfsetroundjoin%
\definecolor{currentfill}{rgb}{0.000000,0.000000,0.000000}%
\pgfsetfillcolor{currentfill}%
\pgfsetlinewidth{0.602250pt}%
\definecolor{currentstroke}{rgb}{0.000000,0.000000,0.000000}%
\pgfsetstrokecolor{currentstroke}%
\pgfsetdash{}{0pt}%
\pgfsys@defobject{currentmarker}{\pgfqpoint{0.000000in}{-0.027778in}}{\pgfqpoint{0.000000in}{0.000000in}}{%
\pgfpathmoveto{\pgfqpoint{0.000000in}{0.000000in}}%
\pgfpathlineto{\pgfqpoint{0.000000in}{-0.027778in}}%
\pgfusepath{stroke,fill}%
}%
\begin{pgfscope}%
\pgfsys@transformshift{2.812524in}{0.467838in}%
\pgfsys@useobject{currentmarker}{}%
\end{pgfscope}%
\end{pgfscope}%
\begin{pgfscope}%
\pgfsetbuttcap%
\pgfsetroundjoin%
\definecolor{currentfill}{rgb}{0.000000,0.000000,0.000000}%
\pgfsetfillcolor{currentfill}%
\pgfsetlinewidth{0.602250pt}%
\definecolor{currentstroke}{rgb}{0.000000,0.000000,0.000000}%
\pgfsetstrokecolor{currentstroke}%
\pgfsetdash{}{0pt}%
\pgfsys@defobject{currentmarker}{\pgfqpoint{0.000000in}{-0.027778in}}{\pgfqpoint{0.000000in}{0.000000in}}{%
\pgfpathmoveto{\pgfqpoint{0.000000in}{0.000000in}}%
\pgfpathlineto{\pgfqpoint{0.000000in}{-0.027778in}}%
\pgfusepath{stroke,fill}%
}%
\begin{pgfscope}%
\pgfsys@transformshift{2.867349in}{0.467838in}%
\pgfsys@useobject{currentmarker}{}%
\end{pgfscope}%
\end{pgfscope}%
\begin{pgfscope}%
\pgfsetbuttcap%
\pgfsetroundjoin%
\definecolor{currentfill}{rgb}{0.000000,0.000000,0.000000}%
\pgfsetfillcolor{currentfill}%
\pgfsetlinewidth{0.602250pt}%
\definecolor{currentstroke}{rgb}{0.000000,0.000000,0.000000}%
\pgfsetstrokecolor{currentstroke}%
\pgfsetdash{}{0pt}%
\pgfsys@defobject{currentmarker}{\pgfqpoint{0.000000in}{-0.027778in}}{\pgfqpoint{0.000000in}{0.000000in}}{%
\pgfpathmoveto{\pgfqpoint{0.000000in}{0.000000in}}%
\pgfpathlineto{\pgfqpoint{0.000000in}{-0.027778in}}%
\pgfusepath{stroke,fill}%
}%
\begin{pgfscope}%
\pgfsys@transformshift{2.914841in}{0.467838in}%
\pgfsys@useobject{currentmarker}{}%
\end{pgfscope}%
\end{pgfscope}%
\begin{pgfscope}%
\pgfsetbuttcap%
\pgfsetroundjoin%
\definecolor{currentfill}{rgb}{0.000000,0.000000,0.000000}%
\pgfsetfillcolor{currentfill}%
\pgfsetlinewidth{0.602250pt}%
\definecolor{currentstroke}{rgb}{0.000000,0.000000,0.000000}%
\pgfsetstrokecolor{currentstroke}%
\pgfsetdash{}{0pt}%
\pgfsys@defobject{currentmarker}{\pgfqpoint{0.000000in}{-0.027778in}}{\pgfqpoint{0.000000in}{0.000000in}}{%
\pgfpathmoveto{\pgfqpoint{0.000000in}{0.000000in}}%
\pgfpathlineto{\pgfqpoint{0.000000in}{-0.027778in}}%
\pgfusepath{stroke,fill}%
}%
\begin{pgfscope}%
\pgfsys@transformshift{2.956732in}{0.467838in}%
\pgfsys@useobject{currentmarker}{}%
\end{pgfscope}%
\end{pgfscope}%
\begin{pgfscope}%
\pgfsetbuttcap%
\pgfsetroundjoin%
\definecolor{currentfill}{rgb}{0.000000,0.000000,0.000000}%
\pgfsetfillcolor{currentfill}%
\pgfsetlinewidth{0.602250pt}%
\definecolor{currentstroke}{rgb}{0.000000,0.000000,0.000000}%
\pgfsetstrokecolor{currentstroke}%
\pgfsetdash{}{0pt}%
\pgfsys@defobject{currentmarker}{\pgfqpoint{0.000000in}{-0.027778in}}{\pgfqpoint{0.000000in}{0.000000in}}{%
\pgfpathmoveto{\pgfqpoint{0.000000in}{0.000000in}}%
\pgfpathlineto{\pgfqpoint{0.000000in}{-0.027778in}}%
\pgfusepath{stroke,fill}%
}%
\begin{pgfscope}%
\pgfsys@transformshift{3.240729in}{0.467838in}%
\pgfsys@useobject{currentmarker}{}%
\end{pgfscope}%
\end{pgfscope}%
\begin{pgfscope}%
\pgfsetbuttcap%
\pgfsetroundjoin%
\definecolor{currentfill}{rgb}{0.000000,0.000000,0.000000}%
\pgfsetfillcolor{currentfill}%
\pgfsetlinewidth{0.602250pt}%
\definecolor{currentstroke}{rgb}{0.000000,0.000000,0.000000}%
\pgfsetstrokecolor{currentstroke}%
\pgfsetdash{}{0pt}%
\pgfsys@defobject{currentmarker}{\pgfqpoint{0.000000in}{-0.027778in}}{\pgfqpoint{0.000000in}{0.000000in}}{%
\pgfpathmoveto{\pgfqpoint{0.000000in}{0.000000in}}%
\pgfpathlineto{\pgfqpoint{0.000000in}{-0.027778in}}%
\pgfusepath{stroke,fill}%
}%
\begin{pgfscope}%
\pgfsys@transformshift{3.384936in}{0.467838in}%
\pgfsys@useobject{currentmarker}{}%
\end{pgfscope}%
\end{pgfscope}%
\begin{pgfscope}%
\pgfsetbuttcap%
\pgfsetroundjoin%
\definecolor{currentfill}{rgb}{0.000000,0.000000,0.000000}%
\pgfsetfillcolor{currentfill}%
\pgfsetlinewidth{0.602250pt}%
\definecolor{currentstroke}{rgb}{0.000000,0.000000,0.000000}%
\pgfsetstrokecolor{currentstroke}%
\pgfsetdash{}{0pt}%
\pgfsys@defobject{currentmarker}{\pgfqpoint{0.000000in}{-0.027778in}}{\pgfqpoint{0.000000in}{0.000000in}}{%
\pgfpathmoveto{\pgfqpoint{0.000000in}{0.000000in}}%
\pgfpathlineto{\pgfqpoint{0.000000in}{-0.027778in}}%
\pgfusepath{stroke,fill}%
}%
\begin{pgfscope}%
\pgfsys@transformshift{3.487253in}{0.467838in}%
\pgfsys@useobject{currentmarker}{}%
\end{pgfscope}%
\end{pgfscope}%
\begin{pgfscope}%
\pgfsetbuttcap%
\pgfsetroundjoin%
\definecolor{currentfill}{rgb}{0.000000,0.000000,0.000000}%
\pgfsetfillcolor{currentfill}%
\pgfsetlinewidth{0.602250pt}%
\definecolor{currentstroke}{rgb}{0.000000,0.000000,0.000000}%
\pgfsetstrokecolor{currentstroke}%
\pgfsetdash{}{0pt}%
\pgfsys@defobject{currentmarker}{\pgfqpoint{0.000000in}{-0.027778in}}{\pgfqpoint{0.000000in}{0.000000in}}{%
\pgfpathmoveto{\pgfqpoint{0.000000in}{0.000000in}}%
\pgfpathlineto{\pgfqpoint{0.000000in}{-0.027778in}}%
\pgfusepath{stroke,fill}%
}%
\begin{pgfscope}%
\pgfsys@transformshift{3.566617in}{0.467838in}%
\pgfsys@useobject{currentmarker}{}%
\end{pgfscope}%
\end{pgfscope}%
\begin{pgfscope}%
\pgfsetbuttcap%
\pgfsetroundjoin%
\definecolor{currentfill}{rgb}{0.000000,0.000000,0.000000}%
\pgfsetfillcolor{currentfill}%
\pgfsetlinewidth{0.602250pt}%
\definecolor{currentstroke}{rgb}{0.000000,0.000000,0.000000}%
\pgfsetstrokecolor{currentstroke}%
\pgfsetdash{}{0pt}%
\pgfsys@defobject{currentmarker}{\pgfqpoint{0.000000in}{-0.027778in}}{\pgfqpoint{0.000000in}{0.000000in}}{%
\pgfpathmoveto{\pgfqpoint{0.000000in}{0.000000in}}%
\pgfpathlineto{\pgfqpoint{0.000000in}{-0.027778in}}%
\pgfusepath{stroke,fill}%
}%
\begin{pgfscope}%
\pgfsys@transformshift{3.631461in}{0.467838in}%
\pgfsys@useobject{currentmarker}{}%
\end{pgfscope}%
\end{pgfscope}%
\begin{pgfscope}%
\pgfsetbuttcap%
\pgfsetroundjoin%
\definecolor{currentfill}{rgb}{0.000000,0.000000,0.000000}%
\pgfsetfillcolor{currentfill}%
\pgfsetlinewidth{0.602250pt}%
\definecolor{currentstroke}{rgb}{0.000000,0.000000,0.000000}%
\pgfsetstrokecolor{currentstroke}%
\pgfsetdash{}{0pt}%
\pgfsys@defobject{currentmarker}{\pgfqpoint{0.000000in}{-0.027778in}}{\pgfqpoint{0.000000in}{0.000000in}}{%
\pgfpathmoveto{\pgfqpoint{0.000000in}{0.000000in}}%
\pgfpathlineto{\pgfqpoint{0.000000in}{-0.027778in}}%
\pgfusepath{stroke,fill}%
}%
\begin{pgfscope}%
\pgfsys@transformshift{3.686286in}{0.467838in}%
\pgfsys@useobject{currentmarker}{}%
\end{pgfscope}%
\end{pgfscope}%
\begin{pgfscope}%
\pgfsetbuttcap%
\pgfsetroundjoin%
\definecolor{currentfill}{rgb}{0.000000,0.000000,0.000000}%
\pgfsetfillcolor{currentfill}%
\pgfsetlinewidth{0.602250pt}%
\definecolor{currentstroke}{rgb}{0.000000,0.000000,0.000000}%
\pgfsetstrokecolor{currentstroke}%
\pgfsetdash{}{0pt}%
\pgfsys@defobject{currentmarker}{\pgfqpoint{0.000000in}{-0.027778in}}{\pgfqpoint{0.000000in}{0.000000in}}{%
\pgfpathmoveto{\pgfqpoint{0.000000in}{0.000000in}}%
\pgfpathlineto{\pgfqpoint{0.000000in}{-0.027778in}}%
\pgfusepath{stroke,fill}%
}%
\begin{pgfscope}%
\pgfsys@transformshift{3.733778in}{0.467838in}%
\pgfsys@useobject{currentmarker}{}%
\end{pgfscope}%
\end{pgfscope}%
\begin{pgfscope}%
\pgfsetbuttcap%
\pgfsetroundjoin%
\definecolor{currentfill}{rgb}{0.000000,0.000000,0.000000}%
\pgfsetfillcolor{currentfill}%
\pgfsetlinewidth{0.602250pt}%
\definecolor{currentstroke}{rgb}{0.000000,0.000000,0.000000}%
\pgfsetstrokecolor{currentstroke}%
\pgfsetdash{}{0pt}%
\pgfsys@defobject{currentmarker}{\pgfqpoint{0.000000in}{-0.027778in}}{\pgfqpoint{0.000000in}{0.000000in}}{%
\pgfpathmoveto{\pgfqpoint{0.000000in}{0.000000in}}%
\pgfpathlineto{\pgfqpoint{0.000000in}{-0.027778in}}%
\pgfusepath{stroke,fill}%
}%
\begin{pgfscope}%
\pgfsys@transformshift{3.775669in}{0.467838in}%
\pgfsys@useobject{currentmarker}{}%
\end{pgfscope}%
\end{pgfscope}%
\begin{pgfscope}%
\pgfsetbuttcap%
\pgfsetroundjoin%
\definecolor{currentfill}{rgb}{0.000000,0.000000,0.000000}%
\pgfsetfillcolor{currentfill}%
\pgfsetlinewidth{0.602250pt}%
\definecolor{currentstroke}{rgb}{0.000000,0.000000,0.000000}%
\pgfsetstrokecolor{currentstroke}%
\pgfsetdash{}{0pt}%
\pgfsys@defobject{currentmarker}{\pgfqpoint{0.000000in}{-0.027778in}}{\pgfqpoint{0.000000in}{0.000000in}}{%
\pgfpathmoveto{\pgfqpoint{0.000000in}{0.000000in}}%
\pgfpathlineto{\pgfqpoint{0.000000in}{-0.027778in}}%
\pgfusepath{stroke,fill}%
}%
\begin{pgfscope}%
\pgfsys@transformshift{4.059666in}{0.467838in}%
\pgfsys@useobject{currentmarker}{}%
\end{pgfscope}%
\end{pgfscope}%
\begin{pgfscope}%
\pgfsetbuttcap%
\pgfsetroundjoin%
\definecolor{currentfill}{rgb}{0.000000,0.000000,0.000000}%
\pgfsetfillcolor{currentfill}%
\pgfsetlinewidth{0.602250pt}%
\definecolor{currentstroke}{rgb}{0.000000,0.000000,0.000000}%
\pgfsetstrokecolor{currentstroke}%
\pgfsetdash{}{0pt}%
\pgfsys@defobject{currentmarker}{\pgfqpoint{0.000000in}{-0.027778in}}{\pgfqpoint{0.000000in}{0.000000in}}{%
\pgfpathmoveto{\pgfqpoint{0.000000in}{0.000000in}}%
\pgfpathlineto{\pgfqpoint{0.000000in}{-0.027778in}}%
\pgfusepath{stroke,fill}%
}%
\begin{pgfscope}%
\pgfsys@transformshift{4.203873in}{0.467838in}%
\pgfsys@useobject{currentmarker}{}%
\end{pgfscope}%
\end{pgfscope}%
\begin{pgfscope}%
\pgfsetbuttcap%
\pgfsetroundjoin%
\definecolor{currentfill}{rgb}{0.000000,0.000000,0.000000}%
\pgfsetfillcolor{currentfill}%
\pgfsetlinewidth{0.602250pt}%
\definecolor{currentstroke}{rgb}{0.000000,0.000000,0.000000}%
\pgfsetstrokecolor{currentstroke}%
\pgfsetdash{}{0pt}%
\pgfsys@defobject{currentmarker}{\pgfqpoint{0.000000in}{-0.027778in}}{\pgfqpoint{0.000000in}{0.000000in}}{%
\pgfpathmoveto{\pgfqpoint{0.000000in}{0.000000in}}%
\pgfpathlineto{\pgfqpoint{0.000000in}{-0.027778in}}%
\pgfusepath{stroke,fill}%
}%
\begin{pgfscope}%
\pgfsys@transformshift{4.306190in}{0.467838in}%
\pgfsys@useobject{currentmarker}{}%
\end{pgfscope}%
\end{pgfscope}%
\begin{pgfscope}%
\pgfsetbuttcap%
\pgfsetroundjoin%
\definecolor{currentfill}{rgb}{0.000000,0.000000,0.000000}%
\pgfsetfillcolor{currentfill}%
\pgfsetlinewidth{0.602250pt}%
\definecolor{currentstroke}{rgb}{0.000000,0.000000,0.000000}%
\pgfsetstrokecolor{currentstroke}%
\pgfsetdash{}{0pt}%
\pgfsys@defobject{currentmarker}{\pgfqpoint{0.000000in}{-0.027778in}}{\pgfqpoint{0.000000in}{0.000000in}}{%
\pgfpathmoveto{\pgfqpoint{0.000000in}{0.000000in}}%
\pgfpathlineto{\pgfqpoint{0.000000in}{-0.027778in}}%
\pgfusepath{stroke,fill}%
}%
\begin{pgfscope}%
\pgfsys@transformshift{4.385553in}{0.467838in}%
\pgfsys@useobject{currentmarker}{}%
\end{pgfscope}%
\end{pgfscope}%
\begin{pgfscope}%
\pgfsetbuttcap%
\pgfsetroundjoin%
\definecolor{currentfill}{rgb}{0.000000,0.000000,0.000000}%
\pgfsetfillcolor{currentfill}%
\pgfsetlinewidth{0.602250pt}%
\definecolor{currentstroke}{rgb}{0.000000,0.000000,0.000000}%
\pgfsetstrokecolor{currentstroke}%
\pgfsetdash{}{0pt}%
\pgfsys@defobject{currentmarker}{\pgfqpoint{0.000000in}{-0.027778in}}{\pgfqpoint{0.000000in}{0.000000in}}{%
\pgfpathmoveto{\pgfqpoint{0.000000in}{0.000000in}}%
\pgfpathlineto{\pgfqpoint{0.000000in}{-0.027778in}}%
\pgfusepath{stroke,fill}%
}%
\begin{pgfscope}%
\pgfsys@transformshift{4.450398in}{0.467838in}%
\pgfsys@useobject{currentmarker}{}%
\end{pgfscope}%
\end{pgfscope}%
\begin{pgfscope}%
\pgfsetbuttcap%
\pgfsetroundjoin%
\definecolor{currentfill}{rgb}{0.000000,0.000000,0.000000}%
\pgfsetfillcolor{currentfill}%
\pgfsetlinewidth{0.602250pt}%
\definecolor{currentstroke}{rgb}{0.000000,0.000000,0.000000}%
\pgfsetstrokecolor{currentstroke}%
\pgfsetdash{}{0pt}%
\pgfsys@defobject{currentmarker}{\pgfqpoint{0.000000in}{-0.027778in}}{\pgfqpoint{0.000000in}{0.000000in}}{%
\pgfpathmoveto{\pgfqpoint{0.000000in}{0.000000in}}%
\pgfpathlineto{\pgfqpoint{0.000000in}{-0.027778in}}%
\pgfusepath{stroke,fill}%
}%
\begin{pgfscope}%
\pgfsys@transformshift{4.505223in}{0.467838in}%
\pgfsys@useobject{currentmarker}{}%
\end{pgfscope}%
\end{pgfscope}%
\begin{pgfscope}%
\pgfsetbuttcap%
\pgfsetroundjoin%
\definecolor{currentfill}{rgb}{0.000000,0.000000,0.000000}%
\pgfsetfillcolor{currentfill}%
\pgfsetlinewidth{0.602250pt}%
\definecolor{currentstroke}{rgb}{0.000000,0.000000,0.000000}%
\pgfsetstrokecolor{currentstroke}%
\pgfsetdash{}{0pt}%
\pgfsys@defobject{currentmarker}{\pgfqpoint{0.000000in}{-0.027778in}}{\pgfqpoint{0.000000in}{0.000000in}}{%
\pgfpathmoveto{\pgfqpoint{0.000000in}{0.000000in}}%
\pgfpathlineto{\pgfqpoint{0.000000in}{-0.027778in}}%
\pgfusepath{stroke,fill}%
}%
\begin{pgfscope}%
\pgfsys@transformshift{4.552715in}{0.467838in}%
\pgfsys@useobject{currentmarker}{}%
\end{pgfscope}%
\end{pgfscope}%
\begin{pgfscope}%
\pgfsetbuttcap%
\pgfsetroundjoin%
\definecolor{currentfill}{rgb}{0.000000,0.000000,0.000000}%
\pgfsetfillcolor{currentfill}%
\pgfsetlinewidth{0.602250pt}%
\definecolor{currentstroke}{rgb}{0.000000,0.000000,0.000000}%
\pgfsetstrokecolor{currentstroke}%
\pgfsetdash{}{0pt}%
\pgfsys@defobject{currentmarker}{\pgfqpoint{0.000000in}{-0.027778in}}{\pgfqpoint{0.000000in}{0.000000in}}{%
\pgfpathmoveto{\pgfqpoint{0.000000in}{0.000000in}}%
\pgfpathlineto{\pgfqpoint{0.000000in}{-0.027778in}}%
\pgfusepath{stroke,fill}%
}%
\begin{pgfscope}%
\pgfsys@transformshift{4.594605in}{0.467838in}%
\pgfsys@useobject{currentmarker}{}%
\end{pgfscope}%
\end{pgfscope}%
\begin{pgfscope}%
\definecolor{textcolor}{rgb}{0.000000,0.000000,0.000000}%
\pgfsetstrokecolor{textcolor}%
\pgfsetfillcolor{textcolor}%
\pgftext[x=2.584736in,y=0.207530in,,top]{\color{textcolor}\sffamily\fontsize{8.000000}{9.600000}\selectfont Longest solving time (seconds)}%
\end{pgfscope}%
\begin{pgfscope}%
\pgfsetbuttcap%
\pgfsetroundjoin%
\definecolor{currentfill}{rgb}{0.000000,0.000000,0.000000}%
\pgfsetfillcolor{currentfill}%
\pgfsetlinewidth{0.803000pt}%
\definecolor{currentstroke}{rgb}{0.000000,0.000000,0.000000}%
\pgfsetstrokecolor{currentstroke}%
\pgfsetdash{}{0pt}%
\pgfsys@defobject{currentmarker}{\pgfqpoint{-0.048611in}{0.000000in}}{\pgfqpoint{-0.000000in}{0.000000in}}{%
\pgfpathmoveto{\pgfqpoint{-0.000000in}{0.000000in}}%
\pgfpathlineto{\pgfqpoint{-0.048611in}{0.000000in}}%
\pgfusepath{stroke,fill}%
}%
\begin{pgfscope}%
\pgfsys@transformshift{0.537394in}{0.467838in}%
\pgfsys@useobject{currentmarker}{}%
\end{pgfscope}%
\end{pgfscope}%
\begin{pgfscope}%
\definecolor{textcolor}{rgb}{0.000000,0.000000,0.000000}%
\pgfsetstrokecolor{textcolor}%
\pgfsetfillcolor{textcolor}%
\pgftext[x=0.381143in, y=0.425629in, left, base]{\color{textcolor}\sffamily\fontsize{8.000000}{9.600000}\selectfont \(\displaystyle {0}\)}%
\end{pgfscope}%
\begin{pgfscope}%
\pgfsetbuttcap%
\pgfsetroundjoin%
\definecolor{currentfill}{rgb}{0.000000,0.000000,0.000000}%
\pgfsetfillcolor{currentfill}%
\pgfsetlinewidth{0.803000pt}%
\definecolor{currentstroke}{rgb}{0.000000,0.000000,0.000000}%
\pgfsetstrokecolor{currentstroke}%
\pgfsetdash{}{0pt}%
\pgfsys@defobject{currentmarker}{\pgfqpoint{-0.048611in}{0.000000in}}{\pgfqpoint{-0.000000in}{0.000000in}}{%
\pgfpathmoveto{\pgfqpoint{-0.000000in}{0.000000in}}%
\pgfpathlineto{\pgfqpoint{-0.048611in}{0.000000in}}%
\pgfusepath{stroke,fill}%
}%
\begin{pgfscope}%
\pgfsys@transformshift{0.537394in}{0.717152in}%
\pgfsys@useobject{currentmarker}{}%
\end{pgfscope}%
\end{pgfscope}%
\begin{pgfscope}%
\definecolor{textcolor}{rgb}{0.000000,0.000000,0.000000}%
\pgfsetstrokecolor{textcolor}%
\pgfsetfillcolor{textcolor}%
\pgftext[x=0.263086in, y=0.674943in, left, base]{\color{textcolor}\sffamily\fontsize{8.000000}{9.600000}\selectfont \(\displaystyle {100}\)}%
\end{pgfscope}%
\begin{pgfscope}%
\pgfsetbuttcap%
\pgfsetroundjoin%
\definecolor{currentfill}{rgb}{0.000000,0.000000,0.000000}%
\pgfsetfillcolor{currentfill}%
\pgfsetlinewidth{0.803000pt}%
\definecolor{currentstroke}{rgb}{0.000000,0.000000,0.000000}%
\pgfsetstrokecolor{currentstroke}%
\pgfsetdash{}{0pt}%
\pgfsys@defobject{currentmarker}{\pgfqpoint{-0.048611in}{0.000000in}}{\pgfqpoint{-0.000000in}{0.000000in}}{%
\pgfpathmoveto{\pgfqpoint{-0.000000in}{0.000000in}}%
\pgfpathlineto{\pgfqpoint{-0.048611in}{0.000000in}}%
\pgfusepath{stroke,fill}%
}%
\begin{pgfscope}%
\pgfsys@transformshift{0.537394in}{0.966465in}%
\pgfsys@useobject{currentmarker}{}%
\end{pgfscope}%
\end{pgfscope}%
\begin{pgfscope}%
\definecolor{textcolor}{rgb}{0.000000,0.000000,0.000000}%
\pgfsetstrokecolor{textcolor}%
\pgfsetfillcolor{textcolor}%
\pgftext[x=0.263086in, y=0.924256in, left, base]{\color{textcolor}\sffamily\fontsize{8.000000}{9.600000}\selectfont \(\displaystyle {200}\)}%
\end{pgfscope}%
\begin{pgfscope}%
\pgfsetbuttcap%
\pgfsetroundjoin%
\definecolor{currentfill}{rgb}{0.000000,0.000000,0.000000}%
\pgfsetfillcolor{currentfill}%
\pgfsetlinewidth{0.803000pt}%
\definecolor{currentstroke}{rgb}{0.000000,0.000000,0.000000}%
\pgfsetstrokecolor{currentstroke}%
\pgfsetdash{}{0pt}%
\pgfsys@defobject{currentmarker}{\pgfqpoint{-0.048611in}{0.000000in}}{\pgfqpoint{-0.000000in}{0.000000in}}{%
\pgfpathmoveto{\pgfqpoint{-0.000000in}{0.000000in}}%
\pgfpathlineto{\pgfqpoint{-0.048611in}{0.000000in}}%
\pgfusepath{stroke,fill}%
}%
\begin{pgfscope}%
\pgfsys@transformshift{0.537394in}{1.215779in}%
\pgfsys@useobject{currentmarker}{}%
\end{pgfscope}%
\end{pgfscope}%
\begin{pgfscope}%
\definecolor{textcolor}{rgb}{0.000000,0.000000,0.000000}%
\pgfsetstrokecolor{textcolor}%
\pgfsetfillcolor{textcolor}%
\pgftext[x=0.263086in, y=1.173569in, left, base]{\color{textcolor}\sffamily\fontsize{8.000000}{9.600000}\selectfont \(\displaystyle {300}\)}%
\end{pgfscope}%
\begin{pgfscope}%
\pgfsetbuttcap%
\pgfsetroundjoin%
\definecolor{currentfill}{rgb}{0.000000,0.000000,0.000000}%
\pgfsetfillcolor{currentfill}%
\pgfsetlinewidth{0.803000pt}%
\definecolor{currentstroke}{rgb}{0.000000,0.000000,0.000000}%
\pgfsetstrokecolor{currentstroke}%
\pgfsetdash{}{0pt}%
\pgfsys@defobject{currentmarker}{\pgfqpoint{-0.048611in}{0.000000in}}{\pgfqpoint{-0.000000in}{0.000000in}}{%
\pgfpathmoveto{\pgfqpoint{-0.000000in}{0.000000in}}%
\pgfpathlineto{\pgfqpoint{-0.048611in}{0.000000in}}%
\pgfusepath{stroke,fill}%
}%
\begin{pgfscope}%
\pgfsys@transformshift{0.537394in}{1.465092in}%
\pgfsys@useobject{currentmarker}{}%
\end{pgfscope}%
\end{pgfscope}%
\begin{pgfscope}%
\definecolor{textcolor}{rgb}{0.000000,0.000000,0.000000}%
\pgfsetstrokecolor{textcolor}%
\pgfsetfillcolor{textcolor}%
\pgftext[x=0.263086in, y=1.422883in, left, base]{\color{textcolor}\sffamily\fontsize{8.000000}{9.600000}\selectfont \(\displaystyle {400}\)}%
\end{pgfscope}%
\begin{pgfscope}%
\definecolor{textcolor}{rgb}{0.000000,0.000000,0.000000}%
\pgfsetstrokecolor{textcolor}%
\pgfsetfillcolor{textcolor}%
\pgftext[x=0.207530in,y=0.966465in,,bottom,rotate=90.000000]{\color{textcolor}\sffamily\fontsize{8.000000}{9.600000}\selectfont Benchmarks solved}%
\end{pgfscope}%
\begin{pgfscope}%
\pgfpathrectangle{\pgfqpoint{0.537394in}{0.467838in}}{\pgfqpoint{4.094684in}{0.997254in}}%
\pgfusepath{clip}%
\pgfsetrectcap%
\pgfsetroundjoin%
\pgfsetlinewidth{1.003750pt}%
\definecolor{currentstroke}{rgb}{0.121569,0.466667,0.705882}%
\pgfsetstrokecolor{currentstroke}%
\pgfsetdash{}{0pt}%
\pgfpathmoveto{\pgfqpoint{0.537394in}{0.482797in}}%
\pgfpathlineto{\pgfqpoint{0.623180in}{0.485290in}}%
\pgfpathlineto{\pgfqpoint{0.652299in}{0.487783in}}%
\pgfpathlineto{\pgfqpoint{1.155535in}{0.490277in}}%
\pgfpathlineto{\pgfqpoint{1.161909in}{0.492770in}}%
\pgfpathlineto{\pgfqpoint{1.184532in}{0.495263in}}%
\pgfpathlineto{\pgfqpoint{1.193365in}{0.497756in}}%
\pgfpathlineto{\pgfqpoint{1.223585in}{0.500249in}}%
\pgfpathlineto{\pgfqpoint{1.256641in}{0.502742in}}%
\pgfpathlineto{\pgfqpoint{1.281347in}{0.505235in}}%
\pgfpathlineto{\pgfqpoint{1.292678in}{0.507728in}}%
\pgfpathlineto{\pgfqpoint{1.295722in}{0.510222in}}%
\pgfpathlineto{\pgfqpoint{1.301003in}{0.512715in}}%
\pgfpathlineto{\pgfqpoint{1.303519in}{0.515208in}}%
\pgfpathlineto{\pgfqpoint{1.320749in}{0.517701in}}%
\pgfpathlineto{\pgfqpoint{1.332065in}{0.520194in}}%
\pgfpathlineto{\pgfqpoint{1.342626in}{0.522687in}}%
\pgfpathlineto{\pgfqpoint{1.345267in}{0.525180in}}%
\pgfpathlineto{\pgfqpoint{1.355900in}{0.527674in}}%
\pgfpathlineto{\pgfqpoint{1.367340in}{0.530167in}}%
\pgfpathlineto{\pgfqpoint{1.368308in}{0.532660in}}%
\pgfpathlineto{\pgfqpoint{1.378260in}{0.535153in}}%
\pgfpathlineto{\pgfqpoint{1.387535in}{0.537646in}}%
\pgfpathlineto{\pgfqpoint{1.390917in}{0.540139in}}%
\pgfpathlineto{\pgfqpoint{1.397462in}{0.542632in}}%
\pgfpathlineto{\pgfqpoint{1.399096in}{0.545125in}}%
\pgfpathlineto{\pgfqpoint{1.407754in}{0.547619in}}%
\pgfpathlineto{\pgfqpoint{1.413951in}{0.550112in}}%
\pgfpathlineto{\pgfqpoint{1.421942in}{0.552605in}}%
\pgfpathlineto{\pgfqpoint{1.425962in}{0.555098in}}%
\pgfpathlineto{\pgfqpoint{1.430414in}{0.557591in}}%
\pgfpathlineto{\pgfqpoint{1.432837in}{0.560084in}}%
\pgfpathlineto{\pgfqpoint{1.435947in}{0.562577in}}%
\pgfpathlineto{\pgfqpoint{1.436441in}{0.565071in}}%
\pgfpathlineto{\pgfqpoint{1.436765in}{0.567564in}}%
\pgfpathlineto{\pgfqpoint{1.441966in}{0.570057in}}%
\pgfpathlineto{\pgfqpoint{1.443158in}{0.572550in}}%
\pgfpathlineto{\pgfqpoint{1.445457in}{0.575043in}}%
\pgfpathlineto{\pgfqpoint{1.446690in}{0.577536in}}%
\pgfpathlineto{\pgfqpoint{1.447142in}{0.580029in}}%
\pgfpathlineto{\pgfqpoint{1.450932in}{0.582523in}}%
\pgfpathlineto{\pgfqpoint{1.451534in}{0.585016in}}%
\pgfpathlineto{\pgfqpoint{1.453453in}{0.587509in}}%
\pgfpathlineto{\pgfqpoint{1.456447in}{0.590002in}}%
\pgfpathlineto{\pgfqpoint{1.456641in}{0.592495in}}%
\pgfpathlineto{\pgfqpoint{1.458359in}{0.594988in}}%
\pgfpathlineto{\pgfqpoint{1.458743in}{0.597481in}}%
\pgfpathlineto{\pgfqpoint{1.469429in}{0.599974in}}%
\pgfpathlineto{\pgfqpoint{1.471863in}{0.602468in}}%
\pgfpathlineto{\pgfqpoint{1.474679in}{0.604961in}}%
\pgfpathlineto{\pgfqpoint{1.474871in}{0.607454in}}%
\pgfpathlineto{\pgfqpoint{1.476853in}{0.609947in}}%
\pgfpathlineto{\pgfqpoint{1.478267in}{0.612440in}}%
\pgfpathlineto{\pgfqpoint{1.480719in}{0.614933in}}%
\pgfpathlineto{\pgfqpoint{1.483143in}{0.617426in}}%
\pgfpathlineto{\pgfqpoint{1.483941in}{0.619920in}}%
\pgfpathlineto{\pgfqpoint{1.488954in}{0.622413in}}%
\pgfpathlineto{\pgfqpoint{1.494438in}{0.624906in}}%
\pgfpathlineto{\pgfqpoint{1.499118in}{0.627399in}}%
\pgfpathlineto{\pgfqpoint{1.509457in}{0.629892in}}%
\pgfpathlineto{\pgfqpoint{1.517462in}{0.632385in}}%
\pgfpathlineto{\pgfqpoint{1.521267in}{0.634878in}}%
\pgfpathlineto{\pgfqpoint{1.526991in}{0.637371in}}%
\pgfpathlineto{\pgfqpoint{1.537478in}{0.639865in}}%
\pgfpathlineto{\pgfqpoint{1.544394in}{0.642358in}}%
\pgfpathlineto{\pgfqpoint{1.548674in}{0.644851in}}%
\pgfpathlineto{\pgfqpoint{1.549197in}{0.647344in}}%
\pgfpathlineto{\pgfqpoint{1.550089in}{0.649837in}}%
\pgfpathlineto{\pgfqpoint{1.550778in}{0.652330in}}%
\pgfpathlineto{\pgfqpoint{1.563149in}{0.654823in}}%
\pgfpathlineto{\pgfqpoint{1.569719in}{0.657317in}}%
\pgfpathlineto{\pgfqpoint{1.575970in}{0.659810in}}%
\pgfpathlineto{\pgfqpoint{1.581049in}{0.662303in}}%
\pgfpathlineto{\pgfqpoint{1.581427in}{0.664796in}}%
\pgfpathlineto{\pgfqpoint{1.585031in}{0.667289in}}%
\pgfpathlineto{\pgfqpoint{1.588977in}{0.669782in}}%
\pgfpathlineto{\pgfqpoint{1.590273in}{0.672275in}}%
\pgfpathlineto{\pgfqpoint{1.604962in}{0.674768in}}%
\pgfpathlineto{\pgfqpoint{1.607996in}{0.677262in}}%
\pgfpathlineto{\pgfqpoint{1.608017in}{0.679755in}}%
\pgfpathlineto{\pgfqpoint{1.609584in}{0.682248in}}%
\pgfpathlineto{\pgfqpoint{1.613808in}{0.684741in}}%
\pgfpathlineto{\pgfqpoint{1.637540in}{0.687234in}}%
\pgfpathlineto{\pgfqpoint{1.644115in}{0.689727in}}%
\pgfpathlineto{\pgfqpoint{1.648692in}{0.692220in}}%
\pgfpathlineto{\pgfqpoint{1.665201in}{0.694714in}}%
\pgfpathlineto{\pgfqpoint{1.679582in}{0.697207in}}%
\pgfpathlineto{\pgfqpoint{1.681824in}{0.699700in}}%
\pgfpathlineto{\pgfqpoint{1.691189in}{0.702193in}}%
\pgfpathlineto{\pgfqpoint{1.691711in}{0.704686in}}%
\pgfpathlineto{\pgfqpoint{1.724243in}{0.707179in}}%
\pgfpathlineto{\pgfqpoint{1.731394in}{0.709672in}}%
\pgfpathlineto{\pgfqpoint{1.733611in}{0.712165in}}%
\pgfpathlineto{\pgfqpoint{1.782049in}{0.714659in}}%
\pgfpathlineto{\pgfqpoint{1.793838in}{0.717152in}}%
\pgfpathlineto{\pgfqpoint{1.828569in}{0.719645in}}%
\pgfpathlineto{\pgfqpoint{1.833818in}{0.722138in}}%
\pgfpathlineto{\pgfqpoint{1.840259in}{0.724631in}}%
\pgfpathlineto{\pgfqpoint{1.845773in}{0.727124in}}%
\pgfpathlineto{\pgfqpoint{1.865641in}{0.729617in}}%
\pgfpathlineto{\pgfqpoint{1.908000in}{0.732111in}}%
\pgfpathlineto{\pgfqpoint{1.913583in}{0.734604in}}%
\pgfusepath{stroke}%
\end{pgfscope}%
\begin{pgfscope}%
\pgfpathrectangle{\pgfqpoint{0.537394in}{0.467838in}}{\pgfqpoint{4.094684in}{0.997254in}}%
\pgfusepath{clip}%
\pgfsetrectcap%
\pgfsetroundjoin%
\pgfsetlinewidth{1.003750pt}%
\definecolor{currentstroke}{rgb}{1.000000,0.498039,0.054902}%
\pgfsetstrokecolor{currentstroke}%
\pgfsetdash{}{0pt}%
\pgfpathmoveto{\pgfqpoint{1.587682in}{0.470331in}}%
\pgfpathlineto{\pgfqpoint{1.606153in}{0.472825in}}%
\pgfpathlineto{\pgfqpoint{1.610623in}{0.475318in}}%
\pgfpathlineto{\pgfqpoint{1.610786in}{0.477811in}}%
\pgfpathlineto{\pgfqpoint{1.614743in}{0.480304in}}%
\pgfpathlineto{\pgfqpoint{1.614896in}{0.482797in}}%
\pgfpathlineto{\pgfqpoint{1.618575in}{0.485290in}}%
\pgfpathlineto{\pgfqpoint{1.620140in}{0.487783in}}%
\pgfpathlineto{\pgfqpoint{1.624884in}{0.490277in}}%
\pgfpathlineto{\pgfqpoint{1.645141in}{0.492770in}}%
\pgfpathlineto{\pgfqpoint{1.650265in}{0.495263in}}%
\pgfpathlineto{\pgfqpoint{1.653662in}{0.497756in}}%
\pgfpathlineto{\pgfqpoint{1.655956in}{0.500249in}}%
\pgfpathlineto{\pgfqpoint{1.656218in}{0.502742in}}%
\pgfpathlineto{\pgfqpoint{1.656255in}{0.505235in}}%
\pgfpathlineto{\pgfqpoint{1.656703in}{0.507728in}}%
\pgfpathlineto{\pgfqpoint{1.658222in}{0.510222in}}%
\pgfpathlineto{\pgfqpoint{1.660531in}{0.512715in}}%
\pgfpathlineto{\pgfqpoint{1.660700in}{0.515208in}}%
\pgfpathlineto{\pgfqpoint{1.660728in}{0.517701in}}%
\pgfpathlineto{\pgfqpoint{1.662344in}{0.520194in}}%
\pgfpathlineto{\pgfqpoint{1.663169in}{0.522687in}}%
\pgfpathlineto{\pgfqpoint{1.663604in}{0.525180in}}%
\pgfpathlineto{\pgfqpoint{1.664528in}{0.527674in}}%
\pgfpathlineto{\pgfqpoint{1.666994in}{0.530167in}}%
\pgfpathlineto{\pgfqpoint{1.667537in}{0.532660in}}%
\pgfpathlineto{\pgfqpoint{1.670047in}{0.535153in}}%
\pgfpathlineto{\pgfqpoint{1.672522in}{0.537646in}}%
\pgfpathlineto{\pgfqpoint{1.673325in}{0.540139in}}%
\pgfpathlineto{\pgfqpoint{1.673396in}{0.542632in}}%
\pgfpathlineto{\pgfqpoint{1.674456in}{0.545125in}}%
\pgfpathlineto{\pgfqpoint{1.675032in}{0.547619in}}%
\pgfpathlineto{\pgfqpoint{1.678001in}{0.550112in}}%
\pgfpathlineto{\pgfqpoint{1.681131in}{0.552605in}}%
\pgfpathlineto{\pgfqpoint{1.684848in}{0.555098in}}%
\pgfpathlineto{\pgfqpoint{1.687347in}{0.557591in}}%
\pgfpathlineto{\pgfqpoint{1.689643in}{0.560084in}}%
\pgfpathlineto{\pgfqpoint{1.691249in}{0.562577in}}%
\pgfpathlineto{\pgfqpoint{1.692737in}{0.565071in}}%
\pgfpathlineto{\pgfqpoint{1.693043in}{0.567564in}}%
\pgfpathlineto{\pgfqpoint{1.695298in}{0.570057in}}%
\pgfpathlineto{\pgfqpoint{1.695580in}{0.572550in}}%
\pgfpathlineto{\pgfqpoint{1.695930in}{0.575043in}}%
\pgfpathlineto{\pgfqpoint{1.696780in}{0.577536in}}%
\pgfpathlineto{\pgfqpoint{1.701227in}{0.580029in}}%
\pgfpathlineto{\pgfqpoint{1.703074in}{0.582523in}}%
\pgfpathlineto{\pgfqpoint{1.708703in}{0.585016in}}%
\pgfpathlineto{\pgfqpoint{1.709323in}{0.587509in}}%
\pgfpathlineto{\pgfqpoint{1.710241in}{0.590002in}}%
\pgfpathlineto{\pgfqpoint{1.710427in}{0.592495in}}%
\pgfpathlineto{\pgfqpoint{1.713202in}{0.594988in}}%
\pgfpathlineto{\pgfqpoint{1.715938in}{0.597481in}}%
\pgfpathlineto{\pgfqpoint{1.716185in}{0.599974in}}%
\pgfpathlineto{\pgfqpoint{1.723054in}{0.602468in}}%
\pgfpathlineto{\pgfqpoint{1.723943in}{0.604961in}}%
\pgfpathlineto{\pgfqpoint{1.726644in}{0.607454in}}%
\pgfpathlineto{\pgfqpoint{1.728389in}{0.609947in}}%
\pgfpathlineto{\pgfqpoint{1.731767in}{0.612440in}}%
\pgfpathlineto{\pgfqpoint{1.732259in}{0.614933in}}%
\pgfpathlineto{\pgfqpoint{1.733018in}{0.617426in}}%
\pgfpathlineto{\pgfqpoint{1.746308in}{0.619920in}}%
\pgfpathlineto{\pgfqpoint{1.750463in}{0.622413in}}%
\pgfpathlineto{\pgfqpoint{1.751782in}{0.624906in}}%
\pgfpathlineto{\pgfqpoint{1.753319in}{0.627399in}}%
\pgfpathlineto{\pgfqpoint{1.758139in}{0.629892in}}%
\pgfpathlineto{\pgfqpoint{1.768082in}{0.632385in}}%
\pgfpathlineto{\pgfqpoint{1.783506in}{0.634878in}}%
\pgfpathlineto{\pgfqpoint{1.784600in}{0.637371in}}%
\pgfpathlineto{\pgfqpoint{1.785287in}{0.639865in}}%
\pgfpathlineto{\pgfqpoint{1.799146in}{0.642358in}}%
\pgfpathlineto{\pgfqpoint{1.844278in}{0.644851in}}%
\pgfpathlineto{\pgfqpoint{1.862830in}{0.647344in}}%
\pgfpathlineto{\pgfqpoint{1.967921in}{0.649837in}}%
\pgfusepath{stroke}%
\end{pgfscope}%
\begin{pgfscope}%
\pgfpathrectangle{\pgfqpoint{0.537394in}{0.467838in}}{\pgfqpoint{4.094684in}{0.997254in}}%
\pgfusepath{clip}%
\pgfsetrectcap%
\pgfsetroundjoin%
\pgfsetlinewidth{1.003750pt}%
\definecolor{currentstroke}{rgb}{0.172549,0.627451,0.172549}%
\pgfsetstrokecolor{currentstroke}%
\pgfsetdash{}{0pt}%
\pgfpathmoveto{\pgfqpoint{2.629336in}{0.470331in}}%
\pgfpathlineto{\pgfqpoint{2.630777in}{0.472825in}}%
\pgfpathlineto{\pgfqpoint{2.632740in}{0.475318in}}%
\pgfpathlineto{\pgfqpoint{2.634423in}{0.477811in}}%
\pgfpathlineto{\pgfqpoint{2.634777in}{0.480304in}}%
\pgfpathlineto{\pgfqpoint{2.638446in}{0.482797in}}%
\pgfpathlineto{\pgfqpoint{2.639466in}{0.485290in}}%
\pgfpathlineto{\pgfqpoint{2.640442in}{0.487783in}}%
\pgfpathlineto{\pgfqpoint{2.645044in}{0.490277in}}%
\pgfpathlineto{\pgfqpoint{2.646381in}{0.492770in}}%
\pgfpathlineto{\pgfqpoint{2.648579in}{0.495263in}}%
\pgfpathlineto{\pgfqpoint{2.649997in}{0.497756in}}%
\pgfpathlineto{\pgfqpoint{2.650508in}{0.500249in}}%
\pgfpathlineto{\pgfqpoint{2.651662in}{0.502742in}}%
\pgfpathlineto{\pgfqpoint{2.652057in}{0.505235in}}%
\pgfpathlineto{\pgfqpoint{2.653167in}{0.507728in}}%
\pgfpathlineto{\pgfqpoint{2.659152in}{0.510222in}}%
\pgfpathlineto{\pgfqpoint{2.689255in}{0.512715in}}%
\pgfpathlineto{\pgfqpoint{2.703416in}{0.515208in}}%
\pgfpathlineto{\pgfqpoint{2.769193in}{0.517701in}}%
\pgfpathlineto{\pgfqpoint{2.771843in}{0.520194in}}%
\pgfpathlineto{\pgfqpoint{2.774546in}{0.522687in}}%
\pgfpathlineto{\pgfqpoint{2.777503in}{0.525180in}}%
\pgfpathlineto{\pgfqpoint{2.780019in}{0.527674in}}%
\pgfpathlineto{\pgfqpoint{2.780519in}{0.530167in}}%
\pgfpathlineto{\pgfqpoint{2.782335in}{0.532660in}}%
\pgfpathlineto{\pgfqpoint{2.791381in}{0.535153in}}%
\pgfpathlineto{\pgfqpoint{2.793183in}{0.537646in}}%
\pgfpathlineto{\pgfqpoint{2.797060in}{0.540139in}}%
\pgfpathlineto{\pgfqpoint{2.800837in}{0.542632in}}%
\pgfpathlineto{\pgfqpoint{2.806680in}{0.545125in}}%
\pgfpathlineto{\pgfqpoint{2.808580in}{0.547619in}}%
\pgfpathlineto{\pgfqpoint{2.808834in}{0.550112in}}%
\pgfpathlineto{\pgfqpoint{2.817504in}{0.552605in}}%
\pgfpathlineto{\pgfqpoint{2.817930in}{0.555098in}}%
\pgfpathlineto{\pgfqpoint{2.820964in}{0.557591in}}%
\pgfpathlineto{\pgfqpoint{2.821953in}{0.560084in}}%
\pgfpathlineto{\pgfqpoint{2.826339in}{0.562577in}}%
\pgfpathlineto{\pgfqpoint{2.827180in}{0.565071in}}%
\pgfpathlineto{\pgfqpoint{2.833731in}{0.567564in}}%
\pgfpathlineto{\pgfqpoint{2.833919in}{0.570057in}}%
\pgfpathlineto{\pgfqpoint{2.836498in}{0.572550in}}%
\pgfpathlineto{\pgfqpoint{2.841966in}{0.575043in}}%
\pgfpathlineto{\pgfqpoint{2.847268in}{0.577536in}}%
\pgfpathlineto{\pgfqpoint{2.849184in}{0.580029in}}%
\pgfpathlineto{\pgfqpoint{2.850101in}{0.582523in}}%
\pgfpathlineto{\pgfqpoint{2.850370in}{0.585016in}}%
\pgfpathlineto{\pgfqpoint{2.856100in}{0.587509in}}%
\pgfpathlineto{\pgfqpoint{2.861029in}{0.590002in}}%
\pgfpathlineto{\pgfqpoint{2.864223in}{0.592495in}}%
\pgfpathlineto{\pgfqpoint{2.870729in}{0.594988in}}%
\pgfpathlineto{\pgfqpoint{2.870856in}{0.597481in}}%
\pgfpathlineto{\pgfqpoint{2.872796in}{0.599974in}}%
\pgfpathlineto{\pgfqpoint{2.873479in}{0.602468in}}%
\pgfpathlineto{\pgfqpoint{2.884455in}{0.604961in}}%
\pgfpathlineto{\pgfqpoint{2.890655in}{0.607454in}}%
\pgfpathlineto{\pgfqpoint{2.896335in}{0.609947in}}%
\pgfpathlineto{\pgfqpoint{2.900271in}{0.612440in}}%
\pgfpathlineto{\pgfqpoint{2.900505in}{0.614933in}}%
\pgfpathlineto{\pgfqpoint{2.902393in}{0.617426in}}%
\pgfpathlineto{\pgfqpoint{2.904381in}{0.619920in}}%
\pgfpathlineto{\pgfqpoint{2.912657in}{0.622413in}}%
\pgfpathlineto{\pgfqpoint{2.946906in}{0.624906in}}%
\pgfpathlineto{\pgfqpoint{2.954005in}{0.627399in}}%
\pgfpathlineto{\pgfqpoint{2.955367in}{0.629892in}}%
\pgfpathlineto{\pgfqpoint{2.964918in}{0.632385in}}%
\pgfpathlineto{\pgfqpoint{2.969194in}{0.634878in}}%
\pgfpathlineto{\pgfqpoint{2.978289in}{0.637371in}}%
\pgfpathlineto{\pgfqpoint{2.978306in}{0.639865in}}%
\pgfpathlineto{\pgfqpoint{2.980802in}{0.642358in}}%
\pgfpathlineto{\pgfqpoint{2.985517in}{0.644851in}}%
\pgfpathlineto{\pgfqpoint{3.005528in}{0.647344in}}%
\pgfpathlineto{\pgfqpoint{3.009720in}{0.649837in}}%
\pgfpathlineto{\pgfqpoint{3.013024in}{0.652330in}}%
\pgfpathlineto{\pgfqpoint{3.021095in}{0.654823in}}%
\pgfpathlineto{\pgfqpoint{3.040409in}{0.657317in}}%
\pgfpathlineto{\pgfqpoint{3.115390in}{0.659810in}}%
\pgfpathlineto{\pgfqpoint{3.146666in}{0.662303in}}%
\pgfpathlineto{\pgfqpoint{3.179080in}{0.664796in}}%
\pgfpathlineto{\pgfqpoint{3.197795in}{0.667289in}}%
\pgfpathlineto{\pgfqpoint{3.199165in}{0.669782in}}%
\pgfpathlineto{\pgfqpoint{3.213815in}{0.672275in}}%
\pgfpathlineto{\pgfqpoint{3.242407in}{0.674768in}}%
\pgfusepath{stroke}%
\end{pgfscope}%
\begin{pgfscope}%
\pgfpathrectangle{\pgfqpoint{0.537394in}{0.467838in}}{\pgfqpoint{4.094684in}{0.997254in}}%
\pgfusepath{clip}%
\pgfsetbuttcap%
\pgfsetroundjoin%
\pgfsetlinewidth{1.003750pt}%
\definecolor{currentstroke}{rgb}{0.839216,0.152941,0.156863}%
\pgfsetstrokecolor{currentstroke}%
\pgfsetdash{{3.700000pt}{1.600000pt}}{0.000000pt}%
\pgfpathmoveto{\pgfqpoint{1.109806in}{0.480304in}}%
\pgfpathlineto{\pgfqpoint{1.174651in}{0.485290in}}%
\pgfpathlineto{\pgfqpoint{1.229476in}{0.490277in}}%
\pgfpathlineto{\pgfqpoint{1.318858in}{0.497756in}}%
\pgfpathlineto{\pgfqpoint{1.356331in}{0.500249in}}%
\pgfpathlineto{\pgfqpoint{1.390229in}{0.502742in}}%
\pgfpathlineto{\pgfqpoint{1.421175in}{0.505235in}}%
\pgfpathlineto{\pgfqpoint{1.476000in}{0.507728in}}%
\pgfpathlineto{\pgfqpoint{1.500538in}{0.512715in}}%
\pgfpathlineto{\pgfqpoint{1.565383in}{0.515208in}}%
\pgfpathlineto{\pgfqpoint{1.584612in}{0.517701in}}%
\pgfpathlineto{\pgfqpoint{1.636753in}{0.525180in}}%
\pgfpathlineto{\pgfqpoint{1.652563in}{0.530167in}}%
\pgfpathlineto{\pgfqpoint{1.667700in}{0.532660in}}%
\pgfpathlineto{\pgfqpoint{1.696168in}{0.535153in}}%
\pgfpathlineto{\pgfqpoint{1.709590in}{0.540139in}}%
\pgfpathlineto{\pgfqpoint{1.722525in}{0.545125in}}%
\pgfpathlineto{\pgfqpoint{1.849380in}{0.547619in}}%
\pgfusepath{stroke}%
\end{pgfscope}%
\begin{pgfscope}%
\pgfpathrectangle{\pgfqpoint{0.537394in}{0.467838in}}{\pgfqpoint{4.094684in}{0.997254in}}%
\pgfusepath{clip}%
\pgfsetbuttcap%
\pgfsetroundjoin%
\pgfsetlinewidth{1.003750pt}%
\definecolor{currentstroke}{rgb}{0.580392,0.403922,0.741176}%
\pgfsetstrokecolor{currentstroke}%
\pgfsetdash{{3.700000pt}{1.600000pt}}{0.000000pt}%
\pgfpathmoveto{\pgfqpoint{1.109806in}{0.477811in}}%
\pgfpathlineto{\pgfqpoint{1.174651in}{0.482797in}}%
\pgfpathlineto{\pgfqpoint{1.276968in}{0.485290in}}%
\pgfpathlineto{\pgfqpoint{1.318858in}{0.487783in}}%
\pgfpathlineto{\pgfqpoint{1.390229in}{0.490277in}}%
\pgfpathlineto{\pgfqpoint{1.421175in}{0.492770in}}%
\pgfpathlineto{\pgfqpoint{1.449643in}{0.497756in}}%
\pgfpathlineto{\pgfqpoint{1.476000in}{0.505235in}}%
\pgfpathlineto{\pgfqpoint{1.500538in}{0.510222in}}%
\pgfpathlineto{\pgfqpoint{1.523492in}{0.517701in}}%
\pgfpathlineto{\pgfqpoint{1.545054in}{0.520194in}}%
\pgfpathlineto{\pgfqpoint{1.565383in}{0.530167in}}%
\pgfpathlineto{\pgfqpoint{1.584612in}{0.535153in}}%
\pgfpathlineto{\pgfqpoint{1.620208in}{0.540139in}}%
\pgfpathlineto{\pgfqpoint{1.652563in}{0.545125in}}%
\pgfpathlineto{\pgfqpoint{1.709590in}{0.550112in}}%
\pgfpathlineto{\pgfqpoint{1.747063in}{0.552605in}}%
\pgfpathlineto{\pgfqpoint{1.770017in}{0.555098in}}%
\pgfpathlineto{\pgfqpoint{1.811907in}{0.557591in}}%
\pgfpathlineto{\pgfqpoint{1.831137in}{0.560084in}}%
\pgfpathlineto{\pgfqpoint{1.840375in}{0.562577in}}%
\pgfpathlineto{\pgfqpoint{1.866732in}{0.565071in}}%
\pgfpathlineto{\pgfqpoint{1.875101in}{0.567564in}}%
\pgfpathlineto{\pgfqpoint{1.899087in}{0.570057in}}%
\pgfpathlineto{\pgfqpoint{1.914224in}{0.572550in}}%
\pgfpathlineto{\pgfqpoint{1.921558in}{0.575043in}}%
\pgfpathlineto{\pgfqpoint{1.987610in}{0.577536in}}%
\pgfpathlineto{\pgfqpoint{1.993587in}{0.580029in}}%
\pgfpathlineto{\pgfqpoint{2.016541in}{0.582523in}}%
\pgfpathlineto{\pgfqpoint{2.038103in}{0.585016in}}%
\pgfpathlineto{\pgfqpoint{2.043295in}{0.587509in}}%
\pgfpathlineto{\pgfqpoint{2.095904in}{0.590002in}}%
\pgfpathlineto{\pgfqpoint{2.121626in}{0.592495in}}%
\pgfpathlineto{\pgfqpoint{2.164434in}{0.594988in}}%
\pgfusepath{stroke}%
\end{pgfscope}%
\begin{pgfscope}%
\pgfsetrectcap%
\pgfsetmiterjoin%
\pgfsetlinewidth{0.803000pt}%
\definecolor{currentstroke}{rgb}{0.000000,0.000000,0.000000}%
\pgfsetstrokecolor{currentstroke}%
\pgfsetdash{}{0pt}%
\pgfpathmoveto{\pgfqpoint{0.537394in}{0.467838in}}%
\pgfpathlineto{\pgfqpoint{0.537394in}{1.465092in}}%
\pgfusepath{stroke}%
\end{pgfscope}%
\begin{pgfscope}%
\pgfsetrectcap%
\pgfsetmiterjoin%
\pgfsetlinewidth{0.803000pt}%
\definecolor{currentstroke}{rgb}{0.000000,0.000000,0.000000}%
\pgfsetstrokecolor{currentstroke}%
\pgfsetdash{}{0pt}%
\pgfpathmoveto{\pgfqpoint{4.632078in}{0.467838in}}%
\pgfpathlineto{\pgfqpoint{4.632078in}{1.465092in}}%
\pgfusepath{stroke}%
\end{pgfscope}%
\begin{pgfscope}%
\pgfsetrectcap%
\pgfsetmiterjoin%
\pgfsetlinewidth{0.803000pt}%
\definecolor{currentstroke}{rgb}{0.000000,0.000000,0.000000}%
\pgfsetstrokecolor{currentstroke}%
\pgfsetdash{}{0pt}%
\pgfpathmoveto{\pgfqpoint{0.537394in}{0.467838in}}%
\pgfpathlineto{\pgfqpoint{4.632078in}{0.467838in}}%
\pgfusepath{stroke}%
\end{pgfscope}%
\begin{pgfscope}%
\pgfsetrectcap%
\pgfsetmiterjoin%
\pgfsetlinewidth{0.803000pt}%
\definecolor{currentstroke}{rgb}{0.000000,0.000000,0.000000}%
\pgfsetstrokecolor{currentstroke}%
\pgfsetdash{}{0pt}%
\pgfpathmoveto{\pgfqpoint{0.537394in}{1.465092in}}%
\pgfpathlineto{\pgfqpoint{4.632078in}{1.465092in}}%
\pgfusepath{stroke}%
\end{pgfscope}%
\begin{pgfscope}%
\pgfsetbuttcap%
\pgfsetmiterjoin%
\definecolor{currentfill}{rgb}{1.000000,1.000000,1.000000}%
\pgfsetfillcolor{currentfill}%
\pgfsetfillopacity{0.800000}%
\pgfsetlinewidth{1.003750pt}%
\definecolor{currentstroke}{rgb}{0.800000,0.800000,0.800000}%
\pgfsetstrokecolor{currentstroke}%
\pgfsetstrokeopacity{0.800000}%
\pgfsetdash{}{0pt}%
\pgfpathmoveto{\pgfqpoint{3.361234in}{0.560774in}}%
\pgfpathlineto{\pgfqpoint{4.554300in}{0.560774in}}%
\pgfpathquadraticcurveto{\pgfqpoint{4.576522in}{0.560774in}}{\pgfqpoint{4.576522in}{0.582996in}}%
\pgfpathlineto{\pgfqpoint{4.576522in}{1.387314in}}%
\pgfpathquadraticcurveto{\pgfqpoint{4.576522in}{1.409536in}}{\pgfqpoint{4.554300in}{1.409536in}}%
\pgfpathlineto{\pgfqpoint{3.361234in}{1.409536in}}%
\pgfpathquadraticcurveto{\pgfqpoint{3.339012in}{1.409536in}}{\pgfqpoint{3.339012in}{1.387314in}}%
\pgfpathlineto{\pgfqpoint{3.339012in}{0.582996in}}%
\pgfpathquadraticcurveto{\pgfqpoint{3.339012in}{0.560774in}}{\pgfqpoint{3.361234in}{0.560774in}}%
\pgfpathclose%
\pgfusepath{stroke,fill}%
\end{pgfscope}%
\begin{pgfscope}%
\pgfsetrectcap%
\pgfsetroundjoin%
\pgfsetlinewidth{1.003750pt}%
\definecolor{currentstroke}{rgb}{0.121569,0.466667,0.705882}%
\pgfsetstrokecolor{currentstroke}%
\pgfsetdash{}{0pt}%
\pgfpathmoveto{\pgfqpoint{3.383456in}{1.319562in}}%
\pgfpathlineto{\pgfqpoint{3.605678in}{1.319562in}}%
\pgfusepath{stroke}%
\end{pgfscope}%
\begin{pgfscope}%
\definecolor{textcolor}{rgb}{0.000000,0.000000,0.000000}%
\pgfsetstrokecolor{textcolor}%
\pgfsetfillcolor{textcolor}%
\pgftext[x=3.694567in,y=1.280674in,left,base]{\color{textcolor}\sffamily\fontsize{8.000000}{9.600000}\selectfont LG(FlowCutter)}%
\end{pgfscope}%
\begin{pgfscope}%
\pgfsetrectcap%
\pgfsetroundjoin%
\pgfsetlinewidth{1.003750pt}%
\definecolor{currentstroke}{rgb}{1.000000,0.498039,0.054902}%
\pgfsetstrokecolor{currentstroke}%
\pgfsetdash{}{0pt}%
\pgfpathmoveto{\pgfqpoint{3.383456in}{1.156477in}}%
\pgfpathlineto{\pgfqpoint{3.605678in}{1.156477in}}%
\pgfusepath{stroke}%
\end{pgfscope}%
\begin{pgfscope}%
\definecolor{textcolor}{rgb}{0.000000,0.000000,0.000000}%
\pgfsetstrokecolor{textcolor}%
\pgfsetfillcolor{textcolor}%
\pgftext[x=3.694567in,y=1.117588in,left,base]{\color{textcolor}\sffamily\fontsize{8.000000}{9.600000}\selectfont LG(htd)}%
\end{pgfscope}%
\begin{pgfscope}%
\pgfsetrectcap%
\pgfsetroundjoin%
\pgfsetlinewidth{1.003750pt}%
\definecolor{currentstroke}{rgb}{0.172549,0.627451,0.172549}%
\pgfsetstrokecolor{currentstroke}%
\pgfsetdash{}{0pt}%
\pgfpathmoveto{\pgfqpoint{3.383456in}{0.993391in}}%
\pgfpathlineto{\pgfqpoint{3.605678in}{0.993391in}}%
\pgfusepath{stroke}%
\end{pgfscope}%
\begin{pgfscope}%
\definecolor{textcolor}{rgb}{0.000000,0.000000,0.000000}%
\pgfsetstrokecolor{textcolor}%
\pgfsetfillcolor{textcolor}%
\pgftext[x=3.694567in,y=0.954502in,left,base]{\color{textcolor}\sffamily\fontsize{8.000000}{9.600000}\selectfont LG(Tamaki)}%
\end{pgfscope}%
\begin{pgfscope}%
\pgfsetbuttcap%
\pgfsetroundjoin%
\pgfsetlinewidth{1.003750pt}%
\definecolor{currentstroke}{rgb}{0.839216,0.152941,0.156863}%
\pgfsetstrokecolor{currentstroke}%
\pgfsetdash{{3.700000pt}{1.600000pt}}{0.000000pt}%
\pgfpathmoveto{\pgfqpoint{3.383456in}{0.830305in}}%
\pgfpathlineto{\pgfqpoint{3.605678in}{0.830305in}}%
\pgfusepath{stroke}%
\end{pgfscope}%
\begin{pgfscope}%
\definecolor{textcolor}{rgb}{0.000000,0.000000,0.000000}%
\pgfsetstrokecolor{textcolor}%
\pgfsetfillcolor{textcolor}%
\pgftext[x=3.694567in,y=0.791416in,left,base]{\color{textcolor}\sffamily\fontsize{8.000000}{9.600000}\selectfont HTB(MCS, BE)}%
\end{pgfscope}%
\begin{pgfscope}%
\pgfsetbuttcap%
\pgfsetroundjoin%
\pgfsetlinewidth{1.003750pt}%
\definecolor{currentstroke}{rgb}{0.580392,0.403922,0.741176}%
\pgfsetstrokecolor{currentstroke}%
\pgfsetdash{{3.700000pt}{1.600000pt}}{0.000000pt}%
\pgfpathmoveto{\pgfqpoint{3.383456in}{0.667219in}}%
\pgfpathlineto{\pgfqpoint{3.605678in}{0.667219in}}%
\pgfusepath{stroke}%
\end{pgfscope}%
\begin{pgfscope}%
\definecolor{textcolor}{rgb}{0.000000,0.000000,0.000000}%
\pgfsetstrokecolor{textcolor}%
\pgfsetfillcolor{textcolor}%
\pgftext[x=3.694567in,y=0.628330in,left,base]{\color{textcolor}\sffamily\fontsize{8.000000}{9.600000}\selectfont HTB(MCS, BM)}%
\end{pgfscope}%
\end{pgfpicture}%
\makeatother%
\endgroup%

    \caption{
        Experiment 1 compares the tree-decomposition-based planner \Lg{} to the constraint-satisfaction-based planner \htb{}.
	    A planner ``solves'' a benchmark when it finds a project-join tree of width \maxWidth{} or lower.
        % \Lg{} invokes a tree decomposer (\flowcutter{}, \htd{}, or \tamaki{}).
        % \Lg{} is an \emph{anytime} tool that produces several trees (of decreasing widths) for each benchmark.
        % We only show an \Lg{} data point if the first tree produced has width at most \maxWidth{}%
        % (we discard an \Lg{} data point when the first tree has width over \maxWidth, even if a later tree has width at most \maxWidth)%
        % .
        % \htb{} requires a variable-ordering heuristic (\mcs{}, \lexp, \lexm, or \minfill{}) and a clause-ordering heuristic (\be{} or \bm).
        For \htb, we only show the variable-ordering heuristic \mcs{}; the \lexp{}, \lexm{}, and \minfill{} curves are qualitatively similar.
    }
    \label{figPlanning}
\end{figure}

%%%%%%%%%%%%%%%%%%%%%%%%%%%%%%%%%%%%%%%%%%%%%%%%%%%%%%%%%%%%%%%%%%%%%%%%%%%%%%%%

\subsection{Experiment 2: Comparing Execution Heuristics}

In this experiment, we take the 346 graded project-join trees produced by \Lg{} with \flowcutter{} in Experiment 1 % (107 trees of widths 1-30 and 239 trees of widths 31-99)
and run \dmc{} once for 100 seconds with each of four ADD variable-ordering heuristics. 
We present the execution time of each heuristic (excluding planning time) in Figure \ref{figExecution}. 
We observe that \mcs{} and \lexp{} outperform \lexm{} and \minfill{}.
We use \dmc{} with \mcs{} in \procount{} for Experiment 3.
\begin{figure}[t]
    \centering
    %% Creator: Matplotlib, PGF backend
%%
%% To include the figure in your LaTeX document, write
%%   \input{<filename>.pgf}
%%
%% Make sure the required packages are loaded in your preamble
%%   \usepackage{pgf}
%%
%% and, on pdftex
%%   \usepackage[utf8]{inputenc}\DeclareUnicodeCharacter{2212}{-}
%%
%% or, on luatex and xetex
%%   \usepackage{unicode-math}
%%
%% Figures using additional raster images can only be included by \input if
%% they are in the same directory as the main LaTeX file. For loading figures
%% from other directories you can use the `import` package
%%   \usepackage{import}
%%
%% and then include the figures with
%%   \import{<path to file>}{<filename>.pgf}
%%
%% Matplotlib used the following preamble
%%   \usepackage{fontspec}
%%   \setmainfont{DejaVuSerif.ttf}[Path=/home/vhp1/.local/lib/python3.8/site-packages/matplotlib/mpl-data/fonts/ttf/]
%%   \setsansfont{DejaVuSans.ttf}[Path=/home/vhp1/.local/lib/python3.8/site-packages/matplotlib/mpl-data/fonts/ttf/]
%%   \setmonofont{DejaVuSansMono.ttf}[Path=/home/vhp1/.local/lib/python3.8/site-packages/matplotlib/mpl-data/fonts/ttf/]
%%
\begingroup%
\makeatletter%
\begin{pgfpicture}%
\pgfpathrectangle{\pgfpointorigin}{\pgfqpoint{4.820041in}{1.610194in}}%
\pgfusepath{use as bounding box, clip}%
\begin{pgfscope}%
\pgfsetbuttcap%
\pgfsetmiterjoin%
\pgfsetlinewidth{0.000000pt}%
\definecolor{currentstroke}{rgb}{1.000000,1.000000,1.000000}%
\pgfsetstrokecolor{currentstroke}%
\pgfsetstrokeopacity{0.000000}%
\pgfsetdash{}{0pt}%
\pgfpathmoveto{\pgfqpoint{0.000000in}{0.000000in}}%
\pgfpathlineto{\pgfqpoint{4.820041in}{0.000000in}}%
\pgfpathlineto{\pgfqpoint{4.820041in}{1.610194in}}%
\pgfpathlineto{\pgfqpoint{0.000000in}{1.610194in}}%
\pgfpathclose%
\pgfusepath{}%
\end{pgfscope}%
\begin{pgfscope}%
\pgfsetbuttcap%
\pgfsetmiterjoin%
\definecolor{currentfill}{rgb}{1.000000,1.000000,1.000000}%
\pgfsetfillcolor{currentfill}%
\pgfsetlinewidth{0.000000pt}%
\definecolor{currentstroke}{rgb}{0.000000,0.000000,0.000000}%
\pgfsetstrokecolor{currentstroke}%
\pgfsetstrokeopacity{0.000000}%
\pgfsetdash{}{0pt}%
\pgfpathmoveto{\pgfqpoint{0.537394in}{0.467838in}}%
\pgfpathlineto{\pgfqpoint{4.632078in}{0.467838in}}%
\pgfpathlineto{\pgfqpoint{4.632078in}{1.465092in}}%
\pgfpathlineto{\pgfqpoint{0.537394in}{1.465092in}}%
\pgfpathclose%
\pgfusepath{fill}%
\end{pgfscope}%
\begin{pgfscope}%
\pgfsetbuttcap%
\pgfsetroundjoin%
\definecolor{currentfill}{rgb}{0.000000,0.000000,0.000000}%
\pgfsetfillcolor{currentfill}%
\pgfsetlinewidth{0.803000pt}%
\definecolor{currentstroke}{rgb}{0.000000,0.000000,0.000000}%
\pgfsetstrokecolor{currentstroke}%
\pgfsetdash{}{0pt}%
\pgfsys@defobject{currentmarker}{\pgfqpoint{0.000000in}{-0.048611in}}{\pgfqpoint{0.000000in}{0.000000in}}{%
\pgfpathmoveto{\pgfqpoint{0.000000in}{0.000000in}}%
\pgfpathlineto{\pgfqpoint{0.000000in}{-0.048611in}}%
\pgfusepath{stroke,fill}%
}%
\begin{pgfscope}%
\pgfsys@transformshift{0.537394in}{0.467838in}%
\pgfsys@useobject{currentmarker}{}%
\end{pgfscope}%
\end{pgfscope}%
\begin{pgfscope}%
\definecolor{textcolor}{rgb}{0.000000,0.000000,0.000000}%
\pgfsetstrokecolor{textcolor}%
\pgfsetfillcolor{textcolor}%
\pgftext[x=0.537394in,y=0.370616in,,top]{\color{textcolor}\sffamily\fontsize{8.000000}{9.600000}\selectfont \(\displaystyle {10^{-3}}\)}%
\end{pgfscope}%
\begin{pgfscope}%
\pgfsetbuttcap%
\pgfsetroundjoin%
\definecolor{currentfill}{rgb}{0.000000,0.000000,0.000000}%
\pgfsetfillcolor{currentfill}%
\pgfsetlinewidth{0.803000pt}%
\definecolor{currentstroke}{rgb}{0.000000,0.000000,0.000000}%
\pgfsetstrokecolor{currentstroke}%
\pgfsetdash{}{0pt}%
\pgfsys@defobject{currentmarker}{\pgfqpoint{0.000000in}{-0.048611in}}{\pgfqpoint{0.000000in}{0.000000in}}{%
\pgfpathmoveto{\pgfqpoint{0.000000in}{0.000000in}}%
\pgfpathlineto{\pgfqpoint{0.000000in}{-0.048611in}}%
\pgfusepath{stroke,fill}%
}%
\begin{pgfscope}%
\pgfsys@transformshift{1.356331in}{0.467838in}%
\pgfsys@useobject{currentmarker}{}%
\end{pgfscope}%
\end{pgfscope}%
\begin{pgfscope}%
\definecolor{textcolor}{rgb}{0.000000,0.000000,0.000000}%
\pgfsetstrokecolor{textcolor}%
\pgfsetfillcolor{textcolor}%
\pgftext[x=1.356331in,y=0.370616in,,top]{\color{textcolor}\sffamily\fontsize{8.000000}{9.600000}\selectfont \(\displaystyle {10^{-2}}\)}%
\end{pgfscope}%
\begin{pgfscope}%
\pgfsetbuttcap%
\pgfsetroundjoin%
\definecolor{currentfill}{rgb}{0.000000,0.000000,0.000000}%
\pgfsetfillcolor{currentfill}%
\pgfsetlinewidth{0.803000pt}%
\definecolor{currentstroke}{rgb}{0.000000,0.000000,0.000000}%
\pgfsetstrokecolor{currentstroke}%
\pgfsetdash{}{0pt}%
\pgfsys@defobject{currentmarker}{\pgfqpoint{0.000000in}{-0.048611in}}{\pgfqpoint{0.000000in}{0.000000in}}{%
\pgfpathmoveto{\pgfqpoint{0.000000in}{0.000000in}}%
\pgfpathlineto{\pgfqpoint{0.000000in}{-0.048611in}}%
\pgfusepath{stroke,fill}%
}%
\begin{pgfscope}%
\pgfsys@transformshift{2.175268in}{0.467838in}%
\pgfsys@useobject{currentmarker}{}%
\end{pgfscope}%
\end{pgfscope}%
\begin{pgfscope}%
\definecolor{textcolor}{rgb}{0.000000,0.000000,0.000000}%
\pgfsetstrokecolor{textcolor}%
\pgfsetfillcolor{textcolor}%
\pgftext[x=2.175268in,y=0.370616in,,top]{\color{textcolor}\sffamily\fontsize{8.000000}{9.600000}\selectfont \(\displaystyle {10^{-1}}\)}%
\end{pgfscope}%
\begin{pgfscope}%
\pgfsetbuttcap%
\pgfsetroundjoin%
\definecolor{currentfill}{rgb}{0.000000,0.000000,0.000000}%
\pgfsetfillcolor{currentfill}%
\pgfsetlinewidth{0.803000pt}%
\definecolor{currentstroke}{rgb}{0.000000,0.000000,0.000000}%
\pgfsetstrokecolor{currentstroke}%
\pgfsetdash{}{0pt}%
\pgfsys@defobject{currentmarker}{\pgfqpoint{0.000000in}{-0.048611in}}{\pgfqpoint{0.000000in}{0.000000in}}{%
\pgfpathmoveto{\pgfqpoint{0.000000in}{0.000000in}}%
\pgfpathlineto{\pgfqpoint{0.000000in}{-0.048611in}}%
\pgfusepath{stroke,fill}%
}%
\begin{pgfscope}%
\pgfsys@transformshift{2.994204in}{0.467838in}%
\pgfsys@useobject{currentmarker}{}%
\end{pgfscope}%
\end{pgfscope}%
\begin{pgfscope}%
\definecolor{textcolor}{rgb}{0.000000,0.000000,0.000000}%
\pgfsetstrokecolor{textcolor}%
\pgfsetfillcolor{textcolor}%
\pgftext[x=2.994204in,y=0.370616in,,top]{\color{textcolor}\sffamily\fontsize{8.000000}{9.600000}\selectfont \(\displaystyle {10^{0}}\)}%
\end{pgfscope}%
\begin{pgfscope}%
\pgfsetbuttcap%
\pgfsetroundjoin%
\definecolor{currentfill}{rgb}{0.000000,0.000000,0.000000}%
\pgfsetfillcolor{currentfill}%
\pgfsetlinewidth{0.803000pt}%
\definecolor{currentstroke}{rgb}{0.000000,0.000000,0.000000}%
\pgfsetstrokecolor{currentstroke}%
\pgfsetdash{}{0pt}%
\pgfsys@defobject{currentmarker}{\pgfqpoint{0.000000in}{-0.048611in}}{\pgfqpoint{0.000000in}{0.000000in}}{%
\pgfpathmoveto{\pgfqpoint{0.000000in}{0.000000in}}%
\pgfpathlineto{\pgfqpoint{0.000000in}{-0.048611in}}%
\pgfusepath{stroke,fill}%
}%
\begin{pgfscope}%
\pgfsys@transformshift{3.813141in}{0.467838in}%
\pgfsys@useobject{currentmarker}{}%
\end{pgfscope}%
\end{pgfscope}%
\begin{pgfscope}%
\definecolor{textcolor}{rgb}{0.000000,0.000000,0.000000}%
\pgfsetstrokecolor{textcolor}%
\pgfsetfillcolor{textcolor}%
\pgftext[x=3.813141in,y=0.370616in,,top]{\color{textcolor}\sffamily\fontsize{8.000000}{9.600000}\selectfont \(\displaystyle {10^{1}}\)}%
\end{pgfscope}%
\begin{pgfscope}%
\pgfsetbuttcap%
\pgfsetroundjoin%
\definecolor{currentfill}{rgb}{0.000000,0.000000,0.000000}%
\pgfsetfillcolor{currentfill}%
\pgfsetlinewidth{0.803000pt}%
\definecolor{currentstroke}{rgb}{0.000000,0.000000,0.000000}%
\pgfsetstrokecolor{currentstroke}%
\pgfsetdash{}{0pt}%
\pgfsys@defobject{currentmarker}{\pgfqpoint{0.000000in}{-0.048611in}}{\pgfqpoint{0.000000in}{0.000000in}}{%
\pgfpathmoveto{\pgfqpoint{0.000000in}{0.000000in}}%
\pgfpathlineto{\pgfqpoint{0.000000in}{-0.048611in}}%
\pgfusepath{stroke,fill}%
}%
\begin{pgfscope}%
\pgfsys@transformshift{4.632078in}{0.467838in}%
\pgfsys@useobject{currentmarker}{}%
\end{pgfscope}%
\end{pgfscope}%
\begin{pgfscope}%
\definecolor{textcolor}{rgb}{0.000000,0.000000,0.000000}%
\pgfsetstrokecolor{textcolor}%
\pgfsetfillcolor{textcolor}%
\pgftext[x=4.632078in,y=0.370616in,,top]{\color{textcolor}\sffamily\fontsize{8.000000}{9.600000}\selectfont \(\displaystyle {10^{2}}\)}%
\end{pgfscope}%
\begin{pgfscope}%
\pgfsetbuttcap%
\pgfsetroundjoin%
\definecolor{currentfill}{rgb}{0.000000,0.000000,0.000000}%
\pgfsetfillcolor{currentfill}%
\pgfsetlinewidth{0.602250pt}%
\definecolor{currentstroke}{rgb}{0.000000,0.000000,0.000000}%
\pgfsetstrokecolor{currentstroke}%
\pgfsetdash{}{0pt}%
\pgfsys@defobject{currentmarker}{\pgfqpoint{0.000000in}{-0.027778in}}{\pgfqpoint{0.000000in}{0.000000in}}{%
\pgfpathmoveto{\pgfqpoint{0.000000in}{0.000000in}}%
\pgfpathlineto{\pgfqpoint{0.000000in}{-0.027778in}}%
\pgfusepath{stroke,fill}%
}%
\begin{pgfscope}%
\pgfsys@transformshift{0.783918in}{0.467838in}%
\pgfsys@useobject{currentmarker}{}%
\end{pgfscope}%
\end{pgfscope}%
\begin{pgfscope}%
\pgfsetbuttcap%
\pgfsetroundjoin%
\definecolor{currentfill}{rgb}{0.000000,0.000000,0.000000}%
\pgfsetfillcolor{currentfill}%
\pgfsetlinewidth{0.602250pt}%
\definecolor{currentstroke}{rgb}{0.000000,0.000000,0.000000}%
\pgfsetstrokecolor{currentstroke}%
\pgfsetdash{}{0pt}%
\pgfsys@defobject{currentmarker}{\pgfqpoint{0.000000in}{-0.027778in}}{\pgfqpoint{0.000000in}{0.000000in}}{%
\pgfpathmoveto{\pgfqpoint{0.000000in}{0.000000in}}%
\pgfpathlineto{\pgfqpoint{0.000000in}{-0.027778in}}%
\pgfusepath{stroke,fill}%
}%
\begin{pgfscope}%
\pgfsys@transformshift{0.928126in}{0.467838in}%
\pgfsys@useobject{currentmarker}{}%
\end{pgfscope}%
\end{pgfscope}%
\begin{pgfscope}%
\pgfsetbuttcap%
\pgfsetroundjoin%
\definecolor{currentfill}{rgb}{0.000000,0.000000,0.000000}%
\pgfsetfillcolor{currentfill}%
\pgfsetlinewidth{0.602250pt}%
\definecolor{currentstroke}{rgb}{0.000000,0.000000,0.000000}%
\pgfsetstrokecolor{currentstroke}%
\pgfsetdash{}{0pt}%
\pgfsys@defobject{currentmarker}{\pgfqpoint{0.000000in}{-0.027778in}}{\pgfqpoint{0.000000in}{0.000000in}}{%
\pgfpathmoveto{\pgfqpoint{0.000000in}{0.000000in}}%
\pgfpathlineto{\pgfqpoint{0.000000in}{-0.027778in}}%
\pgfusepath{stroke,fill}%
}%
\begin{pgfscope}%
\pgfsys@transformshift{1.030443in}{0.467838in}%
\pgfsys@useobject{currentmarker}{}%
\end{pgfscope}%
\end{pgfscope}%
\begin{pgfscope}%
\pgfsetbuttcap%
\pgfsetroundjoin%
\definecolor{currentfill}{rgb}{0.000000,0.000000,0.000000}%
\pgfsetfillcolor{currentfill}%
\pgfsetlinewidth{0.602250pt}%
\definecolor{currentstroke}{rgb}{0.000000,0.000000,0.000000}%
\pgfsetstrokecolor{currentstroke}%
\pgfsetdash{}{0pt}%
\pgfsys@defobject{currentmarker}{\pgfqpoint{0.000000in}{-0.027778in}}{\pgfqpoint{0.000000in}{0.000000in}}{%
\pgfpathmoveto{\pgfqpoint{0.000000in}{0.000000in}}%
\pgfpathlineto{\pgfqpoint{0.000000in}{-0.027778in}}%
\pgfusepath{stroke,fill}%
}%
\begin{pgfscope}%
\pgfsys@transformshift{1.109806in}{0.467838in}%
\pgfsys@useobject{currentmarker}{}%
\end{pgfscope}%
\end{pgfscope}%
\begin{pgfscope}%
\pgfsetbuttcap%
\pgfsetroundjoin%
\definecolor{currentfill}{rgb}{0.000000,0.000000,0.000000}%
\pgfsetfillcolor{currentfill}%
\pgfsetlinewidth{0.602250pt}%
\definecolor{currentstroke}{rgb}{0.000000,0.000000,0.000000}%
\pgfsetstrokecolor{currentstroke}%
\pgfsetdash{}{0pt}%
\pgfsys@defobject{currentmarker}{\pgfqpoint{0.000000in}{-0.027778in}}{\pgfqpoint{0.000000in}{0.000000in}}{%
\pgfpathmoveto{\pgfqpoint{0.000000in}{0.000000in}}%
\pgfpathlineto{\pgfqpoint{0.000000in}{-0.027778in}}%
\pgfusepath{stroke,fill}%
}%
\begin{pgfscope}%
\pgfsys@transformshift{1.174651in}{0.467838in}%
\pgfsys@useobject{currentmarker}{}%
\end{pgfscope}%
\end{pgfscope}%
\begin{pgfscope}%
\pgfsetbuttcap%
\pgfsetroundjoin%
\definecolor{currentfill}{rgb}{0.000000,0.000000,0.000000}%
\pgfsetfillcolor{currentfill}%
\pgfsetlinewidth{0.602250pt}%
\definecolor{currentstroke}{rgb}{0.000000,0.000000,0.000000}%
\pgfsetstrokecolor{currentstroke}%
\pgfsetdash{}{0pt}%
\pgfsys@defobject{currentmarker}{\pgfqpoint{0.000000in}{-0.027778in}}{\pgfqpoint{0.000000in}{0.000000in}}{%
\pgfpathmoveto{\pgfqpoint{0.000000in}{0.000000in}}%
\pgfpathlineto{\pgfqpoint{0.000000in}{-0.027778in}}%
\pgfusepath{stroke,fill}%
}%
\begin{pgfscope}%
\pgfsys@transformshift{1.229476in}{0.467838in}%
\pgfsys@useobject{currentmarker}{}%
\end{pgfscope}%
\end{pgfscope}%
\begin{pgfscope}%
\pgfsetbuttcap%
\pgfsetroundjoin%
\definecolor{currentfill}{rgb}{0.000000,0.000000,0.000000}%
\pgfsetfillcolor{currentfill}%
\pgfsetlinewidth{0.602250pt}%
\definecolor{currentstroke}{rgb}{0.000000,0.000000,0.000000}%
\pgfsetstrokecolor{currentstroke}%
\pgfsetdash{}{0pt}%
\pgfsys@defobject{currentmarker}{\pgfqpoint{0.000000in}{-0.027778in}}{\pgfqpoint{0.000000in}{0.000000in}}{%
\pgfpathmoveto{\pgfqpoint{0.000000in}{0.000000in}}%
\pgfpathlineto{\pgfqpoint{0.000000in}{-0.027778in}}%
\pgfusepath{stroke,fill}%
}%
\begin{pgfscope}%
\pgfsys@transformshift{1.276968in}{0.467838in}%
\pgfsys@useobject{currentmarker}{}%
\end{pgfscope}%
\end{pgfscope}%
\begin{pgfscope}%
\pgfsetbuttcap%
\pgfsetroundjoin%
\definecolor{currentfill}{rgb}{0.000000,0.000000,0.000000}%
\pgfsetfillcolor{currentfill}%
\pgfsetlinewidth{0.602250pt}%
\definecolor{currentstroke}{rgb}{0.000000,0.000000,0.000000}%
\pgfsetstrokecolor{currentstroke}%
\pgfsetdash{}{0pt}%
\pgfsys@defobject{currentmarker}{\pgfqpoint{0.000000in}{-0.027778in}}{\pgfqpoint{0.000000in}{0.000000in}}{%
\pgfpathmoveto{\pgfqpoint{0.000000in}{0.000000in}}%
\pgfpathlineto{\pgfqpoint{0.000000in}{-0.027778in}}%
\pgfusepath{stroke,fill}%
}%
\begin{pgfscope}%
\pgfsys@transformshift{1.318858in}{0.467838in}%
\pgfsys@useobject{currentmarker}{}%
\end{pgfscope}%
\end{pgfscope}%
\begin{pgfscope}%
\pgfsetbuttcap%
\pgfsetroundjoin%
\definecolor{currentfill}{rgb}{0.000000,0.000000,0.000000}%
\pgfsetfillcolor{currentfill}%
\pgfsetlinewidth{0.602250pt}%
\definecolor{currentstroke}{rgb}{0.000000,0.000000,0.000000}%
\pgfsetstrokecolor{currentstroke}%
\pgfsetdash{}{0pt}%
\pgfsys@defobject{currentmarker}{\pgfqpoint{0.000000in}{-0.027778in}}{\pgfqpoint{0.000000in}{0.000000in}}{%
\pgfpathmoveto{\pgfqpoint{0.000000in}{0.000000in}}%
\pgfpathlineto{\pgfqpoint{0.000000in}{-0.027778in}}%
\pgfusepath{stroke,fill}%
}%
\begin{pgfscope}%
\pgfsys@transformshift{1.602855in}{0.467838in}%
\pgfsys@useobject{currentmarker}{}%
\end{pgfscope}%
\end{pgfscope}%
\begin{pgfscope}%
\pgfsetbuttcap%
\pgfsetroundjoin%
\definecolor{currentfill}{rgb}{0.000000,0.000000,0.000000}%
\pgfsetfillcolor{currentfill}%
\pgfsetlinewidth{0.602250pt}%
\definecolor{currentstroke}{rgb}{0.000000,0.000000,0.000000}%
\pgfsetstrokecolor{currentstroke}%
\pgfsetdash{}{0pt}%
\pgfsys@defobject{currentmarker}{\pgfqpoint{0.000000in}{-0.027778in}}{\pgfqpoint{0.000000in}{0.000000in}}{%
\pgfpathmoveto{\pgfqpoint{0.000000in}{0.000000in}}%
\pgfpathlineto{\pgfqpoint{0.000000in}{-0.027778in}}%
\pgfusepath{stroke,fill}%
}%
\begin{pgfscope}%
\pgfsys@transformshift{1.747063in}{0.467838in}%
\pgfsys@useobject{currentmarker}{}%
\end{pgfscope}%
\end{pgfscope}%
\begin{pgfscope}%
\pgfsetbuttcap%
\pgfsetroundjoin%
\definecolor{currentfill}{rgb}{0.000000,0.000000,0.000000}%
\pgfsetfillcolor{currentfill}%
\pgfsetlinewidth{0.602250pt}%
\definecolor{currentstroke}{rgb}{0.000000,0.000000,0.000000}%
\pgfsetstrokecolor{currentstroke}%
\pgfsetdash{}{0pt}%
\pgfsys@defobject{currentmarker}{\pgfqpoint{0.000000in}{-0.027778in}}{\pgfqpoint{0.000000in}{0.000000in}}{%
\pgfpathmoveto{\pgfqpoint{0.000000in}{0.000000in}}%
\pgfpathlineto{\pgfqpoint{0.000000in}{-0.027778in}}%
\pgfusepath{stroke,fill}%
}%
\begin{pgfscope}%
\pgfsys@transformshift{1.849380in}{0.467838in}%
\pgfsys@useobject{currentmarker}{}%
\end{pgfscope}%
\end{pgfscope}%
\begin{pgfscope}%
\pgfsetbuttcap%
\pgfsetroundjoin%
\definecolor{currentfill}{rgb}{0.000000,0.000000,0.000000}%
\pgfsetfillcolor{currentfill}%
\pgfsetlinewidth{0.602250pt}%
\definecolor{currentstroke}{rgb}{0.000000,0.000000,0.000000}%
\pgfsetstrokecolor{currentstroke}%
\pgfsetdash{}{0pt}%
\pgfsys@defobject{currentmarker}{\pgfqpoint{0.000000in}{-0.027778in}}{\pgfqpoint{0.000000in}{0.000000in}}{%
\pgfpathmoveto{\pgfqpoint{0.000000in}{0.000000in}}%
\pgfpathlineto{\pgfqpoint{0.000000in}{-0.027778in}}%
\pgfusepath{stroke,fill}%
}%
\begin{pgfscope}%
\pgfsys@transformshift{1.928743in}{0.467838in}%
\pgfsys@useobject{currentmarker}{}%
\end{pgfscope}%
\end{pgfscope}%
\begin{pgfscope}%
\pgfsetbuttcap%
\pgfsetroundjoin%
\definecolor{currentfill}{rgb}{0.000000,0.000000,0.000000}%
\pgfsetfillcolor{currentfill}%
\pgfsetlinewidth{0.602250pt}%
\definecolor{currentstroke}{rgb}{0.000000,0.000000,0.000000}%
\pgfsetstrokecolor{currentstroke}%
\pgfsetdash{}{0pt}%
\pgfsys@defobject{currentmarker}{\pgfqpoint{0.000000in}{-0.027778in}}{\pgfqpoint{0.000000in}{0.000000in}}{%
\pgfpathmoveto{\pgfqpoint{0.000000in}{0.000000in}}%
\pgfpathlineto{\pgfqpoint{0.000000in}{-0.027778in}}%
\pgfusepath{stroke,fill}%
}%
\begin{pgfscope}%
\pgfsys@transformshift{1.993587in}{0.467838in}%
\pgfsys@useobject{currentmarker}{}%
\end{pgfscope}%
\end{pgfscope}%
\begin{pgfscope}%
\pgfsetbuttcap%
\pgfsetroundjoin%
\definecolor{currentfill}{rgb}{0.000000,0.000000,0.000000}%
\pgfsetfillcolor{currentfill}%
\pgfsetlinewidth{0.602250pt}%
\definecolor{currentstroke}{rgb}{0.000000,0.000000,0.000000}%
\pgfsetstrokecolor{currentstroke}%
\pgfsetdash{}{0pt}%
\pgfsys@defobject{currentmarker}{\pgfqpoint{0.000000in}{-0.027778in}}{\pgfqpoint{0.000000in}{0.000000in}}{%
\pgfpathmoveto{\pgfqpoint{0.000000in}{0.000000in}}%
\pgfpathlineto{\pgfqpoint{0.000000in}{-0.027778in}}%
\pgfusepath{stroke,fill}%
}%
\begin{pgfscope}%
\pgfsys@transformshift{2.048413in}{0.467838in}%
\pgfsys@useobject{currentmarker}{}%
\end{pgfscope}%
\end{pgfscope}%
\begin{pgfscope}%
\pgfsetbuttcap%
\pgfsetroundjoin%
\definecolor{currentfill}{rgb}{0.000000,0.000000,0.000000}%
\pgfsetfillcolor{currentfill}%
\pgfsetlinewidth{0.602250pt}%
\definecolor{currentstroke}{rgb}{0.000000,0.000000,0.000000}%
\pgfsetstrokecolor{currentstroke}%
\pgfsetdash{}{0pt}%
\pgfsys@defobject{currentmarker}{\pgfqpoint{0.000000in}{-0.027778in}}{\pgfqpoint{0.000000in}{0.000000in}}{%
\pgfpathmoveto{\pgfqpoint{0.000000in}{0.000000in}}%
\pgfpathlineto{\pgfqpoint{0.000000in}{-0.027778in}}%
\pgfusepath{stroke,fill}%
}%
\begin{pgfscope}%
\pgfsys@transformshift{2.095904in}{0.467838in}%
\pgfsys@useobject{currentmarker}{}%
\end{pgfscope}%
\end{pgfscope}%
\begin{pgfscope}%
\pgfsetbuttcap%
\pgfsetroundjoin%
\definecolor{currentfill}{rgb}{0.000000,0.000000,0.000000}%
\pgfsetfillcolor{currentfill}%
\pgfsetlinewidth{0.602250pt}%
\definecolor{currentstroke}{rgb}{0.000000,0.000000,0.000000}%
\pgfsetstrokecolor{currentstroke}%
\pgfsetdash{}{0pt}%
\pgfsys@defobject{currentmarker}{\pgfqpoint{0.000000in}{-0.027778in}}{\pgfqpoint{0.000000in}{0.000000in}}{%
\pgfpathmoveto{\pgfqpoint{0.000000in}{0.000000in}}%
\pgfpathlineto{\pgfqpoint{0.000000in}{-0.027778in}}%
\pgfusepath{stroke,fill}%
}%
\begin{pgfscope}%
\pgfsys@transformshift{2.137795in}{0.467838in}%
\pgfsys@useobject{currentmarker}{}%
\end{pgfscope}%
\end{pgfscope}%
\begin{pgfscope}%
\pgfsetbuttcap%
\pgfsetroundjoin%
\definecolor{currentfill}{rgb}{0.000000,0.000000,0.000000}%
\pgfsetfillcolor{currentfill}%
\pgfsetlinewidth{0.602250pt}%
\definecolor{currentstroke}{rgb}{0.000000,0.000000,0.000000}%
\pgfsetstrokecolor{currentstroke}%
\pgfsetdash{}{0pt}%
\pgfsys@defobject{currentmarker}{\pgfqpoint{0.000000in}{-0.027778in}}{\pgfqpoint{0.000000in}{0.000000in}}{%
\pgfpathmoveto{\pgfqpoint{0.000000in}{0.000000in}}%
\pgfpathlineto{\pgfqpoint{0.000000in}{-0.027778in}}%
\pgfusepath{stroke,fill}%
}%
\begin{pgfscope}%
\pgfsys@transformshift{2.421792in}{0.467838in}%
\pgfsys@useobject{currentmarker}{}%
\end{pgfscope}%
\end{pgfscope}%
\begin{pgfscope}%
\pgfsetbuttcap%
\pgfsetroundjoin%
\definecolor{currentfill}{rgb}{0.000000,0.000000,0.000000}%
\pgfsetfillcolor{currentfill}%
\pgfsetlinewidth{0.602250pt}%
\definecolor{currentstroke}{rgb}{0.000000,0.000000,0.000000}%
\pgfsetstrokecolor{currentstroke}%
\pgfsetdash{}{0pt}%
\pgfsys@defobject{currentmarker}{\pgfqpoint{0.000000in}{-0.027778in}}{\pgfqpoint{0.000000in}{0.000000in}}{%
\pgfpathmoveto{\pgfqpoint{0.000000in}{0.000000in}}%
\pgfpathlineto{\pgfqpoint{0.000000in}{-0.027778in}}%
\pgfusepath{stroke,fill}%
}%
\begin{pgfscope}%
\pgfsys@transformshift{2.566000in}{0.467838in}%
\pgfsys@useobject{currentmarker}{}%
\end{pgfscope}%
\end{pgfscope}%
\begin{pgfscope}%
\pgfsetbuttcap%
\pgfsetroundjoin%
\definecolor{currentfill}{rgb}{0.000000,0.000000,0.000000}%
\pgfsetfillcolor{currentfill}%
\pgfsetlinewidth{0.602250pt}%
\definecolor{currentstroke}{rgb}{0.000000,0.000000,0.000000}%
\pgfsetstrokecolor{currentstroke}%
\pgfsetdash{}{0pt}%
\pgfsys@defobject{currentmarker}{\pgfqpoint{0.000000in}{-0.027778in}}{\pgfqpoint{0.000000in}{0.000000in}}{%
\pgfpathmoveto{\pgfqpoint{0.000000in}{0.000000in}}%
\pgfpathlineto{\pgfqpoint{0.000000in}{-0.027778in}}%
\pgfusepath{stroke,fill}%
}%
\begin{pgfscope}%
\pgfsys@transformshift{2.668317in}{0.467838in}%
\pgfsys@useobject{currentmarker}{}%
\end{pgfscope}%
\end{pgfscope}%
\begin{pgfscope}%
\pgfsetbuttcap%
\pgfsetroundjoin%
\definecolor{currentfill}{rgb}{0.000000,0.000000,0.000000}%
\pgfsetfillcolor{currentfill}%
\pgfsetlinewidth{0.602250pt}%
\definecolor{currentstroke}{rgb}{0.000000,0.000000,0.000000}%
\pgfsetstrokecolor{currentstroke}%
\pgfsetdash{}{0pt}%
\pgfsys@defobject{currentmarker}{\pgfqpoint{0.000000in}{-0.027778in}}{\pgfqpoint{0.000000in}{0.000000in}}{%
\pgfpathmoveto{\pgfqpoint{0.000000in}{0.000000in}}%
\pgfpathlineto{\pgfqpoint{0.000000in}{-0.027778in}}%
\pgfusepath{stroke,fill}%
}%
\begin{pgfscope}%
\pgfsys@transformshift{2.747680in}{0.467838in}%
\pgfsys@useobject{currentmarker}{}%
\end{pgfscope}%
\end{pgfscope}%
\begin{pgfscope}%
\pgfsetbuttcap%
\pgfsetroundjoin%
\definecolor{currentfill}{rgb}{0.000000,0.000000,0.000000}%
\pgfsetfillcolor{currentfill}%
\pgfsetlinewidth{0.602250pt}%
\definecolor{currentstroke}{rgb}{0.000000,0.000000,0.000000}%
\pgfsetstrokecolor{currentstroke}%
\pgfsetdash{}{0pt}%
\pgfsys@defobject{currentmarker}{\pgfqpoint{0.000000in}{-0.027778in}}{\pgfqpoint{0.000000in}{0.000000in}}{%
\pgfpathmoveto{\pgfqpoint{0.000000in}{0.000000in}}%
\pgfpathlineto{\pgfqpoint{0.000000in}{-0.027778in}}%
\pgfusepath{stroke,fill}%
}%
\begin{pgfscope}%
\pgfsys@transformshift{2.812524in}{0.467838in}%
\pgfsys@useobject{currentmarker}{}%
\end{pgfscope}%
\end{pgfscope}%
\begin{pgfscope}%
\pgfsetbuttcap%
\pgfsetroundjoin%
\definecolor{currentfill}{rgb}{0.000000,0.000000,0.000000}%
\pgfsetfillcolor{currentfill}%
\pgfsetlinewidth{0.602250pt}%
\definecolor{currentstroke}{rgb}{0.000000,0.000000,0.000000}%
\pgfsetstrokecolor{currentstroke}%
\pgfsetdash{}{0pt}%
\pgfsys@defobject{currentmarker}{\pgfqpoint{0.000000in}{-0.027778in}}{\pgfqpoint{0.000000in}{0.000000in}}{%
\pgfpathmoveto{\pgfqpoint{0.000000in}{0.000000in}}%
\pgfpathlineto{\pgfqpoint{0.000000in}{-0.027778in}}%
\pgfusepath{stroke,fill}%
}%
\begin{pgfscope}%
\pgfsys@transformshift{2.867349in}{0.467838in}%
\pgfsys@useobject{currentmarker}{}%
\end{pgfscope}%
\end{pgfscope}%
\begin{pgfscope}%
\pgfsetbuttcap%
\pgfsetroundjoin%
\definecolor{currentfill}{rgb}{0.000000,0.000000,0.000000}%
\pgfsetfillcolor{currentfill}%
\pgfsetlinewidth{0.602250pt}%
\definecolor{currentstroke}{rgb}{0.000000,0.000000,0.000000}%
\pgfsetstrokecolor{currentstroke}%
\pgfsetdash{}{0pt}%
\pgfsys@defobject{currentmarker}{\pgfqpoint{0.000000in}{-0.027778in}}{\pgfqpoint{0.000000in}{0.000000in}}{%
\pgfpathmoveto{\pgfqpoint{0.000000in}{0.000000in}}%
\pgfpathlineto{\pgfqpoint{0.000000in}{-0.027778in}}%
\pgfusepath{stroke,fill}%
}%
\begin{pgfscope}%
\pgfsys@transformshift{2.914841in}{0.467838in}%
\pgfsys@useobject{currentmarker}{}%
\end{pgfscope}%
\end{pgfscope}%
\begin{pgfscope}%
\pgfsetbuttcap%
\pgfsetroundjoin%
\definecolor{currentfill}{rgb}{0.000000,0.000000,0.000000}%
\pgfsetfillcolor{currentfill}%
\pgfsetlinewidth{0.602250pt}%
\definecolor{currentstroke}{rgb}{0.000000,0.000000,0.000000}%
\pgfsetstrokecolor{currentstroke}%
\pgfsetdash{}{0pt}%
\pgfsys@defobject{currentmarker}{\pgfqpoint{0.000000in}{-0.027778in}}{\pgfqpoint{0.000000in}{0.000000in}}{%
\pgfpathmoveto{\pgfqpoint{0.000000in}{0.000000in}}%
\pgfpathlineto{\pgfqpoint{0.000000in}{-0.027778in}}%
\pgfusepath{stroke,fill}%
}%
\begin{pgfscope}%
\pgfsys@transformshift{2.956732in}{0.467838in}%
\pgfsys@useobject{currentmarker}{}%
\end{pgfscope}%
\end{pgfscope}%
\begin{pgfscope}%
\pgfsetbuttcap%
\pgfsetroundjoin%
\definecolor{currentfill}{rgb}{0.000000,0.000000,0.000000}%
\pgfsetfillcolor{currentfill}%
\pgfsetlinewidth{0.602250pt}%
\definecolor{currentstroke}{rgb}{0.000000,0.000000,0.000000}%
\pgfsetstrokecolor{currentstroke}%
\pgfsetdash{}{0pt}%
\pgfsys@defobject{currentmarker}{\pgfqpoint{0.000000in}{-0.027778in}}{\pgfqpoint{0.000000in}{0.000000in}}{%
\pgfpathmoveto{\pgfqpoint{0.000000in}{0.000000in}}%
\pgfpathlineto{\pgfqpoint{0.000000in}{-0.027778in}}%
\pgfusepath{stroke,fill}%
}%
\begin{pgfscope}%
\pgfsys@transformshift{3.240729in}{0.467838in}%
\pgfsys@useobject{currentmarker}{}%
\end{pgfscope}%
\end{pgfscope}%
\begin{pgfscope}%
\pgfsetbuttcap%
\pgfsetroundjoin%
\definecolor{currentfill}{rgb}{0.000000,0.000000,0.000000}%
\pgfsetfillcolor{currentfill}%
\pgfsetlinewidth{0.602250pt}%
\definecolor{currentstroke}{rgb}{0.000000,0.000000,0.000000}%
\pgfsetstrokecolor{currentstroke}%
\pgfsetdash{}{0pt}%
\pgfsys@defobject{currentmarker}{\pgfqpoint{0.000000in}{-0.027778in}}{\pgfqpoint{0.000000in}{0.000000in}}{%
\pgfpathmoveto{\pgfqpoint{0.000000in}{0.000000in}}%
\pgfpathlineto{\pgfqpoint{0.000000in}{-0.027778in}}%
\pgfusepath{stroke,fill}%
}%
\begin{pgfscope}%
\pgfsys@transformshift{3.384936in}{0.467838in}%
\pgfsys@useobject{currentmarker}{}%
\end{pgfscope}%
\end{pgfscope}%
\begin{pgfscope}%
\pgfsetbuttcap%
\pgfsetroundjoin%
\definecolor{currentfill}{rgb}{0.000000,0.000000,0.000000}%
\pgfsetfillcolor{currentfill}%
\pgfsetlinewidth{0.602250pt}%
\definecolor{currentstroke}{rgb}{0.000000,0.000000,0.000000}%
\pgfsetstrokecolor{currentstroke}%
\pgfsetdash{}{0pt}%
\pgfsys@defobject{currentmarker}{\pgfqpoint{0.000000in}{-0.027778in}}{\pgfqpoint{0.000000in}{0.000000in}}{%
\pgfpathmoveto{\pgfqpoint{0.000000in}{0.000000in}}%
\pgfpathlineto{\pgfqpoint{0.000000in}{-0.027778in}}%
\pgfusepath{stroke,fill}%
}%
\begin{pgfscope}%
\pgfsys@transformshift{3.487253in}{0.467838in}%
\pgfsys@useobject{currentmarker}{}%
\end{pgfscope}%
\end{pgfscope}%
\begin{pgfscope}%
\pgfsetbuttcap%
\pgfsetroundjoin%
\definecolor{currentfill}{rgb}{0.000000,0.000000,0.000000}%
\pgfsetfillcolor{currentfill}%
\pgfsetlinewidth{0.602250pt}%
\definecolor{currentstroke}{rgb}{0.000000,0.000000,0.000000}%
\pgfsetstrokecolor{currentstroke}%
\pgfsetdash{}{0pt}%
\pgfsys@defobject{currentmarker}{\pgfqpoint{0.000000in}{-0.027778in}}{\pgfqpoint{0.000000in}{0.000000in}}{%
\pgfpathmoveto{\pgfqpoint{0.000000in}{0.000000in}}%
\pgfpathlineto{\pgfqpoint{0.000000in}{-0.027778in}}%
\pgfusepath{stroke,fill}%
}%
\begin{pgfscope}%
\pgfsys@transformshift{3.566617in}{0.467838in}%
\pgfsys@useobject{currentmarker}{}%
\end{pgfscope}%
\end{pgfscope}%
\begin{pgfscope}%
\pgfsetbuttcap%
\pgfsetroundjoin%
\definecolor{currentfill}{rgb}{0.000000,0.000000,0.000000}%
\pgfsetfillcolor{currentfill}%
\pgfsetlinewidth{0.602250pt}%
\definecolor{currentstroke}{rgb}{0.000000,0.000000,0.000000}%
\pgfsetstrokecolor{currentstroke}%
\pgfsetdash{}{0pt}%
\pgfsys@defobject{currentmarker}{\pgfqpoint{0.000000in}{-0.027778in}}{\pgfqpoint{0.000000in}{0.000000in}}{%
\pgfpathmoveto{\pgfqpoint{0.000000in}{0.000000in}}%
\pgfpathlineto{\pgfqpoint{0.000000in}{-0.027778in}}%
\pgfusepath{stroke,fill}%
}%
\begin{pgfscope}%
\pgfsys@transformshift{3.631461in}{0.467838in}%
\pgfsys@useobject{currentmarker}{}%
\end{pgfscope}%
\end{pgfscope}%
\begin{pgfscope}%
\pgfsetbuttcap%
\pgfsetroundjoin%
\definecolor{currentfill}{rgb}{0.000000,0.000000,0.000000}%
\pgfsetfillcolor{currentfill}%
\pgfsetlinewidth{0.602250pt}%
\definecolor{currentstroke}{rgb}{0.000000,0.000000,0.000000}%
\pgfsetstrokecolor{currentstroke}%
\pgfsetdash{}{0pt}%
\pgfsys@defobject{currentmarker}{\pgfqpoint{0.000000in}{-0.027778in}}{\pgfqpoint{0.000000in}{0.000000in}}{%
\pgfpathmoveto{\pgfqpoint{0.000000in}{0.000000in}}%
\pgfpathlineto{\pgfqpoint{0.000000in}{-0.027778in}}%
\pgfusepath{stroke,fill}%
}%
\begin{pgfscope}%
\pgfsys@transformshift{3.686286in}{0.467838in}%
\pgfsys@useobject{currentmarker}{}%
\end{pgfscope}%
\end{pgfscope}%
\begin{pgfscope}%
\pgfsetbuttcap%
\pgfsetroundjoin%
\definecolor{currentfill}{rgb}{0.000000,0.000000,0.000000}%
\pgfsetfillcolor{currentfill}%
\pgfsetlinewidth{0.602250pt}%
\definecolor{currentstroke}{rgb}{0.000000,0.000000,0.000000}%
\pgfsetstrokecolor{currentstroke}%
\pgfsetdash{}{0pt}%
\pgfsys@defobject{currentmarker}{\pgfqpoint{0.000000in}{-0.027778in}}{\pgfqpoint{0.000000in}{0.000000in}}{%
\pgfpathmoveto{\pgfqpoint{0.000000in}{0.000000in}}%
\pgfpathlineto{\pgfqpoint{0.000000in}{-0.027778in}}%
\pgfusepath{stroke,fill}%
}%
\begin{pgfscope}%
\pgfsys@transformshift{3.733778in}{0.467838in}%
\pgfsys@useobject{currentmarker}{}%
\end{pgfscope}%
\end{pgfscope}%
\begin{pgfscope}%
\pgfsetbuttcap%
\pgfsetroundjoin%
\definecolor{currentfill}{rgb}{0.000000,0.000000,0.000000}%
\pgfsetfillcolor{currentfill}%
\pgfsetlinewidth{0.602250pt}%
\definecolor{currentstroke}{rgb}{0.000000,0.000000,0.000000}%
\pgfsetstrokecolor{currentstroke}%
\pgfsetdash{}{0pt}%
\pgfsys@defobject{currentmarker}{\pgfqpoint{0.000000in}{-0.027778in}}{\pgfqpoint{0.000000in}{0.000000in}}{%
\pgfpathmoveto{\pgfqpoint{0.000000in}{0.000000in}}%
\pgfpathlineto{\pgfqpoint{0.000000in}{-0.027778in}}%
\pgfusepath{stroke,fill}%
}%
\begin{pgfscope}%
\pgfsys@transformshift{3.775669in}{0.467838in}%
\pgfsys@useobject{currentmarker}{}%
\end{pgfscope}%
\end{pgfscope}%
\begin{pgfscope}%
\pgfsetbuttcap%
\pgfsetroundjoin%
\definecolor{currentfill}{rgb}{0.000000,0.000000,0.000000}%
\pgfsetfillcolor{currentfill}%
\pgfsetlinewidth{0.602250pt}%
\definecolor{currentstroke}{rgb}{0.000000,0.000000,0.000000}%
\pgfsetstrokecolor{currentstroke}%
\pgfsetdash{}{0pt}%
\pgfsys@defobject{currentmarker}{\pgfqpoint{0.000000in}{-0.027778in}}{\pgfqpoint{0.000000in}{0.000000in}}{%
\pgfpathmoveto{\pgfqpoint{0.000000in}{0.000000in}}%
\pgfpathlineto{\pgfqpoint{0.000000in}{-0.027778in}}%
\pgfusepath{stroke,fill}%
}%
\begin{pgfscope}%
\pgfsys@transformshift{4.059666in}{0.467838in}%
\pgfsys@useobject{currentmarker}{}%
\end{pgfscope}%
\end{pgfscope}%
\begin{pgfscope}%
\pgfsetbuttcap%
\pgfsetroundjoin%
\definecolor{currentfill}{rgb}{0.000000,0.000000,0.000000}%
\pgfsetfillcolor{currentfill}%
\pgfsetlinewidth{0.602250pt}%
\definecolor{currentstroke}{rgb}{0.000000,0.000000,0.000000}%
\pgfsetstrokecolor{currentstroke}%
\pgfsetdash{}{0pt}%
\pgfsys@defobject{currentmarker}{\pgfqpoint{0.000000in}{-0.027778in}}{\pgfqpoint{0.000000in}{0.000000in}}{%
\pgfpathmoveto{\pgfqpoint{0.000000in}{0.000000in}}%
\pgfpathlineto{\pgfqpoint{0.000000in}{-0.027778in}}%
\pgfusepath{stroke,fill}%
}%
\begin{pgfscope}%
\pgfsys@transformshift{4.203873in}{0.467838in}%
\pgfsys@useobject{currentmarker}{}%
\end{pgfscope}%
\end{pgfscope}%
\begin{pgfscope}%
\pgfsetbuttcap%
\pgfsetroundjoin%
\definecolor{currentfill}{rgb}{0.000000,0.000000,0.000000}%
\pgfsetfillcolor{currentfill}%
\pgfsetlinewidth{0.602250pt}%
\definecolor{currentstroke}{rgb}{0.000000,0.000000,0.000000}%
\pgfsetstrokecolor{currentstroke}%
\pgfsetdash{}{0pt}%
\pgfsys@defobject{currentmarker}{\pgfqpoint{0.000000in}{-0.027778in}}{\pgfqpoint{0.000000in}{0.000000in}}{%
\pgfpathmoveto{\pgfqpoint{0.000000in}{0.000000in}}%
\pgfpathlineto{\pgfqpoint{0.000000in}{-0.027778in}}%
\pgfusepath{stroke,fill}%
}%
\begin{pgfscope}%
\pgfsys@transformshift{4.306190in}{0.467838in}%
\pgfsys@useobject{currentmarker}{}%
\end{pgfscope}%
\end{pgfscope}%
\begin{pgfscope}%
\pgfsetbuttcap%
\pgfsetroundjoin%
\definecolor{currentfill}{rgb}{0.000000,0.000000,0.000000}%
\pgfsetfillcolor{currentfill}%
\pgfsetlinewidth{0.602250pt}%
\definecolor{currentstroke}{rgb}{0.000000,0.000000,0.000000}%
\pgfsetstrokecolor{currentstroke}%
\pgfsetdash{}{0pt}%
\pgfsys@defobject{currentmarker}{\pgfqpoint{0.000000in}{-0.027778in}}{\pgfqpoint{0.000000in}{0.000000in}}{%
\pgfpathmoveto{\pgfqpoint{0.000000in}{0.000000in}}%
\pgfpathlineto{\pgfqpoint{0.000000in}{-0.027778in}}%
\pgfusepath{stroke,fill}%
}%
\begin{pgfscope}%
\pgfsys@transformshift{4.385553in}{0.467838in}%
\pgfsys@useobject{currentmarker}{}%
\end{pgfscope}%
\end{pgfscope}%
\begin{pgfscope}%
\pgfsetbuttcap%
\pgfsetroundjoin%
\definecolor{currentfill}{rgb}{0.000000,0.000000,0.000000}%
\pgfsetfillcolor{currentfill}%
\pgfsetlinewidth{0.602250pt}%
\definecolor{currentstroke}{rgb}{0.000000,0.000000,0.000000}%
\pgfsetstrokecolor{currentstroke}%
\pgfsetdash{}{0pt}%
\pgfsys@defobject{currentmarker}{\pgfqpoint{0.000000in}{-0.027778in}}{\pgfqpoint{0.000000in}{0.000000in}}{%
\pgfpathmoveto{\pgfqpoint{0.000000in}{0.000000in}}%
\pgfpathlineto{\pgfqpoint{0.000000in}{-0.027778in}}%
\pgfusepath{stroke,fill}%
}%
\begin{pgfscope}%
\pgfsys@transformshift{4.450398in}{0.467838in}%
\pgfsys@useobject{currentmarker}{}%
\end{pgfscope}%
\end{pgfscope}%
\begin{pgfscope}%
\pgfsetbuttcap%
\pgfsetroundjoin%
\definecolor{currentfill}{rgb}{0.000000,0.000000,0.000000}%
\pgfsetfillcolor{currentfill}%
\pgfsetlinewidth{0.602250pt}%
\definecolor{currentstroke}{rgb}{0.000000,0.000000,0.000000}%
\pgfsetstrokecolor{currentstroke}%
\pgfsetdash{}{0pt}%
\pgfsys@defobject{currentmarker}{\pgfqpoint{0.000000in}{-0.027778in}}{\pgfqpoint{0.000000in}{0.000000in}}{%
\pgfpathmoveto{\pgfqpoint{0.000000in}{0.000000in}}%
\pgfpathlineto{\pgfqpoint{0.000000in}{-0.027778in}}%
\pgfusepath{stroke,fill}%
}%
\begin{pgfscope}%
\pgfsys@transformshift{4.505223in}{0.467838in}%
\pgfsys@useobject{currentmarker}{}%
\end{pgfscope}%
\end{pgfscope}%
\begin{pgfscope}%
\pgfsetbuttcap%
\pgfsetroundjoin%
\definecolor{currentfill}{rgb}{0.000000,0.000000,0.000000}%
\pgfsetfillcolor{currentfill}%
\pgfsetlinewidth{0.602250pt}%
\definecolor{currentstroke}{rgb}{0.000000,0.000000,0.000000}%
\pgfsetstrokecolor{currentstroke}%
\pgfsetdash{}{0pt}%
\pgfsys@defobject{currentmarker}{\pgfqpoint{0.000000in}{-0.027778in}}{\pgfqpoint{0.000000in}{0.000000in}}{%
\pgfpathmoveto{\pgfqpoint{0.000000in}{0.000000in}}%
\pgfpathlineto{\pgfqpoint{0.000000in}{-0.027778in}}%
\pgfusepath{stroke,fill}%
}%
\begin{pgfscope}%
\pgfsys@transformshift{4.552715in}{0.467838in}%
\pgfsys@useobject{currentmarker}{}%
\end{pgfscope}%
\end{pgfscope}%
\begin{pgfscope}%
\pgfsetbuttcap%
\pgfsetroundjoin%
\definecolor{currentfill}{rgb}{0.000000,0.000000,0.000000}%
\pgfsetfillcolor{currentfill}%
\pgfsetlinewidth{0.602250pt}%
\definecolor{currentstroke}{rgb}{0.000000,0.000000,0.000000}%
\pgfsetstrokecolor{currentstroke}%
\pgfsetdash{}{0pt}%
\pgfsys@defobject{currentmarker}{\pgfqpoint{0.000000in}{-0.027778in}}{\pgfqpoint{0.000000in}{0.000000in}}{%
\pgfpathmoveto{\pgfqpoint{0.000000in}{0.000000in}}%
\pgfpathlineto{\pgfqpoint{0.000000in}{-0.027778in}}%
\pgfusepath{stroke,fill}%
}%
\begin{pgfscope}%
\pgfsys@transformshift{4.594605in}{0.467838in}%
\pgfsys@useobject{currentmarker}{}%
\end{pgfscope}%
\end{pgfscope}%
\begin{pgfscope}%
\definecolor{textcolor}{rgb}{0.000000,0.000000,0.000000}%
\pgfsetstrokecolor{textcolor}%
\pgfsetfillcolor{textcolor}%
\pgftext[x=2.584736in,y=0.207530in,,top]{\color{textcolor}\sffamily\fontsize{8.000000}{9.600000}\selectfont Longest solving time (seconds)}%
\end{pgfscope}%
\begin{pgfscope}%
\pgfsetbuttcap%
\pgfsetroundjoin%
\definecolor{currentfill}{rgb}{0.000000,0.000000,0.000000}%
\pgfsetfillcolor{currentfill}%
\pgfsetlinewidth{0.803000pt}%
\definecolor{currentstroke}{rgb}{0.000000,0.000000,0.000000}%
\pgfsetstrokecolor{currentstroke}%
\pgfsetdash{}{0pt}%
\pgfsys@defobject{currentmarker}{\pgfqpoint{-0.048611in}{0.000000in}}{\pgfqpoint{-0.000000in}{0.000000in}}{%
\pgfpathmoveto{\pgfqpoint{-0.000000in}{0.000000in}}%
\pgfpathlineto{\pgfqpoint{-0.048611in}{0.000000in}}%
\pgfusepath{stroke,fill}%
}%
\begin{pgfscope}%
\pgfsys@transformshift{0.537394in}{0.467838in}%
\pgfsys@useobject{currentmarker}{}%
\end{pgfscope}%
\end{pgfscope}%
\begin{pgfscope}%
\definecolor{textcolor}{rgb}{0.000000,0.000000,0.000000}%
\pgfsetstrokecolor{textcolor}%
\pgfsetfillcolor{textcolor}%
\pgftext[x=0.381143in, y=0.425629in, left, base]{\color{textcolor}\sffamily\fontsize{8.000000}{9.600000}\selectfont \(\displaystyle {0}\)}%
\end{pgfscope}%
\begin{pgfscope}%
\pgfsetbuttcap%
\pgfsetroundjoin%
\definecolor{currentfill}{rgb}{0.000000,0.000000,0.000000}%
\pgfsetfillcolor{currentfill}%
\pgfsetlinewidth{0.803000pt}%
\definecolor{currentstroke}{rgb}{0.000000,0.000000,0.000000}%
\pgfsetstrokecolor{currentstroke}%
\pgfsetdash{}{0pt}%
\pgfsys@defobject{currentmarker}{\pgfqpoint{-0.048611in}{0.000000in}}{\pgfqpoint{-0.000000in}{0.000000in}}{%
\pgfpathmoveto{\pgfqpoint{-0.000000in}{0.000000in}}%
\pgfpathlineto{\pgfqpoint{-0.048611in}{0.000000in}}%
\pgfusepath{stroke,fill}%
}%
\begin{pgfscope}%
\pgfsys@transformshift{0.537394in}{0.717152in}%
\pgfsys@useobject{currentmarker}{}%
\end{pgfscope}%
\end{pgfscope}%
\begin{pgfscope}%
\definecolor{textcolor}{rgb}{0.000000,0.000000,0.000000}%
\pgfsetstrokecolor{textcolor}%
\pgfsetfillcolor{textcolor}%
\pgftext[x=0.263086in, y=0.674943in, left, base]{\color{textcolor}\sffamily\fontsize{8.000000}{9.600000}\selectfont \(\displaystyle {100}\)}%
\end{pgfscope}%
\begin{pgfscope}%
\pgfsetbuttcap%
\pgfsetroundjoin%
\definecolor{currentfill}{rgb}{0.000000,0.000000,0.000000}%
\pgfsetfillcolor{currentfill}%
\pgfsetlinewidth{0.803000pt}%
\definecolor{currentstroke}{rgb}{0.000000,0.000000,0.000000}%
\pgfsetstrokecolor{currentstroke}%
\pgfsetdash{}{0pt}%
\pgfsys@defobject{currentmarker}{\pgfqpoint{-0.048611in}{0.000000in}}{\pgfqpoint{-0.000000in}{0.000000in}}{%
\pgfpathmoveto{\pgfqpoint{-0.000000in}{0.000000in}}%
\pgfpathlineto{\pgfqpoint{-0.048611in}{0.000000in}}%
\pgfusepath{stroke,fill}%
}%
\begin{pgfscope}%
\pgfsys@transformshift{0.537394in}{0.966465in}%
\pgfsys@useobject{currentmarker}{}%
\end{pgfscope}%
\end{pgfscope}%
\begin{pgfscope}%
\definecolor{textcolor}{rgb}{0.000000,0.000000,0.000000}%
\pgfsetstrokecolor{textcolor}%
\pgfsetfillcolor{textcolor}%
\pgftext[x=0.263086in, y=0.924256in, left, base]{\color{textcolor}\sffamily\fontsize{8.000000}{9.600000}\selectfont \(\displaystyle {200}\)}%
\end{pgfscope}%
\begin{pgfscope}%
\pgfsetbuttcap%
\pgfsetroundjoin%
\definecolor{currentfill}{rgb}{0.000000,0.000000,0.000000}%
\pgfsetfillcolor{currentfill}%
\pgfsetlinewidth{0.803000pt}%
\definecolor{currentstroke}{rgb}{0.000000,0.000000,0.000000}%
\pgfsetstrokecolor{currentstroke}%
\pgfsetdash{}{0pt}%
\pgfsys@defobject{currentmarker}{\pgfqpoint{-0.048611in}{0.000000in}}{\pgfqpoint{-0.000000in}{0.000000in}}{%
\pgfpathmoveto{\pgfqpoint{-0.000000in}{0.000000in}}%
\pgfpathlineto{\pgfqpoint{-0.048611in}{0.000000in}}%
\pgfusepath{stroke,fill}%
}%
\begin{pgfscope}%
\pgfsys@transformshift{0.537394in}{1.215779in}%
\pgfsys@useobject{currentmarker}{}%
\end{pgfscope}%
\end{pgfscope}%
\begin{pgfscope}%
\definecolor{textcolor}{rgb}{0.000000,0.000000,0.000000}%
\pgfsetstrokecolor{textcolor}%
\pgfsetfillcolor{textcolor}%
\pgftext[x=0.263086in, y=1.173569in, left, base]{\color{textcolor}\sffamily\fontsize{8.000000}{9.600000}\selectfont \(\displaystyle {300}\)}%
\end{pgfscope}%
\begin{pgfscope}%
\pgfsetbuttcap%
\pgfsetroundjoin%
\definecolor{currentfill}{rgb}{0.000000,0.000000,0.000000}%
\pgfsetfillcolor{currentfill}%
\pgfsetlinewidth{0.803000pt}%
\definecolor{currentstroke}{rgb}{0.000000,0.000000,0.000000}%
\pgfsetstrokecolor{currentstroke}%
\pgfsetdash{}{0pt}%
\pgfsys@defobject{currentmarker}{\pgfqpoint{-0.048611in}{0.000000in}}{\pgfqpoint{-0.000000in}{0.000000in}}{%
\pgfpathmoveto{\pgfqpoint{-0.000000in}{0.000000in}}%
\pgfpathlineto{\pgfqpoint{-0.048611in}{0.000000in}}%
\pgfusepath{stroke,fill}%
}%
\begin{pgfscope}%
\pgfsys@transformshift{0.537394in}{1.465092in}%
\pgfsys@useobject{currentmarker}{}%
\end{pgfscope}%
\end{pgfscope}%
\begin{pgfscope}%
\definecolor{textcolor}{rgb}{0.000000,0.000000,0.000000}%
\pgfsetstrokecolor{textcolor}%
\pgfsetfillcolor{textcolor}%
\pgftext[x=0.263086in, y=1.422883in, left, base]{\color{textcolor}\sffamily\fontsize{8.000000}{9.600000}\selectfont \(\displaystyle {400}\)}%
\end{pgfscope}%
\begin{pgfscope}%
\definecolor{textcolor}{rgb}{0.000000,0.000000,0.000000}%
\pgfsetstrokecolor{textcolor}%
\pgfsetfillcolor{textcolor}%
\pgftext[x=0.207530in,y=0.966465in,,bottom,rotate=90.000000]{\color{textcolor}\sffamily\fontsize{8.000000}{9.600000}\selectfont Benchmarks solved}%
\end{pgfscope}%
\begin{pgfscope}%
\pgfpathrectangle{\pgfqpoint{0.537394in}{0.467838in}}{\pgfqpoint{4.094684in}{0.997254in}}%
\pgfusepath{clip}%
\pgfsetrectcap%
\pgfsetroundjoin%
\pgfsetlinewidth{1.003750pt}%
\definecolor{currentstroke}{rgb}{0.121569,0.466667,0.705882}%
\pgfsetstrokecolor{currentstroke}%
\pgfsetdash{}{0pt}%
\pgfpathmoveto{\pgfqpoint{1.229476in}{0.475318in}}%
\pgfpathlineto{\pgfqpoint{1.318858in}{0.477811in}}%
\pgfpathlineto{\pgfqpoint{1.356331in}{0.485290in}}%
\pgfpathlineto{\pgfqpoint{1.390229in}{0.497756in}}%
\pgfpathlineto{\pgfqpoint{1.421175in}{0.505235in}}%
\pgfpathlineto{\pgfqpoint{1.449643in}{0.510222in}}%
\pgfpathlineto{\pgfqpoint{1.476000in}{0.525180in}}%
\pgfpathlineto{\pgfqpoint{1.500538in}{0.530167in}}%
\pgfpathlineto{\pgfqpoint{1.523492in}{0.540139in}}%
\pgfpathlineto{\pgfqpoint{1.565383in}{0.550112in}}%
\pgfpathlineto{\pgfqpoint{1.584612in}{0.555098in}}%
\pgfpathlineto{\pgfqpoint{1.602855in}{0.567564in}}%
\pgfpathlineto{\pgfqpoint{1.620208in}{0.572550in}}%
\pgfpathlineto{\pgfqpoint{1.636753in}{0.587509in}}%
\pgfpathlineto{\pgfqpoint{1.652563in}{0.594988in}}%
\pgfpathlineto{\pgfqpoint{1.667700in}{0.602468in}}%
\pgfpathlineto{\pgfqpoint{1.682218in}{0.614933in}}%
\pgfpathlineto{\pgfqpoint{1.696168in}{0.624906in}}%
\pgfpathlineto{\pgfqpoint{1.709590in}{0.639865in}}%
\pgfpathlineto{\pgfqpoint{1.722525in}{0.652330in}}%
\pgfpathlineto{\pgfqpoint{1.735005in}{0.664796in}}%
\pgfpathlineto{\pgfqpoint{1.758725in}{0.679755in}}%
\pgfpathlineto{\pgfqpoint{1.770017in}{0.692220in}}%
\pgfpathlineto{\pgfqpoint{1.780961in}{0.697207in}}%
\pgfpathlineto{\pgfqpoint{1.791578in}{0.704686in}}%
\pgfpathlineto{\pgfqpoint{1.801888in}{0.722138in}}%
\pgfpathlineto{\pgfqpoint{1.811907in}{0.734604in}}%
\pgfpathlineto{\pgfqpoint{1.821652in}{0.749563in}}%
\pgfpathlineto{\pgfqpoint{1.840375in}{0.754549in}}%
\pgfpathlineto{\pgfqpoint{1.849380in}{0.759535in}}%
\pgfpathlineto{\pgfqpoint{1.866732in}{0.774494in}}%
\pgfpathlineto{\pgfqpoint{1.875101in}{0.784466in}}%
\pgfpathlineto{\pgfqpoint{1.883278in}{0.791946in}}%
\pgfpathlineto{\pgfqpoint{1.899087in}{0.796932in}}%
\pgfpathlineto{\pgfqpoint{1.906736in}{0.804411in}}%
\pgfpathlineto{\pgfqpoint{1.914224in}{0.806905in}}%
\pgfpathlineto{\pgfqpoint{1.921558in}{0.814384in}}%
\pgfpathlineto{\pgfqpoint{1.935786in}{0.819370in}}%
\pgfpathlineto{\pgfqpoint{1.962641in}{0.829343in}}%
\pgfpathlineto{\pgfqpoint{1.969049in}{0.834329in}}%
\pgfpathlineto{\pgfqpoint{1.981530in}{0.836822in}}%
\pgfpathlineto{\pgfqpoint{1.987610in}{0.841808in}}%
\pgfpathlineto{\pgfqpoint{1.993587in}{0.844302in}}%
\pgfpathlineto{\pgfqpoint{2.005249in}{0.846795in}}%
\pgfpathlineto{\pgfqpoint{2.016541in}{0.856767in}}%
\pgfpathlineto{\pgfqpoint{2.022055in}{0.859260in}}%
\pgfpathlineto{\pgfqpoint{2.027485in}{0.864247in}}%
\pgfpathlineto{\pgfqpoint{2.032834in}{0.866740in}}%
\pgfpathlineto{\pgfqpoint{2.043295in}{0.876712in}}%
\pgfpathlineto{\pgfqpoint{2.058432in}{0.879205in}}%
\pgfpathlineto{\pgfqpoint{2.068177in}{0.886685in}}%
\pgfpathlineto{\pgfqpoint{2.072951in}{0.889178in}}%
\pgfpathlineto{\pgfqpoint{2.100323in}{0.896657in}}%
\pgfpathlineto{\pgfqpoint{2.113257in}{0.904137in}}%
\pgfpathlineto{\pgfqpoint{2.117466in}{0.909123in}}%
\pgfpathlineto{\pgfqpoint{2.121626in}{0.911616in}}%
\pgfpathlineto{\pgfqpoint{2.129802in}{0.914109in}}%
\pgfpathlineto{\pgfqpoint{2.133821in}{0.919096in}}%
\pgfpathlineto{\pgfqpoint{2.145612in}{0.926575in}}%
\pgfpathlineto{\pgfqpoint{2.175268in}{0.929068in}}%
\pgfpathlineto{\pgfqpoint{2.182311in}{0.931561in}}%
\pgfpathlineto{\pgfqpoint{2.189217in}{0.936548in}}%
\pgfpathlineto{\pgfqpoint{2.195991in}{0.939041in}}%
\pgfpathlineto{\pgfqpoint{2.209166in}{0.941534in}}%
\pgfpathlineto{\pgfqpoint{2.212384in}{0.944027in}}%
\pgfpathlineto{\pgfqpoint{2.221869in}{0.946520in}}%
\pgfpathlineto{\pgfqpoint{2.234134in}{0.956493in}}%
\pgfpathlineto{\pgfqpoint{2.240112in}{0.958986in}}%
\pgfpathlineto{\pgfqpoint{2.254631in}{0.963972in}}%
\pgfpathlineto{\pgfqpoint{2.263066in}{0.966465in}}%
\pgfpathlineto{\pgfqpoint{2.274010in}{0.968958in}}%
\pgfpathlineto{\pgfqpoint{2.289820in}{0.971451in}}%
\pgfpathlineto{\pgfqpoint{2.292388in}{0.973945in}}%
\pgfpathlineto{\pgfqpoint{2.333424in}{0.976438in}}%
\pgfpathlineto{\pgfqpoint{2.342429in}{0.981424in}}%
\pgfpathlineto{\pgfqpoint{2.349036in}{0.988903in}}%
\pgfpathlineto{\pgfqpoint{2.355522in}{0.991397in}}%
\pgfpathlineto{\pgfqpoint{2.359782in}{0.993890in}}%
\pgfpathlineto{\pgfqpoint{2.361892in}{0.996383in}}%
\pgfpathlineto{\pgfqpoint{2.372262in}{0.998876in}}%
\pgfpathlineto{\pgfqpoint{2.397889in}{1.001369in}}%
\pgfpathlineto{\pgfqpoint{2.414607in}{1.003862in}}%
\pgfpathlineto{\pgfqpoint{2.421792in}{1.006355in}}%
\pgfpathlineto{\pgfqpoint{2.440834in}{1.008848in}}%
\pgfpathlineto{\pgfqpoint{2.452442in}{1.011342in}}%
\pgfpathlineto{\pgfqpoint{2.469950in}{1.013835in}}%
\pgfpathlineto{\pgfqpoint{2.498299in}{1.016328in}}%
\pgfpathlineto{\pgfqpoint{2.523219in}{1.018821in}}%
\pgfpathlineto{\pgfqpoint{2.541462in}{1.021314in}}%
\pgfpathlineto{\pgfqpoint{2.552714in}{1.023807in}}%
\pgfpathlineto{\pgfqpoint{2.562425in}{1.026300in}}%
\pgfpathlineto{\pgfqpoint{2.573043in}{1.028794in}}%
\pgfpathlineto{\pgfqpoint{2.611560in}{1.031287in}}%
\pgfpathlineto{\pgfqpoint{2.615707in}{1.033780in}}%
\pgfpathlineto{\pgfqpoint{2.657483in}{1.036273in}}%
\pgfpathlineto{\pgfqpoint{2.675360in}{1.038766in}}%
\pgfpathlineto{\pgfqpoint{2.733161in}{1.041259in}}%
\pgfpathlineto{\pgfqpoint{2.758883in}{1.043752in}}%
\pgfpathlineto{\pgfqpoint{2.765032in}{1.046245in}}%
\pgfpathlineto{\pgfqpoint{2.789886in}{1.048739in}}%
\pgfpathlineto{\pgfqpoint{2.803520in}{1.051232in}}%
\pgfpathlineto{\pgfqpoint{2.842630in}{1.053725in}}%
\pgfpathlineto{\pgfqpoint{2.853888in}{1.056218in}}%
\pgfpathlineto{\pgfqpoint{2.982637in}{1.058711in}}%
\pgfpathlineto{\pgfqpoint{2.990989in}{1.061204in}}%
\pgfpathlineto{\pgfqpoint{2.992064in}{1.063697in}}%
\pgfpathlineto{\pgfqpoint{3.105910in}{1.066191in}}%
\pgfpathlineto{\pgfqpoint{3.165345in}{1.068684in}}%
\pgfpathlineto{\pgfqpoint{3.174245in}{1.071177in}}%
\pgfpathlineto{\pgfqpoint{3.181880in}{1.073670in}}%
\pgfpathlineto{\pgfqpoint{3.185221in}{1.076163in}}%
\pgfpathlineto{\pgfqpoint{3.349434in}{1.078656in}}%
\pgfpathlineto{\pgfqpoint{3.480793in}{1.081149in}}%
\pgfpathlineto{\pgfqpoint{3.507138in}{1.083642in}}%
\pgfpathlineto{\pgfqpoint{3.555930in}{1.086136in}}%
\pgfpathlineto{\pgfqpoint{3.558122in}{1.088629in}}%
\pgfpathlineto{\pgfqpoint{3.651569in}{1.091122in}}%
\pgfpathlineto{\pgfqpoint{3.669804in}{1.093615in}}%
\pgfpathlineto{\pgfqpoint{3.691181in}{1.096108in}}%
\pgfpathlineto{\pgfqpoint{3.696058in}{1.098601in}}%
\pgfpathlineto{\pgfqpoint{3.705039in}{1.101094in}}%
\pgfpathlineto{\pgfqpoint{3.833059in}{1.103588in}}%
\pgfpathlineto{\pgfqpoint{3.923858in}{1.106081in}}%
\pgfpathlineto{\pgfqpoint{4.024851in}{1.108574in}}%
\pgfpathlineto{\pgfqpoint{4.175414in}{1.111067in}}%
\pgfpathlineto{\pgfqpoint{4.311766in}{1.113560in}}%
\pgfpathlineto{\pgfqpoint{4.348586in}{1.116053in}}%
\pgfpathlineto{\pgfqpoint{4.391500in}{1.118546in}}%
\pgfpathlineto{\pgfqpoint{4.487557in}{1.121040in}}%
\pgfpathlineto{\pgfqpoint{4.543350in}{1.123533in}}%
\pgfpathlineto{\pgfqpoint{4.552968in}{1.126026in}}%
\pgfpathlineto{\pgfqpoint{4.552968in}{1.126026in}}%
\pgfusepath{stroke}%
\end{pgfscope}%
\begin{pgfscope}%
\pgfpathrectangle{\pgfqpoint{0.537394in}{0.467838in}}{\pgfqpoint{4.094684in}{0.997254in}}%
\pgfusepath{clip}%
\pgfsetrectcap%
\pgfsetroundjoin%
\pgfsetlinewidth{1.003750pt}%
\definecolor{currentstroke}{rgb}{1.000000,0.498039,0.054902}%
\pgfsetstrokecolor{currentstroke}%
\pgfsetdash{}{0pt}%
\pgfpathmoveto{\pgfqpoint{0.928126in}{0.480304in}}%
\pgfpathlineto{\pgfqpoint{1.030443in}{0.485290in}}%
\pgfpathlineto{\pgfqpoint{1.109806in}{0.490277in}}%
\pgfpathlineto{\pgfqpoint{1.229476in}{0.492770in}}%
\pgfpathlineto{\pgfqpoint{1.276968in}{0.500249in}}%
\pgfpathlineto{\pgfqpoint{1.318858in}{0.502742in}}%
\pgfpathlineto{\pgfqpoint{1.356331in}{0.507728in}}%
\pgfpathlineto{\pgfqpoint{1.390229in}{0.522687in}}%
\pgfpathlineto{\pgfqpoint{1.421175in}{0.527674in}}%
\pgfpathlineto{\pgfqpoint{1.449643in}{0.532660in}}%
\pgfpathlineto{\pgfqpoint{1.476000in}{0.550112in}}%
\pgfpathlineto{\pgfqpoint{1.500538in}{0.557591in}}%
\pgfpathlineto{\pgfqpoint{1.523492in}{0.572550in}}%
\pgfpathlineto{\pgfqpoint{1.545054in}{0.585016in}}%
\pgfpathlineto{\pgfqpoint{1.565383in}{0.592495in}}%
\pgfpathlineto{\pgfqpoint{1.584612in}{0.607454in}}%
\pgfpathlineto{\pgfqpoint{1.602855in}{0.609947in}}%
\pgfpathlineto{\pgfqpoint{1.620208in}{0.632385in}}%
\pgfpathlineto{\pgfqpoint{1.652563in}{0.642358in}}%
\pgfpathlineto{\pgfqpoint{1.667700in}{0.657317in}}%
\pgfpathlineto{\pgfqpoint{1.696168in}{0.667289in}}%
\pgfpathlineto{\pgfqpoint{1.709590in}{0.677262in}}%
\pgfpathlineto{\pgfqpoint{1.722525in}{0.689727in}}%
\pgfpathlineto{\pgfqpoint{1.735005in}{0.699700in}}%
\pgfpathlineto{\pgfqpoint{1.747063in}{0.707179in}}%
\pgfpathlineto{\pgfqpoint{1.758725in}{0.717152in}}%
\pgfpathlineto{\pgfqpoint{1.770017in}{0.729617in}}%
\pgfpathlineto{\pgfqpoint{1.780961in}{0.739590in}}%
\pgfpathlineto{\pgfqpoint{1.791578in}{0.744576in}}%
\pgfpathlineto{\pgfqpoint{1.821652in}{0.754549in}}%
\pgfpathlineto{\pgfqpoint{1.840375in}{0.759535in}}%
\pgfpathlineto{\pgfqpoint{1.858162in}{0.774494in}}%
\pgfpathlineto{\pgfqpoint{1.866732in}{0.779480in}}%
\pgfpathlineto{\pgfqpoint{1.891270in}{0.786960in}}%
\pgfpathlineto{\pgfqpoint{1.899087in}{0.789453in}}%
\pgfpathlineto{\pgfqpoint{1.906736in}{0.794439in}}%
\pgfpathlineto{\pgfqpoint{1.921558in}{0.796932in}}%
\pgfpathlineto{\pgfqpoint{1.942692in}{0.804411in}}%
\pgfpathlineto{\pgfqpoint{1.962641in}{0.811891in}}%
\pgfpathlineto{\pgfqpoint{1.969049in}{0.814384in}}%
\pgfpathlineto{\pgfqpoint{1.981530in}{0.816877in}}%
\pgfpathlineto{\pgfqpoint{2.005249in}{0.819370in}}%
\pgfpathlineto{\pgfqpoint{2.010940in}{0.821863in}}%
\pgfpathlineto{\pgfqpoint{2.016541in}{0.826850in}}%
\pgfpathlineto{\pgfqpoint{2.022055in}{0.829343in}}%
\pgfpathlineto{\pgfqpoint{2.027485in}{0.836822in}}%
\pgfpathlineto{\pgfqpoint{2.048413in}{0.846795in}}%
\pgfpathlineto{\pgfqpoint{2.058432in}{0.854274in}}%
\pgfpathlineto{\pgfqpoint{2.068177in}{0.856767in}}%
\pgfpathlineto{\pgfqpoint{2.082311in}{0.864247in}}%
\pgfpathlineto{\pgfqpoint{2.091431in}{0.866740in}}%
\pgfpathlineto{\pgfqpoint{2.108998in}{0.876712in}}%
\pgfpathlineto{\pgfqpoint{2.113257in}{0.879205in}}%
\pgfpathlineto{\pgfqpoint{2.121626in}{0.881699in}}%
\pgfpathlineto{\pgfqpoint{2.125738in}{0.886685in}}%
\pgfpathlineto{\pgfqpoint{2.137795in}{0.889178in}}%
\pgfpathlineto{\pgfqpoint{2.141725in}{0.891671in}}%
\pgfpathlineto{\pgfqpoint{2.145612in}{0.896657in}}%
\pgfpathlineto{\pgfqpoint{2.149457in}{0.904137in}}%
\pgfpathlineto{\pgfqpoint{2.157025in}{0.909123in}}%
\pgfpathlineto{\pgfqpoint{2.171693in}{0.911616in}}%
\pgfpathlineto{\pgfqpoint{2.175268in}{0.914109in}}%
\pgfpathlineto{\pgfqpoint{2.178806in}{0.921589in}}%
\pgfpathlineto{\pgfqpoint{2.182311in}{0.926575in}}%
\pgfpathlineto{\pgfqpoint{2.192620in}{0.929068in}}%
\pgfpathlineto{\pgfqpoint{2.209166in}{0.931561in}}%
\pgfpathlineto{\pgfqpoint{2.212384in}{0.934054in}}%
\pgfpathlineto{\pgfqpoint{2.240112in}{0.936548in}}%
\pgfpathlineto{\pgfqpoint{2.248894in}{0.939041in}}%
\pgfpathlineto{\pgfqpoint{2.254631in}{0.941534in}}%
\pgfpathlineto{\pgfqpoint{2.276694in}{0.944027in}}%
\pgfpathlineto{\pgfqpoint{2.279358in}{0.946520in}}%
\pgfpathlineto{\pgfqpoint{2.319475in}{0.951506in}}%
\pgfpathlineto{\pgfqpoint{2.321838in}{0.954000in}}%
\pgfpathlineto{\pgfqpoint{2.337955in}{0.956493in}}%
\pgfpathlineto{\pgfqpoint{2.342429in}{0.958986in}}%
\pgfpathlineto{\pgfqpoint{2.344645in}{0.961479in}}%
\pgfpathlineto{\pgfqpoint{2.386290in}{0.963972in}}%
\pgfpathlineto{\pgfqpoint{2.390198in}{0.968958in}}%
\pgfpathlineto{\pgfqpoint{2.403549in}{0.971451in}}%
\pgfpathlineto{\pgfqpoint{2.412788in}{0.973945in}}%
\pgfpathlineto{\pgfqpoint{2.420009in}{0.976438in}}%
\pgfpathlineto{\pgfqpoint{2.435741in}{0.978931in}}%
\pgfpathlineto{\pgfqpoint{2.462099in}{0.981424in}}%
\pgfpathlineto{\pgfqpoint{2.463683in}{0.983917in}}%
\pgfpathlineto{\pgfqpoint{2.469950in}{0.986410in}}%
\pgfpathlineto{\pgfqpoint{2.473043in}{0.988903in}}%
\pgfpathlineto{\pgfqpoint{2.482163in}{0.991397in}}%
\pgfpathlineto{\pgfqpoint{2.503989in}{0.996383in}}%
\pgfpathlineto{\pgfqpoint{2.510977in}{0.998876in}}%
\pgfpathlineto{\pgfqpoint{2.512358in}{1.003862in}}%
\pgfpathlineto{\pgfqpoint{2.515104in}{1.006355in}}%
\pgfpathlineto{\pgfqpoint{2.525883in}{1.008848in}}%
\pgfpathlineto{\pgfqpoint{2.531152in}{1.011342in}}%
\pgfpathlineto{\pgfqpoint{2.542730in}{1.013835in}}%
\pgfpathlineto{\pgfqpoint{2.547757in}{1.016328in}}%
\pgfpathlineto{\pgfqpoint{2.555167in}{1.018821in}}%
\pgfpathlineto{\pgfqpoint{2.613640in}{1.021314in}}%
\pgfpathlineto{\pgfqpoint{2.616737in}{1.023807in}}%
\pgfpathlineto{\pgfqpoint{2.657483in}{1.026300in}}%
\pgfpathlineto{\pgfqpoint{2.669205in}{1.028794in}}%
\pgfpathlineto{\pgfqpoint{2.673612in}{1.031287in}}%
\pgfpathlineto{\pgfqpoint{2.691548in}{1.033780in}}%
\pgfpathlineto{\pgfqpoint{2.715697in}{1.036273in}}%
\pgfpathlineto{\pgfqpoint{2.733161in}{1.038766in}}%
\pgfpathlineto{\pgfqpoint{2.817235in}{1.041259in}}%
\pgfpathlineto{\pgfqpoint{2.927934in}{1.043752in}}%
\pgfpathlineto{\pgfqpoint{2.938489in}{1.046245in}}%
\pgfpathlineto{\pgfqpoint{2.950754in}{1.048739in}}%
\pgfpathlineto{\pgfqpoint{3.059641in}{1.051232in}}%
\pgfpathlineto{\pgfqpoint{3.080889in}{1.053725in}}%
\pgfpathlineto{\pgfqpoint{3.250896in}{1.056218in}}%
\pgfpathlineto{\pgfqpoint{3.259940in}{1.058711in}}%
\pgfpathlineto{\pgfqpoint{3.261621in}{1.061204in}}%
\pgfpathlineto{\pgfqpoint{3.370788in}{1.063697in}}%
\pgfpathlineto{\pgfqpoint{3.420233in}{1.066191in}}%
\pgfpathlineto{\pgfqpoint{3.429243in}{1.068684in}}%
\pgfpathlineto{\pgfqpoint{3.448494in}{1.071177in}}%
\pgfpathlineto{\pgfqpoint{3.511483in}{1.073670in}}%
\pgfpathlineto{\pgfqpoint{3.597984in}{1.076163in}}%
\pgfpathlineto{\pgfqpoint{3.606287in}{1.078656in}}%
\pgfpathlineto{\pgfqpoint{3.744420in}{1.081149in}}%
\pgfpathlineto{\pgfqpoint{3.744679in}{1.083642in}}%
\pgfpathlineto{\pgfqpoint{3.745239in}{1.086136in}}%
\pgfpathlineto{\pgfqpoint{4.049382in}{1.088629in}}%
\pgfpathlineto{\pgfqpoint{4.091553in}{1.091122in}}%
\pgfpathlineto{\pgfqpoint{4.120306in}{1.093615in}}%
\pgfpathlineto{\pgfqpoint{4.140420in}{1.096108in}}%
\pgfpathlineto{\pgfqpoint{4.145346in}{1.098601in}}%
\pgfpathlineto{\pgfqpoint{4.158826in}{1.101094in}}%
\pgfpathlineto{\pgfqpoint{4.180755in}{1.103588in}}%
\pgfpathlineto{\pgfqpoint{4.236097in}{1.106081in}}%
\pgfpathlineto{\pgfqpoint{4.275770in}{1.108574in}}%
\pgfpathlineto{\pgfqpoint{4.291495in}{1.111067in}}%
\pgfpathlineto{\pgfqpoint{4.311407in}{1.113560in}}%
\pgfpathlineto{\pgfqpoint{4.322619in}{1.116053in}}%
\pgfpathlineto{\pgfqpoint{4.405195in}{1.118546in}}%
\pgfpathlineto{\pgfqpoint{4.415629in}{1.121040in}}%
\pgfpathlineto{\pgfqpoint{4.419393in}{1.123533in}}%
\pgfpathlineto{\pgfqpoint{4.419393in}{1.123533in}}%
\pgfusepath{stroke}%
\end{pgfscope}%
\begin{pgfscope}%
\pgfpathrectangle{\pgfqpoint{0.537394in}{0.467838in}}{\pgfqpoint{4.094684in}{0.997254in}}%
\pgfusepath{clip}%
\pgfsetbuttcap%
\pgfsetroundjoin%
\pgfsetlinewidth{1.003750pt}%
\definecolor{currentstroke}{rgb}{0.172549,0.627451,0.172549}%
\pgfsetstrokecolor{currentstroke}%
\pgfsetdash{{3.700000pt}{1.600000pt}}{0.000000pt}%
\pgfpathmoveto{\pgfqpoint{0.928126in}{0.485290in}}%
\pgfpathlineto{\pgfqpoint{1.030443in}{0.495263in}}%
\pgfpathlineto{\pgfqpoint{1.109806in}{0.497756in}}%
\pgfpathlineto{\pgfqpoint{1.174651in}{0.502742in}}%
\pgfpathlineto{\pgfqpoint{1.229476in}{0.505235in}}%
\pgfpathlineto{\pgfqpoint{1.276968in}{0.507728in}}%
\pgfpathlineto{\pgfqpoint{1.390229in}{0.515208in}}%
\pgfpathlineto{\pgfqpoint{1.421175in}{0.517701in}}%
\pgfpathlineto{\pgfqpoint{1.449643in}{0.522687in}}%
\pgfpathlineto{\pgfqpoint{1.500538in}{0.527674in}}%
\pgfpathlineto{\pgfqpoint{1.584612in}{0.542632in}}%
\pgfpathlineto{\pgfqpoint{1.620208in}{0.547619in}}%
\pgfpathlineto{\pgfqpoint{1.667700in}{0.550112in}}%
\pgfpathlineto{\pgfqpoint{1.682218in}{0.552605in}}%
\pgfpathlineto{\pgfqpoint{1.735005in}{0.555098in}}%
\pgfpathlineto{\pgfqpoint{1.801888in}{0.557591in}}%
\pgfpathlineto{\pgfqpoint{1.858162in}{0.565071in}}%
\pgfpathlineto{\pgfqpoint{1.866732in}{0.567564in}}%
\pgfpathlineto{\pgfqpoint{1.928743in}{0.570057in}}%
\pgfpathlineto{\pgfqpoint{1.942692in}{0.572550in}}%
\pgfpathlineto{\pgfqpoint{1.981530in}{0.575043in}}%
\pgfpathlineto{\pgfqpoint{2.063338in}{0.577536in}}%
\pgfpathlineto{\pgfqpoint{2.068177in}{0.582523in}}%
\pgfpathlineto{\pgfqpoint{2.125738in}{0.585016in}}%
\pgfpathlineto{\pgfqpoint{2.129802in}{0.587509in}}%
\pgfpathlineto{\pgfqpoint{2.153261in}{0.590002in}}%
\pgfpathlineto{\pgfqpoint{2.157025in}{0.592495in}}%
\pgfpathlineto{\pgfqpoint{2.168082in}{0.597481in}}%
\pgfpathlineto{\pgfqpoint{2.205917in}{0.599974in}}%
\pgfpathlineto{\pgfqpoint{2.260276in}{0.602468in}}%
\pgfpathlineto{\pgfqpoint{2.276694in}{0.604961in}}%
\pgfpathlineto{\pgfqpoint{2.328835in}{0.607454in}}%
\pgfpathlineto{\pgfqpoint{2.346847in}{0.609947in}}%
\pgfpathlineto{\pgfqpoint{2.454070in}{0.612440in}}%
\pgfpathlineto{\pgfqpoint{2.468394in}{0.614933in}}%
\pgfpathlineto{\pgfqpoint{2.473043in}{0.617426in}}%
\pgfpathlineto{\pgfqpoint{2.502575in}{0.619920in}}%
\pgfpathlineto{\pgfqpoint{2.503989in}{0.624906in}}%
\pgfpathlineto{\pgfqpoint{2.523219in}{0.627399in}}%
\pgfpathlineto{\pgfqpoint{2.541462in}{0.632385in}}%
\pgfpathlineto{\pgfqpoint{2.547757in}{0.634878in}}%
\pgfpathlineto{\pgfqpoint{2.550244in}{0.637371in}}%
\pgfpathlineto{\pgfqpoint{2.586724in}{0.639865in}}%
\pgfpathlineto{\pgfqpoint{2.587840in}{0.642358in}}%
\pgfpathlineto{\pgfqpoint{2.591169in}{0.644851in}}%
\pgfpathlineto{\pgfqpoint{2.593372in}{0.647344in}}%
\pgfpathlineto{\pgfqpoint{2.594468in}{0.649837in}}%
\pgfpathlineto{\pgfqpoint{2.596650in}{0.652330in}}%
\pgfpathlineto{\pgfqpoint{2.598818in}{0.657317in}}%
\pgfpathlineto{\pgfqpoint{2.617763in}{0.659810in}}%
\pgfpathlineto{\pgfqpoint{2.622851in}{0.664796in}}%
\pgfpathlineto{\pgfqpoint{2.629855in}{0.669782in}}%
\pgfpathlineto{\pgfqpoint{2.639626in}{0.672275in}}%
\pgfpathlineto{\pgfqpoint{2.684821in}{0.674768in}}%
\pgfpathlineto{\pgfqpoint{2.749100in}{0.677262in}}%
\pgfpathlineto{\pgfqpoint{2.749807in}{0.679755in}}%
\pgfpathlineto{\pgfqpoint{2.760258in}{0.682248in}}%
\pgfpathlineto{\pgfqpoint{2.801079in}{0.684741in}}%
\pgfpathlineto{\pgfqpoint{2.808350in}{0.687234in}}%
\pgfpathlineto{\pgfqpoint{2.822461in}{0.689727in}}%
\pgfpathlineto{\pgfqpoint{2.850707in}{0.692220in}}%
\pgfpathlineto{\pgfqpoint{2.868364in}{0.694714in}}%
\pgfpathlineto{\pgfqpoint{2.872895in}{0.697207in}}%
\pgfpathlineto{\pgfqpoint{2.882274in}{0.699700in}}%
\pgfpathlineto{\pgfqpoint{2.882761in}{0.702193in}}%
\pgfpathlineto{\pgfqpoint{2.888073in}{0.704686in}}%
\pgfpathlineto{\pgfqpoint{2.894721in}{0.707179in}}%
\pgfpathlineto{\pgfqpoint{2.903090in}{0.709672in}}%
\pgfpathlineto{\pgfqpoint{2.907202in}{0.712165in}}%
\pgfpathlineto{\pgfqpoint{2.910367in}{0.714659in}}%
\pgfpathlineto{\pgfqpoint{2.910817in}{0.719645in}}%
\pgfpathlineto{\pgfqpoint{2.911267in}{0.722138in}}%
\pgfpathlineto{\pgfqpoint{2.912611in}{0.724631in}}%
\pgfpathlineto{\pgfqpoint{2.925354in}{0.727124in}}%
\pgfpathlineto{\pgfqpoint{2.926216in}{0.729617in}}%
\pgfpathlineto{\pgfqpoint{2.926647in}{0.732111in}}%
\pgfpathlineto{\pgfqpoint{2.929644in}{0.734604in}}%
\pgfpathlineto{\pgfqpoint{2.935145in}{0.737097in}}%
\pgfpathlineto{\pgfqpoint{2.937239in}{0.739590in}}%
\pgfpathlineto{\pgfqpoint{2.938905in}{0.742083in}}%
\pgfpathlineto{\pgfqpoint{2.945899in}{0.744576in}}%
\pgfpathlineto{\pgfqpoint{2.955544in}{0.747069in}}%
\pgfpathlineto{\pgfqpoint{2.958309in}{0.749563in}}%
\pgfpathlineto{\pgfqpoint{2.961443in}{0.752056in}}%
\pgfpathlineto{\pgfqpoint{2.962222in}{0.754549in}}%
\pgfpathlineto{\pgfqpoint{2.969920in}{0.757042in}}%
\pgfpathlineto{\pgfqpoint{2.973331in}{0.759535in}}%
\pgfpathlineto{\pgfqpoint{2.974085in}{0.764521in}}%
\pgfpathlineto{\pgfqpoint{2.975587in}{0.769508in}}%
\pgfpathlineto{\pgfqpoint{2.978201in}{0.772001in}}%
\pgfpathlineto{\pgfqpoint{2.984104in}{0.774494in}}%
\pgfpathlineto{\pgfqpoint{2.986292in}{0.776987in}}%
\pgfpathlineto{\pgfqpoint{2.988106in}{0.779480in}}%
\pgfpathlineto{\pgfqpoint{3.005752in}{0.781973in}}%
\pgfpathlineto{\pgfqpoint{3.006440in}{0.784466in}}%
\pgfpathlineto{\pgfqpoint{3.007811in}{0.786960in}}%
\pgfpathlineto{\pgfqpoint{3.010879in}{0.794439in}}%
\pgfpathlineto{\pgfqpoint{3.011218in}{0.796932in}}%
\pgfpathlineto{\pgfqpoint{3.015599in}{0.799425in}}%
\pgfpathlineto{\pgfqpoint{3.015933in}{0.801918in}}%
\pgfpathlineto{\pgfqpoint{3.017935in}{0.804411in}}%
\pgfpathlineto{\pgfqpoint{3.054875in}{0.806905in}}%
\pgfpathlineto{\pgfqpoint{3.076966in}{0.809398in}}%
\pgfpathlineto{\pgfqpoint{3.079213in}{0.811891in}}%
\pgfpathlineto{\pgfqpoint{3.101203in}{0.814384in}}%
\pgfpathlineto{\pgfqpoint{3.101466in}{0.816877in}}%
\pgfpathlineto{\pgfqpoint{3.104348in}{0.819370in}}%
\pgfpathlineto{\pgfqpoint{3.108756in}{0.821863in}}%
\pgfpathlineto{\pgfqpoint{3.136510in}{0.824357in}}%
\pgfpathlineto{\pgfqpoint{3.136748in}{0.826850in}}%
\pgfpathlineto{\pgfqpoint{3.139832in}{0.829343in}}%
\pgfpathlineto{\pgfqpoint{3.140775in}{0.839315in}}%
\pgfpathlineto{\pgfqpoint{3.142420in}{0.841808in}}%
\pgfpathlineto{\pgfqpoint{3.172956in}{0.844302in}}%
\pgfpathlineto{\pgfqpoint{3.184597in}{0.846795in}}%
\pgfpathlineto{\pgfqpoint{3.189149in}{0.849288in}}%
\pgfpathlineto{\pgfqpoint{3.189355in}{0.851781in}}%
\pgfpathlineto{\pgfqpoint{3.190995in}{0.854274in}}%
\pgfpathlineto{\pgfqpoint{3.193846in}{0.856767in}}%
\pgfpathlineto{\pgfqpoint{3.237693in}{0.859260in}}%
\pgfpathlineto{\pgfqpoint{3.281353in}{0.864247in}}%
\pgfpathlineto{\pgfqpoint{3.322502in}{0.866740in}}%
\pgfpathlineto{\pgfqpoint{3.330605in}{0.869233in}}%
\pgfpathlineto{\pgfqpoint{3.331433in}{0.871726in}}%
\pgfpathlineto{\pgfqpoint{3.331570in}{0.874219in}}%
\pgfpathlineto{\pgfqpoint{3.333082in}{0.876712in}}%
\pgfpathlineto{\pgfqpoint{3.351264in}{0.879205in}}%
\pgfpathlineto{\pgfqpoint{3.352305in}{0.881699in}}%
\pgfpathlineto{\pgfqpoint{3.352694in}{0.886685in}}%
\pgfpathlineto{\pgfqpoint{3.352824in}{0.889178in}}%
\pgfpathlineto{\pgfqpoint{3.357337in}{0.891671in}}%
\pgfpathlineto{\pgfqpoint{3.382438in}{0.894164in}}%
\pgfpathlineto{\pgfqpoint{3.384343in}{0.896657in}}%
\pgfpathlineto{\pgfqpoint{3.395679in}{0.899151in}}%
\pgfpathlineto{\pgfqpoint{3.421198in}{0.901644in}}%
\pgfpathlineto{\pgfqpoint{3.422587in}{0.904137in}}%
\pgfpathlineto{\pgfqpoint{3.457114in}{0.906630in}}%
\pgfpathlineto{\pgfqpoint{3.458659in}{0.909123in}}%
\pgfpathlineto{\pgfqpoint{3.464963in}{0.911616in}}%
\pgfpathlineto{\pgfqpoint{3.465436in}{0.914109in}}%
\pgfpathlineto{\pgfqpoint{3.470784in}{0.916602in}}%
\pgfpathlineto{\pgfqpoint{3.471436in}{0.919096in}}%
\pgfpathlineto{\pgfqpoint{3.501032in}{0.921589in}}%
\pgfpathlineto{\pgfqpoint{3.508982in}{0.924082in}}%
\pgfpathlineto{\pgfqpoint{3.510068in}{0.926575in}}%
\pgfpathlineto{\pgfqpoint{3.512727in}{0.929068in}}%
\pgfpathlineto{\pgfqpoint{3.516104in}{0.934054in}}%
\pgfpathlineto{\pgfqpoint{3.609454in}{0.936548in}}%
\pgfpathlineto{\pgfqpoint{3.616757in}{0.939041in}}%
\pgfpathlineto{\pgfqpoint{3.640879in}{0.941534in}}%
\pgfpathlineto{\pgfqpoint{3.642377in}{0.944027in}}%
\pgfpathlineto{\pgfqpoint{3.642951in}{0.946520in}}%
\pgfpathlineto{\pgfqpoint{3.645638in}{0.949013in}}%
\pgfpathlineto{\pgfqpoint{3.646776in}{0.951506in}}%
\pgfpathlineto{\pgfqpoint{3.647059in}{0.954000in}}%
\pgfpathlineto{\pgfqpoint{3.655303in}{0.956493in}}%
\pgfpathlineto{\pgfqpoint{3.655801in}{0.958986in}}%
\pgfpathlineto{\pgfqpoint{3.657127in}{0.961479in}}%
\pgfpathlineto{\pgfqpoint{3.659381in}{0.963972in}}%
\pgfpathlineto{\pgfqpoint{3.659436in}{0.966465in}}%
\pgfpathlineto{\pgfqpoint{3.672402in}{0.968958in}}%
\pgfpathlineto{\pgfqpoint{3.674614in}{0.971451in}}%
\pgfpathlineto{\pgfqpoint{3.683532in}{0.973945in}}%
\pgfpathlineto{\pgfqpoint{3.683737in}{0.976438in}}%
\pgfpathlineto{\pgfqpoint{3.691531in}{0.981424in}}%
\pgfpathlineto{\pgfqpoint{3.692781in}{0.983917in}}%
\pgfpathlineto{\pgfqpoint{3.707106in}{0.986410in}}%
\pgfpathlineto{\pgfqpoint{3.739248in}{0.988903in}}%
\pgfpathlineto{\pgfqpoint{3.740559in}{0.991397in}}%
\pgfpathlineto{\pgfqpoint{3.787751in}{0.993890in}}%
\pgfpathlineto{\pgfqpoint{3.788819in}{0.996383in}}%
\pgfpathlineto{\pgfqpoint{3.793435in}{0.998876in}}%
\pgfpathlineto{\pgfqpoint{3.794561in}{1.001369in}}%
\pgfpathlineto{\pgfqpoint{3.799695in}{1.003862in}}%
\pgfpathlineto{\pgfqpoint{3.800875in}{1.006355in}}%
\pgfpathlineto{\pgfqpoint{3.810428in}{1.008848in}}%
\pgfpathlineto{\pgfqpoint{3.811858in}{1.011342in}}%
\pgfpathlineto{\pgfqpoint{3.832386in}{1.013835in}}%
\pgfpathlineto{\pgfqpoint{3.836173in}{1.016328in}}%
\pgfpathlineto{\pgfqpoint{3.859743in}{1.018821in}}%
\pgfpathlineto{\pgfqpoint{3.924977in}{1.021314in}}%
\pgfpathlineto{\pgfqpoint{3.925936in}{1.023807in}}%
\pgfpathlineto{\pgfqpoint{3.929621in}{1.026300in}}%
\pgfpathlineto{\pgfqpoint{3.929826in}{1.028794in}}%
\pgfpathlineto{\pgfqpoint{3.980947in}{1.031287in}}%
\pgfpathlineto{\pgfqpoint{3.982098in}{1.033780in}}%
\pgfpathlineto{\pgfqpoint{4.012499in}{1.036273in}}%
\pgfpathlineto{\pgfqpoint{4.012702in}{1.038766in}}%
\pgfpathlineto{\pgfqpoint{4.013857in}{1.041259in}}%
\pgfpathlineto{\pgfqpoint{4.015914in}{1.043752in}}%
\pgfpathlineto{\pgfqpoint{4.017319in}{1.046245in}}%
\pgfpathlineto{\pgfqpoint{4.020510in}{1.048739in}}%
\pgfpathlineto{\pgfqpoint{4.025204in}{1.051232in}}%
\pgfpathlineto{\pgfqpoint{4.026416in}{1.053725in}}%
\pgfpathlineto{\pgfqpoint{4.045684in}{1.056218in}}%
\pgfpathlineto{\pgfqpoint{4.048007in}{1.058711in}}%
\pgfpathlineto{\pgfqpoint{4.050351in}{1.063697in}}%
\pgfpathlineto{\pgfqpoint{4.051190in}{1.066191in}}%
\pgfpathlineto{\pgfqpoint{4.056558in}{1.068684in}}%
\pgfpathlineto{\pgfqpoint{4.063557in}{1.071177in}}%
\pgfpathlineto{\pgfqpoint{4.089417in}{1.073670in}}%
\pgfpathlineto{\pgfqpoint{4.119691in}{1.076163in}}%
\pgfpathlineto{\pgfqpoint{4.130608in}{1.078656in}}%
\pgfpathlineto{\pgfqpoint{4.133697in}{1.081149in}}%
\pgfpathlineto{\pgfqpoint{4.184968in}{1.083642in}}%
\pgfpathlineto{\pgfqpoint{4.227660in}{1.086136in}}%
\pgfpathlineto{\pgfqpoint{4.234980in}{1.088629in}}%
\pgfpathlineto{\pgfqpoint{4.235392in}{1.091122in}}%
\pgfpathlineto{\pgfqpoint{4.239341in}{1.093615in}}%
\pgfpathlineto{\pgfqpoint{4.248169in}{1.096108in}}%
\pgfpathlineto{\pgfqpoint{4.278059in}{1.098601in}}%
\pgfpathlineto{\pgfqpoint{4.281115in}{1.101094in}}%
\pgfpathlineto{\pgfqpoint{4.344003in}{1.103588in}}%
\pgfpathlineto{\pgfqpoint{4.357710in}{1.106081in}}%
\pgfpathlineto{\pgfqpoint{4.382482in}{1.108574in}}%
\pgfpathlineto{\pgfqpoint{4.382847in}{1.111067in}}%
\pgfpathlineto{\pgfqpoint{4.386342in}{1.113560in}}%
\pgfpathlineto{\pgfqpoint{4.393279in}{1.116053in}}%
\pgfpathlineto{\pgfqpoint{4.394911in}{1.118546in}}%
\pgfpathlineto{\pgfqpoint{4.405451in}{1.121040in}}%
\pgfpathlineto{\pgfqpoint{4.469755in}{1.123533in}}%
\pgfpathlineto{\pgfqpoint{4.477034in}{1.126026in}}%
\pgfpathlineto{\pgfqpoint{4.477297in}{1.128519in}}%
\pgfpathlineto{\pgfqpoint{4.519182in}{1.131012in}}%
\pgfpathlineto{\pgfqpoint{4.519182in}{1.131012in}}%
\pgfusepath{stroke}%
\end{pgfscope}%
\begin{pgfscope}%
\pgfpathrectangle{\pgfqpoint{0.537394in}{0.467838in}}{\pgfqpoint{4.094684in}{0.997254in}}%
\pgfusepath{clip}%
\pgfsetbuttcap%
\pgfsetroundjoin%
\pgfsetlinewidth{1.003750pt}%
\definecolor{currentstroke}{rgb}{0.839216,0.152941,0.156863}%
\pgfsetstrokecolor{currentstroke}%
\pgfsetdash{{3.700000pt}{1.600000pt}}{0.000000pt}%
\pgfpathmoveto{\pgfqpoint{0.928126in}{0.495263in}}%
\pgfpathlineto{\pgfqpoint{1.030443in}{0.512715in}}%
\pgfpathlineto{\pgfqpoint{1.109806in}{0.525180in}}%
\pgfpathlineto{\pgfqpoint{1.174651in}{0.532660in}}%
\pgfpathlineto{\pgfqpoint{1.229476in}{0.545125in}}%
\pgfpathlineto{\pgfqpoint{1.276968in}{0.547619in}}%
\pgfpathlineto{\pgfqpoint{1.318858in}{0.557591in}}%
\pgfpathlineto{\pgfqpoint{1.421175in}{0.565071in}}%
\pgfpathlineto{\pgfqpoint{1.476000in}{0.570057in}}%
\pgfpathlineto{\pgfqpoint{1.523492in}{0.575043in}}%
\pgfpathlineto{\pgfqpoint{1.565383in}{0.580029in}}%
\pgfpathlineto{\pgfqpoint{1.584612in}{0.587509in}}%
\pgfpathlineto{\pgfqpoint{1.620208in}{0.590002in}}%
\pgfpathlineto{\pgfqpoint{1.652563in}{0.599974in}}%
\pgfpathlineto{\pgfqpoint{1.667700in}{0.607454in}}%
\pgfpathlineto{\pgfqpoint{1.682218in}{0.612440in}}%
\pgfpathlineto{\pgfqpoint{1.709590in}{0.614933in}}%
\pgfpathlineto{\pgfqpoint{1.722525in}{0.617426in}}%
\pgfpathlineto{\pgfqpoint{1.758725in}{0.622413in}}%
\pgfpathlineto{\pgfqpoint{1.791578in}{0.624906in}}%
\pgfpathlineto{\pgfqpoint{1.811907in}{0.634878in}}%
\pgfpathlineto{\pgfqpoint{1.821652in}{0.637371in}}%
\pgfpathlineto{\pgfqpoint{1.831137in}{0.644851in}}%
\pgfpathlineto{\pgfqpoint{1.840375in}{0.647344in}}%
\pgfpathlineto{\pgfqpoint{1.849380in}{0.652330in}}%
\pgfpathlineto{\pgfqpoint{1.858162in}{0.662303in}}%
\pgfpathlineto{\pgfqpoint{1.875101in}{0.667289in}}%
\pgfpathlineto{\pgfqpoint{1.883278in}{0.677262in}}%
\pgfpathlineto{\pgfqpoint{1.899087in}{0.682248in}}%
\pgfpathlineto{\pgfqpoint{1.914224in}{0.684741in}}%
\pgfpathlineto{\pgfqpoint{1.935786in}{0.692220in}}%
\pgfpathlineto{\pgfqpoint{1.942692in}{0.697207in}}%
\pgfpathlineto{\pgfqpoint{1.949467in}{0.699700in}}%
\pgfpathlineto{\pgfqpoint{1.981530in}{0.704686in}}%
\pgfpathlineto{\pgfqpoint{1.987610in}{0.707179in}}%
\pgfpathlineto{\pgfqpoint{1.999466in}{0.709672in}}%
\pgfpathlineto{\pgfqpoint{2.005249in}{0.717152in}}%
\pgfpathlineto{\pgfqpoint{2.058432in}{0.722138in}}%
\pgfpathlineto{\pgfqpoint{2.072951in}{0.727124in}}%
\pgfpathlineto{\pgfqpoint{2.091431in}{0.732111in}}%
\pgfpathlineto{\pgfqpoint{2.095904in}{0.734604in}}%
\pgfpathlineto{\pgfqpoint{2.104687in}{0.737097in}}%
\pgfpathlineto{\pgfqpoint{2.108998in}{0.742083in}}%
\pgfpathlineto{\pgfqpoint{2.121626in}{0.749563in}}%
\pgfpathlineto{\pgfqpoint{2.137795in}{0.752056in}}%
\pgfpathlineto{\pgfqpoint{2.145612in}{0.762028in}}%
\pgfpathlineto{\pgfqpoint{2.164434in}{0.767014in}}%
\pgfpathlineto{\pgfqpoint{2.178806in}{0.772001in}}%
\pgfpathlineto{\pgfqpoint{2.185780in}{0.779480in}}%
\pgfpathlineto{\pgfqpoint{2.189217in}{0.784466in}}%
\pgfpathlineto{\pgfqpoint{2.245991in}{0.786960in}}%
\pgfpathlineto{\pgfqpoint{2.248894in}{0.789453in}}%
\pgfpathlineto{\pgfqpoint{2.314701in}{0.791946in}}%
\pgfpathlineto{\pgfqpoint{2.324186in}{0.794439in}}%
\pgfpathlineto{\pgfqpoint{2.331137in}{0.796932in}}%
\pgfpathlineto{\pgfqpoint{2.337955in}{0.804411in}}%
\pgfpathlineto{\pgfqpoint{2.342429in}{0.806905in}}%
\pgfpathlineto{\pgfqpoint{2.351211in}{0.809398in}}%
\pgfpathlineto{\pgfqpoint{2.397889in}{0.811891in}}%
\pgfpathlineto{\pgfqpoint{2.414607in}{0.814384in}}%
\pgfpathlineto{\pgfqpoint{2.420009in}{0.816877in}}%
\pgfpathlineto{\pgfqpoint{2.423566in}{0.819370in}}%
\pgfpathlineto{\pgfqpoint{2.435741in}{0.821863in}}%
\pgfpathlineto{\pgfqpoint{2.437447in}{0.824357in}}%
\pgfpathlineto{\pgfqpoint{2.462099in}{0.826850in}}%
\pgfpathlineto{\pgfqpoint{2.474579in}{0.829343in}}%
\pgfpathlineto{\pgfqpoint{2.483660in}{0.831836in}}%
\pgfpathlineto{\pgfqpoint{2.517830in}{0.834329in}}%
\pgfpathlineto{\pgfqpoint{2.520534in}{0.839315in}}%
\pgfpathlineto{\pgfqpoint{2.546507in}{0.841808in}}%
\pgfpathlineto{\pgfqpoint{2.552714in}{0.844302in}}%
\pgfpathlineto{\pgfqpoint{2.562425in}{0.846795in}}%
\pgfpathlineto{\pgfqpoint{2.564812in}{0.849288in}}%
\pgfpathlineto{\pgfqpoint{2.566000in}{0.851781in}}%
\pgfpathlineto{\pgfqpoint{2.577662in}{0.854274in}}%
\pgfpathlineto{\pgfqpoint{2.584480in}{0.856767in}}%
\pgfpathlineto{\pgfqpoint{2.585603in}{0.859260in}}%
\pgfpathlineto{\pgfqpoint{2.595560in}{0.861754in}}%
\pgfpathlineto{\pgfqpoint{2.607363in}{0.864247in}}%
\pgfpathlineto{\pgfqpoint{2.611560in}{0.866740in}}%
\pgfpathlineto{\pgfqpoint{2.613640in}{0.871726in}}%
\pgfpathlineto{\pgfqpoint{2.615707in}{0.874219in}}%
\pgfpathlineto{\pgfqpoint{2.642506in}{0.876712in}}%
\pgfpathlineto{\pgfqpoint{2.650074in}{0.879205in}}%
\pgfpathlineto{\pgfqpoint{2.651008in}{0.881699in}}%
\pgfpathlineto{\pgfqpoint{2.654723in}{0.884192in}}%
\pgfpathlineto{\pgfqpoint{2.670090in}{0.886685in}}%
\pgfpathlineto{\pgfqpoint{2.691548in}{0.889178in}}%
\pgfpathlineto{\pgfqpoint{2.703022in}{0.891671in}}%
\pgfpathlineto{\pgfqpoint{2.727183in}{0.894164in}}%
\pgfpathlineto{\pgfqpoint{2.754723in}{0.896657in}}%
\pgfpathlineto{\pgfqpoint{2.762312in}{0.899151in}}%
\pgfpathlineto{\pgfqpoint{2.769074in}{0.901644in}}%
\pgfpathlineto{\pgfqpoint{2.779633in}{0.904137in}}%
\pgfpathlineto{\pgfqpoint{2.798005in}{0.906630in}}%
\pgfpathlineto{\pgfqpoint{2.803520in}{0.909123in}}%
\pgfpathlineto{\pgfqpoint{2.862232in}{0.911616in}}%
\pgfpathlineto{\pgfqpoint{2.883247in}{0.914109in}}%
\pgfpathlineto{\pgfqpoint{2.884218in}{0.916602in}}%
\pgfpathlineto{\pgfqpoint{2.886151in}{0.919096in}}%
\pgfpathlineto{\pgfqpoint{2.886632in}{0.921589in}}%
\pgfpathlineto{\pgfqpoint{2.891413in}{0.926575in}}%
\pgfpathlineto{\pgfqpoint{2.891887in}{0.929068in}}%
\pgfpathlineto{\pgfqpoint{2.901247in}{0.934054in}}%
\pgfpathlineto{\pgfqpoint{2.903090in}{0.936548in}}%
\pgfpathlineto{\pgfqpoint{2.903549in}{0.939041in}}%
\pgfpathlineto{\pgfqpoint{2.909014in}{0.944027in}}%
\pgfpathlineto{\pgfqpoint{2.917057in}{0.946520in}}%
\pgfpathlineto{\pgfqpoint{2.927934in}{0.951506in}}%
\pgfpathlineto{\pgfqpoint{2.929644in}{0.954000in}}%
\pgfpathlineto{\pgfqpoint{2.946306in}{0.956493in}}%
\pgfpathlineto{\pgfqpoint{2.978944in}{0.958986in}}%
\pgfpathlineto{\pgfqpoint{2.991706in}{0.961479in}}%
\pgfpathlineto{\pgfqpoint{3.000200in}{0.963972in}}%
\pgfpathlineto{\pgfqpoint{3.017602in}{0.966465in}}%
\pgfpathlineto{\pgfqpoint{3.056073in}{0.968958in}}%
\pgfpathlineto{\pgfqpoint{3.085046in}{0.971451in}}%
\pgfpathlineto{\pgfqpoint{3.144291in}{0.973945in}}%
\pgfpathlineto{\pgfqpoint{3.144524in}{0.976438in}}%
\pgfpathlineto{\pgfqpoint{3.163140in}{0.978931in}}%
\pgfpathlineto{\pgfqpoint{3.187707in}{0.981424in}}%
\pgfpathlineto{\pgfqpoint{3.321228in}{0.983917in}}%
\pgfpathlineto{\pgfqpoint{3.327832in}{0.986410in}}%
\pgfpathlineto{\pgfqpoint{3.329914in}{0.988903in}}%
\pgfpathlineto{\pgfqpoint{3.339606in}{0.991397in}}%
\pgfpathlineto{\pgfqpoint{3.355539in}{0.993890in}}%
\pgfpathlineto{\pgfqpoint{3.388475in}{0.996383in}}%
\pgfpathlineto{\pgfqpoint{3.415042in}{0.998876in}}%
\pgfpathlineto{\pgfqpoint{3.424819in}{1.001369in}}%
\pgfpathlineto{\pgfqpoint{3.438948in}{1.003862in}}%
\pgfpathlineto{\pgfqpoint{3.440371in}{1.006355in}}%
\pgfpathlineto{\pgfqpoint{3.476237in}{1.008848in}}%
\pgfpathlineto{\pgfqpoint{3.477701in}{1.011342in}}%
\pgfpathlineto{\pgfqpoint{3.481878in}{1.013835in}}%
\pgfpathlineto{\pgfqpoint{3.493074in}{1.016328in}}%
\pgfpathlineto{\pgfqpoint{3.551950in}{1.018821in}}%
\pgfpathlineto{\pgfqpoint{3.578508in}{1.023807in}}%
\pgfpathlineto{\pgfqpoint{3.633824in}{1.026300in}}%
\pgfpathlineto{\pgfqpoint{3.647739in}{1.028794in}}%
\pgfpathlineto{\pgfqpoint{3.651905in}{1.031287in}}%
\pgfpathlineto{\pgfqpoint{3.659436in}{1.033780in}}%
\pgfpathlineto{\pgfqpoint{3.676290in}{1.036273in}}%
\pgfpathlineto{\pgfqpoint{3.743340in}{1.038766in}}%
\pgfpathlineto{\pgfqpoint{3.877926in}{1.041259in}}%
\pgfpathlineto{\pgfqpoint{3.880290in}{1.043752in}}%
\pgfpathlineto{\pgfqpoint{3.884127in}{1.046245in}}%
\pgfpathlineto{\pgfqpoint{3.989150in}{1.048739in}}%
\pgfpathlineto{\pgfqpoint{4.000481in}{1.051232in}}%
\pgfpathlineto{\pgfqpoint{4.009194in}{1.053725in}}%
\pgfpathlineto{\pgfqpoint{4.171761in}{1.056218in}}%
\pgfpathlineto{\pgfqpoint{4.191632in}{1.058711in}}%
\pgfpathlineto{\pgfqpoint{4.449615in}{1.061204in}}%
\pgfpathlineto{\pgfqpoint{4.455559in}{1.063697in}}%
\pgfpathlineto{\pgfqpoint{4.476103in}{1.066191in}}%
\pgfpathlineto{\pgfqpoint{4.580095in}{1.068684in}}%
\pgfpathlineto{\pgfqpoint{4.580103in}{1.071177in}}%
\pgfpathlineto{\pgfqpoint{4.590132in}{1.073670in}}%
\pgfpathlineto{\pgfqpoint{4.607935in}{1.076163in}}%
\pgfpathlineto{\pgfqpoint{4.627781in}{1.078656in}}%
\pgfpathlineto{\pgfqpoint{4.627781in}{1.078656in}}%
\pgfusepath{stroke}%
\end{pgfscope}%
\begin{pgfscope}%
\pgfsetrectcap%
\pgfsetmiterjoin%
\pgfsetlinewidth{0.803000pt}%
\definecolor{currentstroke}{rgb}{0.000000,0.000000,0.000000}%
\pgfsetstrokecolor{currentstroke}%
\pgfsetdash{}{0pt}%
\pgfpathmoveto{\pgfqpoint{0.537394in}{0.467838in}}%
\pgfpathlineto{\pgfqpoint{0.537394in}{1.465092in}}%
\pgfusepath{stroke}%
\end{pgfscope}%
\begin{pgfscope}%
\pgfsetrectcap%
\pgfsetmiterjoin%
\pgfsetlinewidth{0.803000pt}%
\definecolor{currentstroke}{rgb}{0.000000,0.000000,0.000000}%
\pgfsetstrokecolor{currentstroke}%
\pgfsetdash{}{0pt}%
\pgfpathmoveto{\pgfqpoint{4.632078in}{0.467838in}}%
\pgfpathlineto{\pgfqpoint{4.632078in}{1.465092in}}%
\pgfusepath{stroke}%
\end{pgfscope}%
\begin{pgfscope}%
\pgfsetrectcap%
\pgfsetmiterjoin%
\pgfsetlinewidth{0.803000pt}%
\definecolor{currentstroke}{rgb}{0.000000,0.000000,0.000000}%
\pgfsetstrokecolor{currentstroke}%
\pgfsetdash{}{0pt}%
\pgfpathmoveto{\pgfqpoint{0.537394in}{0.467838in}}%
\pgfpathlineto{\pgfqpoint{4.632078in}{0.467838in}}%
\pgfusepath{stroke}%
\end{pgfscope}%
\begin{pgfscope}%
\pgfsetrectcap%
\pgfsetmiterjoin%
\pgfsetlinewidth{0.803000pt}%
\definecolor{currentstroke}{rgb}{0.000000,0.000000,0.000000}%
\pgfsetstrokecolor{currentstroke}%
\pgfsetdash{}{0pt}%
\pgfpathmoveto{\pgfqpoint{0.537394in}{1.465092in}}%
\pgfpathlineto{\pgfqpoint{4.632078in}{1.465092in}}%
\pgfusepath{stroke}%
\end{pgfscope}%
\begin{pgfscope}%
\pgfsetbuttcap%
\pgfsetmiterjoin%
\definecolor{currentfill}{rgb}{1.000000,1.000000,1.000000}%
\pgfsetfillcolor{currentfill}%
\pgfsetfillopacity{0.800000}%
\pgfsetlinewidth{1.003750pt}%
\definecolor{currentstroke}{rgb}{0.800000,0.800000,0.800000}%
\pgfsetstrokecolor{currentstroke}%
\pgfsetstrokeopacity{0.800000}%
\pgfsetdash{}{0pt}%
\pgfpathmoveto{\pgfqpoint{0.615172in}{0.723860in}}%
\pgfpathlineto{\pgfqpoint{1.560408in}{0.723860in}}%
\pgfpathquadraticcurveto{\pgfqpoint{1.582630in}{0.723860in}}{\pgfqpoint{1.582630in}{0.746082in}}%
\pgfpathlineto{\pgfqpoint{1.582630in}{1.387314in}}%
\pgfpathquadraticcurveto{\pgfqpoint{1.582630in}{1.409536in}}{\pgfqpoint{1.560408in}{1.409536in}}%
\pgfpathlineto{\pgfqpoint{0.615172in}{1.409536in}}%
\pgfpathquadraticcurveto{\pgfqpoint{0.592949in}{1.409536in}}{\pgfqpoint{0.592949in}{1.387314in}}%
\pgfpathlineto{\pgfqpoint{0.592949in}{0.746082in}}%
\pgfpathquadraticcurveto{\pgfqpoint{0.592949in}{0.723860in}}{\pgfqpoint{0.615172in}{0.723860in}}%
\pgfpathclose%
\pgfusepath{stroke,fill}%
\end{pgfscope}%
\begin{pgfscope}%
\pgfsetrectcap%
\pgfsetroundjoin%
\pgfsetlinewidth{1.003750pt}%
\definecolor{currentstroke}{rgb}{0.121569,0.466667,0.705882}%
\pgfsetstrokecolor{currentstroke}%
\pgfsetdash{}{0pt}%
\pgfpathmoveto{\pgfqpoint{0.637394in}{1.319562in}}%
\pgfpathlineto{\pgfqpoint{0.859616in}{1.319562in}}%
\pgfusepath{stroke}%
\end{pgfscope}%
\begin{pgfscope}%
\definecolor{textcolor}{rgb}{0.000000,0.000000,0.000000}%
\pgfsetstrokecolor{textcolor}%
\pgfsetfillcolor{textcolor}%
\pgftext[x=0.948505in,y=1.280674in,left,base]{\color{textcolor}\sffamily\fontsize{8.000000}{9.600000}\selectfont DMC(MCS)}%
\end{pgfscope}%
\begin{pgfscope}%
\pgfsetrectcap%
\pgfsetroundjoin%
\pgfsetlinewidth{1.003750pt}%
\definecolor{currentstroke}{rgb}{1.000000,0.498039,0.054902}%
\pgfsetstrokecolor{currentstroke}%
\pgfsetdash{}{0pt}%
\pgfpathmoveto{\pgfqpoint{0.637394in}{1.156477in}}%
\pgfpathlineto{\pgfqpoint{0.859616in}{1.156477in}}%
\pgfusepath{stroke}%
\end{pgfscope}%
\begin{pgfscope}%
\definecolor{textcolor}{rgb}{0.000000,0.000000,0.000000}%
\pgfsetstrokecolor{textcolor}%
\pgfsetfillcolor{textcolor}%
\pgftext[x=0.948505in,y=1.117588in,left,base]{\color{textcolor}\sffamily\fontsize{8.000000}{9.600000}\selectfont DMC(LP)}%
\end{pgfscope}%
\begin{pgfscope}%
\pgfsetbuttcap%
\pgfsetroundjoin%
\pgfsetlinewidth{1.003750pt}%
\definecolor{currentstroke}{rgb}{0.172549,0.627451,0.172549}%
\pgfsetstrokecolor{currentstroke}%
\pgfsetdash{{3.700000pt}{1.600000pt}}{0.000000pt}%
\pgfpathmoveto{\pgfqpoint{0.637394in}{0.993391in}}%
\pgfpathlineto{\pgfqpoint{0.859616in}{0.993391in}}%
\pgfusepath{stroke}%
\end{pgfscope}%
\begin{pgfscope}%
\definecolor{textcolor}{rgb}{0.000000,0.000000,0.000000}%
\pgfsetstrokecolor{textcolor}%
\pgfsetfillcolor{textcolor}%
\pgftext[x=0.948505in,y=0.954502in,left,base]{\color{textcolor}\sffamily\fontsize{8.000000}{9.600000}\selectfont DMC(LM)}%
\end{pgfscope}%
\begin{pgfscope}%
\pgfsetbuttcap%
\pgfsetroundjoin%
\pgfsetlinewidth{1.003750pt}%
\definecolor{currentstroke}{rgb}{0.839216,0.152941,0.156863}%
\pgfsetstrokecolor{currentstroke}%
\pgfsetdash{{3.700000pt}{1.600000pt}}{0.000000pt}%
\pgfpathmoveto{\pgfqpoint{0.637394in}{0.830305in}}%
\pgfpathlineto{\pgfqpoint{0.859616in}{0.830305in}}%
\pgfusepath{stroke}%
\end{pgfscope}%
\begin{pgfscope}%
\definecolor{textcolor}{rgb}{0.000000,0.000000,0.000000}%
\pgfsetstrokecolor{textcolor}%
\pgfsetfillcolor{textcolor}%
\pgftext[x=0.948505in,y=0.791416in,left,base]{\color{textcolor}\sffamily\fontsize{8.000000}{9.600000}\selectfont DMC(MF)}%
\end{pgfscope}%
\end{pgfpicture}%
\makeatother%
\endgroup%

    \caption{
        Experiment 2 compares variable-ordering heuristics (\mcs{}, \lexp, \lexm, and \minfill{}) for the ADD-based executor \dmc.
        \mcs{} and \lexp{} are significantly faster than \lexm{} and \minfill{}.
        % The graded project-join trees here were produced by the planner \Lg{} with the tree decomposer \flowcutter{} from Experiment 1. %JD: Don't need to repeat this
    }
    \label{figExecution}
\end{figure}

%%%%%%%%%%%%%%%%%%%%%%%%%%%%%%%%%%%%%%%%%%%%%%%%%%%%%%%%%%%%%%%%%%%%%%%%%%%%%%%%

\subsection{Experiment 3: Comparing Weighted Projected Model Counters}

Informed by Experiments 1 and 2, we choose \Lg{} with \flowcutter{} as the planner and \dmc{} with \mcs{} as the executor for our framework \procount.
We compare \procount{} with the weighted projected model counters \dfp{}, \projmc{}, and \ssat{}.
Since all benchmarks are satisfiable
% (checked by the SAT solver \sat{} \cite{soos2009extending})
with positive literal weights, the model counts must be positive.
Thus, for all tools, we exclude outputs that are zero (possible floating-point underflow).
We are confident that the remaining results are correct.
Differences in model counts among tools are less than $10^{-6}$.

Figure \ref{figSolving} shows the performance of \procount{}, \dfp{}, \projmc{}, and \ssat{} with a 1000-second timeout. 
Additional statistics are given in Table \ref{tableSolving}. 
Of \benchmarks{} benchmarks, 390 are solved by at least one of four tools.
\procount{} achieves the shortest solving time on 131 benchmarks, including \dpmcUniqueBenchmarks{} solved by none of the other three tools.
Between the two \emph{virtual best solvers} in Figure \ref{figSolving}, \vbs1 (all four tools) is significantly faster than \vbs0 (three existing tools, without \procount).
\begin{figure}[t]
    \centering
    %% Creator: Matplotlib, PGF backend
%%
%% To include the figure in your LaTeX document, write
%%   \input{<filename>.pgf}
%%
%% Make sure the required packages are loaded in your preamble
%%   \usepackage{pgf}
%%
%% and, on pdftex
%%   \usepackage[utf8]{inputenc}\DeclareUnicodeCharacter{2212}{-}
%%
%% or, on luatex and xetex
%%   \usepackage{unicode-math}
%%
%% Figures using additional raster images can only be included by \input if
%% they are in the same directory as the main LaTeX file. For loading figures
%% from other directories you can use the `import` package
%%   \usepackage{import}
%%
%% and then include the figures with
%%   \import{<path to file>}{<filename>.pgf}
%%
%% Matplotlib used the following preamble
%%   \usepackage[utf8x]{inputenc}
%%   \usepackage[T1]{fontenc}
%%
\begingroup%
\makeatletter%
\begin{pgfpicture}%
\pgfpathrectangle{\pgfpointorigin}{\pgfqpoint{6.000000in}{2.500000in}}%
\pgfusepath{use as bounding box, clip}%
\begin{pgfscope}%
\pgfsetbuttcap%
\pgfsetmiterjoin%
\definecolor{currentfill}{rgb}{1.000000,1.000000,1.000000}%
\pgfsetfillcolor{currentfill}%
\pgfsetlinewidth{0.000000pt}%
\definecolor{currentstroke}{rgb}{1.000000,1.000000,1.000000}%
\pgfsetstrokecolor{currentstroke}%
\pgfsetdash{}{0pt}%
\pgfpathmoveto{\pgfqpoint{0.000000in}{0.000000in}}%
\pgfpathlineto{\pgfqpoint{6.000000in}{0.000000in}}%
\pgfpathlineto{\pgfqpoint{6.000000in}{2.500000in}}%
\pgfpathlineto{\pgfqpoint{0.000000in}{2.500000in}}%
\pgfpathclose%
\pgfusepath{fill}%
\end{pgfscope}%
\begin{pgfscope}%
\pgfsetbuttcap%
\pgfsetmiterjoin%
\definecolor{currentfill}{rgb}{1.000000,1.000000,1.000000}%
\pgfsetfillcolor{currentfill}%
\pgfsetlinewidth{0.000000pt}%
\definecolor{currentstroke}{rgb}{0.000000,0.000000,0.000000}%
\pgfsetstrokecolor{currentstroke}%
\pgfsetstrokeopacity{0.000000}%
\pgfsetdash{}{0pt}%
\pgfpathmoveto{\pgfqpoint{0.708220in}{0.535823in}}%
\pgfpathlineto{\pgfqpoint{5.753646in}{0.535823in}}%
\pgfpathlineto{\pgfqpoint{5.753646in}{2.305275in}}%
\pgfpathlineto{\pgfqpoint{0.708220in}{2.305275in}}%
\pgfpathclose%
\pgfusepath{fill}%
\end{pgfscope}%
\begin{pgfscope}%
\pgfsetbuttcap%
\pgfsetroundjoin%
\definecolor{currentfill}{rgb}{0.000000,0.000000,0.000000}%
\pgfsetfillcolor{currentfill}%
\pgfsetlinewidth{0.803000pt}%
\definecolor{currentstroke}{rgb}{0.000000,0.000000,0.000000}%
\pgfsetstrokecolor{currentstroke}%
\pgfsetdash{}{0pt}%
\pgfsys@defobject{currentmarker}{\pgfqpoint{0.000000in}{-0.048611in}}{\pgfqpoint{0.000000in}{0.000000in}}{%
\pgfpathmoveto{\pgfqpoint{0.000000in}{0.000000in}}%
\pgfpathlineto{\pgfqpoint{0.000000in}{-0.048611in}}%
\pgfusepath{stroke,fill}%
}%
\begin{pgfscope}%
\pgfsys@transformshift{0.708220in}{0.535823in}%
\pgfsys@useobject{currentmarker}{}%
\end{pgfscope}%
\end{pgfscope}%
\begin{pgfscope}%
\definecolor{textcolor}{rgb}{0.000000,0.000000,0.000000}%
\pgfsetstrokecolor{textcolor}%
\pgfsetfillcolor{textcolor}%
\pgftext[x=0.708220in,y=0.438600in,,top]{\color{textcolor}\rmfamily\fontsize{9.000000}{10.800000}\selectfont \(\displaystyle {0}\)}%
\end{pgfscope}%
\begin{pgfscope}%
\pgfsetbuttcap%
\pgfsetroundjoin%
\definecolor{currentfill}{rgb}{0.000000,0.000000,0.000000}%
\pgfsetfillcolor{currentfill}%
\pgfsetlinewidth{0.803000pt}%
\definecolor{currentstroke}{rgb}{0.000000,0.000000,0.000000}%
\pgfsetstrokecolor{currentstroke}%
\pgfsetdash{}{0pt}%
\pgfsys@defobject{currentmarker}{\pgfqpoint{0.000000in}{-0.048611in}}{\pgfqpoint{0.000000in}{0.000000in}}{%
\pgfpathmoveto{\pgfqpoint{0.000000in}{0.000000in}}%
\pgfpathlineto{\pgfqpoint{0.000000in}{-0.048611in}}%
\pgfusepath{stroke,fill}%
}%
\begin{pgfscope}%
\pgfsys@transformshift{1.338898in}{0.535823in}%
\pgfsys@useobject{currentmarker}{}%
\end{pgfscope}%
\end{pgfscope}%
\begin{pgfscope}%
\definecolor{textcolor}{rgb}{0.000000,0.000000,0.000000}%
\pgfsetstrokecolor{textcolor}%
\pgfsetfillcolor{textcolor}%
\pgftext[x=1.338898in,y=0.438600in,,top]{\color{textcolor}\rmfamily\fontsize{9.000000}{10.800000}\selectfont \(\displaystyle {50}\)}%
\end{pgfscope}%
\begin{pgfscope}%
\pgfsetbuttcap%
\pgfsetroundjoin%
\definecolor{currentfill}{rgb}{0.000000,0.000000,0.000000}%
\pgfsetfillcolor{currentfill}%
\pgfsetlinewidth{0.803000pt}%
\definecolor{currentstroke}{rgb}{0.000000,0.000000,0.000000}%
\pgfsetstrokecolor{currentstroke}%
\pgfsetdash{}{0pt}%
\pgfsys@defobject{currentmarker}{\pgfqpoint{0.000000in}{-0.048611in}}{\pgfqpoint{0.000000in}{0.000000in}}{%
\pgfpathmoveto{\pgfqpoint{0.000000in}{0.000000in}}%
\pgfpathlineto{\pgfqpoint{0.000000in}{-0.048611in}}%
\pgfusepath{stroke,fill}%
}%
\begin{pgfscope}%
\pgfsys@transformshift{1.969577in}{0.535823in}%
\pgfsys@useobject{currentmarker}{}%
\end{pgfscope}%
\end{pgfscope}%
\begin{pgfscope}%
\definecolor{textcolor}{rgb}{0.000000,0.000000,0.000000}%
\pgfsetstrokecolor{textcolor}%
\pgfsetfillcolor{textcolor}%
\pgftext[x=1.969577in,y=0.438600in,,top]{\color{textcolor}\rmfamily\fontsize{9.000000}{10.800000}\selectfont \(\displaystyle {100}\)}%
\end{pgfscope}%
\begin{pgfscope}%
\pgfsetbuttcap%
\pgfsetroundjoin%
\definecolor{currentfill}{rgb}{0.000000,0.000000,0.000000}%
\pgfsetfillcolor{currentfill}%
\pgfsetlinewidth{0.803000pt}%
\definecolor{currentstroke}{rgb}{0.000000,0.000000,0.000000}%
\pgfsetstrokecolor{currentstroke}%
\pgfsetdash{}{0pt}%
\pgfsys@defobject{currentmarker}{\pgfqpoint{0.000000in}{-0.048611in}}{\pgfqpoint{0.000000in}{0.000000in}}{%
\pgfpathmoveto{\pgfqpoint{0.000000in}{0.000000in}}%
\pgfpathlineto{\pgfqpoint{0.000000in}{-0.048611in}}%
\pgfusepath{stroke,fill}%
}%
\begin{pgfscope}%
\pgfsys@transformshift{2.600255in}{0.535823in}%
\pgfsys@useobject{currentmarker}{}%
\end{pgfscope}%
\end{pgfscope}%
\begin{pgfscope}%
\definecolor{textcolor}{rgb}{0.000000,0.000000,0.000000}%
\pgfsetstrokecolor{textcolor}%
\pgfsetfillcolor{textcolor}%
\pgftext[x=2.600255in,y=0.438600in,,top]{\color{textcolor}\rmfamily\fontsize{9.000000}{10.800000}\selectfont \(\displaystyle {150}\)}%
\end{pgfscope}%
\begin{pgfscope}%
\pgfsetbuttcap%
\pgfsetroundjoin%
\definecolor{currentfill}{rgb}{0.000000,0.000000,0.000000}%
\pgfsetfillcolor{currentfill}%
\pgfsetlinewidth{0.803000pt}%
\definecolor{currentstroke}{rgb}{0.000000,0.000000,0.000000}%
\pgfsetstrokecolor{currentstroke}%
\pgfsetdash{}{0pt}%
\pgfsys@defobject{currentmarker}{\pgfqpoint{0.000000in}{-0.048611in}}{\pgfqpoint{0.000000in}{0.000000in}}{%
\pgfpathmoveto{\pgfqpoint{0.000000in}{0.000000in}}%
\pgfpathlineto{\pgfqpoint{0.000000in}{-0.048611in}}%
\pgfusepath{stroke,fill}%
}%
\begin{pgfscope}%
\pgfsys@transformshift{3.230933in}{0.535823in}%
\pgfsys@useobject{currentmarker}{}%
\end{pgfscope}%
\end{pgfscope}%
\begin{pgfscope}%
\definecolor{textcolor}{rgb}{0.000000,0.000000,0.000000}%
\pgfsetstrokecolor{textcolor}%
\pgfsetfillcolor{textcolor}%
\pgftext[x=3.230933in,y=0.438600in,,top]{\color{textcolor}\rmfamily\fontsize{9.000000}{10.800000}\selectfont \(\displaystyle {200}\)}%
\end{pgfscope}%
\begin{pgfscope}%
\pgfsetbuttcap%
\pgfsetroundjoin%
\definecolor{currentfill}{rgb}{0.000000,0.000000,0.000000}%
\pgfsetfillcolor{currentfill}%
\pgfsetlinewidth{0.803000pt}%
\definecolor{currentstroke}{rgb}{0.000000,0.000000,0.000000}%
\pgfsetstrokecolor{currentstroke}%
\pgfsetdash{}{0pt}%
\pgfsys@defobject{currentmarker}{\pgfqpoint{0.000000in}{-0.048611in}}{\pgfqpoint{0.000000in}{0.000000in}}{%
\pgfpathmoveto{\pgfqpoint{0.000000in}{0.000000in}}%
\pgfpathlineto{\pgfqpoint{0.000000in}{-0.048611in}}%
\pgfusepath{stroke,fill}%
}%
\begin{pgfscope}%
\pgfsys@transformshift{3.861611in}{0.535823in}%
\pgfsys@useobject{currentmarker}{}%
\end{pgfscope}%
\end{pgfscope}%
\begin{pgfscope}%
\definecolor{textcolor}{rgb}{0.000000,0.000000,0.000000}%
\pgfsetstrokecolor{textcolor}%
\pgfsetfillcolor{textcolor}%
\pgftext[x=3.861611in,y=0.438600in,,top]{\color{textcolor}\rmfamily\fontsize{9.000000}{10.800000}\selectfont \(\displaystyle {250}\)}%
\end{pgfscope}%
\begin{pgfscope}%
\pgfsetbuttcap%
\pgfsetroundjoin%
\definecolor{currentfill}{rgb}{0.000000,0.000000,0.000000}%
\pgfsetfillcolor{currentfill}%
\pgfsetlinewidth{0.803000pt}%
\definecolor{currentstroke}{rgb}{0.000000,0.000000,0.000000}%
\pgfsetstrokecolor{currentstroke}%
\pgfsetdash{}{0pt}%
\pgfsys@defobject{currentmarker}{\pgfqpoint{0.000000in}{-0.048611in}}{\pgfqpoint{0.000000in}{0.000000in}}{%
\pgfpathmoveto{\pgfqpoint{0.000000in}{0.000000in}}%
\pgfpathlineto{\pgfqpoint{0.000000in}{-0.048611in}}%
\pgfusepath{stroke,fill}%
}%
\begin{pgfscope}%
\pgfsys@transformshift{4.492290in}{0.535823in}%
\pgfsys@useobject{currentmarker}{}%
\end{pgfscope}%
\end{pgfscope}%
\begin{pgfscope}%
\definecolor{textcolor}{rgb}{0.000000,0.000000,0.000000}%
\pgfsetstrokecolor{textcolor}%
\pgfsetfillcolor{textcolor}%
\pgftext[x=4.492290in,y=0.438600in,,top]{\color{textcolor}\rmfamily\fontsize{9.000000}{10.800000}\selectfont \(\displaystyle {300}\)}%
\end{pgfscope}%
\begin{pgfscope}%
\pgfsetbuttcap%
\pgfsetroundjoin%
\definecolor{currentfill}{rgb}{0.000000,0.000000,0.000000}%
\pgfsetfillcolor{currentfill}%
\pgfsetlinewidth{0.803000pt}%
\definecolor{currentstroke}{rgb}{0.000000,0.000000,0.000000}%
\pgfsetstrokecolor{currentstroke}%
\pgfsetdash{}{0pt}%
\pgfsys@defobject{currentmarker}{\pgfqpoint{0.000000in}{-0.048611in}}{\pgfqpoint{0.000000in}{0.000000in}}{%
\pgfpathmoveto{\pgfqpoint{0.000000in}{0.000000in}}%
\pgfpathlineto{\pgfqpoint{0.000000in}{-0.048611in}}%
\pgfusepath{stroke,fill}%
}%
\begin{pgfscope}%
\pgfsys@transformshift{5.122968in}{0.535823in}%
\pgfsys@useobject{currentmarker}{}%
\end{pgfscope}%
\end{pgfscope}%
\begin{pgfscope}%
\definecolor{textcolor}{rgb}{0.000000,0.000000,0.000000}%
\pgfsetstrokecolor{textcolor}%
\pgfsetfillcolor{textcolor}%
\pgftext[x=5.122968in,y=0.438600in,,top]{\color{textcolor}\rmfamily\fontsize{9.000000}{10.800000}\selectfont \(\displaystyle {350}\)}%
\end{pgfscope}%
\begin{pgfscope}%
\pgfsetbuttcap%
\pgfsetroundjoin%
\definecolor{currentfill}{rgb}{0.000000,0.000000,0.000000}%
\pgfsetfillcolor{currentfill}%
\pgfsetlinewidth{0.803000pt}%
\definecolor{currentstroke}{rgb}{0.000000,0.000000,0.000000}%
\pgfsetstrokecolor{currentstroke}%
\pgfsetdash{}{0pt}%
\pgfsys@defobject{currentmarker}{\pgfqpoint{0.000000in}{-0.048611in}}{\pgfqpoint{0.000000in}{0.000000in}}{%
\pgfpathmoveto{\pgfqpoint{0.000000in}{0.000000in}}%
\pgfpathlineto{\pgfqpoint{0.000000in}{-0.048611in}}%
\pgfusepath{stroke,fill}%
}%
\begin{pgfscope}%
\pgfsys@transformshift{5.753646in}{0.535823in}%
\pgfsys@useobject{currentmarker}{}%
\end{pgfscope}%
\end{pgfscope}%
\begin{pgfscope}%
\definecolor{textcolor}{rgb}{0.000000,0.000000,0.000000}%
\pgfsetstrokecolor{textcolor}%
\pgfsetfillcolor{textcolor}%
\pgftext[x=5.753646in,y=0.438600in,,top]{\color{textcolor}\rmfamily\fontsize{9.000000}{10.800000}\selectfont \(\displaystyle {400}\)}%
\end{pgfscope}%
\begin{pgfscope}%
\definecolor{textcolor}{rgb}{0.000000,0.000000,0.000000}%
\pgfsetstrokecolor{textcolor}%
\pgfsetfillcolor{textcolor}%
\pgftext[x=3.230933in,y=0.272655in,,top]{\color{textcolor}\rmfamily\fontsize{10.000000}{12.000000}\selectfont Number of benchmarks solved}%
\end{pgfscope}%
\begin{pgfscope}%
\pgfsetbuttcap%
\pgfsetroundjoin%
\definecolor{currentfill}{rgb}{0.000000,0.000000,0.000000}%
\pgfsetfillcolor{currentfill}%
\pgfsetlinewidth{0.803000pt}%
\definecolor{currentstroke}{rgb}{0.000000,0.000000,0.000000}%
\pgfsetstrokecolor{currentstroke}%
\pgfsetdash{}{0pt}%
\pgfsys@defobject{currentmarker}{\pgfqpoint{-0.048611in}{0.000000in}}{\pgfqpoint{-0.000000in}{0.000000in}}{%
\pgfpathmoveto{\pgfqpoint{-0.000000in}{0.000000in}}%
\pgfpathlineto{\pgfqpoint{-0.048611in}{0.000000in}}%
\pgfusepath{stroke,fill}%
}%
\begin{pgfscope}%
\pgfsys@transformshift{0.708220in}{0.620358in}%
\pgfsys@useobject{currentmarker}{}%
\end{pgfscope}%
\end{pgfscope}%
\begin{pgfscope}%
\definecolor{textcolor}{rgb}{0.000000,0.000000,0.000000}%
\pgfsetstrokecolor{textcolor}%
\pgfsetfillcolor{textcolor}%
\pgftext[x=0.344411in, y=0.575633in, left, base]{\color{textcolor}\rmfamily\fontsize{9.000000}{10.800000}\selectfont \(\displaystyle {10^{-3}}\)}%
\end{pgfscope}%
\begin{pgfscope}%
\pgfsetbuttcap%
\pgfsetroundjoin%
\definecolor{currentfill}{rgb}{0.000000,0.000000,0.000000}%
\pgfsetfillcolor{currentfill}%
\pgfsetlinewidth{0.803000pt}%
\definecolor{currentstroke}{rgb}{0.000000,0.000000,0.000000}%
\pgfsetstrokecolor{currentstroke}%
\pgfsetdash{}{0pt}%
\pgfsys@defobject{currentmarker}{\pgfqpoint{-0.048611in}{0.000000in}}{\pgfqpoint{-0.000000in}{0.000000in}}{%
\pgfpathmoveto{\pgfqpoint{-0.000000in}{0.000000in}}%
\pgfpathlineto{\pgfqpoint{-0.048611in}{0.000000in}}%
\pgfusepath{stroke,fill}%
}%
\begin{pgfscope}%
\pgfsys@transformshift{0.708220in}{0.901177in}%
\pgfsys@useobject{currentmarker}{}%
\end{pgfscope}%
\end{pgfscope}%
\begin{pgfscope}%
\definecolor{textcolor}{rgb}{0.000000,0.000000,0.000000}%
\pgfsetstrokecolor{textcolor}%
\pgfsetfillcolor{textcolor}%
\pgftext[x=0.344411in, y=0.856453in, left, base]{\color{textcolor}\rmfamily\fontsize{9.000000}{10.800000}\selectfont \(\displaystyle {10^{-2}}\)}%
\end{pgfscope}%
\begin{pgfscope}%
\pgfsetbuttcap%
\pgfsetroundjoin%
\definecolor{currentfill}{rgb}{0.000000,0.000000,0.000000}%
\pgfsetfillcolor{currentfill}%
\pgfsetlinewidth{0.803000pt}%
\definecolor{currentstroke}{rgb}{0.000000,0.000000,0.000000}%
\pgfsetstrokecolor{currentstroke}%
\pgfsetdash{}{0pt}%
\pgfsys@defobject{currentmarker}{\pgfqpoint{-0.048611in}{0.000000in}}{\pgfqpoint{-0.000000in}{0.000000in}}{%
\pgfpathmoveto{\pgfqpoint{-0.000000in}{0.000000in}}%
\pgfpathlineto{\pgfqpoint{-0.048611in}{0.000000in}}%
\pgfusepath{stroke,fill}%
}%
\begin{pgfscope}%
\pgfsys@transformshift{0.708220in}{1.181997in}%
\pgfsys@useobject{currentmarker}{}%
\end{pgfscope}%
\end{pgfscope}%
\begin{pgfscope}%
\definecolor{textcolor}{rgb}{0.000000,0.000000,0.000000}%
\pgfsetstrokecolor{textcolor}%
\pgfsetfillcolor{textcolor}%
\pgftext[x=0.344411in, y=1.137272in, left, base]{\color{textcolor}\rmfamily\fontsize{9.000000}{10.800000}\selectfont \(\displaystyle {10^{-1}}\)}%
\end{pgfscope}%
\begin{pgfscope}%
\pgfsetbuttcap%
\pgfsetroundjoin%
\definecolor{currentfill}{rgb}{0.000000,0.000000,0.000000}%
\pgfsetfillcolor{currentfill}%
\pgfsetlinewidth{0.803000pt}%
\definecolor{currentstroke}{rgb}{0.000000,0.000000,0.000000}%
\pgfsetstrokecolor{currentstroke}%
\pgfsetdash{}{0pt}%
\pgfsys@defobject{currentmarker}{\pgfqpoint{-0.048611in}{0.000000in}}{\pgfqpoint{-0.000000in}{0.000000in}}{%
\pgfpathmoveto{\pgfqpoint{-0.000000in}{0.000000in}}%
\pgfpathlineto{\pgfqpoint{-0.048611in}{0.000000in}}%
\pgfusepath{stroke,fill}%
}%
\begin{pgfscope}%
\pgfsys@transformshift{0.708220in}{1.462816in}%
\pgfsys@useobject{currentmarker}{}%
\end{pgfscope}%
\end{pgfscope}%
\begin{pgfscope}%
\definecolor{textcolor}{rgb}{0.000000,0.000000,0.000000}%
\pgfsetstrokecolor{textcolor}%
\pgfsetfillcolor{textcolor}%
\pgftext[x=0.424657in, y=1.418092in, left, base]{\color{textcolor}\rmfamily\fontsize{9.000000}{10.800000}\selectfont \(\displaystyle {10^{0}}\)}%
\end{pgfscope}%
\begin{pgfscope}%
\pgfsetbuttcap%
\pgfsetroundjoin%
\definecolor{currentfill}{rgb}{0.000000,0.000000,0.000000}%
\pgfsetfillcolor{currentfill}%
\pgfsetlinewidth{0.803000pt}%
\definecolor{currentstroke}{rgb}{0.000000,0.000000,0.000000}%
\pgfsetstrokecolor{currentstroke}%
\pgfsetdash{}{0pt}%
\pgfsys@defobject{currentmarker}{\pgfqpoint{-0.048611in}{0.000000in}}{\pgfqpoint{-0.000000in}{0.000000in}}{%
\pgfpathmoveto{\pgfqpoint{-0.000000in}{0.000000in}}%
\pgfpathlineto{\pgfqpoint{-0.048611in}{0.000000in}}%
\pgfusepath{stroke,fill}%
}%
\begin{pgfscope}%
\pgfsys@transformshift{0.708220in}{1.743636in}%
\pgfsys@useobject{currentmarker}{}%
\end{pgfscope}%
\end{pgfscope}%
\begin{pgfscope}%
\definecolor{textcolor}{rgb}{0.000000,0.000000,0.000000}%
\pgfsetstrokecolor{textcolor}%
\pgfsetfillcolor{textcolor}%
\pgftext[x=0.424657in, y=1.698911in, left, base]{\color{textcolor}\rmfamily\fontsize{9.000000}{10.800000}\selectfont \(\displaystyle {10^{1}}\)}%
\end{pgfscope}%
\begin{pgfscope}%
\pgfsetbuttcap%
\pgfsetroundjoin%
\definecolor{currentfill}{rgb}{0.000000,0.000000,0.000000}%
\pgfsetfillcolor{currentfill}%
\pgfsetlinewidth{0.803000pt}%
\definecolor{currentstroke}{rgb}{0.000000,0.000000,0.000000}%
\pgfsetstrokecolor{currentstroke}%
\pgfsetdash{}{0pt}%
\pgfsys@defobject{currentmarker}{\pgfqpoint{-0.048611in}{0.000000in}}{\pgfqpoint{-0.000000in}{0.000000in}}{%
\pgfpathmoveto{\pgfqpoint{-0.000000in}{0.000000in}}%
\pgfpathlineto{\pgfqpoint{-0.048611in}{0.000000in}}%
\pgfusepath{stroke,fill}%
}%
\begin{pgfscope}%
\pgfsys@transformshift{0.708220in}{2.024456in}%
\pgfsys@useobject{currentmarker}{}%
\end{pgfscope}%
\end{pgfscope}%
\begin{pgfscope}%
\definecolor{textcolor}{rgb}{0.000000,0.000000,0.000000}%
\pgfsetstrokecolor{textcolor}%
\pgfsetfillcolor{textcolor}%
\pgftext[x=0.424657in, y=1.979731in, left, base]{\color{textcolor}\rmfamily\fontsize{9.000000}{10.800000}\selectfont \(\displaystyle {10^{2}}\)}%
\end{pgfscope}%
\begin{pgfscope}%
\pgfsetbuttcap%
\pgfsetroundjoin%
\definecolor{currentfill}{rgb}{0.000000,0.000000,0.000000}%
\pgfsetfillcolor{currentfill}%
\pgfsetlinewidth{0.803000pt}%
\definecolor{currentstroke}{rgb}{0.000000,0.000000,0.000000}%
\pgfsetstrokecolor{currentstroke}%
\pgfsetdash{}{0pt}%
\pgfsys@defobject{currentmarker}{\pgfqpoint{-0.048611in}{0.000000in}}{\pgfqpoint{-0.000000in}{0.000000in}}{%
\pgfpathmoveto{\pgfqpoint{-0.000000in}{0.000000in}}%
\pgfpathlineto{\pgfqpoint{-0.048611in}{0.000000in}}%
\pgfusepath{stroke,fill}%
}%
\begin{pgfscope}%
\pgfsys@transformshift{0.708220in}{2.305275in}%
\pgfsys@useobject{currentmarker}{}%
\end{pgfscope}%
\end{pgfscope}%
\begin{pgfscope}%
\definecolor{textcolor}{rgb}{0.000000,0.000000,0.000000}%
\pgfsetstrokecolor{textcolor}%
\pgfsetfillcolor{textcolor}%
\pgftext[x=0.424657in, y=2.260550in, left, base]{\color{textcolor}\rmfamily\fontsize{9.000000}{10.800000}\selectfont \(\displaystyle {10^{3}}\)}%
\end{pgfscope}%
\begin{pgfscope}%
\pgfsetbuttcap%
\pgfsetroundjoin%
\definecolor{currentfill}{rgb}{0.000000,0.000000,0.000000}%
\pgfsetfillcolor{currentfill}%
\pgfsetlinewidth{0.602250pt}%
\definecolor{currentstroke}{rgb}{0.000000,0.000000,0.000000}%
\pgfsetstrokecolor{currentstroke}%
\pgfsetdash{}{0pt}%
\pgfsys@defobject{currentmarker}{\pgfqpoint{-0.027778in}{0.000000in}}{\pgfqpoint{-0.000000in}{0.000000in}}{%
\pgfpathmoveto{\pgfqpoint{-0.000000in}{0.000000in}}%
\pgfpathlineto{\pgfqpoint{-0.027778in}{0.000000in}}%
\pgfusepath{stroke,fill}%
}%
\begin{pgfscope}%
\pgfsys@transformshift{0.708220in}{0.535823in}%
\pgfsys@useobject{currentmarker}{}%
\end{pgfscope}%
\end{pgfscope}%
\begin{pgfscope}%
\pgfsetbuttcap%
\pgfsetroundjoin%
\definecolor{currentfill}{rgb}{0.000000,0.000000,0.000000}%
\pgfsetfillcolor{currentfill}%
\pgfsetlinewidth{0.602250pt}%
\definecolor{currentstroke}{rgb}{0.000000,0.000000,0.000000}%
\pgfsetstrokecolor{currentstroke}%
\pgfsetdash{}{0pt}%
\pgfsys@defobject{currentmarker}{\pgfqpoint{-0.027778in}{0.000000in}}{\pgfqpoint{-0.000000in}{0.000000in}}{%
\pgfpathmoveto{\pgfqpoint{-0.000000in}{0.000000in}}%
\pgfpathlineto{\pgfqpoint{-0.027778in}{0.000000in}}%
\pgfusepath{stroke,fill}%
}%
\begin{pgfscope}%
\pgfsys@transformshift{0.708220in}{0.558058in}%
\pgfsys@useobject{currentmarker}{}%
\end{pgfscope}%
\end{pgfscope}%
\begin{pgfscope}%
\pgfsetbuttcap%
\pgfsetroundjoin%
\definecolor{currentfill}{rgb}{0.000000,0.000000,0.000000}%
\pgfsetfillcolor{currentfill}%
\pgfsetlinewidth{0.602250pt}%
\definecolor{currentstroke}{rgb}{0.000000,0.000000,0.000000}%
\pgfsetstrokecolor{currentstroke}%
\pgfsetdash{}{0pt}%
\pgfsys@defobject{currentmarker}{\pgfqpoint{-0.027778in}{0.000000in}}{\pgfqpoint{-0.000000in}{0.000000in}}{%
\pgfpathmoveto{\pgfqpoint{-0.000000in}{0.000000in}}%
\pgfpathlineto{\pgfqpoint{-0.027778in}{0.000000in}}%
\pgfusepath{stroke,fill}%
}%
\begin{pgfscope}%
\pgfsys@transformshift{0.708220in}{0.576858in}%
\pgfsys@useobject{currentmarker}{}%
\end{pgfscope}%
\end{pgfscope}%
\begin{pgfscope}%
\pgfsetbuttcap%
\pgfsetroundjoin%
\definecolor{currentfill}{rgb}{0.000000,0.000000,0.000000}%
\pgfsetfillcolor{currentfill}%
\pgfsetlinewidth{0.602250pt}%
\definecolor{currentstroke}{rgb}{0.000000,0.000000,0.000000}%
\pgfsetstrokecolor{currentstroke}%
\pgfsetdash{}{0pt}%
\pgfsys@defobject{currentmarker}{\pgfqpoint{-0.027778in}{0.000000in}}{\pgfqpoint{-0.000000in}{0.000000in}}{%
\pgfpathmoveto{\pgfqpoint{-0.000000in}{0.000000in}}%
\pgfpathlineto{\pgfqpoint{-0.027778in}{0.000000in}}%
\pgfusepath{stroke,fill}%
}%
\begin{pgfscope}%
\pgfsys@transformshift{0.708220in}{0.593144in}%
\pgfsys@useobject{currentmarker}{}%
\end{pgfscope}%
\end{pgfscope}%
\begin{pgfscope}%
\pgfsetbuttcap%
\pgfsetroundjoin%
\definecolor{currentfill}{rgb}{0.000000,0.000000,0.000000}%
\pgfsetfillcolor{currentfill}%
\pgfsetlinewidth{0.602250pt}%
\definecolor{currentstroke}{rgb}{0.000000,0.000000,0.000000}%
\pgfsetstrokecolor{currentstroke}%
\pgfsetdash{}{0pt}%
\pgfsys@defobject{currentmarker}{\pgfqpoint{-0.027778in}{0.000000in}}{\pgfqpoint{-0.000000in}{0.000000in}}{%
\pgfpathmoveto{\pgfqpoint{-0.000000in}{0.000000in}}%
\pgfpathlineto{\pgfqpoint{-0.027778in}{0.000000in}}%
\pgfusepath{stroke,fill}%
}%
\begin{pgfscope}%
\pgfsys@transformshift{0.708220in}{0.607508in}%
\pgfsys@useobject{currentmarker}{}%
\end{pgfscope}%
\end{pgfscope}%
\begin{pgfscope}%
\pgfsetbuttcap%
\pgfsetroundjoin%
\definecolor{currentfill}{rgb}{0.000000,0.000000,0.000000}%
\pgfsetfillcolor{currentfill}%
\pgfsetlinewidth{0.602250pt}%
\definecolor{currentstroke}{rgb}{0.000000,0.000000,0.000000}%
\pgfsetstrokecolor{currentstroke}%
\pgfsetdash{}{0pt}%
\pgfsys@defobject{currentmarker}{\pgfqpoint{-0.027778in}{0.000000in}}{\pgfqpoint{-0.000000in}{0.000000in}}{%
\pgfpathmoveto{\pgfqpoint{-0.000000in}{0.000000in}}%
\pgfpathlineto{\pgfqpoint{-0.027778in}{0.000000in}}%
\pgfusepath{stroke,fill}%
}%
\begin{pgfscope}%
\pgfsys@transformshift{0.708220in}{0.704893in}%
\pgfsys@useobject{currentmarker}{}%
\end{pgfscope}%
\end{pgfscope}%
\begin{pgfscope}%
\pgfsetbuttcap%
\pgfsetroundjoin%
\definecolor{currentfill}{rgb}{0.000000,0.000000,0.000000}%
\pgfsetfillcolor{currentfill}%
\pgfsetlinewidth{0.602250pt}%
\definecolor{currentstroke}{rgb}{0.000000,0.000000,0.000000}%
\pgfsetstrokecolor{currentstroke}%
\pgfsetdash{}{0pt}%
\pgfsys@defobject{currentmarker}{\pgfqpoint{-0.027778in}{0.000000in}}{\pgfqpoint{-0.000000in}{0.000000in}}{%
\pgfpathmoveto{\pgfqpoint{-0.000000in}{0.000000in}}%
\pgfpathlineto{\pgfqpoint{-0.027778in}{0.000000in}}%
\pgfusepath{stroke,fill}%
}%
\begin{pgfscope}%
\pgfsys@transformshift{0.708220in}{0.754343in}%
\pgfsys@useobject{currentmarker}{}%
\end{pgfscope}%
\end{pgfscope}%
\begin{pgfscope}%
\pgfsetbuttcap%
\pgfsetroundjoin%
\definecolor{currentfill}{rgb}{0.000000,0.000000,0.000000}%
\pgfsetfillcolor{currentfill}%
\pgfsetlinewidth{0.602250pt}%
\definecolor{currentstroke}{rgb}{0.000000,0.000000,0.000000}%
\pgfsetstrokecolor{currentstroke}%
\pgfsetdash{}{0pt}%
\pgfsys@defobject{currentmarker}{\pgfqpoint{-0.027778in}{0.000000in}}{\pgfqpoint{-0.000000in}{0.000000in}}{%
\pgfpathmoveto{\pgfqpoint{-0.000000in}{0.000000in}}%
\pgfpathlineto{\pgfqpoint{-0.027778in}{0.000000in}}%
\pgfusepath{stroke,fill}%
}%
\begin{pgfscope}%
\pgfsys@transformshift{0.708220in}{0.789428in}%
\pgfsys@useobject{currentmarker}{}%
\end{pgfscope}%
\end{pgfscope}%
\begin{pgfscope}%
\pgfsetbuttcap%
\pgfsetroundjoin%
\definecolor{currentfill}{rgb}{0.000000,0.000000,0.000000}%
\pgfsetfillcolor{currentfill}%
\pgfsetlinewidth{0.602250pt}%
\definecolor{currentstroke}{rgb}{0.000000,0.000000,0.000000}%
\pgfsetstrokecolor{currentstroke}%
\pgfsetdash{}{0pt}%
\pgfsys@defobject{currentmarker}{\pgfqpoint{-0.027778in}{0.000000in}}{\pgfqpoint{-0.000000in}{0.000000in}}{%
\pgfpathmoveto{\pgfqpoint{-0.000000in}{0.000000in}}%
\pgfpathlineto{\pgfqpoint{-0.027778in}{0.000000in}}%
\pgfusepath{stroke,fill}%
}%
\begin{pgfscope}%
\pgfsys@transformshift{0.708220in}{0.816642in}%
\pgfsys@useobject{currentmarker}{}%
\end{pgfscope}%
\end{pgfscope}%
\begin{pgfscope}%
\pgfsetbuttcap%
\pgfsetroundjoin%
\definecolor{currentfill}{rgb}{0.000000,0.000000,0.000000}%
\pgfsetfillcolor{currentfill}%
\pgfsetlinewidth{0.602250pt}%
\definecolor{currentstroke}{rgb}{0.000000,0.000000,0.000000}%
\pgfsetstrokecolor{currentstroke}%
\pgfsetdash{}{0pt}%
\pgfsys@defobject{currentmarker}{\pgfqpoint{-0.027778in}{0.000000in}}{\pgfqpoint{-0.000000in}{0.000000in}}{%
\pgfpathmoveto{\pgfqpoint{-0.000000in}{0.000000in}}%
\pgfpathlineto{\pgfqpoint{-0.027778in}{0.000000in}}%
\pgfusepath{stroke,fill}%
}%
\begin{pgfscope}%
\pgfsys@transformshift{0.708220in}{0.838878in}%
\pgfsys@useobject{currentmarker}{}%
\end{pgfscope}%
\end{pgfscope}%
\begin{pgfscope}%
\pgfsetbuttcap%
\pgfsetroundjoin%
\definecolor{currentfill}{rgb}{0.000000,0.000000,0.000000}%
\pgfsetfillcolor{currentfill}%
\pgfsetlinewidth{0.602250pt}%
\definecolor{currentstroke}{rgb}{0.000000,0.000000,0.000000}%
\pgfsetstrokecolor{currentstroke}%
\pgfsetdash{}{0pt}%
\pgfsys@defobject{currentmarker}{\pgfqpoint{-0.027778in}{0.000000in}}{\pgfqpoint{-0.000000in}{0.000000in}}{%
\pgfpathmoveto{\pgfqpoint{-0.000000in}{0.000000in}}%
\pgfpathlineto{\pgfqpoint{-0.027778in}{0.000000in}}%
\pgfusepath{stroke,fill}%
}%
\begin{pgfscope}%
\pgfsys@transformshift{0.708220in}{0.857678in}%
\pgfsys@useobject{currentmarker}{}%
\end{pgfscope}%
\end{pgfscope}%
\begin{pgfscope}%
\pgfsetbuttcap%
\pgfsetroundjoin%
\definecolor{currentfill}{rgb}{0.000000,0.000000,0.000000}%
\pgfsetfillcolor{currentfill}%
\pgfsetlinewidth{0.602250pt}%
\definecolor{currentstroke}{rgb}{0.000000,0.000000,0.000000}%
\pgfsetstrokecolor{currentstroke}%
\pgfsetdash{}{0pt}%
\pgfsys@defobject{currentmarker}{\pgfqpoint{-0.027778in}{0.000000in}}{\pgfqpoint{-0.000000in}{0.000000in}}{%
\pgfpathmoveto{\pgfqpoint{-0.000000in}{0.000000in}}%
\pgfpathlineto{\pgfqpoint{-0.027778in}{0.000000in}}%
\pgfusepath{stroke,fill}%
}%
\begin{pgfscope}%
\pgfsys@transformshift{0.708220in}{0.873963in}%
\pgfsys@useobject{currentmarker}{}%
\end{pgfscope}%
\end{pgfscope}%
\begin{pgfscope}%
\pgfsetbuttcap%
\pgfsetroundjoin%
\definecolor{currentfill}{rgb}{0.000000,0.000000,0.000000}%
\pgfsetfillcolor{currentfill}%
\pgfsetlinewidth{0.602250pt}%
\definecolor{currentstroke}{rgb}{0.000000,0.000000,0.000000}%
\pgfsetstrokecolor{currentstroke}%
\pgfsetdash{}{0pt}%
\pgfsys@defobject{currentmarker}{\pgfqpoint{-0.027778in}{0.000000in}}{\pgfqpoint{-0.000000in}{0.000000in}}{%
\pgfpathmoveto{\pgfqpoint{-0.000000in}{0.000000in}}%
\pgfpathlineto{\pgfqpoint{-0.027778in}{0.000000in}}%
\pgfusepath{stroke,fill}%
}%
\begin{pgfscope}%
\pgfsys@transformshift{0.708220in}{0.888328in}%
\pgfsys@useobject{currentmarker}{}%
\end{pgfscope}%
\end{pgfscope}%
\begin{pgfscope}%
\pgfsetbuttcap%
\pgfsetroundjoin%
\definecolor{currentfill}{rgb}{0.000000,0.000000,0.000000}%
\pgfsetfillcolor{currentfill}%
\pgfsetlinewidth{0.602250pt}%
\definecolor{currentstroke}{rgb}{0.000000,0.000000,0.000000}%
\pgfsetstrokecolor{currentstroke}%
\pgfsetdash{}{0pt}%
\pgfsys@defobject{currentmarker}{\pgfqpoint{-0.027778in}{0.000000in}}{\pgfqpoint{-0.000000in}{0.000000in}}{%
\pgfpathmoveto{\pgfqpoint{-0.000000in}{0.000000in}}%
\pgfpathlineto{\pgfqpoint{-0.027778in}{0.000000in}}%
\pgfusepath{stroke,fill}%
}%
\begin{pgfscope}%
\pgfsys@transformshift{0.708220in}{0.985712in}%
\pgfsys@useobject{currentmarker}{}%
\end{pgfscope}%
\end{pgfscope}%
\begin{pgfscope}%
\pgfsetbuttcap%
\pgfsetroundjoin%
\definecolor{currentfill}{rgb}{0.000000,0.000000,0.000000}%
\pgfsetfillcolor{currentfill}%
\pgfsetlinewidth{0.602250pt}%
\definecolor{currentstroke}{rgb}{0.000000,0.000000,0.000000}%
\pgfsetstrokecolor{currentstroke}%
\pgfsetdash{}{0pt}%
\pgfsys@defobject{currentmarker}{\pgfqpoint{-0.027778in}{0.000000in}}{\pgfqpoint{-0.000000in}{0.000000in}}{%
\pgfpathmoveto{\pgfqpoint{-0.000000in}{0.000000in}}%
\pgfpathlineto{\pgfqpoint{-0.027778in}{0.000000in}}%
\pgfusepath{stroke,fill}%
}%
\begin{pgfscope}%
\pgfsys@transformshift{0.708220in}{1.035162in}%
\pgfsys@useobject{currentmarker}{}%
\end{pgfscope}%
\end{pgfscope}%
\begin{pgfscope}%
\pgfsetbuttcap%
\pgfsetroundjoin%
\definecolor{currentfill}{rgb}{0.000000,0.000000,0.000000}%
\pgfsetfillcolor{currentfill}%
\pgfsetlinewidth{0.602250pt}%
\definecolor{currentstroke}{rgb}{0.000000,0.000000,0.000000}%
\pgfsetstrokecolor{currentstroke}%
\pgfsetdash{}{0pt}%
\pgfsys@defobject{currentmarker}{\pgfqpoint{-0.027778in}{0.000000in}}{\pgfqpoint{-0.000000in}{0.000000in}}{%
\pgfpathmoveto{\pgfqpoint{-0.000000in}{0.000000in}}%
\pgfpathlineto{\pgfqpoint{-0.027778in}{0.000000in}}%
\pgfusepath{stroke,fill}%
}%
\begin{pgfscope}%
\pgfsys@transformshift{0.708220in}{1.070248in}%
\pgfsys@useobject{currentmarker}{}%
\end{pgfscope}%
\end{pgfscope}%
\begin{pgfscope}%
\pgfsetbuttcap%
\pgfsetroundjoin%
\definecolor{currentfill}{rgb}{0.000000,0.000000,0.000000}%
\pgfsetfillcolor{currentfill}%
\pgfsetlinewidth{0.602250pt}%
\definecolor{currentstroke}{rgb}{0.000000,0.000000,0.000000}%
\pgfsetstrokecolor{currentstroke}%
\pgfsetdash{}{0pt}%
\pgfsys@defobject{currentmarker}{\pgfqpoint{-0.027778in}{0.000000in}}{\pgfqpoint{-0.000000in}{0.000000in}}{%
\pgfpathmoveto{\pgfqpoint{-0.000000in}{0.000000in}}%
\pgfpathlineto{\pgfqpoint{-0.027778in}{0.000000in}}%
\pgfusepath{stroke,fill}%
}%
\begin{pgfscope}%
\pgfsys@transformshift{0.708220in}{1.097462in}%
\pgfsys@useobject{currentmarker}{}%
\end{pgfscope}%
\end{pgfscope}%
\begin{pgfscope}%
\pgfsetbuttcap%
\pgfsetroundjoin%
\definecolor{currentfill}{rgb}{0.000000,0.000000,0.000000}%
\pgfsetfillcolor{currentfill}%
\pgfsetlinewidth{0.602250pt}%
\definecolor{currentstroke}{rgb}{0.000000,0.000000,0.000000}%
\pgfsetstrokecolor{currentstroke}%
\pgfsetdash{}{0pt}%
\pgfsys@defobject{currentmarker}{\pgfqpoint{-0.027778in}{0.000000in}}{\pgfqpoint{-0.000000in}{0.000000in}}{%
\pgfpathmoveto{\pgfqpoint{-0.000000in}{0.000000in}}%
\pgfpathlineto{\pgfqpoint{-0.027778in}{0.000000in}}%
\pgfusepath{stroke,fill}%
}%
\begin{pgfscope}%
\pgfsys@transformshift{0.708220in}{1.119697in}%
\pgfsys@useobject{currentmarker}{}%
\end{pgfscope}%
\end{pgfscope}%
\begin{pgfscope}%
\pgfsetbuttcap%
\pgfsetroundjoin%
\definecolor{currentfill}{rgb}{0.000000,0.000000,0.000000}%
\pgfsetfillcolor{currentfill}%
\pgfsetlinewidth{0.602250pt}%
\definecolor{currentstroke}{rgb}{0.000000,0.000000,0.000000}%
\pgfsetstrokecolor{currentstroke}%
\pgfsetdash{}{0pt}%
\pgfsys@defobject{currentmarker}{\pgfqpoint{-0.027778in}{0.000000in}}{\pgfqpoint{-0.000000in}{0.000000in}}{%
\pgfpathmoveto{\pgfqpoint{-0.000000in}{0.000000in}}%
\pgfpathlineto{\pgfqpoint{-0.027778in}{0.000000in}}%
\pgfusepath{stroke,fill}%
}%
\begin{pgfscope}%
\pgfsys@transformshift{0.708220in}{1.138497in}%
\pgfsys@useobject{currentmarker}{}%
\end{pgfscope}%
\end{pgfscope}%
\begin{pgfscope}%
\pgfsetbuttcap%
\pgfsetroundjoin%
\definecolor{currentfill}{rgb}{0.000000,0.000000,0.000000}%
\pgfsetfillcolor{currentfill}%
\pgfsetlinewidth{0.602250pt}%
\definecolor{currentstroke}{rgb}{0.000000,0.000000,0.000000}%
\pgfsetstrokecolor{currentstroke}%
\pgfsetdash{}{0pt}%
\pgfsys@defobject{currentmarker}{\pgfqpoint{-0.027778in}{0.000000in}}{\pgfqpoint{-0.000000in}{0.000000in}}{%
\pgfpathmoveto{\pgfqpoint{-0.000000in}{0.000000in}}%
\pgfpathlineto{\pgfqpoint{-0.027778in}{0.000000in}}%
\pgfusepath{stroke,fill}%
}%
\begin{pgfscope}%
\pgfsys@transformshift{0.708220in}{1.154783in}%
\pgfsys@useobject{currentmarker}{}%
\end{pgfscope}%
\end{pgfscope}%
\begin{pgfscope}%
\pgfsetbuttcap%
\pgfsetroundjoin%
\definecolor{currentfill}{rgb}{0.000000,0.000000,0.000000}%
\pgfsetfillcolor{currentfill}%
\pgfsetlinewidth{0.602250pt}%
\definecolor{currentstroke}{rgb}{0.000000,0.000000,0.000000}%
\pgfsetstrokecolor{currentstroke}%
\pgfsetdash{}{0pt}%
\pgfsys@defobject{currentmarker}{\pgfqpoint{-0.027778in}{0.000000in}}{\pgfqpoint{-0.000000in}{0.000000in}}{%
\pgfpathmoveto{\pgfqpoint{-0.000000in}{0.000000in}}%
\pgfpathlineto{\pgfqpoint{-0.027778in}{0.000000in}}%
\pgfusepath{stroke,fill}%
}%
\begin{pgfscope}%
\pgfsys@transformshift{0.708220in}{1.169147in}%
\pgfsys@useobject{currentmarker}{}%
\end{pgfscope}%
\end{pgfscope}%
\begin{pgfscope}%
\pgfsetbuttcap%
\pgfsetroundjoin%
\definecolor{currentfill}{rgb}{0.000000,0.000000,0.000000}%
\pgfsetfillcolor{currentfill}%
\pgfsetlinewidth{0.602250pt}%
\definecolor{currentstroke}{rgb}{0.000000,0.000000,0.000000}%
\pgfsetstrokecolor{currentstroke}%
\pgfsetdash{}{0pt}%
\pgfsys@defobject{currentmarker}{\pgfqpoint{-0.027778in}{0.000000in}}{\pgfqpoint{-0.000000in}{0.000000in}}{%
\pgfpathmoveto{\pgfqpoint{-0.000000in}{0.000000in}}%
\pgfpathlineto{\pgfqpoint{-0.027778in}{0.000000in}}%
\pgfusepath{stroke,fill}%
}%
\begin{pgfscope}%
\pgfsys@transformshift{0.708220in}{1.266532in}%
\pgfsys@useobject{currentmarker}{}%
\end{pgfscope}%
\end{pgfscope}%
\begin{pgfscope}%
\pgfsetbuttcap%
\pgfsetroundjoin%
\definecolor{currentfill}{rgb}{0.000000,0.000000,0.000000}%
\pgfsetfillcolor{currentfill}%
\pgfsetlinewidth{0.602250pt}%
\definecolor{currentstroke}{rgb}{0.000000,0.000000,0.000000}%
\pgfsetstrokecolor{currentstroke}%
\pgfsetdash{}{0pt}%
\pgfsys@defobject{currentmarker}{\pgfqpoint{-0.027778in}{0.000000in}}{\pgfqpoint{-0.000000in}{0.000000in}}{%
\pgfpathmoveto{\pgfqpoint{-0.000000in}{0.000000in}}%
\pgfpathlineto{\pgfqpoint{-0.027778in}{0.000000in}}%
\pgfusepath{stroke,fill}%
}%
\begin{pgfscope}%
\pgfsys@transformshift{0.708220in}{1.315982in}%
\pgfsys@useobject{currentmarker}{}%
\end{pgfscope}%
\end{pgfscope}%
\begin{pgfscope}%
\pgfsetbuttcap%
\pgfsetroundjoin%
\definecolor{currentfill}{rgb}{0.000000,0.000000,0.000000}%
\pgfsetfillcolor{currentfill}%
\pgfsetlinewidth{0.602250pt}%
\definecolor{currentstroke}{rgb}{0.000000,0.000000,0.000000}%
\pgfsetstrokecolor{currentstroke}%
\pgfsetdash{}{0pt}%
\pgfsys@defobject{currentmarker}{\pgfqpoint{-0.027778in}{0.000000in}}{\pgfqpoint{-0.000000in}{0.000000in}}{%
\pgfpathmoveto{\pgfqpoint{-0.000000in}{0.000000in}}%
\pgfpathlineto{\pgfqpoint{-0.027778in}{0.000000in}}%
\pgfusepath{stroke,fill}%
}%
\begin{pgfscope}%
\pgfsys@transformshift{0.708220in}{1.351067in}%
\pgfsys@useobject{currentmarker}{}%
\end{pgfscope}%
\end{pgfscope}%
\begin{pgfscope}%
\pgfsetbuttcap%
\pgfsetroundjoin%
\definecolor{currentfill}{rgb}{0.000000,0.000000,0.000000}%
\pgfsetfillcolor{currentfill}%
\pgfsetlinewidth{0.602250pt}%
\definecolor{currentstroke}{rgb}{0.000000,0.000000,0.000000}%
\pgfsetstrokecolor{currentstroke}%
\pgfsetdash{}{0pt}%
\pgfsys@defobject{currentmarker}{\pgfqpoint{-0.027778in}{0.000000in}}{\pgfqpoint{-0.000000in}{0.000000in}}{%
\pgfpathmoveto{\pgfqpoint{-0.000000in}{0.000000in}}%
\pgfpathlineto{\pgfqpoint{-0.027778in}{0.000000in}}%
\pgfusepath{stroke,fill}%
}%
\begin{pgfscope}%
\pgfsys@transformshift{0.708220in}{1.378281in}%
\pgfsys@useobject{currentmarker}{}%
\end{pgfscope}%
\end{pgfscope}%
\begin{pgfscope}%
\pgfsetbuttcap%
\pgfsetroundjoin%
\definecolor{currentfill}{rgb}{0.000000,0.000000,0.000000}%
\pgfsetfillcolor{currentfill}%
\pgfsetlinewidth{0.602250pt}%
\definecolor{currentstroke}{rgb}{0.000000,0.000000,0.000000}%
\pgfsetstrokecolor{currentstroke}%
\pgfsetdash{}{0pt}%
\pgfsys@defobject{currentmarker}{\pgfqpoint{-0.027778in}{0.000000in}}{\pgfqpoint{-0.000000in}{0.000000in}}{%
\pgfpathmoveto{\pgfqpoint{-0.000000in}{0.000000in}}%
\pgfpathlineto{\pgfqpoint{-0.027778in}{0.000000in}}%
\pgfusepath{stroke,fill}%
}%
\begin{pgfscope}%
\pgfsys@transformshift{0.708220in}{1.400517in}%
\pgfsys@useobject{currentmarker}{}%
\end{pgfscope}%
\end{pgfscope}%
\begin{pgfscope}%
\pgfsetbuttcap%
\pgfsetroundjoin%
\definecolor{currentfill}{rgb}{0.000000,0.000000,0.000000}%
\pgfsetfillcolor{currentfill}%
\pgfsetlinewidth{0.602250pt}%
\definecolor{currentstroke}{rgb}{0.000000,0.000000,0.000000}%
\pgfsetstrokecolor{currentstroke}%
\pgfsetdash{}{0pt}%
\pgfsys@defobject{currentmarker}{\pgfqpoint{-0.027778in}{0.000000in}}{\pgfqpoint{-0.000000in}{0.000000in}}{%
\pgfpathmoveto{\pgfqpoint{-0.000000in}{0.000000in}}%
\pgfpathlineto{\pgfqpoint{-0.027778in}{0.000000in}}%
\pgfusepath{stroke,fill}%
}%
\begin{pgfscope}%
\pgfsys@transformshift{0.708220in}{1.419317in}%
\pgfsys@useobject{currentmarker}{}%
\end{pgfscope}%
\end{pgfscope}%
\begin{pgfscope}%
\pgfsetbuttcap%
\pgfsetroundjoin%
\definecolor{currentfill}{rgb}{0.000000,0.000000,0.000000}%
\pgfsetfillcolor{currentfill}%
\pgfsetlinewidth{0.602250pt}%
\definecolor{currentstroke}{rgb}{0.000000,0.000000,0.000000}%
\pgfsetstrokecolor{currentstroke}%
\pgfsetdash{}{0pt}%
\pgfsys@defobject{currentmarker}{\pgfqpoint{-0.027778in}{0.000000in}}{\pgfqpoint{-0.000000in}{0.000000in}}{%
\pgfpathmoveto{\pgfqpoint{-0.000000in}{0.000000in}}%
\pgfpathlineto{\pgfqpoint{-0.027778in}{0.000000in}}%
\pgfusepath{stroke,fill}%
}%
\begin{pgfscope}%
\pgfsys@transformshift{0.708220in}{1.435602in}%
\pgfsys@useobject{currentmarker}{}%
\end{pgfscope}%
\end{pgfscope}%
\begin{pgfscope}%
\pgfsetbuttcap%
\pgfsetroundjoin%
\definecolor{currentfill}{rgb}{0.000000,0.000000,0.000000}%
\pgfsetfillcolor{currentfill}%
\pgfsetlinewidth{0.602250pt}%
\definecolor{currentstroke}{rgb}{0.000000,0.000000,0.000000}%
\pgfsetstrokecolor{currentstroke}%
\pgfsetdash{}{0pt}%
\pgfsys@defobject{currentmarker}{\pgfqpoint{-0.027778in}{0.000000in}}{\pgfqpoint{-0.000000in}{0.000000in}}{%
\pgfpathmoveto{\pgfqpoint{-0.000000in}{0.000000in}}%
\pgfpathlineto{\pgfqpoint{-0.027778in}{0.000000in}}%
\pgfusepath{stroke,fill}%
}%
\begin{pgfscope}%
\pgfsys@transformshift{0.708220in}{1.449967in}%
\pgfsys@useobject{currentmarker}{}%
\end{pgfscope}%
\end{pgfscope}%
\begin{pgfscope}%
\pgfsetbuttcap%
\pgfsetroundjoin%
\definecolor{currentfill}{rgb}{0.000000,0.000000,0.000000}%
\pgfsetfillcolor{currentfill}%
\pgfsetlinewidth{0.602250pt}%
\definecolor{currentstroke}{rgb}{0.000000,0.000000,0.000000}%
\pgfsetstrokecolor{currentstroke}%
\pgfsetdash{}{0pt}%
\pgfsys@defobject{currentmarker}{\pgfqpoint{-0.027778in}{0.000000in}}{\pgfqpoint{-0.000000in}{0.000000in}}{%
\pgfpathmoveto{\pgfqpoint{-0.000000in}{0.000000in}}%
\pgfpathlineto{\pgfqpoint{-0.027778in}{0.000000in}}%
\pgfusepath{stroke,fill}%
}%
\begin{pgfscope}%
\pgfsys@transformshift{0.708220in}{1.547352in}%
\pgfsys@useobject{currentmarker}{}%
\end{pgfscope}%
\end{pgfscope}%
\begin{pgfscope}%
\pgfsetbuttcap%
\pgfsetroundjoin%
\definecolor{currentfill}{rgb}{0.000000,0.000000,0.000000}%
\pgfsetfillcolor{currentfill}%
\pgfsetlinewidth{0.602250pt}%
\definecolor{currentstroke}{rgb}{0.000000,0.000000,0.000000}%
\pgfsetstrokecolor{currentstroke}%
\pgfsetdash{}{0pt}%
\pgfsys@defobject{currentmarker}{\pgfqpoint{-0.027778in}{0.000000in}}{\pgfqpoint{-0.000000in}{0.000000in}}{%
\pgfpathmoveto{\pgfqpoint{-0.000000in}{0.000000in}}%
\pgfpathlineto{\pgfqpoint{-0.027778in}{0.000000in}}%
\pgfusepath{stroke,fill}%
}%
\begin{pgfscope}%
\pgfsys@transformshift{0.708220in}{1.596801in}%
\pgfsys@useobject{currentmarker}{}%
\end{pgfscope}%
\end{pgfscope}%
\begin{pgfscope}%
\pgfsetbuttcap%
\pgfsetroundjoin%
\definecolor{currentfill}{rgb}{0.000000,0.000000,0.000000}%
\pgfsetfillcolor{currentfill}%
\pgfsetlinewidth{0.602250pt}%
\definecolor{currentstroke}{rgb}{0.000000,0.000000,0.000000}%
\pgfsetstrokecolor{currentstroke}%
\pgfsetdash{}{0pt}%
\pgfsys@defobject{currentmarker}{\pgfqpoint{-0.027778in}{0.000000in}}{\pgfqpoint{-0.000000in}{0.000000in}}{%
\pgfpathmoveto{\pgfqpoint{-0.000000in}{0.000000in}}%
\pgfpathlineto{\pgfqpoint{-0.027778in}{0.000000in}}%
\pgfusepath{stroke,fill}%
}%
\begin{pgfscope}%
\pgfsys@transformshift{0.708220in}{1.631887in}%
\pgfsys@useobject{currentmarker}{}%
\end{pgfscope}%
\end{pgfscope}%
\begin{pgfscope}%
\pgfsetbuttcap%
\pgfsetroundjoin%
\definecolor{currentfill}{rgb}{0.000000,0.000000,0.000000}%
\pgfsetfillcolor{currentfill}%
\pgfsetlinewidth{0.602250pt}%
\definecolor{currentstroke}{rgb}{0.000000,0.000000,0.000000}%
\pgfsetstrokecolor{currentstroke}%
\pgfsetdash{}{0pt}%
\pgfsys@defobject{currentmarker}{\pgfqpoint{-0.027778in}{0.000000in}}{\pgfqpoint{-0.000000in}{0.000000in}}{%
\pgfpathmoveto{\pgfqpoint{-0.000000in}{0.000000in}}%
\pgfpathlineto{\pgfqpoint{-0.027778in}{0.000000in}}%
\pgfusepath{stroke,fill}%
}%
\begin{pgfscope}%
\pgfsys@transformshift{0.708220in}{1.659101in}%
\pgfsys@useobject{currentmarker}{}%
\end{pgfscope}%
\end{pgfscope}%
\begin{pgfscope}%
\pgfsetbuttcap%
\pgfsetroundjoin%
\definecolor{currentfill}{rgb}{0.000000,0.000000,0.000000}%
\pgfsetfillcolor{currentfill}%
\pgfsetlinewidth{0.602250pt}%
\definecolor{currentstroke}{rgb}{0.000000,0.000000,0.000000}%
\pgfsetstrokecolor{currentstroke}%
\pgfsetdash{}{0pt}%
\pgfsys@defobject{currentmarker}{\pgfqpoint{-0.027778in}{0.000000in}}{\pgfqpoint{-0.000000in}{0.000000in}}{%
\pgfpathmoveto{\pgfqpoint{-0.000000in}{0.000000in}}%
\pgfpathlineto{\pgfqpoint{-0.027778in}{0.000000in}}%
\pgfusepath{stroke,fill}%
}%
\begin{pgfscope}%
\pgfsys@transformshift{0.708220in}{1.681337in}%
\pgfsys@useobject{currentmarker}{}%
\end{pgfscope}%
\end{pgfscope}%
\begin{pgfscope}%
\pgfsetbuttcap%
\pgfsetroundjoin%
\definecolor{currentfill}{rgb}{0.000000,0.000000,0.000000}%
\pgfsetfillcolor{currentfill}%
\pgfsetlinewidth{0.602250pt}%
\definecolor{currentstroke}{rgb}{0.000000,0.000000,0.000000}%
\pgfsetstrokecolor{currentstroke}%
\pgfsetdash{}{0pt}%
\pgfsys@defobject{currentmarker}{\pgfqpoint{-0.027778in}{0.000000in}}{\pgfqpoint{-0.000000in}{0.000000in}}{%
\pgfpathmoveto{\pgfqpoint{-0.000000in}{0.000000in}}%
\pgfpathlineto{\pgfqpoint{-0.027778in}{0.000000in}}%
\pgfusepath{stroke,fill}%
}%
\begin{pgfscope}%
\pgfsys@transformshift{0.708220in}{1.700137in}%
\pgfsys@useobject{currentmarker}{}%
\end{pgfscope}%
\end{pgfscope}%
\begin{pgfscope}%
\pgfsetbuttcap%
\pgfsetroundjoin%
\definecolor{currentfill}{rgb}{0.000000,0.000000,0.000000}%
\pgfsetfillcolor{currentfill}%
\pgfsetlinewidth{0.602250pt}%
\definecolor{currentstroke}{rgb}{0.000000,0.000000,0.000000}%
\pgfsetstrokecolor{currentstroke}%
\pgfsetdash{}{0pt}%
\pgfsys@defobject{currentmarker}{\pgfqpoint{-0.027778in}{0.000000in}}{\pgfqpoint{-0.000000in}{0.000000in}}{%
\pgfpathmoveto{\pgfqpoint{-0.000000in}{0.000000in}}%
\pgfpathlineto{\pgfqpoint{-0.027778in}{0.000000in}}%
\pgfusepath{stroke,fill}%
}%
\begin{pgfscope}%
\pgfsys@transformshift{0.708220in}{1.716422in}%
\pgfsys@useobject{currentmarker}{}%
\end{pgfscope}%
\end{pgfscope}%
\begin{pgfscope}%
\pgfsetbuttcap%
\pgfsetroundjoin%
\definecolor{currentfill}{rgb}{0.000000,0.000000,0.000000}%
\pgfsetfillcolor{currentfill}%
\pgfsetlinewidth{0.602250pt}%
\definecolor{currentstroke}{rgb}{0.000000,0.000000,0.000000}%
\pgfsetstrokecolor{currentstroke}%
\pgfsetdash{}{0pt}%
\pgfsys@defobject{currentmarker}{\pgfqpoint{-0.027778in}{0.000000in}}{\pgfqpoint{-0.000000in}{0.000000in}}{%
\pgfpathmoveto{\pgfqpoint{-0.000000in}{0.000000in}}%
\pgfpathlineto{\pgfqpoint{-0.027778in}{0.000000in}}%
\pgfusepath{stroke,fill}%
}%
\begin{pgfscope}%
\pgfsys@transformshift{0.708220in}{1.730786in}%
\pgfsys@useobject{currentmarker}{}%
\end{pgfscope}%
\end{pgfscope}%
\begin{pgfscope}%
\pgfsetbuttcap%
\pgfsetroundjoin%
\definecolor{currentfill}{rgb}{0.000000,0.000000,0.000000}%
\pgfsetfillcolor{currentfill}%
\pgfsetlinewidth{0.602250pt}%
\definecolor{currentstroke}{rgb}{0.000000,0.000000,0.000000}%
\pgfsetstrokecolor{currentstroke}%
\pgfsetdash{}{0pt}%
\pgfsys@defobject{currentmarker}{\pgfqpoint{-0.027778in}{0.000000in}}{\pgfqpoint{-0.000000in}{0.000000in}}{%
\pgfpathmoveto{\pgfqpoint{-0.000000in}{0.000000in}}%
\pgfpathlineto{\pgfqpoint{-0.027778in}{0.000000in}}%
\pgfusepath{stroke,fill}%
}%
\begin{pgfscope}%
\pgfsys@transformshift{0.708220in}{1.828171in}%
\pgfsys@useobject{currentmarker}{}%
\end{pgfscope}%
\end{pgfscope}%
\begin{pgfscope}%
\pgfsetbuttcap%
\pgfsetroundjoin%
\definecolor{currentfill}{rgb}{0.000000,0.000000,0.000000}%
\pgfsetfillcolor{currentfill}%
\pgfsetlinewidth{0.602250pt}%
\definecolor{currentstroke}{rgb}{0.000000,0.000000,0.000000}%
\pgfsetstrokecolor{currentstroke}%
\pgfsetdash{}{0pt}%
\pgfsys@defobject{currentmarker}{\pgfqpoint{-0.027778in}{0.000000in}}{\pgfqpoint{-0.000000in}{0.000000in}}{%
\pgfpathmoveto{\pgfqpoint{-0.000000in}{0.000000in}}%
\pgfpathlineto{\pgfqpoint{-0.027778in}{0.000000in}}%
\pgfusepath{stroke,fill}%
}%
\begin{pgfscope}%
\pgfsys@transformshift{0.708220in}{1.877621in}%
\pgfsys@useobject{currentmarker}{}%
\end{pgfscope}%
\end{pgfscope}%
\begin{pgfscope}%
\pgfsetbuttcap%
\pgfsetroundjoin%
\definecolor{currentfill}{rgb}{0.000000,0.000000,0.000000}%
\pgfsetfillcolor{currentfill}%
\pgfsetlinewidth{0.602250pt}%
\definecolor{currentstroke}{rgb}{0.000000,0.000000,0.000000}%
\pgfsetstrokecolor{currentstroke}%
\pgfsetdash{}{0pt}%
\pgfsys@defobject{currentmarker}{\pgfqpoint{-0.027778in}{0.000000in}}{\pgfqpoint{-0.000000in}{0.000000in}}{%
\pgfpathmoveto{\pgfqpoint{-0.000000in}{0.000000in}}%
\pgfpathlineto{\pgfqpoint{-0.027778in}{0.000000in}}%
\pgfusepath{stroke,fill}%
}%
\begin{pgfscope}%
\pgfsys@transformshift{0.708220in}{1.912706in}%
\pgfsys@useobject{currentmarker}{}%
\end{pgfscope}%
\end{pgfscope}%
\begin{pgfscope}%
\pgfsetbuttcap%
\pgfsetroundjoin%
\definecolor{currentfill}{rgb}{0.000000,0.000000,0.000000}%
\pgfsetfillcolor{currentfill}%
\pgfsetlinewidth{0.602250pt}%
\definecolor{currentstroke}{rgb}{0.000000,0.000000,0.000000}%
\pgfsetstrokecolor{currentstroke}%
\pgfsetdash{}{0pt}%
\pgfsys@defobject{currentmarker}{\pgfqpoint{-0.027778in}{0.000000in}}{\pgfqpoint{-0.000000in}{0.000000in}}{%
\pgfpathmoveto{\pgfqpoint{-0.000000in}{0.000000in}}%
\pgfpathlineto{\pgfqpoint{-0.027778in}{0.000000in}}%
\pgfusepath{stroke,fill}%
}%
\begin{pgfscope}%
\pgfsys@transformshift{0.708220in}{1.939921in}%
\pgfsys@useobject{currentmarker}{}%
\end{pgfscope}%
\end{pgfscope}%
\begin{pgfscope}%
\pgfsetbuttcap%
\pgfsetroundjoin%
\definecolor{currentfill}{rgb}{0.000000,0.000000,0.000000}%
\pgfsetfillcolor{currentfill}%
\pgfsetlinewidth{0.602250pt}%
\definecolor{currentstroke}{rgb}{0.000000,0.000000,0.000000}%
\pgfsetstrokecolor{currentstroke}%
\pgfsetdash{}{0pt}%
\pgfsys@defobject{currentmarker}{\pgfqpoint{-0.027778in}{0.000000in}}{\pgfqpoint{-0.000000in}{0.000000in}}{%
\pgfpathmoveto{\pgfqpoint{-0.000000in}{0.000000in}}%
\pgfpathlineto{\pgfqpoint{-0.027778in}{0.000000in}}%
\pgfusepath{stroke,fill}%
}%
\begin{pgfscope}%
\pgfsys@transformshift{0.708220in}{1.962156in}%
\pgfsys@useobject{currentmarker}{}%
\end{pgfscope}%
\end{pgfscope}%
\begin{pgfscope}%
\pgfsetbuttcap%
\pgfsetroundjoin%
\definecolor{currentfill}{rgb}{0.000000,0.000000,0.000000}%
\pgfsetfillcolor{currentfill}%
\pgfsetlinewidth{0.602250pt}%
\definecolor{currentstroke}{rgb}{0.000000,0.000000,0.000000}%
\pgfsetstrokecolor{currentstroke}%
\pgfsetdash{}{0pt}%
\pgfsys@defobject{currentmarker}{\pgfqpoint{-0.027778in}{0.000000in}}{\pgfqpoint{-0.000000in}{0.000000in}}{%
\pgfpathmoveto{\pgfqpoint{-0.000000in}{0.000000in}}%
\pgfpathlineto{\pgfqpoint{-0.027778in}{0.000000in}}%
\pgfusepath{stroke,fill}%
}%
\begin{pgfscope}%
\pgfsys@transformshift{0.708220in}{1.980956in}%
\pgfsys@useobject{currentmarker}{}%
\end{pgfscope}%
\end{pgfscope}%
\begin{pgfscope}%
\pgfsetbuttcap%
\pgfsetroundjoin%
\definecolor{currentfill}{rgb}{0.000000,0.000000,0.000000}%
\pgfsetfillcolor{currentfill}%
\pgfsetlinewidth{0.602250pt}%
\definecolor{currentstroke}{rgb}{0.000000,0.000000,0.000000}%
\pgfsetstrokecolor{currentstroke}%
\pgfsetdash{}{0pt}%
\pgfsys@defobject{currentmarker}{\pgfqpoint{-0.027778in}{0.000000in}}{\pgfqpoint{-0.000000in}{0.000000in}}{%
\pgfpathmoveto{\pgfqpoint{-0.000000in}{0.000000in}}%
\pgfpathlineto{\pgfqpoint{-0.027778in}{0.000000in}}%
\pgfusepath{stroke,fill}%
}%
\begin{pgfscope}%
\pgfsys@transformshift{0.708220in}{1.997241in}%
\pgfsys@useobject{currentmarker}{}%
\end{pgfscope}%
\end{pgfscope}%
\begin{pgfscope}%
\pgfsetbuttcap%
\pgfsetroundjoin%
\definecolor{currentfill}{rgb}{0.000000,0.000000,0.000000}%
\pgfsetfillcolor{currentfill}%
\pgfsetlinewidth{0.602250pt}%
\definecolor{currentstroke}{rgb}{0.000000,0.000000,0.000000}%
\pgfsetstrokecolor{currentstroke}%
\pgfsetdash{}{0pt}%
\pgfsys@defobject{currentmarker}{\pgfqpoint{-0.027778in}{0.000000in}}{\pgfqpoint{-0.000000in}{0.000000in}}{%
\pgfpathmoveto{\pgfqpoint{-0.000000in}{0.000000in}}%
\pgfpathlineto{\pgfqpoint{-0.027778in}{0.000000in}}%
\pgfusepath{stroke,fill}%
}%
\begin{pgfscope}%
\pgfsys@transformshift{0.708220in}{2.011606in}%
\pgfsys@useobject{currentmarker}{}%
\end{pgfscope}%
\end{pgfscope}%
\begin{pgfscope}%
\pgfsetbuttcap%
\pgfsetroundjoin%
\definecolor{currentfill}{rgb}{0.000000,0.000000,0.000000}%
\pgfsetfillcolor{currentfill}%
\pgfsetlinewidth{0.602250pt}%
\definecolor{currentstroke}{rgb}{0.000000,0.000000,0.000000}%
\pgfsetstrokecolor{currentstroke}%
\pgfsetdash{}{0pt}%
\pgfsys@defobject{currentmarker}{\pgfqpoint{-0.027778in}{0.000000in}}{\pgfqpoint{-0.000000in}{0.000000in}}{%
\pgfpathmoveto{\pgfqpoint{-0.000000in}{0.000000in}}%
\pgfpathlineto{\pgfqpoint{-0.027778in}{0.000000in}}%
\pgfusepath{stroke,fill}%
}%
\begin{pgfscope}%
\pgfsys@transformshift{0.708220in}{2.108991in}%
\pgfsys@useobject{currentmarker}{}%
\end{pgfscope}%
\end{pgfscope}%
\begin{pgfscope}%
\pgfsetbuttcap%
\pgfsetroundjoin%
\definecolor{currentfill}{rgb}{0.000000,0.000000,0.000000}%
\pgfsetfillcolor{currentfill}%
\pgfsetlinewidth{0.602250pt}%
\definecolor{currentstroke}{rgb}{0.000000,0.000000,0.000000}%
\pgfsetstrokecolor{currentstroke}%
\pgfsetdash{}{0pt}%
\pgfsys@defobject{currentmarker}{\pgfqpoint{-0.027778in}{0.000000in}}{\pgfqpoint{-0.000000in}{0.000000in}}{%
\pgfpathmoveto{\pgfqpoint{-0.000000in}{0.000000in}}%
\pgfpathlineto{\pgfqpoint{-0.027778in}{0.000000in}}%
\pgfusepath{stroke,fill}%
}%
\begin{pgfscope}%
\pgfsys@transformshift{0.708220in}{2.158441in}%
\pgfsys@useobject{currentmarker}{}%
\end{pgfscope}%
\end{pgfscope}%
\begin{pgfscope}%
\pgfsetbuttcap%
\pgfsetroundjoin%
\definecolor{currentfill}{rgb}{0.000000,0.000000,0.000000}%
\pgfsetfillcolor{currentfill}%
\pgfsetlinewidth{0.602250pt}%
\definecolor{currentstroke}{rgb}{0.000000,0.000000,0.000000}%
\pgfsetstrokecolor{currentstroke}%
\pgfsetdash{}{0pt}%
\pgfsys@defobject{currentmarker}{\pgfqpoint{-0.027778in}{0.000000in}}{\pgfqpoint{-0.000000in}{0.000000in}}{%
\pgfpathmoveto{\pgfqpoint{-0.000000in}{0.000000in}}%
\pgfpathlineto{\pgfqpoint{-0.027778in}{0.000000in}}%
\pgfusepath{stroke,fill}%
}%
\begin{pgfscope}%
\pgfsys@transformshift{0.708220in}{2.193526in}%
\pgfsys@useobject{currentmarker}{}%
\end{pgfscope}%
\end{pgfscope}%
\begin{pgfscope}%
\pgfsetbuttcap%
\pgfsetroundjoin%
\definecolor{currentfill}{rgb}{0.000000,0.000000,0.000000}%
\pgfsetfillcolor{currentfill}%
\pgfsetlinewidth{0.602250pt}%
\definecolor{currentstroke}{rgb}{0.000000,0.000000,0.000000}%
\pgfsetstrokecolor{currentstroke}%
\pgfsetdash{}{0pt}%
\pgfsys@defobject{currentmarker}{\pgfqpoint{-0.027778in}{0.000000in}}{\pgfqpoint{-0.000000in}{0.000000in}}{%
\pgfpathmoveto{\pgfqpoint{-0.000000in}{0.000000in}}%
\pgfpathlineto{\pgfqpoint{-0.027778in}{0.000000in}}%
\pgfusepath{stroke,fill}%
}%
\begin{pgfscope}%
\pgfsys@transformshift{0.708220in}{2.220740in}%
\pgfsys@useobject{currentmarker}{}%
\end{pgfscope}%
\end{pgfscope}%
\begin{pgfscope}%
\pgfsetbuttcap%
\pgfsetroundjoin%
\definecolor{currentfill}{rgb}{0.000000,0.000000,0.000000}%
\pgfsetfillcolor{currentfill}%
\pgfsetlinewidth{0.602250pt}%
\definecolor{currentstroke}{rgb}{0.000000,0.000000,0.000000}%
\pgfsetstrokecolor{currentstroke}%
\pgfsetdash{}{0pt}%
\pgfsys@defobject{currentmarker}{\pgfqpoint{-0.027778in}{0.000000in}}{\pgfqpoint{-0.000000in}{0.000000in}}{%
\pgfpathmoveto{\pgfqpoint{-0.000000in}{0.000000in}}%
\pgfpathlineto{\pgfqpoint{-0.027778in}{0.000000in}}%
\pgfusepath{stroke,fill}%
}%
\begin{pgfscope}%
\pgfsys@transformshift{0.708220in}{2.242976in}%
\pgfsys@useobject{currentmarker}{}%
\end{pgfscope}%
\end{pgfscope}%
\begin{pgfscope}%
\pgfsetbuttcap%
\pgfsetroundjoin%
\definecolor{currentfill}{rgb}{0.000000,0.000000,0.000000}%
\pgfsetfillcolor{currentfill}%
\pgfsetlinewidth{0.602250pt}%
\definecolor{currentstroke}{rgb}{0.000000,0.000000,0.000000}%
\pgfsetstrokecolor{currentstroke}%
\pgfsetdash{}{0pt}%
\pgfsys@defobject{currentmarker}{\pgfqpoint{-0.027778in}{0.000000in}}{\pgfqpoint{-0.000000in}{0.000000in}}{%
\pgfpathmoveto{\pgfqpoint{-0.000000in}{0.000000in}}%
\pgfpathlineto{\pgfqpoint{-0.027778in}{0.000000in}}%
\pgfusepath{stroke,fill}%
}%
\begin{pgfscope}%
\pgfsys@transformshift{0.708220in}{2.261776in}%
\pgfsys@useobject{currentmarker}{}%
\end{pgfscope}%
\end{pgfscope}%
\begin{pgfscope}%
\pgfsetbuttcap%
\pgfsetroundjoin%
\definecolor{currentfill}{rgb}{0.000000,0.000000,0.000000}%
\pgfsetfillcolor{currentfill}%
\pgfsetlinewidth{0.602250pt}%
\definecolor{currentstroke}{rgb}{0.000000,0.000000,0.000000}%
\pgfsetstrokecolor{currentstroke}%
\pgfsetdash{}{0pt}%
\pgfsys@defobject{currentmarker}{\pgfqpoint{-0.027778in}{0.000000in}}{\pgfqpoint{-0.000000in}{0.000000in}}{%
\pgfpathmoveto{\pgfqpoint{-0.000000in}{0.000000in}}%
\pgfpathlineto{\pgfqpoint{-0.027778in}{0.000000in}}%
\pgfusepath{stroke,fill}%
}%
\begin{pgfscope}%
\pgfsys@transformshift{0.708220in}{2.278061in}%
\pgfsys@useobject{currentmarker}{}%
\end{pgfscope}%
\end{pgfscope}%
\begin{pgfscope}%
\pgfsetbuttcap%
\pgfsetroundjoin%
\definecolor{currentfill}{rgb}{0.000000,0.000000,0.000000}%
\pgfsetfillcolor{currentfill}%
\pgfsetlinewidth{0.602250pt}%
\definecolor{currentstroke}{rgb}{0.000000,0.000000,0.000000}%
\pgfsetstrokecolor{currentstroke}%
\pgfsetdash{}{0pt}%
\pgfsys@defobject{currentmarker}{\pgfqpoint{-0.027778in}{0.000000in}}{\pgfqpoint{-0.000000in}{0.000000in}}{%
\pgfpathmoveto{\pgfqpoint{-0.000000in}{0.000000in}}%
\pgfpathlineto{\pgfqpoint{-0.027778in}{0.000000in}}%
\pgfusepath{stroke,fill}%
}%
\begin{pgfscope}%
\pgfsys@transformshift{0.708220in}{2.292426in}%
\pgfsys@useobject{currentmarker}{}%
\end{pgfscope}%
\end{pgfscope}%
\begin{pgfscope}%
\definecolor{textcolor}{rgb}{0.000000,0.000000,0.000000}%
\pgfsetstrokecolor{textcolor}%
\pgfsetfillcolor{textcolor}%
\pgftext[x=0.288855in,y=1.420549in,,bottom,rotate=90.000000]{\color{textcolor}\rmfamily\fontsize{10.000000}{12.000000}\selectfont Longest solving time (s)}%
\end{pgfscope}%
\begin{pgfscope}%
\pgfpathrectangle{\pgfqpoint{0.708220in}{0.535823in}}{\pgfqpoint{5.045427in}{1.769453in}}%
\pgfusepath{clip}%
\pgfsetrectcap%
\pgfsetroundjoin%
\pgfsetlinewidth{1.003750pt}%
\definecolor{currentstroke}{rgb}{0.121569,0.466667,0.705882}%
\pgfsetstrokecolor{currentstroke}%
\pgfsetdash{}{0pt}%
\pgfpathmoveto{\pgfqpoint{0.708220in}{1.081871in}}%
\pgfpathlineto{\pgfqpoint{0.720833in}{1.119697in}}%
\pgfpathlineto{\pgfqpoint{0.733447in}{1.177018in}}%
\pgfpathlineto{\pgfqpoint{0.771288in}{1.180771in}}%
\pgfpathlineto{\pgfqpoint{0.796515in}{1.180771in}}%
\pgfpathlineto{\pgfqpoint{0.809128in}{1.181997in}}%
\pgfpathlineto{\pgfqpoint{0.846969in}{1.181997in}}%
\pgfpathlineto{\pgfqpoint{0.859583in}{1.183210in}}%
\pgfpathlineto{\pgfqpoint{0.897423in}{1.183210in}}%
\pgfpathlineto{\pgfqpoint{0.910037in}{1.184412in}}%
\pgfpathlineto{\pgfqpoint{0.947878in}{1.184412in}}%
\pgfpathlineto{\pgfqpoint{0.960491in}{1.185602in}}%
\pgfpathlineto{\pgfqpoint{0.985718in}{1.185602in}}%
\pgfpathlineto{\pgfqpoint{0.998332in}{1.186780in}}%
\pgfpathlineto{\pgfqpoint{1.010945in}{1.186780in}}%
\pgfpathlineto{\pgfqpoint{1.023559in}{1.187947in}}%
\pgfpathlineto{\pgfqpoint{1.036173in}{1.187947in}}%
\pgfpathlineto{\pgfqpoint{1.048786in}{1.189103in}}%
\pgfpathlineto{\pgfqpoint{1.074013in}{1.189103in}}%
\pgfpathlineto{\pgfqpoint{1.099240in}{1.191383in}}%
\pgfpathlineto{\pgfqpoint{1.124468in}{1.191383in}}%
\pgfpathlineto{\pgfqpoint{1.137081in}{1.192507in}}%
\pgfpathlineto{\pgfqpoint{1.162308in}{1.192507in}}%
\pgfpathlineto{\pgfqpoint{1.187535in}{1.193621in}}%
\pgfpathlineto{\pgfqpoint{1.263217in}{1.194724in}}%
\pgfpathlineto{\pgfqpoint{1.301057in}{1.196902in}}%
\pgfpathlineto{\pgfqpoint{1.439807in}{1.197977in}}%
\pgfpathlineto{\pgfqpoint{1.465034in}{1.199042in}}%
\pgfpathlineto{\pgfqpoint{1.528102in}{1.200098in}}%
\pgfpathlineto{\pgfqpoint{1.553329in}{1.201145in}}%
\pgfpathlineto{\pgfqpoint{1.578556in}{1.201145in}}%
\pgfpathlineto{\pgfqpoint{1.591170in}{1.203212in}}%
\pgfpathlineto{\pgfqpoint{1.629010in}{1.204233in}}%
\pgfpathlineto{\pgfqpoint{1.654237in}{1.205245in}}%
\pgfpathlineto{\pgfqpoint{1.729919in}{1.206248in}}%
\pgfpathlineto{\pgfqpoint{1.755146in}{1.207244in}}%
\pgfpathlineto{\pgfqpoint{1.805600in}{1.208232in}}%
\pgfpathlineto{\pgfqpoint{1.843441in}{1.210183in}}%
\pgfpathlineto{\pgfqpoint{1.906509in}{1.210183in}}%
\pgfpathlineto{\pgfqpoint{1.919122in}{1.212104in}}%
\pgfpathlineto{\pgfqpoint{1.944349in}{1.213053in}}%
\pgfpathlineto{\pgfqpoint{1.969577in}{1.213994in}}%
\pgfpathlineto{\pgfqpoint{2.045258in}{1.214929in}}%
\pgfpathlineto{\pgfqpoint{2.070485in}{1.215856in}}%
\pgfpathlineto{\pgfqpoint{2.158780in}{1.216777in}}%
\pgfpathlineto{\pgfqpoint{2.184007in}{1.217690in}}%
\pgfpathlineto{\pgfqpoint{2.234461in}{1.218597in}}%
\pgfpathlineto{\pgfqpoint{2.272302in}{1.220391in}}%
\pgfpathlineto{\pgfqpoint{2.297529in}{1.221278in}}%
\pgfpathlineto{\pgfqpoint{2.310143in}{1.224762in}}%
\pgfpathlineto{\pgfqpoint{2.322756in}{1.225618in}}%
\pgfpathlineto{\pgfqpoint{2.335370in}{1.228150in}}%
\pgfpathlineto{\pgfqpoint{2.360597in}{1.228983in}}%
\pgfpathlineto{\pgfqpoint{2.373211in}{1.231447in}}%
\pgfpathlineto{\pgfqpoint{2.385824in}{1.231447in}}%
\pgfpathlineto{\pgfqpoint{2.398438in}{1.233062in}}%
\pgfpathlineto{\pgfqpoint{2.461506in}{1.233862in}}%
\pgfpathlineto{\pgfqpoint{2.549801in}{1.238553in}}%
\pgfpathlineto{\pgfqpoint{2.562414in}{1.238553in}}%
\pgfpathlineto{\pgfqpoint{2.612868in}{1.244540in}}%
\pgfpathlineto{\pgfqpoint{2.650709in}{1.245268in}}%
\pgfpathlineto{\pgfqpoint{2.663323in}{1.246711in}}%
\pgfpathlineto{\pgfqpoint{2.675936in}{1.249548in}}%
\pgfpathlineto{\pgfqpoint{2.688550in}{1.250942in}}%
\pgfpathlineto{\pgfqpoint{2.713777in}{1.251633in}}%
\pgfpathlineto{\pgfqpoint{2.751618in}{1.255030in}}%
\pgfpathlineto{\pgfqpoint{2.991275in}{1.268348in}}%
\pgfpathlineto{\pgfqpoint{3.016503in}{1.270137in}}%
\pgfpathlineto{\pgfqpoint{3.041730in}{1.270728in}}%
\pgfpathlineto{\pgfqpoint{3.054343in}{1.271900in}}%
\pgfpathlineto{\pgfqpoint{3.079570in}{1.275918in}}%
\pgfpathlineto{\pgfqpoint{3.142638in}{1.279260in}}%
\pgfpathlineto{\pgfqpoint{3.167865in}{1.283046in}}%
\pgfpathlineto{\pgfqpoint{3.193092in}{1.283577in}}%
\pgfpathlineto{\pgfqpoint{3.205706in}{1.289780in}}%
\pgfpathlineto{\pgfqpoint{3.218320in}{1.293257in}}%
\pgfpathlineto{\pgfqpoint{3.268774in}{1.298060in}}%
\pgfpathlineto{\pgfqpoint{3.319228in}{1.304926in}}%
\pgfpathlineto{\pgfqpoint{3.331842in}{1.305370in}}%
\pgfpathlineto{\pgfqpoint{3.357069in}{1.319587in}}%
\pgfpathlineto{\pgfqpoint{3.369682in}{1.319981in}}%
\pgfpathlineto{\pgfqpoint{3.394910in}{1.323088in}}%
\pgfpathlineto{\pgfqpoint{3.407523in}{1.326118in}}%
\pgfpathlineto{\pgfqpoint{3.420137in}{1.336512in}}%
\pgfpathlineto{\pgfqpoint{3.457977in}{1.340566in}}%
\pgfpathlineto{\pgfqpoint{3.470591in}{1.344168in}}%
\pgfpathlineto{\pgfqpoint{3.495818in}{1.346089in}}%
\pgfpathlineto{\pgfqpoint{3.508432in}{1.353781in}}%
\pgfpathlineto{\pgfqpoint{3.546272in}{1.355850in}}%
\pgfpathlineto{\pgfqpoint{3.571499in}{1.365702in}}%
\pgfpathlineto{\pgfqpoint{3.584113in}{1.375319in}}%
\pgfpathlineto{\pgfqpoint{3.596727in}{1.375568in}}%
\pgfpathlineto{\pgfqpoint{3.609340in}{1.393187in}}%
\pgfpathlineto{\pgfqpoint{3.621954in}{1.403330in}}%
\pgfpathlineto{\pgfqpoint{3.634567in}{1.410466in}}%
\pgfpathlineto{\pgfqpoint{3.647181in}{1.415422in}}%
\pgfpathlineto{\pgfqpoint{3.659794in}{1.430624in}}%
\pgfpathlineto{\pgfqpoint{3.672408in}{1.438465in}}%
\pgfpathlineto{\pgfqpoint{3.685022in}{1.452780in}}%
\pgfpathlineto{\pgfqpoint{3.697635in}{1.471976in}}%
\pgfpathlineto{\pgfqpoint{3.710249in}{1.475324in}}%
\pgfpathlineto{\pgfqpoint{3.722862in}{1.508552in}}%
\pgfpathlineto{\pgfqpoint{3.735476in}{1.533550in}}%
\pgfpathlineto{\pgfqpoint{3.748089in}{1.535110in}}%
\pgfpathlineto{\pgfqpoint{3.760703in}{1.546311in}}%
\pgfpathlineto{\pgfqpoint{3.773317in}{1.585433in}}%
\pgfpathlineto{\pgfqpoint{3.785930in}{1.632130in}}%
\pgfpathlineto{\pgfqpoint{3.798544in}{1.640820in}}%
\pgfpathlineto{\pgfqpoint{3.811157in}{1.664352in}}%
\pgfpathlineto{\pgfqpoint{3.823771in}{1.675124in}}%
\pgfpathlineto{\pgfqpoint{3.836384in}{1.689341in}}%
\pgfpathlineto{\pgfqpoint{3.848998in}{1.691679in}}%
\pgfpathlineto{\pgfqpoint{3.874225in}{1.708535in}}%
\pgfpathlineto{\pgfqpoint{3.886839in}{1.716772in}}%
\pgfpathlineto{\pgfqpoint{3.912066in}{1.786847in}}%
\pgfpathlineto{\pgfqpoint{3.924679in}{1.788251in}}%
\pgfpathlineto{\pgfqpoint{3.937293in}{1.842763in}}%
\pgfpathlineto{\pgfqpoint{3.949906in}{1.848558in}}%
\pgfpathlineto{\pgfqpoint{3.962520in}{1.869329in}}%
\pgfpathlineto{\pgfqpoint{3.975134in}{1.910590in}}%
\pgfpathlineto{\pgfqpoint{3.987747in}{1.946298in}}%
\pgfpathlineto{\pgfqpoint{4.012974in}{1.993094in}}%
\pgfpathlineto{\pgfqpoint{4.025588in}{2.002672in}}%
\pgfpathlineto{\pgfqpoint{4.038201in}{2.006356in}}%
\pgfpathlineto{\pgfqpoint{4.050815in}{2.021025in}}%
\pgfpathlineto{\pgfqpoint{4.063429in}{2.098041in}}%
\pgfpathlineto{\pgfqpoint{4.076042in}{2.106049in}}%
\pgfpathlineto{\pgfqpoint{4.088656in}{2.142626in}}%
\pgfpathlineto{\pgfqpoint{4.101269in}{2.152286in}}%
\pgfpathlineto{\pgfqpoint{4.113883in}{2.154419in}}%
\pgfpathlineto{\pgfqpoint{4.139110in}{2.167796in}}%
\pgfpathlineto{\pgfqpoint{4.151724in}{2.168054in}}%
\pgfpathlineto{\pgfqpoint{4.164337in}{2.174326in}}%
\pgfpathlineto{\pgfqpoint{4.176951in}{2.175430in}}%
\pgfpathlineto{\pgfqpoint{4.189564in}{2.184958in}}%
\pgfpathlineto{\pgfqpoint{4.202178in}{2.206324in}}%
\pgfpathlineto{\pgfqpoint{4.214791in}{2.206670in}}%
\pgfpathlineto{\pgfqpoint{4.227405in}{2.210631in}}%
\pgfpathlineto{\pgfqpoint{4.240018in}{2.212381in}}%
\pgfpathlineto{\pgfqpoint{4.265246in}{2.281125in}}%
\pgfpathlineto{\pgfqpoint{4.277859in}{2.305275in}}%
\pgfpathlineto{\pgfqpoint{4.277859in}{2.305275in}}%
\pgfusepath{stroke}%
\end{pgfscope}%
\begin{pgfscope}%
\pgfpathrectangle{\pgfqpoint{0.708220in}{0.535823in}}{\pgfqpoint{5.045427in}{1.769453in}}%
\pgfusepath{clip}%
\pgfsetrectcap%
\pgfsetroundjoin%
\pgfsetlinewidth{1.003750pt}%
\definecolor{currentstroke}{rgb}{1.000000,0.498039,0.054902}%
\pgfsetstrokecolor{currentstroke}%
\pgfsetdash{}{0pt}%
\pgfpathmoveto{\pgfqpoint{0.708220in}{0.550512in}}%
\pgfpathlineto{\pgfqpoint{0.720833in}{0.552230in}}%
\pgfpathlineto{\pgfqpoint{0.733447in}{0.570603in}}%
\pgfpathlineto{\pgfqpoint{0.746061in}{0.571880in}}%
\pgfpathlineto{\pgfqpoint{0.758674in}{0.578931in}}%
\pgfpathlineto{\pgfqpoint{0.771288in}{0.588640in}}%
\pgfpathlineto{\pgfqpoint{0.796515in}{0.590835in}}%
\pgfpathlineto{\pgfqpoint{0.809128in}{0.596600in}}%
\pgfpathlineto{\pgfqpoint{0.821742in}{0.598220in}}%
\pgfpathlineto{\pgfqpoint{0.846969in}{0.607644in}}%
\pgfpathlineto{\pgfqpoint{0.872196in}{0.609657in}}%
\pgfpathlineto{\pgfqpoint{0.884810in}{0.616517in}}%
\pgfpathlineto{\pgfqpoint{0.897423in}{0.619869in}}%
\pgfpathlineto{\pgfqpoint{0.910037in}{0.620114in}}%
\pgfpathlineto{\pgfqpoint{0.947878in}{0.625141in}}%
\pgfpathlineto{\pgfqpoint{0.960491in}{0.656505in}}%
\pgfpathlineto{\pgfqpoint{0.973105in}{0.668500in}}%
\pgfpathlineto{\pgfqpoint{0.985718in}{0.687909in}}%
\pgfpathlineto{\pgfqpoint{0.998332in}{0.696892in}}%
\pgfpathlineto{\pgfqpoint{1.010945in}{0.699214in}}%
\pgfpathlineto{\pgfqpoint{1.086627in}{0.703544in}}%
\pgfpathlineto{\pgfqpoint{1.099240in}{0.706227in}}%
\pgfpathlineto{\pgfqpoint{1.162308in}{0.708793in}}%
\pgfpathlineto{\pgfqpoint{1.187535in}{0.711423in}}%
\pgfpathlineto{\pgfqpoint{1.212763in}{0.712459in}}%
\pgfpathlineto{\pgfqpoint{1.225376in}{0.737031in}}%
\pgfpathlineto{\pgfqpoint{1.237990in}{0.737312in}}%
\pgfpathlineto{\pgfqpoint{1.250603in}{0.747530in}}%
\pgfpathlineto{\pgfqpoint{1.263217in}{0.753731in}}%
\pgfpathlineto{\pgfqpoint{1.275830in}{0.754302in}}%
\pgfpathlineto{\pgfqpoint{1.288444in}{0.760872in}}%
\pgfpathlineto{\pgfqpoint{1.326285in}{0.763390in}}%
\pgfpathlineto{\pgfqpoint{1.338898in}{0.770572in}}%
\pgfpathlineto{\pgfqpoint{1.351512in}{0.770608in}}%
\pgfpathlineto{\pgfqpoint{1.364125in}{0.776443in}}%
\pgfpathlineto{\pgfqpoint{1.376739in}{0.779788in}}%
\pgfpathlineto{\pgfqpoint{1.389352in}{0.781068in}}%
\pgfpathlineto{\pgfqpoint{1.401966in}{0.784227in}}%
\pgfpathlineto{\pgfqpoint{1.414580in}{0.797594in}}%
\pgfpathlineto{\pgfqpoint{1.427193in}{0.805782in}}%
\pgfpathlineto{\pgfqpoint{1.439807in}{0.808498in}}%
\pgfpathlineto{\pgfqpoint{1.452420in}{0.808992in}}%
\pgfpathlineto{\pgfqpoint{1.477647in}{0.814551in}}%
\pgfpathlineto{\pgfqpoint{1.502875in}{0.821261in}}%
\pgfpathlineto{\pgfqpoint{1.515488in}{0.832708in}}%
\pgfpathlineto{\pgfqpoint{1.528102in}{0.838185in}}%
\pgfpathlineto{\pgfqpoint{1.578556in}{0.850981in}}%
\pgfpathlineto{\pgfqpoint{1.591170in}{0.854411in}}%
\pgfpathlineto{\pgfqpoint{1.603783in}{0.861418in}}%
\pgfpathlineto{\pgfqpoint{1.616397in}{0.864554in}}%
\pgfpathlineto{\pgfqpoint{1.641624in}{0.867499in}}%
\pgfpathlineto{\pgfqpoint{1.654237in}{0.868587in}}%
\pgfpathlineto{\pgfqpoint{1.666851in}{0.873872in}}%
\pgfpathlineto{\pgfqpoint{1.679464in}{0.874024in}}%
\pgfpathlineto{\pgfqpoint{1.729919in}{0.880377in}}%
\pgfpathlineto{\pgfqpoint{1.755146in}{0.881901in}}%
\pgfpathlineto{\pgfqpoint{1.792987in}{0.887635in}}%
\pgfpathlineto{\pgfqpoint{1.805600in}{0.893267in}}%
\pgfpathlineto{\pgfqpoint{1.818214in}{0.896402in}}%
\pgfpathlineto{\pgfqpoint{1.830827in}{0.902873in}}%
\pgfpathlineto{\pgfqpoint{1.843441in}{0.903939in}}%
\pgfpathlineto{\pgfqpoint{1.868668in}{0.916965in}}%
\pgfpathlineto{\pgfqpoint{1.893895in}{0.926047in}}%
\pgfpathlineto{\pgfqpoint{1.906509in}{0.931313in}}%
\pgfpathlineto{\pgfqpoint{1.919122in}{0.932403in}}%
\pgfpathlineto{\pgfqpoint{1.931736in}{0.934611in}}%
\pgfpathlineto{\pgfqpoint{1.944349in}{0.942143in}}%
\pgfpathlineto{\pgfqpoint{1.969577in}{0.950407in}}%
\pgfpathlineto{\pgfqpoint{1.994804in}{0.951607in}}%
\pgfpathlineto{\pgfqpoint{2.007417in}{0.959296in}}%
\pgfpathlineto{\pgfqpoint{2.020031in}{0.962966in}}%
\pgfpathlineto{\pgfqpoint{2.045258in}{0.979135in}}%
\pgfpathlineto{\pgfqpoint{2.057871in}{0.981815in}}%
\pgfpathlineto{\pgfqpoint{2.108326in}{0.985938in}}%
\pgfpathlineto{\pgfqpoint{2.120939in}{0.997680in}}%
\pgfpathlineto{\pgfqpoint{2.133553in}{1.005878in}}%
\pgfpathlineto{\pgfqpoint{2.146166in}{1.006496in}}%
\pgfpathlineto{\pgfqpoint{2.158780in}{1.020199in}}%
\pgfpathlineto{\pgfqpoint{2.184007in}{1.025267in}}%
\pgfpathlineto{\pgfqpoint{2.209234in}{1.025957in}}%
\pgfpathlineto{\pgfqpoint{2.221848in}{1.028577in}}%
\pgfpathlineto{\pgfqpoint{2.234461in}{1.029261in}}%
\pgfpathlineto{\pgfqpoint{2.259689in}{1.032262in}}%
\pgfpathlineto{\pgfqpoint{2.284916in}{1.034739in}}%
\pgfpathlineto{\pgfqpoint{2.297529in}{1.043892in}}%
\pgfpathlineto{\pgfqpoint{2.310143in}{1.055290in}}%
\pgfpathlineto{\pgfqpoint{2.322756in}{1.056500in}}%
\pgfpathlineto{\pgfqpoint{2.335370in}{1.059174in}}%
\pgfpathlineto{\pgfqpoint{2.347984in}{1.067964in}}%
\pgfpathlineto{\pgfqpoint{2.398438in}{1.073974in}}%
\pgfpathlineto{\pgfqpoint{2.411051in}{1.078100in}}%
\pgfpathlineto{\pgfqpoint{2.423665in}{1.078832in}}%
\pgfpathlineto{\pgfqpoint{2.436278in}{1.084047in}}%
\pgfpathlineto{\pgfqpoint{2.448892in}{1.087497in}}%
\pgfpathlineto{\pgfqpoint{2.461506in}{1.089299in}}%
\pgfpathlineto{\pgfqpoint{2.486733in}{1.106089in}}%
\pgfpathlineto{\pgfqpoint{2.537187in}{1.108699in}}%
\pgfpathlineto{\pgfqpoint{2.549801in}{1.109010in}}%
\pgfpathlineto{\pgfqpoint{2.562414in}{1.111274in}}%
\pgfpathlineto{\pgfqpoint{2.575028in}{1.118244in}}%
\pgfpathlineto{\pgfqpoint{2.587641in}{1.122483in}}%
\pgfpathlineto{\pgfqpoint{2.600255in}{1.124964in}}%
\pgfpathlineto{\pgfqpoint{2.612868in}{1.125277in}}%
\pgfpathlineto{\pgfqpoint{2.625482in}{1.128033in}}%
\pgfpathlineto{\pgfqpoint{2.638096in}{1.132489in}}%
\pgfpathlineto{\pgfqpoint{2.650709in}{1.134837in}}%
\pgfpathlineto{\pgfqpoint{2.663323in}{1.135533in}}%
\pgfpathlineto{\pgfqpoint{2.675936in}{1.138551in}}%
\pgfpathlineto{\pgfqpoint{2.688550in}{1.140071in}}%
\pgfpathlineto{\pgfqpoint{2.713777in}{1.147966in}}%
\pgfpathlineto{\pgfqpoint{2.726390in}{1.154055in}}%
\pgfpathlineto{\pgfqpoint{2.739004in}{1.154639in}}%
\pgfpathlineto{\pgfqpoint{2.751618in}{1.166950in}}%
\pgfpathlineto{\pgfqpoint{2.764231in}{1.174548in}}%
\pgfpathlineto{\pgfqpoint{2.776845in}{1.176049in}}%
\pgfpathlineto{\pgfqpoint{2.789458in}{1.195512in}}%
\pgfpathlineto{\pgfqpoint{2.802072in}{1.195703in}}%
\pgfpathlineto{\pgfqpoint{2.814685in}{1.199502in}}%
\pgfpathlineto{\pgfqpoint{2.827299in}{1.200759in}}%
\pgfpathlineto{\pgfqpoint{2.839913in}{1.205048in}}%
\pgfpathlineto{\pgfqpoint{2.877753in}{1.210268in}}%
\pgfpathlineto{\pgfqpoint{2.890367in}{1.213809in}}%
\pgfpathlineto{\pgfqpoint{2.902980in}{1.214723in}}%
\pgfpathlineto{\pgfqpoint{2.915594in}{1.232096in}}%
\pgfpathlineto{\pgfqpoint{2.928208in}{1.234960in}}%
\pgfpathlineto{\pgfqpoint{2.940821in}{1.236582in}}%
\pgfpathlineto{\pgfqpoint{2.966048in}{1.247741in}}%
\pgfpathlineto{\pgfqpoint{2.978662in}{1.250176in}}%
\pgfpathlineto{\pgfqpoint{3.003889in}{1.251184in}}%
\pgfpathlineto{\pgfqpoint{3.016503in}{1.255891in}}%
\pgfpathlineto{\pgfqpoint{3.029116in}{1.258132in}}%
\pgfpathlineto{\pgfqpoint{3.041730in}{1.264091in}}%
\pgfpathlineto{\pgfqpoint{3.054343in}{1.264425in}}%
\pgfpathlineto{\pgfqpoint{3.066957in}{1.267691in}}%
\pgfpathlineto{\pgfqpoint{3.079570in}{1.267967in}}%
\pgfpathlineto{\pgfqpoint{3.092184in}{1.269855in}}%
\pgfpathlineto{\pgfqpoint{3.104797in}{1.269970in}}%
\pgfpathlineto{\pgfqpoint{3.130025in}{1.282290in}}%
\pgfpathlineto{\pgfqpoint{3.142638in}{1.285275in}}%
\pgfpathlineto{\pgfqpoint{3.155252in}{1.293288in}}%
\pgfpathlineto{\pgfqpoint{3.167865in}{1.293652in}}%
\pgfpathlineto{\pgfqpoint{3.180479in}{1.297453in}}%
\pgfpathlineto{\pgfqpoint{3.193092in}{1.305378in}}%
\pgfpathlineto{\pgfqpoint{3.205706in}{1.325956in}}%
\pgfpathlineto{\pgfqpoint{3.243547in}{1.327652in}}%
\pgfpathlineto{\pgfqpoint{3.256160in}{1.331831in}}%
\pgfpathlineto{\pgfqpoint{3.268774in}{1.337374in}}%
\pgfpathlineto{\pgfqpoint{3.294001in}{1.337708in}}%
\pgfpathlineto{\pgfqpoint{3.306615in}{1.340005in}}%
\pgfpathlineto{\pgfqpoint{3.319228in}{1.365413in}}%
\pgfpathlineto{\pgfqpoint{3.331842in}{1.368793in}}%
\pgfpathlineto{\pgfqpoint{3.344455in}{1.368822in}}%
\pgfpathlineto{\pgfqpoint{3.357069in}{1.375861in}}%
\pgfpathlineto{\pgfqpoint{3.369682in}{1.376095in}}%
\pgfpathlineto{\pgfqpoint{3.407523in}{1.393194in}}%
\pgfpathlineto{\pgfqpoint{3.420137in}{1.396703in}}%
\pgfpathlineto{\pgfqpoint{3.432750in}{1.410219in}}%
\pgfpathlineto{\pgfqpoint{3.457977in}{1.411240in}}%
\pgfpathlineto{\pgfqpoint{3.470591in}{1.416537in}}%
\pgfpathlineto{\pgfqpoint{3.483204in}{1.416857in}}%
\pgfpathlineto{\pgfqpoint{3.495818in}{1.446644in}}%
\pgfpathlineto{\pgfqpoint{3.508432in}{1.454322in}}%
\pgfpathlineto{\pgfqpoint{3.521045in}{1.464424in}}%
\pgfpathlineto{\pgfqpoint{3.533659in}{1.467037in}}%
\pgfpathlineto{\pgfqpoint{3.558886in}{1.467859in}}%
\pgfpathlineto{\pgfqpoint{3.571499in}{1.476830in}}%
\pgfpathlineto{\pgfqpoint{3.584113in}{1.477101in}}%
\pgfpathlineto{\pgfqpoint{3.596727in}{1.481168in}}%
\pgfpathlineto{\pgfqpoint{3.621954in}{1.483818in}}%
\pgfpathlineto{\pgfqpoint{3.647181in}{1.484057in}}%
\pgfpathlineto{\pgfqpoint{3.659794in}{1.487800in}}%
\pgfpathlineto{\pgfqpoint{3.685022in}{1.489037in}}%
\pgfpathlineto{\pgfqpoint{3.697635in}{1.489991in}}%
\pgfpathlineto{\pgfqpoint{3.710249in}{1.536703in}}%
\pgfpathlineto{\pgfqpoint{3.722862in}{1.536940in}}%
\pgfpathlineto{\pgfqpoint{3.735476in}{1.542089in}}%
\pgfpathlineto{\pgfqpoint{3.748089in}{1.557106in}}%
\pgfpathlineto{\pgfqpoint{3.785930in}{1.576081in}}%
\pgfpathlineto{\pgfqpoint{3.798544in}{1.592154in}}%
\pgfpathlineto{\pgfqpoint{3.811157in}{1.594665in}}%
\pgfpathlineto{\pgfqpoint{3.823771in}{1.595401in}}%
\pgfpathlineto{\pgfqpoint{3.836384in}{1.607566in}}%
\pgfpathlineto{\pgfqpoint{3.861611in}{1.612144in}}%
\pgfpathlineto{\pgfqpoint{3.886839in}{1.613399in}}%
\pgfpathlineto{\pgfqpoint{3.899452in}{1.617079in}}%
\pgfpathlineto{\pgfqpoint{3.912066in}{1.622812in}}%
\pgfpathlineto{\pgfqpoint{3.924679in}{1.625178in}}%
\pgfpathlineto{\pgfqpoint{3.937293in}{1.625723in}}%
\pgfpathlineto{\pgfqpoint{3.949906in}{1.635479in}}%
\pgfpathlineto{\pgfqpoint{3.987747in}{1.638780in}}%
\pgfpathlineto{\pgfqpoint{4.000361in}{1.647910in}}%
\pgfpathlineto{\pgfqpoint{4.012974in}{1.650432in}}%
\pgfpathlineto{\pgfqpoint{4.025588in}{1.662456in}}%
\pgfpathlineto{\pgfqpoint{4.038201in}{1.665956in}}%
\pgfpathlineto{\pgfqpoint{4.050815in}{1.666104in}}%
\pgfpathlineto{\pgfqpoint{4.063429in}{1.674107in}}%
\pgfpathlineto{\pgfqpoint{4.088656in}{1.674601in}}%
\pgfpathlineto{\pgfqpoint{4.113883in}{1.711710in}}%
\pgfpathlineto{\pgfqpoint{4.126496in}{1.717785in}}%
\pgfpathlineto{\pgfqpoint{4.139110in}{1.727804in}}%
\pgfpathlineto{\pgfqpoint{4.151724in}{1.728081in}}%
\pgfpathlineto{\pgfqpoint{4.164337in}{1.730867in}}%
\pgfpathlineto{\pgfqpoint{4.176951in}{1.768482in}}%
\pgfpathlineto{\pgfqpoint{4.189564in}{1.771174in}}%
\pgfpathlineto{\pgfqpoint{4.202178in}{1.781149in}}%
\pgfpathlineto{\pgfqpoint{4.214791in}{1.786575in}}%
\pgfpathlineto{\pgfqpoint{4.227405in}{1.796494in}}%
\pgfpathlineto{\pgfqpoint{4.240018in}{1.802091in}}%
\pgfpathlineto{\pgfqpoint{4.252632in}{1.803734in}}%
\pgfpathlineto{\pgfqpoint{4.265246in}{1.806850in}}%
\pgfpathlineto{\pgfqpoint{4.277859in}{1.842477in}}%
\pgfpathlineto{\pgfqpoint{4.290473in}{1.843636in}}%
\pgfpathlineto{\pgfqpoint{4.303086in}{1.847224in}}%
\pgfpathlineto{\pgfqpoint{4.315700in}{1.848602in}}%
\pgfpathlineto{\pgfqpoint{4.328313in}{1.876164in}}%
\pgfpathlineto{\pgfqpoint{4.340927in}{1.880860in}}%
\pgfpathlineto{\pgfqpoint{4.353541in}{1.882711in}}%
\pgfpathlineto{\pgfqpoint{4.378768in}{1.902003in}}%
\pgfpathlineto{\pgfqpoint{4.416608in}{1.903492in}}%
\pgfpathlineto{\pgfqpoint{4.429222in}{1.904091in}}%
\pgfpathlineto{\pgfqpoint{4.441836in}{1.932519in}}%
\pgfpathlineto{\pgfqpoint{4.454449in}{1.932976in}}%
\pgfpathlineto{\pgfqpoint{4.467063in}{1.938309in}}%
\pgfpathlineto{\pgfqpoint{4.479676in}{1.939372in}}%
\pgfpathlineto{\pgfqpoint{4.492290in}{1.954544in}}%
\pgfpathlineto{\pgfqpoint{4.504903in}{1.964985in}}%
\pgfpathlineto{\pgfqpoint{4.517517in}{1.967320in}}%
\pgfpathlineto{\pgfqpoint{4.530131in}{1.996487in}}%
\pgfpathlineto{\pgfqpoint{4.542744in}{1.996549in}}%
\pgfpathlineto{\pgfqpoint{4.555358in}{2.000321in}}%
\pgfpathlineto{\pgfqpoint{4.567971in}{2.000630in}}%
\pgfpathlineto{\pgfqpoint{4.580585in}{2.028029in}}%
\pgfpathlineto{\pgfqpoint{4.605812in}{2.030447in}}%
\pgfpathlineto{\pgfqpoint{4.618425in}{2.041523in}}%
\pgfpathlineto{\pgfqpoint{4.656266in}{2.046620in}}%
\pgfpathlineto{\pgfqpoint{4.668880in}{2.062032in}}%
\pgfpathlineto{\pgfqpoint{4.681493in}{2.073839in}}%
\pgfpathlineto{\pgfqpoint{4.694107in}{2.075880in}}%
\pgfpathlineto{\pgfqpoint{4.706720in}{2.076506in}}%
\pgfpathlineto{\pgfqpoint{4.719334in}{2.087166in}}%
\pgfpathlineto{\pgfqpoint{4.731948in}{2.106760in}}%
\pgfpathlineto{\pgfqpoint{4.744561in}{2.116146in}}%
\pgfpathlineto{\pgfqpoint{4.757175in}{2.120891in}}%
\pgfpathlineto{\pgfqpoint{4.769788in}{2.121880in}}%
\pgfpathlineto{\pgfqpoint{4.782402in}{2.126655in}}%
\pgfpathlineto{\pgfqpoint{4.795015in}{2.132965in}}%
\pgfpathlineto{\pgfqpoint{4.807629in}{2.154631in}}%
\pgfpathlineto{\pgfqpoint{4.820243in}{2.155491in}}%
\pgfpathlineto{\pgfqpoint{4.832856in}{2.162066in}}%
\pgfpathlineto{\pgfqpoint{4.845470in}{2.164343in}}%
\pgfpathlineto{\pgfqpoint{4.870697in}{2.164768in}}%
\pgfpathlineto{\pgfqpoint{4.883310in}{2.170510in}}%
\pgfpathlineto{\pgfqpoint{4.895924in}{2.183989in}}%
\pgfpathlineto{\pgfqpoint{4.908538in}{2.194301in}}%
\pgfpathlineto{\pgfqpoint{4.933765in}{2.195364in}}%
\pgfpathlineto{\pgfqpoint{4.946378in}{2.219943in}}%
\pgfpathlineto{\pgfqpoint{4.958992in}{2.225812in}}%
\pgfpathlineto{\pgfqpoint{4.971605in}{2.229674in}}%
\pgfpathlineto{\pgfqpoint{4.984219in}{2.238325in}}%
\pgfpathlineto{\pgfqpoint{4.996832in}{2.239699in}}%
\pgfpathlineto{\pgfqpoint{5.009446in}{2.263624in}}%
\pgfpathlineto{\pgfqpoint{5.022060in}{2.266698in}}%
\pgfpathlineto{\pgfqpoint{5.034673in}{2.266887in}}%
\pgfpathlineto{\pgfqpoint{5.047287in}{2.290341in}}%
\pgfpathlineto{\pgfqpoint{5.059900in}{2.305275in}}%
\pgfpathlineto{\pgfqpoint{5.059900in}{2.305275in}}%
\pgfusepath{stroke}%
\end{pgfscope}%
\begin{pgfscope}%
\pgfpathrectangle{\pgfqpoint{0.708220in}{0.535823in}}{\pgfqpoint{5.045427in}{1.769453in}}%
\pgfusepath{clip}%
\pgfsetrectcap%
\pgfsetroundjoin%
\pgfsetlinewidth{1.003750pt}%
\definecolor{currentstroke}{rgb}{0.172549,0.627451,0.172549}%
\pgfsetstrokecolor{currentstroke}%
\pgfsetdash{}{0pt}%
\pgfpathmoveto{\pgfqpoint{0.708220in}{0.951518in}}%
\pgfpathlineto{\pgfqpoint{0.720833in}{0.954909in}}%
\pgfpathlineto{\pgfqpoint{0.796515in}{0.958590in}}%
\pgfpathlineto{\pgfqpoint{0.809128in}{0.961925in}}%
\pgfpathlineto{\pgfqpoint{0.821742in}{0.962103in}}%
\pgfpathlineto{\pgfqpoint{0.884810in}{0.969768in}}%
\pgfpathlineto{\pgfqpoint{0.897423in}{0.970233in}}%
\pgfpathlineto{\pgfqpoint{0.910037in}{0.972095in}}%
\pgfpathlineto{\pgfqpoint{0.922650in}{0.975947in}}%
\pgfpathlineto{\pgfqpoint{0.947878in}{0.976376in}}%
\pgfpathlineto{\pgfqpoint{0.985718in}{0.982587in}}%
\pgfpathlineto{\pgfqpoint{0.998332in}{0.990525in}}%
\pgfpathlineto{\pgfqpoint{1.048786in}{0.992306in}}%
\pgfpathlineto{\pgfqpoint{1.086627in}{0.992848in}}%
\pgfpathlineto{\pgfqpoint{1.149695in}{0.997657in}}%
\pgfpathlineto{\pgfqpoint{1.174922in}{0.999097in}}%
\pgfpathlineto{\pgfqpoint{1.187535in}{1.001344in}}%
\pgfpathlineto{\pgfqpoint{1.200149in}{1.006229in}}%
\pgfpathlineto{\pgfqpoint{1.212763in}{1.007755in}}%
\pgfpathlineto{\pgfqpoint{1.237990in}{1.008526in}}%
\pgfpathlineto{\pgfqpoint{1.288444in}{1.013331in}}%
\pgfpathlineto{\pgfqpoint{1.313671in}{1.014565in}}%
\pgfpathlineto{\pgfqpoint{1.326285in}{1.014810in}}%
\pgfpathlineto{\pgfqpoint{1.364125in}{1.019913in}}%
\pgfpathlineto{\pgfqpoint{1.376739in}{1.025320in}}%
\pgfpathlineto{\pgfqpoint{1.389352in}{1.027413in}}%
\pgfpathlineto{\pgfqpoint{1.401966in}{1.031749in}}%
\pgfpathlineto{\pgfqpoint{1.414580in}{1.034064in}}%
\pgfpathlineto{\pgfqpoint{1.439807in}{1.051395in}}%
\pgfpathlineto{\pgfqpoint{1.477647in}{1.053561in}}%
\pgfpathlineto{\pgfqpoint{1.490261in}{1.059640in}}%
\pgfpathlineto{\pgfqpoint{1.502875in}{1.059763in}}%
\pgfpathlineto{\pgfqpoint{1.528102in}{1.064696in}}%
\pgfpathlineto{\pgfqpoint{1.540715in}{1.070494in}}%
\pgfpathlineto{\pgfqpoint{1.553329in}{1.071452in}}%
\pgfpathlineto{\pgfqpoint{1.578556in}{1.078374in}}%
\pgfpathlineto{\pgfqpoint{1.591170in}{1.079164in}}%
\pgfpathlineto{\pgfqpoint{1.603783in}{1.085158in}}%
\pgfpathlineto{\pgfqpoint{1.616397in}{1.087205in}}%
\pgfpathlineto{\pgfqpoint{1.629010in}{1.087303in}}%
\pgfpathlineto{\pgfqpoint{1.666851in}{1.090167in}}%
\pgfpathlineto{\pgfqpoint{1.692078in}{1.090983in}}%
\pgfpathlineto{\pgfqpoint{1.742532in}{1.093958in}}%
\pgfpathlineto{\pgfqpoint{1.755146in}{1.096066in}}%
\pgfpathlineto{\pgfqpoint{1.767759in}{1.103614in}}%
\pgfpathlineto{\pgfqpoint{1.780373in}{1.108325in}}%
\pgfpathlineto{\pgfqpoint{1.792987in}{1.110200in}}%
\pgfpathlineto{\pgfqpoint{1.805600in}{1.115036in}}%
\pgfpathlineto{\pgfqpoint{1.856054in}{1.116812in}}%
\pgfpathlineto{\pgfqpoint{1.868668in}{1.126059in}}%
\pgfpathlineto{\pgfqpoint{1.893895in}{1.128103in}}%
\pgfpathlineto{\pgfqpoint{1.906509in}{1.128701in}}%
\pgfpathlineto{\pgfqpoint{1.919122in}{1.141238in}}%
\pgfpathlineto{\pgfqpoint{1.931736in}{1.141322in}}%
\pgfpathlineto{\pgfqpoint{1.944349in}{1.148976in}}%
\pgfpathlineto{\pgfqpoint{1.956963in}{1.151853in}}%
\pgfpathlineto{\pgfqpoint{1.969577in}{1.152098in}}%
\pgfpathlineto{\pgfqpoint{1.982190in}{1.154783in}}%
\pgfpathlineto{\pgfqpoint{2.057871in}{1.158797in}}%
\pgfpathlineto{\pgfqpoint{2.070485in}{1.162785in}}%
\pgfpathlineto{\pgfqpoint{2.083099in}{1.163401in}}%
\pgfpathlineto{\pgfqpoint{2.120939in}{1.168293in}}%
\pgfpathlineto{\pgfqpoint{2.133553in}{1.174360in}}%
\pgfpathlineto{\pgfqpoint{2.184007in}{1.179513in}}%
\pgfpathlineto{\pgfqpoint{2.196621in}{1.181338in}}%
\pgfpathlineto{\pgfqpoint{2.209234in}{1.185403in}}%
\pgfpathlineto{\pgfqpoint{2.221848in}{1.186150in}}%
\pgfpathlineto{\pgfqpoint{2.259689in}{1.193600in}}%
\pgfpathlineto{\pgfqpoint{2.284916in}{1.195279in}}%
\pgfpathlineto{\pgfqpoint{2.322756in}{1.197600in}}%
\pgfpathlineto{\pgfqpoint{2.335370in}{1.204550in}}%
\pgfpathlineto{\pgfqpoint{2.347984in}{1.205359in}}%
\pgfpathlineto{\pgfqpoint{2.385824in}{1.223579in}}%
\pgfpathlineto{\pgfqpoint{2.398438in}{1.223580in}}%
\pgfpathlineto{\pgfqpoint{2.411051in}{1.229512in}}%
\pgfpathlineto{\pgfqpoint{2.423665in}{1.246372in}}%
\pgfpathlineto{\pgfqpoint{2.436278in}{1.246647in}}%
\pgfpathlineto{\pgfqpoint{2.461506in}{1.250821in}}%
\pgfpathlineto{\pgfqpoint{2.474119in}{1.251187in}}%
\pgfpathlineto{\pgfqpoint{2.486733in}{1.255926in}}%
\pgfpathlineto{\pgfqpoint{2.511960in}{1.258421in}}%
\pgfpathlineto{\pgfqpoint{2.524573in}{1.262098in}}%
\pgfpathlineto{\pgfqpoint{2.537187in}{1.264577in}}%
\pgfpathlineto{\pgfqpoint{2.549801in}{1.273096in}}%
\pgfpathlineto{\pgfqpoint{2.562414in}{1.275201in}}%
\pgfpathlineto{\pgfqpoint{2.575028in}{1.275911in}}%
\pgfpathlineto{\pgfqpoint{2.587641in}{1.279998in}}%
\pgfpathlineto{\pgfqpoint{2.600255in}{1.280606in}}%
\pgfpathlineto{\pgfqpoint{2.625482in}{1.285182in}}%
\pgfpathlineto{\pgfqpoint{2.638096in}{1.290566in}}%
\pgfpathlineto{\pgfqpoint{2.650709in}{1.291330in}}%
\pgfpathlineto{\pgfqpoint{2.663323in}{1.305641in}}%
\pgfpathlineto{\pgfqpoint{2.675936in}{1.306348in}}%
\pgfpathlineto{\pgfqpoint{2.701163in}{1.326623in}}%
\pgfpathlineto{\pgfqpoint{2.713777in}{1.330047in}}%
\pgfpathlineto{\pgfqpoint{2.726390in}{1.336436in}}%
\pgfpathlineto{\pgfqpoint{2.739004in}{1.336976in}}%
\pgfpathlineto{\pgfqpoint{2.751618in}{1.339969in}}%
\pgfpathlineto{\pgfqpoint{2.764231in}{1.340282in}}%
\pgfpathlineto{\pgfqpoint{2.776845in}{1.349627in}}%
\pgfpathlineto{\pgfqpoint{2.814685in}{1.353491in}}%
\pgfpathlineto{\pgfqpoint{2.865140in}{1.354349in}}%
\pgfpathlineto{\pgfqpoint{2.877753in}{1.356713in}}%
\pgfpathlineto{\pgfqpoint{2.890367in}{1.357506in}}%
\pgfpathlineto{\pgfqpoint{2.902980in}{1.367074in}}%
\pgfpathlineto{\pgfqpoint{2.928208in}{1.371229in}}%
\pgfpathlineto{\pgfqpoint{2.940821in}{1.396291in}}%
\pgfpathlineto{\pgfqpoint{2.953435in}{1.396376in}}%
\pgfpathlineto{\pgfqpoint{2.966048in}{1.402838in}}%
\pgfpathlineto{\pgfqpoint{2.978662in}{1.402869in}}%
\pgfpathlineto{\pgfqpoint{2.991275in}{1.439408in}}%
\pgfpathlineto{\pgfqpoint{3.003889in}{1.441410in}}%
\pgfpathlineto{\pgfqpoint{3.029116in}{1.459493in}}%
\pgfpathlineto{\pgfqpoint{3.041730in}{1.464645in}}%
\pgfpathlineto{\pgfqpoint{3.054343in}{1.471507in}}%
\pgfpathlineto{\pgfqpoint{3.066957in}{1.472886in}}%
\pgfpathlineto{\pgfqpoint{3.079570in}{1.476608in}}%
\pgfpathlineto{\pgfqpoint{3.092184in}{1.482023in}}%
\pgfpathlineto{\pgfqpoint{3.104797in}{1.503495in}}%
\pgfpathlineto{\pgfqpoint{3.117411in}{1.509701in}}%
\pgfpathlineto{\pgfqpoint{3.130025in}{1.510686in}}%
\pgfpathlineto{\pgfqpoint{3.142638in}{1.517288in}}%
\pgfpathlineto{\pgfqpoint{3.155252in}{1.533520in}}%
\pgfpathlineto{\pgfqpoint{3.180479in}{1.535333in}}%
\pgfpathlineto{\pgfqpoint{3.193092in}{1.535899in}}%
\pgfpathlineto{\pgfqpoint{3.205706in}{1.558495in}}%
\pgfpathlineto{\pgfqpoint{3.218320in}{1.559094in}}%
\pgfpathlineto{\pgfqpoint{3.230933in}{1.561696in}}%
\pgfpathlineto{\pgfqpoint{3.243547in}{1.568690in}}%
\pgfpathlineto{\pgfqpoint{3.256160in}{1.572267in}}%
\pgfpathlineto{\pgfqpoint{3.268774in}{1.572801in}}%
\pgfpathlineto{\pgfqpoint{3.281387in}{1.578194in}}%
\pgfpathlineto{\pgfqpoint{3.294001in}{1.580378in}}%
\pgfpathlineto{\pgfqpoint{3.319228in}{1.581009in}}%
\pgfpathlineto{\pgfqpoint{3.331842in}{1.585495in}}%
\pgfpathlineto{\pgfqpoint{3.344455in}{1.596275in}}%
\pgfpathlineto{\pgfqpoint{3.369682in}{1.623099in}}%
\pgfpathlineto{\pgfqpoint{3.382296in}{1.623243in}}%
\pgfpathlineto{\pgfqpoint{3.407523in}{1.653592in}}%
\pgfpathlineto{\pgfqpoint{3.420137in}{1.653777in}}%
\pgfpathlineto{\pgfqpoint{3.432750in}{1.663768in}}%
\pgfpathlineto{\pgfqpoint{3.445364in}{1.700094in}}%
\pgfpathlineto{\pgfqpoint{3.457977in}{1.751251in}}%
\pgfpathlineto{\pgfqpoint{3.470591in}{1.766748in}}%
\pgfpathlineto{\pgfqpoint{3.483204in}{1.766855in}}%
\pgfpathlineto{\pgfqpoint{3.508432in}{1.769208in}}%
\pgfpathlineto{\pgfqpoint{3.521045in}{1.784432in}}%
\pgfpathlineto{\pgfqpoint{3.533659in}{1.785274in}}%
\pgfpathlineto{\pgfqpoint{3.546272in}{1.791796in}}%
\pgfpathlineto{\pgfqpoint{3.558886in}{1.795504in}}%
\pgfpathlineto{\pgfqpoint{3.571499in}{1.796667in}}%
\pgfpathlineto{\pgfqpoint{3.584113in}{1.819395in}}%
\pgfpathlineto{\pgfqpoint{3.596727in}{1.831955in}}%
\pgfpathlineto{\pgfqpoint{3.609340in}{1.833330in}}%
\pgfpathlineto{\pgfqpoint{3.621954in}{1.838083in}}%
\pgfpathlineto{\pgfqpoint{3.634567in}{1.838471in}}%
\pgfpathlineto{\pgfqpoint{3.659794in}{1.847565in}}%
\pgfpathlineto{\pgfqpoint{3.672408in}{1.847596in}}%
\pgfpathlineto{\pgfqpoint{3.685022in}{1.884247in}}%
\pgfpathlineto{\pgfqpoint{3.710249in}{1.897807in}}%
\pgfpathlineto{\pgfqpoint{3.722862in}{1.904457in}}%
\pgfpathlineto{\pgfqpoint{3.735476in}{1.905142in}}%
\pgfpathlineto{\pgfqpoint{3.748089in}{1.924038in}}%
\pgfpathlineto{\pgfqpoint{3.760703in}{1.925889in}}%
\pgfpathlineto{\pgfqpoint{3.773317in}{1.954897in}}%
\pgfpathlineto{\pgfqpoint{3.798544in}{1.970227in}}%
\pgfpathlineto{\pgfqpoint{3.836384in}{1.977899in}}%
\pgfpathlineto{\pgfqpoint{3.848998in}{1.988961in}}%
\pgfpathlineto{\pgfqpoint{3.861611in}{2.038733in}}%
\pgfpathlineto{\pgfqpoint{3.874225in}{2.071444in}}%
\pgfpathlineto{\pgfqpoint{3.886839in}{2.091111in}}%
\pgfpathlineto{\pgfqpoint{3.899452in}{2.091488in}}%
\pgfpathlineto{\pgfqpoint{3.912066in}{2.096103in}}%
\pgfpathlineto{\pgfqpoint{3.924679in}{2.103156in}}%
\pgfpathlineto{\pgfqpoint{3.937293in}{2.106026in}}%
\pgfpathlineto{\pgfqpoint{3.949906in}{2.117706in}}%
\pgfpathlineto{\pgfqpoint{3.962520in}{2.120772in}}%
\pgfpathlineto{\pgfqpoint{3.975134in}{2.128290in}}%
\pgfpathlineto{\pgfqpoint{3.987747in}{2.129422in}}%
\pgfpathlineto{\pgfqpoint{4.000361in}{2.144025in}}%
\pgfpathlineto{\pgfqpoint{4.025588in}{2.145464in}}%
\pgfpathlineto{\pgfqpoint{4.038201in}{2.146386in}}%
\pgfpathlineto{\pgfqpoint{4.050815in}{2.154539in}}%
\pgfpathlineto{\pgfqpoint{4.076042in}{2.177227in}}%
\pgfpathlineto{\pgfqpoint{4.088656in}{2.180390in}}%
\pgfpathlineto{\pgfqpoint{4.101269in}{2.187862in}}%
\pgfpathlineto{\pgfqpoint{4.113883in}{2.189422in}}%
\pgfpathlineto{\pgfqpoint{4.126496in}{2.198120in}}%
\pgfpathlineto{\pgfqpoint{4.139110in}{2.259607in}}%
\pgfpathlineto{\pgfqpoint{4.151724in}{2.269747in}}%
\pgfpathlineto{\pgfqpoint{4.164337in}{2.277026in}}%
\pgfpathlineto{\pgfqpoint{4.176951in}{2.305275in}}%
\pgfpathlineto{\pgfqpoint{4.176951in}{2.305275in}}%
\pgfusepath{stroke}%
\end{pgfscope}%
\begin{pgfscope}%
\pgfpathrectangle{\pgfqpoint{0.708220in}{0.535823in}}{\pgfqpoint{5.045427in}{1.769453in}}%
\pgfusepath{clip}%
\pgfsetrectcap%
\pgfsetroundjoin%
\pgfsetlinewidth{1.003750pt}%
\definecolor{currentstroke}{rgb}{0.839216,0.152941,0.156863}%
\pgfsetstrokecolor{currentstroke}%
\pgfsetdash{}{0pt}%
\pgfpathmoveto{\pgfqpoint{0.846969in}{0.901177in}}%
\pgfpathlineto{\pgfqpoint{1.048786in}{0.901177in}}%
\pgfpathlineto{\pgfqpoint{1.061400in}{0.985712in}}%
\pgfpathlineto{\pgfqpoint{1.149695in}{0.985712in}}%
\pgfpathlineto{\pgfqpoint{1.162308in}{1.070248in}}%
\pgfpathlineto{\pgfqpoint{1.174922in}{1.070248in}}%
\pgfpathlineto{\pgfqpoint{1.187535in}{1.097462in}}%
\pgfpathlineto{\pgfqpoint{1.200149in}{1.181997in}}%
\pgfpathlineto{\pgfqpoint{1.212763in}{1.204233in}}%
\pgfpathlineto{\pgfqpoint{1.225376in}{1.204233in}}%
\pgfpathlineto{\pgfqpoint{1.237990in}{1.223033in}}%
\pgfpathlineto{\pgfqpoint{1.250603in}{1.231447in}}%
\pgfpathlineto{\pgfqpoint{1.263217in}{1.231447in}}%
\pgfpathlineto{\pgfqpoint{1.275830in}{1.239318in}}%
\pgfpathlineto{\pgfqpoint{1.288444in}{1.298530in}}%
\pgfpathlineto{\pgfqpoint{1.301057in}{1.298530in}}%
\pgfpathlineto{\pgfqpoint{1.313671in}{1.303132in}}%
\pgfpathlineto{\pgfqpoint{1.326285in}{1.344811in}}%
\pgfpathlineto{\pgfqpoint{1.338898in}{1.359887in}}%
\pgfpathlineto{\pgfqpoint{1.351512in}{1.359887in}}%
\pgfpathlineto{\pgfqpoint{1.364125in}{1.362691in}}%
\pgfpathlineto{\pgfqpoint{1.376739in}{1.396382in}}%
\pgfpathlineto{\pgfqpoint{1.389352in}{1.400517in}}%
\pgfpathlineto{\pgfqpoint{1.401966in}{1.400517in}}%
\pgfpathlineto{\pgfqpoint{1.414580in}{1.402533in}}%
\pgfpathlineto{\pgfqpoint{1.427193in}{1.402533in}}%
\pgfpathlineto{\pgfqpoint{1.439807in}{1.404516in}}%
\pgfpathlineto{\pgfqpoint{1.452420in}{1.408388in}}%
\pgfpathlineto{\pgfqpoint{1.465034in}{1.410279in}}%
\pgfpathlineto{\pgfqpoint{1.490261in}{1.427731in}}%
\pgfpathlineto{\pgfqpoint{1.502875in}{1.476638in}}%
\pgfpathlineto{\pgfqpoint{1.515488in}{1.491002in}}%
\pgfpathlineto{\pgfqpoint{1.540715in}{1.492923in}}%
\pgfpathlineto{\pgfqpoint{1.553329in}{1.498510in}}%
\pgfpathlineto{\pgfqpoint{1.565942in}{1.500317in}}%
\pgfpathlineto{\pgfqpoint{1.578556in}{1.531066in}}%
\pgfpathlineto{\pgfqpoint{1.591170in}{1.533823in}}%
\pgfpathlineto{\pgfqpoint{1.603783in}{1.534502in}}%
\pgfpathlineto{\pgfqpoint{1.616397in}{1.539805in}}%
\pgfpathlineto{\pgfqpoint{1.629010in}{1.543637in}}%
\pgfpathlineto{\pgfqpoint{1.641624in}{1.543637in}}%
\pgfpathlineto{\pgfqpoint{1.654237in}{1.555032in}}%
\pgfpathlineto{\pgfqpoint{1.666851in}{1.557862in}}%
\pgfpathlineto{\pgfqpoint{1.679464in}{1.590546in}}%
\pgfpathlineto{\pgfqpoint{1.692078in}{1.592667in}}%
\pgfpathlineto{\pgfqpoint{1.704692in}{1.597612in}}%
\pgfpathlineto{\pgfqpoint{1.717305in}{1.606188in}}%
\pgfpathlineto{\pgfqpoint{1.729919in}{1.610623in}}%
\pgfpathlineto{\pgfqpoint{1.742532in}{1.610985in}}%
\pgfpathlineto{\pgfqpoint{1.755146in}{1.614903in}}%
\pgfpathlineto{\pgfqpoint{1.767759in}{1.620719in}}%
\pgfpathlineto{\pgfqpoint{1.780373in}{1.621386in}}%
\pgfpathlineto{\pgfqpoint{1.792987in}{1.631581in}}%
\pgfpathlineto{\pgfqpoint{1.805600in}{1.699962in}}%
\pgfpathlineto{\pgfqpoint{1.843441in}{1.706914in}}%
\pgfpathlineto{\pgfqpoint{1.856054in}{1.727907in}}%
\pgfpathlineto{\pgfqpoint{1.893895in}{1.732802in}}%
\pgfpathlineto{\pgfqpoint{1.906509in}{1.733069in}}%
\pgfpathlineto{\pgfqpoint{1.919122in}{1.768982in}}%
\pgfpathlineto{\pgfqpoint{1.931736in}{1.782740in}}%
\pgfpathlineto{\pgfqpoint{1.944349in}{1.808852in}}%
\pgfpathlineto{\pgfqpoint{1.956963in}{1.813478in}}%
\pgfpathlineto{\pgfqpoint{1.969577in}{1.813959in}}%
\pgfpathlineto{\pgfqpoint{1.982190in}{1.817537in}}%
\pgfpathlineto{\pgfqpoint{2.007417in}{1.820819in}}%
\pgfpathlineto{\pgfqpoint{2.032644in}{1.822044in}}%
\pgfpathlineto{\pgfqpoint{2.045258in}{1.835565in}}%
\pgfpathlineto{\pgfqpoint{2.057871in}{1.841283in}}%
\pgfpathlineto{\pgfqpoint{2.070485in}{1.850914in}}%
\pgfpathlineto{\pgfqpoint{2.083099in}{1.870507in}}%
\pgfpathlineto{\pgfqpoint{2.108326in}{1.870980in}}%
\pgfpathlineto{\pgfqpoint{2.120939in}{1.946010in}}%
\pgfpathlineto{\pgfqpoint{2.133553in}{1.968223in}}%
\pgfpathlineto{\pgfqpoint{2.146166in}{1.971260in}}%
\pgfpathlineto{\pgfqpoint{2.158780in}{1.971655in}}%
\pgfpathlineto{\pgfqpoint{2.171394in}{1.975996in}}%
\pgfpathlineto{\pgfqpoint{2.184007in}{1.989598in}}%
\pgfpathlineto{\pgfqpoint{2.196621in}{1.999042in}}%
\pgfpathlineto{\pgfqpoint{2.209234in}{2.010845in}}%
\pgfpathlineto{\pgfqpoint{2.221848in}{2.046304in}}%
\pgfpathlineto{\pgfqpoint{2.234461in}{2.056096in}}%
\pgfpathlineto{\pgfqpoint{2.247075in}{2.061821in}}%
\pgfpathlineto{\pgfqpoint{2.259689in}{2.091936in}}%
\pgfpathlineto{\pgfqpoint{2.272302in}{2.092440in}}%
\pgfpathlineto{\pgfqpoint{2.284916in}{2.098855in}}%
\pgfpathlineto{\pgfqpoint{2.297529in}{2.099060in}}%
\pgfpathlineto{\pgfqpoint{2.310143in}{2.137727in}}%
\pgfpathlineto{\pgfqpoint{2.322756in}{2.140608in}}%
\pgfpathlineto{\pgfqpoint{2.335370in}{2.140838in}}%
\pgfpathlineto{\pgfqpoint{2.347984in}{2.148871in}}%
\pgfpathlineto{\pgfqpoint{2.360597in}{2.154474in}}%
\pgfpathlineto{\pgfqpoint{2.373211in}{2.157059in}}%
\pgfpathlineto{\pgfqpoint{2.385824in}{2.165382in}}%
\pgfpathlineto{\pgfqpoint{2.398438in}{2.178716in}}%
\pgfpathlineto{\pgfqpoint{2.411051in}{2.225928in}}%
\pgfpathlineto{\pgfqpoint{2.423665in}{2.226430in}}%
\pgfpathlineto{\pgfqpoint{2.436278in}{2.254603in}}%
\pgfpathlineto{\pgfqpoint{2.448892in}{2.254871in}}%
\pgfpathlineto{\pgfqpoint{2.461506in}{2.256445in}}%
\pgfpathlineto{\pgfqpoint{2.474119in}{2.265889in}}%
\pgfpathlineto{\pgfqpoint{2.486733in}{2.268296in}}%
\pgfpathlineto{\pgfqpoint{2.499346in}{2.268908in}}%
\pgfpathlineto{\pgfqpoint{2.511960in}{2.274342in}}%
\pgfpathlineto{\pgfqpoint{2.575028in}{2.278636in}}%
\pgfpathlineto{\pgfqpoint{2.587641in}{2.282430in}}%
\pgfpathlineto{\pgfqpoint{2.612868in}{2.284429in}}%
\pgfpathlineto{\pgfqpoint{2.625482in}{2.299781in}}%
\pgfpathlineto{\pgfqpoint{2.638096in}{2.300218in}}%
\pgfpathlineto{\pgfqpoint{2.650709in}{2.305275in}}%
\pgfpathlineto{\pgfqpoint{2.650709in}{2.305275in}}%
\pgfusepath{stroke}%
\end{pgfscope}%
\begin{pgfscope}%
\pgfpathrectangle{\pgfqpoint{0.708220in}{0.535823in}}{\pgfqpoint{5.045427in}{1.769453in}}%
\pgfusepath{clip}%
\pgfsetbuttcap%
\pgfsetroundjoin%
\pgfsetlinewidth{1.003750pt}%
\definecolor{currentstroke}{rgb}{0.580392,0.403922,0.741176}%
\pgfsetstrokecolor{currentstroke}%
\pgfsetdash{{3.700000pt}{1.600000pt}}{0.000000pt}%
\pgfpathmoveto{\pgfqpoint{0.846969in}{0.552230in}}%
\pgfpathlineto{\pgfqpoint{0.859583in}{0.570603in}}%
\pgfpathlineto{\pgfqpoint{0.872196in}{0.571880in}}%
\pgfpathlineto{\pgfqpoint{0.897423in}{0.608319in}}%
\pgfpathlineto{\pgfqpoint{0.910037in}{0.609657in}}%
\pgfpathlineto{\pgfqpoint{0.922650in}{0.619869in}}%
\pgfpathlineto{\pgfqpoint{0.935264in}{0.620114in}}%
\pgfpathlineto{\pgfqpoint{0.947878in}{0.625141in}}%
\pgfpathlineto{\pgfqpoint{0.960491in}{0.656505in}}%
\pgfpathlineto{\pgfqpoint{0.973105in}{0.668500in}}%
\pgfpathlineto{\pgfqpoint{0.985718in}{0.687909in}}%
\pgfpathlineto{\pgfqpoint{0.998332in}{0.696892in}}%
\pgfpathlineto{\pgfqpoint{1.010945in}{0.699214in}}%
\pgfpathlineto{\pgfqpoint{1.086627in}{0.703544in}}%
\pgfpathlineto{\pgfqpoint{1.099240in}{0.706227in}}%
\pgfpathlineto{\pgfqpoint{1.162308in}{0.708793in}}%
\pgfpathlineto{\pgfqpoint{1.187535in}{0.711423in}}%
\pgfpathlineto{\pgfqpoint{1.212763in}{0.712459in}}%
\pgfpathlineto{\pgfqpoint{1.225376in}{0.737031in}}%
\pgfpathlineto{\pgfqpoint{1.237990in}{0.737312in}}%
\pgfpathlineto{\pgfqpoint{1.250603in}{0.747530in}}%
\pgfpathlineto{\pgfqpoint{1.263217in}{0.753731in}}%
\pgfpathlineto{\pgfqpoint{1.275830in}{0.754302in}}%
\pgfpathlineto{\pgfqpoint{1.288444in}{0.760872in}}%
\pgfpathlineto{\pgfqpoint{1.326285in}{0.763390in}}%
\pgfpathlineto{\pgfqpoint{1.338898in}{0.770572in}}%
\pgfpathlineto{\pgfqpoint{1.351512in}{0.770608in}}%
\pgfpathlineto{\pgfqpoint{1.364125in}{0.776443in}}%
\pgfpathlineto{\pgfqpoint{1.376739in}{0.779788in}}%
\pgfpathlineto{\pgfqpoint{1.389352in}{0.781068in}}%
\pgfpathlineto{\pgfqpoint{1.401966in}{0.784227in}}%
\pgfpathlineto{\pgfqpoint{1.414580in}{0.797594in}}%
\pgfpathlineto{\pgfqpoint{1.427193in}{0.805782in}}%
\pgfpathlineto{\pgfqpoint{1.439807in}{0.808498in}}%
\pgfpathlineto{\pgfqpoint{1.452420in}{0.808992in}}%
\pgfpathlineto{\pgfqpoint{1.477647in}{0.814551in}}%
\pgfpathlineto{\pgfqpoint{1.502875in}{0.821261in}}%
\pgfpathlineto{\pgfqpoint{1.515488in}{0.832708in}}%
\pgfpathlineto{\pgfqpoint{1.528102in}{0.838185in}}%
\pgfpathlineto{\pgfqpoint{1.578556in}{0.850981in}}%
\pgfpathlineto{\pgfqpoint{1.591170in}{0.854411in}}%
\pgfpathlineto{\pgfqpoint{1.603783in}{0.861418in}}%
\pgfpathlineto{\pgfqpoint{1.616397in}{0.864554in}}%
\pgfpathlineto{\pgfqpoint{1.641624in}{0.867499in}}%
\pgfpathlineto{\pgfqpoint{1.654237in}{0.868587in}}%
\pgfpathlineto{\pgfqpoint{1.666851in}{0.873872in}}%
\pgfpathlineto{\pgfqpoint{1.679464in}{0.874024in}}%
\pgfpathlineto{\pgfqpoint{1.729919in}{0.880377in}}%
\pgfpathlineto{\pgfqpoint{1.755146in}{0.881901in}}%
\pgfpathlineto{\pgfqpoint{1.792987in}{0.887635in}}%
\pgfpathlineto{\pgfqpoint{1.805600in}{0.893267in}}%
\pgfpathlineto{\pgfqpoint{1.818214in}{0.896402in}}%
\pgfpathlineto{\pgfqpoint{1.830827in}{0.901177in}}%
\pgfpathlineto{\pgfqpoint{1.856054in}{0.903939in}}%
\pgfpathlineto{\pgfqpoint{1.881282in}{0.916965in}}%
\pgfpathlineto{\pgfqpoint{1.906509in}{0.926047in}}%
\pgfpathlineto{\pgfqpoint{1.919122in}{0.931313in}}%
\pgfpathlineto{\pgfqpoint{1.931736in}{0.932403in}}%
\pgfpathlineto{\pgfqpoint{1.944349in}{0.934611in}}%
\pgfpathlineto{\pgfqpoint{1.956963in}{0.942143in}}%
\pgfpathlineto{\pgfqpoint{1.982190in}{0.950407in}}%
\pgfpathlineto{\pgfqpoint{2.007417in}{0.951607in}}%
\pgfpathlineto{\pgfqpoint{2.020031in}{0.959296in}}%
\pgfpathlineto{\pgfqpoint{2.032644in}{0.962966in}}%
\pgfpathlineto{\pgfqpoint{2.045258in}{0.970496in}}%
\pgfpathlineto{\pgfqpoint{2.057871in}{0.976376in}}%
\pgfpathlineto{\pgfqpoint{2.083099in}{0.981815in}}%
\pgfpathlineto{\pgfqpoint{2.133553in}{0.985938in}}%
\pgfpathlineto{\pgfqpoint{2.146166in}{0.997680in}}%
\pgfpathlineto{\pgfqpoint{2.158780in}{1.005878in}}%
\pgfpathlineto{\pgfqpoint{2.171394in}{1.006496in}}%
\pgfpathlineto{\pgfqpoint{2.184007in}{1.020199in}}%
\pgfpathlineto{\pgfqpoint{2.209234in}{1.025267in}}%
\pgfpathlineto{\pgfqpoint{2.234461in}{1.025957in}}%
\pgfpathlineto{\pgfqpoint{2.247075in}{1.028577in}}%
\pgfpathlineto{\pgfqpoint{2.259689in}{1.029261in}}%
\pgfpathlineto{\pgfqpoint{2.284916in}{1.032262in}}%
\pgfpathlineto{\pgfqpoint{2.310143in}{1.034739in}}%
\pgfpathlineto{\pgfqpoint{2.322756in}{1.043892in}}%
\pgfpathlineto{\pgfqpoint{2.335370in}{1.055290in}}%
\pgfpathlineto{\pgfqpoint{2.347984in}{1.056500in}}%
\pgfpathlineto{\pgfqpoint{2.360597in}{1.059174in}}%
\pgfpathlineto{\pgfqpoint{2.373211in}{1.067964in}}%
\pgfpathlineto{\pgfqpoint{2.436278in}{1.073974in}}%
\pgfpathlineto{\pgfqpoint{2.448892in}{1.074523in}}%
\pgfpathlineto{\pgfqpoint{2.461506in}{1.078100in}}%
\pgfpathlineto{\pgfqpoint{2.474119in}{1.078832in}}%
\pgfpathlineto{\pgfqpoint{2.486733in}{1.084047in}}%
\pgfpathlineto{\pgfqpoint{2.499346in}{1.087497in}}%
\pgfpathlineto{\pgfqpoint{2.511960in}{1.097562in}}%
\pgfpathlineto{\pgfqpoint{2.524573in}{1.106089in}}%
\pgfpathlineto{\pgfqpoint{2.575028in}{1.108699in}}%
\pgfpathlineto{\pgfqpoint{2.587641in}{1.109010in}}%
\pgfpathlineto{\pgfqpoint{2.600255in}{1.111274in}}%
\pgfpathlineto{\pgfqpoint{2.612868in}{1.118244in}}%
\pgfpathlineto{\pgfqpoint{2.625482in}{1.122483in}}%
\pgfpathlineto{\pgfqpoint{2.638096in}{1.124964in}}%
\pgfpathlineto{\pgfqpoint{2.650709in}{1.125277in}}%
\pgfpathlineto{\pgfqpoint{2.663323in}{1.128033in}}%
\pgfpathlineto{\pgfqpoint{2.675936in}{1.132489in}}%
\pgfpathlineto{\pgfqpoint{2.688550in}{1.134837in}}%
\pgfpathlineto{\pgfqpoint{2.701163in}{1.135533in}}%
\pgfpathlineto{\pgfqpoint{2.713777in}{1.138551in}}%
\pgfpathlineto{\pgfqpoint{2.726390in}{1.140071in}}%
\pgfpathlineto{\pgfqpoint{2.751618in}{1.147966in}}%
\pgfpathlineto{\pgfqpoint{2.764231in}{1.154055in}}%
\pgfpathlineto{\pgfqpoint{2.776845in}{1.154639in}}%
\pgfpathlineto{\pgfqpoint{2.789458in}{1.166950in}}%
\pgfpathlineto{\pgfqpoint{2.802072in}{1.174548in}}%
\pgfpathlineto{\pgfqpoint{2.814685in}{1.176049in}}%
\pgfpathlineto{\pgfqpoint{2.827299in}{1.181338in}}%
\pgfpathlineto{\pgfqpoint{2.839913in}{1.185403in}}%
\pgfpathlineto{\pgfqpoint{2.852526in}{1.195512in}}%
\pgfpathlineto{\pgfqpoint{2.865140in}{1.195703in}}%
\pgfpathlineto{\pgfqpoint{2.877753in}{1.199502in}}%
\pgfpathlineto{\pgfqpoint{2.890367in}{1.200759in}}%
\pgfpathlineto{\pgfqpoint{2.902980in}{1.204233in}}%
\pgfpathlineto{\pgfqpoint{2.928208in}{1.206425in}}%
\pgfpathlineto{\pgfqpoint{2.940821in}{1.208665in}}%
\pgfpathlineto{\pgfqpoint{2.953435in}{1.214723in}}%
\pgfpathlineto{\pgfqpoint{2.966048in}{1.223033in}}%
\pgfpathlineto{\pgfqpoint{2.978662in}{1.234960in}}%
\pgfpathlineto{\pgfqpoint{2.991275in}{1.236582in}}%
\pgfpathlineto{\pgfqpoint{3.003889in}{1.241828in}}%
\pgfpathlineto{\pgfqpoint{3.016503in}{1.250956in}}%
\pgfpathlineto{\pgfqpoint{3.029116in}{1.251184in}}%
\pgfpathlineto{\pgfqpoint{3.041730in}{1.257530in}}%
\pgfpathlineto{\pgfqpoint{3.054343in}{1.258132in}}%
\pgfpathlineto{\pgfqpoint{3.066957in}{1.262098in}}%
\pgfpathlineto{\pgfqpoint{3.079570in}{1.264091in}}%
\pgfpathlineto{\pgfqpoint{3.092184in}{1.264425in}}%
\pgfpathlineto{\pgfqpoint{3.104797in}{1.269855in}}%
\pgfpathlineto{\pgfqpoint{3.117411in}{1.269970in}}%
\pgfpathlineto{\pgfqpoint{3.142638in}{1.282290in}}%
\pgfpathlineto{\pgfqpoint{3.155252in}{1.285275in}}%
\pgfpathlineto{\pgfqpoint{3.167865in}{1.293288in}}%
\pgfpathlineto{\pgfqpoint{3.180479in}{1.293652in}}%
\pgfpathlineto{\pgfqpoint{3.193092in}{1.297453in}}%
\pgfpathlineto{\pgfqpoint{3.205706in}{1.305378in}}%
\pgfpathlineto{\pgfqpoint{3.218320in}{1.325956in}}%
\pgfpathlineto{\pgfqpoint{3.256160in}{1.327652in}}%
\pgfpathlineto{\pgfqpoint{3.281387in}{1.336976in}}%
\pgfpathlineto{\pgfqpoint{3.319228in}{1.337708in}}%
\pgfpathlineto{\pgfqpoint{3.331842in}{1.339969in}}%
\pgfpathlineto{\pgfqpoint{3.344455in}{1.368793in}}%
\pgfpathlineto{\pgfqpoint{3.357069in}{1.368822in}}%
\pgfpathlineto{\pgfqpoint{3.369682in}{1.375861in}}%
\pgfpathlineto{\pgfqpoint{3.382296in}{1.376095in}}%
\pgfpathlineto{\pgfqpoint{3.420137in}{1.393194in}}%
\pgfpathlineto{\pgfqpoint{3.432750in}{1.396703in}}%
\pgfpathlineto{\pgfqpoint{3.445364in}{1.404516in}}%
\pgfpathlineto{\pgfqpoint{3.457977in}{1.410219in}}%
\pgfpathlineto{\pgfqpoint{3.483204in}{1.411240in}}%
\pgfpathlineto{\pgfqpoint{3.495818in}{1.416537in}}%
\pgfpathlineto{\pgfqpoint{3.508432in}{1.416857in}}%
\pgfpathlineto{\pgfqpoint{3.521045in}{1.446644in}}%
\pgfpathlineto{\pgfqpoint{3.533659in}{1.454322in}}%
\pgfpathlineto{\pgfqpoint{3.558886in}{1.464424in}}%
\pgfpathlineto{\pgfqpoint{3.571499in}{1.464645in}}%
\pgfpathlineto{\pgfqpoint{3.584113in}{1.467037in}}%
\pgfpathlineto{\pgfqpoint{3.609340in}{1.467859in}}%
\pgfpathlineto{\pgfqpoint{3.621954in}{1.471507in}}%
\pgfpathlineto{\pgfqpoint{3.634567in}{1.476830in}}%
\pgfpathlineto{\pgfqpoint{3.647181in}{1.477101in}}%
\pgfpathlineto{\pgfqpoint{3.659794in}{1.481168in}}%
\pgfpathlineto{\pgfqpoint{3.685022in}{1.483818in}}%
\pgfpathlineto{\pgfqpoint{3.710249in}{1.484057in}}%
\pgfpathlineto{\pgfqpoint{3.722862in}{1.487800in}}%
\pgfpathlineto{\pgfqpoint{3.748089in}{1.489037in}}%
\pgfpathlineto{\pgfqpoint{3.760703in}{1.489991in}}%
\pgfpathlineto{\pgfqpoint{3.773317in}{1.503495in}}%
\pgfpathlineto{\pgfqpoint{3.785930in}{1.542089in}}%
\pgfpathlineto{\pgfqpoint{3.798544in}{1.557106in}}%
\pgfpathlineto{\pgfqpoint{3.836384in}{1.576081in}}%
\pgfpathlineto{\pgfqpoint{3.848998in}{1.580378in}}%
\pgfpathlineto{\pgfqpoint{3.861611in}{1.581009in}}%
\pgfpathlineto{\pgfqpoint{3.874225in}{1.592154in}}%
\pgfpathlineto{\pgfqpoint{3.886839in}{1.594665in}}%
\pgfpathlineto{\pgfqpoint{3.899452in}{1.595401in}}%
\pgfpathlineto{\pgfqpoint{3.912066in}{1.607566in}}%
\pgfpathlineto{\pgfqpoint{3.949906in}{1.612144in}}%
\pgfpathlineto{\pgfqpoint{3.975134in}{1.613399in}}%
\pgfpathlineto{\pgfqpoint{3.987747in}{1.617079in}}%
\pgfpathlineto{\pgfqpoint{4.000361in}{1.622812in}}%
\pgfpathlineto{\pgfqpoint{4.012974in}{1.625178in}}%
\pgfpathlineto{\pgfqpoint{4.025588in}{1.625723in}}%
\pgfpathlineto{\pgfqpoint{4.038201in}{1.638780in}}%
\pgfpathlineto{\pgfqpoint{4.050815in}{1.647910in}}%
\pgfpathlineto{\pgfqpoint{4.076042in}{1.653592in}}%
\pgfpathlineto{\pgfqpoint{4.088656in}{1.653777in}}%
\pgfpathlineto{\pgfqpoint{4.101269in}{1.663768in}}%
\pgfpathlineto{\pgfqpoint{4.113883in}{1.665956in}}%
\pgfpathlineto{\pgfqpoint{4.126496in}{1.666104in}}%
\pgfpathlineto{\pgfqpoint{4.139110in}{1.674107in}}%
\pgfpathlineto{\pgfqpoint{4.164337in}{1.674601in}}%
\pgfpathlineto{\pgfqpoint{4.189564in}{1.711710in}}%
\pgfpathlineto{\pgfqpoint{4.202178in}{1.717785in}}%
\pgfpathlineto{\pgfqpoint{4.214791in}{1.727804in}}%
\pgfpathlineto{\pgfqpoint{4.227405in}{1.728081in}}%
\pgfpathlineto{\pgfqpoint{4.240018in}{1.730867in}}%
\pgfpathlineto{\pgfqpoint{4.252632in}{1.766748in}}%
\pgfpathlineto{\pgfqpoint{4.265246in}{1.766855in}}%
\pgfpathlineto{\pgfqpoint{4.277859in}{1.768482in}}%
\pgfpathlineto{\pgfqpoint{4.303086in}{1.769208in}}%
\pgfpathlineto{\pgfqpoint{4.315700in}{1.771174in}}%
\pgfpathlineto{\pgfqpoint{4.328313in}{1.786575in}}%
\pgfpathlineto{\pgfqpoint{4.340927in}{1.795504in}}%
\pgfpathlineto{\pgfqpoint{4.353541in}{1.796667in}}%
\pgfpathlineto{\pgfqpoint{4.366154in}{1.842477in}}%
\pgfpathlineto{\pgfqpoint{4.378768in}{1.843636in}}%
\pgfpathlineto{\pgfqpoint{4.391381in}{1.847565in}}%
\pgfpathlineto{\pgfqpoint{4.403995in}{1.847596in}}%
\pgfpathlineto{\pgfqpoint{4.416608in}{1.876164in}}%
\pgfpathlineto{\pgfqpoint{4.429222in}{1.880860in}}%
\pgfpathlineto{\pgfqpoint{4.441836in}{1.882711in}}%
\pgfpathlineto{\pgfqpoint{4.467063in}{1.902003in}}%
\pgfpathlineto{\pgfqpoint{4.492290in}{1.903492in}}%
\pgfpathlineto{\pgfqpoint{4.504903in}{1.938309in}}%
\pgfpathlineto{\pgfqpoint{4.517517in}{1.939372in}}%
\pgfpathlineto{\pgfqpoint{4.530131in}{1.954544in}}%
\pgfpathlineto{\pgfqpoint{4.542744in}{1.964985in}}%
\pgfpathlineto{\pgfqpoint{4.555358in}{1.967320in}}%
\pgfpathlineto{\pgfqpoint{4.567971in}{1.996487in}}%
\pgfpathlineto{\pgfqpoint{4.580585in}{1.996549in}}%
\pgfpathlineto{\pgfqpoint{4.593198in}{2.000321in}}%
\pgfpathlineto{\pgfqpoint{4.605812in}{2.000630in}}%
\pgfpathlineto{\pgfqpoint{4.618425in}{2.030447in}}%
\pgfpathlineto{\pgfqpoint{4.631039in}{2.041523in}}%
\pgfpathlineto{\pgfqpoint{4.668880in}{2.046620in}}%
\pgfpathlineto{\pgfqpoint{4.681493in}{2.062032in}}%
\pgfpathlineto{\pgfqpoint{4.694107in}{2.073839in}}%
\pgfpathlineto{\pgfqpoint{4.706720in}{2.075880in}}%
\pgfpathlineto{\pgfqpoint{4.719334in}{2.076506in}}%
\pgfpathlineto{\pgfqpoint{4.731948in}{2.087166in}}%
\pgfpathlineto{\pgfqpoint{4.744561in}{2.106026in}}%
\pgfpathlineto{\pgfqpoint{4.757175in}{2.106760in}}%
\pgfpathlineto{\pgfqpoint{4.769788in}{2.116146in}}%
\pgfpathlineto{\pgfqpoint{4.782402in}{2.117706in}}%
\pgfpathlineto{\pgfqpoint{4.795015in}{2.120772in}}%
\pgfpathlineto{\pgfqpoint{4.820243in}{2.121880in}}%
\pgfpathlineto{\pgfqpoint{4.832856in}{2.126655in}}%
\pgfpathlineto{\pgfqpoint{4.858083in}{2.132965in}}%
\pgfpathlineto{\pgfqpoint{4.870697in}{2.154631in}}%
\pgfpathlineto{\pgfqpoint{4.883310in}{2.155491in}}%
\pgfpathlineto{\pgfqpoint{4.895924in}{2.162066in}}%
\pgfpathlineto{\pgfqpoint{4.908538in}{2.170510in}}%
\pgfpathlineto{\pgfqpoint{4.921151in}{2.183989in}}%
\pgfpathlineto{\pgfqpoint{4.933765in}{2.194301in}}%
\pgfpathlineto{\pgfqpoint{4.958992in}{2.195364in}}%
\pgfpathlineto{\pgfqpoint{4.971605in}{2.219943in}}%
\pgfpathlineto{\pgfqpoint{4.984219in}{2.225812in}}%
\pgfpathlineto{\pgfqpoint{4.996832in}{2.229674in}}%
\pgfpathlineto{\pgfqpoint{5.009446in}{2.239699in}}%
\pgfpathlineto{\pgfqpoint{5.022060in}{2.263624in}}%
\pgfpathlineto{\pgfqpoint{5.034673in}{2.266698in}}%
\pgfpathlineto{\pgfqpoint{5.047287in}{2.266887in}}%
\pgfpathlineto{\pgfqpoint{5.059900in}{2.290341in}}%
\pgfpathlineto{\pgfqpoint{5.072514in}{2.305275in}}%
\pgfpathlineto{\pgfqpoint{5.072514in}{2.305275in}}%
\pgfusepath{stroke}%
\end{pgfscope}%
\begin{pgfscope}%
\pgfpathrectangle{\pgfqpoint{0.708220in}{0.535823in}}{\pgfqpoint{5.045427in}{1.769453in}}%
\pgfusepath{clip}%
\pgfsetbuttcap%
\pgfsetroundjoin%
\pgfsetlinewidth{1.003750pt}%
\definecolor{currentstroke}{rgb}{0.549020,0.337255,0.294118}%
\pgfsetstrokecolor{currentstroke}%
\pgfsetdash{{1.000000pt}{1.650000pt}}{0.000000pt}%
\pgfpathmoveto{\pgfqpoint{0.846969in}{0.552230in}}%
\pgfpathlineto{\pgfqpoint{0.859583in}{0.570603in}}%
\pgfpathlineto{\pgfqpoint{0.872196in}{0.571880in}}%
\pgfpathlineto{\pgfqpoint{0.897423in}{0.608319in}}%
\pgfpathlineto{\pgfqpoint{0.910037in}{0.609657in}}%
\pgfpathlineto{\pgfqpoint{0.922650in}{0.619869in}}%
\pgfpathlineto{\pgfqpoint{0.935264in}{0.620114in}}%
\pgfpathlineto{\pgfqpoint{0.947878in}{0.625141in}}%
\pgfpathlineto{\pgfqpoint{0.960491in}{0.656505in}}%
\pgfpathlineto{\pgfqpoint{0.973105in}{0.668500in}}%
\pgfpathlineto{\pgfqpoint{0.985718in}{0.687909in}}%
\pgfpathlineto{\pgfqpoint{0.998332in}{0.696892in}}%
\pgfpathlineto{\pgfqpoint{1.010945in}{0.699214in}}%
\pgfpathlineto{\pgfqpoint{1.086627in}{0.703544in}}%
\pgfpathlineto{\pgfqpoint{1.099240in}{0.706227in}}%
\pgfpathlineto{\pgfqpoint{1.162308in}{0.708793in}}%
\pgfpathlineto{\pgfqpoint{1.187535in}{0.711423in}}%
\pgfpathlineto{\pgfqpoint{1.212763in}{0.712459in}}%
\pgfpathlineto{\pgfqpoint{1.225376in}{0.737031in}}%
\pgfpathlineto{\pgfqpoint{1.237990in}{0.737312in}}%
\pgfpathlineto{\pgfqpoint{1.250603in}{0.747530in}}%
\pgfpathlineto{\pgfqpoint{1.263217in}{0.753731in}}%
\pgfpathlineto{\pgfqpoint{1.275830in}{0.754302in}}%
\pgfpathlineto{\pgfqpoint{1.288444in}{0.760872in}}%
\pgfpathlineto{\pgfqpoint{1.326285in}{0.763390in}}%
\pgfpathlineto{\pgfqpoint{1.338898in}{0.770572in}}%
\pgfpathlineto{\pgfqpoint{1.351512in}{0.770608in}}%
\pgfpathlineto{\pgfqpoint{1.364125in}{0.776443in}}%
\pgfpathlineto{\pgfqpoint{1.376739in}{0.779788in}}%
\pgfpathlineto{\pgfqpoint{1.389352in}{0.781068in}}%
\pgfpathlineto{\pgfqpoint{1.401966in}{0.784227in}}%
\pgfpathlineto{\pgfqpoint{1.414580in}{0.797594in}}%
\pgfpathlineto{\pgfqpoint{1.427193in}{0.805782in}}%
\pgfpathlineto{\pgfqpoint{1.439807in}{0.808498in}}%
\pgfpathlineto{\pgfqpoint{1.452420in}{0.808992in}}%
\pgfpathlineto{\pgfqpoint{1.477647in}{0.814551in}}%
\pgfpathlineto{\pgfqpoint{1.502875in}{0.821261in}}%
\pgfpathlineto{\pgfqpoint{1.515488in}{0.832708in}}%
\pgfpathlineto{\pgfqpoint{1.528102in}{0.838185in}}%
\pgfpathlineto{\pgfqpoint{1.578556in}{0.850981in}}%
\pgfpathlineto{\pgfqpoint{1.591170in}{0.854411in}}%
\pgfpathlineto{\pgfqpoint{1.603783in}{0.861418in}}%
\pgfpathlineto{\pgfqpoint{1.616397in}{0.864554in}}%
\pgfpathlineto{\pgfqpoint{1.641624in}{0.867499in}}%
\pgfpathlineto{\pgfqpoint{1.654237in}{0.868587in}}%
\pgfpathlineto{\pgfqpoint{1.666851in}{0.873872in}}%
\pgfpathlineto{\pgfqpoint{1.679464in}{0.874024in}}%
\pgfpathlineto{\pgfqpoint{1.729919in}{0.880377in}}%
\pgfpathlineto{\pgfqpoint{1.755146in}{0.881901in}}%
\pgfpathlineto{\pgfqpoint{1.792987in}{0.887635in}}%
\pgfpathlineto{\pgfqpoint{1.805600in}{0.893267in}}%
\pgfpathlineto{\pgfqpoint{1.818214in}{0.896402in}}%
\pgfpathlineto{\pgfqpoint{1.830827in}{0.901177in}}%
\pgfpathlineto{\pgfqpoint{1.856054in}{0.903939in}}%
\pgfpathlineto{\pgfqpoint{1.881282in}{0.916965in}}%
\pgfpathlineto{\pgfqpoint{1.906509in}{0.926047in}}%
\pgfpathlineto{\pgfqpoint{1.919122in}{0.931313in}}%
\pgfpathlineto{\pgfqpoint{1.931736in}{0.932403in}}%
\pgfpathlineto{\pgfqpoint{1.944349in}{0.934611in}}%
\pgfpathlineto{\pgfqpoint{1.956963in}{0.942143in}}%
\pgfpathlineto{\pgfqpoint{1.982190in}{0.950407in}}%
\pgfpathlineto{\pgfqpoint{2.007417in}{0.951607in}}%
\pgfpathlineto{\pgfqpoint{2.020031in}{0.959296in}}%
\pgfpathlineto{\pgfqpoint{2.032644in}{0.962966in}}%
\pgfpathlineto{\pgfqpoint{2.045258in}{0.970496in}}%
\pgfpathlineto{\pgfqpoint{2.057871in}{0.976376in}}%
\pgfpathlineto{\pgfqpoint{2.083099in}{0.981815in}}%
\pgfpathlineto{\pgfqpoint{2.133553in}{0.985938in}}%
\pgfpathlineto{\pgfqpoint{2.146166in}{0.997680in}}%
\pgfpathlineto{\pgfqpoint{2.158780in}{1.005878in}}%
\pgfpathlineto{\pgfqpoint{2.171394in}{1.006496in}}%
\pgfpathlineto{\pgfqpoint{2.184007in}{1.020199in}}%
\pgfpathlineto{\pgfqpoint{2.209234in}{1.025267in}}%
\pgfpathlineto{\pgfqpoint{2.234461in}{1.025957in}}%
\pgfpathlineto{\pgfqpoint{2.247075in}{1.028577in}}%
\pgfpathlineto{\pgfqpoint{2.259689in}{1.029261in}}%
\pgfpathlineto{\pgfqpoint{2.284916in}{1.032262in}}%
\pgfpathlineto{\pgfqpoint{2.310143in}{1.034739in}}%
\pgfpathlineto{\pgfqpoint{2.322756in}{1.043892in}}%
\pgfpathlineto{\pgfqpoint{2.335370in}{1.055290in}}%
\pgfpathlineto{\pgfqpoint{2.347984in}{1.056500in}}%
\pgfpathlineto{\pgfqpoint{2.360597in}{1.059174in}}%
\pgfpathlineto{\pgfqpoint{2.373211in}{1.067964in}}%
\pgfpathlineto{\pgfqpoint{2.436278in}{1.073974in}}%
\pgfpathlineto{\pgfqpoint{2.448892in}{1.074523in}}%
\pgfpathlineto{\pgfqpoint{2.461506in}{1.078100in}}%
\pgfpathlineto{\pgfqpoint{2.474119in}{1.078832in}}%
\pgfpathlineto{\pgfqpoint{2.511960in}{1.087497in}}%
\pgfpathlineto{\pgfqpoint{2.524573in}{1.097562in}}%
\pgfpathlineto{\pgfqpoint{2.537187in}{1.106089in}}%
\pgfpathlineto{\pgfqpoint{2.587641in}{1.108699in}}%
\pgfpathlineto{\pgfqpoint{2.600255in}{1.109010in}}%
\pgfpathlineto{\pgfqpoint{2.612868in}{1.111274in}}%
\pgfpathlineto{\pgfqpoint{2.625482in}{1.118244in}}%
\pgfpathlineto{\pgfqpoint{2.638096in}{1.122483in}}%
\pgfpathlineto{\pgfqpoint{2.650709in}{1.124964in}}%
\pgfpathlineto{\pgfqpoint{2.663323in}{1.125277in}}%
\pgfpathlineto{\pgfqpoint{2.675936in}{1.128033in}}%
\pgfpathlineto{\pgfqpoint{2.688550in}{1.132489in}}%
\pgfpathlineto{\pgfqpoint{2.701163in}{1.134837in}}%
\pgfpathlineto{\pgfqpoint{2.713777in}{1.135533in}}%
\pgfpathlineto{\pgfqpoint{2.726390in}{1.138551in}}%
\pgfpathlineto{\pgfqpoint{2.739004in}{1.140071in}}%
\pgfpathlineto{\pgfqpoint{2.764231in}{1.147966in}}%
\pgfpathlineto{\pgfqpoint{2.776845in}{1.154055in}}%
\pgfpathlineto{\pgfqpoint{2.789458in}{1.154639in}}%
\pgfpathlineto{\pgfqpoint{2.802072in}{1.166950in}}%
\pgfpathlineto{\pgfqpoint{2.814685in}{1.174548in}}%
\pgfpathlineto{\pgfqpoint{2.827299in}{1.176049in}}%
\pgfpathlineto{\pgfqpoint{2.839913in}{1.181338in}}%
\pgfpathlineto{\pgfqpoint{2.852526in}{1.185403in}}%
\pgfpathlineto{\pgfqpoint{2.865140in}{1.195512in}}%
\pgfpathlineto{\pgfqpoint{2.877753in}{1.195703in}}%
\pgfpathlineto{\pgfqpoint{2.890367in}{1.199502in}}%
\pgfpathlineto{\pgfqpoint{2.966048in}{1.206248in}}%
\pgfpathlineto{\pgfqpoint{3.003889in}{1.207244in}}%
\pgfpathlineto{\pgfqpoint{3.029116in}{1.209211in}}%
\pgfpathlineto{\pgfqpoint{3.117411in}{1.215856in}}%
\pgfpathlineto{\pgfqpoint{3.155252in}{1.216777in}}%
\pgfpathlineto{\pgfqpoint{3.180479in}{1.217690in}}%
\pgfpathlineto{\pgfqpoint{3.205706in}{1.218597in}}%
\pgfpathlineto{\pgfqpoint{3.243547in}{1.220391in}}%
\pgfpathlineto{\pgfqpoint{3.256160in}{1.220391in}}%
\pgfpathlineto{\pgfqpoint{3.319228in}{1.233062in}}%
\pgfpathlineto{\pgfqpoint{3.369682in}{1.233862in}}%
\pgfpathlineto{\pgfqpoint{3.445364in}{1.238553in}}%
\pgfpathlineto{\pgfqpoint{3.470591in}{1.244540in}}%
\pgfpathlineto{\pgfqpoint{3.483204in}{1.249548in}}%
\pgfpathlineto{\pgfqpoint{3.508432in}{1.253003in}}%
\pgfpathlineto{\pgfqpoint{3.533659in}{1.258335in}}%
\pgfpathlineto{\pgfqpoint{3.584113in}{1.260917in}}%
\pgfpathlineto{\pgfqpoint{3.609340in}{1.264425in}}%
\pgfpathlineto{\pgfqpoint{3.634567in}{1.265921in}}%
\pgfpathlineto{\pgfqpoint{3.659794in}{1.266532in}}%
\pgfpathlineto{\pgfqpoint{3.672408in}{1.269543in}}%
\pgfpathlineto{\pgfqpoint{3.722862in}{1.271900in}}%
\pgfpathlineto{\pgfqpoint{3.735476in}{1.275918in}}%
\pgfpathlineto{\pgfqpoint{3.785930in}{1.277042in}}%
\pgfpathlineto{\pgfqpoint{3.811157in}{1.281437in}}%
\pgfpathlineto{\pgfqpoint{3.848998in}{1.283577in}}%
\pgfpathlineto{\pgfqpoint{3.861611in}{1.283577in}}%
\pgfpathlineto{\pgfqpoint{3.874225in}{1.285275in}}%
\pgfpathlineto{\pgfqpoint{3.899452in}{1.293288in}}%
\pgfpathlineto{\pgfqpoint{3.924679in}{1.294233in}}%
\pgfpathlineto{\pgfqpoint{3.937293in}{1.297453in}}%
\pgfpathlineto{\pgfqpoint{3.949906in}{1.298060in}}%
\pgfpathlineto{\pgfqpoint{4.000361in}{1.304926in}}%
\pgfpathlineto{\pgfqpoint{4.025588in}{1.305378in}}%
\pgfpathlineto{\pgfqpoint{4.038201in}{1.321544in}}%
\pgfpathlineto{\pgfqpoint{4.050815in}{1.323088in}}%
\pgfpathlineto{\pgfqpoint{4.063429in}{1.325956in}}%
\pgfpathlineto{\pgfqpoint{4.076042in}{1.326118in}}%
\pgfpathlineto{\pgfqpoint{4.088656in}{1.336512in}}%
\pgfpathlineto{\pgfqpoint{4.101269in}{1.336976in}}%
\pgfpathlineto{\pgfqpoint{4.113883in}{1.339969in}}%
\pgfpathlineto{\pgfqpoint{4.126496in}{1.344168in}}%
\pgfpathlineto{\pgfqpoint{4.151724in}{1.346089in}}%
\pgfpathlineto{\pgfqpoint{4.164337in}{1.353781in}}%
\pgfpathlineto{\pgfqpoint{4.176951in}{1.354968in}}%
\pgfpathlineto{\pgfqpoint{4.189564in}{1.360453in}}%
\pgfpathlineto{\pgfqpoint{4.202178in}{1.375319in}}%
\pgfpathlineto{\pgfqpoint{4.214791in}{1.375568in}}%
\pgfpathlineto{\pgfqpoint{4.227405in}{1.389049in}}%
\pgfpathlineto{\pgfqpoint{4.240018in}{1.393187in}}%
\pgfpathlineto{\pgfqpoint{4.252632in}{1.393194in}}%
\pgfpathlineto{\pgfqpoint{4.265246in}{1.396703in}}%
\pgfpathlineto{\pgfqpoint{4.277859in}{1.403330in}}%
\pgfpathlineto{\pgfqpoint{4.290473in}{1.404516in}}%
\pgfpathlineto{\pgfqpoint{4.315700in}{1.415422in}}%
\pgfpathlineto{\pgfqpoint{4.328313in}{1.430624in}}%
\pgfpathlineto{\pgfqpoint{4.353541in}{1.446644in}}%
\pgfpathlineto{\pgfqpoint{4.366154in}{1.452780in}}%
\pgfpathlineto{\pgfqpoint{4.378768in}{1.454322in}}%
\pgfpathlineto{\pgfqpoint{4.391381in}{1.459493in}}%
\pgfpathlineto{\pgfqpoint{4.403995in}{1.481168in}}%
\pgfpathlineto{\pgfqpoint{4.416608in}{1.483947in}}%
\pgfpathlineto{\pgfqpoint{4.429222in}{1.484057in}}%
\pgfpathlineto{\pgfqpoint{4.441836in}{1.487800in}}%
\pgfpathlineto{\pgfqpoint{4.454449in}{1.488054in}}%
\pgfpathlineto{\pgfqpoint{4.467063in}{1.508552in}}%
\pgfpathlineto{\pgfqpoint{4.479676in}{1.533550in}}%
\pgfpathlineto{\pgfqpoint{4.492290in}{1.535110in}}%
\pgfpathlineto{\pgfqpoint{4.504903in}{1.542089in}}%
\pgfpathlineto{\pgfqpoint{4.517517in}{1.546311in}}%
\pgfpathlineto{\pgfqpoint{4.530131in}{1.576081in}}%
\pgfpathlineto{\pgfqpoint{4.542744in}{1.585433in}}%
\pgfpathlineto{\pgfqpoint{4.555358in}{1.607566in}}%
\pgfpathlineto{\pgfqpoint{4.567971in}{1.609914in}}%
\pgfpathlineto{\pgfqpoint{4.580585in}{1.617079in}}%
\pgfpathlineto{\pgfqpoint{4.593198in}{1.638780in}}%
\pgfpathlineto{\pgfqpoint{4.605812in}{1.647910in}}%
\pgfpathlineto{\pgfqpoint{4.618425in}{1.650432in}}%
\pgfpathlineto{\pgfqpoint{4.631039in}{1.663768in}}%
\pgfpathlineto{\pgfqpoint{4.668880in}{1.666104in}}%
\pgfpathlineto{\pgfqpoint{4.681493in}{1.689341in}}%
\pgfpathlineto{\pgfqpoint{4.706720in}{1.693436in}}%
\pgfpathlineto{\pgfqpoint{4.731948in}{1.708535in}}%
\pgfpathlineto{\pgfqpoint{4.744561in}{1.711710in}}%
\pgfpathlineto{\pgfqpoint{4.757175in}{1.716772in}}%
\pgfpathlineto{\pgfqpoint{4.769788in}{1.717785in}}%
\pgfpathlineto{\pgfqpoint{4.782402in}{1.730867in}}%
\pgfpathlineto{\pgfqpoint{4.807629in}{1.768482in}}%
\pgfpathlineto{\pgfqpoint{4.820243in}{1.771174in}}%
\pgfpathlineto{\pgfqpoint{4.832856in}{1.786575in}}%
\pgfpathlineto{\pgfqpoint{4.845470in}{1.786847in}}%
\pgfpathlineto{\pgfqpoint{4.858083in}{1.788251in}}%
\pgfpathlineto{\pgfqpoint{4.870697in}{1.842477in}}%
\pgfpathlineto{\pgfqpoint{4.895924in}{1.843636in}}%
\pgfpathlineto{\pgfqpoint{4.908538in}{1.869329in}}%
\pgfpathlineto{\pgfqpoint{4.921151in}{1.880860in}}%
\pgfpathlineto{\pgfqpoint{4.933765in}{1.882711in}}%
\pgfpathlineto{\pgfqpoint{4.946378in}{1.891753in}}%
\pgfpathlineto{\pgfqpoint{4.958992in}{1.910590in}}%
\pgfpathlineto{\pgfqpoint{4.971605in}{1.946298in}}%
\pgfpathlineto{\pgfqpoint{4.984219in}{1.954544in}}%
\pgfpathlineto{\pgfqpoint{4.996832in}{1.964985in}}%
\pgfpathlineto{\pgfqpoint{5.009446in}{1.970244in}}%
\pgfpathlineto{\pgfqpoint{5.022060in}{1.993094in}}%
\pgfpathlineto{\pgfqpoint{5.034673in}{1.996487in}}%
\pgfpathlineto{\pgfqpoint{5.047287in}{1.996549in}}%
\pgfpathlineto{\pgfqpoint{5.059900in}{2.002672in}}%
\pgfpathlineto{\pgfqpoint{5.072514in}{2.006356in}}%
\pgfpathlineto{\pgfqpoint{5.085127in}{2.021025in}}%
\pgfpathlineto{\pgfqpoint{5.097741in}{2.041523in}}%
\pgfpathlineto{\pgfqpoint{5.135582in}{2.046620in}}%
\pgfpathlineto{\pgfqpoint{5.148195in}{2.073839in}}%
\pgfpathlineto{\pgfqpoint{5.160809in}{2.075880in}}%
\pgfpathlineto{\pgfqpoint{5.173422in}{2.076506in}}%
\pgfpathlineto{\pgfqpoint{5.198650in}{2.098041in}}%
\pgfpathlineto{\pgfqpoint{5.211263in}{2.106049in}}%
\pgfpathlineto{\pgfqpoint{5.223877in}{2.106760in}}%
\pgfpathlineto{\pgfqpoint{5.236490in}{2.116146in}}%
\pgfpathlineto{\pgfqpoint{5.249104in}{2.120891in}}%
\pgfpathlineto{\pgfqpoint{5.261717in}{2.121880in}}%
\pgfpathlineto{\pgfqpoint{5.274331in}{2.126655in}}%
\pgfpathlineto{\pgfqpoint{5.286944in}{2.132965in}}%
\pgfpathlineto{\pgfqpoint{5.312172in}{2.152286in}}%
\pgfpathlineto{\pgfqpoint{5.324785in}{2.154419in}}%
\pgfpathlineto{\pgfqpoint{5.350012in}{2.155491in}}%
\pgfpathlineto{\pgfqpoint{5.362626in}{2.161329in}}%
\pgfpathlineto{\pgfqpoint{5.375239in}{2.162066in}}%
\pgfpathlineto{\pgfqpoint{5.387853in}{2.167796in}}%
\pgfpathlineto{\pgfqpoint{5.400467in}{2.168054in}}%
\pgfpathlineto{\pgfqpoint{5.413080in}{2.170510in}}%
\pgfpathlineto{\pgfqpoint{5.425694in}{2.174326in}}%
\pgfpathlineto{\pgfqpoint{5.438307in}{2.175430in}}%
\pgfpathlineto{\pgfqpoint{5.450921in}{2.183989in}}%
\pgfpathlineto{\pgfqpoint{5.463534in}{2.194301in}}%
\pgfpathlineto{\pgfqpoint{5.476148in}{2.194668in}}%
\pgfpathlineto{\pgfqpoint{5.488762in}{2.206670in}}%
\pgfpathlineto{\pgfqpoint{5.501375in}{2.210631in}}%
\pgfpathlineto{\pgfqpoint{5.513989in}{2.212381in}}%
\pgfpathlineto{\pgfqpoint{5.526602in}{2.219943in}}%
\pgfpathlineto{\pgfqpoint{5.551829in}{2.239699in}}%
\pgfpathlineto{\pgfqpoint{5.564443in}{2.247268in}}%
\pgfpathlineto{\pgfqpoint{5.577057in}{2.263624in}}%
\pgfpathlineto{\pgfqpoint{5.589670in}{2.266698in}}%
\pgfpathlineto{\pgfqpoint{5.602284in}{2.266887in}}%
\pgfpathlineto{\pgfqpoint{5.614897in}{2.281125in}}%
\pgfpathlineto{\pgfqpoint{5.627511in}{2.305275in}}%
\pgfpathlineto{\pgfqpoint{5.627511in}{2.305275in}}%
\pgfusepath{stroke}%
\end{pgfscope}%
\begin{pgfscope}%
\pgfsetrectcap%
\pgfsetmiterjoin%
\pgfsetlinewidth{0.803000pt}%
\definecolor{currentstroke}{rgb}{0.000000,0.000000,0.000000}%
\pgfsetstrokecolor{currentstroke}%
\pgfsetdash{}{0pt}%
\pgfpathmoveto{\pgfqpoint{0.708220in}{0.535823in}}%
\pgfpathlineto{\pgfqpoint{0.708220in}{2.305275in}}%
\pgfusepath{stroke}%
\end{pgfscope}%
\begin{pgfscope}%
\pgfsetrectcap%
\pgfsetmiterjoin%
\pgfsetlinewidth{0.803000pt}%
\definecolor{currentstroke}{rgb}{0.000000,0.000000,0.000000}%
\pgfsetstrokecolor{currentstroke}%
\pgfsetdash{}{0pt}%
\pgfpathmoveto{\pgfqpoint{5.753646in}{0.535823in}}%
\pgfpathlineto{\pgfqpoint{5.753646in}{2.305275in}}%
\pgfusepath{stroke}%
\end{pgfscope}%
\begin{pgfscope}%
\pgfsetrectcap%
\pgfsetmiterjoin%
\pgfsetlinewidth{0.803000pt}%
\definecolor{currentstroke}{rgb}{0.000000,0.000000,0.000000}%
\pgfsetstrokecolor{currentstroke}%
\pgfsetdash{}{0pt}%
\pgfpathmoveto{\pgfqpoint{0.708220in}{0.535823in}}%
\pgfpathlineto{\pgfqpoint{5.753646in}{0.535823in}}%
\pgfusepath{stroke}%
\end{pgfscope}%
\begin{pgfscope}%
\pgfsetrectcap%
\pgfsetmiterjoin%
\pgfsetlinewidth{0.803000pt}%
\definecolor{currentstroke}{rgb}{0.000000,0.000000,0.000000}%
\pgfsetstrokecolor{currentstroke}%
\pgfsetdash{}{0pt}%
\pgfpathmoveto{\pgfqpoint{0.708220in}{2.305275in}}%
\pgfpathlineto{\pgfqpoint{5.753646in}{2.305275in}}%
\pgfusepath{stroke}%
\end{pgfscope}%
\begin{pgfscope}%
\pgfsetrectcap%
\pgfsetroundjoin%
\pgfsetlinewidth{1.003750pt}%
\definecolor{currentstroke}{rgb}{0.121569,0.466667,0.705882}%
\pgfsetstrokecolor{currentstroke}%
\pgfsetdash{}{0pt}%
\pgfpathmoveto{\pgfqpoint{4.880710in}{1.462870in}}%
\pgfpathlineto{\pgfqpoint{5.130710in}{1.462870in}}%
\pgfusepath{stroke}%
\end{pgfscope}%
\begin{pgfscope}%
\definecolor{textcolor}{rgb}{0.000000,0.000000,0.000000}%
\pgfsetstrokecolor{textcolor}%
\pgfsetfillcolor{textcolor}%
\pgftext[x=5.155710in,y=1.419120in,left,base]{\color{textcolor}\rmfamily\fontsize{9.000000}{10.800000}\selectfont ProCount}%
\end{pgfscope}%
\begin{pgfscope}%
\pgfsetrectcap%
\pgfsetroundjoin%
\pgfsetlinewidth{1.003750pt}%
\definecolor{currentstroke}{rgb}{1.000000,0.498039,0.054902}%
\pgfsetstrokecolor{currentstroke}%
\pgfsetdash{}{0pt}%
\pgfpathmoveto{\pgfqpoint{4.880710in}{1.301071in}}%
\pgfpathlineto{\pgfqpoint{5.130710in}{1.301071in}}%
\pgfusepath{stroke}%
\end{pgfscope}%
\begin{pgfscope}%
\definecolor{textcolor}{rgb}{0.000000,0.000000,0.000000}%
\pgfsetstrokecolor{textcolor}%
\pgfsetfillcolor{textcolor}%
\pgftext[x=5.155710in,y=1.257321in,left,base]{\color{textcolor}\rmfamily\fontsize{9.000000}{10.800000}\selectfont D4\textsubscript{P}}%
\end{pgfscope}%
\begin{pgfscope}%
\pgfsetrectcap%
\pgfsetroundjoin%
\pgfsetlinewidth{1.003750pt}%
\definecolor{currentstroke}{rgb}{0.172549,0.627451,0.172549}%
\pgfsetstrokecolor{currentstroke}%
\pgfsetdash{}{0pt}%
\pgfpathmoveto{\pgfqpoint{4.880710in}{1.139271in}}%
\pgfpathlineto{\pgfqpoint{5.130710in}{1.139271in}}%
\pgfusepath{stroke}%
\end{pgfscope}%
\begin{pgfscope}%
\definecolor{textcolor}{rgb}{0.000000,0.000000,0.000000}%
\pgfsetstrokecolor{textcolor}%
\pgfsetfillcolor{textcolor}%
\pgftext[x=5.155710in,y=1.095521in,left,base]{\color{textcolor}\rmfamily\fontsize{9.000000}{10.800000}\selectfont projMC}%
\end{pgfscope}%
\begin{pgfscope}%
\pgfsetrectcap%
\pgfsetroundjoin%
\pgfsetlinewidth{1.003750pt}%
\definecolor{currentstroke}{rgb}{0.839216,0.152941,0.156863}%
\pgfsetstrokecolor{currentstroke}%
\pgfsetdash{}{0pt}%
\pgfpathmoveto{\pgfqpoint{4.880710in}{0.977471in}}%
\pgfpathlineto{\pgfqpoint{5.130710in}{0.977471in}}%
\pgfusepath{stroke}%
\end{pgfscope}%
\begin{pgfscope}%
\definecolor{textcolor}{rgb}{0.000000,0.000000,0.000000}%
\pgfsetstrokecolor{textcolor}%
\pgfsetfillcolor{textcolor}%
\pgftext[x=5.155710in,y=0.933721in,left,base]{\color{textcolor}\rmfamily\fontsize{9.000000}{10.800000}\selectfont reSSAT}%
\end{pgfscope}%
\begin{pgfscope}%
\pgfsetbuttcap%
\pgfsetroundjoin%
\pgfsetlinewidth{1.003750pt}%
\definecolor{currentstroke}{rgb}{0.580392,0.403922,0.741176}%
\pgfsetstrokecolor{currentstroke}%
\pgfsetdash{{3.700000pt}{1.600000pt}}{0.000000pt}%
\pgfpathmoveto{\pgfqpoint{4.880710in}{0.815672in}}%
\pgfpathlineto{\pgfqpoint{5.130710in}{0.815672in}}%
\pgfusepath{stroke}%
\end{pgfscope}%
\begin{pgfscope}%
\definecolor{textcolor}{rgb}{0.000000,0.000000,0.000000}%
\pgfsetstrokecolor{textcolor}%
\pgfsetfillcolor{textcolor}%
\pgftext[x=5.155710in,y=0.771922in,left,base]{\color{textcolor}\rmfamily\fontsize{9.000000}{10.800000}\selectfont VBS0}%
\end{pgfscope}%
\begin{pgfscope}%
\pgfsetbuttcap%
\pgfsetroundjoin%
\pgfsetlinewidth{1.003750pt}%
\definecolor{currentstroke}{rgb}{0.549020,0.337255,0.294118}%
\pgfsetstrokecolor{currentstroke}%
\pgfsetdash{{1.000000pt}{1.650000pt}}{0.000000pt}%
\pgfpathmoveto{\pgfqpoint{4.880710in}{0.653872in}}%
\pgfpathlineto{\pgfqpoint{5.130710in}{0.653872in}}%
\pgfusepath{stroke}%
\end{pgfscope}%
\begin{pgfscope}%
\definecolor{textcolor}{rgb}{0.000000,0.000000,0.000000}%
\pgfsetstrokecolor{textcolor}%
\pgfsetfillcolor{textcolor}%
\pgftext[x=5.155710in,y=0.610122in,left,base]{\color{textcolor}\rmfamily\fontsize{9.000000}{10.800000}\selectfont VBS1}%
\end{pgfscope}%
\end{pgfpicture}%
\makeatother%
\endgroup%

    \caption{
        Experiment 3 compares our framework \procount{} to the state-of-the-art exact weighted projected model counters \dfp, \projmc{}, and \ssat{}.
        \vbs0 is the virtual best solver of the three existing tools, excluding \procount.
        \vbs1 includes all four tools.
        Adding \procount{} significantly improves the portfolio of projected model counters.
    }
    \label{figSolving}
\end{figure}
\begin{table}[t]
    \centering
    \caption{
        Experiment 3 compares our framework \procount{} to the state-of-the-art exact weighted projected model counters \dfp{}, \projmc{}, and \ssat{}. 
        For each solver, the PAR-2 score is the cumulative solving time of completed benchmarks plus twice the 1000-second timeout for each unsolved benchmark.
        There are \solvedBenchmarks{} benchmarks solved by at least one of four tools.
        By including \procount, the portfolio of tools solves \dpmcUniqueBenchmarks{} more benchmarks and achieves shorter solving time on 87 other benchmarks.
    }
    \begin{tabular}{|l|r|r|r|r|} \hline
        \multirow{2}{*}{Tool} & \multicolumn{3}{c|}{Number of benchmarks solved (of \benchmarks)} & \multirow{2}{*}{PAR-2 score} \\ \cline{2-4}
        & By no other & In shortest time & In total & \\ \hline
        \procount & \dpmcUniqueBenchmarks & \dpmcFastestBenchmarks & 283 & 1139215 \\ \hline
        \dfp{} & 50 & 235 & 345 & 1021809 \\ \hline
        \projmc{} & 0 & 8 & 275 & 1157018 \\ \hline
        \ssat{} & 1 & 16 & 154 & 1408853 \\ \hline
        \vbs0 & - & - & 346 & 1018784 \\ \hline
        \vbs1 & - & - & 390 & 933494 \\ \hline
    \end{tabular}
    \label{tableSolving}
\end{table}


%%%%%%%%%%%%%%%%%%%%%%%%%%%%%%%%%%%%%%%%%%%%%%%%%%%%%%%%%%%%%%%%%%%%%%%%%%%%%%%%

\subsubsection{Project-Join Tree Width and Computation Time}

To identify which type of benchmarks can be solved efficiently by \procount{}, we study how the performance of each projected model counter varies with the widths of graded project-join trees.
In particular, for each benchmark, we consider the width of the first graded project-join tree produced by the planner \Lg{} in Experiment 1.
% For those 346 project-join trees, the widths range from 1 to 99. Each such project-join tree was produced in less than 29 seconds. % JD: This info was already in 5.2
Figure \ref{time_vs_width} shows how these widths relate to PAR-2 scores of projected model counters. 
\procount{} seems to be the fastest solver on instances for which there exist graded project-join trees of widths between 50 and 100.
\begin{figure}[t]
    \centering
    %% Creator: Matplotlib, PGF backend
%%
%% To include the figure in your LaTeX document, write
%%   \input{<filename>.pgf}
%%
%% Make sure the required packages are loaded in your preamble
%%   \usepackage{pgf}
%%
%% and, on pdftex
%%   \usepackage[utf8]{inputenc}\DeclareUnicodeCharacter{2212}{-}
%%
%% or, on luatex and xetex
%%   \usepackage{unicode-math}
%%
%% Figures using additional raster images can only be included by \input if
%% they are in the same directory as the main LaTeX file. For loading figures
%% from other directories you can use the `import` package
%%   \usepackage{import}
%%
%% and then include the figures with
%%   \import{<path to file>}{<filename>.pgf}
%%
%% Matplotlib used the following preamble
%%   \usepackage{fontspec}
%%   \setmainfont{DejaVuSerif.ttf}[Path=/home/vhp1/.local/lib/python3.8/site-packages/matplotlib/mpl-data/fonts/ttf/]
%%   \setsansfont{DejaVuSans.ttf}[Path=/home/vhp1/.local/lib/python3.8/site-packages/matplotlib/mpl-data/fonts/ttf/]
%%   \setmonofont{DejaVuSansMono.ttf}[Path=/home/vhp1/.local/lib/python3.8/site-packages/matplotlib/mpl-data/fonts/ttf/]
%%
\begingroup%
\makeatletter%
\begin{pgfpicture}%
\pgfpathrectangle{\pgfpointorigin}{\pgfqpoint{4.791107in}{2.220595in}}%
\pgfusepath{use as bounding box, clip}%
\begin{pgfscope}%
\pgfsetbuttcap%
\pgfsetmiterjoin%
\pgfsetlinewidth{0.000000pt}%
\definecolor{currentstroke}{rgb}{1.000000,1.000000,1.000000}%
\pgfsetstrokecolor{currentstroke}%
\pgfsetstrokeopacity{0.000000}%
\pgfsetdash{}{0pt}%
\pgfpathmoveto{\pgfqpoint{0.000000in}{0.000000in}}%
\pgfpathlineto{\pgfqpoint{4.791107in}{0.000000in}}%
\pgfpathlineto{\pgfqpoint{4.791107in}{2.220595in}}%
\pgfpathlineto{\pgfqpoint{0.000000in}{2.220595in}}%
\pgfpathclose%
\pgfusepath{}%
\end{pgfscope}%
\begin{pgfscope}%
\pgfsetbuttcap%
\pgfsetmiterjoin%
\definecolor{currentfill}{rgb}{1.000000,1.000000,1.000000}%
\pgfsetfillcolor{currentfill}%
\pgfsetlinewidth{0.000000pt}%
\definecolor{currentstroke}{rgb}{0.000000,0.000000,0.000000}%
\pgfsetstrokecolor{currentstroke}%
\pgfsetstrokeopacity{0.000000}%
\pgfsetdash{}{0pt}%
\pgfpathmoveto{\pgfqpoint{0.596422in}{0.467838in}}%
\pgfpathlineto{\pgfqpoint{4.691107in}{0.467838in}}%
\pgfpathlineto{\pgfqpoint{4.691107in}{2.063444in}}%
\pgfpathlineto{\pgfqpoint{0.596422in}{2.063444in}}%
\pgfpathclose%
\pgfusepath{fill}%
\end{pgfscope}%
\begin{pgfscope}%
\pgfsetbuttcap%
\pgfsetroundjoin%
\definecolor{currentfill}{rgb}{0.000000,0.000000,0.000000}%
\pgfsetfillcolor{currentfill}%
\pgfsetlinewidth{0.803000pt}%
\definecolor{currentstroke}{rgb}{0.000000,0.000000,0.000000}%
\pgfsetstrokecolor{currentstroke}%
\pgfsetdash{}{0pt}%
\pgfsys@defobject{currentmarker}{\pgfqpoint{0.000000in}{-0.048611in}}{\pgfqpoint{0.000000in}{0.000000in}}{%
\pgfpathmoveto{\pgfqpoint{0.000000in}{0.000000in}}%
\pgfpathlineto{\pgfqpoint{0.000000in}{-0.048611in}}%
\pgfusepath{stroke,fill}%
}%
\begin{pgfscope}%
\pgfsys@transformshift{0.704899in}{0.467838in}%
\pgfsys@useobject{currentmarker}{}%
\end{pgfscope}%
\end{pgfscope}%
\begin{pgfscope}%
\definecolor{textcolor}{rgb}{0.000000,0.000000,0.000000}%
\pgfsetstrokecolor{textcolor}%
\pgfsetfillcolor{textcolor}%
\pgftext[x=0.704899in,y=0.370616in,,top]{\color{textcolor}\sffamily\fontsize{8.000000}{9.600000}\selectfont \(\displaystyle {10}\)}%
\end{pgfscope}%
\begin{pgfscope}%
\pgfsetbuttcap%
\pgfsetroundjoin%
\definecolor{currentfill}{rgb}{0.000000,0.000000,0.000000}%
\pgfsetfillcolor{currentfill}%
\pgfsetlinewidth{0.803000pt}%
\definecolor{currentstroke}{rgb}{0.000000,0.000000,0.000000}%
\pgfsetstrokecolor{currentstroke}%
\pgfsetdash{}{0pt}%
\pgfsys@defobject{currentmarker}{\pgfqpoint{0.000000in}{-0.048611in}}{\pgfqpoint{0.000000in}{0.000000in}}{%
\pgfpathmoveto{\pgfqpoint{0.000000in}{0.000000in}}%
\pgfpathlineto{\pgfqpoint{0.000000in}{-0.048611in}}%
\pgfusepath{stroke,fill}%
}%
\begin{pgfscope}%
\pgfsys@transformshift{1.161640in}{0.467838in}%
\pgfsys@useobject{currentmarker}{}%
\end{pgfscope}%
\end{pgfscope}%
\begin{pgfscope}%
\definecolor{textcolor}{rgb}{0.000000,0.000000,0.000000}%
\pgfsetstrokecolor{textcolor}%
\pgfsetfillcolor{textcolor}%
\pgftext[x=1.161640in,y=0.370616in,,top]{\color{textcolor}\sffamily\fontsize{8.000000}{9.600000}\selectfont \(\displaystyle {20}\)}%
\end{pgfscope}%
\begin{pgfscope}%
\pgfsetbuttcap%
\pgfsetroundjoin%
\definecolor{currentfill}{rgb}{0.000000,0.000000,0.000000}%
\pgfsetfillcolor{currentfill}%
\pgfsetlinewidth{0.803000pt}%
\definecolor{currentstroke}{rgb}{0.000000,0.000000,0.000000}%
\pgfsetstrokecolor{currentstroke}%
\pgfsetdash{}{0pt}%
\pgfsys@defobject{currentmarker}{\pgfqpoint{0.000000in}{-0.048611in}}{\pgfqpoint{0.000000in}{0.000000in}}{%
\pgfpathmoveto{\pgfqpoint{0.000000in}{0.000000in}}%
\pgfpathlineto{\pgfqpoint{0.000000in}{-0.048611in}}%
\pgfusepath{stroke,fill}%
}%
\begin{pgfscope}%
\pgfsys@transformshift{1.618381in}{0.467838in}%
\pgfsys@useobject{currentmarker}{}%
\end{pgfscope}%
\end{pgfscope}%
\begin{pgfscope}%
\definecolor{textcolor}{rgb}{0.000000,0.000000,0.000000}%
\pgfsetstrokecolor{textcolor}%
\pgfsetfillcolor{textcolor}%
\pgftext[x=1.618381in,y=0.370616in,,top]{\color{textcolor}\sffamily\fontsize{8.000000}{9.600000}\selectfont \(\displaystyle {30}\)}%
\end{pgfscope}%
\begin{pgfscope}%
\pgfsetbuttcap%
\pgfsetroundjoin%
\definecolor{currentfill}{rgb}{0.000000,0.000000,0.000000}%
\pgfsetfillcolor{currentfill}%
\pgfsetlinewidth{0.803000pt}%
\definecolor{currentstroke}{rgb}{0.000000,0.000000,0.000000}%
\pgfsetstrokecolor{currentstroke}%
\pgfsetdash{}{0pt}%
\pgfsys@defobject{currentmarker}{\pgfqpoint{0.000000in}{-0.048611in}}{\pgfqpoint{0.000000in}{0.000000in}}{%
\pgfpathmoveto{\pgfqpoint{0.000000in}{0.000000in}}%
\pgfpathlineto{\pgfqpoint{0.000000in}{-0.048611in}}%
\pgfusepath{stroke,fill}%
}%
\begin{pgfscope}%
\pgfsys@transformshift{2.075122in}{0.467838in}%
\pgfsys@useobject{currentmarker}{}%
\end{pgfscope}%
\end{pgfscope}%
\begin{pgfscope}%
\definecolor{textcolor}{rgb}{0.000000,0.000000,0.000000}%
\pgfsetstrokecolor{textcolor}%
\pgfsetfillcolor{textcolor}%
\pgftext[x=2.075122in,y=0.370616in,,top]{\color{textcolor}\sffamily\fontsize{8.000000}{9.600000}\selectfont \(\displaystyle {40}\)}%
\end{pgfscope}%
\begin{pgfscope}%
\pgfsetbuttcap%
\pgfsetroundjoin%
\definecolor{currentfill}{rgb}{0.000000,0.000000,0.000000}%
\pgfsetfillcolor{currentfill}%
\pgfsetlinewidth{0.803000pt}%
\definecolor{currentstroke}{rgb}{0.000000,0.000000,0.000000}%
\pgfsetstrokecolor{currentstroke}%
\pgfsetdash{}{0pt}%
\pgfsys@defobject{currentmarker}{\pgfqpoint{0.000000in}{-0.048611in}}{\pgfqpoint{0.000000in}{0.000000in}}{%
\pgfpathmoveto{\pgfqpoint{0.000000in}{0.000000in}}%
\pgfpathlineto{\pgfqpoint{0.000000in}{-0.048611in}}%
\pgfusepath{stroke,fill}%
}%
\begin{pgfscope}%
\pgfsys@transformshift{2.531863in}{0.467838in}%
\pgfsys@useobject{currentmarker}{}%
\end{pgfscope}%
\end{pgfscope}%
\begin{pgfscope}%
\definecolor{textcolor}{rgb}{0.000000,0.000000,0.000000}%
\pgfsetstrokecolor{textcolor}%
\pgfsetfillcolor{textcolor}%
\pgftext[x=2.531863in,y=0.370616in,,top]{\color{textcolor}\sffamily\fontsize{8.000000}{9.600000}\selectfont \(\displaystyle {50}\)}%
\end{pgfscope}%
\begin{pgfscope}%
\pgfsetbuttcap%
\pgfsetroundjoin%
\definecolor{currentfill}{rgb}{0.000000,0.000000,0.000000}%
\pgfsetfillcolor{currentfill}%
\pgfsetlinewidth{0.803000pt}%
\definecolor{currentstroke}{rgb}{0.000000,0.000000,0.000000}%
\pgfsetstrokecolor{currentstroke}%
\pgfsetdash{}{0pt}%
\pgfsys@defobject{currentmarker}{\pgfqpoint{0.000000in}{-0.048611in}}{\pgfqpoint{0.000000in}{0.000000in}}{%
\pgfpathmoveto{\pgfqpoint{0.000000in}{0.000000in}}%
\pgfpathlineto{\pgfqpoint{0.000000in}{-0.048611in}}%
\pgfusepath{stroke,fill}%
}%
\begin{pgfscope}%
\pgfsys@transformshift{2.988604in}{0.467838in}%
\pgfsys@useobject{currentmarker}{}%
\end{pgfscope}%
\end{pgfscope}%
\begin{pgfscope}%
\definecolor{textcolor}{rgb}{0.000000,0.000000,0.000000}%
\pgfsetstrokecolor{textcolor}%
\pgfsetfillcolor{textcolor}%
\pgftext[x=2.988604in,y=0.370616in,,top]{\color{textcolor}\sffamily\fontsize{8.000000}{9.600000}\selectfont \(\displaystyle {60}\)}%
\end{pgfscope}%
\begin{pgfscope}%
\pgfsetbuttcap%
\pgfsetroundjoin%
\definecolor{currentfill}{rgb}{0.000000,0.000000,0.000000}%
\pgfsetfillcolor{currentfill}%
\pgfsetlinewidth{0.803000pt}%
\definecolor{currentstroke}{rgb}{0.000000,0.000000,0.000000}%
\pgfsetstrokecolor{currentstroke}%
\pgfsetdash{}{0pt}%
\pgfsys@defobject{currentmarker}{\pgfqpoint{0.000000in}{-0.048611in}}{\pgfqpoint{0.000000in}{0.000000in}}{%
\pgfpathmoveto{\pgfqpoint{0.000000in}{0.000000in}}%
\pgfpathlineto{\pgfqpoint{0.000000in}{-0.048611in}}%
\pgfusepath{stroke,fill}%
}%
\begin{pgfscope}%
\pgfsys@transformshift{3.445345in}{0.467838in}%
\pgfsys@useobject{currentmarker}{}%
\end{pgfscope}%
\end{pgfscope}%
\begin{pgfscope}%
\definecolor{textcolor}{rgb}{0.000000,0.000000,0.000000}%
\pgfsetstrokecolor{textcolor}%
\pgfsetfillcolor{textcolor}%
\pgftext[x=3.445345in,y=0.370616in,,top]{\color{textcolor}\sffamily\fontsize{8.000000}{9.600000}\selectfont \(\displaystyle {70}\)}%
\end{pgfscope}%
\begin{pgfscope}%
\pgfsetbuttcap%
\pgfsetroundjoin%
\definecolor{currentfill}{rgb}{0.000000,0.000000,0.000000}%
\pgfsetfillcolor{currentfill}%
\pgfsetlinewidth{0.803000pt}%
\definecolor{currentstroke}{rgb}{0.000000,0.000000,0.000000}%
\pgfsetstrokecolor{currentstroke}%
\pgfsetdash{}{0pt}%
\pgfsys@defobject{currentmarker}{\pgfqpoint{0.000000in}{-0.048611in}}{\pgfqpoint{0.000000in}{0.000000in}}{%
\pgfpathmoveto{\pgfqpoint{0.000000in}{0.000000in}}%
\pgfpathlineto{\pgfqpoint{0.000000in}{-0.048611in}}%
\pgfusepath{stroke,fill}%
}%
\begin{pgfscope}%
\pgfsys@transformshift{3.902086in}{0.467838in}%
\pgfsys@useobject{currentmarker}{}%
\end{pgfscope}%
\end{pgfscope}%
\begin{pgfscope}%
\definecolor{textcolor}{rgb}{0.000000,0.000000,0.000000}%
\pgfsetstrokecolor{textcolor}%
\pgfsetfillcolor{textcolor}%
\pgftext[x=3.902086in,y=0.370616in,,top]{\color{textcolor}\sffamily\fontsize{8.000000}{9.600000}\selectfont \(\displaystyle {80}\)}%
\end{pgfscope}%
\begin{pgfscope}%
\pgfsetbuttcap%
\pgfsetroundjoin%
\definecolor{currentfill}{rgb}{0.000000,0.000000,0.000000}%
\pgfsetfillcolor{currentfill}%
\pgfsetlinewidth{0.803000pt}%
\definecolor{currentstroke}{rgb}{0.000000,0.000000,0.000000}%
\pgfsetstrokecolor{currentstroke}%
\pgfsetdash{}{0pt}%
\pgfsys@defobject{currentmarker}{\pgfqpoint{0.000000in}{-0.048611in}}{\pgfqpoint{0.000000in}{0.000000in}}{%
\pgfpathmoveto{\pgfqpoint{0.000000in}{0.000000in}}%
\pgfpathlineto{\pgfqpoint{0.000000in}{-0.048611in}}%
\pgfusepath{stroke,fill}%
}%
\begin{pgfscope}%
\pgfsys@transformshift{4.358827in}{0.467838in}%
\pgfsys@useobject{currentmarker}{}%
\end{pgfscope}%
\end{pgfscope}%
\begin{pgfscope}%
\definecolor{textcolor}{rgb}{0.000000,0.000000,0.000000}%
\pgfsetstrokecolor{textcolor}%
\pgfsetfillcolor{textcolor}%
\pgftext[x=4.358827in,y=0.370616in,,top]{\color{textcolor}\sffamily\fontsize{8.000000}{9.600000}\selectfont \(\displaystyle {90}\)}%
\end{pgfscope}%
\begin{pgfscope}%
\definecolor{textcolor}{rgb}{0.000000,0.000000,0.000000}%
\pgfsetstrokecolor{textcolor}%
\pgfsetfillcolor{textcolor}%
\pgftext[x=2.643765in,y=0.207530in,,top]{\color{textcolor}\sffamily\fontsize{8.000000}{9.600000}\selectfont Mean of 10 project-join tree widths}%
\end{pgfscope}%
\begin{pgfscope}%
\pgfsetbuttcap%
\pgfsetroundjoin%
\definecolor{currentfill}{rgb}{0.000000,0.000000,0.000000}%
\pgfsetfillcolor{currentfill}%
\pgfsetlinewidth{0.803000pt}%
\definecolor{currentstroke}{rgb}{0.000000,0.000000,0.000000}%
\pgfsetstrokecolor{currentstroke}%
\pgfsetdash{}{0pt}%
\pgfsys@defobject{currentmarker}{\pgfqpoint{-0.048611in}{0.000000in}}{\pgfqpoint{-0.000000in}{0.000000in}}{%
\pgfpathmoveto{\pgfqpoint{-0.000000in}{0.000000in}}%
\pgfpathlineto{\pgfqpoint{-0.048611in}{0.000000in}}%
\pgfusepath{stroke,fill}%
}%
\begin{pgfscope}%
\pgfsys@transformshift{0.596422in}{0.540283in}%
\pgfsys@useobject{currentmarker}{}%
\end{pgfscope}%
\end{pgfscope}%
\begin{pgfscope}%
\definecolor{textcolor}{rgb}{0.000000,0.000000,0.000000}%
\pgfsetstrokecolor{textcolor}%
\pgfsetfillcolor{textcolor}%
\pgftext[x=0.440172in, y=0.498074in, left, base]{\color{textcolor}\sffamily\fontsize{8.000000}{9.600000}\selectfont \(\displaystyle {0}\)}%
\end{pgfscope}%
\begin{pgfscope}%
\pgfsetbuttcap%
\pgfsetroundjoin%
\definecolor{currentfill}{rgb}{0.000000,0.000000,0.000000}%
\pgfsetfillcolor{currentfill}%
\pgfsetlinewidth{0.803000pt}%
\definecolor{currentstroke}{rgb}{0.000000,0.000000,0.000000}%
\pgfsetstrokecolor{currentstroke}%
\pgfsetdash{}{0pt}%
\pgfsys@defobject{currentmarker}{\pgfqpoint{-0.048611in}{0.000000in}}{\pgfqpoint{-0.000000in}{0.000000in}}{%
\pgfpathmoveto{\pgfqpoint{-0.000000in}{0.000000in}}%
\pgfpathlineto{\pgfqpoint{-0.048611in}{0.000000in}}%
\pgfusepath{stroke,fill}%
}%
\begin{pgfscope}%
\pgfsys@transformshift{0.596422in}{0.902942in}%
\pgfsys@useobject{currentmarker}{}%
\end{pgfscope}%
\end{pgfscope}%
\begin{pgfscope}%
\definecolor{textcolor}{rgb}{0.000000,0.000000,0.000000}%
\pgfsetstrokecolor{textcolor}%
\pgfsetfillcolor{textcolor}%
\pgftext[x=0.322114in, y=0.860732in, left, base]{\color{textcolor}\sffamily\fontsize{8.000000}{9.600000}\selectfont \(\displaystyle {500}\)}%
\end{pgfscope}%
\begin{pgfscope}%
\pgfsetbuttcap%
\pgfsetroundjoin%
\definecolor{currentfill}{rgb}{0.000000,0.000000,0.000000}%
\pgfsetfillcolor{currentfill}%
\pgfsetlinewidth{0.803000pt}%
\definecolor{currentstroke}{rgb}{0.000000,0.000000,0.000000}%
\pgfsetstrokecolor{currentstroke}%
\pgfsetdash{}{0pt}%
\pgfsys@defobject{currentmarker}{\pgfqpoint{-0.048611in}{0.000000in}}{\pgfqpoint{-0.000000in}{0.000000in}}{%
\pgfpathmoveto{\pgfqpoint{-0.000000in}{0.000000in}}%
\pgfpathlineto{\pgfqpoint{-0.048611in}{0.000000in}}%
\pgfusepath{stroke,fill}%
}%
\begin{pgfscope}%
\pgfsys@transformshift{0.596422in}{1.265600in}%
\pgfsys@useobject{currentmarker}{}%
\end{pgfscope}%
\end{pgfscope}%
\begin{pgfscope}%
\definecolor{textcolor}{rgb}{0.000000,0.000000,0.000000}%
\pgfsetstrokecolor{textcolor}%
\pgfsetfillcolor{textcolor}%
\pgftext[x=0.263086in, y=1.223391in, left, base]{\color{textcolor}\sffamily\fontsize{8.000000}{9.600000}\selectfont \(\displaystyle {1000}\)}%
\end{pgfscope}%
\begin{pgfscope}%
\pgfsetbuttcap%
\pgfsetroundjoin%
\definecolor{currentfill}{rgb}{0.000000,0.000000,0.000000}%
\pgfsetfillcolor{currentfill}%
\pgfsetlinewidth{0.803000pt}%
\definecolor{currentstroke}{rgb}{0.000000,0.000000,0.000000}%
\pgfsetstrokecolor{currentstroke}%
\pgfsetdash{}{0pt}%
\pgfsys@defobject{currentmarker}{\pgfqpoint{-0.048611in}{0.000000in}}{\pgfqpoint{-0.000000in}{0.000000in}}{%
\pgfpathmoveto{\pgfqpoint{-0.000000in}{0.000000in}}%
\pgfpathlineto{\pgfqpoint{-0.048611in}{0.000000in}}%
\pgfusepath{stroke,fill}%
}%
\begin{pgfscope}%
\pgfsys@transformshift{0.596422in}{1.628258in}%
\pgfsys@useobject{currentmarker}{}%
\end{pgfscope}%
\end{pgfscope}%
\begin{pgfscope}%
\definecolor{textcolor}{rgb}{0.000000,0.000000,0.000000}%
\pgfsetstrokecolor{textcolor}%
\pgfsetfillcolor{textcolor}%
\pgftext[x=0.263086in, y=1.586049in, left, base]{\color{textcolor}\sffamily\fontsize{8.000000}{9.600000}\selectfont \(\displaystyle {1500}\)}%
\end{pgfscope}%
\begin{pgfscope}%
\pgfsetbuttcap%
\pgfsetroundjoin%
\definecolor{currentfill}{rgb}{0.000000,0.000000,0.000000}%
\pgfsetfillcolor{currentfill}%
\pgfsetlinewidth{0.803000pt}%
\definecolor{currentstroke}{rgb}{0.000000,0.000000,0.000000}%
\pgfsetstrokecolor{currentstroke}%
\pgfsetdash{}{0pt}%
\pgfsys@defobject{currentmarker}{\pgfqpoint{-0.048611in}{0.000000in}}{\pgfqpoint{-0.000000in}{0.000000in}}{%
\pgfpathmoveto{\pgfqpoint{-0.000000in}{0.000000in}}%
\pgfpathlineto{\pgfqpoint{-0.048611in}{0.000000in}}%
\pgfusepath{stroke,fill}%
}%
\begin{pgfscope}%
\pgfsys@transformshift{0.596422in}{1.990917in}%
\pgfsys@useobject{currentmarker}{}%
\end{pgfscope}%
\end{pgfscope}%
\begin{pgfscope}%
\definecolor{textcolor}{rgb}{0.000000,0.000000,0.000000}%
\pgfsetstrokecolor{textcolor}%
\pgfsetfillcolor{textcolor}%
\pgftext[x=0.263086in, y=1.948707in, left, base]{\color{textcolor}\sffamily\fontsize{8.000000}{9.600000}\selectfont \(\displaystyle {2000}\)}%
\end{pgfscope}%
\begin{pgfscope}%
\definecolor{textcolor}{rgb}{0.000000,0.000000,0.000000}%
\pgfsetstrokecolor{textcolor}%
\pgfsetfillcolor{textcolor}%
\pgftext[x=0.207530in,y=1.265641in,,bottom,rotate=90.000000]{\color{textcolor}\sffamily\fontsize{8.000000}{9.600000}\selectfont Mean PAR-2 score of 10 widths}%
\end{pgfscope}%
\begin{pgfscope}%
\pgfpathrectangle{\pgfqpoint{0.596422in}{0.467838in}}{\pgfqpoint{4.094684in}{1.595606in}}%
\pgfusepath{clip}%
\pgfsetrectcap%
\pgfsetroundjoin%
\pgfsetlinewidth{1.003750pt}%
\definecolor{currentstroke}{rgb}{0.121569,0.466667,0.705882}%
\pgfsetstrokecolor{currentstroke}%
\pgfsetdash{}{0pt}%
\pgfpathmoveto{\pgfqpoint{0.782545in}{0.612895in}}%
\pgfpathlineto{\pgfqpoint{0.864758in}{0.540369in}}%
\pgfpathlineto{\pgfqpoint{0.924134in}{0.540369in}}%
\pgfpathlineto{\pgfqpoint{0.983511in}{0.540370in}}%
\pgfpathlineto{\pgfqpoint{1.042887in}{0.540366in}}%
\pgfpathlineto{\pgfqpoint{1.093128in}{0.540366in}}%
\pgfpathlineto{\pgfqpoint{1.138803in}{0.540367in}}%
\pgfpathlineto{\pgfqpoint{1.184477in}{0.562019in}}%
\pgfpathlineto{\pgfqpoint{1.230151in}{0.560800in}}%
\pgfpathlineto{\pgfqpoint{1.275825in}{0.581232in}}%
\pgfpathlineto{\pgfqpoint{1.321499in}{0.609446in}}%
\pgfpathlineto{\pgfqpoint{1.367173in}{0.620958in}}%
\pgfpathlineto{\pgfqpoint{1.412847in}{0.629182in}}%
\pgfpathlineto{\pgfqpoint{1.458521in}{0.654144in}}%
\pgfpathlineto{\pgfqpoint{1.504195in}{0.658791in}}%
\pgfpathlineto{\pgfqpoint{1.549870in}{0.656427in}}%
\pgfpathlineto{\pgfqpoint{1.595544in}{0.685440in}}%
\pgfpathlineto{\pgfqpoint{1.641218in}{0.716219in}}%
\pgfpathlineto{\pgfqpoint{1.686892in}{0.753715in}}%
\pgfpathlineto{\pgfqpoint{1.732566in}{0.717291in}}%
\pgfpathlineto{\pgfqpoint{1.778240in}{0.709064in}}%
\pgfpathlineto{\pgfqpoint{1.823914in}{0.688409in}}%
\pgfpathlineto{\pgfqpoint{1.869588in}{0.707766in}}%
\pgfpathlineto{\pgfqpoint{1.915262in}{0.685449in}}%
\pgfpathlineto{\pgfqpoint{1.960937in}{0.679871in}}%
\pgfpathlineto{\pgfqpoint{2.006611in}{0.688408in}}%
\pgfpathlineto{\pgfqpoint{2.052285in}{0.679876in}}%
\pgfpathlineto{\pgfqpoint{2.097959in}{0.763559in}}%
\pgfpathlineto{\pgfqpoint{2.143633in}{0.772487in}}%
\pgfpathlineto{\pgfqpoint{2.189307in}{0.830514in}}%
\pgfpathlineto{\pgfqpoint{2.234981in}{0.911473in}}%
\pgfpathlineto{\pgfqpoint{2.280655in}{0.996295in}}%
\pgfpathlineto{\pgfqpoint{2.326329in}{0.932445in}}%
\pgfpathlineto{\pgfqpoint{2.372004in}{0.903052in}}%
\pgfpathlineto{\pgfqpoint{2.417678in}{0.954864in}}%
\pgfpathlineto{\pgfqpoint{2.463352in}{0.910787in}}%
\pgfpathlineto{\pgfqpoint{2.509026in}{0.929670in}}%
\pgfpathlineto{\pgfqpoint{2.554700in}{0.824357in}}%
\pgfpathlineto{\pgfqpoint{2.600374in}{0.816864in}}%
\pgfpathlineto{\pgfqpoint{2.650616in}{0.837264in}}%
\pgfpathlineto{\pgfqpoint{2.700857in}{0.747792in}}%
\pgfpathlineto{\pgfqpoint{2.760234in}{0.742997in}}%
\pgfpathlineto{\pgfqpoint{2.819610in}{0.796064in}}%
\pgfpathlineto{\pgfqpoint{2.878986in}{0.806333in}}%
\pgfpathlineto{\pgfqpoint{2.938363in}{0.791232in}}%
\pgfpathlineto{\pgfqpoint{2.997739in}{0.842404in}}%
\pgfpathlineto{\pgfqpoint{3.057115in}{0.822291in}}%
\pgfpathlineto{\pgfqpoint{3.116492in}{0.830851in}}%
\pgfpathlineto{\pgfqpoint{3.175868in}{0.830904in}}%
\pgfpathlineto{\pgfqpoint{3.230677in}{0.823456in}}%
\pgfpathlineto{\pgfqpoint{3.285486in}{0.787576in}}%
\pgfpathlineto{\pgfqpoint{3.331160in}{0.848761in}}%
\pgfpathlineto{\pgfqpoint{3.376834in}{0.828116in}}%
\pgfpathlineto{\pgfqpoint{3.422508in}{0.861844in}}%
\pgfpathlineto{\pgfqpoint{3.468182in}{0.894717in}}%
\pgfpathlineto{\pgfqpoint{3.513856in}{0.912812in}}%
\pgfpathlineto{\pgfqpoint{3.564098in}{1.023790in}}%
\pgfpathlineto{\pgfqpoint{3.614339in}{1.110348in}}%
\pgfpathlineto{\pgfqpoint{3.664581in}{1.038009in}}%
\pgfpathlineto{\pgfqpoint{3.714822in}{1.038014in}}%
\pgfpathlineto{\pgfqpoint{3.769631in}{1.144964in}}%
\pgfpathlineto{\pgfqpoint{3.824440in}{1.037033in}}%
\pgfpathlineto{\pgfqpoint{3.879249in}{1.036756in}}%
\pgfpathlineto{\pgfqpoint{3.938626in}{1.023698in}}%
\pgfpathlineto{\pgfqpoint{3.998002in}{1.048498in}}%
\pgfpathlineto{\pgfqpoint{4.057378in}{1.011425in}}%
\pgfpathlineto{\pgfqpoint{4.112187in}{1.000075in}}%
\pgfpathlineto{\pgfqpoint{4.166996in}{1.000075in}}%
\pgfpathlineto{\pgfqpoint{4.221805in}{1.107040in}}%
\pgfpathlineto{\pgfqpoint{4.276614in}{1.093741in}}%
\pgfpathlineto{\pgfqpoint{4.331423in}{1.111457in}}%
\pgfpathlineto{\pgfqpoint{4.390799in}{1.340591in}}%
\pgfpathlineto{\pgfqpoint{4.450176in}{1.418398in}}%
\pgfpathlineto{\pgfqpoint{4.504985in}{1.439001in}}%
\pgfusepath{stroke}%
\end{pgfscope}%
\begin{pgfscope}%
\pgfpathrectangle{\pgfqpoint{0.596422in}{0.467838in}}{\pgfqpoint{4.094684in}{1.595606in}}%
\pgfusepath{clip}%
\pgfsetrectcap%
\pgfsetroundjoin%
\pgfsetlinewidth{1.003750pt}%
\definecolor{currentstroke}{rgb}{1.000000,0.498039,0.054902}%
\pgfsetstrokecolor{currentstroke}%
\pgfsetdash{}{0pt}%
\pgfpathmoveto{\pgfqpoint{0.782545in}{0.576570in}}%
\pgfpathlineto{\pgfqpoint{0.864758in}{0.540490in}}%
\pgfpathlineto{\pgfqpoint{0.924134in}{0.540475in}}%
\pgfpathlineto{\pgfqpoint{0.983511in}{0.540495in}}%
\pgfpathlineto{\pgfqpoint{1.042887in}{0.540504in}}%
\pgfpathlineto{\pgfqpoint{1.093128in}{0.540509in}}%
\pgfpathlineto{\pgfqpoint{1.138803in}{0.540497in}}%
\pgfpathlineto{\pgfqpoint{1.184477in}{0.541150in}}%
\pgfpathlineto{\pgfqpoint{1.230151in}{0.542388in}}%
\pgfpathlineto{\pgfqpoint{1.275825in}{0.543146in}}%
\pgfpathlineto{\pgfqpoint{1.321499in}{0.543497in}}%
\pgfpathlineto{\pgfqpoint{1.367173in}{0.543861in}}%
\pgfpathlineto{\pgfqpoint{1.412847in}{0.544227in}}%
\pgfpathlineto{\pgfqpoint{1.458521in}{0.544076in}}%
\pgfpathlineto{\pgfqpoint{1.504195in}{0.544269in}}%
\pgfpathlineto{\pgfqpoint{1.549870in}{0.544191in}}%
\pgfpathlineto{\pgfqpoint{1.595544in}{0.544190in}}%
\pgfpathlineto{\pgfqpoint{1.641218in}{0.544761in}}%
\pgfpathlineto{\pgfqpoint{1.686892in}{0.543147in}}%
\pgfpathlineto{\pgfqpoint{1.732566in}{0.547210in}}%
\pgfpathlineto{\pgfqpoint{1.778240in}{0.549744in}}%
\pgfpathlineto{\pgfqpoint{1.823914in}{0.548585in}}%
\pgfpathlineto{\pgfqpoint{1.869588in}{0.548119in}}%
\pgfpathlineto{\pgfqpoint{1.915262in}{0.548443in}}%
\pgfpathlineto{\pgfqpoint{1.960937in}{0.548169in}}%
\pgfpathlineto{\pgfqpoint{2.006611in}{0.548650in}}%
\pgfpathlineto{\pgfqpoint{2.052285in}{0.556185in}}%
\pgfpathlineto{\pgfqpoint{2.097959in}{0.621891in}}%
\pgfpathlineto{\pgfqpoint{2.143633in}{0.653597in}}%
\pgfpathlineto{\pgfqpoint{2.189307in}{0.675915in}}%
\pgfpathlineto{\pgfqpoint{2.234981in}{0.771636in}}%
\pgfpathlineto{\pgfqpoint{2.280655in}{0.826644in}}%
\pgfpathlineto{\pgfqpoint{2.326329in}{0.829902in}}%
\pgfpathlineto{\pgfqpoint{2.372004in}{0.784301in}}%
\pgfpathlineto{\pgfqpoint{2.417678in}{0.795837in}}%
\pgfpathlineto{\pgfqpoint{2.463352in}{0.892109in}}%
\pgfpathlineto{\pgfqpoint{2.509026in}{0.933469in}}%
\pgfpathlineto{\pgfqpoint{2.554700in}{1.005726in}}%
\pgfpathlineto{\pgfqpoint{2.600374in}{1.015230in}}%
\pgfpathlineto{\pgfqpoint{2.650616in}{1.003164in}}%
\pgfpathlineto{\pgfqpoint{2.700857in}{0.983670in}}%
\pgfpathlineto{\pgfqpoint{2.760234in}{0.973442in}}%
\pgfpathlineto{\pgfqpoint{2.819610in}{1.027314in}}%
\pgfpathlineto{\pgfqpoint{2.878986in}{1.130034in}}%
\pgfpathlineto{\pgfqpoint{2.938363in}{1.140103in}}%
\pgfpathlineto{\pgfqpoint{2.997739in}{1.029989in}}%
\pgfpathlineto{\pgfqpoint{3.057115in}{1.032100in}}%
\pgfpathlineto{\pgfqpoint{3.116492in}{0.878070in}}%
\pgfpathlineto{\pgfqpoint{3.175868in}{0.909992in}}%
\pgfpathlineto{\pgfqpoint{3.230677in}{0.926988in}}%
\pgfpathlineto{\pgfqpoint{3.285486in}{1.037525in}}%
\pgfpathlineto{\pgfqpoint{3.331160in}{1.099361in}}%
\pgfpathlineto{\pgfqpoint{3.376834in}{1.123069in}}%
\pgfpathlineto{\pgfqpoint{3.422508in}{1.123020in}}%
\pgfpathlineto{\pgfqpoint{3.468182in}{1.180692in}}%
\pgfpathlineto{\pgfqpoint{3.513856in}{1.215095in}}%
\pgfpathlineto{\pgfqpoint{3.564098in}{1.264108in}}%
\pgfpathlineto{\pgfqpoint{3.614339in}{1.287776in}}%
\pgfpathlineto{\pgfqpoint{3.664581in}{1.396012in}}%
\pgfpathlineto{\pgfqpoint{3.714822in}{1.396006in}}%
\pgfpathlineto{\pgfqpoint{3.769631in}{1.346448in}}%
\pgfpathlineto{\pgfqpoint{3.824440in}{1.285455in}}%
\pgfpathlineto{\pgfqpoint{3.879249in}{1.246256in}}%
\pgfpathlineto{\pgfqpoint{3.938626in}{1.265857in}}%
\pgfpathlineto{\pgfqpoint{3.998002in}{1.284449in}}%
\pgfpathlineto{\pgfqpoint{4.057378in}{1.321699in}}%
\pgfpathlineto{\pgfqpoint{4.112187in}{1.338482in}}%
\pgfpathlineto{\pgfqpoint{4.166996in}{1.338408in}}%
\pgfpathlineto{\pgfqpoint{4.221805in}{1.393822in}}%
\pgfpathlineto{\pgfqpoint{4.276614in}{1.507582in}}%
\pgfpathlineto{\pgfqpoint{4.331423in}{1.540931in}}%
\pgfpathlineto{\pgfqpoint{4.390799in}{1.600946in}}%
\pgfpathlineto{\pgfqpoint{4.450176in}{1.561677in}}%
\pgfpathlineto{\pgfqpoint{4.504985in}{1.628807in}}%
\pgfusepath{stroke}%
\end{pgfscope}%
\begin{pgfscope}%
\pgfpathrectangle{\pgfqpoint{0.596422in}{0.467838in}}{\pgfqpoint{4.094684in}{1.595606in}}%
\pgfusepath{clip}%
\pgfsetrectcap%
\pgfsetroundjoin%
\pgfsetlinewidth{1.003750pt}%
\definecolor{currentstroke}{rgb}{0.172549,0.627451,0.172549}%
\pgfsetstrokecolor{currentstroke}%
\pgfsetdash{}{0pt}%
\pgfpathmoveto{\pgfqpoint{0.782545in}{0.576612in}}%
\pgfpathlineto{\pgfqpoint{0.864758in}{0.540480in}}%
\pgfpathlineto{\pgfqpoint{0.924134in}{0.540468in}}%
\pgfpathlineto{\pgfqpoint{0.983511in}{0.540486in}}%
\pgfpathlineto{\pgfqpoint{1.042887in}{0.540499in}}%
\pgfpathlineto{\pgfqpoint{1.093128in}{0.540506in}}%
\pgfpathlineto{\pgfqpoint{1.138803in}{0.540501in}}%
\pgfpathlineto{\pgfqpoint{1.184477in}{0.540887in}}%
\pgfpathlineto{\pgfqpoint{1.230151in}{0.541352in}}%
\pgfpathlineto{\pgfqpoint{1.275825in}{0.541832in}}%
\pgfpathlineto{\pgfqpoint{1.321499in}{0.542025in}}%
\pgfpathlineto{\pgfqpoint{1.367173in}{0.542225in}}%
\pgfpathlineto{\pgfqpoint{1.412847in}{0.542423in}}%
\pgfpathlineto{\pgfqpoint{1.458521in}{0.542354in}}%
\pgfpathlineto{\pgfqpoint{1.504195in}{0.543130in}}%
\pgfpathlineto{\pgfqpoint{1.549870in}{0.543077in}}%
\pgfpathlineto{\pgfqpoint{1.595544in}{0.543073in}}%
\pgfpathlineto{\pgfqpoint{1.641218in}{0.543616in}}%
\pgfpathlineto{\pgfqpoint{1.686892in}{0.546318in}}%
\pgfpathlineto{\pgfqpoint{1.732566in}{0.580287in}}%
\pgfpathlineto{\pgfqpoint{1.778240in}{0.612654in}}%
\pgfpathlineto{\pgfqpoint{1.823914in}{0.603754in}}%
\pgfpathlineto{\pgfqpoint{1.869588in}{0.600357in}}%
\pgfpathlineto{\pgfqpoint{1.915262in}{0.614251in}}%
\pgfpathlineto{\pgfqpoint{1.960937in}{0.613519in}}%
\pgfpathlineto{\pgfqpoint{2.006611in}{0.618008in}}%
\pgfpathlineto{\pgfqpoint{2.052285in}{0.649311in}}%
\pgfpathlineto{\pgfqpoint{2.097959in}{0.732995in}}%
\pgfpathlineto{\pgfqpoint{2.143633in}{0.784339in}}%
\pgfpathlineto{\pgfqpoint{2.189307in}{0.808632in}}%
\pgfpathlineto{\pgfqpoint{2.234981in}{0.893888in}}%
\pgfpathlineto{\pgfqpoint{2.280655in}{0.975015in}}%
\pgfpathlineto{\pgfqpoint{2.326329in}{0.980056in}}%
\pgfpathlineto{\pgfqpoint{2.372004in}{0.902262in}}%
\pgfpathlineto{\pgfqpoint{2.417678in}{0.916086in}}%
\pgfpathlineto{\pgfqpoint{2.463352in}{0.999553in}}%
\pgfpathlineto{\pgfqpoint{2.509026in}{1.024743in}}%
\pgfpathlineto{\pgfqpoint{2.554700in}{1.070389in}}%
\pgfpathlineto{\pgfqpoint{2.600374in}{1.070510in}}%
\pgfpathlineto{\pgfqpoint{2.650616in}{1.049622in}}%
\pgfpathlineto{\pgfqpoint{2.700857in}{1.001414in}}%
\pgfpathlineto{\pgfqpoint{2.760234in}{1.036770in}}%
\pgfpathlineto{\pgfqpoint{2.819610in}{1.119791in}}%
\pgfpathlineto{\pgfqpoint{2.878986in}{1.250837in}}%
\pgfpathlineto{\pgfqpoint{2.938363in}{1.417446in}}%
\pgfpathlineto{\pgfqpoint{2.997739in}{1.443245in}}%
\pgfpathlineto{\pgfqpoint{3.057115in}{1.547109in}}%
\pgfpathlineto{\pgfqpoint{3.116492in}{1.549178in}}%
\pgfpathlineto{\pgfqpoint{3.175868in}{1.487699in}}%
\pgfpathlineto{\pgfqpoint{3.230677in}{1.497110in}}%
\pgfpathlineto{\pgfqpoint{3.285486in}{1.566941in}}%
\pgfpathlineto{\pgfqpoint{3.331160in}{1.550972in}}%
\pgfpathlineto{\pgfqpoint{3.376834in}{1.564245in}}%
\pgfpathlineto{\pgfqpoint{3.422508in}{1.552651in}}%
\pgfpathlineto{\pgfqpoint{3.468182in}{1.507701in}}%
\pgfpathlineto{\pgfqpoint{3.513856in}{1.442390in}}%
\pgfpathlineto{\pgfqpoint{3.564098in}{1.331951in}}%
\pgfpathlineto{\pgfqpoint{3.614339in}{1.288014in}}%
\pgfpathlineto{\pgfqpoint{3.664581in}{1.397628in}}%
\pgfpathlineto{\pgfqpoint{3.714822in}{1.397559in}}%
\pgfpathlineto{\pgfqpoint{3.769631in}{1.348176in}}%
\pgfpathlineto{\pgfqpoint{3.824440in}{1.287164in}}%
\pgfpathlineto{\pgfqpoint{3.879249in}{1.287164in}}%
\pgfpathlineto{\pgfqpoint{3.938626in}{1.343852in}}%
\pgfpathlineto{\pgfqpoint{3.998002in}{1.360444in}}%
\pgfpathlineto{\pgfqpoint{4.057378in}{1.434820in}}%
\pgfpathlineto{\pgfqpoint{4.112187in}{1.484982in}}%
\pgfpathlineto{\pgfqpoint{4.166996in}{1.484965in}}%
\pgfpathlineto{\pgfqpoint{4.221805in}{1.564530in}}%
\pgfpathlineto{\pgfqpoint{4.276614in}{1.683403in}}%
\pgfpathlineto{\pgfqpoint{4.331423in}{1.740921in}}%
\pgfpathlineto{\pgfqpoint{4.390799in}{1.823685in}}%
\pgfpathlineto{\pgfqpoint{4.450176in}{1.723377in}}%
\pgfpathlineto{\pgfqpoint{4.504985in}{1.732932in}}%
\pgfusepath{stroke}%
\end{pgfscope}%
\begin{pgfscope}%
\pgfpathrectangle{\pgfqpoint{0.596422in}{0.467838in}}{\pgfqpoint{4.094684in}{1.595606in}}%
\pgfusepath{clip}%
\pgfsetrectcap%
\pgfsetroundjoin%
\pgfsetlinewidth{1.003750pt}%
\definecolor{currentstroke}{rgb}{0.839216,0.152941,0.156863}%
\pgfsetstrokecolor{currentstroke}%
\pgfsetdash{}{0pt}%
\pgfpathmoveto{\pgfqpoint{0.782545in}{0.577317in}}%
\pgfpathlineto{\pgfqpoint{0.864758in}{0.889050in}}%
\pgfpathlineto{\pgfqpoint{0.924134in}{0.863365in}}%
\pgfpathlineto{\pgfqpoint{0.983511in}{0.896336in}}%
\pgfpathlineto{\pgfqpoint{1.042887in}{0.915666in}}%
\pgfpathlineto{\pgfqpoint{1.093128in}{0.925113in}}%
\pgfpathlineto{\pgfqpoint{1.138803in}{0.901875in}}%
\pgfpathlineto{\pgfqpoint{1.184477in}{1.046451in}}%
\pgfpathlineto{\pgfqpoint{1.230151in}{1.115674in}}%
\pgfpathlineto{\pgfqpoint{1.275825in}{1.176967in}}%
\pgfpathlineto{\pgfqpoint{1.321499in}{1.280399in}}%
\pgfpathlineto{\pgfqpoint{1.367173in}{1.085407in}}%
\pgfpathlineto{\pgfqpoint{1.412847in}{1.142192in}}%
\pgfpathlineto{\pgfqpoint{1.458521in}{1.118938in}}%
\pgfpathlineto{\pgfqpoint{1.504195in}{1.172902in}}%
\pgfpathlineto{\pgfqpoint{1.549870in}{1.203265in}}%
\pgfpathlineto{\pgfqpoint{1.595544in}{1.222172in}}%
\pgfpathlineto{\pgfqpoint{1.641218in}{1.214558in}}%
\pgfpathlineto{\pgfqpoint{1.686892in}{1.051262in}}%
\pgfpathlineto{\pgfqpoint{1.732566in}{1.247072in}}%
\pgfpathlineto{\pgfqpoint{1.778240in}{1.315343in}}%
\pgfpathlineto{\pgfqpoint{1.823914in}{1.452953in}}%
\pgfpathlineto{\pgfqpoint{1.869588in}{1.510639in}}%
\pgfpathlineto{\pgfqpoint{1.915262in}{1.578104in}}%
\pgfpathlineto{\pgfqpoint{1.960937in}{1.645299in}}%
\pgfpathlineto{\pgfqpoint{2.006611in}{1.697457in}}%
\pgfpathlineto{\pgfqpoint{2.052285in}{1.807491in}}%
\pgfpathlineto{\pgfqpoint{2.097959in}{1.807491in}}%
\pgfpathlineto{\pgfqpoint{2.143633in}{1.962008in}}%
\pgfpathlineto{\pgfqpoint{2.189307in}{1.954780in}}%
\pgfpathlineto{\pgfqpoint{2.234981in}{1.957301in}}%
\pgfpathlineto{\pgfqpoint{2.280655in}{1.990917in}}%
\pgfpathlineto{\pgfqpoint{2.326329in}{1.990917in}}%
\pgfpathlineto{\pgfqpoint{2.372004in}{1.760852in}}%
\pgfpathlineto{\pgfqpoint{2.417678in}{1.646538in}}%
\pgfpathlineto{\pgfqpoint{2.463352in}{1.622023in}}%
\pgfpathlineto{\pgfqpoint{2.509026in}{1.568039in}}%
\pgfpathlineto{\pgfqpoint{2.554700in}{1.582516in}}%
\pgfpathlineto{\pgfqpoint{2.600374in}{1.543620in}}%
\pgfpathlineto{\pgfqpoint{2.650616in}{1.498018in}}%
\pgfpathlineto{\pgfqpoint{2.700857in}{1.371544in}}%
\pgfpathlineto{\pgfqpoint{2.760234in}{1.385948in}}%
\pgfpathlineto{\pgfqpoint{2.819610in}{1.304320in}}%
\pgfpathlineto{\pgfqpoint{2.878986in}{1.465431in}}%
\pgfpathlineto{\pgfqpoint{2.938363in}{1.619489in}}%
\pgfpathlineto{\pgfqpoint{2.997739in}{1.650448in}}%
\pgfpathlineto{\pgfqpoint{3.057115in}{1.694098in}}%
\pgfpathlineto{\pgfqpoint{3.116492in}{1.684023in}}%
\pgfpathlineto{\pgfqpoint{3.175868in}{1.724403in}}%
\pgfpathlineto{\pgfqpoint{3.230677in}{1.768432in}}%
\pgfpathlineto{\pgfqpoint{3.285486in}{1.887326in}}%
\pgfpathlineto{\pgfqpoint{3.331160in}{1.894232in}}%
\pgfpathlineto{\pgfqpoint{3.376834in}{1.923461in}}%
\pgfpathlineto{\pgfqpoint{3.422508in}{1.890202in}}%
\pgfpathlineto{\pgfqpoint{3.468182in}{1.879873in}}%
\pgfpathlineto{\pgfqpoint{3.513856in}{1.841481in}}%
\pgfpathlineto{\pgfqpoint{3.564098in}{1.779939in}}%
\pgfpathlineto{\pgfqpoint{3.614339in}{1.737401in}}%
\pgfpathlineto{\pgfqpoint{3.664581in}{1.776404in}}%
\pgfpathlineto{\pgfqpoint{3.714822in}{1.776815in}}%
\pgfpathlineto{\pgfqpoint{3.769631in}{1.758973in}}%
\pgfpathlineto{\pgfqpoint{3.824440in}{1.625692in}}%
\pgfpathlineto{\pgfqpoint{3.879249in}{1.625692in}}%
\pgfpathlineto{\pgfqpoint{3.938626in}{1.672937in}}%
\pgfpathlineto{\pgfqpoint{3.998002in}{1.681091in}}%
\pgfpathlineto{\pgfqpoint{4.057378in}{1.749000in}}%
\pgfpathlineto{\pgfqpoint{4.112187in}{1.790877in}}%
\pgfpathlineto{\pgfqpoint{4.166996in}{1.789712in}}%
\pgfpathlineto{\pgfqpoint{4.221805in}{1.754205in}}%
\pgfpathlineto{\pgfqpoint{4.276614in}{1.790501in}}%
\pgfpathlineto{\pgfqpoint{4.331423in}{1.762857in}}%
\pgfpathlineto{\pgfqpoint{4.390799in}{1.900987in}}%
\pgfpathlineto{\pgfqpoint{4.450176in}{1.816325in}}%
\pgfpathlineto{\pgfqpoint{4.504985in}{1.822560in}}%
\pgfusepath{stroke}%
\end{pgfscope}%
\begin{pgfscope}%
\pgfsetrectcap%
\pgfsetmiterjoin%
\pgfsetlinewidth{0.803000pt}%
\definecolor{currentstroke}{rgb}{0.000000,0.000000,0.000000}%
\pgfsetstrokecolor{currentstroke}%
\pgfsetdash{}{0pt}%
\pgfpathmoveto{\pgfqpoint{0.596422in}{0.467838in}}%
\pgfpathlineto{\pgfqpoint{0.596422in}{2.063444in}}%
\pgfusepath{stroke}%
\end{pgfscope}%
\begin{pgfscope}%
\pgfsetrectcap%
\pgfsetmiterjoin%
\pgfsetlinewidth{0.803000pt}%
\definecolor{currentstroke}{rgb}{0.000000,0.000000,0.000000}%
\pgfsetstrokecolor{currentstroke}%
\pgfsetdash{}{0pt}%
\pgfpathmoveto{\pgfqpoint{4.691107in}{0.467838in}}%
\pgfpathlineto{\pgfqpoint{4.691107in}{2.063444in}}%
\pgfusepath{stroke}%
\end{pgfscope}%
\begin{pgfscope}%
\pgfsetrectcap%
\pgfsetmiterjoin%
\pgfsetlinewidth{0.803000pt}%
\definecolor{currentstroke}{rgb}{0.000000,0.000000,0.000000}%
\pgfsetstrokecolor{currentstroke}%
\pgfsetdash{}{0pt}%
\pgfpathmoveto{\pgfqpoint{0.596422in}{0.467838in}}%
\pgfpathlineto{\pgfqpoint{4.691107in}{0.467838in}}%
\pgfusepath{stroke}%
\end{pgfscope}%
\begin{pgfscope}%
\pgfsetrectcap%
\pgfsetmiterjoin%
\pgfsetlinewidth{0.803000pt}%
\definecolor{currentstroke}{rgb}{0.000000,0.000000,0.000000}%
\pgfsetstrokecolor{currentstroke}%
\pgfsetdash{}{0pt}%
\pgfpathmoveto{\pgfqpoint{0.596422in}{2.063444in}}%
\pgfpathlineto{\pgfqpoint{4.691107in}{2.063444in}}%
\pgfusepath{stroke}%
\end{pgfscope}%
\begin{pgfscope}%
\pgfsetbuttcap%
\pgfsetmiterjoin%
\definecolor{currentfill}{rgb}{1.000000,1.000000,1.000000}%
\pgfsetfillcolor{currentfill}%
\pgfsetfillopacity{0.800000}%
\pgfsetlinewidth{1.003750pt}%
\definecolor{currentstroke}{rgb}{0.800000,0.800000,0.800000}%
\pgfsetstrokecolor{currentstroke}%
\pgfsetstrokeopacity{0.800000}%
\pgfsetdash{}{0pt}%
\pgfpathmoveto{\pgfqpoint{0.674200in}{1.322212in}}%
\pgfpathlineto{\pgfqpoint{1.535995in}{1.322212in}}%
\pgfpathquadraticcurveto{\pgfqpoint{1.558217in}{1.322212in}}{\pgfqpoint{1.558217in}{1.344434in}}%
\pgfpathlineto{\pgfqpoint{1.558217in}{1.985666in}}%
\pgfpathquadraticcurveto{\pgfqpoint{1.558217in}{2.007889in}}{\pgfqpoint{1.535995in}{2.007889in}}%
\pgfpathlineto{\pgfqpoint{0.674200in}{2.007889in}}%
\pgfpathquadraticcurveto{\pgfqpoint{0.651978in}{2.007889in}}{\pgfqpoint{0.651978in}{1.985666in}}%
\pgfpathlineto{\pgfqpoint{0.651978in}{1.344434in}}%
\pgfpathquadraticcurveto{\pgfqpoint{0.651978in}{1.322212in}}{\pgfqpoint{0.674200in}{1.322212in}}%
\pgfpathclose%
\pgfusepath{stroke,fill}%
\end{pgfscope}%
\begin{pgfscope}%
\pgfsetrectcap%
\pgfsetroundjoin%
\pgfsetlinewidth{1.003750pt}%
\definecolor{currentstroke}{rgb}{0.121569,0.466667,0.705882}%
\pgfsetstrokecolor{currentstroke}%
\pgfsetdash{}{0pt}%
\pgfpathmoveto{\pgfqpoint{0.696422in}{1.917915in}}%
\pgfpathlineto{\pgfqpoint{0.918645in}{1.917915in}}%
\pgfusepath{stroke}%
\end{pgfscope}%
\begin{pgfscope}%
\definecolor{textcolor}{rgb}{0.000000,0.000000,0.000000}%
\pgfsetstrokecolor{textcolor}%
\pgfsetfillcolor{textcolor}%
\pgftext[x=1.007534in,y=1.879026in,left,base]{\color{textcolor}\sffamily\fontsize{8.000000}{9.600000}\selectfont ProCount}%
\end{pgfscope}%
\begin{pgfscope}%
\pgfsetrectcap%
\pgfsetroundjoin%
\pgfsetlinewidth{1.003750pt}%
\definecolor{currentstroke}{rgb}{1.000000,0.498039,0.054902}%
\pgfsetstrokecolor{currentstroke}%
\pgfsetdash{}{0pt}%
\pgfpathmoveto{\pgfqpoint{0.696422in}{1.754829in}}%
\pgfpathlineto{\pgfqpoint{0.918645in}{1.754829in}}%
\pgfusepath{stroke}%
\end{pgfscope}%
\begin{pgfscope}%
\definecolor{textcolor}{rgb}{0.000000,0.000000,0.000000}%
\pgfsetstrokecolor{textcolor}%
\pgfsetfillcolor{textcolor}%
\pgftext[x=1.007534in,y=1.715940in,left,base]{\color{textcolor}\sffamily\fontsize{8.000000}{9.600000}\selectfont D4\textsubscript{P}}%
\end{pgfscope}%
\begin{pgfscope}%
\pgfsetrectcap%
\pgfsetroundjoin%
\pgfsetlinewidth{1.003750pt}%
\definecolor{currentstroke}{rgb}{0.172549,0.627451,0.172549}%
\pgfsetstrokecolor{currentstroke}%
\pgfsetdash{}{0pt}%
\pgfpathmoveto{\pgfqpoint{0.696422in}{1.591743in}}%
\pgfpathlineto{\pgfqpoint{0.918645in}{1.591743in}}%
\pgfusepath{stroke}%
\end{pgfscope}%
\begin{pgfscope}%
\definecolor{textcolor}{rgb}{0.000000,0.000000,0.000000}%
\pgfsetstrokecolor{textcolor}%
\pgfsetfillcolor{textcolor}%
\pgftext[x=1.007534in,y=1.552854in,left,base]{\color{textcolor}\sffamily\fontsize{8.000000}{9.600000}\selectfont projMC}%
\end{pgfscope}%
\begin{pgfscope}%
\pgfsetrectcap%
\pgfsetroundjoin%
\pgfsetlinewidth{1.003750pt}%
\definecolor{currentstroke}{rgb}{0.839216,0.152941,0.156863}%
\pgfsetstrokecolor{currentstroke}%
\pgfsetdash{}{0pt}%
\pgfpathmoveto{\pgfqpoint{0.696422in}{1.428657in}}%
\pgfpathlineto{\pgfqpoint{0.918645in}{1.428657in}}%
\pgfusepath{stroke}%
\end{pgfscope}%
\begin{pgfscope}%
\definecolor{textcolor}{rgb}{0.000000,0.000000,0.000000}%
\pgfsetstrokecolor{textcolor}%
\pgfsetfillcolor{textcolor}%
\pgftext[x=1.007534in,y=1.389768in,left,base]{\color{textcolor}\sffamily\fontsize{8.000000}{9.600000}\selectfont reSSAT}%
\end{pgfscope}%
\end{pgfpicture}%
\makeatother%
\endgroup%

    \caption{
        A plot of mean PAR-2 scores (in seconds) against mean project-join tree widths.
        On this plot, each projected model counter (\procount{}, \dfp{}, \projmc, or \ssat) is represented by a curve, on which a point $(x, y)$ indicates that $x$ is the central moving average of 10 consecutive project-join tree widths ($1 \le w_1 < w_2 < \ldots < w_{10} \le 99$) and $y$ is the average PAR-2 score of the benchmarks whose project-join trees have widths $w$ where $w_1 \le w \le w_{10}$.
        We observe that the performance of \procount{} degrades as the project-join tree width increases.
        However, \procount{} tends to be the fastest solver on benchmarks whose graded project-join trees have widths roughly between 50 and 100.
    }
    \label{time_vs_width}
\end{figure}
