\subsection{Planning with Anytime Tree-Decomposition Tools}
\label{sec_td}

In join-query optimization, tree decompositions \cite{RS91} of the \emph{join graph} are used to find optimal join trees \cite{DKV02,MPPV04}.
The \emph{join graph} of a project-join query consists of all attributes of a database as vertices and all tables as cliques.
In this approach, tree decompositions of the join graph of a query are used to find optimal project-join trees; see Algorithm 3 of \cite{MPPV04}.
Similarly, tree decompositions of the \emph{primal graph} of a factor graph, which consists of all variables as vertices and all factors as cliques, can be used to find variable elimination orders \cite{KDLD05}.
Section \ref{sec:tensors:contraction-theory} discussed this technique in the context of tensor networks.

Translated to our project-join-tree framework, these techniques allows us to use tree decompositions of the \textdef{Gaifman graph} of a CNF formula to compute project-join trees.
The Gaifman graph of a CNF formula $\phi$, denoted $\gaifman(\phi)$, has a vertex for each variable of $\phi$, and two vertices are adjacent if the corresponding variables appear together in some clause of $\phi$.
The key idea of the technique is that each clause $c$ of $\phi$ forms a clique in $\gaifman(\phi)$ between the variables of $c$.
Thus all variables of $c$ must appear together in the label of some node of the tree decomposition.
We identify that node with $c$ in the resulting project-join tree.
\begin{algorithm*}
\caption{Using a tree decomposition to build a project-join tree}
\label{alg_td_to_join}
    \DontPrintSemicolon
    \KwIn{$X$: set of Boolean variables}
    \KwIn{$\phi$: CNF formula over $X$}
    \KwIn{$(S, \chi)$: tree decomposition of the Gaifman graph of $\phi$}
    \KwOut{$(T, r, \gamma, \pi)$: project-join tree of $\phi$}
    $(T, \nil, \gamma, \pi) \gets \text{empty project-join tree}$\;
    $found \gets \emptyset$\tcc*{clauses of $\phi$ that have been added to $T$}
    $s \gets$ arbitrary node of $S$ \label{line_arbitrary_node}
        \tcc*{fixing $s$ as root of $S$}
    \Function{\upshape $\func{Process}(n, \ell)$}{
        \KwIn{$n \in \V{S}$: node of $S$ to process}
        \KwIn{$\ell \subseteq X$: variables that must not be projected out here}
        \KwOut{$N \subseteq \V{T}$}
        $clauses \gets \{ c \in \phi : c \notin found~\text{and}~\vars(c) \subseteq \chi(n)\}$\; \label{line_clauses}
        $found \gets found \cup clauses$\; \label{line_found}
        $children \gets \{\leaf(T, c) : c \in clauses\} \cup \bigcup_{o \in \C{S}{s}{n}} \func{Process}(o, \chi(n))$\; \label{line_recur}
            \tcc*{new leaf nodes $p \in \V T$ with $\gamma(p) = c$}
        \If{$children = \emptyset$~\text{\upshape or}~$\chi(n) \subseteq \ell$}{
            \Return{children}
        }
        \Else{
            \Return{\upshape $\{\internal(T, children, \chi(n) \setminus \ell)\}$}\; \label{line_return_singleton}
                \tcc*{new internal node $o \in \V T$ with label $\pi(o) = \chi(n) \setminus \ell$}
        }
    }
    $r \gets \text{only element of}~\func{Process}(s, \emptyset)$\;
    \Return{$(T, r, \gamma, \pi)$}
\end{algorithm*}

We present this tree-decomposition-based technique as Algorithm \ref{alg_td_to_join}.
We first prove in the following theorem that Algorithm \ref{alg_td_to_join} is correct, i.e. that it indeed returns a project-join tree. 
\begin{theorem}
\label{thm_td_to_join_correct}
	Let $\phi$ be a CNF formula over a set $X$ of variables and $(S, \chi)$ be a tree decomposition of $\gaifman(\phi)$.
    Then Algorithm \ref{alg_td_to_join} returns a project-join tree of $\phi$.
\end{theorem}
\begin{proof}
Let $\mathcal{T} = (T, r, \gamma, \pi)$ be the object returned by Algorithm \ref{alg_td_to_join}. For each node $a \in \V{T}$, $a$ was created in some $\func{Process}$ call during the execution of Algorithm \ref{alg_td_to_join}; let $O(a)$ denote the node in $\V{S}$ so that $\func{Process}(O(a), \ell)$ was the call that created $a$ (for some $\ell \subseteq X$).
Moreover, let $s$ denote the value obtained on Line \ref{line_arbitrary_node} of Algorithm \ref{alg_td_to_join}.

We first observe that $T$ is indeed a tree with root $r$. To prove that $\mathcal{T}$ is a project-join tree we must additionally verify that $\gamma$ is a bijection and that both properties from Definition \ref{def_jointree} hold.  

\paragraph{Part 1: $\gamma$ is a bijection.}
    Note that $\gamma$ is an injection since $found$ on Line \ref{line_found} of Algorithm \ref{alg_td_to_join} ensures that we generate at most one leaf node for each clause.
    To show that $\gamma$ is a surjection, consider $c \in \phi$.
    Then $\vars(c)$ forms a clique in the Gaifman graph of $\phi$.
    It follows (since the treewidth of a complete graph on $k$ vertices is $k-1$) that $\vars(c) \subseteq \chi(n)$ for some $n \in \V{S}$.
    Thus $\gamma$ is a surjection as well.
    
\paragraph{Part 2: Property 1 of Definition \ref{def_jointree}.} 
    That is, we must verify that $P = \{\pi(a) : a \in \V{T} \setminus \Lv{T} \}$ is a partition of $X$.
    First, let $x \in X$.
    Then $x \in \vars(c)$ for some $c \in \phi$. 
    Let $n_0 = O(\gamma^{-1}(c)) \in \V{S}$ and observe that $x \in \chi(n_0)$.
    Consider the sequence of nodes $n_0, n_1, n_2, \cdots, n_k = s \in \V{S}$ on the unique shortest path from $n_0$ to $s$ in $S$.
    Observe that, by Property 3 of tree decompositions in Definition \ref{def:treedecomposition}, there is some $0 \leq k' \leq k$ s.t. $x \in \chi(n_i)$ for all $0 \leq i \leq k'$ and $x \not \in \chi(n_i)$ for all $k' < i \leq k$.
    If $k' = k$, then $x \in \chi(s)$ and so $x \in \pi(r)$.
    Otherwise,  $x \in \chi(n_{k'})$ but $x \not\in \chi(n_{k'+1})$.
    Consider the call on Line \ref{line_recur} to $\func{Process}(o, \chi(n))$ when $o = n_{k'}$ and $n = n_{k'+1}$.
    Within this call, $\chi(n_{k'} \not\subseteq \ell = \chi(n_{k'+1}))$ and so Line \ref{line_return_singleton} will return an internal node $a \in \V{T}$ with $O(a) = n_{k'}$ and $x \in \pi(a)$.

    On the other hand, consider distinct $a, b \in \V{T}$. Since $S$ is a tree, there is some node $p \in \V{S}$ on the path between $O(a)$ and $O(b)$ \st{} $p$ is the parent of either $O(a)$ or $O(b)$. By Line \ref{line_recur} and Line \ref{line_return_singleton}, this means that $\chi(p)$ is disjoint from either $\pi(a)$ or $\pi(b)$. In either case, $\chi(p)$ is disjoint from $\pi(a) \cap \pi(b)$. By Line \ref{line_return_singleton} of Algorithm \ref{alg_td_to_join}, we also know that $\pi(a) \cap \pi(b) \subseteq \chi(O(a)) \cap \chi(O(b))$. By Property 3 of tree decompositions in Definition \ref{def:treedecomposition}, $\chi(O(a)) \cap \chi(O(b))$ (and thus $\pi(a) \cap \pi(b)$) must be contained in $\chi(p)$. It follows that $\pi(a) \cap \pi(b) = \emptyset$.

\paragraph{Part 3: Property 2 of Definition \ref{def_jointree}.}
    That is, we must verify that, for each internal node $a \in \V{T} \setminus \Lv{T}$, variable $x \in \pi(a)$, and clause $c \in \phi$ \st{} $x$ appears in $c$, the leaf node $\gamma^{-1}(c)$ is a descendant of $a$ in $T$.
    If $O(a) = s$, then $a$ is the root of $T$, so all leaf nodes are descendants.
    Otherwise, assume for the sake of contradiction that $\gamma^{-1}(c)$ is not a descendant of $a$ in $T$.
    Then $O(\gamma^{-1}(c))$ is not a descendant of $O(a)$ in $S$.
    This means that the parent $p \in \V{S}$ of $O(a)$ is on the path between $O(a)$ and $O(\gamma^{-1}(c))$. 
    By Property 3 of tree decompositions in Definition \ref{def:treedecomposition}, $x \in \chi(p)$.
    But Line \ref{line_recur} and Line \ref{line_return_singleton} then implies that $x \not\in \pi(a)$, a contradiction.
    Thus $\gamma^{-1}(c)$ is a descendant of $a$ in $T$.
    
\paragraph{} It follows that $\mathcal{T}$ is a project-join tree of $\phi$.
\end{proof}

The width of the resulting project-join tree is closely connected to the width of the original tree decomposition.
We formalize this in the following theorem.
\begin{theorem}
\label{thm_td_to_join}
	Let $\phi$ be a CNF formula over a set $X$ of variables and $(S, \chi)$ be a tree decomposition of $\gaifman(\phi)$ of width $w$. Then the project-join tree $\mathcal{T}$ of $\varphi$ returned by Algorithm \ref{alg_td_to_join} has width at most $w+1$.
\end{theorem}
\begin{proof}
The key idea is that for all $n \in \V{S}$ and $\ell \subseteq X$, every node $i \in \func{Process}(n, \ell)$ has $\vars(i) \subseteq \ell$. We prove this by induction on the tree structure of $S$.

Let $A$ denote the set $children$ after Line \ref{line_recur} occurs. 
We begin by proving that $\vars(a) \subseteq \chi(n)$ for each $a \in A$.
First, assume that $a$ is a leaf node of $\mathcal{T}$ corresponding to some $c \in clauses$.
In this case, $\vars(a) = \vars(c) \subseteq \chi(n)$ by Line \ref{line_clauses}.
Otherwise $a \in \func{Process}(o, \chi(n))$ for some $o \in C(n)$.
In this case, notice that $n$ is an internal node, so by the inductive hypothesis, $\vars(a) \subseteq \chi(n)$.

If $A = \emptyset$, then $\func{Process}(n, \ell)$ returns $\emptyset$, so the lemma is vacuously true.
If $\chi(n) \subseteq \ell$, then $A$ is returned by $\func{Process}(n, \ell)$.
So for every $i \in A$, we have $\vars(i) \subseteq \chi(n) \subseteq \ell$.
Otherwise, $A \neq \emptyset$ and $\chi(n) \not\subseteq \ell$.
In this case, $\func{Process}(n, \ell)$ returns a single node $i$ with $\vars(i) = \cup_{a \in A} \vars(a) \setminus (\chi(n) \setminus \ell) \subseteq \chi(n) \setminus (\chi(n) \setminus \ell) \subseteq \ell$.

This completes the induction. To complete the proof, notice that every internal node of $\mathcal{T}$ is returned by some call of 
$\func{Process}$ in Algorithm \ref{alg_td_to_join}. Since every call Algorithm \ref{alg_td_to_join} makes to $\func{Process}$ has $\ell \in \{\emptyset\} \cup \{ \chi(n) : n \in \V{S}\}$, it follows that, for every internal node $i$ of $\mathcal{T}$, $| \vars(i) | \leq w+1$. Moreover, for every leaf node $i$ of $\mathcal{T}$ there is a corresponding clauses $c \in \varphi$ with $\vars(i) = \vars(c)$. Thus there is a clique in $\gaifman(\varphi)$ of $|\vars(i)|$ vertices, which implies that $w \geq |\vars(i)| - 1$. Thus the width of $\mathcal{T}$ is at most $w+1$.
\end{proof}

Theorem \ref{thm_td_to_join} allows us to leverage state-of-the-art tools for finding tree decompositions \cite{Tamaki17,strasser2017computing,AMW17} to construct project-join trees, which we do in Section \ref{sec_experiments_planning}.

On the theoretical front, it is well-known that tree decompositions of the Gaifman graph are actually equivalent to project-join trees \cite{MPPV04}.
That is, one can go in the other direction as well: given a project-join tree of $\phi$, one can construct a tree decomposition of $\gaifman(\phi)$ of equivalent width.
Formally:
\begin{theorem}
\label{thm_join_to_td}
    Let $\phi$ be a CNF formula and $(T, r, \gamma, \pi)$ be a project-join tree of $\phi$ of width $w$.
    Then there is a tree decomposition of $\gaifman(\phi)$ of width $w-1$.
\end{theorem}
\begin{proof}
    Define $\chi: \V{T} \to 2^\vars(\phi)$ by, for all $n \in \V{T}$,
    % $\chi(n) \equiv \vars(f_n)$%
    $\chi(n) \equiv\vars(n)$ if $n \in \Lv{T}$ and $\chi(n) \equiv \vars(n) \cup \pi(n)$ otherwise.
    Then $(T, \chi)$ is a tree decomposition of the Gaifman graph of $\phi$.
    Moreover, the width of $(T, \chi)$ is $\max_{n \in \V{T}} \size{\chi (n)}  - 1 = \max_{n \in \V T} \func{size}(n) - 1 = w - 1$.
\end{proof}
Theorem \ref{thm_join_to_td} is Lemma 1 of \cite{MPPV04} and can be seen as the inverse of Theorem \ref{thm_td_to_join}.

% Notice the difference between Algorithm \ref{alg_td_to_join}, which uses tree decompositions to construct project-join trees, and Theorem \ref{thm:carving-equiv-tree}, which uses tree decompositions to construct carving decompositions. 
% This is because the width of a project-join tree is analogous to the contraction complexity of a contraction tree, while (by Theorem \ref{thm:contraction-equiv-carving}) the width of a carving decomposition is exactly the max rank of the corresponding contraction tree.