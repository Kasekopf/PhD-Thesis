\section{Implementation and Evaluation}
\label{sec:experiments}

We aim to answer the following experimental research questions:
\begin{enumerate}\itemsep0em 
    \item[(RQ1)] Is the planning phase improved by a parallel portfolio of decomposition tools?
    
    \item[(RQ2)] Is the planning phase improved by adding a branch-decomposition tool?
    
    \item[(RQ3)] When should Algorithm \ref{alg:wmc} transition from the planning phase to the execution phase (i.e., what should be the value of the performance factor $\alpha$)?
    
    \item[(RQ4)] Is the execution phase improved by leveraging multiple cores and a GPU?
    
    \item[(RQ5)] Do parallel tensor network approaches improve a portfolio of state-of-the-art weighted model counters (\tool{cachet}, \tool{miniC2D}, \tool{d4}, \tool{ADDMC}, and \tool{gpuSAT2})?
\end{enumerate}

We implement our changes on top of \tool{TensorOrder} \cite{DDV19} (which implements Algorithm \ref{alg:wmc}; see Section \ref{sec:tensors:experiments:implementation}) to produce \tool{TensorOrder2}, a new parallel weighted model counter. Implementation details are described in Section \ref{sec:experiments:impl}. All code is available at  \url{https://github.com/vardigroup/TensorOrder}.

%\paragraph{Benchmarks.} 
We use a standard set\footnote{\url{https://github.com/vardigroup/ADDMC/releases/tag/v1.0.0}} of 1914 weighted model counting benchmarks \cite{DPV20}. Of these, 1091 benchmarks\footnote{\url{https://www.cs.rochester.edu/u/kautz/Cachet/}} are from Bayesian inference problems \cite{SBK05} and 823 benchmarks\footnote{\url{http://www.cril.univ-artois.fr/KC/benchmarks.html}} are unweighted benchmarks (from various domains) that were artificially weighted by \cite{DPV20}. For weighted model counters that cannot handle real weights larger than 1 (\tool{cachet} and \tool{gpusat2}), we rescale the weights of benchmarks with larger weights. In Experiment 3, we also consider preprocessing these 1914 benchmarks by applying $\pkg{pmc-eq}$ \cite{LM14} (which preserves weighted model count). % We apply to each benchmark, with a timeout of 1000 seconds. We remove 61 benchmarks that were fully solved by $\pkg{pmc-eq}$ (either UNSAT., or with a single solution) and 11 benchmarks that timed out, resulting in 1842 preprocessed benchmarks.
We evaluate the performance of each tool using the PAR-2 score, which is the sum of of the wall-clock times for each completed benchmark, plus twice the timeout for each uncompleted benchmark.

%\paragraph{Experimental Setup.}
All counters are run in the Docker images (one for each counter) with Docker 19.03.5. All experiments are run on Google Cloud \tool{n1-standard-8} machines with 8 cores (Intel Haswell, 2.3 GHz) and 30 GB RAM. GPU-based counters are provided an \tool{NVIDIA Tesla V100} GPU (16 GB of onboard RAM) using NVIDIA driver 418.67 and CUDA 10.1.243.

\subsection{Implementation Details of \tool{TensorOrder2}}
\label{sec:experiments:impl}
\tool{TensorOrder2} is primarily implemented in Python 3 as a modified version of the tool from Chapter \ref{ch:tensors}, \tool{TensorOrder}. We replace portions of the Python code with C++ (\tool{g++} v7.4.0) using Cython 0.29.15 for general speedup, especially in $\func{FactorTree}(\cdot)$.

\paragraph{Planning.} 
\tool{TensorOrder} contains an implementation of the planning phase using a choice of three single-core tree-decomposition solvers: \pkg{Tamaki} \cite{Tamaki17}, \pkg{FlowCutter} \cite{HS18}, and \pkg{htd} \cite{AMW17}. We add to \tool{TensorOrder2} an implementation of Theorem \ref{thm:factorable-branch} and use it to add a branch-decomposition solver \pkg{Hicks} \cite{hicks02}.
We implement a parallel portfolio of graph-decomposition solvers in C++ and give \tool{TensorOrder2} access to two portfolios, each with access to all cores: \pkg{P3} (which combines \pkg{Tamaki},  \pkg{FlowCutter}, and \pkg{htd}) and \pkg{P4} (which includes \pkg{Hicks} as well).

\paragraph{Execution.} 
\tool{TensorOrder} is able to perform the execution phase on a single core and on multiple cores using \pkg{numpy} v1.18.1 and \pkg{OpenBLAS} v0.2.20. In \tool{TensorOrder2}, we add the ability to contract tensors on a GPU with \pkg{TensorFlow} v2.1.0 \cite{ABCCDDDGII16}. To avoid GPU kernel calls for small contractions, \tool{TensorOrder2} uses a GPU only for contractions where one of the tensors involved has rank $\geq 20$, and reverts back to using multi-core \pkg{numpy} otherwise. We also add an implementation of Algorithm \ref{alg:tn-sliced}.
Overall, \tool{TensorOrder2} runs the execution phase on three hardware configurations: \pkg{CPU1} (restricted to a single CPU core), \pkg{CPU8} (allowed to use all 8 CPU cores), and \pkg{GPU} (allowed to use all 8 CPU cores and use a GPU).

\begin{figure}
	\centering
	%% Creator: Matplotlib, PGF backend
%%
%% To include the figure in your LaTeX document, write
%%   \input{<filename>.pgf}
%%
%% Make sure the required packages are loaded in your preamble
%%   \usepackage{pgf}
%%
%% and, on pdftex
%%   \usepackage[utf8]{inputenc}\DeclareUnicodeCharacter{2212}{-}
%%
%% or, on luatex and xetex
%%   \usepackage{unicode-math}
%%
%% Figures using additional raster images can only be included by \input if
%% they are in the same directory as the main LaTeX file. For loading figures
%% from other directories you can use the `import` package
%%   \usepackage{import}
%%
%% and then include the figures with
%%   \import{<path to file>}{<filename>.pgf}
%%
%% Matplotlib used the following preamble
%%   \usepackage[utf8x]{inputenc}
%%   \usepackage[T1]{fontenc}
%%
\begingroup%
\makeatletter%
\begin{pgfpicture}%
\pgfpathrectangle{\pgfpointorigin}{\pgfqpoint{4.803148in}{2.021259in}}%
\pgfusepath{use as bounding box, clip}%
\begin{pgfscope}%
\pgfsetbuttcap%
\pgfsetmiterjoin%
\definecolor{currentfill}{rgb}{1.000000,1.000000,1.000000}%
\pgfsetfillcolor{currentfill}%
\pgfsetlinewidth{0.000000pt}%
\definecolor{currentstroke}{rgb}{1.000000,1.000000,1.000000}%
\pgfsetstrokecolor{currentstroke}%
\pgfsetdash{}{0pt}%
\pgfpathmoveto{\pgfqpoint{0.000000in}{0.000000in}}%
\pgfpathlineto{\pgfqpoint{4.803148in}{0.000000in}}%
\pgfpathlineto{\pgfqpoint{4.803148in}{2.021259in}}%
\pgfpathlineto{\pgfqpoint{0.000000in}{2.021259in}}%
\pgfpathclose%
\pgfusepath{fill}%
\end{pgfscope}%
\begin{pgfscope}%
\pgfsetbuttcap%
\pgfsetmiterjoin%
\definecolor{currentfill}{rgb}{1.000000,1.000000,1.000000}%
\pgfsetfillcolor{currentfill}%
\pgfsetlinewidth{0.000000pt}%
\definecolor{currentstroke}{rgb}{0.000000,0.000000,0.000000}%
\pgfsetstrokecolor{currentstroke}%
\pgfsetstrokeopacity{0.000000}%
\pgfsetdash{}{0pt}%
\pgfpathmoveto{\pgfqpoint{0.694334in}{0.523557in}}%
\pgfpathlineto{\pgfqpoint{4.524677in}{0.523557in}}%
\pgfpathlineto{\pgfqpoint{4.524677in}{1.826535in}}%
\pgfpathlineto{\pgfqpoint{0.694334in}{1.826535in}}%
\pgfpathclose%
\pgfusepath{fill}%
\end{pgfscope}%
\begin{pgfscope}%
\pgfsetbuttcap%
\pgfsetroundjoin%
\definecolor{currentfill}{rgb}{0.000000,0.000000,0.000000}%
\pgfsetfillcolor{currentfill}%
\pgfsetlinewidth{0.803000pt}%
\definecolor{currentstroke}{rgb}{0.000000,0.000000,0.000000}%
\pgfsetstrokecolor{currentstroke}%
\pgfsetdash{}{0pt}%
\pgfsys@defobject{currentmarker}{\pgfqpoint{0.000000in}{-0.048611in}}{\pgfqpoint{0.000000in}{0.000000in}}{%
\pgfpathmoveto{\pgfqpoint{0.000000in}{0.000000in}}%
\pgfpathlineto{\pgfqpoint{0.000000in}{-0.048611in}}%
\pgfusepath{stroke,fill}%
}%
\begin{pgfscope}%
\pgfsys@transformshift{0.694334in}{0.523557in}%
\pgfsys@useobject{currentmarker}{}%
\end{pgfscope}%
\end{pgfscope}%
\begin{pgfscope}%
\definecolor{textcolor}{rgb}{0.000000,0.000000,0.000000}%
\pgfsetstrokecolor{textcolor}%
\pgfsetfillcolor{textcolor}%
\pgftext[x=0.694334in,y=0.426335in,,top]{\color{textcolor}\rmfamily\fontsize{9.000000}{10.800000}\selectfont \(\displaystyle 0\)}%
\end{pgfscope}%
\begin{pgfscope}%
\pgfsetbuttcap%
\pgfsetroundjoin%
\definecolor{currentfill}{rgb}{0.000000,0.000000,0.000000}%
\pgfsetfillcolor{currentfill}%
\pgfsetlinewidth{0.803000pt}%
\definecolor{currentstroke}{rgb}{0.000000,0.000000,0.000000}%
\pgfsetstrokecolor{currentstroke}%
\pgfsetdash{}{0pt}%
\pgfsys@defobject{currentmarker}{\pgfqpoint{0.000000in}{-0.048611in}}{\pgfqpoint{0.000000in}{0.000000in}}{%
\pgfpathmoveto{\pgfqpoint{0.000000in}{0.000000in}}%
\pgfpathlineto{\pgfqpoint{0.000000in}{-0.048611in}}%
\pgfusepath{stroke,fill}%
}%
\begin{pgfscope}%
\pgfsys@transformshift{1.173127in}{0.523557in}%
\pgfsys@useobject{currentmarker}{}%
\end{pgfscope}%
\end{pgfscope}%
\begin{pgfscope}%
\definecolor{textcolor}{rgb}{0.000000,0.000000,0.000000}%
\pgfsetstrokecolor{textcolor}%
\pgfsetfillcolor{textcolor}%
\pgftext[x=1.173127in,y=0.426335in,,top]{\color{textcolor}\rmfamily\fontsize{9.000000}{10.800000}\selectfont \(\displaystyle 250\)}%
\end{pgfscope}%
\begin{pgfscope}%
\pgfsetbuttcap%
\pgfsetroundjoin%
\definecolor{currentfill}{rgb}{0.000000,0.000000,0.000000}%
\pgfsetfillcolor{currentfill}%
\pgfsetlinewidth{0.803000pt}%
\definecolor{currentstroke}{rgb}{0.000000,0.000000,0.000000}%
\pgfsetstrokecolor{currentstroke}%
\pgfsetdash{}{0pt}%
\pgfsys@defobject{currentmarker}{\pgfqpoint{0.000000in}{-0.048611in}}{\pgfqpoint{0.000000in}{0.000000in}}{%
\pgfpathmoveto{\pgfqpoint{0.000000in}{0.000000in}}%
\pgfpathlineto{\pgfqpoint{0.000000in}{-0.048611in}}%
\pgfusepath{stroke,fill}%
}%
\begin{pgfscope}%
\pgfsys@transformshift{1.651920in}{0.523557in}%
\pgfsys@useobject{currentmarker}{}%
\end{pgfscope}%
\end{pgfscope}%
\begin{pgfscope}%
\definecolor{textcolor}{rgb}{0.000000,0.000000,0.000000}%
\pgfsetstrokecolor{textcolor}%
\pgfsetfillcolor{textcolor}%
\pgftext[x=1.651920in,y=0.426335in,,top]{\color{textcolor}\rmfamily\fontsize{9.000000}{10.800000}\selectfont \(\displaystyle 500\)}%
\end{pgfscope}%
\begin{pgfscope}%
\pgfsetbuttcap%
\pgfsetroundjoin%
\definecolor{currentfill}{rgb}{0.000000,0.000000,0.000000}%
\pgfsetfillcolor{currentfill}%
\pgfsetlinewidth{0.803000pt}%
\definecolor{currentstroke}{rgb}{0.000000,0.000000,0.000000}%
\pgfsetstrokecolor{currentstroke}%
\pgfsetdash{}{0pt}%
\pgfsys@defobject{currentmarker}{\pgfqpoint{0.000000in}{-0.048611in}}{\pgfqpoint{0.000000in}{0.000000in}}{%
\pgfpathmoveto{\pgfqpoint{0.000000in}{0.000000in}}%
\pgfpathlineto{\pgfqpoint{0.000000in}{-0.048611in}}%
\pgfusepath{stroke,fill}%
}%
\begin{pgfscope}%
\pgfsys@transformshift{2.130713in}{0.523557in}%
\pgfsys@useobject{currentmarker}{}%
\end{pgfscope}%
\end{pgfscope}%
\begin{pgfscope}%
\definecolor{textcolor}{rgb}{0.000000,0.000000,0.000000}%
\pgfsetstrokecolor{textcolor}%
\pgfsetfillcolor{textcolor}%
\pgftext[x=2.130713in,y=0.426335in,,top]{\color{textcolor}\rmfamily\fontsize{9.000000}{10.800000}\selectfont \(\displaystyle 750\)}%
\end{pgfscope}%
\begin{pgfscope}%
\pgfsetbuttcap%
\pgfsetroundjoin%
\definecolor{currentfill}{rgb}{0.000000,0.000000,0.000000}%
\pgfsetfillcolor{currentfill}%
\pgfsetlinewidth{0.803000pt}%
\definecolor{currentstroke}{rgb}{0.000000,0.000000,0.000000}%
\pgfsetstrokecolor{currentstroke}%
\pgfsetdash{}{0pt}%
\pgfsys@defobject{currentmarker}{\pgfqpoint{0.000000in}{-0.048611in}}{\pgfqpoint{0.000000in}{0.000000in}}{%
\pgfpathmoveto{\pgfqpoint{0.000000in}{0.000000in}}%
\pgfpathlineto{\pgfqpoint{0.000000in}{-0.048611in}}%
\pgfusepath{stroke,fill}%
}%
\begin{pgfscope}%
\pgfsys@transformshift{2.609506in}{0.523557in}%
\pgfsys@useobject{currentmarker}{}%
\end{pgfscope}%
\end{pgfscope}%
\begin{pgfscope}%
\definecolor{textcolor}{rgb}{0.000000,0.000000,0.000000}%
\pgfsetstrokecolor{textcolor}%
\pgfsetfillcolor{textcolor}%
\pgftext[x=2.609506in,y=0.426335in,,top]{\color{textcolor}\rmfamily\fontsize{9.000000}{10.800000}\selectfont \(\displaystyle 1000\)}%
\end{pgfscope}%
\begin{pgfscope}%
\pgfsetbuttcap%
\pgfsetroundjoin%
\definecolor{currentfill}{rgb}{0.000000,0.000000,0.000000}%
\pgfsetfillcolor{currentfill}%
\pgfsetlinewidth{0.803000pt}%
\definecolor{currentstroke}{rgb}{0.000000,0.000000,0.000000}%
\pgfsetstrokecolor{currentstroke}%
\pgfsetdash{}{0pt}%
\pgfsys@defobject{currentmarker}{\pgfqpoint{0.000000in}{-0.048611in}}{\pgfqpoint{0.000000in}{0.000000in}}{%
\pgfpathmoveto{\pgfqpoint{0.000000in}{0.000000in}}%
\pgfpathlineto{\pgfqpoint{0.000000in}{-0.048611in}}%
\pgfusepath{stroke,fill}%
}%
\begin{pgfscope}%
\pgfsys@transformshift{3.088299in}{0.523557in}%
\pgfsys@useobject{currentmarker}{}%
\end{pgfscope}%
\end{pgfscope}%
\begin{pgfscope}%
\definecolor{textcolor}{rgb}{0.000000,0.000000,0.000000}%
\pgfsetstrokecolor{textcolor}%
\pgfsetfillcolor{textcolor}%
\pgftext[x=3.088299in,y=0.426335in,,top]{\color{textcolor}\rmfamily\fontsize{9.000000}{10.800000}\selectfont \(\displaystyle 1250\)}%
\end{pgfscope}%
\begin{pgfscope}%
\pgfsetbuttcap%
\pgfsetroundjoin%
\definecolor{currentfill}{rgb}{0.000000,0.000000,0.000000}%
\pgfsetfillcolor{currentfill}%
\pgfsetlinewidth{0.803000pt}%
\definecolor{currentstroke}{rgb}{0.000000,0.000000,0.000000}%
\pgfsetstrokecolor{currentstroke}%
\pgfsetdash{}{0pt}%
\pgfsys@defobject{currentmarker}{\pgfqpoint{0.000000in}{-0.048611in}}{\pgfqpoint{0.000000in}{0.000000in}}{%
\pgfpathmoveto{\pgfqpoint{0.000000in}{0.000000in}}%
\pgfpathlineto{\pgfqpoint{0.000000in}{-0.048611in}}%
\pgfusepath{stroke,fill}%
}%
\begin{pgfscope}%
\pgfsys@transformshift{3.567091in}{0.523557in}%
\pgfsys@useobject{currentmarker}{}%
\end{pgfscope}%
\end{pgfscope}%
\begin{pgfscope}%
\definecolor{textcolor}{rgb}{0.000000,0.000000,0.000000}%
\pgfsetstrokecolor{textcolor}%
\pgfsetfillcolor{textcolor}%
\pgftext[x=3.567091in,y=0.426335in,,top]{\color{textcolor}\rmfamily\fontsize{9.000000}{10.800000}\selectfont \(\displaystyle 1500\)}%
\end{pgfscope}%
\begin{pgfscope}%
\pgfsetbuttcap%
\pgfsetroundjoin%
\definecolor{currentfill}{rgb}{0.000000,0.000000,0.000000}%
\pgfsetfillcolor{currentfill}%
\pgfsetlinewidth{0.803000pt}%
\definecolor{currentstroke}{rgb}{0.000000,0.000000,0.000000}%
\pgfsetstrokecolor{currentstroke}%
\pgfsetdash{}{0pt}%
\pgfsys@defobject{currentmarker}{\pgfqpoint{0.000000in}{-0.048611in}}{\pgfqpoint{0.000000in}{0.000000in}}{%
\pgfpathmoveto{\pgfqpoint{0.000000in}{0.000000in}}%
\pgfpathlineto{\pgfqpoint{0.000000in}{-0.048611in}}%
\pgfusepath{stroke,fill}%
}%
\begin{pgfscope}%
\pgfsys@transformshift{4.045884in}{0.523557in}%
\pgfsys@useobject{currentmarker}{}%
\end{pgfscope}%
\end{pgfscope}%
\begin{pgfscope}%
\definecolor{textcolor}{rgb}{0.000000,0.000000,0.000000}%
\pgfsetstrokecolor{textcolor}%
\pgfsetfillcolor{textcolor}%
\pgftext[x=4.045884in,y=0.426335in,,top]{\color{textcolor}\rmfamily\fontsize{9.000000}{10.800000}\selectfont \(\displaystyle 1750\)}%
\end{pgfscope}%
\begin{pgfscope}%
\pgfsetbuttcap%
\pgfsetroundjoin%
\definecolor{currentfill}{rgb}{0.000000,0.000000,0.000000}%
\pgfsetfillcolor{currentfill}%
\pgfsetlinewidth{0.803000pt}%
\definecolor{currentstroke}{rgb}{0.000000,0.000000,0.000000}%
\pgfsetstrokecolor{currentstroke}%
\pgfsetdash{}{0pt}%
\pgfsys@defobject{currentmarker}{\pgfqpoint{0.000000in}{-0.048611in}}{\pgfqpoint{0.000000in}{0.000000in}}{%
\pgfpathmoveto{\pgfqpoint{0.000000in}{0.000000in}}%
\pgfpathlineto{\pgfqpoint{0.000000in}{-0.048611in}}%
\pgfusepath{stroke,fill}%
}%
\begin{pgfscope}%
\pgfsys@transformshift{4.524677in}{0.523557in}%
\pgfsys@useobject{currentmarker}{}%
\end{pgfscope}%
\end{pgfscope}%
\begin{pgfscope}%
\definecolor{textcolor}{rgb}{0.000000,0.000000,0.000000}%
\pgfsetstrokecolor{textcolor}%
\pgfsetfillcolor{textcolor}%
\pgftext[x=4.524677in,y=0.426335in,,top]{\color{textcolor}\rmfamily\fontsize{9.000000}{10.800000}\selectfont \(\displaystyle 2000\)}%
\end{pgfscope}%
\begin{pgfscope}%
\definecolor{textcolor}{rgb}{0.000000,0.000000,0.000000}%
\pgfsetstrokecolor{textcolor}%
\pgfsetfillcolor{textcolor}%
\pgftext[x=2.609506in,y=0.260390in,,top]{\color{textcolor}\rmfamily\fontsize{9.000000}{10.800000}\selectfont Number of benchmarks solved}%
\end{pgfscope}%
\begin{pgfscope}%
\pgfsetbuttcap%
\pgfsetroundjoin%
\definecolor{currentfill}{rgb}{0.000000,0.000000,0.000000}%
\pgfsetfillcolor{currentfill}%
\pgfsetlinewidth{0.803000pt}%
\definecolor{currentstroke}{rgb}{0.000000,0.000000,0.000000}%
\pgfsetstrokecolor{currentstroke}%
\pgfsetdash{}{0pt}%
\pgfsys@defobject{currentmarker}{\pgfqpoint{-0.048611in}{0.000000in}}{\pgfqpoint{0.000000in}{0.000000in}}{%
\pgfpathmoveto{\pgfqpoint{0.000000in}{0.000000in}}%
\pgfpathlineto{\pgfqpoint{-0.048611in}{0.000000in}}%
\pgfusepath{stroke,fill}%
}%
\begin{pgfscope}%
\pgfsys@transformshift{0.694334in}{0.843347in}%
\pgfsys@useobject{currentmarker}{}%
\end{pgfscope}%
\end{pgfscope}%
\begin{pgfscope}%
\definecolor{textcolor}{rgb}{0.000000,0.000000,0.000000}%
\pgfsetstrokecolor{textcolor}%
\pgfsetfillcolor{textcolor}%
\pgftext[x=0.330525in, y=0.798622in, left, base]{\color{textcolor}\rmfamily\fontsize{9.000000}{10.800000}\selectfont \(\displaystyle 10^{-1}\)}%
\end{pgfscope}%
\begin{pgfscope}%
\pgfsetbuttcap%
\pgfsetroundjoin%
\definecolor{currentfill}{rgb}{0.000000,0.000000,0.000000}%
\pgfsetfillcolor{currentfill}%
\pgfsetlinewidth{0.803000pt}%
\definecolor{currentstroke}{rgb}{0.000000,0.000000,0.000000}%
\pgfsetstrokecolor{currentstroke}%
\pgfsetdash{}{0pt}%
\pgfsys@defobject{currentmarker}{\pgfqpoint{-0.048611in}{0.000000in}}{\pgfqpoint{0.000000in}{0.000000in}}{%
\pgfpathmoveto{\pgfqpoint{0.000000in}{0.000000in}}%
\pgfpathlineto{\pgfqpoint{-0.048611in}{0.000000in}}%
\pgfusepath{stroke,fill}%
}%
\begin{pgfscope}%
\pgfsys@transformshift{0.694334in}{1.334941in}%
\pgfsys@useobject{currentmarker}{}%
\end{pgfscope}%
\end{pgfscope}%
\begin{pgfscope}%
\definecolor{textcolor}{rgb}{0.000000,0.000000,0.000000}%
\pgfsetstrokecolor{textcolor}%
\pgfsetfillcolor{textcolor}%
\pgftext[x=0.410771in, y=1.290216in, left, base]{\color{textcolor}\rmfamily\fontsize{9.000000}{10.800000}\selectfont \(\displaystyle 10^{1}\)}%
\end{pgfscope}%
\begin{pgfscope}%
\pgfsetbuttcap%
\pgfsetroundjoin%
\definecolor{currentfill}{rgb}{0.000000,0.000000,0.000000}%
\pgfsetfillcolor{currentfill}%
\pgfsetlinewidth{0.803000pt}%
\definecolor{currentstroke}{rgb}{0.000000,0.000000,0.000000}%
\pgfsetstrokecolor{currentstroke}%
\pgfsetdash{}{0pt}%
\pgfsys@defobject{currentmarker}{\pgfqpoint{-0.048611in}{0.000000in}}{\pgfqpoint{0.000000in}{0.000000in}}{%
\pgfpathmoveto{\pgfqpoint{0.000000in}{0.000000in}}%
\pgfpathlineto{\pgfqpoint{-0.048611in}{0.000000in}}%
\pgfusepath{stroke,fill}%
}%
\begin{pgfscope}%
\pgfsys@transformshift{0.694334in}{1.826535in}%
\pgfsys@useobject{currentmarker}{}%
\end{pgfscope}%
\end{pgfscope}%
\begin{pgfscope}%
\definecolor{textcolor}{rgb}{0.000000,0.000000,0.000000}%
\pgfsetstrokecolor{textcolor}%
\pgfsetfillcolor{textcolor}%
\pgftext[x=0.410771in, y=1.781810in, left, base]{\color{textcolor}\rmfamily\fontsize{9.000000}{10.800000}\selectfont \(\displaystyle 10^{3}\)}%
\end{pgfscope}%
\begin{pgfscope}%
\definecolor{textcolor}{rgb}{0.000000,0.000000,0.000000}%
\pgfsetstrokecolor{textcolor}%
\pgfsetfillcolor{textcolor}%
\pgftext[x=0.274969in,y=1.175046in,,bottom,rotate=90.000000]{\color{textcolor}\rmfamily\fontsize{9.000000}{10.800000}\selectfont Longest solving time (s)}%
\end{pgfscope}%
\begin{pgfscope}%
\pgfpathrectangle{\pgfqpoint{0.694334in}{0.523557in}}{\pgfqpoint{3.830343in}{1.302977in}}%
\pgfusepath{clip}%
\pgfsetrectcap%
\pgfsetroundjoin%
\pgfsetlinewidth{1.003750pt}%
\definecolor{currentstroke}{rgb}{0.878431,0.878431,0.815686}%
\pgfsetstrokecolor{currentstroke}%
\pgfsetdash{}{0pt}%
\pgfpathmoveto{\pgfqpoint{0.694334in}{0.948144in}}%
\pgfpathlineto{\pgfqpoint{0.696249in}{0.955934in}}%
\pgfpathlineto{\pgfqpoint{0.701995in}{0.961880in}}%
\pgfpathlineto{\pgfqpoint{0.703910in}{0.964864in}}%
\pgfpathlineto{\pgfqpoint{0.705825in}{0.965524in}}%
\pgfpathlineto{\pgfqpoint{0.709656in}{0.973680in}}%
\pgfpathlineto{\pgfqpoint{0.711571in}{0.974495in}}%
\pgfpathlineto{\pgfqpoint{0.713486in}{0.976961in}}%
\pgfpathlineto{\pgfqpoint{0.719232in}{0.978816in}}%
\pgfpathlineto{\pgfqpoint{0.721147in}{0.980178in}}%
\pgfpathlineto{\pgfqpoint{0.724977in}{1.000991in}}%
\pgfpathlineto{\pgfqpoint{0.728807in}{1.015379in}}%
\pgfpathlineto{\pgfqpoint{0.730723in}{1.015482in}}%
\pgfpathlineto{\pgfqpoint{0.732638in}{1.022216in}}%
\pgfpathlineto{\pgfqpoint{0.734553in}{1.022392in}}%
\pgfpathlineto{\pgfqpoint{0.736468in}{1.024621in}}%
\pgfpathlineto{\pgfqpoint{0.738383in}{1.024629in}}%
\pgfpathlineto{\pgfqpoint{0.742214in}{1.030245in}}%
\pgfpathlineto{\pgfqpoint{0.744129in}{1.030409in}}%
\pgfpathlineto{\pgfqpoint{0.746044in}{1.033013in}}%
\pgfpathlineto{\pgfqpoint{0.747959in}{1.039599in}}%
\pgfpathlineto{\pgfqpoint{0.749874in}{1.040605in}}%
\pgfpathlineto{\pgfqpoint{0.755620in}{1.049992in}}%
\pgfpathlineto{\pgfqpoint{0.772856in}{1.056340in}}%
\pgfpathlineto{\pgfqpoint{0.786263in}{1.061350in}}%
\pgfpathlineto{\pgfqpoint{0.801584in}{1.062980in}}%
\pgfpathlineto{\pgfqpoint{0.805414in}{1.063733in}}%
\pgfpathlineto{\pgfqpoint{0.814990in}{1.065163in}}%
\pgfpathlineto{\pgfqpoint{0.882021in}{1.076339in}}%
\pgfpathlineto{\pgfqpoint{0.885851in}{1.077456in}}%
\pgfpathlineto{\pgfqpoint{0.895427in}{1.078655in}}%
\pgfpathlineto{\pgfqpoint{0.906918in}{1.080409in}}%
\pgfpathlineto{\pgfqpoint{0.987356in}{1.089407in}}%
\pgfpathlineto{\pgfqpoint{1.008422in}{1.090894in}}%
\pgfpathlineto{\pgfqpoint{1.052471in}{1.095137in}}%
\pgfpathlineto{\pgfqpoint{1.056302in}{1.095792in}}%
\pgfpathlineto{\pgfqpoint{1.077369in}{1.097817in}}%
\pgfpathlineto{\pgfqpoint{1.081199in}{1.099401in}}%
\pgfpathlineto{\pgfqpoint{1.096520in}{1.100833in}}%
\pgfpathlineto{\pgfqpoint{1.106096in}{1.102076in}}%
\pgfpathlineto{\pgfqpoint{1.159721in}{1.107209in}}%
\pgfpathlineto{\pgfqpoint{1.163551in}{1.109041in}}%
\pgfpathlineto{\pgfqpoint{1.173127in}{1.110874in}}%
\pgfpathlineto{\pgfqpoint{1.180788in}{1.111938in}}%
\pgfpathlineto{\pgfqpoint{1.186533in}{1.112998in}}%
\pgfpathlineto{\pgfqpoint{1.194194in}{1.114050in}}%
\pgfpathlineto{\pgfqpoint{1.213346in}{1.115660in}}%
\pgfpathlineto{\pgfqpoint{1.221006in}{1.118786in}}%
\pgfpathlineto{\pgfqpoint{1.228667in}{1.120114in}}%
\pgfpathlineto{\pgfqpoint{1.255480in}{1.127277in}}%
\pgfpathlineto{\pgfqpoint{1.265055in}{1.131122in}}%
\pgfpathlineto{\pgfqpoint{1.274631in}{1.132047in}}%
\pgfpathlineto{\pgfqpoint{1.278462in}{1.133795in}}%
\pgfpathlineto{\pgfqpoint{1.295698in}{1.138666in}}%
\pgfpathlineto{\pgfqpoint{1.299528in}{1.139717in}}%
\pgfpathlineto{\pgfqpoint{1.301444in}{1.139860in}}%
\pgfpathlineto{\pgfqpoint{1.303359in}{1.142985in}}%
\pgfpathlineto{\pgfqpoint{1.309104in}{1.143567in}}%
\pgfpathlineto{\pgfqpoint{1.311019in}{1.145858in}}%
\pgfpathlineto{\pgfqpoint{1.316765in}{1.146442in}}%
\pgfpathlineto{\pgfqpoint{1.320595in}{1.149572in}}%
\pgfpathlineto{\pgfqpoint{1.326341in}{1.150947in}}%
\pgfpathlineto{\pgfqpoint{1.334002in}{1.152019in}}%
\pgfpathlineto{\pgfqpoint{1.349323in}{1.155558in}}%
\pgfpathlineto{\pgfqpoint{1.351238in}{1.157653in}}%
\pgfpathlineto{\pgfqpoint{1.353153in}{1.157732in}}%
\pgfpathlineto{\pgfqpoint{1.356984in}{1.160260in}}%
\pgfpathlineto{\pgfqpoint{1.360814in}{1.160544in}}%
\pgfpathlineto{\pgfqpoint{1.368475in}{1.163791in}}%
\pgfpathlineto{\pgfqpoint{1.372305in}{1.164419in}}%
\pgfpathlineto{\pgfqpoint{1.381881in}{1.167150in}}%
\pgfpathlineto{\pgfqpoint{1.395287in}{1.168617in}}%
\pgfpathlineto{\pgfqpoint{1.402948in}{1.170470in}}%
\pgfpathlineto{\pgfqpoint{1.410608in}{1.171407in}}%
\pgfpathlineto{\pgfqpoint{1.420184in}{1.176099in}}%
\pgfpathlineto{\pgfqpoint{1.441251in}{1.178997in}}%
\pgfpathlineto{\pgfqpoint{1.446997in}{1.180239in}}%
\pgfpathlineto{\pgfqpoint{1.458488in}{1.181758in}}%
\pgfpathlineto{\pgfqpoint{1.466148in}{1.183530in}}%
\pgfpathlineto{\pgfqpoint{1.475724in}{1.184325in}}%
\pgfpathlineto{\pgfqpoint{1.485300in}{1.185970in}}%
\pgfpathlineto{\pgfqpoint{1.496791in}{1.186897in}}%
\pgfpathlineto{\pgfqpoint{1.500621in}{1.188076in}}%
\pgfpathlineto{\pgfqpoint{1.510197in}{1.189051in}}%
\pgfpathlineto{\pgfqpoint{1.517858in}{1.191555in}}%
\pgfpathlineto{\pgfqpoint{1.529349in}{1.192277in}}%
\pgfpathlineto{\pgfqpoint{1.542755in}{1.193417in}}%
\pgfpathlineto{\pgfqpoint{1.556161in}{1.195160in}}%
\pgfpathlineto{\pgfqpoint{1.569568in}{1.196028in}}%
\pgfpathlineto{\pgfqpoint{1.573398in}{1.197232in}}%
\pgfpathlineto{\pgfqpoint{1.590635in}{1.200437in}}%
\pgfpathlineto{\pgfqpoint{1.592550in}{1.202315in}}%
\pgfpathlineto{\pgfqpoint{1.598295in}{1.203693in}}%
\pgfpathlineto{\pgfqpoint{1.627023in}{1.210857in}}%
\pgfpathlineto{\pgfqpoint{1.630853in}{1.211879in}}%
\pgfpathlineto{\pgfqpoint{1.632768in}{1.212608in}}%
\pgfpathlineto{\pgfqpoint{1.634683in}{1.214894in}}%
\pgfpathlineto{\pgfqpoint{1.650005in}{1.216939in}}%
\pgfpathlineto{\pgfqpoint{1.653835in}{1.217843in}}%
\pgfpathlineto{\pgfqpoint{1.667241in}{1.220406in}}%
\pgfpathlineto{\pgfqpoint{1.671072in}{1.222320in}}%
\pgfpathlineto{\pgfqpoint{1.678732in}{1.223561in}}%
\pgfpathlineto{\pgfqpoint{1.682563in}{1.225762in}}%
\pgfpathlineto{\pgfqpoint{1.690223in}{1.227473in}}%
\pgfpathlineto{\pgfqpoint{1.694054in}{1.228995in}}%
\pgfpathlineto{\pgfqpoint{1.695969in}{1.229430in}}%
\pgfpathlineto{\pgfqpoint{1.697884in}{1.231143in}}%
\pgfpathlineto{\pgfqpoint{1.701714in}{1.232314in}}%
\pgfpathlineto{\pgfqpoint{1.709375in}{1.233737in}}%
\pgfpathlineto{\pgfqpoint{1.713205in}{1.235570in}}%
\pgfpathlineto{\pgfqpoint{1.728527in}{1.241910in}}%
\pgfpathlineto{\pgfqpoint{1.736188in}{1.243184in}}%
\pgfpathlineto{\pgfqpoint{1.745763in}{1.246246in}}%
\pgfpathlineto{\pgfqpoint{1.757254in}{1.248056in}}%
\pgfpathlineto{\pgfqpoint{1.761085in}{1.248943in}}%
\pgfpathlineto{\pgfqpoint{1.764915in}{1.249471in}}%
\pgfpathlineto{\pgfqpoint{1.774491in}{1.251515in}}%
\pgfpathlineto{\pgfqpoint{1.785982in}{1.252902in}}%
\pgfpathlineto{\pgfqpoint{1.789812in}{1.254325in}}%
\pgfpathlineto{\pgfqpoint{1.793643in}{1.254431in}}%
\pgfpathlineto{\pgfqpoint{1.797473in}{1.256010in}}%
\pgfpathlineto{\pgfqpoint{1.803219in}{1.257885in}}%
\pgfpathlineto{\pgfqpoint{1.835776in}{1.262884in}}%
\pgfpathlineto{\pgfqpoint{1.887486in}{1.270353in}}%
\pgfpathlineto{\pgfqpoint{1.893232in}{1.271230in}}%
\pgfpathlineto{\pgfqpoint{1.902807in}{1.275463in}}%
\pgfpathlineto{\pgfqpoint{1.906638in}{1.276300in}}%
\pgfpathlineto{\pgfqpoint{1.908553in}{1.280411in}}%
\pgfpathlineto{\pgfqpoint{1.916214in}{1.282545in}}%
\pgfpathlineto{\pgfqpoint{1.927705in}{1.284625in}}%
\pgfpathlineto{\pgfqpoint{1.929620in}{1.286283in}}%
\pgfpathlineto{\pgfqpoint{1.937281in}{1.287696in}}%
\pgfpathlineto{\pgfqpoint{1.956432in}{1.296836in}}%
\pgfpathlineto{\pgfqpoint{1.960263in}{1.301263in}}%
\pgfpathlineto{\pgfqpoint{1.964093in}{1.302226in}}%
\pgfpathlineto{\pgfqpoint{1.971754in}{1.304660in}}%
\pgfpathlineto{\pgfqpoint{1.973669in}{1.306666in}}%
\pgfpathlineto{\pgfqpoint{1.981329in}{1.307451in}}%
\pgfpathlineto{\pgfqpoint{1.988990in}{1.309458in}}%
\pgfpathlineto{\pgfqpoint{2.000481in}{1.310640in}}%
\pgfpathlineto{\pgfqpoint{2.002396in}{1.310865in}}%
\pgfpathlineto{\pgfqpoint{2.004312in}{1.313052in}}%
\pgfpathlineto{\pgfqpoint{2.006227in}{1.313203in}}%
\pgfpathlineto{\pgfqpoint{2.010057in}{1.315514in}}%
\pgfpathlineto{\pgfqpoint{2.017718in}{1.316843in}}%
\pgfpathlineto{\pgfqpoint{2.021548in}{1.318677in}}%
\pgfpathlineto{\pgfqpoint{2.031124in}{1.321009in}}%
\pgfpathlineto{\pgfqpoint{2.036869in}{1.325537in}}%
\pgfpathlineto{\pgfqpoint{2.046445in}{1.327598in}}%
\pgfpathlineto{\pgfqpoint{2.048360in}{1.329940in}}%
\pgfpathlineto{\pgfqpoint{2.056021in}{1.331163in}}%
\pgfpathlineto{\pgfqpoint{2.059852in}{1.331423in}}%
\pgfpathlineto{\pgfqpoint{2.063682in}{1.333203in}}%
\pgfpathlineto{\pgfqpoint{2.077088in}{1.334794in}}%
\pgfpathlineto{\pgfqpoint{2.082834in}{1.335836in}}%
\pgfpathlineto{\pgfqpoint{2.086664in}{1.337222in}}%
\pgfpathlineto{\pgfqpoint{2.088579in}{1.339358in}}%
\pgfpathlineto{\pgfqpoint{2.100070in}{1.340897in}}%
\pgfpathlineto{\pgfqpoint{2.103900in}{1.341009in}}%
\pgfpathlineto{\pgfqpoint{2.107731in}{1.345043in}}%
\pgfpathlineto{\pgfqpoint{2.111561in}{1.346544in}}%
\pgfpathlineto{\pgfqpoint{2.132628in}{1.353093in}}%
\pgfpathlineto{\pgfqpoint{2.134543in}{1.354830in}}%
\pgfpathlineto{\pgfqpoint{2.140289in}{1.355716in}}%
\pgfpathlineto{\pgfqpoint{2.144119in}{1.357213in}}%
\pgfpathlineto{\pgfqpoint{2.151780in}{1.358472in}}%
\pgfpathlineto{\pgfqpoint{2.155610in}{1.359652in}}%
\pgfpathlineto{\pgfqpoint{2.157525in}{1.359983in}}%
\pgfpathlineto{\pgfqpoint{2.159440in}{1.362600in}}%
\pgfpathlineto{\pgfqpoint{2.207320in}{1.374146in}}%
\pgfpathlineto{\pgfqpoint{2.216896in}{1.377913in}}%
\pgfpathlineto{\pgfqpoint{2.220726in}{1.378002in}}%
\pgfpathlineto{\pgfqpoint{2.222641in}{1.380086in}}%
\pgfpathlineto{\pgfqpoint{2.236047in}{1.382030in}}%
\pgfpathlineto{\pgfqpoint{2.241793in}{1.384763in}}%
\pgfpathlineto{\pgfqpoint{2.243708in}{1.387612in}}%
\pgfpathlineto{\pgfqpoint{2.249453in}{1.389313in}}%
\pgfpathlineto{\pgfqpoint{2.253284in}{1.390935in}}%
\pgfpathlineto{\pgfqpoint{2.260945in}{1.395304in}}%
\pgfpathlineto{\pgfqpoint{2.262860in}{1.398980in}}%
\pgfpathlineto{\pgfqpoint{2.266690in}{1.400113in}}%
\pgfpathlineto{\pgfqpoint{2.268605in}{1.400377in}}%
\pgfpathlineto{\pgfqpoint{2.272436in}{1.403796in}}%
\pgfpathlineto{\pgfqpoint{2.274351in}{1.403902in}}%
\pgfpathlineto{\pgfqpoint{2.276266in}{1.407289in}}%
\pgfpathlineto{\pgfqpoint{2.287757in}{1.409395in}}%
\pgfpathlineto{\pgfqpoint{2.289672in}{1.410866in}}%
\pgfpathlineto{\pgfqpoint{2.291587in}{1.410933in}}%
\pgfpathlineto{\pgfqpoint{2.293502in}{1.413835in}}%
\pgfpathlineto{\pgfqpoint{2.299248in}{1.414269in}}%
\pgfpathlineto{\pgfqpoint{2.304993in}{1.418648in}}%
\pgfpathlineto{\pgfqpoint{2.310739in}{1.419236in}}%
\pgfpathlineto{\pgfqpoint{2.312654in}{1.421411in}}%
\pgfpathlineto{\pgfqpoint{2.316484in}{1.422608in}}%
\pgfpathlineto{\pgfqpoint{2.320315in}{1.424557in}}%
\pgfpathlineto{\pgfqpoint{2.329891in}{1.429193in}}%
\pgfpathlineto{\pgfqpoint{2.331806in}{1.432102in}}%
\pgfpathlineto{\pgfqpoint{2.335636in}{1.433470in}}%
\pgfpathlineto{\pgfqpoint{2.339467in}{1.437411in}}%
\pgfpathlineto{\pgfqpoint{2.358618in}{1.441416in}}%
\pgfpathlineto{\pgfqpoint{2.362449in}{1.444526in}}%
\pgfpathlineto{\pgfqpoint{2.366279in}{1.447267in}}%
\pgfpathlineto{\pgfqpoint{2.370109in}{1.448358in}}%
\pgfpathlineto{\pgfqpoint{2.373940in}{1.449298in}}%
\pgfpathlineto{\pgfqpoint{2.379685in}{1.450516in}}%
\pgfpathlineto{\pgfqpoint{2.395006in}{1.452221in}}%
\pgfpathlineto{\pgfqpoint{2.410328in}{1.458475in}}%
\pgfpathlineto{\pgfqpoint{2.414158in}{1.459391in}}%
\pgfpathlineto{\pgfqpoint{2.427564in}{1.465748in}}%
\pgfpathlineto{\pgfqpoint{2.433310in}{1.466424in}}%
\pgfpathlineto{\pgfqpoint{2.437140in}{1.471209in}}%
\pgfpathlineto{\pgfqpoint{2.444801in}{1.478878in}}%
\pgfpathlineto{\pgfqpoint{2.448631in}{1.479727in}}%
\pgfpathlineto{\pgfqpoint{2.454377in}{1.483926in}}%
\pgfpathlineto{\pgfqpoint{2.458207in}{1.485287in}}%
\pgfpathlineto{\pgfqpoint{2.463953in}{1.486920in}}%
\pgfpathlineto{\pgfqpoint{2.467783in}{1.488548in}}%
\pgfpathlineto{\pgfqpoint{2.481189in}{1.497385in}}%
\pgfpathlineto{\pgfqpoint{2.483104in}{1.499642in}}%
\pgfpathlineto{\pgfqpoint{2.488850in}{1.501012in}}%
\pgfpathlineto{\pgfqpoint{2.500341in}{1.504464in}}%
\pgfpathlineto{\pgfqpoint{2.504171in}{1.506477in}}%
\pgfpathlineto{\pgfqpoint{2.506086in}{1.512215in}}%
\pgfpathlineto{\pgfqpoint{2.509917in}{1.513305in}}%
\pgfpathlineto{\pgfqpoint{2.511832in}{1.513716in}}%
\pgfpathlineto{\pgfqpoint{2.513747in}{1.517236in}}%
\pgfpathlineto{\pgfqpoint{2.517577in}{1.518534in}}%
\pgfpathlineto{\pgfqpoint{2.519493in}{1.521075in}}%
\pgfpathlineto{\pgfqpoint{2.521408in}{1.521471in}}%
\pgfpathlineto{\pgfqpoint{2.525238in}{1.523827in}}%
\pgfpathlineto{\pgfqpoint{2.542475in}{1.527227in}}%
\pgfpathlineto{\pgfqpoint{2.550135in}{1.531370in}}%
\pgfpathlineto{\pgfqpoint{2.557796in}{1.534325in}}%
\pgfpathlineto{\pgfqpoint{2.561626in}{1.535827in}}%
\pgfpathlineto{\pgfqpoint{2.567372in}{1.539053in}}%
\pgfpathlineto{\pgfqpoint{2.569287in}{1.542904in}}%
\pgfpathlineto{\pgfqpoint{2.575033in}{1.544833in}}%
\pgfpathlineto{\pgfqpoint{2.580778in}{1.546787in}}%
\pgfpathlineto{\pgfqpoint{2.592269in}{1.551344in}}%
\pgfpathlineto{\pgfqpoint{2.594184in}{1.553310in}}%
\pgfpathlineto{\pgfqpoint{2.596099in}{1.553663in}}%
\pgfpathlineto{\pgfqpoint{2.598015in}{1.555244in}}%
\pgfpathlineto{\pgfqpoint{2.601845in}{1.560810in}}%
\pgfpathlineto{\pgfqpoint{2.603760in}{1.562633in}}%
\pgfpathlineto{\pgfqpoint{2.605675in}{1.562837in}}%
\pgfpathlineto{\pgfqpoint{2.611421in}{1.566155in}}%
\pgfpathlineto{\pgfqpoint{2.626742in}{1.570817in}}%
\pgfpathlineto{\pgfqpoint{2.630573in}{1.571164in}}%
\pgfpathlineto{\pgfqpoint{2.638233in}{1.574685in}}%
\pgfpathlineto{\pgfqpoint{2.642064in}{1.579641in}}%
\pgfpathlineto{\pgfqpoint{2.647809in}{1.581096in}}%
\pgfpathlineto{\pgfqpoint{2.649724in}{1.581447in}}%
\pgfpathlineto{\pgfqpoint{2.653555in}{1.584317in}}%
\pgfpathlineto{\pgfqpoint{2.663130in}{1.586069in}}%
\pgfpathlineto{\pgfqpoint{2.666961in}{1.587435in}}%
\pgfpathlineto{\pgfqpoint{2.676537in}{1.590321in}}%
\pgfpathlineto{\pgfqpoint{2.678452in}{1.593693in}}%
\pgfpathlineto{\pgfqpoint{2.682282in}{1.594974in}}%
\pgfpathlineto{\pgfqpoint{2.691858in}{1.597882in}}%
\pgfpathlineto{\pgfqpoint{2.701434in}{1.600655in}}%
\pgfpathlineto{\pgfqpoint{2.703349in}{1.600878in}}%
\pgfpathlineto{\pgfqpoint{2.707179in}{1.602882in}}%
\pgfpathlineto{\pgfqpoint{2.714840in}{1.606766in}}%
\pgfpathlineto{\pgfqpoint{2.716755in}{1.610526in}}%
\pgfpathlineto{\pgfqpoint{2.724416in}{1.612056in}}%
\pgfpathlineto{\pgfqpoint{2.726331in}{1.612349in}}%
\pgfpathlineto{\pgfqpoint{2.728246in}{1.614897in}}%
\pgfpathlineto{\pgfqpoint{2.739737in}{1.616467in}}%
\pgfpathlineto{\pgfqpoint{2.741653in}{1.618264in}}%
\pgfpathlineto{\pgfqpoint{2.749313in}{1.619465in}}%
\pgfpathlineto{\pgfqpoint{2.755059in}{1.621218in}}%
\pgfpathlineto{\pgfqpoint{2.758889in}{1.622390in}}%
\pgfpathlineto{\pgfqpoint{2.762719in}{1.623297in}}%
\pgfpathlineto{\pgfqpoint{2.764635in}{1.625681in}}%
\pgfpathlineto{\pgfqpoint{2.774210in}{1.626737in}}%
\pgfpathlineto{\pgfqpoint{2.781871in}{1.628336in}}%
\pgfpathlineto{\pgfqpoint{2.785701in}{1.630408in}}%
\pgfpathlineto{\pgfqpoint{2.789532in}{1.634811in}}%
\pgfpathlineto{\pgfqpoint{2.793362in}{1.635914in}}%
\pgfpathlineto{\pgfqpoint{2.804853in}{1.639762in}}%
\pgfpathlineto{\pgfqpoint{2.806768in}{1.642283in}}%
\pgfpathlineto{\pgfqpoint{2.810599in}{1.643193in}}%
\pgfpathlineto{\pgfqpoint{2.818259in}{1.645990in}}%
\pgfpathlineto{\pgfqpoint{2.824005in}{1.647253in}}%
\pgfpathlineto{\pgfqpoint{2.825920in}{1.647678in}}%
\pgfpathlineto{\pgfqpoint{2.827835in}{1.651306in}}%
\pgfpathlineto{\pgfqpoint{2.843157in}{1.658318in}}%
\pgfpathlineto{\pgfqpoint{2.846987in}{1.662826in}}%
\pgfpathlineto{\pgfqpoint{2.852732in}{1.665312in}}%
\pgfpathlineto{\pgfqpoint{2.856563in}{1.666470in}}%
\pgfpathlineto{\pgfqpoint{2.862308in}{1.668568in}}%
\pgfpathlineto{\pgfqpoint{2.864223in}{1.669565in}}%
\pgfpathlineto{\pgfqpoint{2.868054in}{1.673167in}}%
\pgfpathlineto{\pgfqpoint{2.875715in}{1.674294in}}%
\pgfpathlineto{\pgfqpoint{2.877630in}{1.676224in}}%
\pgfpathlineto{\pgfqpoint{2.881460in}{1.676644in}}%
\pgfpathlineto{\pgfqpoint{2.898697in}{1.684935in}}%
\pgfpathlineto{\pgfqpoint{2.904442in}{1.692800in}}%
\pgfpathlineto{\pgfqpoint{2.912103in}{1.693895in}}%
\pgfpathlineto{\pgfqpoint{2.921679in}{1.696221in}}%
\pgfpathlineto{\pgfqpoint{2.927424in}{1.697790in}}%
\pgfpathlineto{\pgfqpoint{2.929339in}{1.702518in}}%
\pgfpathlineto{\pgfqpoint{2.942746in}{1.712964in}}%
\pgfpathlineto{\pgfqpoint{2.946576in}{1.714324in}}%
\pgfpathlineto{\pgfqpoint{2.950406in}{1.716044in}}%
\pgfpathlineto{\pgfqpoint{2.958067in}{1.717145in}}%
\pgfpathlineto{\pgfqpoint{2.961897in}{1.722932in}}%
\pgfpathlineto{\pgfqpoint{2.971473in}{1.725541in}}%
\pgfpathlineto{\pgfqpoint{2.975303in}{1.730058in}}%
\pgfpathlineto{\pgfqpoint{2.986794in}{1.737420in}}%
\pgfpathlineto{\pgfqpoint{2.992540in}{1.744238in}}%
\pgfpathlineto{\pgfqpoint{2.998285in}{1.760158in}}%
\pgfpathlineto{\pgfqpoint{3.000201in}{1.763486in}}%
\pgfpathlineto{\pgfqpoint{3.002116in}{1.763574in}}%
\pgfpathlineto{\pgfqpoint{3.013607in}{1.800754in}}%
\pgfpathlineto{\pgfqpoint{3.015522in}{1.819575in}}%
\pgfpathlineto{\pgfqpoint{3.017437in}{1.826535in}}%
\pgfpathlineto{\pgfqpoint{3.017437in}{1.826535in}}%
\pgfusepath{stroke}%
\end{pgfscope}%
\begin{pgfscope}%
\pgfpathrectangle{\pgfqpoint{0.694334in}{0.523557in}}{\pgfqpoint{3.830343in}{1.302977in}}%
\pgfusepath{clip}%
\pgfsetbuttcap%
\pgfsetroundjoin%
\pgfsetlinewidth{1.003750pt}%
\definecolor{currentstroke}{rgb}{0.941176,0.627451,0.188235}%
\pgfsetstrokecolor{currentstroke}%
\pgfsetdash{{1.000000pt}{1.650000pt}}{0.000000pt}%
\pgfpathmoveto{\pgfqpoint{0.694334in}{0.567141in}}%
\pgfpathlineto{\pgfqpoint{0.696249in}{0.582562in}}%
\pgfpathlineto{\pgfqpoint{0.698165in}{0.582755in}}%
\pgfpathlineto{\pgfqpoint{0.701995in}{0.599148in}}%
\pgfpathlineto{\pgfqpoint{0.703910in}{0.606735in}}%
\pgfpathlineto{\pgfqpoint{0.711571in}{0.615604in}}%
\pgfpathlineto{\pgfqpoint{0.713486in}{0.616358in}}%
\pgfpathlineto{\pgfqpoint{0.715401in}{0.627746in}}%
\pgfpathlineto{\pgfqpoint{0.719232in}{0.630247in}}%
\pgfpathlineto{\pgfqpoint{0.721147in}{0.632363in}}%
\pgfpathlineto{\pgfqpoint{0.724977in}{0.633447in}}%
\pgfpathlineto{\pgfqpoint{0.726892in}{0.635337in}}%
\pgfpathlineto{\pgfqpoint{0.732638in}{0.646877in}}%
\pgfpathlineto{\pgfqpoint{0.736468in}{0.649507in}}%
\pgfpathlineto{\pgfqpoint{0.738383in}{0.655445in}}%
\pgfpathlineto{\pgfqpoint{0.740298in}{0.655930in}}%
\pgfpathlineto{\pgfqpoint{0.747959in}{0.671221in}}%
\pgfpathlineto{\pgfqpoint{0.749874in}{0.675620in}}%
\pgfpathlineto{\pgfqpoint{0.751789in}{0.675686in}}%
\pgfpathlineto{\pgfqpoint{0.753705in}{0.678423in}}%
\pgfpathlineto{\pgfqpoint{0.759450in}{0.690456in}}%
\pgfpathlineto{\pgfqpoint{0.770941in}{0.700710in}}%
\pgfpathlineto{\pgfqpoint{0.774772in}{0.703735in}}%
\pgfpathlineto{\pgfqpoint{0.801584in}{0.709532in}}%
\pgfpathlineto{\pgfqpoint{0.805414in}{0.711596in}}%
\pgfpathlineto{\pgfqpoint{0.814990in}{0.713390in}}%
\pgfpathlineto{\pgfqpoint{0.820736in}{0.714546in}}%
\pgfpathlineto{\pgfqpoint{0.836057in}{0.716555in}}%
\pgfpathlineto{\pgfqpoint{0.843718in}{0.719765in}}%
\pgfpathlineto{\pgfqpoint{0.855209in}{0.720566in}}%
\pgfpathlineto{\pgfqpoint{0.857124in}{0.720645in}}%
\pgfpathlineto{\pgfqpoint{0.860954in}{0.722678in}}%
\pgfpathlineto{\pgfqpoint{0.870530in}{0.723679in}}%
\pgfpathlineto{\pgfqpoint{0.878191in}{0.724279in}}%
\pgfpathlineto{\pgfqpoint{0.908834in}{0.725460in}}%
\pgfpathlineto{\pgfqpoint{0.920325in}{0.726063in}}%
\pgfpathlineto{\pgfqpoint{0.945222in}{0.727385in}}%
\pgfpathlineto{\pgfqpoint{0.972034in}{0.728886in}}%
\pgfpathlineto{\pgfqpoint{1.044811in}{0.732918in}}%
\pgfpathlineto{\pgfqpoint{1.083114in}{0.735423in}}%
\pgfpathlineto{\pgfqpoint{1.100351in}{0.737113in}}%
\pgfpathlineto{\pgfqpoint{1.192279in}{0.741689in}}%
\pgfpathlineto{\pgfqpoint{1.213346in}{0.742889in}}%
\pgfpathlineto{\pgfqpoint{1.236328in}{0.744723in}}%
\pgfpathlineto{\pgfqpoint{1.240158in}{0.745473in}}%
\pgfpathlineto{\pgfqpoint{1.284207in}{0.747171in}}%
\pgfpathlineto{\pgfqpoint{1.414439in}{0.759272in}}%
\pgfpathlineto{\pgfqpoint{1.420184in}{0.760773in}}%
\pgfpathlineto{\pgfqpoint{1.431675in}{0.762508in}}%
\pgfpathlineto{\pgfqpoint{1.441251in}{0.763654in}}%
\pgfpathlineto{\pgfqpoint{1.445081in}{0.765852in}}%
\pgfpathlineto{\pgfqpoint{1.450827in}{0.766734in}}%
\pgfpathlineto{\pgfqpoint{1.462318in}{0.771921in}}%
\pgfpathlineto{\pgfqpoint{1.466148in}{0.772684in}}%
\pgfpathlineto{\pgfqpoint{1.475724in}{0.780723in}}%
\pgfpathlineto{\pgfqpoint{1.477639in}{0.784911in}}%
\pgfpathlineto{\pgfqpoint{1.481470in}{0.785898in}}%
\pgfpathlineto{\pgfqpoint{1.483385in}{0.788811in}}%
\pgfpathlineto{\pgfqpoint{1.492961in}{0.790671in}}%
\pgfpathlineto{\pgfqpoint{1.510197in}{0.798592in}}%
\pgfpathlineto{\pgfqpoint{1.521688in}{0.799587in}}%
\pgfpathlineto{\pgfqpoint{1.527434in}{0.802172in}}%
\pgfpathlineto{\pgfqpoint{1.540840in}{0.804764in}}%
\pgfpathlineto{\pgfqpoint{1.544670in}{0.806461in}}%
\pgfpathlineto{\pgfqpoint{1.550416in}{0.807943in}}%
\pgfpathlineto{\pgfqpoint{1.579143in}{0.812868in}}%
\pgfpathlineto{\pgfqpoint{1.582974in}{0.815132in}}%
\pgfpathlineto{\pgfqpoint{1.584889in}{0.815211in}}%
\pgfpathlineto{\pgfqpoint{1.588719in}{0.816491in}}%
\pgfpathlineto{\pgfqpoint{1.596380in}{0.818241in}}%
\pgfpathlineto{\pgfqpoint{1.615532in}{0.820945in}}%
\pgfpathlineto{\pgfqpoint{1.623192in}{0.822119in}}%
\pgfpathlineto{\pgfqpoint{1.634683in}{0.824956in}}%
\pgfpathlineto{\pgfqpoint{1.650005in}{0.827186in}}%
\pgfpathlineto{\pgfqpoint{1.661496in}{0.828179in}}%
\pgfpathlineto{\pgfqpoint{1.680648in}{0.832850in}}%
\pgfpathlineto{\pgfqpoint{1.695969in}{0.834803in}}%
\pgfpathlineto{\pgfqpoint{1.701714in}{0.836460in}}%
\pgfpathlineto{\pgfqpoint{1.718951in}{0.838267in}}%
\pgfpathlineto{\pgfqpoint{1.724697in}{0.841013in}}%
\pgfpathlineto{\pgfqpoint{1.730442in}{0.841765in}}%
\pgfpathlineto{\pgfqpoint{1.738103in}{0.843347in}}%
\pgfpathlineto{\pgfqpoint{1.766830in}{0.847835in}}%
\pgfpathlineto{\pgfqpoint{1.768745in}{0.848125in}}%
\pgfpathlineto{\pgfqpoint{1.772576in}{0.850816in}}%
\pgfpathlineto{\pgfqpoint{1.778321in}{0.852136in}}%
\pgfpathlineto{\pgfqpoint{1.789812in}{0.855151in}}%
\pgfpathlineto{\pgfqpoint{1.793643in}{0.858744in}}%
\pgfpathlineto{\pgfqpoint{1.799388in}{0.860410in}}%
\pgfpathlineto{\pgfqpoint{1.808964in}{0.861690in}}%
\pgfpathlineto{\pgfqpoint{1.897062in}{0.876194in}}%
\pgfpathlineto{\pgfqpoint{1.902807in}{0.879436in}}%
\pgfpathlineto{\pgfqpoint{1.962178in}{0.889986in}}%
\pgfpathlineto{\pgfqpoint{1.998566in}{0.895279in}}%
\pgfpathlineto{\pgfqpoint{2.002396in}{0.897136in}}%
\pgfpathlineto{\pgfqpoint{2.019633in}{0.899457in}}%
\pgfpathlineto{\pgfqpoint{2.023463in}{0.900421in}}%
\pgfpathlineto{\pgfqpoint{2.046445in}{0.903270in}}%
\pgfpathlineto{\pgfqpoint{2.063682in}{0.905826in}}%
\pgfpathlineto{\pgfqpoint{2.077088in}{0.907498in}}%
\pgfpathlineto{\pgfqpoint{2.092409in}{0.909303in}}%
\pgfpathlineto{\pgfqpoint{2.121137in}{0.912436in}}%
\pgfpathlineto{\pgfqpoint{2.124967in}{0.913570in}}%
\pgfpathlineto{\pgfqpoint{2.136458in}{0.915106in}}%
\pgfpathlineto{\pgfqpoint{2.138374in}{0.916643in}}%
\pgfpathlineto{\pgfqpoint{2.144119in}{0.917605in}}%
\pgfpathlineto{\pgfqpoint{2.149865in}{0.918496in}}%
\pgfpathlineto{\pgfqpoint{2.151780in}{0.918737in}}%
\pgfpathlineto{\pgfqpoint{2.153695in}{0.920150in}}%
\pgfpathlineto{\pgfqpoint{2.163271in}{0.920991in}}%
\pgfpathlineto{\pgfqpoint{2.214980in}{0.931176in}}%
\pgfpathlineto{\pgfqpoint{2.224556in}{0.931968in}}%
\pgfpathlineto{\pgfqpoint{2.236047in}{0.935912in}}%
\pgfpathlineto{\pgfqpoint{2.243708in}{0.936754in}}%
\pgfpathlineto{\pgfqpoint{2.280096in}{0.941676in}}%
\pgfpathlineto{\pgfqpoint{2.289672in}{0.942902in}}%
\pgfpathlineto{\pgfqpoint{2.316484in}{0.945052in}}%
\pgfpathlineto{\pgfqpoint{2.329891in}{0.945964in}}%
\pgfpathlineto{\pgfqpoint{2.335636in}{0.947179in}}%
\pgfpathlineto{\pgfqpoint{2.358618in}{0.948809in}}%
\pgfpathlineto{\pgfqpoint{2.366279in}{0.950171in}}%
\pgfpathlineto{\pgfqpoint{2.373940in}{0.951947in}}%
\pgfpathlineto{\pgfqpoint{2.377770in}{0.954446in}}%
\pgfpathlineto{\pgfqpoint{2.389261in}{0.956433in}}%
\pgfpathlineto{\pgfqpoint{2.391176in}{0.957190in}}%
\pgfpathlineto{\pgfqpoint{2.395006in}{0.960086in}}%
\pgfpathlineto{\pgfqpoint{2.406498in}{0.961412in}}%
\pgfpathlineto{\pgfqpoint{2.410328in}{0.963197in}}%
\pgfpathlineto{\pgfqpoint{2.419904in}{0.964791in}}%
\pgfpathlineto{\pgfqpoint{2.423734in}{0.967067in}}%
\pgfpathlineto{\pgfqpoint{2.439055in}{0.974701in}}%
\pgfpathlineto{\pgfqpoint{2.450546in}{0.980023in}}%
\pgfpathlineto{\pgfqpoint{2.458207in}{0.982179in}}%
\pgfpathlineto{\pgfqpoint{2.460122in}{0.986058in}}%
\pgfpathlineto{\pgfqpoint{2.465868in}{0.988210in}}%
\pgfpathlineto{\pgfqpoint{2.471613in}{0.989814in}}%
\pgfpathlineto{\pgfqpoint{2.477359in}{0.992420in}}%
\pgfpathlineto{\pgfqpoint{2.486935in}{0.993619in}}%
\pgfpathlineto{\pgfqpoint{2.498426in}{0.997123in}}%
\pgfpathlineto{\pgfqpoint{2.500341in}{0.999870in}}%
\pgfpathlineto{\pgfqpoint{2.509917in}{1.001205in}}%
\pgfpathlineto{\pgfqpoint{2.513747in}{1.003966in}}%
\pgfpathlineto{\pgfqpoint{2.523323in}{1.005179in}}%
\pgfpathlineto{\pgfqpoint{2.529068in}{1.006342in}}%
\pgfpathlineto{\pgfqpoint{2.532899in}{1.007849in}}%
\pgfpathlineto{\pgfqpoint{2.536729in}{1.008569in}}%
\pgfpathlineto{\pgfqpoint{2.557796in}{1.012606in}}%
\pgfpathlineto{\pgfqpoint{2.561626in}{1.012872in}}%
\pgfpathlineto{\pgfqpoint{2.565457in}{1.015743in}}%
\pgfpathlineto{\pgfqpoint{2.569287in}{1.016717in}}%
\pgfpathlineto{\pgfqpoint{2.573117in}{1.018092in}}%
\pgfpathlineto{\pgfqpoint{2.575033in}{1.018320in}}%
\pgfpathlineto{\pgfqpoint{2.576948in}{1.020656in}}%
\pgfpathlineto{\pgfqpoint{2.580778in}{1.021639in}}%
\pgfpathlineto{\pgfqpoint{2.584608in}{1.023294in}}%
\pgfpathlineto{\pgfqpoint{2.590354in}{1.024158in}}%
\pgfpathlineto{\pgfqpoint{2.592269in}{1.026817in}}%
\pgfpathlineto{\pgfqpoint{2.594184in}{1.027094in}}%
\pgfpathlineto{\pgfqpoint{2.596099in}{1.029692in}}%
\pgfpathlineto{\pgfqpoint{2.598015in}{1.029857in}}%
\pgfpathlineto{\pgfqpoint{2.599930in}{1.032096in}}%
\pgfpathlineto{\pgfqpoint{2.605675in}{1.034423in}}%
\pgfpathlineto{\pgfqpoint{2.611421in}{1.039133in}}%
\pgfpathlineto{\pgfqpoint{2.626742in}{1.043293in}}%
\pgfpathlineto{\pgfqpoint{2.628657in}{1.044111in}}%
\pgfpathlineto{\pgfqpoint{2.630573in}{1.046152in}}%
\pgfpathlineto{\pgfqpoint{2.632488in}{1.051263in}}%
\pgfpathlineto{\pgfqpoint{2.642064in}{1.053898in}}%
\pgfpathlineto{\pgfqpoint{2.645894in}{1.058431in}}%
\pgfpathlineto{\pgfqpoint{2.647809in}{1.059291in}}%
\pgfpathlineto{\pgfqpoint{2.651639in}{1.062244in}}%
\pgfpathlineto{\pgfqpoint{2.659300in}{1.066143in}}%
\pgfpathlineto{\pgfqpoint{2.682282in}{1.071259in}}%
\pgfpathlineto{\pgfqpoint{2.684197in}{1.073698in}}%
\pgfpathlineto{\pgfqpoint{2.686113in}{1.074109in}}%
\pgfpathlineto{\pgfqpoint{2.688028in}{1.077390in}}%
\pgfpathlineto{\pgfqpoint{2.691858in}{1.078985in}}%
\pgfpathlineto{\pgfqpoint{2.695688in}{1.081315in}}%
\pgfpathlineto{\pgfqpoint{2.697604in}{1.081341in}}%
\pgfpathlineto{\pgfqpoint{2.699519in}{1.084661in}}%
\pgfpathlineto{\pgfqpoint{2.703349in}{1.086777in}}%
\pgfpathlineto{\pgfqpoint{2.709095in}{1.092806in}}%
\pgfpathlineto{\pgfqpoint{2.712925in}{1.094980in}}%
\pgfpathlineto{\pgfqpoint{2.718670in}{1.108459in}}%
\pgfpathlineto{\pgfqpoint{2.722501in}{1.116280in}}%
\pgfpathlineto{\pgfqpoint{2.724416in}{1.118961in}}%
\pgfpathlineto{\pgfqpoint{2.726331in}{1.119126in}}%
\pgfpathlineto{\pgfqpoint{2.732077in}{1.121740in}}%
\pgfpathlineto{\pgfqpoint{2.737822in}{1.122041in}}%
\pgfpathlineto{\pgfqpoint{2.739737in}{1.128837in}}%
\pgfpathlineto{\pgfqpoint{2.741653in}{1.131149in}}%
\pgfpathlineto{\pgfqpoint{2.743568in}{1.142341in}}%
\pgfpathlineto{\pgfqpoint{2.747398in}{1.143926in}}%
\pgfpathlineto{\pgfqpoint{2.749313in}{1.152235in}}%
\pgfpathlineto{\pgfqpoint{2.753144in}{1.152754in}}%
\pgfpathlineto{\pgfqpoint{2.755059in}{1.163088in}}%
\pgfpathlineto{\pgfqpoint{2.756974in}{1.164581in}}%
\pgfpathlineto{\pgfqpoint{2.760804in}{1.170521in}}%
\pgfpathlineto{\pgfqpoint{2.762719in}{1.178607in}}%
\pgfpathlineto{\pgfqpoint{2.764635in}{1.180313in}}%
\pgfpathlineto{\pgfqpoint{2.768465in}{1.189905in}}%
\pgfpathlineto{\pgfqpoint{2.770380in}{1.191539in}}%
\pgfpathlineto{\pgfqpoint{2.776126in}{1.202914in}}%
\pgfpathlineto{\pgfqpoint{2.779956in}{1.219104in}}%
\pgfpathlineto{\pgfqpoint{2.781871in}{1.223605in}}%
\pgfpathlineto{\pgfqpoint{2.785701in}{1.240588in}}%
\pgfpathlineto{\pgfqpoint{2.787617in}{1.240613in}}%
\pgfpathlineto{\pgfqpoint{2.791447in}{1.244129in}}%
\pgfpathlineto{\pgfqpoint{2.793362in}{1.248530in}}%
\pgfpathlineto{\pgfqpoint{2.795277in}{1.258087in}}%
\pgfpathlineto{\pgfqpoint{2.797192in}{1.276958in}}%
\pgfpathlineto{\pgfqpoint{2.799108in}{1.278513in}}%
\pgfpathlineto{\pgfqpoint{2.802938in}{1.290776in}}%
\pgfpathlineto{\pgfqpoint{2.806768in}{1.339754in}}%
\pgfpathlineto{\pgfqpoint{2.808684in}{1.379698in}}%
\pgfpathlineto{\pgfqpoint{2.810599in}{1.381391in}}%
\pgfpathlineto{\pgfqpoint{2.812514in}{1.385237in}}%
\pgfpathlineto{\pgfqpoint{2.822090in}{1.589779in}}%
\pgfpathlineto{\pgfqpoint{2.825920in}{1.796014in}}%
\pgfpathlineto{\pgfqpoint{2.827835in}{1.826535in}}%
\pgfpathlineto{\pgfqpoint{2.827835in}{1.826535in}}%
\pgfusepath{stroke}%
\end{pgfscope}%
\begin{pgfscope}%
\pgfpathrectangle{\pgfqpoint{0.694334in}{0.523557in}}{\pgfqpoint{3.830343in}{1.302977in}}%
\pgfusepath{clip}%
\pgfsetbuttcap%
\pgfsetroundjoin%
\pgfsetlinewidth{1.003750pt}%
\definecolor{currentstroke}{rgb}{0.062745,0.000000,0.062745}%
\pgfsetstrokecolor{currentstroke}%
\pgfsetdash{{3.700000pt}{1.600000pt}}{0.000000pt}%
\pgfpathmoveto{\pgfqpoint{0.694334in}{0.613292in}}%
\pgfpathlineto{\pgfqpoint{0.696249in}{0.621949in}}%
\pgfpathlineto{\pgfqpoint{0.698165in}{0.622951in}}%
\pgfpathlineto{\pgfqpoint{0.700080in}{0.636570in}}%
\pgfpathlineto{\pgfqpoint{0.703910in}{0.641552in}}%
\pgfpathlineto{\pgfqpoint{0.705825in}{0.649251in}}%
\pgfpathlineto{\pgfqpoint{0.711571in}{0.652708in}}%
\pgfpathlineto{\pgfqpoint{0.713486in}{0.652973in}}%
\pgfpathlineto{\pgfqpoint{0.715401in}{0.660913in}}%
\pgfpathlineto{\pgfqpoint{0.721147in}{0.664229in}}%
\pgfpathlineto{\pgfqpoint{0.723062in}{0.668212in}}%
\pgfpathlineto{\pgfqpoint{0.724977in}{0.668256in}}%
\pgfpathlineto{\pgfqpoint{0.730723in}{0.673020in}}%
\pgfpathlineto{\pgfqpoint{0.734553in}{0.683462in}}%
\pgfpathlineto{\pgfqpoint{0.742214in}{0.687870in}}%
\pgfpathlineto{\pgfqpoint{0.744129in}{0.690797in}}%
\pgfpathlineto{\pgfqpoint{0.753705in}{0.693659in}}%
\pgfpathlineto{\pgfqpoint{0.755620in}{0.697917in}}%
\pgfpathlineto{\pgfqpoint{0.757535in}{0.699322in}}%
\pgfpathlineto{\pgfqpoint{0.759450in}{0.711299in}}%
\pgfpathlineto{\pgfqpoint{0.774772in}{0.714803in}}%
\pgfpathlineto{\pgfqpoint{0.782432in}{0.715851in}}%
\pgfpathlineto{\pgfqpoint{0.803499in}{0.718816in}}%
\pgfpathlineto{\pgfqpoint{0.813075in}{0.719973in}}%
\pgfpathlineto{\pgfqpoint{0.836057in}{0.724418in}}%
\pgfpathlineto{\pgfqpoint{0.843718in}{0.725105in}}%
\pgfpathlineto{\pgfqpoint{0.864785in}{0.726369in}}%
\pgfpathlineto{\pgfqpoint{0.870530in}{0.728363in}}%
\pgfpathlineto{\pgfqpoint{0.885851in}{0.729985in}}%
\pgfpathlineto{\pgfqpoint{0.935646in}{0.733917in}}%
\pgfpathlineto{\pgfqpoint{0.966289in}{0.735553in}}%
\pgfpathlineto{\pgfqpoint{0.970119in}{0.736329in}}%
\pgfpathlineto{\pgfqpoint{1.002677in}{0.738756in}}%
\pgfpathlineto{\pgfqpoint{1.027574in}{0.739852in}}%
\pgfpathlineto{\pgfqpoint{1.040980in}{0.740505in}}%
\pgfpathlineto{\pgfqpoint{1.104181in}{0.744093in}}%
\pgfpathlineto{\pgfqpoint{1.303359in}{0.758814in}}%
\pgfpathlineto{\pgfqpoint{1.332086in}{0.760504in}}%
\pgfpathlineto{\pgfqpoint{1.347408in}{0.761997in}}%
\pgfpathlineto{\pgfqpoint{1.358899in}{0.762976in}}%
\pgfpathlineto{\pgfqpoint{1.410608in}{0.769284in}}%
\pgfpathlineto{\pgfqpoint{1.420184in}{0.773766in}}%
\pgfpathlineto{\pgfqpoint{1.424015in}{0.775241in}}%
\pgfpathlineto{\pgfqpoint{1.431675in}{0.776193in}}%
\pgfpathlineto{\pgfqpoint{1.435506in}{0.777845in}}%
\pgfpathlineto{\pgfqpoint{1.439336in}{0.778714in}}%
\pgfpathlineto{\pgfqpoint{1.441251in}{0.781600in}}%
\pgfpathlineto{\pgfqpoint{1.443166in}{0.781687in}}%
\pgfpathlineto{\pgfqpoint{1.445081in}{0.783849in}}%
\pgfpathlineto{\pgfqpoint{1.446997in}{0.784167in}}%
\pgfpathlineto{\pgfqpoint{1.448912in}{0.786177in}}%
\pgfpathlineto{\pgfqpoint{1.462318in}{0.789214in}}%
\pgfpathlineto{\pgfqpoint{1.468064in}{0.790727in}}%
\pgfpathlineto{\pgfqpoint{1.471894in}{0.791007in}}%
\pgfpathlineto{\pgfqpoint{1.473809in}{0.793216in}}%
\pgfpathlineto{\pgfqpoint{1.477639in}{0.795139in}}%
\pgfpathlineto{\pgfqpoint{1.479555in}{0.803461in}}%
\pgfpathlineto{\pgfqpoint{1.483385in}{0.806286in}}%
\pgfpathlineto{\pgfqpoint{1.498706in}{0.812237in}}%
\pgfpathlineto{\pgfqpoint{1.502537in}{0.812525in}}%
\pgfpathlineto{\pgfqpoint{1.506367in}{0.814271in}}%
\pgfpathlineto{\pgfqpoint{1.512112in}{0.814900in}}%
\pgfpathlineto{\pgfqpoint{1.517858in}{0.816308in}}%
\pgfpathlineto{\pgfqpoint{1.523604in}{0.817857in}}%
\pgfpathlineto{\pgfqpoint{1.525519in}{0.820031in}}%
\pgfpathlineto{\pgfqpoint{1.542755in}{0.823617in}}%
\pgfpathlineto{\pgfqpoint{1.586804in}{0.833309in}}%
\pgfpathlineto{\pgfqpoint{1.602126in}{0.834527in}}%
\pgfpathlineto{\pgfqpoint{1.619362in}{0.837079in}}%
\pgfpathlineto{\pgfqpoint{1.650005in}{0.842212in}}%
\pgfpathlineto{\pgfqpoint{1.657666in}{0.843519in}}%
\pgfpathlineto{\pgfqpoint{1.665326in}{0.845086in}}%
\pgfpathlineto{\pgfqpoint{1.672987in}{0.845492in}}%
\pgfpathlineto{\pgfqpoint{1.676817in}{0.846960in}}%
\pgfpathlineto{\pgfqpoint{1.690223in}{0.848621in}}%
\pgfpathlineto{\pgfqpoint{1.699799in}{0.850941in}}%
\pgfpathlineto{\pgfqpoint{1.711290in}{0.852917in}}%
\pgfpathlineto{\pgfqpoint{1.720866in}{0.855801in}}%
\pgfpathlineto{\pgfqpoint{1.730442in}{0.856953in}}%
\pgfpathlineto{\pgfqpoint{1.768745in}{0.864715in}}%
\pgfpathlineto{\pgfqpoint{1.776406in}{0.866159in}}%
\pgfpathlineto{\pgfqpoint{1.822370in}{0.871388in}}%
\pgfpathlineto{\pgfqpoint{1.824285in}{0.873073in}}%
\pgfpathlineto{\pgfqpoint{1.830031in}{0.874370in}}%
\pgfpathlineto{\pgfqpoint{1.833861in}{0.876147in}}%
\pgfpathlineto{\pgfqpoint{1.877910in}{0.882911in}}%
\pgfpathlineto{\pgfqpoint{1.883656in}{0.885176in}}%
\pgfpathlineto{\pgfqpoint{1.895147in}{0.887468in}}%
\pgfpathlineto{\pgfqpoint{1.900892in}{0.888286in}}%
\pgfpathlineto{\pgfqpoint{1.923874in}{0.892108in}}%
\pgfpathlineto{\pgfqpoint{1.927705in}{0.893412in}}%
\pgfpathlineto{\pgfqpoint{1.937281in}{0.894443in}}%
\pgfpathlineto{\pgfqpoint{1.950687in}{0.895245in}}%
\pgfpathlineto{\pgfqpoint{1.954517in}{0.896470in}}%
\pgfpathlineto{\pgfqpoint{1.958347in}{0.897567in}}%
\pgfpathlineto{\pgfqpoint{1.966008in}{0.899326in}}%
\pgfpathlineto{\pgfqpoint{1.994736in}{0.904957in}}%
\pgfpathlineto{\pgfqpoint{2.006227in}{0.906883in}}%
\pgfpathlineto{\pgfqpoint{2.011972in}{0.907787in}}%
\pgfpathlineto{\pgfqpoint{2.021548in}{0.909620in}}%
\pgfpathlineto{\pgfqpoint{2.025378in}{0.910426in}}%
\pgfpathlineto{\pgfqpoint{2.040700in}{0.911965in}}%
\pgfpathlineto{\pgfqpoint{2.088579in}{0.921193in}}%
\pgfpathlineto{\pgfqpoint{2.094325in}{0.923094in}}%
\pgfpathlineto{\pgfqpoint{2.107731in}{0.924941in}}%
\pgfpathlineto{\pgfqpoint{2.115391in}{0.925691in}}%
\pgfpathlineto{\pgfqpoint{2.121137in}{0.926982in}}%
\pgfpathlineto{\pgfqpoint{2.123052in}{0.927051in}}%
\pgfpathlineto{\pgfqpoint{2.126883in}{0.928355in}}%
\pgfpathlineto{\pgfqpoint{2.165186in}{0.934156in}}%
\pgfpathlineto{\pgfqpoint{2.169016in}{0.935670in}}%
\pgfpathlineto{\pgfqpoint{2.209235in}{0.943307in}}%
\pgfpathlineto{\pgfqpoint{2.213065in}{0.944656in}}%
\pgfpathlineto{\pgfqpoint{2.283927in}{0.953096in}}%
\pgfpathlineto{\pgfqpoint{2.295418in}{0.956536in}}%
\pgfpathlineto{\pgfqpoint{2.312654in}{0.959066in}}%
\pgfpathlineto{\pgfqpoint{2.318400in}{0.960021in}}%
\pgfpathlineto{\pgfqpoint{2.322230in}{0.961899in}}%
\pgfpathlineto{\pgfqpoint{2.326060in}{0.962330in}}%
\pgfpathlineto{\pgfqpoint{2.327975in}{0.964487in}}%
\pgfpathlineto{\pgfqpoint{2.343297in}{0.967160in}}%
\pgfpathlineto{\pgfqpoint{2.352873in}{0.970888in}}%
\pgfpathlineto{\pgfqpoint{2.354788in}{0.971079in}}%
\pgfpathlineto{\pgfqpoint{2.356703in}{0.972769in}}%
\pgfpathlineto{\pgfqpoint{2.360533in}{0.973870in}}%
\pgfpathlineto{\pgfqpoint{2.368194in}{0.975749in}}%
\pgfpathlineto{\pgfqpoint{2.373940in}{0.978564in}}%
\pgfpathlineto{\pgfqpoint{2.385431in}{0.981230in}}%
\pgfpathlineto{\pgfqpoint{2.389261in}{0.982244in}}%
\pgfpathlineto{\pgfqpoint{2.400752in}{0.987375in}}%
\pgfpathlineto{\pgfqpoint{2.406498in}{0.989127in}}%
\pgfpathlineto{\pgfqpoint{2.435225in}{0.998211in}}%
\pgfpathlineto{\pgfqpoint{2.442886in}{0.999568in}}%
\pgfpathlineto{\pgfqpoint{2.446716in}{1.000610in}}%
\pgfpathlineto{\pgfqpoint{2.454377in}{1.001857in}}%
\pgfpathlineto{\pgfqpoint{2.467783in}{1.008305in}}%
\pgfpathlineto{\pgfqpoint{2.473529in}{1.008820in}}%
\pgfpathlineto{\pgfqpoint{2.479274in}{1.011252in}}%
\pgfpathlineto{\pgfqpoint{2.486935in}{1.011900in}}%
\pgfpathlineto{\pgfqpoint{2.490765in}{1.013914in}}%
\pgfpathlineto{\pgfqpoint{2.496511in}{1.014687in}}%
\pgfpathlineto{\pgfqpoint{2.500341in}{1.015832in}}%
\pgfpathlineto{\pgfqpoint{2.515662in}{1.018394in}}%
\pgfpathlineto{\pgfqpoint{2.517577in}{1.018503in}}%
\pgfpathlineto{\pgfqpoint{2.523323in}{1.021593in}}%
\pgfpathlineto{\pgfqpoint{2.529068in}{1.022097in}}%
\pgfpathlineto{\pgfqpoint{2.530984in}{1.024576in}}%
\pgfpathlineto{\pgfqpoint{2.534814in}{1.026401in}}%
\pgfpathlineto{\pgfqpoint{2.536729in}{1.029089in}}%
\pgfpathlineto{\pgfqpoint{2.542475in}{1.029334in}}%
\pgfpathlineto{\pgfqpoint{2.546305in}{1.031708in}}%
\pgfpathlineto{\pgfqpoint{2.553966in}{1.034094in}}%
\pgfpathlineto{\pgfqpoint{2.557796in}{1.034626in}}%
\pgfpathlineto{\pgfqpoint{2.561626in}{1.036665in}}%
\pgfpathlineto{\pgfqpoint{2.569287in}{1.039114in}}%
\pgfpathlineto{\pgfqpoint{2.571202in}{1.039116in}}%
\pgfpathlineto{\pgfqpoint{2.575033in}{1.043055in}}%
\pgfpathlineto{\pgfqpoint{2.594184in}{1.046713in}}%
\pgfpathlineto{\pgfqpoint{2.596099in}{1.048649in}}%
\pgfpathlineto{\pgfqpoint{2.601845in}{1.049867in}}%
\pgfpathlineto{\pgfqpoint{2.603760in}{1.052652in}}%
\pgfpathlineto{\pgfqpoint{2.609506in}{1.054760in}}%
\pgfpathlineto{\pgfqpoint{2.611421in}{1.056654in}}%
\pgfpathlineto{\pgfqpoint{2.615251in}{1.056941in}}%
\pgfpathlineto{\pgfqpoint{2.619082in}{1.058611in}}%
\pgfpathlineto{\pgfqpoint{2.620997in}{1.058659in}}%
\pgfpathlineto{\pgfqpoint{2.626742in}{1.063202in}}%
\pgfpathlineto{\pgfqpoint{2.632488in}{1.068559in}}%
\pgfpathlineto{\pgfqpoint{2.638233in}{1.070863in}}%
\pgfpathlineto{\pgfqpoint{2.640148in}{1.073210in}}%
\pgfpathlineto{\pgfqpoint{2.642064in}{1.073376in}}%
\pgfpathlineto{\pgfqpoint{2.653555in}{1.079961in}}%
\pgfpathlineto{\pgfqpoint{2.659300in}{1.081531in}}%
\pgfpathlineto{\pgfqpoint{2.665046in}{1.083021in}}%
\pgfpathlineto{\pgfqpoint{2.668876in}{1.086147in}}%
\pgfpathlineto{\pgfqpoint{2.672706in}{1.089786in}}%
\pgfpathlineto{\pgfqpoint{2.674622in}{1.092776in}}%
\pgfpathlineto{\pgfqpoint{2.676537in}{1.093057in}}%
\pgfpathlineto{\pgfqpoint{2.680367in}{1.095473in}}%
\pgfpathlineto{\pgfqpoint{2.684197in}{1.096841in}}%
\pgfpathlineto{\pgfqpoint{2.693773in}{1.098690in}}%
\pgfpathlineto{\pgfqpoint{2.697604in}{1.100264in}}%
\pgfpathlineto{\pgfqpoint{2.705264in}{1.104803in}}%
\pgfpathlineto{\pgfqpoint{2.707179in}{1.108273in}}%
\pgfpathlineto{\pgfqpoint{2.720586in}{1.115609in}}%
\pgfpathlineto{\pgfqpoint{2.726331in}{1.125363in}}%
\pgfpathlineto{\pgfqpoint{2.728246in}{1.129427in}}%
\pgfpathlineto{\pgfqpoint{2.730161in}{1.129533in}}%
\pgfpathlineto{\pgfqpoint{2.737822in}{1.135682in}}%
\pgfpathlineto{\pgfqpoint{2.741653in}{1.139500in}}%
\pgfpathlineto{\pgfqpoint{2.743568in}{1.139923in}}%
\pgfpathlineto{\pgfqpoint{2.745483in}{1.149433in}}%
\pgfpathlineto{\pgfqpoint{2.747398in}{1.150761in}}%
\pgfpathlineto{\pgfqpoint{2.749313in}{1.159555in}}%
\pgfpathlineto{\pgfqpoint{2.753144in}{1.163879in}}%
\pgfpathlineto{\pgfqpoint{2.758889in}{1.169167in}}%
\pgfpathlineto{\pgfqpoint{2.760804in}{1.176447in}}%
\pgfpathlineto{\pgfqpoint{2.764635in}{1.179226in}}%
\pgfpathlineto{\pgfqpoint{2.774210in}{1.200352in}}%
\pgfpathlineto{\pgfqpoint{2.776126in}{1.200960in}}%
\pgfpathlineto{\pgfqpoint{2.779956in}{1.211885in}}%
\pgfpathlineto{\pgfqpoint{2.781871in}{1.223561in}}%
\pgfpathlineto{\pgfqpoint{2.787617in}{1.233278in}}%
\pgfpathlineto{\pgfqpoint{2.791447in}{1.235612in}}%
\pgfpathlineto{\pgfqpoint{2.793362in}{1.239034in}}%
\pgfpathlineto{\pgfqpoint{2.795277in}{1.239183in}}%
\pgfpathlineto{\pgfqpoint{2.797192in}{1.241743in}}%
\pgfpathlineto{\pgfqpoint{2.802938in}{1.258053in}}%
\pgfpathlineto{\pgfqpoint{2.804853in}{1.259950in}}%
\pgfpathlineto{\pgfqpoint{2.806768in}{1.260051in}}%
\pgfpathlineto{\pgfqpoint{2.808684in}{1.265087in}}%
\pgfpathlineto{\pgfqpoint{2.810599in}{1.266597in}}%
\pgfpathlineto{\pgfqpoint{2.812514in}{1.266653in}}%
\pgfpathlineto{\pgfqpoint{2.814429in}{1.294932in}}%
\pgfpathlineto{\pgfqpoint{2.816344in}{1.296020in}}%
\pgfpathlineto{\pgfqpoint{2.820175in}{1.422730in}}%
\pgfpathlineto{\pgfqpoint{2.822090in}{1.516447in}}%
\pgfpathlineto{\pgfqpoint{2.824005in}{1.826535in}}%
\pgfpathlineto{\pgfqpoint{2.824005in}{1.826535in}}%
\pgfusepath{stroke}%
\end{pgfscope}%
\begin{pgfscope}%
\pgfpathrectangle{\pgfqpoint{0.694334in}{0.523557in}}{\pgfqpoint{3.830343in}{1.302977in}}%
\pgfusepath{clip}%
\pgfsetbuttcap%
\pgfsetroundjoin%
\pgfsetlinewidth{1.003750pt}%
\definecolor{currentstroke}{rgb}{0.811765,0.125490,0.125490}%
\pgfsetstrokecolor{currentstroke}%
\pgfsetdash{{1.000000pt}{1.650000pt}}{0.000000pt}%
\pgfpathmoveto{\pgfqpoint{0.694334in}{1.092842in}}%
\pgfpathlineto{\pgfqpoint{0.700080in}{1.095601in}}%
\pgfpathlineto{\pgfqpoint{0.713486in}{1.097828in}}%
\pgfpathlineto{\pgfqpoint{0.715401in}{1.099365in}}%
\pgfpathlineto{\pgfqpoint{0.723062in}{1.100590in}}%
\pgfpathlineto{\pgfqpoint{0.738383in}{1.104836in}}%
\pgfpathlineto{\pgfqpoint{0.740298in}{1.108199in}}%
\pgfpathlineto{\pgfqpoint{0.744129in}{1.109019in}}%
\pgfpathlineto{\pgfqpoint{0.747959in}{1.109596in}}%
\pgfpathlineto{\pgfqpoint{0.749874in}{1.118682in}}%
\pgfpathlineto{\pgfqpoint{0.751789in}{1.118957in}}%
\pgfpathlineto{\pgfqpoint{0.755620in}{1.131559in}}%
\pgfpathlineto{\pgfqpoint{0.757535in}{1.132092in}}%
\pgfpathlineto{\pgfqpoint{0.759450in}{1.137051in}}%
\pgfpathlineto{\pgfqpoint{0.761365in}{1.138654in}}%
\pgfpathlineto{\pgfqpoint{0.765196in}{1.145963in}}%
\pgfpathlineto{\pgfqpoint{0.767111in}{1.146688in}}%
\pgfpathlineto{\pgfqpoint{0.769026in}{1.150685in}}%
\pgfpathlineto{\pgfqpoint{0.772856in}{1.152392in}}%
\pgfpathlineto{\pgfqpoint{0.776687in}{1.160254in}}%
\pgfpathlineto{\pgfqpoint{0.784347in}{1.161935in}}%
\pgfpathlineto{\pgfqpoint{0.788178in}{1.173187in}}%
\pgfpathlineto{\pgfqpoint{0.792008in}{1.176899in}}%
\pgfpathlineto{\pgfqpoint{0.795838in}{1.177668in}}%
\pgfpathlineto{\pgfqpoint{0.797754in}{1.180982in}}%
\pgfpathlineto{\pgfqpoint{0.799669in}{1.187098in}}%
\pgfpathlineto{\pgfqpoint{0.805414in}{1.190346in}}%
\pgfpathlineto{\pgfqpoint{0.811160in}{1.192110in}}%
\pgfpathlineto{\pgfqpoint{0.814990in}{1.195951in}}%
\pgfpathlineto{\pgfqpoint{0.818820in}{1.196792in}}%
\pgfpathlineto{\pgfqpoint{0.860954in}{1.207986in}}%
\pgfpathlineto{\pgfqpoint{0.870530in}{1.209480in}}%
\pgfpathlineto{\pgfqpoint{0.883936in}{1.212655in}}%
\pgfpathlineto{\pgfqpoint{0.899258in}{1.214008in}}%
\pgfpathlineto{\pgfqpoint{0.908834in}{1.214543in}}%
\pgfpathlineto{\pgfqpoint{0.914579in}{1.215715in}}%
\pgfpathlineto{\pgfqpoint{0.941391in}{1.218454in}}%
\pgfpathlineto{\pgfqpoint{0.956713in}{1.219640in}}%
\pgfpathlineto{\pgfqpoint{0.964373in}{1.220722in}}%
\pgfpathlineto{\pgfqpoint{0.995016in}{1.223496in}}%
\pgfpathlineto{\pgfqpoint{1.010338in}{1.225303in}}%
\pgfpathlineto{\pgfqpoint{1.276546in}{1.260355in}}%
\pgfpathlineto{\pgfqpoint{1.301444in}{1.262781in}}%
\pgfpathlineto{\pgfqpoint{1.309104in}{1.263265in}}%
\pgfpathlineto{\pgfqpoint{1.314850in}{1.263874in}}%
\pgfpathlineto{\pgfqpoint{1.318680in}{1.265400in}}%
\pgfpathlineto{\pgfqpoint{1.326341in}{1.266687in}}%
\pgfpathlineto{\pgfqpoint{1.347408in}{1.271336in}}%
\pgfpathlineto{\pgfqpoint{1.353153in}{1.272187in}}%
\pgfpathlineto{\pgfqpoint{1.356984in}{1.273188in}}%
\pgfpathlineto{\pgfqpoint{1.368475in}{1.275324in}}%
\pgfpathlineto{\pgfqpoint{1.379966in}{1.276812in}}%
\pgfpathlineto{\pgfqpoint{1.391457in}{1.278655in}}%
\pgfpathlineto{\pgfqpoint{1.406778in}{1.281795in}}%
\pgfpathlineto{\pgfqpoint{1.416354in}{1.282600in}}%
\pgfpathlineto{\pgfqpoint{1.420184in}{1.284273in}}%
\pgfpathlineto{\pgfqpoint{1.431675in}{1.286989in}}%
\pgfpathlineto{\pgfqpoint{1.454657in}{1.290085in}}%
\pgfpathlineto{\pgfqpoint{1.456573in}{1.291995in}}%
\pgfpathlineto{\pgfqpoint{1.462318in}{1.292473in}}%
\pgfpathlineto{\pgfqpoint{1.466148in}{1.293576in}}%
\pgfpathlineto{\pgfqpoint{1.475724in}{1.295512in}}%
\pgfpathlineto{\pgfqpoint{1.485300in}{1.297704in}}%
\pgfpathlineto{\pgfqpoint{1.491046in}{1.298449in}}%
\pgfpathlineto{\pgfqpoint{1.502537in}{1.304469in}}%
\pgfpathlineto{\pgfqpoint{1.515943in}{1.308966in}}%
\pgfpathlineto{\pgfqpoint{1.521688in}{1.311486in}}%
\pgfpathlineto{\pgfqpoint{1.550416in}{1.319829in}}%
\pgfpathlineto{\pgfqpoint{1.556161in}{1.322115in}}%
\pgfpathlineto{\pgfqpoint{1.561907in}{1.323577in}}%
\pgfpathlineto{\pgfqpoint{1.565737in}{1.325470in}}%
\pgfpathlineto{\pgfqpoint{1.567652in}{1.326031in}}%
\pgfpathlineto{\pgfqpoint{1.569568in}{1.327861in}}%
\pgfpathlineto{\pgfqpoint{1.577228in}{1.328459in}}%
\pgfpathlineto{\pgfqpoint{1.581059in}{1.330616in}}%
\pgfpathlineto{\pgfqpoint{1.586804in}{1.330921in}}%
\pgfpathlineto{\pgfqpoint{1.588719in}{1.332514in}}%
\pgfpathlineto{\pgfqpoint{1.590635in}{1.332541in}}%
\pgfpathlineto{\pgfqpoint{1.594465in}{1.336745in}}%
\pgfpathlineto{\pgfqpoint{1.598295in}{1.339926in}}%
\pgfpathlineto{\pgfqpoint{1.604041in}{1.342014in}}%
\pgfpathlineto{\pgfqpoint{1.605956in}{1.343043in}}%
\pgfpathlineto{\pgfqpoint{1.607871in}{1.346368in}}%
\pgfpathlineto{\pgfqpoint{1.617447in}{1.348989in}}%
\pgfpathlineto{\pgfqpoint{1.621277in}{1.351011in}}%
\pgfpathlineto{\pgfqpoint{1.623192in}{1.351528in}}%
\pgfpathlineto{\pgfqpoint{1.628938in}{1.355828in}}%
\pgfpathlineto{\pgfqpoint{1.630853in}{1.356072in}}%
\pgfpathlineto{\pgfqpoint{1.632768in}{1.359363in}}%
\pgfpathlineto{\pgfqpoint{1.638514in}{1.361024in}}%
\pgfpathlineto{\pgfqpoint{1.640429in}{1.366651in}}%
\pgfpathlineto{\pgfqpoint{1.646174in}{1.369896in}}%
\pgfpathlineto{\pgfqpoint{1.653835in}{1.371709in}}%
\pgfpathlineto{\pgfqpoint{1.655750in}{1.372445in}}%
\pgfpathlineto{\pgfqpoint{1.659581in}{1.376545in}}%
\pgfpathlineto{\pgfqpoint{1.661496in}{1.376664in}}%
\pgfpathlineto{\pgfqpoint{1.663411in}{1.378528in}}%
\pgfpathlineto{\pgfqpoint{1.665326in}{1.382384in}}%
\pgfpathlineto{\pgfqpoint{1.671072in}{1.385381in}}%
\pgfpathlineto{\pgfqpoint{1.672987in}{1.385605in}}%
\pgfpathlineto{\pgfqpoint{1.676817in}{1.388631in}}%
\pgfpathlineto{\pgfqpoint{1.678732in}{1.389252in}}%
\pgfpathlineto{\pgfqpoint{1.684478in}{1.394453in}}%
\pgfpathlineto{\pgfqpoint{1.686393in}{1.394889in}}%
\pgfpathlineto{\pgfqpoint{1.690223in}{1.398688in}}%
\pgfpathlineto{\pgfqpoint{1.697884in}{1.403132in}}%
\pgfpathlineto{\pgfqpoint{1.701714in}{1.409587in}}%
\pgfpathlineto{\pgfqpoint{1.713205in}{1.415145in}}%
\pgfpathlineto{\pgfqpoint{1.715121in}{1.416608in}}%
\pgfpathlineto{\pgfqpoint{1.717036in}{1.421963in}}%
\pgfpathlineto{\pgfqpoint{1.726612in}{1.424488in}}%
\pgfpathlineto{\pgfqpoint{1.730442in}{1.427041in}}%
\pgfpathlineto{\pgfqpoint{1.734272in}{1.428226in}}%
\pgfpathlineto{\pgfqpoint{1.738103in}{1.434547in}}%
\pgfpathlineto{\pgfqpoint{1.757254in}{1.441305in}}%
\pgfpathlineto{\pgfqpoint{1.761085in}{1.442517in}}%
\pgfpathlineto{\pgfqpoint{1.764915in}{1.443902in}}%
\pgfpathlineto{\pgfqpoint{1.768745in}{1.444849in}}%
\pgfpathlineto{\pgfqpoint{1.772576in}{1.449568in}}%
\pgfpathlineto{\pgfqpoint{1.782152in}{1.453292in}}%
\pgfpathlineto{\pgfqpoint{1.785982in}{1.456892in}}%
\pgfpathlineto{\pgfqpoint{1.791728in}{1.458672in}}%
\pgfpathlineto{\pgfqpoint{1.793643in}{1.460292in}}%
\pgfpathlineto{\pgfqpoint{1.797473in}{1.460309in}}%
\pgfpathlineto{\pgfqpoint{1.799388in}{1.463854in}}%
\pgfpathlineto{\pgfqpoint{1.803219in}{1.464567in}}%
\pgfpathlineto{\pgfqpoint{1.807049in}{1.464956in}}%
\pgfpathlineto{\pgfqpoint{1.810879in}{1.467354in}}%
\pgfpathlineto{\pgfqpoint{1.814710in}{1.475400in}}%
\pgfpathlineto{\pgfqpoint{1.818540in}{1.476345in}}%
\pgfpathlineto{\pgfqpoint{1.820455in}{1.479490in}}%
\pgfpathlineto{\pgfqpoint{1.831946in}{1.483182in}}%
\pgfpathlineto{\pgfqpoint{1.835776in}{1.486299in}}%
\pgfpathlineto{\pgfqpoint{1.837692in}{1.486766in}}%
\pgfpathlineto{\pgfqpoint{1.841522in}{1.489589in}}%
\pgfpathlineto{\pgfqpoint{1.843437in}{1.489910in}}%
\pgfpathlineto{\pgfqpoint{1.847267in}{1.493070in}}%
\pgfpathlineto{\pgfqpoint{1.860674in}{1.493814in}}%
\pgfpathlineto{\pgfqpoint{1.862589in}{1.495401in}}%
\pgfpathlineto{\pgfqpoint{1.868334in}{1.495891in}}%
\pgfpathlineto{\pgfqpoint{1.872165in}{1.497930in}}%
\pgfpathlineto{\pgfqpoint{1.875995in}{1.499961in}}%
\pgfpathlineto{\pgfqpoint{1.877910in}{1.505456in}}%
\pgfpathlineto{\pgfqpoint{1.885571in}{1.507597in}}%
\pgfpathlineto{\pgfqpoint{1.889401in}{1.511675in}}%
\pgfpathlineto{\pgfqpoint{1.904723in}{1.522836in}}%
\pgfpathlineto{\pgfqpoint{1.910468in}{1.531346in}}%
\pgfpathlineto{\pgfqpoint{1.914298in}{1.531408in}}%
\pgfpathlineto{\pgfqpoint{1.923874in}{1.537233in}}%
\pgfpathlineto{\pgfqpoint{1.925790in}{1.540900in}}%
\pgfpathlineto{\pgfqpoint{1.931535in}{1.543587in}}%
\pgfpathlineto{\pgfqpoint{1.935365in}{1.544622in}}%
\pgfpathlineto{\pgfqpoint{1.944941in}{1.549946in}}%
\pgfpathlineto{\pgfqpoint{1.948772in}{1.550939in}}%
\pgfpathlineto{\pgfqpoint{1.952602in}{1.551894in}}%
\pgfpathlineto{\pgfqpoint{1.956432in}{1.553405in}}%
\pgfpathlineto{\pgfqpoint{1.966008in}{1.557216in}}%
\pgfpathlineto{\pgfqpoint{1.969838in}{1.568010in}}%
\pgfpathlineto{\pgfqpoint{1.973669in}{1.573545in}}%
\pgfpathlineto{\pgfqpoint{1.977499in}{1.573994in}}%
\pgfpathlineto{\pgfqpoint{1.983245in}{1.575787in}}%
\pgfpathlineto{\pgfqpoint{1.985160in}{1.575964in}}%
\pgfpathlineto{\pgfqpoint{1.988990in}{1.578691in}}%
\pgfpathlineto{\pgfqpoint{1.996651in}{1.580364in}}%
\pgfpathlineto{\pgfqpoint{2.010057in}{1.585292in}}%
\pgfpathlineto{\pgfqpoint{2.013887in}{1.595369in}}%
\pgfpathlineto{\pgfqpoint{2.019633in}{1.598076in}}%
\pgfpathlineto{\pgfqpoint{2.025378in}{1.601973in}}%
\pgfpathlineto{\pgfqpoint{2.031124in}{1.603684in}}%
\pgfpathlineto{\pgfqpoint{2.040700in}{1.607022in}}%
\pgfpathlineto{\pgfqpoint{2.050276in}{1.620180in}}%
\pgfpathlineto{\pgfqpoint{2.054106in}{1.621879in}}%
\pgfpathlineto{\pgfqpoint{2.061767in}{1.622929in}}%
\pgfpathlineto{\pgfqpoint{2.065597in}{1.626878in}}%
\pgfpathlineto{\pgfqpoint{2.082834in}{1.631262in}}%
\pgfpathlineto{\pgfqpoint{2.086664in}{1.633537in}}%
\pgfpathlineto{\pgfqpoint{2.088579in}{1.633905in}}%
\pgfpathlineto{\pgfqpoint{2.090494in}{1.636687in}}%
\pgfpathlineto{\pgfqpoint{2.092409in}{1.641522in}}%
\pgfpathlineto{\pgfqpoint{2.098155in}{1.643104in}}%
\pgfpathlineto{\pgfqpoint{2.103900in}{1.647931in}}%
\pgfpathlineto{\pgfqpoint{2.105816in}{1.648707in}}%
\pgfpathlineto{\pgfqpoint{2.107731in}{1.652064in}}%
\pgfpathlineto{\pgfqpoint{2.113476in}{1.653067in}}%
\pgfpathlineto{\pgfqpoint{2.124967in}{1.657318in}}%
\pgfpathlineto{\pgfqpoint{2.126883in}{1.660554in}}%
\pgfpathlineto{\pgfqpoint{2.132628in}{1.663152in}}%
\pgfpathlineto{\pgfqpoint{2.136458in}{1.667398in}}%
\pgfpathlineto{\pgfqpoint{2.146034in}{1.672284in}}%
\pgfpathlineto{\pgfqpoint{2.153695in}{1.677501in}}%
\pgfpathlineto{\pgfqpoint{2.165186in}{1.681182in}}%
\pgfpathlineto{\pgfqpoint{2.174762in}{1.687722in}}%
\pgfpathlineto{\pgfqpoint{2.182422in}{1.698022in}}%
\pgfpathlineto{\pgfqpoint{2.186253in}{1.699619in}}%
\pgfpathlineto{\pgfqpoint{2.188168in}{1.702234in}}%
\pgfpathlineto{\pgfqpoint{2.191998in}{1.702478in}}%
\pgfpathlineto{\pgfqpoint{2.203489in}{1.713686in}}%
\pgfpathlineto{\pgfqpoint{2.207320in}{1.714462in}}%
\pgfpathlineto{\pgfqpoint{2.209235in}{1.719630in}}%
\pgfpathlineto{\pgfqpoint{2.216896in}{1.722760in}}%
\pgfpathlineto{\pgfqpoint{2.220726in}{1.726786in}}%
\pgfpathlineto{\pgfqpoint{2.222641in}{1.727246in}}%
\pgfpathlineto{\pgfqpoint{2.226471in}{1.730974in}}%
\pgfpathlineto{\pgfqpoint{2.230302in}{1.731978in}}%
\pgfpathlineto{\pgfqpoint{2.237962in}{1.734587in}}%
\pgfpathlineto{\pgfqpoint{2.241793in}{1.736091in}}%
\pgfpathlineto{\pgfqpoint{2.247538in}{1.739614in}}%
\pgfpathlineto{\pgfqpoint{2.253284in}{1.741282in}}%
\pgfpathlineto{\pgfqpoint{2.262860in}{1.746025in}}%
\pgfpathlineto{\pgfqpoint{2.268605in}{1.747129in}}%
\pgfpathlineto{\pgfqpoint{2.272436in}{1.749142in}}%
\pgfpathlineto{\pgfqpoint{2.278181in}{1.759746in}}%
\pgfpathlineto{\pgfqpoint{2.285842in}{1.765681in}}%
\pgfpathlineto{\pgfqpoint{2.291587in}{1.766759in}}%
\pgfpathlineto{\pgfqpoint{2.293502in}{1.767371in}}%
\pgfpathlineto{\pgfqpoint{2.295418in}{1.772118in}}%
\pgfpathlineto{\pgfqpoint{2.301163in}{1.775959in}}%
\pgfpathlineto{\pgfqpoint{2.303078in}{1.776410in}}%
\pgfpathlineto{\pgfqpoint{2.306909in}{1.778441in}}%
\pgfpathlineto{\pgfqpoint{2.312654in}{1.780935in}}%
\pgfpathlineto{\pgfqpoint{2.320315in}{1.782097in}}%
\pgfpathlineto{\pgfqpoint{2.327975in}{1.786705in}}%
\pgfpathlineto{\pgfqpoint{2.331806in}{1.788304in}}%
\pgfpathlineto{\pgfqpoint{2.333721in}{1.789322in}}%
\pgfpathlineto{\pgfqpoint{2.335636in}{1.791670in}}%
\pgfpathlineto{\pgfqpoint{2.337551in}{1.792129in}}%
\pgfpathlineto{\pgfqpoint{2.339467in}{1.796183in}}%
\pgfpathlineto{\pgfqpoint{2.343297in}{1.797173in}}%
\pgfpathlineto{\pgfqpoint{2.345212in}{1.799983in}}%
\pgfpathlineto{\pgfqpoint{2.347127in}{1.800002in}}%
\pgfpathlineto{\pgfqpoint{2.349042in}{1.803257in}}%
\pgfpathlineto{\pgfqpoint{2.358618in}{1.807884in}}%
\pgfpathlineto{\pgfqpoint{2.360533in}{1.809997in}}%
\pgfpathlineto{\pgfqpoint{2.362449in}{1.810025in}}%
\pgfpathlineto{\pgfqpoint{2.366279in}{1.815338in}}%
\pgfpathlineto{\pgfqpoint{2.373940in}{1.816209in}}%
\pgfpathlineto{\pgfqpoint{2.377770in}{1.818488in}}%
\pgfpathlineto{\pgfqpoint{2.381600in}{1.823308in}}%
\pgfpathlineto{\pgfqpoint{2.385431in}{1.823974in}}%
\pgfpathlineto{\pgfqpoint{2.387346in}{1.826535in}}%
\pgfpathlineto{\pgfqpoint{2.387346in}{1.826535in}}%
\pgfusepath{stroke}%
\end{pgfscope}%
\begin{pgfscope}%
\pgfpathrectangle{\pgfqpoint{0.694334in}{0.523557in}}{\pgfqpoint{3.830343in}{1.302977in}}%
\pgfusepath{clip}%
\pgfsetrectcap%
\pgfsetroundjoin%
\pgfsetlinewidth{1.003750pt}%
\definecolor{currentstroke}{rgb}{0.000000,0.000000,0.376471}%
\pgfsetstrokecolor{currentstroke}%
\pgfsetdash{}{0pt}%
\pgfpathmoveto{\pgfqpoint{0.694334in}{0.599366in}}%
\pgfpathlineto{\pgfqpoint{0.696249in}{0.616722in}}%
\pgfpathlineto{\pgfqpoint{0.698165in}{0.623343in}}%
\pgfpathlineto{\pgfqpoint{0.700080in}{0.624942in}}%
\pgfpathlineto{\pgfqpoint{0.701995in}{0.628627in}}%
\pgfpathlineto{\pgfqpoint{0.707741in}{0.675393in}}%
\pgfpathlineto{\pgfqpoint{0.709656in}{0.679304in}}%
\pgfpathlineto{\pgfqpoint{0.713486in}{0.680868in}}%
\pgfpathlineto{\pgfqpoint{0.715401in}{0.684930in}}%
\pgfpathlineto{\pgfqpoint{0.717316in}{0.685863in}}%
\pgfpathlineto{\pgfqpoint{0.721147in}{0.699237in}}%
\pgfpathlineto{\pgfqpoint{0.724977in}{0.724415in}}%
\pgfpathlineto{\pgfqpoint{0.728807in}{0.731356in}}%
\pgfpathlineto{\pgfqpoint{0.730723in}{0.731663in}}%
\pgfpathlineto{\pgfqpoint{0.732638in}{0.733313in}}%
\pgfpathlineto{\pgfqpoint{0.736468in}{0.745375in}}%
\pgfpathlineto{\pgfqpoint{0.749874in}{0.754558in}}%
\pgfpathlineto{\pgfqpoint{0.757535in}{0.755842in}}%
\pgfpathlineto{\pgfqpoint{0.765196in}{0.759516in}}%
\pgfpathlineto{\pgfqpoint{0.769026in}{0.761092in}}%
\pgfpathlineto{\pgfqpoint{0.778602in}{0.763699in}}%
\pgfpathlineto{\pgfqpoint{0.780517in}{0.765949in}}%
\pgfpathlineto{\pgfqpoint{0.782432in}{0.766314in}}%
\pgfpathlineto{\pgfqpoint{0.784347in}{0.770570in}}%
\pgfpathlineto{\pgfqpoint{0.795838in}{0.772555in}}%
\pgfpathlineto{\pgfqpoint{0.834142in}{0.780708in}}%
\pgfpathlineto{\pgfqpoint{0.837972in}{0.781987in}}%
\pgfpathlineto{\pgfqpoint{0.841803in}{0.783419in}}%
\pgfpathlineto{\pgfqpoint{0.843718in}{0.784883in}}%
\pgfpathlineto{\pgfqpoint{0.847548in}{0.785885in}}%
\pgfpathlineto{\pgfqpoint{0.853294in}{0.788128in}}%
\pgfpathlineto{\pgfqpoint{0.855209in}{0.791418in}}%
\pgfpathlineto{\pgfqpoint{0.857124in}{0.791448in}}%
\pgfpathlineto{\pgfqpoint{0.860954in}{0.793652in}}%
\pgfpathlineto{\pgfqpoint{0.866700in}{0.795597in}}%
\pgfpathlineto{\pgfqpoint{0.870530in}{0.798359in}}%
\pgfpathlineto{\pgfqpoint{0.872445in}{0.798803in}}%
\pgfpathlineto{\pgfqpoint{0.876276in}{0.802324in}}%
\pgfpathlineto{\pgfqpoint{0.885851in}{0.804749in}}%
\pgfpathlineto{\pgfqpoint{0.903088in}{0.810705in}}%
\pgfpathlineto{\pgfqpoint{0.905003in}{0.813318in}}%
\pgfpathlineto{\pgfqpoint{0.912664in}{0.815343in}}%
\pgfpathlineto{\pgfqpoint{0.914579in}{0.817710in}}%
\pgfpathlineto{\pgfqpoint{0.918409in}{0.818198in}}%
\pgfpathlineto{\pgfqpoint{0.920325in}{0.820676in}}%
\pgfpathlineto{\pgfqpoint{0.926070in}{0.821629in}}%
\pgfpathlineto{\pgfqpoint{0.929900in}{0.823259in}}%
\pgfpathlineto{\pgfqpoint{0.939476in}{0.826427in}}%
\pgfpathlineto{\pgfqpoint{0.943307in}{0.830369in}}%
\pgfpathlineto{\pgfqpoint{0.952882in}{0.834160in}}%
\pgfpathlineto{\pgfqpoint{0.964373in}{0.836130in}}%
\pgfpathlineto{\pgfqpoint{0.966289in}{0.837963in}}%
\pgfpathlineto{\pgfqpoint{0.972034in}{0.838534in}}%
\pgfpathlineto{\pgfqpoint{0.975864in}{0.841301in}}%
\pgfpathlineto{\pgfqpoint{0.977780in}{0.841407in}}%
\pgfpathlineto{\pgfqpoint{0.979695in}{0.843140in}}%
\pgfpathlineto{\pgfqpoint{0.985440in}{0.844131in}}%
\pgfpathlineto{\pgfqpoint{0.989271in}{0.845658in}}%
\pgfpathlineto{\pgfqpoint{1.008422in}{0.852766in}}%
\pgfpathlineto{\pgfqpoint{1.021829in}{0.854652in}}%
\pgfpathlineto{\pgfqpoint{1.025659in}{0.856194in}}%
\pgfpathlineto{\pgfqpoint{1.042895in}{0.858909in}}%
\pgfpathlineto{\pgfqpoint{1.056302in}{0.860634in}}%
\pgfpathlineto{\pgfqpoint{1.060132in}{0.862067in}}%
\pgfpathlineto{\pgfqpoint{1.088860in}{0.868666in}}%
\pgfpathlineto{\pgfqpoint{1.100351in}{0.871832in}}%
\pgfpathlineto{\pgfqpoint{1.108011in}{0.873189in}}%
\pgfpathlineto{\pgfqpoint{1.111842in}{0.873923in}}%
\pgfpathlineto{\pgfqpoint{1.121418in}{0.875284in}}%
\pgfpathlineto{\pgfqpoint{1.130993in}{0.877922in}}%
\pgfpathlineto{\pgfqpoint{1.140569in}{0.879026in}}%
\pgfpathlineto{\pgfqpoint{1.146315in}{0.880442in}}%
\pgfpathlineto{\pgfqpoint{1.152060in}{0.881287in}}%
\pgfpathlineto{\pgfqpoint{1.159721in}{0.883155in}}%
\pgfpathlineto{\pgfqpoint{1.199940in}{0.889064in}}%
\pgfpathlineto{\pgfqpoint{1.213346in}{0.889892in}}%
\pgfpathlineto{\pgfqpoint{1.217176in}{0.891013in}}%
\pgfpathlineto{\pgfqpoint{1.228667in}{0.892147in}}%
\pgfpathlineto{\pgfqpoint{1.266971in}{0.897178in}}%
\pgfpathlineto{\pgfqpoint{1.280377in}{0.898879in}}%
\pgfpathlineto{\pgfqpoint{1.286122in}{0.899993in}}%
\pgfpathlineto{\pgfqpoint{1.299528in}{0.902931in}}%
\pgfpathlineto{\pgfqpoint{1.312935in}{0.905879in}}%
\pgfpathlineto{\pgfqpoint{1.316765in}{0.906863in}}%
\pgfpathlineto{\pgfqpoint{1.322511in}{0.908110in}}%
\pgfpathlineto{\pgfqpoint{1.349323in}{0.911752in}}%
\pgfpathlineto{\pgfqpoint{1.355068in}{0.912981in}}%
\pgfpathlineto{\pgfqpoint{1.387626in}{0.917568in}}%
\pgfpathlineto{\pgfqpoint{1.454657in}{0.922597in}}%
\pgfpathlineto{\pgfqpoint{1.550416in}{0.933264in}}%
\pgfpathlineto{\pgfqpoint{1.565737in}{0.935599in}}%
\pgfpathlineto{\pgfqpoint{1.575313in}{0.936435in}}%
\pgfpathlineto{\pgfqpoint{1.579143in}{0.937425in}}%
\pgfpathlineto{\pgfqpoint{1.588719in}{0.938693in}}%
\pgfpathlineto{\pgfqpoint{1.607871in}{0.940878in}}%
\pgfpathlineto{\pgfqpoint{1.625108in}{0.944931in}}%
\pgfpathlineto{\pgfqpoint{1.646174in}{0.947189in}}%
\pgfpathlineto{\pgfqpoint{1.650005in}{0.948491in}}%
\pgfpathlineto{\pgfqpoint{1.653835in}{0.950144in}}%
\pgfpathlineto{\pgfqpoint{1.674902in}{0.953346in}}%
\pgfpathlineto{\pgfqpoint{1.686393in}{0.954526in}}%
\pgfpathlineto{\pgfqpoint{1.695969in}{0.955641in}}%
\pgfpathlineto{\pgfqpoint{1.703630in}{0.956799in}}%
\pgfpathlineto{\pgfqpoint{1.722781in}{0.959828in}}%
\pgfpathlineto{\pgfqpoint{1.740018in}{0.964303in}}%
\pgfpathlineto{\pgfqpoint{1.741933in}{0.964532in}}%
\pgfpathlineto{\pgfqpoint{1.743848in}{0.967466in}}%
\pgfpathlineto{\pgfqpoint{1.755339in}{0.969525in}}%
\pgfpathlineto{\pgfqpoint{1.774491in}{0.971453in}}%
\pgfpathlineto{\pgfqpoint{1.778321in}{0.973937in}}%
\pgfpathlineto{\pgfqpoint{1.780236in}{0.974197in}}%
\pgfpathlineto{\pgfqpoint{1.785982in}{0.977752in}}%
\pgfpathlineto{\pgfqpoint{1.797473in}{0.980035in}}%
\pgfpathlineto{\pgfqpoint{1.820455in}{0.983071in}}%
\pgfpathlineto{\pgfqpoint{1.824285in}{0.983797in}}%
\pgfpathlineto{\pgfqpoint{1.833861in}{0.984717in}}%
\pgfpathlineto{\pgfqpoint{1.837692in}{0.986598in}}%
\pgfpathlineto{\pgfqpoint{1.860674in}{0.989409in}}%
\pgfpathlineto{\pgfqpoint{1.868334in}{0.991345in}}%
\pgfpathlineto{\pgfqpoint{1.877910in}{0.992458in}}%
\pgfpathlineto{\pgfqpoint{1.900892in}{0.997358in}}%
\pgfpathlineto{\pgfqpoint{1.904723in}{0.998557in}}%
\pgfpathlineto{\pgfqpoint{1.918129in}{1.001810in}}%
\pgfpathlineto{\pgfqpoint{1.929620in}{1.003261in}}%
\pgfpathlineto{\pgfqpoint{1.941111in}{1.004191in}}%
\pgfpathlineto{\pgfqpoint{1.954517in}{1.006170in}}%
\pgfpathlineto{\pgfqpoint{1.960263in}{1.007225in}}%
\pgfpathlineto{\pgfqpoint{1.964093in}{1.009090in}}%
\pgfpathlineto{\pgfqpoint{1.992821in}{1.012236in}}%
\pgfpathlineto{\pgfqpoint{1.996651in}{1.013673in}}%
\pgfpathlineto{\pgfqpoint{2.021548in}{1.017232in}}%
\pgfpathlineto{\pgfqpoint{2.027294in}{1.018388in}}%
\pgfpathlineto{\pgfqpoint{2.036869in}{1.019759in}}%
\pgfpathlineto{\pgfqpoint{2.038785in}{1.022034in}}%
\pgfpathlineto{\pgfqpoint{2.140289in}{1.034822in}}%
\pgfpathlineto{\pgfqpoint{2.149865in}{1.036153in}}%
\pgfpathlineto{\pgfqpoint{2.155610in}{1.036768in}}%
\pgfpathlineto{\pgfqpoint{2.161356in}{1.038141in}}%
\pgfpathlineto{\pgfqpoint{2.172847in}{1.039984in}}%
\pgfpathlineto{\pgfqpoint{2.188168in}{1.042432in}}%
\pgfpathlineto{\pgfqpoint{2.193914in}{1.044353in}}%
\pgfpathlineto{\pgfqpoint{2.203489in}{1.045628in}}%
\pgfpathlineto{\pgfqpoint{2.222641in}{1.049509in}}%
\pgfpathlineto{\pgfqpoint{2.228387in}{1.050142in}}%
\pgfpathlineto{\pgfqpoint{2.297333in}{1.061340in}}%
\pgfpathlineto{\pgfqpoint{2.301163in}{1.062700in}}%
\pgfpathlineto{\pgfqpoint{2.303078in}{1.062824in}}%
\pgfpathlineto{\pgfqpoint{2.304993in}{1.064582in}}%
\pgfpathlineto{\pgfqpoint{2.314569in}{1.065431in}}%
\pgfpathlineto{\pgfqpoint{2.318400in}{1.066746in}}%
\pgfpathlineto{\pgfqpoint{2.324145in}{1.067012in}}%
\pgfpathlineto{\pgfqpoint{2.327975in}{1.068510in}}%
\pgfpathlineto{\pgfqpoint{2.339467in}{1.069371in}}%
\pgfpathlineto{\pgfqpoint{2.358618in}{1.073888in}}%
\pgfpathlineto{\pgfqpoint{2.387346in}{1.079643in}}%
\pgfpathlineto{\pgfqpoint{2.393091in}{1.080477in}}%
\pgfpathlineto{\pgfqpoint{2.398837in}{1.082285in}}%
\pgfpathlineto{\pgfqpoint{2.402667in}{1.083946in}}%
\pgfpathlineto{\pgfqpoint{2.406498in}{1.084690in}}%
\pgfpathlineto{\pgfqpoint{2.416073in}{1.086223in}}%
\pgfpathlineto{\pgfqpoint{2.431395in}{1.089847in}}%
\pgfpathlineto{\pgfqpoint{2.433310in}{1.089886in}}%
\pgfpathlineto{\pgfqpoint{2.435225in}{1.091309in}}%
\pgfpathlineto{\pgfqpoint{2.439055in}{1.091773in}}%
\pgfpathlineto{\pgfqpoint{2.442886in}{1.092888in}}%
\pgfpathlineto{\pgfqpoint{2.450546in}{1.093513in}}%
\pgfpathlineto{\pgfqpoint{2.456292in}{1.095811in}}%
\pgfpathlineto{\pgfqpoint{2.465868in}{1.099734in}}%
\pgfpathlineto{\pgfqpoint{2.483104in}{1.104601in}}%
\pgfpathlineto{\pgfqpoint{2.496511in}{1.107612in}}%
\pgfpathlineto{\pgfqpoint{2.500341in}{1.112103in}}%
\pgfpathlineto{\pgfqpoint{2.527153in}{1.117385in}}%
\pgfpathlineto{\pgfqpoint{2.546305in}{1.123202in}}%
\pgfpathlineto{\pgfqpoint{2.553966in}{1.124777in}}%
\pgfpathlineto{\pgfqpoint{2.559711in}{1.127149in}}%
\pgfpathlineto{\pgfqpoint{2.567372in}{1.128651in}}%
\pgfpathlineto{\pgfqpoint{2.569287in}{1.132174in}}%
\pgfpathlineto{\pgfqpoint{2.571202in}{1.132231in}}%
\pgfpathlineto{\pgfqpoint{2.575033in}{1.133807in}}%
\pgfpathlineto{\pgfqpoint{2.582693in}{1.135698in}}%
\pgfpathlineto{\pgfqpoint{2.584608in}{1.136519in}}%
\pgfpathlineto{\pgfqpoint{2.586524in}{1.139103in}}%
\pgfpathlineto{\pgfqpoint{2.598015in}{1.141933in}}%
\pgfpathlineto{\pgfqpoint{2.599930in}{1.145281in}}%
\pgfpathlineto{\pgfqpoint{2.609506in}{1.148967in}}%
\pgfpathlineto{\pgfqpoint{2.615251in}{1.151012in}}%
\pgfpathlineto{\pgfqpoint{2.632488in}{1.160730in}}%
\pgfpathlineto{\pgfqpoint{2.634403in}{1.162259in}}%
\pgfpathlineto{\pgfqpoint{2.636318in}{1.162292in}}%
\pgfpathlineto{\pgfqpoint{2.640148in}{1.164127in}}%
\pgfpathlineto{\pgfqpoint{2.642064in}{1.164768in}}%
\pgfpathlineto{\pgfqpoint{2.645894in}{1.171670in}}%
\pgfpathlineto{\pgfqpoint{2.649724in}{1.173699in}}%
\pgfpathlineto{\pgfqpoint{2.651639in}{1.177104in}}%
\pgfpathlineto{\pgfqpoint{2.655470in}{1.178221in}}%
\pgfpathlineto{\pgfqpoint{2.661215in}{1.181616in}}%
\pgfpathlineto{\pgfqpoint{2.665046in}{1.182529in}}%
\pgfpathlineto{\pgfqpoint{2.668876in}{1.183959in}}%
\pgfpathlineto{\pgfqpoint{2.672706in}{1.185299in}}%
\pgfpathlineto{\pgfqpoint{2.674622in}{1.189981in}}%
\pgfpathlineto{\pgfqpoint{2.678452in}{1.191422in}}%
\pgfpathlineto{\pgfqpoint{2.693773in}{1.196168in}}%
\pgfpathlineto{\pgfqpoint{2.697604in}{1.201777in}}%
\pgfpathlineto{\pgfqpoint{2.699519in}{1.203232in}}%
\pgfpathlineto{\pgfqpoint{2.709095in}{1.204549in}}%
\pgfpathlineto{\pgfqpoint{2.716755in}{1.206028in}}%
\pgfpathlineto{\pgfqpoint{2.720586in}{1.207480in}}%
\pgfpathlineto{\pgfqpoint{2.724416in}{1.207804in}}%
\pgfpathlineto{\pgfqpoint{2.737822in}{1.216964in}}%
\pgfpathlineto{\pgfqpoint{2.741653in}{1.217531in}}%
\pgfpathlineto{\pgfqpoint{2.743568in}{1.220838in}}%
\pgfpathlineto{\pgfqpoint{2.745483in}{1.221483in}}%
\pgfpathlineto{\pgfqpoint{2.747398in}{1.224490in}}%
\pgfpathlineto{\pgfqpoint{2.755059in}{1.226493in}}%
\pgfpathlineto{\pgfqpoint{2.756974in}{1.228797in}}%
\pgfpathlineto{\pgfqpoint{2.758889in}{1.233639in}}%
\pgfpathlineto{\pgfqpoint{2.762719in}{1.234828in}}%
\pgfpathlineto{\pgfqpoint{2.766550in}{1.236943in}}%
\pgfpathlineto{\pgfqpoint{2.770380in}{1.237936in}}%
\pgfpathlineto{\pgfqpoint{2.772295in}{1.239920in}}%
\pgfpathlineto{\pgfqpoint{2.774210in}{1.239988in}}%
\pgfpathlineto{\pgfqpoint{2.776126in}{1.242697in}}%
\pgfpathlineto{\pgfqpoint{2.778041in}{1.242864in}}%
\pgfpathlineto{\pgfqpoint{2.785701in}{1.248064in}}%
\pgfpathlineto{\pgfqpoint{2.797192in}{1.252870in}}%
\pgfpathlineto{\pgfqpoint{2.806768in}{1.254761in}}%
\pgfpathlineto{\pgfqpoint{2.808684in}{1.258943in}}%
\pgfpathlineto{\pgfqpoint{2.812514in}{1.260611in}}%
\pgfpathlineto{\pgfqpoint{2.820175in}{1.268660in}}%
\pgfpathlineto{\pgfqpoint{2.822090in}{1.275625in}}%
\pgfpathlineto{\pgfqpoint{2.831666in}{1.277604in}}%
\pgfpathlineto{\pgfqpoint{2.833581in}{1.282838in}}%
\pgfpathlineto{\pgfqpoint{2.839326in}{1.286615in}}%
\pgfpathlineto{\pgfqpoint{2.843157in}{1.288602in}}%
\pgfpathlineto{\pgfqpoint{2.845072in}{1.289591in}}%
\pgfpathlineto{\pgfqpoint{2.850817in}{1.297029in}}%
\pgfpathlineto{\pgfqpoint{2.854648in}{1.299721in}}%
\pgfpathlineto{\pgfqpoint{2.856563in}{1.300402in}}%
\pgfpathlineto{\pgfqpoint{2.862308in}{1.305160in}}%
\pgfpathlineto{\pgfqpoint{2.866139in}{1.306455in}}%
\pgfpathlineto{\pgfqpoint{2.871884in}{1.310414in}}%
\pgfpathlineto{\pgfqpoint{2.875715in}{1.314566in}}%
\pgfpathlineto{\pgfqpoint{2.877630in}{1.315000in}}%
\pgfpathlineto{\pgfqpoint{2.885290in}{1.327105in}}%
\pgfpathlineto{\pgfqpoint{2.889121in}{1.327696in}}%
\pgfpathlineto{\pgfqpoint{2.891036in}{1.332498in}}%
\pgfpathlineto{\pgfqpoint{2.892951in}{1.333169in}}%
\pgfpathlineto{\pgfqpoint{2.896781in}{1.335924in}}%
\pgfpathlineto{\pgfqpoint{2.898697in}{1.341167in}}%
\pgfpathlineto{\pgfqpoint{2.906357in}{1.344628in}}%
\pgfpathlineto{\pgfqpoint{2.908272in}{1.351538in}}%
\pgfpathlineto{\pgfqpoint{2.910188in}{1.352266in}}%
\pgfpathlineto{\pgfqpoint{2.915933in}{1.363609in}}%
\pgfpathlineto{\pgfqpoint{2.919763in}{1.365061in}}%
\pgfpathlineto{\pgfqpoint{2.921679in}{1.373283in}}%
\pgfpathlineto{\pgfqpoint{2.929339in}{1.380897in}}%
\pgfpathlineto{\pgfqpoint{2.931254in}{1.381325in}}%
\pgfpathlineto{\pgfqpoint{2.933170in}{1.387435in}}%
\pgfpathlineto{\pgfqpoint{2.935085in}{1.387793in}}%
\pgfpathlineto{\pgfqpoint{2.937000in}{1.393773in}}%
\pgfpathlineto{\pgfqpoint{2.938915in}{1.394062in}}%
\pgfpathlineto{\pgfqpoint{2.940830in}{1.401763in}}%
\pgfpathlineto{\pgfqpoint{2.942746in}{1.402862in}}%
\pgfpathlineto{\pgfqpoint{2.944661in}{1.419801in}}%
\pgfpathlineto{\pgfqpoint{2.948491in}{1.431082in}}%
\pgfpathlineto{\pgfqpoint{2.954237in}{1.443016in}}%
\pgfpathlineto{\pgfqpoint{2.956152in}{1.450338in}}%
\pgfpathlineto{\pgfqpoint{2.958067in}{1.471589in}}%
\pgfpathlineto{\pgfqpoint{2.963812in}{1.493713in}}%
\pgfpathlineto{\pgfqpoint{2.967643in}{1.504276in}}%
\pgfpathlineto{\pgfqpoint{2.971473in}{1.540108in}}%
\pgfpathlineto{\pgfqpoint{2.973388in}{1.541925in}}%
\pgfpathlineto{\pgfqpoint{2.975303in}{1.546393in}}%
\pgfpathlineto{\pgfqpoint{2.977219in}{1.558329in}}%
\pgfpathlineto{\pgfqpoint{2.979134in}{1.561688in}}%
\pgfpathlineto{\pgfqpoint{2.981049in}{1.569034in}}%
\pgfpathlineto{\pgfqpoint{2.982964in}{1.597494in}}%
\pgfpathlineto{\pgfqpoint{2.984879in}{1.602936in}}%
\pgfpathlineto{\pgfqpoint{2.986794in}{1.622419in}}%
\pgfpathlineto{\pgfqpoint{2.992540in}{1.641153in}}%
\pgfpathlineto{\pgfqpoint{2.996370in}{1.651511in}}%
\pgfpathlineto{\pgfqpoint{3.000201in}{1.657346in}}%
\pgfpathlineto{\pgfqpoint{3.004031in}{1.665559in}}%
\pgfpathlineto{\pgfqpoint{3.005946in}{1.668235in}}%
\pgfpathlineto{\pgfqpoint{3.009777in}{1.700837in}}%
\pgfpathlineto{\pgfqpoint{3.013607in}{1.713141in}}%
\pgfpathlineto{\pgfqpoint{3.017437in}{1.723441in}}%
\pgfpathlineto{\pgfqpoint{3.019352in}{1.728682in}}%
\pgfpathlineto{\pgfqpoint{3.021268in}{1.730441in}}%
\pgfpathlineto{\pgfqpoint{3.025098in}{1.753429in}}%
\pgfpathlineto{\pgfqpoint{3.027013in}{1.794188in}}%
\pgfpathlineto{\pgfqpoint{3.028928in}{1.794719in}}%
\pgfpathlineto{\pgfqpoint{3.030843in}{1.797123in}}%
\pgfpathlineto{\pgfqpoint{3.032759in}{1.826535in}}%
\pgfpathlineto{\pgfqpoint{3.032759in}{1.826535in}}%
\pgfusepath{stroke}%
\end{pgfscope}%
\begin{pgfscope}%
\pgfpathrectangle{\pgfqpoint{0.694334in}{0.523557in}}{\pgfqpoint{3.830343in}{1.302977in}}%
\pgfusepath{clip}%
\pgfsetrectcap%
\pgfsetroundjoin%
\pgfsetlinewidth{1.003750pt}%
\definecolor{currentstroke}{rgb}{0.564706,0.564706,1.000000}%
\pgfsetstrokecolor{currentstroke}%
\pgfsetdash{}{0pt}%
\pgfpathmoveto{\pgfqpoint{0.694334in}{0.703806in}}%
\pgfpathlineto{\pgfqpoint{0.696249in}{0.737312in}}%
\pgfpathlineto{\pgfqpoint{0.698165in}{0.740239in}}%
\pgfpathlineto{\pgfqpoint{0.700080in}{0.745465in}}%
\pgfpathlineto{\pgfqpoint{0.703910in}{0.746981in}}%
\pgfpathlineto{\pgfqpoint{0.705825in}{0.754666in}}%
\pgfpathlineto{\pgfqpoint{0.713486in}{0.757532in}}%
\pgfpathlineto{\pgfqpoint{0.719232in}{0.759770in}}%
\pgfpathlineto{\pgfqpoint{0.724977in}{0.761277in}}%
\pgfpathlineto{\pgfqpoint{0.728807in}{0.761602in}}%
\pgfpathlineto{\pgfqpoint{0.732638in}{0.762857in}}%
\pgfpathlineto{\pgfqpoint{0.736468in}{0.763733in}}%
\pgfpathlineto{\pgfqpoint{0.740298in}{0.766512in}}%
\pgfpathlineto{\pgfqpoint{0.744129in}{0.768190in}}%
\pgfpathlineto{\pgfqpoint{0.747959in}{0.768811in}}%
\pgfpathlineto{\pgfqpoint{0.757535in}{0.769539in}}%
\pgfpathlineto{\pgfqpoint{0.763280in}{0.771747in}}%
\pgfpathlineto{\pgfqpoint{0.769026in}{0.774631in}}%
\pgfpathlineto{\pgfqpoint{0.770941in}{0.774736in}}%
\pgfpathlineto{\pgfqpoint{0.772856in}{0.777195in}}%
\pgfpathlineto{\pgfqpoint{0.780517in}{0.779321in}}%
\pgfpathlineto{\pgfqpoint{0.790093in}{0.782434in}}%
\pgfpathlineto{\pgfqpoint{0.793923in}{0.784767in}}%
\pgfpathlineto{\pgfqpoint{0.795838in}{0.784811in}}%
\pgfpathlineto{\pgfqpoint{0.799669in}{0.788278in}}%
\pgfpathlineto{\pgfqpoint{0.805414in}{0.789452in}}%
\pgfpathlineto{\pgfqpoint{0.809245in}{0.791820in}}%
\pgfpathlineto{\pgfqpoint{0.816905in}{0.793698in}}%
\pgfpathlineto{\pgfqpoint{0.818820in}{0.795425in}}%
\pgfpathlineto{\pgfqpoint{0.824566in}{0.796047in}}%
\pgfpathlineto{\pgfqpoint{0.832227in}{0.799687in}}%
\pgfpathlineto{\pgfqpoint{0.841803in}{0.802601in}}%
\pgfpathlineto{\pgfqpoint{0.845633in}{0.803979in}}%
\pgfpathlineto{\pgfqpoint{0.847548in}{0.803993in}}%
\pgfpathlineto{\pgfqpoint{0.851378in}{0.807098in}}%
\pgfpathlineto{\pgfqpoint{0.862869in}{0.808678in}}%
\pgfpathlineto{\pgfqpoint{0.866700in}{0.810160in}}%
\pgfpathlineto{\pgfqpoint{0.874360in}{0.812265in}}%
\pgfpathlineto{\pgfqpoint{0.880106in}{0.813507in}}%
\pgfpathlineto{\pgfqpoint{0.882021in}{0.813875in}}%
\pgfpathlineto{\pgfqpoint{0.887767in}{0.818447in}}%
\pgfpathlineto{\pgfqpoint{0.891597in}{0.819748in}}%
\pgfpathlineto{\pgfqpoint{0.906918in}{0.828475in}}%
\pgfpathlineto{\pgfqpoint{0.908834in}{0.828597in}}%
\pgfpathlineto{\pgfqpoint{0.914579in}{0.834186in}}%
\pgfpathlineto{\pgfqpoint{0.920325in}{0.837295in}}%
\pgfpathlineto{\pgfqpoint{0.924155in}{0.839796in}}%
\pgfpathlineto{\pgfqpoint{0.927985in}{0.840826in}}%
\pgfpathlineto{\pgfqpoint{0.929900in}{0.842079in}}%
\pgfpathlineto{\pgfqpoint{0.935646in}{0.849780in}}%
\pgfpathlineto{\pgfqpoint{0.949052in}{0.854951in}}%
\pgfpathlineto{\pgfqpoint{0.954798in}{0.855791in}}%
\pgfpathlineto{\pgfqpoint{0.962458in}{0.861168in}}%
\pgfpathlineto{\pgfqpoint{0.966289in}{0.861769in}}%
\pgfpathlineto{\pgfqpoint{0.968204in}{0.864003in}}%
\pgfpathlineto{\pgfqpoint{0.973949in}{0.866159in}}%
\pgfpathlineto{\pgfqpoint{0.979695in}{0.867453in}}%
\pgfpathlineto{\pgfqpoint{0.985440in}{0.869988in}}%
\pgfpathlineto{\pgfqpoint{0.987356in}{0.873680in}}%
\pgfpathlineto{\pgfqpoint{0.989271in}{0.873771in}}%
\pgfpathlineto{\pgfqpoint{0.993101in}{0.875298in}}%
\pgfpathlineto{\pgfqpoint{1.023744in}{0.880894in}}%
\pgfpathlineto{\pgfqpoint{1.029489in}{0.883667in}}%
\pgfpathlineto{\pgfqpoint{1.033320in}{0.884552in}}%
\pgfpathlineto{\pgfqpoint{1.037150in}{0.885936in}}%
\pgfpathlineto{\pgfqpoint{1.040980in}{0.887013in}}%
\pgfpathlineto{\pgfqpoint{1.044811in}{0.887763in}}%
\pgfpathlineto{\pgfqpoint{1.050556in}{0.889467in}}%
\pgfpathlineto{\pgfqpoint{1.054387in}{0.889942in}}%
\pgfpathlineto{\pgfqpoint{1.058217in}{0.891196in}}%
\pgfpathlineto{\pgfqpoint{1.065878in}{0.892859in}}%
\pgfpathlineto{\pgfqpoint{1.088860in}{0.896070in}}%
\pgfpathlineto{\pgfqpoint{1.090775in}{0.897788in}}%
\pgfpathlineto{\pgfqpoint{1.094605in}{0.898895in}}%
\pgfpathlineto{\pgfqpoint{1.098435in}{0.900369in}}%
\pgfpathlineto{\pgfqpoint{1.106096in}{0.901823in}}%
\pgfpathlineto{\pgfqpoint{1.113757in}{0.903021in}}%
\pgfpathlineto{\pgfqpoint{1.153975in}{0.909369in}}%
\pgfpathlineto{\pgfqpoint{1.157806in}{0.911192in}}%
\pgfpathlineto{\pgfqpoint{1.167382in}{0.912204in}}%
\pgfpathlineto{\pgfqpoint{1.173127in}{0.912697in}}%
\pgfpathlineto{\pgfqpoint{1.178873in}{0.913300in}}%
\pgfpathlineto{\pgfqpoint{1.184618in}{0.914770in}}%
\pgfpathlineto{\pgfqpoint{1.205685in}{0.916886in}}%
\pgfpathlineto{\pgfqpoint{1.215261in}{0.918381in}}%
\pgfpathlineto{\pgfqpoint{1.284207in}{0.927502in}}%
\pgfpathlineto{\pgfqpoint{1.301444in}{0.929047in}}%
\pgfpathlineto{\pgfqpoint{1.307189in}{0.930186in}}%
\pgfpathlineto{\pgfqpoint{1.312935in}{0.931042in}}%
\pgfpathlineto{\pgfqpoint{1.374220in}{0.937970in}}%
\pgfpathlineto{\pgfqpoint{1.378050in}{0.938874in}}%
\pgfpathlineto{\pgfqpoint{1.393372in}{0.940231in}}%
\pgfpathlineto{\pgfqpoint{1.437421in}{0.945512in}}%
\pgfpathlineto{\pgfqpoint{1.445081in}{0.946576in}}%
\pgfpathlineto{\pgfqpoint{1.517858in}{0.952724in}}%
\pgfpathlineto{\pgfqpoint{1.521688in}{0.954409in}}%
\pgfpathlineto{\pgfqpoint{1.554246in}{0.957380in}}%
\pgfpathlineto{\pgfqpoint{1.582974in}{0.959969in}}%
\pgfpathlineto{\pgfqpoint{1.590635in}{0.960756in}}%
\pgfpathlineto{\pgfqpoint{1.623192in}{0.964336in}}%
\pgfpathlineto{\pgfqpoint{1.628938in}{0.965136in}}%
\pgfpathlineto{\pgfqpoint{1.636599in}{0.966438in}}%
\pgfpathlineto{\pgfqpoint{1.642344in}{0.967911in}}%
\pgfpathlineto{\pgfqpoint{1.686393in}{0.974032in}}%
\pgfpathlineto{\pgfqpoint{1.690223in}{0.975041in}}%
\pgfpathlineto{\pgfqpoint{1.734272in}{0.981467in}}%
\pgfpathlineto{\pgfqpoint{1.740018in}{0.983447in}}%
\pgfpathlineto{\pgfqpoint{1.745763in}{0.984686in}}%
\pgfpathlineto{\pgfqpoint{1.755339in}{0.986555in}}%
\pgfpathlineto{\pgfqpoint{1.761085in}{0.987270in}}%
\pgfpathlineto{\pgfqpoint{1.772576in}{0.988294in}}%
\pgfpathlineto{\pgfqpoint{1.785982in}{0.991567in}}%
\pgfpathlineto{\pgfqpoint{1.816625in}{0.997086in}}%
\pgfpathlineto{\pgfqpoint{1.820455in}{0.997821in}}%
\pgfpathlineto{\pgfqpoint{1.826201in}{0.998985in}}%
\pgfpathlineto{\pgfqpoint{1.921959in}{1.013104in}}%
\pgfpathlineto{\pgfqpoint{1.935365in}{1.014109in}}%
\pgfpathlineto{\pgfqpoint{1.990905in}{1.021609in}}%
\pgfpathlineto{\pgfqpoint{1.994736in}{1.023228in}}%
\pgfpathlineto{\pgfqpoint{2.004312in}{1.023816in}}%
\pgfpathlineto{\pgfqpoint{2.008142in}{1.025233in}}%
\pgfpathlineto{\pgfqpoint{2.023463in}{1.027506in}}%
\pgfpathlineto{\pgfqpoint{2.046445in}{1.029353in}}%
\pgfpathlineto{\pgfqpoint{2.071343in}{1.032582in}}%
\pgfpathlineto{\pgfqpoint{2.082834in}{1.034329in}}%
\pgfpathlineto{\pgfqpoint{2.142204in}{1.042093in}}%
\pgfpathlineto{\pgfqpoint{2.147949in}{1.043497in}}%
\pgfpathlineto{\pgfqpoint{2.163271in}{1.045316in}}%
\pgfpathlineto{\pgfqpoint{2.176677in}{1.047736in}}%
\pgfpathlineto{\pgfqpoint{2.182422in}{1.048601in}}%
\pgfpathlineto{\pgfqpoint{2.197744in}{1.049935in}}%
\pgfpathlineto{\pgfqpoint{2.214980in}{1.053074in}}%
\pgfpathlineto{\pgfqpoint{2.237962in}{1.055008in}}%
\pgfpathlineto{\pgfqpoint{2.262860in}{1.056430in}}%
\pgfpathlineto{\pgfqpoint{2.268605in}{1.057931in}}%
\pgfpathlineto{\pgfqpoint{2.289672in}{1.060641in}}%
\pgfpathlineto{\pgfqpoint{2.308824in}{1.064654in}}%
\pgfpathlineto{\pgfqpoint{2.396922in}{1.078320in}}%
\pgfpathlineto{\pgfqpoint{2.400752in}{1.079652in}}%
\pgfpathlineto{\pgfqpoint{2.417989in}{1.081793in}}%
\pgfpathlineto{\pgfqpoint{2.419904in}{1.084435in}}%
\pgfpathlineto{\pgfqpoint{2.439055in}{1.089532in}}%
\pgfpathlineto{\pgfqpoint{2.442886in}{1.091604in}}%
\pgfpathlineto{\pgfqpoint{2.446716in}{1.092017in}}%
\pgfpathlineto{\pgfqpoint{2.448631in}{1.094610in}}%
\pgfpathlineto{\pgfqpoint{2.454377in}{1.095374in}}%
\pgfpathlineto{\pgfqpoint{2.469698in}{1.105210in}}%
\pgfpathlineto{\pgfqpoint{2.523323in}{1.120572in}}%
\pgfpathlineto{\pgfqpoint{2.540560in}{1.129362in}}%
\pgfpathlineto{\pgfqpoint{2.544390in}{1.130803in}}%
\pgfpathlineto{\pgfqpoint{2.546305in}{1.131678in}}%
\pgfpathlineto{\pgfqpoint{2.548220in}{1.135472in}}%
\pgfpathlineto{\pgfqpoint{2.550135in}{1.135501in}}%
\pgfpathlineto{\pgfqpoint{2.552051in}{1.138245in}}%
\pgfpathlineto{\pgfqpoint{2.569287in}{1.140683in}}%
\pgfpathlineto{\pgfqpoint{2.584608in}{1.145341in}}%
\pgfpathlineto{\pgfqpoint{2.599930in}{1.150057in}}%
\pgfpathlineto{\pgfqpoint{2.603760in}{1.153358in}}%
\pgfpathlineto{\pgfqpoint{2.613336in}{1.154237in}}%
\pgfpathlineto{\pgfqpoint{2.620997in}{1.157395in}}%
\pgfpathlineto{\pgfqpoint{2.624827in}{1.158424in}}%
\pgfpathlineto{\pgfqpoint{2.628657in}{1.161592in}}%
\pgfpathlineto{\pgfqpoint{2.634403in}{1.162027in}}%
\pgfpathlineto{\pgfqpoint{2.638233in}{1.164186in}}%
\pgfpathlineto{\pgfqpoint{2.643979in}{1.165551in}}%
\pgfpathlineto{\pgfqpoint{2.647809in}{1.168084in}}%
\pgfpathlineto{\pgfqpoint{2.655470in}{1.170708in}}%
\pgfpathlineto{\pgfqpoint{2.659300in}{1.174661in}}%
\pgfpathlineto{\pgfqpoint{2.663130in}{1.176342in}}%
\pgfpathlineto{\pgfqpoint{2.672706in}{1.178980in}}%
\pgfpathlineto{\pgfqpoint{2.674622in}{1.184777in}}%
\pgfpathlineto{\pgfqpoint{2.680367in}{1.188055in}}%
\pgfpathlineto{\pgfqpoint{2.684197in}{1.189445in}}%
\pgfpathlineto{\pgfqpoint{2.688028in}{1.190641in}}%
\pgfpathlineto{\pgfqpoint{2.693773in}{1.194410in}}%
\pgfpathlineto{\pgfqpoint{2.705264in}{1.206707in}}%
\pgfpathlineto{\pgfqpoint{2.707179in}{1.207159in}}%
\pgfpathlineto{\pgfqpoint{2.709095in}{1.208957in}}%
\pgfpathlineto{\pgfqpoint{2.718670in}{1.210562in}}%
\pgfpathlineto{\pgfqpoint{2.720586in}{1.212025in}}%
\pgfpathlineto{\pgfqpoint{2.722501in}{1.217560in}}%
\pgfpathlineto{\pgfqpoint{2.733992in}{1.219462in}}%
\pgfpathlineto{\pgfqpoint{2.753144in}{1.226461in}}%
\pgfpathlineto{\pgfqpoint{2.756974in}{1.229147in}}%
\pgfpathlineto{\pgfqpoint{2.766550in}{1.231501in}}%
\pgfpathlineto{\pgfqpoint{2.768465in}{1.232466in}}%
\pgfpathlineto{\pgfqpoint{2.772295in}{1.235922in}}%
\pgfpathlineto{\pgfqpoint{2.779956in}{1.237844in}}%
\pgfpathlineto{\pgfqpoint{2.785701in}{1.243512in}}%
\pgfpathlineto{\pgfqpoint{2.789532in}{1.248620in}}%
\pgfpathlineto{\pgfqpoint{2.795277in}{1.250336in}}%
\pgfpathlineto{\pgfqpoint{2.797192in}{1.254998in}}%
\pgfpathlineto{\pgfqpoint{2.799108in}{1.255714in}}%
\pgfpathlineto{\pgfqpoint{2.808684in}{1.266275in}}%
\pgfpathlineto{\pgfqpoint{2.814429in}{1.267728in}}%
\pgfpathlineto{\pgfqpoint{2.816344in}{1.270971in}}%
\pgfpathlineto{\pgfqpoint{2.818259in}{1.271617in}}%
\pgfpathlineto{\pgfqpoint{2.820175in}{1.274300in}}%
\pgfpathlineto{\pgfqpoint{2.822090in}{1.274873in}}%
\pgfpathlineto{\pgfqpoint{2.846987in}{1.297343in}}%
\pgfpathlineto{\pgfqpoint{2.848902in}{1.298045in}}%
\pgfpathlineto{\pgfqpoint{2.850817in}{1.301241in}}%
\pgfpathlineto{\pgfqpoint{2.852732in}{1.301421in}}%
\pgfpathlineto{\pgfqpoint{2.858478in}{1.306625in}}%
\pgfpathlineto{\pgfqpoint{2.860393in}{1.306735in}}%
\pgfpathlineto{\pgfqpoint{2.866139in}{1.314238in}}%
\pgfpathlineto{\pgfqpoint{2.868054in}{1.314547in}}%
\pgfpathlineto{\pgfqpoint{2.869969in}{1.316651in}}%
\pgfpathlineto{\pgfqpoint{2.879545in}{1.318419in}}%
\pgfpathlineto{\pgfqpoint{2.881460in}{1.323457in}}%
\pgfpathlineto{\pgfqpoint{2.883375in}{1.325154in}}%
\pgfpathlineto{\pgfqpoint{2.891036in}{1.338332in}}%
\pgfpathlineto{\pgfqpoint{2.892951in}{1.339528in}}%
\pgfpathlineto{\pgfqpoint{2.896781in}{1.348804in}}%
\pgfpathlineto{\pgfqpoint{2.900612in}{1.349348in}}%
\pgfpathlineto{\pgfqpoint{2.904442in}{1.352064in}}%
\pgfpathlineto{\pgfqpoint{2.906357in}{1.352728in}}%
\pgfpathlineto{\pgfqpoint{2.910188in}{1.362177in}}%
\pgfpathlineto{\pgfqpoint{2.912103in}{1.362651in}}%
\pgfpathlineto{\pgfqpoint{2.915933in}{1.368245in}}%
\pgfpathlineto{\pgfqpoint{2.917848in}{1.368411in}}%
\pgfpathlineto{\pgfqpoint{2.919763in}{1.372686in}}%
\pgfpathlineto{\pgfqpoint{2.921679in}{1.383317in}}%
\pgfpathlineto{\pgfqpoint{2.923594in}{1.383544in}}%
\pgfpathlineto{\pgfqpoint{2.925509in}{1.385876in}}%
\pgfpathlineto{\pgfqpoint{2.927424in}{1.390546in}}%
\pgfpathlineto{\pgfqpoint{2.929339in}{1.392253in}}%
\pgfpathlineto{\pgfqpoint{2.931254in}{1.400743in}}%
\pgfpathlineto{\pgfqpoint{2.935085in}{1.402566in}}%
\pgfpathlineto{\pgfqpoint{2.937000in}{1.426236in}}%
\pgfpathlineto{\pgfqpoint{2.938915in}{1.434149in}}%
\pgfpathlineto{\pgfqpoint{2.940830in}{1.449912in}}%
\pgfpathlineto{\pgfqpoint{2.942746in}{1.453522in}}%
\pgfpathlineto{\pgfqpoint{2.944661in}{1.453739in}}%
\pgfpathlineto{\pgfqpoint{2.946576in}{1.460477in}}%
\pgfpathlineto{\pgfqpoint{2.948491in}{1.463050in}}%
\pgfpathlineto{\pgfqpoint{2.950406in}{1.477048in}}%
\pgfpathlineto{\pgfqpoint{2.952321in}{1.479890in}}%
\pgfpathlineto{\pgfqpoint{2.954237in}{1.500516in}}%
\pgfpathlineto{\pgfqpoint{2.956152in}{1.502706in}}%
\pgfpathlineto{\pgfqpoint{2.958067in}{1.514112in}}%
\pgfpathlineto{\pgfqpoint{2.959982in}{1.514813in}}%
\pgfpathlineto{\pgfqpoint{2.963812in}{1.551074in}}%
\pgfpathlineto{\pgfqpoint{2.967643in}{1.554889in}}%
\pgfpathlineto{\pgfqpoint{2.977219in}{1.604232in}}%
\pgfpathlineto{\pgfqpoint{2.979134in}{1.604287in}}%
\pgfpathlineto{\pgfqpoint{2.981049in}{1.609002in}}%
\pgfpathlineto{\pgfqpoint{2.982964in}{1.610477in}}%
\pgfpathlineto{\pgfqpoint{2.984879in}{1.615768in}}%
\pgfpathlineto{\pgfqpoint{2.986794in}{1.616285in}}%
\pgfpathlineto{\pgfqpoint{2.988710in}{1.627537in}}%
\pgfpathlineto{\pgfqpoint{2.990625in}{1.627787in}}%
\pgfpathlineto{\pgfqpoint{2.992540in}{1.636913in}}%
\pgfpathlineto{\pgfqpoint{2.994455in}{1.653743in}}%
\pgfpathlineto{\pgfqpoint{2.996370in}{1.655288in}}%
\pgfpathlineto{\pgfqpoint{2.998285in}{1.671393in}}%
\pgfpathlineto{\pgfqpoint{3.000201in}{1.675089in}}%
\pgfpathlineto{\pgfqpoint{3.002116in}{1.699109in}}%
\pgfpathlineto{\pgfqpoint{3.004031in}{1.703650in}}%
\pgfpathlineto{\pgfqpoint{3.007861in}{1.726928in}}%
\pgfpathlineto{\pgfqpoint{3.009777in}{1.727205in}}%
\pgfpathlineto{\pgfqpoint{3.011692in}{1.738607in}}%
\pgfpathlineto{\pgfqpoint{3.017437in}{1.747913in}}%
\pgfpathlineto{\pgfqpoint{3.021268in}{1.758869in}}%
\pgfpathlineto{\pgfqpoint{3.025098in}{1.766004in}}%
\pgfpathlineto{\pgfqpoint{3.027013in}{1.786911in}}%
\pgfpathlineto{\pgfqpoint{3.032759in}{1.799776in}}%
\pgfpathlineto{\pgfqpoint{3.034674in}{1.801144in}}%
\pgfpathlineto{\pgfqpoint{3.036589in}{1.826535in}}%
\pgfpathlineto{\pgfqpoint{3.036589in}{1.826535in}}%
\pgfusepath{stroke}%
\end{pgfscope}%
\begin{pgfscope}%
\pgfpathrectangle{\pgfqpoint{0.694334in}{0.523557in}}{\pgfqpoint{3.830343in}{1.302977in}}%
\pgfusepath{clip}%
\pgfsetbuttcap%
\pgfsetroundjoin%
\pgfsetlinewidth{1.003750pt}%
\definecolor{currentstroke}{rgb}{0.000000,0.000000,0.000000}%
\pgfsetstrokecolor{currentstroke}%
\pgfsetdash{{1.000000pt}{1.650000pt}}{0.000000pt}%
\pgfpathmoveto{\pgfqpoint{0.694334in}{0.567141in}}%
\pgfpathlineto{\pgfqpoint{0.696249in}{0.582562in}}%
\pgfpathlineto{\pgfqpoint{0.698165in}{0.582755in}}%
\pgfpathlineto{\pgfqpoint{0.701995in}{0.599148in}}%
\pgfpathlineto{\pgfqpoint{0.703910in}{0.606735in}}%
\pgfpathlineto{\pgfqpoint{0.711571in}{0.615604in}}%
\pgfpathlineto{\pgfqpoint{0.713486in}{0.616358in}}%
\pgfpathlineto{\pgfqpoint{0.715401in}{0.627746in}}%
\pgfpathlineto{\pgfqpoint{0.719232in}{0.630247in}}%
\pgfpathlineto{\pgfqpoint{0.721147in}{0.632363in}}%
\pgfpathlineto{\pgfqpoint{0.724977in}{0.633447in}}%
\pgfpathlineto{\pgfqpoint{0.726892in}{0.635337in}}%
\pgfpathlineto{\pgfqpoint{0.732638in}{0.646877in}}%
\pgfpathlineto{\pgfqpoint{0.736468in}{0.649507in}}%
\pgfpathlineto{\pgfqpoint{0.738383in}{0.655445in}}%
\pgfpathlineto{\pgfqpoint{0.740298in}{0.655930in}}%
\pgfpathlineto{\pgfqpoint{0.747959in}{0.671221in}}%
\pgfpathlineto{\pgfqpoint{0.749874in}{0.675620in}}%
\pgfpathlineto{\pgfqpoint{0.751789in}{0.675686in}}%
\pgfpathlineto{\pgfqpoint{0.753705in}{0.678423in}}%
\pgfpathlineto{\pgfqpoint{0.755620in}{0.683416in}}%
\pgfpathlineto{\pgfqpoint{0.759450in}{0.685036in}}%
\pgfpathlineto{\pgfqpoint{0.761365in}{0.690456in}}%
\pgfpathlineto{\pgfqpoint{0.770941in}{0.697917in}}%
\pgfpathlineto{\pgfqpoint{0.772856in}{0.698286in}}%
\pgfpathlineto{\pgfqpoint{0.778602in}{0.704153in}}%
\pgfpathlineto{\pgfqpoint{0.793923in}{0.708539in}}%
\pgfpathlineto{\pgfqpoint{0.801584in}{0.709532in}}%
\pgfpathlineto{\pgfqpoint{0.803499in}{0.711186in}}%
\pgfpathlineto{\pgfqpoint{0.811160in}{0.712162in}}%
\pgfpathlineto{\pgfqpoint{0.814990in}{0.713108in}}%
\pgfpathlineto{\pgfqpoint{0.837972in}{0.715177in}}%
\pgfpathlineto{\pgfqpoint{0.853294in}{0.716555in}}%
\pgfpathlineto{\pgfqpoint{0.859039in}{0.717527in}}%
\pgfpathlineto{\pgfqpoint{0.864785in}{0.718514in}}%
\pgfpathlineto{\pgfqpoint{0.889682in}{0.720470in}}%
\pgfpathlineto{\pgfqpoint{0.903088in}{0.721801in}}%
\pgfpathlineto{\pgfqpoint{0.920325in}{0.724066in}}%
\pgfpathlineto{\pgfqpoint{0.947137in}{0.725055in}}%
\pgfpathlineto{\pgfqpoint{1.010338in}{0.726970in}}%
\pgfpathlineto{\pgfqpoint{1.027574in}{0.727996in}}%
\pgfpathlineto{\pgfqpoint{1.067793in}{0.730388in}}%
\pgfpathlineto{\pgfqpoint{1.075453in}{0.730995in}}%
\pgfpathlineto{\pgfqpoint{1.299528in}{0.744595in}}%
\pgfpathlineto{\pgfqpoint{1.303359in}{0.745473in}}%
\pgfpathlineto{\pgfqpoint{1.328256in}{0.746903in}}%
\pgfpathlineto{\pgfqpoint{1.370390in}{0.750858in}}%
\pgfpathlineto{\pgfqpoint{1.385711in}{0.751900in}}%
\pgfpathlineto{\pgfqpoint{1.404863in}{0.753175in}}%
\pgfpathlineto{\pgfqpoint{1.416354in}{0.754161in}}%
\pgfpathlineto{\pgfqpoint{1.422099in}{0.755207in}}%
\pgfpathlineto{\pgfqpoint{1.433590in}{0.756474in}}%
\pgfpathlineto{\pgfqpoint{1.437421in}{0.757393in}}%
\pgfpathlineto{\pgfqpoint{1.452742in}{0.758832in}}%
\pgfpathlineto{\pgfqpoint{1.491046in}{0.765101in}}%
\pgfpathlineto{\pgfqpoint{1.498706in}{0.765845in}}%
\pgfpathlineto{\pgfqpoint{1.508282in}{0.766470in}}%
\pgfpathlineto{\pgfqpoint{1.519773in}{0.771921in}}%
\pgfpathlineto{\pgfqpoint{1.523604in}{0.772598in}}%
\pgfpathlineto{\pgfqpoint{1.531264in}{0.777376in}}%
\pgfpathlineto{\pgfqpoint{1.535095in}{0.778714in}}%
\pgfpathlineto{\pgfqpoint{1.537010in}{0.783849in}}%
\pgfpathlineto{\pgfqpoint{1.538925in}{0.784911in}}%
\pgfpathlineto{\pgfqpoint{1.540840in}{0.788811in}}%
\pgfpathlineto{\pgfqpoint{1.552331in}{0.791007in}}%
\pgfpathlineto{\pgfqpoint{1.554246in}{0.791602in}}%
\pgfpathlineto{\pgfqpoint{1.561907in}{0.798592in}}%
\pgfpathlineto{\pgfqpoint{1.569568in}{0.799587in}}%
\pgfpathlineto{\pgfqpoint{1.573398in}{0.802172in}}%
\pgfpathlineto{\pgfqpoint{1.577228in}{0.802534in}}%
\pgfpathlineto{\pgfqpoint{1.582974in}{0.806461in}}%
\pgfpathlineto{\pgfqpoint{1.590635in}{0.808069in}}%
\pgfpathlineto{\pgfqpoint{1.607871in}{0.810257in}}%
\pgfpathlineto{\pgfqpoint{1.613617in}{0.811082in}}%
\pgfpathlineto{\pgfqpoint{1.623192in}{0.811795in}}%
\pgfpathlineto{\pgfqpoint{1.636599in}{0.816468in}}%
\pgfpathlineto{\pgfqpoint{1.642344in}{0.816816in}}%
\pgfpathlineto{\pgfqpoint{1.646174in}{0.817868in}}%
\pgfpathlineto{\pgfqpoint{1.703630in}{0.828093in}}%
\pgfpathlineto{\pgfqpoint{1.715121in}{0.829443in}}%
\pgfpathlineto{\pgfqpoint{1.722781in}{0.831705in}}%
\pgfpathlineto{\pgfqpoint{1.730442in}{0.832697in}}%
\pgfpathlineto{\pgfqpoint{1.759170in}{0.835613in}}%
\pgfpathlineto{\pgfqpoint{1.763000in}{0.836460in}}%
\pgfpathlineto{\pgfqpoint{1.778321in}{0.838136in}}%
\pgfpathlineto{\pgfqpoint{1.782152in}{0.839452in}}%
\pgfpathlineto{\pgfqpoint{1.803219in}{0.843519in}}%
\pgfpathlineto{\pgfqpoint{1.818540in}{0.845975in}}%
\pgfpathlineto{\pgfqpoint{1.822370in}{0.846987in}}%
\pgfpathlineto{\pgfqpoint{1.826201in}{0.847835in}}%
\pgfpathlineto{\pgfqpoint{1.831946in}{0.849615in}}%
\pgfpathlineto{\pgfqpoint{1.835776in}{0.851432in}}%
\pgfpathlineto{\pgfqpoint{1.839607in}{0.852136in}}%
\pgfpathlineto{\pgfqpoint{1.843437in}{0.854187in}}%
\pgfpathlineto{\pgfqpoint{1.847267in}{0.855217in}}%
\pgfpathlineto{\pgfqpoint{1.864504in}{0.859580in}}%
\pgfpathlineto{\pgfqpoint{1.879825in}{0.860892in}}%
\pgfpathlineto{\pgfqpoint{1.891316in}{0.863940in}}%
\pgfpathlineto{\pgfqpoint{1.908553in}{0.865804in}}%
\pgfpathlineto{\pgfqpoint{1.944941in}{0.870873in}}%
\pgfpathlineto{\pgfqpoint{1.950687in}{0.872051in}}%
\pgfpathlineto{\pgfqpoint{2.002396in}{0.880407in}}%
\pgfpathlineto{\pgfqpoint{2.006227in}{0.881360in}}%
\pgfpathlineto{\pgfqpoint{2.013887in}{0.882640in}}%
\pgfpathlineto{\pgfqpoint{2.019633in}{0.882946in}}%
\pgfpathlineto{\pgfqpoint{2.023463in}{0.885084in}}%
\pgfpathlineto{\pgfqpoint{2.027294in}{0.885827in}}%
\pgfpathlineto{\pgfqpoint{2.031124in}{0.887468in}}%
\pgfpathlineto{\pgfqpoint{2.046445in}{0.889403in}}%
\pgfpathlineto{\pgfqpoint{2.113476in}{0.897402in}}%
\pgfpathlineto{\pgfqpoint{2.117307in}{0.898536in}}%
\pgfpathlineto{\pgfqpoint{2.130713in}{0.900570in}}%
\pgfpathlineto{\pgfqpoint{2.134543in}{0.901537in}}%
\pgfpathlineto{\pgfqpoint{2.142204in}{0.902500in}}%
\pgfpathlineto{\pgfqpoint{2.147949in}{0.903786in}}%
\pgfpathlineto{\pgfqpoint{2.193914in}{0.910651in}}%
\pgfpathlineto{\pgfqpoint{2.207320in}{0.911405in}}%
\pgfpathlineto{\pgfqpoint{2.213065in}{0.913407in}}%
\pgfpathlineto{\pgfqpoint{2.218811in}{0.914494in}}%
\pgfpathlineto{\pgfqpoint{2.222641in}{0.915106in}}%
\pgfpathlineto{\pgfqpoint{2.230302in}{0.917520in}}%
\pgfpathlineto{\pgfqpoint{2.239878in}{0.918703in}}%
\pgfpathlineto{\pgfqpoint{2.243708in}{0.920150in}}%
\pgfpathlineto{\pgfqpoint{2.264775in}{0.923191in}}%
\pgfpathlineto{\pgfqpoint{2.326060in}{0.934156in}}%
\pgfpathlineto{\pgfqpoint{2.329891in}{0.935857in}}%
\pgfpathlineto{\pgfqpoint{2.343297in}{0.937293in}}%
\pgfpathlineto{\pgfqpoint{2.350958in}{0.938427in}}%
\pgfpathlineto{\pgfqpoint{2.366279in}{0.939420in}}%
\pgfpathlineto{\pgfqpoint{2.372024in}{0.941202in}}%
\pgfpathlineto{\pgfqpoint{2.383515in}{0.942902in}}%
\pgfpathlineto{\pgfqpoint{2.410328in}{0.945045in}}%
\pgfpathlineto{\pgfqpoint{2.423734in}{0.945656in}}%
\pgfpathlineto{\pgfqpoint{2.429480in}{0.946929in}}%
\pgfpathlineto{\pgfqpoint{2.444801in}{0.947880in}}%
\pgfpathlineto{\pgfqpoint{2.467783in}{0.950815in}}%
\pgfpathlineto{\pgfqpoint{2.479274in}{0.954592in}}%
\pgfpathlineto{\pgfqpoint{2.511832in}{0.964688in}}%
\pgfpathlineto{\pgfqpoint{2.517577in}{0.970839in}}%
\pgfpathlineto{\pgfqpoint{2.519493in}{0.970888in}}%
\pgfpathlineto{\pgfqpoint{2.521408in}{0.972769in}}%
\pgfpathlineto{\pgfqpoint{2.527153in}{0.973870in}}%
\pgfpathlineto{\pgfqpoint{2.530984in}{0.975766in}}%
\pgfpathlineto{\pgfqpoint{2.534814in}{0.980023in}}%
\pgfpathlineto{\pgfqpoint{2.548220in}{0.982244in}}%
\pgfpathlineto{\pgfqpoint{2.555881in}{0.987072in}}%
\pgfpathlineto{\pgfqpoint{2.559711in}{0.987375in}}%
\pgfpathlineto{\pgfqpoint{2.561626in}{0.989794in}}%
\pgfpathlineto{\pgfqpoint{2.565457in}{0.990829in}}%
\pgfpathlineto{\pgfqpoint{2.575033in}{0.992834in}}%
\pgfpathlineto{\pgfqpoint{2.580778in}{0.995368in}}%
\pgfpathlineto{\pgfqpoint{2.584608in}{0.996058in}}%
\pgfpathlineto{\pgfqpoint{2.586524in}{0.999027in}}%
\pgfpathlineto{\pgfqpoint{2.592269in}{1.000545in}}%
\pgfpathlineto{\pgfqpoint{2.601845in}{1.001857in}}%
\pgfpathlineto{\pgfqpoint{2.607591in}{1.004198in}}%
\pgfpathlineto{\pgfqpoint{2.615251in}{1.006342in}}%
\pgfpathlineto{\pgfqpoint{2.617166in}{1.007769in}}%
\pgfpathlineto{\pgfqpoint{2.624827in}{1.008569in}}%
\pgfpathlineto{\pgfqpoint{2.647809in}{1.015743in}}%
\pgfpathlineto{\pgfqpoint{2.653555in}{1.017738in}}%
\pgfpathlineto{\pgfqpoint{2.661215in}{1.018503in}}%
\pgfpathlineto{\pgfqpoint{2.666961in}{1.021593in}}%
\pgfpathlineto{\pgfqpoint{2.668876in}{1.021639in}}%
\pgfpathlineto{\pgfqpoint{2.676537in}{1.029089in}}%
\pgfpathlineto{\pgfqpoint{2.682282in}{1.029692in}}%
\pgfpathlineto{\pgfqpoint{2.684197in}{1.032096in}}%
\pgfpathlineto{\pgfqpoint{2.688028in}{1.032551in}}%
\pgfpathlineto{\pgfqpoint{2.689943in}{1.036236in}}%
\pgfpathlineto{\pgfqpoint{2.693773in}{1.037453in}}%
\pgfpathlineto{\pgfqpoint{2.699519in}{1.041072in}}%
\pgfpathlineto{\pgfqpoint{2.705264in}{1.041844in}}%
\pgfpathlineto{\pgfqpoint{2.711010in}{1.046404in}}%
\pgfpathlineto{\pgfqpoint{2.712925in}{1.046713in}}%
\pgfpathlineto{\pgfqpoint{2.714840in}{1.048649in}}%
\pgfpathlineto{\pgfqpoint{2.718670in}{1.048973in}}%
\pgfpathlineto{\pgfqpoint{2.722501in}{1.051883in}}%
\pgfpathlineto{\pgfqpoint{2.730161in}{1.053131in}}%
\pgfpathlineto{\pgfqpoint{2.733992in}{1.058431in}}%
\pgfpathlineto{\pgfqpoint{2.737822in}{1.060002in}}%
\pgfpathlineto{\pgfqpoint{2.739737in}{1.062244in}}%
\pgfpathlineto{\pgfqpoint{2.745483in}{1.063362in}}%
\pgfpathlineto{\pgfqpoint{2.749313in}{1.066143in}}%
\pgfpathlineto{\pgfqpoint{2.756974in}{1.068559in}}%
\pgfpathlineto{\pgfqpoint{2.758889in}{1.070863in}}%
\pgfpathlineto{\pgfqpoint{2.760804in}{1.071197in}}%
\pgfpathlineto{\pgfqpoint{2.762719in}{1.073376in}}%
\pgfpathlineto{\pgfqpoint{2.764635in}{1.073698in}}%
\pgfpathlineto{\pgfqpoint{2.766550in}{1.076698in}}%
\pgfpathlineto{\pgfqpoint{2.781871in}{1.081531in}}%
\pgfpathlineto{\pgfqpoint{2.789532in}{1.091942in}}%
\pgfpathlineto{\pgfqpoint{2.802938in}{1.097733in}}%
\pgfpathlineto{\pgfqpoint{2.816344in}{1.113734in}}%
\pgfpathlineto{\pgfqpoint{2.820175in}{1.115461in}}%
\pgfpathlineto{\pgfqpoint{2.822090in}{1.118249in}}%
\pgfpathlineto{\pgfqpoint{2.824005in}{1.118961in}}%
\pgfpathlineto{\pgfqpoint{2.825920in}{1.121740in}}%
\pgfpathlineto{\pgfqpoint{2.827835in}{1.121876in}}%
\pgfpathlineto{\pgfqpoint{2.831666in}{1.128837in}}%
\pgfpathlineto{\pgfqpoint{2.841241in}{1.132505in}}%
\pgfpathlineto{\pgfqpoint{2.843157in}{1.135682in}}%
\pgfpathlineto{\pgfqpoint{2.845072in}{1.142341in}}%
\pgfpathlineto{\pgfqpoint{2.846987in}{1.143000in}}%
\pgfpathlineto{\pgfqpoint{2.848902in}{1.150761in}}%
\pgfpathlineto{\pgfqpoint{2.850817in}{1.152351in}}%
\pgfpathlineto{\pgfqpoint{2.852732in}{1.162566in}}%
\pgfpathlineto{\pgfqpoint{2.858478in}{1.165194in}}%
\pgfpathlineto{\pgfqpoint{2.860393in}{1.166312in}}%
\pgfpathlineto{\pgfqpoint{2.862308in}{1.176447in}}%
\pgfpathlineto{\pgfqpoint{2.875715in}{1.189905in}}%
\pgfpathlineto{\pgfqpoint{2.877630in}{1.200352in}}%
\pgfpathlineto{\pgfqpoint{2.879545in}{1.200960in}}%
\pgfpathlineto{\pgfqpoint{2.883375in}{1.214235in}}%
\pgfpathlineto{\pgfqpoint{2.885290in}{1.223561in}}%
\pgfpathlineto{\pgfqpoint{2.887206in}{1.223605in}}%
\pgfpathlineto{\pgfqpoint{2.892951in}{1.233278in}}%
\pgfpathlineto{\pgfqpoint{2.896781in}{1.235146in}}%
\pgfpathlineto{\pgfqpoint{2.898697in}{1.235612in}}%
\pgfpathlineto{\pgfqpoint{2.902527in}{1.240602in}}%
\pgfpathlineto{\pgfqpoint{2.908272in}{1.241978in}}%
\pgfpathlineto{\pgfqpoint{2.910188in}{1.245462in}}%
\pgfpathlineto{\pgfqpoint{2.914018in}{1.258053in}}%
\pgfpathlineto{\pgfqpoint{2.915933in}{1.258087in}}%
\pgfpathlineto{\pgfqpoint{2.917848in}{1.259950in}}%
\pgfpathlineto{\pgfqpoint{2.919763in}{1.265087in}}%
\pgfpathlineto{\pgfqpoint{2.929339in}{1.266600in}}%
\pgfpathlineto{\pgfqpoint{2.931254in}{1.269136in}}%
\pgfpathlineto{\pgfqpoint{2.933170in}{1.274322in}}%
\pgfpathlineto{\pgfqpoint{2.935085in}{1.275324in}}%
\pgfpathlineto{\pgfqpoint{2.937000in}{1.286711in}}%
\pgfpathlineto{\pgfqpoint{2.942746in}{1.296020in}}%
\pgfpathlineto{\pgfqpoint{2.950406in}{1.339754in}}%
\pgfpathlineto{\pgfqpoint{2.954237in}{1.353078in}}%
\pgfpathlineto{\pgfqpoint{2.956152in}{1.366320in}}%
\pgfpathlineto{\pgfqpoint{2.958067in}{1.367354in}}%
\pgfpathlineto{\pgfqpoint{2.961897in}{1.375775in}}%
\pgfpathlineto{\pgfqpoint{2.963812in}{1.401096in}}%
\pgfpathlineto{\pgfqpoint{2.965728in}{1.402566in}}%
\pgfpathlineto{\pgfqpoint{2.967643in}{1.422730in}}%
\pgfpathlineto{\pgfqpoint{2.969558in}{1.427997in}}%
\pgfpathlineto{\pgfqpoint{2.971473in}{1.438897in}}%
\pgfpathlineto{\pgfqpoint{2.973388in}{1.463050in}}%
\pgfpathlineto{\pgfqpoint{2.977219in}{1.472723in}}%
\pgfpathlineto{\pgfqpoint{2.979134in}{1.500516in}}%
\pgfpathlineto{\pgfqpoint{2.982964in}{1.515955in}}%
\pgfpathlineto{\pgfqpoint{2.994455in}{1.564788in}}%
\pgfpathlineto{\pgfqpoint{2.996370in}{1.566155in}}%
\pgfpathlineto{\pgfqpoint{2.998285in}{1.571164in}}%
\pgfpathlineto{\pgfqpoint{3.000201in}{1.572768in}}%
\pgfpathlineto{\pgfqpoint{3.002116in}{1.579641in}}%
\pgfpathlineto{\pgfqpoint{3.005946in}{1.602936in}}%
\pgfpathlineto{\pgfqpoint{3.007861in}{1.606766in}}%
\pgfpathlineto{\pgfqpoint{3.009777in}{1.616119in}}%
\pgfpathlineto{\pgfqpoint{3.011692in}{1.616285in}}%
\pgfpathlineto{\pgfqpoint{3.013607in}{1.623297in}}%
\pgfpathlineto{\pgfqpoint{3.017437in}{1.627683in}}%
\pgfpathlineto{\pgfqpoint{3.019352in}{1.627787in}}%
\pgfpathlineto{\pgfqpoint{3.025098in}{1.654026in}}%
\pgfpathlineto{\pgfqpoint{3.027013in}{1.673286in}}%
\pgfpathlineto{\pgfqpoint{3.028928in}{1.679975in}}%
\pgfpathlineto{\pgfqpoint{3.032759in}{1.703650in}}%
\pgfpathlineto{\pgfqpoint{3.034674in}{1.709293in}}%
\pgfpathlineto{\pgfqpoint{3.036589in}{1.709480in}}%
\pgfpathlineto{\pgfqpoint{3.044250in}{1.716044in}}%
\pgfpathlineto{\pgfqpoint{3.046165in}{1.741674in}}%
\pgfpathlineto{\pgfqpoint{3.048080in}{1.742266in}}%
\pgfpathlineto{\pgfqpoint{3.051910in}{1.791670in}}%
\pgfpathlineto{\pgfqpoint{3.053825in}{1.826535in}}%
\pgfpathlineto{\pgfqpoint{3.053825in}{1.826535in}}%
\pgfusepath{stroke}%
\end{pgfscope}%
\begin{pgfscope}%
\pgfsetrectcap%
\pgfsetmiterjoin%
\pgfsetlinewidth{0.803000pt}%
\definecolor{currentstroke}{rgb}{0.000000,0.000000,0.000000}%
\pgfsetstrokecolor{currentstroke}%
\pgfsetdash{}{0pt}%
\pgfpathmoveto{\pgfqpoint{0.694334in}{0.523557in}}%
\pgfpathlineto{\pgfqpoint{0.694334in}{1.826535in}}%
\pgfusepath{stroke}%
\end{pgfscope}%
\begin{pgfscope}%
\pgfsetrectcap%
\pgfsetmiterjoin%
\pgfsetlinewidth{0.803000pt}%
\definecolor{currentstroke}{rgb}{0.000000,0.000000,0.000000}%
\pgfsetstrokecolor{currentstroke}%
\pgfsetdash{}{0pt}%
\pgfpathmoveto{\pgfqpoint{4.524677in}{0.523557in}}%
\pgfpathlineto{\pgfqpoint{4.524677in}{1.826535in}}%
\pgfusepath{stroke}%
\end{pgfscope}%
\begin{pgfscope}%
\pgfsetrectcap%
\pgfsetmiterjoin%
\pgfsetlinewidth{0.803000pt}%
\definecolor{currentstroke}{rgb}{0.000000,0.000000,0.000000}%
\pgfsetstrokecolor{currentstroke}%
\pgfsetdash{}{0pt}%
\pgfpathmoveto{\pgfqpoint{0.694334in}{0.523557in}}%
\pgfpathlineto{\pgfqpoint{4.524677in}{0.523557in}}%
\pgfusepath{stroke}%
\end{pgfscope}%
\begin{pgfscope}%
\pgfsetrectcap%
\pgfsetmiterjoin%
\pgfsetlinewidth{0.803000pt}%
\definecolor{currentstroke}{rgb}{0.000000,0.000000,0.000000}%
\pgfsetstrokecolor{currentstroke}%
\pgfsetdash{}{0pt}%
\pgfpathmoveto{\pgfqpoint{0.694334in}{1.826535in}}%
\pgfpathlineto{\pgfqpoint{4.524677in}{1.826535in}}%
\pgfusepath{stroke}%
\end{pgfscope}%
\begin{pgfscope}%
\pgfsetrectcap%
\pgfsetroundjoin%
\pgfsetlinewidth{1.003750pt}%
\definecolor{currentstroke}{rgb}{0.878431,0.878431,0.815686}%
\pgfsetstrokecolor{currentstroke}%
\pgfsetdash{}{0pt}%
\pgfpathmoveto{\pgfqpoint{3.867012in}{1.491422in}}%
\pgfpathlineto{\pgfqpoint{4.089235in}{1.491422in}}%
\pgfusepath{stroke}%
\end{pgfscope}%
\begin{pgfscope}%
\definecolor{textcolor}{rgb}{0.000000,0.000000,0.000000}%
\pgfsetstrokecolor{textcolor}%
\pgfsetfillcolor{textcolor}%
\pgftext[x=4.111457in,y=1.452533in,left,base]{\color{textcolor}\rmfamily\fontsize{8.000000}{9.600000}\selectfont T.}%
\end{pgfscope}%
\begin{pgfscope}%
\pgfsetbuttcap%
\pgfsetroundjoin%
\pgfsetlinewidth{1.003750pt}%
\definecolor{currentstroke}{rgb}{0.941176,0.627451,0.188235}%
\pgfsetstrokecolor{currentstroke}%
\pgfsetdash{{1.000000pt}{1.650000pt}}{0.000000pt}%
\pgfpathmoveto{\pgfqpoint{3.867012in}{1.347600in}}%
\pgfpathlineto{\pgfqpoint{4.089235in}{1.347600in}}%
\pgfusepath{stroke}%
\end{pgfscope}%
\begin{pgfscope}%
\definecolor{textcolor}{rgb}{0.000000,0.000000,0.000000}%
\pgfsetstrokecolor{textcolor}%
\pgfsetfillcolor{textcolor}%
\pgftext[x=4.111457in,y=1.308711in,left,base]{\color{textcolor}\rmfamily\fontsize{8.000000}{9.600000}\selectfont FlowC.}%
\end{pgfscope}%
\begin{pgfscope}%
\pgfsetbuttcap%
\pgfsetroundjoin%
\pgfsetlinewidth{1.003750pt}%
\definecolor{currentstroke}{rgb}{0.062745,0.000000,0.062745}%
\pgfsetstrokecolor{currentstroke}%
\pgfsetdash{{3.700000pt}{1.600000pt}}{0.000000pt}%
\pgfpathmoveto{\pgfqpoint{3.867012in}{1.203778in}}%
\pgfpathlineto{\pgfqpoint{4.089235in}{1.203778in}}%
\pgfusepath{stroke}%
\end{pgfscope}%
\begin{pgfscope}%
\definecolor{textcolor}{rgb}{0.000000,0.000000,0.000000}%
\pgfsetstrokecolor{textcolor}%
\pgfsetfillcolor{textcolor}%
\pgftext[x=4.111457in,y=1.164889in,left,base]{\color{textcolor}\rmfamily\fontsize{8.000000}{9.600000}\selectfont htd}%
\end{pgfscope}%
\begin{pgfscope}%
\pgfsetbuttcap%
\pgfsetroundjoin%
\pgfsetlinewidth{1.003750pt}%
\definecolor{currentstroke}{rgb}{0.811765,0.125490,0.125490}%
\pgfsetstrokecolor{currentstroke}%
\pgfsetdash{{1.000000pt}{1.650000pt}}{0.000000pt}%
\pgfpathmoveto{\pgfqpoint{3.867012in}{1.059956in}}%
\pgfpathlineto{\pgfqpoint{4.089235in}{1.059956in}}%
\pgfusepath{stroke}%
\end{pgfscope}%
\begin{pgfscope}%
\definecolor{textcolor}{rgb}{0.000000,0.000000,0.000000}%
\pgfsetstrokecolor{textcolor}%
\pgfsetfillcolor{textcolor}%
\pgftext[x=4.111457in,y=1.021067in,left,base]{\color{textcolor}\rmfamily\fontsize{8.000000}{9.600000}\selectfont Hicks}%
\end{pgfscope}%
\begin{pgfscope}%
\pgfsetrectcap%
\pgfsetroundjoin%
\pgfsetlinewidth{1.003750pt}%
\definecolor{currentstroke}{rgb}{0.000000,0.000000,0.376471}%
\pgfsetstrokecolor{currentstroke}%
\pgfsetdash{}{0pt}%
\pgfpathmoveto{\pgfqpoint{3.867012in}{0.916134in}}%
\pgfpathlineto{\pgfqpoint{4.089235in}{0.916134in}}%
\pgfusepath{stroke}%
\end{pgfscope}%
\begin{pgfscope}%
\definecolor{textcolor}{rgb}{0.000000,0.000000,0.000000}%
\pgfsetstrokecolor{textcolor}%
\pgfsetfillcolor{textcolor}%
\pgftext[x=4.111457in,y=0.877245in,left,base]{\color{textcolor}\rmfamily\fontsize{8.000000}{9.600000}\selectfont P3}%
\end{pgfscope}%
\begin{pgfscope}%
\pgfsetrectcap%
\pgfsetroundjoin%
\pgfsetlinewidth{1.003750pt}%
\definecolor{currentstroke}{rgb}{0.564706,0.564706,1.000000}%
\pgfsetstrokecolor{currentstroke}%
\pgfsetdash{}{0pt}%
\pgfpathmoveto{\pgfqpoint{3.867012in}{0.772312in}}%
\pgfpathlineto{\pgfqpoint{4.089235in}{0.772312in}}%
\pgfusepath{stroke}%
\end{pgfscope}%
\begin{pgfscope}%
\definecolor{textcolor}{rgb}{0.000000,0.000000,0.000000}%
\pgfsetstrokecolor{textcolor}%
\pgfsetfillcolor{textcolor}%
\pgftext[x=4.111457in,y=0.733423in,left,base]{\color{textcolor}\rmfamily\fontsize{8.000000}{9.600000}\selectfont P4}%
\end{pgfscope}%
\begin{pgfscope}%
\pgfsetbuttcap%
\pgfsetroundjoin%
\pgfsetlinewidth{1.003750pt}%
\definecolor{currentstroke}{rgb}{0.000000,0.000000,0.000000}%
\pgfsetstrokecolor{currentstroke}%
\pgfsetdash{{1.000000pt}{1.650000pt}}{0.000000pt}%
\pgfpathmoveto{\pgfqpoint{3.867012in}{0.628490in}}%
\pgfpathlineto{\pgfqpoint{4.089235in}{0.628490in}}%
\pgfusepath{stroke}%
\end{pgfscope}%
\begin{pgfscope}%
\definecolor{textcolor}{rgb}{0.000000,0.000000,0.000000}%
\pgfsetstrokecolor{textcolor}%
\pgfsetfillcolor{textcolor}%
\pgftext[x=4.111457in,y=0.589601in,left,base]{\color{textcolor}\rmfamily\fontsize{8.000000}{9.600000}\selectfont VBS}%
\end{pgfscope}%
\end{pgfpicture}%
\makeatother%
\endgroup%

    \vspace*{-0.7cm}
	\caption{\label{fig:parallel:planning} A cactus plot of the performance of various planners. A planner ``solves'' a benchmark when it finds a contraction tree of max rank 30 or smaller.}
\end{figure}

\subsection{Experiment 1: The Planning Phase (RQ1 and RQ2)}
We run each planning implementation (\pkg{FlowCutter}, \pkg{htd}, \pkg{Tamaki}, \pkg{Hicks}, \pkg{P3}, and \pkg{P4}) once on each of our 1914 benchmarks and save all contraction trees found within 1000 seconds (without executing the contractions). Results are summarized in Figure \ref{fig:parallel:planning}. 

% In this figure, we consider a benchmark to be solved by a planning implementation when the planner is able to find contraction tree whose max-rank is 30 or smaller.

%\begin{figure}[t]
%\begin{center}
%%% Creator: Matplotlib, PGF backend
%%
%% To include the figure in your LaTeX document, write
%%   \input{<filename>.pgf}
%%
%% Make sure the required packages are loaded in your preamble
%%   \usepackage{pgf}
%%
%% and, on pdftex
%%   \usepackage[utf8]{inputenc}\DeclareUnicodeCharacter{2212}{-}
%%
%% or, on luatex and xetex
%%   \usepackage{unicode-math}
%%
%% Figures using additional raster images can only be included by \input if
%% they are in the same directory as the main LaTeX file. For loading figures
%% from other directories you can use the `import` package
%%   \usepackage{import}
%%
%% and then include the figures with
%%   \import{<path to file>}{<filename>.pgf}
%%
%% Matplotlib used the following preamble
%%   \usepackage[utf8x]{inputenc}
%%   \usepackage[T1]{fontenc}
%%
\begingroup%
\makeatletter%
\begin{pgfpicture}%
\pgfpathrectangle{\pgfpointorigin}{\pgfqpoint{4.803148in}{2.021259in}}%
\pgfusepath{use as bounding box, clip}%
\begin{pgfscope}%
\pgfsetbuttcap%
\pgfsetmiterjoin%
\definecolor{currentfill}{rgb}{1.000000,1.000000,1.000000}%
\pgfsetfillcolor{currentfill}%
\pgfsetlinewidth{0.000000pt}%
\definecolor{currentstroke}{rgb}{1.000000,1.000000,1.000000}%
\pgfsetstrokecolor{currentstroke}%
\pgfsetdash{}{0pt}%
\pgfpathmoveto{\pgfqpoint{0.000000in}{0.000000in}}%
\pgfpathlineto{\pgfqpoint{4.803148in}{0.000000in}}%
\pgfpathlineto{\pgfqpoint{4.803148in}{2.021259in}}%
\pgfpathlineto{\pgfqpoint{0.000000in}{2.021259in}}%
\pgfpathclose%
\pgfusepath{fill}%
\end{pgfscope}%
\begin{pgfscope}%
\pgfsetbuttcap%
\pgfsetmiterjoin%
\definecolor{currentfill}{rgb}{1.000000,1.000000,1.000000}%
\pgfsetfillcolor{currentfill}%
\pgfsetlinewidth{0.000000pt}%
\definecolor{currentstroke}{rgb}{0.000000,0.000000,0.000000}%
\pgfsetstrokecolor{currentstroke}%
\pgfsetstrokeopacity{0.000000}%
\pgfsetdash{}{0pt}%
\pgfpathmoveto{\pgfqpoint{0.694334in}{0.523557in}}%
\pgfpathlineto{\pgfqpoint{4.524677in}{0.523557in}}%
\pgfpathlineto{\pgfqpoint{4.524677in}{1.826535in}}%
\pgfpathlineto{\pgfqpoint{0.694334in}{1.826535in}}%
\pgfpathclose%
\pgfusepath{fill}%
\end{pgfscope}%
\begin{pgfscope}%
\pgfsetbuttcap%
\pgfsetroundjoin%
\definecolor{currentfill}{rgb}{0.000000,0.000000,0.000000}%
\pgfsetfillcolor{currentfill}%
\pgfsetlinewidth{0.803000pt}%
\definecolor{currentstroke}{rgb}{0.000000,0.000000,0.000000}%
\pgfsetstrokecolor{currentstroke}%
\pgfsetdash{}{0pt}%
\pgfsys@defobject{currentmarker}{\pgfqpoint{0.000000in}{-0.048611in}}{\pgfqpoint{0.000000in}{0.000000in}}{%
\pgfpathmoveto{\pgfqpoint{0.000000in}{0.000000in}}%
\pgfpathlineto{\pgfqpoint{0.000000in}{-0.048611in}}%
\pgfusepath{stroke,fill}%
}%
\begin{pgfscope}%
\pgfsys@transformshift{0.694334in}{0.523557in}%
\pgfsys@useobject{currentmarker}{}%
\end{pgfscope}%
\end{pgfscope}%
\begin{pgfscope}%
\definecolor{textcolor}{rgb}{0.000000,0.000000,0.000000}%
\pgfsetstrokecolor{textcolor}%
\pgfsetfillcolor{textcolor}%
\pgftext[x=0.694334in,y=0.426335in,,top]{\color{textcolor}\rmfamily\fontsize{9.000000}{10.800000}\selectfont \(\displaystyle 0\)}%
\end{pgfscope}%
\begin{pgfscope}%
\pgfsetbuttcap%
\pgfsetroundjoin%
\definecolor{currentfill}{rgb}{0.000000,0.000000,0.000000}%
\pgfsetfillcolor{currentfill}%
\pgfsetlinewidth{0.803000pt}%
\definecolor{currentstroke}{rgb}{0.000000,0.000000,0.000000}%
\pgfsetstrokecolor{currentstroke}%
\pgfsetdash{}{0pt}%
\pgfsys@defobject{currentmarker}{\pgfqpoint{0.000000in}{-0.048611in}}{\pgfqpoint{0.000000in}{0.000000in}}{%
\pgfpathmoveto{\pgfqpoint{0.000000in}{0.000000in}}%
\pgfpathlineto{\pgfqpoint{0.000000in}{-0.048611in}}%
\pgfusepath{stroke,fill}%
}%
\begin{pgfscope}%
\pgfsys@transformshift{1.173127in}{0.523557in}%
\pgfsys@useobject{currentmarker}{}%
\end{pgfscope}%
\end{pgfscope}%
\begin{pgfscope}%
\definecolor{textcolor}{rgb}{0.000000,0.000000,0.000000}%
\pgfsetstrokecolor{textcolor}%
\pgfsetfillcolor{textcolor}%
\pgftext[x=1.173127in,y=0.426335in,,top]{\color{textcolor}\rmfamily\fontsize{9.000000}{10.800000}\selectfont \(\displaystyle 250\)}%
\end{pgfscope}%
\begin{pgfscope}%
\pgfsetbuttcap%
\pgfsetroundjoin%
\definecolor{currentfill}{rgb}{0.000000,0.000000,0.000000}%
\pgfsetfillcolor{currentfill}%
\pgfsetlinewidth{0.803000pt}%
\definecolor{currentstroke}{rgb}{0.000000,0.000000,0.000000}%
\pgfsetstrokecolor{currentstroke}%
\pgfsetdash{}{0pt}%
\pgfsys@defobject{currentmarker}{\pgfqpoint{0.000000in}{-0.048611in}}{\pgfqpoint{0.000000in}{0.000000in}}{%
\pgfpathmoveto{\pgfqpoint{0.000000in}{0.000000in}}%
\pgfpathlineto{\pgfqpoint{0.000000in}{-0.048611in}}%
\pgfusepath{stroke,fill}%
}%
\begin{pgfscope}%
\pgfsys@transformshift{1.651920in}{0.523557in}%
\pgfsys@useobject{currentmarker}{}%
\end{pgfscope}%
\end{pgfscope}%
\begin{pgfscope}%
\definecolor{textcolor}{rgb}{0.000000,0.000000,0.000000}%
\pgfsetstrokecolor{textcolor}%
\pgfsetfillcolor{textcolor}%
\pgftext[x=1.651920in,y=0.426335in,,top]{\color{textcolor}\rmfamily\fontsize{9.000000}{10.800000}\selectfont \(\displaystyle 500\)}%
\end{pgfscope}%
\begin{pgfscope}%
\pgfsetbuttcap%
\pgfsetroundjoin%
\definecolor{currentfill}{rgb}{0.000000,0.000000,0.000000}%
\pgfsetfillcolor{currentfill}%
\pgfsetlinewidth{0.803000pt}%
\definecolor{currentstroke}{rgb}{0.000000,0.000000,0.000000}%
\pgfsetstrokecolor{currentstroke}%
\pgfsetdash{}{0pt}%
\pgfsys@defobject{currentmarker}{\pgfqpoint{0.000000in}{-0.048611in}}{\pgfqpoint{0.000000in}{0.000000in}}{%
\pgfpathmoveto{\pgfqpoint{0.000000in}{0.000000in}}%
\pgfpathlineto{\pgfqpoint{0.000000in}{-0.048611in}}%
\pgfusepath{stroke,fill}%
}%
\begin{pgfscope}%
\pgfsys@transformshift{2.130713in}{0.523557in}%
\pgfsys@useobject{currentmarker}{}%
\end{pgfscope}%
\end{pgfscope}%
\begin{pgfscope}%
\definecolor{textcolor}{rgb}{0.000000,0.000000,0.000000}%
\pgfsetstrokecolor{textcolor}%
\pgfsetfillcolor{textcolor}%
\pgftext[x=2.130713in,y=0.426335in,,top]{\color{textcolor}\rmfamily\fontsize{9.000000}{10.800000}\selectfont \(\displaystyle 750\)}%
\end{pgfscope}%
\begin{pgfscope}%
\pgfsetbuttcap%
\pgfsetroundjoin%
\definecolor{currentfill}{rgb}{0.000000,0.000000,0.000000}%
\pgfsetfillcolor{currentfill}%
\pgfsetlinewidth{0.803000pt}%
\definecolor{currentstroke}{rgb}{0.000000,0.000000,0.000000}%
\pgfsetstrokecolor{currentstroke}%
\pgfsetdash{}{0pt}%
\pgfsys@defobject{currentmarker}{\pgfqpoint{0.000000in}{-0.048611in}}{\pgfqpoint{0.000000in}{0.000000in}}{%
\pgfpathmoveto{\pgfqpoint{0.000000in}{0.000000in}}%
\pgfpathlineto{\pgfqpoint{0.000000in}{-0.048611in}}%
\pgfusepath{stroke,fill}%
}%
\begin{pgfscope}%
\pgfsys@transformshift{2.609506in}{0.523557in}%
\pgfsys@useobject{currentmarker}{}%
\end{pgfscope}%
\end{pgfscope}%
\begin{pgfscope}%
\definecolor{textcolor}{rgb}{0.000000,0.000000,0.000000}%
\pgfsetstrokecolor{textcolor}%
\pgfsetfillcolor{textcolor}%
\pgftext[x=2.609506in,y=0.426335in,,top]{\color{textcolor}\rmfamily\fontsize{9.000000}{10.800000}\selectfont \(\displaystyle 1000\)}%
\end{pgfscope}%
\begin{pgfscope}%
\pgfsetbuttcap%
\pgfsetroundjoin%
\definecolor{currentfill}{rgb}{0.000000,0.000000,0.000000}%
\pgfsetfillcolor{currentfill}%
\pgfsetlinewidth{0.803000pt}%
\definecolor{currentstroke}{rgb}{0.000000,0.000000,0.000000}%
\pgfsetstrokecolor{currentstroke}%
\pgfsetdash{}{0pt}%
\pgfsys@defobject{currentmarker}{\pgfqpoint{0.000000in}{-0.048611in}}{\pgfqpoint{0.000000in}{0.000000in}}{%
\pgfpathmoveto{\pgfqpoint{0.000000in}{0.000000in}}%
\pgfpathlineto{\pgfqpoint{0.000000in}{-0.048611in}}%
\pgfusepath{stroke,fill}%
}%
\begin{pgfscope}%
\pgfsys@transformshift{3.088299in}{0.523557in}%
\pgfsys@useobject{currentmarker}{}%
\end{pgfscope}%
\end{pgfscope}%
\begin{pgfscope}%
\definecolor{textcolor}{rgb}{0.000000,0.000000,0.000000}%
\pgfsetstrokecolor{textcolor}%
\pgfsetfillcolor{textcolor}%
\pgftext[x=3.088299in,y=0.426335in,,top]{\color{textcolor}\rmfamily\fontsize{9.000000}{10.800000}\selectfont \(\displaystyle 1250\)}%
\end{pgfscope}%
\begin{pgfscope}%
\pgfsetbuttcap%
\pgfsetroundjoin%
\definecolor{currentfill}{rgb}{0.000000,0.000000,0.000000}%
\pgfsetfillcolor{currentfill}%
\pgfsetlinewidth{0.803000pt}%
\definecolor{currentstroke}{rgb}{0.000000,0.000000,0.000000}%
\pgfsetstrokecolor{currentstroke}%
\pgfsetdash{}{0pt}%
\pgfsys@defobject{currentmarker}{\pgfqpoint{0.000000in}{-0.048611in}}{\pgfqpoint{0.000000in}{0.000000in}}{%
\pgfpathmoveto{\pgfqpoint{0.000000in}{0.000000in}}%
\pgfpathlineto{\pgfqpoint{0.000000in}{-0.048611in}}%
\pgfusepath{stroke,fill}%
}%
\begin{pgfscope}%
\pgfsys@transformshift{3.567091in}{0.523557in}%
\pgfsys@useobject{currentmarker}{}%
\end{pgfscope}%
\end{pgfscope}%
\begin{pgfscope}%
\definecolor{textcolor}{rgb}{0.000000,0.000000,0.000000}%
\pgfsetstrokecolor{textcolor}%
\pgfsetfillcolor{textcolor}%
\pgftext[x=3.567091in,y=0.426335in,,top]{\color{textcolor}\rmfamily\fontsize{9.000000}{10.800000}\selectfont \(\displaystyle 1500\)}%
\end{pgfscope}%
\begin{pgfscope}%
\pgfsetbuttcap%
\pgfsetroundjoin%
\definecolor{currentfill}{rgb}{0.000000,0.000000,0.000000}%
\pgfsetfillcolor{currentfill}%
\pgfsetlinewidth{0.803000pt}%
\definecolor{currentstroke}{rgb}{0.000000,0.000000,0.000000}%
\pgfsetstrokecolor{currentstroke}%
\pgfsetdash{}{0pt}%
\pgfsys@defobject{currentmarker}{\pgfqpoint{0.000000in}{-0.048611in}}{\pgfqpoint{0.000000in}{0.000000in}}{%
\pgfpathmoveto{\pgfqpoint{0.000000in}{0.000000in}}%
\pgfpathlineto{\pgfqpoint{0.000000in}{-0.048611in}}%
\pgfusepath{stroke,fill}%
}%
\begin{pgfscope}%
\pgfsys@transformshift{4.045884in}{0.523557in}%
\pgfsys@useobject{currentmarker}{}%
\end{pgfscope}%
\end{pgfscope}%
\begin{pgfscope}%
\definecolor{textcolor}{rgb}{0.000000,0.000000,0.000000}%
\pgfsetstrokecolor{textcolor}%
\pgfsetfillcolor{textcolor}%
\pgftext[x=4.045884in,y=0.426335in,,top]{\color{textcolor}\rmfamily\fontsize{9.000000}{10.800000}\selectfont \(\displaystyle 1750\)}%
\end{pgfscope}%
\begin{pgfscope}%
\pgfsetbuttcap%
\pgfsetroundjoin%
\definecolor{currentfill}{rgb}{0.000000,0.000000,0.000000}%
\pgfsetfillcolor{currentfill}%
\pgfsetlinewidth{0.803000pt}%
\definecolor{currentstroke}{rgb}{0.000000,0.000000,0.000000}%
\pgfsetstrokecolor{currentstroke}%
\pgfsetdash{}{0pt}%
\pgfsys@defobject{currentmarker}{\pgfqpoint{0.000000in}{-0.048611in}}{\pgfqpoint{0.000000in}{0.000000in}}{%
\pgfpathmoveto{\pgfqpoint{0.000000in}{0.000000in}}%
\pgfpathlineto{\pgfqpoint{0.000000in}{-0.048611in}}%
\pgfusepath{stroke,fill}%
}%
\begin{pgfscope}%
\pgfsys@transformshift{4.524677in}{0.523557in}%
\pgfsys@useobject{currentmarker}{}%
\end{pgfscope}%
\end{pgfscope}%
\begin{pgfscope}%
\definecolor{textcolor}{rgb}{0.000000,0.000000,0.000000}%
\pgfsetstrokecolor{textcolor}%
\pgfsetfillcolor{textcolor}%
\pgftext[x=4.524677in,y=0.426335in,,top]{\color{textcolor}\rmfamily\fontsize{9.000000}{10.800000}\selectfont \(\displaystyle 2000\)}%
\end{pgfscope}%
\begin{pgfscope}%
\definecolor{textcolor}{rgb}{0.000000,0.000000,0.000000}%
\pgfsetstrokecolor{textcolor}%
\pgfsetfillcolor{textcolor}%
\pgftext[x=2.609506in,y=0.260390in,,top]{\color{textcolor}\rmfamily\fontsize{9.000000}{10.800000}\selectfont Number of benchmarks solved}%
\end{pgfscope}%
\begin{pgfscope}%
\pgfsetbuttcap%
\pgfsetroundjoin%
\definecolor{currentfill}{rgb}{0.000000,0.000000,0.000000}%
\pgfsetfillcolor{currentfill}%
\pgfsetlinewidth{0.803000pt}%
\definecolor{currentstroke}{rgb}{0.000000,0.000000,0.000000}%
\pgfsetstrokecolor{currentstroke}%
\pgfsetdash{}{0pt}%
\pgfsys@defobject{currentmarker}{\pgfqpoint{-0.048611in}{0.000000in}}{\pgfqpoint{0.000000in}{0.000000in}}{%
\pgfpathmoveto{\pgfqpoint{0.000000in}{0.000000in}}%
\pgfpathlineto{\pgfqpoint{-0.048611in}{0.000000in}}%
\pgfusepath{stroke,fill}%
}%
\begin{pgfscope}%
\pgfsys@transformshift{0.694334in}{0.843347in}%
\pgfsys@useobject{currentmarker}{}%
\end{pgfscope}%
\end{pgfscope}%
\begin{pgfscope}%
\definecolor{textcolor}{rgb}{0.000000,0.000000,0.000000}%
\pgfsetstrokecolor{textcolor}%
\pgfsetfillcolor{textcolor}%
\pgftext[x=0.330525in, y=0.798622in, left, base]{\color{textcolor}\rmfamily\fontsize{9.000000}{10.800000}\selectfont \(\displaystyle 10^{-1}\)}%
\end{pgfscope}%
\begin{pgfscope}%
\pgfsetbuttcap%
\pgfsetroundjoin%
\definecolor{currentfill}{rgb}{0.000000,0.000000,0.000000}%
\pgfsetfillcolor{currentfill}%
\pgfsetlinewidth{0.803000pt}%
\definecolor{currentstroke}{rgb}{0.000000,0.000000,0.000000}%
\pgfsetstrokecolor{currentstroke}%
\pgfsetdash{}{0pt}%
\pgfsys@defobject{currentmarker}{\pgfqpoint{-0.048611in}{0.000000in}}{\pgfqpoint{0.000000in}{0.000000in}}{%
\pgfpathmoveto{\pgfqpoint{0.000000in}{0.000000in}}%
\pgfpathlineto{\pgfqpoint{-0.048611in}{0.000000in}}%
\pgfusepath{stroke,fill}%
}%
\begin{pgfscope}%
\pgfsys@transformshift{0.694334in}{1.334941in}%
\pgfsys@useobject{currentmarker}{}%
\end{pgfscope}%
\end{pgfscope}%
\begin{pgfscope}%
\definecolor{textcolor}{rgb}{0.000000,0.000000,0.000000}%
\pgfsetstrokecolor{textcolor}%
\pgfsetfillcolor{textcolor}%
\pgftext[x=0.410771in, y=1.290216in, left, base]{\color{textcolor}\rmfamily\fontsize{9.000000}{10.800000}\selectfont \(\displaystyle 10^{1}\)}%
\end{pgfscope}%
\begin{pgfscope}%
\pgfsetbuttcap%
\pgfsetroundjoin%
\definecolor{currentfill}{rgb}{0.000000,0.000000,0.000000}%
\pgfsetfillcolor{currentfill}%
\pgfsetlinewidth{0.803000pt}%
\definecolor{currentstroke}{rgb}{0.000000,0.000000,0.000000}%
\pgfsetstrokecolor{currentstroke}%
\pgfsetdash{}{0pt}%
\pgfsys@defobject{currentmarker}{\pgfqpoint{-0.048611in}{0.000000in}}{\pgfqpoint{0.000000in}{0.000000in}}{%
\pgfpathmoveto{\pgfqpoint{0.000000in}{0.000000in}}%
\pgfpathlineto{\pgfqpoint{-0.048611in}{0.000000in}}%
\pgfusepath{stroke,fill}%
}%
\begin{pgfscope}%
\pgfsys@transformshift{0.694334in}{1.826535in}%
\pgfsys@useobject{currentmarker}{}%
\end{pgfscope}%
\end{pgfscope}%
\begin{pgfscope}%
\definecolor{textcolor}{rgb}{0.000000,0.000000,0.000000}%
\pgfsetstrokecolor{textcolor}%
\pgfsetfillcolor{textcolor}%
\pgftext[x=0.410771in, y=1.781810in, left, base]{\color{textcolor}\rmfamily\fontsize{9.000000}{10.800000}\selectfont \(\displaystyle 10^{3}\)}%
\end{pgfscope}%
\begin{pgfscope}%
\definecolor{textcolor}{rgb}{0.000000,0.000000,0.000000}%
\pgfsetstrokecolor{textcolor}%
\pgfsetfillcolor{textcolor}%
\pgftext[x=0.274969in,y=1.175046in,,bottom,rotate=90.000000]{\color{textcolor}\rmfamily\fontsize{9.000000}{10.800000}\selectfont Longest solving time (s)}%
\end{pgfscope}%
\begin{pgfscope}%
\pgfpathrectangle{\pgfqpoint{0.694334in}{0.523557in}}{\pgfqpoint{3.830343in}{1.302977in}}%
\pgfusepath{clip}%
\pgfsetrectcap%
\pgfsetroundjoin%
\pgfsetlinewidth{1.003750pt}%
\definecolor{currentstroke}{rgb}{0.878431,0.878431,0.815686}%
\pgfsetstrokecolor{currentstroke}%
\pgfsetdash{}{0pt}%
\pgfpathmoveto{\pgfqpoint{0.694334in}{0.948144in}}%
\pgfpathlineto{\pgfqpoint{0.696249in}{0.955934in}}%
\pgfpathlineto{\pgfqpoint{0.701995in}{0.961880in}}%
\pgfpathlineto{\pgfqpoint{0.703910in}{0.964864in}}%
\pgfpathlineto{\pgfqpoint{0.705825in}{0.965524in}}%
\pgfpathlineto{\pgfqpoint{0.709656in}{0.973680in}}%
\pgfpathlineto{\pgfqpoint{0.711571in}{0.974495in}}%
\pgfpathlineto{\pgfqpoint{0.713486in}{0.976961in}}%
\pgfpathlineto{\pgfqpoint{0.719232in}{0.978816in}}%
\pgfpathlineto{\pgfqpoint{0.721147in}{0.980178in}}%
\pgfpathlineto{\pgfqpoint{0.724977in}{1.000991in}}%
\pgfpathlineto{\pgfqpoint{0.728807in}{1.015379in}}%
\pgfpathlineto{\pgfqpoint{0.730723in}{1.015482in}}%
\pgfpathlineto{\pgfqpoint{0.732638in}{1.022216in}}%
\pgfpathlineto{\pgfqpoint{0.734553in}{1.022392in}}%
\pgfpathlineto{\pgfqpoint{0.736468in}{1.024621in}}%
\pgfpathlineto{\pgfqpoint{0.738383in}{1.024629in}}%
\pgfpathlineto{\pgfqpoint{0.742214in}{1.030245in}}%
\pgfpathlineto{\pgfqpoint{0.744129in}{1.030409in}}%
\pgfpathlineto{\pgfqpoint{0.746044in}{1.033013in}}%
\pgfpathlineto{\pgfqpoint{0.747959in}{1.039599in}}%
\pgfpathlineto{\pgfqpoint{0.749874in}{1.040605in}}%
\pgfpathlineto{\pgfqpoint{0.755620in}{1.049992in}}%
\pgfpathlineto{\pgfqpoint{0.772856in}{1.056340in}}%
\pgfpathlineto{\pgfqpoint{0.786263in}{1.061350in}}%
\pgfpathlineto{\pgfqpoint{0.801584in}{1.062980in}}%
\pgfpathlineto{\pgfqpoint{0.805414in}{1.063733in}}%
\pgfpathlineto{\pgfqpoint{0.814990in}{1.065163in}}%
\pgfpathlineto{\pgfqpoint{0.882021in}{1.076339in}}%
\pgfpathlineto{\pgfqpoint{0.885851in}{1.077456in}}%
\pgfpathlineto{\pgfqpoint{0.895427in}{1.078655in}}%
\pgfpathlineto{\pgfqpoint{0.906918in}{1.080409in}}%
\pgfpathlineto{\pgfqpoint{0.987356in}{1.089407in}}%
\pgfpathlineto{\pgfqpoint{1.008422in}{1.090894in}}%
\pgfpathlineto{\pgfqpoint{1.052471in}{1.095137in}}%
\pgfpathlineto{\pgfqpoint{1.056302in}{1.095792in}}%
\pgfpathlineto{\pgfqpoint{1.077369in}{1.097817in}}%
\pgfpathlineto{\pgfqpoint{1.081199in}{1.099401in}}%
\pgfpathlineto{\pgfqpoint{1.096520in}{1.100833in}}%
\pgfpathlineto{\pgfqpoint{1.106096in}{1.102076in}}%
\pgfpathlineto{\pgfqpoint{1.159721in}{1.107209in}}%
\pgfpathlineto{\pgfqpoint{1.163551in}{1.109041in}}%
\pgfpathlineto{\pgfqpoint{1.173127in}{1.110874in}}%
\pgfpathlineto{\pgfqpoint{1.180788in}{1.111938in}}%
\pgfpathlineto{\pgfqpoint{1.186533in}{1.112998in}}%
\pgfpathlineto{\pgfqpoint{1.194194in}{1.114050in}}%
\pgfpathlineto{\pgfqpoint{1.213346in}{1.115660in}}%
\pgfpathlineto{\pgfqpoint{1.221006in}{1.118786in}}%
\pgfpathlineto{\pgfqpoint{1.228667in}{1.120114in}}%
\pgfpathlineto{\pgfqpoint{1.255480in}{1.127277in}}%
\pgfpathlineto{\pgfqpoint{1.265055in}{1.131122in}}%
\pgfpathlineto{\pgfqpoint{1.274631in}{1.132047in}}%
\pgfpathlineto{\pgfqpoint{1.278462in}{1.133795in}}%
\pgfpathlineto{\pgfqpoint{1.295698in}{1.138666in}}%
\pgfpathlineto{\pgfqpoint{1.299528in}{1.139717in}}%
\pgfpathlineto{\pgfqpoint{1.301444in}{1.139860in}}%
\pgfpathlineto{\pgfqpoint{1.303359in}{1.142985in}}%
\pgfpathlineto{\pgfqpoint{1.309104in}{1.143567in}}%
\pgfpathlineto{\pgfqpoint{1.311019in}{1.145858in}}%
\pgfpathlineto{\pgfqpoint{1.316765in}{1.146442in}}%
\pgfpathlineto{\pgfqpoint{1.320595in}{1.149572in}}%
\pgfpathlineto{\pgfqpoint{1.326341in}{1.150947in}}%
\pgfpathlineto{\pgfqpoint{1.334002in}{1.152019in}}%
\pgfpathlineto{\pgfqpoint{1.349323in}{1.155558in}}%
\pgfpathlineto{\pgfqpoint{1.351238in}{1.157653in}}%
\pgfpathlineto{\pgfqpoint{1.353153in}{1.157732in}}%
\pgfpathlineto{\pgfqpoint{1.356984in}{1.160260in}}%
\pgfpathlineto{\pgfqpoint{1.360814in}{1.160544in}}%
\pgfpathlineto{\pgfqpoint{1.368475in}{1.163791in}}%
\pgfpathlineto{\pgfqpoint{1.372305in}{1.164419in}}%
\pgfpathlineto{\pgfqpoint{1.381881in}{1.167150in}}%
\pgfpathlineto{\pgfqpoint{1.395287in}{1.168617in}}%
\pgfpathlineto{\pgfqpoint{1.402948in}{1.170470in}}%
\pgfpathlineto{\pgfqpoint{1.410608in}{1.171407in}}%
\pgfpathlineto{\pgfqpoint{1.420184in}{1.176099in}}%
\pgfpathlineto{\pgfqpoint{1.441251in}{1.178997in}}%
\pgfpathlineto{\pgfqpoint{1.446997in}{1.180239in}}%
\pgfpathlineto{\pgfqpoint{1.458488in}{1.181758in}}%
\pgfpathlineto{\pgfqpoint{1.466148in}{1.183530in}}%
\pgfpathlineto{\pgfqpoint{1.475724in}{1.184325in}}%
\pgfpathlineto{\pgfqpoint{1.485300in}{1.185970in}}%
\pgfpathlineto{\pgfqpoint{1.496791in}{1.186897in}}%
\pgfpathlineto{\pgfqpoint{1.500621in}{1.188076in}}%
\pgfpathlineto{\pgfqpoint{1.510197in}{1.189051in}}%
\pgfpathlineto{\pgfqpoint{1.517858in}{1.191555in}}%
\pgfpathlineto{\pgfqpoint{1.529349in}{1.192277in}}%
\pgfpathlineto{\pgfqpoint{1.542755in}{1.193417in}}%
\pgfpathlineto{\pgfqpoint{1.556161in}{1.195160in}}%
\pgfpathlineto{\pgfqpoint{1.569568in}{1.196028in}}%
\pgfpathlineto{\pgfqpoint{1.573398in}{1.197232in}}%
\pgfpathlineto{\pgfqpoint{1.590635in}{1.200437in}}%
\pgfpathlineto{\pgfqpoint{1.592550in}{1.202315in}}%
\pgfpathlineto{\pgfqpoint{1.598295in}{1.203693in}}%
\pgfpathlineto{\pgfqpoint{1.627023in}{1.210857in}}%
\pgfpathlineto{\pgfqpoint{1.630853in}{1.211879in}}%
\pgfpathlineto{\pgfqpoint{1.632768in}{1.212608in}}%
\pgfpathlineto{\pgfqpoint{1.634683in}{1.214894in}}%
\pgfpathlineto{\pgfqpoint{1.650005in}{1.216939in}}%
\pgfpathlineto{\pgfqpoint{1.653835in}{1.217843in}}%
\pgfpathlineto{\pgfqpoint{1.667241in}{1.220406in}}%
\pgfpathlineto{\pgfqpoint{1.671072in}{1.222320in}}%
\pgfpathlineto{\pgfqpoint{1.678732in}{1.223561in}}%
\pgfpathlineto{\pgfqpoint{1.682563in}{1.225762in}}%
\pgfpathlineto{\pgfqpoint{1.690223in}{1.227473in}}%
\pgfpathlineto{\pgfqpoint{1.694054in}{1.228995in}}%
\pgfpathlineto{\pgfqpoint{1.695969in}{1.229430in}}%
\pgfpathlineto{\pgfqpoint{1.697884in}{1.231143in}}%
\pgfpathlineto{\pgfqpoint{1.701714in}{1.232314in}}%
\pgfpathlineto{\pgfqpoint{1.709375in}{1.233737in}}%
\pgfpathlineto{\pgfqpoint{1.713205in}{1.235570in}}%
\pgfpathlineto{\pgfqpoint{1.728527in}{1.241910in}}%
\pgfpathlineto{\pgfqpoint{1.736188in}{1.243184in}}%
\pgfpathlineto{\pgfqpoint{1.745763in}{1.246246in}}%
\pgfpathlineto{\pgfqpoint{1.757254in}{1.248056in}}%
\pgfpathlineto{\pgfqpoint{1.761085in}{1.248943in}}%
\pgfpathlineto{\pgfqpoint{1.764915in}{1.249471in}}%
\pgfpathlineto{\pgfqpoint{1.774491in}{1.251515in}}%
\pgfpathlineto{\pgfqpoint{1.785982in}{1.252902in}}%
\pgfpathlineto{\pgfqpoint{1.789812in}{1.254325in}}%
\pgfpathlineto{\pgfqpoint{1.793643in}{1.254431in}}%
\pgfpathlineto{\pgfqpoint{1.797473in}{1.256010in}}%
\pgfpathlineto{\pgfqpoint{1.803219in}{1.257885in}}%
\pgfpathlineto{\pgfqpoint{1.835776in}{1.262884in}}%
\pgfpathlineto{\pgfqpoint{1.887486in}{1.270353in}}%
\pgfpathlineto{\pgfqpoint{1.893232in}{1.271230in}}%
\pgfpathlineto{\pgfqpoint{1.902807in}{1.275463in}}%
\pgfpathlineto{\pgfqpoint{1.906638in}{1.276300in}}%
\pgfpathlineto{\pgfqpoint{1.908553in}{1.280411in}}%
\pgfpathlineto{\pgfqpoint{1.916214in}{1.282545in}}%
\pgfpathlineto{\pgfqpoint{1.927705in}{1.284625in}}%
\pgfpathlineto{\pgfqpoint{1.929620in}{1.286283in}}%
\pgfpathlineto{\pgfqpoint{1.937281in}{1.287696in}}%
\pgfpathlineto{\pgfqpoint{1.956432in}{1.296836in}}%
\pgfpathlineto{\pgfqpoint{1.960263in}{1.301263in}}%
\pgfpathlineto{\pgfqpoint{1.964093in}{1.302226in}}%
\pgfpathlineto{\pgfqpoint{1.971754in}{1.304660in}}%
\pgfpathlineto{\pgfqpoint{1.973669in}{1.306666in}}%
\pgfpathlineto{\pgfqpoint{1.981329in}{1.307451in}}%
\pgfpathlineto{\pgfqpoint{1.988990in}{1.309458in}}%
\pgfpathlineto{\pgfqpoint{2.000481in}{1.310640in}}%
\pgfpathlineto{\pgfqpoint{2.002396in}{1.310865in}}%
\pgfpathlineto{\pgfqpoint{2.004312in}{1.313052in}}%
\pgfpathlineto{\pgfqpoint{2.006227in}{1.313203in}}%
\pgfpathlineto{\pgfqpoint{2.010057in}{1.315514in}}%
\pgfpathlineto{\pgfqpoint{2.017718in}{1.316843in}}%
\pgfpathlineto{\pgfqpoint{2.021548in}{1.318677in}}%
\pgfpathlineto{\pgfqpoint{2.031124in}{1.321009in}}%
\pgfpathlineto{\pgfqpoint{2.036869in}{1.325537in}}%
\pgfpathlineto{\pgfqpoint{2.046445in}{1.327598in}}%
\pgfpathlineto{\pgfqpoint{2.048360in}{1.329940in}}%
\pgfpathlineto{\pgfqpoint{2.056021in}{1.331163in}}%
\pgfpathlineto{\pgfqpoint{2.059852in}{1.331423in}}%
\pgfpathlineto{\pgfqpoint{2.063682in}{1.333203in}}%
\pgfpathlineto{\pgfqpoint{2.077088in}{1.334794in}}%
\pgfpathlineto{\pgfqpoint{2.082834in}{1.335836in}}%
\pgfpathlineto{\pgfqpoint{2.086664in}{1.337222in}}%
\pgfpathlineto{\pgfqpoint{2.088579in}{1.339358in}}%
\pgfpathlineto{\pgfqpoint{2.100070in}{1.340897in}}%
\pgfpathlineto{\pgfqpoint{2.103900in}{1.341009in}}%
\pgfpathlineto{\pgfqpoint{2.107731in}{1.345043in}}%
\pgfpathlineto{\pgfqpoint{2.111561in}{1.346544in}}%
\pgfpathlineto{\pgfqpoint{2.132628in}{1.353093in}}%
\pgfpathlineto{\pgfqpoint{2.134543in}{1.354830in}}%
\pgfpathlineto{\pgfqpoint{2.140289in}{1.355716in}}%
\pgfpathlineto{\pgfqpoint{2.144119in}{1.357213in}}%
\pgfpathlineto{\pgfqpoint{2.151780in}{1.358472in}}%
\pgfpathlineto{\pgfqpoint{2.155610in}{1.359652in}}%
\pgfpathlineto{\pgfqpoint{2.157525in}{1.359983in}}%
\pgfpathlineto{\pgfqpoint{2.159440in}{1.362600in}}%
\pgfpathlineto{\pgfqpoint{2.207320in}{1.374146in}}%
\pgfpathlineto{\pgfqpoint{2.216896in}{1.377913in}}%
\pgfpathlineto{\pgfqpoint{2.220726in}{1.378002in}}%
\pgfpathlineto{\pgfqpoint{2.222641in}{1.380086in}}%
\pgfpathlineto{\pgfqpoint{2.236047in}{1.382030in}}%
\pgfpathlineto{\pgfqpoint{2.241793in}{1.384763in}}%
\pgfpathlineto{\pgfqpoint{2.243708in}{1.387612in}}%
\pgfpathlineto{\pgfqpoint{2.249453in}{1.389313in}}%
\pgfpathlineto{\pgfqpoint{2.253284in}{1.390935in}}%
\pgfpathlineto{\pgfqpoint{2.260945in}{1.395304in}}%
\pgfpathlineto{\pgfqpoint{2.262860in}{1.398980in}}%
\pgfpathlineto{\pgfqpoint{2.266690in}{1.400113in}}%
\pgfpathlineto{\pgfqpoint{2.268605in}{1.400377in}}%
\pgfpathlineto{\pgfqpoint{2.272436in}{1.403796in}}%
\pgfpathlineto{\pgfqpoint{2.274351in}{1.403902in}}%
\pgfpathlineto{\pgfqpoint{2.276266in}{1.407289in}}%
\pgfpathlineto{\pgfqpoint{2.287757in}{1.409395in}}%
\pgfpathlineto{\pgfqpoint{2.289672in}{1.410866in}}%
\pgfpathlineto{\pgfqpoint{2.291587in}{1.410933in}}%
\pgfpathlineto{\pgfqpoint{2.293502in}{1.413835in}}%
\pgfpathlineto{\pgfqpoint{2.299248in}{1.414269in}}%
\pgfpathlineto{\pgfqpoint{2.304993in}{1.418648in}}%
\pgfpathlineto{\pgfqpoint{2.310739in}{1.419236in}}%
\pgfpathlineto{\pgfqpoint{2.312654in}{1.421411in}}%
\pgfpathlineto{\pgfqpoint{2.316484in}{1.422608in}}%
\pgfpathlineto{\pgfqpoint{2.320315in}{1.424557in}}%
\pgfpathlineto{\pgfqpoint{2.329891in}{1.429193in}}%
\pgfpathlineto{\pgfqpoint{2.331806in}{1.432102in}}%
\pgfpathlineto{\pgfqpoint{2.335636in}{1.433470in}}%
\pgfpathlineto{\pgfqpoint{2.339467in}{1.437411in}}%
\pgfpathlineto{\pgfqpoint{2.358618in}{1.441416in}}%
\pgfpathlineto{\pgfqpoint{2.362449in}{1.444526in}}%
\pgfpathlineto{\pgfqpoint{2.366279in}{1.447267in}}%
\pgfpathlineto{\pgfqpoint{2.370109in}{1.448358in}}%
\pgfpathlineto{\pgfqpoint{2.373940in}{1.449298in}}%
\pgfpathlineto{\pgfqpoint{2.379685in}{1.450516in}}%
\pgfpathlineto{\pgfqpoint{2.395006in}{1.452221in}}%
\pgfpathlineto{\pgfqpoint{2.410328in}{1.458475in}}%
\pgfpathlineto{\pgfqpoint{2.414158in}{1.459391in}}%
\pgfpathlineto{\pgfqpoint{2.427564in}{1.465748in}}%
\pgfpathlineto{\pgfqpoint{2.433310in}{1.466424in}}%
\pgfpathlineto{\pgfqpoint{2.437140in}{1.471209in}}%
\pgfpathlineto{\pgfqpoint{2.444801in}{1.478878in}}%
\pgfpathlineto{\pgfqpoint{2.448631in}{1.479727in}}%
\pgfpathlineto{\pgfqpoint{2.454377in}{1.483926in}}%
\pgfpathlineto{\pgfqpoint{2.458207in}{1.485287in}}%
\pgfpathlineto{\pgfqpoint{2.463953in}{1.486920in}}%
\pgfpathlineto{\pgfqpoint{2.467783in}{1.488548in}}%
\pgfpathlineto{\pgfqpoint{2.481189in}{1.497385in}}%
\pgfpathlineto{\pgfqpoint{2.483104in}{1.499642in}}%
\pgfpathlineto{\pgfqpoint{2.488850in}{1.501012in}}%
\pgfpathlineto{\pgfqpoint{2.500341in}{1.504464in}}%
\pgfpathlineto{\pgfqpoint{2.504171in}{1.506477in}}%
\pgfpathlineto{\pgfqpoint{2.506086in}{1.512215in}}%
\pgfpathlineto{\pgfqpoint{2.509917in}{1.513305in}}%
\pgfpathlineto{\pgfqpoint{2.511832in}{1.513716in}}%
\pgfpathlineto{\pgfqpoint{2.513747in}{1.517236in}}%
\pgfpathlineto{\pgfqpoint{2.517577in}{1.518534in}}%
\pgfpathlineto{\pgfqpoint{2.519493in}{1.521075in}}%
\pgfpathlineto{\pgfqpoint{2.521408in}{1.521471in}}%
\pgfpathlineto{\pgfqpoint{2.525238in}{1.523827in}}%
\pgfpathlineto{\pgfqpoint{2.542475in}{1.527227in}}%
\pgfpathlineto{\pgfqpoint{2.550135in}{1.531370in}}%
\pgfpathlineto{\pgfqpoint{2.557796in}{1.534325in}}%
\pgfpathlineto{\pgfqpoint{2.561626in}{1.535827in}}%
\pgfpathlineto{\pgfqpoint{2.567372in}{1.539053in}}%
\pgfpathlineto{\pgfqpoint{2.569287in}{1.542904in}}%
\pgfpathlineto{\pgfqpoint{2.575033in}{1.544833in}}%
\pgfpathlineto{\pgfqpoint{2.580778in}{1.546787in}}%
\pgfpathlineto{\pgfqpoint{2.592269in}{1.551344in}}%
\pgfpathlineto{\pgfqpoint{2.594184in}{1.553310in}}%
\pgfpathlineto{\pgfqpoint{2.596099in}{1.553663in}}%
\pgfpathlineto{\pgfqpoint{2.598015in}{1.555244in}}%
\pgfpathlineto{\pgfqpoint{2.601845in}{1.560810in}}%
\pgfpathlineto{\pgfqpoint{2.603760in}{1.562633in}}%
\pgfpathlineto{\pgfqpoint{2.605675in}{1.562837in}}%
\pgfpathlineto{\pgfqpoint{2.611421in}{1.566155in}}%
\pgfpathlineto{\pgfqpoint{2.626742in}{1.570817in}}%
\pgfpathlineto{\pgfqpoint{2.630573in}{1.571164in}}%
\pgfpathlineto{\pgfqpoint{2.638233in}{1.574685in}}%
\pgfpathlineto{\pgfqpoint{2.642064in}{1.579641in}}%
\pgfpathlineto{\pgfqpoint{2.647809in}{1.581096in}}%
\pgfpathlineto{\pgfqpoint{2.649724in}{1.581447in}}%
\pgfpathlineto{\pgfqpoint{2.653555in}{1.584317in}}%
\pgfpathlineto{\pgfqpoint{2.663130in}{1.586069in}}%
\pgfpathlineto{\pgfqpoint{2.666961in}{1.587435in}}%
\pgfpathlineto{\pgfqpoint{2.676537in}{1.590321in}}%
\pgfpathlineto{\pgfqpoint{2.678452in}{1.593693in}}%
\pgfpathlineto{\pgfqpoint{2.682282in}{1.594974in}}%
\pgfpathlineto{\pgfqpoint{2.691858in}{1.597882in}}%
\pgfpathlineto{\pgfqpoint{2.701434in}{1.600655in}}%
\pgfpathlineto{\pgfqpoint{2.703349in}{1.600878in}}%
\pgfpathlineto{\pgfqpoint{2.707179in}{1.602882in}}%
\pgfpathlineto{\pgfqpoint{2.714840in}{1.606766in}}%
\pgfpathlineto{\pgfqpoint{2.716755in}{1.610526in}}%
\pgfpathlineto{\pgfqpoint{2.724416in}{1.612056in}}%
\pgfpathlineto{\pgfqpoint{2.726331in}{1.612349in}}%
\pgfpathlineto{\pgfqpoint{2.728246in}{1.614897in}}%
\pgfpathlineto{\pgfqpoint{2.739737in}{1.616467in}}%
\pgfpathlineto{\pgfqpoint{2.741653in}{1.618264in}}%
\pgfpathlineto{\pgfqpoint{2.749313in}{1.619465in}}%
\pgfpathlineto{\pgfqpoint{2.755059in}{1.621218in}}%
\pgfpathlineto{\pgfqpoint{2.758889in}{1.622390in}}%
\pgfpathlineto{\pgfqpoint{2.762719in}{1.623297in}}%
\pgfpathlineto{\pgfqpoint{2.764635in}{1.625681in}}%
\pgfpathlineto{\pgfqpoint{2.774210in}{1.626737in}}%
\pgfpathlineto{\pgfqpoint{2.781871in}{1.628336in}}%
\pgfpathlineto{\pgfqpoint{2.785701in}{1.630408in}}%
\pgfpathlineto{\pgfqpoint{2.789532in}{1.634811in}}%
\pgfpathlineto{\pgfqpoint{2.793362in}{1.635914in}}%
\pgfpathlineto{\pgfqpoint{2.804853in}{1.639762in}}%
\pgfpathlineto{\pgfqpoint{2.806768in}{1.642283in}}%
\pgfpathlineto{\pgfqpoint{2.810599in}{1.643193in}}%
\pgfpathlineto{\pgfqpoint{2.818259in}{1.645990in}}%
\pgfpathlineto{\pgfqpoint{2.824005in}{1.647253in}}%
\pgfpathlineto{\pgfqpoint{2.825920in}{1.647678in}}%
\pgfpathlineto{\pgfqpoint{2.827835in}{1.651306in}}%
\pgfpathlineto{\pgfqpoint{2.843157in}{1.658318in}}%
\pgfpathlineto{\pgfqpoint{2.846987in}{1.662826in}}%
\pgfpathlineto{\pgfqpoint{2.852732in}{1.665312in}}%
\pgfpathlineto{\pgfqpoint{2.856563in}{1.666470in}}%
\pgfpathlineto{\pgfqpoint{2.862308in}{1.668568in}}%
\pgfpathlineto{\pgfqpoint{2.864223in}{1.669565in}}%
\pgfpathlineto{\pgfqpoint{2.868054in}{1.673167in}}%
\pgfpathlineto{\pgfqpoint{2.875715in}{1.674294in}}%
\pgfpathlineto{\pgfqpoint{2.877630in}{1.676224in}}%
\pgfpathlineto{\pgfqpoint{2.881460in}{1.676644in}}%
\pgfpathlineto{\pgfqpoint{2.898697in}{1.684935in}}%
\pgfpathlineto{\pgfqpoint{2.904442in}{1.692800in}}%
\pgfpathlineto{\pgfqpoint{2.912103in}{1.693895in}}%
\pgfpathlineto{\pgfqpoint{2.921679in}{1.696221in}}%
\pgfpathlineto{\pgfqpoint{2.927424in}{1.697790in}}%
\pgfpathlineto{\pgfqpoint{2.929339in}{1.702518in}}%
\pgfpathlineto{\pgfqpoint{2.942746in}{1.712964in}}%
\pgfpathlineto{\pgfqpoint{2.946576in}{1.714324in}}%
\pgfpathlineto{\pgfqpoint{2.950406in}{1.716044in}}%
\pgfpathlineto{\pgfqpoint{2.958067in}{1.717145in}}%
\pgfpathlineto{\pgfqpoint{2.961897in}{1.722932in}}%
\pgfpathlineto{\pgfqpoint{2.971473in}{1.725541in}}%
\pgfpathlineto{\pgfqpoint{2.975303in}{1.730058in}}%
\pgfpathlineto{\pgfqpoint{2.986794in}{1.737420in}}%
\pgfpathlineto{\pgfqpoint{2.992540in}{1.744238in}}%
\pgfpathlineto{\pgfqpoint{2.998285in}{1.760158in}}%
\pgfpathlineto{\pgfqpoint{3.000201in}{1.763486in}}%
\pgfpathlineto{\pgfqpoint{3.002116in}{1.763574in}}%
\pgfpathlineto{\pgfqpoint{3.013607in}{1.800754in}}%
\pgfpathlineto{\pgfqpoint{3.015522in}{1.819575in}}%
\pgfpathlineto{\pgfqpoint{3.017437in}{1.826535in}}%
\pgfpathlineto{\pgfqpoint{3.017437in}{1.826535in}}%
\pgfusepath{stroke}%
\end{pgfscope}%
\begin{pgfscope}%
\pgfpathrectangle{\pgfqpoint{0.694334in}{0.523557in}}{\pgfqpoint{3.830343in}{1.302977in}}%
\pgfusepath{clip}%
\pgfsetbuttcap%
\pgfsetroundjoin%
\pgfsetlinewidth{1.003750pt}%
\definecolor{currentstroke}{rgb}{0.941176,0.627451,0.188235}%
\pgfsetstrokecolor{currentstroke}%
\pgfsetdash{{1.000000pt}{1.650000pt}}{0.000000pt}%
\pgfpathmoveto{\pgfqpoint{0.694334in}{0.567141in}}%
\pgfpathlineto{\pgfqpoint{0.696249in}{0.582562in}}%
\pgfpathlineto{\pgfqpoint{0.698165in}{0.582755in}}%
\pgfpathlineto{\pgfqpoint{0.701995in}{0.599148in}}%
\pgfpathlineto{\pgfqpoint{0.703910in}{0.606735in}}%
\pgfpathlineto{\pgfqpoint{0.711571in}{0.615604in}}%
\pgfpathlineto{\pgfqpoint{0.713486in}{0.616358in}}%
\pgfpathlineto{\pgfqpoint{0.715401in}{0.627746in}}%
\pgfpathlineto{\pgfqpoint{0.719232in}{0.630247in}}%
\pgfpathlineto{\pgfqpoint{0.721147in}{0.632363in}}%
\pgfpathlineto{\pgfqpoint{0.724977in}{0.633447in}}%
\pgfpathlineto{\pgfqpoint{0.726892in}{0.635337in}}%
\pgfpathlineto{\pgfqpoint{0.732638in}{0.646877in}}%
\pgfpathlineto{\pgfqpoint{0.736468in}{0.649507in}}%
\pgfpathlineto{\pgfqpoint{0.738383in}{0.655445in}}%
\pgfpathlineto{\pgfqpoint{0.740298in}{0.655930in}}%
\pgfpathlineto{\pgfqpoint{0.747959in}{0.671221in}}%
\pgfpathlineto{\pgfqpoint{0.749874in}{0.675620in}}%
\pgfpathlineto{\pgfqpoint{0.751789in}{0.675686in}}%
\pgfpathlineto{\pgfqpoint{0.753705in}{0.678423in}}%
\pgfpathlineto{\pgfqpoint{0.759450in}{0.690456in}}%
\pgfpathlineto{\pgfqpoint{0.770941in}{0.700710in}}%
\pgfpathlineto{\pgfqpoint{0.774772in}{0.703735in}}%
\pgfpathlineto{\pgfqpoint{0.801584in}{0.709532in}}%
\pgfpathlineto{\pgfqpoint{0.805414in}{0.711596in}}%
\pgfpathlineto{\pgfqpoint{0.814990in}{0.713390in}}%
\pgfpathlineto{\pgfqpoint{0.820736in}{0.714546in}}%
\pgfpathlineto{\pgfqpoint{0.836057in}{0.716555in}}%
\pgfpathlineto{\pgfqpoint{0.843718in}{0.719765in}}%
\pgfpathlineto{\pgfqpoint{0.855209in}{0.720566in}}%
\pgfpathlineto{\pgfqpoint{0.857124in}{0.720645in}}%
\pgfpathlineto{\pgfqpoint{0.860954in}{0.722678in}}%
\pgfpathlineto{\pgfqpoint{0.870530in}{0.723679in}}%
\pgfpathlineto{\pgfqpoint{0.878191in}{0.724279in}}%
\pgfpathlineto{\pgfqpoint{0.908834in}{0.725460in}}%
\pgfpathlineto{\pgfqpoint{0.920325in}{0.726063in}}%
\pgfpathlineto{\pgfqpoint{0.945222in}{0.727385in}}%
\pgfpathlineto{\pgfqpoint{0.972034in}{0.728886in}}%
\pgfpathlineto{\pgfqpoint{1.044811in}{0.732918in}}%
\pgfpathlineto{\pgfqpoint{1.083114in}{0.735423in}}%
\pgfpathlineto{\pgfqpoint{1.100351in}{0.737113in}}%
\pgfpathlineto{\pgfqpoint{1.192279in}{0.741689in}}%
\pgfpathlineto{\pgfqpoint{1.213346in}{0.742889in}}%
\pgfpathlineto{\pgfqpoint{1.236328in}{0.744723in}}%
\pgfpathlineto{\pgfqpoint{1.240158in}{0.745473in}}%
\pgfpathlineto{\pgfqpoint{1.284207in}{0.747171in}}%
\pgfpathlineto{\pgfqpoint{1.414439in}{0.759272in}}%
\pgfpathlineto{\pgfqpoint{1.420184in}{0.760773in}}%
\pgfpathlineto{\pgfqpoint{1.431675in}{0.762508in}}%
\pgfpathlineto{\pgfqpoint{1.441251in}{0.763654in}}%
\pgfpathlineto{\pgfqpoint{1.445081in}{0.765852in}}%
\pgfpathlineto{\pgfqpoint{1.450827in}{0.766734in}}%
\pgfpathlineto{\pgfqpoint{1.462318in}{0.771921in}}%
\pgfpathlineto{\pgfqpoint{1.466148in}{0.772684in}}%
\pgfpathlineto{\pgfqpoint{1.475724in}{0.780723in}}%
\pgfpathlineto{\pgfqpoint{1.477639in}{0.784911in}}%
\pgfpathlineto{\pgfqpoint{1.481470in}{0.785898in}}%
\pgfpathlineto{\pgfqpoint{1.483385in}{0.788811in}}%
\pgfpathlineto{\pgfqpoint{1.492961in}{0.790671in}}%
\pgfpathlineto{\pgfqpoint{1.510197in}{0.798592in}}%
\pgfpathlineto{\pgfqpoint{1.521688in}{0.799587in}}%
\pgfpathlineto{\pgfqpoint{1.527434in}{0.802172in}}%
\pgfpathlineto{\pgfqpoint{1.540840in}{0.804764in}}%
\pgfpathlineto{\pgfqpoint{1.544670in}{0.806461in}}%
\pgfpathlineto{\pgfqpoint{1.550416in}{0.807943in}}%
\pgfpathlineto{\pgfqpoint{1.579143in}{0.812868in}}%
\pgfpathlineto{\pgfqpoint{1.582974in}{0.815132in}}%
\pgfpathlineto{\pgfqpoint{1.584889in}{0.815211in}}%
\pgfpathlineto{\pgfqpoint{1.588719in}{0.816491in}}%
\pgfpathlineto{\pgfqpoint{1.596380in}{0.818241in}}%
\pgfpathlineto{\pgfqpoint{1.615532in}{0.820945in}}%
\pgfpathlineto{\pgfqpoint{1.623192in}{0.822119in}}%
\pgfpathlineto{\pgfqpoint{1.634683in}{0.824956in}}%
\pgfpathlineto{\pgfqpoint{1.650005in}{0.827186in}}%
\pgfpathlineto{\pgfqpoint{1.661496in}{0.828179in}}%
\pgfpathlineto{\pgfqpoint{1.680648in}{0.832850in}}%
\pgfpathlineto{\pgfqpoint{1.695969in}{0.834803in}}%
\pgfpathlineto{\pgfqpoint{1.701714in}{0.836460in}}%
\pgfpathlineto{\pgfqpoint{1.718951in}{0.838267in}}%
\pgfpathlineto{\pgfqpoint{1.724697in}{0.841013in}}%
\pgfpathlineto{\pgfqpoint{1.730442in}{0.841765in}}%
\pgfpathlineto{\pgfqpoint{1.738103in}{0.843347in}}%
\pgfpathlineto{\pgfqpoint{1.766830in}{0.847835in}}%
\pgfpathlineto{\pgfqpoint{1.768745in}{0.848125in}}%
\pgfpathlineto{\pgfqpoint{1.772576in}{0.850816in}}%
\pgfpathlineto{\pgfqpoint{1.778321in}{0.852136in}}%
\pgfpathlineto{\pgfqpoint{1.789812in}{0.855151in}}%
\pgfpathlineto{\pgfqpoint{1.793643in}{0.858744in}}%
\pgfpathlineto{\pgfqpoint{1.799388in}{0.860410in}}%
\pgfpathlineto{\pgfqpoint{1.808964in}{0.861690in}}%
\pgfpathlineto{\pgfqpoint{1.897062in}{0.876194in}}%
\pgfpathlineto{\pgfqpoint{1.902807in}{0.879436in}}%
\pgfpathlineto{\pgfqpoint{1.962178in}{0.889986in}}%
\pgfpathlineto{\pgfqpoint{1.998566in}{0.895279in}}%
\pgfpathlineto{\pgfqpoint{2.002396in}{0.897136in}}%
\pgfpathlineto{\pgfqpoint{2.019633in}{0.899457in}}%
\pgfpathlineto{\pgfqpoint{2.023463in}{0.900421in}}%
\pgfpathlineto{\pgfqpoint{2.046445in}{0.903270in}}%
\pgfpathlineto{\pgfqpoint{2.063682in}{0.905826in}}%
\pgfpathlineto{\pgfqpoint{2.077088in}{0.907498in}}%
\pgfpathlineto{\pgfqpoint{2.092409in}{0.909303in}}%
\pgfpathlineto{\pgfqpoint{2.121137in}{0.912436in}}%
\pgfpathlineto{\pgfqpoint{2.124967in}{0.913570in}}%
\pgfpathlineto{\pgfqpoint{2.136458in}{0.915106in}}%
\pgfpathlineto{\pgfqpoint{2.138374in}{0.916643in}}%
\pgfpathlineto{\pgfqpoint{2.144119in}{0.917605in}}%
\pgfpathlineto{\pgfqpoint{2.149865in}{0.918496in}}%
\pgfpathlineto{\pgfqpoint{2.151780in}{0.918737in}}%
\pgfpathlineto{\pgfqpoint{2.153695in}{0.920150in}}%
\pgfpathlineto{\pgfqpoint{2.163271in}{0.920991in}}%
\pgfpathlineto{\pgfqpoint{2.214980in}{0.931176in}}%
\pgfpathlineto{\pgfqpoint{2.224556in}{0.931968in}}%
\pgfpathlineto{\pgfqpoint{2.236047in}{0.935912in}}%
\pgfpathlineto{\pgfqpoint{2.243708in}{0.936754in}}%
\pgfpathlineto{\pgfqpoint{2.280096in}{0.941676in}}%
\pgfpathlineto{\pgfqpoint{2.289672in}{0.942902in}}%
\pgfpathlineto{\pgfqpoint{2.316484in}{0.945052in}}%
\pgfpathlineto{\pgfqpoint{2.329891in}{0.945964in}}%
\pgfpathlineto{\pgfqpoint{2.335636in}{0.947179in}}%
\pgfpathlineto{\pgfqpoint{2.358618in}{0.948809in}}%
\pgfpathlineto{\pgfqpoint{2.366279in}{0.950171in}}%
\pgfpathlineto{\pgfqpoint{2.373940in}{0.951947in}}%
\pgfpathlineto{\pgfqpoint{2.377770in}{0.954446in}}%
\pgfpathlineto{\pgfqpoint{2.389261in}{0.956433in}}%
\pgfpathlineto{\pgfqpoint{2.391176in}{0.957190in}}%
\pgfpathlineto{\pgfqpoint{2.395006in}{0.960086in}}%
\pgfpathlineto{\pgfqpoint{2.406498in}{0.961412in}}%
\pgfpathlineto{\pgfqpoint{2.410328in}{0.963197in}}%
\pgfpathlineto{\pgfqpoint{2.419904in}{0.964791in}}%
\pgfpathlineto{\pgfqpoint{2.423734in}{0.967067in}}%
\pgfpathlineto{\pgfqpoint{2.439055in}{0.974701in}}%
\pgfpathlineto{\pgfqpoint{2.450546in}{0.980023in}}%
\pgfpathlineto{\pgfqpoint{2.458207in}{0.982179in}}%
\pgfpathlineto{\pgfqpoint{2.460122in}{0.986058in}}%
\pgfpathlineto{\pgfqpoint{2.465868in}{0.988210in}}%
\pgfpathlineto{\pgfqpoint{2.471613in}{0.989814in}}%
\pgfpathlineto{\pgfqpoint{2.477359in}{0.992420in}}%
\pgfpathlineto{\pgfqpoint{2.486935in}{0.993619in}}%
\pgfpathlineto{\pgfqpoint{2.498426in}{0.997123in}}%
\pgfpathlineto{\pgfqpoint{2.500341in}{0.999870in}}%
\pgfpathlineto{\pgfqpoint{2.509917in}{1.001205in}}%
\pgfpathlineto{\pgfqpoint{2.513747in}{1.003966in}}%
\pgfpathlineto{\pgfqpoint{2.523323in}{1.005179in}}%
\pgfpathlineto{\pgfqpoint{2.529068in}{1.006342in}}%
\pgfpathlineto{\pgfqpoint{2.532899in}{1.007849in}}%
\pgfpathlineto{\pgfqpoint{2.536729in}{1.008569in}}%
\pgfpathlineto{\pgfqpoint{2.557796in}{1.012606in}}%
\pgfpathlineto{\pgfqpoint{2.561626in}{1.012872in}}%
\pgfpathlineto{\pgfqpoint{2.565457in}{1.015743in}}%
\pgfpathlineto{\pgfqpoint{2.569287in}{1.016717in}}%
\pgfpathlineto{\pgfqpoint{2.573117in}{1.018092in}}%
\pgfpathlineto{\pgfqpoint{2.575033in}{1.018320in}}%
\pgfpathlineto{\pgfqpoint{2.576948in}{1.020656in}}%
\pgfpathlineto{\pgfqpoint{2.580778in}{1.021639in}}%
\pgfpathlineto{\pgfqpoint{2.584608in}{1.023294in}}%
\pgfpathlineto{\pgfqpoint{2.590354in}{1.024158in}}%
\pgfpathlineto{\pgfqpoint{2.592269in}{1.026817in}}%
\pgfpathlineto{\pgfqpoint{2.594184in}{1.027094in}}%
\pgfpathlineto{\pgfqpoint{2.596099in}{1.029692in}}%
\pgfpathlineto{\pgfqpoint{2.598015in}{1.029857in}}%
\pgfpathlineto{\pgfqpoint{2.599930in}{1.032096in}}%
\pgfpathlineto{\pgfqpoint{2.605675in}{1.034423in}}%
\pgfpathlineto{\pgfqpoint{2.611421in}{1.039133in}}%
\pgfpathlineto{\pgfqpoint{2.626742in}{1.043293in}}%
\pgfpathlineto{\pgfqpoint{2.628657in}{1.044111in}}%
\pgfpathlineto{\pgfqpoint{2.630573in}{1.046152in}}%
\pgfpathlineto{\pgfqpoint{2.632488in}{1.051263in}}%
\pgfpathlineto{\pgfqpoint{2.642064in}{1.053898in}}%
\pgfpathlineto{\pgfqpoint{2.645894in}{1.058431in}}%
\pgfpathlineto{\pgfqpoint{2.647809in}{1.059291in}}%
\pgfpathlineto{\pgfqpoint{2.651639in}{1.062244in}}%
\pgfpathlineto{\pgfqpoint{2.659300in}{1.066143in}}%
\pgfpathlineto{\pgfqpoint{2.682282in}{1.071259in}}%
\pgfpathlineto{\pgfqpoint{2.684197in}{1.073698in}}%
\pgfpathlineto{\pgfqpoint{2.686113in}{1.074109in}}%
\pgfpathlineto{\pgfqpoint{2.688028in}{1.077390in}}%
\pgfpathlineto{\pgfqpoint{2.691858in}{1.078985in}}%
\pgfpathlineto{\pgfqpoint{2.695688in}{1.081315in}}%
\pgfpathlineto{\pgfqpoint{2.697604in}{1.081341in}}%
\pgfpathlineto{\pgfqpoint{2.699519in}{1.084661in}}%
\pgfpathlineto{\pgfqpoint{2.703349in}{1.086777in}}%
\pgfpathlineto{\pgfqpoint{2.709095in}{1.092806in}}%
\pgfpathlineto{\pgfqpoint{2.712925in}{1.094980in}}%
\pgfpathlineto{\pgfqpoint{2.718670in}{1.108459in}}%
\pgfpathlineto{\pgfqpoint{2.722501in}{1.116280in}}%
\pgfpathlineto{\pgfqpoint{2.724416in}{1.118961in}}%
\pgfpathlineto{\pgfqpoint{2.726331in}{1.119126in}}%
\pgfpathlineto{\pgfqpoint{2.732077in}{1.121740in}}%
\pgfpathlineto{\pgfqpoint{2.737822in}{1.122041in}}%
\pgfpathlineto{\pgfqpoint{2.739737in}{1.128837in}}%
\pgfpathlineto{\pgfqpoint{2.741653in}{1.131149in}}%
\pgfpathlineto{\pgfqpoint{2.743568in}{1.142341in}}%
\pgfpathlineto{\pgfqpoint{2.747398in}{1.143926in}}%
\pgfpathlineto{\pgfqpoint{2.749313in}{1.152235in}}%
\pgfpathlineto{\pgfqpoint{2.753144in}{1.152754in}}%
\pgfpathlineto{\pgfqpoint{2.755059in}{1.163088in}}%
\pgfpathlineto{\pgfqpoint{2.756974in}{1.164581in}}%
\pgfpathlineto{\pgfqpoint{2.760804in}{1.170521in}}%
\pgfpathlineto{\pgfqpoint{2.762719in}{1.178607in}}%
\pgfpathlineto{\pgfqpoint{2.764635in}{1.180313in}}%
\pgfpathlineto{\pgfqpoint{2.768465in}{1.189905in}}%
\pgfpathlineto{\pgfqpoint{2.770380in}{1.191539in}}%
\pgfpathlineto{\pgfqpoint{2.776126in}{1.202914in}}%
\pgfpathlineto{\pgfqpoint{2.779956in}{1.219104in}}%
\pgfpathlineto{\pgfqpoint{2.781871in}{1.223605in}}%
\pgfpathlineto{\pgfqpoint{2.785701in}{1.240588in}}%
\pgfpathlineto{\pgfqpoint{2.787617in}{1.240613in}}%
\pgfpathlineto{\pgfqpoint{2.791447in}{1.244129in}}%
\pgfpathlineto{\pgfqpoint{2.793362in}{1.248530in}}%
\pgfpathlineto{\pgfqpoint{2.795277in}{1.258087in}}%
\pgfpathlineto{\pgfqpoint{2.797192in}{1.276958in}}%
\pgfpathlineto{\pgfqpoint{2.799108in}{1.278513in}}%
\pgfpathlineto{\pgfqpoint{2.802938in}{1.290776in}}%
\pgfpathlineto{\pgfqpoint{2.806768in}{1.339754in}}%
\pgfpathlineto{\pgfqpoint{2.808684in}{1.379698in}}%
\pgfpathlineto{\pgfqpoint{2.810599in}{1.381391in}}%
\pgfpathlineto{\pgfqpoint{2.812514in}{1.385237in}}%
\pgfpathlineto{\pgfqpoint{2.822090in}{1.589779in}}%
\pgfpathlineto{\pgfqpoint{2.825920in}{1.796014in}}%
\pgfpathlineto{\pgfqpoint{2.827835in}{1.826535in}}%
\pgfpathlineto{\pgfqpoint{2.827835in}{1.826535in}}%
\pgfusepath{stroke}%
\end{pgfscope}%
\begin{pgfscope}%
\pgfpathrectangle{\pgfqpoint{0.694334in}{0.523557in}}{\pgfqpoint{3.830343in}{1.302977in}}%
\pgfusepath{clip}%
\pgfsetbuttcap%
\pgfsetroundjoin%
\pgfsetlinewidth{1.003750pt}%
\definecolor{currentstroke}{rgb}{0.062745,0.000000,0.062745}%
\pgfsetstrokecolor{currentstroke}%
\pgfsetdash{{3.700000pt}{1.600000pt}}{0.000000pt}%
\pgfpathmoveto{\pgfqpoint{0.694334in}{0.613292in}}%
\pgfpathlineto{\pgfqpoint{0.696249in}{0.621949in}}%
\pgfpathlineto{\pgfqpoint{0.698165in}{0.622951in}}%
\pgfpathlineto{\pgfqpoint{0.700080in}{0.636570in}}%
\pgfpathlineto{\pgfqpoint{0.703910in}{0.641552in}}%
\pgfpathlineto{\pgfqpoint{0.705825in}{0.649251in}}%
\pgfpathlineto{\pgfqpoint{0.711571in}{0.652708in}}%
\pgfpathlineto{\pgfqpoint{0.713486in}{0.652973in}}%
\pgfpathlineto{\pgfqpoint{0.715401in}{0.660913in}}%
\pgfpathlineto{\pgfqpoint{0.721147in}{0.664229in}}%
\pgfpathlineto{\pgfqpoint{0.723062in}{0.668212in}}%
\pgfpathlineto{\pgfqpoint{0.724977in}{0.668256in}}%
\pgfpathlineto{\pgfqpoint{0.730723in}{0.673020in}}%
\pgfpathlineto{\pgfqpoint{0.734553in}{0.683462in}}%
\pgfpathlineto{\pgfqpoint{0.742214in}{0.687870in}}%
\pgfpathlineto{\pgfqpoint{0.744129in}{0.690797in}}%
\pgfpathlineto{\pgfqpoint{0.753705in}{0.693659in}}%
\pgfpathlineto{\pgfqpoint{0.755620in}{0.697917in}}%
\pgfpathlineto{\pgfqpoint{0.757535in}{0.699322in}}%
\pgfpathlineto{\pgfqpoint{0.759450in}{0.711299in}}%
\pgfpathlineto{\pgfqpoint{0.774772in}{0.714803in}}%
\pgfpathlineto{\pgfqpoint{0.782432in}{0.715851in}}%
\pgfpathlineto{\pgfqpoint{0.803499in}{0.718816in}}%
\pgfpathlineto{\pgfqpoint{0.813075in}{0.719973in}}%
\pgfpathlineto{\pgfqpoint{0.836057in}{0.724418in}}%
\pgfpathlineto{\pgfqpoint{0.843718in}{0.725105in}}%
\pgfpathlineto{\pgfqpoint{0.864785in}{0.726369in}}%
\pgfpathlineto{\pgfqpoint{0.870530in}{0.728363in}}%
\pgfpathlineto{\pgfqpoint{0.885851in}{0.729985in}}%
\pgfpathlineto{\pgfqpoint{0.935646in}{0.733917in}}%
\pgfpathlineto{\pgfqpoint{0.966289in}{0.735553in}}%
\pgfpathlineto{\pgfqpoint{0.970119in}{0.736329in}}%
\pgfpathlineto{\pgfqpoint{1.002677in}{0.738756in}}%
\pgfpathlineto{\pgfqpoint{1.027574in}{0.739852in}}%
\pgfpathlineto{\pgfqpoint{1.040980in}{0.740505in}}%
\pgfpathlineto{\pgfqpoint{1.104181in}{0.744093in}}%
\pgfpathlineto{\pgfqpoint{1.303359in}{0.758814in}}%
\pgfpathlineto{\pgfqpoint{1.332086in}{0.760504in}}%
\pgfpathlineto{\pgfqpoint{1.347408in}{0.761997in}}%
\pgfpathlineto{\pgfqpoint{1.358899in}{0.762976in}}%
\pgfpathlineto{\pgfqpoint{1.410608in}{0.769284in}}%
\pgfpathlineto{\pgfqpoint{1.420184in}{0.773766in}}%
\pgfpathlineto{\pgfqpoint{1.424015in}{0.775241in}}%
\pgfpathlineto{\pgfqpoint{1.431675in}{0.776193in}}%
\pgfpathlineto{\pgfqpoint{1.435506in}{0.777845in}}%
\pgfpathlineto{\pgfqpoint{1.439336in}{0.778714in}}%
\pgfpathlineto{\pgfqpoint{1.441251in}{0.781600in}}%
\pgfpathlineto{\pgfqpoint{1.443166in}{0.781687in}}%
\pgfpathlineto{\pgfqpoint{1.445081in}{0.783849in}}%
\pgfpathlineto{\pgfqpoint{1.446997in}{0.784167in}}%
\pgfpathlineto{\pgfqpoint{1.448912in}{0.786177in}}%
\pgfpathlineto{\pgfqpoint{1.462318in}{0.789214in}}%
\pgfpathlineto{\pgfqpoint{1.468064in}{0.790727in}}%
\pgfpathlineto{\pgfqpoint{1.471894in}{0.791007in}}%
\pgfpathlineto{\pgfqpoint{1.473809in}{0.793216in}}%
\pgfpathlineto{\pgfqpoint{1.477639in}{0.795139in}}%
\pgfpathlineto{\pgfqpoint{1.479555in}{0.803461in}}%
\pgfpathlineto{\pgfqpoint{1.483385in}{0.806286in}}%
\pgfpathlineto{\pgfqpoint{1.498706in}{0.812237in}}%
\pgfpathlineto{\pgfqpoint{1.502537in}{0.812525in}}%
\pgfpathlineto{\pgfqpoint{1.506367in}{0.814271in}}%
\pgfpathlineto{\pgfqpoint{1.512112in}{0.814900in}}%
\pgfpathlineto{\pgfqpoint{1.517858in}{0.816308in}}%
\pgfpathlineto{\pgfqpoint{1.523604in}{0.817857in}}%
\pgfpathlineto{\pgfqpoint{1.525519in}{0.820031in}}%
\pgfpathlineto{\pgfqpoint{1.542755in}{0.823617in}}%
\pgfpathlineto{\pgfqpoint{1.586804in}{0.833309in}}%
\pgfpathlineto{\pgfqpoint{1.602126in}{0.834527in}}%
\pgfpathlineto{\pgfqpoint{1.619362in}{0.837079in}}%
\pgfpathlineto{\pgfqpoint{1.650005in}{0.842212in}}%
\pgfpathlineto{\pgfqpoint{1.657666in}{0.843519in}}%
\pgfpathlineto{\pgfqpoint{1.665326in}{0.845086in}}%
\pgfpathlineto{\pgfqpoint{1.672987in}{0.845492in}}%
\pgfpathlineto{\pgfqpoint{1.676817in}{0.846960in}}%
\pgfpathlineto{\pgfqpoint{1.690223in}{0.848621in}}%
\pgfpathlineto{\pgfqpoint{1.699799in}{0.850941in}}%
\pgfpathlineto{\pgfqpoint{1.711290in}{0.852917in}}%
\pgfpathlineto{\pgfqpoint{1.720866in}{0.855801in}}%
\pgfpathlineto{\pgfqpoint{1.730442in}{0.856953in}}%
\pgfpathlineto{\pgfqpoint{1.768745in}{0.864715in}}%
\pgfpathlineto{\pgfqpoint{1.776406in}{0.866159in}}%
\pgfpathlineto{\pgfqpoint{1.822370in}{0.871388in}}%
\pgfpathlineto{\pgfqpoint{1.824285in}{0.873073in}}%
\pgfpathlineto{\pgfqpoint{1.830031in}{0.874370in}}%
\pgfpathlineto{\pgfqpoint{1.833861in}{0.876147in}}%
\pgfpathlineto{\pgfqpoint{1.877910in}{0.882911in}}%
\pgfpathlineto{\pgfqpoint{1.883656in}{0.885176in}}%
\pgfpathlineto{\pgfqpoint{1.895147in}{0.887468in}}%
\pgfpathlineto{\pgfqpoint{1.900892in}{0.888286in}}%
\pgfpathlineto{\pgfqpoint{1.923874in}{0.892108in}}%
\pgfpathlineto{\pgfqpoint{1.927705in}{0.893412in}}%
\pgfpathlineto{\pgfqpoint{1.937281in}{0.894443in}}%
\pgfpathlineto{\pgfqpoint{1.950687in}{0.895245in}}%
\pgfpathlineto{\pgfqpoint{1.954517in}{0.896470in}}%
\pgfpathlineto{\pgfqpoint{1.958347in}{0.897567in}}%
\pgfpathlineto{\pgfqpoint{1.966008in}{0.899326in}}%
\pgfpathlineto{\pgfqpoint{1.994736in}{0.904957in}}%
\pgfpathlineto{\pgfqpoint{2.006227in}{0.906883in}}%
\pgfpathlineto{\pgfqpoint{2.011972in}{0.907787in}}%
\pgfpathlineto{\pgfqpoint{2.021548in}{0.909620in}}%
\pgfpathlineto{\pgfqpoint{2.025378in}{0.910426in}}%
\pgfpathlineto{\pgfqpoint{2.040700in}{0.911965in}}%
\pgfpathlineto{\pgfqpoint{2.088579in}{0.921193in}}%
\pgfpathlineto{\pgfqpoint{2.094325in}{0.923094in}}%
\pgfpathlineto{\pgfqpoint{2.107731in}{0.924941in}}%
\pgfpathlineto{\pgfqpoint{2.115391in}{0.925691in}}%
\pgfpathlineto{\pgfqpoint{2.121137in}{0.926982in}}%
\pgfpathlineto{\pgfqpoint{2.123052in}{0.927051in}}%
\pgfpathlineto{\pgfqpoint{2.126883in}{0.928355in}}%
\pgfpathlineto{\pgfqpoint{2.165186in}{0.934156in}}%
\pgfpathlineto{\pgfqpoint{2.169016in}{0.935670in}}%
\pgfpathlineto{\pgfqpoint{2.209235in}{0.943307in}}%
\pgfpathlineto{\pgfqpoint{2.213065in}{0.944656in}}%
\pgfpathlineto{\pgfqpoint{2.283927in}{0.953096in}}%
\pgfpathlineto{\pgfqpoint{2.295418in}{0.956536in}}%
\pgfpathlineto{\pgfqpoint{2.312654in}{0.959066in}}%
\pgfpathlineto{\pgfqpoint{2.318400in}{0.960021in}}%
\pgfpathlineto{\pgfqpoint{2.322230in}{0.961899in}}%
\pgfpathlineto{\pgfqpoint{2.326060in}{0.962330in}}%
\pgfpathlineto{\pgfqpoint{2.327975in}{0.964487in}}%
\pgfpathlineto{\pgfqpoint{2.343297in}{0.967160in}}%
\pgfpathlineto{\pgfqpoint{2.352873in}{0.970888in}}%
\pgfpathlineto{\pgfqpoint{2.354788in}{0.971079in}}%
\pgfpathlineto{\pgfqpoint{2.356703in}{0.972769in}}%
\pgfpathlineto{\pgfqpoint{2.360533in}{0.973870in}}%
\pgfpathlineto{\pgfqpoint{2.368194in}{0.975749in}}%
\pgfpathlineto{\pgfqpoint{2.373940in}{0.978564in}}%
\pgfpathlineto{\pgfqpoint{2.385431in}{0.981230in}}%
\pgfpathlineto{\pgfqpoint{2.389261in}{0.982244in}}%
\pgfpathlineto{\pgfqpoint{2.400752in}{0.987375in}}%
\pgfpathlineto{\pgfqpoint{2.406498in}{0.989127in}}%
\pgfpathlineto{\pgfqpoint{2.435225in}{0.998211in}}%
\pgfpathlineto{\pgfqpoint{2.442886in}{0.999568in}}%
\pgfpathlineto{\pgfqpoint{2.446716in}{1.000610in}}%
\pgfpathlineto{\pgfqpoint{2.454377in}{1.001857in}}%
\pgfpathlineto{\pgfqpoint{2.467783in}{1.008305in}}%
\pgfpathlineto{\pgfqpoint{2.473529in}{1.008820in}}%
\pgfpathlineto{\pgfqpoint{2.479274in}{1.011252in}}%
\pgfpathlineto{\pgfqpoint{2.486935in}{1.011900in}}%
\pgfpathlineto{\pgfqpoint{2.490765in}{1.013914in}}%
\pgfpathlineto{\pgfqpoint{2.496511in}{1.014687in}}%
\pgfpathlineto{\pgfqpoint{2.500341in}{1.015832in}}%
\pgfpathlineto{\pgfqpoint{2.515662in}{1.018394in}}%
\pgfpathlineto{\pgfqpoint{2.517577in}{1.018503in}}%
\pgfpathlineto{\pgfqpoint{2.523323in}{1.021593in}}%
\pgfpathlineto{\pgfqpoint{2.529068in}{1.022097in}}%
\pgfpathlineto{\pgfqpoint{2.530984in}{1.024576in}}%
\pgfpathlineto{\pgfqpoint{2.534814in}{1.026401in}}%
\pgfpathlineto{\pgfqpoint{2.536729in}{1.029089in}}%
\pgfpathlineto{\pgfqpoint{2.542475in}{1.029334in}}%
\pgfpathlineto{\pgfqpoint{2.546305in}{1.031708in}}%
\pgfpathlineto{\pgfqpoint{2.553966in}{1.034094in}}%
\pgfpathlineto{\pgfqpoint{2.557796in}{1.034626in}}%
\pgfpathlineto{\pgfqpoint{2.561626in}{1.036665in}}%
\pgfpathlineto{\pgfqpoint{2.569287in}{1.039114in}}%
\pgfpathlineto{\pgfqpoint{2.571202in}{1.039116in}}%
\pgfpathlineto{\pgfqpoint{2.575033in}{1.043055in}}%
\pgfpathlineto{\pgfqpoint{2.594184in}{1.046713in}}%
\pgfpathlineto{\pgfqpoint{2.596099in}{1.048649in}}%
\pgfpathlineto{\pgfqpoint{2.601845in}{1.049867in}}%
\pgfpathlineto{\pgfqpoint{2.603760in}{1.052652in}}%
\pgfpathlineto{\pgfqpoint{2.609506in}{1.054760in}}%
\pgfpathlineto{\pgfqpoint{2.611421in}{1.056654in}}%
\pgfpathlineto{\pgfqpoint{2.615251in}{1.056941in}}%
\pgfpathlineto{\pgfqpoint{2.619082in}{1.058611in}}%
\pgfpathlineto{\pgfqpoint{2.620997in}{1.058659in}}%
\pgfpathlineto{\pgfqpoint{2.626742in}{1.063202in}}%
\pgfpathlineto{\pgfqpoint{2.632488in}{1.068559in}}%
\pgfpathlineto{\pgfqpoint{2.638233in}{1.070863in}}%
\pgfpathlineto{\pgfqpoint{2.640148in}{1.073210in}}%
\pgfpathlineto{\pgfqpoint{2.642064in}{1.073376in}}%
\pgfpathlineto{\pgfqpoint{2.653555in}{1.079961in}}%
\pgfpathlineto{\pgfqpoint{2.659300in}{1.081531in}}%
\pgfpathlineto{\pgfqpoint{2.665046in}{1.083021in}}%
\pgfpathlineto{\pgfqpoint{2.668876in}{1.086147in}}%
\pgfpathlineto{\pgfqpoint{2.672706in}{1.089786in}}%
\pgfpathlineto{\pgfqpoint{2.674622in}{1.092776in}}%
\pgfpathlineto{\pgfqpoint{2.676537in}{1.093057in}}%
\pgfpathlineto{\pgfqpoint{2.680367in}{1.095473in}}%
\pgfpathlineto{\pgfqpoint{2.684197in}{1.096841in}}%
\pgfpathlineto{\pgfqpoint{2.693773in}{1.098690in}}%
\pgfpathlineto{\pgfqpoint{2.697604in}{1.100264in}}%
\pgfpathlineto{\pgfqpoint{2.705264in}{1.104803in}}%
\pgfpathlineto{\pgfqpoint{2.707179in}{1.108273in}}%
\pgfpathlineto{\pgfqpoint{2.720586in}{1.115609in}}%
\pgfpathlineto{\pgfqpoint{2.726331in}{1.125363in}}%
\pgfpathlineto{\pgfqpoint{2.728246in}{1.129427in}}%
\pgfpathlineto{\pgfqpoint{2.730161in}{1.129533in}}%
\pgfpathlineto{\pgfqpoint{2.737822in}{1.135682in}}%
\pgfpathlineto{\pgfqpoint{2.741653in}{1.139500in}}%
\pgfpathlineto{\pgfqpoint{2.743568in}{1.139923in}}%
\pgfpathlineto{\pgfqpoint{2.745483in}{1.149433in}}%
\pgfpathlineto{\pgfqpoint{2.747398in}{1.150761in}}%
\pgfpathlineto{\pgfqpoint{2.749313in}{1.159555in}}%
\pgfpathlineto{\pgfqpoint{2.753144in}{1.163879in}}%
\pgfpathlineto{\pgfqpoint{2.758889in}{1.169167in}}%
\pgfpathlineto{\pgfqpoint{2.760804in}{1.176447in}}%
\pgfpathlineto{\pgfqpoint{2.764635in}{1.179226in}}%
\pgfpathlineto{\pgfqpoint{2.774210in}{1.200352in}}%
\pgfpathlineto{\pgfqpoint{2.776126in}{1.200960in}}%
\pgfpathlineto{\pgfqpoint{2.779956in}{1.211885in}}%
\pgfpathlineto{\pgfqpoint{2.781871in}{1.223561in}}%
\pgfpathlineto{\pgfqpoint{2.787617in}{1.233278in}}%
\pgfpathlineto{\pgfqpoint{2.791447in}{1.235612in}}%
\pgfpathlineto{\pgfqpoint{2.793362in}{1.239034in}}%
\pgfpathlineto{\pgfqpoint{2.795277in}{1.239183in}}%
\pgfpathlineto{\pgfqpoint{2.797192in}{1.241743in}}%
\pgfpathlineto{\pgfqpoint{2.802938in}{1.258053in}}%
\pgfpathlineto{\pgfqpoint{2.804853in}{1.259950in}}%
\pgfpathlineto{\pgfqpoint{2.806768in}{1.260051in}}%
\pgfpathlineto{\pgfqpoint{2.808684in}{1.265087in}}%
\pgfpathlineto{\pgfqpoint{2.810599in}{1.266597in}}%
\pgfpathlineto{\pgfqpoint{2.812514in}{1.266653in}}%
\pgfpathlineto{\pgfqpoint{2.814429in}{1.294932in}}%
\pgfpathlineto{\pgfqpoint{2.816344in}{1.296020in}}%
\pgfpathlineto{\pgfqpoint{2.820175in}{1.422730in}}%
\pgfpathlineto{\pgfqpoint{2.822090in}{1.516447in}}%
\pgfpathlineto{\pgfqpoint{2.824005in}{1.826535in}}%
\pgfpathlineto{\pgfqpoint{2.824005in}{1.826535in}}%
\pgfusepath{stroke}%
\end{pgfscope}%
\begin{pgfscope}%
\pgfpathrectangle{\pgfqpoint{0.694334in}{0.523557in}}{\pgfqpoint{3.830343in}{1.302977in}}%
\pgfusepath{clip}%
\pgfsetbuttcap%
\pgfsetroundjoin%
\pgfsetlinewidth{1.003750pt}%
\definecolor{currentstroke}{rgb}{0.811765,0.125490,0.125490}%
\pgfsetstrokecolor{currentstroke}%
\pgfsetdash{{1.000000pt}{1.650000pt}}{0.000000pt}%
\pgfpathmoveto{\pgfqpoint{0.694334in}{1.092842in}}%
\pgfpathlineto{\pgfqpoint{0.700080in}{1.095601in}}%
\pgfpathlineto{\pgfqpoint{0.713486in}{1.097828in}}%
\pgfpathlineto{\pgfqpoint{0.715401in}{1.099365in}}%
\pgfpathlineto{\pgfqpoint{0.723062in}{1.100590in}}%
\pgfpathlineto{\pgfqpoint{0.738383in}{1.104836in}}%
\pgfpathlineto{\pgfqpoint{0.740298in}{1.108199in}}%
\pgfpathlineto{\pgfqpoint{0.744129in}{1.109019in}}%
\pgfpathlineto{\pgfqpoint{0.747959in}{1.109596in}}%
\pgfpathlineto{\pgfqpoint{0.749874in}{1.118682in}}%
\pgfpathlineto{\pgfqpoint{0.751789in}{1.118957in}}%
\pgfpathlineto{\pgfqpoint{0.755620in}{1.131559in}}%
\pgfpathlineto{\pgfqpoint{0.757535in}{1.132092in}}%
\pgfpathlineto{\pgfqpoint{0.759450in}{1.137051in}}%
\pgfpathlineto{\pgfqpoint{0.761365in}{1.138654in}}%
\pgfpathlineto{\pgfqpoint{0.765196in}{1.145963in}}%
\pgfpathlineto{\pgfqpoint{0.767111in}{1.146688in}}%
\pgfpathlineto{\pgfqpoint{0.769026in}{1.150685in}}%
\pgfpathlineto{\pgfqpoint{0.772856in}{1.152392in}}%
\pgfpathlineto{\pgfqpoint{0.776687in}{1.160254in}}%
\pgfpathlineto{\pgfqpoint{0.784347in}{1.161935in}}%
\pgfpathlineto{\pgfqpoint{0.788178in}{1.173187in}}%
\pgfpathlineto{\pgfqpoint{0.792008in}{1.176899in}}%
\pgfpathlineto{\pgfqpoint{0.795838in}{1.177668in}}%
\pgfpathlineto{\pgfqpoint{0.797754in}{1.180982in}}%
\pgfpathlineto{\pgfqpoint{0.799669in}{1.187098in}}%
\pgfpathlineto{\pgfqpoint{0.805414in}{1.190346in}}%
\pgfpathlineto{\pgfqpoint{0.811160in}{1.192110in}}%
\pgfpathlineto{\pgfqpoint{0.814990in}{1.195951in}}%
\pgfpathlineto{\pgfqpoint{0.818820in}{1.196792in}}%
\pgfpathlineto{\pgfqpoint{0.860954in}{1.207986in}}%
\pgfpathlineto{\pgfqpoint{0.870530in}{1.209480in}}%
\pgfpathlineto{\pgfqpoint{0.883936in}{1.212655in}}%
\pgfpathlineto{\pgfqpoint{0.899258in}{1.214008in}}%
\pgfpathlineto{\pgfqpoint{0.908834in}{1.214543in}}%
\pgfpathlineto{\pgfqpoint{0.914579in}{1.215715in}}%
\pgfpathlineto{\pgfqpoint{0.941391in}{1.218454in}}%
\pgfpathlineto{\pgfqpoint{0.956713in}{1.219640in}}%
\pgfpathlineto{\pgfqpoint{0.964373in}{1.220722in}}%
\pgfpathlineto{\pgfqpoint{0.995016in}{1.223496in}}%
\pgfpathlineto{\pgfqpoint{1.010338in}{1.225303in}}%
\pgfpathlineto{\pgfqpoint{1.276546in}{1.260355in}}%
\pgfpathlineto{\pgfqpoint{1.301444in}{1.262781in}}%
\pgfpathlineto{\pgfqpoint{1.309104in}{1.263265in}}%
\pgfpathlineto{\pgfqpoint{1.314850in}{1.263874in}}%
\pgfpathlineto{\pgfqpoint{1.318680in}{1.265400in}}%
\pgfpathlineto{\pgfqpoint{1.326341in}{1.266687in}}%
\pgfpathlineto{\pgfqpoint{1.347408in}{1.271336in}}%
\pgfpathlineto{\pgfqpoint{1.353153in}{1.272187in}}%
\pgfpathlineto{\pgfqpoint{1.356984in}{1.273188in}}%
\pgfpathlineto{\pgfqpoint{1.368475in}{1.275324in}}%
\pgfpathlineto{\pgfqpoint{1.379966in}{1.276812in}}%
\pgfpathlineto{\pgfqpoint{1.391457in}{1.278655in}}%
\pgfpathlineto{\pgfqpoint{1.406778in}{1.281795in}}%
\pgfpathlineto{\pgfqpoint{1.416354in}{1.282600in}}%
\pgfpathlineto{\pgfqpoint{1.420184in}{1.284273in}}%
\pgfpathlineto{\pgfqpoint{1.431675in}{1.286989in}}%
\pgfpathlineto{\pgfqpoint{1.454657in}{1.290085in}}%
\pgfpathlineto{\pgfqpoint{1.456573in}{1.291995in}}%
\pgfpathlineto{\pgfqpoint{1.462318in}{1.292473in}}%
\pgfpathlineto{\pgfqpoint{1.466148in}{1.293576in}}%
\pgfpathlineto{\pgfqpoint{1.475724in}{1.295512in}}%
\pgfpathlineto{\pgfqpoint{1.485300in}{1.297704in}}%
\pgfpathlineto{\pgfqpoint{1.491046in}{1.298449in}}%
\pgfpathlineto{\pgfqpoint{1.502537in}{1.304469in}}%
\pgfpathlineto{\pgfqpoint{1.515943in}{1.308966in}}%
\pgfpathlineto{\pgfqpoint{1.521688in}{1.311486in}}%
\pgfpathlineto{\pgfqpoint{1.550416in}{1.319829in}}%
\pgfpathlineto{\pgfqpoint{1.556161in}{1.322115in}}%
\pgfpathlineto{\pgfqpoint{1.561907in}{1.323577in}}%
\pgfpathlineto{\pgfqpoint{1.565737in}{1.325470in}}%
\pgfpathlineto{\pgfqpoint{1.567652in}{1.326031in}}%
\pgfpathlineto{\pgfqpoint{1.569568in}{1.327861in}}%
\pgfpathlineto{\pgfqpoint{1.577228in}{1.328459in}}%
\pgfpathlineto{\pgfqpoint{1.581059in}{1.330616in}}%
\pgfpathlineto{\pgfqpoint{1.586804in}{1.330921in}}%
\pgfpathlineto{\pgfqpoint{1.588719in}{1.332514in}}%
\pgfpathlineto{\pgfqpoint{1.590635in}{1.332541in}}%
\pgfpathlineto{\pgfqpoint{1.594465in}{1.336745in}}%
\pgfpathlineto{\pgfqpoint{1.598295in}{1.339926in}}%
\pgfpathlineto{\pgfqpoint{1.604041in}{1.342014in}}%
\pgfpathlineto{\pgfqpoint{1.605956in}{1.343043in}}%
\pgfpathlineto{\pgfqpoint{1.607871in}{1.346368in}}%
\pgfpathlineto{\pgfqpoint{1.617447in}{1.348989in}}%
\pgfpathlineto{\pgfqpoint{1.621277in}{1.351011in}}%
\pgfpathlineto{\pgfqpoint{1.623192in}{1.351528in}}%
\pgfpathlineto{\pgfqpoint{1.628938in}{1.355828in}}%
\pgfpathlineto{\pgfqpoint{1.630853in}{1.356072in}}%
\pgfpathlineto{\pgfqpoint{1.632768in}{1.359363in}}%
\pgfpathlineto{\pgfqpoint{1.638514in}{1.361024in}}%
\pgfpathlineto{\pgfqpoint{1.640429in}{1.366651in}}%
\pgfpathlineto{\pgfqpoint{1.646174in}{1.369896in}}%
\pgfpathlineto{\pgfqpoint{1.653835in}{1.371709in}}%
\pgfpathlineto{\pgfqpoint{1.655750in}{1.372445in}}%
\pgfpathlineto{\pgfqpoint{1.659581in}{1.376545in}}%
\pgfpathlineto{\pgfqpoint{1.661496in}{1.376664in}}%
\pgfpathlineto{\pgfqpoint{1.663411in}{1.378528in}}%
\pgfpathlineto{\pgfqpoint{1.665326in}{1.382384in}}%
\pgfpathlineto{\pgfqpoint{1.671072in}{1.385381in}}%
\pgfpathlineto{\pgfqpoint{1.672987in}{1.385605in}}%
\pgfpathlineto{\pgfqpoint{1.676817in}{1.388631in}}%
\pgfpathlineto{\pgfqpoint{1.678732in}{1.389252in}}%
\pgfpathlineto{\pgfqpoint{1.684478in}{1.394453in}}%
\pgfpathlineto{\pgfqpoint{1.686393in}{1.394889in}}%
\pgfpathlineto{\pgfqpoint{1.690223in}{1.398688in}}%
\pgfpathlineto{\pgfqpoint{1.697884in}{1.403132in}}%
\pgfpathlineto{\pgfqpoint{1.701714in}{1.409587in}}%
\pgfpathlineto{\pgfqpoint{1.713205in}{1.415145in}}%
\pgfpathlineto{\pgfqpoint{1.715121in}{1.416608in}}%
\pgfpathlineto{\pgfqpoint{1.717036in}{1.421963in}}%
\pgfpathlineto{\pgfqpoint{1.726612in}{1.424488in}}%
\pgfpathlineto{\pgfqpoint{1.730442in}{1.427041in}}%
\pgfpathlineto{\pgfqpoint{1.734272in}{1.428226in}}%
\pgfpathlineto{\pgfqpoint{1.738103in}{1.434547in}}%
\pgfpathlineto{\pgfqpoint{1.757254in}{1.441305in}}%
\pgfpathlineto{\pgfqpoint{1.761085in}{1.442517in}}%
\pgfpathlineto{\pgfqpoint{1.764915in}{1.443902in}}%
\pgfpathlineto{\pgfqpoint{1.768745in}{1.444849in}}%
\pgfpathlineto{\pgfqpoint{1.772576in}{1.449568in}}%
\pgfpathlineto{\pgfqpoint{1.782152in}{1.453292in}}%
\pgfpathlineto{\pgfqpoint{1.785982in}{1.456892in}}%
\pgfpathlineto{\pgfqpoint{1.791728in}{1.458672in}}%
\pgfpathlineto{\pgfqpoint{1.793643in}{1.460292in}}%
\pgfpathlineto{\pgfqpoint{1.797473in}{1.460309in}}%
\pgfpathlineto{\pgfqpoint{1.799388in}{1.463854in}}%
\pgfpathlineto{\pgfqpoint{1.803219in}{1.464567in}}%
\pgfpathlineto{\pgfqpoint{1.807049in}{1.464956in}}%
\pgfpathlineto{\pgfqpoint{1.810879in}{1.467354in}}%
\pgfpathlineto{\pgfqpoint{1.814710in}{1.475400in}}%
\pgfpathlineto{\pgfqpoint{1.818540in}{1.476345in}}%
\pgfpathlineto{\pgfqpoint{1.820455in}{1.479490in}}%
\pgfpathlineto{\pgfqpoint{1.831946in}{1.483182in}}%
\pgfpathlineto{\pgfqpoint{1.835776in}{1.486299in}}%
\pgfpathlineto{\pgfqpoint{1.837692in}{1.486766in}}%
\pgfpathlineto{\pgfqpoint{1.841522in}{1.489589in}}%
\pgfpathlineto{\pgfqpoint{1.843437in}{1.489910in}}%
\pgfpathlineto{\pgfqpoint{1.847267in}{1.493070in}}%
\pgfpathlineto{\pgfqpoint{1.860674in}{1.493814in}}%
\pgfpathlineto{\pgfqpoint{1.862589in}{1.495401in}}%
\pgfpathlineto{\pgfqpoint{1.868334in}{1.495891in}}%
\pgfpathlineto{\pgfqpoint{1.872165in}{1.497930in}}%
\pgfpathlineto{\pgfqpoint{1.875995in}{1.499961in}}%
\pgfpathlineto{\pgfqpoint{1.877910in}{1.505456in}}%
\pgfpathlineto{\pgfqpoint{1.885571in}{1.507597in}}%
\pgfpathlineto{\pgfqpoint{1.889401in}{1.511675in}}%
\pgfpathlineto{\pgfqpoint{1.904723in}{1.522836in}}%
\pgfpathlineto{\pgfqpoint{1.910468in}{1.531346in}}%
\pgfpathlineto{\pgfqpoint{1.914298in}{1.531408in}}%
\pgfpathlineto{\pgfqpoint{1.923874in}{1.537233in}}%
\pgfpathlineto{\pgfqpoint{1.925790in}{1.540900in}}%
\pgfpathlineto{\pgfqpoint{1.931535in}{1.543587in}}%
\pgfpathlineto{\pgfqpoint{1.935365in}{1.544622in}}%
\pgfpathlineto{\pgfqpoint{1.944941in}{1.549946in}}%
\pgfpathlineto{\pgfqpoint{1.948772in}{1.550939in}}%
\pgfpathlineto{\pgfqpoint{1.952602in}{1.551894in}}%
\pgfpathlineto{\pgfqpoint{1.956432in}{1.553405in}}%
\pgfpathlineto{\pgfqpoint{1.966008in}{1.557216in}}%
\pgfpathlineto{\pgfqpoint{1.969838in}{1.568010in}}%
\pgfpathlineto{\pgfqpoint{1.973669in}{1.573545in}}%
\pgfpathlineto{\pgfqpoint{1.977499in}{1.573994in}}%
\pgfpathlineto{\pgfqpoint{1.983245in}{1.575787in}}%
\pgfpathlineto{\pgfqpoint{1.985160in}{1.575964in}}%
\pgfpathlineto{\pgfqpoint{1.988990in}{1.578691in}}%
\pgfpathlineto{\pgfqpoint{1.996651in}{1.580364in}}%
\pgfpathlineto{\pgfqpoint{2.010057in}{1.585292in}}%
\pgfpathlineto{\pgfqpoint{2.013887in}{1.595369in}}%
\pgfpathlineto{\pgfqpoint{2.019633in}{1.598076in}}%
\pgfpathlineto{\pgfqpoint{2.025378in}{1.601973in}}%
\pgfpathlineto{\pgfqpoint{2.031124in}{1.603684in}}%
\pgfpathlineto{\pgfqpoint{2.040700in}{1.607022in}}%
\pgfpathlineto{\pgfqpoint{2.050276in}{1.620180in}}%
\pgfpathlineto{\pgfqpoint{2.054106in}{1.621879in}}%
\pgfpathlineto{\pgfqpoint{2.061767in}{1.622929in}}%
\pgfpathlineto{\pgfqpoint{2.065597in}{1.626878in}}%
\pgfpathlineto{\pgfqpoint{2.082834in}{1.631262in}}%
\pgfpathlineto{\pgfqpoint{2.086664in}{1.633537in}}%
\pgfpathlineto{\pgfqpoint{2.088579in}{1.633905in}}%
\pgfpathlineto{\pgfqpoint{2.090494in}{1.636687in}}%
\pgfpathlineto{\pgfqpoint{2.092409in}{1.641522in}}%
\pgfpathlineto{\pgfqpoint{2.098155in}{1.643104in}}%
\pgfpathlineto{\pgfqpoint{2.103900in}{1.647931in}}%
\pgfpathlineto{\pgfqpoint{2.105816in}{1.648707in}}%
\pgfpathlineto{\pgfqpoint{2.107731in}{1.652064in}}%
\pgfpathlineto{\pgfqpoint{2.113476in}{1.653067in}}%
\pgfpathlineto{\pgfqpoint{2.124967in}{1.657318in}}%
\pgfpathlineto{\pgfqpoint{2.126883in}{1.660554in}}%
\pgfpathlineto{\pgfqpoint{2.132628in}{1.663152in}}%
\pgfpathlineto{\pgfqpoint{2.136458in}{1.667398in}}%
\pgfpathlineto{\pgfqpoint{2.146034in}{1.672284in}}%
\pgfpathlineto{\pgfqpoint{2.153695in}{1.677501in}}%
\pgfpathlineto{\pgfqpoint{2.165186in}{1.681182in}}%
\pgfpathlineto{\pgfqpoint{2.174762in}{1.687722in}}%
\pgfpathlineto{\pgfqpoint{2.182422in}{1.698022in}}%
\pgfpathlineto{\pgfqpoint{2.186253in}{1.699619in}}%
\pgfpathlineto{\pgfqpoint{2.188168in}{1.702234in}}%
\pgfpathlineto{\pgfqpoint{2.191998in}{1.702478in}}%
\pgfpathlineto{\pgfqpoint{2.203489in}{1.713686in}}%
\pgfpathlineto{\pgfqpoint{2.207320in}{1.714462in}}%
\pgfpathlineto{\pgfqpoint{2.209235in}{1.719630in}}%
\pgfpathlineto{\pgfqpoint{2.216896in}{1.722760in}}%
\pgfpathlineto{\pgfqpoint{2.220726in}{1.726786in}}%
\pgfpathlineto{\pgfqpoint{2.222641in}{1.727246in}}%
\pgfpathlineto{\pgfqpoint{2.226471in}{1.730974in}}%
\pgfpathlineto{\pgfqpoint{2.230302in}{1.731978in}}%
\pgfpathlineto{\pgfqpoint{2.237962in}{1.734587in}}%
\pgfpathlineto{\pgfqpoint{2.241793in}{1.736091in}}%
\pgfpathlineto{\pgfqpoint{2.247538in}{1.739614in}}%
\pgfpathlineto{\pgfqpoint{2.253284in}{1.741282in}}%
\pgfpathlineto{\pgfqpoint{2.262860in}{1.746025in}}%
\pgfpathlineto{\pgfqpoint{2.268605in}{1.747129in}}%
\pgfpathlineto{\pgfqpoint{2.272436in}{1.749142in}}%
\pgfpathlineto{\pgfqpoint{2.278181in}{1.759746in}}%
\pgfpathlineto{\pgfqpoint{2.285842in}{1.765681in}}%
\pgfpathlineto{\pgfqpoint{2.291587in}{1.766759in}}%
\pgfpathlineto{\pgfqpoint{2.293502in}{1.767371in}}%
\pgfpathlineto{\pgfqpoint{2.295418in}{1.772118in}}%
\pgfpathlineto{\pgfqpoint{2.301163in}{1.775959in}}%
\pgfpathlineto{\pgfqpoint{2.303078in}{1.776410in}}%
\pgfpathlineto{\pgfqpoint{2.306909in}{1.778441in}}%
\pgfpathlineto{\pgfqpoint{2.312654in}{1.780935in}}%
\pgfpathlineto{\pgfqpoint{2.320315in}{1.782097in}}%
\pgfpathlineto{\pgfqpoint{2.327975in}{1.786705in}}%
\pgfpathlineto{\pgfqpoint{2.331806in}{1.788304in}}%
\pgfpathlineto{\pgfqpoint{2.333721in}{1.789322in}}%
\pgfpathlineto{\pgfqpoint{2.335636in}{1.791670in}}%
\pgfpathlineto{\pgfqpoint{2.337551in}{1.792129in}}%
\pgfpathlineto{\pgfqpoint{2.339467in}{1.796183in}}%
\pgfpathlineto{\pgfqpoint{2.343297in}{1.797173in}}%
\pgfpathlineto{\pgfqpoint{2.345212in}{1.799983in}}%
\pgfpathlineto{\pgfqpoint{2.347127in}{1.800002in}}%
\pgfpathlineto{\pgfqpoint{2.349042in}{1.803257in}}%
\pgfpathlineto{\pgfqpoint{2.358618in}{1.807884in}}%
\pgfpathlineto{\pgfqpoint{2.360533in}{1.809997in}}%
\pgfpathlineto{\pgfqpoint{2.362449in}{1.810025in}}%
\pgfpathlineto{\pgfqpoint{2.366279in}{1.815338in}}%
\pgfpathlineto{\pgfqpoint{2.373940in}{1.816209in}}%
\pgfpathlineto{\pgfqpoint{2.377770in}{1.818488in}}%
\pgfpathlineto{\pgfqpoint{2.381600in}{1.823308in}}%
\pgfpathlineto{\pgfqpoint{2.385431in}{1.823974in}}%
\pgfpathlineto{\pgfqpoint{2.387346in}{1.826535in}}%
\pgfpathlineto{\pgfqpoint{2.387346in}{1.826535in}}%
\pgfusepath{stroke}%
\end{pgfscope}%
\begin{pgfscope}%
\pgfpathrectangle{\pgfqpoint{0.694334in}{0.523557in}}{\pgfqpoint{3.830343in}{1.302977in}}%
\pgfusepath{clip}%
\pgfsetrectcap%
\pgfsetroundjoin%
\pgfsetlinewidth{1.003750pt}%
\definecolor{currentstroke}{rgb}{0.000000,0.000000,0.376471}%
\pgfsetstrokecolor{currentstroke}%
\pgfsetdash{}{0pt}%
\pgfpathmoveto{\pgfqpoint{0.694334in}{0.599366in}}%
\pgfpathlineto{\pgfqpoint{0.696249in}{0.616722in}}%
\pgfpathlineto{\pgfqpoint{0.698165in}{0.623343in}}%
\pgfpathlineto{\pgfqpoint{0.700080in}{0.624942in}}%
\pgfpathlineto{\pgfqpoint{0.701995in}{0.628627in}}%
\pgfpathlineto{\pgfqpoint{0.707741in}{0.675393in}}%
\pgfpathlineto{\pgfqpoint{0.709656in}{0.679304in}}%
\pgfpathlineto{\pgfqpoint{0.713486in}{0.680868in}}%
\pgfpathlineto{\pgfqpoint{0.715401in}{0.684930in}}%
\pgfpathlineto{\pgfqpoint{0.717316in}{0.685863in}}%
\pgfpathlineto{\pgfqpoint{0.721147in}{0.699237in}}%
\pgfpathlineto{\pgfqpoint{0.724977in}{0.724415in}}%
\pgfpathlineto{\pgfqpoint{0.728807in}{0.731356in}}%
\pgfpathlineto{\pgfqpoint{0.730723in}{0.731663in}}%
\pgfpathlineto{\pgfqpoint{0.732638in}{0.733313in}}%
\pgfpathlineto{\pgfqpoint{0.736468in}{0.745375in}}%
\pgfpathlineto{\pgfqpoint{0.749874in}{0.754558in}}%
\pgfpathlineto{\pgfqpoint{0.757535in}{0.755842in}}%
\pgfpathlineto{\pgfqpoint{0.765196in}{0.759516in}}%
\pgfpathlineto{\pgfqpoint{0.769026in}{0.761092in}}%
\pgfpathlineto{\pgfqpoint{0.778602in}{0.763699in}}%
\pgfpathlineto{\pgfqpoint{0.780517in}{0.765949in}}%
\pgfpathlineto{\pgfqpoint{0.782432in}{0.766314in}}%
\pgfpathlineto{\pgfqpoint{0.784347in}{0.770570in}}%
\pgfpathlineto{\pgfqpoint{0.795838in}{0.772555in}}%
\pgfpathlineto{\pgfqpoint{0.834142in}{0.780708in}}%
\pgfpathlineto{\pgfqpoint{0.837972in}{0.781987in}}%
\pgfpathlineto{\pgfqpoint{0.841803in}{0.783419in}}%
\pgfpathlineto{\pgfqpoint{0.843718in}{0.784883in}}%
\pgfpathlineto{\pgfqpoint{0.847548in}{0.785885in}}%
\pgfpathlineto{\pgfqpoint{0.853294in}{0.788128in}}%
\pgfpathlineto{\pgfqpoint{0.855209in}{0.791418in}}%
\pgfpathlineto{\pgfqpoint{0.857124in}{0.791448in}}%
\pgfpathlineto{\pgfqpoint{0.860954in}{0.793652in}}%
\pgfpathlineto{\pgfqpoint{0.866700in}{0.795597in}}%
\pgfpathlineto{\pgfqpoint{0.870530in}{0.798359in}}%
\pgfpathlineto{\pgfqpoint{0.872445in}{0.798803in}}%
\pgfpathlineto{\pgfqpoint{0.876276in}{0.802324in}}%
\pgfpathlineto{\pgfqpoint{0.885851in}{0.804749in}}%
\pgfpathlineto{\pgfqpoint{0.903088in}{0.810705in}}%
\pgfpathlineto{\pgfqpoint{0.905003in}{0.813318in}}%
\pgfpathlineto{\pgfqpoint{0.912664in}{0.815343in}}%
\pgfpathlineto{\pgfqpoint{0.914579in}{0.817710in}}%
\pgfpathlineto{\pgfqpoint{0.918409in}{0.818198in}}%
\pgfpathlineto{\pgfqpoint{0.920325in}{0.820676in}}%
\pgfpathlineto{\pgfqpoint{0.926070in}{0.821629in}}%
\pgfpathlineto{\pgfqpoint{0.929900in}{0.823259in}}%
\pgfpathlineto{\pgfqpoint{0.939476in}{0.826427in}}%
\pgfpathlineto{\pgfqpoint{0.943307in}{0.830369in}}%
\pgfpathlineto{\pgfqpoint{0.952882in}{0.834160in}}%
\pgfpathlineto{\pgfqpoint{0.964373in}{0.836130in}}%
\pgfpathlineto{\pgfqpoint{0.966289in}{0.837963in}}%
\pgfpathlineto{\pgfqpoint{0.972034in}{0.838534in}}%
\pgfpathlineto{\pgfqpoint{0.975864in}{0.841301in}}%
\pgfpathlineto{\pgfqpoint{0.977780in}{0.841407in}}%
\pgfpathlineto{\pgfqpoint{0.979695in}{0.843140in}}%
\pgfpathlineto{\pgfqpoint{0.985440in}{0.844131in}}%
\pgfpathlineto{\pgfqpoint{0.989271in}{0.845658in}}%
\pgfpathlineto{\pgfqpoint{1.008422in}{0.852766in}}%
\pgfpathlineto{\pgfqpoint{1.021829in}{0.854652in}}%
\pgfpathlineto{\pgfqpoint{1.025659in}{0.856194in}}%
\pgfpathlineto{\pgfqpoint{1.042895in}{0.858909in}}%
\pgfpathlineto{\pgfqpoint{1.056302in}{0.860634in}}%
\pgfpathlineto{\pgfqpoint{1.060132in}{0.862067in}}%
\pgfpathlineto{\pgfqpoint{1.088860in}{0.868666in}}%
\pgfpathlineto{\pgfqpoint{1.100351in}{0.871832in}}%
\pgfpathlineto{\pgfqpoint{1.108011in}{0.873189in}}%
\pgfpathlineto{\pgfqpoint{1.111842in}{0.873923in}}%
\pgfpathlineto{\pgfqpoint{1.121418in}{0.875284in}}%
\pgfpathlineto{\pgfqpoint{1.130993in}{0.877922in}}%
\pgfpathlineto{\pgfqpoint{1.140569in}{0.879026in}}%
\pgfpathlineto{\pgfqpoint{1.146315in}{0.880442in}}%
\pgfpathlineto{\pgfqpoint{1.152060in}{0.881287in}}%
\pgfpathlineto{\pgfqpoint{1.159721in}{0.883155in}}%
\pgfpathlineto{\pgfqpoint{1.199940in}{0.889064in}}%
\pgfpathlineto{\pgfqpoint{1.213346in}{0.889892in}}%
\pgfpathlineto{\pgfqpoint{1.217176in}{0.891013in}}%
\pgfpathlineto{\pgfqpoint{1.228667in}{0.892147in}}%
\pgfpathlineto{\pgfqpoint{1.266971in}{0.897178in}}%
\pgfpathlineto{\pgfqpoint{1.280377in}{0.898879in}}%
\pgfpathlineto{\pgfqpoint{1.286122in}{0.899993in}}%
\pgfpathlineto{\pgfqpoint{1.299528in}{0.902931in}}%
\pgfpathlineto{\pgfqpoint{1.312935in}{0.905879in}}%
\pgfpathlineto{\pgfqpoint{1.316765in}{0.906863in}}%
\pgfpathlineto{\pgfqpoint{1.322511in}{0.908110in}}%
\pgfpathlineto{\pgfqpoint{1.349323in}{0.911752in}}%
\pgfpathlineto{\pgfqpoint{1.355068in}{0.912981in}}%
\pgfpathlineto{\pgfqpoint{1.387626in}{0.917568in}}%
\pgfpathlineto{\pgfqpoint{1.454657in}{0.922597in}}%
\pgfpathlineto{\pgfqpoint{1.550416in}{0.933264in}}%
\pgfpathlineto{\pgfqpoint{1.565737in}{0.935599in}}%
\pgfpathlineto{\pgfqpoint{1.575313in}{0.936435in}}%
\pgfpathlineto{\pgfqpoint{1.579143in}{0.937425in}}%
\pgfpathlineto{\pgfqpoint{1.588719in}{0.938693in}}%
\pgfpathlineto{\pgfqpoint{1.607871in}{0.940878in}}%
\pgfpathlineto{\pgfqpoint{1.625108in}{0.944931in}}%
\pgfpathlineto{\pgfqpoint{1.646174in}{0.947189in}}%
\pgfpathlineto{\pgfqpoint{1.650005in}{0.948491in}}%
\pgfpathlineto{\pgfqpoint{1.653835in}{0.950144in}}%
\pgfpathlineto{\pgfqpoint{1.674902in}{0.953346in}}%
\pgfpathlineto{\pgfqpoint{1.686393in}{0.954526in}}%
\pgfpathlineto{\pgfqpoint{1.695969in}{0.955641in}}%
\pgfpathlineto{\pgfqpoint{1.703630in}{0.956799in}}%
\pgfpathlineto{\pgfqpoint{1.722781in}{0.959828in}}%
\pgfpathlineto{\pgfqpoint{1.740018in}{0.964303in}}%
\pgfpathlineto{\pgfqpoint{1.741933in}{0.964532in}}%
\pgfpathlineto{\pgfqpoint{1.743848in}{0.967466in}}%
\pgfpathlineto{\pgfqpoint{1.755339in}{0.969525in}}%
\pgfpathlineto{\pgfqpoint{1.774491in}{0.971453in}}%
\pgfpathlineto{\pgfqpoint{1.778321in}{0.973937in}}%
\pgfpathlineto{\pgfqpoint{1.780236in}{0.974197in}}%
\pgfpathlineto{\pgfqpoint{1.785982in}{0.977752in}}%
\pgfpathlineto{\pgfqpoint{1.797473in}{0.980035in}}%
\pgfpathlineto{\pgfqpoint{1.820455in}{0.983071in}}%
\pgfpathlineto{\pgfqpoint{1.824285in}{0.983797in}}%
\pgfpathlineto{\pgfqpoint{1.833861in}{0.984717in}}%
\pgfpathlineto{\pgfqpoint{1.837692in}{0.986598in}}%
\pgfpathlineto{\pgfqpoint{1.860674in}{0.989409in}}%
\pgfpathlineto{\pgfqpoint{1.868334in}{0.991345in}}%
\pgfpathlineto{\pgfqpoint{1.877910in}{0.992458in}}%
\pgfpathlineto{\pgfqpoint{1.900892in}{0.997358in}}%
\pgfpathlineto{\pgfqpoint{1.904723in}{0.998557in}}%
\pgfpathlineto{\pgfqpoint{1.918129in}{1.001810in}}%
\pgfpathlineto{\pgfqpoint{1.929620in}{1.003261in}}%
\pgfpathlineto{\pgfqpoint{1.941111in}{1.004191in}}%
\pgfpathlineto{\pgfqpoint{1.954517in}{1.006170in}}%
\pgfpathlineto{\pgfqpoint{1.960263in}{1.007225in}}%
\pgfpathlineto{\pgfqpoint{1.964093in}{1.009090in}}%
\pgfpathlineto{\pgfqpoint{1.992821in}{1.012236in}}%
\pgfpathlineto{\pgfqpoint{1.996651in}{1.013673in}}%
\pgfpathlineto{\pgfqpoint{2.021548in}{1.017232in}}%
\pgfpathlineto{\pgfqpoint{2.027294in}{1.018388in}}%
\pgfpathlineto{\pgfqpoint{2.036869in}{1.019759in}}%
\pgfpathlineto{\pgfqpoint{2.038785in}{1.022034in}}%
\pgfpathlineto{\pgfqpoint{2.140289in}{1.034822in}}%
\pgfpathlineto{\pgfqpoint{2.149865in}{1.036153in}}%
\pgfpathlineto{\pgfqpoint{2.155610in}{1.036768in}}%
\pgfpathlineto{\pgfqpoint{2.161356in}{1.038141in}}%
\pgfpathlineto{\pgfqpoint{2.172847in}{1.039984in}}%
\pgfpathlineto{\pgfqpoint{2.188168in}{1.042432in}}%
\pgfpathlineto{\pgfqpoint{2.193914in}{1.044353in}}%
\pgfpathlineto{\pgfqpoint{2.203489in}{1.045628in}}%
\pgfpathlineto{\pgfqpoint{2.222641in}{1.049509in}}%
\pgfpathlineto{\pgfqpoint{2.228387in}{1.050142in}}%
\pgfpathlineto{\pgfqpoint{2.297333in}{1.061340in}}%
\pgfpathlineto{\pgfqpoint{2.301163in}{1.062700in}}%
\pgfpathlineto{\pgfqpoint{2.303078in}{1.062824in}}%
\pgfpathlineto{\pgfqpoint{2.304993in}{1.064582in}}%
\pgfpathlineto{\pgfqpoint{2.314569in}{1.065431in}}%
\pgfpathlineto{\pgfqpoint{2.318400in}{1.066746in}}%
\pgfpathlineto{\pgfqpoint{2.324145in}{1.067012in}}%
\pgfpathlineto{\pgfqpoint{2.327975in}{1.068510in}}%
\pgfpathlineto{\pgfqpoint{2.339467in}{1.069371in}}%
\pgfpathlineto{\pgfqpoint{2.358618in}{1.073888in}}%
\pgfpathlineto{\pgfqpoint{2.387346in}{1.079643in}}%
\pgfpathlineto{\pgfqpoint{2.393091in}{1.080477in}}%
\pgfpathlineto{\pgfqpoint{2.398837in}{1.082285in}}%
\pgfpathlineto{\pgfqpoint{2.402667in}{1.083946in}}%
\pgfpathlineto{\pgfqpoint{2.406498in}{1.084690in}}%
\pgfpathlineto{\pgfqpoint{2.416073in}{1.086223in}}%
\pgfpathlineto{\pgfqpoint{2.431395in}{1.089847in}}%
\pgfpathlineto{\pgfqpoint{2.433310in}{1.089886in}}%
\pgfpathlineto{\pgfqpoint{2.435225in}{1.091309in}}%
\pgfpathlineto{\pgfqpoint{2.439055in}{1.091773in}}%
\pgfpathlineto{\pgfqpoint{2.442886in}{1.092888in}}%
\pgfpathlineto{\pgfqpoint{2.450546in}{1.093513in}}%
\pgfpathlineto{\pgfqpoint{2.456292in}{1.095811in}}%
\pgfpathlineto{\pgfqpoint{2.465868in}{1.099734in}}%
\pgfpathlineto{\pgfqpoint{2.483104in}{1.104601in}}%
\pgfpathlineto{\pgfqpoint{2.496511in}{1.107612in}}%
\pgfpathlineto{\pgfqpoint{2.500341in}{1.112103in}}%
\pgfpathlineto{\pgfqpoint{2.527153in}{1.117385in}}%
\pgfpathlineto{\pgfqpoint{2.546305in}{1.123202in}}%
\pgfpathlineto{\pgfqpoint{2.553966in}{1.124777in}}%
\pgfpathlineto{\pgfqpoint{2.559711in}{1.127149in}}%
\pgfpathlineto{\pgfqpoint{2.567372in}{1.128651in}}%
\pgfpathlineto{\pgfqpoint{2.569287in}{1.132174in}}%
\pgfpathlineto{\pgfqpoint{2.571202in}{1.132231in}}%
\pgfpathlineto{\pgfqpoint{2.575033in}{1.133807in}}%
\pgfpathlineto{\pgfqpoint{2.582693in}{1.135698in}}%
\pgfpathlineto{\pgfqpoint{2.584608in}{1.136519in}}%
\pgfpathlineto{\pgfqpoint{2.586524in}{1.139103in}}%
\pgfpathlineto{\pgfqpoint{2.598015in}{1.141933in}}%
\pgfpathlineto{\pgfqpoint{2.599930in}{1.145281in}}%
\pgfpathlineto{\pgfqpoint{2.609506in}{1.148967in}}%
\pgfpathlineto{\pgfqpoint{2.615251in}{1.151012in}}%
\pgfpathlineto{\pgfqpoint{2.632488in}{1.160730in}}%
\pgfpathlineto{\pgfqpoint{2.634403in}{1.162259in}}%
\pgfpathlineto{\pgfqpoint{2.636318in}{1.162292in}}%
\pgfpathlineto{\pgfqpoint{2.640148in}{1.164127in}}%
\pgfpathlineto{\pgfqpoint{2.642064in}{1.164768in}}%
\pgfpathlineto{\pgfqpoint{2.645894in}{1.171670in}}%
\pgfpathlineto{\pgfqpoint{2.649724in}{1.173699in}}%
\pgfpathlineto{\pgfqpoint{2.651639in}{1.177104in}}%
\pgfpathlineto{\pgfqpoint{2.655470in}{1.178221in}}%
\pgfpathlineto{\pgfqpoint{2.661215in}{1.181616in}}%
\pgfpathlineto{\pgfqpoint{2.665046in}{1.182529in}}%
\pgfpathlineto{\pgfqpoint{2.668876in}{1.183959in}}%
\pgfpathlineto{\pgfqpoint{2.672706in}{1.185299in}}%
\pgfpathlineto{\pgfqpoint{2.674622in}{1.189981in}}%
\pgfpathlineto{\pgfqpoint{2.678452in}{1.191422in}}%
\pgfpathlineto{\pgfqpoint{2.693773in}{1.196168in}}%
\pgfpathlineto{\pgfqpoint{2.697604in}{1.201777in}}%
\pgfpathlineto{\pgfqpoint{2.699519in}{1.203232in}}%
\pgfpathlineto{\pgfqpoint{2.709095in}{1.204549in}}%
\pgfpathlineto{\pgfqpoint{2.716755in}{1.206028in}}%
\pgfpathlineto{\pgfqpoint{2.720586in}{1.207480in}}%
\pgfpathlineto{\pgfqpoint{2.724416in}{1.207804in}}%
\pgfpathlineto{\pgfqpoint{2.737822in}{1.216964in}}%
\pgfpathlineto{\pgfqpoint{2.741653in}{1.217531in}}%
\pgfpathlineto{\pgfqpoint{2.743568in}{1.220838in}}%
\pgfpathlineto{\pgfqpoint{2.745483in}{1.221483in}}%
\pgfpathlineto{\pgfqpoint{2.747398in}{1.224490in}}%
\pgfpathlineto{\pgfqpoint{2.755059in}{1.226493in}}%
\pgfpathlineto{\pgfqpoint{2.756974in}{1.228797in}}%
\pgfpathlineto{\pgfqpoint{2.758889in}{1.233639in}}%
\pgfpathlineto{\pgfqpoint{2.762719in}{1.234828in}}%
\pgfpathlineto{\pgfqpoint{2.766550in}{1.236943in}}%
\pgfpathlineto{\pgfqpoint{2.770380in}{1.237936in}}%
\pgfpathlineto{\pgfqpoint{2.772295in}{1.239920in}}%
\pgfpathlineto{\pgfqpoint{2.774210in}{1.239988in}}%
\pgfpathlineto{\pgfqpoint{2.776126in}{1.242697in}}%
\pgfpathlineto{\pgfqpoint{2.778041in}{1.242864in}}%
\pgfpathlineto{\pgfqpoint{2.785701in}{1.248064in}}%
\pgfpathlineto{\pgfqpoint{2.797192in}{1.252870in}}%
\pgfpathlineto{\pgfqpoint{2.806768in}{1.254761in}}%
\pgfpathlineto{\pgfqpoint{2.808684in}{1.258943in}}%
\pgfpathlineto{\pgfqpoint{2.812514in}{1.260611in}}%
\pgfpathlineto{\pgfqpoint{2.820175in}{1.268660in}}%
\pgfpathlineto{\pgfqpoint{2.822090in}{1.275625in}}%
\pgfpathlineto{\pgfqpoint{2.831666in}{1.277604in}}%
\pgfpathlineto{\pgfqpoint{2.833581in}{1.282838in}}%
\pgfpathlineto{\pgfqpoint{2.839326in}{1.286615in}}%
\pgfpathlineto{\pgfqpoint{2.843157in}{1.288602in}}%
\pgfpathlineto{\pgfqpoint{2.845072in}{1.289591in}}%
\pgfpathlineto{\pgfqpoint{2.850817in}{1.297029in}}%
\pgfpathlineto{\pgfqpoint{2.854648in}{1.299721in}}%
\pgfpathlineto{\pgfqpoint{2.856563in}{1.300402in}}%
\pgfpathlineto{\pgfqpoint{2.862308in}{1.305160in}}%
\pgfpathlineto{\pgfqpoint{2.866139in}{1.306455in}}%
\pgfpathlineto{\pgfqpoint{2.871884in}{1.310414in}}%
\pgfpathlineto{\pgfqpoint{2.875715in}{1.314566in}}%
\pgfpathlineto{\pgfqpoint{2.877630in}{1.315000in}}%
\pgfpathlineto{\pgfqpoint{2.885290in}{1.327105in}}%
\pgfpathlineto{\pgfqpoint{2.889121in}{1.327696in}}%
\pgfpathlineto{\pgfqpoint{2.891036in}{1.332498in}}%
\pgfpathlineto{\pgfqpoint{2.892951in}{1.333169in}}%
\pgfpathlineto{\pgfqpoint{2.896781in}{1.335924in}}%
\pgfpathlineto{\pgfqpoint{2.898697in}{1.341167in}}%
\pgfpathlineto{\pgfqpoint{2.906357in}{1.344628in}}%
\pgfpathlineto{\pgfqpoint{2.908272in}{1.351538in}}%
\pgfpathlineto{\pgfqpoint{2.910188in}{1.352266in}}%
\pgfpathlineto{\pgfqpoint{2.915933in}{1.363609in}}%
\pgfpathlineto{\pgfqpoint{2.919763in}{1.365061in}}%
\pgfpathlineto{\pgfqpoint{2.921679in}{1.373283in}}%
\pgfpathlineto{\pgfqpoint{2.929339in}{1.380897in}}%
\pgfpathlineto{\pgfqpoint{2.931254in}{1.381325in}}%
\pgfpathlineto{\pgfqpoint{2.933170in}{1.387435in}}%
\pgfpathlineto{\pgfqpoint{2.935085in}{1.387793in}}%
\pgfpathlineto{\pgfqpoint{2.937000in}{1.393773in}}%
\pgfpathlineto{\pgfqpoint{2.938915in}{1.394062in}}%
\pgfpathlineto{\pgfqpoint{2.940830in}{1.401763in}}%
\pgfpathlineto{\pgfqpoint{2.942746in}{1.402862in}}%
\pgfpathlineto{\pgfqpoint{2.944661in}{1.419801in}}%
\pgfpathlineto{\pgfqpoint{2.948491in}{1.431082in}}%
\pgfpathlineto{\pgfqpoint{2.954237in}{1.443016in}}%
\pgfpathlineto{\pgfqpoint{2.956152in}{1.450338in}}%
\pgfpathlineto{\pgfqpoint{2.958067in}{1.471589in}}%
\pgfpathlineto{\pgfqpoint{2.963812in}{1.493713in}}%
\pgfpathlineto{\pgfqpoint{2.967643in}{1.504276in}}%
\pgfpathlineto{\pgfqpoint{2.971473in}{1.540108in}}%
\pgfpathlineto{\pgfqpoint{2.973388in}{1.541925in}}%
\pgfpathlineto{\pgfqpoint{2.975303in}{1.546393in}}%
\pgfpathlineto{\pgfqpoint{2.977219in}{1.558329in}}%
\pgfpathlineto{\pgfqpoint{2.979134in}{1.561688in}}%
\pgfpathlineto{\pgfqpoint{2.981049in}{1.569034in}}%
\pgfpathlineto{\pgfqpoint{2.982964in}{1.597494in}}%
\pgfpathlineto{\pgfqpoint{2.984879in}{1.602936in}}%
\pgfpathlineto{\pgfqpoint{2.986794in}{1.622419in}}%
\pgfpathlineto{\pgfqpoint{2.992540in}{1.641153in}}%
\pgfpathlineto{\pgfqpoint{2.996370in}{1.651511in}}%
\pgfpathlineto{\pgfqpoint{3.000201in}{1.657346in}}%
\pgfpathlineto{\pgfqpoint{3.004031in}{1.665559in}}%
\pgfpathlineto{\pgfqpoint{3.005946in}{1.668235in}}%
\pgfpathlineto{\pgfqpoint{3.009777in}{1.700837in}}%
\pgfpathlineto{\pgfqpoint{3.013607in}{1.713141in}}%
\pgfpathlineto{\pgfqpoint{3.017437in}{1.723441in}}%
\pgfpathlineto{\pgfqpoint{3.019352in}{1.728682in}}%
\pgfpathlineto{\pgfqpoint{3.021268in}{1.730441in}}%
\pgfpathlineto{\pgfqpoint{3.025098in}{1.753429in}}%
\pgfpathlineto{\pgfqpoint{3.027013in}{1.794188in}}%
\pgfpathlineto{\pgfqpoint{3.028928in}{1.794719in}}%
\pgfpathlineto{\pgfqpoint{3.030843in}{1.797123in}}%
\pgfpathlineto{\pgfqpoint{3.032759in}{1.826535in}}%
\pgfpathlineto{\pgfqpoint{3.032759in}{1.826535in}}%
\pgfusepath{stroke}%
\end{pgfscope}%
\begin{pgfscope}%
\pgfpathrectangle{\pgfqpoint{0.694334in}{0.523557in}}{\pgfqpoint{3.830343in}{1.302977in}}%
\pgfusepath{clip}%
\pgfsetrectcap%
\pgfsetroundjoin%
\pgfsetlinewidth{1.003750pt}%
\definecolor{currentstroke}{rgb}{0.564706,0.564706,1.000000}%
\pgfsetstrokecolor{currentstroke}%
\pgfsetdash{}{0pt}%
\pgfpathmoveto{\pgfqpoint{0.694334in}{0.703806in}}%
\pgfpathlineto{\pgfqpoint{0.696249in}{0.737312in}}%
\pgfpathlineto{\pgfqpoint{0.698165in}{0.740239in}}%
\pgfpathlineto{\pgfqpoint{0.700080in}{0.745465in}}%
\pgfpathlineto{\pgfqpoint{0.703910in}{0.746981in}}%
\pgfpathlineto{\pgfqpoint{0.705825in}{0.754666in}}%
\pgfpathlineto{\pgfqpoint{0.713486in}{0.757532in}}%
\pgfpathlineto{\pgfqpoint{0.719232in}{0.759770in}}%
\pgfpathlineto{\pgfqpoint{0.724977in}{0.761277in}}%
\pgfpathlineto{\pgfqpoint{0.728807in}{0.761602in}}%
\pgfpathlineto{\pgfqpoint{0.732638in}{0.762857in}}%
\pgfpathlineto{\pgfqpoint{0.736468in}{0.763733in}}%
\pgfpathlineto{\pgfqpoint{0.740298in}{0.766512in}}%
\pgfpathlineto{\pgfqpoint{0.744129in}{0.768190in}}%
\pgfpathlineto{\pgfqpoint{0.747959in}{0.768811in}}%
\pgfpathlineto{\pgfqpoint{0.757535in}{0.769539in}}%
\pgfpathlineto{\pgfqpoint{0.763280in}{0.771747in}}%
\pgfpathlineto{\pgfqpoint{0.769026in}{0.774631in}}%
\pgfpathlineto{\pgfqpoint{0.770941in}{0.774736in}}%
\pgfpathlineto{\pgfqpoint{0.772856in}{0.777195in}}%
\pgfpathlineto{\pgfqpoint{0.780517in}{0.779321in}}%
\pgfpathlineto{\pgfqpoint{0.790093in}{0.782434in}}%
\pgfpathlineto{\pgfqpoint{0.793923in}{0.784767in}}%
\pgfpathlineto{\pgfqpoint{0.795838in}{0.784811in}}%
\pgfpathlineto{\pgfqpoint{0.799669in}{0.788278in}}%
\pgfpathlineto{\pgfqpoint{0.805414in}{0.789452in}}%
\pgfpathlineto{\pgfqpoint{0.809245in}{0.791820in}}%
\pgfpathlineto{\pgfqpoint{0.816905in}{0.793698in}}%
\pgfpathlineto{\pgfqpoint{0.818820in}{0.795425in}}%
\pgfpathlineto{\pgfqpoint{0.824566in}{0.796047in}}%
\pgfpathlineto{\pgfqpoint{0.832227in}{0.799687in}}%
\pgfpathlineto{\pgfqpoint{0.841803in}{0.802601in}}%
\pgfpathlineto{\pgfqpoint{0.845633in}{0.803979in}}%
\pgfpathlineto{\pgfqpoint{0.847548in}{0.803993in}}%
\pgfpathlineto{\pgfqpoint{0.851378in}{0.807098in}}%
\pgfpathlineto{\pgfqpoint{0.862869in}{0.808678in}}%
\pgfpathlineto{\pgfqpoint{0.866700in}{0.810160in}}%
\pgfpathlineto{\pgfqpoint{0.874360in}{0.812265in}}%
\pgfpathlineto{\pgfqpoint{0.880106in}{0.813507in}}%
\pgfpathlineto{\pgfqpoint{0.882021in}{0.813875in}}%
\pgfpathlineto{\pgfqpoint{0.887767in}{0.818447in}}%
\pgfpathlineto{\pgfqpoint{0.891597in}{0.819748in}}%
\pgfpathlineto{\pgfqpoint{0.906918in}{0.828475in}}%
\pgfpathlineto{\pgfqpoint{0.908834in}{0.828597in}}%
\pgfpathlineto{\pgfqpoint{0.914579in}{0.834186in}}%
\pgfpathlineto{\pgfqpoint{0.920325in}{0.837295in}}%
\pgfpathlineto{\pgfqpoint{0.924155in}{0.839796in}}%
\pgfpathlineto{\pgfqpoint{0.927985in}{0.840826in}}%
\pgfpathlineto{\pgfqpoint{0.929900in}{0.842079in}}%
\pgfpathlineto{\pgfqpoint{0.935646in}{0.849780in}}%
\pgfpathlineto{\pgfqpoint{0.949052in}{0.854951in}}%
\pgfpathlineto{\pgfqpoint{0.954798in}{0.855791in}}%
\pgfpathlineto{\pgfqpoint{0.962458in}{0.861168in}}%
\pgfpathlineto{\pgfqpoint{0.966289in}{0.861769in}}%
\pgfpathlineto{\pgfqpoint{0.968204in}{0.864003in}}%
\pgfpathlineto{\pgfqpoint{0.973949in}{0.866159in}}%
\pgfpathlineto{\pgfqpoint{0.979695in}{0.867453in}}%
\pgfpathlineto{\pgfqpoint{0.985440in}{0.869988in}}%
\pgfpathlineto{\pgfqpoint{0.987356in}{0.873680in}}%
\pgfpathlineto{\pgfqpoint{0.989271in}{0.873771in}}%
\pgfpathlineto{\pgfqpoint{0.993101in}{0.875298in}}%
\pgfpathlineto{\pgfqpoint{1.023744in}{0.880894in}}%
\pgfpathlineto{\pgfqpoint{1.029489in}{0.883667in}}%
\pgfpathlineto{\pgfqpoint{1.033320in}{0.884552in}}%
\pgfpathlineto{\pgfqpoint{1.037150in}{0.885936in}}%
\pgfpathlineto{\pgfqpoint{1.040980in}{0.887013in}}%
\pgfpathlineto{\pgfqpoint{1.044811in}{0.887763in}}%
\pgfpathlineto{\pgfqpoint{1.050556in}{0.889467in}}%
\pgfpathlineto{\pgfqpoint{1.054387in}{0.889942in}}%
\pgfpathlineto{\pgfqpoint{1.058217in}{0.891196in}}%
\pgfpathlineto{\pgfqpoint{1.065878in}{0.892859in}}%
\pgfpathlineto{\pgfqpoint{1.088860in}{0.896070in}}%
\pgfpathlineto{\pgfqpoint{1.090775in}{0.897788in}}%
\pgfpathlineto{\pgfqpoint{1.094605in}{0.898895in}}%
\pgfpathlineto{\pgfqpoint{1.098435in}{0.900369in}}%
\pgfpathlineto{\pgfqpoint{1.106096in}{0.901823in}}%
\pgfpathlineto{\pgfqpoint{1.113757in}{0.903021in}}%
\pgfpathlineto{\pgfqpoint{1.153975in}{0.909369in}}%
\pgfpathlineto{\pgfqpoint{1.157806in}{0.911192in}}%
\pgfpathlineto{\pgfqpoint{1.167382in}{0.912204in}}%
\pgfpathlineto{\pgfqpoint{1.173127in}{0.912697in}}%
\pgfpathlineto{\pgfqpoint{1.178873in}{0.913300in}}%
\pgfpathlineto{\pgfqpoint{1.184618in}{0.914770in}}%
\pgfpathlineto{\pgfqpoint{1.205685in}{0.916886in}}%
\pgfpathlineto{\pgfqpoint{1.215261in}{0.918381in}}%
\pgfpathlineto{\pgfqpoint{1.284207in}{0.927502in}}%
\pgfpathlineto{\pgfqpoint{1.301444in}{0.929047in}}%
\pgfpathlineto{\pgfqpoint{1.307189in}{0.930186in}}%
\pgfpathlineto{\pgfqpoint{1.312935in}{0.931042in}}%
\pgfpathlineto{\pgfqpoint{1.374220in}{0.937970in}}%
\pgfpathlineto{\pgfqpoint{1.378050in}{0.938874in}}%
\pgfpathlineto{\pgfqpoint{1.393372in}{0.940231in}}%
\pgfpathlineto{\pgfqpoint{1.437421in}{0.945512in}}%
\pgfpathlineto{\pgfqpoint{1.445081in}{0.946576in}}%
\pgfpathlineto{\pgfqpoint{1.517858in}{0.952724in}}%
\pgfpathlineto{\pgfqpoint{1.521688in}{0.954409in}}%
\pgfpathlineto{\pgfqpoint{1.554246in}{0.957380in}}%
\pgfpathlineto{\pgfqpoint{1.582974in}{0.959969in}}%
\pgfpathlineto{\pgfqpoint{1.590635in}{0.960756in}}%
\pgfpathlineto{\pgfqpoint{1.623192in}{0.964336in}}%
\pgfpathlineto{\pgfqpoint{1.628938in}{0.965136in}}%
\pgfpathlineto{\pgfqpoint{1.636599in}{0.966438in}}%
\pgfpathlineto{\pgfqpoint{1.642344in}{0.967911in}}%
\pgfpathlineto{\pgfqpoint{1.686393in}{0.974032in}}%
\pgfpathlineto{\pgfqpoint{1.690223in}{0.975041in}}%
\pgfpathlineto{\pgfqpoint{1.734272in}{0.981467in}}%
\pgfpathlineto{\pgfqpoint{1.740018in}{0.983447in}}%
\pgfpathlineto{\pgfqpoint{1.745763in}{0.984686in}}%
\pgfpathlineto{\pgfqpoint{1.755339in}{0.986555in}}%
\pgfpathlineto{\pgfqpoint{1.761085in}{0.987270in}}%
\pgfpathlineto{\pgfqpoint{1.772576in}{0.988294in}}%
\pgfpathlineto{\pgfqpoint{1.785982in}{0.991567in}}%
\pgfpathlineto{\pgfqpoint{1.816625in}{0.997086in}}%
\pgfpathlineto{\pgfqpoint{1.820455in}{0.997821in}}%
\pgfpathlineto{\pgfqpoint{1.826201in}{0.998985in}}%
\pgfpathlineto{\pgfqpoint{1.921959in}{1.013104in}}%
\pgfpathlineto{\pgfqpoint{1.935365in}{1.014109in}}%
\pgfpathlineto{\pgfqpoint{1.990905in}{1.021609in}}%
\pgfpathlineto{\pgfqpoint{1.994736in}{1.023228in}}%
\pgfpathlineto{\pgfqpoint{2.004312in}{1.023816in}}%
\pgfpathlineto{\pgfqpoint{2.008142in}{1.025233in}}%
\pgfpathlineto{\pgfqpoint{2.023463in}{1.027506in}}%
\pgfpathlineto{\pgfqpoint{2.046445in}{1.029353in}}%
\pgfpathlineto{\pgfqpoint{2.071343in}{1.032582in}}%
\pgfpathlineto{\pgfqpoint{2.082834in}{1.034329in}}%
\pgfpathlineto{\pgfqpoint{2.142204in}{1.042093in}}%
\pgfpathlineto{\pgfqpoint{2.147949in}{1.043497in}}%
\pgfpathlineto{\pgfqpoint{2.163271in}{1.045316in}}%
\pgfpathlineto{\pgfqpoint{2.176677in}{1.047736in}}%
\pgfpathlineto{\pgfqpoint{2.182422in}{1.048601in}}%
\pgfpathlineto{\pgfqpoint{2.197744in}{1.049935in}}%
\pgfpathlineto{\pgfqpoint{2.214980in}{1.053074in}}%
\pgfpathlineto{\pgfqpoint{2.237962in}{1.055008in}}%
\pgfpathlineto{\pgfqpoint{2.262860in}{1.056430in}}%
\pgfpathlineto{\pgfqpoint{2.268605in}{1.057931in}}%
\pgfpathlineto{\pgfqpoint{2.289672in}{1.060641in}}%
\pgfpathlineto{\pgfqpoint{2.308824in}{1.064654in}}%
\pgfpathlineto{\pgfqpoint{2.396922in}{1.078320in}}%
\pgfpathlineto{\pgfqpoint{2.400752in}{1.079652in}}%
\pgfpathlineto{\pgfqpoint{2.417989in}{1.081793in}}%
\pgfpathlineto{\pgfqpoint{2.419904in}{1.084435in}}%
\pgfpathlineto{\pgfqpoint{2.439055in}{1.089532in}}%
\pgfpathlineto{\pgfqpoint{2.442886in}{1.091604in}}%
\pgfpathlineto{\pgfqpoint{2.446716in}{1.092017in}}%
\pgfpathlineto{\pgfqpoint{2.448631in}{1.094610in}}%
\pgfpathlineto{\pgfqpoint{2.454377in}{1.095374in}}%
\pgfpathlineto{\pgfqpoint{2.469698in}{1.105210in}}%
\pgfpathlineto{\pgfqpoint{2.523323in}{1.120572in}}%
\pgfpathlineto{\pgfqpoint{2.540560in}{1.129362in}}%
\pgfpathlineto{\pgfqpoint{2.544390in}{1.130803in}}%
\pgfpathlineto{\pgfqpoint{2.546305in}{1.131678in}}%
\pgfpathlineto{\pgfqpoint{2.548220in}{1.135472in}}%
\pgfpathlineto{\pgfqpoint{2.550135in}{1.135501in}}%
\pgfpathlineto{\pgfqpoint{2.552051in}{1.138245in}}%
\pgfpathlineto{\pgfqpoint{2.569287in}{1.140683in}}%
\pgfpathlineto{\pgfqpoint{2.584608in}{1.145341in}}%
\pgfpathlineto{\pgfqpoint{2.599930in}{1.150057in}}%
\pgfpathlineto{\pgfqpoint{2.603760in}{1.153358in}}%
\pgfpathlineto{\pgfqpoint{2.613336in}{1.154237in}}%
\pgfpathlineto{\pgfqpoint{2.620997in}{1.157395in}}%
\pgfpathlineto{\pgfqpoint{2.624827in}{1.158424in}}%
\pgfpathlineto{\pgfqpoint{2.628657in}{1.161592in}}%
\pgfpathlineto{\pgfqpoint{2.634403in}{1.162027in}}%
\pgfpathlineto{\pgfqpoint{2.638233in}{1.164186in}}%
\pgfpathlineto{\pgfqpoint{2.643979in}{1.165551in}}%
\pgfpathlineto{\pgfqpoint{2.647809in}{1.168084in}}%
\pgfpathlineto{\pgfqpoint{2.655470in}{1.170708in}}%
\pgfpathlineto{\pgfqpoint{2.659300in}{1.174661in}}%
\pgfpathlineto{\pgfqpoint{2.663130in}{1.176342in}}%
\pgfpathlineto{\pgfqpoint{2.672706in}{1.178980in}}%
\pgfpathlineto{\pgfqpoint{2.674622in}{1.184777in}}%
\pgfpathlineto{\pgfqpoint{2.680367in}{1.188055in}}%
\pgfpathlineto{\pgfqpoint{2.684197in}{1.189445in}}%
\pgfpathlineto{\pgfqpoint{2.688028in}{1.190641in}}%
\pgfpathlineto{\pgfqpoint{2.693773in}{1.194410in}}%
\pgfpathlineto{\pgfqpoint{2.705264in}{1.206707in}}%
\pgfpathlineto{\pgfqpoint{2.707179in}{1.207159in}}%
\pgfpathlineto{\pgfqpoint{2.709095in}{1.208957in}}%
\pgfpathlineto{\pgfqpoint{2.718670in}{1.210562in}}%
\pgfpathlineto{\pgfqpoint{2.720586in}{1.212025in}}%
\pgfpathlineto{\pgfqpoint{2.722501in}{1.217560in}}%
\pgfpathlineto{\pgfqpoint{2.733992in}{1.219462in}}%
\pgfpathlineto{\pgfqpoint{2.753144in}{1.226461in}}%
\pgfpathlineto{\pgfqpoint{2.756974in}{1.229147in}}%
\pgfpathlineto{\pgfqpoint{2.766550in}{1.231501in}}%
\pgfpathlineto{\pgfqpoint{2.768465in}{1.232466in}}%
\pgfpathlineto{\pgfqpoint{2.772295in}{1.235922in}}%
\pgfpathlineto{\pgfqpoint{2.779956in}{1.237844in}}%
\pgfpathlineto{\pgfqpoint{2.785701in}{1.243512in}}%
\pgfpathlineto{\pgfqpoint{2.789532in}{1.248620in}}%
\pgfpathlineto{\pgfqpoint{2.795277in}{1.250336in}}%
\pgfpathlineto{\pgfqpoint{2.797192in}{1.254998in}}%
\pgfpathlineto{\pgfqpoint{2.799108in}{1.255714in}}%
\pgfpathlineto{\pgfqpoint{2.808684in}{1.266275in}}%
\pgfpathlineto{\pgfqpoint{2.814429in}{1.267728in}}%
\pgfpathlineto{\pgfqpoint{2.816344in}{1.270971in}}%
\pgfpathlineto{\pgfqpoint{2.818259in}{1.271617in}}%
\pgfpathlineto{\pgfqpoint{2.820175in}{1.274300in}}%
\pgfpathlineto{\pgfqpoint{2.822090in}{1.274873in}}%
\pgfpathlineto{\pgfqpoint{2.846987in}{1.297343in}}%
\pgfpathlineto{\pgfqpoint{2.848902in}{1.298045in}}%
\pgfpathlineto{\pgfqpoint{2.850817in}{1.301241in}}%
\pgfpathlineto{\pgfqpoint{2.852732in}{1.301421in}}%
\pgfpathlineto{\pgfqpoint{2.858478in}{1.306625in}}%
\pgfpathlineto{\pgfqpoint{2.860393in}{1.306735in}}%
\pgfpathlineto{\pgfqpoint{2.866139in}{1.314238in}}%
\pgfpathlineto{\pgfqpoint{2.868054in}{1.314547in}}%
\pgfpathlineto{\pgfqpoint{2.869969in}{1.316651in}}%
\pgfpathlineto{\pgfqpoint{2.879545in}{1.318419in}}%
\pgfpathlineto{\pgfqpoint{2.881460in}{1.323457in}}%
\pgfpathlineto{\pgfqpoint{2.883375in}{1.325154in}}%
\pgfpathlineto{\pgfqpoint{2.891036in}{1.338332in}}%
\pgfpathlineto{\pgfqpoint{2.892951in}{1.339528in}}%
\pgfpathlineto{\pgfqpoint{2.896781in}{1.348804in}}%
\pgfpathlineto{\pgfqpoint{2.900612in}{1.349348in}}%
\pgfpathlineto{\pgfqpoint{2.904442in}{1.352064in}}%
\pgfpathlineto{\pgfqpoint{2.906357in}{1.352728in}}%
\pgfpathlineto{\pgfqpoint{2.910188in}{1.362177in}}%
\pgfpathlineto{\pgfqpoint{2.912103in}{1.362651in}}%
\pgfpathlineto{\pgfqpoint{2.915933in}{1.368245in}}%
\pgfpathlineto{\pgfqpoint{2.917848in}{1.368411in}}%
\pgfpathlineto{\pgfqpoint{2.919763in}{1.372686in}}%
\pgfpathlineto{\pgfqpoint{2.921679in}{1.383317in}}%
\pgfpathlineto{\pgfqpoint{2.923594in}{1.383544in}}%
\pgfpathlineto{\pgfqpoint{2.925509in}{1.385876in}}%
\pgfpathlineto{\pgfqpoint{2.927424in}{1.390546in}}%
\pgfpathlineto{\pgfqpoint{2.929339in}{1.392253in}}%
\pgfpathlineto{\pgfqpoint{2.931254in}{1.400743in}}%
\pgfpathlineto{\pgfqpoint{2.935085in}{1.402566in}}%
\pgfpathlineto{\pgfqpoint{2.937000in}{1.426236in}}%
\pgfpathlineto{\pgfqpoint{2.938915in}{1.434149in}}%
\pgfpathlineto{\pgfqpoint{2.940830in}{1.449912in}}%
\pgfpathlineto{\pgfqpoint{2.942746in}{1.453522in}}%
\pgfpathlineto{\pgfqpoint{2.944661in}{1.453739in}}%
\pgfpathlineto{\pgfqpoint{2.946576in}{1.460477in}}%
\pgfpathlineto{\pgfqpoint{2.948491in}{1.463050in}}%
\pgfpathlineto{\pgfqpoint{2.950406in}{1.477048in}}%
\pgfpathlineto{\pgfqpoint{2.952321in}{1.479890in}}%
\pgfpathlineto{\pgfqpoint{2.954237in}{1.500516in}}%
\pgfpathlineto{\pgfqpoint{2.956152in}{1.502706in}}%
\pgfpathlineto{\pgfqpoint{2.958067in}{1.514112in}}%
\pgfpathlineto{\pgfqpoint{2.959982in}{1.514813in}}%
\pgfpathlineto{\pgfqpoint{2.963812in}{1.551074in}}%
\pgfpathlineto{\pgfqpoint{2.967643in}{1.554889in}}%
\pgfpathlineto{\pgfqpoint{2.977219in}{1.604232in}}%
\pgfpathlineto{\pgfqpoint{2.979134in}{1.604287in}}%
\pgfpathlineto{\pgfqpoint{2.981049in}{1.609002in}}%
\pgfpathlineto{\pgfqpoint{2.982964in}{1.610477in}}%
\pgfpathlineto{\pgfqpoint{2.984879in}{1.615768in}}%
\pgfpathlineto{\pgfqpoint{2.986794in}{1.616285in}}%
\pgfpathlineto{\pgfqpoint{2.988710in}{1.627537in}}%
\pgfpathlineto{\pgfqpoint{2.990625in}{1.627787in}}%
\pgfpathlineto{\pgfqpoint{2.992540in}{1.636913in}}%
\pgfpathlineto{\pgfqpoint{2.994455in}{1.653743in}}%
\pgfpathlineto{\pgfqpoint{2.996370in}{1.655288in}}%
\pgfpathlineto{\pgfqpoint{2.998285in}{1.671393in}}%
\pgfpathlineto{\pgfqpoint{3.000201in}{1.675089in}}%
\pgfpathlineto{\pgfqpoint{3.002116in}{1.699109in}}%
\pgfpathlineto{\pgfqpoint{3.004031in}{1.703650in}}%
\pgfpathlineto{\pgfqpoint{3.007861in}{1.726928in}}%
\pgfpathlineto{\pgfqpoint{3.009777in}{1.727205in}}%
\pgfpathlineto{\pgfqpoint{3.011692in}{1.738607in}}%
\pgfpathlineto{\pgfqpoint{3.017437in}{1.747913in}}%
\pgfpathlineto{\pgfqpoint{3.021268in}{1.758869in}}%
\pgfpathlineto{\pgfqpoint{3.025098in}{1.766004in}}%
\pgfpathlineto{\pgfqpoint{3.027013in}{1.786911in}}%
\pgfpathlineto{\pgfqpoint{3.032759in}{1.799776in}}%
\pgfpathlineto{\pgfqpoint{3.034674in}{1.801144in}}%
\pgfpathlineto{\pgfqpoint{3.036589in}{1.826535in}}%
\pgfpathlineto{\pgfqpoint{3.036589in}{1.826535in}}%
\pgfusepath{stroke}%
\end{pgfscope}%
\begin{pgfscope}%
\pgfpathrectangle{\pgfqpoint{0.694334in}{0.523557in}}{\pgfqpoint{3.830343in}{1.302977in}}%
\pgfusepath{clip}%
\pgfsetbuttcap%
\pgfsetroundjoin%
\pgfsetlinewidth{1.003750pt}%
\definecolor{currentstroke}{rgb}{0.000000,0.000000,0.000000}%
\pgfsetstrokecolor{currentstroke}%
\pgfsetdash{{1.000000pt}{1.650000pt}}{0.000000pt}%
\pgfpathmoveto{\pgfqpoint{0.694334in}{0.567141in}}%
\pgfpathlineto{\pgfqpoint{0.696249in}{0.582562in}}%
\pgfpathlineto{\pgfqpoint{0.698165in}{0.582755in}}%
\pgfpathlineto{\pgfqpoint{0.701995in}{0.599148in}}%
\pgfpathlineto{\pgfqpoint{0.703910in}{0.606735in}}%
\pgfpathlineto{\pgfqpoint{0.711571in}{0.615604in}}%
\pgfpathlineto{\pgfqpoint{0.713486in}{0.616358in}}%
\pgfpathlineto{\pgfqpoint{0.715401in}{0.627746in}}%
\pgfpathlineto{\pgfqpoint{0.719232in}{0.630247in}}%
\pgfpathlineto{\pgfqpoint{0.721147in}{0.632363in}}%
\pgfpathlineto{\pgfqpoint{0.724977in}{0.633447in}}%
\pgfpathlineto{\pgfqpoint{0.726892in}{0.635337in}}%
\pgfpathlineto{\pgfqpoint{0.732638in}{0.646877in}}%
\pgfpathlineto{\pgfqpoint{0.736468in}{0.649507in}}%
\pgfpathlineto{\pgfqpoint{0.738383in}{0.655445in}}%
\pgfpathlineto{\pgfqpoint{0.740298in}{0.655930in}}%
\pgfpathlineto{\pgfqpoint{0.747959in}{0.671221in}}%
\pgfpathlineto{\pgfqpoint{0.749874in}{0.675620in}}%
\pgfpathlineto{\pgfqpoint{0.751789in}{0.675686in}}%
\pgfpathlineto{\pgfqpoint{0.753705in}{0.678423in}}%
\pgfpathlineto{\pgfqpoint{0.755620in}{0.683416in}}%
\pgfpathlineto{\pgfqpoint{0.759450in}{0.685036in}}%
\pgfpathlineto{\pgfqpoint{0.761365in}{0.690456in}}%
\pgfpathlineto{\pgfqpoint{0.770941in}{0.697917in}}%
\pgfpathlineto{\pgfqpoint{0.772856in}{0.698286in}}%
\pgfpathlineto{\pgfqpoint{0.778602in}{0.704153in}}%
\pgfpathlineto{\pgfqpoint{0.793923in}{0.708539in}}%
\pgfpathlineto{\pgfqpoint{0.801584in}{0.709532in}}%
\pgfpathlineto{\pgfqpoint{0.803499in}{0.711186in}}%
\pgfpathlineto{\pgfqpoint{0.811160in}{0.712162in}}%
\pgfpathlineto{\pgfqpoint{0.814990in}{0.713108in}}%
\pgfpathlineto{\pgfqpoint{0.837972in}{0.715177in}}%
\pgfpathlineto{\pgfqpoint{0.853294in}{0.716555in}}%
\pgfpathlineto{\pgfqpoint{0.859039in}{0.717527in}}%
\pgfpathlineto{\pgfqpoint{0.864785in}{0.718514in}}%
\pgfpathlineto{\pgfqpoint{0.889682in}{0.720470in}}%
\pgfpathlineto{\pgfqpoint{0.903088in}{0.721801in}}%
\pgfpathlineto{\pgfqpoint{0.920325in}{0.724066in}}%
\pgfpathlineto{\pgfqpoint{0.947137in}{0.725055in}}%
\pgfpathlineto{\pgfqpoint{1.010338in}{0.726970in}}%
\pgfpathlineto{\pgfqpoint{1.027574in}{0.727996in}}%
\pgfpathlineto{\pgfqpoint{1.067793in}{0.730388in}}%
\pgfpathlineto{\pgfqpoint{1.075453in}{0.730995in}}%
\pgfpathlineto{\pgfqpoint{1.299528in}{0.744595in}}%
\pgfpathlineto{\pgfqpoint{1.303359in}{0.745473in}}%
\pgfpathlineto{\pgfqpoint{1.328256in}{0.746903in}}%
\pgfpathlineto{\pgfqpoint{1.370390in}{0.750858in}}%
\pgfpathlineto{\pgfqpoint{1.385711in}{0.751900in}}%
\pgfpathlineto{\pgfqpoint{1.404863in}{0.753175in}}%
\pgfpathlineto{\pgfqpoint{1.416354in}{0.754161in}}%
\pgfpathlineto{\pgfqpoint{1.422099in}{0.755207in}}%
\pgfpathlineto{\pgfqpoint{1.433590in}{0.756474in}}%
\pgfpathlineto{\pgfqpoint{1.437421in}{0.757393in}}%
\pgfpathlineto{\pgfqpoint{1.452742in}{0.758832in}}%
\pgfpathlineto{\pgfqpoint{1.491046in}{0.765101in}}%
\pgfpathlineto{\pgfqpoint{1.498706in}{0.765845in}}%
\pgfpathlineto{\pgfqpoint{1.508282in}{0.766470in}}%
\pgfpathlineto{\pgfqpoint{1.519773in}{0.771921in}}%
\pgfpathlineto{\pgfqpoint{1.523604in}{0.772598in}}%
\pgfpathlineto{\pgfqpoint{1.531264in}{0.777376in}}%
\pgfpathlineto{\pgfqpoint{1.535095in}{0.778714in}}%
\pgfpathlineto{\pgfqpoint{1.537010in}{0.783849in}}%
\pgfpathlineto{\pgfqpoint{1.538925in}{0.784911in}}%
\pgfpathlineto{\pgfqpoint{1.540840in}{0.788811in}}%
\pgfpathlineto{\pgfqpoint{1.552331in}{0.791007in}}%
\pgfpathlineto{\pgfqpoint{1.554246in}{0.791602in}}%
\pgfpathlineto{\pgfqpoint{1.561907in}{0.798592in}}%
\pgfpathlineto{\pgfqpoint{1.569568in}{0.799587in}}%
\pgfpathlineto{\pgfqpoint{1.573398in}{0.802172in}}%
\pgfpathlineto{\pgfqpoint{1.577228in}{0.802534in}}%
\pgfpathlineto{\pgfqpoint{1.582974in}{0.806461in}}%
\pgfpathlineto{\pgfqpoint{1.590635in}{0.808069in}}%
\pgfpathlineto{\pgfqpoint{1.607871in}{0.810257in}}%
\pgfpathlineto{\pgfqpoint{1.613617in}{0.811082in}}%
\pgfpathlineto{\pgfqpoint{1.623192in}{0.811795in}}%
\pgfpathlineto{\pgfqpoint{1.636599in}{0.816468in}}%
\pgfpathlineto{\pgfqpoint{1.642344in}{0.816816in}}%
\pgfpathlineto{\pgfqpoint{1.646174in}{0.817868in}}%
\pgfpathlineto{\pgfqpoint{1.703630in}{0.828093in}}%
\pgfpathlineto{\pgfqpoint{1.715121in}{0.829443in}}%
\pgfpathlineto{\pgfqpoint{1.722781in}{0.831705in}}%
\pgfpathlineto{\pgfqpoint{1.730442in}{0.832697in}}%
\pgfpathlineto{\pgfqpoint{1.759170in}{0.835613in}}%
\pgfpathlineto{\pgfqpoint{1.763000in}{0.836460in}}%
\pgfpathlineto{\pgfqpoint{1.778321in}{0.838136in}}%
\pgfpathlineto{\pgfqpoint{1.782152in}{0.839452in}}%
\pgfpathlineto{\pgfqpoint{1.803219in}{0.843519in}}%
\pgfpathlineto{\pgfqpoint{1.818540in}{0.845975in}}%
\pgfpathlineto{\pgfqpoint{1.822370in}{0.846987in}}%
\pgfpathlineto{\pgfqpoint{1.826201in}{0.847835in}}%
\pgfpathlineto{\pgfqpoint{1.831946in}{0.849615in}}%
\pgfpathlineto{\pgfqpoint{1.835776in}{0.851432in}}%
\pgfpathlineto{\pgfqpoint{1.839607in}{0.852136in}}%
\pgfpathlineto{\pgfqpoint{1.843437in}{0.854187in}}%
\pgfpathlineto{\pgfqpoint{1.847267in}{0.855217in}}%
\pgfpathlineto{\pgfqpoint{1.864504in}{0.859580in}}%
\pgfpathlineto{\pgfqpoint{1.879825in}{0.860892in}}%
\pgfpathlineto{\pgfqpoint{1.891316in}{0.863940in}}%
\pgfpathlineto{\pgfqpoint{1.908553in}{0.865804in}}%
\pgfpathlineto{\pgfqpoint{1.944941in}{0.870873in}}%
\pgfpathlineto{\pgfqpoint{1.950687in}{0.872051in}}%
\pgfpathlineto{\pgfqpoint{2.002396in}{0.880407in}}%
\pgfpathlineto{\pgfqpoint{2.006227in}{0.881360in}}%
\pgfpathlineto{\pgfqpoint{2.013887in}{0.882640in}}%
\pgfpathlineto{\pgfqpoint{2.019633in}{0.882946in}}%
\pgfpathlineto{\pgfqpoint{2.023463in}{0.885084in}}%
\pgfpathlineto{\pgfqpoint{2.027294in}{0.885827in}}%
\pgfpathlineto{\pgfqpoint{2.031124in}{0.887468in}}%
\pgfpathlineto{\pgfqpoint{2.046445in}{0.889403in}}%
\pgfpathlineto{\pgfqpoint{2.113476in}{0.897402in}}%
\pgfpathlineto{\pgfqpoint{2.117307in}{0.898536in}}%
\pgfpathlineto{\pgfqpoint{2.130713in}{0.900570in}}%
\pgfpathlineto{\pgfqpoint{2.134543in}{0.901537in}}%
\pgfpathlineto{\pgfqpoint{2.142204in}{0.902500in}}%
\pgfpathlineto{\pgfqpoint{2.147949in}{0.903786in}}%
\pgfpathlineto{\pgfqpoint{2.193914in}{0.910651in}}%
\pgfpathlineto{\pgfqpoint{2.207320in}{0.911405in}}%
\pgfpathlineto{\pgfqpoint{2.213065in}{0.913407in}}%
\pgfpathlineto{\pgfqpoint{2.218811in}{0.914494in}}%
\pgfpathlineto{\pgfqpoint{2.222641in}{0.915106in}}%
\pgfpathlineto{\pgfqpoint{2.230302in}{0.917520in}}%
\pgfpathlineto{\pgfqpoint{2.239878in}{0.918703in}}%
\pgfpathlineto{\pgfqpoint{2.243708in}{0.920150in}}%
\pgfpathlineto{\pgfqpoint{2.264775in}{0.923191in}}%
\pgfpathlineto{\pgfqpoint{2.326060in}{0.934156in}}%
\pgfpathlineto{\pgfqpoint{2.329891in}{0.935857in}}%
\pgfpathlineto{\pgfqpoint{2.343297in}{0.937293in}}%
\pgfpathlineto{\pgfqpoint{2.350958in}{0.938427in}}%
\pgfpathlineto{\pgfqpoint{2.366279in}{0.939420in}}%
\pgfpathlineto{\pgfqpoint{2.372024in}{0.941202in}}%
\pgfpathlineto{\pgfqpoint{2.383515in}{0.942902in}}%
\pgfpathlineto{\pgfqpoint{2.410328in}{0.945045in}}%
\pgfpathlineto{\pgfqpoint{2.423734in}{0.945656in}}%
\pgfpathlineto{\pgfqpoint{2.429480in}{0.946929in}}%
\pgfpathlineto{\pgfqpoint{2.444801in}{0.947880in}}%
\pgfpathlineto{\pgfqpoint{2.467783in}{0.950815in}}%
\pgfpathlineto{\pgfqpoint{2.479274in}{0.954592in}}%
\pgfpathlineto{\pgfqpoint{2.511832in}{0.964688in}}%
\pgfpathlineto{\pgfqpoint{2.517577in}{0.970839in}}%
\pgfpathlineto{\pgfqpoint{2.519493in}{0.970888in}}%
\pgfpathlineto{\pgfqpoint{2.521408in}{0.972769in}}%
\pgfpathlineto{\pgfqpoint{2.527153in}{0.973870in}}%
\pgfpathlineto{\pgfqpoint{2.530984in}{0.975766in}}%
\pgfpathlineto{\pgfqpoint{2.534814in}{0.980023in}}%
\pgfpathlineto{\pgfqpoint{2.548220in}{0.982244in}}%
\pgfpathlineto{\pgfqpoint{2.555881in}{0.987072in}}%
\pgfpathlineto{\pgfqpoint{2.559711in}{0.987375in}}%
\pgfpathlineto{\pgfqpoint{2.561626in}{0.989794in}}%
\pgfpathlineto{\pgfqpoint{2.565457in}{0.990829in}}%
\pgfpathlineto{\pgfqpoint{2.575033in}{0.992834in}}%
\pgfpathlineto{\pgfqpoint{2.580778in}{0.995368in}}%
\pgfpathlineto{\pgfqpoint{2.584608in}{0.996058in}}%
\pgfpathlineto{\pgfqpoint{2.586524in}{0.999027in}}%
\pgfpathlineto{\pgfqpoint{2.592269in}{1.000545in}}%
\pgfpathlineto{\pgfqpoint{2.601845in}{1.001857in}}%
\pgfpathlineto{\pgfqpoint{2.607591in}{1.004198in}}%
\pgfpathlineto{\pgfqpoint{2.615251in}{1.006342in}}%
\pgfpathlineto{\pgfqpoint{2.617166in}{1.007769in}}%
\pgfpathlineto{\pgfqpoint{2.624827in}{1.008569in}}%
\pgfpathlineto{\pgfqpoint{2.647809in}{1.015743in}}%
\pgfpathlineto{\pgfqpoint{2.653555in}{1.017738in}}%
\pgfpathlineto{\pgfqpoint{2.661215in}{1.018503in}}%
\pgfpathlineto{\pgfqpoint{2.666961in}{1.021593in}}%
\pgfpathlineto{\pgfqpoint{2.668876in}{1.021639in}}%
\pgfpathlineto{\pgfqpoint{2.676537in}{1.029089in}}%
\pgfpathlineto{\pgfqpoint{2.682282in}{1.029692in}}%
\pgfpathlineto{\pgfqpoint{2.684197in}{1.032096in}}%
\pgfpathlineto{\pgfqpoint{2.688028in}{1.032551in}}%
\pgfpathlineto{\pgfqpoint{2.689943in}{1.036236in}}%
\pgfpathlineto{\pgfqpoint{2.693773in}{1.037453in}}%
\pgfpathlineto{\pgfqpoint{2.699519in}{1.041072in}}%
\pgfpathlineto{\pgfqpoint{2.705264in}{1.041844in}}%
\pgfpathlineto{\pgfqpoint{2.711010in}{1.046404in}}%
\pgfpathlineto{\pgfqpoint{2.712925in}{1.046713in}}%
\pgfpathlineto{\pgfqpoint{2.714840in}{1.048649in}}%
\pgfpathlineto{\pgfqpoint{2.718670in}{1.048973in}}%
\pgfpathlineto{\pgfqpoint{2.722501in}{1.051883in}}%
\pgfpathlineto{\pgfqpoint{2.730161in}{1.053131in}}%
\pgfpathlineto{\pgfqpoint{2.733992in}{1.058431in}}%
\pgfpathlineto{\pgfqpoint{2.737822in}{1.060002in}}%
\pgfpathlineto{\pgfqpoint{2.739737in}{1.062244in}}%
\pgfpathlineto{\pgfqpoint{2.745483in}{1.063362in}}%
\pgfpathlineto{\pgfqpoint{2.749313in}{1.066143in}}%
\pgfpathlineto{\pgfqpoint{2.756974in}{1.068559in}}%
\pgfpathlineto{\pgfqpoint{2.758889in}{1.070863in}}%
\pgfpathlineto{\pgfqpoint{2.760804in}{1.071197in}}%
\pgfpathlineto{\pgfqpoint{2.762719in}{1.073376in}}%
\pgfpathlineto{\pgfqpoint{2.764635in}{1.073698in}}%
\pgfpathlineto{\pgfqpoint{2.766550in}{1.076698in}}%
\pgfpathlineto{\pgfqpoint{2.781871in}{1.081531in}}%
\pgfpathlineto{\pgfqpoint{2.789532in}{1.091942in}}%
\pgfpathlineto{\pgfqpoint{2.802938in}{1.097733in}}%
\pgfpathlineto{\pgfqpoint{2.816344in}{1.113734in}}%
\pgfpathlineto{\pgfqpoint{2.820175in}{1.115461in}}%
\pgfpathlineto{\pgfqpoint{2.822090in}{1.118249in}}%
\pgfpathlineto{\pgfqpoint{2.824005in}{1.118961in}}%
\pgfpathlineto{\pgfqpoint{2.825920in}{1.121740in}}%
\pgfpathlineto{\pgfqpoint{2.827835in}{1.121876in}}%
\pgfpathlineto{\pgfqpoint{2.831666in}{1.128837in}}%
\pgfpathlineto{\pgfqpoint{2.841241in}{1.132505in}}%
\pgfpathlineto{\pgfqpoint{2.843157in}{1.135682in}}%
\pgfpathlineto{\pgfqpoint{2.845072in}{1.142341in}}%
\pgfpathlineto{\pgfqpoint{2.846987in}{1.143000in}}%
\pgfpathlineto{\pgfqpoint{2.848902in}{1.150761in}}%
\pgfpathlineto{\pgfqpoint{2.850817in}{1.152351in}}%
\pgfpathlineto{\pgfqpoint{2.852732in}{1.162566in}}%
\pgfpathlineto{\pgfqpoint{2.858478in}{1.165194in}}%
\pgfpathlineto{\pgfqpoint{2.860393in}{1.166312in}}%
\pgfpathlineto{\pgfqpoint{2.862308in}{1.176447in}}%
\pgfpathlineto{\pgfqpoint{2.875715in}{1.189905in}}%
\pgfpathlineto{\pgfqpoint{2.877630in}{1.200352in}}%
\pgfpathlineto{\pgfqpoint{2.879545in}{1.200960in}}%
\pgfpathlineto{\pgfqpoint{2.883375in}{1.214235in}}%
\pgfpathlineto{\pgfqpoint{2.885290in}{1.223561in}}%
\pgfpathlineto{\pgfqpoint{2.887206in}{1.223605in}}%
\pgfpathlineto{\pgfqpoint{2.892951in}{1.233278in}}%
\pgfpathlineto{\pgfqpoint{2.896781in}{1.235146in}}%
\pgfpathlineto{\pgfqpoint{2.898697in}{1.235612in}}%
\pgfpathlineto{\pgfqpoint{2.902527in}{1.240602in}}%
\pgfpathlineto{\pgfqpoint{2.908272in}{1.241978in}}%
\pgfpathlineto{\pgfqpoint{2.910188in}{1.245462in}}%
\pgfpathlineto{\pgfqpoint{2.914018in}{1.258053in}}%
\pgfpathlineto{\pgfqpoint{2.915933in}{1.258087in}}%
\pgfpathlineto{\pgfqpoint{2.917848in}{1.259950in}}%
\pgfpathlineto{\pgfqpoint{2.919763in}{1.265087in}}%
\pgfpathlineto{\pgfqpoint{2.929339in}{1.266600in}}%
\pgfpathlineto{\pgfqpoint{2.931254in}{1.269136in}}%
\pgfpathlineto{\pgfqpoint{2.933170in}{1.274322in}}%
\pgfpathlineto{\pgfqpoint{2.935085in}{1.275324in}}%
\pgfpathlineto{\pgfqpoint{2.937000in}{1.286711in}}%
\pgfpathlineto{\pgfqpoint{2.942746in}{1.296020in}}%
\pgfpathlineto{\pgfqpoint{2.950406in}{1.339754in}}%
\pgfpathlineto{\pgfqpoint{2.954237in}{1.353078in}}%
\pgfpathlineto{\pgfqpoint{2.956152in}{1.366320in}}%
\pgfpathlineto{\pgfqpoint{2.958067in}{1.367354in}}%
\pgfpathlineto{\pgfqpoint{2.961897in}{1.375775in}}%
\pgfpathlineto{\pgfqpoint{2.963812in}{1.401096in}}%
\pgfpathlineto{\pgfqpoint{2.965728in}{1.402566in}}%
\pgfpathlineto{\pgfqpoint{2.967643in}{1.422730in}}%
\pgfpathlineto{\pgfqpoint{2.969558in}{1.427997in}}%
\pgfpathlineto{\pgfqpoint{2.971473in}{1.438897in}}%
\pgfpathlineto{\pgfqpoint{2.973388in}{1.463050in}}%
\pgfpathlineto{\pgfqpoint{2.977219in}{1.472723in}}%
\pgfpathlineto{\pgfqpoint{2.979134in}{1.500516in}}%
\pgfpathlineto{\pgfqpoint{2.982964in}{1.515955in}}%
\pgfpathlineto{\pgfqpoint{2.994455in}{1.564788in}}%
\pgfpathlineto{\pgfqpoint{2.996370in}{1.566155in}}%
\pgfpathlineto{\pgfqpoint{2.998285in}{1.571164in}}%
\pgfpathlineto{\pgfqpoint{3.000201in}{1.572768in}}%
\pgfpathlineto{\pgfqpoint{3.002116in}{1.579641in}}%
\pgfpathlineto{\pgfqpoint{3.005946in}{1.602936in}}%
\pgfpathlineto{\pgfqpoint{3.007861in}{1.606766in}}%
\pgfpathlineto{\pgfqpoint{3.009777in}{1.616119in}}%
\pgfpathlineto{\pgfqpoint{3.011692in}{1.616285in}}%
\pgfpathlineto{\pgfqpoint{3.013607in}{1.623297in}}%
\pgfpathlineto{\pgfqpoint{3.017437in}{1.627683in}}%
\pgfpathlineto{\pgfqpoint{3.019352in}{1.627787in}}%
\pgfpathlineto{\pgfqpoint{3.025098in}{1.654026in}}%
\pgfpathlineto{\pgfqpoint{3.027013in}{1.673286in}}%
\pgfpathlineto{\pgfqpoint{3.028928in}{1.679975in}}%
\pgfpathlineto{\pgfqpoint{3.032759in}{1.703650in}}%
\pgfpathlineto{\pgfqpoint{3.034674in}{1.709293in}}%
\pgfpathlineto{\pgfqpoint{3.036589in}{1.709480in}}%
\pgfpathlineto{\pgfqpoint{3.044250in}{1.716044in}}%
\pgfpathlineto{\pgfqpoint{3.046165in}{1.741674in}}%
\pgfpathlineto{\pgfqpoint{3.048080in}{1.742266in}}%
\pgfpathlineto{\pgfqpoint{3.051910in}{1.791670in}}%
\pgfpathlineto{\pgfqpoint{3.053825in}{1.826535in}}%
\pgfpathlineto{\pgfqpoint{3.053825in}{1.826535in}}%
\pgfusepath{stroke}%
\end{pgfscope}%
\begin{pgfscope}%
\pgfsetrectcap%
\pgfsetmiterjoin%
\pgfsetlinewidth{0.803000pt}%
\definecolor{currentstroke}{rgb}{0.000000,0.000000,0.000000}%
\pgfsetstrokecolor{currentstroke}%
\pgfsetdash{}{0pt}%
\pgfpathmoveto{\pgfqpoint{0.694334in}{0.523557in}}%
\pgfpathlineto{\pgfqpoint{0.694334in}{1.826535in}}%
\pgfusepath{stroke}%
\end{pgfscope}%
\begin{pgfscope}%
\pgfsetrectcap%
\pgfsetmiterjoin%
\pgfsetlinewidth{0.803000pt}%
\definecolor{currentstroke}{rgb}{0.000000,0.000000,0.000000}%
\pgfsetstrokecolor{currentstroke}%
\pgfsetdash{}{0pt}%
\pgfpathmoveto{\pgfqpoint{4.524677in}{0.523557in}}%
\pgfpathlineto{\pgfqpoint{4.524677in}{1.826535in}}%
\pgfusepath{stroke}%
\end{pgfscope}%
\begin{pgfscope}%
\pgfsetrectcap%
\pgfsetmiterjoin%
\pgfsetlinewidth{0.803000pt}%
\definecolor{currentstroke}{rgb}{0.000000,0.000000,0.000000}%
\pgfsetstrokecolor{currentstroke}%
\pgfsetdash{}{0pt}%
\pgfpathmoveto{\pgfqpoint{0.694334in}{0.523557in}}%
\pgfpathlineto{\pgfqpoint{4.524677in}{0.523557in}}%
\pgfusepath{stroke}%
\end{pgfscope}%
\begin{pgfscope}%
\pgfsetrectcap%
\pgfsetmiterjoin%
\pgfsetlinewidth{0.803000pt}%
\definecolor{currentstroke}{rgb}{0.000000,0.000000,0.000000}%
\pgfsetstrokecolor{currentstroke}%
\pgfsetdash{}{0pt}%
\pgfpathmoveto{\pgfqpoint{0.694334in}{1.826535in}}%
\pgfpathlineto{\pgfqpoint{4.524677in}{1.826535in}}%
\pgfusepath{stroke}%
\end{pgfscope}%
\begin{pgfscope}%
\pgfsetrectcap%
\pgfsetroundjoin%
\pgfsetlinewidth{1.003750pt}%
\definecolor{currentstroke}{rgb}{0.878431,0.878431,0.815686}%
\pgfsetstrokecolor{currentstroke}%
\pgfsetdash{}{0pt}%
\pgfpathmoveto{\pgfqpoint{3.867012in}{1.491422in}}%
\pgfpathlineto{\pgfqpoint{4.089235in}{1.491422in}}%
\pgfusepath{stroke}%
\end{pgfscope}%
\begin{pgfscope}%
\definecolor{textcolor}{rgb}{0.000000,0.000000,0.000000}%
\pgfsetstrokecolor{textcolor}%
\pgfsetfillcolor{textcolor}%
\pgftext[x=4.111457in,y=1.452533in,left,base]{\color{textcolor}\rmfamily\fontsize{8.000000}{9.600000}\selectfont T.}%
\end{pgfscope}%
\begin{pgfscope}%
\pgfsetbuttcap%
\pgfsetroundjoin%
\pgfsetlinewidth{1.003750pt}%
\definecolor{currentstroke}{rgb}{0.941176,0.627451,0.188235}%
\pgfsetstrokecolor{currentstroke}%
\pgfsetdash{{1.000000pt}{1.650000pt}}{0.000000pt}%
\pgfpathmoveto{\pgfqpoint{3.867012in}{1.347600in}}%
\pgfpathlineto{\pgfqpoint{4.089235in}{1.347600in}}%
\pgfusepath{stroke}%
\end{pgfscope}%
\begin{pgfscope}%
\definecolor{textcolor}{rgb}{0.000000,0.000000,0.000000}%
\pgfsetstrokecolor{textcolor}%
\pgfsetfillcolor{textcolor}%
\pgftext[x=4.111457in,y=1.308711in,left,base]{\color{textcolor}\rmfamily\fontsize{8.000000}{9.600000}\selectfont FlowC.}%
\end{pgfscope}%
\begin{pgfscope}%
\pgfsetbuttcap%
\pgfsetroundjoin%
\pgfsetlinewidth{1.003750pt}%
\definecolor{currentstroke}{rgb}{0.062745,0.000000,0.062745}%
\pgfsetstrokecolor{currentstroke}%
\pgfsetdash{{3.700000pt}{1.600000pt}}{0.000000pt}%
\pgfpathmoveto{\pgfqpoint{3.867012in}{1.203778in}}%
\pgfpathlineto{\pgfqpoint{4.089235in}{1.203778in}}%
\pgfusepath{stroke}%
\end{pgfscope}%
\begin{pgfscope}%
\definecolor{textcolor}{rgb}{0.000000,0.000000,0.000000}%
\pgfsetstrokecolor{textcolor}%
\pgfsetfillcolor{textcolor}%
\pgftext[x=4.111457in,y=1.164889in,left,base]{\color{textcolor}\rmfamily\fontsize{8.000000}{9.600000}\selectfont htd}%
\end{pgfscope}%
\begin{pgfscope}%
\pgfsetbuttcap%
\pgfsetroundjoin%
\pgfsetlinewidth{1.003750pt}%
\definecolor{currentstroke}{rgb}{0.811765,0.125490,0.125490}%
\pgfsetstrokecolor{currentstroke}%
\pgfsetdash{{1.000000pt}{1.650000pt}}{0.000000pt}%
\pgfpathmoveto{\pgfqpoint{3.867012in}{1.059956in}}%
\pgfpathlineto{\pgfqpoint{4.089235in}{1.059956in}}%
\pgfusepath{stroke}%
\end{pgfscope}%
\begin{pgfscope}%
\definecolor{textcolor}{rgb}{0.000000,0.000000,0.000000}%
\pgfsetstrokecolor{textcolor}%
\pgfsetfillcolor{textcolor}%
\pgftext[x=4.111457in,y=1.021067in,left,base]{\color{textcolor}\rmfamily\fontsize{8.000000}{9.600000}\selectfont Hicks}%
\end{pgfscope}%
\begin{pgfscope}%
\pgfsetrectcap%
\pgfsetroundjoin%
\pgfsetlinewidth{1.003750pt}%
\definecolor{currentstroke}{rgb}{0.000000,0.000000,0.376471}%
\pgfsetstrokecolor{currentstroke}%
\pgfsetdash{}{0pt}%
\pgfpathmoveto{\pgfqpoint{3.867012in}{0.916134in}}%
\pgfpathlineto{\pgfqpoint{4.089235in}{0.916134in}}%
\pgfusepath{stroke}%
\end{pgfscope}%
\begin{pgfscope}%
\definecolor{textcolor}{rgb}{0.000000,0.000000,0.000000}%
\pgfsetstrokecolor{textcolor}%
\pgfsetfillcolor{textcolor}%
\pgftext[x=4.111457in,y=0.877245in,left,base]{\color{textcolor}\rmfamily\fontsize{8.000000}{9.600000}\selectfont P3}%
\end{pgfscope}%
\begin{pgfscope}%
\pgfsetrectcap%
\pgfsetroundjoin%
\pgfsetlinewidth{1.003750pt}%
\definecolor{currentstroke}{rgb}{0.564706,0.564706,1.000000}%
\pgfsetstrokecolor{currentstroke}%
\pgfsetdash{}{0pt}%
\pgfpathmoveto{\pgfqpoint{3.867012in}{0.772312in}}%
\pgfpathlineto{\pgfqpoint{4.089235in}{0.772312in}}%
\pgfusepath{stroke}%
\end{pgfscope}%
\begin{pgfscope}%
\definecolor{textcolor}{rgb}{0.000000,0.000000,0.000000}%
\pgfsetstrokecolor{textcolor}%
\pgfsetfillcolor{textcolor}%
\pgftext[x=4.111457in,y=0.733423in,left,base]{\color{textcolor}\rmfamily\fontsize{8.000000}{9.600000}\selectfont P4}%
\end{pgfscope}%
\begin{pgfscope}%
\pgfsetbuttcap%
\pgfsetroundjoin%
\pgfsetlinewidth{1.003750pt}%
\definecolor{currentstroke}{rgb}{0.000000,0.000000,0.000000}%
\pgfsetstrokecolor{currentstroke}%
\pgfsetdash{{1.000000pt}{1.650000pt}}{0.000000pt}%
\pgfpathmoveto{\pgfqpoint{3.867012in}{0.628490in}}%
\pgfpathlineto{\pgfqpoint{4.089235in}{0.628490in}}%
\pgfusepath{stroke}%
\end{pgfscope}%
\begin{pgfscope}%
\definecolor{textcolor}{rgb}{0.000000,0.000000,0.000000}%
\pgfsetstrokecolor{textcolor}%
\pgfsetfillcolor{textcolor}%
\pgftext[x=4.111457in,y=0.589601in,left,base]{\color{textcolor}\rmfamily\fontsize{8.000000}{9.600000}\selectfont VBS}%
\end{pgfscope}%
\end{pgfpicture}%
\makeatother%
\endgroup%

%\includegraphics[height=2in,width=2.5in]{figures/planning.pdf}
%\caption{\label{fig:parallel:planning} A cactus plot of the performance of various planners. A planner ``solves'' a benchmark when it finds a contraction tree of max rank 30 or smaller.}
%\end{center}
%\end{figure}


We observe that the parallel portfolio planners outperform all four single-core planners after 5 seconds. In fact, after 20 seconds both portfolios perform almost as well as the virtual best solver. We conclude that portfolio solvers significantly speed up the planning phase.

We also observe that \pkg{P3} and \pkg{P4} perform almost identically in Figure \ref{fig:parallel:planning}. Although after 1000 seconds \pkg{P4} has found better contraction trees than \pkg{P3} on 407 benchmarks, most improvements are small (reducing the max-rank by 1 or 2) or still do not result in good-enough contraction trees. We conclude that adding \pkg{Hicks} improves the portfolio slightly, but not significantly.
 
\subsection{Experiment 2: Determining the Performance Factor (RQ3)}
\label{sec:experiments:pf}
We take each contraction tree discovered in Experiment 1 (with max-rank below 36) and use \tool{TensorOrder2} to execute the tree with a timeout of 1000 seconds on each of three hardware configurations (\pkg{CPU1}, \pkg{CPU8}, and \pkg{GPU}). We observe that the max-rank of almost all solved contraction trees is 30 or smaller.

Given a performance factor, a benchmark, and a planner, we use the planning times from Experiment 1 to determine which contraction tree would have been chosen in step 4 of Algorithm \ref{alg:wmc}. We then add the execution time of the relevant contraction tree on each hardware. In this way, we simulate Algorithm \ref{alg:wmc} for a given planner and hardware with many performance factors. 

\begin{figure}[t]
\begin{center}
%% Creator: Matplotlib, PGF backend
%%
%% To include the figure in your LaTeX document, write
%%   \input{<filename>.pgf}
%%
%% Make sure the required packages are loaded in your preamble
%%   \usepackage{pgf}
%%
%% and, on pdftex
%%   \usepackage[utf8]{inputenc}\DeclareUnicodeCharacter{2212}{-}
%%
%% or, on luatex and xetex
%%   \usepackage{unicode-math}
%%
%% Figures using additional raster images can only be included by \input if
%% they are in the same directory as the main LaTeX file. For loading figures
%% from other directories you can use the `import` package
%%   \usepackage{import}
%%
%% and then include the figures with
%%   \import{<path to file>}{<filename>.pgf}
%%
%% Matplotlib used the following preamble
%%   \usepackage[utf8x]{inputenc}
%%   \usepackage[T1]{fontenc}
%%
\begingroup%
\makeatletter%
\begin{pgfpicture}%
\pgfpathrectangle{\pgfpointorigin}{\pgfqpoint{6.000000in}{2.500000in}}%
\pgfusepath{use as bounding box, clip}%
\begin{pgfscope}%
\pgfsetbuttcap%
\pgfsetmiterjoin%
\definecolor{currentfill}{rgb}{1.000000,1.000000,1.000000}%
\pgfsetfillcolor{currentfill}%
\pgfsetlinewidth{0.000000pt}%
\definecolor{currentstroke}{rgb}{1.000000,1.000000,1.000000}%
\pgfsetstrokecolor{currentstroke}%
\pgfsetdash{}{0pt}%
\pgfpathmoveto{\pgfqpoint{0.000000in}{0.000000in}}%
\pgfpathlineto{\pgfqpoint{6.000000in}{0.000000in}}%
\pgfpathlineto{\pgfqpoint{6.000000in}{2.500000in}}%
\pgfpathlineto{\pgfqpoint{0.000000in}{2.500000in}}%
\pgfpathclose%
\pgfusepath{fill}%
\end{pgfscope}%
\begin{pgfscope}%
\pgfsetbuttcap%
\pgfsetmiterjoin%
\definecolor{currentfill}{rgb}{1.000000,1.000000,1.000000}%
\pgfsetfillcolor{currentfill}%
\pgfsetlinewidth{0.000000pt}%
\definecolor{currentstroke}{rgb}{0.000000,0.000000,0.000000}%
\pgfsetstrokecolor{currentstroke}%
\pgfsetstrokeopacity{0.000000}%
\pgfsetdash{}{0pt}%
\pgfpathmoveto{\pgfqpoint{0.589591in}{0.539182in}}%
\pgfpathlineto{\pgfqpoint{5.756830in}{0.539182in}}%
\pgfpathlineto{\pgfqpoint{5.756830in}{2.207310in}}%
\pgfpathlineto{\pgfqpoint{0.589591in}{2.207310in}}%
\pgfpathclose%
\pgfusepath{fill}%
\end{pgfscope}%
\begin{pgfscope}%
\pgfsetbuttcap%
\pgfsetroundjoin%
\definecolor{currentfill}{rgb}{0.000000,0.000000,0.000000}%
\pgfsetfillcolor{currentfill}%
\pgfsetlinewidth{0.803000pt}%
\definecolor{currentstroke}{rgb}{0.000000,0.000000,0.000000}%
\pgfsetstrokecolor{currentstroke}%
\pgfsetdash{}{0pt}%
\pgfsys@defobject{currentmarker}{\pgfqpoint{0.000000in}{-0.048611in}}{\pgfqpoint{0.000000in}{0.000000in}}{%
\pgfpathmoveto{\pgfqpoint{0.000000in}{0.000000in}}%
\pgfpathlineto{\pgfqpoint{0.000000in}{-0.048611in}}%
\pgfusepath{stroke,fill}%
}%
\begin{pgfscope}%
\pgfsys@transformshift{0.589591in}{0.539182in}%
\pgfsys@useobject{currentmarker}{}%
\end{pgfscope}%
\end{pgfscope}%
\begin{pgfscope}%
\definecolor{textcolor}{rgb}{0.000000,0.000000,0.000000}%
\pgfsetstrokecolor{textcolor}%
\pgfsetfillcolor{textcolor}%
\pgftext[x=0.589591in,y=0.441960in,,top]{\color{textcolor}\rmfamily\fontsize{9.000000}{10.800000}\selectfont \(\displaystyle {10^{-21}}\)}%
\end{pgfscope}%
\begin{pgfscope}%
\pgfsetbuttcap%
\pgfsetroundjoin%
\definecolor{currentfill}{rgb}{0.000000,0.000000,0.000000}%
\pgfsetfillcolor{currentfill}%
\pgfsetlinewidth{0.803000pt}%
\definecolor{currentstroke}{rgb}{0.000000,0.000000,0.000000}%
\pgfsetstrokecolor{currentstroke}%
\pgfsetdash{}{0pt}%
\pgfsys@defobject{currentmarker}{\pgfqpoint{0.000000in}{-0.048611in}}{\pgfqpoint{0.000000in}{0.000000in}}{%
\pgfpathmoveto{\pgfqpoint{0.000000in}{0.000000in}}%
\pgfpathlineto{\pgfqpoint{0.000000in}{-0.048611in}}%
\pgfusepath{stroke,fill}%
}%
\begin{pgfscope}%
\pgfsys@transformshift{1.327768in}{0.539182in}%
\pgfsys@useobject{currentmarker}{}%
\end{pgfscope}%
\end{pgfscope}%
\begin{pgfscope}%
\definecolor{textcolor}{rgb}{0.000000,0.000000,0.000000}%
\pgfsetstrokecolor{textcolor}%
\pgfsetfillcolor{textcolor}%
\pgftext[x=1.327768in,y=0.441960in,,top]{\color{textcolor}\rmfamily\fontsize{9.000000}{10.800000}\selectfont \(\displaystyle {10^{-18}}\)}%
\end{pgfscope}%
\begin{pgfscope}%
\pgfsetbuttcap%
\pgfsetroundjoin%
\definecolor{currentfill}{rgb}{0.000000,0.000000,0.000000}%
\pgfsetfillcolor{currentfill}%
\pgfsetlinewidth{0.803000pt}%
\definecolor{currentstroke}{rgb}{0.000000,0.000000,0.000000}%
\pgfsetstrokecolor{currentstroke}%
\pgfsetdash{}{0pt}%
\pgfsys@defobject{currentmarker}{\pgfqpoint{0.000000in}{-0.048611in}}{\pgfqpoint{0.000000in}{0.000000in}}{%
\pgfpathmoveto{\pgfqpoint{0.000000in}{0.000000in}}%
\pgfpathlineto{\pgfqpoint{0.000000in}{-0.048611in}}%
\pgfusepath{stroke,fill}%
}%
\begin{pgfscope}%
\pgfsys@transformshift{2.065945in}{0.539182in}%
\pgfsys@useobject{currentmarker}{}%
\end{pgfscope}%
\end{pgfscope}%
\begin{pgfscope}%
\definecolor{textcolor}{rgb}{0.000000,0.000000,0.000000}%
\pgfsetstrokecolor{textcolor}%
\pgfsetfillcolor{textcolor}%
\pgftext[x=2.065945in,y=0.441960in,,top]{\color{textcolor}\rmfamily\fontsize{9.000000}{10.800000}\selectfont \(\displaystyle {10^{-15}}\)}%
\end{pgfscope}%
\begin{pgfscope}%
\pgfsetbuttcap%
\pgfsetroundjoin%
\definecolor{currentfill}{rgb}{0.000000,0.000000,0.000000}%
\pgfsetfillcolor{currentfill}%
\pgfsetlinewidth{0.803000pt}%
\definecolor{currentstroke}{rgb}{0.000000,0.000000,0.000000}%
\pgfsetstrokecolor{currentstroke}%
\pgfsetdash{}{0pt}%
\pgfsys@defobject{currentmarker}{\pgfqpoint{0.000000in}{-0.048611in}}{\pgfqpoint{0.000000in}{0.000000in}}{%
\pgfpathmoveto{\pgfqpoint{0.000000in}{0.000000in}}%
\pgfpathlineto{\pgfqpoint{0.000000in}{-0.048611in}}%
\pgfusepath{stroke,fill}%
}%
\begin{pgfscope}%
\pgfsys@transformshift{2.804122in}{0.539182in}%
\pgfsys@useobject{currentmarker}{}%
\end{pgfscope}%
\end{pgfscope}%
\begin{pgfscope}%
\definecolor{textcolor}{rgb}{0.000000,0.000000,0.000000}%
\pgfsetstrokecolor{textcolor}%
\pgfsetfillcolor{textcolor}%
\pgftext[x=2.804122in,y=0.441960in,,top]{\color{textcolor}\rmfamily\fontsize{9.000000}{10.800000}\selectfont \(\displaystyle {10^{-12}}\)}%
\end{pgfscope}%
\begin{pgfscope}%
\pgfsetbuttcap%
\pgfsetroundjoin%
\definecolor{currentfill}{rgb}{0.000000,0.000000,0.000000}%
\pgfsetfillcolor{currentfill}%
\pgfsetlinewidth{0.803000pt}%
\definecolor{currentstroke}{rgb}{0.000000,0.000000,0.000000}%
\pgfsetstrokecolor{currentstroke}%
\pgfsetdash{}{0pt}%
\pgfsys@defobject{currentmarker}{\pgfqpoint{0.000000in}{-0.048611in}}{\pgfqpoint{0.000000in}{0.000000in}}{%
\pgfpathmoveto{\pgfqpoint{0.000000in}{0.000000in}}%
\pgfpathlineto{\pgfqpoint{0.000000in}{-0.048611in}}%
\pgfusepath{stroke,fill}%
}%
\begin{pgfscope}%
\pgfsys@transformshift{3.542299in}{0.539182in}%
\pgfsys@useobject{currentmarker}{}%
\end{pgfscope}%
\end{pgfscope}%
\begin{pgfscope}%
\definecolor{textcolor}{rgb}{0.000000,0.000000,0.000000}%
\pgfsetstrokecolor{textcolor}%
\pgfsetfillcolor{textcolor}%
\pgftext[x=3.542299in,y=0.441960in,,top]{\color{textcolor}\rmfamily\fontsize{9.000000}{10.800000}\selectfont \(\displaystyle {10^{-9}}\)}%
\end{pgfscope}%
\begin{pgfscope}%
\pgfsetbuttcap%
\pgfsetroundjoin%
\definecolor{currentfill}{rgb}{0.000000,0.000000,0.000000}%
\pgfsetfillcolor{currentfill}%
\pgfsetlinewidth{0.803000pt}%
\definecolor{currentstroke}{rgb}{0.000000,0.000000,0.000000}%
\pgfsetstrokecolor{currentstroke}%
\pgfsetdash{}{0pt}%
\pgfsys@defobject{currentmarker}{\pgfqpoint{0.000000in}{-0.048611in}}{\pgfqpoint{0.000000in}{0.000000in}}{%
\pgfpathmoveto{\pgfqpoint{0.000000in}{0.000000in}}%
\pgfpathlineto{\pgfqpoint{0.000000in}{-0.048611in}}%
\pgfusepath{stroke,fill}%
}%
\begin{pgfscope}%
\pgfsys@transformshift{4.280476in}{0.539182in}%
\pgfsys@useobject{currentmarker}{}%
\end{pgfscope}%
\end{pgfscope}%
\begin{pgfscope}%
\definecolor{textcolor}{rgb}{0.000000,0.000000,0.000000}%
\pgfsetstrokecolor{textcolor}%
\pgfsetfillcolor{textcolor}%
\pgftext[x=4.280476in,y=0.441960in,,top]{\color{textcolor}\rmfamily\fontsize{9.000000}{10.800000}\selectfont \(\displaystyle {10^{-6}}\)}%
\end{pgfscope}%
\begin{pgfscope}%
\pgfsetbuttcap%
\pgfsetroundjoin%
\definecolor{currentfill}{rgb}{0.000000,0.000000,0.000000}%
\pgfsetfillcolor{currentfill}%
\pgfsetlinewidth{0.803000pt}%
\definecolor{currentstroke}{rgb}{0.000000,0.000000,0.000000}%
\pgfsetstrokecolor{currentstroke}%
\pgfsetdash{}{0pt}%
\pgfsys@defobject{currentmarker}{\pgfqpoint{0.000000in}{-0.048611in}}{\pgfqpoint{0.000000in}{0.000000in}}{%
\pgfpathmoveto{\pgfqpoint{0.000000in}{0.000000in}}%
\pgfpathlineto{\pgfqpoint{0.000000in}{-0.048611in}}%
\pgfusepath{stroke,fill}%
}%
\begin{pgfscope}%
\pgfsys@transformshift{5.018653in}{0.539182in}%
\pgfsys@useobject{currentmarker}{}%
\end{pgfscope}%
\end{pgfscope}%
\begin{pgfscope}%
\definecolor{textcolor}{rgb}{0.000000,0.000000,0.000000}%
\pgfsetstrokecolor{textcolor}%
\pgfsetfillcolor{textcolor}%
\pgftext[x=5.018653in,y=0.441960in,,top]{\color{textcolor}\rmfamily\fontsize{9.000000}{10.800000}\selectfont \(\displaystyle {10^{-3}}\)}%
\end{pgfscope}%
\begin{pgfscope}%
\pgfsetbuttcap%
\pgfsetroundjoin%
\definecolor{currentfill}{rgb}{0.000000,0.000000,0.000000}%
\pgfsetfillcolor{currentfill}%
\pgfsetlinewidth{0.803000pt}%
\definecolor{currentstroke}{rgb}{0.000000,0.000000,0.000000}%
\pgfsetstrokecolor{currentstroke}%
\pgfsetdash{}{0pt}%
\pgfsys@defobject{currentmarker}{\pgfqpoint{0.000000in}{-0.048611in}}{\pgfqpoint{0.000000in}{0.000000in}}{%
\pgfpathmoveto{\pgfqpoint{0.000000in}{0.000000in}}%
\pgfpathlineto{\pgfqpoint{0.000000in}{-0.048611in}}%
\pgfusepath{stroke,fill}%
}%
\begin{pgfscope}%
\pgfsys@transformshift{5.756830in}{0.539182in}%
\pgfsys@useobject{currentmarker}{}%
\end{pgfscope}%
\end{pgfscope}%
\begin{pgfscope}%
\definecolor{textcolor}{rgb}{0.000000,0.000000,0.000000}%
\pgfsetstrokecolor{textcolor}%
\pgfsetfillcolor{textcolor}%
\pgftext[x=5.756830in,y=0.441960in,,top]{\color{textcolor}\rmfamily\fontsize{9.000000}{10.800000}\selectfont \(\displaystyle {10^{0}}\)}%
\end{pgfscope}%
\begin{pgfscope}%
\definecolor{textcolor}{rgb}{0.000000,0.000000,0.000000}%
\pgfsetstrokecolor{textcolor}%
\pgfsetfillcolor{textcolor}%
\pgftext[x=3.173210in,y=0.272655in,,top]{\color{textcolor}\rmfamily\fontsize{10.000000}{12.000000}\selectfont Performance factor}%
\end{pgfscope}%
\begin{pgfscope}%
\pgfsetbuttcap%
\pgfsetroundjoin%
\definecolor{currentfill}{rgb}{0.000000,0.000000,0.000000}%
\pgfsetfillcolor{currentfill}%
\pgfsetlinewidth{0.803000pt}%
\definecolor{currentstroke}{rgb}{0.000000,0.000000,0.000000}%
\pgfsetstrokecolor{currentstroke}%
\pgfsetdash{}{0pt}%
\pgfsys@defobject{currentmarker}{\pgfqpoint{-0.048611in}{0.000000in}}{\pgfqpoint{-0.000000in}{0.000000in}}{%
\pgfpathmoveto{\pgfqpoint{-0.000000in}{0.000000in}}%
\pgfpathlineto{\pgfqpoint{-0.048611in}{0.000000in}}%
\pgfusepath{stroke,fill}%
}%
\begin{pgfscope}%
\pgfsys@transformshift{0.589591in}{0.539182in}%
\pgfsys@useobject{currentmarker}{}%
\end{pgfscope}%
\end{pgfscope}%
\begin{pgfscope}%
\definecolor{textcolor}{rgb}{0.000000,0.000000,0.000000}%
\pgfsetstrokecolor{textcolor}%
\pgfsetfillcolor{textcolor}%
\pgftext[x=0.328211in, y=0.496137in, left, base]{\color{textcolor}\rmfamily\fontsize{9.000000}{10.800000}\selectfont \(\displaystyle {1.0}\)}%
\end{pgfscope}%
\begin{pgfscope}%
\pgfsetbuttcap%
\pgfsetroundjoin%
\definecolor{currentfill}{rgb}{0.000000,0.000000,0.000000}%
\pgfsetfillcolor{currentfill}%
\pgfsetlinewidth{0.803000pt}%
\definecolor{currentstroke}{rgb}{0.000000,0.000000,0.000000}%
\pgfsetstrokecolor{currentstroke}%
\pgfsetdash{}{0pt}%
\pgfsys@defobject{currentmarker}{\pgfqpoint{-0.048611in}{0.000000in}}{\pgfqpoint{-0.000000in}{0.000000in}}{%
\pgfpathmoveto{\pgfqpoint{-0.000000in}{0.000000in}}%
\pgfpathlineto{\pgfqpoint{-0.048611in}{0.000000in}}%
\pgfusepath{stroke,fill}%
}%
\begin{pgfscope}%
\pgfsys@transformshift{0.589591in}{0.817203in}%
\pgfsys@useobject{currentmarker}{}%
\end{pgfscope}%
\end{pgfscope}%
\begin{pgfscope}%
\definecolor{textcolor}{rgb}{0.000000,0.000000,0.000000}%
\pgfsetstrokecolor{textcolor}%
\pgfsetfillcolor{textcolor}%
\pgftext[x=0.328211in, y=0.774158in, left, base]{\color{textcolor}\rmfamily\fontsize{9.000000}{10.800000}\selectfont \(\displaystyle {1.5}\)}%
\end{pgfscope}%
\begin{pgfscope}%
\pgfsetbuttcap%
\pgfsetroundjoin%
\definecolor{currentfill}{rgb}{0.000000,0.000000,0.000000}%
\pgfsetfillcolor{currentfill}%
\pgfsetlinewidth{0.803000pt}%
\definecolor{currentstroke}{rgb}{0.000000,0.000000,0.000000}%
\pgfsetstrokecolor{currentstroke}%
\pgfsetdash{}{0pt}%
\pgfsys@defobject{currentmarker}{\pgfqpoint{-0.048611in}{0.000000in}}{\pgfqpoint{-0.000000in}{0.000000in}}{%
\pgfpathmoveto{\pgfqpoint{-0.000000in}{0.000000in}}%
\pgfpathlineto{\pgfqpoint{-0.048611in}{0.000000in}}%
\pgfusepath{stroke,fill}%
}%
\begin{pgfscope}%
\pgfsys@transformshift{0.589591in}{1.095225in}%
\pgfsys@useobject{currentmarker}{}%
\end{pgfscope}%
\end{pgfscope}%
\begin{pgfscope}%
\definecolor{textcolor}{rgb}{0.000000,0.000000,0.000000}%
\pgfsetstrokecolor{textcolor}%
\pgfsetfillcolor{textcolor}%
\pgftext[x=0.328211in, y=1.052180in, left, base]{\color{textcolor}\rmfamily\fontsize{9.000000}{10.800000}\selectfont \(\displaystyle {2.0}\)}%
\end{pgfscope}%
\begin{pgfscope}%
\pgfsetbuttcap%
\pgfsetroundjoin%
\definecolor{currentfill}{rgb}{0.000000,0.000000,0.000000}%
\pgfsetfillcolor{currentfill}%
\pgfsetlinewidth{0.803000pt}%
\definecolor{currentstroke}{rgb}{0.000000,0.000000,0.000000}%
\pgfsetstrokecolor{currentstroke}%
\pgfsetdash{}{0pt}%
\pgfsys@defobject{currentmarker}{\pgfqpoint{-0.048611in}{0.000000in}}{\pgfqpoint{-0.000000in}{0.000000in}}{%
\pgfpathmoveto{\pgfqpoint{-0.000000in}{0.000000in}}%
\pgfpathlineto{\pgfqpoint{-0.048611in}{0.000000in}}%
\pgfusepath{stroke,fill}%
}%
\begin{pgfscope}%
\pgfsys@transformshift{0.589591in}{1.373246in}%
\pgfsys@useobject{currentmarker}{}%
\end{pgfscope}%
\end{pgfscope}%
\begin{pgfscope}%
\definecolor{textcolor}{rgb}{0.000000,0.000000,0.000000}%
\pgfsetstrokecolor{textcolor}%
\pgfsetfillcolor{textcolor}%
\pgftext[x=0.328211in, y=1.330201in, left, base]{\color{textcolor}\rmfamily\fontsize{9.000000}{10.800000}\selectfont \(\displaystyle {2.5}\)}%
\end{pgfscope}%
\begin{pgfscope}%
\pgfsetbuttcap%
\pgfsetroundjoin%
\definecolor{currentfill}{rgb}{0.000000,0.000000,0.000000}%
\pgfsetfillcolor{currentfill}%
\pgfsetlinewidth{0.803000pt}%
\definecolor{currentstroke}{rgb}{0.000000,0.000000,0.000000}%
\pgfsetstrokecolor{currentstroke}%
\pgfsetdash{}{0pt}%
\pgfsys@defobject{currentmarker}{\pgfqpoint{-0.048611in}{0.000000in}}{\pgfqpoint{-0.000000in}{0.000000in}}{%
\pgfpathmoveto{\pgfqpoint{-0.000000in}{0.000000in}}%
\pgfpathlineto{\pgfqpoint{-0.048611in}{0.000000in}}%
\pgfusepath{stroke,fill}%
}%
\begin{pgfscope}%
\pgfsys@transformshift{0.589591in}{1.651267in}%
\pgfsys@useobject{currentmarker}{}%
\end{pgfscope}%
\end{pgfscope}%
\begin{pgfscope}%
\definecolor{textcolor}{rgb}{0.000000,0.000000,0.000000}%
\pgfsetstrokecolor{textcolor}%
\pgfsetfillcolor{textcolor}%
\pgftext[x=0.328211in, y=1.608222in, left, base]{\color{textcolor}\rmfamily\fontsize{9.000000}{10.800000}\selectfont \(\displaystyle {3.0}\)}%
\end{pgfscope}%
\begin{pgfscope}%
\pgfsetbuttcap%
\pgfsetroundjoin%
\definecolor{currentfill}{rgb}{0.000000,0.000000,0.000000}%
\pgfsetfillcolor{currentfill}%
\pgfsetlinewidth{0.803000pt}%
\definecolor{currentstroke}{rgb}{0.000000,0.000000,0.000000}%
\pgfsetstrokecolor{currentstroke}%
\pgfsetdash{}{0pt}%
\pgfsys@defobject{currentmarker}{\pgfqpoint{-0.048611in}{0.000000in}}{\pgfqpoint{-0.000000in}{0.000000in}}{%
\pgfpathmoveto{\pgfqpoint{-0.000000in}{0.000000in}}%
\pgfpathlineto{\pgfqpoint{-0.048611in}{0.000000in}}%
\pgfusepath{stroke,fill}%
}%
\begin{pgfscope}%
\pgfsys@transformshift{0.589591in}{1.929289in}%
\pgfsys@useobject{currentmarker}{}%
\end{pgfscope}%
\end{pgfscope}%
\begin{pgfscope}%
\definecolor{textcolor}{rgb}{0.000000,0.000000,0.000000}%
\pgfsetstrokecolor{textcolor}%
\pgfsetfillcolor{textcolor}%
\pgftext[x=0.328211in, y=1.886244in, left, base]{\color{textcolor}\rmfamily\fontsize{9.000000}{10.800000}\selectfont \(\displaystyle {3.5}\)}%
\end{pgfscope}%
\begin{pgfscope}%
\pgfsetbuttcap%
\pgfsetroundjoin%
\definecolor{currentfill}{rgb}{0.000000,0.000000,0.000000}%
\pgfsetfillcolor{currentfill}%
\pgfsetlinewidth{0.803000pt}%
\definecolor{currentstroke}{rgb}{0.000000,0.000000,0.000000}%
\pgfsetstrokecolor{currentstroke}%
\pgfsetdash{}{0pt}%
\pgfsys@defobject{currentmarker}{\pgfqpoint{-0.048611in}{0.000000in}}{\pgfqpoint{-0.000000in}{0.000000in}}{%
\pgfpathmoveto{\pgfqpoint{-0.000000in}{0.000000in}}%
\pgfpathlineto{\pgfqpoint{-0.048611in}{0.000000in}}%
\pgfusepath{stroke,fill}%
}%
\begin{pgfscope}%
\pgfsys@transformshift{0.589591in}{2.207310in}%
\pgfsys@useobject{currentmarker}{}%
\end{pgfscope}%
\end{pgfscope}%
\begin{pgfscope}%
\definecolor{textcolor}{rgb}{0.000000,0.000000,0.000000}%
\pgfsetstrokecolor{textcolor}%
\pgfsetfillcolor{textcolor}%
\pgftext[x=0.328211in, y=2.164265in, left, base]{\color{textcolor}\rmfamily\fontsize{9.000000}{10.800000}\selectfont \(\displaystyle {4.0}\)}%
\end{pgfscope}%
\begin{pgfscope}%
\definecolor{textcolor}{rgb}{0.000000,0.000000,0.000000}%
\pgfsetstrokecolor{textcolor}%
\pgfsetfillcolor{textcolor}%
\pgftext[x=0.272655in,y=1.373246in,,bottom,rotate=90.000000]{\color{textcolor}\rmfamily\fontsize{10.000000}{12.000000}\selectfont Par-2 Score}%
\end{pgfscope}%
\begin{pgfscope}%
\definecolor{textcolor}{rgb}{0.000000,0.000000,0.000000}%
\pgfsetstrokecolor{textcolor}%
\pgfsetfillcolor{textcolor}%
\pgftext[x=0.589591in,y=2.248977in,left,base]{\color{textcolor}\rmfamily\fontsize{9.000000}{10.800000}\selectfont \(\displaystyle \times{10^{6}}{}\)}%
\end{pgfscope}%
\begin{pgfscope}%
\pgfpathrectangle{\pgfqpoint{0.589591in}{0.539182in}}{\pgfqpoint{5.167239in}{1.668128in}}%
\pgfusepath{clip}%
\pgfsetrectcap%
\pgfsetroundjoin%
\pgfsetlinewidth{2.007500pt}%
\definecolor{currentstroke}{rgb}{0.878431,0.878431,0.815686}%
\pgfsetstrokecolor{currentstroke}%
\pgfsetdash{}{0pt}%
\pgfpathmoveto{\pgfqpoint{0.589591in}{1.040488in}}%
\pgfpathlineto{\pgfqpoint{2.480360in}{1.039663in}}%
\pgfpathlineto{\pgfqpoint{2.493310in}{1.037549in}}%
\pgfpathlineto{\pgfqpoint{2.519211in}{1.026804in}}%
\pgfpathlineto{\pgfqpoint{2.532162in}{1.024864in}}%
\pgfpathlineto{\pgfqpoint{2.558063in}{1.013854in}}%
\pgfpathlineto{\pgfqpoint{2.583964in}{1.012934in}}%
\pgfpathlineto{\pgfqpoint{2.596914in}{1.012936in}}%
\pgfpathlineto{\pgfqpoint{2.622815in}{1.006427in}}%
\pgfpathlineto{\pgfqpoint{2.635766in}{1.004050in}}%
\pgfpathlineto{\pgfqpoint{2.648716in}{1.003890in}}%
\pgfpathlineto{\pgfqpoint{2.661666in}{1.000309in}}%
\pgfpathlineto{\pgfqpoint{2.674617in}{0.998233in}}%
\pgfpathlineto{\pgfqpoint{2.687567in}{0.992822in}}%
\pgfpathlineto{\pgfqpoint{2.713468in}{0.985138in}}%
\pgfpathlineto{\pgfqpoint{2.739369in}{0.983393in}}%
\pgfpathlineto{\pgfqpoint{2.778221in}{0.982029in}}%
\pgfpathlineto{\pgfqpoint{2.791171in}{0.980952in}}%
\pgfpathlineto{\pgfqpoint{2.804122in}{0.978287in}}%
\pgfpathlineto{\pgfqpoint{2.817072in}{0.977749in}}%
\pgfpathlineto{\pgfqpoint{2.830023in}{0.974856in}}%
\pgfpathlineto{\pgfqpoint{2.855924in}{0.966158in}}%
\pgfpathlineto{\pgfqpoint{2.868874in}{0.963842in}}%
\pgfpathlineto{\pgfqpoint{2.920676in}{0.960755in}}%
\pgfpathlineto{\pgfqpoint{2.972478in}{0.957262in}}%
\pgfpathlineto{\pgfqpoint{3.011329in}{0.957113in}}%
\pgfpathlineto{\pgfqpoint{3.024280in}{0.954684in}}%
\pgfpathlineto{\pgfqpoint{3.037230in}{0.954823in}}%
\pgfpathlineto{\pgfqpoint{3.050181in}{0.953507in}}%
\pgfpathlineto{\pgfqpoint{3.140834in}{0.953633in}}%
\pgfpathlineto{\pgfqpoint{3.153784in}{0.948381in}}%
\pgfpathlineto{\pgfqpoint{3.179685in}{0.948061in}}%
\pgfpathlineto{\pgfqpoint{3.192636in}{0.946396in}}%
\pgfpathlineto{\pgfqpoint{3.218537in}{0.949757in}}%
\pgfpathlineto{\pgfqpoint{3.257388in}{0.955597in}}%
\pgfpathlineto{\pgfqpoint{3.309190in}{0.964278in}}%
\pgfpathlineto{\pgfqpoint{3.322141in}{0.967193in}}%
\pgfpathlineto{\pgfqpoint{3.335091in}{0.972583in}}%
\pgfpathlineto{\pgfqpoint{3.348042in}{0.981004in}}%
\pgfpathlineto{\pgfqpoint{3.360992in}{0.983059in}}%
\pgfpathlineto{\pgfqpoint{3.373943in}{0.986532in}}%
\pgfpathlineto{\pgfqpoint{3.386893in}{0.992476in}}%
\pgfpathlineto{\pgfqpoint{3.412794in}{1.002130in}}%
\pgfpathlineto{\pgfqpoint{3.425744in}{1.007404in}}%
\pgfpathlineto{\pgfqpoint{3.451645in}{1.022222in}}%
\pgfpathlineto{\pgfqpoint{3.464596in}{1.025961in}}%
\pgfpathlineto{\pgfqpoint{3.477546in}{1.031902in}}%
\pgfpathlineto{\pgfqpoint{3.490497in}{1.036027in}}%
\pgfpathlineto{\pgfqpoint{3.503447in}{1.041360in}}%
\pgfpathlineto{\pgfqpoint{3.516398in}{1.049876in}}%
\pgfpathlineto{\pgfqpoint{3.529348in}{1.056469in}}%
\pgfpathlineto{\pgfqpoint{3.555249in}{1.064256in}}%
\pgfpathlineto{\pgfqpoint{3.568200in}{1.069172in}}%
\pgfpathlineto{\pgfqpoint{3.594101in}{1.087396in}}%
\pgfpathlineto{\pgfqpoint{3.620001in}{1.095614in}}%
\pgfpathlineto{\pgfqpoint{3.632952in}{1.102032in}}%
\pgfpathlineto{\pgfqpoint{3.658853in}{1.111848in}}%
\pgfpathlineto{\pgfqpoint{3.671803in}{1.121528in}}%
\pgfpathlineto{\pgfqpoint{3.710655in}{1.139906in}}%
\pgfpathlineto{\pgfqpoint{3.723605in}{1.150725in}}%
\pgfpathlineto{\pgfqpoint{3.736556in}{1.164632in}}%
\pgfpathlineto{\pgfqpoint{3.749506in}{1.176624in}}%
\pgfpathlineto{\pgfqpoint{3.775407in}{1.195470in}}%
\pgfpathlineto{\pgfqpoint{3.788358in}{1.206382in}}%
\pgfpathlineto{\pgfqpoint{3.801308in}{1.213930in}}%
\pgfpathlineto{\pgfqpoint{3.814259in}{1.223198in}}%
\pgfpathlineto{\pgfqpoint{3.827209in}{1.241666in}}%
\pgfpathlineto{\pgfqpoint{3.840160in}{1.253000in}}%
\pgfpathlineto{\pgfqpoint{3.853110in}{1.262622in}}%
\pgfpathlineto{\pgfqpoint{3.866060in}{1.274948in}}%
\pgfpathlineto{\pgfqpoint{3.879011in}{1.288843in}}%
\pgfpathlineto{\pgfqpoint{3.891961in}{1.295646in}}%
\pgfpathlineto{\pgfqpoint{3.904912in}{1.312084in}}%
\pgfpathlineto{\pgfqpoint{3.917862in}{1.317889in}}%
\pgfpathlineto{\pgfqpoint{3.930813in}{1.330949in}}%
\pgfpathlineto{\pgfqpoint{3.969664in}{1.360481in}}%
\pgfpathlineto{\pgfqpoint{3.982615in}{1.372694in}}%
\pgfpathlineto{\pgfqpoint{3.995565in}{1.388388in}}%
\pgfpathlineto{\pgfqpoint{4.021466in}{1.405892in}}%
\pgfpathlineto{\pgfqpoint{4.034417in}{1.418422in}}%
\pgfpathlineto{\pgfqpoint{4.060318in}{1.433322in}}%
\pgfpathlineto{\pgfqpoint{4.073268in}{1.442226in}}%
\pgfpathlineto{\pgfqpoint{4.086219in}{1.454238in}}%
\pgfpathlineto{\pgfqpoint{4.099169in}{1.474156in}}%
\pgfpathlineto{\pgfqpoint{4.125070in}{1.488553in}}%
\pgfpathlineto{\pgfqpoint{4.138020in}{1.496902in}}%
\pgfpathlineto{\pgfqpoint{4.150971in}{1.503411in}}%
\pgfpathlineto{\pgfqpoint{4.163921in}{1.511942in}}%
\pgfpathlineto{\pgfqpoint{4.176872in}{1.534191in}}%
\pgfpathlineto{\pgfqpoint{4.189822in}{1.542153in}}%
\pgfpathlineto{\pgfqpoint{4.202773in}{1.547057in}}%
\pgfpathlineto{\pgfqpoint{4.228674in}{1.566072in}}%
\pgfpathlineto{\pgfqpoint{4.241624in}{1.575114in}}%
\pgfpathlineto{\pgfqpoint{4.254575in}{1.588344in}}%
\pgfpathlineto{\pgfqpoint{4.267525in}{1.596950in}}%
\pgfpathlineto{\pgfqpoint{4.280476in}{1.608742in}}%
\pgfpathlineto{\pgfqpoint{4.293426in}{1.613631in}}%
\pgfpathlineto{\pgfqpoint{4.306377in}{1.620590in}}%
\pgfpathlineto{\pgfqpoint{4.319327in}{1.631089in}}%
\pgfpathlineto{\pgfqpoint{4.345228in}{1.645738in}}%
\pgfpathlineto{\pgfqpoint{4.358178in}{1.662941in}}%
\pgfpathlineto{\pgfqpoint{4.384079in}{1.671680in}}%
\pgfpathlineto{\pgfqpoint{4.397030in}{1.681043in}}%
\pgfpathlineto{\pgfqpoint{4.409980in}{1.686577in}}%
\pgfpathlineto{\pgfqpoint{4.422931in}{1.694544in}}%
\pgfpathlineto{\pgfqpoint{4.435881in}{1.705597in}}%
\pgfpathlineto{\pgfqpoint{4.448832in}{1.718792in}}%
\pgfpathlineto{\pgfqpoint{4.474733in}{1.728012in}}%
\pgfpathlineto{\pgfqpoint{4.500634in}{1.742895in}}%
\pgfpathlineto{\pgfqpoint{4.513584in}{1.754285in}}%
\pgfpathlineto{\pgfqpoint{4.565386in}{1.766337in}}%
\pgfpathlineto{\pgfqpoint{4.604237in}{1.783060in}}%
\pgfpathlineto{\pgfqpoint{4.630138in}{1.797654in}}%
\pgfpathlineto{\pgfqpoint{4.643089in}{1.801893in}}%
\pgfpathlineto{\pgfqpoint{4.656039in}{1.807402in}}%
\pgfpathlineto{\pgfqpoint{4.668990in}{1.814779in}}%
\pgfpathlineto{\pgfqpoint{4.681940in}{1.826236in}}%
\pgfpathlineto{\pgfqpoint{4.694891in}{1.831666in}}%
\pgfpathlineto{\pgfqpoint{4.733742in}{1.844044in}}%
\pgfpathlineto{\pgfqpoint{4.746693in}{1.850300in}}%
\pgfpathlineto{\pgfqpoint{4.759643in}{1.862724in}}%
\pgfpathlineto{\pgfqpoint{4.772594in}{1.866630in}}%
\pgfpathlineto{\pgfqpoint{4.785544in}{1.871712in}}%
\pgfpathlineto{\pgfqpoint{4.798495in}{1.880722in}}%
\pgfpathlineto{\pgfqpoint{4.811445in}{1.887308in}}%
\pgfpathlineto{\pgfqpoint{4.837346in}{1.897288in}}%
\pgfpathlineto{\pgfqpoint{4.850296in}{1.908231in}}%
\pgfpathlineto{\pgfqpoint{4.863247in}{1.912059in}}%
\pgfpathlineto{\pgfqpoint{4.889148in}{1.916021in}}%
\pgfpathlineto{\pgfqpoint{4.940950in}{1.927327in}}%
\pgfpathlineto{\pgfqpoint{4.966851in}{1.936331in}}%
\pgfpathlineto{\pgfqpoint{4.992752in}{1.944095in}}%
\pgfpathlineto{\pgfqpoint{5.005702in}{1.964786in}}%
\pgfpathlineto{\pgfqpoint{5.031603in}{1.970852in}}%
\pgfpathlineto{\pgfqpoint{5.057504in}{1.975974in}}%
\pgfpathlineto{\pgfqpoint{5.083405in}{1.983277in}}%
\pgfpathlineto{\pgfqpoint{5.096355in}{1.987557in}}%
\pgfpathlineto{\pgfqpoint{5.109306in}{1.993093in}}%
\pgfpathlineto{\pgfqpoint{5.122256in}{2.001531in}}%
\pgfpathlineto{\pgfqpoint{5.148157in}{2.013109in}}%
\pgfpathlineto{\pgfqpoint{5.174058in}{2.035884in}}%
\pgfpathlineto{\pgfqpoint{5.187009in}{2.043029in}}%
\pgfpathlineto{\pgfqpoint{5.212910in}{2.049343in}}%
\pgfpathlineto{\pgfqpoint{5.238811in}{2.056006in}}%
\pgfpathlineto{\pgfqpoint{5.251761in}{2.057933in}}%
\pgfpathlineto{\pgfqpoint{5.277662in}{2.065237in}}%
\pgfpathlineto{\pgfqpoint{5.290612in}{2.069674in}}%
\pgfpathlineto{\pgfqpoint{5.355365in}{2.080347in}}%
\pgfpathlineto{\pgfqpoint{5.368315in}{2.081183in}}%
\pgfpathlineto{\pgfqpoint{5.381266in}{2.084245in}}%
\pgfpathlineto{\pgfqpoint{5.420117in}{2.088768in}}%
\pgfpathlineto{\pgfqpoint{5.484870in}{2.096828in}}%
\pgfpathlineto{\pgfqpoint{5.588473in}{2.104973in}}%
\pgfpathlineto{\pgfqpoint{5.601424in}{2.107539in}}%
\pgfpathlineto{\pgfqpoint{5.666176in}{2.109487in}}%
\pgfpathlineto{\pgfqpoint{5.705028in}{2.110835in}}%
\pgfpathlineto{\pgfqpoint{5.756830in}{2.111008in}}%
\pgfpathlineto{\pgfqpoint{5.756830in}{2.111008in}}%
\pgfusepath{stroke}%
\end{pgfscope}%
\begin{pgfscope}%
\pgfpathrectangle{\pgfqpoint{0.589591in}{0.539182in}}{\pgfqpoint{5.167239in}{1.668128in}}%
\pgfusepath{clip}%
\pgfsetrectcap%
\pgfsetroundjoin%
\pgfsetlinewidth{2.007500pt}%
\definecolor{currentstroke}{rgb}{0.564706,0.564706,1.000000}%
\pgfsetstrokecolor{currentstroke}%
\pgfsetdash{}{0pt}%
\pgfpathmoveto{\pgfqpoint{0.589591in}{1.141162in}}%
\pgfpathlineto{\pgfqpoint{2.091846in}{1.140419in}}%
\pgfpathlineto{\pgfqpoint{2.117747in}{1.139720in}}%
\pgfpathlineto{\pgfqpoint{2.143648in}{1.138656in}}%
\pgfpathlineto{\pgfqpoint{2.169549in}{1.135874in}}%
\pgfpathlineto{\pgfqpoint{2.195449in}{1.132690in}}%
\pgfpathlineto{\pgfqpoint{2.208400in}{1.129944in}}%
\pgfpathlineto{\pgfqpoint{2.221350in}{1.128853in}}%
\pgfpathlineto{\pgfqpoint{2.234301in}{1.122390in}}%
\pgfpathlineto{\pgfqpoint{2.247251in}{1.120663in}}%
\pgfpathlineto{\pgfqpoint{2.260202in}{1.117789in}}%
\pgfpathlineto{\pgfqpoint{2.273152in}{1.110911in}}%
\pgfpathlineto{\pgfqpoint{2.299053in}{1.100582in}}%
\pgfpathlineto{\pgfqpoint{2.337905in}{1.086978in}}%
\pgfpathlineto{\pgfqpoint{2.350855in}{1.083801in}}%
\pgfpathlineto{\pgfqpoint{2.363806in}{1.076697in}}%
\pgfpathlineto{\pgfqpoint{2.376756in}{1.071233in}}%
\pgfpathlineto{\pgfqpoint{2.389707in}{1.063530in}}%
\pgfpathlineto{\pgfqpoint{2.402657in}{1.060845in}}%
\pgfpathlineto{\pgfqpoint{2.415608in}{1.049817in}}%
\pgfpathlineto{\pgfqpoint{2.428558in}{1.044466in}}%
\pgfpathlineto{\pgfqpoint{2.441508in}{1.037224in}}%
\pgfpathlineto{\pgfqpoint{2.454459in}{1.036138in}}%
\pgfpathlineto{\pgfqpoint{2.493310in}{1.027783in}}%
\pgfpathlineto{\pgfqpoint{2.519211in}{1.019563in}}%
\pgfpathlineto{\pgfqpoint{2.532162in}{1.013324in}}%
\pgfpathlineto{\pgfqpoint{2.545112in}{1.010740in}}%
\pgfpathlineto{\pgfqpoint{2.558063in}{1.002716in}}%
\pgfpathlineto{\pgfqpoint{2.583964in}{0.994151in}}%
\pgfpathlineto{\pgfqpoint{2.596914in}{0.992079in}}%
\pgfpathlineto{\pgfqpoint{2.609865in}{0.987055in}}%
\pgfpathlineto{\pgfqpoint{2.622815in}{0.985092in}}%
\pgfpathlineto{\pgfqpoint{2.674617in}{0.965235in}}%
\pgfpathlineto{\pgfqpoint{2.713468in}{0.952577in}}%
\pgfpathlineto{\pgfqpoint{2.726419in}{0.950639in}}%
\pgfpathlineto{\pgfqpoint{2.739369in}{0.950322in}}%
\pgfpathlineto{\pgfqpoint{2.765270in}{0.938983in}}%
\pgfpathlineto{\pgfqpoint{2.778221in}{0.930496in}}%
\pgfpathlineto{\pgfqpoint{2.791171in}{0.924158in}}%
\pgfpathlineto{\pgfqpoint{2.804122in}{0.920495in}}%
\pgfpathlineto{\pgfqpoint{2.842973in}{0.917418in}}%
\pgfpathlineto{\pgfqpoint{2.868874in}{0.914342in}}%
\pgfpathlineto{\pgfqpoint{2.907725in}{0.910463in}}%
\pgfpathlineto{\pgfqpoint{2.920676in}{0.907141in}}%
\pgfpathlineto{\pgfqpoint{2.959527in}{0.904252in}}%
\pgfpathlineto{\pgfqpoint{3.050181in}{0.902195in}}%
\pgfpathlineto{\pgfqpoint{3.089032in}{0.901814in}}%
\pgfpathlineto{\pgfqpoint{3.101983in}{0.900079in}}%
\pgfpathlineto{\pgfqpoint{3.114933in}{0.900099in}}%
\pgfpathlineto{\pgfqpoint{3.127884in}{0.902018in}}%
\pgfpathlineto{\pgfqpoint{3.153784in}{0.902608in}}%
\pgfpathlineto{\pgfqpoint{3.166735in}{0.904886in}}%
\pgfpathlineto{\pgfqpoint{3.179685in}{0.908433in}}%
\pgfpathlineto{\pgfqpoint{3.192636in}{0.908297in}}%
\pgfpathlineto{\pgfqpoint{3.205586in}{0.910932in}}%
\pgfpathlineto{\pgfqpoint{3.257388in}{0.915278in}}%
\pgfpathlineto{\pgfqpoint{3.283289in}{0.920625in}}%
\pgfpathlineto{\pgfqpoint{3.296240in}{0.922503in}}%
\pgfpathlineto{\pgfqpoint{3.309190in}{0.926340in}}%
\pgfpathlineto{\pgfqpoint{3.322141in}{0.932551in}}%
\pgfpathlineto{\pgfqpoint{3.335091in}{0.941026in}}%
\pgfpathlineto{\pgfqpoint{3.348042in}{0.947534in}}%
\pgfpathlineto{\pgfqpoint{3.399843in}{0.959711in}}%
\pgfpathlineto{\pgfqpoint{3.438695in}{0.972242in}}%
\pgfpathlineto{\pgfqpoint{3.464596in}{0.982498in}}%
\pgfpathlineto{\pgfqpoint{3.477546in}{0.991676in}}%
\pgfpathlineto{\pgfqpoint{3.529348in}{1.019748in}}%
\pgfpathlineto{\pgfqpoint{3.542299in}{1.024188in}}%
\pgfpathlineto{\pgfqpoint{3.555249in}{1.035560in}}%
\pgfpathlineto{\pgfqpoint{3.568200in}{1.042492in}}%
\pgfpathlineto{\pgfqpoint{3.581150in}{1.054465in}}%
\pgfpathlineto{\pgfqpoint{3.632952in}{1.087766in}}%
\pgfpathlineto{\pgfqpoint{3.658853in}{1.102159in}}%
\pgfpathlineto{\pgfqpoint{3.671803in}{1.112971in}}%
\pgfpathlineto{\pgfqpoint{3.684754in}{1.120690in}}%
\pgfpathlineto{\pgfqpoint{3.697704in}{1.129840in}}%
\pgfpathlineto{\pgfqpoint{3.723605in}{1.154406in}}%
\pgfpathlineto{\pgfqpoint{3.749506in}{1.184057in}}%
\pgfpathlineto{\pgfqpoint{3.762457in}{1.192734in}}%
\pgfpathlineto{\pgfqpoint{3.775407in}{1.199450in}}%
\pgfpathlineto{\pgfqpoint{3.801308in}{1.220873in}}%
\pgfpathlineto{\pgfqpoint{3.814259in}{1.229454in}}%
\pgfpathlineto{\pgfqpoint{3.827209in}{1.245456in}}%
\pgfpathlineto{\pgfqpoint{3.840160in}{1.255485in}}%
\pgfpathlineto{\pgfqpoint{3.866060in}{1.273254in}}%
\pgfpathlineto{\pgfqpoint{3.879011in}{1.284423in}}%
\pgfpathlineto{\pgfqpoint{3.891961in}{1.292769in}}%
\pgfpathlineto{\pgfqpoint{3.904912in}{1.304929in}}%
\pgfpathlineto{\pgfqpoint{3.930813in}{1.321810in}}%
\pgfpathlineto{\pgfqpoint{3.943763in}{1.337512in}}%
\pgfpathlineto{\pgfqpoint{3.956714in}{1.349176in}}%
\pgfpathlineto{\pgfqpoint{3.969664in}{1.364440in}}%
\pgfpathlineto{\pgfqpoint{4.008516in}{1.401353in}}%
\pgfpathlineto{\pgfqpoint{4.086219in}{1.457848in}}%
\pgfpathlineto{\pgfqpoint{4.099169in}{1.469127in}}%
\pgfpathlineto{\pgfqpoint{4.112119in}{1.477693in}}%
\pgfpathlineto{\pgfqpoint{4.125070in}{1.487733in}}%
\pgfpathlineto{\pgfqpoint{4.163921in}{1.509046in}}%
\pgfpathlineto{\pgfqpoint{4.176872in}{1.524900in}}%
\pgfpathlineto{\pgfqpoint{4.189822in}{1.533802in}}%
\pgfpathlineto{\pgfqpoint{4.202773in}{1.541255in}}%
\pgfpathlineto{\pgfqpoint{4.241624in}{1.572368in}}%
\pgfpathlineto{\pgfqpoint{4.254575in}{1.586445in}}%
\pgfpathlineto{\pgfqpoint{4.267525in}{1.595374in}}%
\pgfpathlineto{\pgfqpoint{4.280476in}{1.607877in}}%
\pgfpathlineto{\pgfqpoint{4.293426in}{1.611901in}}%
\pgfpathlineto{\pgfqpoint{4.319327in}{1.630656in}}%
\pgfpathlineto{\pgfqpoint{4.332277in}{1.635457in}}%
\pgfpathlineto{\pgfqpoint{4.345228in}{1.643201in}}%
\pgfpathlineto{\pgfqpoint{4.358178in}{1.655988in}}%
\pgfpathlineto{\pgfqpoint{4.371129in}{1.661941in}}%
\pgfpathlineto{\pgfqpoint{4.397030in}{1.678987in}}%
\pgfpathlineto{\pgfqpoint{4.422931in}{1.695267in}}%
\pgfpathlineto{\pgfqpoint{4.435881in}{1.705715in}}%
\pgfpathlineto{\pgfqpoint{4.448832in}{1.713981in}}%
\pgfpathlineto{\pgfqpoint{4.461782in}{1.720798in}}%
\pgfpathlineto{\pgfqpoint{4.474733in}{1.724198in}}%
\pgfpathlineto{\pgfqpoint{4.487683in}{1.730033in}}%
\pgfpathlineto{\pgfqpoint{4.513584in}{1.748603in}}%
\pgfpathlineto{\pgfqpoint{4.526535in}{1.751192in}}%
\pgfpathlineto{\pgfqpoint{4.539485in}{1.755275in}}%
\pgfpathlineto{\pgfqpoint{4.552436in}{1.761050in}}%
\pgfpathlineto{\pgfqpoint{4.565386in}{1.768470in}}%
\pgfpathlineto{\pgfqpoint{4.578336in}{1.772277in}}%
\pgfpathlineto{\pgfqpoint{4.591287in}{1.779979in}}%
\pgfpathlineto{\pgfqpoint{4.604237in}{1.786004in}}%
\pgfpathlineto{\pgfqpoint{4.617188in}{1.794701in}}%
\pgfpathlineto{\pgfqpoint{4.630138in}{1.801451in}}%
\pgfpathlineto{\pgfqpoint{4.668990in}{1.816012in}}%
\pgfpathlineto{\pgfqpoint{4.681940in}{1.825544in}}%
\pgfpathlineto{\pgfqpoint{4.694891in}{1.832404in}}%
\pgfpathlineto{\pgfqpoint{4.720792in}{1.839876in}}%
\pgfpathlineto{\pgfqpoint{4.733742in}{1.843014in}}%
\pgfpathlineto{\pgfqpoint{4.746693in}{1.853995in}}%
\pgfpathlineto{\pgfqpoint{4.759643in}{1.861663in}}%
\pgfpathlineto{\pgfqpoint{4.772594in}{1.864908in}}%
\pgfpathlineto{\pgfqpoint{4.785544in}{1.869442in}}%
\pgfpathlineto{\pgfqpoint{4.811445in}{1.881714in}}%
\pgfpathlineto{\pgfqpoint{4.824395in}{1.891520in}}%
\pgfpathlineto{\pgfqpoint{4.837346in}{1.895146in}}%
\pgfpathlineto{\pgfqpoint{4.850296in}{1.903538in}}%
\pgfpathlineto{\pgfqpoint{4.876197in}{1.913769in}}%
\pgfpathlineto{\pgfqpoint{4.915049in}{1.922200in}}%
\pgfpathlineto{\pgfqpoint{4.927999in}{1.924724in}}%
\pgfpathlineto{\pgfqpoint{4.953900in}{1.936560in}}%
\pgfpathlineto{\pgfqpoint{4.966851in}{1.941313in}}%
\pgfpathlineto{\pgfqpoint{4.979801in}{1.944552in}}%
\pgfpathlineto{\pgfqpoint{4.992752in}{1.949605in}}%
\pgfpathlineto{\pgfqpoint{5.005702in}{1.958893in}}%
\pgfpathlineto{\pgfqpoint{5.031603in}{1.970028in}}%
\pgfpathlineto{\pgfqpoint{5.070454in}{1.978878in}}%
\pgfpathlineto{\pgfqpoint{5.096355in}{1.986132in}}%
\pgfpathlineto{\pgfqpoint{5.109306in}{1.991246in}}%
\pgfpathlineto{\pgfqpoint{5.135207in}{2.009087in}}%
\pgfpathlineto{\pgfqpoint{5.148157in}{2.014953in}}%
\pgfpathlineto{\pgfqpoint{5.161108in}{2.023980in}}%
\pgfpathlineto{\pgfqpoint{5.174058in}{2.034666in}}%
\pgfpathlineto{\pgfqpoint{5.187009in}{2.040902in}}%
\pgfpathlineto{\pgfqpoint{5.225860in}{2.051656in}}%
\pgfpathlineto{\pgfqpoint{5.342414in}{2.079073in}}%
\pgfpathlineto{\pgfqpoint{5.355365in}{2.081466in}}%
\pgfpathlineto{\pgfqpoint{5.381266in}{2.083481in}}%
\pgfpathlineto{\pgfqpoint{5.394216in}{2.086440in}}%
\pgfpathlineto{\pgfqpoint{5.420117in}{2.087756in}}%
\pgfpathlineto{\pgfqpoint{5.471919in}{2.094791in}}%
\pgfpathlineto{\pgfqpoint{5.601424in}{2.106250in}}%
\pgfpathlineto{\pgfqpoint{5.614374in}{2.107600in}}%
\pgfpathlineto{\pgfqpoint{5.679127in}{2.109504in}}%
\pgfpathlineto{\pgfqpoint{5.717978in}{2.110859in}}%
\pgfpathlineto{\pgfqpoint{5.756830in}{2.110990in}}%
\pgfpathlineto{\pgfqpoint{5.756830in}{2.110990in}}%
\pgfusepath{stroke}%
\end{pgfscope}%
\begin{pgfscope}%
\pgfpathrectangle{\pgfqpoint{0.589591in}{0.539182in}}{\pgfqpoint{5.167239in}{1.668128in}}%
\pgfusepath{clip}%
\pgfsetbuttcap%
\pgfsetroundjoin%
\pgfsetlinewidth{2.007500pt}%
\definecolor{currentstroke}{rgb}{0.564706,0.564706,1.000000}%
\pgfsetstrokecolor{currentstroke}%
\pgfsetdash{{2.000000pt}{3.300000pt}}{0.000000pt}%
\pgfpathmoveto{\pgfqpoint{0.589591in}{1.050327in}}%
\pgfpathlineto{\pgfqpoint{2.091846in}{1.049330in}}%
\pgfpathlineto{\pgfqpoint{2.104796in}{1.047552in}}%
\pgfpathlineto{\pgfqpoint{2.143648in}{1.046458in}}%
\pgfpathlineto{\pgfqpoint{2.169549in}{1.042364in}}%
\pgfpathlineto{\pgfqpoint{2.182499in}{1.041276in}}%
\pgfpathlineto{\pgfqpoint{2.208400in}{1.035823in}}%
\pgfpathlineto{\pgfqpoint{2.221350in}{1.034719in}}%
\pgfpathlineto{\pgfqpoint{2.234301in}{1.027287in}}%
\pgfpathlineto{\pgfqpoint{2.247251in}{1.025980in}}%
\pgfpathlineto{\pgfqpoint{2.260202in}{1.022789in}}%
\pgfpathlineto{\pgfqpoint{2.299053in}{1.006474in}}%
\pgfpathlineto{\pgfqpoint{2.312004in}{1.003919in}}%
\pgfpathlineto{\pgfqpoint{2.324954in}{1.000128in}}%
\pgfpathlineto{\pgfqpoint{2.337905in}{0.994993in}}%
\pgfpathlineto{\pgfqpoint{2.350855in}{0.992490in}}%
\pgfpathlineto{\pgfqpoint{2.363806in}{0.985832in}}%
\pgfpathlineto{\pgfqpoint{2.376756in}{0.983975in}}%
\pgfpathlineto{\pgfqpoint{2.389707in}{0.979788in}}%
\pgfpathlineto{\pgfqpoint{2.402657in}{0.978464in}}%
\pgfpathlineto{\pgfqpoint{2.415608in}{0.971027in}}%
\pgfpathlineto{\pgfqpoint{2.428558in}{0.968169in}}%
\pgfpathlineto{\pgfqpoint{2.441508in}{0.963045in}}%
\pgfpathlineto{\pgfqpoint{2.467409in}{0.962311in}}%
\pgfpathlineto{\pgfqpoint{2.519211in}{0.952475in}}%
\pgfpathlineto{\pgfqpoint{2.532162in}{0.946014in}}%
\pgfpathlineto{\pgfqpoint{2.545112in}{0.943694in}}%
\pgfpathlineto{\pgfqpoint{2.558063in}{0.936770in}}%
\pgfpathlineto{\pgfqpoint{2.571013in}{0.932297in}}%
\pgfpathlineto{\pgfqpoint{2.596914in}{0.927354in}}%
\pgfpathlineto{\pgfqpoint{2.609865in}{0.922762in}}%
\pgfpathlineto{\pgfqpoint{2.622815in}{0.921439in}}%
\pgfpathlineto{\pgfqpoint{2.674617in}{0.903901in}}%
\pgfpathlineto{\pgfqpoint{2.726419in}{0.893119in}}%
\pgfpathlineto{\pgfqpoint{2.739369in}{0.893079in}}%
\pgfpathlineto{\pgfqpoint{2.752320in}{0.890910in}}%
\pgfpathlineto{\pgfqpoint{2.791171in}{0.872065in}}%
\pgfpathlineto{\pgfqpoint{2.817072in}{0.869109in}}%
\pgfpathlineto{\pgfqpoint{2.868874in}{0.867715in}}%
\pgfpathlineto{\pgfqpoint{2.933626in}{0.862224in}}%
\pgfpathlineto{\pgfqpoint{2.946577in}{0.860065in}}%
\pgfpathlineto{\pgfqpoint{3.024280in}{0.861771in}}%
\pgfpathlineto{\pgfqpoint{3.114933in}{0.865673in}}%
\pgfpathlineto{\pgfqpoint{3.166735in}{0.870800in}}%
\pgfpathlineto{\pgfqpoint{3.192636in}{0.879886in}}%
\pgfpathlineto{\pgfqpoint{3.205586in}{0.881368in}}%
\pgfpathlineto{\pgfqpoint{3.231487in}{0.886216in}}%
\pgfpathlineto{\pgfqpoint{3.244438in}{0.888378in}}%
\pgfpathlineto{\pgfqpoint{3.257388in}{0.893338in}}%
\pgfpathlineto{\pgfqpoint{3.270339in}{0.896721in}}%
\pgfpathlineto{\pgfqpoint{3.283289in}{0.902952in}}%
\pgfpathlineto{\pgfqpoint{3.348042in}{0.923971in}}%
\pgfpathlineto{\pgfqpoint{3.360992in}{0.932688in}}%
\pgfpathlineto{\pgfqpoint{3.373943in}{0.943268in}}%
\pgfpathlineto{\pgfqpoint{3.399843in}{0.951139in}}%
\pgfpathlineto{\pgfqpoint{3.412794in}{0.956240in}}%
\pgfpathlineto{\pgfqpoint{3.425744in}{0.959907in}}%
\pgfpathlineto{\pgfqpoint{3.477546in}{0.982475in}}%
\pgfpathlineto{\pgfqpoint{3.490497in}{0.992242in}}%
\pgfpathlineto{\pgfqpoint{3.503447in}{0.999084in}}%
\pgfpathlineto{\pgfqpoint{3.516398in}{1.008236in}}%
\pgfpathlineto{\pgfqpoint{3.529348in}{1.015406in}}%
\pgfpathlineto{\pgfqpoint{3.542299in}{1.021046in}}%
\pgfpathlineto{\pgfqpoint{3.555249in}{1.028713in}}%
\pgfpathlineto{\pgfqpoint{3.568200in}{1.039160in}}%
\pgfpathlineto{\pgfqpoint{3.581150in}{1.047976in}}%
\pgfpathlineto{\pgfqpoint{3.607051in}{1.068521in}}%
\pgfpathlineto{\pgfqpoint{3.645902in}{1.092856in}}%
\pgfpathlineto{\pgfqpoint{3.658853in}{1.100234in}}%
\pgfpathlineto{\pgfqpoint{3.671803in}{1.110572in}}%
\pgfpathlineto{\pgfqpoint{3.697704in}{1.128077in}}%
\pgfpathlineto{\pgfqpoint{3.723605in}{1.152153in}}%
\pgfpathlineto{\pgfqpoint{3.736556in}{1.167844in}}%
\pgfpathlineto{\pgfqpoint{3.749506in}{1.180878in}}%
\pgfpathlineto{\pgfqpoint{3.762457in}{1.191834in}}%
\pgfpathlineto{\pgfqpoint{3.775407in}{1.199117in}}%
\pgfpathlineto{\pgfqpoint{3.801308in}{1.220632in}}%
\pgfpathlineto{\pgfqpoint{3.814259in}{1.229222in}}%
\pgfpathlineto{\pgfqpoint{3.827209in}{1.244730in}}%
\pgfpathlineto{\pgfqpoint{3.840160in}{1.254799in}}%
\pgfpathlineto{\pgfqpoint{3.866060in}{1.273138in}}%
\pgfpathlineto{\pgfqpoint{3.879011in}{1.284320in}}%
\pgfpathlineto{\pgfqpoint{3.891961in}{1.292669in}}%
\pgfpathlineto{\pgfqpoint{3.904912in}{1.304848in}}%
\pgfpathlineto{\pgfqpoint{3.930813in}{1.321735in}}%
\pgfpathlineto{\pgfqpoint{3.943763in}{1.337446in}}%
\pgfpathlineto{\pgfqpoint{3.956714in}{1.348562in}}%
\pgfpathlineto{\pgfqpoint{3.969664in}{1.364397in}}%
\pgfpathlineto{\pgfqpoint{4.008516in}{1.401330in}}%
\pgfpathlineto{\pgfqpoint{4.086219in}{1.457830in}}%
\pgfpathlineto{\pgfqpoint{4.099169in}{1.469117in}}%
\pgfpathlineto{\pgfqpoint{4.112119in}{1.477680in}}%
\pgfpathlineto{\pgfqpoint{4.125070in}{1.487722in}}%
\pgfpathlineto{\pgfqpoint{4.163921in}{1.509033in}}%
\pgfpathlineto{\pgfqpoint{4.176872in}{1.524894in}}%
\pgfpathlineto{\pgfqpoint{4.189822in}{1.533794in}}%
\pgfpathlineto{\pgfqpoint{4.202773in}{1.541247in}}%
\pgfpathlineto{\pgfqpoint{4.241624in}{1.572359in}}%
\pgfpathlineto{\pgfqpoint{4.254575in}{1.586438in}}%
\pgfpathlineto{\pgfqpoint{4.267525in}{1.595368in}}%
\pgfpathlineto{\pgfqpoint{4.280476in}{1.607871in}}%
\pgfpathlineto{\pgfqpoint{4.293426in}{1.611895in}}%
\pgfpathlineto{\pgfqpoint{4.319327in}{1.630649in}}%
\pgfpathlineto{\pgfqpoint{4.332277in}{1.635450in}}%
\pgfpathlineto{\pgfqpoint{4.345228in}{1.643193in}}%
\pgfpathlineto{\pgfqpoint{4.358178in}{1.655981in}}%
\pgfpathlineto{\pgfqpoint{4.371129in}{1.661934in}}%
\pgfpathlineto{\pgfqpoint{4.397030in}{1.678980in}}%
\pgfpathlineto{\pgfqpoint{4.422931in}{1.695259in}}%
\pgfpathlineto{\pgfqpoint{4.435881in}{1.705707in}}%
\pgfpathlineto{\pgfqpoint{4.448832in}{1.713972in}}%
\pgfpathlineto{\pgfqpoint{4.461782in}{1.720789in}}%
\pgfpathlineto{\pgfqpoint{4.474733in}{1.724190in}}%
\pgfpathlineto{\pgfqpoint{4.487683in}{1.730024in}}%
\pgfpathlineto{\pgfqpoint{4.513584in}{1.748594in}}%
\pgfpathlineto{\pgfqpoint{4.526535in}{1.751184in}}%
\pgfpathlineto{\pgfqpoint{4.539485in}{1.755267in}}%
\pgfpathlineto{\pgfqpoint{4.552436in}{1.761042in}}%
\pgfpathlineto{\pgfqpoint{4.565386in}{1.768463in}}%
\pgfpathlineto{\pgfqpoint{4.578336in}{1.772269in}}%
\pgfpathlineto{\pgfqpoint{4.591287in}{1.779972in}}%
\pgfpathlineto{\pgfqpoint{4.604237in}{1.785996in}}%
\pgfpathlineto{\pgfqpoint{4.617188in}{1.794694in}}%
\pgfpathlineto{\pgfqpoint{4.630138in}{1.801443in}}%
\pgfpathlineto{\pgfqpoint{4.668990in}{1.816004in}}%
\pgfpathlineto{\pgfqpoint{4.681940in}{1.825537in}}%
\pgfpathlineto{\pgfqpoint{4.694891in}{1.832397in}}%
\pgfpathlineto{\pgfqpoint{4.720792in}{1.839869in}}%
\pgfpathlineto{\pgfqpoint{4.733742in}{1.843007in}}%
\pgfpathlineto{\pgfqpoint{4.746693in}{1.853988in}}%
\pgfpathlineto{\pgfqpoint{4.759643in}{1.861656in}}%
\pgfpathlineto{\pgfqpoint{4.772594in}{1.864901in}}%
\pgfpathlineto{\pgfqpoint{4.785544in}{1.869435in}}%
\pgfpathlineto{\pgfqpoint{4.811445in}{1.881707in}}%
\pgfpathlineto{\pgfqpoint{4.824395in}{1.891514in}}%
\pgfpathlineto{\pgfqpoint{4.837346in}{1.895140in}}%
\pgfpathlineto{\pgfqpoint{4.850296in}{1.903532in}}%
\pgfpathlineto{\pgfqpoint{4.876197in}{1.913764in}}%
\pgfpathlineto{\pgfqpoint{4.915049in}{1.922195in}}%
\pgfpathlineto{\pgfqpoint{4.927999in}{1.924718in}}%
\pgfpathlineto{\pgfqpoint{4.953900in}{1.936555in}}%
\pgfpathlineto{\pgfqpoint{4.966851in}{1.941308in}}%
\pgfpathlineto{\pgfqpoint{4.979801in}{1.944547in}}%
\pgfpathlineto{\pgfqpoint{4.992752in}{1.949601in}}%
\pgfpathlineto{\pgfqpoint{5.005702in}{1.958888in}}%
\pgfpathlineto{\pgfqpoint{5.031603in}{1.970024in}}%
\pgfpathlineto{\pgfqpoint{5.070454in}{1.978874in}}%
\pgfpathlineto{\pgfqpoint{5.096355in}{1.986129in}}%
\pgfpathlineto{\pgfqpoint{5.109306in}{1.991242in}}%
\pgfpathlineto{\pgfqpoint{5.135207in}{2.009084in}}%
\pgfpathlineto{\pgfqpoint{5.148157in}{2.014951in}}%
\pgfpathlineto{\pgfqpoint{5.161108in}{2.023977in}}%
\pgfpathlineto{\pgfqpoint{5.174058in}{2.034665in}}%
\pgfpathlineto{\pgfqpoint{5.187009in}{2.040901in}}%
\pgfpathlineto{\pgfqpoint{5.225860in}{2.051655in}}%
\pgfpathlineto{\pgfqpoint{5.342414in}{2.079072in}}%
\pgfpathlineto{\pgfqpoint{5.355365in}{2.081466in}}%
\pgfpathlineto{\pgfqpoint{5.381266in}{2.083481in}}%
\pgfpathlineto{\pgfqpoint{5.394216in}{2.086440in}}%
\pgfpathlineto{\pgfqpoint{5.420117in}{2.087756in}}%
\pgfpathlineto{\pgfqpoint{5.471919in}{2.094791in}}%
\pgfpathlineto{\pgfqpoint{5.601424in}{2.106250in}}%
\pgfpathlineto{\pgfqpoint{5.614374in}{2.107600in}}%
\pgfpathlineto{\pgfqpoint{5.679127in}{2.109504in}}%
\pgfpathlineto{\pgfqpoint{5.717978in}{2.110859in}}%
\pgfpathlineto{\pgfqpoint{5.756830in}{2.110990in}}%
\pgfpathlineto{\pgfqpoint{5.756830in}{2.110990in}}%
\pgfusepath{stroke}%
\end{pgfscope}%
\begin{pgfscope}%
\pgfpathrectangle{\pgfqpoint{0.589591in}{0.539182in}}{\pgfqpoint{5.167239in}{1.668128in}}%
\pgfusepath{clip}%
\pgfsetbuttcap%
\pgfsetroundjoin%
\pgfsetlinewidth{2.007500pt}%
\definecolor{currentstroke}{rgb}{0.564706,0.564706,1.000000}%
\pgfsetstrokecolor{currentstroke}%
\pgfsetdash{{7.400000pt}{3.200000pt}}{0.000000pt}%
\pgfpathmoveto{\pgfqpoint{0.589591in}{0.924834in}}%
\pgfpathlineto{\pgfqpoint{2.091846in}{0.923725in}}%
\pgfpathlineto{\pgfqpoint{2.104796in}{0.921507in}}%
\pgfpathlineto{\pgfqpoint{2.143648in}{0.920397in}}%
\pgfpathlineto{\pgfqpoint{2.182499in}{0.917066in}}%
\pgfpathlineto{\pgfqpoint{2.208400in}{0.911511in}}%
\pgfpathlineto{\pgfqpoint{2.221350in}{0.911504in}}%
\pgfpathlineto{\pgfqpoint{2.234301in}{0.908161in}}%
\pgfpathlineto{\pgfqpoint{2.260202in}{0.904827in}}%
\pgfpathlineto{\pgfqpoint{2.286103in}{0.892621in}}%
\pgfpathlineto{\pgfqpoint{2.312004in}{0.885957in}}%
\pgfpathlineto{\pgfqpoint{2.324954in}{0.885947in}}%
\pgfpathlineto{\pgfqpoint{2.350855in}{0.882612in}}%
\pgfpathlineto{\pgfqpoint{2.402657in}{0.880808in}}%
\pgfpathlineto{\pgfqpoint{2.428558in}{0.877461in}}%
\pgfpathlineto{\pgfqpoint{2.454459in}{0.875236in}}%
\pgfpathlineto{\pgfqpoint{2.480360in}{0.874134in}}%
\pgfpathlineto{\pgfqpoint{2.506261in}{0.872989in}}%
\pgfpathlineto{\pgfqpoint{2.519211in}{0.870776in}}%
\pgfpathlineto{\pgfqpoint{2.532162in}{0.865270in}}%
\pgfpathlineto{\pgfqpoint{2.545112in}{0.864173in}}%
\pgfpathlineto{\pgfqpoint{2.558063in}{0.858647in}}%
\pgfpathlineto{\pgfqpoint{2.571013in}{0.857548in}}%
\pgfpathlineto{\pgfqpoint{2.583964in}{0.855324in}}%
\pgfpathlineto{\pgfqpoint{2.622815in}{0.852019in}}%
\pgfpathlineto{\pgfqpoint{2.635766in}{0.847625in}}%
\pgfpathlineto{\pgfqpoint{2.674617in}{0.839971in}}%
\pgfpathlineto{\pgfqpoint{2.687567in}{0.836836in}}%
\pgfpathlineto{\pgfqpoint{2.726419in}{0.834809in}}%
\pgfpathlineto{\pgfqpoint{2.739369in}{0.834888in}}%
\pgfpathlineto{\pgfqpoint{2.765270in}{0.829634in}}%
\pgfpathlineto{\pgfqpoint{2.778221in}{0.825354in}}%
\pgfpathlineto{\pgfqpoint{2.817072in}{0.822579in}}%
\pgfpathlineto{\pgfqpoint{2.855924in}{0.822157in}}%
\pgfpathlineto{\pgfqpoint{2.894775in}{0.821143in}}%
\pgfpathlineto{\pgfqpoint{2.907725in}{0.822091in}}%
\pgfpathlineto{\pgfqpoint{2.933626in}{0.820092in}}%
\pgfpathlineto{\pgfqpoint{2.946577in}{0.817028in}}%
\pgfpathlineto{\pgfqpoint{2.972478in}{0.819496in}}%
\pgfpathlineto{\pgfqpoint{3.011329in}{0.822691in}}%
\pgfpathlineto{\pgfqpoint{3.101983in}{0.833567in}}%
\pgfpathlineto{\pgfqpoint{3.205586in}{0.856848in}}%
\pgfpathlineto{\pgfqpoint{3.218537in}{0.861327in}}%
\pgfpathlineto{\pgfqpoint{3.231487in}{0.867991in}}%
\pgfpathlineto{\pgfqpoint{3.283289in}{0.884557in}}%
\pgfpathlineto{\pgfqpoint{3.296240in}{0.890556in}}%
\pgfpathlineto{\pgfqpoint{3.309190in}{0.898101in}}%
\pgfpathlineto{\pgfqpoint{3.322141in}{0.900643in}}%
\pgfpathlineto{\pgfqpoint{3.360992in}{0.918624in}}%
\pgfpathlineto{\pgfqpoint{3.386893in}{0.937790in}}%
\pgfpathlineto{\pgfqpoint{3.425744in}{0.952923in}}%
\pgfpathlineto{\pgfqpoint{3.438695in}{0.957571in}}%
\pgfpathlineto{\pgfqpoint{3.464596in}{0.971396in}}%
\pgfpathlineto{\pgfqpoint{3.477546in}{0.976135in}}%
\pgfpathlineto{\pgfqpoint{3.490497in}{0.985555in}}%
\pgfpathlineto{\pgfqpoint{3.503447in}{0.993108in}}%
\pgfpathlineto{\pgfqpoint{3.516398in}{1.002968in}}%
\pgfpathlineto{\pgfqpoint{3.529348in}{1.010250in}}%
\pgfpathlineto{\pgfqpoint{3.542299in}{1.015947in}}%
\pgfpathlineto{\pgfqpoint{3.555249in}{1.023797in}}%
\pgfpathlineto{\pgfqpoint{3.568200in}{1.034490in}}%
\pgfpathlineto{\pgfqpoint{3.581150in}{1.043461in}}%
\pgfpathlineto{\pgfqpoint{3.620001in}{1.073963in}}%
\pgfpathlineto{\pgfqpoint{3.658853in}{1.095581in}}%
\pgfpathlineto{\pgfqpoint{3.671803in}{1.106495in}}%
\pgfpathlineto{\pgfqpoint{3.697704in}{1.123498in}}%
\pgfpathlineto{\pgfqpoint{3.723605in}{1.147223in}}%
\pgfpathlineto{\pgfqpoint{3.736556in}{1.163633in}}%
\pgfpathlineto{\pgfqpoint{3.762457in}{1.188424in}}%
\pgfpathlineto{\pgfqpoint{3.775407in}{1.195724in}}%
\pgfpathlineto{\pgfqpoint{3.788358in}{1.206521in}}%
\pgfpathlineto{\pgfqpoint{3.801308in}{1.219706in}}%
\pgfpathlineto{\pgfqpoint{3.814259in}{1.228304in}}%
\pgfpathlineto{\pgfqpoint{3.840160in}{1.253610in}}%
\pgfpathlineto{\pgfqpoint{3.866060in}{1.272002in}}%
\pgfpathlineto{\pgfqpoint{3.879011in}{1.283228in}}%
\pgfpathlineto{\pgfqpoint{3.891961in}{1.292129in}}%
\pgfpathlineto{\pgfqpoint{3.904912in}{1.304936in}}%
\pgfpathlineto{\pgfqpoint{3.930813in}{1.321815in}}%
\pgfpathlineto{\pgfqpoint{3.943763in}{1.337530in}}%
\pgfpathlineto{\pgfqpoint{3.956714in}{1.348683in}}%
\pgfpathlineto{\pgfqpoint{3.969664in}{1.363433in}}%
\pgfpathlineto{\pgfqpoint{4.008516in}{1.401529in}}%
\pgfpathlineto{\pgfqpoint{4.086219in}{1.457430in}}%
\pgfpathlineto{\pgfqpoint{4.099169in}{1.469275in}}%
\pgfpathlineto{\pgfqpoint{4.112119in}{1.477849in}}%
\pgfpathlineto{\pgfqpoint{4.125070in}{1.487881in}}%
\pgfpathlineto{\pgfqpoint{4.163921in}{1.509170in}}%
\pgfpathlineto{\pgfqpoint{4.176872in}{1.525042in}}%
\pgfpathlineto{\pgfqpoint{4.189822in}{1.533937in}}%
\pgfpathlineto{\pgfqpoint{4.202773in}{1.541386in}}%
\pgfpathlineto{\pgfqpoint{4.241624in}{1.572451in}}%
\pgfpathlineto{\pgfqpoint{4.254575in}{1.586520in}}%
\pgfpathlineto{\pgfqpoint{4.267525in}{1.595438in}}%
\pgfpathlineto{\pgfqpoint{4.280476in}{1.607929in}}%
\pgfpathlineto{\pgfqpoint{4.293426in}{1.611949in}}%
\pgfpathlineto{\pgfqpoint{4.319327in}{1.630682in}}%
\pgfpathlineto{\pgfqpoint{4.332277in}{1.635482in}}%
\pgfpathlineto{\pgfqpoint{4.345228in}{1.643219in}}%
\pgfpathlineto{\pgfqpoint{4.358178in}{1.655990in}}%
\pgfpathlineto{\pgfqpoint{4.371129in}{1.661941in}}%
\pgfpathlineto{\pgfqpoint{4.397030in}{1.678986in}}%
\pgfpathlineto{\pgfqpoint{4.422931in}{1.695261in}}%
\pgfpathlineto{\pgfqpoint{4.435881in}{1.705708in}}%
\pgfpathlineto{\pgfqpoint{4.448832in}{1.713974in}}%
\pgfpathlineto{\pgfqpoint{4.461782in}{1.720791in}}%
\pgfpathlineto{\pgfqpoint{4.474733in}{1.724191in}}%
\pgfpathlineto{\pgfqpoint{4.487683in}{1.730026in}}%
\pgfpathlineto{\pgfqpoint{4.513584in}{1.748596in}}%
\pgfpathlineto{\pgfqpoint{4.526535in}{1.751185in}}%
\pgfpathlineto{\pgfqpoint{4.539485in}{1.755268in}}%
\pgfpathlineto{\pgfqpoint{4.552436in}{1.761044in}}%
\pgfpathlineto{\pgfqpoint{4.565386in}{1.768464in}}%
\pgfpathlineto{\pgfqpoint{4.578336in}{1.772270in}}%
\pgfpathlineto{\pgfqpoint{4.591287in}{1.779973in}}%
\pgfpathlineto{\pgfqpoint{4.604237in}{1.785998in}}%
\pgfpathlineto{\pgfqpoint{4.617188in}{1.794696in}}%
\pgfpathlineto{\pgfqpoint{4.630138in}{1.801445in}}%
\pgfpathlineto{\pgfqpoint{4.668990in}{1.816007in}}%
\pgfpathlineto{\pgfqpoint{4.681940in}{1.825539in}}%
\pgfpathlineto{\pgfqpoint{4.694891in}{1.832399in}}%
\pgfpathlineto{\pgfqpoint{4.720792in}{1.839871in}}%
\pgfpathlineto{\pgfqpoint{4.733742in}{1.843009in}}%
\pgfpathlineto{\pgfqpoint{4.746693in}{1.853990in}}%
\pgfpathlineto{\pgfqpoint{4.759643in}{1.861658in}}%
\pgfpathlineto{\pgfqpoint{4.772594in}{1.864903in}}%
\pgfpathlineto{\pgfqpoint{4.785544in}{1.869437in}}%
\pgfpathlineto{\pgfqpoint{4.811445in}{1.881709in}}%
\pgfpathlineto{\pgfqpoint{4.824395in}{1.891515in}}%
\pgfpathlineto{\pgfqpoint{4.837346in}{1.895142in}}%
\pgfpathlineto{\pgfqpoint{4.850296in}{1.903534in}}%
\pgfpathlineto{\pgfqpoint{4.876197in}{1.913765in}}%
\pgfpathlineto{\pgfqpoint{4.915049in}{1.922197in}}%
\pgfpathlineto{\pgfqpoint{4.927999in}{1.924720in}}%
\pgfpathlineto{\pgfqpoint{4.953900in}{1.936556in}}%
\pgfpathlineto{\pgfqpoint{4.966851in}{1.941310in}}%
\pgfpathlineto{\pgfqpoint{4.979801in}{1.944548in}}%
\pgfpathlineto{\pgfqpoint{4.992752in}{1.949602in}}%
\pgfpathlineto{\pgfqpoint{5.005702in}{1.958890in}}%
\pgfpathlineto{\pgfqpoint{5.031603in}{1.970025in}}%
\pgfpathlineto{\pgfqpoint{5.070454in}{1.978875in}}%
\pgfpathlineto{\pgfqpoint{5.096355in}{1.986130in}}%
\pgfpathlineto{\pgfqpoint{5.109306in}{1.991243in}}%
\pgfpathlineto{\pgfqpoint{5.135207in}{2.009085in}}%
\pgfpathlineto{\pgfqpoint{5.148157in}{2.014951in}}%
\pgfpathlineto{\pgfqpoint{5.161108in}{2.023978in}}%
\pgfpathlineto{\pgfqpoint{5.174058in}{2.034665in}}%
\pgfpathlineto{\pgfqpoint{5.187009in}{2.040901in}}%
\pgfpathlineto{\pgfqpoint{5.225860in}{2.051655in}}%
\pgfpathlineto{\pgfqpoint{5.342414in}{2.079072in}}%
\pgfpathlineto{\pgfqpoint{5.355365in}{2.081466in}}%
\pgfpathlineto{\pgfqpoint{5.381266in}{2.083481in}}%
\pgfpathlineto{\pgfqpoint{5.394216in}{2.086440in}}%
\pgfpathlineto{\pgfqpoint{5.420117in}{2.087756in}}%
\pgfpathlineto{\pgfqpoint{5.471919in}{2.094791in}}%
\pgfpathlineto{\pgfqpoint{5.601424in}{2.106250in}}%
\pgfpathlineto{\pgfqpoint{5.614374in}{2.107600in}}%
\pgfpathlineto{\pgfqpoint{5.679127in}{2.109504in}}%
\pgfpathlineto{\pgfqpoint{5.717978in}{2.110859in}}%
\pgfpathlineto{\pgfqpoint{5.756830in}{2.110990in}}%
\pgfpathlineto{\pgfqpoint{5.756830in}{2.110990in}}%
\pgfusepath{stroke}%
\end{pgfscope}%
\begin{pgfscope}%
\pgfsetrectcap%
\pgfsetmiterjoin%
\pgfsetlinewidth{0.803000pt}%
\definecolor{currentstroke}{rgb}{0.000000,0.000000,0.000000}%
\pgfsetstrokecolor{currentstroke}%
\pgfsetdash{}{0pt}%
\pgfpathmoveto{\pgfqpoint{0.589591in}{0.539182in}}%
\pgfpathlineto{\pgfqpoint{0.589591in}{2.207310in}}%
\pgfusepath{stroke}%
\end{pgfscope}%
\begin{pgfscope}%
\pgfsetrectcap%
\pgfsetmiterjoin%
\pgfsetlinewidth{0.803000pt}%
\definecolor{currentstroke}{rgb}{0.000000,0.000000,0.000000}%
\pgfsetstrokecolor{currentstroke}%
\pgfsetdash{}{0pt}%
\pgfpathmoveto{\pgfqpoint{5.756830in}{0.539182in}}%
\pgfpathlineto{\pgfqpoint{5.756830in}{2.207310in}}%
\pgfusepath{stroke}%
\end{pgfscope}%
\begin{pgfscope}%
\pgfsetrectcap%
\pgfsetmiterjoin%
\pgfsetlinewidth{0.803000pt}%
\definecolor{currentstroke}{rgb}{0.000000,0.000000,0.000000}%
\pgfsetstrokecolor{currentstroke}%
\pgfsetdash{}{0pt}%
\pgfpathmoveto{\pgfqpoint{0.589591in}{0.539182in}}%
\pgfpathlineto{\pgfqpoint{5.756830in}{0.539182in}}%
\pgfusepath{stroke}%
\end{pgfscope}%
\begin{pgfscope}%
\pgfsetrectcap%
\pgfsetmiterjoin%
\pgfsetlinewidth{0.803000pt}%
\definecolor{currentstroke}{rgb}{0.000000,0.000000,0.000000}%
\pgfsetstrokecolor{currentstroke}%
\pgfsetdash{}{0pt}%
\pgfpathmoveto{\pgfqpoint{0.589591in}{2.207310in}}%
\pgfpathlineto{\pgfqpoint{5.756830in}{2.207310in}}%
\pgfusepath{stroke}%
\end{pgfscope}%
\begin{pgfscope}%
\pgfsetrectcap%
\pgfsetroundjoin%
\pgfsetlinewidth{2.007500pt}%
\definecolor{currentstroke}{rgb}{0.878431,0.878431,0.815686}%
\pgfsetstrokecolor{currentstroke}%
\pgfsetdash{}{0pt}%
\pgfpathmoveto{\pgfqpoint{4.839613in}{1.142630in}}%
\pgfpathlineto{\pgfqpoint{5.089613in}{1.142630in}}%
\pgfusepath{stroke}%
\end{pgfscope}%
\begin{pgfscope}%
\definecolor{textcolor}{rgb}{0.000000,0.000000,0.000000}%
\pgfsetstrokecolor{textcolor}%
\pgfsetfillcolor{textcolor}%
\pgftext[x=5.114613in,y=1.098880in,left,base]{\color{textcolor}\rmfamily\fontsize{9.000000}{10.800000}\selectfont T.+CPU1}%
\end{pgfscope}%
\begin{pgfscope}%
\pgfsetrectcap%
\pgfsetroundjoin%
\pgfsetlinewidth{2.007500pt}%
\definecolor{currentstroke}{rgb}{0.564706,0.564706,1.000000}%
\pgfsetstrokecolor{currentstroke}%
\pgfsetdash{}{0pt}%
\pgfpathmoveto{\pgfqpoint{4.839613in}{0.980831in}}%
\pgfpathlineto{\pgfqpoint{5.089613in}{0.980831in}}%
\pgfusepath{stroke}%
\end{pgfscope}%
\begin{pgfscope}%
\definecolor{textcolor}{rgb}{0.000000,0.000000,0.000000}%
\pgfsetstrokecolor{textcolor}%
\pgfsetfillcolor{textcolor}%
\pgftext[x=5.114613in,y=0.937081in,left,base]{\color{textcolor}\rmfamily\fontsize{9.000000}{10.800000}\selectfont P4+CPU1}%
\end{pgfscope}%
\begin{pgfscope}%
\pgfsetbuttcap%
\pgfsetroundjoin%
\pgfsetlinewidth{2.007500pt}%
\definecolor{currentstroke}{rgb}{0.564706,0.564706,1.000000}%
\pgfsetstrokecolor{currentstroke}%
\pgfsetdash{{2.000000pt}{3.300000pt}}{0.000000pt}%
\pgfpathmoveto{\pgfqpoint{4.839613in}{0.819031in}}%
\pgfpathlineto{\pgfqpoint{5.089613in}{0.819031in}}%
\pgfusepath{stroke}%
\end{pgfscope}%
\begin{pgfscope}%
\definecolor{textcolor}{rgb}{0.000000,0.000000,0.000000}%
\pgfsetstrokecolor{textcolor}%
\pgfsetfillcolor{textcolor}%
\pgftext[x=5.114613in,y=0.775281in,left,base]{\color{textcolor}\rmfamily\fontsize{9.000000}{10.800000}\selectfont P4+CPU8}%
\end{pgfscope}%
\begin{pgfscope}%
\pgfsetbuttcap%
\pgfsetroundjoin%
\pgfsetlinewidth{2.007500pt}%
\definecolor{currentstroke}{rgb}{0.564706,0.564706,1.000000}%
\pgfsetstrokecolor{currentstroke}%
\pgfsetdash{{7.400000pt}{3.200000pt}}{0.000000pt}%
\pgfpathmoveto{\pgfqpoint{4.839613in}{0.657232in}}%
\pgfpathlineto{\pgfqpoint{5.089613in}{0.657232in}}%
\pgfusepath{stroke}%
\end{pgfscope}%
\begin{pgfscope}%
\definecolor{textcolor}{rgb}{0.000000,0.000000,0.000000}%
\pgfsetstrokecolor{textcolor}%
\pgfsetfillcolor{textcolor}%
\pgftext[x=5.114613in,y=0.613482in,left,base]{\color{textcolor}\rmfamily\fontsize{9.000000}{10.800000}\selectfont P4+GPU}%
\end{pgfscope}%
\end{pgfpicture}%
\makeatother%
\endgroup%

\caption{\label{fig:performance-factor} A graph of the simulated PAR-2 score for various combinations of planners and hardware as the performance factor varies.}
\end{center}
\end{figure}

\begin{table}[t]
  \caption{\label{tab:performance_factor} The performance factor for each combination of planner and hardware that minimizes the simulated PAR-2 score.}
  \centering
    \begin{tabular}{|l|c|c|c|c|c|c|} \hline
 & \pkg{Tamaki} & \pkg{FlowCutter} & \pkg{htd} & \pkg{Hicks} & \pkg{P3} & \pkg{P4}\\ \hline 
\pkg{CPU1} & $3.8\cdot 10^{-11}$ & $4.8\cdot 10^{-12}$ & $1.6\cdot 10^{-12}$ & $1.0\cdot 10^{-21}$ & $1.4\cdot 10^{-11}$ & $1.6\cdot 10^{-11}$\\ \hline 
\pkg{CPU8} & $7.8\cdot 10^{-12}$ & $1.8\cdot 10^{-12}$ & $1.3\cdot 10^{-12}$ & $1.0\cdot 10^{-21}$ & $5.5\cdot 10^{-12}$ & $6.2\cdot 10^{-12}$\\ \hline 
\pkg{GPU} & $2.1\cdot 10^{-12}$ & $5.5\cdot 10^{-13}$ & $1.3\cdot 10^{-12}$ & $1.0\cdot 10^{-21}$ & $3.0 \cdot 10^{-12}$ & $3.8\cdot 10^{-12}$\\ \hline 
    \end{tabular}
\end{table}

Figure \ref{fig:performance-factor} indicates how varying the performance factor affects the simulated PAR-2 score for various combinations of planners and hardware. For each planner and hardware, Table 2 shows the performance factor $\alpha$ that minimizes the simulated PAR-2 score. We observe that the performance factor for \pkg{CPU8} is lower than for \pkg{CPU1}, but not necessarily higher or lower than for \pkg{GPU}. We conclude that different combinations of planners and hardware are optimized by different performance factors. % Further details are available in the supplemental material.

% Selected results are summarized in Figure \ref{fig:performance-factor}. For each planner and hardware, we select the performance factor $\alpha$ that minimizes the simulated PAR-2 score (i.e., the sum of of the wall-clock times for each completed benchmark, plus 2000 for each uncompleted benchmark). We observe that the performance factor for $\pkg{GPU}$ is typically lower than for $\pkg{CPU8}$, which is lower than for $\pkg{CPU1}$. We conclude that different combinations of planners and hardware are optimized by different performance factors. The full set of selected values is available in the supplemental material.

\subsection{Experiment 3: End-to-End Performance (RQ4 and RQ5)}
Finally, we compare \tool{TensorOrder2} with state-of-the-art weighted model counters \tool{cachet}, \tool{miniC2D}, \tool{d4}, \tool{ADDMC}, and \tool{gpuSAT2}. We consider \tool{TensorOrder2} using \pkg{P4} combined with each hardware configuration (\pkg{CPU1}, \pkg{CPU8}, and \pkg{GPU}), along with \pkg{Tamaki} + \pkg{CPU1} as the best non-parallel configuration from \cite{DDV19}. Note that \tool{P4}+\tool{CPU1} still leverages multiple cores in the planning phase. The performance factor from Experiment 2 is used for each \tool{TensorOrder2} configuration.

We run each counter once on each benchmark (both with and without \pkg{pmc-eq} preprocessing) with a timeout of 1000 seconds and record the wall-clock time taken. When preprocessing is used, both the timeout and the recorded time include preprocessing time. For \tool{TensorOrder2}, recorded times include all of Algorithm \ref{alg:wmc}. Results are summarized in Figure \ref{fig:parallel:comparison} and Table \ref{tab:comparison}. 

\begin{figure}[t]
\begin{center}
%% Creator: Matplotlib, PGF backend
%%
%% To include the figure in your LaTeX document, write
%%   \input{<filename>.pgf}
%%
%% Make sure the required packages are loaded in your preamble
%%   \usepackage{pgf}
%%
%% and, on pdftex
%%   \usepackage[utf8]{inputenc}\DeclareUnicodeCharacter{2212}{-}
%%
%% or, on luatex and xetex
%%   \usepackage{unicode-math}
%%
%% Figures using additional raster images can only be included by \input if
%% they are in the same directory as the main LaTeX file. For loading figures
%% from other directories you can use the `import` package
%%   \usepackage{import}
%%
%% and then include the figures with
%%   \import{<path to file>}{<filename>.pgf}
%%
%% Matplotlib used the following preamble
%%   \usepackage[utf8x]{inputenc}
%%   \usepackage[T1]{fontenc}
%%
\begingroup%
\makeatletter%
\begin{pgfpicture}%
\pgfpathrectangle{\pgfpointorigin}{\pgfqpoint{6.000000in}{5.500000in}}%
\pgfusepath{use as bounding box, clip}%
\begin{pgfscope}%
\pgfsetbuttcap%
\pgfsetmiterjoin%
\definecolor{currentfill}{rgb}{1.000000,1.000000,1.000000}%
\pgfsetfillcolor{currentfill}%
\pgfsetlinewidth{0.000000pt}%
\definecolor{currentstroke}{rgb}{1.000000,1.000000,1.000000}%
\pgfsetstrokecolor{currentstroke}%
\pgfsetdash{}{0pt}%
\pgfpathmoveto{\pgfqpoint{0.000000in}{0.000000in}}%
\pgfpathlineto{\pgfqpoint{6.000000in}{0.000000in}}%
\pgfpathlineto{\pgfqpoint{6.000000in}{5.500000in}}%
\pgfpathlineto{\pgfqpoint{0.000000in}{5.500000in}}%
\pgfpathclose%
\pgfusepath{fill}%
\end{pgfscope}%
\begin{pgfscope}%
\pgfsetbuttcap%
\pgfsetmiterjoin%
\definecolor{currentfill}{rgb}{1.000000,1.000000,1.000000}%
\pgfsetfillcolor{currentfill}%
\pgfsetlinewidth{0.000000pt}%
\definecolor{currentstroke}{rgb}{0.000000,0.000000,0.000000}%
\pgfsetstrokecolor{currentstroke}%
\pgfsetstrokeopacity{0.000000}%
\pgfsetdash{}{0pt}%
\pgfpathmoveto{\pgfqpoint{0.708220in}{3.210823in}}%
\pgfpathlineto{\pgfqpoint{5.721529in}{3.210823in}}%
\pgfpathlineto{\pgfqpoint{5.721529in}{5.305275in}}%
\pgfpathlineto{\pgfqpoint{0.708220in}{5.305275in}}%
\pgfpathclose%
\pgfusepath{fill}%
\end{pgfscope}%
\begin{pgfscope}%
\pgfsetbuttcap%
\pgfsetroundjoin%
\definecolor{currentfill}{rgb}{0.000000,0.000000,0.000000}%
\pgfsetfillcolor{currentfill}%
\pgfsetlinewidth{0.803000pt}%
\definecolor{currentstroke}{rgb}{0.000000,0.000000,0.000000}%
\pgfsetstrokecolor{currentstroke}%
\pgfsetdash{}{0pt}%
\pgfsys@defobject{currentmarker}{\pgfqpoint{0.000000in}{-0.048611in}}{\pgfqpoint{0.000000in}{0.000000in}}{%
\pgfpathmoveto{\pgfqpoint{0.000000in}{0.000000in}}%
\pgfpathlineto{\pgfqpoint{0.000000in}{-0.048611in}}%
\pgfusepath{stroke,fill}%
}%
\begin{pgfscope}%
\pgfsys@transformshift{0.708220in}{3.210823in}%
\pgfsys@useobject{currentmarker}{}%
\end{pgfscope}%
\end{pgfscope}%
\begin{pgfscope}%
\definecolor{textcolor}{rgb}{0.000000,0.000000,0.000000}%
\pgfsetstrokecolor{textcolor}%
\pgfsetfillcolor{textcolor}%
\pgftext[x=0.708220in,y=3.113600in,,top]{\color{textcolor}\rmfamily\fontsize{9.000000}{10.800000}\selectfont \(\displaystyle {0}\)}%
\end{pgfscope}%
\begin{pgfscope}%
\pgfsetbuttcap%
\pgfsetroundjoin%
\definecolor{currentfill}{rgb}{0.000000,0.000000,0.000000}%
\pgfsetfillcolor{currentfill}%
\pgfsetlinewidth{0.803000pt}%
\definecolor{currentstroke}{rgb}{0.000000,0.000000,0.000000}%
\pgfsetstrokecolor{currentstroke}%
\pgfsetdash{}{0pt}%
\pgfsys@defobject{currentmarker}{\pgfqpoint{0.000000in}{-0.048611in}}{\pgfqpoint{0.000000in}{0.000000in}}{%
\pgfpathmoveto{\pgfqpoint{0.000000in}{0.000000in}}%
\pgfpathlineto{\pgfqpoint{0.000000in}{-0.048611in}}%
\pgfusepath{stroke,fill}%
}%
\begin{pgfscope}%
\pgfsys@transformshift{1.334883in}{3.210823in}%
\pgfsys@useobject{currentmarker}{}%
\end{pgfscope}%
\end{pgfscope}%
\begin{pgfscope}%
\definecolor{textcolor}{rgb}{0.000000,0.000000,0.000000}%
\pgfsetstrokecolor{textcolor}%
\pgfsetfillcolor{textcolor}%
\pgftext[x=1.334883in,y=3.113600in,,top]{\color{textcolor}\rmfamily\fontsize{9.000000}{10.800000}\selectfont \(\displaystyle {250}\)}%
\end{pgfscope}%
\begin{pgfscope}%
\pgfsetbuttcap%
\pgfsetroundjoin%
\definecolor{currentfill}{rgb}{0.000000,0.000000,0.000000}%
\pgfsetfillcolor{currentfill}%
\pgfsetlinewidth{0.803000pt}%
\definecolor{currentstroke}{rgb}{0.000000,0.000000,0.000000}%
\pgfsetstrokecolor{currentstroke}%
\pgfsetdash{}{0pt}%
\pgfsys@defobject{currentmarker}{\pgfqpoint{0.000000in}{-0.048611in}}{\pgfqpoint{0.000000in}{0.000000in}}{%
\pgfpathmoveto{\pgfqpoint{0.000000in}{0.000000in}}%
\pgfpathlineto{\pgfqpoint{0.000000in}{-0.048611in}}%
\pgfusepath{stroke,fill}%
}%
\begin{pgfscope}%
\pgfsys@transformshift{1.961547in}{3.210823in}%
\pgfsys@useobject{currentmarker}{}%
\end{pgfscope}%
\end{pgfscope}%
\begin{pgfscope}%
\definecolor{textcolor}{rgb}{0.000000,0.000000,0.000000}%
\pgfsetstrokecolor{textcolor}%
\pgfsetfillcolor{textcolor}%
\pgftext[x=1.961547in,y=3.113600in,,top]{\color{textcolor}\rmfamily\fontsize{9.000000}{10.800000}\selectfont \(\displaystyle {500}\)}%
\end{pgfscope}%
\begin{pgfscope}%
\pgfsetbuttcap%
\pgfsetroundjoin%
\definecolor{currentfill}{rgb}{0.000000,0.000000,0.000000}%
\pgfsetfillcolor{currentfill}%
\pgfsetlinewidth{0.803000pt}%
\definecolor{currentstroke}{rgb}{0.000000,0.000000,0.000000}%
\pgfsetstrokecolor{currentstroke}%
\pgfsetdash{}{0pt}%
\pgfsys@defobject{currentmarker}{\pgfqpoint{0.000000in}{-0.048611in}}{\pgfqpoint{0.000000in}{0.000000in}}{%
\pgfpathmoveto{\pgfqpoint{0.000000in}{0.000000in}}%
\pgfpathlineto{\pgfqpoint{0.000000in}{-0.048611in}}%
\pgfusepath{stroke,fill}%
}%
\begin{pgfscope}%
\pgfsys@transformshift{2.588211in}{3.210823in}%
\pgfsys@useobject{currentmarker}{}%
\end{pgfscope}%
\end{pgfscope}%
\begin{pgfscope}%
\definecolor{textcolor}{rgb}{0.000000,0.000000,0.000000}%
\pgfsetstrokecolor{textcolor}%
\pgfsetfillcolor{textcolor}%
\pgftext[x=2.588211in,y=3.113600in,,top]{\color{textcolor}\rmfamily\fontsize{9.000000}{10.800000}\selectfont \(\displaystyle {750}\)}%
\end{pgfscope}%
\begin{pgfscope}%
\pgfsetbuttcap%
\pgfsetroundjoin%
\definecolor{currentfill}{rgb}{0.000000,0.000000,0.000000}%
\pgfsetfillcolor{currentfill}%
\pgfsetlinewidth{0.803000pt}%
\definecolor{currentstroke}{rgb}{0.000000,0.000000,0.000000}%
\pgfsetstrokecolor{currentstroke}%
\pgfsetdash{}{0pt}%
\pgfsys@defobject{currentmarker}{\pgfqpoint{0.000000in}{-0.048611in}}{\pgfqpoint{0.000000in}{0.000000in}}{%
\pgfpathmoveto{\pgfqpoint{0.000000in}{0.000000in}}%
\pgfpathlineto{\pgfqpoint{0.000000in}{-0.048611in}}%
\pgfusepath{stroke,fill}%
}%
\begin{pgfscope}%
\pgfsys@transformshift{3.214874in}{3.210823in}%
\pgfsys@useobject{currentmarker}{}%
\end{pgfscope}%
\end{pgfscope}%
\begin{pgfscope}%
\definecolor{textcolor}{rgb}{0.000000,0.000000,0.000000}%
\pgfsetstrokecolor{textcolor}%
\pgfsetfillcolor{textcolor}%
\pgftext[x=3.214874in,y=3.113600in,,top]{\color{textcolor}\rmfamily\fontsize{9.000000}{10.800000}\selectfont \(\displaystyle {1000}\)}%
\end{pgfscope}%
\begin{pgfscope}%
\pgfsetbuttcap%
\pgfsetroundjoin%
\definecolor{currentfill}{rgb}{0.000000,0.000000,0.000000}%
\pgfsetfillcolor{currentfill}%
\pgfsetlinewidth{0.803000pt}%
\definecolor{currentstroke}{rgb}{0.000000,0.000000,0.000000}%
\pgfsetstrokecolor{currentstroke}%
\pgfsetdash{}{0pt}%
\pgfsys@defobject{currentmarker}{\pgfqpoint{0.000000in}{-0.048611in}}{\pgfqpoint{0.000000in}{0.000000in}}{%
\pgfpathmoveto{\pgfqpoint{0.000000in}{0.000000in}}%
\pgfpathlineto{\pgfqpoint{0.000000in}{-0.048611in}}%
\pgfusepath{stroke,fill}%
}%
\begin{pgfscope}%
\pgfsys@transformshift{3.841538in}{3.210823in}%
\pgfsys@useobject{currentmarker}{}%
\end{pgfscope}%
\end{pgfscope}%
\begin{pgfscope}%
\definecolor{textcolor}{rgb}{0.000000,0.000000,0.000000}%
\pgfsetstrokecolor{textcolor}%
\pgfsetfillcolor{textcolor}%
\pgftext[x=3.841538in,y=3.113600in,,top]{\color{textcolor}\rmfamily\fontsize{9.000000}{10.800000}\selectfont \(\displaystyle {1250}\)}%
\end{pgfscope}%
\begin{pgfscope}%
\pgfsetbuttcap%
\pgfsetroundjoin%
\definecolor{currentfill}{rgb}{0.000000,0.000000,0.000000}%
\pgfsetfillcolor{currentfill}%
\pgfsetlinewidth{0.803000pt}%
\definecolor{currentstroke}{rgb}{0.000000,0.000000,0.000000}%
\pgfsetstrokecolor{currentstroke}%
\pgfsetdash{}{0pt}%
\pgfsys@defobject{currentmarker}{\pgfqpoint{0.000000in}{-0.048611in}}{\pgfqpoint{0.000000in}{0.000000in}}{%
\pgfpathmoveto{\pgfqpoint{0.000000in}{0.000000in}}%
\pgfpathlineto{\pgfqpoint{0.000000in}{-0.048611in}}%
\pgfusepath{stroke,fill}%
}%
\begin{pgfscope}%
\pgfsys@transformshift{4.468201in}{3.210823in}%
\pgfsys@useobject{currentmarker}{}%
\end{pgfscope}%
\end{pgfscope}%
\begin{pgfscope}%
\definecolor{textcolor}{rgb}{0.000000,0.000000,0.000000}%
\pgfsetstrokecolor{textcolor}%
\pgfsetfillcolor{textcolor}%
\pgftext[x=4.468201in,y=3.113600in,,top]{\color{textcolor}\rmfamily\fontsize{9.000000}{10.800000}\selectfont \(\displaystyle {1500}\)}%
\end{pgfscope}%
\begin{pgfscope}%
\pgfsetbuttcap%
\pgfsetroundjoin%
\definecolor{currentfill}{rgb}{0.000000,0.000000,0.000000}%
\pgfsetfillcolor{currentfill}%
\pgfsetlinewidth{0.803000pt}%
\definecolor{currentstroke}{rgb}{0.000000,0.000000,0.000000}%
\pgfsetstrokecolor{currentstroke}%
\pgfsetdash{}{0pt}%
\pgfsys@defobject{currentmarker}{\pgfqpoint{0.000000in}{-0.048611in}}{\pgfqpoint{0.000000in}{0.000000in}}{%
\pgfpathmoveto{\pgfqpoint{0.000000in}{0.000000in}}%
\pgfpathlineto{\pgfqpoint{0.000000in}{-0.048611in}}%
\pgfusepath{stroke,fill}%
}%
\begin{pgfscope}%
\pgfsys@transformshift{5.094865in}{3.210823in}%
\pgfsys@useobject{currentmarker}{}%
\end{pgfscope}%
\end{pgfscope}%
\begin{pgfscope}%
\definecolor{textcolor}{rgb}{0.000000,0.000000,0.000000}%
\pgfsetstrokecolor{textcolor}%
\pgfsetfillcolor{textcolor}%
\pgftext[x=5.094865in,y=3.113600in,,top]{\color{textcolor}\rmfamily\fontsize{9.000000}{10.800000}\selectfont \(\displaystyle {1750}\)}%
\end{pgfscope}%
\begin{pgfscope}%
\pgfsetbuttcap%
\pgfsetroundjoin%
\definecolor{currentfill}{rgb}{0.000000,0.000000,0.000000}%
\pgfsetfillcolor{currentfill}%
\pgfsetlinewidth{0.803000pt}%
\definecolor{currentstroke}{rgb}{0.000000,0.000000,0.000000}%
\pgfsetstrokecolor{currentstroke}%
\pgfsetdash{}{0pt}%
\pgfsys@defobject{currentmarker}{\pgfqpoint{0.000000in}{-0.048611in}}{\pgfqpoint{0.000000in}{0.000000in}}{%
\pgfpathmoveto{\pgfqpoint{0.000000in}{0.000000in}}%
\pgfpathlineto{\pgfqpoint{0.000000in}{-0.048611in}}%
\pgfusepath{stroke,fill}%
}%
\begin{pgfscope}%
\pgfsys@transformshift{5.721529in}{3.210823in}%
\pgfsys@useobject{currentmarker}{}%
\end{pgfscope}%
\end{pgfscope}%
\begin{pgfscope}%
\definecolor{textcolor}{rgb}{0.000000,0.000000,0.000000}%
\pgfsetstrokecolor{textcolor}%
\pgfsetfillcolor{textcolor}%
\pgftext[x=5.721529in,y=3.113600in,,top]{\color{textcolor}\rmfamily\fontsize{9.000000}{10.800000}\selectfont \(\displaystyle {2000}\)}%
\end{pgfscope}%
\begin{pgfscope}%
\definecolor{textcolor}{rgb}{0.000000,0.000000,0.000000}%
\pgfsetstrokecolor{textcolor}%
\pgfsetfillcolor{textcolor}%
\pgftext[x=3.214874in,y=2.947655in,,top]{\color{textcolor}\rmfamily\fontsize{10.000000}{12.000000}\selectfont Number of benchmarks solved}%
\end{pgfscope}%
\begin{pgfscope}%
\pgfsetbuttcap%
\pgfsetroundjoin%
\definecolor{currentfill}{rgb}{0.000000,0.000000,0.000000}%
\pgfsetfillcolor{currentfill}%
\pgfsetlinewidth{0.803000pt}%
\definecolor{currentstroke}{rgb}{0.000000,0.000000,0.000000}%
\pgfsetstrokecolor{currentstroke}%
\pgfsetdash{}{0pt}%
\pgfsys@defobject{currentmarker}{\pgfqpoint{-0.048611in}{0.000000in}}{\pgfqpoint{-0.000000in}{0.000000in}}{%
\pgfpathmoveto{\pgfqpoint{-0.000000in}{0.000000in}}%
\pgfpathlineto{\pgfqpoint{-0.048611in}{0.000000in}}%
\pgfusepath{stroke,fill}%
}%
\begin{pgfscope}%
\pgfsys@transformshift{0.708220in}{3.329760in}%
\pgfsys@useobject{currentmarker}{}%
\end{pgfscope}%
\end{pgfscope}%
\begin{pgfscope}%
\definecolor{textcolor}{rgb}{0.000000,0.000000,0.000000}%
\pgfsetstrokecolor{textcolor}%
\pgfsetfillcolor{textcolor}%
\pgftext[x=0.344411in, y=3.285036in, left, base]{\color{textcolor}\rmfamily\fontsize{9.000000}{10.800000}\selectfont \(\displaystyle {10^{-2}}\)}%
\end{pgfscope}%
\begin{pgfscope}%
\pgfsetbuttcap%
\pgfsetroundjoin%
\definecolor{currentfill}{rgb}{0.000000,0.000000,0.000000}%
\pgfsetfillcolor{currentfill}%
\pgfsetlinewidth{0.803000pt}%
\definecolor{currentstroke}{rgb}{0.000000,0.000000,0.000000}%
\pgfsetstrokecolor{currentstroke}%
\pgfsetdash{}{0pt}%
\pgfsys@defobject{currentmarker}{\pgfqpoint{-0.048611in}{0.000000in}}{\pgfqpoint{-0.000000in}{0.000000in}}{%
\pgfpathmoveto{\pgfqpoint{-0.000000in}{0.000000in}}%
\pgfpathlineto{\pgfqpoint{-0.048611in}{0.000000in}}%
\pgfusepath{stroke,fill}%
}%
\begin{pgfscope}%
\pgfsys@transformshift{0.708220in}{3.724863in}%
\pgfsys@useobject{currentmarker}{}%
\end{pgfscope}%
\end{pgfscope}%
\begin{pgfscope}%
\definecolor{textcolor}{rgb}{0.000000,0.000000,0.000000}%
\pgfsetstrokecolor{textcolor}%
\pgfsetfillcolor{textcolor}%
\pgftext[x=0.344411in, y=3.680139in, left, base]{\color{textcolor}\rmfamily\fontsize{9.000000}{10.800000}\selectfont \(\displaystyle {10^{-1}}\)}%
\end{pgfscope}%
\begin{pgfscope}%
\pgfsetbuttcap%
\pgfsetroundjoin%
\definecolor{currentfill}{rgb}{0.000000,0.000000,0.000000}%
\pgfsetfillcolor{currentfill}%
\pgfsetlinewidth{0.803000pt}%
\definecolor{currentstroke}{rgb}{0.000000,0.000000,0.000000}%
\pgfsetstrokecolor{currentstroke}%
\pgfsetdash{}{0pt}%
\pgfsys@defobject{currentmarker}{\pgfqpoint{-0.048611in}{0.000000in}}{\pgfqpoint{-0.000000in}{0.000000in}}{%
\pgfpathmoveto{\pgfqpoint{-0.000000in}{0.000000in}}%
\pgfpathlineto{\pgfqpoint{-0.048611in}{0.000000in}}%
\pgfusepath{stroke,fill}%
}%
\begin{pgfscope}%
\pgfsys@transformshift{0.708220in}{4.119966in}%
\pgfsys@useobject{currentmarker}{}%
\end{pgfscope}%
\end{pgfscope}%
\begin{pgfscope}%
\definecolor{textcolor}{rgb}{0.000000,0.000000,0.000000}%
\pgfsetstrokecolor{textcolor}%
\pgfsetfillcolor{textcolor}%
\pgftext[x=0.424657in, y=4.075242in, left, base]{\color{textcolor}\rmfamily\fontsize{9.000000}{10.800000}\selectfont \(\displaystyle {10^{0}}\)}%
\end{pgfscope}%
\begin{pgfscope}%
\pgfsetbuttcap%
\pgfsetroundjoin%
\definecolor{currentfill}{rgb}{0.000000,0.000000,0.000000}%
\pgfsetfillcolor{currentfill}%
\pgfsetlinewidth{0.803000pt}%
\definecolor{currentstroke}{rgb}{0.000000,0.000000,0.000000}%
\pgfsetstrokecolor{currentstroke}%
\pgfsetdash{}{0pt}%
\pgfsys@defobject{currentmarker}{\pgfqpoint{-0.048611in}{0.000000in}}{\pgfqpoint{-0.000000in}{0.000000in}}{%
\pgfpathmoveto{\pgfqpoint{-0.000000in}{0.000000in}}%
\pgfpathlineto{\pgfqpoint{-0.048611in}{0.000000in}}%
\pgfusepath{stroke,fill}%
}%
\begin{pgfscope}%
\pgfsys@transformshift{0.708220in}{4.515069in}%
\pgfsys@useobject{currentmarker}{}%
\end{pgfscope}%
\end{pgfscope}%
\begin{pgfscope}%
\definecolor{textcolor}{rgb}{0.000000,0.000000,0.000000}%
\pgfsetstrokecolor{textcolor}%
\pgfsetfillcolor{textcolor}%
\pgftext[x=0.424657in, y=4.470345in, left, base]{\color{textcolor}\rmfamily\fontsize{9.000000}{10.800000}\selectfont \(\displaystyle {10^{1}}\)}%
\end{pgfscope}%
\begin{pgfscope}%
\pgfsetbuttcap%
\pgfsetroundjoin%
\definecolor{currentfill}{rgb}{0.000000,0.000000,0.000000}%
\pgfsetfillcolor{currentfill}%
\pgfsetlinewidth{0.803000pt}%
\definecolor{currentstroke}{rgb}{0.000000,0.000000,0.000000}%
\pgfsetstrokecolor{currentstroke}%
\pgfsetdash{}{0pt}%
\pgfsys@defobject{currentmarker}{\pgfqpoint{-0.048611in}{0.000000in}}{\pgfqpoint{-0.000000in}{0.000000in}}{%
\pgfpathmoveto{\pgfqpoint{-0.000000in}{0.000000in}}%
\pgfpathlineto{\pgfqpoint{-0.048611in}{0.000000in}}%
\pgfusepath{stroke,fill}%
}%
\begin{pgfscope}%
\pgfsys@transformshift{0.708220in}{4.910172in}%
\pgfsys@useobject{currentmarker}{}%
\end{pgfscope}%
\end{pgfscope}%
\begin{pgfscope}%
\definecolor{textcolor}{rgb}{0.000000,0.000000,0.000000}%
\pgfsetstrokecolor{textcolor}%
\pgfsetfillcolor{textcolor}%
\pgftext[x=0.424657in, y=4.865447in, left, base]{\color{textcolor}\rmfamily\fontsize{9.000000}{10.800000}\selectfont \(\displaystyle {10^{2}}\)}%
\end{pgfscope}%
\begin{pgfscope}%
\pgfsetbuttcap%
\pgfsetroundjoin%
\definecolor{currentfill}{rgb}{0.000000,0.000000,0.000000}%
\pgfsetfillcolor{currentfill}%
\pgfsetlinewidth{0.803000pt}%
\definecolor{currentstroke}{rgb}{0.000000,0.000000,0.000000}%
\pgfsetstrokecolor{currentstroke}%
\pgfsetdash{}{0pt}%
\pgfsys@defobject{currentmarker}{\pgfqpoint{-0.048611in}{0.000000in}}{\pgfqpoint{-0.000000in}{0.000000in}}{%
\pgfpathmoveto{\pgfqpoint{-0.000000in}{0.000000in}}%
\pgfpathlineto{\pgfqpoint{-0.048611in}{0.000000in}}%
\pgfusepath{stroke,fill}%
}%
\begin{pgfscope}%
\pgfsys@transformshift{0.708220in}{5.305275in}%
\pgfsys@useobject{currentmarker}{}%
\end{pgfscope}%
\end{pgfscope}%
\begin{pgfscope}%
\definecolor{textcolor}{rgb}{0.000000,0.000000,0.000000}%
\pgfsetstrokecolor{textcolor}%
\pgfsetfillcolor{textcolor}%
\pgftext[x=0.424657in, y=5.260550in, left, base]{\color{textcolor}\rmfamily\fontsize{9.000000}{10.800000}\selectfont \(\displaystyle {10^{3}}\)}%
\end{pgfscope}%
\begin{pgfscope}%
\pgfsetbuttcap%
\pgfsetroundjoin%
\definecolor{currentfill}{rgb}{0.000000,0.000000,0.000000}%
\pgfsetfillcolor{currentfill}%
\pgfsetlinewidth{0.602250pt}%
\definecolor{currentstroke}{rgb}{0.000000,0.000000,0.000000}%
\pgfsetstrokecolor{currentstroke}%
\pgfsetdash{}{0pt}%
\pgfsys@defobject{currentmarker}{\pgfqpoint{-0.027778in}{0.000000in}}{\pgfqpoint{-0.000000in}{0.000000in}}{%
\pgfpathmoveto{\pgfqpoint{-0.000000in}{0.000000in}}%
\pgfpathlineto{\pgfqpoint{-0.027778in}{0.000000in}}%
\pgfusepath{stroke,fill}%
}%
\begin{pgfscope}%
\pgfsys@transformshift{0.708220in}{3.210823in}%
\pgfsys@useobject{currentmarker}{}%
\end{pgfscope}%
\end{pgfscope}%
\begin{pgfscope}%
\pgfsetbuttcap%
\pgfsetroundjoin%
\definecolor{currentfill}{rgb}{0.000000,0.000000,0.000000}%
\pgfsetfillcolor{currentfill}%
\pgfsetlinewidth{0.602250pt}%
\definecolor{currentstroke}{rgb}{0.000000,0.000000,0.000000}%
\pgfsetstrokecolor{currentstroke}%
\pgfsetdash{}{0pt}%
\pgfsys@defobject{currentmarker}{\pgfqpoint{-0.027778in}{0.000000in}}{\pgfqpoint{-0.000000in}{0.000000in}}{%
\pgfpathmoveto{\pgfqpoint{-0.000000in}{0.000000in}}%
\pgfpathlineto{\pgfqpoint{-0.027778in}{0.000000in}}%
\pgfusepath{stroke,fill}%
}%
\begin{pgfscope}%
\pgfsys@transformshift{0.708220in}{3.242107in}%
\pgfsys@useobject{currentmarker}{}%
\end{pgfscope}%
\end{pgfscope}%
\begin{pgfscope}%
\pgfsetbuttcap%
\pgfsetroundjoin%
\definecolor{currentfill}{rgb}{0.000000,0.000000,0.000000}%
\pgfsetfillcolor{currentfill}%
\pgfsetlinewidth{0.602250pt}%
\definecolor{currentstroke}{rgb}{0.000000,0.000000,0.000000}%
\pgfsetstrokecolor{currentstroke}%
\pgfsetdash{}{0pt}%
\pgfsys@defobject{currentmarker}{\pgfqpoint{-0.027778in}{0.000000in}}{\pgfqpoint{-0.000000in}{0.000000in}}{%
\pgfpathmoveto{\pgfqpoint{-0.000000in}{0.000000in}}%
\pgfpathlineto{\pgfqpoint{-0.027778in}{0.000000in}}%
\pgfusepath{stroke,fill}%
}%
\begin{pgfscope}%
\pgfsys@transformshift{0.708220in}{3.268558in}%
\pgfsys@useobject{currentmarker}{}%
\end{pgfscope}%
\end{pgfscope}%
\begin{pgfscope}%
\pgfsetbuttcap%
\pgfsetroundjoin%
\definecolor{currentfill}{rgb}{0.000000,0.000000,0.000000}%
\pgfsetfillcolor{currentfill}%
\pgfsetlinewidth{0.602250pt}%
\definecolor{currentstroke}{rgb}{0.000000,0.000000,0.000000}%
\pgfsetstrokecolor{currentstroke}%
\pgfsetdash{}{0pt}%
\pgfsys@defobject{currentmarker}{\pgfqpoint{-0.027778in}{0.000000in}}{\pgfqpoint{-0.000000in}{0.000000in}}{%
\pgfpathmoveto{\pgfqpoint{-0.000000in}{0.000000in}}%
\pgfpathlineto{\pgfqpoint{-0.027778in}{0.000000in}}%
\pgfusepath{stroke,fill}%
}%
\begin{pgfscope}%
\pgfsys@transformshift{0.708220in}{3.291471in}%
\pgfsys@useobject{currentmarker}{}%
\end{pgfscope}%
\end{pgfscope}%
\begin{pgfscope}%
\pgfsetbuttcap%
\pgfsetroundjoin%
\definecolor{currentfill}{rgb}{0.000000,0.000000,0.000000}%
\pgfsetfillcolor{currentfill}%
\pgfsetlinewidth{0.602250pt}%
\definecolor{currentstroke}{rgb}{0.000000,0.000000,0.000000}%
\pgfsetstrokecolor{currentstroke}%
\pgfsetdash{}{0pt}%
\pgfsys@defobject{currentmarker}{\pgfqpoint{-0.027778in}{0.000000in}}{\pgfqpoint{-0.000000in}{0.000000in}}{%
\pgfpathmoveto{\pgfqpoint{-0.000000in}{0.000000in}}%
\pgfpathlineto{\pgfqpoint{-0.027778in}{0.000000in}}%
\pgfusepath{stroke,fill}%
}%
\begin{pgfscope}%
\pgfsys@transformshift{0.708220in}{3.311682in}%
\pgfsys@useobject{currentmarker}{}%
\end{pgfscope}%
\end{pgfscope}%
\begin{pgfscope}%
\pgfsetbuttcap%
\pgfsetroundjoin%
\definecolor{currentfill}{rgb}{0.000000,0.000000,0.000000}%
\pgfsetfillcolor{currentfill}%
\pgfsetlinewidth{0.602250pt}%
\definecolor{currentstroke}{rgb}{0.000000,0.000000,0.000000}%
\pgfsetstrokecolor{currentstroke}%
\pgfsetdash{}{0pt}%
\pgfsys@defobject{currentmarker}{\pgfqpoint{-0.027778in}{0.000000in}}{\pgfqpoint{-0.000000in}{0.000000in}}{%
\pgfpathmoveto{\pgfqpoint{-0.000000in}{0.000000in}}%
\pgfpathlineto{\pgfqpoint{-0.027778in}{0.000000in}}%
\pgfusepath{stroke,fill}%
}%
\begin{pgfscope}%
\pgfsys@transformshift{0.708220in}{3.448698in}%
\pgfsys@useobject{currentmarker}{}%
\end{pgfscope}%
\end{pgfscope}%
\begin{pgfscope}%
\pgfsetbuttcap%
\pgfsetroundjoin%
\definecolor{currentfill}{rgb}{0.000000,0.000000,0.000000}%
\pgfsetfillcolor{currentfill}%
\pgfsetlinewidth{0.602250pt}%
\definecolor{currentstroke}{rgb}{0.000000,0.000000,0.000000}%
\pgfsetstrokecolor{currentstroke}%
\pgfsetdash{}{0pt}%
\pgfsys@defobject{currentmarker}{\pgfqpoint{-0.027778in}{0.000000in}}{\pgfqpoint{-0.000000in}{0.000000in}}{%
\pgfpathmoveto{\pgfqpoint{-0.000000in}{0.000000in}}%
\pgfpathlineto{\pgfqpoint{-0.027778in}{0.000000in}}%
\pgfusepath{stroke,fill}%
}%
\begin{pgfscope}%
\pgfsys@transformshift{0.708220in}{3.518272in}%
\pgfsys@useobject{currentmarker}{}%
\end{pgfscope}%
\end{pgfscope}%
\begin{pgfscope}%
\pgfsetbuttcap%
\pgfsetroundjoin%
\definecolor{currentfill}{rgb}{0.000000,0.000000,0.000000}%
\pgfsetfillcolor{currentfill}%
\pgfsetlinewidth{0.602250pt}%
\definecolor{currentstroke}{rgb}{0.000000,0.000000,0.000000}%
\pgfsetstrokecolor{currentstroke}%
\pgfsetdash{}{0pt}%
\pgfsys@defobject{currentmarker}{\pgfqpoint{-0.027778in}{0.000000in}}{\pgfqpoint{-0.000000in}{0.000000in}}{%
\pgfpathmoveto{\pgfqpoint{-0.000000in}{0.000000in}}%
\pgfpathlineto{\pgfqpoint{-0.027778in}{0.000000in}}%
\pgfusepath{stroke,fill}%
}%
\begin{pgfscope}%
\pgfsys@transformshift{0.708220in}{3.567636in}%
\pgfsys@useobject{currentmarker}{}%
\end{pgfscope}%
\end{pgfscope}%
\begin{pgfscope}%
\pgfsetbuttcap%
\pgfsetroundjoin%
\definecolor{currentfill}{rgb}{0.000000,0.000000,0.000000}%
\pgfsetfillcolor{currentfill}%
\pgfsetlinewidth{0.602250pt}%
\definecolor{currentstroke}{rgb}{0.000000,0.000000,0.000000}%
\pgfsetstrokecolor{currentstroke}%
\pgfsetdash{}{0pt}%
\pgfsys@defobject{currentmarker}{\pgfqpoint{-0.027778in}{0.000000in}}{\pgfqpoint{-0.000000in}{0.000000in}}{%
\pgfpathmoveto{\pgfqpoint{-0.000000in}{0.000000in}}%
\pgfpathlineto{\pgfqpoint{-0.027778in}{0.000000in}}%
\pgfusepath{stroke,fill}%
}%
\begin{pgfscope}%
\pgfsys@transformshift{0.708220in}{3.605926in}%
\pgfsys@useobject{currentmarker}{}%
\end{pgfscope}%
\end{pgfscope}%
\begin{pgfscope}%
\pgfsetbuttcap%
\pgfsetroundjoin%
\definecolor{currentfill}{rgb}{0.000000,0.000000,0.000000}%
\pgfsetfillcolor{currentfill}%
\pgfsetlinewidth{0.602250pt}%
\definecolor{currentstroke}{rgb}{0.000000,0.000000,0.000000}%
\pgfsetstrokecolor{currentstroke}%
\pgfsetdash{}{0pt}%
\pgfsys@defobject{currentmarker}{\pgfqpoint{-0.027778in}{0.000000in}}{\pgfqpoint{-0.000000in}{0.000000in}}{%
\pgfpathmoveto{\pgfqpoint{-0.000000in}{0.000000in}}%
\pgfpathlineto{\pgfqpoint{-0.027778in}{0.000000in}}%
\pgfusepath{stroke,fill}%
}%
\begin{pgfscope}%
\pgfsys@transformshift{0.708220in}{3.637210in}%
\pgfsys@useobject{currentmarker}{}%
\end{pgfscope}%
\end{pgfscope}%
\begin{pgfscope}%
\pgfsetbuttcap%
\pgfsetroundjoin%
\definecolor{currentfill}{rgb}{0.000000,0.000000,0.000000}%
\pgfsetfillcolor{currentfill}%
\pgfsetlinewidth{0.602250pt}%
\definecolor{currentstroke}{rgb}{0.000000,0.000000,0.000000}%
\pgfsetstrokecolor{currentstroke}%
\pgfsetdash{}{0pt}%
\pgfsys@defobject{currentmarker}{\pgfqpoint{-0.027778in}{0.000000in}}{\pgfqpoint{-0.000000in}{0.000000in}}{%
\pgfpathmoveto{\pgfqpoint{-0.000000in}{0.000000in}}%
\pgfpathlineto{\pgfqpoint{-0.027778in}{0.000000in}}%
\pgfusepath{stroke,fill}%
}%
\begin{pgfscope}%
\pgfsys@transformshift{0.708220in}{3.663661in}%
\pgfsys@useobject{currentmarker}{}%
\end{pgfscope}%
\end{pgfscope}%
\begin{pgfscope}%
\pgfsetbuttcap%
\pgfsetroundjoin%
\definecolor{currentfill}{rgb}{0.000000,0.000000,0.000000}%
\pgfsetfillcolor{currentfill}%
\pgfsetlinewidth{0.602250pt}%
\definecolor{currentstroke}{rgb}{0.000000,0.000000,0.000000}%
\pgfsetstrokecolor{currentstroke}%
\pgfsetdash{}{0pt}%
\pgfsys@defobject{currentmarker}{\pgfqpoint{-0.027778in}{0.000000in}}{\pgfqpoint{-0.000000in}{0.000000in}}{%
\pgfpathmoveto{\pgfqpoint{-0.000000in}{0.000000in}}%
\pgfpathlineto{\pgfqpoint{-0.027778in}{0.000000in}}%
\pgfusepath{stroke,fill}%
}%
\begin{pgfscope}%
\pgfsys@transformshift{0.708220in}{3.686574in}%
\pgfsys@useobject{currentmarker}{}%
\end{pgfscope}%
\end{pgfscope}%
\begin{pgfscope}%
\pgfsetbuttcap%
\pgfsetroundjoin%
\definecolor{currentfill}{rgb}{0.000000,0.000000,0.000000}%
\pgfsetfillcolor{currentfill}%
\pgfsetlinewidth{0.602250pt}%
\definecolor{currentstroke}{rgb}{0.000000,0.000000,0.000000}%
\pgfsetstrokecolor{currentstroke}%
\pgfsetdash{}{0pt}%
\pgfsys@defobject{currentmarker}{\pgfqpoint{-0.027778in}{0.000000in}}{\pgfqpoint{-0.000000in}{0.000000in}}{%
\pgfpathmoveto{\pgfqpoint{-0.000000in}{0.000000in}}%
\pgfpathlineto{\pgfqpoint{-0.027778in}{0.000000in}}%
\pgfusepath{stroke,fill}%
}%
\begin{pgfscope}%
\pgfsys@transformshift{0.708220in}{3.706785in}%
\pgfsys@useobject{currentmarker}{}%
\end{pgfscope}%
\end{pgfscope}%
\begin{pgfscope}%
\pgfsetbuttcap%
\pgfsetroundjoin%
\definecolor{currentfill}{rgb}{0.000000,0.000000,0.000000}%
\pgfsetfillcolor{currentfill}%
\pgfsetlinewidth{0.602250pt}%
\definecolor{currentstroke}{rgb}{0.000000,0.000000,0.000000}%
\pgfsetstrokecolor{currentstroke}%
\pgfsetdash{}{0pt}%
\pgfsys@defobject{currentmarker}{\pgfqpoint{-0.027778in}{0.000000in}}{\pgfqpoint{-0.000000in}{0.000000in}}{%
\pgfpathmoveto{\pgfqpoint{-0.000000in}{0.000000in}}%
\pgfpathlineto{\pgfqpoint{-0.027778in}{0.000000in}}%
\pgfusepath{stroke,fill}%
}%
\begin{pgfscope}%
\pgfsys@transformshift{0.708220in}{3.843801in}%
\pgfsys@useobject{currentmarker}{}%
\end{pgfscope}%
\end{pgfscope}%
\begin{pgfscope}%
\pgfsetbuttcap%
\pgfsetroundjoin%
\definecolor{currentfill}{rgb}{0.000000,0.000000,0.000000}%
\pgfsetfillcolor{currentfill}%
\pgfsetlinewidth{0.602250pt}%
\definecolor{currentstroke}{rgb}{0.000000,0.000000,0.000000}%
\pgfsetstrokecolor{currentstroke}%
\pgfsetdash{}{0pt}%
\pgfsys@defobject{currentmarker}{\pgfqpoint{-0.027778in}{0.000000in}}{\pgfqpoint{-0.000000in}{0.000000in}}{%
\pgfpathmoveto{\pgfqpoint{-0.000000in}{0.000000in}}%
\pgfpathlineto{\pgfqpoint{-0.027778in}{0.000000in}}%
\pgfusepath{stroke,fill}%
}%
\begin{pgfscope}%
\pgfsys@transformshift{0.708220in}{3.913375in}%
\pgfsys@useobject{currentmarker}{}%
\end{pgfscope}%
\end{pgfscope}%
\begin{pgfscope}%
\pgfsetbuttcap%
\pgfsetroundjoin%
\definecolor{currentfill}{rgb}{0.000000,0.000000,0.000000}%
\pgfsetfillcolor{currentfill}%
\pgfsetlinewidth{0.602250pt}%
\definecolor{currentstroke}{rgb}{0.000000,0.000000,0.000000}%
\pgfsetstrokecolor{currentstroke}%
\pgfsetdash{}{0pt}%
\pgfsys@defobject{currentmarker}{\pgfqpoint{-0.027778in}{0.000000in}}{\pgfqpoint{-0.000000in}{0.000000in}}{%
\pgfpathmoveto{\pgfqpoint{-0.000000in}{0.000000in}}%
\pgfpathlineto{\pgfqpoint{-0.027778in}{0.000000in}}%
\pgfusepath{stroke,fill}%
}%
\begin{pgfscope}%
\pgfsys@transformshift{0.708220in}{3.962739in}%
\pgfsys@useobject{currentmarker}{}%
\end{pgfscope}%
\end{pgfscope}%
\begin{pgfscope}%
\pgfsetbuttcap%
\pgfsetroundjoin%
\definecolor{currentfill}{rgb}{0.000000,0.000000,0.000000}%
\pgfsetfillcolor{currentfill}%
\pgfsetlinewidth{0.602250pt}%
\definecolor{currentstroke}{rgb}{0.000000,0.000000,0.000000}%
\pgfsetstrokecolor{currentstroke}%
\pgfsetdash{}{0pt}%
\pgfsys@defobject{currentmarker}{\pgfqpoint{-0.027778in}{0.000000in}}{\pgfqpoint{-0.000000in}{0.000000in}}{%
\pgfpathmoveto{\pgfqpoint{-0.000000in}{0.000000in}}%
\pgfpathlineto{\pgfqpoint{-0.027778in}{0.000000in}}%
\pgfusepath{stroke,fill}%
}%
\begin{pgfscope}%
\pgfsys@transformshift{0.708220in}{4.001029in}%
\pgfsys@useobject{currentmarker}{}%
\end{pgfscope}%
\end{pgfscope}%
\begin{pgfscope}%
\pgfsetbuttcap%
\pgfsetroundjoin%
\definecolor{currentfill}{rgb}{0.000000,0.000000,0.000000}%
\pgfsetfillcolor{currentfill}%
\pgfsetlinewidth{0.602250pt}%
\definecolor{currentstroke}{rgb}{0.000000,0.000000,0.000000}%
\pgfsetstrokecolor{currentstroke}%
\pgfsetdash{}{0pt}%
\pgfsys@defobject{currentmarker}{\pgfqpoint{-0.027778in}{0.000000in}}{\pgfqpoint{-0.000000in}{0.000000in}}{%
\pgfpathmoveto{\pgfqpoint{-0.000000in}{0.000000in}}%
\pgfpathlineto{\pgfqpoint{-0.027778in}{0.000000in}}%
\pgfusepath{stroke,fill}%
}%
\begin{pgfscope}%
\pgfsys@transformshift{0.708220in}{4.032313in}%
\pgfsys@useobject{currentmarker}{}%
\end{pgfscope}%
\end{pgfscope}%
\begin{pgfscope}%
\pgfsetbuttcap%
\pgfsetroundjoin%
\definecolor{currentfill}{rgb}{0.000000,0.000000,0.000000}%
\pgfsetfillcolor{currentfill}%
\pgfsetlinewidth{0.602250pt}%
\definecolor{currentstroke}{rgb}{0.000000,0.000000,0.000000}%
\pgfsetstrokecolor{currentstroke}%
\pgfsetdash{}{0pt}%
\pgfsys@defobject{currentmarker}{\pgfqpoint{-0.027778in}{0.000000in}}{\pgfqpoint{-0.000000in}{0.000000in}}{%
\pgfpathmoveto{\pgfqpoint{-0.000000in}{0.000000in}}%
\pgfpathlineto{\pgfqpoint{-0.027778in}{0.000000in}}%
\pgfusepath{stroke,fill}%
}%
\begin{pgfscope}%
\pgfsys@transformshift{0.708220in}{4.058764in}%
\pgfsys@useobject{currentmarker}{}%
\end{pgfscope}%
\end{pgfscope}%
\begin{pgfscope}%
\pgfsetbuttcap%
\pgfsetroundjoin%
\definecolor{currentfill}{rgb}{0.000000,0.000000,0.000000}%
\pgfsetfillcolor{currentfill}%
\pgfsetlinewidth{0.602250pt}%
\definecolor{currentstroke}{rgb}{0.000000,0.000000,0.000000}%
\pgfsetstrokecolor{currentstroke}%
\pgfsetdash{}{0pt}%
\pgfsys@defobject{currentmarker}{\pgfqpoint{-0.027778in}{0.000000in}}{\pgfqpoint{-0.000000in}{0.000000in}}{%
\pgfpathmoveto{\pgfqpoint{-0.000000in}{0.000000in}}%
\pgfpathlineto{\pgfqpoint{-0.027778in}{0.000000in}}%
\pgfusepath{stroke,fill}%
}%
\begin{pgfscope}%
\pgfsys@transformshift{0.708220in}{4.081677in}%
\pgfsys@useobject{currentmarker}{}%
\end{pgfscope}%
\end{pgfscope}%
\begin{pgfscope}%
\pgfsetbuttcap%
\pgfsetroundjoin%
\definecolor{currentfill}{rgb}{0.000000,0.000000,0.000000}%
\pgfsetfillcolor{currentfill}%
\pgfsetlinewidth{0.602250pt}%
\definecolor{currentstroke}{rgb}{0.000000,0.000000,0.000000}%
\pgfsetstrokecolor{currentstroke}%
\pgfsetdash{}{0pt}%
\pgfsys@defobject{currentmarker}{\pgfqpoint{-0.027778in}{0.000000in}}{\pgfqpoint{-0.000000in}{0.000000in}}{%
\pgfpathmoveto{\pgfqpoint{-0.000000in}{0.000000in}}%
\pgfpathlineto{\pgfqpoint{-0.027778in}{0.000000in}}%
\pgfusepath{stroke,fill}%
}%
\begin{pgfscope}%
\pgfsys@transformshift{0.708220in}{4.101887in}%
\pgfsys@useobject{currentmarker}{}%
\end{pgfscope}%
\end{pgfscope}%
\begin{pgfscope}%
\pgfsetbuttcap%
\pgfsetroundjoin%
\definecolor{currentfill}{rgb}{0.000000,0.000000,0.000000}%
\pgfsetfillcolor{currentfill}%
\pgfsetlinewidth{0.602250pt}%
\definecolor{currentstroke}{rgb}{0.000000,0.000000,0.000000}%
\pgfsetstrokecolor{currentstroke}%
\pgfsetdash{}{0pt}%
\pgfsys@defobject{currentmarker}{\pgfqpoint{-0.027778in}{0.000000in}}{\pgfqpoint{-0.000000in}{0.000000in}}{%
\pgfpathmoveto{\pgfqpoint{-0.000000in}{0.000000in}}%
\pgfpathlineto{\pgfqpoint{-0.027778in}{0.000000in}}%
\pgfusepath{stroke,fill}%
}%
\begin{pgfscope}%
\pgfsys@transformshift{0.708220in}{4.238904in}%
\pgfsys@useobject{currentmarker}{}%
\end{pgfscope}%
\end{pgfscope}%
\begin{pgfscope}%
\pgfsetbuttcap%
\pgfsetroundjoin%
\definecolor{currentfill}{rgb}{0.000000,0.000000,0.000000}%
\pgfsetfillcolor{currentfill}%
\pgfsetlinewidth{0.602250pt}%
\definecolor{currentstroke}{rgb}{0.000000,0.000000,0.000000}%
\pgfsetstrokecolor{currentstroke}%
\pgfsetdash{}{0pt}%
\pgfsys@defobject{currentmarker}{\pgfqpoint{-0.027778in}{0.000000in}}{\pgfqpoint{-0.000000in}{0.000000in}}{%
\pgfpathmoveto{\pgfqpoint{-0.000000in}{0.000000in}}%
\pgfpathlineto{\pgfqpoint{-0.027778in}{0.000000in}}%
\pgfusepath{stroke,fill}%
}%
\begin{pgfscope}%
\pgfsys@transformshift{0.708220in}{4.308478in}%
\pgfsys@useobject{currentmarker}{}%
\end{pgfscope}%
\end{pgfscope}%
\begin{pgfscope}%
\pgfsetbuttcap%
\pgfsetroundjoin%
\definecolor{currentfill}{rgb}{0.000000,0.000000,0.000000}%
\pgfsetfillcolor{currentfill}%
\pgfsetlinewidth{0.602250pt}%
\definecolor{currentstroke}{rgb}{0.000000,0.000000,0.000000}%
\pgfsetstrokecolor{currentstroke}%
\pgfsetdash{}{0pt}%
\pgfsys@defobject{currentmarker}{\pgfqpoint{-0.027778in}{0.000000in}}{\pgfqpoint{-0.000000in}{0.000000in}}{%
\pgfpathmoveto{\pgfqpoint{-0.000000in}{0.000000in}}%
\pgfpathlineto{\pgfqpoint{-0.027778in}{0.000000in}}%
\pgfusepath{stroke,fill}%
}%
\begin{pgfscope}%
\pgfsys@transformshift{0.708220in}{4.357842in}%
\pgfsys@useobject{currentmarker}{}%
\end{pgfscope}%
\end{pgfscope}%
\begin{pgfscope}%
\pgfsetbuttcap%
\pgfsetroundjoin%
\definecolor{currentfill}{rgb}{0.000000,0.000000,0.000000}%
\pgfsetfillcolor{currentfill}%
\pgfsetlinewidth{0.602250pt}%
\definecolor{currentstroke}{rgb}{0.000000,0.000000,0.000000}%
\pgfsetstrokecolor{currentstroke}%
\pgfsetdash{}{0pt}%
\pgfsys@defobject{currentmarker}{\pgfqpoint{-0.027778in}{0.000000in}}{\pgfqpoint{-0.000000in}{0.000000in}}{%
\pgfpathmoveto{\pgfqpoint{-0.000000in}{0.000000in}}%
\pgfpathlineto{\pgfqpoint{-0.027778in}{0.000000in}}%
\pgfusepath{stroke,fill}%
}%
\begin{pgfscope}%
\pgfsys@transformshift{0.708220in}{4.396131in}%
\pgfsys@useobject{currentmarker}{}%
\end{pgfscope}%
\end{pgfscope}%
\begin{pgfscope}%
\pgfsetbuttcap%
\pgfsetroundjoin%
\definecolor{currentfill}{rgb}{0.000000,0.000000,0.000000}%
\pgfsetfillcolor{currentfill}%
\pgfsetlinewidth{0.602250pt}%
\definecolor{currentstroke}{rgb}{0.000000,0.000000,0.000000}%
\pgfsetstrokecolor{currentstroke}%
\pgfsetdash{}{0pt}%
\pgfsys@defobject{currentmarker}{\pgfqpoint{-0.027778in}{0.000000in}}{\pgfqpoint{-0.000000in}{0.000000in}}{%
\pgfpathmoveto{\pgfqpoint{-0.000000in}{0.000000in}}%
\pgfpathlineto{\pgfqpoint{-0.027778in}{0.000000in}}%
\pgfusepath{stroke,fill}%
}%
\begin{pgfscope}%
\pgfsys@transformshift{0.708220in}{4.427416in}%
\pgfsys@useobject{currentmarker}{}%
\end{pgfscope}%
\end{pgfscope}%
\begin{pgfscope}%
\pgfsetbuttcap%
\pgfsetroundjoin%
\definecolor{currentfill}{rgb}{0.000000,0.000000,0.000000}%
\pgfsetfillcolor{currentfill}%
\pgfsetlinewidth{0.602250pt}%
\definecolor{currentstroke}{rgb}{0.000000,0.000000,0.000000}%
\pgfsetstrokecolor{currentstroke}%
\pgfsetdash{}{0pt}%
\pgfsys@defobject{currentmarker}{\pgfqpoint{-0.027778in}{0.000000in}}{\pgfqpoint{-0.000000in}{0.000000in}}{%
\pgfpathmoveto{\pgfqpoint{-0.000000in}{0.000000in}}%
\pgfpathlineto{\pgfqpoint{-0.027778in}{0.000000in}}%
\pgfusepath{stroke,fill}%
}%
\begin{pgfscope}%
\pgfsys@transformshift{0.708220in}{4.453867in}%
\pgfsys@useobject{currentmarker}{}%
\end{pgfscope}%
\end{pgfscope}%
\begin{pgfscope}%
\pgfsetbuttcap%
\pgfsetroundjoin%
\definecolor{currentfill}{rgb}{0.000000,0.000000,0.000000}%
\pgfsetfillcolor{currentfill}%
\pgfsetlinewidth{0.602250pt}%
\definecolor{currentstroke}{rgb}{0.000000,0.000000,0.000000}%
\pgfsetstrokecolor{currentstroke}%
\pgfsetdash{}{0pt}%
\pgfsys@defobject{currentmarker}{\pgfqpoint{-0.027778in}{0.000000in}}{\pgfqpoint{-0.000000in}{0.000000in}}{%
\pgfpathmoveto{\pgfqpoint{-0.000000in}{0.000000in}}%
\pgfpathlineto{\pgfqpoint{-0.027778in}{0.000000in}}%
\pgfusepath{stroke,fill}%
}%
\begin{pgfscope}%
\pgfsys@transformshift{0.708220in}{4.476780in}%
\pgfsys@useobject{currentmarker}{}%
\end{pgfscope}%
\end{pgfscope}%
\begin{pgfscope}%
\pgfsetbuttcap%
\pgfsetroundjoin%
\definecolor{currentfill}{rgb}{0.000000,0.000000,0.000000}%
\pgfsetfillcolor{currentfill}%
\pgfsetlinewidth{0.602250pt}%
\definecolor{currentstroke}{rgb}{0.000000,0.000000,0.000000}%
\pgfsetstrokecolor{currentstroke}%
\pgfsetdash{}{0pt}%
\pgfsys@defobject{currentmarker}{\pgfqpoint{-0.027778in}{0.000000in}}{\pgfqpoint{-0.000000in}{0.000000in}}{%
\pgfpathmoveto{\pgfqpoint{-0.000000in}{0.000000in}}%
\pgfpathlineto{\pgfqpoint{-0.027778in}{0.000000in}}%
\pgfusepath{stroke,fill}%
}%
\begin{pgfscope}%
\pgfsys@transformshift{0.708220in}{4.496990in}%
\pgfsys@useobject{currentmarker}{}%
\end{pgfscope}%
\end{pgfscope}%
\begin{pgfscope}%
\pgfsetbuttcap%
\pgfsetroundjoin%
\definecolor{currentfill}{rgb}{0.000000,0.000000,0.000000}%
\pgfsetfillcolor{currentfill}%
\pgfsetlinewidth{0.602250pt}%
\definecolor{currentstroke}{rgb}{0.000000,0.000000,0.000000}%
\pgfsetstrokecolor{currentstroke}%
\pgfsetdash{}{0pt}%
\pgfsys@defobject{currentmarker}{\pgfqpoint{-0.027778in}{0.000000in}}{\pgfqpoint{-0.000000in}{0.000000in}}{%
\pgfpathmoveto{\pgfqpoint{-0.000000in}{0.000000in}}%
\pgfpathlineto{\pgfqpoint{-0.027778in}{0.000000in}}%
\pgfusepath{stroke,fill}%
}%
\begin{pgfscope}%
\pgfsys@transformshift{0.708220in}{4.634007in}%
\pgfsys@useobject{currentmarker}{}%
\end{pgfscope}%
\end{pgfscope}%
\begin{pgfscope}%
\pgfsetbuttcap%
\pgfsetroundjoin%
\definecolor{currentfill}{rgb}{0.000000,0.000000,0.000000}%
\pgfsetfillcolor{currentfill}%
\pgfsetlinewidth{0.602250pt}%
\definecolor{currentstroke}{rgb}{0.000000,0.000000,0.000000}%
\pgfsetstrokecolor{currentstroke}%
\pgfsetdash{}{0pt}%
\pgfsys@defobject{currentmarker}{\pgfqpoint{-0.027778in}{0.000000in}}{\pgfqpoint{-0.000000in}{0.000000in}}{%
\pgfpathmoveto{\pgfqpoint{-0.000000in}{0.000000in}}%
\pgfpathlineto{\pgfqpoint{-0.027778in}{0.000000in}}%
\pgfusepath{stroke,fill}%
}%
\begin{pgfscope}%
\pgfsys@transformshift{0.708220in}{4.703581in}%
\pgfsys@useobject{currentmarker}{}%
\end{pgfscope}%
\end{pgfscope}%
\begin{pgfscope}%
\pgfsetbuttcap%
\pgfsetroundjoin%
\definecolor{currentfill}{rgb}{0.000000,0.000000,0.000000}%
\pgfsetfillcolor{currentfill}%
\pgfsetlinewidth{0.602250pt}%
\definecolor{currentstroke}{rgb}{0.000000,0.000000,0.000000}%
\pgfsetstrokecolor{currentstroke}%
\pgfsetdash{}{0pt}%
\pgfsys@defobject{currentmarker}{\pgfqpoint{-0.027778in}{0.000000in}}{\pgfqpoint{-0.000000in}{0.000000in}}{%
\pgfpathmoveto{\pgfqpoint{-0.000000in}{0.000000in}}%
\pgfpathlineto{\pgfqpoint{-0.027778in}{0.000000in}}%
\pgfusepath{stroke,fill}%
}%
\begin{pgfscope}%
\pgfsys@transformshift{0.708220in}{4.752945in}%
\pgfsys@useobject{currentmarker}{}%
\end{pgfscope}%
\end{pgfscope}%
\begin{pgfscope}%
\pgfsetbuttcap%
\pgfsetroundjoin%
\definecolor{currentfill}{rgb}{0.000000,0.000000,0.000000}%
\pgfsetfillcolor{currentfill}%
\pgfsetlinewidth{0.602250pt}%
\definecolor{currentstroke}{rgb}{0.000000,0.000000,0.000000}%
\pgfsetstrokecolor{currentstroke}%
\pgfsetdash{}{0pt}%
\pgfsys@defobject{currentmarker}{\pgfqpoint{-0.027778in}{0.000000in}}{\pgfqpoint{-0.000000in}{0.000000in}}{%
\pgfpathmoveto{\pgfqpoint{-0.000000in}{0.000000in}}%
\pgfpathlineto{\pgfqpoint{-0.027778in}{0.000000in}}%
\pgfusepath{stroke,fill}%
}%
\begin{pgfscope}%
\pgfsys@transformshift{0.708220in}{4.791234in}%
\pgfsys@useobject{currentmarker}{}%
\end{pgfscope}%
\end{pgfscope}%
\begin{pgfscope}%
\pgfsetbuttcap%
\pgfsetroundjoin%
\definecolor{currentfill}{rgb}{0.000000,0.000000,0.000000}%
\pgfsetfillcolor{currentfill}%
\pgfsetlinewidth{0.602250pt}%
\definecolor{currentstroke}{rgb}{0.000000,0.000000,0.000000}%
\pgfsetstrokecolor{currentstroke}%
\pgfsetdash{}{0pt}%
\pgfsys@defobject{currentmarker}{\pgfqpoint{-0.027778in}{0.000000in}}{\pgfqpoint{-0.000000in}{0.000000in}}{%
\pgfpathmoveto{\pgfqpoint{-0.000000in}{0.000000in}}%
\pgfpathlineto{\pgfqpoint{-0.027778in}{0.000000in}}%
\pgfusepath{stroke,fill}%
}%
\begin{pgfscope}%
\pgfsys@transformshift{0.708220in}{4.822519in}%
\pgfsys@useobject{currentmarker}{}%
\end{pgfscope}%
\end{pgfscope}%
\begin{pgfscope}%
\pgfsetbuttcap%
\pgfsetroundjoin%
\definecolor{currentfill}{rgb}{0.000000,0.000000,0.000000}%
\pgfsetfillcolor{currentfill}%
\pgfsetlinewidth{0.602250pt}%
\definecolor{currentstroke}{rgb}{0.000000,0.000000,0.000000}%
\pgfsetstrokecolor{currentstroke}%
\pgfsetdash{}{0pt}%
\pgfsys@defobject{currentmarker}{\pgfqpoint{-0.027778in}{0.000000in}}{\pgfqpoint{-0.000000in}{0.000000in}}{%
\pgfpathmoveto{\pgfqpoint{-0.000000in}{0.000000in}}%
\pgfpathlineto{\pgfqpoint{-0.027778in}{0.000000in}}%
\pgfusepath{stroke,fill}%
}%
\begin{pgfscope}%
\pgfsys@transformshift{0.708220in}{4.848970in}%
\pgfsys@useobject{currentmarker}{}%
\end{pgfscope}%
\end{pgfscope}%
\begin{pgfscope}%
\pgfsetbuttcap%
\pgfsetroundjoin%
\definecolor{currentfill}{rgb}{0.000000,0.000000,0.000000}%
\pgfsetfillcolor{currentfill}%
\pgfsetlinewidth{0.602250pt}%
\definecolor{currentstroke}{rgb}{0.000000,0.000000,0.000000}%
\pgfsetstrokecolor{currentstroke}%
\pgfsetdash{}{0pt}%
\pgfsys@defobject{currentmarker}{\pgfqpoint{-0.027778in}{0.000000in}}{\pgfqpoint{-0.000000in}{0.000000in}}{%
\pgfpathmoveto{\pgfqpoint{-0.000000in}{0.000000in}}%
\pgfpathlineto{\pgfqpoint{-0.027778in}{0.000000in}}%
\pgfusepath{stroke,fill}%
}%
\begin{pgfscope}%
\pgfsys@transformshift{0.708220in}{4.871883in}%
\pgfsys@useobject{currentmarker}{}%
\end{pgfscope}%
\end{pgfscope}%
\begin{pgfscope}%
\pgfsetbuttcap%
\pgfsetroundjoin%
\definecolor{currentfill}{rgb}{0.000000,0.000000,0.000000}%
\pgfsetfillcolor{currentfill}%
\pgfsetlinewidth{0.602250pt}%
\definecolor{currentstroke}{rgb}{0.000000,0.000000,0.000000}%
\pgfsetstrokecolor{currentstroke}%
\pgfsetdash{}{0pt}%
\pgfsys@defobject{currentmarker}{\pgfqpoint{-0.027778in}{0.000000in}}{\pgfqpoint{-0.000000in}{0.000000in}}{%
\pgfpathmoveto{\pgfqpoint{-0.000000in}{0.000000in}}%
\pgfpathlineto{\pgfqpoint{-0.027778in}{0.000000in}}%
\pgfusepath{stroke,fill}%
}%
\begin{pgfscope}%
\pgfsys@transformshift{0.708220in}{4.892093in}%
\pgfsys@useobject{currentmarker}{}%
\end{pgfscope}%
\end{pgfscope}%
\begin{pgfscope}%
\pgfsetbuttcap%
\pgfsetroundjoin%
\definecolor{currentfill}{rgb}{0.000000,0.000000,0.000000}%
\pgfsetfillcolor{currentfill}%
\pgfsetlinewidth{0.602250pt}%
\definecolor{currentstroke}{rgb}{0.000000,0.000000,0.000000}%
\pgfsetstrokecolor{currentstroke}%
\pgfsetdash{}{0pt}%
\pgfsys@defobject{currentmarker}{\pgfqpoint{-0.027778in}{0.000000in}}{\pgfqpoint{-0.000000in}{0.000000in}}{%
\pgfpathmoveto{\pgfqpoint{-0.000000in}{0.000000in}}%
\pgfpathlineto{\pgfqpoint{-0.027778in}{0.000000in}}%
\pgfusepath{stroke,fill}%
}%
\begin{pgfscope}%
\pgfsys@transformshift{0.708220in}{5.029110in}%
\pgfsys@useobject{currentmarker}{}%
\end{pgfscope}%
\end{pgfscope}%
\begin{pgfscope}%
\pgfsetbuttcap%
\pgfsetroundjoin%
\definecolor{currentfill}{rgb}{0.000000,0.000000,0.000000}%
\pgfsetfillcolor{currentfill}%
\pgfsetlinewidth{0.602250pt}%
\definecolor{currentstroke}{rgb}{0.000000,0.000000,0.000000}%
\pgfsetstrokecolor{currentstroke}%
\pgfsetdash{}{0pt}%
\pgfsys@defobject{currentmarker}{\pgfqpoint{-0.027778in}{0.000000in}}{\pgfqpoint{-0.000000in}{0.000000in}}{%
\pgfpathmoveto{\pgfqpoint{-0.000000in}{0.000000in}}%
\pgfpathlineto{\pgfqpoint{-0.027778in}{0.000000in}}%
\pgfusepath{stroke,fill}%
}%
\begin{pgfscope}%
\pgfsys@transformshift{0.708220in}{5.098684in}%
\pgfsys@useobject{currentmarker}{}%
\end{pgfscope}%
\end{pgfscope}%
\begin{pgfscope}%
\pgfsetbuttcap%
\pgfsetroundjoin%
\definecolor{currentfill}{rgb}{0.000000,0.000000,0.000000}%
\pgfsetfillcolor{currentfill}%
\pgfsetlinewidth{0.602250pt}%
\definecolor{currentstroke}{rgb}{0.000000,0.000000,0.000000}%
\pgfsetstrokecolor{currentstroke}%
\pgfsetdash{}{0pt}%
\pgfsys@defobject{currentmarker}{\pgfqpoint{-0.027778in}{0.000000in}}{\pgfqpoint{-0.000000in}{0.000000in}}{%
\pgfpathmoveto{\pgfqpoint{-0.000000in}{0.000000in}}%
\pgfpathlineto{\pgfqpoint{-0.027778in}{0.000000in}}%
\pgfusepath{stroke,fill}%
}%
\begin{pgfscope}%
\pgfsys@transformshift{0.708220in}{5.148048in}%
\pgfsys@useobject{currentmarker}{}%
\end{pgfscope}%
\end{pgfscope}%
\begin{pgfscope}%
\pgfsetbuttcap%
\pgfsetroundjoin%
\definecolor{currentfill}{rgb}{0.000000,0.000000,0.000000}%
\pgfsetfillcolor{currentfill}%
\pgfsetlinewidth{0.602250pt}%
\definecolor{currentstroke}{rgb}{0.000000,0.000000,0.000000}%
\pgfsetstrokecolor{currentstroke}%
\pgfsetdash{}{0pt}%
\pgfsys@defobject{currentmarker}{\pgfqpoint{-0.027778in}{0.000000in}}{\pgfqpoint{-0.000000in}{0.000000in}}{%
\pgfpathmoveto{\pgfqpoint{-0.000000in}{0.000000in}}%
\pgfpathlineto{\pgfqpoint{-0.027778in}{0.000000in}}%
\pgfusepath{stroke,fill}%
}%
\begin{pgfscope}%
\pgfsys@transformshift{0.708220in}{5.186337in}%
\pgfsys@useobject{currentmarker}{}%
\end{pgfscope}%
\end{pgfscope}%
\begin{pgfscope}%
\pgfsetbuttcap%
\pgfsetroundjoin%
\definecolor{currentfill}{rgb}{0.000000,0.000000,0.000000}%
\pgfsetfillcolor{currentfill}%
\pgfsetlinewidth{0.602250pt}%
\definecolor{currentstroke}{rgb}{0.000000,0.000000,0.000000}%
\pgfsetstrokecolor{currentstroke}%
\pgfsetdash{}{0pt}%
\pgfsys@defobject{currentmarker}{\pgfqpoint{-0.027778in}{0.000000in}}{\pgfqpoint{-0.000000in}{0.000000in}}{%
\pgfpathmoveto{\pgfqpoint{-0.000000in}{0.000000in}}%
\pgfpathlineto{\pgfqpoint{-0.027778in}{0.000000in}}%
\pgfusepath{stroke,fill}%
}%
\begin{pgfscope}%
\pgfsys@transformshift{0.708220in}{5.217622in}%
\pgfsys@useobject{currentmarker}{}%
\end{pgfscope}%
\end{pgfscope}%
\begin{pgfscope}%
\pgfsetbuttcap%
\pgfsetroundjoin%
\definecolor{currentfill}{rgb}{0.000000,0.000000,0.000000}%
\pgfsetfillcolor{currentfill}%
\pgfsetlinewidth{0.602250pt}%
\definecolor{currentstroke}{rgb}{0.000000,0.000000,0.000000}%
\pgfsetstrokecolor{currentstroke}%
\pgfsetdash{}{0pt}%
\pgfsys@defobject{currentmarker}{\pgfqpoint{-0.027778in}{0.000000in}}{\pgfqpoint{-0.000000in}{0.000000in}}{%
\pgfpathmoveto{\pgfqpoint{-0.000000in}{0.000000in}}%
\pgfpathlineto{\pgfqpoint{-0.027778in}{0.000000in}}%
\pgfusepath{stroke,fill}%
}%
\begin{pgfscope}%
\pgfsys@transformshift{0.708220in}{5.244073in}%
\pgfsys@useobject{currentmarker}{}%
\end{pgfscope}%
\end{pgfscope}%
\begin{pgfscope}%
\pgfsetbuttcap%
\pgfsetroundjoin%
\definecolor{currentfill}{rgb}{0.000000,0.000000,0.000000}%
\pgfsetfillcolor{currentfill}%
\pgfsetlinewidth{0.602250pt}%
\definecolor{currentstroke}{rgb}{0.000000,0.000000,0.000000}%
\pgfsetstrokecolor{currentstroke}%
\pgfsetdash{}{0pt}%
\pgfsys@defobject{currentmarker}{\pgfqpoint{-0.027778in}{0.000000in}}{\pgfqpoint{-0.000000in}{0.000000in}}{%
\pgfpathmoveto{\pgfqpoint{-0.000000in}{0.000000in}}%
\pgfpathlineto{\pgfqpoint{-0.027778in}{0.000000in}}%
\pgfusepath{stroke,fill}%
}%
\begin{pgfscope}%
\pgfsys@transformshift{0.708220in}{5.266986in}%
\pgfsys@useobject{currentmarker}{}%
\end{pgfscope}%
\end{pgfscope}%
\begin{pgfscope}%
\pgfsetbuttcap%
\pgfsetroundjoin%
\definecolor{currentfill}{rgb}{0.000000,0.000000,0.000000}%
\pgfsetfillcolor{currentfill}%
\pgfsetlinewidth{0.602250pt}%
\definecolor{currentstroke}{rgb}{0.000000,0.000000,0.000000}%
\pgfsetstrokecolor{currentstroke}%
\pgfsetdash{}{0pt}%
\pgfsys@defobject{currentmarker}{\pgfqpoint{-0.027778in}{0.000000in}}{\pgfqpoint{-0.000000in}{0.000000in}}{%
\pgfpathmoveto{\pgfqpoint{-0.000000in}{0.000000in}}%
\pgfpathlineto{\pgfqpoint{-0.027778in}{0.000000in}}%
\pgfusepath{stroke,fill}%
}%
\begin{pgfscope}%
\pgfsys@transformshift{0.708220in}{5.287196in}%
\pgfsys@useobject{currentmarker}{}%
\end{pgfscope}%
\end{pgfscope}%
\begin{pgfscope}%
\definecolor{textcolor}{rgb}{0.000000,0.000000,0.000000}%
\pgfsetstrokecolor{textcolor}%
\pgfsetfillcolor{textcolor}%
\pgftext[x=0.288855in,y=4.258049in,,bottom,rotate=90.000000]{\color{textcolor}\rmfamily\fontsize{10.000000}{12.000000}\selectfont Longest solving time (s)}%
\end{pgfscope}%
\begin{pgfscope}%
\pgfpathrectangle{\pgfqpoint{0.708220in}{3.210823in}}{\pgfqpoint{5.013309in}{2.094453in}}%
\pgfusepath{clip}%
\pgfsetrectcap%
\pgfsetroundjoin%
\pgfsetlinewidth{1.003750pt}%
\definecolor{currentstroke}{rgb}{0.878431,0.878431,0.815686}%
\pgfsetstrokecolor{currentstroke}%
\pgfsetdash{}{0pt}%
\pgfpathmoveto{\pgfqpoint{0.708220in}{3.903600in}}%
\pgfpathlineto{\pgfqpoint{0.710727in}{3.909932in}}%
\pgfpathlineto{\pgfqpoint{0.713233in}{3.911791in}}%
\pgfpathlineto{\pgfqpoint{0.715740in}{3.918896in}}%
\pgfpathlineto{\pgfqpoint{0.718246in}{3.921274in}}%
\pgfpathlineto{\pgfqpoint{0.723260in}{3.930842in}}%
\pgfpathlineto{\pgfqpoint{0.725766in}{3.933218in}}%
\pgfpathlineto{\pgfqpoint{0.728273in}{3.933485in}}%
\pgfpathlineto{\pgfqpoint{0.730780in}{3.936293in}}%
\pgfpathlineto{\pgfqpoint{0.740806in}{3.940309in}}%
\pgfpathlineto{\pgfqpoint{0.745820in}{3.942951in}}%
\pgfpathlineto{\pgfqpoint{0.748326in}{3.943148in}}%
\pgfpathlineto{\pgfqpoint{0.755846in}{3.948494in}}%
\pgfpathlineto{\pgfqpoint{0.758353in}{3.958125in}}%
\pgfpathlineto{\pgfqpoint{0.763366in}{3.959435in}}%
\pgfpathlineto{\pgfqpoint{0.765873in}{3.965391in}}%
\pgfpathlineto{\pgfqpoint{0.768380in}{3.966111in}}%
\pgfpathlineto{\pgfqpoint{0.770886in}{3.983793in}}%
\pgfpathlineto{\pgfqpoint{0.773393in}{3.991844in}}%
\pgfpathlineto{\pgfqpoint{0.775900in}{3.994728in}}%
\pgfpathlineto{\pgfqpoint{0.778406in}{3.995437in}}%
\pgfpathlineto{\pgfqpoint{0.783419in}{4.016036in}}%
\pgfpathlineto{\pgfqpoint{0.788433in}{4.018756in}}%
\pgfpathlineto{\pgfqpoint{0.790939in}{4.023728in}}%
\pgfpathlineto{\pgfqpoint{0.793446in}{4.023954in}}%
\pgfpathlineto{\pgfqpoint{0.795953in}{4.028716in}}%
\pgfpathlineto{\pgfqpoint{0.800966in}{4.029975in}}%
\pgfpathlineto{\pgfqpoint{0.803473in}{4.041154in}}%
\pgfpathlineto{\pgfqpoint{0.805979in}{4.044958in}}%
\pgfpathlineto{\pgfqpoint{0.808486in}{4.045612in}}%
\pgfpathlineto{\pgfqpoint{0.810993in}{4.051201in}}%
\pgfpathlineto{\pgfqpoint{0.821019in}{4.056890in}}%
\pgfpathlineto{\pgfqpoint{0.823526in}{4.057526in}}%
\pgfpathlineto{\pgfqpoint{0.828539in}{4.066083in}}%
\pgfpathlineto{\pgfqpoint{0.831046in}{4.073228in}}%
\pgfpathlineto{\pgfqpoint{0.838566in}{4.078775in}}%
\pgfpathlineto{\pgfqpoint{0.841073in}{4.092190in}}%
\pgfpathlineto{\pgfqpoint{0.843579in}{4.092497in}}%
\pgfpathlineto{\pgfqpoint{0.856112in}{4.102350in}}%
\pgfpathlineto{\pgfqpoint{0.858619in}{4.102399in}}%
\pgfpathlineto{\pgfqpoint{0.861126in}{4.107738in}}%
\pgfpathlineto{\pgfqpoint{0.863632in}{4.109721in}}%
\pgfpathlineto{\pgfqpoint{0.868646in}{4.110608in}}%
\pgfpathlineto{\pgfqpoint{0.876166in}{4.117645in}}%
\pgfpathlineto{\pgfqpoint{0.878672in}{4.118670in}}%
\pgfpathlineto{\pgfqpoint{0.881179in}{4.121365in}}%
\pgfpathlineto{\pgfqpoint{0.883686in}{4.125695in}}%
\pgfpathlineto{\pgfqpoint{0.888699in}{4.129932in}}%
\pgfpathlineto{\pgfqpoint{0.898726in}{4.131136in}}%
\pgfpathlineto{\pgfqpoint{0.901232in}{4.136463in}}%
\pgfpathlineto{\pgfqpoint{0.903739in}{4.137136in}}%
\pgfpathlineto{\pgfqpoint{0.906246in}{4.140528in}}%
\pgfpathlineto{\pgfqpoint{0.913766in}{4.144766in}}%
\pgfpathlineto{\pgfqpoint{0.916272in}{4.148250in}}%
\pgfpathlineto{\pgfqpoint{0.923792in}{4.151644in}}%
\pgfpathlineto{\pgfqpoint{0.928805in}{4.156243in}}%
\pgfpathlineto{\pgfqpoint{0.933819in}{4.158121in}}%
\pgfpathlineto{\pgfqpoint{0.936325in}{4.158470in}}%
\pgfpathlineto{\pgfqpoint{0.938832in}{4.162035in}}%
\pgfpathlineto{\pgfqpoint{0.951365in}{4.166809in}}%
\pgfpathlineto{\pgfqpoint{0.956379in}{4.168865in}}%
\pgfpathlineto{\pgfqpoint{0.961392in}{4.170947in}}%
\pgfpathlineto{\pgfqpoint{0.966405in}{4.172261in}}%
\pgfpathlineto{\pgfqpoint{0.971419in}{4.174001in}}%
\pgfpathlineto{\pgfqpoint{0.976432in}{4.178966in}}%
\pgfpathlineto{\pgfqpoint{0.981445in}{4.180255in}}%
\pgfpathlineto{\pgfqpoint{0.983952in}{4.180267in}}%
\pgfpathlineto{\pgfqpoint{0.986458in}{4.181873in}}%
\pgfpathlineto{\pgfqpoint{0.991472in}{4.182356in}}%
\pgfpathlineto{\pgfqpoint{0.993978in}{4.188103in}}%
\pgfpathlineto{\pgfqpoint{1.001498in}{4.190522in}}%
\pgfpathlineto{\pgfqpoint{1.006512in}{4.191334in}}%
\pgfpathlineto{\pgfqpoint{1.009018in}{4.194860in}}%
\pgfpathlineto{\pgfqpoint{1.014032in}{4.196179in}}%
\pgfpathlineto{\pgfqpoint{1.021552in}{4.198282in}}%
\pgfpathlineto{\pgfqpoint{1.024058in}{4.201120in}}%
\pgfpathlineto{\pgfqpoint{1.026565in}{4.201131in}}%
\pgfpathlineto{\pgfqpoint{1.029072in}{4.205643in}}%
\pgfpathlineto{\pgfqpoint{1.031578in}{4.207564in}}%
\pgfpathlineto{\pgfqpoint{1.044112in}{4.210016in}}%
\pgfpathlineto{\pgfqpoint{1.064165in}{4.216360in}}%
\pgfpathlineto{\pgfqpoint{1.069178in}{4.218113in}}%
\pgfpathlineto{\pgfqpoint{1.076698in}{4.222088in}}%
\pgfpathlineto{\pgfqpoint{1.081711in}{4.223914in}}%
\pgfpathlineto{\pgfqpoint{1.091738in}{4.226458in}}%
\pgfpathlineto{\pgfqpoint{1.099258in}{4.233362in}}%
\pgfpathlineto{\pgfqpoint{1.109285in}{4.237828in}}%
\pgfpathlineto{\pgfqpoint{1.114298in}{4.243898in}}%
\pgfpathlineto{\pgfqpoint{1.119311in}{4.245047in}}%
\pgfpathlineto{\pgfqpoint{1.121818in}{4.245347in}}%
\pgfpathlineto{\pgfqpoint{1.126831in}{4.247184in}}%
\pgfpathlineto{\pgfqpoint{1.161924in}{4.255824in}}%
\pgfpathlineto{\pgfqpoint{1.166938in}{4.261415in}}%
\pgfpathlineto{\pgfqpoint{1.169444in}{4.262126in}}%
\pgfpathlineto{\pgfqpoint{1.174458in}{4.266408in}}%
\pgfpathlineto{\pgfqpoint{1.179471in}{4.267044in}}%
\pgfpathlineto{\pgfqpoint{1.184484in}{4.272452in}}%
\pgfpathlineto{\pgfqpoint{1.186991in}{4.274948in}}%
\pgfpathlineto{\pgfqpoint{1.192004in}{4.276213in}}%
\pgfpathlineto{\pgfqpoint{1.194511in}{4.281324in}}%
\pgfpathlineto{\pgfqpoint{1.204537in}{4.284649in}}%
\pgfpathlineto{\pgfqpoint{1.212057in}{4.289240in}}%
\pgfpathlineto{\pgfqpoint{1.214564in}{4.289430in}}%
\pgfpathlineto{\pgfqpoint{1.217071in}{4.292542in}}%
\pgfpathlineto{\pgfqpoint{1.219577in}{4.292776in}}%
\pgfpathlineto{\pgfqpoint{1.222084in}{4.297043in}}%
\pgfpathlineto{\pgfqpoint{1.229604in}{4.298301in}}%
\pgfpathlineto{\pgfqpoint{1.232111in}{4.298590in}}%
\pgfpathlineto{\pgfqpoint{1.237124in}{4.302811in}}%
\pgfpathlineto{\pgfqpoint{1.244644in}{4.307725in}}%
\pgfpathlineto{\pgfqpoint{1.247151in}{4.308700in}}%
\pgfpathlineto{\pgfqpoint{1.249657in}{4.313925in}}%
\pgfpathlineto{\pgfqpoint{1.252164in}{4.323242in}}%
\pgfpathlineto{\pgfqpoint{1.264697in}{4.336360in}}%
\pgfpathlineto{\pgfqpoint{1.267204in}{4.336572in}}%
\pgfpathlineto{\pgfqpoint{1.292270in}{4.349911in}}%
\pgfpathlineto{\pgfqpoint{1.297284in}{4.350137in}}%
\pgfpathlineto{\pgfqpoint{1.302297in}{4.352204in}}%
\pgfpathlineto{\pgfqpoint{1.307310in}{4.352622in}}%
\pgfpathlineto{\pgfqpoint{1.314830in}{4.354382in}}%
\pgfpathlineto{\pgfqpoint{1.324857in}{4.355796in}}%
\pgfpathlineto{\pgfqpoint{1.327363in}{4.357765in}}%
\pgfpathlineto{\pgfqpoint{1.339897in}{4.361421in}}%
\pgfpathlineto{\pgfqpoint{1.344910in}{4.365463in}}%
\pgfpathlineto{\pgfqpoint{1.352430in}{4.370660in}}%
\pgfpathlineto{\pgfqpoint{1.359950in}{4.373645in}}%
\pgfpathlineto{\pgfqpoint{1.372483in}{4.380462in}}%
\pgfpathlineto{\pgfqpoint{1.377497in}{4.380729in}}%
\pgfpathlineto{\pgfqpoint{1.382510in}{4.383044in}}%
\pgfpathlineto{\pgfqpoint{1.392537in}{4.385479in}}%
\pgfpathlineto{\pgfqpoint{1.395043in}{4.392543in}}%
\pgfpathlineto{\pgfqpoint{1.425123in}{4.405254in}}%
\pgfpathlineto{\pgfqpoint{1.430136in}{4.405520in}}%
\pgfpathlineto{\pgfqpoint{1.432643in}{4.408722in}}%
\pgfpathlineto{\pgfqpoint{1.440163in}{4.409629in}}%
\pgfpathlineto{\pgfqpoint{1.445176in}{4.410920in}}%
\pgfpathlineto{\pgfqpoint{1.450190in}{4.411755in}}%
\pgfpathlineto{\pgfqpoint{1.452696in}{4.412917in}}%
\pgfpathlineto{\pgfqpoint{1.455203in}{4.415629in}}%
\pgfpathlineto{\pgfqpoint{1.467736in}{4.419681in}}%
\pgfpathlineto{\pgfqpoint{1.470243in}{4.421465in}}%
\pgfpathlineto{\pgfqpoint{1.482776in}{4.423815in}}%
\pgfpathlineto{\pgfqpoint{1.487789in}{4.425192in}}%
\pgfpathlineto{\pgfqpoint{1.512856in}{4.428704in}}%
\pgfpathlineto{\pgfqpoint{1.517869in}{4.430663in}}%
\pgfpathlineto{\pgfqpoint{1.527896in}{4.436160in}}%
\pgfpathlineto{\pgfqpoint{1.532909in}{4.437158in}}%
\pgfpathlineto{\pgfqpoint{1.542936in}{4.438329in}}%
\pgfpathlineto{\pgfqpoint{1.545442in}{4.440794in}}%
\pgfpathlineto{\pgfqpoint{1.557976in}{4.443364in}}%
\pgfpathlineto{\pgfqpoint{1.565496in}{4.444730in}}%
\pgfpathlineto{\pgfqpoint{1.573016in}{4.445666in}}%
\pgfpathlineto{\pgfqpoint{1.575522in}{4.447893in}}%
\pgfpathlineto{\pgfqpoint{1.583042in}{4.448411in}}%
\pgfpathlineto{\pgfqpoint{1.585549in}{4.450843in}}%
\pgfpathlineto{\pgfqpoint{1.600589in}{4.453218in}}%
\pgfpathlineto{\pgfqpoint{1.608109in}{4.453802in}}%
\pgfpathlineto{\pgfqpoint{1.615629in}{4.455933in}}%
\pgfpathlineto{\pgfqpoint{1.620642in}{4.456716in}}%
\pgfpathlineto{\pgfqpoint{1.633175in}{4.459478in}}%
\pgfpathlineto{\pgfqpoint{1.640695in}{4.460245in}}%
\pgfpathlineto{\pgfqpoint{1.645709in}{4.460572in}}%
\pgfpathlineto{\pgfqpoint{1.650722in}{4.464041in}}%
\pgfpathlineto{\pgfqpoint{1.660749in}{4.466988in}}%
\pgfpathlineto{\pgfqpoint{1.675788in}{4.474055in}}%
\pgfpathlineto{\pgfqpoint{1.678295in}{4.474288in}}%
\pgfpathlineto{\pgfqpoint{1.683308in}{4.477391in}}%
\pgfpathlineto{\pgfqpoint{1.685815in}{4.477440in}}%
\pgfpathlineto{\pgfqpoint{1.690828in}{4.479709in}}%
\pgfpathlineto{\pgfqpoint{1.693335in}{4.480372in}}%
\pgfpathlineto{\pgfqpoint{1.698348in}{4.483094in}}%
\pgfpathlineto{\pgfqpoint{1.703362in}{4.485514in}}%
\pgfpathlineto{\pgfqpoint{1.705868in}{4.490734in}}%
\pgfpathlineto{\pgfqpoint{1.708375in}{4.492816in}}%
\pgfpathlineto{\pgfqpoint{1.713388in}{4.499823in}}%
\pgfpathlineto{\pgfqpoint{1.720908in}{4.504444in}}%
\pgfpathlineto{\pgfqpoint{1.723415in}{4.504586in}}%
\pgfpathlineto{\pgfqpoint{1.728428in}{4.508977in}}%
\pgfpathlineto{\pgfqpoint{1.733442in}{4.509632in}}%
\pgfpathlineto{\pgfqpoint{1.743468in}{4.515151in}}%
\pgfpathlineto{\pgfqpoint{1.753495in}{4.518387in}}%
\pgfpathlineto{\pgfqpoint{1.758508in}{4.520919in}}%
\pgfpathlineto{\pgfqpoint{1.761015in}{4.522337in}}%
\pgfpathlineto{\pgfqpoint{1.763521in}{4.522381in}}%
\pgfpathlineto{\pgfqpoint{1.768535in}{4.525065in}}%
\pgfpathlineto{\pgfqpoint{1.773548in}{4.525788in}}%
\pgfpathlineto{\pgfqpoint{1.776055in}{4.530631in}}%
\pgfpathlineto{\pgfqpoint{1.788588in}{4.534507in}}%
\pgfpathlineto{\pgfqpoint{1.791095in}{4.534771in}}%
\pgfpathlineto{\pgfqpoint{1.793601in}{4.536473in}}%
\pgfpathlineto{\pgfqpoint{1.806134in}{4.537902in}}%
\pgfpathlineto{\pgfqpoint{1.811148in}{4.541818in}}%
\pgfpathlineto{\pgfqpoint{1.816161in}{4.542658in}}%
\pgfpathlineto{\pgfqpoint{1.826188in}{4.553357in}}%
\pgfpathlineto{\pgfqpoint{1.833708in}{4.554715in}}%
\pgfpathlineto{\pgfqpoint{1.838721in}{4.555269in}}%
\pgfpathlineto{\pgfqpoint{1.841228in}{4.558286in}}%
\pgfpathlineto{\pgfqpoint{1.858774in}{4.561113in}}%
\pgfpathlineto{\pgfqpoint{1.866294in}{4.564937in}}%
\pgfpathlineto{\pgfqpoint{1.868801in}{4.569057in}}%
\pgfpathlineto{\pgfqpoint{1.876321in}{4.569959in}}%
\pgfpathlineto{\pgfqpoint{1.891361in}{4.574496in}}%
\pgfpathlineto{\pgfqpoint{1.898881in}{4.575031in}}%
\pgfpathlineto{\pgfqpoint{1.903894in}{4.576251in}}%
\pgfpathlineto{\pgfqpoint{1.908907in}{4.577253in}}%
\pgfpathlineto{\pgfqpoint{1.911414in}{4.579125in}}%
\pgfpathlineto{\pgfqpoint{1.931467in}{4.581351in}}%
\pgfpathlineto{\pgfqpoint{1.936481in}{4.583853in}}%
\pgfpathlineto{\pgfqpoint{1.946507in}{4.585819in}}%
\pgfpathlineto{\pgfqpoint{1.954027in}{4.586350in}}%
\pgfpathlineto{\pgfqpoint{1.959040in}{4.590069in}}%
\pgfpathlineto{\pgfqpoint{1.979094in}{4.597484in}}%
\pgfpathlineto{\pgfqpoint{1.984107in}{4.598346in}}%
\pgfpathlineto{\pgfqpoint{1.989120in}{4.600446in}}%
\pgfpathlineto{\pgfqpoint{1.994134in}{4.602745in}}%
\pgfpathlineto{\pgfqpoint{1.996640in}{4.604675in}}%
\pgfpathlineto{\pgfqpoint{2.006667in}{4.606069in}}%
\pgfpathlineto{\pgfqpoint{2.024213in}{4.610429in}}%
\pgfpathlineto{\pgfqpoint{2.034240in}{4.611873in}}%
\pgfpathlineto{\pgfqpoint{2.036747in}{4.614796in}}%
\pgfpathlineto{\pgfqpoint{2.044267in}{4.616027in}}%
\pgfpathlineto{\pgfqpoint{2.046773in}{4.618058in}}%
\pgfpathlineto{\pgfqpoint{2.074346in}{4.620666in}}%
\pgfpathlineto{\pgfqpoint{2.076853in}{4.622474in}}%
\pgfpathlineto{\pgfqpoint{2.089386in}{4.623818in}}%
\pgfpathlineto{\pgfqpoint{2.109440in}{4.628043in}}%
\pgfpathlineto{\pgfqpoint{2.126986in}{4.630639in}}%
\pgfpathlineto{\pgfqpoint{2.132000in}{4.633213in}}%
\pgfpathlineto{\pgfqpoint{2.137013in}{4.633429in}}%
\pgfpathlineto{\pgfqpoint{2.144533in}{4.636526in}}%
\pgfpathlineto{\pgfqpoint{2.147039in}{4.636840in}}%
\pgfpathlineto{\pgfqpoint{2.149546in}{4.638602in}}%
\pgfpathlineto{\pgfqpoint{2.152053in}{4.638631in}}%
\pgfpathlineto{\pgfqpoint{2.157066in}{4.642059in}}%
\pgfpathlineto{\pgfqpoint{2.162079in}{4.643325in}}%
\pgfpathlineto{\pgfqpoint{2.167093in}{4.643790in}}%
\pgfpathlineto{\pgfqpoint{2.174613in}{4.650343in}}%
\pgfpathlineto{\pgfqpoint{2.177119in}{4.652844in}}%
\pgfpathlineto{\pgfqpoint{2.179626in}{4.653038in}}%
\pgfpathlineto{\pgfqpoint{2.182133in}{4.657106in}}%
\pgfpathlineto{\pgfqpoint{2.184639in}{4.657445in}}%
\pgfpathlineto{\pgfqpoint{2.187146in}{4.663290in}}%
\pgfpathlineto{\pgfqpoint{2.194666in}{4.663942in}}%
\pgfpathlineto{\pgfqpoint{2.199679in}{4.668549in}}%
\pgfpathlineto{\pgfqpoint{2.207199in}{4.673995in}}%
\pgfpathlineto{\pgfqpoint{2.209706in}{4.674322in}}%
\pgfpathlineto{\pgfqpoint{2.212212in}{4.676738in}}%
\pgfpathlineto{\pgfqpoint{2.219732in}{4.677313in}}%
\pgfpathlineto{\pgfqpoint{2.227252in}{4.684129in}}%
\pgfpathlineto{\pgfqpoint{2.242292in}{4.687218in}}%
\pgfpathlineto{\pgfqpoint{2.267359in}{4.693991in}}%
\pgfpathlineto{\pgfqpoint{2.269866in}{4.696334in}}%
\pgfpathlineto{\pgfqpoint{2.274879in}{4.697864in}}%
\pgfpathlineto{\pgfqpoint{2.277385in}{4.701393in}}%
\pgfpathlineto{\pgfqpoint{2.287412in}{4.704096in}}%
\pgfpathlineto{\pgfqpoint{2.294932in}{4.704911in}}%
\pgfpathlineto{\pgfqpoint{2.327519in}{4.715052in}}%
\pgfpathlineto{\pgfqpoint{2.330025in}{4.719088in}}%
\pgfpathlineto{\pgfqpoint{2.335039in}{4.721479in}}%
\pgfpathlineto{\pgfqpoint{2.337545in}{4.721966in}}%
\pgfpathlineto{\pgfqpoint{2.340052in}{4.724037in}}%
\pgfpathlineto{\pgfqpoint{2.347572in}{4.725731in}}%
\pgfpathlineto{\pgfqpoint{2.350078in}{4.727661in}}%
\pgfpathlineto{\pgfqpoint{2.352585in}{4.727970in}}%
\pgfpathlineto{\pgfqpoint{2.355092in}{4.729555in}}%
\pgfpathlineto{\pgfqpoint{2.360105in}{4.730828in}}%
\pgfpathlineto{\pgfqpoint{2.372638in}{4.734744in}}%
\pgfpathlineto{\pgfqpoint{2.375145in}{4.737100in}}%
\pgfpathlineto{\pgfqpoint{2.405225in}{4.745661in}}%
\pgfpathlineto{\pgfqpoint{2.415251in}{4.754641in}}%
\pgfpathlineto{\pgfqpoint{2.420265in}{4.756641in}}%
\pgfpathlineto{\pgfqpoint{2.427785in}{4.762210in}}%
\pgfpathlineto{\pgfqpoint{2.430291in}{4.764950in}}%
\pgfpathlineto{\pgfqpoint{2.442825in}{4.768179in}}%
\pgfpathlineto{\pgfqpoint{2.450345in}{4.770389in}}%
\pgfpathlineto{\pgfqpoint{2.455358in}{4.775675in}}%
\pgfpathlineto{\pgfqpoint{2.460371in}{4.777823in}}%
\pgfpathlineto{\pgfqpoint{2.470398in}{4.785151in}}%
\pgfpathlineto{\pgfqpoint{2.472905in}{4.787841in}}%
\pgfpathlineto{\pgfqpoint{2.475411in}{4.788166in}}%
\pgfpathlineto{\pgfqpoint{2.477918in}{4.789914in}}%
\pgfpathlineto{\pgfqpoint{2.482931in}{4.791053in}}%
\pgfpathlineto{\pgfqpoint{2.495464in}{4.802222in}}%
\pgfpathlineto{\pgfqpoint{2.497971in}{4.809109in}}%
\pgfpathlineto{\pgfqpoint{2.502984in}{4.816979in}}%
\pgfpathlineto{\pgfqpoint{2.507998in}{4.817158in}}%
\pgfpathlineto{\pgfqpoint{2.513011in}{4.819770in}}%
\pgfpathlineto{\pgfqpoint{2.518024in}{4.819880in}}%
\pgfpathlineto{\pgfqpoint{2.523038in}{4.820998in}}%
\pgfpathlineto{\pgfqpoint{2.540584in}{4.825178in}}%
\pgfpathlineto{\pgfqpoint{2.563144in}{4.829446in}}%
\pgfpathlineto{\pgfqpoint{2.565651in}{4.831346in}}%
\pgfpathlineto{\pgfqpoint{2.568157in}{4.831614in}}%
\pgfpathlineto{\pgfqpoint{2.570664in}{4.834292in}}%
\pgfpathlineto{\pgfqpoint{2.578184in}{4.836361in}}%
\pgfpathlineto{\pgfqpoint{2.583197in}{4.837926in}}%
\pgfpathlineto{\pgfqpoint{2.590717in}{4.839839in}}%
\pgfpathlineto{\pgfqpoint{2.593224in}{4.839939in}}%
\pgfpathlineto{\pgfqpoint{2.598237in}{4.843147in}}%
\pgfpathlineto{\pgfqpoint{2.610771in}{4.845319in}}%
\pgfpathlineto{\pgfqpoint{2.630824in}{4.848272in}}%
\pgfpathlineto{\pgfqpoint{2.635837in}{4.849435in}}%
\pgfpathlineto{\pgfqpoint{2.645864in}{4.850288in}}%
\pgfpathlineto{\pgfqpoint{2.670930in}{4.854905in}}%
\pgfpathlineto{\pgfqpoint{2.673437in}{4.856520in}}%
\pgfpathlineto{\pgfqpoint{2.678450in}{4.862248in}}%
\pgfpathlineto{\pgfqpoint{2.688477in}{4.862821in}}%
\pgfpathlineto{\pgfqpoint{2.711037in}{4.871689in}}%
\pgfpathlineto{\pgfqpoint{2.713543in}{4.876416in}}%
\pgfpathlineto{\pgfqpoint{2.718557in}{4.877591in}}%
\pgfpathlineto{\pgfqpoint{2.721063in}{4.883332in}}%
\pgfpathlineto{\pgfqpoint{2.728583in}{4.889028in}}%
\pgfpathlineto{\pgfqpoint{2.738610in}{4.891457in}}%
\pgfpathlineto{\pgfqpoint{2.746130in}{4.899568in}}%
\pgfpathlineto{\pgfqpoint{2.748637in}{4.902218in}}%
\pgfpathlineto{\pgfqpoint{2.751143in}{4.902482in}}%
\pgfpathlineto{\pgfqpoint{2.753650in}{4.906091in}}%
\pgfpathlineto{\pgfqpoint{2.761170in}{4.908855in}}%
\pgfpathlineto{\pgfqpoint{2.763676in}{4.908949in}}%
\pgfpathlineto{\pgfqpoint{2.768690in}{4.912211in}}%
\pgfpathlineto{\pgfqpoint{2.821329in}{4.930373in}}%
\pgfpathlineto{\pgfqpoint{2.826343in}{4.934278in}}%
\pgfpathlineto{\pgfqpoint{2.833863in}{4.937656in}}%
\pgfpathlineto{\pgfqpoint{2.836369in}{4.939913in}}%
\pgfpathlineto{\pgfqpoint{2.851409in}{4.941164in}}%
\pgfpathlineto{\pgfqpoint{2.853916in}{4.942390in}}%
\pgfpathlineto{\pgfqpoint{2.856423in}{4.945582in}}%
\pgfpathlineto{\pgfqpoint{2.863943in}{4.946895in}}%
\pgfpathlineto{\pgfqpoint{2.866449in}{4.947643in}}%
\pgfpathlineto{\pgfqpoint{2.868956in}{4.951692in}}%
\pgfpathlineto{\pgfqpoint{2.873969in}{4.954511in}}%
\pgfpathlineto{\pgfqpoint{2.876476in}{4.954918in}}%
\pgfpathlineto{\pgfqpoint{2.878983in}{4.957749in}}%
\pgfpathlineto{\pgfqpoint{2.906556in}{4.966194in}}%
\pgfpathlineto{\pgfqpoint{2.909062in}{4.968329in}}%
\pgfpathlineto{\pgfqpoint{2.914076in}{4.969090in}}%
\pgfpathlineto{\pgfqpoint{2.924102in}{4.970254in}}%
\pgfpathlineto{\pgfqpoint{2.926609in}{4.971209in}}%
\pgfpathlineto{\pgfqpoint{2.929116in}{4.974771in}}%
\pgfpathlineto{\pgfqpoint{2.946662in}{4.980074in}}%
\pgfpathlineto{\pgfqpoint{2.949169in}{4.981959in}}%
\pgfpathlineto{\pgfqpoint{2.959195in}{4.984546in}}%
\pgfpathlineto{\pgfqpoint{2.994289in}{4.996675in}}%
\pgfpathlineto{\pgfqpoint{2.996795in}{4.999897in}}%
\pgfpathlineto{\pgfqpoint{2.999302in}{4.999919in}}%
\pgfpathlineto{\pgfqpoint{3.004315in}{5.001303in}}%
\pgfpathlineto{\pgfqpoint{3.011835in}{5.003419in}}%
\pgfpathlineto{\pgfqpoint{3.014342in}{5.006731in}}%
\pgfpathlineto{\pgfqpoint{3.036902in}{5.017321in}}%
\pgfpathlineto{\pgfqpoint{3.039408in}{5.023509in}}%
\pgfpathlineto{\pgfqpoint{3.044422in}{5.026600in}}%
\pgfpathlineto{\pgfqpoint{3.049435in}{5.033957in}}%
\pgfpathlineto{\pgfqpoint{3.051942in}{5.034059in}}%
\pgfpathlineto{\pgfqpoint{3.056955in}{5.035880in}}%
\pgfpathlineto{\pgfqpoint{3.061968in}{5.036932in}}%
\pgfpathlineto{\pgfqpoint{3.069488in}{5.045763in}}%
\pgfpathlineto{\pgfqpoint{3.077008in}{5.047498in}}%
\pgfpathlineto{\pgfqpoint{3.087035in}{5.051923in}}%
\pgfpathlineto{\pgfqpoint{3.092048in}{5.052892in}}%
\pgfpathlineto{\pgfqpoint{3.094555in}{5.055849in}}%
\pgfpathlineto{\pgfqpoint{3.122128in}{5.061645in}}%
\pgfpathlineto{\pgfqpoint{3.124635in}{5.065412in}}%
\pgfpathlineto{\pgfqpoint{3.134661in}{5.071448in}}%
\pgfpathlineto{\pgfqpoint{3.137168in}{5.074702in}}%
\pgfpathlineto{\pgfqpoint{3.144688in}{5.075616in}}%
\pgfpathlineto{\pgfqpoint{3.147195in}{5.081217in}}%
\pgfpathlineto{\pgfqpoint{3.157221in}{5.083251in}}%
\pgfpathlineto{\pgfqpoint{3.164741in}{5.085164in}}%
\pgfpathlineto{\pgfqpoint{3.169754in}{5.087417in}}%
\pgfpathlineto{\pgfqpoint{3.174768in}{5.089238in}}%
\pgfpathlineto{\pgfqpoint{3.177274in}{5.089357in}}%
\pgfpathlineto{\pgfqpoint{3.182288in}{5.094174in}}%
\pgfpathlineto{\pgfqpoint{3.189808in}{5.095815in}}%
\pgfpathlineto{\pgfqpoint{3.194821in}{5.096248in}}%
\pgfpathlineto{\pgfqpoint{3.197328in}{5.100934in}}%
\pgfpathlineto{\pgfqpoint{3.199834in}{5.102815in}}%
\pgfpathlineto{\pgfqpoint{3.204848in}{5.104552in}}%
\pgfpathlineto{\pgfqpoint{3.209861in}{5.105373in}}%
\pgfpathlineto{\pgfqpoint{3.222394in}{5.107958in}}%
\pgfpathlineto{\pgfqpoint{3.229914in}{5.116710in}}%
\pgfpathlineto{\pgfqpoint{3.239941in}{5.118270in}}%
\pgfpathlineto{\pgfqpoint{3.242447in}{5.126072in}}%
\pgfpathlineto{\pgfqpoint{3.244954in}{5.127036in}}%
\pgfpathlineto{\pgfqpoint{3.249967in}{5.130761in}}%
\pgfpathlineto{\pgfqpoint{3.259994in}{5.135916in}}%
\pgfpathlineto{\pgfqpoint{3.262501in}{5.141359in}}%
\pgfpathlineto{\pgfqpoint{3.280047in}{5.145652in}}%
\pgfpathlineto{\pgfqpoint{3.287567in}{5.147284in}}%
\pgfpathlineto{\pgfqpoint{3.292581in}{5.154634in}}%
\pgfpathlineto{\pgfqpoint{3.295087in}{5.155159in}}%
\pgfpathlineto{\pgfqpoint{3.297594in}{5.157200in}}%
\pgfpathlineto{\pgfqpoint{3.305114in}{5.158187in}}%
\pgfpathlineto{\pgfqpoint{3.312634in}{5.160916in}}%
\pgfpathlineto{\pgfqpoint{3.315140in}{5.164414in}}%
\pgfpathlineto{\pgfqpoint{3.327674in}{5.168282in}}%
\pgfpathlineto{\pgfqpoint{3.335194in}{5.174055in}}%
\pgfpathlineto{\pgfqpoint{3.337700in}{5.174484in}}%
\pgfpathlineto{\pgfqpoint{3.342714in}{5.177934in}}%
\pgfpathlineto{\pgfqpoint{3.345220in}{5.178836in}}%
\pgfpathlineto{\pgfqpoint{3.347727in}{5.181423in}}%
\pgfpathlineto{\pgfqpoint{3.355247in}{5.183940in}}%
\pgfpathlineto{\pgfqpoint{3.357754in}{5.186207in}}%
\pgfpathlineto{\pgfqpoint{3.360260in}{5.186506in}}%
\pgfpathlineto{\pgfqpoint{3.362767in}{5.187994in}}%
\pgfpathlineto{\pgfqpoint{3.367780in}{5.189039in}}%
\pgfpathlineto{\pgfqpoint{3.370287in}{5.190778in}}%
\pgfpathlineto{\pgfqpoint{3.372793in}{5.190837in}}%
\pgfpathlineto{\pgfqpoint{3.375300in}{5.192636in}}%
\pgfpathlineto{\pgfqpoint{3.390340in}{5.211598in}}%
\pgfpathlineto{\pgfqpoint{3.392847in}{5.213979in}}%
\pgfpathlineto{\pgfqpoint{3.400367in}{5.215306in}}%
\pgfpathlineto{\pgfqpoint{3.412900in}{5.222451in}}%
\pgfpathlineto{\pgfqpoint{3.417913in}{5.223645in}}%
\pgfpathlineto{\pgfqpoint{3.432953in}{5.232106in}}%
\pgfpathlineto{\pgfqpoint{3.435460in}{5.233858in}}%
\pgfpathlineto{\pgfqpoint{3.440473in}{5.234492in}}%
\pgfpathlineto{\pgfqpoint{3.442980in}{5.237010in}}%
\pgfpathlineto{\pgfqpoint{3.445486in}{5.237225in}}%
\pgfpathlineto{\pgfqpoint{3.453006in}{5.244537in}}%
\pgfpathlineto{\pgfqpoint{3.455513in}{5.244654in}}%
\pgfpathlineto{\pgfqpoint{3.460526in}{5.249355in}}%
\pgfpathlineto{\pgfqpoint{3.465540in}{5.249917in}}%
\pgfpathlineto{\pgfqpoint{3.468046in}{5.251388in}}%
\pgfpathlineto{\pgfqpoint{3.473060in}{5.257802in}}%
\pgfpathlineto{\pgfqpoint{3.480580in}{5.260557in}}%
\pgfpathlineto{\pgfqpoint{3.483086in}{5.270047in}}%
\pgfpathlineto{\pgfqpoint{3.488100in}{5.275748in}}%
\pgfpathlineto{\pgfqpoint{3.490606in}{5.276887in}}%
\pgfpathlineto{\pgfqpoint{3.493113in}{5.296598in}}%
\pgfpathlineto{\pgfqpoint{3.508153in}{5.301085in}}%
\pgfpathlineto{\pgfqpoint{3.510659in}{5.302033in}}%
\pgfpathlineto{\pgfqpoint{3.513166in}{5.305275in}}%
\pgfpathlineto{\pgfqpoint{3.513166in}{5.305275in}}%
\pgfusepath{stroke}%
\end{pgfscope}%
\begin{pgfscope}%
\pgfpathrectangle{\pgfqpoint{0.708220in}{3.210823in}}{\pgfqpoint{5.013309in}{2.094453in}}%
\pgfusepath{clip}%
\pgfsetrectcap%
\pgfsetroundjoin%
\pgfsetlinewidth{1.003750pt}%
\definecolor{currentstroke}{rgb}{0.564706,0.564706,1.000000}%
\pgfsetstrokecolor{currentstroke}%
\pgfsetdash{}{0pt}%
\pgfpathmoveto{\pgfqpoint{0.708220in}{3.443431in}}%
\pgfpathlineto{\pgfqpoint{0.713233in}{3.477920in}}%
\pgfpathlineto{\pgfqpoint{0.715740in}{3.480950in}}%
\pgfpathlineto{\pgfqpoint{0.718246in}{3.520307in}}%
\pgfpathlineto{\pgfqpoint{0.723260in}{3.542640in}}%
\pgfpathlineto{\pgfqpoint{0.725766in}{3.544392in}}%
\pgfpathlineto{\pgfqpoint{0.728273in}{3.555494in}}%
\pgfpathlineto{\pgfqpoint{0.730780in}{3.556534in}}%
\pgfpathlineto{\pgfqpoint{0.733286in}{3.561538in}}%
\pgfpathlineto{\pgfqpoint{0.748326in}{3.611359in}}%
\pgfpathlineto{\pgfqpoint{0.750833in}{3.616166in}}%
\pgfpathlineto{\pgfqpoint{0.753340in}{3.617145in}}%
\pgfpathlineto{\pgfqpoint{0.763366in}{3.633490in}}%
\pgfpathlineto{\pgfqpoint{0.765873in}{3.638590in}}%
\pgfpathlineto{\pgfqpoint{0.768380in}{3.640364in}}%
\pgfpathlineto{\pgfqpoint{0.773393in}{3.649135in}}%
\pgfpathlineto{\pgfqpoint{0.778406in}{3.669332in}}%
\pgfpathlineto{\pgfqpoint{0.780913in}{3.670778in}}%
\pgfpathlineto{\pgfqpoint{0.783419in}{3.676100in}}%
\pgfpathlineto{\pgfqpoint{0.785926in}{3.677405in}}%
\pgfpathlineto{\pgfqpoint{0.788433in}{3.692695in}}%
\pgfpathlineto{\pgfqpoint{0.790939in}{3.693266in}}%
\pgfpathlineto{\pgfqpoint{0.793446in}{3.695989in}}%
\pgfpathlineto{\pgfqpoint{0.795953in}{3.696126in}}%
\pgfpathlineto{\pgfqpoint{0.800966in}{3.713279in}}%
\pgfpathlineto{\pgfqpoint{0.803473in}{3.713869in}}%
\pgfpathlineto{\pgfqpoint{0.805979in}{3.724160in}}%
\pgfpathlineto{\pgfqpoint{0.808486in}{3.724975in}}%
\pgfpathlineto{\pgfqpoint{0.810993in}{3.727096in}}%
\pgfpathlineto{\pgfqpoint{0.813499in}{3.738841in}}%
\pgfpathlineto{\pgfqpoint{0.816006in}{3.739205in}}%
\pgfpathlineto{\pgfqpoint{0.818513in}{3.749467in}}%
\pgfpathlineto{\pgfqpoint{0.821019in}{3.753338in}}%
\pgfpathlineto{\pgfqpoint{0.823526in}{3.753719in}}%
\pgfpathlineto{\pgfqpoint{0.831046in}{3.762280in}}%
\pgfpathlineto{\pgfqpoint{0.833553in}{3.771809in}}%
\pgfpathlineto{\pgfqpoint{0.838566in}{3.772704in}}%
\pgfpathlineto{\pgfqpoint{0.843579in}{3.774971in}}%
\pgfpathlineto{\pgfqpoint{0.848593in}{3.782548in}}%
\pgfpathlineto{\pgfqpoint{0.851099in}{3.786916in}}%
\pgfpathlineto{\pgfqpoint{0.853606in}{3.788799in}}%
\pgfpathlineto{\pgfqpoint{0.856112in}{3.793477in}}%
\pgfpathlineto{\pgfqpoint{0.861126in}{3.795753in}}%
\pgfpathlineto{\pgfqpoint{0.863632in}{3.795810in}}%
\pgfpathlineto{\pgfqpoint{0.868646in}{3.804285in}}%
\pgfpathlineto{\pgfqpoint{0.873659in}{3.805066in}}%
\pgfpathlineto{\pgfqpoint{0.876166in}{3.808291in}}%
\pgfpathlineto{\pgfqpoint{0.881179in}{3.809913in}}%
\pgfpathlineto{\pgfqpoint{0.906246in}{3.825483in}}%
\pgfpathlineto{\pgfqpoint{0.908752in}{3.828646in}}%
\pgfpathlineto{\pgfqpoint{0.911259in}{3.829994in}}%
\pgfpathlineto{\pgfqpoint{0.913766in}{3.834319in}}%
\pgfpathlineto{\pgfqpoint{0.916272in}{3.834525in}}%
\pgfpathlineto{\pgfqpoint{0.918779in}{3.839325in}}%
\pgfpathlineto{\pgfqpoint{0.923792in}{3.843146in}}%
\pgfpathlineto{\pgfqpoint{0.926299in}{3.844550in}}%
\pgfpathlineto{\pgfqpoint{0.928805in}{3.861054in}}%
\pgfpathlineto{\pgfqpoint{0.931312in}{3.864473in}}%
\pgfpathlineto{\pgfqpoint{0.941339in}{3.866146in}}%
\pgfpathlineto{\pgfqpoint{0.943845in}{3.866762in}}%
\pgfpathlineto{\pgfqpoint{0.948859in}{3.875065in}}%
\pgfpathlineto{\pgfqpoint{0.958885in}{3.878624in}}%
\pgfpathlineto{\pgfqpoint{0.961392in}{3.881910in}}%
\pgfpathlineto{\pgfqpoint{0.971419in}{3.884363in}}%
\pgfpathlineto{\pgfqpoint{0.976432in}{3.888759in}}%
\pgfpathlineto{\pgfqpoint{0.978939in}{3.888877in}}%
\pgfpathlineto{\pgfqpoint{0.981445in}{3.890745in}}%
\pgfpathlineto{\pgfqpoint{0.983952in}{3.894165in}}%
\pgfpathlineto{\pgfqpoint{0.993978in}{3.896525in}}%
\pgfpathlineto{\pgfqpoint{0.998992in}{3.901187in}}%
\pgfpathlineto{\pgfqpoint{1.001498in}{3.901431in}}%
\pgfpathlineto{\pgfqpoint{1.004005in}{3.905863in}}%
\pgfpathlineto{\pgfqpoint{1.009018in}{3.907501in}}%
\pgfpathlineto{\pgfqpoint{1.011525in}{3.909422in}}%
\pgfpathlineto{\pgfqpoint{1.026565in}{3.910860in}}%
\pgfpathlineto{\pgfqpoint{1.029072in}{3.913280in}}%
\pgfpathlineto{\pgfqpoint{1.031578in}{3.913398in}}%
\pgfpathlineto{\pgfqpoint{1.034085in}{3.916195in}}%
\pgfpathlineto{\pgfqpoint{1.039098in}{3.917667in}}%
\pgfpathlineto{\pgfqpoint{1.044112in}{3.919008in}}%
\pgfpathlineto{\pgfqpoint{1.051632in}{3.919816in}}%
\pgfpathlineto{\pgfqpoint{1.059151in}{3.920690in}}%
\pgfpathlineto{\pgfqpoint{1.091738in}{3.931810in}}%
\pgfpathlineto{\pgfqpoint{1.099258in}{3.934713in}}%
\pgfpathlineto{\pgfqpoint{1.104271in}{3.935210in}}%
\pgfpathlineto{\pgfqpoint{1.109285in}{3.936328in}}%
\pgfpathlineto{\pgfqpoint{1.111791in}{3.937065in}}%
\pgfpathlineto{\pgfqpoint{1.114298in}{3.939405in}}%
\pgfpathlineto{\pgfqpoint{1.119311in}{3.940523in}}%
\pgfpathlineto{\pgfqpoint{1.124324in}{3.941749in}}%
\pgfpathlineto{\pgfqpoint{1.129338in}{3.943022in}}%
\pgfpathlineto{\pgfqpoint{1.131844in}{3.943413in}}%
\pgfpathlineto{\pgfqpoint{1.139364in}{3.951891in}}%
\pgfpathlineto{\pgfqpoint{1.141871in}{3.952954in}}%
\pgfpathlineto{\pgfqpoint{1.144378in}{3.955708in}}%
\pgfpathlineto{\pgfqpoint{1.151898in}{3.957640in}}%
\pgfpathlineto{\pgfqpoint{1.159418in}{3.960423in}}%
\pgfpathlineto{\pgfqpoint{1.161924in}{3.963009in}}%
\pgfpathlineto{\pgfqpoint{1.186991in}{3.970424in}}%
\pgfpathlineto{\pgfqpoint{1.192004in}{3.971448in}}%
\pgfpathlineto{\pgfqpoint{1.199524in}{3.972291in}}%
\pgfpathlineto{\pgfqpoint{1.202031in}{3.974407in}}%
\pgfpathlineto{\pgfqpoint{1.214564in}{3.977195in}}%
\pgfpathlineto{\pgfqpoint{1.217071in}{3.978631in}}%
\pgfpathlineto{\pgfqpoint{1.222084in}{3.978938in}}%
\pgfpathlineto{\pgfqpoint{1.224591in}{3.981053in}}%
\pgfpathlineto{\pgfqpoint{1.242137in}{3.982545in}}%
\pgfpathlineto{\pgfqpoint{1.244644in}{3.984234in}}%
\pgfpathlineto{\pgfqpoint{1.249657in}{3.984685in}}%
\pgfpathlineto{\pgfqpoint{1.254671in}{3.988665in}}%
\pgfpathlineto{\pgfqpoint{1.259684in}{3.989549in}}%
\pgfpathlineto{\pgfqpoint{1.269710in}{3.991847in}}%
\pgfpathlineto{\pgfqpoint{1.282244in}{3.993454in}}%
\pgfpathlineto{\pgfqpoint{1.287257in}{3.994277in}}%
\pgfpathlineto{\pgfqpoint{1.297284in}{3.994909in}}%
\pgfpathlineto{\pgfqpoint{1.302297in}{3.997554in}}%
\pgfpathlineto{\pgfqpoint{1.304804in}{3.998026in}}%
\pgfpathlineto{\pgfqpoint{1.314830in}{4.005623in}}%
\pgfpathlineto{\pgfqpoint{1.317337in}{4.005766in}}%
\pgfpathlineto{\pgfqpoint{1.322350in}{4.008177in}}%
\pgfpathlineto{\pgfqpoint{1.334883in}{4.010037in}}%
\pgfpathlineto{\pgfqpoint{1.342403in}{4.012302in}}%
\pgfpathlineto{\pgfqpoint{1.347417in}{4.014525in}}%
\pgfpathlineto{\pgfqpoint{1.364963in}{4.019996in}}%
\pgfpathlineto{\pgfqpoint{1.372483in}{4.022219in}}%
\pgfpathlineto{\pgfqpoint{1.377497in}{4.023015in}}%
\pgfpathlineto{\pgfqpoint{1.380003in}{4.023280in}}%
\pgfpathlineto{\pgfqpoint{1.382510in}{4.025224in}}%
\pgfpathlineto{\pgfqpoint{1.387523in}{4.026194in}}%
\pgfpathlineto{\pgfqpoint{1.392537in}{4.027295in}}%
\pgfpathlineto{\pgfqpoint{1.402563in}{4.029093in}}%
\pgfpathlineto{\pgfqpoint{1.405070in}{4.029748in}}%
\pgfpathlineto{\pgfqpoint{1.407576in}{4.031643in}}%
\pgfpathlineto{\pgfqpoint{1.410083in}{4.032073in}}%
\pgfpathlineto{\pgfqpoint{1.415096in}{4.034234in}}%
\pgfpathlineto{\pgfqpoint{1.430136in}{4.036567in}}%
\pgfpathlineto{\pgfqpoint{1.437656in}{4.038493in}}%
\pgfpathlineto{\pgfqpoint{1.440163in}{4.039115in}}%
\pgfpathlineto{\pgfqpoint{1.442670in}{4.042354in}}%
\pgfpathlineto{\pgfqpoint{1.457710in}{4.046675in}}%
\pgfpathlineto{\pgfqpoint{1.462723in}{4.049986in}}%
\pgfpathlineto{\pgfqpoint{1.485283in}{4.054194in}}%
\pgfpathlineto{\pgfqpoint{1.487789in}{4.055950in}}%
\pgfpathlineto{\pgfqpoint{1.492803in}{4.056123in}}%
\pgfpathlineto{\pgfqpoint{1.495309in}{4.057898in}}%
\pgfpathlineto{\pgfqpoint{1.497816in}{4.058244in}}%
\pgfpathlineto{\pgfqpoint{1.500323in}{4.060072in}}%
\pgfpathlineto{\pgfqpoint{1.502829in}{4.063856in}}%
\pgfpathlineto{\pgfqpoint{1.507843in}{4.066142in}}%
\pgfpathlineto{\pgfqpoint{1.512856in}{4.070069in}}%
\pgfpathlineto{\pgfqpoint{1.517869in}{4.070958in}}%
\pgfpathlineto{\pgfqpoint{1.520376in}{4.074181in}}%
\pgfpathlineto{\pgfqpoint{1.525389in}{4.075848in}}%
\pgfpathlineto{\pgfqpoint{1.537922in}{4.078384in}}%
\pgfpathlineto{\pgfqpoint{1.542936in}{4.079731in}}%
\pgfpathlineto{\pgfqpoint{1.547949in}{4.081161in}}%
\pgfpathlineto{\pgfqpoint{1.562989in}{4.085880in}}%
\pgfpathlineto{\pgfqpoint{1.568002in}{4.087864in}}%
\pgfpathlineto{\pgfqpoint{1.575522in}{4.089176in}}%
\pgfpathlineto{\pgfqpoint{1.578029in}{4.092938in}}%
\pgfpathlineto{\pgfqpoint{1.608109in}{4.100727in}}%
\pgfpathlineto{\pgfqpoint{1.613122in}{4.105365in}}%
\pgfpathlineto{\pgfqpoint{1.623149in}{4.110931in}}%
\pgfpathlineto{\pgfqpoint{1.635682in}{4.115199in}}%
\pgfpathlineto{\pgfqpoint{1.640695in}{4.117111in}}%
\pgfpathlineto{\pgfqpoint{1.645709in}{4.118170in}}%
\pgfpathlineto{\pgfqpoint{1.655735in}{4.119721in}}%
\pgfpathlineto{\pgfqpoint{1.675788in}{4.121843in}}%
\pgfpathlineto{\pgfqpoint{1.678295in}{4.123416in}}%
\pgfpathlineto{\pgfqpoint{1.680802in}{4.123598in}}%
\pgfpathlineto{\pgfqpoint{1.685815in}{4.125850in}}%
\pgfpathlineto{\pgfqpoint{1.690828in}{4.126304in}}%
\pgfpathlineto{\pgfqpoint{1.695842in}{4.131528in}}%
\pgfpathlineto{\pgfqpoint{1.698348in}{4.135155in}}%
\pgfpathlineto{\pgfqpoint{1.703362in}{4.135333in}}%
\pgfpathlineto{\pgfqpoint{1.705868in}{4.137271in}}%
\pgfpathlineto{\pgfqpoint{1.710882in}{4.138104in}}%
\pgfpathlineto{\pgfqpoint{1.720908in}{4.140106in}}%
\pgfpathlineto{\pgfqpoint{1.728428in}{4.145628in}}%
\pgfpathlineto{\pgfqpoint{1.730935in}{4.145924in}}%
\pgfpathlineto{\pgfqpoint{1.738455in}{4.151065in}}%
\pgfpathlineto{\pgfqpoint{1.761015in}{4.158404in}}%
\pgfpathlineto{\pgfqpoint{1.763521in}{4.161765in}}%
\pgfpathlineto{\pgfqpoint{1.773548in}{4.163176in}}%
\pgfpathlineto{\pgfqpoint{1.776055in}{4.167140in}}%
\pgfpathlineto{\pgfqpoint{1.788588in}{4.169820in}}%
\pgfpathlineto{\pgfqpoint{1.793601in}{4.171784in}}%
\pgfpathlineto{\pgfqpoint{1.796108in}{4.175112in}}%
\pgfpathlineto{\pgfqpoint{1.798615in}{4.175521in}}%
\pgfpathlineto{\pgfqpoint{1.803628in}{4.179505in}}%
\pgfpathlineto{\pgfqpoint{1.818668in}{4.184059in}}%
\pgfpathlineto{\pgfqpoint{1.823681in}{4.188480in}}%
\pgfpathlineto{\pgfqpoint{1.836214in}{4.190617in}}%
\pgfpathlineto{\pgfqpoint{1.846241in}{4.194552in}}%
\pgfpathlineto{\pgfqpoint{1.848748in}{4.197373in}}%
\pgfpathlineto{\pgfqpoint{1.851254in}{4.197766in}}%
\pgfpathlineto{\pgfqpoint{1.853761in}{4.202851in}}%
\pgfpathlineto{\pgfqpoint{1.861281in}{4.204216in}}%
\pgfpathlineto{\pgfqpoint{1.866294in}{4.207399in}}%
\pgfpathlineto{\pgfqpoint{1.871307in}{4.208166in}}%
\pgfpathlineto{\pgfqpoint{1.873814in}{4.210249in}}%
\pgfpathlineto{\pgfqpoint{1.886347in}{4.214322in}}%
\pgfpathlineto{\pgfqpoint{1.888854in}{4.215050in}}%
\pgfpathlineto{\pgfqpoint{1.891361in}{4.217245in}}%
\pgfpathlineto{\pgfqpoint{1.908907in}{4.221305in}}%
\pgfpathlineto{\pgfqpoint{1.911414in}{4.226688in}}%
\pgfpathlineto{\pgfqpoint{1.918934in}{4.231067in}}%
\pgfpathlineto{\pgfqpoint{1.921441in}{4.231356in}}%
\pgfpathlineto{\pgfqpoint{1.923947in}{4.233380in}}%
\pgfpathlineto{\pgfqpoint{1.926454in}{4.233565in}}%
\pgfpathlineto{\pgfqpoint{1.931467in}{4.236183in}}%
\pgfpathlineto{\pgfqpoint{1.933974in}{4.236444in}}%
\pgfpathlineto{\pgfqpoint{1.938987in}{4.243495in}}%
\pgfpathlineto{\pgfqpoint{1.941494in}{4.245205in}}%
\pgfpathlineto{\pgfqpoint{1.944000in}{4.248328in}}%
\pgfpathlineto{\pgfqpoint{1.951520in}{4.249294in}}%
\pgfpathlineto{\pgfqpoint{1.954027in}{4.252595in}}%
\pgfpathlineto{\pgfqpoint{1.956534in}{4.253257in}}%
\pgfpathlineto{\pgfqpoint{1.959040in}{4.256254in}}%
\pgfpathlineto{\pgfqpoint{1.969067in}{4.257306in}}%
\pgfpathlineto{\pgfqpoint{1.974080in}{4.259921in}}%
\pgfpathlineto{\pgfqpoint{1.981600in}{4.260813in}}%
\pgfpathlineto{\pgfqpoint{1.986614in}{4.262299in}}%
\pgfpathlineto{\pgfqpoint{1.994134in}{4.271667in}}%
\pgfpathlineto{\pgfqpoint{1.996640in}{4.272126in}}%
\pgfpathlineto{\pgfqpoint{1.999147in}{4.274927in}}%
\pgfpathlineto{\pgfqpoint{2.004160in}{4.275702in}}%
\pgfpathlineto{\pgfqpoint{2.011680in}{4.279963in}}%
\pgfpathlineto{\pgfqpoint{2.014187in}{4.283578in}}%
\pgfpathlineto{\pgfqpoint{2.021707in}{4.284755in}}%
\pgfpathlineto{\pgfqpoint{2.026720in}{4.286113in}}%
\pgfpathlineto{\pgfqpoint{2.034240in}{4.290535in}}%
\pgfpathlineto{\pgfqpoint{2.039253in}{4.291608in}}%
\pgfpathlineto{\pgfqpoint{2.041760in}{4.292049in}}%
\pgfpathlineto{\pgfqpoint{2.044267in}{4.294171in}}%
\pgfpathlineto{\pgfqpoint{2.049280in}{4.295630in}}%
\pgfpathlineto{\pgfqpoint{2.054293in}{4.297122in}}%
\pgfpathlineto{\pgfqpoint{2.061813in}{4.300496in}}%
\pgfpathlineto{\pgfqpoint{2.064320in}{4.300848in}}%
\pgfpathlineto{\pgfqpoint{2.071840in}{4.305264in}}%
\pgfpathlineto{\pgfqpoint{2.076853in}{4.306530in}}%
\pgfpathlineto{\pgfqpoint{2.081866in}{4.308363in}}%
\pgfpathlineto{\pgfqpoint{2.086880in}{4.311949in}}%
\pgfpathlineto{\pgfqpoint{2.091893in}{4.312163in}}%
\pgfpathlineto{\pgfqpoint{2.094400in}{4.314641in}}%
\pgfpathlineto{\pgfqpoint{2.096906in}{4.315162in}}%
\pgfpathlineto{\pgfqpoint{2.099413in}{4.318032in}}%
\pgfpathlineto{\pgfqpoint{2.111946in}{4.322083in}}%
\pgfpathlineto{\pgfqpoint{2.119466in}{4.326480in}}%
\pgfpathlineto{\pgfqpoint{2.121973in}{4.326793in}}%
\pgfpathlineto{\pgfqpoint{2.126986in}{4.331563in}}%
\pgfpathlineto{\pgfqpoint{2.132000in}{4.337461in}}%
\pgfpathlineto{\pgfqpoint{2.134506in}{4.342775in}}%
\pgfpathlineto{\pgfqpoint{2.139520in}{4.344239in}}%
\pgfpathlineto{\pgfqpoint{2.142026in}{4.346994in}}%
\pgfpathlineto{\pgfqpoint{2.159573in}{4.349770in}}%
\pgfpathlineto{\pgfqpoint{2.169599in}{4.356556in}}%
\pgfpathlineto{\pgfqpoint{2.174613in}{4.360200in}}%
\pgfpathlineto{\pgfqpoint{2.182133in}{4.363183in}}%
\pgfpathlineto{\pgfqpoint{2.199679in}{4.366899in}}%
\pgfpathlineto{\pgfqpoint{2.204693in}{4.367436in}}%
\pgfpathlineto{\pgfqpoint{2.209706in}{4.370819in}}%
\pgfpathlineto{\pgfqpoint{2.219732in}{4.371112in}}%
\pgfpathlineto{\pgfqpoint{2.222239in}{4.373576in}}%
\pgfpathlineto{\pgfqpoint{2.247306in}{4.380571in}}%
\pgfpathlineto{\pgfqpoint{2.249812in}{4.380810in}}%
\pgfpathlineto{\pgfqpoint{2.254826in}{4.384370in}}%
\pgfpathlineto{\pgfqpoint{2.262346in}{4.387091in}}%
\pgfpathlineto{\pgfqpoint{2.274879in}{4.396641in}}%
\pgfpathlineto{\pgfqpoint{2.282399in}{4.400478in}}%
\pgfpathlineto{\pgfqpoint{2.292425in}{4.402761in}}%
\pgfpathlineto{\pgfqpoint{2.304959in}{4.406268in}}%
\pgfpathlineto{\pgfqpoint{2.307465in}{4.409816in}}%
\pgfpathlineto{\pgfqpoint{2.314985in}{4.411196in}}%
\pgfpathlineto{\pgfqpoint{2.319999in}{4.417028in}}%
\pgfpathlineto{\pgfqpoint{2.322505in}{4.418097in}}%
\pgfpathlineto{\pgfqpoint{2.325012in}{4.423602in}}%
\pgfpathlineto{\pgfqpoint{2.330025in}{4.425645in}}%
\pgfpathlineto{\pgfqpoint{2.340052in}{4.426824in}}%
\pgfpathlineto{\pgfqpoint{2.342559in}{4.428717in}}%
\pgfpathlineto{\pgfqpoint{2.352585in}{4.430279in}}%
\pgfpathlineto{\pgfqpoint{2.357598in}{4.432099in}}%
\pgfpathlineto{\pgfqpoint{2.362612in}{4.433342in}}%
\pgfpathlineto{\pgfqpoint{2.367625in}{4.433605in}}%
\pgfpathlineto{\pgfqpoint{2.370132in}{4.436008in}}%
\pgfpathlineto{\pgfqpoint{2.375145in}{4.437223in}}%
\pgfpathlineto{\pgfqpoint{2.377652in}{4.438913in}}%
\pgfpathlineto{\pgfqpoint{2.382665in}{4.439419in}}%
\pgfpathlineto{\pgfqpoint{2.385172in}{4.442770in}}%
\pgfpathlineto{\pgfqpoint{2.395198in}{4.446105in}}%
\pgfpathlineto{\pgfqpoint{2.420265in}{4.451937in}}%
\pgfpathlineto{\pgfqpoint{2.427785in}{4.457105in}}%
\pgfpathlineto{\pgfqpoint{2.430291in}{4.458114in}}%
\pgfpathlineto{\pgfqpoint{2.432798in}{4.460366in}}%
\pgfpathlineto{\pgfqpoint{2.435305in}{4.460594in}}%
\pgfpathlineto{\pgfqpoint{2.440318in}{4.462985in}}%
\pgfpathlineto{\pgfqpoint{2.442825in}{4.463616in}}%
\pgfpathlineto{\pgfqpoint{2.447838in}{4.466656in}}%
\pgfpathlineto{\pgfqpoint{2.455358in}{4.468019in}}%
\pgfpathlineto{\pgfqpoint{2.467891in}{4.472534in}}%
\pgfpathlineto{\pgfqpoint{2.470398in}{4.477411in}}%
\pgfpathlineto{\pgfqpoint{2.482931in}{4.479317in}}%
\pgfpathlineto{\pgfqpoint{2.487944in}{4.482026in}}%
\pgfpathlineto{\pgfqpoint{2.495464in}{4.483362in}}%
\pgfpathlineto{\pgfqpoint{2.497971in}{4.485156in}}%
\pgfpathlineto{\pgfqpoint{2.500478in}{4.489113in}}%
\pgfpathlineto{\pgfqpoint{2.513011in}{4.490440in}}%
\pgfpathlineto{\pgfqpoint{2.520531in}{4.492763in}}%
\pgfpathlineto{\pgfqpoint{2.523038in}{4.495560in}}%
\pgfpathlineto{\pgfqpoint{2.528051in}{4.497113in}}%
\pgfpathlineto{\pgfqpoint{2.530558in}{4.499400in}}%
\pgfpathlineto{\pgfqpoint{2.533064in}{4.499650in}}%
\pgfpathlineto{\pgfqpoint{2.538078in}{4.502242in}}%
\pgfpathlineto{\pgfqpoint{2.548104in}{4.503507in}}%
\pgfpathlineto{\pgfqpoint{2.558131in}{4.512028in}}%
\pgfpathlineto{\pgfqpoint{2.560637in}{4.512344in}}%
\pgfpathlineto{\pgfqpoint{2.563144in}{4.514671in}}%
\pgfpathlineto{\pgfqpoint{2.575677in}{4.517911in}}%
\pgfpathlineto{\pgfqpoint{2.583197in}{4.522724in}}%
\pgfpathlineto{\pgfqpoint{2.588211in}{4.523817in}}%
\pgfpathlineto{\pgfqpoint{2.595731in}{4.525525in}}%
\pgfpathlineto{\pgfqpoint{2.603251in}{4.528343in}}%
\pgfpathlineto{\pgfqpoint{2.610771in}{4.528580in}}%
\pgfpathlineto{\pgfqpoint{2.620797in}{4.536222in}}%
\pgfpathlineto{\pgfqpoint{2.628317in}{4.540235in}}%
\pgfpathlineto{\pgfqpoint{2.635837in}{4.543289in}}%
\pgfpathlineto{\pgfqpoint{2.640850in}{4.546492in}}%
\pgfpathlineto{\pgfqpoint{2.650877in}{4.548614in}}%
\pgfpathlineto{\pgfqpoint{2.655890in}{4.554130in}}%
\pgfpathlineto{\pgfqpoint{2.663410in}{4.557385in}}%
\pgfpathlineto{\pgfqpoint{2.673437in}{4.564619in}}%
\pgfpathlineto{\pgfqpoint{2.675944in}{4.568911in}}%
\pgfpathlineto{\pgfqpoint{2.680957in}{4.570786in}}%
\pgfpathlineto{\pgfqpoint{2.690983in}{4.573774in}}%
\pgfpathlineto{\pgfqpoint{2.693490in}{4.576802in}}%
\pgfpathlineto{\pgfqpoint{2.698503in}{4.577819in}}%
\pgfpathlineto{\pgfqpoint{2.706023in}{4.578948in}}%
\pgfpathlineto{\pgfqpoint{2.721063in}{4.587470in}}%
\pgfpathlineto{\pgfqpoint{2.746130in}{4.594867in}}%
\pgfpathlineto{\pgfqpoint{2.756156in}{4.595572in}}%
\pgfpathlineto{\pgfqpoint{2.758663in}{4.602119in}}%
\pgfpathlineto{\pgfqpoint{2.763676in}{4.603526in}}%
\pgfpathlineto{\pgfqpoint{2.766183in}{4.605271in}}%
\pgfpathlineto{\pgfqpoint{2.771196in}{4.606437in}}%
\pgfpathlineto{\pgfqpoint{2.773703in}{4.609745in}}%
\pgfpathlineto{\pgfqpoint{2.776210in}{4.610175in}}%
\pgfpathlineto{\pgfqpoint{2.781223in}{4.614004in}}%
\pgfpathlineto{\pgfqpoint{2.786236in}{4.614713in}}%
\pgfpathlineto{\pgfqpoint{2.788743in}{4.618856in}}%
\pgfpathlineto{\pgfqpoint{2.791250in}{4.619940in}}%
\pgfpathlineto{\pgfqpoint{2.793756in}{4.622782in}}%
\pgfpathlineto{\pgfqpoint{2.798770in}{4.624955in}}%
\pgfpathlineto{\pgfqpoint{2.801276in}{4.626834in}}%
\pgfpathlineto{\pgfqpoint{2.803783in}{4.627309in}}%
\pgfpathlineto{\pgfqpoint{2.806290in}{4.631287in}}%
\pgfpathlineto{\pgfqpoint{2.816316in}{4.632911in}}%
\pgfpathlineto{\pgfqpoint{2.821329in}{4.634874in}}%
\pgfpathlineto{\pgfqpoint{2.823836in}{4.640529in}}%
\pgfpathlineto{\pgfqpoint{2.831356in}{4.645947in}}%
\pgfpathlineto{\pgfqpoint{2.833863in}{4.646598in}}%
\pgfpathlineto{\pgfqpoint{2.836369in}{4.651603in}}%
\pgfpathlineto{\pgfqpoint{2.841383in}{4.652753in}}%
\pgfpathlineto{\pgfqpoint{2.843889in}{4.652854in}}%
\pgfpathlineto{\pgfqpoint{2.848903in}{4.657024in}}%
\pgfpathlineto{\pgfqpoint{2.853916in}{4.661220in}}%
\pgfpathlineto{\pgfqpoint{2.861436in}{4.662753in}}%
\pgfpathlineto{\pgfqpoint{2.863943in}{4.662958in}}%
\pgfpathlineto{\pgfqpoint{2.868956in}{4.667869in}}%
\pgfpathlineto{\pgfqpoint{2.876476in}{4.670116in}}%
\pgfpathlineto{\pgfqpoint{2.878983in}{4.673946in}}%
\pgfpathlineto{\pgfqpoint{2.881489in}{4.674279in}}%
\pgfpathlineto{\pgfqpoint{2.889009in}{4.681486in}}%
\pgfpathlineto{\pgfqpoint{2.891516in}{4.685847in}}%
\pgfpathlineto{\pgfqpoint{2.894022in}{4.687350in}}%
\pgfpathlineto{\pgfqpoint{2.901542in}{4.694512in}}%
\pgfpathlineto{\pgfqpoint{2.911569in}{4.696095in}}%
\pgfpathlineto{\pgfqpoint{2.921596in}{4.702446in}}%
\pgfpathlineto{\pgfqpoint{2.924102in}{4.706443in}}%
\pgfpathlineto{\pgfqpoint{2.929116in}{4.707007in}}%
\pgfpathlineto{\pgfqpoint{2.931622in}{4.708199in}}%
\pgfpathlineto{\pgfqpoint{2.936636in}{4.712834in}}%
\pgfpathlineto{\pgfqpoint{2.941649in}{4.716143in}}%
\pgfpathlineto{\pgfqpoint{2.946662in}{4.716569in}}%
\pgfpathlineto{\pgfqpoint{2.949169in}{4.719302in}}%
\pgfpathlineto{\pgfqpoint{2.951676in}{4.719459in}}%
\pgfpathlineto{\pgfqpoint{2.956689in}{4.722868in}}%
\pgfpathlineto{\pgfqpoint{2.959195in}{4.723895in}}%
\pgfpathlineto{\pgfqpoint{2.961702in}{4.732706in}}%
\pgfpathlineto{\pgfqpoint{2.969222in}{4.735945in}}%
\pgfpathlineto{\pgfqpoint{2.971729in}{4.745465in}}%
\pgfpathlineto{\pgfqpoint{2.974235in}{4.750006in}}%
\pgfpathlineto{\pgfqpoint{2.976742in}{4.751737in}}%
\pgfpathlineto{\pgfqpoint{2.979249in}{4.757607in}}%
\pgfpathlineto{\pgfqpoint{2.984262in}{4.757988in}}%
\pgfpathlineto{\pgfqpoint{2.986769in}{4.760759in}}%
\pgfpathlineto{\pgfqpoint{2.989275in}{4.765468in}}%
\pgfpathlineto{\pgfqpoint{3.001809in}{4.771176in}}%
\pgfpathlineto{\pgfqpoint{3.006822in}{4.773139in}}%
\pgfpathlineto{\pgfqpoint{3.009329in}{4.773363in}}%
\pgfpathlineto{\pgfqpoint{3.011835in}{4.776992in}}%
\pgfpathlineto{\pgfqpoint{3.021862in}{4.782567in}}%
\pgfpathlineto{\pgfqpoint{3.024369in}{4.786791in}}%
\pgfpathlineto{\pgfqpoint{3.034395in}{4.789494in}}%
\pgfpathlineto{\pgfqpoint{3.039408in}{4.795513in}}%
\pgfpathlineto{\pgfqpoint{3.041915in}{4.796333in}}%
\pgfpathlineto{\pgfqpoint{3.044422in}{4.799222in}}%
\pgfpathlineto{\pgfqpoint{3.046928in}{4.800266in}}%
\pgfpathlineto{\pgfqpoint{3.049435in}{4.802773in}}%
\pgfpathlineto{\pgfqpoint{3.054448in}{4.803363in}}%
\pgfpathlineto{\pgfqpoint{3.056955in}{4.805252in}}%
\pgfpathlineto{\pgfqpoint{3.059462in}{4.814286in}}%
\pgfpathlineto{\pgfqpoint{3.061968in}{4.817589in}}%
\pgfpathlineto{\pgfqpoint{3.064475in}{4.817636in}}%
\pgfpathlineto{\pgfqpoint{3.066982in}{4.820210in}}%
\pgfpathlineto{\pgfqpoint{3.069488in}{4.820367in}}%
\pgfpathlineto{\pgfqpoint{3.094555in}{4.835180in}}%
\pgfpathlineto{\pgfqpoint{3.097061in}{4.841563in}}%
\pgfpathlineto{\pgfqpoint{3.104581in}{4.842319in}}%
\pgfpathlineto{\pgfqpoint{3.109595in}{4.851402in}}%
\pgfpathlineto{\pgfqpoint{3.114608in}{4.853140in}}%
\pgfpathlineto{\pgfqpoint{3.117115in}{4.856872in}}%
\pgfpathlineto{\pgfqpoint{3.122128in}{4.857014in}}%
\pgfpathlineto{\pgfqpoint{3.124635in}{4.858716in}}%
\pgfpathlineto{\pgfqpoint{3.127141in}{4.864688in}}%
\pgfpathlineto{\pgfqpoint{3.137168in}{4.873897in}}%
\pgfpathlineto{\pgfqpoint{3.139675in}{4.880965in}}%
\pgfpathlineto{\pgfqpoint{3.144688in}{4.884122in}}%
\pgfpathlineto{\pgfqpoint{3.147195in}{4.885283in}}%
\pgfpathlineto{\pgfqpoint{3.152208in}{4.892417in}}%
\pgfpathlineto{\pgfqpoint{3.154715in}{4.893822in}}%
\pgfpathlineto{\pgfqpoint{3.157221in}{4.896576in}}%
\pgfpathlineto{\pgfqpoint{3.162234in}{4.906872in}}%
\pgfpathlineto{\pgfqpoint{3.167248in}{4.908224in}}%
\pgfpathlineto{\pgfqpoint{3.172261in}{4.914239in}}%
\pgfpathlineto{\pgfqpoint{3.177274in}{4.916455in}}%
\pgfpathlineto{\pgfqpoint{3.197328in}{4.923129in}}%
\pgfpathlineto{\pgfqpoint{3.199834in}{4.935413in}}%
\pgfpathlineto{\pgfqpoint{3.202341in}{4.935424in}}%
\pgfpathlineto{\pgfqpoint{3.204848in}{4.941382in}}%
\pgfpathlineto{\pgfqpoint{3.207354in}{4.941452in}}%
\pgfpathlineto{\pgfqpoint{3.227408in}{4.957235in}}%
\pgfpathlineto{\pgfqpoint{3.229914in}{4.957287in}}%
\pgfpathlineto{\pgfqpoint{3.232421in}{4.959270in}}%
\pgfpathlineto{\pgfqpoint{3.234927in}{4.959422in}}%
\pgfpathlineto{\pgfqpoint{3.247461in}{4.969993in}}%
\pgfpathlineto{\pgfqpoint{3.249967in}{4.975338in}}%
\pgfpathlineto{\pgfqpoint{3.252474in}{4.975769in}}%
\pgfpathlineto{\pgfqpoint{3.254981in}{4.979488in}}%
\pgfpathlineto{\pgfqpoint{3.267514in}{4.983497in}}%
\pgfpathlineto{\pgfqpoint{3.280047in}{4.989683in}}%
\pgfpathlineto{\pgfqpoint{3.285061in}{4.990165in}}%
\pgfpathlineto{\pgfqpoint{3.287567in}{4.998730in}}%
\pgfpathlineto{\pgfqpoint{3.290074in}{5.003126in}}%
\pgfpathlineto{\pgfqpoint{3.292581in}{5.004117in}}%
\pgfpathlineto{\pgfqpoint{3.295087in}{5.008369in}}%
\pgfpathlineto{\pgfqpoint{3.297594in}{5.008516in}}%
\pgfpathlineto{\pgfqpoint{3.300100in}{5.010549in}}%
\pgfpathlineto{\pgfqpoint{3.302607in}{5.024663in}}%
\pgfpathlineto{\pgfqpoint{3.307620in}{5.029118in}}%
\pgfpathlineto{\pgfqpoint{3.310127in}{5.036361in}}%
\pgfpathlineto{\pgfqpoint{3.317647in}{5.039932in}}%
\pgfpathlineto{\pgfqpoint{3.320154in}{5.040163in}}%
\pgfpathlineto{\pgfqpoint{3.327674in}{5.046682in}}%
\pgfpathlineto{\pgfqpoint{3.340207in}{5.052637in}}%
\pgfpathlineto{\pgfqpoint{3.345220in}{5.062581in}}%
\pgfpathlineto{\pgfqpoint{3.350234in}{5.078245in}}%
\pgfpathlineto{\pgfqpoint{3.357754in}{5.084982in}}%
\pgfpathlineto{\pgfqpoint{3.360260in}{5.085258in}}%
\pgfpathlineto{\pgfqpoint{3.362767in}{5.093776in}}%
\pgfpathlineto{\pgfqpoint{3.365273in}{5.094355in}}%
\pgfpathlineto{\pgfqpoint{3.367780in}{5.100337in}}%
\pgfpathlineto{\pgfqpoint{3.370287in}{5.101548in}}%
\pgfpathlineto{\pgfqpoint{3.377807in}{5.111131in}}%
\pgfpathlineto{\pgfqpoint{3.380313in}{5.115286in}}%
\pgfpathlineto{\pgfqpoint{3.382820in}{5.115437in}}%
\pgfpathlineto{\pgfqpoint{3.385327in}{5.120399in}}%
\pgfpathlineto{\pgfqpoint{3.390340in}{5.122153in}}%
\pgfpathlineto{\pgfqpoint{3.397860in}{5.124389in}}%
\pgfpathlineto{\pgfqpoint{3.400367in}{5.127279in}}%
\pgfpathlineto{\pgfqpoint{3.402873in}{5.132723in}}%
\pgfpathlineto{\pgfqpoint{3.405380in}{5.133109in}}%
\pgfpathlineto{\pgfqpoint{3.410393in}{5.144956in}}%
\pgfpathlineto{\pgfqpoint{3.412900in}{5.145067in}}%
\pgfpathlineto{\pgfqpoint{3.415407in}{5.147168in}}%
\pgfpathlineto{\pgfqpoint{3.430447in}{5.148050in}}%
\pgfpathlineto{\pgfqpoint{3.440473in}{5.152811in}}%
\pgfpathlineto{\pgfqpoint{3.442980in}{5.155823in}}%
\pgfpathlineto{\pgfqpoint{3.445486in}{5.156021in}}%
\pgfpathlineto{\pgfqpoint{3.450500in}{5.160120in}}%
\pgfpathlineto{\pgfqpoint{3.463033in}{5.166012in}}%
\pgfpathlineto{\pgfqpoint{3.465540in}{5.167192in}}%
\pgfpathlineto{\pgfqpoint{3.468046in}{5.178700in}}%
\pgfpathlineto{\pgfqpoint{3.470553in}{5.179932in}}%
\pgfpathlineto{\pgfqpoint{3.473060in}{5.183894in}}%
\pgfpathlineto{\pgfqpoint{3.475566in}{5.184987in}}%
\pgfpathlineto{\pgfqpoint{3.480580in}{5.188429in}}%
\pgfpathlineto{\pgfqpoint{3.483086in}{5.188814in}}%
\pgfpathlineto{\pgfqpoint{3.488100in}{5.196725in}}%
\pgfpathlineto{\pgfqpoint{3.490606in}{5.204891in}}%
\pgfpathlineto{\pgfqpoint{3.493113in}{5.205829in}}%
\pgfpathlineto{\pgfqpoint{3.495620in}{5.212911in}}%
\pgfpathlineto{\pgfqpoint{3.498126in}{5.213483in}}%
\pgfpathlineto{\pgfqpoint{3.503139in}{5.216529in}}%
\pgfpathlineto{\pgfqpoint{3.518179in}{5.255681in}}%
\pgfpathlineto{\pgfqpoint{3.520686in}{5.256899in}}%
\pgfpathlineto{\pgfqpoint{3.523193in}{5.259819in}}%
\pgfpathlineto{\pgfqpoint{3.525699in}{5.260363in}}%
\pgfpathlineto{\pgfqpoint{3.530713in}{5.273843in}}%
\pgfpathlineto{\pgfqpoint{3.535726in}{5.276398in}}%
\pgfpathlineto{\pgfqpoint{3.538233in}{5.278848in}}%
\pgfpathlineto{\pgfqpoint{3.540739in}{5.278945in}}%
\pgfpathlineto{\pgfqpoint{3.545753in}{5.291724in}}%
\pgfpathlineto{\pgfqpoint{3.548259in}{5.299234in}}%
\pgfpathlineto{\pgfqpoint{3.550766in}{5.299294in}}%
\pgfpathlineto{\pgfqpoint{3.553273in}{5.303445in}}%
\pgfpathlineto{\pgfqpoint{3.555779in}{5.305275in}}%
\pgfpathlineto{\pgfqpoint{3.555779in}{5.305275in}}%
\pgfusepath{stroke}%
\end{pgfscope}%
\begin{pgfscope}%
\pgfpathrectangle{\pgfqpoint{0.708220in}{3.210823in}}{\pgfqpoint{5.013309in}{2.094453in}}%
\pgfusepath{clip}%
\pgfsetbuttcap%
\pgfsetroundjoin%
\pgfsetlinewidth{1.003750pt}%
\definecolor{currentstroke}{rgb}{0.564706,0.564706,1.000000}%
\pgfsetstrokecolor{currentstroke}%
\pgfsetdash{{1.000000pt}{1.650000pt}}{0.000000pt}%
\pgfpathmoveto{\pgfqpoint{0.708220in}{3.361771in}}%
\pgfpathlineto{\pgfqpoint{0.710727in}{3.396401in}}%
\pgfpathlineto{\pgfqpoint{0.713233in}{3.401186in}}%
\pgfpathlineto{\pgfqpoint{0.718246in}{3.445191in}}%
\pgfpathlineto{\pgfqpoint{0.725766in}{3.462441in}}%
\pgfpathlineto{\pgfqpoint{0.728273in}{3.466881in}}%
\pgfpathlineto{\pgfqpoint{0.730780in}{3.473730in}}%
\pgfpathlineto{\pgfqpoint{0.738300in}{3.517917in}}%
\pgfpathlineto{\pgfqpoint{0.740806in}{3.526347in}}%
\pgfpathlineto{\pgfqpoint{0.743313in}{3.526587in}}%
\pgfpathlineto{\pgfqpoint{0.745820in}{3.533139in}}%
\pgfpathlineto{\pgfqpoint{0.748326in}{3.548108in}}%
\pgfpathlineto{\pgfqpoint{0.750833in}{3.549229in}}%
\pgfpathlineto{\pgfqpoint{0.753340in}{3.554315in}}%
\pgfpathlineto{\pgfqpoint{0.758353in}{3.570022in}}%
\pgfpathlineto{\pgfqpoint{0.763366in}{3.573248in}}%
\pgfpathlineto{\pgfqpoint{0.765873in}{3.584642in}}%
\pgfpathlineto{\pgfqpoint{0.768380in}{3.589117in}}%
\pgfpathlineto{\pgfqpoint{0.770886in}{3.590978in}}%
\pgfpathlineto{\pgfqpoint{0.775900in}{3.605080in}}%
\pgfpathlineto{\pgfqpoint{0.778406in}{3.605160in}}%
\pgfpathlineto{\pgfqpoint{0.780913in}{3.609856in}}%
\pgfpathlineto{\pgfqpoint{0.783419in}{3.628796in}}%
\pgfpathlineto{\pgfqpoint{0.788433in}{3.632548in}}%
\pgfpathlineto{\pgfqpoint{0.798459in}{3.656931in}}%
\pgfpathlineto{\pgfqpoint{0.800966in}{3.660588in}}%
\pgfpathlineto{\pgfqpoint{0.810993in}{3.666910in}}%
\pgfpathlineto{\pgfqpoint{0.816006in}{3.678972in}}%
\pgfpathlineto{\pgfqpoint{0.818513in}{3.693770in}}%
\pgfpathlineto{\pgfqpoint{0.826033in}{3.696359in}}%
\pgfpathlineto{\pgfqpoint{0.833553in}{3.712866in}}%
\pgfpathlineto{\pgfqpoint{0.838566in}{3.714523in}}%
\pgfpathlineto{\pgfqpoint{0.841073in}{3.730700in}}%
\pgfpathlineto{\pgfqpoint{0.843579in}{3.737698in}}%
\pgfpathlineto{\pgfqpoint{0.846086in}{3.748415in}}%
\pgfpathlineto{\pgfqpoint{0.851099in}{3.752070in}}%
\pgfpathlineto{\pgfqpoint{0.853606in}{3.752352in}}%
\pgfpathlineto{\pgfqpoint{0.858619in}{3.756661in}}%
\pgfpathlineto{\pgfqpoint{0.866139in}{3.761362in}}%
\pgfpathlineto{\pgfqpoint{0.868646in}{3.772655in}}%
\pgfpathlineto{\pgfqpoint{0.873659in}{3.778497in}}%
\pgfpathlineto{\pgfqpoint{0.878672in}{3.779246in}}%
\pgfpathlineto{\pgfqpoint{0.881179in}{3.784291in}}%
\pgfpathlineto{\pgfqpoint{0.886192in}{3.784581in}}%
\pgfpathlineto{\pgfqpoint{0.888699in}{3.787015in}}%
\pgfpathlineto{\pgfqpoint{0.891206in}{3.799925in}}%
\pgfpathlineto{\pgfqpoint{0.896219in}{3.800906in}}%
\pgfpathlineto{\pgfqpoint{0.901232in}{3.802881in}}%
\pgfpathlineto{\pgfqpoint{0.911259in}{3.809718in}}%
\pgfpathlineto{\pgfqpoint{0.916272in}{3.811631in}}%
\pgfpathlineto{\pgfqpoint{0.918779in}{3.814591in}}%
\pgfpathlineto{\pgfqpoint{0.921285in}{3.815426in}}%
\pgfpathlineto{\pgfqpoint{0.923792in}{3.819615in}}%
\pgfpathlineto{\pgfqpoint{0.928805in}{3.820962in}}%
\pgfpathlineto{\pgfqpoint{0.933819in}{3.822484in}}%
\pgfpathlineto{\pgfqpoint{0.936325in}{3.822684in}}%
\pgfpathlineto{\pgfqpoint{0.938832in}{3.827733in}}%
\pgfpathlineto{\pgfqpoint{0.953872in}{3.837067in}}%
\pgfpathlineto{\pgfqpoint{0.961392in}{3.849695in}}%
\pgfpathlineto{\pgfqpoint{0.966405in}{3.852853in}}%
\pgfpathlineto{\pgfqpoint{0.968912in}{3.853680in}}%
\pgfpathlineto{\pgfqpoint{0.971419in}{3.858312in}}%
\pgfpathlineto{\pgfqpoint{0.976432in}{3.862140in}}%
\pgfpathlineto{\pgfqpoint{0.978939in}{3.866627in}}%
\pgfpathlineto{\pgfqpoint{0.988965in}{3.871057in}}%
\pgfpathlineto{\pgfqpoint{0.993978in}{3.880102in}}%
\pgfpathlineto{\pgfqpoint{1.001498in}{3.880696in}}%
\pgfpathlineto{\pgfqpoint{1.006512in}{3.883922in}}%
\pgfpathlineto{\pgfqpoint{1.011525in}{3.885762in}}%
\pgfpathlineto{\pgfqpoint{1.014032in}{3.891439in}}%
\pgfpathlineto{\pgfqpoint{1.016538in}{3.893612in}}%
\pgfpathlineto{\pgfqpoint{1.019045in}{3.894224in}}%
\pgfpathlineto{\pgfqpoint{1.024058in}{3.897256in}}%
\pgfpathlineto{\pgfqpoint{1.041605in}{3.902784in}}%
\pgfpathlineto{\pgfqpoint{1.046618in}{3.904499in}}%
\pgfpathlineto{\pgfqpoint{1.049125in}{3.907484in}}%
\pgfpathlineto{\pgfqpoint{1.056645in}{3.908536in}}%
\pgfpathlineto{\pgfqpoint{1.061658in}{3.915427in}}%
\pgfpathlineto{\pgfqpoint{1.066671in}{3.916315in}}%
\pgfpathlineto{\pgfqpoint{1.079205in}{3.923258in}}%
\pgfpathlineto{\pgfqpoint{1.084218in}{3.932413in}}%
\pgfpathlineto{\pgfqpoint{1.094245in}{3.936279in}}%
\pgfpathlineto{\pgfqpoint{1.096751in}{3.938072in}}%
\pgfpathlineto{\pgfqpoint{1.099258in}{3.938339in}}%
\pgfpathlineto{\pgfqpoint{1.119311in}{3.949550in}}%
\pgfpathlineto{\pgfqpoint{1.129338in}{3.952386in}}%
\pgfpathlineto{\pgfqpoint{1.134351in}{3.953960in}}%
\pgfpathlineto{\pgfqpoint{1.144378in}{3.955152in}}%
\pgfpathlineto{\pgfqpoint{1.149391in}{3.959025in}}%
\pgfpathlineto{\pgfqpoint{1.154404in}{3.959660in}}%
\pgfpathlineto{\pgfqpoint{1.161924in}{3.961639in}}%
\pgfpathlineto{\pgfqpoint{1.166938in}{3.962303in}}%
\pgfpathlineto{\pgfqpoint{1.169444in}{3.964565in}}%
\pgfpathlineto{\pgfqpoint{1.174458in}{3.964921in}}%
\pgfpathlineto{\pgfqpoint{1.184484in}{3.974804in}}%
\pgfpathlineto{\pgfqpoint{1.199524in}{3.977408in}}%
\pgfpathlineto{\pgfqpoint{1.202031in}{3.982505in}}%
\pgfpathlineto{\pgfqpoint{1.267204in}{3.998658in}}%
\pgfpathlineto{\pgfqpoint{1.279737in}{4.000662in}}%
\pgfpathlineto{\pgfqpoint{1.284750in}{4.007130in}}%
\pgfpathlineto{\pgfqpoint{1.297284in}{4.010408in}}%
\pgfpathlineto{\pgfqpoint{1.302297in}{4.012044in}}%
\pgfpathlineto{\pgfqpoint{1.304804in}{4.013000in}}%
\pgfpathlineto{\pgfqpoint{1.309817in}{4.017553in}}%
\pgfpathlineto{\pgfqpoint{1.312324in}{4.017972in}}%
\pgfpathlineto{\pgfqpoint{1.319844in}{4.021763in}}%
\pgfpathlineto{\pgfqpoint{1.332377in}{4.025329in}}%
\pgfpathlineto{\pgfqpoint{1.334883in}{4.026806in}}%
\pgfpathlineto{\pgfqpoint{1.342403in}{4.027656in}}%
\pgfpathlineto{\pgfqpoint{1.364963in}{4.034622in}}%
\pgfpathlineto{\pgfqpoint{1.425123in}{4.044339in}}%
\pgfpathlineto{\pgfqpoint{1.430136in}{4.050149in}}%
\pgfpathlineto{\pgfqpoint{1.445176in}{4.052621in}}%
\pgfpathlineto{\pgfqpoint{1.450190in}{4.054204in}}%
\pgfpathlineto{\pgfqpoint{1.452696in}{4.054267in}}%
\pgfpathlineto{\pgfqpoint{1.457710in}{4.056871in}}%
\pgfpathlineto{\pgfqpoint{1.462723in}{4.058165in}}%
\pgfpathlineto{\pgfqpoint{1.472749in}{4.060441in}}%
\pgfpathlineto{\pgfqpoint{1.482776in}{4.062720in}}%
\pgfpathlineto{\pgfqpoint{1.490296in}{4.063760in}}%
\pgfpathlineto{\pgfqpoint{1.492803in}{4.065875in}}%
\pgfpathlineto{\pgfqpoint{1.502829in}{4.066500in}}%
\pgfpathlineto{\pgfqpoint{1.507843in}{4.067828in}}%
\pgfpathlineto{\pgfqpoint{1.515363in}{4.068849in}}%
\pgfpathlineto{\pgfqpoint{1.520376in}{4.072487in}}%
\pgfpathlineto{\pgfqpoint{1.530402in}{4.074288in}}%
\pgfpathlineto{\pgfqpoint{1.532909in}{4.075938in}}%
\pgfpathlineto{\pgfqpoint{1.535416in}{4.079853in}}%
\pgfpathlineto{\pgfqpoint{1.542936in}{4.081898in}}%
\pgfpathlineto{\pgfqpoint{1.547949in}{4.083845in}}%
\pgfpathlineto{\pgfqpoint{1.550456in}{4.087174in}}%
\pgfpathlineto{\pgfqpoint{1.555469in}{4.088511in}}%
\pgfpathlineto{\pgfqpoint{1.562989in}{4.089495in}}%
\pgfpathlineto{\pgfqpoint{1.570509in}{4.092421in}}%
\pgfpathlineto{\pgfqpoint{1.580536in}{4.095660in}}%
\pgfpathlineto{\pgfqpoint{1.585549in}{4.097613in}}%
\pgfpathlineto{\pgfqpoint{1.598082in}{4.100873in}}%
\pgfpathlineto{\pgfqpoint{1.603095in}{4.104013in}}%
\pgfpathlineto{\pgfqpoint{1.610615in}{4.105724in}}%
\pgfpathlineto{\pgfqpoint{1.620642in}{4.109033in}}%
\pgfpathlineto{\pgfqpoint{1.630669in}{4.110614in}}%
\pgfpathlineto{\pgfqpoint{1.658242in}{4.121615in}}%
\pgfpathlineto{\pgfqpoint{1.665762in}{4.125061in}}%
\pgfpathlineto{\pgfqpoint{1.670775in}{4.125942in}}%
\pgfpathlineto{\pgfqpoint{1.678295in}{4.129977in}}%
\pgfpathlineto{\pgfqpoint{1.683308in}{4.130944in}}%
\pgfpathlineto{\pgfqpoint{1.688322in}{4.132789in}}%
\pgfpathlineto{\pgfqpoint{1.695842in}{4.136438in}}%
\pgfpathlineto{\pgfqpoint{1.698348in}{4.136613in}}%
\pgfpathlineto{\pgfqpoint{1.700855in}{4.139807in}}%
\pgfpathlineto{\pgfqpoint{1.703362in}{4.139930in}}%
\pgfpathlineto{\pgfqpoint{1.708375in}{4.143049in}}%
\pgfpathlineto{\pgfqpoint{1.713388in}{4.144345in}}%
\pgfpathlineto{\pgfqpoint{1.718402in}{4.147608in}}%
\pgfpathlineto{\pgfqpoint{1.728428in}{4.150835in}}%
\pgfpathlineto{\pgfqpoint{1.730935in}{4.153927in}}%
\pgfpathlineto{\pgfqpoint{1.743468in}{4.157072in}}%
\pgfpathlineto{\pgfqpoint{1.745975in}{4.159171in}}%
\pgfpathlineto{\pgfqpoint{1.753495in}{4.159706in}}%
\pgfpathlineto{\pgfqpoint{1.758508in}{4.165686in}}%
\pgfpathlineto{\pgfqpoint{1.761015in}{4.165811in}}%
\pgfpathlineto{\pgfqpoint{1.763521in}{4.168942in}}%
\pgfpathlineto{\pgfqpoint{1.768535in}{4.170276in}}%
\pgfpathlineto{\pgfqpoint{1.771041in}{4.174297in}}%
\pgfpathlineto{\pgfqpoint{1.776055in}{4.175775in}}%
\pgfpathlineto{\pgfqpoint{1.778561in}{4.176011in}}%
\pgfpathlineto{\pgfqpoint{1.783575in}{4.178132in}}%
\pgfpathlineto{\pgfqpoint{1.793601in}{4.180093in}}%
\pgfpathlineto{\pgfqpoint{1.798615in}{4.181472in}}%
\pgfpathlineto{\pgfqpoint{1.808641in}{4.183615in}}%
\pgfpathlineto{\pgfqpoint{1.811148in}{4.185335in}}%
\pgfpathlineto{\pgfqpoint{1.813654in}{4.189024in}}%
\pgfpathlineto{\pgfqpoint{1.823681in}{4.192378in}}%
\pgfpathlineto{\pgfqpoint{1.826188in}{4.194189in}}%
\pgfpathlineto{\pgfqpoint{1.833708in}{4.194871in}}%
\pgfpathlineto{\pgfqpoint{1.838721in}{4.197210in}}%
\pgfpathlineto{\pgfqpoint{1.841228in}{4.197290in}}%
\pgfpathlineto{\pgfqpoint{1.843734in}{4.199641in}}%
\pgfpathlineto{\pgfqpoint{1.856268in}{4.201726in}}%
\pgfpathlineto{\pgfqpoint{1.863788in}{4.203370in}}%
\pgfpathlineto{\pgfqpoint{1.883841in}{4.206252in}}%
\pgfpathlineto{\pgfqpoint{1.888854in}{4.209542in}}%
\pgfpathlineto{\pgfqpoint{1.898881in}{4.214462in}}%
\pgfpathlineto{\pgfqpoint{1.913921in}{4.216975in}}%
\pgfpathlineto{\pgfqpoint{1.921441in}{4.217210in}}%
\pgfpathlineto{\pgfqpoint{1.923947in}{4.220457in}}%
\pgfpathlineto{\pgfqpoint{1.928961in}{4.221108in}}%
\pgfpathlineto{\pgfqpoint{1.933974in}{4.222113in}}%
\pgfpathlineto{\pgfqpoint{1.944000in}{4.223527in}}%
\pgfpathlineto{\pgfqpoint{1.946507in}{4.225478in}}%
\pgfpathlineto{\pgfqpoint{1.949014in}{4.228893in}}%
\pgfpathlineto{\pgfqpoint{1.954027in}{4.230558in}}%
\pgfpathlineto{\pgfqpoint{1.956534in}{4.233689in}}%
\pgfpathlineto{\pgfqpoint{1.961547in}{4.234117in}}%
\pgfpathlineto{\pgfqpoint{1.971574in}{4.237867in}}%
\pgfpathlineto{\pgfqpoint{1.974080in}{4.237937in}}%
\pgfpathlineto{\pgfqpoint{1.979094in}{4.240345in}}%
\pgfpathlineto{\pgfqpoint{1.981600in}{4.240459in}}%
\pgfpathlineto{\pgfqpoint{1.986614in}{4.243720in}}%
\pgfpathlineto{\pgfqpoint{1.991627in}{4.246495in}}%
\pgfpathlineto{\pgfqpoint{1.996640in}{4.247815in}}%
\pgfpathlineto{\pgfqpoint{2.001654in}{4.252265in}}%
\pgfpathlineto{\pgfqpoint{2.006667in}{4.252988in}}%
\pgfpathlineto{\pgfqpoint{2.011680in}{4.254771in}}%
\pgfpathlineto{\pgfqpoint{2.014187in}{4.255080in}}%
\pgfpathlineto{\pgfqpoint{2.021707in}{4.266861in}}%
\pgfpathlineto{\pgfqpoint{2.031733in}{4.270095in}}%
\pgfpathlineto{\pgfqpoint{2.034240in}{4.270518in}}%
\pgfpathlineto{\pgfqpoint{2.041760in}{4.277349in}}%
\pgfpathlineto{\pgfqpoint{2.044267in}{4.277688in}}%
\pgfpathlineto{\pgfqpoint{2.049280in}{4.280776in}}%
\pgfpathlineto{\pgfqpoint{2.059307in}{4.282489in}}%
\pgfpathlineto{\pgfqpoint{2.079360in}{4.289062in}}%
\pgfpathlineto{\pgfqpoint{2.086880in}{4.293824in}}%
\pgfpathlineto{\pgfqpoint{2.091893in}{4.297510in}}%
\pgfpathlineto{\pgfqpoint{2.101920in}{4.299770in}}%
\pgfpathlineto{\pgfqpoint{2.106933in}{4.300695in}}%
\pgfpathlineto{\pgfqpoint{2.109440in}{4.305113in}}%
\pgfpathlineto{\pgfqpoint{2.114453in}{4.306400in}}%
\pgfpathlineto{\pgfqpoint{2.119466in}{4.307816in}}%
\pgfpathlineto{\pgfqpoint{2.121973in}{4.309168in}}%
\pgfpathlineto{\pgfqpoint{2.124480in}{4.312510in}}%
\pgfpathlineto{\pgfqpoint{2.129493in}{4.315483in}}%
\pgfpathlineto{\pgfqpoint{2.132000in}{4.317598in}}%
\pgfpathlineto{\pgfqpoint{2.144533in}{4.319181in}}%
\pgfpathlineto{\pgfqpoint{2.154559in}{4.326253in}}%
\pgfpathlineto{\pgfqpoint{2.164586in}{4.327096in}}%
\pgfpathlineto{\pgfqpoint{2.167093in}{4.330163in}}%
\pgfpathlineto{\pgfqpoint{2.172106in}{4.330691in}}%
\pgfpathlineto{\pgfqpoint{2.174613in}{4.332919in}}%
\pgfpathlineto{\pgfqpoint{2.199679in}{4.336562in}}%
\pgfpathlineto{\pgfqpoint{2.204693in}{4.337920in}}%
\pgfpathlineto{\pgfqpoint{2.212212in}{4.339694in}}%
\pgfpathlineto{\pgfqpoint{2.214719in}{4.342650in}}%
\pgfpathlineto{\pgfqpoint{2.224746in}{4.343451in}}%
\pgfpathlineto{\pgfqpoint{2.242292in}{4.348249in}}%
\pgfpathlineto{\pgfqpoint{2.247306in}{4.351868in}}%
\pgfpathlineto{\pgfqpoint{2.254826in}{4.353909in}}%
\pgfpathlineto{\pgfqpoint{2.259839in}{4.354761in}}%
\pgfpathlineto{\pgfqpoint{2.272372in}{4.359974in}}%
\pgfpathlineto{\pgfqpoint{2.277385in}{4.361355in}}%
\pgfpathlineto{\pgfqpoint{2.294932in}{4.364142in}}%
\pgfpathlineto{\pgfqpoint{2.297439in}{4.365814in}}%
\pgfpathlineto{\pgfqpoint{2.309972in}{4.368346in}}%
\pgfpathlineto{\pgfqpoint{2.312479in}{4.370033in}}%
\pgfpathlineto{\pgfqpoint{2.319999in}{4.371718in}}%
\pgfpathlineto{\pgfqpoint{2.322505in}{4.373645in}}%
\pgfpathlineto{\pgfqpoint{2.330025in}{4.374438in}}%
\pgfpathlineto{\pgfqpoint{2.350078in}{4.378792in}}%
\pgfpathlineto{\pgfqpoint{2.352585in}{4.378850in}}%
\pgfpathlineto{\pgfqpoint{2.355092in}{4.383667in}}%
\pgfpathlineto{\pgfqpoint{2.360105in}{4.385825in}}%
\pgfpathlineto{\pgfqpoint{2.370132in}{4.389417in}}%
\pgfpathlineto{\pgfqpoint{2.375145in}{4.389473in}}%
\pgfpathlineto{\pgfqpoint{2.387678in}{4.394816in}}%
\pgfpathlineto{\pgfqpoint{2.417758in}{4.404054in}}%
\pgfpathlineto{\pgfqpoint{2.422771in}{4.405014in}}%
\pgfpathlineto{\pgfqpoint{2.427785in}{4.407814in}}%
\pgfpathlineto{\pgfqpoint{2.435305in}{4.408907in}}%
\pgfpathlineto{\pgfqpoint{2.445331in}{4.411241in}}%
\pgfpathlineto{\pgfqpoint{2.450345in}{4.411718in}}%
\pgfpathlineto{\pgfqpoint{2.452851in}{4.413486in}}%
\pgfpathlineto{\pgfqpoint{2.460371in}{4.414440in}}%
\pgfpathlineto{\pgfqpoint{2.467891in}{4.416312in}}%
\pgfpathlineto{\pgfqpoint{2.485438in}{4.420966in}}%
\pgfpathlineto{\pgfqpoint{2.490451in}{4.424481in}}%
\pgfpathlineto{\pgfqpoint{2.500478in}{4.429267in}}%
\pgfpathlineto{\pgfqpoint{2.505491in}{4.432301in}}%
\pgfpathlineto{\pgfqpoint{2.540584in}{4.441157in}}%
\pgfpathlineto{\pgfqpoint{2.545598in}{4.444119in}}%
\pgfpathlineto{\pgfqpoint{2.550611in}{4.445146in}}%
\pgfpathlineto{\pgfqpoint{2.553117in}{4.445151in}}%
\pgfpathlineto{\pgfqpoint{2.555624in}{4.446358in}}%
\pgfpathlineto{\pgfqpoint{2.560637in}{4.451147in}}%
\pgfpathlineto{\pgfqpoint{2.563144in}{4.451156in}}%
\pgfpathlineto{\pgfqpoint{2.565651in}{4.456524in}}%
\pgfpathlineto{\pgfqpoint{2.573171in}{4.458974in}}%
\pgfpathlineto{\pgfqpoint{2.593224in}{4.466484in}}%
\pgfpathlineto{\pgfqpoint{2.598237in}{4.469582in}}%
\pgfpathlineto{\pgfqpoint{2.608264in}{4.471316in}}%
\pgfpathlineto{\pgfqpoint{2.613277in}{4.473331in}}%
\pgfpathlineto{\pgfqpoint{2.623304in}{4.477288in}}%
\pgfpathlineto{\pgfqpoint{2.628317in}{4.477688in}}%
\pgfpathlineto{\pgfqpoint{2.630824in}{4.480953in}}%
\pgfpathlineto{\pgfqpoint{2.658397in}{4.487563in}}%
\pgfpathlineto{\pgfqpoint{2.660904in}{4.487642in}}%
\pgfpathlineto{\pgfqpoint{2.673437in}{4.495193in}}%
\pgfpathlineto{\pgfqpoint{2.675944in}{4.495313in}}%
\pgfpathlineto{\pgfqpoint{2.680957in}{4.498236in}}%
\pgfpathlineto{\pgfqpoint{2.683464in}{4.498456in}}%
\pgfpathlineto{\pgfqpoint{2.685970in}{4.500723in}}%
\pgfpathlineto{\pgfqpoint{2.695997in}{4.502274in}}%
\pgfpathlineto{\pgfqpoint{2.701010in}{4.505442in}}%
\pgfpathlineto{\pgfqpoint{2.716050in}{4.508228in}}%
\pgfpathlineto{\pgfqpoint{2.718557in}{4.512456in}}%
\pgfpathlineto{\pgfqpoint{2.733597in}{4.518181in}}%
\pgfpathlineto{\pgfqpoint{2.738610in}{4.520263in}}%
\pgfpathlineto{\pgfqpoint{2.746130in}{4.520696in}}%
\pgfpathlineto{\pgfqpoint{2.748637in}{4.523161in}}%
\pgfpathlineto{\pgfqpoint{2.751143in}{4.530257in}}%
\pgfpathlineto{\pgfqpoint{2.756156in}{4.533323in}}%
\pgfpathlineto{\pgfqpoint{2.766183in}{4.536729in}}%
\pgfpathlineto{\pgfqpoint{2.771196in}{4.543516in}}%
\pgfpathlineto{\pgfqpoint{2.778716in}{4.544401in}}%
\pgfpathlineto{\pgfqpoint{2.788743in}{4.548715in}}%
\pgfpathlineto{\pgfqpoint{2.791250in}{4.551008in}}%
\pgfpathlineto{\pgfqpoint{2.796263in}{4.552578in}}%
\pgfpathlineto{\pgfqpoint{2.798770in}{4.555645in}}%
\pgfpathlineto{\pgfqpoint{2.803783in}{4.556815in}}%
\pgfpathlineto{\pgfqpoint{2.806290in}{4.558350in}}%
\pgfpathlineto{\pgfqpoint{2.808796in}{4.566573in}}%
\pgfpathlineto{\pgfqpoint{2.813810in}{4.568140in}}%
\pgfpathlineto{\pgfqpoint{2.816316in}{4.568589in}}%
\pgfpathlineto{\pgfqpoint{2.818823in}{4.570930in}}%
\pgfpathlineto{\pgfqpoint{2.841383in}{4.577073in}}%
\pgfpathlineto{\pgfqpoint{2.843889in}{4.580874in}}%
\pgfpathlineto{\pgfqpoint{2.889009in}{4.592374in}}%
\pgfpathlineto{\pgfqpoint{2.894022in}{4.593210in}}%
\pgfpathlineto{\pgfqpoint{2.899036in}{4.594892in}}%
\pgfpathlineto{\pgfqpoint{2.904049in}{4.595986in}}%
\pgfpathlineto{\pgfqpoint{2.906556in}{4.597618in}}%
\pgfpathlineto{\pgfqpoint{2.916582in}{4.598577in}}%
\pgfpathlineto{\pgfqpoint{2.929116in}{4.607637in}}%
\pgfpathlineto{\pgfqpoint{2.936636in}{4.608532in}}%
\pgfpathlineto{\pgfqpoint{2.944156in}{4.611317in}}%
\pgfpathlineto{\pgfqpoint{2.949169in}{4.615988in}}%
\pgfpathlineto{\pgfqpoint{2.951676in}{4.618211in}}%
\pgfpathlineto{\pgfqpoint{2.954182in}{4.618676in}}%
\pgfpathlineto{\pgfqpoint{2.959195in}{4.624618in}}%
\pgfpathlineto{\pgfqpoint{2.964209in}{4.627198in}}%
\pgfpathlineto{\pgfqpoint{2.969222in}{4.627732in}}%
\pgfpathlineto{\pgfqpoint{2.971729in}{4.632300in}}%
\pgfpathlineto{\pgfqpoint{2.981755in}{4.635184in}}%
\pgfpathlineto{\pgfqpoint{2.984262in}{4.637024in}}%
\pgfpathlineto{\pgfqpoint{2.986769in}{4.646850in}}%
\pgfpathlineto{\pgfqpoint{2.991782in}{4.656524in}}%
\pgfpathlineto{\pgfqpoint{2.996795in}{4.656895in}}%
\pgfpathlineto{\pgfqpoint{2.999302in}{4.658973in}}%
\pgfpathlineto{\pgfqpoint{3.004315in}{4.659148in}}%
\pgfpathlineto{\pgfqpoint{3.006822in}{4.661791in}}%
\pgfpathlineto{\pgfqpoint{3.009329in}{4.670407in}}%
\pgfpathlineto{\pgfqpoint{3.019355in}{4.672189in}}%
\pgfpathlineto{\pgfqpoint{3.021862in}{4.675197in}}%
\pgfpathlineto{\pgfqpoint{3.024369in}{4.675427in}}%
\pgfpathlineto{\pgfqpoint{3.026875in}{4.678464in}}%
\pgfpathlineto{\pgfqpoint{3.029382in}{4.679352in}}%
\pgfpathlineto{\pgfqpoint{3.031888in}{4.681889in}}%
\pgfpathlineto{\pgfqpoint{3.034395in}{4.687002in}}%
\pgfpathlineto{\pgfqpoint{3.036902in}{4.688091in}}%
\pgfpathlineto{\pgfqpoint{3.044422in}{4.694278in}}%
\pgfpathlineto{\pgfqpoint{3.049435in}{4.695116in}}%
\pgfpathlineto{\pgfqpoint{3.051942in}{4.695894in}}%
\pgfpathlineto{\pgfqpoint{3.054448in}{4.699724in}}%
\pgfpathlineto{\pgfqpoint{3.056955in}{4.700108in}}%
\pgfpathlineto{\pgfqpoint{3.061968in}{4.704978in}}%
\pgfpathlineto{\pgfqpoint{3.064475in}{4.707362in}}%
\pgfpathlineto{\pgfqpoint{3.069488in}{4.723961in}}%
\pgfpathlineto{\pgfqpoint{3.071995in}{4.725272in}}%
\pgfpathlineto{\pgfqpoint{3.074502in}{4.731946in}}%
\pgfpathlineto{\pgfqpoint{3.084528in}{4.733813in}}%
\pgfpathlineto{\pgfqpoint{3.087035in}{4.736264in}}%
\pgfpathlineto{\pgfqpoint{3.089542in}{4.736426in}}%
\pgfpathlineto{\pgfqpoint{3.097061in}{4.741211in}}%
\pgfpathlineto{\pgfqpoint{3.107088in}{4.744190in}}%
\pgfpathlineto{\pgfqpoint{3.109595in}{4.748404in}}%
\pgfpathlineto{\pgfqpoint{3.114608in}{4.750669in}}%
\pgfpathlineto{\pgfqpoint{3.117115in}{4.750823in}}%
\pgfpathlineto{\pgfqpoint{3.122128in}{4.757579in}}%
\pgfpathlineto{\pgfqpoint{3.124635in}{4.757964in}}%
\pgfpathlineto{\pgfqpoint{3.127141in}{4.765527in}}%
\pgfpathlineto{\pgfqpoint{3.132155in}{4.773073in}}%
\pgfpathlineto{\pgfqpoint{3.137168in}{4.774384in}}%
\pgfpathlineto{\pgfqpoint{3.139675in}{4.777308in}}%
\pgfpathlineto{\pgfqpoint{3.144688in}{4.778311in}}%
\pgfpathlineto{\pgfqpoint{3.157221in}{4.781387in}}%
\pgfpathlineto{\pgfqpoint{3.162234in}{4.787157in}}%
\pgfpathlineto{\pgfqpoint{3.167248in}{4.790213in}}%
\pgfpathlineto{\pgfqpoint{3.169754in}{4.793112in}}%
\pgfpathlineto{\pgfqpoint{3.179781in}{4.794347in}}%
\pgfpathlineto{\pgfqpoint{3.189808in}{4.799607in}}%
\pgfpathlineto{\pgfqpoint{3.192314in}{4.803020in}}%
\pgfpathlineto{\pgfqpoint{3.199834in}{4.807069in}}%
\pgfpathlineto{\pgfqpoint{3.202341in}{4.810720in}}%
\pgfpathlineto{\pgfqpoint{3.207354in}{4.811272in}}%
\pgfpathlineto{\pgfqpoint{3.209861in}{4.817540in}}%
\pgfpathlineto{\pgfqpoint{3.214874in}{4.818382in}}%
\pgfpathlineto{\pgfqpoint{3.217381in}{4.821814in}}%
\pgfpathlineto{\pgfqpoint{3.219888in}{4.822546in}}%
\pgfpathlineto{\pgfqpoint{3.222394in}{4.825564in}}%
\pgfpathlineto{\pgfqpoint{3.224901in}{4.825727in}}%
\pgfpathlineto{\pgfqpoint{3.227408in}{4.831707in}}%
\pgfpathlineto{\pgfqpoint{3.229914in}{4.831747in}}%
\pgfpathlineto{\pgfqpoint{3.232421in}{4.838133in}}%
\pgfpathlineto{\pgfqpoint{3.237434in}{4.840344in}}%
\pgfpathlineto{\pgfqpoint{3.242447in}{4.845172in}}%
\pgfpathlineto{\pgfqpoint{3.247461in}{4.846323in}}%
\pgfpathlineto{\pgfqpoint{3.259994in}{4.854841in}}%
\pgfpathlineto{\pgfqpoint{3.262501in}{4.859620in}}%
\pgfpathlineto{\pgfqpoint{3.265007in}{4.861765in}}%
\pgfpathlineto{\pgfqpoint{3.267514in}{4.869728in}}%
\pgfpathlineto{\pgfqpoint{3.270021in}{4.870102in}}%
\pgfpathlineto{\pgfqpoint{3.275034in}{4.874728in}}%
\pgfpathlineto{\pgfqpoint{3.280047in}{4.876346in}}%
\pgfpathlineto{\pgfqpoint{3.282554in}{4.883589in}}%
\pgfpathlineto{\pgfqpoint{3.285061in}{4.886356in}}%
\pgfpathlineto{\pgfqpoint{3.287567in}{4.886592in}}%
\pgfpathlineto{\pgfqpoint{3.297594in}{4.895910in}}%
\pgfpathlineto{\pgfqpoint{3.300100in}{4.898037in}}%
\pgfpathlineto{\pgfqpoint{3.305114in}{4.903780in}}%
\pgfpathlineto{\pgfqpoint{3.307620in}{4.906753in}}%
\pgfpathlineto{\pgfqpoint{3.310127in}{4.907776in}}%
\pgfpathlineto{\pgfqpoint{3.317647in}{4.920683in}}%
\pgfpathlineto{\pgfqpoint{3.320154in}{4.922608in}}%
\pgfpathlineto{\pgfqpoint{3.322660in}{4.927232in}}%
\pgfpathlineto{\pgfqpoint{3.325167in}{4.927554in}}%
\pgfpathlineto{\pgfqpoint{3.337700in}{4.935208in}}%
\pgfpathlineto{\pgfqpoint{3.340207in}{4.939921in}}%
\pgfpathlineto{\pgfqpoint{3.350234in}{4.941782in}}%
\pgfpathlineto{\pgfqpoint{3.352740in}{4.947063in}}%
\pgfpathlineto{\pgfqpoint{3.360260in}{4.954382in}}%
\pgfpathlineto{\pgfqpoint{3.382820in}{4.960135in}}%
\pgfpathlineto{\pgfqpoint{3.390340in}{4.961569in}}%
\pgfpathlineto{\pgfqpoint{3.392847in}{4.963245in}}%
\pgfpathlineto{\pgfqpoint{3.400367in}{4.965201in}}%
\pgfpathlineto{\pgfqpoint{3.410393in}{4.969730in}}%
\pgfpathlineto{\pgfqpoint{3.412900in}{4.974990in}}%
\pgfpathlineto{\pgfqpoint{3.415407in}{4.974999in}}%
\pgfpathlineto{\pgfqpoint{3.420420in}{4.977844in}}%
\pgfpathlineto{\pgfqpoint{3.422927in}{4.980098in}}%
\pgfpathlineto{\pgfqpoint{3.435460in}{4.999995in}}%
\pgfpathlineto{\pgfqpoint{3.437966in}{5.000789in}}%
\pgfpathlineto{\pgfqpoint{3.440473in}{5.021350in}}%
\pgfpathlineto{\pgfqpoint{3.442980in}{5.025022in}}%
\pgfpathlineto{\pgfqpoint{3.445486in}{5.025393in}}%
\pgfpathlineto{\pgfqpoint{3.447993in}{5.028575in}}%
\pgfpathlineto{\pgfqpoint{3.453006in}{5.036576in}}%
\pgfpathlineto{\pgfqpoint{3.455513in}{5.039375in}}%
\pgfpathlineto{\pgfqpoint{3.458020in}{5.040302in}}%
\pgfpathlineto{\pgfqpoint{3.465540in}{5.046645in}}%
\pgfpathlineto{\pgfqpoint{3.470553in}{5.055697in}}%
\pgfpathlineto{\pgfqpoint{3.475566in}{5.058813in}}%
\pgfpathlineto{\pgfqpoint{3.478073in}{5.066035in}}%
\pgfpathlineto{\pgfqpoint{3.480580in}{5.068097in}}%
\pgfpathlineto{\pgfqpoint{3.483086in}{5.068602in}}%
\pgfpathlineto{\pgfqpoint{3.485593in}{5.070442in}}%
\pgfpathlineto{\pgfqpoint{3.488100in}{5.076162in}}%
\pgfpathlineto{\pgfqpoint{3.490606in}{5.077894in}}%
\pgfpathlineto{\pgfqpoint{3.498126in}{5.091043in}}%
\pgfpathlineto{\pgfqpoint{3.505646in}{5.092256in}}%
\pgfpathlineto{\pgfqpoint{3.510659in}{5.103439in}}%
\pgfpathlineto{\pgfqpoint{3.513166in}{5.108940in}}%
\pgfpathlineto{\pgfqpoint{3.518179in}{5.111312in}}%
\pgfpathlineto{\pgfqpoint{3.533219in}{5.136896in}}%
\pgfpathlineto{\pgfqpoint{3.535726in}{5.137548in}}%
\pgfpathlineto{\pgfqpoint{3.538233in}{5.156362in}}%
\pgfpathlineto{\pgfqpoint{3.540739in}{5.156603in}}%
\pgfpathlineto{\pgfqpoint{3.543246in}{5.158467in}}%
\pgfpathlineto{\pgfqpoint{3.548259in}{5.167790in}}%
\pgfpathlineto{\pgfqpoint{3.553273in}{5.172490in}}%
\pgfpathlineto{\pgfqpoint{3.555779in}{5.173866in}}%
\pgfpathlineto{\pgfqpoint{3.560793in}{5.180751in}}%
\pgfpathlineto{\pgfqpoint{3.568313in}{5.181741in}}%
\pgfpathlineto{\pgfqpoint{3.570819in}{5.182943in}}%
\pgfpathlineto{\pgfqpoint{3.575832in}{5.196139in}}%
\pgfpathlineto{\pgfqpoint{3.578339in}{5.210385in}}%
\pgfpathlineto{\pgfqpoint{3.583352in}{5.214701in}}%
\pgfpathlineto{\pgfqpoint{3.585859in}{5.215375in}}%
\pgfpathlineto{\pgfqpoint{3.588366in}{5.236623in}}%
\pgfpathlineto{\pgfqpoint{3.593379in}{5.242300in}}%
\pgfpathlineto{\pgfqpoint{3.598392in}{5.249400in}}%
\pgfpathlineto{\pgfqpoint{3.603406in}{5.271240in}}%
\pgfpathlineto{\pgfqpoint{3.605912in}{5.274946in}}%
\pgfpathlineto{\pgfqpoint{3.608419in}{5.305275in}}%
\pgfpathlineto{\pgfqpoint{3.608419in}{5.305275in}}%
\pgfusepath{stroke}%
\end{pgfscope}%
\begin{pgfscope}%
\pgfpathrectangle{\pgfqpoint{0.708220in}{3.210823in}}{\pgfqpoint{5.013309in}{2.094453in}}%
\pgfusepath{clip}%
\pgfsetbuttcap%
\pgfsetroundjoin%
\pgfsetlinewidth{1.003750pt}%
\definecolor{currentstroke}{rgb}{0.564706,0.564706,1.000000}%
\pgfsetstrokecolor{currentstroke}%
\pgfsetdash{{3.700000pt}{1.600000pt}}{0.000000pt}%
\pgfpathmoveto{\pgfqpoint{0.708220in}{3.439833in}}%
\pgfpathlineto{\pgfqpoint{0.710727in}{3.477329in}}%
\pgfpathlineto{\pgfqpoint{0.713233in}{3.479727in}}%
\pgfpathlineto{\pgfqpoint{0.715740in}{3.497808in}}%
\pgfpathlineto{\pgfqpoint{0.718246in}{3.503319in}}%
\pgfpathlineto{\pgfqpoint{0.720753in}{3.505914in}}%
\pgfpathlineto{\pgfqpoint{0.723260in}{3.517673in}}%
\pgfpathlineto{\pgfqpoint{0.725766in}{3.521163in}}%
\pgfpathlineto{\pgfqpoint{0.728273in}{3.531068in}}%
\pgfpathlineto{\pgfqpoint{0.730780in}{3.549673in}}%
\pgfpathlineto{\pgfqpoint{0.735793in}{3.550877in}}%
\pgfpathlineto{\pgfqpoint{0.738300in}{3.555013in}}%
\pgfpathlineto{\pgfqpoint{0.740806in}{3.555265in}}%
\pgfpathlineto{\pgfqpoint{0.745820in}{3.559660in}}%
\pgfpathlineto{\pgfqpoint{0.748326in}{3.564312in}}%
\pgfpathlineto{\pgfqpoint{0.753340in}{3.578473in}}%
\pgfpathlineto{\pgfqpoint{0.755846in}{3.582015in}}%
\pgfpathlineto{\pgfqpoint{0.760860in}{3.597059in}}%
\pgfpathlineto{\pgfqpoint{0.763366in}{3.598158in}}%
\pgfpathlineto{\pgfqpoint{0.765873in}{3.600502in}}%
\pgfpathlineto{\pgfqpoint{0.768380in}{3.612209in}}%
\pgfpathlineto{\pgfqpoint{0.770886in}{3.618375in}}%
\pgfpathlineto{\pgfqpoint{0.773393in}{3.618541in}}%
\pgfpathlineto{\pgfqpoint{0.775900in}{3.620543in}}%
\pgfpathlineto{\pgfqpoint{0.778406in}{3.629431in}}%
\pgfpathlineto{\pgfqpoint{0.783419in}{3.636373in}}%
\pgfpathlineto{\pgfqpoint{0.785926in}{3.643625in}}%
\pgfpathlineto{\pgfqpoint{0.788433in}{3.647413in}}%
\pgfpathlineto{\pgfqpoint{0.790939in}{3.657161in}}%
\pgfpathlineto{\pgfqpoint{0.793446in}{3.658592in}}%
\pgfpathlineto{\pgfqpoint{0.795953in}{3.658670in}}%
\pgfpathlineto{\pgfqpoint{0.798459in}{3.661646in}}%
\pgfpathlineto{\pgfqpoint{0.800966in}{3.666542in}}%
\pgfpathlineto{\pgfqpoint{0.803473in}{3.676255in}}%
\pgfpathlineto{\pgfqpoint{0.805979in}{3.676693in}}%
\pgfpathlineto{\pgfqpoint{0.808486in}{3.680586in}}%
\pgfpathlineto{\pgfqpoint{0.810993in}{3.680948in}}%
\pgfpathlineto{\pgfqpoint{0.813499in}{3.687874in}}%
\pgfpathlineto{\pgfqpoint{0.816006in}{3.691430in}}%
\pgfpathlineto{\pgfqpoint{0.818513in}{3.692571in}}%
\pgfpathlineto{\pgfqpoint{0.823526in}{3.707746in}}%
\pgfpathlineto{\pgfqpoint{0.826033in}{3.708370in}}%
\pgfpathlineto{\pgfqpoint{0.831046in}{3.716183in}}%
\pgfpathlineto{\pgfqpoint{0.833553in}{3.723292in}}%
\pgfpathlineto{\pgfqpoint{0.843579in}{3.728626in}}%
\pgfpathlineto{\pgfqpoint{0.846086in}{3.736648in}}%
\pgfpathlineto{\pgfqpoint{0.848593in}{3.739972in}}%
\pgfpathlineto{\pgfqpoint{0.851099in}{3.746455in}}%
\pgfpathlineto{\pgfqpoint{0.858619in}{3.752432in}}%
\pgfpathlineto{\pgfqpoint{0.861126in}{3.752906in}}%
\pgfpathlineto{\pgfqpoint{0.863632in}{3.756750in}}%
\pgfpathlineto{\pgfqpoint{0.868646in}{3.757590in}}%
\pgfpathlineto{\pgfqpoint{0.871152in}{3.759437in}}%
\pgfpathlineto{\pgfqpoint{0.873659in}{3.765606in}}%
\pgfpathlineto{\pgfqpoint{0.876166in}{3.766364in}}%
\pgfpathlineto{\pgfqpoint{0.881179in}{3.774633in}}%
\pgfpathlineto{\pgfqpoint{0.883686in}{3.774880in}}%
\pgfpathlineto{\pgfqpoint{0.888699in}{3.785182in}}%
\pgfpathlineto{\pgfqpoint{0.893712in}{3.786293in}}%
\pgfpathlineto{\pgfqpoint{0.898726in}{3.795020in}}%
\pgfpathlineto{\pgfqpoint{0.901232in}{3.795021in}}%
\pgfpathlineto{\pgfqpoint{0.906246in}{3.796840in}}%
\pgfpathlineto{\pgfqpoint{0.911259in}{3.797807in}}%
\pgfpathlineto{\pgfqpoint{0.913766in}{3.800741in}}%
\pgfpathlineto{\pgfqpoint{0.916272in}{3.801187in}}%
\pgfpathlineto{\pgfqpoint{0.918779in}{3.808295in}}%
\pgfpathlineto{\pgfqpoint{0.926299in}{3.809592in}}%
\pgfpathlineto{\pgfqpoint{0.928805in}{3.811369in}}%
\pgfpathlineto{\pgfqpoint{0.931312in}{3.815427in}}%
\pgfpathlineto{\pgfqpoint{0.956379in}{3.826439in}}%
\pgfpathlineto{\pgfqpoint{0.961392in}{3.826665in}}%
\pgfpathlineto{\pgfqpoint{0.963899in}{3.830463in}}%
\pgfpathlineto{\pgfqpoint{0.968912in}{3.831658in}}%
\pgfpathlineto{\pgfqpoint{0.976432in}{3.832484in}}%
\pgfpathlineto{\pgfqpoint{0.978939in}{3.833870in}}%
\pgfpathlineto{\pgfqpoint{0.983952in}{3.834280in}}%
\pgfpathlineto{\pgfqpoint{0.991472in}{3.837231in}}%
\pgfpathlineto{\pgfqpoint{0.993978in}{3.841584in}}%
\pgfpathlineto{\pgfqpoint{0.996485in}{3.843062in}}%
\pgfpathlineto{\pgfqpoint{0.998992in}{3.848413in}}%
\pgfpathlineto{\pgfqpoint{1.004005in}{3.848939in}}%
\pgfpathlineto{\pgfqpoint{1.009018in}{3.850585in}}%
\pgfpathlineto{\pgfqpoint{1.011525in}{3.850943in}}%
\pgfpathlineto{\pgfqpoint{1.016538in}{3.855963in}}%
\pgfpathlineto{\pgfqpoint{1.021552in}{3.856416in}}%
\pgfpathlineto{\pgfqpoint{1.029072in}{3.865403in}}%
\pgfpathlineto{\pgfqpoint{1.034085in}{3.866849in}}%
\pgfpathlineto{\pgfqpoint{1.036592in}{3.866883in}}%
\pgfpathlineto{\pgfqpoint{1.039098in}{3.871017in}}%
\pgfpathlineto{\pgfqpoint{1.041605in}{3.871193in}}%
\pgfpathlineto{\pgfqpoint{1.044112in}{3.874680in}}%
\pgfpathlineto{\pgfqpoint{1.054138in}{3.877613in}}%
\pgfpathlineto{\pgfqpoint{1.056645in}{3.880195in}}%
\pgfpathlineto{\pgfqpoint{1.059151in}{3.880304in}}%
\pgfpathlineto{\pgfqpoint{1.061658in}{3.883016in}}%
\pgfpathlineto{\pgfqpoint{1.066671in}{3.883804in}}%
\pgfpathlineto{\pgfqpoint{1.071685in}{3.890435in}}%
\pgfpathlineto{\pgfqpoint{1.074191in}{3.890858in}}%
\pgfpathlineto{\pgfqpoint{1.076698in}{3.893086in}}%
\pgfpathlineto{\pgfqpoint{1.081711in}{3.902709in}}%
\pgfpathlineto{\pgfqpoint{1.086725in}{3.903007in}}%
\pgfpathlineto{\pgfqpoint{1.089231in}{3.906600in}}%
\pgfpathlineto{\pgfqpoint{1.101765in}{3.907858in}}%
\pgfpathlineto{\pgfqpoint{1.104271in}{3.908500in}}%
\pgfpathlineto{\pgfqpoint{1.106778in}{3.910766in}}%
\pgfpathlineto{\pgfqpoint{1.111791in}{3.912902in}}%
\pgfpathlineto{\pgfqpoint{1.121818in}{3.919999in}}%
\pgfpathlineto{\pgfqpoint{1.126831in}{3.934471in}}%
\pgfpathlineto{\pgfqpoint{1.131844in}{3.936422in}}%
\pgfpathlineto{\pgfqpoint{1.134351in}{3.940548in}}%
\pgfpathlineto{\pgfqpoint{1.136858in}{3.948490in}}%
\pgfpathlineto{\pgfqpoint{1.139364in}{3.949281in}}%
\pgfpathlineto{\pgfqpoint{1.144378in}{3.952465in}}%
\pgfpathlineto{\pgfqpoint{1.146884in}{3.953202in}}%
\pgfpathlineto{\pgfqpoint{1.149391in}{3.955757in}}%
\pgfpathlineto{\pgfqpoint{1.151898in}{3.955869in}}%
\pgfpathlineto{\pgfqpoint{1.156911in}{3.957360in}}%
\pgfpathlineto{\pgfqpoint{1.164431in}{3.961622in}}%
\pgfpathlineto{\pgfqpoint{1.166938in}{3.969252in}}%
\pgfpathlineto{\pgfqpoint{1.171951in}{3.970184in}}%
\pgfpathlineto{\pgfqpoint{1.174458in}{3.973681in}}%
\pgfpathlineto{\pgfqpoint{1.179471in}{3.974870in}}%
\pgfpathlineto{\pgfqpoint{1.181978in}{3.977695in}}%
\pgfpathlineto{\pgfqpoint{1.192004in}{3.981624in}}%
\pgfpathlineto{\pgfqpoint{1.197017in}{3.986739in}}%
\pgfpathlineto{\pgfqpoint{1.202031in}{3.988819in}}%
\pgfpathlineto{\pgfqpoint{1.204537in}{3.988888in}}%
\pgfpathlineto{\pgfqpoint{1.207044in}{3.992761in}}%
\pgfpathlineto{\pgfqpoint{1.219577in}{3.998407in}}%
\pgfpathlineto{\pgfqpoint{1.222084in}{4.002451in}}%
\pgfpathlineto{\pgfqpoint{1.227097in}{4.002647in}}%
\pgfpathlineto{\pgfqpoint{1.232111in}{4.004747in}}%
\pgfpathlineto{\pgfqpoint{1.237124in}{4.012711in}}%
\pgfpathlineto{\pgfqpoint{1.239631in}{4.014934in}}%
\pgfpathlineto{\pgfqpoint{1.249657in}{4.016715in}}%
\pgfpathlineto{\pgfqpoint{1.252164in}{4.019452in}}%
\pgfpathlineto{\pgfqpoint{1.257177in}{4.020023in}}%
\pgfpathlineto{\pgfqpoint{1.262190in}{4.021887in}}%
\pgfpathlineto{\pgfqpoint{1.274724in}{4.024379in}}%
\pgfpathlineto{\pgfqpoint{1.279737in}{4.025157in}}%
\pgfpathlineto{\pgfqpoint{1.289764in}{4.034331in}}%
\pgfpathlineto{\pgfqpoint{1.299790in}{4.036741in}}%
\pgfpathlineto{\pgfqpoint{1.302297in}{4.039551in}}%
\pgfpathlineto{\pgfqpoint{1.312324in}{4.041433in}}%
\pgfpathlineto{\pgfqpoint{1.314830in}{4.044843in}}%
\pgfpathlineto{\pgfqpoint{1.339897in}{4.054066in}}%
\pgfpathlineto{\pgfqpoint{1.342403in}{4.057314in}}%
\pgfpathlineto{\pgfqpoint{1.347417in}{4.060055in}}%
\pgfpathlineto{\pgfqpoint{1.349923in}{4.065384in}}%
\pgfpathlineto{\pgfqpoint{1.352430in}{4.067088in}}%
\pgfpathlineto{\pgfqpoint{1.354937in}{4.071188in}}%
\pgfpathlineto{\pgfqpoint{1.362457in}{4.088356in}}%
\pgfpathlineto{\pgfqpoint{1.364963in}{4.089583in}}%
\pgfpathlineto{\pgfqpoint{1.367470in}{4.094927in}}%
\pgfpathlineto{\pgfqpoint{1.369977in}{4.096448in}}%
\pgfpathlineto{\pgfqpoint{1.372483in}{4.101227in}}%
\pgfpathlineto{\pgfqpoint{1.374990in}{4.101345in}}%
\pgfpathlineto{\pgfqpoint{1.377497in}{4.107614in}}%
\pgfpathlineto{\pgfqpoint{1.380003in}{4.109193in}}%
\pgfpathlineto{\pgfqpoint{1.382510in}{4.119975in}}%
\pgfpathlineto{\pgfqpoint{1.387523in}{4.129229in}}%
\pgfpathlineto{\pgfqpoint{1.392537in}{4.146282in}}%
\pgfpathlineto{\pgfqpoint{1.395043in}{4.150771in}}%
\pgfpathlineto{\pgfqpoint{1.397550in}{4.158442in}}%
\pgfpathlineto{\pgfqpoint{1.402563in}{4.185633in}}%
\pgfpathlineto{\pgfqpoint{1.405070in}{4.186612in}}%
\pgfpathlineto{\pgfqpoint{1.410083in}{4.219870in}}%
\pgfpathlineto{\pgfqpoint{1.412590in}{4.223073in}}%
\pgfpathlineto{\pgfqpoint{1.415096in}{4.284334in}}%
\pgfpathlineto{\pgfqpoint{1.435150in}{4.288848in}}%
\pgfpathlineto{\pgfqpoint{1.445176in}{4.289205in}}%
\pgfpathlineto{\pgfqpoint{1.452696in}{4.290600in}}%
\pgfpathlineto{\pgfqpoint{1.515363in}{4.294178in}}%
\pgfpathlineto{\pgfqpoint{1.555469in}{4.295935in}}%
\pgfpathlineto{\pgfqpoint{1.562989in}{4.297071in}}%
\pgfpathlineto{\pgfqpoint{1.658242in}{4.302508in}}%
\pgfpathlineto{\pgfqpoint{1.678295in}{4.303479in}}%
\pgfpathlineto{\pgfqpoint{1.690828in}{4.304453in}}%
\pgfpathlineto{\pgfqpoint{1.720908in}{4.306641in}}%
\pgfpathlineto{\pgfqpoint{1.728428in}{4.307646in}}%
\pgfpathlineto{\pgfqpoint{1.735948in}{4.308364in}}%
\pgfpathlineto{\pgfqpoint{1.745975in}{4.309535in}}%
\pgfpathlineto{\pgfqpoint{1.768535in}{4.310713in}}%
\pgfpathlineto{\pgfqpoint{1.773548in}{4.311418in}}%
\pgfpathlineto{\pgfqpoint{1.793601in}{4.312976in}}%
\pgfpathlineto{\pgfqpoint{1.808641in}{4.314833in}}%
\pgfpathlineto{\pgfqpoint{1.984107in}{4.325239in}}%
\pgfpathlineto{\pgfqpoint{1.994134in}{4.327056in}}%
\pgfpathlineto{\pgfqpoint{2.031733in}{4.329612in}}%
\pgfpathlineto{\pgfqpoint{2.061813in}{4.333171in}}%
\pgfpathlineto{\pgfqpoint{2.086880in}{4.334698in}}%
\pgfpathlineto{\pgfqpoint{2.109440in}{4.335686in}}%
\pgfpathlineto{\pgfqpoint{2.134506in}{4.336936in}}%
\pgfpathlineto{\pgfqpoint{2.142026in}{4.337857in}}%
\pgfpathlineto{\pgfqpoint{2.182133in}{4.340667in}}%
\pgfpathlineto{\pgfqpoint{2.194666in}{4.341884in}}%
\pgfpathlineto{\pgfqpoint{2.219732in}{4.343224in}}%
\pgfpathlineto{\pgfqpoint{2.239786in}{4.344930in}}%
\pgfpathlineto{\pgfqpoint{2.244799in}{4.345937in}}%
\pgfpathlineto{\pgfqpoint{2.267359in}{4.347564in}}%
\pgfpathlineto{\pgfqpoint{2.272372in}{4.348306in}}%
\pgfpathlineto{\pgfqpoint{2.279892in}{4.349114in}}%
\pgfpathlineto{\pgfqpoint{2.284905in}{4.350183in}}%
\pgfpathlineto{\pgfqpoint{2.304959in}{4.351205in}}%
\pgfpathlineto{\pgfqpoint{2.322505in}{4.353497in}}%
\pgfpathlineto{\pgfqpoint{2.342559in}{4.354959in}}%
\pgfpathlineto{\pgfqpoint{2.347572in}{4.355957in}}%
\pgfpathlineto{\pgfqpoint{2.417758in}{4.362280in}}%
\pgfpathlineto{\pgfqpoint{2.440318in}{4.363112in}}%
\pgfpathlineto{\pgfqpoint{2.445331in}{4.364163in}}%
\pgfpathlineto{\pgfqpoint{2.455358in}{4.365383in}}%
\pgfpathlineto{\pgfqpoint{2.480425in}{4.368471in}}%
\pgfpathlineto{\pgfqpoint{2.492958in}{4.369047in}}%
\pgfpathlineto{\pgfqpoint{2.497971in}{4.370746in}}%
\pgfpathlineto{\pgfqpoint{2.523038in}{4.372827in}}%
\pgfpathlineto{\pgfqpoint{2.550611in}{4.378745in}}%
\pgfpathlineto{\pgfqpoint{2.573171in}{4.381049in}}%
\pgfpathlineto{\pgfqpoint{2.578184in}{4.382055in}}%
\pgfpathlineto{\pgfqpoint{2.585704in}{4.383461in}}%
\pgfpathlineto{\pgfqpoint{2.598237in}{4.384650in}}%
\pgfpathlineto{\pgfqpoint{2.615784in}{4.386604in}}%
\pgfpathlineto{\pgfqpoint{2.630824in}{4.388316in}}%
\pgfpathlineto{\pgfqpoint{2.643357in}{4.389233in}}%
\pgfpathlineto{\pgfqpoint{2.660904in}{4.390523in}}%
\pgfpathlineto{\pgfqpoint{2.668424in}{4.392753in}}%
\pgfpathlineto{\pgfqpoint{2.718557in}{4.401779in}}%
\pgfpathlineto{\pgfqpoint{2.728583in}{4.403909in}}%
\pgfpathlineto{\pgfqpoint{2.743623in}{4.405414in}}%
\pgfpathlineto{\pgfqpoint{2.748637in}{4.407043in}}%
\pgfpathlineto{\pgfqpoint{2.766183in}{4.410083in}}%
\pgfpathlineto{\pgfqpoint{2.808796in}{4.414111in}}%
\pgfpathlineto{\pgfqpoint{2.823836in}{4.417890in}}%
\pgfpathlineto{\pgfqpoint{2.833863in}{4.419421in}}%
\pgfpathlineto{\pgfqpoint{2.841383in}{4.421681in}}%
\pgfpathlineto{\pgfqpoint{2.853916in}{4.422833in}}%
\pgfpathlineto{\pgfqpoint{2.856423in}{4.425125in}}%
\pgfpathlineto{\pgfqpoint{2.858929in}{4.425184in}}%
\pgfpathlineto{\pgfqpoint{2.863943in}{4.426827in}}%
\pgfpathlineto{\pgfqpoint{2.876476in}{4.428071in}}%
\pgfpathlineto{\pgfqpoint{2.886503in}{4.428962in}}%
\pgfpathlineto{\pgfqpoint{2.891516in}{4.430294in}}%
\pgfpathlineto{\pgfqpoint{2.904049in}{4.432601in}}%
\pgfpathlineto{\pgfqpoint{2.911569in}{4.432969in}}%
\pgfpathlineto{\pgfqpoint{2.916582in}{4.434631in}}%
\pgfpathlineto{\pgfqpoint{2.954182in}{4.439740in}}%
\pgfpathlineto{\pgfqpoint{2.959195in}{4.445493in}}%
\pgfpathlineto{\pgfqpoint{2.976742in}{4.448575in}}%
\pgfpathlineto{\pgfqpoint{2.979249in}{4.453060in}}%
\pgfpathlineto{\pgfqpoint{2.984262in}{4.453882in}}%
\pgfpathlineto{\pgfqpoint{2.986769in}{4.456197in}}%
\pgfpathlineto{\pgfqpoint{3.001809in}{4.459047in}}%
\pgfpathlineto{\pgfqpoint{3.004315in}{4.461064in}}%
\pgfpathlineto{\pgfqpoint{3.011835in}{4.462291in}}%
\pgfpathlineto{\pgfqpoint{3.026875in}{4.465643in}}%
\pgfpathlineto{\pgfqpoint{3.029382in}{4.465749in}}%
\pgfpathlineto{\pgfqpoint{3.034395in}{4.467940in}}%
\pgfpathlineto{\pgfqpoint{3.044422in}{4.468683in}}%
\pgfpathlineto{\pgfqpoint{3.049435in}{4.469911in}}%
\pgfpathlineto{\pgfqpoint{3.051942in}{4.469912in}}%
\pgfpathlineto{\pgfqpoint{3.054448in}{4.472080in}}%
\pgfpathlineto{\pgfqpoint{3.074502in}{4.476601in}}%
\pgfpathlineto{\pgfqpoint{3.082022in}{4.478125in}}%
\pgfpathlineto{\pgfqpoint{3.102075in}{4.483585in}}%
\pgfpathlineto{\pgfqpoint{3.104581in}{4.485182in}}%
\pgfpathlineto{\pgfqpoint{3.117115in}{4.486648in}}%
\pgfpathlineto{\pgfqpoint{3.119621in}{4.492342in}}%
\pgfpathlineto{\pgfqpoint{3.124635in}{4.494261in}}%
\pgfpathlineto{\pgfqpoint{3.127141in}{4.496775in}}%
\pgfpathlineto{\pgfqpoint{3.132155in}{4.497614in}}%
\pgfpathlineto{\pgfqpoint{3.134661in}{4.497743in}}%
\pgfpathlineto{\pgfqpoint{3.137168in}{4.501655in}}%
\pgfpathlineto{\pgfqpoint{3.142181in}{4.502863in}}%
\pgfpathlineto{\pgfqpoint{3.147195in}{4.507359in}}%
\pgfpathlineto{\pgfqpoint{3.149701in}{4.507462in}}%
\pgfpathlineto{\pgfqpoint{3.152208in}{4.509657in}}%
\pgfpathlineto{\pgfqpoint{3.172261in}{4.514088in}}%
\pgfpathlineto{\pgfqpoint{3.174768in}{4.516873in}}%
\pgfpathlineto{\pgfqpoint{3.179781in}{4.517816in}}%
\pgfpathlineto{\pgfqpoint{3.184794in}{4.521512in}}%
\pgfpathlineto{\pgfqpoint{3.192314in}{4.525954in}}%
\pgfpathlineto{\pgfqpoint{3.194821in}{4.526106in}}%
\pgfpathlineto{\pgfqpoint{3.197328in}{4.528306in}}%
\pgfpathlineto{\pgfqpoint{3.207354in}{4.530766in}}%
\pgfpathlineto{\pgfqpoint{3.209861in}{4.532094in}}%
\pgfpathlineto{\pgfqpoint{3.212368in}{4.537835in}}%
\pgfpathlineto{\pgfqpoint{3.219888in}{4.538450in}}%
\pgfpathlineto{\pgfqpoint{3.222394in}{4.541491in}}%
\pgfpathlineto{\pgfqpoint{3.227408in}{4.542302in}}%
\pgfpathlineto{\pgfqpoint{3.229914in}{4.543002in}}%
\pgfpathlineto{\pgfqpoint{3.234927in}{4.548087in}}%
\pgfpathlineto{\pgfqpoint{3.242447in}{4.549196in}}%
\pgfpathlineto{\pgfqpoint{3.249967in}{4.550454in}}%
\pgfpathlineto{\pgfqpoint{3.252474in}{4.553340in}}%
\pgfpathlineto{\pgfqpoint{3.254981in}{4.553697in}}%
\pgfpathlineto{\pgfqpoint{3.257487in}{4.555219in}}%
\pgfpathlineto{\pgfqpoint{3.259994in}{4.561057in}}%
\pgfpathlineto{\pgfqpoint{3.267514in}{4.562554in}}%
\pgfpathlineto{\pgfqpoint{3.275034in}{4.568222in}}%
\pgfpathlineto{\pgfqpoint{3.277541in}{4.576779in}}%
\pgfpathlineto{\pgfqpoint{3.280047in}{4.578495in}}%
\pgfpathlineto{\pgfqpoint{3.285061in}{4.595076in}}%
\pgfpathlineto{\pgfqpoint{3.287567in}{4.595373in}}%
\pgfpathlineto{\pgfqpoint{3.290074in}{4.601721in}}%
\pgfpathlineto{\pgfqpoint{3.292581in}{4.603720in}}%
\pgfpathlineto{\pgfqpoint{3.295087in}{4.603969in}}%
\pgfpathlineto{\pgfqpoint{3.300100in}{4.608800in}}%
\pgfpathlineto{\pgfqpoint{3.305114in}{4.618131in}}%
\pgfpathlineto{\pgfqpoint{3.310127in}{4.619912in}}%
\pgfpathlineto{\pgfqpoint{3.315140in}{4.630491in}}%
\pgfpathlineto{\pgfqpoint{3.317647in}{4.631935in}}%
\pgfpathlineto{\pgfqpoint{3.320154in}{4.635588in}}%
\pgfpathlineto{\pgfqpoint{3.322660in}{4.636117in}}%
\pgfpathlineto{\pgfqpoint{3.325167in}{4.643971in}}%
\pgfpathlineto{\pgfqpoint{3.335194in}{4.646470in}}%
\pgfpathlineto{\pgfqpoint{3.340207in}{4.650650in}}%
\pgfpathlineto{\pgfqpoint{3.350234in}{4.651005in}}%
\pgfpathlineto{\pgfqpoint{3.367780in}{4.658168in}}%
\pgfpathlineto{\pgfqpoint{3.370287in}{4.659532in}}%
\pgfpathlineto{\pgfqpoint{3.372793in}{4.659624in}}%
\pgfpathlineto{\pgfqpoint{3.375300in}{4.661225in}}%
\pgfpathlineto{\pgfqpoint{3.380313in}{4.662304in}}%
\pgfpathlineto{\pgfqpoint{3.395353in}{4.664236in}}%
\pgfpathlineto{\pgfqpoint{3.400367in}{4.669418in}}%
\pgfpathlineto{\pgfqpoint{3.402873in}{4.669836in}}%
\pgfpathlineto{\pgfqpoint{3.405380in}{4.686286in}}%
\pgfpathlineto{\pgfqpoint{3.407887in}{4.688481in}}%
\pgfpathlineto{\pgfqpoint{3.410393in}{4.688824in}}%
\pgfpathlineto{\pgfqpoint{3.412900in}{4.696905in}}%
\pgfpathlineto{\pgfqpoint{3.415407in}{4.697794in}}%
\pgfpathlineto{\pgfqpoint{3.420420in}{4.705183in}}%
\pgfpathlineto{\pgfqpoint{3.425433in}{4.705857in}}%
\pgfpathlineto{\pgfqpoint{3.427940in}{4.716537in}}%
\pgfpathlineto{\pgfqpoint{3.430447in}{4.718203in}}%
\pgfpathlineto{\pgfqpoint{3.432953in}{4.718357in}}%
\pgfpathlineto{\pgfqpoint{3.435460in}{4.722759in}}%
\pgfpathlineto{\pgfqpoint{3.437966in}{4.723082in}}%
\pgfpathlineto{\pgfqpoint{3.440473in}{4.724589in}}%
\pgfpathlineto{\pgfqpoint{3.442980in}{4.731911in}}%
\pgfpathlineto{\pgfqpoint{3.445486in}{4.732777in}}%
\pgfpathlineto{\pgfqpoint{3.447993in}{4.735182in}}%
\pgfpathlineto{\pgfqpoint{3.450500in}{4.739350in}}%
\pgfpathlineto{\pgfqpoint{3.453006in}{4.747934in}}%
\pgfpathlineto{\pgfqpoint{3.458020in}{4.755925in}}%
\pgfpathlineto{\pgfqpoint{3.460526in}{4.764157in}}%
\pgfpathlineto{\pgfqpoint{3.463033in}{4.767479in}}%
\pgfpathlineto{\pgfqpoint{3.465540in}{4.790653in}}%
\pgfpathlineto{\pgfqpoint{3.468046in}{4.791512in}}%
\pgfpathlineto{\pgfqpoint{3.470553in}{4.793986in}}%
\pgfpathlineto{\pgfqpoint{3.473060in}{4.801253in}}%
\pgfpathlineto{\pgfqpoint{3.475566in}{4.801705in}}%
\pgfpathlineto{\pgfqpoint{3.480580in}{4.817657in}}%
\pgfpathlineto{\pgfqpoint{3.483086in}{4.849163in}}%
\pgfpathlineto{\pgfqpoint{3.488100in}{4.850197in}}%
\pgfpathlineto{\pgfqpoint{3.490606in}{4.866908in}}%
\pgfpathlineto{\pgfqpoint{3.493113in}{4.871954in}}%
\pgfpathlineto{\pgfqpoint{3.498126in}{4.873880in}}%
\pgfpathlineto{\pgfqpoint{3.503139in}{4.875654in}}%
\pgfpathlineto{\pgfqpoint{3.505646in}{4.883035in}}%
\pgfpathlineto{\pgfqpoint{3.508153in}{4.884294in}}%
\pgfpathlineto{\pgfqpoint{3.510659in}{4.889922in}}%
\pgfpathlineto{\pgfqpoint{3.513166in}{4.891135in}}%
\pgfpathlineto{\pgfqpoint{3.515673in}{4.894469in}}%
\pgfpathlineto{\pgfqpoint{3.520686in}{4.912845in}}%
\pgfpathlineto{\pgfqpoint{3.528206in}{4.914718in}}%
\pgfpathlineto{\pgfqpoint{3.530713in}{4.921118in}}%
\pgfpathlineto{\pgfqpoint{3.535726in}{4.924132in}}%
\pgfpathlineto{\pgfqpoint{3.543246in}{4.958043in}}%
\pgfpathlineto{\pgfqpoint{3.548259in}{4.975110in}}%
\pgfpathlineto{\pgfqpoint{3.553273in}{4.975430in}}%
\pgfpathlineto{\pgfqpoint{3.555779in}{4.978097in}}%
\pgfpathlineto{\pgfqpoint{3.563299in}{4.978421in}}%
\pgfpathlineto{\pgfqpoint{3.565806in}{4.988219in}}%
\pgfpathlineto{\pgfqpoint{3.573326in}{4.992489in}}%
\pgfpathlineto{\pgfqpoint{3.575832in}{5.004697in}}%
\pgfpathlineto{\pgfqpoint{3.578339in}{5.007420in}}%
\pgfpathlineto{\pgfqpoint{3.580846in}{5.014687in}}%
\pgfpathlineto{\pgfqpoint{3.585859in}{5.037809in}}%
\pgfpathlineto{\pgfqpoint{3.588366in}{5.043705in}}%
\pgfpathlineto{\pgfqpoint{3.590872in}{5.060654in}}%
\pgfpathlineto{\pgfqpoint{3.593379in}{5.062340in}}%
\pgfpathlineto{\pgfqpoint{3.595886in}{5.068260in}}%
\pgfpathlineto{\pgfqpoint{3.598392in}{5.079825in}}%
\pgfpathlineto{\pgfqpoint{3.600899in}{5.084815in}}%
\pgfpathlineto{\pgfqpoint{3.603406in}{5.084960in}}%
\pgfpathlineto{\pgfqpoint{3.608419in}{5.091921in}}%
\pgfpathlineto{\pgfqpoint{3.610926in}{5.101651in}}%
\pgfpathlineto{\pgfqpoint{3.613432in}{5.104976in}}%
\pgfpathlineto{\pgfqpoint{3.618446in}{5.108088in}}%
\pgfpathlineto{\pgfqpoint{3.620952in}{5.112037in}}%
\pgfpathlineto{\pgfqpoint{3.625966in}{5.200146in}}%
\pgfpathlineto{\pgfqpoint{3.628472in}{5.200618in}}%
\pgfpathlineto{\pgfqpoint{3.633486in}{5.206859in}}%
\pgfpathlineto{\pgfqpoint{3.635992in}{5.220059in}}%
\pgfpathlineto{\pgfqpoint{3.638499in}{5.220881in}}%
\pgfpathlineto{\pgfqpoint{3.641005in}{5.223441in}}%
\pgfpathlineto{\pgfqpoint{3.643512in}{5.228526in}}%
\pgfpathlineto{\pgfqpoint{3.646019in}{5.244922in}}%
\pgfpathlineto{\pgfqpoint{3.653539in}{5.250072in}}%
\pgfpathlineto{\pgfqpoint{3.658552in}{5.261622in}}%
\pgfpathlineto{\pgfqpoint{3.661059in}{5.262765in}}%
\pgfpathlineto{\pgfqpoint{3.663565in}{5.273639in}}%
\pgfpathlineto{\pgfqpoint{3.666072in}{5.278941in}}%
\pgfpathlineto{\pgfqpoint{3.671085in}{5.302163in}}%
\pgfpathlineto{\pgfqpoint{3.673592in}{5.305275in}}%
\pgfpathlineto{\pgfqpoint{3.673592in}{5.305275in}}%
\pgfusepath{stroke}%
\end{pgfscope}%
\begin{pgfscope}%
\pgfpathrectangle{\pgfqpoint{0.708220in}{3.210823in}}{\pgfqpoint{5.013309in}{2.094453in}}%
\pgfusepath{clip}%
\pgfsetrectcap%
\pgfsetroundjoin%
\pgfsetlinewidth{1.003750pt}%
\definecolor{currentstroke}{rgb}{0.811765,0.125490,0.125490}%
\pgfsetstrokecolor{currentstroke}%
\pgfsetdash{}{0pt}%
\pgfpathmoveto{\pgfqpoint{0.708220in}{3.650945in}}%
\pgfpathlineto{\pgfqpoint{0.710727in}{3.653565in}}%
\pgfpathlineto{\pgfqpoint{0.713233in}{3.653565in}}%
\pgfpathlineto{\pgfqpoint{0.715740in}{3.656145in}}%
\pgfpathlineto{\pgfqpoint{0.723260in}{3.656145in}}%
\pgfpathlineto{\pgfqpoint{0.725766in}{3.661192in}}%
\pgfpathlineto{\pgfqpoint{0.728273in}{3.661192in}}%
\pgfpathlineto{\pgfqpoint{0.730780in}{3.663661in}}%
\pgfpathlineto{\pgfqpoint{0.735793in}{3.663661in}}%
\pgfpathlineto{\pgfqpoint{0.740806in}{3.668495in}}%
\pgfpathlineto{\pgfqpoint{0.753340in}{3.668495in}}%
\pgfpathlineto{\pgfqpoint{0.758353in}{3.673196in}}%
\pgfpathlineto{\pgfqpoint{0.763366in}{3.673196in}}%
\pgfpathlineto{\pgfqpoint{0.765873in}{3.675500in}}%
\pgfpathlineto{\pgfqpoint{0.788433in}{3.675500in}}%
\pgfpathlineto{\pgfqpoint{0.790939in}{3.677773in}}%
\pgfpathlineto{\pgfqpoint{0.816006in}{3.677773in}}%
\pgfpathlineto{\pgfqpoint{0.818513in}{3.680016in}}%
\pgfpathlineto{\pgfqpoint{0.833553in}{3.680016in}}%
\pgfpathlineto{\pgfqpoint{0.836059in}{3.682230in}}%
\pgfpathlineto{\pgfqpoint{0.843579in}{3.682230in}}%
\pgfpathlineto{\pgfqpoint{0.846086in}{3.684416in}}%
\pgfpathlineto{\pgfqpoint{0.863632in}{3.684416in}}%
\pgfpathlineto{\pgfqpoint{0.866139in}{3.686574in}}%
\pgfpathlineto{\pgfqpoint{0.878672in}{3.686574in}}%
\pgfpathlineto{\pgfqpoint{0.881179in}{3.688706in}}%
\pgfpathlineto{\pgfqpoint{0.888699in}{3.688706in}}%
\pgfpathlineto{\pgfqpoint{0.891206in}{3.690811in}}%
\pgfpathlineto{\pgfqpoint{0.898726in}{3.690811in}}%
\pgfpathlineto{\pgfqpoint{0.901232in}{3.692891in}}%
\pgfpathlineto{\pgfqpoint{0.926299in}{3.692891in}}%
\pgfpathlineto{\pgfqpoint{0.928805in}{3.694946in}}%
\pgfpathlineto{\pgfqpoint{0.956379in}{3.694946in}}%
\pgfpathlineto{\pgfqpoint{0.958885in}{3.696977in}}%
\pgfpathlineto{\pgfqpoint{0.973925in}{3.696977in}}%
\pgfpathlineto{\pgfqpoint{0.976432in}{3.698984in}}%
\pgfpathlineto{\pgfqpoint{0.983952in}{3.698984in}}%
\pgfpathlineto{\pgfqpoint{0.986458in}{3.700967in}}%
\pgfpathlineto{\pgfqpoint{1.009018in}{3.700967in}}%
\pgfpathlineto{\pgfqpoint{1.011525in}{3.702928in}}%
\pgfpathlineto{\pgfqpoint{1.024058in}{3.702928in}}%
\pgfpathlineto{\pgfqpoint{1.026565in}{3.704867in}}%
\pgfpathlineto{\pgfqpoint{1.044112in}{3.704867in}}%
\pgfpathlineto{\pgfqpoint{1.046618in}{3.706785in}}%
\pgfpathlineto{\pgfqpoint{1.064165in}{3.706785in}}%
\pgfpathlineto{\pgfqpoint{1.066671in}{3.708681in}}%
\pgfpathlineto{\pgfqpoint{1.069178in}{3.708681in}}%
\pgfpathlineto{\pgfqpoint{1.071685in}{3.710556in}}%
\pgfpathlineto{\pgfqpoint{1.086725in}{3.710556in}}%
\pgfpathlineto{\pgfqpoint{1.089231in}{3.712411in}}%
\pgfpathlineto{\pgfqpoint{1.091738in}{3.712411in}}%
\pgfpathlineto{\pgfqpoint{1.096751in}{3.716062in}}%
\pgfpathlineto{\pgfqpoint{1.099258in}{3.716062in}}%
\pgfpathlineto{\pgfqpoint{1.101765in}{3.717859in}}%
\pgfpathlineto{\pgfqpoint{1.111791in}{3.717859in}}%
\pgfpathlineto{\pgfqpoint{1.114298in}{3.719637in}}%
\pgfpathlineto{\pgfqpoint{1.129338in}{3.719637in}}%
\pgfpathlineto{\pgfqpoint{1.131844in}{3.721397in}}%
\pgfpathlineto{\pgfqpoint{1.139364in}{3.721397in}}%
\pgfpathlineto{\pgfqpoint{1.141871in}{3.723139in}}%
\pgfpathlineto{\pgfqpoint{1.151898in}{3.723139in}}%
\pgfpathlineto{\pgfqpoint{1.154404in}{3.724863in}}%
\pgfpathlineto{\pgfqpoint{1.161924in}{3.724863in}}%
\pgfpathlineto{\pgfqpoint{1.164431in}{3.726571in}}%
\pgfpathlineto{\pgfqpoint{1.171951in}{3.726571in}}%
\pgfpathlineto{\pgfqpoint{1.174458in}{3.728261in}}%
\pgfpathlineto{\pgfqpoint{1.176964in}{3.728261in}}%
\pgfpathlineto{\pgfqpoint{1.179471in}{3.729935in}}%
\pgfpathlineto{\pgfqpoint{1.184484in}{3.729935in}}%
\pgfpathlineto{\pgfqpoint{1.186991in}{3.731593in}}%
\pgfpathlineto{\pgfqpoint{1.189498in}{3.731593in}}%
\pgfpathlineto{\pgfqpoint{1.192004in}{3.733235in}}%
\pgfpathlineto{\pgfqpoint{1.202031in}{3.733235in}}%
\pgfpathlineto{\pgfqpoint{1.204537in}{3.734862in}}%
\pgfpathlineto{\pgfqpoint{1.219577in}{3.734862in}}%
\pgfpathlineto{\pgfqpoint{1.222084in}{3.736473in}}%
\pgfpathlineto{\pgfqpoint{1.242137in}{3.736473in}}%
\pgfpathlineto{\pgfqpoint{1.244644in}{3.738069in}}%
\pgfpathlineto{\pgfqpoint{1.264697in}{3.738069in}}%
\pgfpathlineto{\pgfqpoint{1.267204in}{3.739651in}}%
\pgfpathlineto{\pgfqpoint{1.277230in}{3.739651in}}%
\pgfpathlineto{\pgfqpoint{1.279737in}{3.741218in}}%
\pgfpathlineto{\pgfqpoint{1.307310in}{3.741218in}}%
\pgfpathlineto{\pgfqpoint{1.309817in}{3.742771in}}%
\pgfpathlineto{\pgfqpoint{1.319844in}{3.742771in}}%
\pgfpathlineto{\pgfqpoint{1.322350in}{3.744310in}}%
\pgfpathlineto{\pgfqpoint{1.357443in}{3.744310in}}%
\pgfpathlineto{\pgfqpoint{1.359950in}{3.745835in}}%
\pgfpathlineto{\pgfqpoint{1.387523in}{3.745835in}}%
\pgfpathlineto{\pgfqpoint{1.390030in}{3.747347in}}%
\pgfpathlineto{\pgfqpoint{1.410083in}{3.747347in}}%
\pgfpathlineto{\pgfqpoint{1.412590in}{3.748845in}}%
\pgfpathlineto{\pgfqpoint{1.422616in}{3.748845in}}%
\pgfpathlineto{\pgfqpoint{1.425123in}{3.750331in}}%
\pgfpathlineto{\pgfqpoint{1.432643in}{3.750331in}}%
\pgfpathlineto{\pgfqpoint{1.435150in}{3.751804in}}%
\pgfpathlineto{\pgfqpoint{1.440163in}{3.751804in}}%
\pgfpathlineto{\pgfqpoint{1.442670in}{3.753264in}}%
\pgfpathlineto{\pgfqpoint{1.457710in}{3.753264in}}%
\pgfpathlineto{\pgfqpoint{1.460216in}{3.754712in}}%
\pgfpathlineto{\pgfqpoint{1.467736in}{3.754712in}}%
\pgfpathlineto{\pgfqpoint{1.470243in}{3.756148in}}%
\pgfpathlineto{\pgfqpoint{1.492803in}{3.756148in}}%
\pgfpathlineto{\pgfqpoint{1.495309in}{3.757572in}}%
\pgfpathlineto{\pgfqpoint{1.507843in}{3.757572in}}%
\pgfpathlineto{\pgfqpoint{1.510349in}{3.758984in}}%
\pgfpathlineto{\pgfqpoint{1.530402in}{3.758984in}}%
\pgfpathlineto{\pgfqpoint{1.532909in}{3.760385in}}%
\pgfpathlineto{\pgfqpoint{1.545442in}{3.760385in}}%
\pgfpathlineto{\pgfqpoint{1.547949in}{3.761775in}}%
\pgfpathlineto{\pgfqpoint{1.552962in}{3.761775in}}%
\pgfpathlineto{\pgfqpoint{1.555469in}{3.763153in}}%
\pgfpathlineto{\pgfqpoint{1.568002in}{3.763153in}}%
\pgfpathlineto{\pgfqpoint{1.570509in}{3.764520in}}%
\pgfpathlineto{\pgfqpoint{1.583042in}{3.764520in}}%
\pgfpathlineto{\pgfqpoint{1.585549in}{3.765877in}}%
\pgfpathlineto{\pgfqpoint{1.618135in}{3.765877in}}%
\pgfpathlineto{\pgfqpoint{1.620642in}{3.767222in}}%
\pgfpathlineto{\pgfqpoint{1.630669in}{3.767222in}}%
\pgfpathlineto{\pgfqpoint{1.633175in}{3.768558in}}%
\pgfpathlineto{\pgfqpoint{1.653229in}{3.768558in}}%
\pgfpathlineto{\pgfqpoint{1.655735in}{3.769883in}}%
\pgfpathlineto{\pgfqpoint{1.660749in}{3.769883in}}%
\pgfpathlineto{\pgfqpoint{1.663255in}{3.771198in}}%
\pgfpathlineto{\pgfqpoint{1.673282in}{3.771198in}}%
\pgfpathlineto{\pgfqpoint{1.675788in}{3.772503in}}%
\pgfpathlineto{\pgfqpoint{1.685815in}{3.772503in}}%
\pgfpathlineto{\pgfqpoint{1.688322in}{3.773798in}}%
\pgfpathlineto{\pgfqpoint{1.708375in}{3.773798in}}%
\pgfpathlineto{\pgfqpoint{1.710882in}{3.775083in}}%
\pgfpathlineto{\pgfqpoint{1.720908in}{3.775083in}}%
\pgfpathlineto{\pgfqpoint{1.723415in}{3.776359in}}%
\pgfpathlineto{\pgfqpoint{1.728428in}{3.776359in}}%
\pgfpathlineto{\pgfqpoint{1.730935in}{3.777625in}}%
\pgfpathlineto{\pgfqpoint{1.743468in}{3.777625in}}%
\pgfpathlineto{\pgfqpoint{1.745975in}{3.778882in}}%
\pgfpathlineto{\pgfqpoint{1.758508in}{3.778882in}}%
\pgfpathlineto{\pgfqpoint{1.761015in}{3.780130in}}%
\pgfpathlineto{\pgfqpoint{1.776055in}{3.780130in}}%
\pgfpathlineto{\pgfqpoint{1.781068in}{3.781369in}}%
\pgfpathlineto{\pgfqpoint{1.786081in}{3.781369in}}%
\pgfpathlineto{\pgfqpoint{1.791095in}{3.782599in}}%
\pgfpathlineto{\pgfqpoint{1.796108in}{3.782599in}}%
\pgfpathlineto{\pgfqpoint{1.801121in}{3.783820in}}%
\pgfpathlineto{\pgfqpoint{1.803628in}{3.783820in}}%
\pgfpathlineto{\pgfqpoint{1.811148in}{3.786237in}}%
\pgfpathlineto{\pgfqpoint{1.821174in}{3.789800in}}%
\pgfpathlineto{\pgfqpoint{1.826188in}{3.789800in}}%
\pgfpathlineto{\pgfqpoint{1.831201in}{3.790971in}}%
\pgfpathlineto{\pgfqpoint{1.833708in}{3.790971in}}%
\pgfpathlineto{\pgfqpoint{1.846241in}{3.795578in}}%
\pgfpathlineto{\pgfqpoint{1.853761in}{3.797836in}}%
\pgfpathlineto{\pgfqpoint{1.856268in}{3.797836in}}%
\pgfpathlineto{\pgfqpoint{1.858774in}{3.800064in}}%
\pgfpathlineto{\pgfqpoint{1.861281in}{3.800064in}}%
\pgfpathlineto{\pgfqpoint{1.863788in}{3.802264in}}%
\pgfpathlineto{\pgfqpoint{1.868801in}{3.803353in}}%
\pgfpathlineto{\pgfqpoint{1.873814in}{3.804436in}}%
\pgfpathlineto{\pgfqpoint{1.883841in}{3.805512in}}%
\pgfpathlineto{\pgfqpoint{1.891361in}{3.807643in}}%
\pgfpathlineto{\pgfqpoint{1.913921in}{3.808699in}}%
\pgfpathlineto{\pgfqpoint{1.921441in}{3.810792in}}%
\pgfpathlineto{\pgfqpoint{1.938987in}{3.811829in}}%
\pgfpathlineto{\pgfqpoint{1.946507in}{3.813884in}}%
\pgfpathlineto{\pgfqpoint{1.951520in}{3.814902in}}%
\pgfpathlineto{\pgfqpoint{1.956534in}{3.815914in}}%
\pgfpathlineto{\pgfqpoint{1.966560in}{3.816921in}}%
\pgfpathlineto{\pgfqpoint{1.976587in}{3.819905in}}%
\pgfpathlineto{\pgfqpoint{1.979094in}{3.819905in}}%
\pgfpathlineto{\pgfqpoint{1.981600in}{3.821866in}}%
\pgfpathlineto{\pgfqpoint{1.991627in}{3.822838in}}%
\pgfpathlineto{\pgfqpoint{2.004160in}{3.826673in}}%
\pgfpathlineto{\pgfqpoint{2.011680in}{3.827618in}}%
\pgfpathlineto{\pgfqpoint{2.016693in}{3.828559in}}%
\pgfpathlineto{\pgfqpoint{2.024213in}{3.829494in}}%
\pgfpathlineto{\pgfqpoint{2.031733in}{3.831349in}}%
\pgfpathlineto{\pgfqpoint{2.039253in}{3.832269in}}%
\pgfpathlineto{\pgfqpoint{2.051787in}{3.835901in}}%
\pgfpathlineto{\pgfqpoint{2.056800in}{3.836797in}}%
\pgfpathlineto{\pgfqpoint{2.059307in}{3.839457in}}%
\pgfpathlineto{\pgfqpoint{2.066827in}{3.841208in}}%
\pgfpathlineto{\pgfqpoint{2.071840in}{3.842077in}}%
\pgfpathlineto{\pgfqpoint{2.076853in}{3.843801in}}%
\pgfpathlineto{\pgfqpoint{2.079360in}{3.847199in}}%
\pgfpathlineto{\pgfqpoint{2.084373in}{3.848038in}}%
\pgfpathlineto{\pgfqpoint{2.091893in}{3.848873in}}%
\pgfpathlineto{\pgfqpoint{2.101920in}{3.854607in}}%
\pgfpathlineto{\pgfqpoint{2.109440in}{3.856211in}}%
\pgfpathlineto{\pgfqpoint{2.114453in}{3.857007in}}%
\pgfpathlineto{\pgfqpoint{2.121973in}{3.860156in}}%
\pgfpathlineto{\pgfqpoint{2.126986in}{3.862480in}}%
\pgfpathlineto{\pgfqpoint{2.129493in}{3.863247in}}%
\pgfpathlineto{\pgfqpoint{2.134506in}{3.867035in}}%
\pgfpathlineto{\pgfqpoint{2.137013in}{3.868528in}}%
\pgfpathlineto{\pgfqpoint{2.144533in}{3.869269in}}%
\pgfpathlineto{\pgfqpoint{2.154559in}{3.872202in}}%
\pgfpathlineto{\pgfqpoint{2.157066in}{3.874370in}}%
\pgfpathlineto{\pgfqpoint{2.159573in}{3.874370in}}%
\pgfpathlineto{\pgfqpoint{2.162079in}{3.875799in}}%
\pgfpathlineto{\pgfqpoint{2.167093in}{3.875799in}}%
\pgfpathlineto{\pgfqpoint{2.172106in}{3.877922in}}%
\pgfpathlineto{\pgfqpoint{2.177119in}{3.878624in}}%
\pgfpathlineto{\pgfqpoint{2.179626in}{3.880019in}}%
\pgfpathlineto{\pgfqpoint{2.182133in}{3.880019in}}%
\pgfpathlineto{\pgfqpoint{2.194666in}{3.889479in}}%
\pgfpathlineto{\pgfqpoint{2.227252in}{3.897193in}}%
\pgfpathlineto{\pgfqpoint{2.232266in}{3.897193in}}%
\pgfpathlineto{\pgfqpoint{2.237279in}{3.901537in}}%
\pgfpathlineto{\pgfqpoint{2.239786in}{3.901537in}}%
\pgfpathlineto{\pgfqpoint{2.244799in}{3.905175in}}%
\pgfpathlineto{\pgfqpoint{2.247306in}{3.908149in}}%
\pgfpathlineto{\pgfqpoint{2.249812in}{3.908738in}}%
\pgfpathlineto{\pgfqpoint{2.252319in}{3.911072in}}%
\pgfpathlineto{\pgfqpoint{2.259839in}{3.911651in}}%
\pgfpathlineto{\pgfqpoint{2.272372in}{3.916212in}}%
\pgfpathlineto{\pgfqpoint{2.277385in}{3.919555in}}%
\pgfpathlineto{\pgfqpoint{2.289919in}{3.927637in}}%
\pgfpathlineto{\pgfqpoint{2.292425in}{3.931797in}}%
\pgfpathlineto{\pgfqpoint{2.294932in}{3.932310in}}%
\pgfpathlineto{\pgfqpoint{2.297439in}{3.936360in}}%
\pgfpathlineto{\pgfqpoint{2.304959in}{3.937357in}}%
\pgfpathlineto{\pgfqpoint{2.307465in}{3.938349in}}%
\pgfpathlineto{\pgfqpoint{2.309972in}{3.941776in}}%
\pgfpathlineto{\pgfqpoint{2.314985in}{3.942743in}}%
\pgfpathlineto{\pgfqpoint{2.325012in}{3.945136in}}%
\pgfpathlineto{\pgfqpoint{2.327519in}{3.947496in}}%
\pgfpathlineto{\pgfqpoint{2.340052in}{3.948432in}}%
\pgfpathlineto{\pgfqpoint{2.347572in}{3.949362in}}%
\pgfpathlineto{\pgfqpoint{2.350078in}{3.952122in}}%
\pgfpathlineto{\pgfqpoint{2.352585in}{3.962739in}}%
\pgfpathlineto{\pgfqpoint{2.357598in}{3.963168in}}%
\pgfpathlineto{\pgfqpoint{2.360105in}{3.964871in}}%
\pgfpathlineto{\pgfqpoint{2.367625in}{3.965716in}}%
\pgfpathlineto{\pgfqpoint{2.370132in}{3.969881in}}%
\pgfpathlineto{\pgfqpoint{2.380158in}{3.973947in}}%
\pgfpathlineto{\pgfqpoint{2.382665in}{3.974749in}}%
\pgfpathlineto{\pgfqpoint{2.385172in}{3.977132in}}%
\pgfpathlineto{\pgfqpoint{2.390185in}{3.977920in}}%
\pgfpathlineto{\pgfqpoint{2.395198in}{3.979093in}}%
\pgfpathlineto{\pgfqpoint{2.400212in}{3.981418in}}%
\pgfpathlineto{\pgfqpoint{2.402718in}{3.981418in}}%
\pgfpathlineto{\pgfqpoint{2.407732in}{3.985973in}}%
\pgfpathlineto{\pgfqpoint{2.410238in}{3.990046in}}%
\pgfpathlineto{\pgfqpoint{2.415251in}{3.990776in}}%
\pgfpathlineto{\pgfqpoint{2.417758in}{3.994024in}}%
\pgfpathlineto{\pgfqpoint{2.422771in}{3.996155in}}%
\pgfpathlineto{\pgfqpoint{2.427785in}{3.996860in}}%
\pgfpathlineto{\pgfqpoint{2.430291in}{3.999304in}}%
\pgfpathlineto{\pgfqpoint{2.432798in}{3.999996in}}%
\pgfpathlineto{\pgfqpoint{2.435305in}{4.003414in}}%
\pgfpathlineto{\pgfqpoint{2.442825in}{4.006433in}}%
\pgfpathlineto{\pgfqpoint{2.445331in}{4.009073in}}%
\pgfpathlineto{\pgfqpoint{2.452851in}{4.010703in}}%
\pgfpathlineto{\pgfqpoint{2.457865in}{4.012638in}}%
\pgfpathlineto{\pgfqpoint{2.462878in}{4.013916in}}%
\pgfpathlineto{\pgfqpoint{2.470398in}{4.015501in}}%
\pgfpathlineto{\pgfqpoint{2.472905in}{4.018316in}}%
\pgfpathlineto{\pgfqpoint{2.480425in}{4.019553in}}%
\pgfpathlineto{\pgfqpoint{2.485438in}{4.025309in}}%
\pgfpathlineto{\pgfqpoint{2.495464in}{4.041229in}}%
\pgfpathlineto{\pgfqpoint{2.500478in}{4.043655in}}%
\pgfpathlineto{\pgfqpoint{2.502984in}{4.046838in}}%
\pgfpathlineto{\pgfqpoint{2.507998in}{4.048407in}}%
\pgfpathlineto{\pgfqpoint{2.513011in}{4.050478in}}%
\pgfpathlineto{\pgfqpoint{2.525544in}{4.056046in}}%
\pgfpathlineto{\pgfqpoint{2.538078in}{4.060714in}}%
\pgfpathlineto{\pgfqpoint{2.540584in}{4.062402in}}%
\pgfpathlineto{\pgfqpoint{2.545598in}{4.063836in}}%
\pgfpathlineto{\pgfqpoint{2.568157in}{4.074449in}}%
\pgfpathlineto{\pgfqpoint{2.570664in}{4.077772in}}%
\pgfpathlineto{\pgfqpoint{2.575677in}{4.078429in}}%
\pgfpathlineto{\pgfqpoint{2.580691in}{4.082533in}}%
\pgfpathlineto{\pgfqpoint{2.590717in}{4.086541in}}%
\pgfpathlineto{\pgfqpoint{2.593224in}{4.089844in}}%
\pgfpathlineto{\pgfqpoint{2.595731in}{4.091270in}}%
\pgfpathlineto{\pgfqpoint{2.598237in}{4.094485in}}%
\pgfpathlineto{\pgfqpoint{2.600744in}{4.095873in}}%
\pgfpathlineto{\pgfqpoint{2.603251in}{4.098810in}}%
\pgfpathlineto{\pgfqpoint{2.615784in}{4.103972in}}%
\pgfpathlineto{\pgfqpoint{2.618290in}{4.106403in}}%
\pgfpathlineto{\pgfqpoint{2.623304in}{4.107514in}}%
\pgfpathlineto{\pgfqpoint{2.625810in}{4.108250in}}%
\pgfpathlineto{\pgfqpoint{2.628317in}{4.112425in}}%
\pgfpathlineto{\pgfqpoint{2.630824in}{4.112604in}}%
\pgfpathlineto{\pgfqpoint{2.633330in}{4.116850in}}%
\pgfpathlineto{\pgfqpoint{2.635837in}{4.117721in}}%
\pgfpathlineto{\pgfqpoint{2.638344in}{4.120480in}}%
\pgfpathlineto{\pgfqpoint{2.645864in}{4.122013in}}%
\pgfpathlineto{\pgfqpoint{2.658397in}{4.124036in}}%
\pgfpathlineto{\pgfqpoint{2.660904in}{4.125703in}}%
\pgfpathlineto{\pgfqpoint{2.663410in}{4.125703in}}%
\pgfpathlineto{\pgfqpoint{2.668424in}{4.129153in}}%
\pgfpathlineto{\pgfqpoint{2.670930in}{4.129316in}}%
\pgfpathlineto{\pgfqpoint{2.678450in}{4.133013in}}%
\pgfpathlineto{\pgfqpoint{2.690983in}{4.136632in}}%
\pgfpathlineto{\pgfqpoint{2.698503in}{4.142148in}}%
\pgfpathlineto{\pgfqpoint{2.703517in}{4.142299in}}%
\pgfpathlineto{\pgfqpoint{2.706023in}{4.144097in}}%
\pgfpathlineto{\pgfqpoint{2.716050in}{4.145286in}}%
\pgfpathlineto{\pgfqpoint{2.721063in}{4.149959in}}%
\pgfpathlineto{\pgfqpoint{2.728583in}{4.158804in}}%
\pgfpathlineto{\pgfqpoint{2.733597in}{4.162191in}}%
\pgfpathlineto{\pgfqpoint{2.743623in}{4.163261in}}%
\pgfpathlineto{\pgfqpoint{2.761170in}{4.169158in}}%
\pgfpathlineto{\pgfqpoint{2.763676in}{4.176101in}}%
\pgfpathlineto{\pgfqpoint{2.768690in}{4.178802in}}%
\pgfpathlineto{\pgfqpoint{2.771196in}{4.183960in}}%
\pgfpathlineto{\pgfqpoint{2.778716in}{4.185255in}}%
\pgfpathlineto{\pgfqpoint{2.781223in}{4.187700in}}%
\pgfpathlineto{\pgfqpoint{2.783730in}{4.194724in}}%
\pgfpathlineto{\pgfqpoint{2.791250in}{4.195609in}}%
\pgfpathlineto{\pgfqpoint{2.801276in}{4.199539in}}%
\pgfpathlineto{\pgfqpoint{2.813810in}{4.203064in}}%
\pgfpathlineto{\pgfqpoint{2.816316in}{4.205061in}}%
\pgfpathlineto{\pgfqpoint{2.818823in}{4.205061in}}%
\pgfpathlineto{\pgfqpoint{2.821329in}{4.209497in}}%
\pgfpathlineto{\pgfqpoint{2.826343in}{4.210208in}}%
\pgfpathlineto{\pgfqpoint{2.828849in}{4.213423in}}%
\pgfpathlineto{\pgfqpoint{2.831356in}{4.213522in}}%
\pgfpathlineto{\pgfqpoint{2.838876in}{4.218908in}}%
\pgfpathlineto{\pgfqpoint{2.841383in}{4.222533in}}%
\pgfpathlineto{\pgfqpoint{2.846396in}{4.224317in}}%
\pgfpathlineto{\pgfqpoint{2.851409in}{4.226912in}}%
\pgfpathlineto{\pgfqpoint{2.863943in}{4.230914in}}%
\pgfpathlineto{\pgfqpoint{2.871463in}{4.232435in}}%
\pgfpathlineto{\pgfqpoint{2.889009in}{4.247113in}}%
\pgfpathlineto{\pgfqpoint{2.899036in}{4.249388in}}%
\pgfpathlineto{\pgfqpoint{2.904049in}{4.250834in}}%
\pgfpathlineto{\pgfqpoint{2.906556in}{4.258350in}}%
\pgfpathlineto{\pgfqpoint{2.914076in}{4.266576in}}%
\pgfpathlineto{\pgfqpoint{2.924102in}{4.267523in}}%
\pgfpathlineto{\pgfqpoint{2.931622in}{4.268897in}}%
\pgfpathlineto{\pgfqpoint{2.934129in}{4.272109in}}%
\pgfpathlineto{\pgfqpoint{2.936636in}{4.273447in}}%
\pgfpathlineto{\pgfqpoint{2.939142in}{4.280255in}}%
\pgfpathlineto{\pgfqpoint{2.944156in}{4.282798in}}%
\pgfpathlineto{\pgfqpoint{2.949169in}{4.291918in}}%
\pgfpathlineto{\pgfqpoint{2.951676in}{4.292421in}}%
\pgfpathlineto{\pgfqpoint{2.956689in}{4.296579in}}%
\pgfpathlineto{\pgfqpoint{2.964209in}{4.298590in}}%
\pgfpathlineto{\pgfqpoint{2.969222in}{4.300757in}}%
\pgfpathlineto{\pgfqpoint{2.971729in}{4.305827in}}%
\pgfpathlineto{\pgfqpoint{2.974235in}{4.306117in}}%
\pgfpathlineto{\pgfqpoint{2.976742in}{4.311202in}}%
\pgfpathlineto{\pgfqpoint{2.981755in}{4.313273in}}%
\pgfpathlineto{\pgfqpoint{2.984262in}{4.316905in}}%
\pgfpathlineto{\pgfqpoint{2.994289in}{4.319660in}}%
\pgfpathlineto{\pgfqpoint{3.001809in}{4.325196in}}%
\pgfpathlineto{\pgfqpoint{3.006822in}{4.328384in}}%
\pgfpathlineto{\pgfqpoint{3.016849in}{4.338471in}}%
\pgfpathlineto{\pgfqpoint{3.026875in}{4.342177in}}%
\pgfpathlineto{\pgfqpoint{3.031888in}{4.342974in}}%
\pgfpathlineto{\pgfqpoint{3.036902in}{4.344233in}}%
\pgfpathlineto{\pgfqpoint{3.039408in}{4.349041in}}%
\pgfpathlineto{\pgfqpoint{3.041915in}{4.349582in}}%
\pgfpathlineto{\pgfqpoint{3.044422in}{4.352748in}}%
\pgfpathlineto{\pgfqpoint{3.051942in}{4.355249in}}%
\pgfpathlineto{\pgfqpoint{3.054448in}{4.358056in}}%
\pgfpathlineto{\pgfqpoint{3.056955in}{4.358741in}}%
\pgfpathlineto{\pgfqpoint{3.064475in}{4.366541in}}%
\pgfpathlineto{\pgfqpoint{3.069488in}{4.368728in}}%
\pgfpathlineto{\pgfqpoint{3.071995in}{4.368809in}}%
\pgfpathlineto{\pgfqpoint{3.074502in}{4.371365in}}%
\pgfpathlineto{\pgfqpoint{3.084528in}{4.373415in}}%
\pgfpathlineto{\pgfqpoint{3.087035in}{4.376520in}}%
\pgfpathlineto{\pgfqpoint{3.089542in}{4.377020in}}%
\pgfpathlineto{\pgfqpoint{3.092048in}{4.381189in}}%
\pgfpathlineto{\pgfqpoint{3.094555in}{4.381413in}}%
\pgfpathlineto{\pgfqpoint{3.099568in}{4.386243in}}%
\pgfpathlineto{\pgfqpoint{3.104581in}{4.389911in}}%
\pgfpathlineto{\pgfqpoint{3.107088in}{4.390409in}}%
\pgfpathlineto{\pgfqpoint{3.109595in}{4.395788in}}%
\pgfpathlineto{\pgfqpoint{3.117115in}{4.397975in}}%
\pgfpathlineto{\pgfqpoint{3.124635in}{4.399058in}}%
\pgfpathlineto{\pgfqpoint{3.129648in}{4.400033in}}%
\pgfpathlineto{\pgfqpoint{3.134661in}{4.402729in}}%
\pgfpathlineto{\pgfqpoint{3.137168in}{4.403881in}}%
\pgfpathlineto{\pgfqpoint{3.142181in}{4.409591in}}%
\pgfpathlineto{\pgfqpoint{3.147195in}{4.413698in}}%
\pgfpathlineto{\pgfqpoint{3.149701in}{4.416189in}}%
\pgfpathlineto{\pgfqpoint{3.152208in}{4.416403in}}%
\pgfpathlineto{\pgfqpoint{3.157221in}{4.420233in}}%
\pgfpathlineto{\pgfqpoint{3.162234in}{4.420947in}}%
\pgfpathlineto{\pgfqpoint{3.164741in}{4.423658in}}%
\pgfpathlineto{\pgfqpoint{3.167248in}{4.424066in}}%
\pgfpathlineto{\pgfqpoint{3.177274in}{4.430926in}}%
\pgfpathlineto{\pgfqpoint{3.182288in}{4.438115in}}%
\pgfpathlineto{\pgfqpoint{3.184794in}{4.438732in}}%
\pgfpathlineto{\pgfqpoint{3.187301in}{4.443224in}}%
\pgfpathlineto{\pgfqpoint{3.189808in}{4.443484in}}%
\pgfpathlineto{\pgfqpoint{3.192314in}{4.448033in}}%
\pgfpathlineto{\pgfqpoint{3.194821in}{4.448109in}}%
\pgfpathlineto{\pgfqpoint{3.199834in}{4.449523in}}%
\pgfpathlineto{\pgfqpoint{3.204848in}{4.451671in}}%
\pgfpathlineto{\pgfqpoint{3.207354in}{4.454577in}}%
\pgfpathlineto{\pgfqpoint{3.209861in}{4.455574in}}%
\pgfpathlineto{\pgfqpoint{3.212368in}{4.458844in}}%
\pgfpathlineto{\pgfqpoint{3.214874in}{4.464097in}}%
\pgfpathlineto{\pgfqpoint{3.219888in}{4.465568in}}%
\pgfpathlineto{\pgfqpoint{3.222394in}{4.471022in}}%
\pgfpathlineto{\pgfqpoint{3.229914in}{4.474078in}}%
\pgfpathlineto{\pgfqpoint{3.239941in}{4.492802in}}%
\pgfpathlineto{\pgfqpoint{3.244954in}{4.493543in}}%
\pgfpathlineto{\pgfqpoint{3.247461in}{4.496073in}}%
\pgfpathlineto{\pgfqpoint{3.254981in}{4.499451in}}%
\pgfpathlineto{\pgfqpoint{3.257487in}{4.504032in}}%
\pgfpathlineto{\pgfqpoint{3.259994in}{4.504050in}}%
\pgfpathlineto{\pgfqpoint{3.265007in}{4.520058in}}%
\pgfpathlineto{\pgfqpoint{3.267514in}{4.521121in}}%
\pgfpathlineto{\pgfqpoint{3.270021in}{4.524744in}}%
\pgfpathlineto{\pgfqpoint{3.280047in}{4.527607in}}%
\pgfpathlineto{\pgfqpoint{3.282554in}{4.528434in}}%
\pgfpathlineto{\pgfqpoint{3.287567in}{4.532977in}}%
\pgfpathlineto{\pgfqpoint{3.290074in}{4.532992in}}%
\pgfpathlineto{\pgfqpoint{3.292581in}{4.537055in}}%
\pgfpathlineto{\pgfqpoint{3.295087in}{4.538827in}}%
\pgfpathlineto{\pgfqpoint{3.297594in}{4.539036in}}%
\pgfpathlineto{\pgfqpoint{3.302607in}{4.543106in}}%
\pgfpathlineto{\pgfqpoint{3.307620in}{4.543325in}}%
\pgfpathlineto{\pgfqpoint{3.310127in}{4.547181in}}%
\pgfpathlineto{\pgfqpoint{3.312634in}{4.547452in}}%
\pgfpathlineto{\pgfqpoint{3.315140in}{4.552285in}}%
\pgfpathlineto{\pgfqpoint{3.317647in}{4.552781in}}%
\pgfpathlineto{\pgfqpoint{3.320154in}{4.555202in}}%
\pgfpathlineto{\pgfqpoint{3.327674in}{4.555650in}}%
\pgfpathlineto{\pgfqpoint{3.330180in}{4.564968in}}%
\pgfpathlineto{\pgfqpoint{3.332687in}{4.565430in}}%
\pgfpathlineto{\pgfqpoint{3.337700in}{4.573111in}}%
\pgfpathlineto{\pgfqpoint{3.342714in}{4.573783in}}%
\pgfpathlineto{\pgfqpoint{3.350234in}{4.578067in}}%
\pgfpathlineto{\pgfqpoint{3.352740in}{4.578530in}}%
\pgfpathlineto{\pgfqpoint{3.360260in}{4.582896in}}%
\pgfpathlineto{\pgfqpoint{3.390340in}{4.598272in}}%
\pgfpathlineto{\pgfqpoint{3.395353in}{4.601631in}}%
\pgfpathlineto{\pgfqpoint{3.402873in}{4.603916in}}%
\pgfpathlineto{\pgfqpoint{3.405380in}{4.608685in}}%
\pgfpathlineto{\pgfqpoint{3.407887in}{4.609449in}}%
\pgfpathlineto{\pgfqpoint{3.410393in}{4.612140in}}%
\pgfpathlineto{\pgfqpoint{3.412900in}{4.623015in}}%
\pgfpathlineto{\pgfqpoint{3.417913in}{4.624636in}}%
\pgfpathlineto{\pgfqpoint{3.420420in}{4.627725in}}%
\pgfpathlineto{\pgfqpoint{3.422927in}{4.628258in}}%
\pgfpathlineto{\pgfqpoint{3.425433in}{4.630172in}}%
\pgfpathlineto{\pgfqpoint{3.430447in}{4.631144in}}%
\pgfpathlineto{\pgfqpoint{3.432953in}{4.632759in}}%
\pgfpathlineto{\pgfqpoint{3.435460in}{4.637051in}}%
\pgfpathlineto{\pgfqpoint{3.445486in}{4.644030in}}%
\pgfpathlineto{\pgfqpoint{3.447993in}{4.644038in}}%
\pgfpathlineto{\pgfqpoint{3.450500in}{4.647308in}}%
\pgfpathlineto{\pgfqpoint{3.455513in}{4.649345in}}%
\pgfpathlineto{\pgfqpoint{3.460526in}{4.660051in}}%
\pgfpathlineto{\pgfqpoint{3.463033in}{4.660257in}}%
\pgfpathlineto{\pgfqpoint{3.468046in}{4.663039in}}%
\pgfpathlineto{\pgfqpoint{3.470553in}{4.663748in}}%
\pgfpathlineto{\pgfqpoint{3.473060in}{4.665699in}}%
\pgfpathlineto{\pgfqpoint{3.475566in}{4.669654in}}%
\pgfpathlineto{\pgfqpoint{3.478073in}{4.671381in}}%
\pgfpathlineto{\pgfqpoint{3.480580in}{4.677422in}}%
\pgfpathlineto{\pgfqpoint{3.483086in}{4.677575in}}%
\pgfpathlineto{\pgfqpoint{3.488100in}{4.681412in}}%
\pgfpathlineto{\pgfqpoint{3.493113in}{4.682360in}}%
\pgfpathlineto{\pgfqpoint{3.495620in}{4.682483in}}%
\pgfpathlineto{\pgfqpoint{3.498126in}{4.684259in}}%
\pgfpathlineto{\pgfqpoint{3.500633in}{4.687769in}}%
\pgfpathlineto{\pgfqpoint{3.505646in}{4.690513in}}%
\pgfpathlineto{\pgfqpoint{3.508153in}{4.692769in}}%
\pgfpathlineto{\pgfqpoint{3.510659in}{4.692788in}}%
\pgfpathlineto{\pgfqpoint{3.513166in}{4.699225in}}%
\pgfpathlineto{\pgfqpoint{3.518179in}{4.700226in}}%
\pgfpathlineto{\pgfqpoint{3.525699in}{4.709705in}}%
\pgfpathlineto{\pgfqpoint{3.528206in}{4.712785in}}%
\pgfpathlineto{\pgfqpoint{3.530713in}{4.720626in}}%
\pgfpathlineto{\pgfqpoint{3.535726in}{4.723298in}}%
\pgfpathlineto{\pgfqpoint{3.538233in}{4.723777in}}%
\pgfpathlineto{\pgfqpoint{3.540739in}{4.730453in}}%
\pgfpathlineto{\pgfqpoint{3.543246in}{4.730605in}}%
\pgfpathlineto{\pgfqpoint{3.545753in}{4.732002in}}%
\pgfpathlineto{\pgfqpoint{3.548259in}{4.736248in}}%
\pgfpathlineto{\pgfqpoint{3.550766in}{4.736885in}}%
\pgfpathlineto{\pgfqpoint{3.553273in}{4.739015in}}%
\pgfpathlineto{\pgfqpoint{3.558286in}{4.739363in}}%
\pgfpathlineto{\pgfqpoint{3.563299in}{4.747904in}}%
\pgfpathlineto{\pgfqpoint{3.565806in}{4.749308in}}%
\pgfpathlineto{\pgfqpoint{3.568313in}{4.752050in}}%
\pgfpathlineto{\pgfqpoint{3.573326in}{4.752606in}}%
\pgfpathlineto{\pgfqpoint{3.575832in}{4.754465in}}%
\pgfpathlineto{\pgfqpoint{3.583352in}{4.755745in}}%
\pgfpathlineto{\pgfqpoint{3.593379in}{4.759237in}}%
\pgfpathlineto{\pgfqpoint{3.595886in}{4.766270in}}%
\pgfpathlineto{\pgfqpoint{3.610926in}{4.770663in}}%
\pgfpathlineto{\pgfqpoint{3.613432in}{4.774731in}}%
\pgfpathlineto{\pgfqpoint{3.615939in}{4.775259in}}%
\pgfpathlineto{\pgfqpoint{3.618446in}{4.780113in}}%
\pgfpathlineto{\pgfqpoint{3.623459in}{4.783639in}}%
\pgfpathlineto{\pgfqpoint{3.628472in}{4.785046in}}%
\pgfpathlineto{\pgfqpoint{3.641005in}{4.796875in}}%
\pgfpathlineto{\pgfqpoint{3.646019in}{4.797558in}}%
\pgfpathlineto{\pgfqpoint{3.648525in}{4.799982in}}%
\pgfpathlineto{\pgfqpoint{3.653539in}{4.801165in}}%
\pgfpathlineto{\pgfqpoint{3.656045in}{4.803679in}}%
\pgfpathlineto{\pgfqpoint{3.658552in}{4.804173in}}%
\pgfpathlineto{\pgfqpoint{3.661059in}{4.806707in}}%
\pgfpathlineto{\pgfqpoint{3.663565in}{4.811225in}}%
\pgfpathlineto{\pgfqpoint{3.668579in}{4.812324in}}%
\pgfpathlineto{\pgfqpoint{3.671085in}{4.813465in}}%
\pgfpathlineto{\pgfqpoint{3.673592in}{4.818151in}}%
\pgfpathlineto{\pgfqpoint{3.676099in}{4.818644in}}%
\pgfpathlineto{\pgfqpoint{3.681112in}{4.824088in}}%
\pgfpathlineto{\pgfqpoint{3.683619in}{4.827458in}}%
\pgfpathlineto{\pgfqpoint{3.696152in}{4.833263in}}%
\pgfpathlineto{\pgfqpoint{3.701165in}{4.836423in}}%
\pgfpathlineto{\pgfqpoint{3.706178in}{4.837961in}}%
\pgfpathlineto{\pgfqpoint{3.708685in}{4.839040in}}%
\pgfpathlineto{\pgfqpoint{3.711192in}{4.842198in}}%
\pgfpathlineto{\pgfqpoint{3.713698in}{4.842279in}}%
\pgfpathlineto{\pgfqpoint{3.721218in}{4.848437in}}%
\pgfpathlineto{\pgfqpoint{3.723725in}{4.849039in}}%
\pgfpathlineto{\pgfqpoint{3.731245in}{4.860761in}}%
\pgfpathlineto{\pgfqpoint{3.733752in}{4.864393in}}%
\pgfpathlineto{\pgfqpoint{3.741272in}{4.868103in}}%
\pgfpathlineto{\pgfqpoint{3.753805in}{4.869796in}}%
\pgfpathlineto{\pgfqpoint{3.758818in}{4.883825in}}%
\pgfpathlineto{\pgfqpoint{3.761325in}{4.883967in}}%
\pgfpathlineto{\pgfqpoint{3.763832in}{4.888714in}}%
\pgfpathlineto{\pgfqpoint{3.776365in}{4.896902in}}%
\pgfpathlineto{\pgfqpoint{3.778871in}{4.902758in}}%
\pgfpathlineto{\pgfqpoint{3.801431in}{4.910926in}}%
\pgfpathlineto{\pgfqpoint{3.821485in}{4.917730in}}%
\pgfpathlineto{\pgfqpoint{3.826498in}{4.924327in}}%
\pgfpathlineto{\pgfqpoint{3.829005in}{4.925251in}}%
\pgfpathlineto{\pgfqpoint{3.831511in}{4.927539in}}%
\pgfpathlineto{\pgfqpoint{3.834018in}{4.936090in}}%
\pgfpathlineto{\pgfqpoint{3.841538in}{4.944119in}}%
\pgfpathlineto{\pgfqpoint{3.844044in}{4.950823in}}%
\pgfpathlineto{\pgfqpoint{3.859084in}{4.954119in}}%
\pgfpathlineto{\pgfqpoint{3.861591in}{4.958668in}}%
\pgfpathlineto{\pgfqpoint{3.869111in}{4.961544in}}%
\pgfpathlineto{\pgfqpoint{3.874124in}{4.974685in}}%
\pgfpathlineto{\pgfqpoint{3.876631in}{4.975689in}}%
\pgfpathlineto{\pgfqpoint{3.879138in}{4.984328in}}%
\pgfpathlineto{\pgfqpoint{3.884151in}{4.989661in}}%
\pgfpathlineto{\pgfqpoint{3.889164in}{4.990954in}}%
\pgfpathlineto{\pgfqpoint{3.894178in}{4.992464in}}%
\pgfpathlineto{\pgfqpoint{3.899191in}{5.000210in}}%
\pgfpathlineto{\pgfqpoint{3.904204in}{5.001654in}}%
\pgfpathlineto{\pgfqpoint{3.906711in}{5.007049in}}%
\pgfpathlineto{\pgfqpoint{3.909217in}{5.009370in}}%
\pgfpathlineto{\pgfqpoint{3.914231in}{5.027262in}}%
\pgfpathlineto{\pgfqpoint{3.916737in}{5.028513in}}%
\pgfpathlineto{\pgfqpoint{3.919244in}{5.032759in}}%
\pgfpathlineto{\pgfqpoint{3.921751in}{5.040188in}}%
\pgfpathlineto{\pgfqpoint{3.926764in}{5.043436in}}%
\pgfpathlineto{\pgfqpoint{3.929271in}{5.043662in}}%
\pgfpathlineto{\pgfqpoint{3.931777in}{5.050880in}}%
\pgfpathlineto{\pgfqpoint{3.934284in}{5.051704in}}%
\pgfpathlineto{\pgfqpoint{3.941804in}{5.059023in}}%
\pgfpathlineto{\pgfqpoint{3.946817in}{5.068978in}}%
\pgfpathlineto{\pgfqpoint{3.951831in}{5.069642in}}%
\pgfpathlineto{\pgfqpoint{3.954337in}{5.073239in}}%
\pgfpathlineto{\pgfqpoint{3.959351in}{5.075490in}}%
\pgfpathlineto{\pgfqpoint{3.961857in}{5.079750in}}%
\pgfpathlineto{\pgfqpoint{3.964364in}{5.079860in}}%
\pgfpathlineto{\pgfqpoint{3.969377in}{5.081847in}}%
\pgfpathlineto{\pgfqpoint{3.976897in}{5.095623in}}%
\pgfpathlineto{\pgfqpoint{3.981910in}{5.097067in}}%
\pgfpathlineto{\pgfqpoint{3.986924in}{5.102073in}}%
\pgfpathlineto{\pgfqpoint{3.999457in}{5.109955in}}%
\pgfpathlineto{\pgfqpoint{4.001964in}{5.114631in}}%
\pgfpathlineto{\pgfqpoint{4.004470in}{5.116024in}}%
\pgfpathlineto{\pgfqpoint{4.011990in}{5.117158in}}%
\pgfpathlineto{\pgfqpoint{4.017004in}{5.123399in}}%
\pgfpathlineto{\pgfqpoint{4.019510in}{5.129884in}}%
\pgfpathlineto{\pgfqpoint{4.022017in}{5.130672in}}%
\pgfpathlineto{\pgfqpoint{4.024524in}{5.135550in}}%
\pgfpathlineto{\pgfqpoint{4.029537in}{5.138128in}}%
\pgfpathlineto{\pgfqpoint{4.034550in}{5.150383in}}%
\pgfpathlineto{\pgfqpoint{4.037057in}{5.150976in}}%
\pgfpathlineto{\pgfqpoint{4.039564in}{5.155564in}}%
\pgfpathlineto{\pgfqpoint{4.042070in}{5.157291in}}%
\pgfpathlineto{\pgfqpoint{4.044577in}{5.160634in}}%
\pgfpathlineto{\pgfqpoint{4.047083in}{5.175880in}}%
\pgfpathlineto{\pgfqpoint{4.052097in}{5.182866in}}%
\pgfpathlineto{\pgfqpoint{4.057110in}{5.185034in}}%
\pgfpathlineto{\pgfqpoint{4.059617in}{5.188398in}}%
\pgfpathlineto{\pgfqpoint{4.064630in}{5.189637in}}%
\pgfpathlineto{\pgfqpoint{4.067137in}{5.197708in}}%
\pgfpathlineto{\pgfqpoint{4.072150in}{5.198539in}}%
\pgfpathlineto{\pgfqpoint{4.074657in}{5.201559in}}%
\pgfpathlineto{\pgfqpoint{4.077163in}{5.207945in}}%
\pgfpathlineto{\pgfqpoint{4.082177in}{5.211152in}}%
\pgfpathlineto{\pgfqpoint{4.084683in}{5.216929in}}%
\pgfpathlineto{\pgfqpoint{4.089697in}{5.218359in}}%
\pgfpathlineto{\pgfqpoint{4.092203in}{5.219686in}}%
\pgfpathlineto{\pgfqpoint{4.097217in}{5.230251in}}%
\pgfpathlineto{\pgfqpoint{4.099723in}{5.230559in}}%
\pgfpathlineto{\pgfqpoint{4.104737in}{5.244363in}}%
\pgfpathlineto{\pgfqpoint{4.109750in}{5.247985in}}%
\pgfpathlineto{\pgfqpoint{4.117270in}{5.252102in}}%
\pgfpathlineto{\pgfqpoint{4.122283in}{5.262811in}}%
\pgfpathlineto{\pgfqpoint{4.124790in}{5.265038in}}%
\pgfpathlineto{\pgfqpoint{4.132310in}{5.266104in}}%
\pgfpathlineto{\pgfqpoint{4.137323in}{5.270183in}}%
\pgfpathlineto{\pgfqpoint{4.142336in}{5.274320in}}%
\pgfpathlineto{\pgfqpoint{4.144843in}{5.275949in}}%
\pgfpathlineto{\pgfqpoint{4.147350in}{5.283108in}}%
\pgfpathlineto{\pgfqpoint{4.154870in}{5.286647in}}%
\pgfpathlineto{\pgfqpoint{4.157376in}{5.287123in}}%
\pgfpathlineto{\pgfqpoint{4.159883in}{5.289308in}}%
\pgfpathlineto{\pgfqpoint{4.162390in}{5.295455in}}%
\pgfpathlineto{\pgfqpoint{4.167403in}{5.298876in}}%
\pgfpathlineto{\pgfqpoint{4.169910in}{5.305275in}}%
\pgfpathlineto{\pgfqpoint{4.169910in}{5.305275in}}%
\pgfusepath{stroke}%
\end{pgfscope}%
\begin{pgfscope}%
\pgfpathrectangle{\pgfqpoint{0.708220in}{3.210823in}}{\pgfqpoint{5.013309in}{2.094453in}}%
\pgfusepath{clip}%
\pgfsetbuttcap%
\pgfsetroundjoin%
\pgfsetlinewidth{1.003750pt}%
\definecolor{currentstroke}{rgb}{0.811765,0.125490,0.125490}%
\pgfsetstrokecolor{currentstroke}%
\pgfsetdash{{3.700000pt}{1.600000pt}}{0.000000pt}%
\pgfpathmoveto{\pgfqpoint{0.724376in}{3.200823in}}%
\pgfpathlineto{\pgfqpoint{0.728273in}{3.205525in}}%
\pgfpathlineto{\pgfqpoint{0.738300in}{3.213445in}}%
\pgfpathlineto{\pgfqpoint{0.743313in}{3.213816in}}%
\pgfpathlineto{\pgfqpoint{0.750833in}{3.219260in}}%
\pgfpathlineto{\pgfqpoint{0.753340in}{3.230054in}}%
\pgfpathlineto{\pgfqpoint{0.755846in}{3.235043in}}%
\pgfpathlineto{\pgfqpoint{0.763366in}{3.238407in}}%
\pgfpathlineto{\pgfqpoint{0.765873in}{3.238524in}}%
\pgfpathlineto{\pgfqpoint{0.770886in}{3.240758in}}%
\pgfpathlineto{\pgfqpoint{0.788433in}{3.242650in}}%
\pgfpathlineto{\pgfqpoint{0.793446in}{3.245869in}}%
\pgfpathlineto{\pgfqpoint{0.803473in}{3.246818in}}%
\pgfpathlineto{\pgfqpoint{0.805979in}{3.249249in}}%
\pgfpathlineto{\pgfqpoint{0.808486in}{3.250098in}}%
\pgfpathlineto{\pgfqpoint{0.810993in}{3.258566in}}%
\pgfpathlineto{\pgfqpoint{0.821019in}{3.265417in}}%
\pgfpathlineto{\pgfqpoint{0.823526in}{3.265541in}}%
\pgfpathlineto{\pgfqpoint{0.828539in}{3.266982in}}%
\pgfpathlineto{\pgfqpoint{0.838566in}{3.268362in}}%
\pgfpathlineto{\pgfqpoint{0.843579in}{3.270338in}}%
\pgfpathlineto{\pgfqpoint{0.858619in}{3.274248in}}%
\pgfpathlineto{\pgfqpoint{0.861126in}{3.274580in}}%
\pgfpathlineto{\pgfqpoint{0.863632in}{3.283121in}}%
\pgfpathlineto{\pgfqpoint{0.868646in}{3.286178in}}%
\pgfpathlineto{\pgfqpoint{0.873659in}{3.286973in}}%
\pgfpathlineto{\pgfqpoint{0.883686in}{3.290438in}}%
\pgfpathlineto{\pgfqpoint{0.891206in}{3.291364in}}%
\pgfpathlineto{\pgfqpoint{0.896219in}{3.292561in}}%
\pgfpathlineto{\pgfqpoint{0.908752in}{3.293941in}}%
\pgfpathlineto{\pgfqpoint{0.921285in}{3.295896in}}%
\pgfpathlineto{\pgfqpoint{0.931312in}{3.298798in}}%
\pgfpathlineto{\pgfqpoint{0.936325in}{3.302156in}}%
\pgfpathlineto{\pgfqpoint{0.938832in}{3.302257in}}%
\pgfpathlineto{\pgfqpoint{0.941339in}{3.304279in}}%
\pgfpathlineto{\pgfqpoint{0.951365in}{3.305568in}}%
\pgfpathlineto{\pgfqpoint{0.953872in}{3.307239in}}%
\pgfpathlineto{\pgfqpoint{0.956379in}{3.307318in}}%
\pgfpathlineto{\pgfqpoint{0.961392in}{3.308991in}}%
\pgfpathlineto{\pgfqpoint{0.963899in}{3.309108in}}%
\pgfpathlineto{\pgfqpoint{0.968912in}{3.310860in}}%
\pgfpathlineto{\pgfqpoint{0.978939in}{3.311929in}}%
\pgfpathlineto{\pgfqpoint{0.993978in}{3.312954in}}%
\pgfpathlineto{\pgfqpoint{0.998992in}{3.314349in}}%
\pgfpathlineto{\pgfqpoint{1.016538in}{3.317013in}}%
\pgfpathlineto{\pgfqpoint{1.019045in}{3.321013in}}%
\pgfpathlineto{\pgfqpoint{1.026565in}{3.322273in}}%
\pgfpathlineto{\pgfqpoint{1.064165in}{3.328244in}}%
\pgfpathlineto{\pgfqpoint{1.079205in}{3.329211in}}%
\pgfpathlineto{\pgfqpoint{1.091738in}{3.331111in}}%
\pgfpathlineto{\pgfqpoint{1.101765in}{3.335431in}}%
\pgfpathlineto{\pgfqpoint{1.109285in}{3.337920in}}%
\pgfpathlineto{\pgfqpoint{1.119311in}{3.340405in}}%
\pgfpathlineto{\pgfqpoint{1.151898in}{3.347048in}}%
\pgfpathlineto{\pgfqpoint{1.161924in}{3.348946in}}%
\pgfpathlineto{\pgfqpoint{1.164431in}{3.350702in}}%
\pgfpathlineto{\pgfqpoint{1.171951in}{3.351308in}}%
\pgfpathlineto{\pgfqpoint{1.179471in}{3.356789in}}%
\pgfpathlineto{\pgfqpoint{1.197017in}{3.359003in}}%
\pgfpathlineto{\pgfqpoint{1.212057in}{3.360702in}}%
\pgfpathlineto{\pgfqpoint{1.217071in}{3.362143in}}%
\pgfpathlineto{\pgfqpoint{1.222084in}{3.366103in}}%
\pgfpathlineto{\pgfqpoint{1.234617in}{3.370990in}}%
\pgfpathlineto{\pgfqpoint{1.249657in}{3.377504in}}%
\pgfpathlineto{\pgfqpoint{1.257177in}{3.378746in}}%
\pgfpathlineto{\pgfqpoint{1.264697in}{3.380287in}}%
\pgfpathlineto{\pgfqpoint{1.267204in}{3.382812in}}%
\pgfpathlineto{\pgfqpoint{1.269710in}{3.382913in}}%
\pgfpathlineto{\pgfqpoint{1.274724in}{3.385139in}}%
\pgfpathlineto{\pgfqpoint{1.279737in}{3.386352in}}%
\pgfpathlineto{\pgfqpoint{1.294777in}{3.391410in}}%
\pgfpathlineto{\pgfqpoint{1.302297in}{3.392770in}}%
\pgfpathlineto{\pgfqpoint{1.304804in}{3.396881in}}%
\pgfpathlineto{\pgfqpoint{1.317337in}{3.399495in}}%
\pgfpathlineto{\pgfqpoint{1.327363in}{3.401359in}}%
\pgfpathlineto{\pgfqpoint{1.329870in}{3.402923in}}%
\pgfpathlineto{\pgfqpoint{1.332377in}{3.406953in}}%
\pgfpathlineto{\pgfqpoint{1.337390in}{3.408522in}}%
\pgfpathlineto{\pgfqpoint{1.339897in}{3.410912in}}%
\pgfpathlineto{\pgfqpoint{1.342403in}{3.410944in}}%
\pgfpathlineto{\pgfqpoint{1.344910in}{3.414719in}}%
\pgfpathlineto{\pgfqpoint{1.369977in}{3.419260in}}%
\pgfpathlineto{\pgfqpoint{1.372483in}{3.421928in}}%
\pgfpathlineto{\pgfqpoint{1.382510in}{3.423217in}}%
\pgfpathlineto{\pgfqpoint{1.385017in}{3.425776in}}%
\pgfpathlineto{\pgfqpoint{1.390030in}{3.427464in}}%
\pgfpathlineto{\pgfqpoint{1.395043in}{3.429328in}}%
\pgfpathlineto{\pgfqpoint{1.410083in}{3.432053in}}%
\pgfpathlineto{\pgfqpoint{1.415096in}{3.435033in}}%
\pgfpathlineto{\pgfqpoint{1.417603in}{3.437725in}}%
\pgfpathlineto{\pgfqpoint{1.435150in}{3.440699in}}%
\pgfpathlineto{\pgfqpoint{1.440163in}{3.443649in}}%
\pgfpathlineto{\pgfqpoint{1.445176in}{3.447095in}}%
\pgfpathlineto{\pgfqpoint{1.452696in}{3.448123in}}%
\pgfpathlineto{\pgfqpoint{1.462723in}{3.453103in}}%
\pgfpathlineto{\pgfqpoint{1.470243in}{3.454394in}}%
\pgfpathlineto{\pgfqpoint{1.475256in}{3.458145in}}%
\pgfpathlineto{\pgfqpoint{1.487789in}{3.461371in}}%
\pgfpathlineto{\pgfqpoint{1.490296in}{3.464967in}}%
\pgfpathlineto{\pgfqpoint{1.502829in}{3.466157in}}%
\pgfpathlineto{\pgfqpoint{1.507843in}{3.466876in}}%
\pgfpathlineto{\pgfqpoint{1.512856in}{3.470782in}}%
\pgfpathlineto{\pgfqpoint{1.517869in}{3.471730in}}%
\pgfpathlineto{\pgfqpoint{1.522883in}{3.473030in}}%
\pgfpathlineto{\pgfqpoint{1.527896in}{3.474321in}}%
\pgfpathlineto{\pgfqpoint{1.532909in}{3.478482in}}%
\pgfpathlineto{\pgfqpoint{1.537922in}{3.480960in}}%
\pgfpathlineto{\pgfqpoint{1.542936in}{3.481166in}}%
\pgfpathlineto{\pgfqpoint{1.550456in}{3.484297in}}%
\pgfpathlineto{\pgfqpoint{1.552962in}{3.489265in}}%
\pgfpathlineto{\pgfqpoint{1.562989in}{3.494528in}}%
\pgfpathlineto{\pgfqpoint{1.565496in}{3.494770in}}%
\pgfpathlineto{\pgfqpoint{1.583042in}{3.509652in}}%
\pgfpathlineto{\pgfqpoint{1.585549in}{3.511083in}}%
\pgfpathlineto{\pgfqpoint{1.590562in}{3.517217in}}%
\pgfpathlineto{\pgfqpoint{1.595576in}{3.518718in}}%
\pgfpathlineto{\pgfqpoint{1.598082in}{3.518832in}}%
\pgfpathlineto{\pgfqpoint{1.603095in}{3.523039in}}%
\pgfpathlineto{\pgfqpoint{1.608109in}{3.524672in}}%
\pgfpathlineto{\pgfqpoint{1.610615in}{3.527384in}}%
\pgfpathlineto{\pgfqpoint{1.615629in}{3.528923in}}%
\pgfpathlineto{\pgfqpoint{1.618135in}{3.529379in}}%
\pgfpathlineto{\pgfqpoint{1.623149in}{3.535016in}}%
\pgfpathlineto{\pgfqpoint{1.633175in}{3.537749in}}%
\pgfpathlineto{\pgfqpoint{1.638189in}{3.540268in}}%
\pgfpathlineto{\pgfqpoint{1.645709in}{3.542721in}}%
\pgfpathlineto{\pgfqpoint{1.663255in}{3.550522in}}%
\pgfpathlineto{\pgfqpoint{1.675788in}{3.554564in}}%
\pgfpathlineto{\pgfqpoint{1.678295in}{3.557611in}}%
\pgfpathlineto{\pgfqpoint{1.695842in}{3.560551in}}%
\pgfpathlineto{\pgfqpoint{1.698348in}{3.564794in}}%
\pgfpathlineto{\pgfqpoint{1.703362in}{3.565977in}}%
\pgfpathlineto{\pgfqpoint{1.708375in}{3.570111in}}%
\pgfpathlineto{\pgfqpoint{1.713388in}{3.571727in}}%
\pgfpathlineto{\pgfqpoint{1.715895in}{3.571944in}}%
\pgfpathlineto{\pgfqpoint{1.718402in}{3.576139in}}%
\pgfpathlineto{\pgfqpoint{1.723415in}{3.577124in}}%
\pgfpathlineto{\pgfqpoint{1.728428in}{3.579798in}}%
\pgfpathlineto{\pgfqpoint{1.730935in}{3.581227in}}%
\pgfpathlineto{\pgfqpoint{1.735948in}{3.588288in}}%
\pgfpathlineto{\pgfqpoint{1.740961in}{3.588725in}}%
\pgfpathlineto{\pgfqpoint{1.743468in}{3.596444in}}%
\pgfpathlineto{\pgfqpoint{1.750988in}{3.598319in}}%
\pgfpathlineto{\pgfqpoint{1.753495in}{3.602882in}}%
\pgfpathlineto{\pgfqpoint{1.761015in}{3.606381in}}%
\pgfpathlineto{\pgfqpoint{1.763521in}{3.608575in}}%
\pgfpathlineto{\pgfqpoint{1.768535in}{3.609148in}}%
\pgfpathlineto{\pgfqpoint{1.771041in}{3.612547in}}%
\pgfpathlineto{\pgfqpoint{1.778561in}{3.614575in}}%
\pgfpathlineto{\pgfqpoint{1.783575in}{3.619981in}}%
\pgfpathlineto{\pgfqpoint{1.788588in}{3.625767in}}%
\pgfpathlineto{\pgfqpoint{1.791095in}{3.625846in}}%
\pgfpathlineto{\pgfqpoint{1.806134in}{3.634597in}}%
\pgfpathlineto{\pgfqpoint{1.811148in}{3.642978in}}%
\pgfpathlineto{\pgfqpoint{1.818668in}{3.644832in}}%
\pgfpathlineto{\pgfqpoint{1.833708in}{3.647788in}}%
\pgfpathlineto{\pgfqpoint{1.836214in}{3.649474in}}%
\pgfpathlineto{\pgfqpoint{1.838721in}{3.649500in}}%
\pgfpathlineto{\pgfqpoint{1.841228in}{3.651667in}}%
\pgfpathlineto{\pgfqpoint{1.848748in}{3.652406in}}%
\pgfpathlineto{\pgfqpoint{1.853761in}{3.654909in}}%
\pgfpathlineto{\pgfqpoint{1.856268in}{3.655622in}}%
\pgfpathlineto{\pgfqpoint{1.858774in}{3.657776in}}%
\pgfpathlineto{\pgfqpoint{1.866294in}{3.659148in}}%
\pgfpathlineto{\pgfqpoint{1.871307in}{3.660921in}}%
\pgfpathlineto{\pgfqpoint{1.876321in}{3.663128in}}%
\pgfpathlineto{\pgfqpoint{1.881334in}{3.663605in}}%
\pgfpathlineto{\pgfqpoint{1.883841in}{3.668099in}}%
\pgfpathlineto{\pgfqpoint{1.893867in}{3.670518in}}%
\pgfpathlineto{\pgfqpoint{1.901387in}{3.675062in}}%
\pgfpathlineto{\pgfqpoint{1.906401in}{3.678574in}}%
\pgfpathlineto{\pgfqpoint{1.908907in}{3.682183in}}%
\pgfpathlineto{\pgfqpoint{1.911414in}{3.682252in}}%
\pgfpathlineto{\pgfqpoint{1.921441in}{3.686957in}}%
\pgfpathlineto{\pgfqpoint{1.923947in}{3.687445in}}%
\pgfpathlineto{\pgfqpoint{1.933974in}{3.693617in}}%
\pgfpathlineto{\pgfqpoint{1.944000in}{3.697824in}}%
\pgfpathlineto{\pgfqpoint{1.951520in}{3.702963in}}%
\pgfpathlineto{\pgfqpoint{1.966560in}{3.706517in}}%
\pgfpathlineto{\pgfqpoint{1.969067in}{3.706670in}}%
\pgfpathlineto{\pgfqpoint{1.974080in}{3.710397in}}%
\pgfpathlineto{\pgfqpoint{1.976587in}{3.713672in}}%
\pgfpathlineto{\pgfqpoint{1.979094in}{3.714691in}}%
\pgfpathlineto{\pgfqpoint{1.984107in}{3.718697in}}%
\pgfpathlineto{\pgfqpoint{1.986614in}{3.720697in}}%
\pgfpathlineto{\pgfqpoint{1.989120in}{3.724573in}}%
\pgfpathlineto{\pgfqpoint{1.994134in}{3.727520in}}%
\pgfpathlineto{\pgfqpoint{1.996640in}{3.729954in}}%
\pgfpathlineto{\pgfqpoint{2.001654in}{3.730444in}}%
\pgfpathlineto{\pgfqpoint{2.006667in}{3.735200in}}%
\pgfpathlineto{\pgfqpoint{2.011680in}{3.736736in}}%
\pgfpathlineto{\pgfqpoint{2.019200in}{3.737850in}}%
\pgfpathlineto{\pgfqpoint{2.024213in}{3.741559in}}%
\pgfpathlineto{\pgfqpoint{2.031733in}{3.742933in}}%
\pgfpathlineto{\pgfqpoint{2.046773in}{3.746952in}}%
\pgfpathlineto{\pgfqpoint{2.059307in}{3.754920in}}%
\pgfpathlineto{\pgfqpoint{2.061813in}{3.755067in}}%
\pgfpathlineto{\pgfqpoint{2.069333in}{3.759424in}}%
\pgfpathlineto{\pgfqpoint{2.074346in}{3.765090in}}%
\pgfpathlineto{\pgfqpoint{2.076853in}{3.765480in}}%
\pgfpathlineto{\pgfqpoint{2.079360in}{3.767116in}}%
\pgfpathlineto{\pgfqpoint{2.081866in}{3.767336in}}%
\pgfpathlineto{\pgfqpoint{2.086880in}{3.772117in}}%
\pgfpathlineto{\pgfqpoint{2.089386in}{3.772565in}}%
\pgfpathlineto{\pgfqpoint{2.091893in}{3.775810in}}%
\pgfpathlineto{\pgfqpoint{2.094400in}{3.776813in}}%
\pgfpathlineto{\pgfqpoint{2.099413in}{3.781893in}}%
\pgfpathlineto{\pgfqpoint{2.101920in}{3.784014in}}%
\pgfpathlineto{\pgfqpoint{2.106933in}{3.784754in}}%
\pgfpathlineto{\pgfqpoint{2.109440in}{3.787297in}}%
\pgfpathlineto{\pgfqpoint{2.116960in}{3.789747in}}%
\pgfpathlineto{\pgfqpoint{2.119466in}{3.791080in}}%
\pgfpathlineto{\pgfqpoint{2.121973in}{3.794811in}}%
\pgfpathlineto{\pgfqpoint{2.124480in}{3.794929in}}%
\pgfpathlineto{\pgfqpoint{2.126986in}{3.798919in}}%
\pgfpathlineto{\pgfqpoint{2.129493in}{3.799264in}}%
\pgfpathlineto{\pgfqpoint{2.134506in}{3.801991in}}%
\pgfpathlineto{\pgfqpoint{2.144533in}{3.805437in}}%
\pgfpathlineto{\pgfqpoint{2.154559in}{3.814941in}}%
\pgfpathlineto{\pgfqpoint{2.174613in}{3.819684in}}%
\pgfpathlineto{\pgfqpoint{2.177119in}{3.823199in}}%
\pgfpathlineto{\pgfqpoint{2.182133in}{3.826057in}}%
\pgfpathlineto{\pgfqpoint{2.184639in}{3.829297in}}%
\pgfpathlineto{\pgfqpoint{2.194666in}{3.833687in}}%
\pgfpathlineto{\pgfqpoint{2.197173in}{3.836510in}}%
\pgfpathlineto{\pgfqpoint{2.204693in}{3.837292in}}%
\pgfpathlineto{\pgfqpoint{2.207199in}{3.840752in}}%
\pgfpathlineto{\pgfqpoint{2.209706in}{3.846679in}}%
\pgfpathlineto{\pgfqpoint{2.214719in}{3.849420in}}%
\pgfpathlineto{\pgfqpoint{2.219732in}{3.858026in}}%
\pgfpathlineto{\pgfqpoint{2.222239in}{3.858715in}}%
\pgfpathlineto{\pgfqpoint{2.227252in}{3.861569in}}%
\pgfpathlineto{\pgfqpoint{2.229759in}{3.861719in}}%
\pgfpathlineto{\pgfqpoint{2.232266in}{3.863847in}}%
\pgfpathlineto{\pgfqpoint{2.234772in}{3.867973in}}%
\pgfpathlineto{\pgfqpoint{2.242292in}{3.870715in}}%
\pgfpathlineto{\pgfqpoint{2.249812in}{3.872121in}}%
\pgfpathlineto{\pgfqpoint{2.252319in}{3.874563in}}%
\pgfpathlineto{\pgfqpoint{2.254826in}{3.875056in}}%
\pgfpathlineto{\pgfqpoint{2.257332in}{3.877641in}}%
\pgfpathlineto{\pgfqpoint{2.274879in}{3.882262in}}%
\pgfpathlineto{\pgfqpoint{2.277385in}{3.884851in}}%
\pgfpathlineto{\pgfqpoint{2.279892in}{3.884887in}}%
\pgfpathlineto{\pgfqpoint{2.282399in}{3.888357in}}%
\pgfpathlineto{\pgfqpoint{2.284905in}{3.888781in}}%
\pgfpathlineto{\pgfqpoint{2.287412in}{3.890734in}}%
\pgfpathlineto{\pgfqpoint{2.307465in}{3.893746in}}%
\pgfpathlineto{\pgfqpoint{2.309972in}{3.898041in}}%
\pgfpathlineto{\pgfqpoint{2.314985in}{3.899104in}}%
\pgfpathlineto{\pgfqpoint{2.325012in}{3.904640in}}%
\pgfpathlineto{\pgfqpoint{2.327519in}{3.904670in}}%
\pgfpathlineto{\pgfqpoint{2.335039in}{3.909397in}}%
\pgfpathlineto{\pgfqpoint{2.337545in}{3.909786in}}%
\pgfpathlineto{\pgfqpoint{2.342559in}{3.914624in}}%
\pgfpathlineto{\pgfqpoint{2.352585in}{3.916712in}}%
\pgfpathlineto{\pgfqpoint{2.355092in}{3.923692in}}%
\pgfpathlineto{\pgfqpoint{2.367625in}{3.926429in}}%
\pgfpathlineto{\pgfqpoint{2.370132in}{3.930712in}}%
\pgfpathlineto{\pgfqpoint{2.375145in}{3.931923in}}%
\pgfpathlineto{\pgfqpoint{2.377652in}{3.932892in}}%
\pgfpathlineto{\pgfqpoint{2.382665in}{3.937346in}}%
\pgfpathlineto{\pgfqpoint{2.390185in}{3.937850in}}%
\pgfpathlineto{\pgfqpoint{2.395198in}{3.944169in}}%
\pgfpathlineto{\pgfqpoint{2.400212in}{3.947195in}}%
\pgfpathlineto{\pgfqpoint{2.405225in}{3.950836in}}%
\pgfpathlineto{\pgfqpoint{2.410238in}{3.952331in}}%
\pgfpathlineto{\pgfqpoint{2.412745in}{3.955864in}}%
\pgfpathlineto{\pgfqpoint{2.420265in}{3.958873in}}%
\pgfpathlineto{\pgfqpoint{2.422771in}{3.958963in}}%
\pgfpathlineto{\pgfqpoint{2.430291in}{3.964248in}}%
\pgfpathlineto{\pgfqpoint{2.445331in}{3.969525in}}%
\pgfpathlineto{\pgfqpoint{2.457865in}{3.971689in}}%
\pgfpathlineto{\pgfqpoint{2.465385in}{3.979132in}}%
\pgfpathlineto{\pgfqpoint{2.470398in}{3.980659in}}%
\pgfpathlineto{\pgfqpoint{2.472905in}{3.983881in}}%
\pgfpathlineto{\pgfqpoint{2.490451in}{3.992530in}}%
\pgfpathlineto{\pgfqpoint{2.492958in}{3.993320in}}%
\pgfpathlineto{\pgfqpoint{2.497971in}{4.002516in}}%
\pgfpathlineto{\pgfqpoint{2.505491in}{4.003848in}}%
\pgfpathlineto{\pgfqpoint{2.507998in}{4.004097in}}%
\pgfpathlineto{\pgfqpoint{2.515518in}{4.011605in}}%
\pgfpathlineto{\pgfqpoint{2.520531in}{4.012130in}}%
\pgfpathlineto{\pgfqpoint{2.523038in}{4.018217in}}%
\pgfpathlineto{\pgfqpoint{2.525544in}{4.019661in}}%
\pgfpathlineto{\pgfqpoint{2.530558in}{4.020394in}}%
\pgfpathlineto{\pgfqpoint{2.538078in}{4.028540in}}%
\pgfpathlineto{\pgfqpoint{2.545598in}{4.030123in}}%
\pgfpathlineto{\pgfqpoint{2.548104in}{4.033116in}}%
\pgfpathlineto{\pgfqpoint{2.550611in}{4.034298in}}%
\pgfpathlineto{\pgfqpoint{2.555624in}{4.038163in}}%
\pgfpathlineto{\pgfqpoint{2.560637in}{4.040613in}}%
\pgfpathlineto{\pgfqpoint{2.565651in}{4.044285in}}%
\pgfpathlineto{\pgfqpoint{2.578184in}{4.045549in}}%
\pgfpathlineto{\pgfqpoint{2.585704in}{4.046593in}}%
\pgfpathlineto{\pgfqpoint{2.588211in}{4.058683in}}%
\pgfpathlineto{\pgfqpoint{2.593224in}{4.059852in}}%
\pgfpathlineto{\pgfqpoint{2.595731in}{4.066305in}}%
\pgfpathlineto{\pgfqpoint{2.608264in}{4.072824in}}%
\pgfpathlineto{\pgfqpoint{2.610771in}{4.073536in}}%
\pgfpathlineto{\pgfqpoint{2.615784in}{4.081410in}}%
\pgfpathlineto{\pgfqpoint{2.633330in}{4.084966in}}%
\pgfpathlineto{\pgfqpoint{2.638344in}{4.086764in}}%
\pgfpathlineto{\pgfqpoint{2.645864in}{4.089565in}}%
\pgfpathlineto{\pgfqpoint{2.648370in}{4.089749in}}%
\pgfpathlineto{\pgfqpoint{2.653384in}{4.091945in}}%
\pgfpathlineto{\pgfqpoint{2.658397in}{4.092927in}}%
\pgfpathlineto{\pgfqpoint{2.660904in}{4.093507in}}%
\pgfpathlineto{\pgfqpoint{2.663410in}{4.097714in}}%
\pgfpathlineto{\pgfqpoint{2.665917in}{4.098624in}}%
\pgfpathlineto{\pgfqpoint{2.668424in}{4.101408in}}%
\pgfpathlineto{\pgfqpoint{2.678450in}{4.103101in}}%
\pgfpathlineto{\pgfqpoint{2.680957in}{4.103979in}}%
\pgfpathlineto{\pgfqpoint{2.685970in}{4.109300in}}%
\pgfpathlineto{\pgfqpoint{2.688477in}{4.114405in}}%
\pgfpathlineto{\pgfqpoint{2.690983in}{4.114727in}}%
\pgfpathlineto{\pgfqpoint{2.693490in}{4.117601in}}%
\pgfpathlineto{\pgfqpoint{2.706023in}{4.121358in}}%
\pgfpathlineto{\pgfqpoint{2.708530in}{4.121442in}}%
\pgfpathlineto{\pgfqpoint{2.711037in}{4.123651in}}%
\pgfpathlineto{\pgfqpoint{2.718557in}{4.124811in}}%
\pgfpathlineto{\pgfqpoint{2.728583in}{4.133904in}}%
\pgfpathlineto{\pgfqpoint{2.736103in}{4.135428in}}%
\pgfpathlineto{\pgfqpoint{2.743623in}{4.136406in}}%
\pgfpathlineto{\pgfqpoint{2.746130in}{4.146307in}}%
\pgfpathlineto{\pgfqpoint{2.751143in}{4.147853in}}%
\pgfpathlineto{\pgfqpoint{2.753650in}{4.152669in}}%
\pgfpathlineto{\pgfqpoint{2.776210in}{4.157485in}}%
\pgfpathlineto{\pgfqpoint{2.781223in}{4.158684in}}%
\pgfpathlineto{\pgfqpoint{2.783730in}{4.159617in}}%
\pgfpathlineto{\pgfqpoint{2.786236in}{4.162416in}}%
\pgfpathlineto{\pgfqpoint{2.788743in}{4.163266in}}%
\pgfpathlineto{\pgfqpoint{2.791250in}{4.166597in}}%
\pgfpathlineto{\pgfqpoint{2.793756in}{4.167196in}}%
\pgfpathlineto{\pgfqpoint{2.796263in}{4.170056in}}%
\pgfpathlineto{\pgfqpoint{2.798770in}{4.170268in}}%
\pgfpathlineto{\pgfqpoint{2.806290in}{4.180404in}}%
\pgfpathlineto{\pgfqpoint{2.816316in}{4.186058in}}%
\pgfpathlineto{\pgfqpoint{2.823836in}{4.191857in}}%
\pgfpathlineto{\pgfqpoint{2.826343in}{4.192462in}}%
\pgfpathlineto{\pgfqpoint{2.831356in}{4.196194in}}%
\pgfpathlineto{\pgfqpoint{2.836369in}{4.196455in}}%
\pgfpathlineto{\pgfqpoint{2.838876in}{4.199184in}}%
\pgfpathlineto{\pgfqpoint{2.843889in}{4.200686in}}%
\pgfpathlineto{\pgfqpoint{2.853916in}{4.206449in}}%
\pgfpathlineto{\pgfqpoint{2.856423in}{4.209853in}}%
\pgfpathlineto{\pgfqpoint{2.861436in}{4.212118in}}%
\pgfpathlineto{\pgfqpoint{2.871463in}{4.218301in}}%
\pgfpathlineto{\pgfqpoint{2.876476in}{4.219945in}}%
\pgfpathlineto{\pgfqpoint{2.878983in}{4.223725in}}%
\pgfpathlineto{\pgfqpoint{2.881489in}{4.223848in}}%
\pgfpathlineto{\pgfqpoint{2.883996in}{4.231376in}}%
\pgfpathlineto{\pgfqpoint{2.886503in}{4.233877in}}%
\pgfpathlineto{\pgfqpoint{2.894022in}{4.236229in}}%
\pgfpathlineto{\pgfqpoint{2.896529in}{4.237927in}}%
\pgfpathlineto{\pgfqpoint{2.899036in}{4.244163in}}%
\pgfpathlineto{\pgfqpoint{2.904049in}{4.246827in}}%
\pgfpathlineto{\pgfqpoint{2.906556in}{4.250349in}}%
\pgfpathlineto{\pgfqpoint{2.914076in}{4.252250in}}%
\pgfpathlineto{\pgfqpoint{2.916582in}{4.259333in}}%
\pgfpathlineto{\pgfqpoint{2.919089in}{4.261462in}}%
\pgfpathlineto{\pgfqpoint{2.929116in}{4.261861in}}%
\pgfpathlineto{\pgfqpoint{2.931622in}{4.264469in}}%
\pgfpathlineto{\pgfqpoint{2.934129in}{4.264892in}}%
\pgfpathlineto{\pgfqpoint{2.941649in}{4.276318in}}%
\pgfpathlineto{\pgfqpoint{2.946662in}{4.277373in}}%
\pgfpathlineto{\pgfqpoint{2.949169in}{4.278650in}}%
\pgfpathlineto{\pgfqpoint{2.951676in}{4.281375in}}%
\pgfpathlineto{\pgfqpoint{2.954182in}{4.288231in}}%
\pgfpathlineto{\pgfqpoint{2.956689in}{4.288277in}}%
\pgfpathlineto{\pgfqpoint{2.961702in}{4.294149in}}%
\pgfpathlineto{\pgfqpoint{2.969222in}{4.298956in}}%
\pgfpathlineto{\pgfqpoint{2.976742in}{4.300068in}}%
\pgfpathlineto{\pgfqpoint{2.979249in}{4.300623in}}%
\pgfpathlineto{\pgfqpoint{2.981755in}{4.303600in}}%
\pgfpathlineto{\pgfqpoint{2.989275in}{4.304690in}}%
\pgfpathlineto{\pgfqpoint{2.994289in}{4.306624in}}%
\pgfpathlineto{\pgfqpoint{2.996795in}{4.314916in}}%
\pgfpathlineto{\pgfqpoint{3.004315in}{4.315642in}}%
\pgfpathlineto{\pgfqpoint{3.011835in}{4.317792in}}%
\pgfpathlineto{\pgfqpoint{3.014342in}{4.322047in}}%
\pgfpathlineto{\pgfqpoint{3.016849in}{4.322468in}}%
\pgfpathlineto{\pgfqpoint{3.024369in}{4.328185in}}%
\pgfpathlineto{\pgfqpoint{3.046928in}{4.337352in}}%
\pgfpathlineto{\pgfqpoint{3.049435in}{4.337949in}}%
\pgfpathlineto{\pgfqpoint{3.051942in}{4.339735in}}%
\pgfpathlineto{\pgfqpoint{3.059462in}{4.341234in}}%
\pgfpathlineto{\pgfqpoint{3.061968in}{4.345924in}}%
\pgfpathlineto{\pgfqpoint{3.066982in}{4.347784in}}%
\pgfpathlineto{\pgfqpoint{3.071995in}{4.348399in}}%
\pgfpathlineto{\pgfqpoint{3.074502in}{4.350787in}}%
\pgfpathlineto{\pgfqpoint{3.077008in}{4.350886in}}%
\pgfpathlineto{\pgfqpoint{3.079515in}{4.353152in}}%
\pgfpathlineto{\pgfqpoint{3.084528in}{4.355460in}}%
\pgfpathlineto{\pgfqpoint{3.087035in}{4.358156in}}%
\pgfpathlineto{\pgfqpoint{3.094555in}{4.359682in}}%
\pgfpathlineto{\pgfqpoint{3.097061in}{4.362298in}}%
\pgfpathlineto{\pgfqpoint{3.099568in}{4.363148in}}%
\pgfpathlineto{\pgfqpoint{3.107088in}{4.371224in}}%
\pgfpathlineto{\pgfqpoint{3.114608in}{4.371774in}}%
\pgfpathlineto{\pgfqpoint{3.117115in}{4.375944in}}%
\pgfpathlineto{\pgfqpoint{3.124635in}{4.378134in}}%
\pgfpathlineto{\pgfqpoint{3.134661in}{4.381270in}}%
\pgfpathlineto{\pgfqpoint{3.137168in}{4.384141in}}%
\pgfpathlineto{\pgfqpoint{3.139675in}{4.389861in}}%
\pgfpathlineto{\pgfqpoint{3.149701in}{4.392171in}}%
\pgfpathlineto{\pgfqpoint{3.154715in}{4.395257in}}%
\pgfpathlineto{\pgfqpoint{3.164741in}{4.397855in}}%
\pgfpathlineto{\pgfqpoint{3.169754in}{4.400201in}}%
\pgfpathlineto{\pgfqpoint{3.184794in}{4.402211in}}%
\pgfpathlineto{\pgfqpoint{3.189808in}{4.405349in}}%
\pgfpathlineto{\pgfqpoint{3.197328in}{4.406802in}}%
\pgfpathlineto{\pgfqpoint{3.199834in}{4.409376in}}%
\pgfpathlineto{\pgfqpoint{3.202341in}{4.409513in}}%
\pgfpathlineto{\pgfqpoint{3.204848in}{4.412578in}}%
\pgfpathlineto{\pgfqpoint{3.209861in}{4.413490in}}%
\pgfpathlineto{\pgfqpoint{3.217381in}{4.421947in}}%
\pgfpathlineto{\pgfqpoint{3.219888in}{4.421993in}}%
\pgfpathlineto{\pgfqpoint{3.227408in}{4.426559in}}%
\pgfpathlineto{\pgfqpoint{3.234927in}{4.429609in}}%
\pgfpathlineto{\pgfqpoint{3.237434in}{4.432330in}}%
\pgfpathlineto{\pgfqpoint{3.242447in}{4.433226in}}%
\pgfpathlineto{\pgfqpoint{3.244954in}{4.438332in}}%
\pgfpathlineto{\pgfqpoint{3.249967in}{4.439274in}}%
\pgfpathlineto{\pgfqpoint{3.254981in}{4.441102in}}%
\pgfpathlineto{\pgfqpoint{3.262501in}{4.443648in}}%
\pgfpathlineto{\pgfqpoint{3.265007in}{4.445975in}}%
\pgfpathlineto{\pgfqpoint{3.267514in}{4.450234in}}%
\pgfpathlineto{\pgfqpoint{3.292581in}{4.458492in}}%
\pgfpathlineto{\pgfqpoint{3.295087in}{4.458635in}}%
\pgfpathlineto{\pgfqpoint{3.297594in}{4.466794in}}%
\pgfpathlineto{\pgfqpoint{3.302607in}{4.468131in}}%
\pgfpathlineto{\pgfqpoint{3.305114in}{4.470911in}}%
\pgfpathlineto{\pgfqpoint{3.320154in}{4.477009in}}%
\pgfpathlineto{\pgfqpoint{3.332687in}{4.487375in}}%
\pgfpathlineto{\pgfqpoint{3.335194in}{4.488049in}}%
\pgfpathlineto{\pgfqpoint{3.337700in}{4.491008in}}%
\pgfpathlineto{\pgfqpoint{3.340207in}{4.491480in}}%
\pgfpathlineto{\pgfqpoint{3.342714in}{4.496007in}}%
\pgfpathlineto{\pgfqpoint{3.355247in}{4.498840in}}%
\pgfpathlineto{\pgfqpoint{3.362767in}{4.499984in}}%
\pgfpathlineto{\pgfqpoint{3.365273in}{4.500674in}}%
\pgfpathlineto{\pgfqpoint{3.367780in}{4.502834in}}%
\pgfpathlineto{\pgfqpoint{3.372793in}{4.503898in}}%
\pgfpathlineto{\pgfqpoint{3.375300in}{4.508478in}}%
\pgfpathlineto{\pgfqpoint{3.380313in}{4.509314in}}%
\pgfpathlineto{\pgfqpoint{3.385327in}{4.514269in}}%
\pgfpathlineto{\pgfqpoint{3.412900in}{4.523520in}}%
\pgfpathlineto{\pgfqpoint{3.417913in}{4.528404in}}%
\pgfpathlineto{\pgfqpoint{3.445486in}{4.536163in}}%
\pgfpathlineto{\pgfqpoint{3.447993in}{4.539397in}}%
\pgfpathlineto{\pgfqpoint{3.453006in}{4.540048in}}%
\pgfpathlineto{\pgfqpoint{3.463033in}{4.543456in}}%
\pgfpathlineto{\pgfqpoint{3.465540in}{4.546605in}}%
\pgfpathlineto{\pgfqpoint{3.475566in}{4.548212in}}%
\pgfpathlineto{\pgfqpoint{3.478073in}{4.550963in}}%
\pgfpathlineto{\pgfqpoint{3.485593in}{4.551351in}}%
\pgfpathlineto{\pgfqpoint{3.490606in}{4.554787in}}%
\pgfpathlineto{\pgfqpoint{3.500633in}{4.559178in}}%
\pgfpathlineto{\pgfqpoint{3.513166in}{4.561685in}}%
\pgfpathlineto{\pgfqpoint{3.515673in}{4.564999in}}%
\pgfpathlineto{\pgfqpoint{3.518179in}{4.565379in}}%
\pgfpathlineto{\pgfqpoint{3.523193in}{4.568653in}}%
\pgfpathlineto{\pgfqpoint{3.530713in}{4.569756in}}%
\pgfpathlineto{\pgfqpoint{3.538233in}{4.574017in}}%
\pgfpathlineto{\pgfqpoint{3.548259in}{4.578975in}}%
\pgfpathlineto{\pgfqpoint{3.550766in}{4.580693in}}%
\pgfpathlineto{\pgfqpoint{3.553273in}{4.584671in}}%
\pgfpathlineto{\pgfqpoint{3.558286in}{4.585918in}}%
\pgfpathlineto{\pgfqpoint{3.568313in}{4.587865in}}%
\pgfpathlineto{\pgfqpoint{3.575832in}{4.594099in}}%
\pgfpathlineto{\pgfqpoint{3.583352in}{4.595953in}}%
\pgfpathlineto{\pgfqpoint{3.585859in}{4.598565in}}%
\pgfpathlineto{\pgfqpoint{3.588366in}{4.599096in}}%
\pgfpathlineto{\pgfqpoint{3.590872in}{4.603593in}}%
\pgfpathlineto{\pgfqpoint{3.593379in}{4.603746in}}%
\pgfpathlineto{\pgfqpoint{3.595886in}{4.607398in}}%
\pgfpathlineto{\pgfqpoint{3.600899in}{4.607834in}}%
\pgfpathlineto{\pgfqpoint{3.608419in}{4.616913in}}%
\pgfpathlineto{\pgfqpoint{3.615939in}{4.620937in}}%
\pgfpathlineto{\pgfqpoint{3.620952in}{4.621132in}}%
\pgfpathlineto{\pgfqpoint{3.625966in}{4.623967in}}%
\pgfpathlineto{\pgfqpoint{3.633486in}{4.625126in}}%
\pgfpathlineto{\pgfqpoint{3.651032in}{4.634457in}}%
\pgfpathlineto{\pgfqpoint{3.658552in}{4.636516in}}%
\pgfpathlineto{\pgfqpoint{3.661059in}{4.636624in}}%
\pgfpathlineto{\pgfqpoint{3.666072in}{4.639361in}}%
\pgfpathlineto{\pgfqpoint{3.671085in}{4.641030in}}%
\pgfpathlineto{\pgfqpoint{3.678605in}{4.644601in}}%
\pgfpathlineto{\pgfqpoint{3.681112in}{4.649941in}}%
\pgfpathlineto{\pgfqpoint{3.683619in}{4.650019in}}%
\pgfpathlineto{\pgfqpoint{3.686125in}{4.651732in}}%
\pgfpathlineto{\pgfqpoint{3.688632in}{4.658172in}}%
\pgfpathlineto{\pgfqpoint{3.703672in}{4.662124in}}%
\pgfpathlineto{\pgfqpoint{3.706178in}{4.665617in}}%
\pgfpathlineto{\pgfqpoint{3.708685in}{4.666014in}}%
\pgfpathlineto{\pgfqpoint{3.711192in}{4.668534in}}%
\pgfpathlineto{\pgfqpoint{3.713698in}{4.669005in}}%
\pgfpathlineto{\pgfqpoint{3.718712in}{4.673389in}}%
\pgfpathlineto{\pgfqpoint{3.721218in}{4.674053in}}%
\pgfpathlineto{\pgfqpoint{3.723725in}{4.676934in}}%
\pgfpathlineto{\pgfqpoint{3.728738in}{4.678058in}}%
\pgfpathlineto{\pgfqpoint{3.733752in}{4.684123in}}%
\pgfpathlineto{\pgfqpoint{3.738765in}{4.685841in}}%
\pgfpathlineto{\pgfqpoint{3.741272in}{4.690004in}}%
\pgfpathlineto{\pgfqpoint{3.743778in}{4.690545in}}%
\pgfpathlineto{\pgfqpoint{3.751298in}{4.695189in}}%
\pgfpathlineto{\pgfqpoint{3.766338in}{4.702204in}}%
\pgfpathlineto{\pgfqpoint{3.768845in}{4.702248in}}%
\pgfpathlineto{\pgfqpoint{3.771352in}{4.704273in}}%
\pgfpathlineto{\pgfqpoint{3.776365in}{4.704540in}}%
\pgfpathlineto{\pgfqpoint{3.778871in}{4.707023in}}%
\pgfpathlineto{\pgfqpoint{3.783885in}{4.707816in}}%
\pgfpathlineto{\pgfqpoint{3.786391in}{4.711208in}}%
\pgfpathlineto{\pgfqpoint{3.788898in}{4.712403in}}%
\pgfpathlineto{\pgfqpoint{3.791405in}{4.717391in}}%
\pgfpathlineto{\pgfqpoint{3.793911in}{4.717528in}}%
\pgfpathlineto{\pgfqpoint{3.798925in}{4.724710in}}%
\pgfpathlineto{\pgfqpoint{3.806445in}{4.725699in}}%
\pgfpathlineto{\pgfqpoint{3.811458in}{4.727788in}}%
\pgfpathlineto{\pgfqpoint{3.816471in}{4.728258in}}%
\pgfpathlineto{\pgfqpoint{3.818978in}{4.732066in}}%
\pgfpathlineto{\pgfqpoint{3.826498in}{4.735853in}}%
\pgfpathlineto{\pgfqpoint{3.829005in}{4.739172in}}%
\pgfpathlineto{\pgfqpoint{3.831511in}{4.744375in}}%
\pgfpathlineto{\pgfqpoint{3.839031in}{4.750913in}}%
\pgfpathlineto{\pgfqpoint{3.841538in}{4.751605in}}%
\pgfpathlineto{\pgfqpoint{3.844044in}{4.755656in}}%
\pgfpathlineto{\pgfqpoint{3.846551in}{4.756549in}}%
\pgfpathlineto{\pgfqpoint{3.849058in}{4.761759in}}%
\pgfpathlineto{\pgfqpoint{3.854071in}{4.763833in}}%
\pgfpathlineto{\pgfqpoint{3.856578in}{4.767250in}}%
\pgfpathlineto{\pgfqpoint{3.861591in}{4.770446in}}%
\pgfpathlineto{\pgfqpoint{3.866604in}{4.772142in}}%
\pgfpathlineto{\pgfqpoint{3.869111in}{4.772543in}}%
\pgfpathlineto{\pgfqpoint{3.876631in}{4.777297in}}%
\pgfpathlineto{\pgfqpoint{3.879138in}{4.777392in}}%
\pgfpathlineto{\pgfqpoint{3.881644in}{4.781867in}}%
\pgfpathlineto{\pgfqpoint{3.886658in}{4.784029in}}%
\pgfpathlineto{\pgfqpoint{3.889164in}{4.787685in}}%
\pgfpathlineto{\pgfqpoint{3.891671in}{4.788035in}}%
\pgfpathlineto{\pgfqpoint{3.896684in}{4.792641in}}%
\pgfpathlineto{\pgfqpoint{3.899191in}{4.792833in}}%
\pgfpathlineto{\pgfqpoint{3.906711in}{4.800574in}}%
\pgfpathlineto{\pgfqpoint{3.911724in}{4.802263in}}%
\pgfpathlineto{\pgfqpoint{3.914231in}{4.804178in}}%
\pgfpathlineto{\pgfqpoint{3.919244in}{4.805228in}}%
\pgfpathlineto{\pgfqpoint{3.929271in}{4.810353in}}%
\pgfpathlineto{\pgfqpoint{3.931777in}{4.813390in}}%
\pgfpathlineto{\pgfqpoint{3.936791in}{4.814715in}}%
\pgfpathlineto{\pgfqpoint{3.941804in}{4.817155in}}%
\pgfpathlineto{\pgfqpoint{3.946817in}{4.818205in}}%
\pgfpathlineto{\pgfqpoint{3.951831in}{4.820107in}}%
\pgfpathlineto{\pgfqpoint{3.964364in}{4.826658in}}%
\pgfpathlineto{\pgfqpoint{3.966871in}{4.826832in}}%
\pgfpathlineto{\pgfqpoint{3.969377in}{4.828306in}}%
\pgfpathlineto{\pgfqpoint{3.971884in}{4.840864in}}%
\pgfpathlineto{\pgfqpoint{3.981910in}{4.849786in}}%
\pgfpathlineto{\pgfqpoint{3.986924in}{4.850820in}}%
\pgfpathlineto{\pgfqpoint{3.989430in}{4.857181in}}%
\pgfpathlineto{\pgfqpoint{3.991937in}{4.858125in}}%
\pgfpathlineto{\pgfqpoint{3.994444in}{4.864324in}}%
\pgfpathlineto{\pgfqpoint{3.996950in}{4.864538in}}%
\pgfpathlineto{\pgfqpoint{3.999457in}{4.866857in}}%
\pgfpathlineto{\pgfqpoint{4.001964in}{4.867409in}}%
\pgfpathlineto{\pgfqpoint{4.004470in}{4.872022in}}%
\pgfpathlineto{\pgfqpoint{4.011990in}{4.872482in}}%
\pgfpathlineto{\pgfqpoint{4.017004in}{4.876711in}}%
\pgfpathlineto{\pgfqpoint{4.019510in}{4.876745in}}%
\pgfpathlineto{\pgfqpoint{4.022017in}{4.881145in}}%
\pgfpathlineto{\pgfqpoint{4.027030in}{4.882925in}}%
\pgfpathlineto{\pgfqpoint{4.032044in}{4.884694in}}%
\pgfpathlineto{\pgfqpoint{4.044577in}{4.892601in}}%
\pgfpathlineto{\pgfqpoint{4.054603in}{4.902450in}}%
\pgfpathlineto{\pgfqpoint{4.057110in}{4.907016in}}%
\pgfpathlineto{\pgfqpoint{4.069643in}{4.913401in}}%
\pgfpathlineto{\pgfqpoint{4.084683in}{4.915142in}}%
\pgfpathlineto{\pgfqpoint{4.087190in}{4.917815in}}%
\pgfpathlineto{\pgfqpoint{4.092203in}{4.918150in}}%
\pgfpathlineto{\pgfqpoint{4.094710in}{4.924586in}}%
\pgfpathlineto{\pgfqpoint{4.104737in}{4.934317in}}%
\pgfpathlineto{\pgfqpoint{4.107243in}{4.934385in}}%
\pgfpathlineto{\pgfqpoint{4.114763in}{4.949169in}}%
\pgfpathlineto{\pgfqpoint{4.124790in}{4.955081in}}%
\pgfpathlineto{\pgfqpoint{4.129803in}{4.960335in}}%
\pgfpathlineto{\pgfqpoint{4.134816in}{4.960377in}}%
\pgfpathlineto{\pgfqpoint{4.137323in}{4.961702in}}%
\pgfpathlineto{\pgfqpoint{4.144843in}{4.971081in}}%
\pgfpathlineto{\pgfqpoint{4.147350in}{4.976921in}}%
\pgfpathlineto{\pgfqpoint{4.149856in}{4.977702in}}%
\pgfpathlineto{\pgfqpoint{4.157376in}{4.984518in}}%
\pgfpathlineto{\pgfqpoint{4.159883in}{4.997432in}}%
\pgfpathlineto{\pgfqpoint{4.162390in}{4.999570in}}%
\pgfpathlineto{\pgfqpoint{4.167403in}{5.006071in}}%
\pgfpathlineto{\pgfqpoint{4.169910in}{5.006350in}}%
\pgfpathlineto{\pgfqpoint{4.172416in}{5.012791in}}%
\pgfpathlineto{\pgfqpoint{4.179936in}{5.019745in}}%
\pgfpathlineto{\pgfqpoint{4.182443in}{5.023332in}}%
\pgfpathlineto{\pgfqpoint{4.189963in}{5.025282in}}%
\pgfpathlineto{\pgfqpoint{4.194976in}{5.028013in}}%
\pgfpathlineto{\pgfqpoint{4.197483in}{5.035773in}}%
\pgfpathlineto{\pgfqpoint{4.202496in}{5.036749in}}%
\pgfpathlineto{\pgfqpoint{4.225056in}{5.056075in}}%
\pgfpathlineto{\pgfqpoint{4.235083in}{5.063007in}}%
\pgfpathlineto{\pgfqpoint{4.242603in}{5.065318in}}%
\pgfpathlineto{\pgfqpoint{4.252629in}{5.075255in}}%
\pgfpathlineto{\pgfqpoint{4.255136in}{5.076587in}}%
\pgfpathlineto{\pgfqpoint{4.257642in}{5.080963in}}%
\pgfpathlineto{\pgfqpoint{4.262656in}{5.081728in}}%
\pgfpathlineto{\pgfqpoint{4.267669in}{5.084762in}}%
\pgfpathlineto{\pgfqpoint{4.272682in}{5.085872in}}%
\pgfpathlineto{\pgfqpoint{4.275189in}{5.089686in}}%
\pgfpathlineto{\pgfqpoint{4.277696in}{5.090522in}}%
\pgfpathlineto{\pgfqpoint{4.280202in}{5.093537in}}%
\pgfpathlineto{\pgfqpoint{4.282709in}{5.094411in}}%
\pgfpathlineto{\pgfqpoint{4.285216in}{5.101231in}}%
\pgfpathlineto{\pgfqpoint{4.292736in}{5.109901in}}%
\pgfpathlineto{\pgfqpoint{4.295242in}{5.115948in}}%
\pgfpathlineto{\pgfqpoint{4.297749in}{5.116251in}}%
\pgfpathlineto{\pgfqpoint{4.300256in}{5.122939in}}%
\pgfpathlineto{\pgfqpoint{4.307776in}{5.128143in}}%
\pgfpathlineto{\pgfqpoint{4.310282in}{5.129128in}}%
\pgfpathlineto{\pgfqpoint{4.312789in}{5.131766in}}%
\pgfpathlineto{\pgfqpoint{4.315296in}{5.138149in}}%
\pgfpathlineto{\pgfqpoint{4.317802in}{5.140105in}}%
\pgfpathlineto{\pgfqpoint{4.322815in}{5.145372in}}%
\pgfpathlineto{\pgfqpoint{4.325322in}{5.146139in}}%
\pgfpathlineto{\pgfqpoint{4.327829in}{5.152189in}}%
\pgfpathlineto{\pgfqpoint{4.332842in}{5.153396in}}%
\pgfpathlineto{\pgfqpoint{4.335349in}{5.162413in}}%
\pgfpathlineto{\pgfqpoint{4.340362in}{5.163696in}}%
\pgfpathlineto{\pgfqpoint{4.345375in}{5.164216in}}%
\pgfpathlineto{\pgfqpoint{4.347882in}{5.171610in}}%
\pgfpathlineto{\pgfqpoint{4.350389in}{5.172038in}}%
\pgfpathlineto{\pgfqpoint{4.357909in}{5.177980in}}%
\pgfpathlineto{\pgfqpoint{4.360415in}{5.181383in}}%
\pgfpathlineto{\pgfqpoint{4.365429in}{5.191925in}}%
\pgfpathlineto{\pgfqpoint{4.367935in}{5.192027in}}%
\pgfpathlineto{\pgfqpoint{4.372949in}{5.195449in}}%
\pgfpathlineto{\pgfqpoint{4.377962in}{5.196441in}}%
\pgfpathlineto{\pgfqpoint{4.380469in}{5.199513in}}%
\pgfpathlineto{\pgfqpoint{4.385482in}{5.200735in}}%
\pgfpathlineto{\pgfqpoint{4.387988in}{5.207268in}}%
\pgfpathlineto{\pgfqpoint{4.393002in}{5.208335in}}%
\pgfpathlineto{\pgfqpoint{4.400522in}{5.211240in}}%
\pgfpathlineto{\pgfqpoint{4.403028in}{5.211759in}}%
\pgfpathlineto{\pgfqpoint{4.408042in}{5.215758in}}%
\pgfpathlineto{\pgfqpoint{4.410548in}{5.215933in}}%
\pgfpathlineto{\pgfqpoint{4.413055in}{5.219506in}}%
\pgfpathlineto{\pgfqpoint{4.415562in}{5.219931in}}%
\pgfpathlineto{\pgfqpoint{4.418068in}{5.223819in}}%
\pgfpathlineto{\pgfqpoint{4.420575in}{5.224222in}}%
\pgfpathlineto{\pgfqpoint{4.423082in}{5.226269in}}%
\pgfpathlineto{\pgfqpoint{4.433108in}{5.227942in}}%
\pgfpathlineto{\pgfqpoint{4.438122in}{5.231756in}}%
\pgfpathlineto{\pgfqpoint{4.440628in}{5.232525in}}%
\pgfpathlineto{\pgfqpoint{4.445642in}{5.243318in}}%
\pgfpathlineto{\pgfqpoint{4.448148in}{5.243756in}}%
\pgfpathlineto{\pgfqpoint{4.450655in}{5.246880in}}%
\pgfpathlineto{\pgfqpoint{4.455668in}{5.255638in}}%
\pgfpathlineto{\pgfqpoint{4.460681in}{5.258604in}}%
\pgfpathlineto{\pgfqpoint{4.463188in}{5.265556in}}%
\pgfpathlineto{\pgfqpoint{4.465695in}{5.266923in}}%
\pgfpathlineto{\pgfqpoint{4.470708in}{5.273619in}}%
\pgfpathlineto{\pgfqpoint{4.475721in}{5.278478in}}%
\pgfpathlineto{\pgfqpoint{4.480735in}{5.282585in}}%
\pgfpathlineto{\pgfqpoint{4.485748in}{5.293608in}}%
\pgfpathlineto{\pgfqpoint{4.488255in}{5.305275in}}%
\pgfpathlineto{\pgfqpoint{4.488255in}{5.305275in}}%
\pgfusepath{stroke}%
\end{pgfscope}%
\begin{pgfscope}%
\pgfpathrectangle{\pgfqpoint{0.708220in}{3.210823in}}{\pgfqpoint{5.013309in}{2.094453in}}%
\pgfusepath{clip}%
\pgfsetbuttcap%
\pgfsetroundjoin%
\pgfsetlinewidth{1.003750pt}%
\definecolor{currentstroke}{rgb}{0.811765,0.125490,0.125490}%
\pgfsetstrokecolor{currentstroke}%
\pgfsetdash{{1.000000pt}{1.650000pt}}{0.000000pt}%
\pgfpathmoveto{\pgfqpoint{0.708220in}{3.476961in}}%
\pgfpathlineto{\pgfqpoint{0.718246in}{3.477615in}}%
\pgfpathlineto{\pgfqpoint{0.725766in}{3.478706in}}%
\pgfpathlineto{\pgfqpoint{0.760860in}{3.480019in}}%
\pgfpathlineto{\pgfqpoint{0.816006in}{3.481577in}}%
\pgfpathlineto{\pgfqpoint{1.164431in}{3.492260in}}%
\pgfpathlineto{\pgfqpoint{1.171951in}{3.492877in}}%
\pgfpathlineto{\pgfqpoint{1.227097in}{3.496980in}}%
\pgfpathlineto{\pgfqpoint{1.234617in}{3.497761in}}%
\pgfpathlineto{\pgfqpoint{1.259684in}{3.498463in}}%
\pgfpathlineto{\pgfqpoint{1.264697in}{3.499984in}}%
\pgfpathlineto{\pgfqpoint{1.380003in}{3.509188in}}%
\pgfpathlineto{\pgfqpoint{1.392537in}{3.510515in}}%
\pgfpathlineto{\pgfqpoint{1.402563in}{3.511071in}}%
\pgfpathlineto{\pgfqpoint{1.417603in}{3.512485in}}%
\pgfpathlineto{\pgfqpoint{1.422616in}{3.513517in}}%
\pgfpathlineto{\pgfqpoint{1.442670in}{3.515045in}}%
\pgfpathlineto{\pgfqpoint{1.452696in}{3.516583in}}%
\pgfpathlineto{\pgfqpoint{1.470243in}{3.518244in}}%
\pgfpathlineto{\pgfqpoint{1.487789in}{3.520404in}}%
\pgfpathlineto{\pgfqpoint{1.507843in}{3.524567in}}%
\pgfpathlineto{\pgfqpoint{1.517869in}{3.524832in}}%
\pgfpathlineto{\pgfqpoint{1.522883in}{3.527009in}}%
\pgfpathlineto{\pgfqpoint{1.547949in}{3.529293in}}%
\pgfpathlineto{\pgfqpoint{1.605602in}{3.534819in}}%
\pgfpathlineto{\pgfqpoint{1.625655in}{3.536066in}}%
\pgfpathlineto{\pgfqpoint{1.633175in}{3.537938in}}%
\pgfpathlineto{\pgfqpoint{1.645709in}{3.540580in}}%
\pgfpathlineto{\pgfqpoint{1.653229in}{3.541602in}}%
\pgfpathlineto{\pgfqpoint{1.665762in}{3.542632in}}%
\pgfpathlineto{\pgfqpoint{1.685815in}{3.544954in}}%
\pgfpathlineto{\pgfqpoint{1.688322in}{3.546761in}}%
\pgfpathlineto{\pgfqpoint{1.693335in}{3.548318in}}%
\pgfpathlineto{\pgfqpoint{1.698348in}{3.549924in}}%
\pgfpathlineto{\pgfqpoint{1.703362in}{3.551047in}}%
\pgfpathlineto{\pgfqpoint{1.715895in}{3.553948in}}%
\pgfpathlineto{\pgfqpoint{1.718402in}{3.555893in}}%
\pgfpathlineto{\pgfqpoint{1.735948in}{3.558681in}}%
\pgfpathlineto{\pgfqpoint{1.740961in}{3.560211in}}%
\pgfpathlineto{\pgfqpoint{1.748481in}{3.561096in}}%
\pgfpathlineto{\pgfqpoint{1.778561in}{3.568748in}}%
\pgfpathlineto{\pgfqpoint{1.781068in}{3.570689in}}%
\pgfpathlineto{\pgfqpoint{1.786081in}{3.571496in}}%
\pgfpathlineto{\pgfqpoint{1.793601in}{3.575247in}}%
\pgfpathlineto{\pgfqpoint{1.808641in}{3.578418in}}%
\pgfpathlineto{\pgfqpoint{1.816161in}{3.579462in}}%
\pgfpathlineto{\pgfqpoint{1.831201in}{3.584477in}}%
\pgfpathlineto{\pgfqpoint{1.838721in}{3.586188in}}%
\pgfpathlineto{\pgfqpoint{1.843734in}{3.587155in}}%
\pgfpathlineto{\pgfqpoint{1.848748in}{3.589290in}}%
\pgfpathlineto{\pgfqpoint{1.853761in}{3.590127in}}%
\pgfpathlineto{\pgfqpoint{1.858774in}{3.595979in}}%
\pgfpathlineto{\pgfqpoint{1.868801in}{3.597315in}}%
\pgfpathlineto{\pgfqpoint{1.873814in}{3.599542in}}%
\pgfpathlineto{\pgfqpoint{1.876321in}{3.599563in}}%
\pgfpathlineto{\pgfqpoint{1.883841in}{3.605221in}}%
\pgfpathlineto{\pgfqpoint{1.896374in}{3.607395in}}%
\pgfpathlineto{\pgfqpoint{1.898881in}{3.610220in}}%
\pgfpathlineto{\pgfqpoint{1.916427in}{3.612652in}}%
\pgfpathlineto{\pgfqpoint{1.918934in}{3.618271in}}%
\pgfpathlineto{\pgfqpoint{1.923947in}{3.621213in}}%
\pgfpathlineto{\pgfqpoint{1.926454in}{3.623576in}}%
\pgfpathlineto{\pgfqpoint{1.941494in}{3.627043in}}%
\pgfpathlineto{\pgfqpoint{1.944000in}{3.627691in}}%
\pgfpathlineto{\pgfqpoint{1.946507in}{3.630993in}}%
\pgfpathlineto{\pgfqpoint{1.954027in}{3.633758in}}%
\pgfpathlineto{\pgfqpoint{1.959040in}{3.636901in}}%
\pgfpathlineto{\pgfqpoint{1.961547in}{3.638399in}}%
\pgfpathlineto{\pgfqpoint{1.966560in}{3.639477in}}%
\pgfpathlineto{\pgfqpoint{1.969067in}{3.639996in}}%
\pgfpathlineto{\pgfqpoint{1.971574in}{3.642382in}}%
\pgfpathlineto{\pgfqpoint{1.974080in}{3.643080in}}%
\pgfpathlineto{\pgfqpoint{1.979094in}{3.646091in}}%
\pgfpathlineto{\pgfqpoint{1.986614in}{3.646939in}}%
\pgfpathlineto{\pgfqpoint{1.994134in}{3.651774in}}%
\pgfpathlineto{\pgfqpoint{2.014187in}{3.657009in}}%
\pgfpathlineto{\pgfqpoint{2.019200in}{3.661624in}}%
\pgfpathlineto{\pgfqpoint{2.039253in}{3.669806in}}%
\pgfpathlineto{\pgfqpoint{2.044267in}{3.672513in}}%
\pgfpathlineto{\pgfqpoint{2.049280in}{3.673576in}}%
\pgfpathlineto{\pgfqpoint{2.051787in}{3.676876in}}%
\pgfpathlineto{\pgfqpoint{2.061813in}{3.680851in}}%
\pgfpathlineto{\pgfqpoint{2.069333in}{3.681939in}}%
\pgfpathlineto{\pgfqpoint{2.071840in}{3.683911in}}%
\pgfpathlineto{\pgfqpoint{2.074346in}{3.683994in}}%
\pgfpathlineto{\pgfqpoint{2.081866in}{3.688424in}}%
\pgfpathlineto{\pgfqpoint{2.084373in}{3.690509in}}%
\pgfpathlineto{\pgfqpoint{2.089386in}{3.690934in}}%
\pgfpathlineto{\pgfqpoint{2.091893in}{3.693172in}}%
\pgfpathlineto{\pgfqpoint{2.099413in}{3.694119in}}%
\pgfpathlineto{\pgfqpoint{2.104426in}{3.696250in}}%
\pgfpathlineto{\pgfqpoint{2.121973in}{3.701690in}}%
\pgfpathlineto{\pgfqpoint{2.124480in}{3.706328in}}%
\pgfpathlineto{\pgfqpoint{2.129493in}{3.707143in}}%
\pgfpathlineto{\pgfqpoint{2.134506in}{3.707640in}}%
\pgfpathlineto{\pgfqpoint{2.147039in}{3.721862in}}%
\pgfpathlineto{\pgfqpoint{2.157066in}{3.723615in}}%
\pgfpathlineto{\pgfqpoint{2.162079in}{3.728666in}}%
\pgfpathlineto{\pgfqpoint{2.167093in}{3.730032in}}%
\pgfpathlineto{\pgfqpoint{2.174613in}{3.731865in}}%
\pgfpathlineto{\pgfqpoint{2.177119in}{3.734779in}}%
\pgfpathlineto{\pgfqpoint{2.182133in}{3.736179in}}%
\pgfpathlineto{\pgfqpoint{2.197173in}{3.745724in}}%
\pgfpathlineto{\pgfqpoint{2.204693in}{3.746467in}}%
\pgfpathlineto{\pgfqpoint{2.207199in}{3.749111in}}%
\pgfpathlineto{\pgfqpoint{2.209706in}{3.749847in}}%
\pgfpathlineto{\pgfqpoint{2.214719in}{3.752508in}}%
\pgfpathlineto{\pgfqpoint{2.229759in}{3.757554in}}%
\pgfpathlineto{\pgfqpoint{2.242292in}{3.760219in}}%
\pgfpathlineto{\pgfqpoint{2.244799in}{3.762560in}}%
\pgfpathlineto{\pgfqpoint{2.249812in}{3.771071in}}%
\pgfpathlineto{\pgfqpoint{2.254826in}{3.772780in}}%
\pgfpathlineto{\pgfqpoint{2.257332in}{3.776341in}}%
\pgfpathlineto{\pgfqpoint{2.262346in}{3.777221in}}%
\pgfpathlineto{\pgfqpoint{2.264852in}{3.777967in}}%
\pgfpathlineto{\pgfqpoint{2.267359in}{3.782004in}}%
\pgfpathlineto{\pgfqpoint{2.272372in}{3.783027in}}%
\pgfpathlineto{\pgfqpoint{2.282399in}{3.786141in}}%
\pgfpathlineto{\pgfqpoint{2.284905in}{3.788964in}}%
\pgfpathlineto{\pgfqpoint{2.287412in}{3.794642in}}%
\pgfpathlineto{\pgfqpoint{2.294932in}{3.798798in}}%
\pgfpathlineto{\pgfqpoint{2.299945in}{3.799869in}}%
\pgfpathlineto{\pgfqpoint{2.302452in}{3.803046in}}%
\pgfpathlineto{\pgfqpoint{2.304959in}{3.803138in}}%
\pgfpathlineto{\pgfqpoint{2.312479in}{3.807353in}}%
\pgfpathlineto{\pgfqpoint{2.314985in}{3.810692in}}%
\pgfpathlineto{\pgfqpoint{2.322505in}{3.811922in}}%
\pgfpathlineto{\pgfqpoint{2.327519in}{3.814096in}}%
\pgfpathlineto{\pgfqpoint{2.330025in}{3.818208in}}%
\pgfpathlineto{\pgfqpoint{2.342559in}{3.819979in}}%
\pgfpathlineto{\pgfqpoint{2.347572in}{3.821328in}}%
\pgfpathlineto{\pgfqpoint{2.372638in}{3.828485in}}%
\pgfpathlineto{\pgfqpoint{2.375145in}{3.830823in}}%
\pgfpathlineto{\pgfqpoint{2.377652in}{3.835574in}}%
\pgfpathlineto{\pgfqpoint{2.382665in}{3.837144in}}%
\pgfpathlineto{\pgfqpoint{2.390185in}{3.841904in}}%
\pgfpathlineto{\pgfqpoint{2.400212in}{3.845870in}}%
\pgfpathlineto{\pgfqpoint{2.410238in}{3.846861in}}%
\pgfpathlineto{\pgfqpoint{2.415251in}{3.853941in}}%
\pgfpathlineto{\pgfqpoint{2.420265in}{3.854881in}}%
\pgfpathlineto{\pgfqpoint{2.422771in}{3.854912in}}%
\pgfpathlineto{\pgfqpoint{2.430291in}{3.860631in}}%
\pgfpathlineto{\pgfqpoint{2.452851in}{3.873984in}}%
\pgfpathlineto{\pgfqpoint{2.457865in}{3.874479in}}%
\pgfpathlineto{\pgfqpoint{2.460371in}{3.883739in}}%
\pgfpathlineto{\pgfqpoint{2.462878in}{3.884905in}}%
\pgfpathlineto{\pgfqpoint{2.467891in}{3.890734in}}%
\pgfpathlineto{\pgfqpoint{2.475411in}{3.893327in}}%
\pgfpathlineto{\pgfqpoint{2.480425in}{3.896435in}}%
\pgfpathlineto{\pgfqpoint{2.487944in}{3.905701in}}%
\pgfpathlineto{\pgfqpoint{2.492958in}{3.906875in}}%
\pgfpathlineto{\pgfqpoint{2.495464in}{3.910530in}}%
\pgfpathlineto{\pgfqpoint{2.500478in}{3.911876in}}%
\pgfpathlineto{\pgfqpoint{2.505491in}{3.913151in}}%
\pgfpathlineto{\pgfqpoint{2.518024in}{3.916505in}}%
\pgfpathlineto{\pgfqpoint{2.520531in}{3.918281in}}%
\pgfpathlineto{\pgfqpoint{2.523038in}{3.918324in}}%
\pgfpathlineto{\pgfqpoint{2.525544in}{3.924405in}}%
\pgfpathlineto{\pgfqpoint{2.528051in}{3.925165in}}%
\pgfpathlineto{\pgfqpoint{2.530558in}{3.927708in}}%
\pgfpathlineto{\pgfqpoint{2.533064in}{3.928366in}}%
\pgfpathlineto{\pgfqpoint{2.535571in}{3.932443in}}%
\pgfpathlineto{\pgfqpoint{2.540584in}{3.933359in}}%
\pgfpathlineto{\pgfqpoint{2.545598in}{3.944823in}}%
\pgfpathlineto{\pgfqpoint{2.548104in}{3.945042in}}%
\pgfpathlineto{\pgfqpoint{2.550611in}{3.946571in}}%
\pgfpathlineto{\pgfqpoint{2.555624in}{3.947789in}}%
\pgfpathlineto{\pgfqpoint{2.563144in}{3.952152in}}%
\pgfpathlineto{\pgfqpoint{2.565651in}{3.954861in}}%
\pgfpathlineto{\pgfqpoint{2.568157in}{3.961306in}}%
\pgfpathlineto{\pgfqpoint{2.570664in}{3.961405in}}%
\pgfpathlineto{\pgfqpoint{2.580691in}{3.967648in}}%
\pgfpathlineto{\pgfqpoint{2.585704in}{3.972331in}}%
\pgfpathlineto{\pgfqpoint{2.590717in}{3.974121in}}%
\pgfpathlineto{\pgfqpoint{2.593224in}{3.982882in}}%
\pgfpathlineto{\pgfqpoint{2.595731in}{3.987419in}}%
\pgfpathlineto{\pgfqpoint{2.603251in}{3.990152in}}%
\pgfpathlineto{\pgfqpoint{2.605757in}{3.993852in}}%
\pgfpathlineto{\pgfqpoint{2.610771in}{3.996883in}}%
\pgfpathlineto{\pgfqpoint{2.613277in}{4.002803in}}%
\pgfpathlineto{\pgfqpoint{2.615784in}{4.003163in}}%
\pgfpathlineto{\pgfqpoint{2.625810in}{4.011802in}}%
\pgfpathlineto{\pgfqpoint{2.628317in}{4.015946in}}%
\pgfpathlineto{\pgfqpoint{2.638344in}{4.021022in}}%
\pgfpathlineto{\pgfqpoint{2.640850in}{4.023611in}}%
\pgfpathlineto{\pgfqpoint{2.648370in}{4.025739in}}%
\pgfpathlineto{\pgfqpoint{2.650877in}{4.031920in}}%
\pgfpathlineto{\pgfqpoint{2.655890in}{4.033911in}}%
\pgfpathlineto{\pgfqpoint{2.663410in}{4.036472in}}%
\pgfpathlineto{\pgfqpoint{2.665917in}{4.036568in}}%
\pgfpathlineto{\pgfqpoint{2.668424in}{4.040562in}}%
\pgfpathlineto{\pgfqpoint{2.670930in}{4.041138in}}%
\pgfpathlineto{\pgfqpoint{2.675944in}{4.044981in}}%
\pgfpathlineto{\pgfqpoint{2.680957in}{4.045726in}}%
\pgfpathlineto{\pgfqpoint{2.683464in}{4.048779in}}%
\pgfpathlineto{\pgfqpoint{2.688477in}{4.049168in}}%
\pgfpathlineto{\pgfqpoint{2.690983in}{4.054277in}}%
\pgfpathlineto{\pgfqpoint{2.693490in}{4.055222in}}%
\pgfpathlineto{\pgfqpoint{2.698503in}{4.059281in}}%
\pgfpathlineto{\pgfqpoint{2.703517in}{4.061767in}}%
\pgfpathlineto{\pgfqpoint{2.706023in}{4.062724in}}%
\pgfpathlineto{\pgfqpoint{2.708530in}{4.067396in}}%
\pgfpathlineto{\pgfqpoint{2.716050in}{4.070123in}}%
\pgfpathlineto{\pgfqpoint{2.721063in}{4.077113in}}%
\pgfpathlineto{\pgfqpoint{2.733597in}{4.081563in}}%
\pgfpathlineto{\pgfqpoint{2.738610in}{4.087500in}}%
\pgfpathlineto{\pgfqpoint{2.741117in}{4.097873in}}%
\pgfpathlineto{\pgfqpoint{2.746130in}{4.102243in}}%
\pgfpathlineto{\pgfqpoint{2.748637in}{4.102884in}}%
\pgfpathlineto{\pgfqpoint{2.751143in}{4.106664in}}%
\pgfpathlineto{\pgfqpoint{2.758663in}{4.110585in}}%
\pgfpathlineto{\pgfqpoint{2.761170in}{4.112867in}}%
\pgfpathlineto{\pgfqpoint{2.763676in}{4.118039in}}%
\pgfpathlineto{\pgfqpoint{2.768690in}{4.119960in}}%
\pgfpathlineto{\pgfqpoint{2.773703in}{4.124101in}}%
\pgfpathlineto{\pgfqpoint{2.776210in}{4.127146in}}%
\pgfpathlineto{\pgfqpoint{2.778716in}{4.127952in}}%
\pgfpathlineto{\pgfqpoint{2.791250in}{4.138846in}}%
\pgfpathlineto{\pgfqpoint{2.793756in}{4.139115in}}%
\pgfpathlineto{\pgfqpoint{2.796263in}{4.141475in}}%
\pgfpathlineto{\pgfqpoint{2.798770in}{4.141893in}}%
\pgfpathlineto{\pgfqpoint{2.801276in}{4.145033in}}%
\pgfpathlineto{\pgfqpoint{2.808796in}{4.147665in}}%
\pgfpathlineto{\pgfqpoint{2.811303in}{4.152715in}}%
\pgfpathlineto{\pgfqpoint{2.816316in}{4.153393in}}%
\pgfpathlineto{\pgfqpoint{2.821329in}{4.159278in}}%
\pgfpathlineto{\pgfqpoint{2.823836in}{4.167768in}}%
\pgfpathlineto{\pgfqpoint{2.831356in}{4.169300in}}%
\pgfpathlineto{\pgfqpoint{2.838876in}{4.176768in}}%
\pgfpathlineto{\pgfqpoint{2.841383in}{4.177003in}}%
\pgfpathlineto{\pgfqpoint{2.843889in}{4.179904in}}%
\pgfpathlineto{\pgfqpoint{2.846396in}{4.187752in}}%
\pgfpathlineto{\pgfqpoint{2.848903in}{4.188022in}}%
\pgfpathlineto{\pgfqpoint{2.851409in}{4.192072in}}%
\pgfpathlineto{\pgfqpoint{2.853916in}{4.192952in}}%
\pgfpathlineto{\pgfqpoint{2.858929in}{4.200167in}}%
\pgfpathlineto{\pgfqpoint{2.866449in}{4.204082in}}%
\pgfpathlineto{\pgfqpoint{2.871463in}{4.207469in}}%
\pgfpathlineto{\pgfqpoint{2.876476in}{4.210297in}}%
\pgfpathlineto{\pgfqpoint{2.878983in}{4.217181in}}%
\pgfpathlineto{\pgfqpoint{2.881489in}{4.219649in}}%
\pgfpathlineto{\pgfqpoint{2.883996in}{4.223782in}}%
\pgfpathlineto{\pgfqpoint{2.886503in}{4.233061in}}%
\pgfpathlineto{\pgfqpoint{2.889009in}{4.234467in}}%
\pgfpathlineto{\pgfqpoint{2.891516in}{4.238041in}}%
\pgfpathlineto{\pgfqpoint{2.894022in}{4.243729in}}%
\pgfpathlineto{\pgfqpoint{2.899036in}{4.246054in}}%
\pgfpathlineto{\pgfqpoint{2.906556in}{4.253927in}}%
\pgfpathlineto{\pgfqpoint{2.909062in}{4.254648in}}%
\pgfpathlineto{\pgfqpoint{2.911569in}{4.262371in}}%
\pgfpathlineto{\pgfqpoint{2.914076in}{4.265457in}}%
\pgfpathlineto{\pgfqpoint{2.919089in}{4.265851in}}%
\pgfpathlineto{\pgfqpoint{2.921596in}{4.268881in}}%
\pgfpathlineto{\pgfqpoint{2.931622in}{4.271155in}}%
\pgfpathlineto{\pgfqpoint{2.934129in}{4.276525in}}%
\pgfpathlineto{\pgfqpoint{2.936636in}{4.276571in}}%
\pgfpathlineto{\pgfqpoint{2.941649in}{4.278815in}}%
\pgfpathlineto{\pgfqpoint{2.944156in}{4.278880in}}%
\pgfpathlineto{\pgfqpoint{2.949169in}{4.287491in}}%
\pgfpathlineto{\pgfqpoint{2.954182in}{4.290126in}}%
\pgfpathlineto{\pgfqpoint{2.956689in}{4.292403in}}%
\pgfpathlineto{\pgfqpoint{2.959195in}{4.292672in}}%
\pgfpathlineto{\pgfqpoint{2.966715in}{4.297179in}}%
\pgfpathlineto{\pgfqpoint{2.969222in}{4.297347in}}%
\pgfpathlineto{\pgfqpoint{2.971729in}{4.304362in}}%
\pgfpathlineto{\pgfqpoint{2.974235in}{4.304573in}}%
\pgfpathlineto{\pgfqpoint{2.976742in}{4.308133in}}%
\pgfpathlineto{\pgfqpoint{2.979249in}{4.308590in}}%
\pgfpathlineto{\pgfqpoint{2.984262in}{4.317810in}}%
\pgfpathlineto{\pgfqpoint{2.989275in}{4.319704in}}%
\pgfpathlineto{\pgfqpoint{2.994289in}{4.325328in}}%
\pgfpathlineto{\pgfqpoint{2.999302in}{4.327308in}}%
\pgfpathlineto{\pgfqpoint{3.001809in}{4.329200in}}%
\pgfpathlineto{\pgfqpoint{3.004315in}{4.340682in}}%
\pgfpathlineto{\pgfqpoint{3.006822in}{4.342386in}}%
\pgfpathlineto{\pgfqpoint{3.011835in}{4.343211in}}%
\pgfpathlineto{\pgfqpoint{3.016849in}{4.349918in}}%
\pgfpathlineto{\pgfqpoint{3.019355in}{4.355077in}}%
\pgfpathlineto{\pgfqpoint{3.021862in}{4.356762in}}%
\pgfpathlineto{\pgfqpoint{3.026875in}{4.357825in}}%
\pgfpathlineto{\pgfqpoint{3.034395in}{4.362964in}}%
\pgfpathlineto{\pgfqpoint{3.036902in}{4.363512in}}%
\pgfpathlineto{\pgfqpoint{3.041915in}{4.369528in}}%
\pgfpathlineto{\pgfqpoint{3.044422in}{4.371453in}}%
\pgfpathlineto{\pgfqpoint{3.046928in}{4.380970in}}%
\pgfpathlineto{\pgfqpoint{3.051942in}{4.387960in}}%
\pgfpathlineto{\pgfqpoint{3.054448in}{4.391464in}}%
\pgfpathlineto{\pgfqpoint{3.061968in}{4.395777in}}%
\pgfpathlineto{\pgfqpoint{3.066982in}{4.397578in}}%
\pgfpathlineto{\pgfqpoint{3.069488in}{4.403292in}}%
\pgfpathlineto{\pgfqpoint{3.071995in}{4.404520in}}%
\pgfpathlineto{\pgfqpoint{3.077008in}{4.417657in}}%
\pgfpathlineto{\pgfqpoint{3.079515in}{4.420512in}}%
\pgfpathlineto{\pgfqpoint{3.082022in}{4.421527in}}%
\pgfpathlineto{\pgfqpoint{3.084528in}{4.424113in}}%
\pgfpathlineto{\pgfqpoint{3.094555in}{4.440103in}}%
\pgfpathlineto{\pgfqpoint{3.097061in}{4.441382in}}%
\pgfpathlineto{\pgfqpoint{3.104581in}{4.450325in}}%
\pgfpathlineto{\pgfqpoint{3.112101in}{4.451630in}}%
\pgfpathlineto{\pgfqpoint{3.117115in}{4.452239in}}%
\pgfpathlineto{\pgfqpoint{3.119621in}{4.454561in}}%
\pgfpathlineto{\pgfqpoint{3.122128in}{4.458904in}}%
\pgfpathlineto{\pgfqpoint{3.129648in}{4.462998in}}%
\pgfpathlineto{\pgfqpoint{3.134661in}{4.470495in}}%
\pgfpathlineto{\pgfqpoint{3.137168in}{4.470791in}}%
\pgfpathlineto{\pgfqpoint{3.139675in}{4.477645in}}%
\pgfpathlineto{\pgfqpoint{3.147195in}{4.480825in}}%
\pgfpathlineto{\pgfqpoint{3.149701in}{4.485252in}}%
\pgfpathlineto{\pgfqpoint{3.154715in}{4.488529in}}%
\pgfpathlineto{\pgfqpoint{3.157221in}{4.489017in}}%
\pgfpathlineto{\pgfqpoint{3.159728in}{4.494817in}}%
\pgfpathlineto{\pgfqpoint{3.164741in}{4.495894in}}%
\pgfpathlineto{\pgfqpoint{3.167248in}{4.502752in}}%
\pgfpathlineto{\pgfqpoint{3.174768in}{4.507201in}}%
\pgfpathlineto{\pgfqpoint{3.177274in}{4.508802in}}%
\pgfpathlineto{\pgfqpoint{3.179781in}{4.512320in}}%
\pgfpathlineto{\pgfqpoint{3.184794in}{4.513816in}}%
\pgfpathlineto{\pgfqpoint{3.189808in}{4.520055in}}%
\pgfpathlineto{\pgfqpoint{3.194821in}{4.523252in}}%
\pgfpathlineto{\pgfqpoint{3.197328in}{4.527167in}}%
\pgfpathlineto{\pgfqpoint{3.199834in}{4.527479in}}%
\pgfpathlineto{\pgfqpoint{3.204848in}{4.532151in}}%
\pgfpathlineto{\pgfqpoint{3.207354in}{4.533165in}}%
\pgfpathlineto{\pgfqpoint{3.209861in}{4.538817in}}%
\pgfpathlineto{\pgfqpoint{3.214874in}{4.539035in}}%
\pgfpathlineto{\pgfqpoint{3.219888in}{4.542937in}}%
\pgfpathlineto{\pgfqpoint{3.224901in}{4.544993in}}%
\pgfpathlineto{\pgfqpoint{3.227408in}{4.550503in}}%
\pgfpathlineto{\pgfqpoint{3.232421in}{4.552016in}}%
\pgfpathlineto{\pgfqpoint{3.234927in}{4.552043in}}%
\pgfpathlineto{\pgfqpoint{3.237434in}{4.555961in}}%
\pgfpathlineto{\pgfqpoint{3.239941in}{4.556525in}}%
\pgfpathlineto{\pgfqpoint{3.252474in}{4.565708in}}%
\pgfpathlineto{\pgfqpoint{3.257487in}{4.567081in}}%
\pgfpathlineto{\pgfqpoint{3.267514in}{4.569840in}}%
\pgfpathlineto{\pgfqpoint{3.270021in}{4.570207in}}%
\pgfpathlineto{\pgfqpoint{3.272527in}{4.571925in}}%
\pgfpathlineto{\pgfqpoint{3.275034in}{4.581127in}}%
\pgfpathlineto{\pgfqpoint{3.277541in}{4.581965in}}%
\pgfpathlineto{\pgfqpoint{3.280047in}{4.586235in}}%
\pgfpathlineto{\pgfqpoint{3.285061in}{4.587331in}}%
\pgfpathlineto{\pgfqpoint{3.287567in}{4.588836in}}%
\pgfpathlineto{\pgfqpoint{3.290074in}{4.593223in}}%
\pgfpathlineto{\pgfqpoint{3.292581in}{4.593548in}}%
\pgfpathlineto{\pgfqpoint{3.297594in}{4.599206in}}%
\pgfpathlineto{\pgfqpoint{3.300100in}{4.601654in}}%
\pgfpathlineto{\pgfqpoint{3.302607in}{4.602388in}}%
\pgfpathlineto{\pgfqpoint{3.315140in}{4.612814in}}%
\pgfpathlineto{\pgfqpoint{3.327674in}{4.618562in}}%
\pgfpathlineto{\pgfqpoint{3.332687in}{4.619550in}}%
\pgfpathlineto{\pgfqpoint{3.335194in}{4.623224in}}%
\pgfpathlineto{\pgfqpoint{3.340207in}{4.625638in}}%
\pgfpathlineto{\pgfqpoint{3.342714in}{4.633018in}}%
\pgfpathlineto{\pgfqpoint{3.347727in}{4.634859in}}%
\pgfpathlineto{\pgfqpoint{3.352740in}{4.637481in}}%
\pgfpathlineto{\pgfqpoint{3.355247in}{4.642326in}}%
\pgfpathlineto{\pgfqpoint{3.360260in}{4.643217in}}%
\pgfpathlineto{\pgfqpoint{3.365273in}{4.653550in}}%
\pgfpathlineto{\pgfqpoint{3.370287in}{4.653886in}}%
\pgfpathlineto{\pgfqpoint{3.375300in}{4.659866in}}%
\pgfpathlineto{\pgfqpoint{3.377807in}{4.661092in}}%
\pgfpathlineto{\pgfqpoint{3.380313in}{4.665198in}}%
\pgfpathlineto{\pgfqpoint{3.382820in}{4.665645in}}%
\pgfpathlineto{\pgfqpoint{3.385327in}{4.675836in}}%
\pgfpathlineto{\pgfqpoint{3.392847in}{4.678384in}}%
\pgfpathlineto{\pgfqpoint{3.397860in}{4.682752in}}%
\pgfpathlineto{\pgfqpoint{3.400367in}{4.682825in}}%
\pgfpathlineto{\pgfqpoint{3.410393in}{4.687735in}}%
\pgfpathlineto{\pgfqpoint{3.412900in}{4.687898in}}%
\pgfpathlineto{\pgfqpoint{3.415407in}{4.689607in}}%
\pgfpathlineto{\pgfqpoint{3.422927in}{4.690957in}}%
\pgfpathlineto{\pgfqpoint{3.430447in}{4.693325in}}%
\pgfpathlineto{\pgfqpoint{3.435460in}{4.695076in}}%
\pgfpathlineto{\pgfqpoint{3.440473in}{4.697688in}}%
\pgfpathlineto{\pgfqpoint{3.445486in}{4.700954in}}%
\pgfpathlineto{\pgfqpoint{3.460526in}{4.711019in}}%
\pgfpathlineto{\pgfqpoint{3.463033in}{4.715025in}}%
\pgfpathlineto{\pgfqpoint{3.470553in}{4.716352in}}%
\pgfpathlineto{\pgfqpoint{3.473060in}{4.723283in}}%
\pgfpathlineto{\pgfqpoint{3.478073in}{4.727164in}}%
\pgfpathlineto{\pgfqpoint{3.488100in}{4.730882in}}%
\pgfpathlineto{\pgfqpoint{3.493113in}{4.732149in}}%
\pgfpathlineto{\pgfqpoint{3.503139in}{4.733395in}}%
\pgfpathlineto{\pgfqpoint{3.510659in}{4.737841in}}%
\pgfpathlineto{\pgfqpoint{3.513166in}{4.742653in}}%
\pgfpathlineto{\pgfqpoint{3.520686in}{4.745449in}}%
\pgfpathlineto{\pgfqpoint{3.523193in}{4.745502in}}%
\pgfpathlineto{\pgfqpoint{3.528206in}{4.748394in}}%
\pgfpathlineto{\pgfqpoint{3.535726in}{4.750916in}}%
\pgfpathlineto{\pgfqpoint{3.540739in}{4.752906in}}%
\pgfpathlineto{\pgfqpoint{3.545753in}{4.753799in}}%
\pgfpathlineto{\pgfqpoint{3.550766in}{4.757713in}}%
\pgfpathlineto{\pgfqpoint{3.555779in}{4.763125in}}%
\pgfpathlineto{\pgfqpoint{3.563299in}{4.766525in}}%
\pgfpathlineto{\pgfqpoint{3.565806in}{4.766956in}}%
\pgfpathlineto{\pgfqpoint{3.568313in}{4.770410in}}%
\pgfpathlineto{\pgfqpoint{3.573326in}{4.772694in}}%
\pgfpathlineto{\pgfqpoint{3.575832in}{4.776468in}}%
\pgfpathlineto{\pgfqpoint{3.578339in}{4.777964in}}%
\pgfpathlineto{\pgfqpoint{3.585859in}{4.779062in}}%
\pgfpathlineto{\pgfqpoint{3.588366in}{4.782174in}}%
\pgfpathlineto{\pgfqpoint{3.590872in}{4.782530in}}%
\pgfpathlineto{\pgfqpoint{3.598392in}{4.790884in}}%
\pgfpathlineto{\pgfqpoint{3.605912in}{4.793437in}}%
\pgfpathlineto{\pgfqpoint{3.608419in}{4.793761in}}%
\pgfpathlineto{\pgfqpoint{3.610926in}{4.795956in}}%
\pgfpathlineto{\pgfqpoint{3.615939in}{4.797638in}}%
\pgfpathlineto{\pgfqpoint{3.628472in}{4.804378in}}%
\pgfpathlineto{\pgfqpoint{3.635992in}{4.806435in}}%
\pgfpathlineto{\pgfqpoint{3.638499in}{4.809232in}}%
\pgfpathlineto{\pgfqpoint{3.641005in}{4.809642in}}%
\pgfpathlineto{\pgfqpoint{3.643512in}{4.815502in}}%
\pgfpathlineto{\pgfqpoint{3.648525in}{4.817155in}}%
\pgfpathlineto{\pgfqpoint{3.663565in}{4.821986in}}%
\pgfpathlineto{\pgfqpoint{3.668579in}{4.823524in}}%
\pgfpathlineto{\pgfqpoint{3.671085in}{4.825113in}}%
\pgfpathlineto{\pgfqpoint{3.676099in}{4.825952in}}%
\pgfpathlineto{\pgfqpoint{3.681112in}{4.830066in}}%
\pgfpathlineto{\pgfqpoint{3.686125in}{4.834504in}}%
\pgfpathlineto{\pgfqpoint{3.688632in}{4.834566in}}%
\pgfpathlineto{\pgfqpoint{3.693645in}{4.840164in}}%
\pgfpathlineto{\pgfqpoint{3.703672in}{4.841437in}}%
\pgfpathlineto{\pgfqpoint{3.708685in}{4.848094in}}%
\pgfpathlineto{\pgfqpoint{3.713698in}{4.848870in}}%
\pgfpathlineto{\pgfqpoint{3.716205in}{4.850196in}}%
\pgfpathlineto{\pgfqpoint{3.723725in}{4.860323in}}%
\pgfpathlineto{\pgfqpoint{3.726232in}{4.860324in}}%
\pgfpathlineto{\pgfqpoint{3.728738in}{4.865732in}}%
\pgfpathlineto{\pgfqpoint{3.733752in}{4.866240in}}%
\pgfpathlineto{\pgfqpoint{3.738765in}{4.874194in}}%
\pgfpathlineto{\pgfqpoint{3.746285in}{4.883464in}}%
\pgfpathlineto{\pgfqpoint{3.748792in}{4.883677in}}%
\pgfpathlineto{\pgfqpoint{3.753805in}{4.888054in}}%
\pgfpathlineto{\pgfqpoint{3.766338in}{4.895492in}}%
\pgfpathlineto{\pgfqpoint{3.768845in}{4.896629in}}%
\pgfpathlineto{\pgfqpoint{3.771352in}{4.900907in}}%
\pgfpathlineto{\pgfqpoint{3.773858in}{4.900917in}}%
\pgfpathlineto{\pgfqpoint{3.781378in}{4.906135in}}%
\pgfpathlineto{\pgfqpoint{3.788898in}{4.908507in}}%
\pgfpathlineto{\pgfqpoint{3.793911in}{4.910333in}}%
\pgfpathlineto{\pgfqpoint{3.796418in}{4.911796in}}%
\pgfpathlineto{\pgfqpoint{3.798925in}{4.919543in}}%
\pgfpathlineto{\pgfqpoint{3.803938in}{4.921086in}}%
\pgfpathlineto{\pgfqpoint{3.806445in}{4.929918in}}%
\pgfpathlineto{\pgfqpoint{3.811458in}{4.931276in}}%
\pgfpathlineto{\pgfqpoint{3.813965in}{4.933685in}}%
\pgfpathlineto{\pgfqpoint{3.818978in}{4.935101in}}%
\pgfpathlineto{\pgfqpoint{3.823991in}{4.939888in}}%
\pgfpathlineto{\pgfqpoint{3.826498in}{4.940489in}}%
\pgfpathlineto{\pgfqpoint{3.829005in}{4.946132in}}%
\pgfpathlineto{\pgfqpoint{3.836525in}{4.951977in}}%
\pgfpathlineto{\pgfqpoint{3.839031in}{4.952707in}}%
\pgfpathlineto{\pgfqpoint{3.841538in}{4.954994in}}%
\pgfpathlineto{\pgfqpoint{3.846551in}{4.962784in}}%
\pgfpathlineto{\pgfqpoint{3.854071in}{4.965513in}}%
\pgfpathlineto{\pgfqpoint{3.861591in}{4.969326in}}%
\pgfpathlineto{\pgfqpoint{3.864098in}{4.973909in}}%
\pgfpathlineto{\pgfqpoint{3.866604in}{4.974065in}}%
\pgfpathlineto{\pgfqpoint{3.869111in}{4.975457in}}%
\pgfpathlineto{\pgfqpoint{3.871618in}{4.979265in}}%
\pgfpathlineto{\pgfqpoint{3.879138in}{4.980307in}}%
\pgfpathlineto{\pgfqpoint{3.881644in}{4.985630in}}%
\pgfpathlineto{\pgfqpoint{3.884151in}{4.986055in}}%
\pgfpathlineto{\pgfqpoint{3.886658in}{4.989661in}}%
\pgfpathlineto{\pgfqpoint{3.889164in}{4.996562in}}%
\pgfpathlineto{\pgfqpoint{3.891671in}{4.999482in}}%
\pgfpathlineto{\pgfqpoint{3.899191in}{4.999860in}}%
\pgfpathlineto{\pgfqpoint{3.904204in}{5.005423in}}%
\pgfpathlineto{\pgfqpoint{3.909217in}{5.006627in}}%
\pgfpathlineto{\pgfqpoint{3.911724in}{5.006916in}}%
\pgfpathlineto{\pgfqpoint{3.916737in}{5.015475in}}%
\pgfpathlineto{\pgfqpoint{3.919244in}{5.017993in}}%
\pgfpathlineto{\pgfqpoint{3.924257in}{5.026737in}}%
\pgfpathlineto{\pgfqpoint{3.934284in}{5.033177in}}%
\pgfpathlineto{\pgfqpoint{3.941804in}{5.045432in}}%
\pgfpathlineto{\pgfqpoint{3.949324in}{5.047467in}}%
\pgfpathlineto{\pgfqpoint{3.959351in}{5.062360in}}%
\pgfpathlineto{\pgfqpoint{3.964364in}{5.062711in}}%
\pgfpathlineto{\pgfqpoint{3.966871in}{5.072611in}}%
\pgfpathlineto{\pgfqpoint{3.969377in}{5.075992in}}%
\pgfpathlineto{\pgfqpoint{3.971884in}{5.076132in}}%
\pgfpathlineto{\pgfqpoint{3.974391in}{5.077856in}}%
\pgfpathlineto{\pgfqpoint{3.979404in}{5.084687in}}%
\pgfpathlineto{\pgfqpoint{3.981910in}{5.085983in}}%
\pgfpathlineto{\pgfqpoint{3.984417in}{5.088940in}}%
\pgfpathlineto{\pgfqpoint{3.986924in}{5.089400in}}%
\pgfpathlineto{\pgfqpoint{3.994444in}{5.095287in}}%
\pgfpathlineto{\pgfqpoint{3.999457in}{5.110079in}}%
\pgfpathlineto{\pgfqpoint{4.004470in}{5.118705in}}%
\pgfpathlineto{\pgfqpoint{4.006977in}{5.120060in}}%
\pgfpathlineto{\pgfqpoint{4.011990in}{5.126071in}}%
\pgfpathlineto{\pgfqpoint{4.014497in}{5.135859in}}%
\pgfpathlineto{\pgfqpoint{4.019510in}{5.139744in}}%
\pgfpathlineto{\pgfqpoint{4.022017in}{5.144154in}}%
\pgfpathlineto{\pgfqpoint{4.024524in}{5.154739in}}%
\pgfpathlineto{\pgfqpoint{4.027030in}{5.157426in}}%
\pgfpathlineto{\pgfqpoint{4.032044in}{5.159824in}}%
\pgfpathlineto{\pgfqpoint{4.037057in}{5.182072in}}%
\pgfpathlineto{\pgfqpoint{4.039564in}{5.182316in}}%
\pgfpathlineto{\pgfqpoint{4.042070in}{5.184196in}}%
\pgfpathlineto{\pgfqpoint{4.044577in}{5.190206in}}%
\pgfpathlineto{\pgfqpoint{4.049590in}{5.193542in}}%
\pgfpathlineto{\pgfqpoint{4.052097in}{5.193638in}}%
\pgfpathlineto{\pgfqpoint{4.057110in}{5.204169in}}%
\pgfpathlineto{\pgfqpoint{4.059617in}{5.213692in}}%
\pgfpathlineto{\pgfqpoint{4.067137in}{5.218403in}}%
\pgfpathlineto{\pgfqpoint{4.069643in}{5.226561in}}%
\pgfpathlineto{\pgfqpoint{4.074657in}{5.229713in}}%
\pgfpathlineto{\pgfqpoint{4.082177in}{5.235261in}}%
\pgfpathlineto{\pgfqpoint{4.087190in}{5.236195in}}%
\pgfpathlineto{\pgfqpoint{4.089697in}{5.240848in}}%
\pgfpathlineto{\pgfqpoint{4.092203in}{5.251456in}}%
\pgfpathlineto{\pgfqpoint{4.094710in}{5.253295in}}%
\pgfpathlineto{\pgfqpoint{4.097217in}{5.258915in}}%
\pgfpathlineto{\pgfqpoint{4.102230in}{5.260419in}}%
\pgfpathlineto{\pgfqpoint{4.109750in}{5.264669in}}%
\pgfpathlineto{\pgfqpoint{4.112257in}{5.272707in}}%
\pgfpathlineto{\pgfqpoint{4.114763in}{5.274497in}}%
\pgfpathlineto{\pgfqpoint{4.122283in}{5.292448in}}%
\pgfpathlineto{\pgfqpoint{4.124790in}{5.305275in}}%
\pgfpathlineto{\pgfqpoint{4.124790in}{5.305275in}}%
\pgfusepath{stroke}%
\end{pgfscope}%
\begin{pgfscope}%
\pgfpathrectangle{\pgfqpoint{0.708220in}{3.210823in}}{\pgfqpoint{5.013309in}{2.094453in}}%
\pgfusepath{clip}%
\pgfsetrectcap%
\pgfsetroundjoin%
\pgfsetlinewidth{1.003750pt}%
\definecolor{currentstroke}{rgb}{0.062745,0.000000,0.062745}%
\pgfsetstrokecolor{currentstroke}%
\pgfsetdash{}{0pt}%
\pgfpathmoveto{\pgfqpoint{0.708220in}{3.268558in}}%
\pgfpathlineto{\pgfqpoint{0.710727in}{3.291471in}}%
\pgfpathlineto{\pgfqpoint{0.735793in}{3.291471in}}%
\pgfpathlineto{\pgfqpoint{0.738300in}{3.311682in}}%
\pgfpathlineto{\pgfqpoint{0.743313in}{3.311682in}}%
\pgfpathlineto{\pgfqpoint{0.745820in}{3.329760in}}%
\pgfpathlineto{\pgfqpoint{0.748326in}{3.329760in}}%
\pgfpathlineto{\pgfqpoint{0.750833in}{3.346115in}}%
\pgfpathlineto{\pgfqpoint{0.758353in}{3.346115in}}%
\pgfpathlineto{\pgfqpoint{0.760860in}{3.361045in}}%
\pgfpathlineto{\pgfqpoint{0.765873in}{3.361045in}}%
\pgfpathlineto{\pgfqpoint{0.768380in}{3.374780in}}%
\pgfpathlineto{\pgfqpoint{0.773393in}{3.374780in}}%
\pgfpathlineto{\pgfqpoint{0.775900in}{3.387496in}}%
\pgfpathlineto{\pgfqpoint{0.778406in}{3.420812in}}%
\pgfpathlineto{\pgfqpoint{0.780913in}{3.420812in}}%
\pgfpathlineto{\pgfqpoint{0.785926in}{3.439897in}}%
\pgfpathlineto{\pgfqpoint{0.816006in}{3.439897in}}%
\pgfpathlineto{\pgfqpoint{0.818513in}{3.448698in}}%
\pgfpathlineto{\pgfqpoint{0.918779in}{3.448698in}}%
\pgfpathlineto{\pgfqpoint{0.921285in}{3.457070in}}%
\pgfpathlineto{\pgfqpoint{0.988965in}{3.457070in}}%
\pgfpathlineto{\pgfqpoint{0.991472in}{3.465053in}}%
\pgfpathlineto{\pgfqpoint{1.051632in}{3.465053in}}%
\pgfpathlineto{\pgfqpoint{1.054138in}{3.472680in}}%
\pgfpathlineto{\pgfqpoint{1.106778in}{3.472680in}}%
\pgfpathlineto{\pgfqpoint{1.109285in}{3.479983in}}%
\pgfpathlineto{\pgfqpoint{1.134351in}{3.479983in}}%
\pgfpathlineto{\pgfqpoint{1.136858in}{3.486988in}}%
\pgfpathlineto{\pgfqpoint{1.217071in}{3.486988in}}%
\pgfpathlineto{\pgfqpoint{1.219577in}{3.493718in}}%
\pgfpathlineto{\pgfqpoint{1.329870in}{3.493718in}}%
\pgfpathlineto{\pgfqpoint{1.332377in}{3.500194in}}%
\pgfpathlineto{\pgfqpoint{1.390030in}{3.500194in}}%
\pgfpathlineto{\pgfqpoint{1.392537in}{3.506434in}}%
\pgfpathlineto{\pgfqpoint{1.435150in}{3.506434in}}%
\pgfpathlineto{\pgfqpoint{1.437656in}{3.512455in}}%
\pgfpathlineto{\pgfqpoint{1.490296in}{3.512455in}}%
\pgfpathlineto{\pgfqpoint{1.492803in}{3.518272in}}%
\pgfpathlineto{\pgfqpoint{1.542936in}{3.518272in}}%
\pgfpathlineto{\pgfqpoint{1.545442in}{3.523899in}}%
\pgfpathlineto{\pgfqpoint{1.618135in}{3.523899in}}%
\pgfpathlineto{\pgfqpoint{1.620642in}{3.529347in}}%
\pgfpathlineto{\pgfqpoint{1.693335in}{3.529347in}}%
\pgfpathlineto{\pgfqpoint{1.695842in}{3.534627in}}%
\pgfpathlineto{\pgfqpoint{1.763521in}{3.534627in}}%
\pgfpathlineto{\pgfqpoint{1.766028in}{3.539749in}}%
\pgfpathlineto{\pgfqpoint{1.801121in}{3.539749in}}%
\pgfpathlineto{\pgfqpoint{1.803628in}{3.544723in}}%
\pgfpathlineto{\pgfqpoint{1.833708in}{3.544723in}}%
\pgfpathlineto{\pgfqpoint{1.836214in}{3.549557in}}%
\pgfpathlineto{\pgfqpoint{1.871307in}{3.549557in}}%
\pgfpathlineto{\pgfqpoint{1.873814in}{3.554259in}}%
\pgfpathlineto{\pgfqpoint{1.911414in}{3.554259in}}%
\pgfpathlineto{\pgfqpoint{1.913921in}{3.558835in}}%
\pgfpathlineto{\pgfqpoint{1.931467in}{3.558835in}}%
\pgfpathlineto{\pgfqpoint{1.933974in}{3.563292in}}%
\pgfpathlineto{\pgfqpoint{1.941494in}{3.563292in}}%
\pgfpathlineto{\pgfqpoint{1.944000in}{3.567636in}}%
\pgfpathlineto{\pgfqpoint{1.946507in}{3.567636in}}%
\pgfpathlineto{\pgfqpoint{1.951520in}{3.576008in}}%
\pgfpathlineto{\pgfqpoint{1.956534in}{3.576008in}}%
\pgfpathlineto{\pgfqpoint{1.959040in}{3.580046in}}%
\pgfpathlineto{\pgfqpoint{1.964054in}{3.580046in}}%
\pgfpathlineto{\pgfqpoint{1.969067in}{3.587847in}}%
\pgfpathlineto{\pgfqpoint{1.974080in}{3.587847in}}%
\pgfpathlineto{\pgfqpoint{1.979094in}{3.595308in}}%
\pgfpathlineto{\pgfqpoint{1.981600in}{3.595308in}}%
\pgfpathlineto{\pgfqpoint{1.984107in}{3.598921in}}%
\pgfpathlineto{\pgfqpoint{1.986614in}{3.598921in}}%
\pgfpathlineto{\pgfqpoint{1.989120in}{3.602459in}}%
\pgfpathlineto{\pgfqpoint{1.991627in}{3.602459in}}%
\pgfpathlineto{\pgfqpoint{1.994134in}{3.612656in}}%
\pgfpathlineto{\pgfqpoint{1.996640in}{3.615924in}}%
\pgfpathlineto{\pgfqpoint{2.004160in}{3.615924in}}%
\pgfpathlineto{\pgfqpoint{2.006667in}{3.619131in}}%
\pgfpathlineto{\pgfqpoint{2.016693in}{3.619131in}}%
\pgfpathlineto{\pgfqpoint{2.019200in}{3.622280in}}%
\pgfpathlineto{\pgfqpoint{2.076853in}{3.622280in}}%
\pgfpathlineto{\pgfqpoint{2.079360in}{3.625372in}}%
\pgfpathlineto{\pgfqpoint{2.109440in}{3.625372in}}%
\pgfpathlineto{\pgfqpoint{2.111946in}{3.628409in}}%
\pgfpathlineto{\pgfqpoint{2.162079in}{3.628409in}}%
\pgfpathlineto{\pgfqpoint{2.164586in}{3.631393in}}%
\pgfpathlineto{\pgfqpoint{2.199679in}{3.631393in}}%
\pgfpathlineto{\pgfqpoint{2.202186in}{3.634326in}}%
\pgfpathlineto{\pgfqpoint{2.244799in}{3.634326in}}%
\pgfpathlineto{\pgfqpoint{2.247306in}{3.637210in}}%
\pgfpathlineto{\pgfqpoint{2.284905in}{3.637210in}}%
\pgfpathlineto{\pgfqpoint{2.287412in}{3.640047in}}%
\pgfpathlineto{\pgfqpoint{2.314985in}{3.640047in}}%
\pgfpathlineto{\pgfqpoint{2.317492in}{3.642837in}}%
\pgfpathlineto{\pgfqpoint{2.355092in}{3.642837in}}%
\pgfpathlineto{\pgfqpoint{2.357598in}{3.645582in}}%
\pgfpathlineto{\pgfqpoint{2.382665in}{3.645582in}}%
\pgfpathlineto{\pgfqpoint{2.385172in}{3.648285in}}%
\pgfpathlineto{\pgfqpoint{2.392692in}{3.648285in}}%
\pgfpathlineto{\pgfqpoint{2.395198in}{3.650945in}}%
\pgfpathlineto{\pgfqpoint{2.412745in}{3.650945in}}%
\pgfpathlineto{\pgfqpoint{2.415251in}{3.653565in}}%
\pgfpathlineto{\pgfqpoint{2.420265in}{3.653565in}}%
\pgfpathlineto{\pgfqpoint{2.422771in}{3.656145in}}%
\pgfpathlineto{\pgfqpoint{2.430291in}{3.656145in}}%
\pgfpathlineto{\pgfqpoint{2.432798in}{3.658687in}}%
\pgfpathlineto{\pgfqpoint{2.445331in}{3.658687in}}%
\pgfpathlineto{\pgfqpoint{2.447838in}{3.661192in}}%
\pgfpathlineto{\pgfqpoint{2.452851in}{3.661192in}}%
\pgfpathlineto{\pgfqpoint{2.455358in}{3.663661in}}%
\pgfpathlineto{\pgfqpoint{2.460371in}{3.663661in}}%
\pgfpathlineto{\pgfqpoint{2.462878in}{3.666095in}}%
\pgfpathlineto{\pgfqpoint{2.465385in}{3.666095in}}%
\pgfpathlineto{\pgfqpoint{2.470398in}{3.670862in}}%
\pgfpathlineto{\pgfqpoint{2.475411in}{3.670862in}}%
\pgfpathlineto{\pgfqpoint{2.477918in}{3.673196in}}%
\pgfpathlineto{\pgfqpoint{2.485438in}{3.673196in}}%
\pgfpathlineto{\pgfqpoint{2.487944in}{3.675500in}}%
\pgfpathlineto{\pgfqpoint{2.492958in}{3.675500in}}%
\pgfpathlineto{\pgfqpoint{2.495464in}{3.677773in}}%
\pgfpathlineto{\pgfqpoint{2.500478in}{3.677773in}}%
\pgfpathlineto{\pgfqpoint{2.502984in}{3.680016in}}%
\pgfpathlineto{\pgfqpoint{2.507998in}{3.680016in}}%
\pgfpathlineto{\pgfqpoint{2.510504in}{3.682230in}}%
\pgfpathlineto{\pgfqpoint{2.513011in}{3.682230in}}%
\pgfpathlineto{\pgfqpoint{2.518024in}{3.686574in}}%
\pgfpathlineto{\pgfqpoint{2.520531in}{3.686574in}}%
\pgfpathlineto{\pgfqpoint{2.525544in}{3.690811in}}%
\pgfpathlineto{\pgfqpoint{2.528051in}{3.690811in}}%
\pgfpathlineto{\pgfqpoint{2.538078in}{3.708681in}}%
\pgfpathlineto{\pgfqpoint{2.540584in}{3.716062in}}%
\pgfpathlineto{\pgfqpoint{2.543091in}{3.717859in}}%
\pgfpathlineto{\pgfqpoint{2.548104in}{3.717859in}}%
\pgfpathlineto{\pgfqpoint{2.553117in}{3.723139in}}%
\pgfpathlineto{\pgfqpoint{2.555624in}{3.731593in}}%
\pgfpathlineto{\pgfqpoint{2.560637in}{3.731593in}}%
\pgfpathlineto{\pgfqpoint{2.563144in}{3.736473in}}%
\pgfpathlineto{\pgfqpoint{2.565651in}{3.738069in}}%
\pgfpathlineto{\pgfqpoint{2.568157in}{3.741218in}}%
\pgfpathlineto{\pgfqpoint{2.570664in}{3.741218in}}%
\pgfpathlineto{\pgfqpoint{2.573171in}{3.747347in}}%
\pgfpathlineto{\pgfqpoint{2.575677in}{3.747347in}}%
\pgfpathlineto{\pgfqpoint{2.585704in}{3.753264in}}%
\pgfpathlineto{\pgfqpoint{2.588211in}{3.753264in}}%
\pgfpathlineto{\pgfqpoint{2.593224in}{3.756148in}}%
\pgfpathlineto{\pgfqpoint{2.595731in}{3.758984in}}%
\pgfpathlineto{\pgfqpoint{2.598237in}{3.758984in}}%
\pgfpathlineto{\pgfqpoint{2.603251in}{3.761775in}}%
\pgfpathlineto{\pgfqpoint{2.605757in}{3.764520in}}%
\pgfpathlineto{\pgfqpoint{2.610771in}{3.764520in}}%
\pgfpathlineto{\pgfqpoint{2.620797in}{3.780130in}}%
\pgfpathlineto{\pgfqpoint{2.623304in}{3.782599in}}%
\pgfpathlineto{\pgfqpoint{2.625810in}{3.787433in}}%
\pgfpathlineto{\pgfqpoint{2.630824in}{3.788620in}}%
\pgfpathlineto{\pgfqpoint{2.633330in}{3.788620in}}%
\pgfpathlineto{\pgfqpoint{2.635837in}{3.792134in}}%
\pgfpathlineto{\pgfqpoint{2.645864in}{3.795578in}}%
\pgfpathlineto{\pgfqpoint{2.653384in}{3.795578in}}%
\pgfpathlineto{\pgfqpoint{2.658397in}{3.808699in}}%
\pgfpathlineto{\pgfqpoint{2.660904in}{3.808699in}}%
\pgfpathlineto{\pgfqpoint{2.663410in}{3.811829in}}%
\pgfpathlineto{\pgfqpoint{2.665917in}{3.817921in}}%
\pgfpathlineto{\pgfqpoint{2.668424in}{3.819905in}}%
\pgfpathlineto{\pgfqpoint{2.670930in}{3.823805in}}%
\pgfpathlineto{\pgfqpoint{2.680957in}{3.827618in}}%
\pgfpathlineto{\pgfqpoint{2.683464in}{3.830424in}}%
\pgfpathlineto{\pgfqpoint{2.688477in}{3.832269in}}%
\pgfpathlineto{\pgfqpoint{2.690983in}{3.835000in}}%
\pgfpathlineto{\pgfqpoint{2.706023in}{3.839457in}}%
\pgfpathlineto{\pgfqpoint{2.716050in}{3.853800in}}%
\pgfpathlineto{\pgfqpoint{2.723570in}{3.854607in}}%
\pgfpathlineto{\pgfqpoint{2.726077in}{3.864012in}}%
\pgfpathlineto{\pgfqpoint{2.728583in}{3.864773in}}%
\pgfpathlineto{\pgfqpoint{2.733597in}{3.869269in}}%
\pgfpathlineto{\pgfqpoint{2.736103in}{3.871473in}}%
\pgfpathlineto{\pgfqpoint{2.741117in}{3.872202in}}%
\pgfpathlineto{\pgfqpoint{2.746130in}{3.875086in}}%
\pgfpathlineto{\pgfqpoint{2.748637in}{3.875086in}}%
\pgfpathlineto{\pgfqpoint{2.753650in}{3.877922in}}%
\pgfpathlineto{\pgfqpoint{2.758663in}{3.879323in}}%
\pgfpathlineto{\pgfqpoint{2.761170in}{3.885489in}}%
\pgfpathlineto{\pgfqpoint{2.763676in}{3.886160in}}%
\pgfpathlineto{\pgfqpoint{2.766183in}{3.888821in}}%
\pgfpathlineto{\pgfqpoint{2.768690in}{3.889479in}}%
\pgfpathlineto{\pgfqpoint{2.778716in}{3.895931in}}%
\pgfpathlineto{\pgfqpoint{2.783730in}{3.895931in}}%
\pgfpathlineto{\pgfqpoint{2.788743in}{3.899068in}}%
\pgfpathlineto{\pgfqpoint{2.791250in}{3.899688in}}%
\pgfpathlineto{\pgfqpoint{2.796263in}{3.902149in}}%
\pgfpathlineto{\pgfqpoint{2.808796in}{3.908149in}}%
\pgfpathlineto{\pgfqpoint{2.811303in}{3.908149in}}%
\pgfpathlineto{\pgfqpoint{2.813810in}{3.912803in}}%
\pgfpathlineto{\pgfqpoint{2.818823in}{3.913375in}}%
\pgfpathlineto{\pgfqpoint{2.821329in}{3.917891in}}%
\pgfpathlineto{\pgfqpoint{2.823836in}{3.917891in}}%
\pgfpathlineto{\pgfqpoint{2.826343in}{3.920654in}}%
\pgfpathlineto{\pgfqpoint{2.828849in}{3.921202in}}%
\pgfpathlineto{\pgfqpoint{2.838876in}{3.932822in}}%
\pgfpathlineto{\pgfqpoint{2.841383in}{3.932822in}}%
\pgfpathlineto{\pgfqpoint{2.846396in}{3.938349in}}%
\pgfpathlineto{\pgfqpoint{2.848903in}{3.938843in}}%
\pgfpathlineto{\pgfqpoint{2.851409in}{3.942260in}}%
\pgfpathlineto{\pgfqpoint{2.853916in}{3.948897in}}%
\pgfpathlineto{\pgfqpoint{2.863943in}{3.957954in}}%
\pgfpathlineto{\pgfqpoint{2.871463in}{3.959710in}}%
\pgfpathlineto{\pgfqpoint{2.876476in}{3.964021in}}%
\pgfpathlineto{\pgfqpoint{2.886503in}{3.965716in}}%
\pgfpathlineto{\pgfqpoint{2.889009in}{3.967811in}}%
\pgfpathlineto{\pgfqpoint{2.894022in}{3.968227in}}%
\pgfpathlineto{\pgfqpoint{2.896529in}{3.971111in}}%
\pgfpathlineto{\pgfqpoint{2.909062in}{3.975547in}}%
\pgfpathlineto{\pgfqpoint{2.911569in}{3.978703in}}%
\pgfpathlineto{\pgfqpoint{2.914076in}{3.979483in}}%
\pgfpathlineto{\pgfqpoint{2.921596in}{3.985598in}}%
\pgfpathlineto{\pgfqpoint{2.926609in}{3.986721in}}%
\pgfpathlineto{\pgfqpoint{2.929116in}{3.989680in}}%
\pgfpathlineto{\pgfqpoint{2.934129in}{3.990046in}}%
\pgfpathlineto{\pgfqpoint{2.944156in}{3.995093in}}%
\pgfpathlineto{\pgfqpoint{2.949169in}{3.996860in}}%
\pgfpathlineto{\pgfqpoint{2.954182in}{3.998609in}}%
\pgfpathlineto{\pgfqpoint{2.959195in}{3.998609in}}%
\pgfpathlineto{\pgfqpoint{2.961702in}{4.006433in}}%
\pgfpathlineto{\pgfqpoint{2.971729in}{4.013916in}}%
\pgfpathlineto{\pgfqpoint{2.976742in}{4.021392in}}%
\pgfpathlineto{\pgfqpoint{2.986769in}{4.025606in}}%
\pgfpathlineto{\pgfqpoint{2.989275in}{4.029429in}}%
\pgfpathlineto{\pgfqpoint{2.991782in}{4.030010in}}%
\pgfpathlineto{\pgfqpoint{2.994289in}{4.034021in}}%
\pgfpathlineto{\pgfqpoint{3.009329in}{4.039866in}}%
\pgfpathlineto{\pgfqpoint{3.019355in}{4.041229in}}%
\pgfpathlineto{\pgfqpoint{3.029382in}{4.051504in}}%
\pgfpathlineto{\pgfqpoint{3.031888in}{4.052778in}}%
\pgfpathlineto{\pgfqpoint{3.034395in}{4.052778in}}%
\pgfpathlineto{\pgfqpoint{3.039408in}{4.056295in}}%
\pgfpathlineto{\pgfqpoint{3.041915in}{4.060229in}}%
\pgfpathlineto{\pgfqpoint{3.054448in}{4.063836in}}%
\pgfpathlineto{\pgfqpoint{3.056955in}{4.069455in}}%
\pgfpathlineto{\pgfqpoint{3.061968in}{4.070374in}}%
\pgfpathlineto{\pgfqpoint{3.066982in}{4.074225in}}%
\pgfpathlineto{\pgfqpoint{3.074502in}{4.075786in}}%
\pgfpathlineto{\pgfqpoint{3.084528in}{4.085705in}}%
\pgfpathlineto{\pgfqpoint{3.087035in}{4.085705in}}%
\pgfpathlineto{\pgfqpoint{3.094555in}{4.091675in}}%
\pgfpathlineto{\pgfqpoint{3.109595in}{4.095873in}}%
\pgfpathlineto{\pgfqpoint{3.114608in}{4.102078in}}%
\pgfpathlineto{\pgfqpoint{3.122128in}{4.104911in}}%
\pgfpathlineto{\pgfqpoint{3.124635in}{4.112066in}}%
\pgfpathlineto{\pgfqpoint{3.127141in}{4.114208in}}%
\pgfpathlineto{\pgfqpoint{3.132155in}{4.115446in}}%
\pgfpathlineto{\pgfqpoint{3.134661in}{4.120993in}}%
\pgfpathlineto{\pgfqpoint{3.147195in}{4.123532in}}%
\pgfpathlineto{\pgfqpoint{3.154715in}{4.126696in}}%
\pgfpathlineto{\pgfqpoint{3.162234in}{4.132535in}}%
\pgfpathlineto{\pgfqpoint{3.169754in}{4.135539in}}%
\pgfpathlineto{\pgfqpoint{3.177274in}{4.142148in}}%
\pgfpathlineto{\pgfqpoint{3.179781in}{4.142450in}}%
\pgfpathlineto{\pgfqpoint{3.182288in}{4.145730in}}%
\pgfpathlineto{\pgfqpoint{3.187301in}{4.147639in}}%
\pgfpathlineto{\pgfqpoint{3.197328in}{4.152107in}}%
\pgfpathlineto{\pgfqpoint{3.199834in}{4.154929in}}%
\pgfpathlineto{\pgfqpoint{3.204848in}{4.155349in}}%
\pgfpathlineto{\pgfqpoint{3.214874in}{4.164059in}}%
\pgfpathlineto{\pgfqpoint{3.222394in}{4.174485in}}%
\pgfpathlineto{\pgfqpoint{3.227408in}{4.175854in}}%
\pgfpathlineto{\pgfqpoint{3.232421in}{4.181460in}}%
\pgfpathlineto{\pgfqpoint{3.239941in}{4.182178in}}%
\pgfpathlineto{\pgfqpoint{3.244954in}{4.188162in}}%
\pgfpathlineto{\pgfqpoint{3.247461in}{4.188623in}}%
\pgfpathlineto{\pgfqpoint{3.252474in}{4.191248in}}%
\pgfpathlineto{\pgfqpoint{3.254981in}{4.191700in}}%
\pgfpathlineto{\pgfqpoint{3.262501in}{4.200400in}}%
\pgfpathlineto{\pgfqpoint{3.265007in}{4.200400in}}%
\pgfpathlineto{\pgfqpoint{3.267514in}{4.204328in}}%
\pgfpathlineto{\pgfqpoint{3.272527in}{4.218715in}}%
\pgfpathlineto{\pgfqpoint{3.275034in}{4.220444in}}%
\pgfpathlineto{\pgfqpoint{3.277541in}{4.220539in}}%
\pgfpathlineto{\pgfqpoint{3.280047in}{4.223192in}}%
\pgfpathlineto{\pgfqpoint{3.282554in}{4.223568in}}%
\pgfpathlineto{\pgfqpoint{3.290074in}{4.235700in}}%
\pgfpathlineto{\pgfqpoint{3.292581in}{4.237266in}}%
\pgfpathlineto{\pgfqpoint{3.297594in}{4.258197in}}%
\pgfpathlineto{\pgfqpoint{3.305114in}{4.260179in}}%
\pgfpathlineto{\pgfqpoint{3.307620in}{4.269688in}}%
\pgfpathlineto{\pgfqpoint{3.310127in}{4.269831in}}%
\pgfpathlineto{\pgfqpoint{3.315140in}{4.274077in}}%
\pgfpathlineto{\pgfqpoint{3.317647in}{4.276299in}}%
\pgfpathlineto{\pgfqpoint{3.320154in}{4.276988in}}%
\pgfpathlineto{\pgfqpoint{3.322660in}{4.281665in}}%
\pgfpathlineto{\pgfqpoint{3.325167in}{4.282332in}}%
\pgfpathlineto{\pgfqpoint{3.330180in}{4.286413in}}%
\pgfpathlineto{\pgfqpoint{3.332687in}{4.289954in}}%
\pgfpathlineto{\pgfqpoint{3.335194in}{4.290145in}}%
\pgfpathlineto{\pgfqpoint{3.337700in}{4.309391in}}%
\pgfpathlineto{\pgfqpoint{3.340207in}{4.315593in}}%
\pgfpathlineto{\pgfqpoint{3.345220in}{4.318531in}}%
\pgfpathlineto{\pgfqpoint{3.347727in}{4.322266in}}%
\pgfpathlineto{\pgfqpoint{3.355247in}{4.323999in}}%
\pgfpathlineto{\pgfqpoint{3.360260in}{4.328180in}}%
\pgfpathlineto{\pgfqpoint{3.370287in}{4.330811in}}%
\pgfpathlineto{\pgfqpoint{3.375300in}{4.337942in}}%
\pgfpathlineto{\pgfqpoint{3.377807in}{4.338855in}}%
\pgfpathlineto{\pgfqpoint{3.380313in}{4.343021in}}%
\pgfpathlineto{\pgfqpoint{3.382820in}{4.353234in}}%
\pgfpathlineto{\pgfqpoint{3.385327in}{4.355553in}}%
\pgfpathlineto{\pgfqpoint{3.387833in}{4.355553in}}%
\pgfpathlineto{\pgfqpoint{3.390340in}{4.357370in}}%
\pgfpathlineto{\pgfqpoint{3.395353in}{4.358228in}}%
\pgfpathlineto{\pgfqpoint{3.400367in}{4.367435in}}%
\pgfpathlineto{\pgfqpoint{3.402873in}{4.368688in}}%
\pgfpathlineto{\pgfqpoint{3.405380in}{4.371959in}}%
\pgfpathlineto{\pgfqpoint{3.407887in}{4.379268in}}%
\pgfpathlineto{\pgfqpoint{3.410393in}{4.380739in}}%
\pgfpathlineto{\pgfqpoint{3.412900in}{4.389413in}}%
\pgfpathlineto{\pgfqpoint{3.415407in}{4.391082in}}%
\pgfpathlineto{\pgfqpoint{3.417913in}{4.391152in}}%
\pgfpathlineto{\pgfqpoint{3.422927in}{4.394546in}}%
\pgfpathlineto{\pgfqpoint{3.425433in}{4.397431in}}%
\pgfpathlineto{\pgfqpoint{3.427940in}{4.397499in}}%
\pgfpathlineto{\pgfqpoint{3.430447in}{4.398855in}}%
\pgfpathlineto{\pgfqpoint{3.440473in}{4.410824in}}%
\pgfpathlineto{\pgfqpoint{3.445486in}{4.415210in}}%
\pgfpathlineto{\pgfqpoint{3.460526in}{4.421303in}}%
\pgfpathlineto{\pgfqpoint{3.465540in}{4.422896in}}%
\pgfpathlineto{\pgfqpoint{3.470553in}{4.441151in}}%
\pgfpathlineto{\pgfqpoint{3.475566in}{4.446274in}}%
\pgfpathlineto{\pgfqpoint{3.480580in}{4.448236in}}%
\pgfpathlineto{\pgfqpoint{3.485593in}{4.450100in}}%
\pgfpathlineto{\pgfqpoint{3.490606in}{4.454259in}}%
\pgfpathlineto{\pgfqpoint{3.495620in}{4.456615in}}%
\pgfpathlineto{\pgfqpoint{3.498126in}{4.459628in}}%
\pgfpathlineto{\pgfqpoint{3.500633in}{4.460078in}}%
\pgfpathlineto{\pgfqpoint{3.503139in}{4.462775in}}%
\pgfpathlineto{\pgfqpoint{3.505646in}{4.463750in}}%
\pgfpathlineto{\pgfqpoint{3.508153in}{4.466140in}}%
\pgfpathlineto{\pgfqpoint{3.510659in}{4.466891in}}%
\pgfpathlineto{\pgfqpoint{3.515673in}{4.470333in}}%
\pgfpathlineto{\pgfqpoint{3.518179in}{4.471133in}}%
\pgfpathlineto{\pgfqpoint{3.520686in}{4.474187in}}%
\pgfpathlineto{\pgfqpoint{3.523193in}{4.480346in}}%
\pgfpathlineto{\pgfqpoint{3.525699in}{4.482931in}}%
\pgfpathlineto{\pgfqpoint{3.528206in}{4.487787in}}%
\pgfpathlineto{\pgfqpoint{3.533219in}{4.489229in}}%
\pgfpathlineto{\pgfqpoint{3.538233in}{4.490303in}}%
\pgfpathlineto{\pgfqpoint{3.553273in}{4.497219in}}%
\pgfpathlineto{\pgfqpoint{3.558286in}{4.505435in}}%
\pgfpathlineto{\pgfqpoint{3.560793in}{4.505834in}}%
\pgfpathlineto{\pgfqpoint{3.563299in}{4.509825in}}%
\pgfpathlineto{\pgfqpoint{3.565806in}{4.510108in}}%
\pgfpathlineto{\pgfqpoint{3.568313in}{4.513015in}}%
\pgfpathlineto{\pgfqpoint{3.570819in}{4.522096in}}%
\pgfpathlineto{\pgfqpoint{3.573326in}{4.525601in}}%
\pgfpathlineto{\pgfqpoint{3.575832in}{4.526406in}}%
\pgfpathlineto{\pgfqpoint{3.583352in}{4.536647in}}%
\pgfpathlineto{\pgfqpoint{3.585859in}{4.542186in}}%
\pgfpathlineto{\pgfqpoint{3.588366in}{4.543092in}}%
\pgfpathlineto{\pgfqpoint{3.590872in}{4.550298in}}%
\pgfpathlineto{\pgfqpoint{3.593379in}{4.552726in}}%
\pgfpathlineto{\pgfqpoint{3.595886in}{4.552809in}}%
\pgfpathlineto{\pgfqpoint{3.598392in}{4.555039in}}%
\pgfpathlineto{\pgfqpoint{3.600899in}{4.555066in}}%
\pgfpathlineto{\pgfqpoint{3.603406in}{4.556258in}}%
\pgfpathlineto{\pgfqpoint{3.608419in}{4.561259in}}%
\pgfpathlineto{\pgfqpoint{3.613432in}{4.563085in}}%
\pgfpathlineto{\pgfqpoint{3.615939in}{4.565839in}}%
\pgfpathlineto{\pgfqpoint{3.618446in}{4.565915in}}%
\pgfpathlineto{\pgfqpoint{3.620952in}{4.569063in}}%
\pgfpathlineto{\pgfqpoint{3.623459in}{4.574755in}}%
\pgfpathlineto{\pgfqpoint{3.625966in}{4.575094in}}%
\pgfpathlineto{\pgfqpoint{3.628472in}{4.584826in}}%
\pgfpathlineto{\pgfqpoint{3.635992in}{4.586407in}}%
\pgfpathlineto{\pgfqpoint{3.638499in}{4.588791in}}%
\pgfpathlineto{\pgfqpoint{3.641005in}{4.599095in}}%
\pgfpathlineto{\pgfqpoint{3.653539in}{4.618755in}}%
\pgfpathlineto{\pgfqpoint{3.656045in}{4.625612in}}%
\pgfpathlineto{\pgfqpoint{3.658552in}{4.625729in}}%
\pgfpathlineto{\pgfqpoint{3.661059in}{4.628807in}}%
\pgfpathlineto{\pgfqpoint{3.666072in}{4.640299in}}%
\pgfpathlineto{\pgfqpoint{3.673592in}{4.641708in}}%
\pgfpathlineto{\pgfqpoint{3.676099in}{4.645119in}}%
\pgfpathlineto{\pgfqpoint{3.681112in}{4.656370in}}%
\pgfpathlineto{\pgfqpoint{3.688632in}{4.663068in}}%
\pgfpathlineto{\pgfqpoint{3.696152in}{4.674953in}}%
\pgfpathlineto{\pgfqpoint{3.701165in}{4.676373in}}%
\pgfpathlineto{\pgfqpoint{3.711192in}{4.689721in}}%
\pgfpathlineto{\pgfqpoint{3.713698in}{4.698655in}}%
\pgfpathlineto{\pgfqpoint{3.718712in}{4.703867in}}%
\pgfpathlineto{\pgfqpoint{3.721218in}{4.709738in}}%
\pgfpathlineto{\pgfqpoint{3.723725in}{4.711828in}}%
\pgfpathlineto{\pgfqpoint{3.726232in}{4.717843in}}%
\pgfpathlineto{\pgfqpoint{3.731245in}{4.720714in}}%
\pgfpathlineto{\pgfqpoint{3.736258in}{4.728797in}}%
\pgfpathlineto{\pgfqpoint{3.738765in}{4.728980in}}%
\pgfpathlineto{\pgfqpoint{3.741272in}{4.739526in}}%
\pgfpathlineto{\pgfqpoint{3.743778in}{4.744360in}}%
\pgfpathlineto{\pgfqpoint{3.746285in}{4.753369in}}%
\pgfpathlineto{\pgfqpoint{3.748792in}{4.753694in}}%
\pgfpathlineto{\pgfqpoint{3.751298in}{4.757900in}}%
\pgfpathlineto{\pgfqpoint{3.756312in}{4.760904in}}%
\pgfpathlineto{\pgfqpoint{3.761325in}{4.772751in}}%
\pgfpathlineto{\pgfqpoint{3.763832in}{4.778889in}}%
\pgfpathlineto{\pgfqpoint{3.776365in}{4.784173in}}%
\pgfpathlineto{\pgfqpoint{3.781378in}{4.800626in}}%
\pgfpathlineto{\pgfqpoint{3.786391in}{4.804262in}}%
\pgfpathlineto{\pgfqpoint{3.788898in}{4.806299in}}%
\pgfpathlineto{\pgfqpoint{3.801431in}{4.844573in}}%
\pgfpathlineto{\pgfqpoint{3.806445in}{4.846023in}}%
\pgfpathlineto{\pgfqpoint{3.808951in}{4.864992in}}%
\pgfpathlineto{\pgfqpoint{3.811458in}{4.865088in}}%
\pgfpathlineto{\pgfqpoint{3.813965in}{4.874997in}}%
\pgfpathlineto{\pgfqpoint{3.821485in}{4.879315in}}%
\pgfpathlineto{\pgfqpoint{3.829005in}{4.898812in}}%
\pgfpathlineto{\pgfqpoint{3.831511in}{4.899666in}}%
\pgfpathlineto{\pgfqpoint{3.834018in}{4.902971in}}%
\pgfpathlineto{\pgfqpoint{3.836525in}{4.916648in}}%
\pgfpathlineto{\pgfqpoint{3.839031in}{4.922001in}}%
\pgfpathlineto{\pgfqpoint{3.841538in}{4.922499in}}%
\pgfpathlineto{\pgfqpoint{3.844044in}{4.924457in}}%
\pgfpathlineto{\pgfqpoint{3.846551in}{4.928966in}}%
\pgfpathlineto{\pgfqpoint{3.849058in}{4.936738in}}%
\pgfpathlineto{\pgfqpoint{3.854071in}{4.942539in}}%
\pgfpathlineto{\pgfqpoint{3.856578in}{4.942731in}}%
\pgfpathlineto{\pgfqpoint{3.859084in}{4.949088in}}%
\pgfpathlineto{\pgfqpoint{3.861591in}{4.964285in}}%
\pgfpathlineto{\pgfqpoint{3.864098in}{4.965467in}}%
\pgfpathlineto{\pgfqpoint{3.866604in}{4.968895in}}%
\pgfpathlineto{\pgfqpoint{3.869111in}{4.969297in}}%
\pgfpathlineto{\pgfqpoint{3.871618in}{4.971276in}}%
\pgfpathlineto{\pgfqpoint{3.874124in}{4.976531in}}%
\pgfpathlineto{\pgfqpoint{3.881644in}{4.978930in}}%
\pgfpathlineto{\pgfqpoint{3.884151in}{5.001852in}}%
\pgfpathlineto{\pgfqpoint{3.886658in}{5.006415in}}%
\pgfpathlineto{\pgfqpoint{3.889164in}{5.026559in}}%
\pgfpathlineto{\pgfqpoint{3.894178in}{5.030935in}}%
\pgfpathlineto{\pgfqpoint{3.899191in}{5.047554in}}%
\pgfpathlineto{\pgfqpoint{3.901698in}{5.049867in}}%
\pgfpathlineto{\pgfqpoint{3.904204in}{5.049889in}}%
\pgfpathlineto{\pgfqpoint{3.906711in}{5.053016in}}%
\pgfpathlineto{\pgfqpoint{3.909217in}{5.054180in}}%
\pgfpathlineto{\pgfqpoint{3.911724in}{5.060381in}}%
\pgfpathlineto{\pgfqpoint{3.916737in}{5.062701in}}%
\pgfpathlineto{\pgfqpoint{3.919244in}{5.074238in}}%
\pgfpathlineto{\pgfqpoint{3.921751in}{5.108220in}}%
\pgfpathlineto{\pgfqpoint{3.924257in}{5.113823in}}%
\pgfpathlineto{\pgfqpoint{3.926764in}{5.115045in}}%
\pgfpathlineto{\pgfqpoint{3.929271in}{5.120518in}}%
\pgfpathlineto{\pgfqpoint{3.931777in}{5.135778in}}%
\pgfpathlineto{\pgfqpoint{3.934284in}{5.136274in}}%
\pgfpathlineto{\pgfqpoint{3.936791in}{5.154993in}}%
\pgfpathlineto{\pgfqpoint{3.939297in}{5.156115in}}%
\pgfpathlineto{\pgfqpoint{3.941804in}{5.202451in}}%
\pgfpathlineto{\pgfqpoint{3.944311in}{5.202484in}}%
\pgfpathlineto{\pgfqpoint{3.946817in}{5.204968in}}%
\pgfpathlineto{\pgfqpoint{3.949324in}{5.205792in}}%
\pgfpathlineto{\pgfqpoint{3.951831in}{5.226289in}}%
\pgfpathlineto{\pgfqpoint{3.956844in}{5.243955in}}%
\pgfpathlineto{\pgfqpoint{3.959351in}{5.268717in}}%
\pgfpathlineto{\pgfqpoint{3.961857in}{5.272476in}}%
\pgfpathlineto{\pgfqpoint{3.966871in}{5.275156in}}%
\pgfpathlineto{\pgfqpoint{3.969377in}{5.280977in}}%
\pgfpathlineto{\pgfqpoint{3.971884in}{5.305275in}}%
\pgfpathlineto{\pgfqpoint{3.971884in}{5.305275in}}%
\pgfusepath{stroke}%
\end{pgfscope}%
\begin{pgfscope}%
\pgfpathrectangle{\pgfqpoint{0.708220in}{3.210823in}}{\pgfqpoint{5.013309in}{2.094453in}}%
\pgfusepath{clip}%
\pgfsetbuttcap%
\pgfsetroundjoin%
\pgfsetlinewidth{1.003750pt}%
\definecolor{currentstroke}{rgb}{0.000000,0.000000,0.000000}%
\pgfsetstrokecolor{currentstroke}%
\pgfsetdash{{3.700000pt}{1.600000pt}}{0.000000pt}%
\pgfpathmoveto{\pgfqpoint{0.708220in}{3.751804in}}%
\pgfpathlineto{\pgfqpoint{0.710727in}{3.751804in}}%
\pgfpathlineto{\pgfqpoint{0.713233in}{3.754712in}}%
\pgfpathlineto{\pgfqpoint{0.715740in}{3.763153in}}%
\pgfpathlineto{\pgfqpoint{0.720753in}{3.768558in}}%
\pgfpathlineto{\pgfqpoint{0.723260in}{3.776359in}}%
\pgfpathlineto{\pgfqpoint{0.725766in}{3.776359in}}%
\pgfpathlineto{\pgfqpoint{0.730780in}{3.783820in}}%
\pgfpathlineto{\pgfqpoint{0.733286in}{3.785033in}}%
\pgfpathlineto{\pgfqpoint{0.735793in}{3.793290in}}%
\pgfpathlineto{\pgfqpoint{0.740806in}{3.798953in}}%
\pgfpathlineto{\pgfqpoint{0.743313in}{3.807643in}}%
\pgfpathlineto{\pgfqpoint{0.748326in}{3.812859in}}%
\pgfpathlineto{\pgfqpoint{0.753340in}{3.813884in}}%
\pgfpathlineto{\pgfqpoint{0.758353in}{3.821866in}}%
\pgfpathlineto{\pgfqpoint{0.760860in}{3.833184in}}%
\pgfpathlineto{\pgfqpoint{0.768380in}{3.842077in}}%
\pgfpathlineto{\pgfqpoint{0.770886in}{3.857800in}}%
\pgfpathlineto{\pgfqpoint{0.773393in}{3.857800in}}%
\pgfpathlineto{\pgfqpoint{0.778406in}{3.861709in}}%
\pgfpathlineto{\pgfqpoint{0.780913in}{3.865530in}}%
\pgfpathlineto{\pgfqpoint{0.783419in}{3.865530in}}%
\pgfpathlineto{\pgfqpoint{0.785926in}{3.867783in}}%
\pgfpathlineto{\pgfqpoint{0.788433in}{3.874370in}}%
\pgfpathlineto{\pgfqpoint{0.790939in}{3.892735in}}%
\pgfpathlineto{\pgfqpoint{0.793446in}{3.897820in}}%
\pgfpathlineto{\pgfqpoint{0.795953in}{3.913375in}}%
\pgfpathlineto{\pgfqpoint{0.798459in}{3.915648in}}%
\pgfpathlineto{\pgfqpoint{0.800966in}{3.927110in}}%
\pgfpathlineto{\pgfqpoint{0.805979in}{3.931283in}}%
\pgfpathlineto{\pgfqpoint{0.808486in}{3.951207in}}%
\pgfpathlineto{\pgfqpoint{0.810993in}{3.961015in}}%
\pgfpathlineto{\pgfqpoint{0.816006in}{3.964871in}}%
\pgfpathlineto{\pgfqpoint{0.818513in}{3.969056in}}%
\pgfpathlineto{\pgfqpoint{0.823526in}{3.995448in}}%
\pgfpathlineto{\pgfqpoint{0.826033in}{4.027087in}}%
\pgfpathlineto{\pgfqpoint{0.831046in}{4.035711in}}%
\pgfpathlineto{\pgfqpoint{0.833553in}{4.035991in}}%
\pgfpathlineto{\pgfqpoint{0.836059in}{4.041500in}}%
\pgfpathlineto{\pgfqpoint{0.838566in}{4.042042in}}%
\pgfpathlineto{\pgfqpoint{0.841073in}{4.061198in}}%
\pgfpathlineto{\pgfqpoint{0.843579in}{4.062162in}}%
\pgfpathlineto{\pgfqpoint{0.848593in}{4.078866in}}%
\pgfpathlineto{\pgfqpoint{0.851099in}{4.083172in}}%
\pgfpathlineto{\pgfqpoint{0.858619in}{4.083597in}}%
\pgfpathlineto{\pgfqpoint{0.863632in}{4.084443in}}%
\pgfpathlineto{\pgfqpoint{0.881179in}{4.086332in}}%
\pgfpathlineto{\pgfqpoint{0.891206in}{4.087165in}}%
\pgfpathlineto{\pgfqpoint{0.898726in}{4.087994in}}%
\pgfpathlineto{\pgfqpoint{0.993978in}{4.094883in}}%
\pgfpathlineto{\pgfqpoint{1.016538in}{4.095873in}}%
\pgfpathlineto{\pgfqpoint{1.039098in}{4.096661in}}%
\pgfpathlineto{\pgfqpoint{1.069178in}{4.097641in}}%
\pgfpathlineto{\pgfqpoint{1.091738in}{4.098421in}}%
\pgfpathlineto{\pgfqpoint{1.131844in}{4.099391in}}%
\pgfpathlineto{\pgfqpoint{1.156911in}{4.100355in}}%
\pgfpathlineto{\pgfqpoint{1.204537in}{4.101315in}}%
\pgfpathlineto{\pgfqpoint{1.224591in}{4.101887in}}%
\pgfpathlineto{\pgfqpoint{1.277230in}{4.102838in}}%
\pgfpathlineto{\pgfqpoint{1.307310in}{4.104160in}}%
\pgfpathlineto{\pgfqpoint{1.369977in}{4.105098in}}%
\pgfpathlineto{\pgfqpoint{1.385017in}{4.105285in}}%
\pgfpathlineto{\pgfqpoint{1.412590in}{4.106217in}}%
\pgfpathlineto{\pgfqpoint{1.435150in}{4.106959in}}%
\pgfpathlineto{\pgfqpoint{1.487789in}{4.108067in}}%
\pgfpathlineto{\pgfqpoint{1.512856in}{4.108801in}}%
\pgfpathlineto{\pgfqpoint{1.565496in}{4.109896in}}%
\pgfpathlineto{\pgfqpoint{1.575522in}{4.110441in}}%
\pgfpathlineto{\pgfqpoint{1.605602in}{4.111526in}}%
\pgfpathlineto{\pgfqpoint{1.643202in}{4.113140in}}%
\pgfpathlineto{\pgfqpoint{1.663255in}{4.114208in}}%
\pgfpathlineto{\pgfqpoint{1.690828in}{4.115093in}}%
\pgfpathlineto{\pgfqpoint{1.740961in}{4.116149in}}%
\pgfpathlineto{\pgfqpoint{1.763521in}{4.116675in}}%
\pgfpathlineto{\pgfqpoint{1.806134in}{4.117547in}}%
\pgfpathlineto{\pgfqpoint{1.818668in}{4.118242in}}%
\pgfpathlineto{\pgfqpoint{1.853761in}{4.119279in}}%
\pgfpathlineto{\pgfqpoint{1.863788in}{4.119795in}}%
\pgfpathlineto{\pgfqpoint{1.901387in}{4.120822in}}%
\pgfpathlineto{\pgfqpoint{1.931467in}{4.121844in}}%
\pgfpathlineto{\pgfqpoint{1.959040in}{4.122859in}}%
\pgfpathlineto{\pgfqpoint{2.021707in}{4.126035in}}%
\pgfpathlineto{\pgfqpoint{2.034240in}{4.126861in}}%
\pgfpathlineto{\pgfqpoint{2.046773in}{4.127683in}}%
\pgfpathlineto{\pgfqpoint{2.071840in}{4.128665in}}%
\pgfpathlineto{\pgfqpoint{2.149546in}{4.134754in}}%
\pgfpathlineto{\pgfqpoint{2.177119in}{4.137254in}}%
\pgfpathlineto{\pgfqpoint{2.194666in}{4.138337in}}%
\pgfpathlineto{\pgfqpoint{2.252319in}{4.142600in}}%
\pgfpathlineto{\pgfqpoint{2.272372in}{4.146466in}}%
\pgfpathlineto{\pgfqpoint{2.307465in}{4.149527in}}%
\pgfpathlineto{\pgfqpoint{2.319999in}{4.150535in}}%
\pgfpathlineto{\pgfqpoint{2.322505in}{4.152249in}}%
\pgfpathlineto{\pgfqpoint{2.342559in}{4.155628in}}%
\pgfpathlineto{\pgfqpoint{2.357598in}{4.158118in}}%
\pgfpathlineto{\pgfqpoint{2.367625in}{4.159214in}}%
\pgfpathlineto{\pgfqpoint{2.372638in}{4.161923in}}%
\pgfpathlineto{\pgfqpoint{2.380158in}{4.163261in}}%
\pgfpathlineto{\pgfqpoint{2.385172in}{4.164324in}}%
\pgfpathlineto{\pgfqpoint{2.392692in}{4.165381in}}%
\pgfpathlineto{\pgfqpoint{2.397705in}{4.168254in}}%
\pgfpathlineto{\pgfqpoint{2.425278in}{4.174235in}}%
\pgfpathlineto{\pgfqpoint{2.430291in}{4.175977in}}%
\pgfpathlineto{\pgfqpoint{2.432798in}{4.176101in}}%
\pgfpathlineto{\pgfqpoint{2.435305in}{4.178314in}}%
\pgfpathlineto{\pgfqpoint{2.442825in}{4.179288in}}%
\pgfpathlineto{\pgfqpoint{2.447838in}{4.182655in}}%
\pgfpathlineto{\pgfqpoint{2.460371in}{4.186657in}}%
\pgfpathlineto{\pgfqpoint{2.467891in}{4.190567in}}%
\pgfpathlineto{\pgfqpoint{2.475411in}{4.193722in}}%
\pgfpathlineto{\pgfqpoint{2.482931in}{4.195167in}}%
\pgfpathlineto{\pgfqpoint{2.495464in}{4.196050in}}%
\pgfpathlineto{\pgfqpoint{2.502984in}{4.198565in}}%
\pgfpathlineto{\pgfqpoint{2.510504in}{4.199323in}}%
\pgfpathlineto{\pgfqpoint{2.513011in}{4.202322in}}%
\pgfpathlineto{\pgfqpoint{2.523038in}{4.204118in}}%
\pgfpathlineto{\pgfqpoint{2.533064in}{4.206518in}}%
\pgfpathlineto{\pgfqpoint{2.553117in}{4.209089in}}%
\pgfpathlineto{\pgfqpoint{2.563144in}{4.214316in}}%
\pgfpathlineto{\pgfqpoint{2.573171in}{4.216089in}}%
\pgfpathlineto{\pgfqpoint{2.578184in}{4.216872in}}%
\pgfpathlineto{\pgfqpoint{2.580691in}{4.219293in}}%
\pgfpathlineto{\pgfqpoint{2.588211in}{4.220252in}}%
\pgfpathlineto{\pgfqpoint{2.590717in}{4.220635in}}%
\pgfpathlineto{\pgfqpoint{2.595731in}{4.223849in}}%
\pgfpathlineto{\pgfqpoint{2.613277in}{4.226820in}}%
\pgfpathlineto{\pgfqpoint{2.618290in}{4.231183in}}%
\pgfpathlineto{\pgfqpoint{2.625810in}{4.234912in}}%
\pgfpathlineto{\pgfqpoint{2.628317in}{4.237612in}}%
\pgfpathlineto{\pgfqpoint{2.643357in}{4.242134in}}%
\pgfpathlineto{\pgfqpoint{2.648370in}{4.244724in}}%
\pgfpathlineto{\pgfqpoint{2.650877in}{4.244807in}}%
\pgfpathlineto{\pgfqpoint{2.655890in}{4.247847in}}%
\pgfpathlineto{\pgfqpoint{2.660904in}{4.248579in}}%
\pgfpathlineto{\pgfqpoint{2.683464in}{4.257506in}}%
\pgfpathlineto{\pgfqpoint{2.685970in}{4.257506in}}%
\pgfpathlineto{\pgfqpoint{2.688477in}{4.259419in}}%
\pgfpathlineto{\pgfqpoint{2.695997in}{4.260633in}}%
\pgfpathlineto{\pgfqpoint{2.713543in}{4.269257in}}%
\pgfpathlineto{\pgfqpoint{2.716050in}{4.269616in}}%
\pgfpathlineto{\pgfqpoint{2.721063in}{4.272462in}}%
\pgfpathlineto{\pgfqpoint{2.723570in}{4.273587in}}%
\pgfpathlineto{\pgfqpoint{2.726077in}{4.277742in}}%
\pgfpathlineto{\pgfqpoint{2.736103in}{4.281196in}}%
\pgfpathlineto{\pgfqpoint{2.741117in}{4.282332in}}%
\pgfpathlineto{\pgfqpoint{2.756156in}{4.287903in}}%
\pgfpathlineto{\pgfqpoint{2.761170in}{4.294357in}}%
\pgfpathlineto{\pgfqpoint{2.773703in}{4.296149in}}%
\pgfpathlineto{\pgfqpoint{2.776210in}{4.298287in}}%
\pgfpathlineto{\pgfqpoint{2.781223in}{4.299013in}}%
\pgfpathlineto{\pgfqpoint{2.801276in}{4.311089in}}%
\pgfpathlineto{\pgfqpoint{2.803783in}{4.315812in}}%
\pgfpathlineto{\pgfqpoint{2.806290in}{4.316086in}}%
\pgfpathlineto{\pgfqpoint{2.808796in}{4.318746in}}%
\pgfpathlineto{\pgfqpoint{2.821329in}{4.321048in}}%
\pgfpathlineto{\pgfqpoint{2.833863in}{4.326437in}}%
\pgfpathlineto{\pgfqpoint{2.851409in}{4.332659in}}%
\pgfpathlineto{\pgfqpoint{2.853916in}{4.339047in}}%
\pgfpathlineto{\pgfqpoint{2.871463in}{4.341659in}}%
\pgfpathlineto{\pgfqpoint{2.876476in}{4.354463in}}%
\pgfpathlineto{\pgfqpoint{2.883996in}{4.356117in}}%
\pgfpathlineto{\pgfqpoint{2.891516in}{4.363247in}}%
\pgfpathlineto{\pgfqpoint{2.904049in}{4.368285in}}%
\pgfpathlineto{\pgfqpoint{2.906556in}{4.368446in}}%
\pgfpathlineto{\pgfqpoint{2.914076in}{4.379117in}}%
\pgfpathlineto{\pgfqpoint{2.916582in}{4.385478in}}%
\pgfpathlineto{\pgfqpoint{2.919089in}{4.386570in}}%
\pgfpathlineto{\pgfqpoint{2.921596in}{4.394615in}}%
\pgfpathlineto{\pgfqpoint{2.924102in}{4.396987in}}%
\pgfpathlineto{\pgfqpoint{2.926609in}{4.404928in}}%
\pgfpathlineto{\pgfqpoint{2.929116in}{4.406776in}}%
\pgfpathlineto{\pgfqpoint{2.931622in}{4.407195in}}%
\pgfpathlineto{\pgfqpoint{2.934129in}{4.410635in}}%
\pgfpathlineto{\pgfqpoint{2.936636in}{4.410887in}}%
\pgfpathlineto{\pgfqpoint{2.939142in}{4.413481in}}%
\pgfpathlineto{\pgfqpoint{2.949169in}{4.416677in}}%
\pgfpathlineto{\pgfqpoint{2.956689in}{4.422896in}}%
\pgfpathlineto{\pgfqpoint{2.959195in}{4.424765in}}%
\pgfpathlineto{\pgfqpoint{2.961702in}{4.429180in}}%
\pgfpathlineto{\pgfqpoint{2.974235in}{4.434448in}}%
\pgfpathlineto{\pgfqpoint{2.976742in}{4.441836in}}%
\pgfpathlineto{\pgfqpoint{2.981755in}{4.442780in}}%
\pgfpathlineto{\pgfqpoint{2.986769in}{4.448843in}}%
\pgfpathlineto{\pgfqpoint{2.994289in}{4.450275in}}%
\pgfpathlineto{\pgfqpoint{2.996795in}{4.452933in}}%
\pgfpathlineto{\pgfqpoint{3.001809in}{4.453843in}}%
\pgfpathlineto{\pgfqpoint{3.006822in}{4.454308in}}%
\pgfpathlineto{\pgfqpoint{3.009329in}{4.459675in}}%
\pgfpathlineto{\pgfqpoint{3.011835in}{4.459770in}}%
\pgfpathlineto{\pgfqpoint{3.014342in}{4.461232in}}%
\pgfpathlineto{\pgfqpoint{3.016849in}{4.466345in}}%
\pgfpathlineto{\pgfqpoint{3.029382in}{4.470155in}}%
\pgfpathlineto{\pgfqpoint{3.036902in}{4.471266in}}%
\pgfpathlineto{\pgfqpoint{3.039408in}{4.471465in}}%
\pgfpathlineto{\pgfqpoint{3.056955in}{4.482766in}}%
\pgfpathlineto{\pgfqpoint{3.059462in}{4.483839in}}%
\pgfpathlineto{\pgfqpoint{3.061968in}{4.489010in}}%
\pgfpathlineto{\pgfqpoint{3.074502in}{4.501840in}}%
\pgfpathlineto{\pgfqpoint{3.077008in}{4.505399in}}%
\pgfpathlineto{\pgfqpoint{3.082022in}{4.516334in}}%
\pgfpathlineto{\pgfqpoint{3.084528in}{4.521072in}}%
\pgfpathlineto{\pgfqpoint{3.087035in}{4.530297in}}%
\pgfpathlineto{\pgfqpoint{3.089542in}{4.533886in}}%
\pgfpathlineto{\pgfqpoint{3.092048in}{4.534209in}}%
\pgfpathlineto{\pgfqpoint{3.094555in}{4.539662in}}%
\pgfpathlineto{\pgfqpoint{3.099568in}{4.541819in}}%
\pgfpathlineto{\pgfqpoint{3.102075in}{4.547750in}}%
\pgfpathlineto{\pgfqpoint{3.104581in}{4.550130in}}%
\pgfpathlineto{\pgfqpoint{3.112101in}{4.564403in}}%
\pgfpathlineto{\pgfqpoint{3.119621in}{4.565634in}}%
\pgfpathlineto{\pgfqpoint{3.124635in}{4.571488in}}%
\pgfpathlineto{\pgfqpoint{3.132155in}{4.575203in}}%
\pgfpathlineto{\pgfqpoint{3.137168in}{4.587637in}}%
\pgfpathlineto{\pgfqpoint{3.144688in}{4.591186in}}%
\pgfpathlineto{\pgfqpoint{3.152208in}{4.591670in}}%
\pgfpathlineto{\pgfqpoint{3.164741in}{4.612218in}}%
\pgfpathlineto{\pgfqpoint{3.167248in}{4.612530in}}%
\pgfpathlineto{\pgfqpoint{3.172261in}{4.618436in}}%
\pgfpathlineto{\pgfqpoint{3.174768in}{4.629116in}}%
\pgfpathlineto{\pgfqpoint{3.182288in}{4.637186in}}%
\pgfpathlineto{\pgfqpoint{3.189808in}{4.637733in}}%
\pgfpathlineto{\pgfqpoint{3.192314in}{4.640225in}}%
\pgfpathlineto{\pgfqpoint{3.194821in}{4.646560in}}%
\pgfpathlineto{\pgfqpoint{3.202341in}{4.649140in}}%
\pgfpathlineto{\pgfqpoint{3.204848in}{4.649188in}}%
\pgfpathlineto{\pgfqpoint{3.209861in}{4.654431in}}%
\pgfpathlineto{\pgfqpoint{3.214874in}{4.658399in}}%
\pgfpathlineto{\pgfqpoint{3.219888in}{4.659615in}}%
\pgfpathlineto{\pgfqpoint{3.222394in}{4.670496in}}%
\pgfpathlineto{\pgfqpoint{3.224901in}{4.671147in}}%
\pgfpathlineto{\pgfqpoint{3.227408in}{4.680453in}}%
\pgfpathlineto{\pgfqpoint{3.232421in}{4.681913in}}%
\pgfpathlineto{\pgfqpoint{3.234927in}{4.686567in}}%
\pgfpathlineto{\pgfqpoint{3.237434in}{4.697367in}}%
\pgfpathlineto{\pgfqpoint{3.239941in}{4.699712in}}%
\pgfpathlineto{\pgfqpoint{3.247461in}{4.711222in}}%
\pgfpathlineto{\pgfqpoint{3.254981in}{4.722542in}}%
\pgfpathlineto{\pgfqpoint{3.259994in}{4.726090in}}%
\pgfpathlineto{\pgfqpoint{3.262501in}{4.728738in}}%
\pgfpathlineto{\pgfqpoint{3.267514in}{4.743841in}}%
\pgfpathlineto{\pgfqpoint{3.275034in}{4.747767in}}%
\pgfpathlineto{\pgfqpoint{3.277541in}{4.749273in}}%
\pgfpathlineto{\pgfqpoint{3.282554in}{4.756893in}}%
\pgfpathlineto{\pgfqpoint{3.290074in}{4.761652in}}%
\pgfpathlineto{\pgfqpoint{3.292581in}{4.768302in}}%
\pgfpathlineto{\pgfqpoint{3.295087in}{4.770624in}}%
\pgfpathlineto{\pgfqpoint{3.297594in}{4.777586in}}%
\pgfpathlineto{\pgfqpoint{3.305114in}{4.781091in}}%
\pgfpathlineto{\pgfqpoint{3.307620in}{4.785391in}}%
\pgfpathlineto{\pgfqpoint{3.315140in}{4.788373in}}%
\pgfpathlineto{\pgfqpoint{3.317647in}{4.790361in}}%
\pgfpathlineto{\pgfqpoint{3.320154in}{4.797406in}}%
\pgfpathlineto{\pgfqpoint{3.325167in}{4.799897in}}%
\pgfpathlineto{\pgfqpoint{3.327674in}{4.801589in}}%
\pgfpathlineto{\pgfqpoint{3.332687in}{4.814460in}}%
\pgfpathlineto{\pgfqpoint{3.335194in}{4.818904in}}%
\pgfpathlineto{\pgfqpoint{3.340207in}{4.823156in}}%
\pgfpathlineto{\pgfqpoint{3.342714in}{4.823850in}}%
\pgfpathlineto{\pgfqpoint{3.347727in}{4.828949in}}%
\pgfpathlineto{\pgfqpoint{3.350234in}{4.846267in}}%
\pgfpathlineto{\pgfqpoint{3.355247in}{4.846563in}}%
\pgfpathlineto{\pgfqpoint{3.357754in}{4.851307in}}%
\pgfpathlineto{\pgfqpoint{3.360260in}{4.851489in}}%
\pgfpathlineto{\pgfqpoint{3.362767in}{4.862062in}}%
\pgfpathlineto{\pgfqpoint{3.365273in}{4.864370in}}%
\pgfpathlineto{\pgfqpoint{3.370287in}{4.872226in}}%
\pgfpathlineto{\pgfqpoint{3.382820in}{4.878435in}}%
\pgfpathlineto{\pgfqpoint{3.385327in}{4.882725in}}%
\pgfpathlineto{\pgfqpoint{3.390340in}{4.882878in}}%
\pgfpathlineto{\pgfqpoint{3.400367in}{4.893180in}}%
\pgfpathlineto{\pgfqpoint{3.405380in}{4.896129in}}%
\pgfpathlineto{\pgfqpoint{3.410393in}{4.902810in}}%
\pgfpathlineto{\pgfqpoint{3.412900in}{4.905170in}}%
\pgfpathlineto{\pgfqpoint{3.415407in}{4.914739in}}%
\pgfpathlineto{\pgfqpoint{3.420420in}{4.917687in}}%
\pgfpathlineto{\pgfqpoint{3.422927in}{4.921793in}}%
\pgfpathlineto{\pgfqpoint{3.427940in}{4.923511in}}%
\pgfpathlineto{\pgfqpoint{3.430447in}{4.924520in}}%
\pgfpathlineto{\pgfqpoint{3.435460in}{4.930087in}}%
\pgfpathlineto{\pgfqpoint{3.442980in}{4.933851in}}%
\pgfpathlineto{\pgfqpoint{3.445486in}{4.940954in}}%
\pgfpathlineto{\pgfqpoint{3.453006in}{4.943370in}}%
\pgfpathlineto{\pgfqpoint{3.455513in}{4.945483in}}%
\pgfpathlineto{\pgfqpoint{3.458020in}{4.954183in}}%
\pgfpathlineto{\pgfqpoint{3.463033in}{4.961892in}}%
\pgfpathlineto{\pgfqpoint{3.465540in}{4.961975in}}%
\pgfpathlineto{\pgfqpoint{3.468046in}{4.964877in}}%
\pgfpathlineto{\pgfqpoint{3.473060in}{4.976446in}}%
\pgfpathlineto{\pgfqpoint{3.475566in}{4.978020in}}%
\pgfpathlineto{\pgfqpoint{3.478073in}{4.978070in}}%
\pgfpathlineto{\pgfqpoint{3.483086in}{4.981911in}}%
\pgfpathlineto{\pgfqpoint{3.485593in}{4.982927in}}%
\pgfpathlineto{\pgfqpoint{3.488100in}{4.990901in}}%
\pgfpathlineto{\pgfqpoint{3.490606in}{4.993812in}}%
\pgfpathlineto{\pgfqpoint{3.493113in}{4.994003in}}%
\pgfpathlineto{\pgfqpoint{3.495620in}{5.001337in}}%
\pgfpathlineto{\pgfqpoint{3.498126in}{5.002055in}}%
\pgfpathlineto{\pgfqpoint{3.503139in}{5.019614in}}%
\pgfpathlineto{\pgfqpoint{3.505646in}{5.021535in}}%
\pgfpathlineto{\pgfqpoint{3.515673in}{5.046860in}}%
\pgfpathlineto{\pgfqpoint{3.523193in}{5.052670in}}%
\pgfpathlineto{\pgfqpoint{3.525699in}{5.053067in}}%
\pgfpathlineto{\pgfqpoint{3.528206in}{5.058022in}}%
\pgfpathlineto{\pgfqpoint{3.530713in}{5.075497in}}%
\pgfpathlineto{\pgfqpoint{3.538233in}{5.081079in}}%
\pgfpathlineto{\pgfqpoint{3.543246in}{5.086079in}}%
\pgfpathlineto{\pgfqpoint{3.545753in}{5.086368in}}%
\pgfpathlineto{\pgfqpoint{3.548259in}{5.092180in}}%
\pgfpathlineto{\pgfqpoint{3.550766in}{5.092219in}}%
\pgfpathlineto{\pgfqpoint{3.553273in}{5.099223in}}%
\pgfpathlineto{\pgfqpoint{3.558286in}{5.121051in}}%
\pgfpathlineto{\pgfqpoint{3.560793in}{5.126781in}}%
\pgfpathlineto{\pgfqpoint{3.565806in}{5.146614in}}%
\pgfpathlineto{\pgfqpoint{3.568313in}{5.148236in}}%
\pgfpathlineto{\pgfqpoint{3.570819in}{5.161212in}}%
\pgfpathlineto{\pgfqpoint{3.575832in}{5.161840in}}%
\pgfpathlineto{\pgfqpoint{3.578339in}{5.176159in}}%
\pgfpathlineto{\pgfqpoint{3.583352in}{5.181004in}}%
\pgfpathlineto{\pgfqpoint{3.585859in}{5.199306in}}%
\pgfpathlineto{\pgfqpoint{3.588366in}{5.208718in}}%
\pgfpathlineto{\pgfqpoint{3.590872in}{5.227338in}}%
\pgfpathlineto{\pgfqpoint{3.593379in}{5.234336in}}%
\pgfpathlineto{\pgfqpoint{3.595886in}{5.305275in}}%
\pgfpathlineto{\pgfqpoint{3.595886in}{5.305275in}}%
\pgfusepath{stroke}%
\end{pgfscope}%
\begin{pgfscope}%
\pgfpathrectangle{\pgfqpoint{0.708220in}{3.210823in}}{\pgfqpoint{5.013309in}{2.094453in}}%
\pgfusepath{clip}%
\pgfsetbuttcap%
\pgfsetroundjoin%
\pgfsetlinewidth{1.003750pt}%
\definecolor{currentstroke}{rgb}{0.000000,0.000000,0.000000}%
\pgfsetstrokecolor{currentstroke}%
\pgfsetdash{{1.000000pt}{1.650000pt}}{0.000000pt}%
\pgfpathmoveto{\pgfqpoint{0.708220in}{3.291471in}}%
\pgfpathlineto{\pgfqpoint{0.713233in}{3.291471in}}%
\pgfpathlineto{\pgfqpoint{0.715740in}{3.311682in}}%
\pgfpathlineto{\pgfqpoint{0.745820in}{3.311682in}}%
\pgfpathlineto{\pgfqpoint{0.748326in}{3.329760in}}%
\pgfpathlineto{\pgfqpoint{0.750833in}{3.329760in}}%
\pgfpathlineto{\pgfqpoint{0.753340in}{3.346115in}}%
\pgfpathlineto{\pgfqpoint{0.763366in}{3.346115in}}%
\pgfpathlineto{\pgfqpoint{0.765873in}{3.361045in}}%
\pgfpathlineto{\pgfqpoint{0.785926in}{3.361045in}}%
\pgfpathlineto{\pgfqpoint{0.788433in}{3.374780in}}%
\pgfpathlineto{\pgfqpoint{0.838566in}{3.374780in}}%
\pgfpathlineto{\pgfqpoint{0.841073in}{3.387496in}}%
\pgfpathlineto{\pgfqpoint{0.941339in}{3.387496in}}%
\pgfpathlineto{\pgfqpoint{0.943845in}{3.399335in}}%
\pgfpathlineto{\pgfqpoint{1.089231in}{3.399335in}}%
\pgfpathlineto{\pgfqpoint{1.091738in}{3.410409in}}%
\pgfpathlineto{\pgfqpoint{1.287257in}{3.410409in}}%
\pgfpathlineto{\pgfqpoint{1.289764in}{3.420812in}}%
\pgfpathlineto{\pgfqpoint{1.537922in}{3.420812in}}%
\pgfpathlineto{\pgfqpoint{1.540429in}{3.430619in}}%
\pgfpathlineto{\pgfqpoint{1.685815in}{3.430619in}}%
\pgfpathlineto{\pgfqpoint{1.688322in}{3.439897in}}%
\pgfpathlineto{\pgfqpoint{1.793601in}{3.439897in}}%
\pgfpathlineto{\pgfqpoint{1.796108in}{3.448698in}}%
\pgfpathlineto{\pgfqpoint{1.863788in}{3.448698in}}%
\pgfpathlineto{\pgfqpoint{1.866294in}{3.457070in}}%
\pgfpathlineto{\pgfqpoint{1.921441in}{3.457070in}}%
\pgfpathlineto{\pgfqpoint{1.923947in}{3.465053in}}%
\pgfpathlineto{\pgfqpoint{1.971574in}{3.465053in}}%
\pgfpathlineto{\pgfqpoint{1.974080in}{3.472680in}}%
\pgfpathlineto{\pgfqpoint{1.994134in}{3.472680in}}%
\pgfpathlineto{\pgfqpoint{1.996640in}{3.479983in}}%
\pgfpathlineto{\pgfqpoint{2.026720in}{3.479983in}}%
\pgfpathlineto{\pgfqpoint{2.029227in}{3.486988in}}%
\pgfpathlineto{\pgfqpoint{2.071840in}{3.486988in}}%
\pgfpathlineto{\pgfqpoint{2.074346in}{3.493718in}}%
\pgfpathlineto{\pgfqpoint{2.099413in}{3.493718in}}%
\pgfpathlineto{\pgfqpoint{2.101920in}{3.500194in}}%
\pgfpathlineto{\pgfqpoint{2.132000in}{3.500194in}}%
\pgfpathlineto{\pgfqpoint{2.134506in}{3.506434in}}%
\pgfpathlineto{\pgfqpoint{2.149546in}{3.506434in}}%
\pgfpathlineto{\pgfqpoint{2.152053in}{3.512455in}}%
\pgfpathlineto{\pgfqpoint{2.172106in}{3.512455in}}%
\pgfpathlineto{\pgfqpoint{2.174613in}{3.518272in}}%
\pgfpathlineto{\pgfqpoint{2.197173in}{3.518272in}}%
\pgfpathlineto{\pgfqpoint{2.199679in}{3.523899in}}%
\pgfpathlineto{\pgfqpoint{2.219732in}{3.523899in}}%
\pgfpathlineto{\pgfqpoint{2.222239in}{3.529347in}}%
\pgfpathlineto{\pgfqpoint{2.247306in}{3.529347in}}%
\pgfpathlineto{\pgfqpoint{2.249812in}{3.534627in}}%
\pgfpathlineto{\pgfqpoint{2.254826in}{3.534627in}}%
\pgfpathlineto{\pgfqpoint{2.257332in}{3.539749in}}%
\pgfpathlineto{\pgfqpoint{2.269866in}{3.539749in}}%
\pgfpathlineto{\pgfqpoint{2.272372in}{3.544723in}}%
\pgfpathlineto{\pgfqpoint{2.297439in}{3.544723in}}%
\pgfpathlineto{\pgfqpoint{2.299945in}{3.549557in}}%
\pgfpathlineto{\pgfqpoint{2.319999in}{3.549557in}}%
\pgfpathlineto{\pgfqpoint{2.322505in}{3.554259in}}%
\pgfpathlineto{\pgfqpoint{2.347572in}{3.554259in}}%
\pgfpathlineto{\pgfqpoint{2.350078in}{3.558835in}}%
\pgfpathlineto{\pgfqpoint{2.387678in}{3.558835in}}%
\pgfpathlineto{\pgfqpoint{2.390185in}{3.563292in}}%
\pgfpathlineto{\pgfqpoint{2.402718in}{3.563292in}}%
\pgfpathlineto{\pgfqpoint{2.405225in}{3.567636in}}%
\pgfpathlineto{\pgfqpoint{2.410238in}{3.567636in}}%
\pgfpathlineto{\pgfqpoint{2.412745in}{3.571873in}}%
\pgfpathlineto{\pgfqpoint{2.427785in}{3.571873in}}%
\pgfpathlineto{\pgfqpoint{2.430291in}{3.576008in}}%
\pgfpathlineto{\pgfqpoint{2.442825in}{3.576008in}}%
\pgfpathlineto{\pgfqpoint{2.445331in}{3.580046in}}%
\pgfpathlineto{\pgfqpoint{2.457865in}{3.580046in}}%
\pgfpathlineto{\pgfqpoint{2.460371in}{3.583991in}}%
\pgfpathlineto{\pgfqpoint{2.462878in}{3.583991in}}%
\pgfpathlineto{\pgfqpoint{2.465385in}{3.587847in}}%
\pgfpathlineto{\pgfqpoint{2.470398in}{3.587847in}}%
\pgfpathlineto{\pgfqpoint{2.472905in}{3.591618in}}%
\pgfpathlineto{\pgfqpoint{2.485438in}{3.591618in}}%
\pgfpathlineto{\pgfqpoint{2.487944in}{3.595308in}}%
\pgfpathlineto{\pgfqpoint{2.500478in}{3.595308in}}%
\pgfpathlineto{\pgfqpoint{2.502984in}{3.598921in}}%
\pgfpathlineto{\pgfqpoint{2.523038in}{3.598921in}}%
\pgfpathlineto{\pgfqpoint{2.525544in}{3.602459in}}%
\pgfpathlineto{\pgfqpoint{2.535571in}{3.602459in}}%
\pgfpathlineto{\pgfqpoint{2.538078in}{3.605926in}}%
\pgfpathlineto{\pgfqpoint{2.548104in}{3.605926in}}%
\pgfpathlineto{\pgfqpoint{2.553117in}{3.612656in}}%
\pgfpathlineto{\pgfqpoint{2.558131in}{3.612656in}}%
\pgfpathlineto{\pgfqpoint{2.560637in}{3.615924in}}%
\pgfpathlineto{\pgfqpoint{2.563144in}{3.615924in}}%
\pgfpathlineto{\pgfqpoint{2.565651in}{3.619131in}}%
\pgfpathlineto{\pgfqpoint{2.573171in}{3.619131in}}%
\pgfpathlineto{\pgfqpoint{2.578184in}{3.625372in}}%
\pgfpathlineto{\pgfqpoint{2.585704in}{3.625372in}}%
\pgfpathlineto{\pgfqpoint{2.588211in}{3.628409in}}%
\pgfpathlineto{\pgfqpoint{2.590717in}{3.628409in}}%
\pgfpathlineto{\pgfqpoint{2.593224in}{3.631393in}}%
\pgfpathlineto{\pgfqpoint{2.595731in}{3.631393in}}%
\pgfpathlineto{\pgfqpoint{2.598237in}{3.634326in}}%
\pgfpathlineto{\pgfqpoint{2.605757in}{3.634326in}}%
\pgfpathlineto{\pgfqpoint{2.608264in}{3.637210in}}%
\pgfpathlineto{\pgfqpoint{2.610771in}{3.637210in}}%
\pgfpathlineto{\pgfqpoint{2.613277in}{3.640047in}}%
\pgfpathlineto{\pgfqpoint{2.618290in}{3.640047in}}%
\pgfpathlineto{\pgfqpoint{2.620797in}{3.645582in}}%
\pgfpathlineto{\pgfqpoint{2.623304in}{3.648285in}}%
\pgfpathlineto{\pgfqpoint{2.625810in}{3.648285in}}%
\pgfpathlineto{\pgfqpoint{2.628317in}{3.650945in}}%
\pgfpathlineto{\pgfqpoint{2.630824in}{3.650945in}}%
\pgfpathlineto{\pgfqpoint{2.635837in}{3.656145in}}%
\pgfpathlineto{\pgfqpoint{2.640850in}{3.656145in}}%
\pgfpathlineto{\pgfqpoint{2.643357in}{3.658687in}}%
\pgfpathlineto{\pgfqpoint{2.645864in}{3.658687in}}%
\pgfpathlineto{\pgfqpoint{2.648370in}{3.661192in}}%
\pgfpathlineto{\pgfqpoint{2.650877in}{3.661192in}}%
\pgfpathlineto{\pgfqpoint{2.653384in}{3.663661in}}%
\pgfpathlineto{\pgfqpoint{2.658397in}{3.663661in}}%
\pgfpathlineto{\pgfqpoint{2.663410in}{3.668495in}}%
\pgfpathlineto{\pgfqpoint{2.670930in}{3.668495in}}%
\pgfpathlineto{\pgfqpoint{2.675944in}{3.673196in}}%
\pgfpathlineto{\pgfqpoint{2.685970in}{3.673196in}}%
\pgfpathlineto{\pgfqpoint{2.695997in}{3.682230in}}%
\pgfpathlineto{\pgfqpoint{2.706023in}{3.682230in}}%
\pgfpathlineto{\pgfqpoint{2.708530in}{3.686574in}}%
\pgfpathlineto{\pgfqpoint{2.711037in}{3.686574in}}%
\pgfpathlineto{\pgfqpoint{2.713543in}{3.688706in}}%
\pgfpathlineto{\pgfqpoint{2.718557in}{3.688706in}}%
\pgfpathlineto{\pgfqpoint{2.721063in}{3.694946in}}%
\pgfpathlineto{\pgfqpoint{2.726077in}{3.694946in}}%
\pgfpathlineto{\pgfqpoint{2.728583in}{3.696977in}}%
\pgfpathlineto{\pgfqpoint{2.731090in}{3.696977in}}%
\pgfpathlineto{\pgfqpoint{2.733597in}{3.698984in}}%
\pgfpathlineto{\pgfqpoint{2.736103in}{3.698984in}}%
\pgfpathlineto{\pgfqpoint{2.738610in}{3.700967in}}%
\pgfpathlineto{\pgfqpoint{2.741117in}{3.704867in}}%
\pgfpathlineto{\pgfqpoint{2.746130in}{3.704867in}}%
\pgfpathlineto{\pgfqpoint{2.748637in}{3.706785in}}%
\pgfpathlineto{\pgfqpoint{2.751143in}{3.706785in}}%
\pgfpathlineto{\pgfqpoint{2.753650in}{3.708681in}}%
\pgfpathlineto{\pgfqpoint{2.758663in}{3.708681in}}%
\pgfpathlineto{\pgfqpoint{2.763676in}{3.712411in}}%
\pgfpathlineto{\pgfqpoint{2.768690in}{3.719637in}}%
\pgfpathlineto{\pgfqpoint{2.776210in}{3.724863in}}%
\pgfpathlineto{\pgfqpoint{2.778716in}{3.724863in}}%
\pgfpathlineto{\pgfqpoint{2.781223in}{3.726571in}}%
\pgfpathlineto{\pgfqpoint{2.783730in}{3.726571in}}%
\pgfpathlineto{\pgfqpoint{2.786236in}{3.728261in}}%
\pgfpathlineto{\pgfqpoint{2.788743in}{3.728261in}}%
\pgfpathlineto{\pgfqpoint{2.791250in}{3.731593in}}%
\pgfpathlineto{\pgfqpoint{2.796263in}{3.731593in}}%
\pgfpathlineto{\pgfqpoint{2.798770in}{3.733235in}}%
\pgfpathlineto{\pgfqpoint{2.803783in}{3.733235in}}%
\pgfpathlineto{\pgfqpoint{2.808796in}{3.736473in}}%
\pgfpathlineto{\pgfqpoint{2.813810in}{3.736473in}}%
\pgfpathlineto{\pgfqpoint{2.818823in}{3.739651in}}%
\pgfpathlineto{\pgfqpoint{2.823836in}{3.739651in}}%
\pgfpathlineto{\pgfqpoint{2.826343in}{3.741218in}}%
\pgfpathlineto{\pgfqpoint{2.833863in}{3.753264in}}%
\pgfpathlineto{\pgfqpoint{2.836369in}{3.753264in}}%
\pgfpathlineto{\pgfqpoint{2.838876in}{3.756148in}}%
\pgfpathlineto{\pgfqpoint{2.841383in}{3.756148in}}%
\pgfpathlineto{\pgfqpoint{2.846396in}{3.760385in}}%
\pgfpathlineto{\pgfqpoint{2.848903in}{3.761775in}}%
\pgfpathlineto{\pgfqpoint{2.851409in}{3.764520in}}%
\pgfpathlineto{\pgfqpoint{2.853916in}{3.764520in}}%
\pgfpathlineto{\pgfqpoint{2.856423in}{3.767222in}}%
\pgfpathlineto{\pgfqpoint{2.858929in}{3.773798in}}%
\pgfpathlineto{\pgfqpoint{2.871463in}{3.781369in}}%
\pgfpathlineto{\pgfqpoint{2.883996in}{3.794438in}}%
\pgfpathlineto{\pgfqpoint{2.886503in}{3.794438in}}%
\pgfpathlineto{\pgfqpoint{2.894022in}{3.800064in}}%
\pgfpathlineto{\pgfqpoint{2.896529in}{3.800064in}}%
\pgfpathlineto{\pgfqpoint{2.899036in}{3.804436in}}%
\pgfpathlineto{\pgfqpoint{2.901542in}{3.817921in}}%
\pgfpathlineto{\pgfqpoint{2.909062in}{3.821866in}}%
\pgfpathlineto{\pgfqpoint{2.914076in}{3.826673in}}%
\pgfpathlineto{\pgfqpoint{2.916582in}{3.832269in}}%
\pgfpathlineto{\pgfqpoint{2.921596in}{3.833184in}}%
\pgfpathlineto{\pgfqpoint{2.924102in}{3.836797in}}%
\pgfpathlineto{\pgfqpoint{2.926609in}{3.848873in}}%
\pgfpathlineto{\pgfqpoint{2.931622in}{3.851354in}}%
\pgfpathlineto{\pgfqpoint{2.939142in}{3.852988in}}%
\pgfpathlineto{\pgfqpoint{2.944156in}{3.853800in}}%
\pgfpathlineto{\pgfqpoint{2.951676in}{3.858589in}}%
\pgfpathlineto{\pgfqpoint{2.954182in}{3.858589in}}%
\pgfpathlineto{\pgfqpoint{2.956689in}{3.862480in}}%
\pgfpathlineto{\pgfqpoint{2.959195in}{3.862480in}}%
\pgfpathlineto{\pgfqpoint{2.961702in}{3.865530in}}%
\pgfpathlineto{\pgfqpoint{2.964209in}{3.866285in}}%
\pgfpathlineto{\pgfqpoint{2.971729in}{3.875799in}}%
\pgfpathlineto{\pgfqpoint{2.974235in}{3.876510in}}%
\pgfpathlineto{\pgfqpoint{2.976742in}{3.880712in}}%
\pgfpathlineto{\pgfqpoint{2.979249in}{3.881403in}}%
\pgfpathlineto{\pgfqpoint{2.981755in}{3.885489in}}%
\pgfpathlineto{\pgfqpoint{2.984262in}{3.886160in}}%
\pgfpathlineto{\pgfqpoint{2.986769in}{3.889479in}}%
\pgfpathlineto{\pgfqpoint{2.989275in}{3.890789in}}%
\pgfpathlineto{\pgfqpoint{2.991782in}{3.890789in}}%
\pgfpathlineto{\pgfqpoint{2.994289in}{3.899688in}}%
\pgfpathlineto{\pgfqpoint{2.996795in}{3.903971in}}%
\pgfpathlineto{\pgfqpoint{2.999302in}{3.911651in}}%
\pgfpathlineto{\pgfqpoint{3.001809in}{3.912228in}}%
\pgfpathlineto{\pgfqpoint{3.006822in}{3.919002in}}%
\pgfpathlineto{\pgfqpoint{3.011835in}{3.920105in}}%
\pgfpathlineto{\pgfqpoint{3.016849in}{3.921202in}}%
\pgfpathlineto{\pgfqpoint{3.019355in}{3.923374in}}%
\pgfpathlineto{\pgfqpoint{3.021862in}{3.923374in}}%
\pgfpathlineto{\pgfqpoint{3.026875in}{3.927110in}}%
\pgfpathlineto{\pgfqpoint{3.029382in}{3.937357in}}%
\pgfpathlineto{\pgfqpoint{3.031888in}{3.937854in}}%
\pgfpathlineto{\pgfqpoint{3.034395in}{3.943704in}}%
\pgfpathlineto{\pgfqpoint{3.036902in}{3.944660in}}%
\pgfpathlineto{\pgfqpoint{3.039408in}{3.952122in}}%
\pgfpathlineto{\pgfqpoint{3.041915in}{3.953032in}}%
\pgfpathlineto{\pgfqpoint{3.044422in}{3.955734in}}%
\pgfpathlineto{\pgfqpoint{3.049435in}{3.956181in}}%
\pgfpathlineto{\pgfqpoint{3.051942in}{3.961447in}}%
\pgfpathlineto{\pgfqpoint{3.054448in}{3.961447in}}%
\pgfpathlineto{\pgfqpoint{3.056955in}{3.964871in}}%
\pgfpathlineto{\pgfqpoint{3.061968in}{3.966137in}}%
\pgfpathlineto{\pgfqpoint{3.064475in}{3.969469in}}%
\pgfpathlineto{\pgfqpoint{3.066982in}{3.970292in}}%
\pgfpathlineto{\pgfqpoint{3.069488in}{3.976737in}}%
\pgfpathlineto{\pgfqpoint{3.071995in}{3.978312in}}%
\pgfpathlineto{\pgfqpoint{3.074502in}{3.982185in}}%
\pgfpathlineto{\pgfqpoint{3.082022in}{3.985598in}}%
\pgfpathlineto{\pgfqpoint{3.084528in}{3.988576in}}%
\pgfpathlineto{\pgfqpoint{3.087035in}{3.988576in}}%
\pgfpathlineto{\pgfqpoint{3.097061in}{3.999650in}}%
\pgfpathlineto{\pgfqpoint{3.102075in}{4.001371in}}%
\pgfpathlineto{\pgfqpoint{3.104581in}{4.009400in}}%
\pgfpathlineto{\pgfqpoint{3.112101in}{4.012638in}}%
\pgfpathlineto{\pgfqpoint{3.114608in}{4.016130in}}%
\pgfpathlineto{\pgfqpoint{3.117115in}{4.017695in}}%
\pgfpathlineto{\pgfqpoint{3.119621in}{4.017695in}}%
\pgfpathlineto{\pgfqpoint{3.122128in}{4.027381in}}%
\pgfpathlineto{\pgfqpoint{3.124635in}{4.032599in}}%
\pgfpathlineto{\pgfqpoint{3.127141in}{4.033737in}}%
\pgfpathlineto{\pgfqpoint{3.129648in}{4.036829in}}%
\pgfpathlineto{\pgfqpoint{3.132155in}{4.042042in}}%
\pgfpathlineto{\pgfqpoint{3.134661in}{4.042581in}}%
\pgfpathlineto{\pgfqpoint{3.137168in}{4.045519in}}%
\pgfpathlineto{\pgfqpoint{3.139675in}{4.046312in}}%
\pgfpathlineto{\pgfqpoint{3.142181in}{4.051759in}}%
\pgfpathlineto{\pgfqpoint{3.144688in}{4.052015in}}%
\pgfpathlineto{\pgfqpoint{3.147195in}{4.054797in}}%
\pgfpathlineto{\pgfqpoint{3.149701in}{4.061681in}}%
\pgfpathlineto{\pgfqpoint{3.157221in}{4.064074in}}%
\pgfpathlineto{\pgfqpoint{3.164741in}{4.073101in}}%
\pgfpathlineto{\pgfqpoint{3.172261in}{4.094286in}}%
\pgfpathlineto{\pgfqpoint{3.177274in}{4.097641in}}%
\pgfpathlineto{\pgfqpoint{3.179781in}{4.109166in}}%
\pgfpathlineto{\pgfqpoint{3.182288in}{4.109714in}}%
\pgfpathlineto{\pgfqpoint{3.184794in}{4.114208in}}%
\pgfpathlineto{\pgfqpoint{3.192314in}{4.117721in}}%
\pgfpathlineto{\pgfqpoint{3.194821in}{4.123028in}}%
\pgfpathlineto{\pgfqpoint{3.197328in}{4.131896in}}%
\pgfpathlineto{\pgfqpoint{3.202341in}{4.136477in}}%
\pgfpathlineto{\pgfqpoint{3.204848in}{4.136944in}}%
\pgfpathlineto{\pgfqpoint{3.207354in}{4.150247in}}%
\pgfpathlineto{\pgfqpoint{3.212368in}{4.158530in}}%
\pgfpathlineto{\pgfqpoint{3.214874in}{4.168125in}}%
\pgfpathlineto{\pgfqpoint{3.217381in}{4.170570in}}%
\pgfpathlineto{\pgfqpoint{3.224901in}{4.172728in}}%
\pgfpathlineto{\pgfqpoint{3.227408in}{4.173609in}}%
\pgfpathlineto{\pgfqpoint{3.237434in}{4.190339in}}%
\pgfpathlineto{\pgfqpoint{3.239941in}{4.192152in}}%
\pgfpathlineto{\pgfqpoint{3.242447in}{4.196710in}}%
\pgfpathlineto{\pgfqpoint{3.244954in}{4.198782in}}%
\pgfpathlineto{\pgfqpoint{3.247461in}{4.198890in}}%
\pgfpathlineto{\pgfqpoint{3.249967in}{4.204013in}}%
\pgfpathlineto{\pgfqpoint{3.254981in}{4.221965in}}%
\pgfpathlineto{\pgfqpoint{3.259994in}{4.232613in}}%
\pgfpathlineto{\pgfqpoint{3.267514in}{4.241121in}}%
\pgfpathlineto{\pgfqpoint{3.270021in}{4.246293in}}%
\pgfpathlineto{\pgfqpoint{3.277541in}{4.255024in}}%
\pgfpathlineto{\pgfqpoint{3.280047in}{4.263927in}}%
\pgfpathlineto{\pgfqpoint{3.285061in}{4.265183in}}%
\pgfpathlineto{\pgfqpoint{3.295087in}{4.274565in}}%
\pgfpathlineto{\pgfqpoint{3.300100in}{4.286933in}}%
\pgfpathlineto{\pgfqpoint{3.302607in}{4.287386in}}%
\pgfpathlineto{\pgfqpoint{3.305114in}{4.289827in}}%
\pgfpathlineto{\pgfqpoint{3.307620in}{4.297739in}}%
\pgfpathlineto{\pgfqpoint{3.310127in}{4.300398in}}%
\pgfpathlineto{\pgfqpoint{3.315140in}{4.308764in}}%
\pgfpathlineto{\pgfqpoint{3.317647in}{4.310186in}}%
\pgfpathlineto{\pgfqpoint{3.320154in}{4.314823in}}%
\pgfpathlineto{\pgfqpoint{3.322660in}{4.316741in}}%
\pgfpathlineto{\pgfqpoint{3.325167in}{4.321207in}}%
\pgfpathlineto{\pgfqpoint{3.330180in}{4.322424in}}%
\pgfpathlineto{\pgfqpoint{3.332687in}{4.323685in}}%
\pgfpathlineto{\pgfqpoint{3.337700in}{4.338951in}}%
\pgfpathlineto{\pgfqpoint{3.340207in}{4.340998in}}%
\pgfpathlineto{\pgfqpoint{3.342714in}{4.344881in}}%
\pgfpathlineto{\pgfqpoint{3.352740in}{4.369371in}}%
\pgfpathlineto{\pgfqpoint{3.372793in}{4.375517in}}%
\pgfpathlineto{\pgfqpoint{3.377807in}{4.380212in}}%
\pgfpathlineto{\pgfqpoint{3.380313in}{4.389733in}}%
\pgfpathlineto{\pgfqpoint{3.382820in}{4.392069in}}%
\pgfpathlineto{\pgfqpoint{3.387833in}{4.393154in}}%
\pgfpathlineto{\pgfqpoint{3.395353in}{4.412455in}}%
\pgfpathlineto{\pgfqpoint{3.400367in}{4.418795in}}%
\pgfpathlineto{\pgfqpoint{3.407887in}{4.420501in}}%
\pgfpathlineto{\pgfqpoint{3.410393in}{4.431793in}}%
\pgfpathlineto{\pgfqpoint{3.412900in}{4.432238in}}%
\pgfpathlineto{\pgfqpoint{3.422927in}{4.440992in}}%
\pgfpathlineto{\pgfqpoint{3.425433in}{4.447474in}}%
\pgfpathlineto{\pgfqpoint{3.427940in}{4.449824in}}%
\pgfpathlineto{\pgfqpoint{3.430447in}{4.455841in}}%
\pgfpathlineto{\pgfqpoint{3.432953in}{4.455987in}}%
\pgfpathlineto{\pgfqpoint{3.435460in}{4.461138in}}%
\pgfpathlineto{\pgfqpoint{3.442980in}{4.463634in}}%
\pgfpathlineto{\pgfqpoint{3.450500in}{4.480724in}}%
\pgfpathlineto{\pgfqpoint{3.453006in}{4.496398in}}%
\pgfpathlineto{\pgfqpoint{3.455513in}{4.501023in}}%
\pgfpathlineto{\pgfqpoint{3.458020in}{4.501209in}}%
\pgfpathlineto{\pgfqpoint{3.463033in}{4.507007in}}%
\pgfpathlineto{\pgfqpoint{3.468046in}{4.520707in}}%
\pgfpathlineto{\pgfqpoint{3.470553in}{4.521304in}}%
\pgfpathlineto{\pgfqpoint{3.473060in}{4.530830in}}%
\pgfpathlineto{\pgfqpoint{3.475566in}{4.535645in}}%
\pgfpathlineto{\pgfqpoint{3.478073in}{4.535995in}}%
\pgfpathlineto{\pgfqpoint{3.480580in}{4.538438in}}%
\pgfpathlineto{\pgfqpoint{3.485593in}{4.561482in}}%
\pgfpathlineto{\pgfqpoint{3.488100in}{4.562812in}}%
\pgfpathlineto{\pgfqpoint{3.490606in}{4.562838in}}%
\pgfpathlineto{\pgfqpoint{3.493113in}{4.568020in}}%
\pgfpathlineto{\pgfqpoint{3.495620in}{4.568637in}}%
\pgfpathlineto{\pgfqpoint{3.498126in}{4.585215in}}%
\pgfpathlineto{\pgfqpoint{3.500633in}{4.588780in}}%
\pgfpathlineto{\pgfqpoint{3.503139in}{4.597245in}}%
\pgfpathlineto{\pgfqpoint{3.510659in}{4.600998in}}%
\pgfpathlineto{\pgfqpoint{3.515673in}{4.606534in}}%
\pgfpathlineto{\pgfqpoint{3.518179in}{4.613035in}}%
\pgfpathlineto{\pgfqpoint{3.520686in}{4.613808in}}%
\pgfpathlineto{\pgfqpoint{3.523193in}{4.618961in}}%
\pgfpathlineto{\pgfqpoint{3.525699in}{4.619569in}}%
\pgfpathlineto{\pgfqpoint{3.530713in}{4.623864in}}%
\pgfpathlineto{\pgfqpoint{3.533219in}{4.627119in}}%
\pgfpathlineto{\pgfqpoint{3.535726in}{4.627867in}}%
\pgfpathlineto{\pgfqpoint{3.538233in}{4.634495in}}%
\pgfpathlineto{\pgfqpoint{3.543246in}{4.635255in}}%
\pgfpathlineto{\pgfqpoint{3.548259in}{4.647831in}}%
\pgfpathlineto{\pgfqpoint{3.553273in}{4.649274in}}%
\pgfpathlineto{\pgfqpoint{3.555779in}{4.656754in}}%
\pgfpathlineto{\pgfqpoint{3.558286in}{4.671278in}}%
\pgfpathlineto{\pgfqpoint{3.560793in}{4.678908in}}%
\pgfpathlineto{\pgfqpoint{3.563299in}{4.680806in}}%
\pgfpathlineto{\pgfqpoint{3.565806in}{4.685102in}}%
\pgfpathlineto{\pgfqpoint{3.580846in}{4.692982in}}%
\pgfpathlineto{\pgfqpoint{3.585859in}{4.700540in}}%
\pgfpathlineto{\pgfqpoint{3.588366in}{4.700715in}}%
\pgfpathlineto{\pgfqpoint{3.590872in}{4.708764in}}%
\pgfpathlineto{\pgfqpoint{3.593379in}{4.709390in}}%
\pgfpathlineto{\pgfqpoint{3.595886in}{4.711970in}}%
\pgfpathlineto{\pgfqpoint{3.598392in}{4.716289in}}%
\pgfpathlineto{\pgfqpoint{3.603406in}{4.717300in}}%
\pgfpathlineto{\pgfqpoint{3.605912in}{4.720812in}}%
\pgfpathlineto{\pgfqpoint{3.608419in}{4.722039in}}%
\pgfpathlineto{\pgfqpoint{3.613432in}{4.729900in}}%
\pgfpathlineto{\pgfqpoint{3.620952in}{4.735347in}}%
\pgfpathlineto{\pgfqpoint{3.625966in}{4.737665in}}%
\pgfpathlineto{\pgfqpoint{3.628472in}{4.742674in}}%
\pgfpathlineto{\pgfqpoint{3.641005in}{4.748535in}}%
\pgfpathlineto{\pgfqpoint{3.643512in}{4.752240in}}%
\pgfpathlineto{\pgfqpoint{3.646019in}{4.752580in}}%
\pgfpathlineto{\pgfqpoint{3.648525in}{4.755123in}}%
\pgfpathlineto{\pgfqpoint{3.651032in}{4.767901in}}%
\pgfpathlineto{\pgfqpoint{3.653539in}{4.771265in}}%
\pgfpathlineto{\pgfqpoint{3.658552in}{4.775048in}}%
\pgfpathlineto{\pgfqpoint{3.668579in}{4.784715in}}%
\pgfpathlineto{\pgfqpoint{3.676099in}{4.787708in}}%
\pgfpathlineto{\pgfqpoint{3.678605in}{4.792751in}}%
\pgfpathlineto{\pgfqpoint{3.681112in}{4.793491in}}%
\pgfpathlineto{\pgfqpoint{3.686125in}{4.802815in}}%
\pgfpathlineto{\pgfqpoint{3.688632in}{4.803334in}}%
\pgfpathlineto{\pgfqpoint{3.691139in}{4.808755in}}%
\pgfpathlineto{\pgfqpoint{3.693645in}{4.811201in}}%
\pgfpathlineto{\pgfqpoint{3.696152in}{4.815672in}}%
\pgfpathlineto{\pgfqpoint{3.703672in}{4.820890in}}%
\pgfpathlineto{\pgfqpoint{3.706178in}{4.827230in}}%
\pgfpathlineto{\pgfqpoint{3.711192in}{4.829987in}}%
\pgfpathlineto{\pgfqpoint{3.713698in}{4.830277in}}%
\pgfpathlineto{\pgfqpoint{3.716205in}{4.835388in}}%
\pgfpathlineto{\pgfqpoint{3.718712in}{4.836399in}}%
\pgfpathlineto{\pgfqpoint{3.728738in}{4.860383in}}%
\pgfpathlineto{\pgfqpoint{3.731245in}{4.874091in}}%
\pgfpathlineto{\pgfqpoint{3.736258in}{4.875994in}}%
\pgfpathlineto{\pgfqpoint{3.738765in}{4.889037in}}%
\pgfpathlineto{\pgfqpoint{3.743778in}{4.890827in}}%
\pgfpathlineto{\pgfqpoint{3.746285in}{4.893381in}}%
\pgfpathlineto{\pgfqpoint{3.748792in}{4.894336in}}%
\pgfpathlineto{\pgfqpoint{3.751298in}{4.898401in}}%
\pgfpathlineto{\pgfqpoint{3.753805in}{4.898590in}}%
\pgfpathlineto{\pgfqpoint{3.756312in}{4.902983in}}%
\pgfpathlineto{\pgfqpoint{3.758818in}{4.903175in}}%
\pgfpathlineto{\pgfqpoint{3.761325in}{4.920854in}}%
\pgfpathlineto{\pgfqpoint{3.763832in}{4.928563in}}%
\pgfpathlineto{\pgfqpoint{3.768845in}{4.929664in}}%
\pgfpathlineto{\pgfqpoint{3.771352in}{4.939252in}}%
\pgfpathlineto{\pgfqpoint{3.776365in}{4.942868in}}%
\pgfpathlineto{\pgfqpoint{3.783885in}{4.961055in}}%
\pgfpathlineto{\pgfqpoint{3.788898in}{4.962385in}}%
\pgfpathlineto{\pgfqpoint{3.791405in}{4.973354in}}%
\pgfpathlineto{\pgfqpoint{3.793911in}{4.977241in}}%
\pgfpathlineto{\pgfqpoint{3.796418in}{4.984375in}}%
\pgfpathlineto{\pgfqpoint{3.801431in}{5.042423in}}%
\pgfpathlineto{\pgfqpoint{3.803938in}{5.045290in}}%
\pgfpathlineto{\pgfqpoint{3.806445in}{5.058235in}}%
\pgfpathlineto{\pgfqpoint{3.808951in}{5.083812in}}%
\pgfpathlineto{\pgfqpoint{3.813965in}{5.087080in}}%
\pgfpathlineto{\pgfqpoint{3.818978in}{5.108326in}}%
\pgfpathlineto{\pgfqpoint{3.821485in}{5.109028in}}%
\pgfpathlineto{\pgfqpoint{3.823991in}{5.112486in}}%
\pgfpathlineto{\pgfqpoint{3.826498in}{5.113010in}}%
\pgfpathlineto{\pgfqpoint{3.841538in}{5.141870in}}%
\pgfpathlineto{\pgfqpoint{3.844044in}{5.142035in}}%
\pgfpathlineto{\pgfqpoint{3.849058in}{5.147024in}}%
\pgfpathlineto{\pgfqpoint{3.851564in}{5.148448in}}%
\pgfpathlineto{\pgfqpoint{3.856578in}{5.163343in}}%
\pgfpathlineto{\pgfqpoint{3.861591in}{5.166227in}}%
\pgfpathlineto{\pgfqpoint{3.866604in}{5.185242in}}%
\pgfpathlineto{\pgfqpoint{3.871618in}{5.229249in}}%
\pgfpathlineto{\pgfqpoint{3.876631in}{5.235643in}}%
\pgfpathlineto{\pgfqpoint{3.879138in}{5.252963in}}%
\pgfpathlineto{\pgfqpoint{3.881644in}{5.254079in}}%
\pgfpathlineto{\pgfqpoint{3.884151in}{5.263320in}}%
\pgfpathlineto{\pgfqpoint{3.886658in}{5.265031in}}%
\pgfpathlineto{\pgfqpoint{3.889164in}{5.274165in}}%
\pgfpathlineto{\pgfqpoint{3.891671in}{5.274215in}}%
\pgfpathlineto{\pgfqpoint{3.894178in}{5.276239in}}%
\pgfpathlineto{\pgfqpoint{3.896684in}{5.276528in}}%
\pgfpathlineto{\pgfqpoint{3.904204in}{5.292278in}}%
\pgfpathlineto{\pgfqpoint{3.906711in}{5.299049in}}%
\pgfpathlineto{\pgfqpoint{3.909217in}{5.300306in}}%
\pgfpathlineto{\pgfqpoint{3.911724in}{5.303797in}}%
\pgfpathlineto{\pgfqpoint{3.914231in}{5.305275in}}%
\pgfpathlineto{\pgfqpoint{3.914231in}{5.305275in}}%
\pgfusepath{stroke}%
\end{pgfscope}%
\begin{pgfscope}%
\pgfsetrectcap%
\pgfsetmiterjoin%
\pgfsetlinewidth{0.803000pt}%
\definecolor{currentstroke}{rgb}{0.000000,0.000000,0.000000}%
\pgfsetstrokecolor{currentstroke}%
\pgfsetdash{}{0pt}%
\pgfpathmoveto{\pgfqpoint{0.708220in}{3.210823in}}%
\pgfpathlineto{\pgfqpoint{0.708220in}{5.305275in}}%
\pgfusepath{stroke}%
\end{pgfscope}%
\begin{pgfscope}%
\pgfsetrectcap%
\pgfsetmiterjoin%
\pgfsetlinewidth{0.803000pt}%
\definecolor{currentstroke}{rgb}{0.000000,0.000000,0.000000}%
\pgfsetstrokecolor{currentstroke}%
\pgfsetdash{}{0pt}%
\pgfpathmoveto{\pgfqpoint{5.721529in}{3.210823in}}%
\pgfpathlineto{\pgfqpoint{5.721529in}{5.305275in}}%
\pgfusepath{stroke}%
\end{pgfscope}%
\begin{pgfscope}%
\pgfsetrectcap%
\pgfsetmiterjoin%
\pgfsetlinewidth{0.803000pt}%
\definecolor{currentstroke}{rgb}{0.000000,0.000000,0.000000}%
\pgfsetstrokecolor{currentstroke}%
\pgfsetdash{}{0pt}%
\pgfpathmoveto{\pgfqpoint{0.708220in}{3.210823in}}%
\pgfpathlineto{\pgfqpoint{5.721529in}{3.210823in}}%
\pgfusepath{stroke}%
\end{pgfscope}%
\begin{pgfscope}%
\pgfsetrectcap%
\pgfsetmiterjoin%
\pgfsetlinewidth{0.803000pt}%
\definecolor{currentstroke}{rgb}{0.000000,0.000000,0.000000}%
\pgfsetstrokecolor{currentstroke}%
\pgfsetdash{}{0pt}%
\pgfpathmoveto{\pgfqpoint{0.708220in}{5.305275in}}%
\pgfpathlineto{\pgfqpoint{5.721529in}{5.305275in}}%
\pgfusepath{stroke}%
\end{pgfscope}%
\begin{pgfscope}%
\pgfsetrectcap%
\pgfsetroundjoin%
\pgfsetlinewidth{1.003750pt}%
\definecolor{currentstroke}{rgb}{0.878431,0.878431,0.815686}%
\pgfsetstrokecolor{currentstroke}%
\pgfsetdash{}{0pt}%
\pgfpathmoveto{\pgfqpoint{4.804312in}{4.785068in}}%
\pgfpathlineto{\pgfqpoint{5.054312in}{4.785068in}}%
\pgfusepath{stroke}%
\end{pgfscope}%
\begin{pgfscope}%
\definecolor{textcolor}{rgb}{0.000000,0.000000,0.000000}%
\pgfsetstrokecolor{textcolor}%
\pgfsetfillcolor{textcolor}%
\pgftext[x=5.079312in,y=4.741318in,left,base]{\color{textcolor}\rmfamily\fontsize{9.000000}{10.800000}\selectfont T.+CPU1}%
\end{pgfscope}%
\begin{pgfscope}%
\pgfsetrectcap%
\pgfsetroundjoin%
\pgfsetlinewidth{1.003750pt}%
\definecolor{currentstroke}{rgb}{0.564706,0.564706,1.000000}%
\pgfsetstrokecolor{currentstroke}%
\pgfsetdash{}{0pt}%
\pgfpathmoveto{\pgfqpoint{4.804312in}{4.623269in}}%
\pgfpathlineto{\pgfqpoint{5.054312in}{4.623269in}}%
\pgfusepath{stroke}%
\end{pgfscope}%
\begin{pgfscope}%
\definecolor{textcolor}{rgb}{0.000000,0.000000,0.000000}%
\pgfsetstrokecolor{textcolor}%
\pgfsetfillcolor{textcolor}%
\pgftext[x=5.079312in,y=4.579519in,left,base]{\color{textcolor}\rmfamily\fontsize{9.000000}{10.800000}\selectfont P4+CPU1}%
\end{pgfscope}%
\begin{pgfscope}%
\pgfsetbuttcap%
\pgfsetroundjoin%
\pgfsetlinewidth{1.003750pt}%
\definecolor{currentstroke}{rgb}{0.564706,0.564706,1.000000}%
\pgfsetstrokecolor{currentstroke}%
\pgfsetdash{{1.000000pt}{1.650000pt}}{0.000000pt}%
\pgfpathmoveto{\pgfqpoint{4.804312in}{4.461469in}}%
\pgfpathlineto{\pgfqpoint{5.054312in}{4.461469in}}%
\pgfusepath{stroke}%
\end{pgfscope}%
\begin{pgfscope}%
\definecolor{textcolor}{rgb}{0.000000,0.000000,0.000000}%
\pgfsetstrokecolor{textcolor}%
\pgfsetfillcolor{textcolor}%
\pgftext[x=5.079312in,y=4.417719in,left,base]{\color{textcolor}\rmfamily\fontsize{9.000000}{10.800000}\selectfont P4+CPU8}%
\end{pgfscope}%
\begin{pgfscope}%
\pgfsetbuttcap%
\pgfsetroundjoin%
\pgfsetlinewidth{1.003750pt}%
\definecolor{currentstroke}{rgb}{0.564706,0.564706,1.000000}%
\pgfsetstrokecolor{currentstroke}%
\pgfsetdash{{3.700000pt}{1.600000pt}}{0.000000pt}%
\pgfpathmoveto{\pgfqpoint{4.804312in}{4.299670in}}%
\pgfpathlineto{\pgfqpoint{5.054312in}{4.299670in}}%
\pgfusepath{stroke}%
\end{pgfscope}%
\begin{pgfscope}%
\definecolor{textcolor}{rgb}{0.000000,0.000000,0.000000}%
\pgfsetstrokecolor{textcolor}%
\pgfsetfillcolor{textcolor}%
\pgftext[x=5.079312in,y=4.255920in,left,base]{\color{textcolor}\rmfamily\fontsize{9.000000}{10.800000}\selectfont P4+GPU}%
\end{pgfscope}%
\begin{pgfscope}%
\pgfsetrectcap%
\pgfsetroundjoin%
\pgfsetlinewidth{1.003750pt}%
\definecolor{currentstroke}{rgb}{0.811765,0.125490,0.125490}%
\pgfsetstrokecolor{currentstroke}%
\pgfsetdash{}{0pt}%
\pgfpathmoveto{\pgfqpoint{4.804312in}{4.137870in}}%
\pgfpathlineto{\pgfqpoint{5.054312in}{4.137870in}}%
\pgfusepath{stroke}%
\end{pgfscope}%
\begin{pgfscope}%
\definecolor{textcolor}{rgb}{0.000000,0.000000,0.000000}%
\pgfsetstrokecolor{textcolor}%
\pgfsetfillcolor{textcolor}%
\pgftext[x=5.079312in,y=4.094120in,left,base]{\color{textcolor}\rmfamily\fontsize{9.000000}{10.800000}\selectfont miniC2D}%
\end{pgfscope}%
\begin{pgfscope}%
\pgfsetbuttcap%
\pgfsetroundjoin%
\pgfsetlinewidth{1.003750pt}%
\definecolor{currentstroke}{rgb}{0.811765,0.125490,0.125490}%
\pgfsetstrokecolor{currentstroke}%
\pgfsetdash{{3.700000pt}{1.600000pt}}{0.000000pt}%
\pgfpathmoveto{\pgfqpoint{4.804312in}{3.976071in}}%
\pgfpathlineto{\pgfqpoint{5.054312in}{3.976071in}}%
\pgfusepath{stroke}%
\end{pgfscope}%
\begin{pgfscope}%
\definecolor{textcolor}{rgb}{0.000000,0.000000,0.000000}%
\pgfsetstrokecolor{textcolor}%
\pgfsetfillcolor{textcolor}%
\pgftext[x=5.079312in,y=3.932321in,left,base]{\color{textcolor}\rmfamily\fontsize{9.000000}{10.800000}\selectfont d4}%
\end{pgfscope}%
\begin{pgfscope}%
\pgfsetbuttcap%
\pgfsetroundjoin%
\pgfsetlinewidth{1.003750pt}%
\definecolor{currentstroke}{rgb}{0.811765,0.125490,0.125490}%
\pgfsetstrokecolor{currentstroke}%
\pgfsetdash{{1.000000pt}{1.650000pt}}{0.000000pt}%
\pgfpathmoveto{\pgfqpoint{4.804312in}{3.814271in}}%
\pgfpathlineto{\pgfqpoint{5.054312in}{3.814271in}}%
\pgfusepath{stroke}%
\end{pgfscope}%
\begin{pgfscope}%
\definecolor{textcolor}{rgb}{0.000000,0.000000,0.000000}%
\pgfsetstrokecolor{textcolor}%
\pgfsetfillcolor{textcolor}%
\pgftext[x=5.079312in,y=3.770521in,left,base]{\color{textcolor}\rmfamily\fontsize{9.000000}{10.800000}\selectfont cachet}%
\end{pgfscope}%
\begin{pgfscope}%
\pgfsetrectcap%
\pgfsetroundjoin%
\pgfsetlinewidth{1.003750pt}%
\definecolor{currentstroke}{rgb}{0.062745,0.000000,0.062745}%
\pgfsetstrokecolor{currentstroke}%
\pgfsetdash{}{0pt}%
\pgfpathmoveto{\pgfqpoint{4.804312in}{3.652471in}}%
\pgfpathlineto{\pgfqpoint{5.054312in}{3.652471in}}%
\pgfusepath{stroke}%
\end{pgfscope}%
\begin{pgfscope}%
\definecolor{textcolor}{rgb}{0.000000,0.000000,0.000000}%
\pgfsetstrokecolor{textcolor}%
\pgfsetfillcolor{textcolor}%
\pgftext[x=5.079312in,y=3.608721in,left,base]{\color{textcolor}\rmfamily\fontsize{9.000000}{10.800000}\selectfont ADDMC}%
\end{pgfscope}%
\begin{pgfscope}%
\pgfsetbuttcap%
\pgfsetroundjoin%
\pgfsetlinewidth{1.003750pt}%
\definecolor{currentstroke}{rgb}{0.000000,0.000000,0.000000}%
\pgfsetstrokecolor{currentstroke}%
\pgfsetdash{{3.700000pt}{1.600000pt}}{0.000000pt}%
\pgfpathmoveto{\pgfqpoint{4.804312in}{3.490672in}}%
\pgfpathlineto{\pgfqpoint{5.054312in}{3.490672in}}%
\pgfusepath{stroke}%
\end{pgfscope}%
\begin{pgfscope}%
\definecolor{textcolor}{rgb}{0.000000,0.000000,0.000000}%
\pgfsetstrokecolor{textcolor}%
\pgfsetfillcolor{textcolor}%
\pgftext[x=5.079312in,y=3.446922in,left,base]{\color{textcolor}\rmfamily\fontsize{9.000000}{10.800000}\selectfont gpusat2}%
\end{pgfscope}%
\begin{pgfscope}%
\pgfsetbuttcap%
\pgfsetroundjoin%
\pgfsetlinewidth{1.003750pt}%
\definecolor{currentstroke}{rgb}{0.000000,0.000000,0.000000}%
\pgfsetstrokecolor{currentstroke}%
\pgfsetdash{{1.000000pt}{1.650000pt}}{0.000000pt}%
\pgfpathmoveto{\pgfqpoint{4.804312in}{3.328872in}}%
\pgfpathlineto{\pgfqpoint{5.054312in}{3.328872in}}%
\pgfusepath{stroke}%
\end{pgfscope}%
\begin{pgfscope}%
\definecolor{textcolor}{rgb}{0.000000,0.000000,0.000000}%
\pgfsetstrokecolor{textcolor}%
\pgfsetfillcolor{textcolor}%
\pgftext[x=5.079312in,y=3.285122in,left,base]{\color{textcolor}\rmfamily\fontsize{9.000000}{10.800000}\selectfont DPMC}%
\end{pgfscope}%
\begin{pgfscope}%
\pgfsetbuttcap%
\pgfsetmiterjoin%
\definecolor{currentfill}{rgb}{1.000000,1.000000,1.000000}%
\pgfsetfillcolor{currentfill}%
\pgfsetlinewidth{0.000000pt}%
\definecolor{currentstroke}{rgb}{0.000000,0.000000,0.000000}%
\pgfsetstrokecolor{currentstroke}%
\pgfsetstrokeopacity{0.000000}%
\pgfsetdash{}{0pt}%
\pgfpathmoveto{\pgfqpoint{0.708220in}{0.535823in}}%
\pgfpathlineto{\pgfqpoint{5.721529in}{0.535823in}}%
\pgfpathlineto{\pgfqpoint{5.721529in}{2.630275in}}%
\pgfpathlineto{\pgfqpoint{0.708220in}{2.630275in}}%
\pgfpathclose%
\pgfusepath{fill}%
\end{pgfscope}%
\begin{pgfscope}%
\pgfsetbuttcap%
\pgfsetroundjoin%
\definecolor{currentfill}{rgb}{0.000000,0.000000,0.000000}%
\pgfsetfillcolor{currentfill}%
\pgfsetlinewidth{0.803000pt}%
\definecolor{currentstroke}{rgb}{0.000000,0.000000,0.000000}%
\pgfsetstrokecolor{currentstroke}%
\pgfsetdash{}{0pt}%
\pgfsys@defobject{currentmarker}{\pgfqpoint{0.000000in}{-0.048611in}}{\pgfqpoint{0.000000in}{0.000000in}}{%
\pgfpathmoveto{\pgfqpoint{0.000000in}{0.000000in}}%
\pgfpathlineto{\pgfqpoint{0.000000in}{-0.048611in}}%
\pgfusepath{stroke,fill}%
}%
\begin{pgfscope}%
\pgfsys@transformshift{0.708220in}{0.535823in}%
\pgfsys@useobject{currentmarker}{}%
\end{pgfscope}%
\end{pgfscope}%
\begin{pgfscope}%
\definecolor{textcolor}{rgb}{0.000000,0.000000,0.000000}%
\pgfsetstrokecolor{textcolor}%
\pgfsetfillcolor{textcolor}%
\pgftext[x=0.708220in,y=0.438600in,,top]{\color{textcolor}\rmfamily\fontsize{9.000000}{10.800000}\selectfont \(\displaystyle {0}\)}%
\end{pgfscope}%
\begin{pgfscope}%
\pgfsetbuttcap%
\pgfsetroundjoin%
\definecolor{currentfill}{rgb}{0.000000,0.000000,0.000000}%
\pgfsetfillcolor{currentfill}%
\pgfsetlinewidth{0.803000pt}%
\definecolor{currentstroke}{rgb}{0.000000,0.000000,0.000000}%
\pgfsetstrokecolor{currentstroke}%
\pgfsetdash{}{0pt}%
\pgfsys@defobject{currentmarker}{\pgfqpoint{0.000000in}{-0.048611in}}{\pgfqpoint{0.000000in}{0.000000in}}{%
\pgfpathmoveto{\pgfqpoint{0.000000in}{0.000000in}}%
\pgfpathlineto{\pgfqpoint{0.000000in}{-0.048611in}}%
\pgfusepath{stroke,fill}%
}%
\begin{pgfscope}%
\pgfsys@transformshift{1.334883in}{0.535823in}%
\pgfsys@useobject{currentmarker}{}%
\end{pgfscope}%
\end{pgfscope}%
\begin{pgfscope}%
\definecolor{textcolor}{rgb}{0.000000,0.000000,0.000000}%
\pgfsetstrokecolor{textcolor}%
\pgfsetfillcolor{textcolor}%
\pgftext[x=1.334883in,y=0.438600in,,top]{\color{textcolor}\rmfamily\fontsize{9.000000}{10.800000}\selectfont \(\displaystyle {250}\)}%
\end{pgfscope}%
\begin{pgfscope}%
\pgfsetbuttcap%
\pgfsetroundjoin%
\definecolor{currentfill}{rgb}{0.000000,0.000000,0.000000}%
\pgfsetfillcolor{currentfill}%
\pgfsetlinewidth{0.803000pt}%
\definecolor{currentstroke}{rgb}{0.000000,0.000000,0.000000}%
\pgfsetstrokecolor{currentstroke}%
\pgfsetdash{}{0pt}%
\pgfsys@defobject{currentmarker}{\pgfqpoint{0.000000in}{-0.048611in}}{\pgfqpoint{0.000000in}{0.000000in}}{%
\pgfpathmoveto{\pgfqpoint{0.000000in}{0.000000in}}%
\pgfpathlineto{\pgfqpoint{0.000000in}{-0.048611in}}%
\pgfusepath{stroke,fill}%
}%
\begin{pgfscope}%
\pgfsys@transformshift{1.961547in}{0.535823in}%
\pgfsys@useobject{currentmarker}{}%
\end{pgfscope}%
\end{pgfscope}%
\begin{pgfscope}%
\definecolor{textcolor}{rgb}{0.000000,0.000000,0.000000}%
\pgfsetstrokecolor{textcolor}%
\pgfsetfillcolor{textcolor}%
\pgftext[x=1.961547in,y=0.438600in,,top]{\color{textcolor}\rmfamily\fontsize{9.000000}{10.800000}\selectfont \(\displaystyle {500}\)}%
\end{pgfscope}%
\begin{pgfscope}%
\pgfsetbuttcap%
\pgfsetroundjoin%
\definecolor{currentfill}{rgb}{0.000000,0.000000,0.000000}%
\pgfsetfillcolor{currentfill}%
\pgfsetlinewidth{0.803000pt}%
\definecolor{currentstroke}{rgb}{0.000000,0.000000,0.000000}%
\pgfsetstrokecolor{currentstroke}%
\pgfsetdash{}{0pt}%
\pgfsys@defobject{currentmarker}{\pgfqpoint{0.000000in}{-0.048611in}}{\pgfqpoint{0.000000in}{0.000000in}}{%
\pgfpathmoveto{\pgfqpoint{0.000000in}{0.000000in}}%
\pgfpathlineto{\pgfqpoint{0.000000in}{-0.048611in}}%
\pgfusepath{stroke,fill}%
}%
\begin{pgfscope}%
\pgfsys@transformshift{2.588211in}{0.535823in}%
\pgfsys@useobject{currentmarker}{}%
\end{pgfscope}%
\end{pgfscope}%
\begin{pgfscope}%
\definecolor{textcolor}{rgb}{0.000000,0.000000,0.000000}%
\pgfsetstrokecolor{textcolor}%
\pgfsetfillcolor{textcolor}%
\pgftext[x=2.588211in,y=0.438600in,,top]{\color{textcolor}\rmfamily\fontsize{9.000000}{10.800000}\selectfont \(\displaystyle {750}\)}%
\end{pgfscope}%
\begin{pgfscope}%
\pgfsetbuttcap%
\pgfsetroundjoin%
\definecolor{currentfill}{rgb}{0.000000,0.000000,0.000000}%
\pgfsetfillcolor{currentfill}%
\pgfsetlinewidth{0.803000pt}%
\definecolor{currentstroke}{rgb}{0.000000,0.000000,0.000000}%
\pgfsetstrokecolor{currentstroke}%
\pgfsetdash{}{0pt}%
\pgfsys@defobject{currentmarker}{\pgfqpoint{0.000000in}{-0.048611in}}{\pgfqpoint{0.000000in}{0.000000in}}{%
\pgfpathmoveto{\pgfqpoint{0.000000in}{0.000000in}}%
\pgfpathlineto{\pgfqpoint{0.000000in}{-0.048611in}}%
\pgfusepath{stroke,fill}%
}%
\begin{pgfscope}%
\pgfsys@transformshift{3.214874in}{0.535823in}%
\pgfsys@useobject{currentmarker}{}%
\end{pgfscope}%
\end{pgfscope}%
\begin{pgfscope}%
\definecolor{textcolor}{rgb}{0.000000,0.000000,0.000000}%
\pgfsetstrokecolor{textcolor}%
\pgfsetfillcolor{textcolor}%
\pgftext[x=3.214874in,y=0.438600in,,top]{\color{textcolor}\rmfamily\fontsize{9.000000}{10.800000}\selectfont \(\displaystyle {1000}\)}%
\end{pgfscope}%
\begin{pgfscope}%
\pgfsetbuttcap%
\pgfsetroundjoin%
\definecolor{currentfill}{rgb}{0.000000,0.000000,0.000000}%
\pgfsetfillcolor{currentfill}%
\pgfsetlinewidth{0.803000pt}%
\definecolor{currentstroke}{rgb}{0.000000,0.000000,0.000000}%
\pgfsetstrokecolor{currentstroke}%
\pgfsetdash{}{0pt}%
\pgfsys@defobject{currentmarker}{\pgfqpoint{0.000000in}{-0.048611in}}{\pgfqpoint{0.000000in}{0.000000in}}{%
\pgfpathmoveto{\pgfqpoint{0.000000in}{0.000000in}}%
\pgfpathlineto{\pgfqpoint{0.000000in}{-0.048611in}}%
\pgfusepath{stroke,fill}%
}%
\begin{pgfscope}%
\pgfsys@transformshift{3.841538in}{0.535823in}%
\pgfsys@useobject{currentmarker}{}%
\end{pgfscope}%
\end{pgfscope}%
\begin{pgfscope}%
\definecolor{textcolor}{rgb}{0.000000,0.000000,0.000000}%
\pgfsetstrokecolor{textcolor}%
\pgfsetfillcolor{textcolor}%
\pgftext[x=3.841538in,y=0.438600in,,top]{\color{textcolor}\rmfamily\fontsize{9.000000}{10.800000}\selectfont \(\displaystyle {1250}\)}%
\end{pgfscope}%
\begin{pgfscope}%
\pgfsetbuttcap%
\pgfsetroundjoin%
\definecolor{currentfill}{rgb}{0.000000,0.000000,0.000000}%
\pgfsetfillcolor{currentfill}%
\pgfsetlinewidth{0.803000pt}%
\definecolor{currentstroke}{rgb}{0.000000,0.000000,0.000000}%
\pgfsetstrokecolor{currentstroke}%
\pgfsetdash{}{0pt}%
\pgfsys@defobject{currentmarker}{\pgfqpoint{0.000000in}{-0.048611in}}{\pgfqpoint{0.000000in}{0.000000in}}{%
\pgfpathmoveto{\pgfqpoint{0.000000in}{0.000000in}}%
\pgfpathlineto{\pgfqpoint{0.000000in}{-0.048611in}}%
\pgfusepath{stroke,fill}%
}%
\begin{pgfscope}%
\pgfsys@transformshift{4.468201in}{0.535823in}%
\pgfsys@useobject{currentmarker}{}%
\end{pgfscope}%
\end{pgfscope}%
\begin{pgfscope}%
\definecolor{textcolor}{rgb}{0.000000,0.000000,0.000000}%
\pgfsetstrokecolor{textcolor}%
\pgfsetfillcolor{textcolor}%
\pgftext[x=4.468201in,y=0.438600in,,top]{\color{textcolor}\rmfamily\fontsize{9.000000}{10.800000}\selectfont \(\displaystyle {1500}\)}%
\end{pgfscope}%
\begin{pgfscope}%
\pgfsetbuttcap%
\pgfsetroundjoin%
\definecolor{currentfill}{rgb}{0.000000,0.000000,0.000000}%
\pgfsetfillcolor{currentfill}%
\pgfsetlinewidth{0.803000pt}%
\definecolor{currentstroke}{rgb}{0.000000,0.000000,0.000000}%
\pgfsetstrokecolor{currentstroke}%
\pgfsetdash{}{0pt}%
\pgfsys@defobject{currentmarker}{\pgfqpoint{0.000000in}{-0.048611in}}{\pgfqpoint{0.000000in}{0.000000in}}{%
\pgfpathmoveto{\pgfqpoint{0.000000in}{0.000000in}}%
\pgfpathlineto{\pgfqpoint{0.000000in}{-0.048611in}}%
\pgfusepath{stroke,fill}%
}%
\begin{pgfscope}%
\pgfsys@transformshift{5.094865in}{0.535823in}%
\pgfsys@useobject{currentmarker}{}%
\end{pgfscope}%
\end{pgfscope}%
\begin{pgfscope}%
\definecolor{textcolor}{rgb}{0.000000,0.000000,0.000000}%
\pgfsetstrokecolor{textcolor}%
\pgfsetfillcolor{textcolor}%
\pgftext[x=5.094865in,y=0.438600in,,top]{\color{textcolor}\rmfamily\fontsize{9.000000}{10.800000}\selectfont \(\displaystyle {1750}\)}%
\end{pgfscope}%
\begin{pgfscope}%
\pgfsetbuttcap%
\pgfsetroundjoin%
\definecolor{currentfill}{rgb}{0.000000,0.000000,0.000000}%
\pgfsetfillcolor{currentfill}%
\pgfsetlinewidth{0.803000pt}%
\definecolor{currentstroke}{rgb}{0.000000,0.000000,0.000000}%
\pgfsetstrokecolor{currentstroke}%
\pgfsetdash{}{0pt}%
\pgfsys@defobject{currentmarker}{\pgfqpoint{0.000000in}{-0.048611in}}{\pgfqpoint{0.000000in}{0.000000in}}{%
\pgfpathmoveto{\pgfqpoint{0.000000in}{0.000000in}}%
\pgfpathlineto{\pgfqpoint{0.000000in}{-0.048611in}}%
\pgfusepath{stroke,fill}%
}%
\begin{pgfscope}%
\pgfsys@transformshift{5.721529in}{0.535823in}%
\pgfsys@useobject{currentmarker}{}%
\end{pgfscope}%
\end{pgfscope}%
\begin{pgfscope}%
\definecolor{textcolor}{rgb}{0.000000,0.000000,0.000000}%
\pgfsetstrokecolor{textcolor}%
\pgfsetfillcolor{textcolor}%
\pgftext[x=5.721529in,y=0.438600in,,top]{\color{textcolor}\rmfamily\fontsize{9.000000}{10.800000}\selectfont \(\displaystyle {2000}\)}%
\end{pgfscope}%
\begin{pgfscope}%
\definecolor{textcolor}{rgb}{0.000000,0.000000,0.000000}%
\pgfsetstrokecolor{textcolor}%
\pgfsetfillcolor{textcolor}%
\pgftext[x=3.214874in,y=0.272655in,,top]{\color{textcolor}\rmfamily\fontsize{10.000000}{12.000000}\selectfont Number of benchmarks solved}%
\end{pgfscope}%
\begin{pgfscope}%
\pgfsetbuttcap%
\pgfsetroundjoin%
\definecolor{currentfill}{rgb}{0.000000,0.000000,0.000000}%
\pgfsetfillcolor{currentfill}%
\pgfsetlinewidth{0.803000pt}%
\definecolor{currentstroke}{rgb}{0.000000,0.000000,0.000000}%
\pgfsetstrokecolor{currentstroke}%
\pgfsetdash{}{0pt}%
\pgfsys@defobject{currentmarker}{\pgfqpoint{-0.048611in}{0.000000in}}{\pgfqpoint{-0.000000in}{0.000000in}}{%
\pgfpathmoveto{\pgfqpoint{-0.000000in}{0.000000in}}%
\pgfpathlineto{\pgfqpoint{-0.048611in}{0.000000in}}%
\pgfusepath{stroke,fill}%
}%
\begin{pgfscope}%
\pgfsys@transformshift{0.708220in}{0.654760in}%
\pgfsys@useobject{currentmarker}{}%
\end{pgfscope}%
\end{pgfscope}%
\begin{pgfscope}%
\definecolor{textcolor}{rgb}{0.000000,0.000000,0.000000}%
\pgfsetstrokecolor{textcolor}%
\pgfsetfillcolor{textcolor}%
\pgftext[x=0.344411in, y=0.610036in, left, base]{\color{textcolor}\rmfamily\fontsize{9.000000}{10.800000}\selectfont \(\displaystyle {10^{-2}}\)}%
\end{pgfscope}%
\begin{pgfscope}%
\pgfsetbuttcap%
\pgfsetroundjoin%
\definecolor{currentfill}{rgb}{0.000000,0.000000,0.000000}%
\pgfsetfillcolor{currentfill}%
\pgfsetlinewidth{0.803000pt}%
\definecolor{currentstroke}{rgb}{0.000000,0.000000,0.000000}%
\pgfsetstrokecolor{currentstroke}%
\pgfsetdash{}{0pt}%
\pgfsys@defobject{currentmarker}{\pgfqpoint{-0.048611in}{0.000000in}}{\pgfqpoint{-0.000000in}{0.000000in}}{%
\pgfpathmoveto{\pgfqpoint{-0.000000in}{0.000000in}}%
\pgfpathlineto{\pgfqpoint{-0.048611in}{0.000000in}}%
\pgfusepath{stroke,fill}%
}%
\begin{pgfscope}%
\pgfsys@transformshift{0.708220in}{1.049863in}%
\pgfsys@useobject{currentmarker}{}%
\end{pgfscope}%
\end{pgfscope}%
\begin{pgfscope}%
\definecolor{textcolor}{rgb}{0.000000,0.000000,0.000000}%
\pgfsetstrokecolor{textcolor}%
\pgfsetfillcolor{textcolor}%
\pgftext[x=0.344411in, y=1.005139in, left, base]{\color{textcolor}\rmfamily\fontsize{9.000000}{10.800000}\selectfont \(\displaystyle {10^{-1}}\)}%
\end{pgfscope}%
\begin{pgfscope}%
\pgfsetbuttcap%
\pgfsetroundjoin%
\definecolor{currentfill}{rgb}{0.000000,0.000000,0.000000}%
\pgfsetfillcolor{currentfill}%
\pgfsetlinewidth{0.803000pt}%
\definecolor{currentstroke}{rgb}{0.000000,0.000000,0.000000}%
\pgfsetstrokecolor{currentstroke}%
\pgfsetdash{}{0pt}%
\pgfsys@defobject{currentmarker}{\pgfqpoint{-0.048611in}{0.000000in}}{\pgfqpoint{-0.000000in}{0.000000in}}{%
\pgfpathmoveto{\pgfqpoint{-0.000000in}{0.000000in}}%
\pgfpathlineto{\pgfqpoint{-0.048611in}{0.000000in}}%
\pgfusepath{stroke,fill}%
}%
\begin{pgfscope}%
\pgfsys@transformshift{0.708220in}{1.444966in}%
\pgfsys@useobject{currentmarker}{}%
\end{pgfscope}%
\end{pgfscope}%
\begin{pgfscope}%
\definecolor{textcolor}{rgb}{0.000000,0.000000,0.000000}%
\pgfsetstrokecolor{textcolor}%
\pgfsetfillcolor{textcolor}%
\pgftext[x=0.424657in, y=1.400242in, left, base]{\color{textcolor}\rmfamily\fontsize{9.000000}{10.800000}\selectfont \(\displaystyle {10^{0}}\)}%
\end{pgfscope}%
\begin{pgfscope}%
\pgfsetbuttcap%
\pgfsetroundjoin%
\definecolor{currentfill}{rgb}{0.000000,0.000000,0.000000}%
\pgfsetfillcolor{currentfill}%
\pgfsetlinewidth{0.803000pt}%
\definecolor{currentstroke}{rgb}{0.000000,0.000000,0.000000}%
\pgfsetstrokecolor{currentstroke}%
\pgfsetdash{}{0pt}%
\pgfsys@defobject{currentmarker}{\pgfqpoint{-0.048611in}{0.000000in}}{\pgfqpoint{-0.000000in}{0.000000in}}{%
\pgfpathmoveto{\pgfqpoint{-0.000000in}{0.000000in}}%
\pgfpathlineto{\pgfqpoint{-0.048611in}{0.000000in}}%
\pgfusepath{stroke,fill}%
}%
\begin{pgfscope}%
\pgfsys@transformshift{0.708220in}{1.840069in}%
\pgfsys@useobject{currentmarker}{}%
\end{pgfscope}%
\end{pgfscope}%
\begin{pgfscope}%
\definecolor{textcolor}{rgb}{0.000000,0.000000,0.000000}%
\pgfsetstrokecolor{textcolor}%
\pgfsetfillcolor{textcolor}%
\pgftext[x=0.424657in, y=1.795345in, left, base]{\color{textcolor}\rmfamily\fontsize{9.000000}{10.800000}\selectfont \(\displaystyle {10^{1}}\)}%
\end{pgfscope}%
\begin{pgfscope}%
\pgfsetbuttcap%
\pgfsetroundjoin%
\definecolor{currentfill}{rgb}{0.000000,0.000000,0.000000}%
\pgfsetfillcolor{currentfill}%
\pgfsetlinewidth{0.803000pt}%
\definecolor{currentstroke}{rgb}{0.000000,0.000000,0.000000}%
\pgfsetstrokecolor{currentstroke}%
\pgfsetdash{}{0pt}%
\pgfsys@defobject{currentmarker}{\pgfqpoint{-0.048611in}{0.000000in}}{\pgfqpoint{-0.000000in}{0.000000in}}{%
\pgfpathmoveto{\pgfqpoint{-0.000000in}{0.000000in}}%
\pgfpathlineto{\pgfqpoint{-0.048611in}{0.000000in}}%
\pgfusepath{stroke,fill}%
}%
\begin{pgfscope}%
\pgfsys@transformshift{0.708220in}{2.235172in}%
\pgfsys@useobject{currentmarker}{}%
\end{pgfscope}%
\end{pgfscope}%
\begin{pgfscope}%
\definecolor{textcolor}{rgb}{0.000000,0.000000,0.000000}%
\pgfsetstrokecolor{textcolor}%
\pgfsetfillcolor{textcolor}%
\pgftext[x=0.424657in, y=2.190447in, left, base]{\color{textcolor}\rmfamily\fontsize{9.000000}{10.800000}\selectfont \(\displaystyle {10^{2}}\)}%
\end{pgfscope}%
\begin{pgfscope}%
\pgfsetbuttcap%
\pgfsetroundjoin%
\definecolor{currentfill}{rgb}{0.000000,0.000000,0.000000}%
\pgfsetfillcolor{currentfill}%
\pgfsetlinewidth{0.803000pt}%
\definecolor{currentstroke}{rgb}{0.000000,0.000000,0.000000}%
\pgfsetstrokecolor{currentstroke}%
\pgfsetdash{}{0pt}%
\pgfsys@defobject{currentmarker}{\pgfqpoint{-0.048611in}{0.000000in}}{\pgfqpoint{-0.000000in}{0.000000in}}{%
\pgfpathmoveto{\pgfqpoint{-0.000000in}{0.000000in}}%
\pgfpathlineto{\pgfqpoint{-0.048611in}{0.000000in}}%
\pgfusepath{stroke,fill}%
}%
\begin{pgfscope}%
\pgfsys@transformshift{0.708220in}{2.630275in}%
\pgfsys@useobject{currentmarker}{}%
\end{pgfscope}%
\end{pgfscope}%
\begin{pgfscope}%
\definecolor{textcolor}{rgb}{0.000000,0.000000,0.000000}%
\pgfsetstrokecolor{textcolor}%
\pgfsetfillcolor{textcolor}%
\pgftext[x=0.424657in, y=2.585550in, left, base]{\color{textcolor}\rmfamily\fontsize{9.000000}{10.800000}\selectfont \(\displaystyle {10^{3}}\)}%
\end{pgfscope}%
\begin{pgfscope}%
\pgfsetbuttcap%
\pgfsetroundjoin%
\definecolor{currentfill}{rgb}{0.000000,0.000000,0.000000}%
\pgfsetfillcolor{currentfill}%
\pgfsetlinewidth{0.602250pt}%
\definecolor{currentstroke}{rgb}{0.000000,0.000000,0.000000}%
\pgfsetstrokecolor{currentstroke}%
\pgfsetdash{}{0pt}%
\pgfsys@defobject{currentmarker}{\pgfqpoint{-0.027778in}{0.000000in}}{\pgfqpoint{-0.000000in}{0.000000in}}{%
\pgfpathmoveto{\pgfqpoint{-0.000000in}{0.000000in}}%
\pgfpathlineto{\pgfqpoint{-0.027778in}{0.000000in}}%
\pgfusepath{stroke,fill}%
}%
\begin{pgfscope}%
\pgfsys@transformshift{0.708220in}{0.535823in}%
\pgfsys@useobject{currentmarker}{}%
\end{pgfscope}%
\end{pgfscope}%
\begin{pgfscope}%
\pgfsetbuttcap%
\pgfsetroundjoin%
\definecolor{currentfill}{rgb}{0.000000,0.000000,0.000000}%
\pgfsetfillcolor{currentfill}%
\pgfsetlinewidth{0.602250pt}%
\definecolor{currentstroke}{rgb}{0.000000,0.000000,0.000000}%
\pgfsetstrokecolor{currentstroke}%
\pgfsetdash{}{0pt}%
\pgfsys@defobject{currentmarker}{\pgfqpoint{-0.027778in}{0.000000in}}{\pgfqpoint{-0.000000in}{0.000000in}}{%
\pgfpathmoveto{\pgfqpoint{-0.000000in}{0.000000in}}%
\pgfpathlineto{\pgfqpoint{-0.027778in}{0.000000in}}%
\pgfusepath{stroke,fill}%
}%
\begin{pgfscope}%
\pgfsys@transformshift{0.708220in}{0.567107in}%
\pgfsys@useobject{currentmarker}{}%
\end{pgfscope}%
\end{pgfscope}%
\begin{pgfscope}%
\pgfsetbuttcap%
\pgfsetroundjoin%
\definecolor{currentfill}{rgb}{0.000000,0.000000,0.000000}%
\pgfsetfillcolor{currentfill}%
\pgfsetlinewidth{0.602250pt}%
\definecolor{currentstroke}{rgb}{0.000000,0.000000,0.000000}%
\pgfsetstrokecolor{currentstroke}%
\pgfsetdash{}{0pt}%
\pgfsys@defobject{currentmarker}{\pgfqpoint{-0.027778in}{0.000000in}}{\pgfqpoint{-0.000000in}{0.000000in}}{%
\pgfpathmoveto{\pgfqpoint{-0.000000in}{0.000000in}}%
\pgfpathlineto{\pgfqpoint{-0.027778in}{0.000000in}}%
\pgfusepath{stroke,fill}%
}%
\begin{pgfscope}%
\pgfsys@transformshift{0.708220in}{0.593558in}%
\pgfsys@useobject{currentmarker}{}%
\end{pgfscope}%
\end{pgfscope}%
\begin{pgfscope}%
\pgfsetbuttcap%
\pgfsetroundjoin%
\definecolor{currentfill}{rgb}{0.000000,0.000000,0.000000}%
\pgfsetfillcolor{currentfill}%
\pgfsetlinewidth{0.602250pt}%
\definecolor{currentstroke}{rgb}{0.000000,0.000000,0.000000}%
\pgfsetstrokecolor{currentstroke}%
\pgfsetdash{}{0pt}%
\pgfsys@defobject{currentmarker}{\pgfqpoint{-0.027778in}{0.000000in}}{\pgfqpoint{-0.000000in}{0.000000in}}{%
\pgfpathmoveto{\pgfqpoint{-0.000000in}{0.000000in}}%
\pgfpathlineto{\pgfqpoint{-0.027778in}{0.000000in}}%
\pgfusepath{stroke,fill}%
}%
\begin{pgfscope}%
\pgfsys@transformshift{0.708220in}{0.616471in}%
\pgfsys@useobject{currentmarker}{}%
\end{pgfscope}%
\end{pgfscope}%
\begin{pgfscope}%
\pgfsetbuttcap%
\pgfsetroundjoin%
\definecolor{currentfill}{rgb}{0.000000,0.000000,0.000000}%
\pgfsetfillcolor{currentfill}%
\pgfsetlinewidth{0.602250pt}%
\definecolor{currentstroke}{rgb}{0.000000,0.000000,0.000000}%
\pgfsetstrokecolor{currentstroke}%
\pgfsetdash{}{0pt}%
\pgfsys@defobject{currentmarker}{\pgfqpoint{-0.027778in}{0.000000in}}{\pgfqpoint{-0.000000in}{0.000000in}}{%
\pgfpathmoveto{\pgfqpoint{-0.000000in}{0.000000in}}%
\pgfpathlineto{\pgfqpoint{-0.027778in}{0.000000in}}%
\pgfusepath{stroke,fill}%
}%
\begin{pgfscope}%
\pgfsys@transformshift{0.708220in}{0.636682in}%
\pgfsys@useobject{currentmarker}{}%
\end{pgfscope}%
\end{pgfscope}%
\begin{pgfscope}%
\pgfsetbuttcap%
\pgfsetroundjoin%
\definecolor{currentfill}{rgb}{0.000000,0.000000,0.000000}%
\pgfsetfillcolor{currentfill}%
\pgfsetlinewidth{0.602250pt}%
\definecolor{currentstroke}{rgb}{0.000000,0.000000,0.000000}%
\pgfsetstrokecolor{currentstroke}%
\pgfsetdash{}{0pt}%
\pgfsys@defobject{currentmarker}{\pgfqpoint{-0.027778in}{0.000000in}}{\pgfqpoint{-0.000000in}{0.000000in}}{%
\pgfpathmoveto{\pgfqpoint{-0.000000in}{0.000000in}}%
\pgfpathlineto{\pgfqpoint{-0.027778in}{0.000000in}}%
\pgfusepath{stroke,fill}%
}%
\begin{pgfscope}%
\pgfsys@transformshift{0.708220in}{0.773698in}%
\pgfsys@useobject{currentmarker}{}%
\end{pgfscope}%
\end{pgfscope}%
\begin{pgfscope}%
\pgfsetbuttcap%
\pgfsetroundjoin%
\definecolor{currentfill}{rgb}{0.000000,0.000000,0.000000}%
\pgfsetfillcolor{currentfill}%
\pgfsetlinewidth{0.602250pt}%
\definecolor{currentstroke}{rgb}{0.000000,0.000000,0.000000}%
\pgfsetstrokecolor{currentstroke}%
\pgfsetdash{}{0pt}%
\pgfsys@defobject{currentmarker}{\pgfqpoint{-0.027778in}{0.000000in}}{\pgfqpoint{-0.000000in}{0.000000in}}{%
\pgfpathmoveto{\pgfqpoint{-0.000000in}{0.000000in}}%
\pgfpathlineto{\pgfqpoint{-0.027778in}{0.000000in}}%
\pgfusepath{stroke,fill}%
}%
\begin{pgfscope}%
\pgfsys@transformshift{0.708220in}{0.843272in}%
\pgfsys@useobject{currentmarker}{}%
\end{pgfscope}%
\end{pgfscope}%
\begin{pgfscope}%
\pgfsetbuttcap%
\pgfsetroundjoin%
\definecolor{currentfill}{rgb}{0.000000,0.000000,0.000000}%
\pgfsetfillcolor{currentfill}%
\pgfsetlinewidth{0.602250pt}%
\definecolor{currentstroke}{rgb}{0.000000,0.000000,0.000000}%
\pgfsetstrokecolor{currentstroke}%
\pgfsetdash{}{0pt}%
\pgfsys@defobject{currentmarker}{\pgfqpoint{-0.027778in}{0.000000in}}{\pgfqpoint{-0.000000in}{0.000000in}}{%
\pgfpathmoveto{\pgfqpoint{-0.000000in}{0.000000in}}%
\pgfpathlineto{\pgfqpoint{-0.027778in}{0.000000in}}%
\pgfusepath{stroke,fill}%
}%
\begin{pgfscope}%
\pgfsys@transformshift{0.708220in}{0.892636in}%
\pgfsys@useobject{currentmarker}{}%
\end{pgfscope}%
\end{pgfscope}%
\begin{pgfscope}%
\pgfsetbuttcap%
\pgfsetroundjoin%
\definecolor{currentfill}{rgb}{0.000000,0.000000,0.000000}%
\pgfsetfillcolor{currentfill}%
\pgfsetlinewidth{0.602250pt}%
\definecolor{currentstroke}{rgb}{0.000000,0.000000,0.000000}%
\pgfsetstrokecolor{currentstroke}%
\pgfsetdash{}{0pt}%
\pgfsys@defobject{currentmarker}{\pgfqpoint{-0.027778in}{0.000000in}}{\pgfqpoint{-0.000000in}{0.000000in}}{%
\pgfpathmoveto{\pgfqpoint{-0.000000in}{0.000000in}}%
\pgfpathlineto{\pgfqpoint{-0.027778in}{0.000000in}}%
\pgfusepath{stroke,fill}%
}%
\begin{pgfscope}%
\pgfsys@transformshift{0.708220in}{0.930926in}%
\pgfsys@useobject{currentmarker}{}%
\end{pgfscope}%
\end{pgfscope}%
\begin{pgfscope}%
\pgfsetbuttcap%
\pgfsetroundjoin%
\definecolor{currentfill}{rgb}{0.000000,0.000000,0.000000}%
\pgfsetfillcolor{currentfill}%
\pgfsetlinewidth{0.602250pt}%
\definecolor{currentstroke}{rgb}{0.000000,0.000000,0.000000}%
\pgfsetstrokecolor{currentstroke}%
\pgfsetdash{}{0pt}%
\pgfsys@defobject{currentmarker}{\pgfqpoint{-0.027778in}{0.000000in}}{\pgfqpoint{-0.000000in}{0.000000in}}{%
\pgfpathmoveto{\pgfqpoint{-0.000000in}{0.000000in}}%
\pgfpathlineto{\pgfqpoint{-0.027778in}{0.000000in}}%
\pgfusepath{stroke,fill}%
}%
\begin{pgfscope}%
\pgfsys@transformshift{0.708220in}{0.962210in}%
\pgfsys@useobject{currentmarker}{}%
\end{pgfscope}%
\end{pgfscope}%
\begin{pgfscope}%
\pgfsetbuttcap%
\pgfsetroundjoin%
\definecolor{currentfill}{rgb}{0.000000,0.000000,0.000000}%
\pgfsetfillcolor{currentfill}%
\pgfsetlinewidth{0.602250pt}%
\definecolor{currentstroke}{rgb}{0.000000,0.000000,0.000000}%
\pgfsetstrokecolor{currentstroke}%
\pgfsetdash{}{0pt}%
\pgfsys@defobject{currentmarker}{\pgfqpoint{-0.027778in}{0.000000in}}{\pgfqpoint{-0.000000in}{0.000000in}}{%
\pgfpathmoveto{\pgfqpoint{-0.000000in}{0.000000in}}%
\pgfpathlineto{\pgfqpoint{-0.027778in}{0.000000in}}%
\pgfusepath{stroke,fill}%
}%
\begin{pgfscope}%
\pgfsys@transformshift{0.708220in}{0.988661in}%
\pgfsys@useobject{currentmarker}{}%
\end{pgfscope}%
\end{pgfscope}%
\begin{pgfscope}%
\pgfsetbuttcap%
\pgfsetroundjoin%
\definecolor{currentfill}{rgb}{0.000000,0.000000,0.000000}%
\pgfsetfillcolor{currentfill}%
\pgfsetlinewidth{0.602250pt}%
\definecolor{currentstroke}{rgb}{0.000000,0.000000,0.000000}%
\pgfsetstrokecolor{currentstroke}%
\pgfsetdash{}{0pt}%
\pgfsys@defobject{currentmarker}{\pgfqpoint{-0.027778in}{0.000000in}}{\pgfqpoint{-0.000000in}{0.000000in}}{%
\pgfpathmoveto{\pgfqpoint{-0.000000in}{0.000000in}}%
\pgfpathlineto{\pgfqpoint{-0.027778in}{0.000000in}}%
\pgfusepath{stroke,fill}%
}%
\begin{pgfscope}%
\pgfsys@transformshift{0.708220in}{1.011574in}%
\pgfsys@useobject{currentmarker}{}%
\end{pgfscope}%
\end{pgfscope}%
\begin{pgfscope}%
\pgfsetbuttcap%
\pgfsetroundjoin%
\definecolor{currentfill}{rgb}{0.000000,0.000000,0.000000}%
\pgfsetfillcolor{currentfill}%
\pgfsetlinewidth{0.602250pt}%
\definecolor{currentstroke}{rgb}{0.000000,0.000000,0.000000}%
\pgfsetstrokecolor{currentstroke}%
\pgfsetdash{}{0pt}%
\pgfsys@defobject{currentmarker}{\pgfqpoint{-0.027778in}{0.000000in}}{\pgfqpoint{-0.000000in}{0.000000in}}{%
\pgfpathmoveto{\pgfqpoint{-0.000000in}{0.000000in}}%
\pgfpathlineto{\pgfqpoint{-0.027778in}{0.000000in}}%
\pgfusepath{stroke,fill}%
}%
\begin{pgfscope}%
\pgfsys@transformshift{0.708220in}{1.031785in}%
\pgfsys@useobject{currentmarker}{}%
\end{pgfscope}%
\end{pgfscope}%
\begin{pgfscope}%
\pgfsetbuttcap%
\pgfsetroundjoin%
\definecolor{currentfill}{rgb}{0.000000,0.000000,0.000000}%
\pgfsetfillcolor{currentfill}%
\pgfsetlinewidth{0.602250pt}%
\definecolor{currentstroke}{rgb}{0.000000,0.000000,0.000000}%
\pgfsetstrokecolor{currentstroke}%
\pgfsetdash{}{0pt}%
\pgfsys@defobject{currentmarker}{\pgfqpoint{-0.027778in}{0.000000in}}{\pgfqpoint{-0.000000in}{0.000000in}}{%
\pgfpathmoveto{\pgfqpoint{-0.000000in}{0.000000in}}%
\pgfpathlineto{\pgfqpoint{-0.027778in}{0.000000in}}%
\pgfusepath{stroke,fill}%
}%
\begin{pgfscope}%
\pgfsys@transformshift{0.708220in}{1.168801in}%
\pgfsys@useobject{currentmarker}{}%
\end{pgfscope}%
\end{pgfscope}%
\begin{pgfscope}%
\pgfsetbuttcap%
\pgfsetroundjoin%
\definecolor{currentfill}{rgb}{0.000000,0.000000,0.000000}%
\pgfsetfillcolor{currentfill}%
\pgfsetlinewidth{0.602250pt}%
\definecolor{currentstroke}{rgb}{0.000000,0.000000,0.000000}%
\pgfsetstrokecolor{currentstroke}%
\pgfsetdash{}{0pt}%
\pgfsys@defobject{currentmarker}{\pgfqpoint{-0.027778in}{0.000000in}}{\pgfqpoint{-0.000000in}{0.000000in}}{%
\pgfpathmoveto{\pgfqpoint{-0.000000in}{0.000000in}}%
\pgfpathlineto{\pgfqpoint{-0.027778in}{0.000000in}}%
\pgfusepath{stroke,fill}%
}%
\begin{pgfscope}%
\pgfsys@transformshift{0.708220in}{1.238375in}%
\pgfsys@useobject{currentmarker}{}%
\end{pgfscope}%
\end{pgfscope}%
\begin{pgfscope}%
\pgfsetbuttcap%
\pgfsetroundjoin%
\definecolor{currentfill}{rgb}{0.000000,0.000000,0.000000}%
\pgfsetfillcolor{currentfill}%
\pgfsetlinewidth{0.602250pt}%
\definecolor{currentstroke}{rgb}{0.000000,0.000000,0.000000}%
\pgfsetstrokecolor{currentstroke}%
\pgfsetdash{}{0pt}%
\pgfsys@defobject{currentmarker}{\pgfqpoint{-0.027778in}{0.000000in}}{\pgfqpoint{-0.000000in}{0.000000in}}{%
\pgfpathmoveto{\pgfqpoint{-0.000000in}{0.000000in}}%
\pgfpathlineto{\pgfqpoint{-0.027778in}{0.000000in}}%
\pgfusepath{stroke,fill}%
}%
\begin{pgfscope}%
\pgfsys@transformshift{0.708220in}{1.287739in}%
\pgfsys@useobject{currentmarker}{}%
\end{pgfscope}%
\end{pgfscope}%
\begin{pgfscope}%
\pgfsetbuttcap%
\pgfsetroundjoin%
\definecolor{currentfill}{rgb}{0.000000,0.000000,0.000000}%
\pgfsetfillcolor{currentfill}%
\pgfsetlinewidth{0.602250pt}%
\definecolor{currentstroke}{rgb}{0.000000,0.000000,0.000000}%
\pgfsetstrokecolor{currentstroke}%
\pgfsetdash{}{0pt}%
\pgfsys@defobject{currentmarker}{\pgfqpoint{-0.027778in}{0.000000in}}{\pgfqpoint{-0.000000in}{0.000000in}}{%
\pgfpathmoveto{\pgfqpoint{-0.000000in}{0.000000in}}%
\pgfpathlineto{\pgfqpoint{-0.027778in}{0.000000in}}%
\pgfusepath{stroke,fill}%
}%
\begin{pgfscope}%
\pgfsys@transformshift{0.708220in}{1.326029in}%
\pgfsys@useobject{currentmarker}{}%
\end{pgfscope}%
\end{pgfscope}%
\begin{pgfscope}%
\pgfsetbuttcap%
\pgfsetroundjoin%
\definecolor{currentfill}{rgb}{0.000000,0.000000,0.000000}%
\pgfsetfillcolor{currentfill}%
\pgfsetlinewidth{0.602250pt}%
\definecolor{currentstroke}{rgb}{0.000000,0.000000,0.000000}%
\pgfsetstrokecolor{currentstroke}%
\pgfsetdash{}{0pt}%
\pgfsys@defobject{currentmarker}{\pgfqpoint{-0.027778in}{0.000000in}}{\pgfqpoint{-0.000000in}{0.000000in}}{%
\pgfpathmoveto{\pgfqpoint{-0.000000in}{0.000000in}}%
\pgfpathlineto{\pgfqpoint{-0.027778in}{0.000000in}}%
\pgfusepath{stroke,fill}%
}%
\begin{pgfscope}%
\pgfsys@transformshift{0.708220in}{1.357313in}%
\pgfsys@useobject{currentmarker}{}%
\end{pgfscope}%
\end{pgfscope}%
\begin{pgfscope}%
\pgfsetbuttcap%
\pgfsetroundjoin%
\definecolor{currentfill}{rgb}{0.000000,0.000000,0.000000}%
\pgfsetfillcolor{currentfill}%
\pgfsetlinewidth{0.602250pt}%
\definecolor{currentstroke}{rgb}{0.000000,0.000000,0.000000}%
\pgfsetstrokecolor{currentstroke}%
\pgfsetdash{}{0pt}%
\pgfsys@defobject{currentmarker}{\pgfqpoint{-0.027778in}{0.000000in}}{\pgfqpoint{-0.000000in}{0.000000in}}{%
\pgfpathmoveto{\pgfqpoint{-0.000000in}{0.000000in}}%
\pgfpathlineto{\pgfqpoint{-0.027778in}{0.000000in}}%
\pgfusepath{stroke,fill}%
}%
\begin{pgfscope}%
\pgfsys@transformshift{0.708220in}{1.383764in}%
\pgfsys@useobject{currentmarker}{}%
\end{pgfscope}%
\end{pgfscope}%
\begin{pgfscope}%
\pgfsetbuttcap%
\pgfsetroundjoin%
\definecolor{currentfill}{rgb}{0.000000,0.000000,0.000000}%
\pgfsetfillcolor{currentfill}%
\pgfsetlinewidth{0.602250pt}%
\definecolor{currentstroke}{rgb}{0.000000,0.000000,0.000000}%
\pgfsetstrokecolor{currentstroke}%
\pgfsetdash{}{0pt}%
\pgfsys@defobject{currentmarker}{\pgfqpoint{-0.027778in}{0.000000in}}{\pgfqpoint{-0.000000in}{0.000000in}}{%
\pgfpathmoveto{\pgfqpoint{-0.000000in}{0.000000in}}%
\pgfpathlineto{\pgfqpoint{-0.027778in}{0.000000in}}%
\pgfusepath{stroke,fill}%
}%
\begin{pgfscope}%
\pgfsys@transformshift{0.708220in}{1.406677in}%
\pgfsys@useobject{currentmarker}{}%
\end{pgfscope}%
\end{pgfscope}%
\begin{pgfscope}%
\pgfsetbuttcap%
\pgfsetroundjoin%
\definecolor{currentfill}{rgb}{0.000000,0.000000,0.000000}%
\pgfsetfillcolor{currentfill}%
\pgfsetlinewidth{0.602250pt}%
\definecolor{currentstroke}{rgb}{0.000000,0.000000,0.000000}%
\pgfsetstrokecolor{currentstroke}%
\pgfsetdash{}{0pt}%
\pgfsys@defobject{currentmarker}{\pgfqpoint{-0.027778in}{0.000000in}}{\pgfqpoint{-0.000000in}{0.000000in}}{%
\pgfpathmoveto{\pgfqpoint{-0.000000in}{0.000000in}}%
\pgfpathlineto{\pgfqpoint{-0.027778in}{0.000000in}}%
\pgfusepath{stroke,fill}%
}%
\begin{pgfscope}%
\pgfsys@transformshift{0.708220in}{1.426887in}%
\pgfsys@useobject{currentmarker}{}%
\end{pgfscope}%
\end{pgfscope}%
\begin{pgfscope}%
\pgfsetbuttcap%
\pgfsetroundjoin%
\definecolor{currentfill}{rgb}{0.000000,0.000000,0.000000}%
\pgfsetfillcolor{currentfill}%
\pgfsetlinewidth{0.602250pt}%
\definecolor{currentstroke}{rgb}{0.000000,0.000000,0.000000}%
\pgfsetstrokecolor{currentstroke}%
\pgfsetdash{}{0pt}%
\pgfsys@defobject{currentmarker}{\pgfqpoint{-0.027778in}{0.000000in}}{\pgfqpoint{-0.000000in}{0.000000in}}{%
\pgfpathmoveto{\pgfqpoint{-0.000000in}{0.000000in}}%
\pgfpathlineto{\pgfqpoint{-0.027778in}{0.000000in}}%
\pgfusepath{stroke,fill}%
}%
\begin{pgfscope}%
\pgfsys@transformshift{0.708220in}{1.563904in}%
\pgfsys@useobject{currentmarker}{}%
\end{pgfscope}%
\end{pgfscope}%
\begin{pgfscope}%
\pgfsetbuttcap%
\pgfsetroundjoin%
\definecolor{currentfill}{rgb}{0.000000,0.000000,0.000000}%
\pgfsetfillcolor{currentfill}%
\pgfsetlinewidth{0.602250pt}%
\definecolor{currentstroke}{rgb}{0.000000,0.000000,0.000000}%
\pgfsetstrokecolor{currentstroke}%
\pgfsetdash{}{0pt}%
\pgfsys@defobject{currentmarker}{\pgfqpoint{-0.027778in}{0.000000in}}{\pgfqpoint{-0.000000in}{0.000000in}}{%
\pgfpathmoveto{\pgfqpoint{-0.000000in}{0.000000in}}%
\pgfpathlineto{\pgfqpoint{-0.027778in}{0.000000in}}%
\pgfusepath{stroke,fill}%
}%
\begin{pgfscope}%
\pgfsys@transformshift{0.708220in}{1.633478in}%
\pgfsys@useobject{currentmarker}{}%
\end{pgfscope}%
\end{pgfscope}%
\begin{pgfscope}%
\pgfsetbuttcap%
\pgfsetroundjoin%
\definecolor{currentfill}{rgb}{0.000000,0.000000,0.000000}%
\pgfsetfillcolor{currentfill}%
\pgfsetlinewidth{0.602250pt}%
\definecolor{currentstroke}{rgb}{0.000000,0.000000,0.000000}%
\pgfsetstrokecolor{currentstroke}%
\pgfsetdash{}{0pt}%
\pgfsys@defobject{currentmarker}{\pgfqpoint{-0.027778in}{0.000000in}}{\pgfqpoint{-0.000000in}{0.000000in}}{%
\pgfpathmoveto{\pgfqpoint{-0.000000in}{0.000000in}}%
\pgfpathlineto{\pgfqpoint{-0.027778in}{0.000000in}}%
\pgfusepath{stroke,fill}%
}%
\begin{pgfscope}%
\pgfsys@transformshift{0.708220in}{1.682842in}%
\pgfsys@useobject{currentmarker}{}%
\end{pgfscope}%
\end{pgfscope}%
\begin{pgfscope}%
\pgfsetbuttcap%
\pgfsetroundjoin%
\definecolor{currentfill}{rgb}{0.000000,0.000000,0.000000}%
\pgfsetfillcolor{currentfill}%
\pgfsetlinewidth{0.602250pt}%
\definecolor{currentstroke}{rgb}{0.000000,0.000000,0.000000}%
\pgfsetstrokecolor{currentstroke}%
\pgfsetdash{}{0pt}%
\pgfsys@defobject{currentmarker}{\pgfqpoint{-0.027778in}{0.000000in}}{\pgfqpoint{-0.000000in}{0.000000in}}{%
\pgfpathmoveto{\pgfqpoint{-0.000000in}{0.000000in}}%
\pgfpathlineto{\pgfqpoint{-0.027778in}{0.000000in}}%
\pgfusepath{stroke,fill}%
}%
\begin{pgfscope}%
\pgfsys@transformshift{0.708220in}{1.721131in}%
\pgfsys@useobject{currentmarker}{}%
\end{pgfscope}%
\end{pgfscope}%
\begin{pgfscope}%
\pgfsetbuttcap%
\pgfsetroundjoin%
\definecolor{currentfill}{rgb}{0.000000,0.000000,0.000000}%
\pgfsetfillcolor{currentfill}%
\pgfsetlinewidth{0.602250pt}%
\definecolor{currentstroke}{rgb}{0.000000,0.000000,0.000000}%
\pgfsetstrokecolor{currentstroke}%
\pgfsetdash{}{0pt}%
\pgfsys@defobject{currentmarker}{\pgfqpoint{-0.027778in}{0.000000in}}{\pgfqpoint{-0.000000in}{0.000000in}}{%
\pgfpathmoveto{\pgfqpoint{-0.000000in}{0.000000in}}%
\pgfpathlineto{\pgfqpoint{-0.027778in}{0.000000in}}%
\pgfusepath{stroke,fill}%
}%
\begin{pgfscope}%
\pgfsys@transformshift{0.708220in}{1.752416in}%
\pgfsys@useobject{currentmarker}{}%
\end{pgfscope}%
\end{pgfscope}%
\begin{pgfscope}%
\pgfsetbuttcap%
\pgfsetroundjoin%
\definecolor{currentfill}{rgb}{0.000000,0.000000,0.000000}%
\pgfsetfillcolor{currentfill}%
\pgfsetlinewidth{0.602250pt}%
\definecolor{currentstroke}{rgb}{0.000000,0.000000,0.000000}%
\pgfsetstrokecolor{currentstroke}%
\pgfsetdash{}{0pt}%
\pgfsys@defobject{currentmarker}{\pgfqpoint{-0.027778in}{0.000000in}}{\pgfqpoint{-0.000000in}{0.000000in}}{%
\pgfpathmoveto{\pgfqpoint{-0.000000in}{0.000000in}}%
\pgfpathlineto{\pgfqpoint{-0.027778in}{0.000000in}}%
\pgfusepath{stroke,fill}%
}%
\begin{pgfscope}%
\pgfsys@transformshift{0.708220in}{1.778867in}%
\pgfsys@useobject{currentmarker}{}%
\end{pgfscope}%
\end{pgfscope}%
\begin{pgfscope}%
\pgfsetbuttcap%
\pgfsetroundjoin%
\definecolor{currentfill}{rgb}{0.000000,0.000000,0.000000}%
\pgfsetfillcolor{currentfill}%
\pgfsetlinewidth{0.602250pt}%
\definecolor{currentstroke}{rgb}{0.000000,0.000000,0.000000}%
\pgfsetstrokecolor{currentstroke}%
\pgfsetdash{}{0pt}%
\pgfsys@defobject{currentmarker}{\pgfqpoint{-0.027778in}{0.000000in}}{\pgfqpoint{-0.000000in}{0.000000in}}{%
\pgfpathmoveto{\pgfqpoint{-0.000000in}{0.000000in}}%
\pgfpathlineto{\pgfqpoint{-0.027778in}{0.000000in}}%
\pgfusepath{stroke,fill}%
}%
\begin{pgfscope}%
\pgfsys@transformshift{0.708220in}{1.801780in}%
\pgfsys@useobject{currentmarker}{}%
\end{pgfscope}%
\end{pgfscope}%
\begin{pgfscope}%
\pgfsetbuttcap%
\pgfsetroundjoin%
\definecolor{currentfill}{rgb}{0.000000,0.000000,0.000000}%
\pgfsetfillcolor{currentfill}%
\pgfsetlinewidth{0.602250pt}%
\definecolor{currentstroke}{rgb}{0.000000,0.000000,0.000000}%
\pgfsetstrokecolor{currentstroke}%
\pgfsetdash{}{0pt}%
\pgfsys@defobject{currentmarker}{\pgfqpoint{-0.027778in}{0.000000in}}{\pgfqpoint{-0.000000in}{0.000000in}}{%
\pgfpathmoveto{\pgfqpoint{-0.000000in}{0.000000in}}%
\pgfpathlineto{\pgfqpoint{-0.027778in}{0.000000in}}%
\pgfusepath{stroke,fill}%
}%
\begin{pgfscope}%
\pgfsys@transformshift{0.708220in}{1.821990in}%
\pgfsys@useobject{currentmarker}{}%
\end{pgfscope}%
\end{pgfscope}%
\begin{pgfscope}%
\pgfsetbuttcap%
\pgfsetroundjoin%
\definecolor{currentfill}{rgb}{0.000000,0.000000,0.000000}%
\pgfsetfillcolor{currentfill}%
\pgfsetlinewidth{0.602250pt}%
\definecolor{currentstroke}{rgb}{0.000000,0.000000,0.000000}%
\pgfsetstrokecolor{currentstroke}%
\pgfsetdash{}{0pt}%
\pgfsys@defobject{currentmarker}{\pgfqpoint{-0.027778in}{0.000000in}}{\pgfqpoint{-0.000000in}{0.000000in}}{%
\pgfpathmoveto{\pgfqpoint{-0.000000in}{0.000000in}}%
\pgfpathlineto{\pgfqpoint{-0.027778in}{0.000000in}}%
\pgfusepath{stroke,fill}%
}%
\begin{pgfscope}%
\pgfsys@transformshift{0.708220in}{1.959007in}%
\pgfsys@useobject{currentmarker}{}%
\end{pgfscope}%
\end{pgfscope}%
\begin{pgfscope}%
\pgfsetbuttcap%
\pgfsetroundjoin%
\definecolor{currentfill}{rgb}{0.000000,0.000000,0.000000}%
\pgfsetfillcolor{currentfill}%
\pgfsetlinewidth{0.602250pt}%
\definecolor{currentstroke}{rgb}{0.000000,0.000000,0.000000}%
\pgfsetstrokecolor{currentstroke}%
\pgfsetdash{}{0pt}%
\pgfsys@defobject{currentmarker}{\pgfqpoint{-0.027778in}{0.000000in}}{\pgfqpoint{-0.000000in}{0.000000in}}{%
\pgfpathmoveto{\pgfqpoint{-0.000000in}{0.000000in}}%
\pgfpathlineto{\pgfqpoint{-0.027778in}{0.000000in}}%
\pgfusepath{stroke,fill}%
}%
\begin{pgfscope}%
\pgfsys@transformshift{0.708220in}{2.028581in}%
\pgfsys@useobject{currentmarker}{}%
\end{pgfscope}%
\end{pgfscope}%
\begin{pgfscope}%
\pgfsetbuttcap%
\pgfsetroundjoin%
\definecolor{currentfill}{rgb}{0.000000,0.000000,0.000000}%
\pgfsetfillcolor{currentfill}%
\pgfsetlinewidth{0.602250pt}%
\definecolor{currentstroke}{rgb}{0.000000,0.000000,0.000000}%
\pgfsetstrokecolor{currentstroke}%
\pgfsetdash{}{0pt}%
\pgfsys@defobject{currentmarker}{\pgfqpoint{-0.027778in}{0.000000in}}{\pgfqpoint{-0.000000in}{0.000000in}}{%
\pgfpathmoveto{\pgfqpoint{-0.000000in}{0.000000in}}%
\pgfpathlineto{\pgfqpoint{-0.027778in}{0.000000in}}%
\pgfusepath{stroke,fill}%
}%
\begin{pgfscope}%
\pgfsys@transformshift{0.708220in}{2.077945in}%
\pgfsys@useobject{currentmarker}{}%
\end{pgfscope}%
\end{pgfscope}%
\begin{pgfscope}%
\pgfsetbuttcap%
\pgfsetroundjoin%
\definecolor{currentfill}{rgb}{0.000000,0.000000,0.000000}%
\pgfsetfillcolor{currentfill}%
\pgfsetlinewidth{0.602250pt}%
\definecolor{currentstroke}{rgb}{0.000000,0.000000,0.000000}%
\pgfsetstrokecolor{currentstroke}%
\pgfsetdash{}{0pt}%
\pgfsys@defobject{currentmarker}{\pgfqpoint{-0.027778in}{0.000000in}}{\pgfqpoint{-0.000000in}{0.000000in}}{%
\pgfpathmoveto{\pgfqpoint{-0.000000in}{0.000000in}}%
\pgfpathlineto{\pgfqpoint{-0.027778in}{0.000000in}}%
\pgfusepath{stroke,fill}%
}%
\begin{pgfscope}%
\pgfsys@transformshift{0.708220in}{2.116234in}%
\pgfsys@useobject{currentmarker}{}%
\end{pgfscope}%
\end{pgfscope}%
\begin{pgfscope}%
\pgfsetbuttcap%
\pgfsetroundjoin%
\definecolor{currentfill}{rgb}{0.000000,0.000000,0.000000}%
\pgfsetfillcolor{currentfill}%
\pgfsetlinewidth{0.602250pt}%
\definecolor{currentstroke}{rgb}{0.000000,0.000000,0.000000}%
\pgfsetstrokecolor{currentstroke}%
\pgfsetdash{}{0pt}%
\pgfsys@defobject{currentmarker}{\pgfqpoint{-0.027778in}{0.000000in}}{\pgfqpoint{-0.000000in}{0.000000in}}{%
\pgfpathmoveto{\pgfqpoint{-0.000000in}{0.000000in}}%
\pgfpathlineto{\pgfqpoint{-0.027778in}{0.000000in}}%
\pgfusepath{stroke,fill}%
}%
\begin{pgfscope}%
\pgfsys@transformshift{0.708220in}{2.147519in}%
\pgfsys@useobject{currentmarker}{}%
\end{pgfscope}%
\end{pgfscope}%
\begin{pgfscope}%
\pgfsetbuttcap%
\pgfsetroundjoin%
\definecolor{currentfill}{rgb}{0.000000,0.000000,0.000000}%
\pgfsetfillcolor{currentfill}%
\pgfsetlinewidth{0.602250pt}%
\definecolor{currentstroke}{rgb}{0.000000,0.000000,0.000000}%
\pgfsetstrokecolor{currentstroke}%
\pgfsetdash{}{0pt}%
\pgfsys@defobject{currentmarker}{\pgfqpoint{-0.027778in}{0.000000in}}{\pgfqpoint{-0.000000in}{0.000000in}}{%
\pgfpathmoveto{\pgfqpoint{-0.000000in}{0.000000in}}%
\pgfpathlineto{\pgfqpoint{-0.027778in}{0.000000in}}%
\pgfusepath{stroke,fill}%
}%
\begin{pgfscope}%
\pgfsys@transformshift{0.708220in}{2.173970in}%
\pgfsys@useobject{currentmarker}{}%
\end{pgfscope}%
\end{pgfscope}%
\begin{pgfscope}%
\pgfsetbuttcap%
\pgfsetroundjoin%
\definecolor{currentfill}{rgb}{0.000000,0.000000,0.000000}%
\pgfsetfillcolor{currentfill}%
\pgfsetlinewidth{0.602250pt}%
\definecolor{currentstroke}{rgb}{0.000000,0.000000,0.000000}%
\pgfsetstrokecolor{currentstroke}%
\pgfsetdash{}{0pt}%
\pgfsys@defobject{currentmarker}{\pgfqpoint{-0.027778in}{0.000000in}}{\pgfqpoint{-0.000000in}{0.000000in}}{%
\pgfpathmoveto{\pgfqpoint{-0.000000in}{0.000000in}}%
\pgfpathlineto{\pgfqpoint{-0.027778in}{0.000000in}}%
\pgfusepath{stroke,fill}%
}%
\begin{pgfscope}%
\pgfsys@transformshift{0.708220in}{2.196883in}%
\pgfsys@useobject{currentmarker}{}%
\end{pgfscope}%
\end{pgfscope}%
\begin{pgfscope}%
\pgfsetbuttcap%
\pgfsetroundjoin%
\definecolor{currentfill}{rgb}{0.000000,0.000000,0.000000}%
\pgfsetfillcolor{currentfill}%
\pgfsetlinewidth{0.602250pt}%
\definecolor{currentstroke}{rgb}{0.000000,0.000000,0.000000}%
\pgfsetstrokecolor{currentstroke}%
\pgfsetdash{}{0pt}%
\pgfsys@defobject{currentmarker}{\pgfqpoint{-0.027778in}{0.000000in}}{\pgfqpoint{-0.000000in}{0.000000in}}{%
\pgfpathmoveto{\pgfqpoint{-0.000000in}{0.000000in}}%
\pgfpathlineto{\pgfqpoint{-0.027778in}{0.000000in}}%
\pgfusepath{stroke,fill}%
}%
\begin{pgfscope}%
\pgfsys@transformshift{0.708220in}{2.217093in}%
\pgfsys@useobject{currentmarker}{}%
\end{pgfscope}%
\end{pgfscope}%
\begin{pgfscope}%
\pgfsetbuttcap%
\pgfsetroundjoin%
\definecolor{currentfill}{rgb}{0.000000,0.000000,0.000000}%
\pgfsetfillcolor{currentfill}%
\pgfsetlinewidth{0.602250pt}%
\definecolor{currentstroke}{rgb}{0.000000,0.000000,0.000000}%
\pgfsetstrokecolor{currentstroke}%
\pgfsetdash{}{0pt}%
\pgfsys@defobject{currentmarker}{\pgfqpoint{-0.027778in}{0.000000in}}{\pgfqpoint{-0.000000in}{0.000000in}}{%
\pgfpathmoveto{\pgfqpoint{-0.000000in}{0.000000in}}%
\pgfpathlineto{\pgfqpoint{-0.027778in}{0.000000in}}%
\pgfusepath{stroke,fill}%
}%
\begin{pgfscope}%
\pgfsys@transformshift{0.708220in}{2.354110in}%
\pgfsys@useobject{currentmarker}{}%
\end{pgfscope}%
\end{pgfscope}%
\begin{pgfscope}%
\pgfsetbuttcap%
\pgfsetroundjoin%
\definecolor{currentfill}{rgb}{0.000000,0.000000,0.000000}%
\pgfsetfillcolor{currentfill}%
\pgfsetlinewidth{0.602250pt}%
\definecolor{currentstroke}{rgb}{0.000000,0.000000,0.000000}%
\pgfsetstrokecolor{currentstroke}%
\pgfsetdash{}{0pt}%
\pgfsys@defobject{currentmarker}{\pgfqpoint{-0.027778in}{0.000000in}}{\pgfqpoint{-0.000000in}{0.000000in}}{%
\pgfpathmoveto{\pgfqpoint{-0.000000in}{0.000000in}}%
\pgfpathlineto{\pgfqpoint{-0.027778in}{0.000000in}}%
\pgfusepath{stroke,fill}%
}%
\begin{pgfscope}%
\pgfsys@transformshift{0.708220in}{2.423684in}%
\pgfsys@useobject{currentmarker}{}%
\end{pgfscope}%
\end{pgfscope}%
\begin{pgfscope}%
\pgfsetbuttcap%
\pgfsetroundjoin%
\definecolor{currentfill}{rgb}{0.000000,0.000000,0.000000}%
\pgfsetfillcolor{currentfill}%
\pgfsetlinewidth{0.602250pt}%
\definecolor{currentstroke}{rgb}{0.000000,0.000000,0.000000}%
\pgfsetstrokecolor{currentstroke}%
\pgfsetdash{}{0pt}%
\pgfsys@defobject{currentmarker}{\pgfqpoint{-0.027778in}{0.000000in}}{\pgfqpoint{-0.000000in}{0.000000in}}{%
\pgfpathmoveto{\pgfqpoint{-0.000000in}{0.000000in}}%
\pgfpathlineto{\pgfqpoint{-0.027778in}{0.000000in}}%
\pgfusepath{stroke,fill}%
}%
\begin{pgfscope}%
\pgfsys@transformshift{0.708220in}{2.473048in}%
\pgfsys@useobject{currentmarker}{}%
\end{pgfscope}%
\end{pgfscope}%
\begin{pgfscope}%
\pgfsetbuttcap%
\pgfsetroundjoin%
\definecolor{currentfill}{rgb}{0.000000,0.000000,0.000000}%
\pgfsetfillcolor{currentfill}%
\pgfsetlinewidth{0.602250pt}%
\definecolor{currentstroke}{rgb}{0.000000,0.000000,0.000000}%
\pgfsetstrokecolor{currentstroke}%
\pgfsetdash{}{0pt}%
\pgfsys@defobject{currentmarker}{\pgfqpoint{-0.027778in}{0.000000in}}{\pgfqpoint{-0.000000in}{0.000000in}}{%
\pgfpathmoveto{\pgfqpoint{-0.000000in}{0.000000in}}%
\pgfpathlineto{\pgfqpoint{-0.027778in}{0.000000in}}%
\pgfusepath{stroke,fill}%
}%
\begin{pgfscope}%
\pgfsys@transformshift{0.708220in}{2.511337in}%
\pgfsys@useobject{currentmarker}{}%
\end{pgfscope}%
\end{pgfscope}%
\begin{pgfscope}%
\pgfsetbuttcap%
\pgfsetroundjoin%
\definecolor{currentfill}{rgb}{0.000000,0.000000,0.000000}%
\pgfsetfillcolor{currentfill}%
\pgfsetlinewidth{0.602250pt}%
\definecolor{currentstroke}{rgb}{0.000000,0.000000,0.000000}%
\pgfsetstrokecolor{currentstroke}%
\pgfsetdash{}{0pt}%
\pgfsys@defobject{currentmarker}{\pgfqpoint{-0.027778in}{0.000000in}}{\pgfqpoint{-0.000000in}{0.000000in}}{%
\pgfpathmoveto{\pgfqpoint{-0.000000in}{0.000000in}}%
\pgfpathlineto{\pgfqpoint{-0.027778in}{0.000000in}}%
\pgfusepath{stroke,fill}%
}%
\begin{pgfscope}%
\pgfsys@transformshift{0.708220in}{2.542622in}%
\pgfsys@useobject{currentmarker}{}%
\end{pgfscope}%
\end{pgfscope}%
\begin{pgfscope}%
\pgfsetbuttcap%
\pgfsetroundjoin%
\definecolor{currentfill}{rgb}{0.000000,0.000000,0.000000}%
\pgfsetfillcolor{currentfill}%
\pgfsetlinewidth{0.602250pt}%
\definecolor{currentstroke}{rgb}{0.000000,0.000000,0.000000}%
\pgfsetstrokecolor{currentstroke}%
\pgfsetdash{}{0pt}%
\pgfsys@defobject{currentmarker}{\pgfqpoint{-0.027778in}{0.000000in}}{\pgfqpoint{-0.000000in}{0.000000in}}{%
\pgfpathmoveto{\pgfqpoint{-0.000000in}{0.000000in}}%
\pgfpathlineto{\pgfqpoint{-0.027778in}{0.000000in}}%
\pgfusepath{stroke,fill}%
}%
\begin{pgfscope}%
\pgfsys@transformshift{0.708220in}{2.569073in}%
\pgfsys@useobject{currentmarker}{}%
\end{pgfscope}%
\end{pgfscope}%
\begin{pgfscope}%
\pgfsetbuttcap%
\pgfsetroundjoin%
\definecolor{currentfill}{rgb}{0.000000,0.000000,0.000000}%
\pgfsetfillcolor{currentfill}%
\pgfsetlinewidth{0.602250pt}%
\definecolor{currentstroke}{rgb}{0.000000,0.000000,0.000000}%
\pgfsetstrokecolor{currentstroke}%
\pgfsetdash{}{0pt}%
\pgfsys@defobject{currentmarker}{\pgfqpoint{-0.027778in}{0.000000in}}{\pgfqpoint{-0.000000in}{0.000000in}}{%
\pgfpathmoveto{\pgfqpoint{-0.000000in}{0.000000in}}%
\pgfpathlineto{\pgfqpoint{-0.027778in}{0.000000in}}%
\pgfusepath{stroke,fill}%
}%
\begin{pgfscope}%
\pgfsys@transformshift{0.708220in}{2.591986in}%
\pgfsys@useobject{currentmarker}{}%
\end{pgfscope}%
\end{pgfscope}%
\begin{pgfscope}%
\pgfsetbuttcap%
\pgfsetroundjoin%
\definecolor{currentfill}{rgb}{0.000000,0.000000,0.000000}%
\pgfsetfillcolor{currentfill}%
\pgfsetlinewidth{0.602250pt}%
\definecolor{currentstroke}{rgb}{0.000000,0.000000,0.000000}%
\pgfsetstrokecolor{currentstroke}%
\pgfsetdash{}{0pt}%
\pgfsys@defobject{currentmarker}{\pgfqpoint{-0.027778in}{0.000000in}}{\pgfqpoint{-0.000000in}{0.000000in}}{%
\pgfpathmoveto{\pgfqpoint{-0.000000in}{0.000000in}}%
\pgfpathlineto{\pgfqpoint{-0.027778in}{0.000000in}}%
\pgfusepath{stroke,fill}%
}%
\begin{pgfscope}%
\pgfsys@transformshift{0.708220in}{2.612196in}%
\pgfsys@useobject{currentmarker}{}%
\end{pgfscope}%
\end{pgfscope}%
\begin{pgfscope}%
\definecolor{textcolor}{rgb}{0.000000,0.000000,0.000000}%
\pgfsetstrokecolor{textcolor}%
\pgfsetfillcolor{textcolor}%
\pgftext[x=0.288855in,y=1.583049in,,bottom,rotate=90.000000]{\color{textcolor}\rmfamily\fontsize{10.000000}{12.000000}\selectfont Longest solving time (s)}%
\end{pgfscope}%
\begin{pgfscope}%
\pgfpathrectangle{\pgfqpoint{0.708220in}{0.535823in}}{\pgfqpoint{5.013309in}{2.094453in}}%
\pgfusepath{clip}%
\pgfsetrectcap%
\pgfsetroundjoin%
\pgfsetlinewidth{1.003750pt}%
\definecolor{currentstroke}{rgb}{0.878431,0.878431,0.815686}%
\pgfsetstrokecolor{currentstroke}%
\pgfsetdash{}{0pt}%
\pgfpathmoveto{\pgfqpoint{0.708220in}{1.219323in}}%
\pgfpathlineto{\pgfqpoint{0.710727in}{1.226023in}}%
\pgfpathlineto{\pgfqpoint{0.715740in}{1.231500in}}%
\pgfpathlineto{\pgfqpoint{0.725766in}{1.233735in}}%
\pgfpathlineto{\pgfqpoint{0.730780in}{1.235228in}}%
\pgfpathlineto{\pgfqpoint{0.735793in}{1.236119in}}%
\pgfpathlineto{\pgfqpoint{0.738300in}{1.238747in}}%
\pgfpathlineto{\pgfqpoint{0.745820in}{1.240666in}}%
\pgfpathlineto{\pgfqpoint{0.760860in}{1.248623in}}%
\pgfpathlineto{\pgfqpoint{0.765873in}{1.250920in}}%
\pgfpathlineto{\pgfqpoint{0.768380in}{1.251503in}}%
\pgfpathlineto{\pgfqpoint{0.773393in}{1.254280in}}%
\pgfpathlineto{\pgfqpoint{0.778406in}{1.255318in}}%
\pgfpathlineto{\pgfqpoint{0.788433in}{1.257559in}}%
\pgfpathlineto{\pgfqpoint{0.848593in}{1.263333in}}%
\pgfpathlineto{\pgfqpoint{0.861126in}{1.264573in}}%
\pgfpathlineto{\pgfqpoint{0.871152in}{1.265452in}}%
\pgfpathlineto{\pgfqpoint{0.876166in}{1.267033in}}%
\pgfpathlineto{\pgfqpoint{0.888699in}{1.268372in}}%
\pgfpathlineto{\pgfqpoint{0.893712in}{1.269567in}}%
\pgfpathlineto{\pgfqpoint{1.066671in}{1.285305in}}%
\pgfpathlineto{\pgfqpoint{1.071685in}{1.286171in}}%
\pgfpathlineto{\pgfqpoint{1.094245in}{1.288049in}}%
\pgfpathlineto{\pgfqpoint{1.124324in}{1.290581in}}%
\pgfpathlineto{\pgfqpoint{1.134351in}{1.291353in}}%
\pgfpathlineto{\pgfqpoint{1.144378in}{1.292100in}}%
\pgfpathlineto{\pgfqpoint{1.151898in}{1.293340in}}%
\pgfpathlineto{\pgfqpoint{1.161924in}{1.295095in}}%
\pgfpathlineto{\pgfqpoint{1.171951in}{1.296181in}}%
\pgfpathlineto{\pgfqpoint{1.202031in}{1.298319in}}%
\pgfpathlineto{\pgfqpoint{1.217071in}{1.300161in}}%
\pgfpathlineto{\pgfqpoint{1.237124in}{1.302256in}}%
\pgfpathlineto{\pgfqpoint{1.247151in}{1.303893in}}%
\pgfpathlineto{\pgfqpoint{1.259684in}{1.305353in}}%
\pgfpathlineto{\pgfqpoint{1.267204in}{1.306064in}}%
\pgfpathlineto{\pgfqpoint{1.274724in}{1.306790in}}%
\pgfpathlineto{\pgfqpoint{1.287257in}{1.307231in}}%
\pgfpathlineto{\pgfqpoint{1.289764in}{1.309642in}}%
\pgfpathlineto{\pgfqpoint{1.304804in}{1.311096in}}%
\pgfpathlineto{\pgfqpoint{1.319844in}{1.313279in}}%
\pgfpathlineto{\pgfqpoint{1.324857in}{1.314636in}}%
\pgfpathlineto{\pgfqpoint{1.334883in}{1.315814in}}%
\pgfpathlineto{\pgfqpoint{1.352430in}{1.319140in}}%
\pgfpathlineto{\pgfqpoint{1.357443in}{1.321539in}}%
\pgfpathlineto{\pgfqpoint{1.380003in}{1.326959in}}%
\pgfpathlineto{\pgfqpoint{1.385017in}{1.327901in}}%
\pgfpathlineto{\pgfqpoint{1.387523in}{1.329302in}}%
\pgfpathlineto{\pgfqpoint{1.392537in}{1.330470in}}%
\pgfpathlineto{\pgfqpoint{1.402563in}{1.331999in}}%
\pgfpathlineto{\pgfqpoint{1.425123in}{1.336070in}}%
\pgfpathlineto{\pgfqpoint{1.442670in}{1.337530in}}%
\pgfpathlineto{\pgfqpoint{1.447683in}{1.338980in}}%
\pgfpathlineto{\pgfqpoint{1.472749in}{1.340642in}}%
\pgfpathlineto{\pgfqpoint{1.497816in}{1.345052in}}%
\pgfpathlineto{\pgfqpoint{1.507843in}{1.347123in}}%
\pgfpathlineto{\pgfqpoint{1.515363in}{1.349399in}}%
\pgfpathlineto{\pgfqpoint{1.522883in}{1.352683in}}%
\pgfpathlineto{\pgfqpoint{1.530402in}{1.353949in}}%
\pgfpathlineto{\pgfqpoint{1.535416in}{1.355706in}}%
\pgfpathlineto{\pgfqpoint{1.542936in}{1.356651in}}%
\pgfpathlineto{\pgfqpoint{1.547949in}{1.357987in}}%
\pgfpathlineto{\pgfqpoint{1.562989in}{1.359046in}}%
\pgfpathlineto{\pgfqpoint{1.583042in}{1.362112in}}%
\pgfpathlineto{\pgfqpoint{1.588056in}{1.363313in}}%
\pgfpathlineto{\pgfqpoint{1.598082in}{1.364878in}}%
\pgfpathlineto{\pgfqpoint{1.615629in}{1.368408in}}%
\pgfpathlineto{\pgfqpoint{1.645709in}{1.373049in}}%
\pgfpathlineto{\pgfqpoint{1.650722in}{1.374239in}}%
\pgfpathlineto{\pgfqpoint{1.803628in}{1.395165in}}%
\pgfpathlineto{\pgfqpoint{1.806134in}{1.397147in}}%
\pgfpathlineto{\pgfqpoint{1.821174in}{1.398538in}}%
\pgfpathlineto{\pgfqpoint{1.828694in}{1.400695in}}%
\pgfpathlineto{\pgfqpoint{1.831201in}{1.402877in}}%
\pgfpathlineto{\pgfqpoint{1.838721in}{1.403778in}}%
\pgfpathlineto{\pgfqpoint{1.841228in}{1.406225in}}%
\pgfpathlineto{\pgfqpoint{1.873814in}{1.411298in}}%
\pgfpathlineto{\pgfqpoint{1.883841in}{1.412225in}}%
\pgfpathlineto{\pgfqpoint{1.891361in}{1.412695in}}%
\pgfpathlineto{\pgfqpoint{1.896374in}{1.415142in}}%
\pgfpathlineto{\pgfqpoint{1.911414in}{1.417529in}}%
\pgfpathlineto{\pgfqpoint{1.949014in}{1.424655in}}%
\pgfpathlineto{\pgfqpoint{1.951520in}{1.424742in}}%
\pgfpathlineto{\pgfqpoint{1.956534in}{1.426254in}}%
\pgfpathlineto{\pgfqpoint{1.966560in}{1.428921in}}%
\pgfpathlineto{\pgfqpoint{1.974080in}{1.430679in}}%
\pgfpathlineto{\pgfqpoint{1.979094in}{1.431010in}}%
\pgfpathlineto{\pgfqpoint{1.984107in}{1.434793in}}%
\pgfpathlineto{\pgfqpoint{1.991627in}{1.435648in}}%
\pgfpathlineto{\pgfqpoint{1.996640in}{1.436492in}}%
\pgfpathlineto{\pgfqpoint{1.999147in}{1.438521in}}%
\pgfpathlineto{\pgfqpoint{2.004160in}{1.439037in}}%
\pgfpathlineto{\pgfqpoint{2.009173in}{1.440329in}}%
\pgfpathlineto{\pgfqpoint{2.024213in}{1.443904in}}%
\pgfpathlineto{\pgfqpoint{2.029227in}{1.444918in}}%
\pgfpathlineto{\pgfqpoint{2.036747in}{1.446934in}}%
\pgfpathlineto{\pgfqpoint{2.041760in}{1.447291in}}%
\pgfpathlineto{\pgfqpoint{2.046773in}{1.449054in}}%
\pgfpathlineto{\pgfqpoint{2.101920in}{1.453795in}}%
\pgfpathlineto{\pgfqpoint{2.106933in}{1.458536in}}%
\pgfpathlineto{\pgfqpoint{2.116960in}{1.460084in}}%
\pgfpathlineto{\pgfqpoint{2.129493in}{1.463182in}}%
\pgfpathlineto{\pgfqpoint{2.132000in}{1.463416in}}%
\pgfpathlineto{\pgfqpoint{2.134506in}{1.465237in}}%
\pgfpathlineto{\pgfqpoint{2.137013in}{1.465503in}}%
\pgfpathlineto{\pgfqpoint{2.139520in}{1.467066in}}%
\pgfpathlineto{\pgfqpoint{2.142026in}{1.467259in}}%
\pgfpathlineto{\pgfqpoint{2.144533in}{1.469402in}}%
\pgfpathlineto{\pgfqpoint{2.152053in}{1.471135in}}%
\pgfpathlineto{\pgfqpoint{2.172106in}{1.475354in}}%
\pgfpathlineto{\pgfqpoint{2.177119in}{1.478484in}}%
\pgfpathlineto{\pgfqpoint{2.182133in}{1.478716in}}%
\pgfpathlineto{\pgfqpoint{2.184639in}{1.481238in}}%
\pgfpathlineto{\pgfqpoint{2.187146in}{1.481348in}}%
\pgfpathlineto{\pgfqpoint{2.192159in}{1.482862in}}%
\pgfpathlineto{\pgfqpoint{2.199679in}{1.483521in}}%
\pgfpathlineto{\pgfqpoint{2.207199in}{1.484895in}}%
\pgfpathlineto{\pgfqpoint{2.217226in}{1.486405in}}%
\pgfpathlineto{\pgfqpoint{2.239786in}{1.492706in}}%
\pgfpathlineto{\pgfqpoint{2.257332in}{1.493697in}}%
\pgfpathlineto{\pgfqpoint{2.259839in}{1.495815in}}%
\pgfpathlineto{\pgfqpoint{2.264852in}{1.496371in}}%
\pgfpathlineto{\pgfqpoint{2.267359in}{1.500404in}}%
\pgfpathlineto{\pgfqpoint{2.279892in}{1.502311in}}%
\pgfpathlineto{\pgfqpoint{2.294932in}{1.505183in}}%
\pgfpathlineto{\pgfqpoint{2.297439in}{1.505518in}}%
\pgfpathlineto{\pgfqpoint{2.302452in}{1.508480in}}%
\pgfpathlineto{\pgfqpoint{2.314985in}{1.510921in}}%
\pgfpathlineto{\pgfqpoint{2.322505in}{1.513361in}}%
\pgfpathlineto{\pgfqpoint{2.332532in}{1.514030in}}%
\pgfpathlineto{\pgfqpoint{2.337545in}{1.516401in}}%
\pgfpathlineto{\pgfqpoint{2.357598in}{1.518867in}}%
\pgfpathlineto{\pgfqpoint{2.365118in}{1.520859in}}%
\pgfpathlineto{\pgfqpoint{2.417758in}{1.528687in}}%
\pgfpathlineto{\pgfqpoint{2.427785in}{1.530185in}}%
\pgfpathlineto{\pgfqpoint{2.437811in}{1.532041in}}%
\pgfpathlineto{\pgfqpoint{2.447838in}{1.533201in}}%
\pgfpathlineto{\pgfqpoint{2.452851in}{1.535849in}}%
\pgfpathlineto{\pgfqpoint{2.460371in}{1.537102in}}%
\pgfpathlineto{\pgfqpoint{2.472905in}{1.541412in}}%
\pgfpathlineto{\pgfqpoint{2.480425in}{1.542869in}}%
\pgfpathlineto{\pgfqpoint{2.485438in}{1.543281in}}%
\pgfpathlineto{\pgfqpoint{2.490451in}{1.544591in}}%
\pgfpathlineto{\pgfqpoint{2.495464in}{1.544768in}}%
\pgfpathlineto{\pgfqpoint{2.500478in}{1.547117in}}%
\pgfpathlineto{\pgfqpoint{2.528051in}{1.552631in}}%
\pgfpathlineto{\pgfqpoint{2.533064in}{1.553858in}}%
\pgfpathlineto{\pgfqpoint{2.540584in}{1.555057in}}%
\pgfpathlineto{\pgfqpoint{2.543091in}{1.558925in}}%
\pgfpathlineto{\pgfqpoint{2.555624in}{1.561462in}}%
\pgfpathlineto{\pgfqpoint{2.558131in}{1.564604in}}%
\pgfpathlineto{\pgfqpoint{2.575677in}{1.570141in}}%
\pgfpathlineto{\pgfqpoint{2.580691in}{1.571056in}}%
\pgfpathlineto{\pgfqpoint{2.583197in}{1.573990in}}%
\pgfpathlineto{\pgfqpoint{2.590717in}{1.574820in}}%
\pgfpathlineto{\pgfqpoint{2.605757in}{1.578456in}}%
\pgfpathlineto{\pgfqpoint{2.613277in}{1.579848in}}%
\pgfpathlineto{\pgfqpoint{2.623304in}{1.582651in}}%
\pgfpathlineto{\pgfqpoint{2.625810in}{1.582729in}}%
\pgfpathlineto{\pgfqpoint{2.633330in}{1.585533in}}%
\pgfpathlineto{\pgfqpoint{2.645864in}{1.588192in}}%
\pgfpathlineto{\pgfqpoint{2.655890in}{1.589199in}}%
\pgfpathlineto{\pgfqpoint{2.685970in}{1.593619in}}%
\pgfpathlineto{\pgfqpoint{2.688477in}{1.595873in}}%
\pgfpathlineto{\pgfqpoint{2.695997in}{1.596879in}}%
\pgfpathlineto{\pgfqpoint{2.701010in}{1.599162in}}%
\pgfpathlineto{\pgfqpoint{2.713543in}{1.600853in}}%
\pgfpathlineto{\pgfqpoint{2.718557in}{1.602722in}}%
\pgfpathlineto{\pgfqpoint{2.721063in}{1.602842in}}%
\pgfpathlineto{\pgfqpoint{2.726077in}{1.605137in}}%
\pgfpathlineto{\pgfqpoint{2.741117in}{1.608470in}}%
\pgfpathlineto{\pgfqpoint{2.746130in}{1.608965in}}%
\pgfpathlineto{\pgfqpoint{2.751143in}{1.611371in}}%
\pgfpathlineto{\pgfqpoint{2.753650in}{1.611430in}}%
\pgfpathlineto{\pgfqpoint{2.758663in}{1.614527in}}%
\pgfpathlineto{\pgfqpoint{2.768690in}{1.616599in}}%
\pgfpathlineto{\pgfqpoint{2.771196in}{1.619803in}}%
\pgfpathlineto{\pgfqpoint{2.776210in}{1.620120in}}%
\pgfpathlineto{\pgfqpoint{2.778716in}{1.622228in}}%
\pgfpathlineto{\pgfqpoint{2.788743in}{1.623851in}}%
\pgfpathlineto{\pgfqpoint{2.798770in}{1.628063in}}%
\pgfpathlineto{\pgfqpoint{2.806290in}{1.629332in}}%
\pgfpathlineto{\pgfqpoint{2.811303in}{1.630183in}}%
\pgfpathlineto{\pgfqpoint{2.813810in}{1.634688in}}%
\pgfpathlineto{\pgfqpoint{2.843889in}{1.639007in}}%
\pgfpathlineto{\pgfqpoint{2.851409in}{1.640218in}}%
\pgfpathlineto{\pgfqpoint{2.861436in}{1.641925in}}%
\pgfpathlineto{\pgfqpoint{2.866449in}{1.643510in}}%
\pgfpathlineto{\pgfqpoint{2.871463in}{1.645888in}}%
\pgfpathlineto{\pgfqpoint{2.876476in}{1.647342in}}%
\pgfpathlineto{\pgfqpoint{2.891516in}{1.651080in}}%
\pgfpathlineto{\pgfqpoint{2.894022in}{1.651282in}}%
\pgfpathlineto{\pgfqpoint{2.896529in}{1.654179in}}%
\pgfpathlineto{\pgfqpoint{2.909062in}{1.655971in}}%
\pgfpathlineto{\pgfqpoint{2.921596in}{1.659234in}}%
\pgfpathlineto{\pgfqpoint{2.926609in}{1.662048in}}%
\pgfpathlineto{\pgfqpoint{2.929116in}{1.662254in}}%
\pgfpathlineto{\pgfqpoint{2.931622in}{1.664094in}}%
\pgfpathlineto{\pgfqpoint{2.934129in}{1.664342in}}%
\pgfpathlineto{\pgfqpoint{2.936636in}{1.667458in}}%
\pgfpathlineto{\pgfqpoint{2.944156in}{1.669276in}}%
\pgfpathlineto{\pgfqpoint{2.949169in}{1.670020in}}%
\pgfpathlineto{\pgfqpoint{2.951676in}{1.671701in}}%
\pgfpathlineto{\pgfqpoint{2.959195in}{1.672062in}}%
\pgfpathlineto{\pgfqpoint{2.961702in}{1.673871in}}%
\pgfpathlineto{\pgfqpoint{2.969222in}{1.674583in}}%
\pgfpathlineto{\pgfqpoint{2.974235in}{1.679773in}}%
\pgfpathlineto{\pgfqpoint{2.991782in}{1.683790in}}%
\pgfpathlineto{\pgfqpoint{2.994289in}{1.685811in}}%
\pgfpathlineto{\pgfqpoint{3.001809in}{1.687789in}}%
\pgfpathlineto{\pgfqpoint{3.006822in}{1.690768in}}%
\pgfpathlineto{\pgfqpoint{3.019355in}{1.693466in}}%
\pgfpathlineto{\pgfqpoint{3.029382in}{1.698334in}}%
\pgfpathlineto{\pgfqpoint{3.036902in}{1.699485in}}%
\pgfpathlineto{\pgfqpoint{3.039408in}{1.700992in}}%
\pgfpathlineto{\pgfqpoint{3.044422in}{1.701445in}}%
\pgfpathlineto{\pgfqpoint{3.049435in}{1.703136in}}%
\pgfpathlineto{\pgfqpoint{3.054448in}{1.703919in}}%
\pgfpathlineto{\pgfqpoint{3.059462in}{1.705674in}}%
\pgfpathlineto{\pgfqpoint{3.069488in}{1.706896in}}%
\pgfpathlineto{\pgfqpoint{3.079515in}{1.709098in}}%
\pgfpathlineto{\pgfqpoint{3.082022in}{1.712250in}}%
\pgfpathlineto{\pgfqpoint{3.094555in}{1.713921in}}%
\pgfpathlineto{\pgfqpoint{3.099568in}{1.716142in}}%
\pgfpathlineto{\pgfqpoint{3.107088in}{1.719227in}}%
\pgfpathlineto{\pgfqpoint{3.119621in}{1.723243in}}%
\pgfpathlineto{\pgfqpoint{3.124635in}{1.725405in}}%
\pgfpathlineto{\pgfqpoint{3.134661in}{1.726577in}}%
\pgfpathlineto{\pgfqpoint{3.139675in}{1.727771in}}%
\pgfpathlineto{\pgfqpoint{3.149701in}{1.729519in}}%
\pgfpathlineto{\pgfqpoint{3.159728in}{1.730389in}}%
\pgfpathlineto{\pgfqpoint{3.182288in}{1.735081in}}%
\pgfpathlineto{\pgfqpoint{3.184794in}{1.736046in}}%
\pgfpathlineto{\pgfqpoint{3.187301in}{1.738485in}}%
\pgfpathlineto{\pgfqpoint{3.189808in}{1.738950in}}%
\pgfpathlineto{\pgfqpoint{3.194821in}{1.742459in}}%
\pgfpathlineto{\pgfqpoint{3.219888in}{1.747984in}}%
\pgfpathlineto{\pgfqpoint{3.229914in}{1.753320in}}%
\pgfpathlineto{\pgfqpoint{3.249967in}{1.757763in}}%
\pgfpathlineto{\pgfqpoint{3.252474in}{1.759697in}}%
\pgfpathlineto{\pgfqpoint{3.267514in}{1.761512in}}%
\pgfpathlineto{\pgfqpoint{3.272527in}{1.762680in}}%
\pgfpathlineto{\pgfqpoint{3.280047in}{1.765871in}}%
\pgfpathlineto{\pgfqpoint{3.290074in}{1.766925in}}%
\pgfpathlineto{\pgfqpoint{3.292581in}{1.768396in}}%
\pgfpathlineto{\pgfqpoint{3.300100in}{1.769586in}}%
\pgfpathlineto{\pgfqpoint{3.307620in}{1.770945in}}%
\pgfpathlineto{\pgfqpoint{3.312634in}{1.771628in}}%
\pgfpathlineto{\pgfqpoint{3.315140in}{1.773527in}}%
\pgfpathlineto{\pgfqpoint{3.325167in}{1.775072in}}%
\pgfpathlineto{\pgfqpoint{3.330180in}{1.778090in}}%
\pgfpathlineto{\pgfqpoint{3.342714in}{1.780355in}}%
\pgfpathlineto{\pgfqpoint{3.345220in}{1.784177in}}%
\pgfpathlineto{\pgfqpoint{3.350234in}{1.786952in}}%
\pgfpathlineto{\pgfqpoint{3.355247in}{1.787685in}}%
\pgfpathlineto{\pgfqpoint{3.360260in}{1.789085in}}%
\pgfpathlineto{\pgfqpoint{3.370287in}{1.790884in}}%
\pgfpathlineto{\pgfqpoint{3.395353in}{1.795618in}}%
\pgfpathlineto{\pgfqpoint{3.400367in}{1.799004in}}%
\pgfpathlineto{\pgfqpoint{3.410393in}{1.800490in}}%
\pgfpathlineto{\pgfqpoint{3.412900in}{1.802352in}}%
\pgfpathlineto{\pgfqpoint{3.422927in}{1.803881in}}%
\pgfpathlineto{\pgfqpoint{3.442980in}{1.809293in}}%
\pgfpathlineto{\pgfqpoint{3.445486in}{1.810875in}}%
\pgfpathlineto{\pgfqpoint{3.458020in}{1.812142in}}%
\pgfpathlineto{\pgfqpoint{3.465540in}{1.814024in}}%
\pgfpathlineto{\pgfqpoint{3.473060in}{1.815058in}}%
\pgfpathlineto{\pgfqpoint{3.478073in}{1.815813in}}%
\pgfpathlineto{\pgfqpoint{3.505646in}{1.821940in}}%
\pgfpathlineto{\pgfqpoint{3.510659in}{1.822808in}}%
\pgfpathlineto{\pgfqpoint{3.515673in}{1.823705in}}%
\pgfpathlineto{\pgfqpoint{3.520686in}{1.824397in}}%
\pgfpathlineto{\pgfqpoint{3.528206in}{1.829321in}}%
\pgfpathlineto{\pgfqpoint{3.530713in}{1.830480in}}%
\pgfpathlineto{\pgfqpoint{3.533219in}{1.834096in}}%
\pgfpathlineto{\pgfqpoint{3.538233in}{1.834542in}}%
\pgfpathlineto{\pgfqpoint{3.540739in}{1.837270in}}%
\pgfpathlineto{\pgfqpoint{3.550766in}{1.838707in}}%
\pgfpathlineto{\pgfqpoint{3.553273in}{1.840772in}}%
\pgfpathlineto{\pgfqpoint{3.558286in}{1.841068in}}%
\pgfpathlineto{\pgfqpoint{3.560793in}{1.845348in}}%
\pgfpathlineto{\pgfqpoint{3.563299in}{1.846195in}}%
\pgfpathlineto{\pgfqpoint{3.568313in}{1.850341in}}%
\pgfpathlineto{\pgfqpoint{3.590872in}{1.856364in}}%
\pgfpathlineto{\pgfqpoint{3.600899in}{1.860864in}}%
\pgfpathlineto{\pgfqpoint{3.605912in}{1.863812in}}%
\pgfpathlineto{\pgfqpoint{3.608419in}{1.864002in}}%
\pgfpathlineto{\pgfqpoint{3.613432in}{1.865802in}}%
\pgfpathlineto{\pgfqpoint{3.620952in}{1.867658in}}%
\pgfpathlineto{\pgfqpoint{3.633486in}{1.869056in}}%
\pgfpathlineto{\pgfqpoint{3.638499in}{1.870812in}}%
\pgfpathlineto{\pgfqpoint{3.643512in}{1.871825in}}%
\pgfpathlineto{\pgfqpoint{3.646019in}{1.872465in}}%
\pgfpathlineto{\pgfqpoint{3.648525in}{1.874276in}}%
\pgfpathlineto{\pgfqpoint{3.653539in}{1.874927in}}%
\pgfpathlineto{\pgfqpoint{3.656045in}{1.876987in}}%
\pgfpathlineto{\pgfqpoint{3.663565in}{1.878377in}}%
\pgfpathlineto{\pgfqpoint{3.668579in}{1.880465in}}%
\pgfpathlineto{\pgfqpoint{3.681112in}{1.882877in}}%
\pgfpathlineto{\pgfqpoint{3.691139in}{1.890492in}}%
\pgfpathlineto{\pgfqpoint{3.696152in}{1.891797in}}%
\pgfpathlineto{\pgfqpoint{3.698659in}{1.894395in}}%
\pgfpathlineto{\pgfqpoint{3.701165in}{1.894486in}}%
\pgfpathlineto{\pgfqpoint{3.711192in}{1.898951in}}%
\pgfpathlineto{\pgfqpoint{3.713698in}{1.898990in}}%
\pgfpathlineto{\pgfqpoint{3.718712in}{1.900451in}}%
\pgfpathlineto{\pgfqpoint{3.721218in}{1.904149in}}%
\pgfpathlineto{\pgfqpoint{3.731245in}{1.908253in}}%
\pgfpathlineto{\pgfqpoint{3.738765in}{1.909319in}}%
\pgfpathlineto{\pgfqpoint{3.741272in}{1.912796in}}%
\pgfpathlineto{\pgfqpoint{3.761325in}{1.915293in}}%
\pgfpathlineto{\pgfqpoint{3.766338in}{1.917046in}}%
\pgfpathlineto{\pgfqpoint{3.768845in}{1.918397in}}%
\pgfpathlineto{\pgfqpoint{3.773858in}{1.918509in}}%
\pgfpathlineto{\pgfqpoint{3.786391in}{1.924021in}}%
\pgfpathlineto{\pgfqpoint{3.791405in}{1.924073in}}%
\pgfpathlineto{\pgfqpoint{3.793911in}{1.926508in}}%
\pgfpathlineto{\pgfqpoint{3.798925in}{1.927203in}}%
\pgfpathlineto{\pgfqpoint{3.806445in}{1.927890in}}%
\pgfpathlineto{\pgfqpoint{3.808951in}{1.928668in}}%
\pgfpathlineto{\pgfqpoint{3.811458in}{1.932456in}}%
\pgfpathlineto{\pgfqpoint{3.821485in}{1.933331in}}%
\pgfpathlineto{\pgfqpoint{3.823991in}{1.934986in}}%
\pgfpathlineto{\pgfqpoint{3.829005in}{1.936260in}}%
\pgfpathlineto{\pgfqpoint{3.831511in}{1.938794in}}%
\pgfpathlineto{\pgfqpoint{3.849058in}{1.942936in}}%
\pgfpathlineto{\pgfqpoint{3.851564in}{1.943143in}}%
\pgfpathlineto{\pgfqpoint{3.856578in}{1.946593in}}%
\pgfpathlineto{\pgfqpoint{3.861591in}{1.948928in}}%
\pgfpathlineto{\pgfqpoint{3.874124in}{1.957062in}}%
\pgfpathlineto{\pgfqpoint{3.876631in}{1.957236in}}%
\pgfpathlineto{\pgfqpoint{3.881644in}{1.962945in}}%
\pgfpathlineto{\pgfqpoint{3.889164in}{1.965217in}}%
\pgfpathlineto{\pgfqpoint{3.891671in}{1.972249in}}%
\pgfpathlineto{\pgfqpoint{3.896684in}{1.972636in}}%
\pgfpathlineto{\pgfqpoint{3.901698in}{1.975836in}}%
\pgfpathlineto{\pgfqpoint{3.904204in}{1.976941in}}%
\pgfpathlineto{\pgfqpoint{3.906711in}{1.979896in}}%
\pgfpathlineto{\pgfqpoint{3.911724in}{1.980583in}}%
\pgfpathlineto{\pgfqpoint{3.916737in}{1.982852in}}%
\pgfpathlineto{\pgfqpoint{3.924257in}{1.983738in}}%
\pgfpathlineto{\pgfqpoint{3.926764in}{1.985905in}}%
\pgfpathlineto{\pgfqpoint{3.931777in}{1.986956in}}%
\pgfpathlineto{\pgfqpoint{3.936791in}{1.987956in}}%
\pgfpathlineto{\pgfqpoint{3.939297in}{1.989464in}}%
\pgfpathlineto{\pgfqpoint{3.944311in}{1.989742in}}%
\pgfpathlineto{\pgfqpoint{3.946817in}{1.995072in}}%
\pgfpathlineto{\pgfqpoint{3.949324in}{1.995192in}}%
\pgfpathlineto{\pgfqpoint{3.954337in}{1.996765in}}%
\pgfpathlineto{\pgfqpoint{3.956844in}{1.997010in}}%
\pgfpathlineto{\pgfqpoint{3.964364in}{2.000241in}}%
\pgfpathlineto{\pgfqpoint{3.966871in}{2.000429in}}%
\pgfpathlineto{\pgfqpoint{3.969377in}{2.002956in}}%
\pgfpathlineto{\pgfqpoint{3.974391in}{2.003348in}}%
\pgfpathlineto{\pgfqpoint{3.979404in}{2.007986in}}%
\pgfpathlineto{\pgfqpoint{3.989430in}{2.013562in}}%
\pgfpathlineto{\pgfqpoint{3.991937in}{2.019280in}}%
\pgfpathlineto{\pgfqpoint{3.994444in}{2.021358in}}%
\pgfpathlineto{\pgfqpoint{3.999457in}{2.029803in}}%
\pgfpathlineto{\pgfqpoint{4.006977in}{2.033150in}}%
\pgfpathlineto{\pgfqpoint{4.014497in}{2.039518in}}%
\pgfpathlineto{\pgfqpoint{4.017004in}{2.043704in}}%
\pgfpathlineto{\pgfqpoint{4.022017in}{2.045569in}}%
\pgfpathlineto{\pgfqpoint{4.024524in}{2.047730in}}%
\pgfpathlineto{\pgfqpoint{4.034550in}{2.049214in}}%
\pgfpathlineto{\pgfqpoint{4.037057in}{2.050161in}}%
\pgfpathlineto{\pgfqpoint{4.039564in}{2.055338in}}%
\pgfpathlineto{\pgfqpoint{4.042070in}{2.056076in}}%
\pgfpathlineto{\pgfqpoint{4.044577in}{2.058853in}}%
\pgfpathlineto{\pgfqpoint{4.052097in}{2.062991in}}%
\pgfpathlineto{\pgfqpoint{4.059617in}{2.067115in}}%
\pgfpathlineto{\pgfqpoint{4.067137in}{2.068477in}}%
\pgfpathlineto{\pgfqpoint{4.069643in}{2.075516in}}%
\pgfpathlineto{\pgfqpoint{4.072150in}{2.078541in}}%
\pgfpathlineto{\pgfqpoint{4.074657in}{2.078986in}}%
\pgfpathlineto{\pgfqpoint{4.077163in}{2.084038in}}%
\pgfpathlineto{\pgfqpoint{4.079670in}{2.084241in}}%
\pgfpathlineto{\pgfqpoint{4.082177in}{2.088970in}}%
\pgfpathlineto{\pgfqpoint{4.084683in}{2.098874in}}%
\pgfpathlineto{\pgfqpoint{4.087190in}{2.101975in}}%
\pgfpathlineto{\pgfqpoint{4.094710in}{2.104861in}}%
\pgfpathlineto{\pgfqpoint{4.102230in}{2.110778in}}%
\pgfpathlineto{\pgfqpoint{4.107243in}{2.112020in}}%
\pgfpathlineto{\pgfqpoint{4.109750in}{2.116505in}}%
\pgfpathlineto{\pgfqpoint{4.114763in}{2.119871in}}%
\pgfpathlineto{\pgfqpoint{4.117270in}{2.121023in}}%
\pgfpathlineto{\pgfqpoint{4.122283in}{2.128146in}}%
\pgfpathlineto{\pgfqpoint{4.124790in}{2.139098in}}%
\pgfpathlineto{\pgfqpoint{4.129803in}{2.144863in}}%
\pgfpathlineto{\pgfqpoint{4.134816in}{2.146383in}}%
\pgfpathlineto{\pgfqpoint{4.139830in}{2.152707in}}%
\pgfpathlineto{\pgfqpoint{4.144843in}{2.153829in}}%
\pgfpathlineto{\pgfqpoint{4.147350in}{2.156789in}}%
\pgfpathlineto{\pgfqpoint{4.149856in}{2.162374in}}%
\pgfpathlineto{\pgfqpoint{4.154870in}{2.163423in}}%
\pgfpathlineto{\pgfqpoint{4.157376in}{2.167218in}}%
\pgfpathlineto{\pgfqpoint{4.159883in}{2.168914in}}%
\pgfpathlineto{\pgfqpoint{4.164896in}{2.182075in}}%
\pgfpathlineto{\pgfqpoint{4.169910in}{2.185502in}}%
\pgfpathlineto{\pgfqpoint{4.172416in}{2.192653in}}%
\pgfpathlineto{\pgfqpoint{4.174923in}{2.210106in}}%
\pgfpathlineto{\pgfqpoint{4.177430in}{2.216655in}}%
\pgfpathlineto{\pgfqpoint{4.179936in}{2.217923in}}%
\pgfpathlineto{\pgfqpoint{4.182443in}{2.227211in}}%
\pgfpathlineto{\pgfqpoint{4.184949in}{2.227518in}}%
\pgfpathlineto{\pgfqpoint{4.192469in}{2.252161in}}%
\pgfpathlineto{\pgfqpoint{4.194976in}{2.273737in}}%
\pgfpathlineto{\pgfqpoint{4.197483in}{2.280232in}}%
\pgfpathlineto{\pgfqpoint{4.199989in}{2.290978in}}%
\pgfpathlineto{\pgfqpoint{4.202496in}{2.291814in}}%
\pgfpathlineto{\pgfqpoint{4.207509in}{2.319805in}}%
\pgfpathlineto{\pgfqpoint{4.210016in}{2.326974in}}%
\pgfpathlineto{\pgfqpoint{4.212523in}{2.329304in}}%
\pgfpathlineto{\pgfqpoint{4.215029in}{2.329412in}}%
\pgfpathlineto{\pgfqpoint{4.217536in}{2.333316in}}%
\pgfpathlineto{\pgfqpoint{4.220043in}{2.333572in}}%
\pgfpathlineto{\pgfqpoint{4.222549in}{2.338071in}}%
\pgfpathlineto{\pgfqpoint{4.227563in}{2.342918in}}%
\pgfpathlineto{\pgfqpoint{4.232576in}{2.352861in}}%
\pgfpathlineto{\pgfqpoint{4.235083in}{2.371792in}}%
\pgfpathlineto{\pgfqpoint{4.237589in}{2.375664in}}%
\pgfpathlineto{\pgfqpoint{4.242603in}{2.378413in}}%
\pgfpathlineto{\pgfqpoint{4.247616in}{2.379224in}}%
\pgfpathlineto{\pgfqpoint{4.252629in}{2.391098in}}%
\pgfpathlineto{\pgfqpoint{4.255136in}{2.391501in}}%
\pgfpathlineto{\pgfqpoint{4.257642in}{2.395329in}}%
\pgfpathlineto{\pgfqpoint{4.260149in}{2.401835in}}%
\pgfpathlineto{\pgfqpoint{4.262656in}{2.401925in}}%
\pgfpathlineto{\pgfqpoint{4.265162in}{2.422145in}}%
\pgfpathlineto{\pgfqpoint{4.270176in}{2.423585in}}%
\pgfpathlineto{\pgfqpoint{4.272682in}{2.434158in}}%
\pgfpathlineto{\pgfqpoint{4.280202in}{2.435402in}}%
\pgfpathlineto{\pgfqpoint{4.282709in}{2.458054in}}%
\pgfpathlineto{\pgfqpoint{4.285216in}{2.459492in}}%
\pgfpathlineto{\pgfqpoint{4.287722in}{2.463985in}}%
\pgfpathlineto{\pgfqpoint{4.290229in}{2.482438in}}%
\pgfpathlineto{\pgfqpoint{4.292736in}{2.483432in}}%
\pgfpathlineto{\pgfqpoint{4.295242in}{2.487531in}}%
\pgfpathlineto{\pgfqpoint{4.300256in}{2.489220in}}%
\pgfpathlineto{\pgfqpoint{4.302762in}{2.497488in}}%
\pgfpathlineto{\pgfqpoint{4.307776in}{2.502192in}}%
\pgfpathlineto{\pgfqpoint{4.310282in}{2.509350in}}%
\pgfpathlineto{\pgfqpoint{4.312789in}{2.509649in}}%
\pgfpathlineto{\pgfqpoint{4.315296in}{2.515070in}}%
\pgfpathlineto{\pgfqpoint{4.317802in}{2.515914in}}%
\pgfpathlineto{\pgfqpoint{4.320309in}{2.522478in}}%
\pgfpathlineto{\pgfqpoint{4.322815in}{2.532811in}}%
\pgfpathlineto{\pgfqpoint{4.325322in}{2.537383in}}%
\pgfpathlineto{\pgfqpoint{4.327829in}{2.562582in}}%
\pgfpathlineto{\pgfqpoint{4.330335in}{2.571835in}}%
\pgfpathlineto{\pgfqpoint{4.332842in}{2.572224in}}%
\pgfpathlineto{\pgfqpoint{4.337855in}{2.601837in}}%
\pgfpathlineto{\pgfqpoint{4.342869in}{2.607101in}}%
\pgfpathlineto{\pgfqpoint{4.345375in}{2.627961in}}%
\pgfpathlineto{\pgfqpoint{4.350389in}{2.630275in}}%
\pgfpathlineto{\pgfqpoint{4.350389in}{2.630275in}}%
\pgfusepath{stroke}%
\end{pgfscope}%
\begin{pgfscope}%
\pgfpathrectangle{\pgfqpoint{0.708220in}{0.535823in}}{\pgfqpoint{5.013309in}{2.094453in}}%
\pgfusepath{clip}%
\pgfsetrectcap%
\pgfsetroundjoin%
\pgfsetlinewidth{1.003750pt}%
\definecolor{currentstroke}{rgb}{0.564706,0.564706,1.000000}%
\pgfsetstrokecolor{currentstroke}%
\pgfsetdash{}{0pt}%
\pgfpathmoveto{\pgfqpoint{0.708220in}{0.795513in}}%
\pgfpathlineto{\pgfqpoint{0.710727in}{0.813926in}}%
\pgfpathlineto{\pgfqpoint{0.713233in}{0.820308in}}%
\pgfpathlineto{\pgfqpoint{0.715740in}{0.835550in}}%
\pgfpathlineto{\pgfqpoint{0.723260in}{0.856797in}}%
\pgfpathlineto{\pgfqpoint{0.725766in}{0.858635in}}%
\pgfpathlineto{\pgfqpoint{0.728273in}{0.867195in}}%
\pgfpathlineto{\pgfqpoint{0.733286in}{0.869332in}}%
\pgfpathlineto{\pgfqpoint{0.735793in}{0.873768in}}%
\pgfpathlineto{\pgfqpoint{0.745820in}{0.880515in}}%
\pgfpathlineto{\pgfqpoint{0.748326in}{0.895100in}}%
\pgfpathlineto{\pgfqpoint{0.750833in}{0.898406in}}%
\pgfpathlineto{\pgfqpoint{0.760860in}{0.902227in}}%
\pgfpathlineto{\pgfqpoint{0.768380in}{0.912487in}}%
\pgfpathlineto{\pgfqpoint{0.780913in}{0.914733in}}%
\pgfpathlineto{\pgfqpoint{0.788433in}{0.919202in}}%
\pgfpathlineto{\pgfqpoint{0.798459in}{0.920221in}}%
\pgfpathlineto{\pgfqpoint{0.813499in}{0.927767in}}%
\pgfpathlineto{\pgfqpoint{0.818513in}{0.928689in}}%
\pgfpathlineto{\pgfqpoint{0.823526in}{0.930086in}}%
\pgfpathlineto{\pgfqpoint{0.828539in}{0.931065in}}%
\pgfpathlineto{\pgfqpoint{0.868646in}{0.936737in}}%
\pgfpathlineto{\pgfqpoint{0.871152in}{0.936790in}}%
\pgfpathlineto{\pgfqpoint{0.876166in}{0.938155in}}%
\pgfpathlineto{\pgfqpoint{0.886192in}{0.939208in}}%
\pgfpathlineto{\pgfqpoint{0.893712in}{0.941150in}}%
\pgfpathlineto{\pgfqpoint{0.898726in}{0.941676in}}%
\pgfpathlineto{\pgfqpoint{0.901232in}{0.943003in}}%
\pgfpathlineto{\pgfqpoint{0.903739in}{0.943028in}}%
\pgfpathlineto{\pgfqpoint{0.911259in}{0.946120in}}%
\pgfpathlineto{\pgfqpoint{0.956379in}{0.951750in}}%
\pgfpathlineto{\pgfqpoint{0.963899in}{0.953432in}}%
\pgfpathlineto{\pgfqpoint{0.966405in}{0.955069in}}%
\pgfpathlineto{\pgfqpoint{0.983952in}{0.956338in}}%
\pgfpathlineto{\pgfqpoint{0.991472in}{0.957379in}}%
\pgfpathlineto{\pgfqpoint{0.998992in}{0.958618in}}%
\pgfpathlineto{\pgfqpoint{1.026565in}{0.961066in}}%
\pgfpathlineto{\pgfqpoint{1.114298in}{0.971394in}}%
\pgfpathlineto{\pgfqpoint{1.119311in}{0.972469in}}%
\pgfpathlineto{\pgfqpoint{1.126831in}{0.973749in}}%
\pgfpathlineto{\pgfqpoint{1.136858in}{0.975583in}}%
\pgfpathlineto{\pgfqpoint{1.149391in}{0.977534in}}%
\pgfpathlineto{\pgfqpoint{1.184484in}{0.982236in}}%
\pgfpathlineto{\pgfqpoint{1.189498in}{0.983433in}}%
\pgfpathlineto{\pgfqpoint{1.244644in}{0.988864in}}%
\pgfpathlineto{\pgfqpoint{1.249657in}{0.990664in}}%
\pgfpathlineto{\pgfqpoint{1.262190in}{0.991270in}}%
\pgfpathlineto{\pgfqpoint{1.277230in}{0.992359in}}%
\pgfpathlineto{\pgfqpoint{1.292270in}{0.994417in}}%
\pgfpathlineto{\pgfqpoint{1.294777in}{0.996194in}}%
\pgfpathlineto{\pgfqpoint{1.307310in}{0.997232in}}%
\pgfpathlineto{\pgfqpoint{1.367470in}{1.001972in}}%
\pgfpathlineto{\pgfqpoint{1.380003in}{1.003224in}}%
\pgfpathlineto{\pgfqpoint{1.392537in}{1.004640in}}%
\pgfpathlineto{\pgfqpoint{1.447683in}{1.010208in}}%
\pgfpathlineto{\pgfqpoint{1.465229in}{1.011665in}}%
\pgfpathlineto{\pgfqpoint{1.470243in}{1.013160in}}%
\pgfpathlineto{\pgfqpoint{1.547949in}{1.022210in}}%
\pgfpathlineto{\pgfqpoint{1.555469in}{1.023792in}}%
\pgfpathlineto{\pgfqpoint{1.562989in}{1.024861in}}%
\pgfpathlineto{\pgfqpoint{1.570509in}{1.025366in}}%
\pgfpathlineto{\pgfqpoint{1.573016in}{1.027297in}}%
\pgfpathlineto{\pgfqpoint{1.593069in}{1.030317in}}%
\pgfpathlineto{\pgfqpoint{1.598082in}{1.031332in}}%
\pgfpathlineto{\pgfqpoint{1.648215in}{1.034974in}}%
\pgfpathlineto{\pgfqpoint{1.655735in}{1.037665in}}%
\pgfpathlineto{\pgfqpoint{1.660749in}{1.038998in}}%
\pgfpathlineto{\pgfqpoint{1.670775in}{1.040546in}}%
\pgfpathlineto{\pgfqpoint{1.698348in}{1.045075in}}%
\pgfpathlineto{\pgfqpoint{1.708375in}{1.045565in}}%
\pgfpathlineto{\pgfqpoint{1.715895in}{1.048264in}}%
\pgfpathlineto{\pgfqpoint{1.725922in}{1.050666in}}%
\pgfpathlineto{\pgfqpoint{1.748481in}{1.054790in}}%
\pgfpathlineto{\pgfqpoint{1.756001in}{1.058404in}}%
\pgfpathlineto{\pgfqpoint{1.761015in}{1.059063in}}%
\pgfpathlineto{\pgfqpoint{1.781068in}{1.065585in}}%
\pgfpathlineto{\pgfqpoint{1.786081in}{1.067655in}}%
\pgfpathlineto{\pgfqpoint{1.808641in}{1.073444in}}%
\pgfpathlineto{\pgfqpoint{1.816161in}{1.074982in}}%
\pgfpathlineto{\pgfqpoint{1.823681in}{1.077246in}}%
\pgfpathlineto{\pgfqpoint{1.836214in}{1.081576in}}%
\pgfpathlineto{\pgfqpoint{1.853761in}{1.083630in}}%
\pgfpathlineto{\pgfqpoint{1.858774in}{1.085819in}}%
\pgfpathlineto{\pgfqpoint{1.871307in}{1.086760in}}%
\pgfpathlineto{\pgfqpoint{1.873814in}{1.088178in}}%
\pgfpathlineto{\pgfqpoint{1.878827in}{1.089262in}}%
\pgfpathlineto{\pgfqpoint{1.883841in}{1.090294in}}%
\pgfpathlineto{\pgfqpoint{1.893867in}{1.097996in}}%
\pgfpathlineto{\pgfqpoint{1.901387in}{1.099851in}}%
\pgfpathlineto{\pgfqpoint{1.906401in}{1.100265in}}%
\pgfpathlineto{\pgfqpoint{1.908907in}{1.101334in}}%
\pgfpathlineto{\pgfqpoint{1.911414in}{1.103806in}}%
\pgfpathlineto{\pgfqpoint{1.928961in}{1.105788in}}%
\pgfpathlineto{\pgfqpoint{1.931467in}{1.108177in}}%
\pgfpathlineto{\pgfqpoint{1.933974in}{1.108356in}}%
\pgfpathlineto{\pgfqpoint{1.941494in}{1.111993in}}%
\pgfpathlineto{\pgfqpoint{1.944000in}{1.115072in}}%
\pgfpathlineto{\pgfqpoint{1.971574in}{1.124891in}}%
\pgfpathlineto{\pgfqpoint{1.974080in}{1.126102in}}%
\pgfpathlineto{\pgfqpoint{1.976587in}{1.128722in}}%
\pgfpathlineto{\pgfqpoint{1.984107in}{1.129528in}}%
\pgfpathlineto{\pgfqpoint{1.989120in}{1.131175in}}%
\pgfpathlineto{\pgfqpoint{1.991627in}{1.135012in}}%
\pgfpathlineto{\pgfqpoint{2.004160in}{1.138525in}}%
\pgfpathlineto{\pgfqpoint{2.006667in}{1.142486in}}%
\pgfpathlineto{\pgfqpoint{2.011680in}{1.143831in}}%
\pgfpathlineto{\pgfqpoint{2.019200in}{1.145982in}}%
\pgfpathlineto{\pgfqpoint{2.021707in}{1.146073in}}%
\pgfpathlineto{\pgfqpoint{2.026720in}{1.149005in}}%
\pgfpathlineto{\pgfqpoint{2.029227in}{1.149594in}}%
\pgfpathlineto{\pgfqpoint{2.031733in}{1.152382in}}%
\pgfpathlineto{\pgfqpoint{2.041760in}{1.153854in}}%
\pgfpathlineto{\pgfqpoint{2.051787in}{1.158133in}}%
\pgfpathlineto{\pgfqpoint{2.059307in}{1.159346in}}%
\pgfpathlineto{\pgfqpoint{2.071840in}{1.161574in}}%
\pgfpathlineto{\pgfqpoint{2.076853in}{1.162960in}}%
\pgfpathlineto{\pgfqpoint{2.099413in}{1.165798in}}%
\pgfpathlineto{\pgfqpoint{2.116960in}{1.171077in}}%
\pgfpathlineto{\pgfqpoint{2.134506in}{1.173582in}}%
\pgfpathlineto{\pgfqpoint{2.139520in}{1.174936in}}%
\pgfpathlineto{\pgfqpoint{2.149546in}{1.176836in}}%
\pgfpathlineto{\pgfqpoint{2.154559in}{1.179743in}}%
\pgfpathlineto{\pgfqpoint{2.164586in}{1.180876in}}%
\pgfpathlineto{\pgfqpoint{2.177119in}{1.182271in}}%
\pgfpathlineto{\pgfqpoint{2.182133in}{1.185429in}}%
\pgfpathlineto{\pgfqpoint{2.192159in}{1.186848in}}%
\pgfpathlineto{\pgfqpoint{2.199679in}{1.188113in}}%
\pgfpathlineto{\pgfqpoint{2.204693in}{1.190936in}}%
\pgfpathlineto{\pgfqpoint{2.209706in}{1.193552in}}%
\pgfpathlineto{\pgfqpoint{2.217226in}{1.194368in}}%
\pgfpathlineto{\pgfqpoint{2.224746in}{1.198388in}}%
\pgfpathlineto{\pgfqpoint{2.229759in}{1.199658in}}%
\pgfpathlineto{\pgfqpoint{2.232266in}{1.201598in}}%
\pgfpathlineto{\pgfqpoint{2.234772in}{1.201747in}}%
\pgfpathlineto{\pgfqpoint{2.237279in}{1.204614in}}%
\pgfpathlineto{\pgfqpoint{2.244799in}{1.205847in}}%
\pgfpathlineto{\pgfqpoint{2.247306in}{1.209122in}}%
\pgfpathlineto{\pgfqpoint{2.254826in}{1.210406in}}%
\pgfpathlineto{\pgfqpoint{2.262346in}{1.211998in}}%
\pgfpathlineto{\pgfqpoint{2.267359in}{1.217040in}}%
\pgfpathlineto{\pgfqpoint{2.269866in}{1.217574in}}%
\pgfpathlineto{\pgfqpoint{2.272372in}{1.220702in}}%
\pgfpathlineto{\pgfqpoint{2.279892in}{1.221323in}}%
\pgfpathlineto{\pgfqpoint{2.289919in}{1.223667in}}%
\pgfpathlineto{\pgfqpoint{2.309972in}{1.226974in}}%
\pgfpathlineto{\pgfqpoint{2.317492in}{1.230144in}}%
\pgfpathlineto{\pgfqpoint{2.325012in}{1.231516in}}%
\pgfpathlineto{\pgfqpoint{2.340052in}{1.235074in}}%
\pgfpathlineto{\pgfqpoint{2.347572in}{1.236541in}}%
\pgfpathlineto{\pgfqpoint{2.380158in}{1.241260in}}%
\pgfpathlineto{\pgfqpoint{2.390185in}{1.243039in}}%
\pgfpathlineto{\pgfqpoint{2.405225in}{1.245090in}}%
\pgfpathlineto{\pgfqpoint{2.407732in}{1.246487in}}%
\pgfpathlineto{\pgfqpoint{2.422771in}{1.247443in}}%
\pgfpathlineto{\pgfqpoint{2.480425in}{1.254617in}}%
\pgfpathlineto{\pgfqpoint{2.485438in}{1.255920in}}%
\pgfpathlineto{\pgfqpoint{2.490451in}{1.256515in}}%
\pgfpathlineto{\pgfqpoint{2.492958in}{1.258599in}}%
\pgfpathlineto{\pgfqpoint{2.500478in}{1.259350in}}%
\pgfpathlineto{\pgfqpoint{2.573171in}{1.270873in}}%
\pgfpathlineto{\pgfqpoint{2.583197in}{1.272292in}}%
\pgfpathlineto{\pgfqpoint{2.588211in}{1.274323in}}%
\pgfpathlineto{\pgfqpoint{2.595731in}{1.275099in}}%
\pgfpathlineto{\pgfqpoint{2.598237in}{1.276391in}}%
\pgfpathlineto{\pgfqpoint{2.603251in}{1.277316in}}%
\pgfpathlineto{\pgfqpoint{2.663410in}{1.286341in}}%
\pgfpathlineto{\pgfqpoint{2.690983in}{1.289020in}}%
\pgfpathlineto{\pgfqpoint{2.695997in}{1.290747in}}%
\pgfpathlineto{\pgfqpoint{2.713543in}{1.292469in}}%
\pgfpathlineto{\pgfqpoint{2.756156in}{1.302059in}}%
\pgfpathlineto{\pgfqpoint{2.758663in}{1.305196in}}%
\pgfpathlineto{\pgfqpoint{2.768690in}{1.308119in}}%
\pgfpathlineto{\pgfqpoint{2.773703in}{1.308411in}}%
\pgfpathlineto{\pgfqpoint{2.776210in}{1.310746in}}%
\pgfpathlineto{\pgfqpoint{2.786236in}{1.311991in}}%
\pgfpathlineto{\pgfqpoint{2.788743in}{1.314789in}}%
\pgfpathlineto{\pgfqpoint{2.796263in}{1.316464in}}%
\pgfpathlineto{\pgfqpoint{2.801276in}{1.316707in}}%
\pgfpathlineto{\pgfqpoint{2.808796in}{1.319666in}}%
\pgfpathlineto{\pgfqpoint{2.831356in}{1.322158in}}%
\pgfpathlineto{\pgfqpoint{2.836369in}{1.323520in}}%
\pgfpathlineto{\pgfqpoint{2.876476in}{1.328459in}}%
\pgfpathlineto{\pgfqpoint{2.881489in}{1.329632in}}%
\pgfpathlineto{\pgfqpoint{2.894022in}{1.332118in}}%
\pgfpathlineto{\pgfqpoint{2.904049in}{1.333173in}}%
\pgfpathlineto{\pgfqpoint{2.914076in}{1.334587in}}%
\pgfpathlineto{\pgfqpoint{2.921596in}{1.335277in}}%
\pgfpathlineto{\pgfqpoint{2.939142in}{1.338280in}}%
\pgfpathlineto{\pgfqpoint{2.964209in}{1.340951in}}%
\pgfpathlineto{\pgfqpoint{2.971729in}{1.344473in}}%
\pgfpathlineto{\pgfqpoint{2.976742in}{1.345681in}}%
\pgfpathlineto{\pgfqpoint{2.986769in}{1.349061in}}%
\pgfpathlineto{\pgfqpoint{2.989275in}{1.349084in}}%
\pgfpathlineto{\pgfqpoint{2.991782in}{1.350650in}}%
\pgfpathlineto{\pgfqpoint{3.004315in}{1.352098in}}%
\pgfpathlineto{\pgfqpoint{3.016849in}{1.354048in}}%
\pgfpathlineto{\pgfqpoint{3.031888in}{1.356927in}}%
\pgfpathlineto{\pgfqpoint{3.039408in}{1.360452in}}%
\pgfpathlineto{\pgfqpoint{3.046928in}{1.361853in}}%
\pgfpathlineto{\pgfqpoint{3.051942in}{1.362809in}}%
\pgfpathlineto{\pgfqpoint{3.059462in}{1.364136in}}%
\pgfpathlineto{\pgfqpoint{3.066982in}{1.365550in}}%
\pgfpathlineto{\pgfqpoint{3.074502in}{1.367574in}}%
\pgfpathlineto{\pgfqpoint{3.077008in}{1.367845in}}%
\pgfpathlineto{\pgfqpoint{3.079515in}{1.371028in}}%
\pgfpathlineto{\pgfqpoint{3.082022in}{1.371528in}}%
\pgfpathlineto{\pgfqpoint{3.084528in}{1.374705in}}%
\pgfpathlineto{\pgfqpoint{3.099568in}{1.378020in}}%
\pgfpathlineto{\pgfqpoint{3.119621in}{1.380280in}}%
\pgfpathlineto{\pgfqpoint{3.122128in}{1.382779in}}%
\pgfpathlineto{\pgfqpoint{3.142181in}{1.384399in}}%
\pgfpathlineto{\pgfqpoint{3.144688in}{1.384646in}}%
\pgfpathlineto{\pgfqpoint{3.147195in}{1.389476in}}%
\pgfpathlineto{\pgfqpoint{3.152208in}{1.390148in}}%
\pgfpathlineto{\pgfqpoint{3.159728in}{1.392295in}}%
\pgfpathlineto{\pgfqpoint{3.162234in}{1.394483in}}%
\pgfpathlineto{\pgfqpoint{3.164741in}{1.394890in}}%
\pgfpathlineto{\pgfqpoint{3.169754in}{1.396823in}}%
\pgfpathlineto{\pgfqpoint{3.172261in}{1.396948in}}%
\pgfpathlineto{\pgfqpoint{3.179781in}{1.401267in}}%
\pgfpathlineto{\pgfqpoint{3.184794in}{1.402242in}}%
\pgfpathlineto{\pgfqpoint{3.199834in}{1.403585in}}%
\pgfpathlineto{\pgfqpoint{3.204848in}{1.405068in}}%
\pgfpathlineto{\pgfqpoint{3.214874in}{1.406490in}}%
\pgfpathlineto{\pgfqpoint{3.217381in}{1.406572in}}%
\pgfpathlineto{\pgfqpoint{3.219888in}{1.408566in}}%
\pgfpathlineto{\pgfqpoint{3.227408in}{1.410571in}}%
\pgfpathlineto{\pgfqpoint{3.234927in}{1.415023in}}%
\pgfpathlineto{\pgfqpoint{3.239941in}{1.415733in}}%
\pgfpathlineto{\pgfqpoint{3.244954in}{1.417729in}}%
\pgfpathlineto{\pgfqpoint{3.265007in}{1.420604in}}%
\pgfpathlineto{\pgfqpoint{3.292581in}{1.422493in}}%
\pgfpathlineto{\pgfqpoint{3.297594in}{1.423451in}}%
\pgfpathlineto{\pgfqpoint{3.302607in}{1.424152in}}%
\pgfpathlineto{\pgfqpoint{3.307620in}{1.426337in}}%
\pgfpathlineto{\pgfqpoint{3.312634in}{1.427297in}}%
\pgfpathlineto{\pgfqpoint{3.320154in}{1.436319in}}%
\pgfpathlineto{\pgfqpoint{3.325167in}{1.436792in}}%
\pgfpathlineto{\pgfqpoint{3.330180in}{1.439604in}}%
\pgfpathlineto{\pgfqpoint{3.357754in}{1.449665in}}%
\pgfpathlineto{\pgfqpoint{3.370287in}{1.451760in}}%
\pgfpathlineto{\pgfqpoint{3.382820in}{1.457998in}}%
\pgfpathlineto{\pgfqpoint{3.385327in}{1.464836in}}%
\pgfpathlineto{\pgfqpoint{3.390340in}{1.465061in}}%
\pgfpathlineto{\pgfqpoint{3.395353in}{1.467343in}}%
\pgfpathlineto{\pgfqpoint{3.417913in}{1.473489in}}%
\pgfpathlineto{\pgfqpoint{3.430447in}{1.475593in}}%
\pgfpathlineto{\pgfqpoint{3.432953in}{1.478786in}}%
\pgfpathlineto{\pgfqpoint{3.437966in}{1.480193in}}%
\pgfpathlineto{\pgfqpoint{3.442980in}{1.484126in}}%
\pgfpathlineto{\pgfqpoint{3.475566in}{1.490891in}}%
\pgfpathlineto{\pgfqpoint{3.483086in}{1.492240in}}%
\pgfpathlineto{\pgfqpoint{3.490606in}{1.499077in}}%
\pgfpathlineto{\pgfqpoint{3.493113in}{1.500059in}}%
\pgfpathlineto{\pgfqpoint{3.500633in}{1.507657in}}%
\pgfpathlineto{\pgfqpoint{3.503139in}{1.507875in}}%
\pgfpathlineto{\pgfqpoint{3.505646in}{1.510091in}}%
\pgfpathlineto{\pgfqpoint{3.510659in}{1.511426in}}%
\pgfpathlineto{\pgfqpoint{3.520686in}{1.515826in}}%
\pgfpathlineto{\pgfqpoint{3.523193in}{1.517654in}}%
\pgfpathlineto{\pgfqpoint{3.535726in}{1.520130in}}%
\pgfpathlineto{\pgfqpoint{3.543246in}{1.524668in}}%
\pgfpathlineto{\pgfqpoint{3.553273in}{1.526845in}}%
\pgfpathlineto{\pgfqpoint{3.558286in}{1.529082in}}%
\pgfpathlineto{\pgfqpoint{3.565806in}{1.529888in}}%
\pgfpathlineto{\pgfqpoint{3.568313in}{1.530724in}}%
\pgfpathlineto{\pgfqpoint{3.573326in}{1.534189in}}%
\pgfpathlineto{\pgfqpoint{3.578339in}{1.536107in}}%
\pgfpathlineto{\pgfqpoint{3.585859in}{1.540913in}}%
\pgfpathlineto{\pgfqpoint{3.593379in}{1.543655in}}%
\pgfpathlineto{\pgfqpoint{3.595886in}{1.545361in}}%
\pgfpathlineto{\pgfqpoint{3.598392in}{1.551751in}}%
\pgfpathlineto{\pgfqpoint{3.605912in}{1.555018in}}%
\pgfpathlineto{\pgfqpoint{3.608419in}{1.555052in}}%
\pgfpathlineto{\pgfqpoint{3.610926in}{1.563228in}}%
\pgfpathlineto{\pgfqpoint{3.618446in}{1.569711in}}%
\pgfpathlineto{\pgfqpoint{3.620952in}{1.572970in}}%
\pgfpathlineto{\pgfqpoint{3.625966in}{1.575881in}}%
\pgfpathlineto{\pgfqpoint{3.628472in}{1.578327in}}%
\pgfpathlineto{\pgfqpoint{3.630979in}{1.578488in}}%
\pgfpathlineto{\pgfqpoint{3.635992in}{1.586777in}}%
\pgfpathlineto{\pgfqpoint{3.643512in}{1.591218in}}%
\pgfpathlineto{\pgfqpoint{3.646019in}{1.596248in}}%
\pgfpathlineto{\pgfqpoint{3.648525in}{1.596541in}}%
\pgfpathlineto{\pgfqpoint{3.651032in}{1.598162in}}%
\pgfpathlineto{\pgfqpoint{3.653539in}{1.603641in}}%
\pgfpathlineto{\pgfqpoint{3.656045in}{1.604293in}}%
\pgfpathlineto{\pgfqpoint{3.661059in}{1.609574in}}%
\pgfpathlineto{\pgfqpoint{3.663565in}{1.610122in}}%
\pgfpathlineto{\pgfqpoint{3.668579in}{1.614365in}}%
\pgfpathlineto{\pgfqpoint{3.676099in}{1.616886in}}%
\pgfpathlineto{\pgfqpoint{3.681112in}{1.617553in}}%
\pgfpathlineto{\pgfqpoint{3.683619in}{1.619099in}}%
\pgfpathlineto{\pgfqpoint{3.688632in}{1.624097in}}%
\pgfpathlineto{\pgfqpoint{3.693645in}{1.629791in}}%
\pgfpathlineto{\pgfqpoint{3.698659in}{1.639183in}}%
\pgfpathlineto{\pgfqpoint{3.708685in}{1.641278in}}%
\pgfpathlineto{\pgfqpoint{3.718712in}{1.651000in}}%
\pgfpathlineto{\pgfqpoint{3.721218in}{1.655171in}}%
\pgfpathlineto{\pgfqpoint{3.726232in}{1.659738in}}%
\pgfpathlineto{\pgfqpoint{3.728738in}{1.665133in}}%
\pgfpathlineto{\pgfqpoint{3.738765in}{1.668559in}}%
\pgfpathlineto{\pgfqpoint{3.746285in}{1.676515in}}%
\pgfpathlineto{\pgfqpoint{3.748792in}{1.678359in}}%
\pgfpathlineto{\pgfqpoint{3.751298in}{1.685488in}}%
\pgfpathlineto{\pgfqpoint{3.756312in}{1.688985in}}%
\pgfpathlineto{\pgfqpoint{3.761325in}{1.695067in}}%
\pgfpathlineto{\pgfqpoint{3.768845in}{1.696622in}}%
\pgfpathlineto{\pgfqpoint{3.773858in}{1.698778in}}%
\pgfpathlineto{\pgfqpoint{3.776365in}{1.703564in}}%
\pgfpathlineto{\pgfqpoint{3.781378in}{1.705902in}}%
\pgfpathlineto{\pgfqpoint{3.783885in}{1.710075in}}%
\pgfpathlineto{\pgfqpoint{3.786391in}{1.710628in}}%
\pgfpathlineto{\pgfqpoint{3.793911in}{1.716665in}}%
\pgfpathlineto{\pgfqpoint{3.801431in}{1.719289in}}%
\pgfpathlineto{\pgfqpoint{3.803938in}{1.720584in}}%
\pgfpathlineto{\pgfqpoint{3.808951in}{1.731513in}}%
\pgfpathlineto{\pgfqpoint{3.811458in}{1.744237in}}%
\pgfpathlineto{\pgfqpoint{3.816471in}{1.749889in}}%
\pgfpathlineto{\pgfqpoint{3.826498in}{1.774266in}}%
\pgfpathlineto{\pgfqpoint{3.831511in}{1.775534in}}%
\pgfpathlineto{\pgfqpoint{3.834018in}{1.787022in}}%
\pgfpathlineto{\pgfqpoint{3.836525in}{1.787636in}}%
\pgfpathlineto{\pgfqpoint{3.839031in}{1.792599in}}%
\pgfpathlineto{\pgfqpoint{3.841538in}{1.793671in}}%
\pgfpathlineto{\pgfqpoint{3.844044in}{1.796698in}}%
\pgfpathlineto{\pgfqpoint{3.854071in}{1.801593in}}%
\pgfpathlineto{\pgfqpoint{3.859084in}{1.801962in}}%
\pgfpathlineto{\pgfqpoint{3.866604in}{1.805934in}}%
\pgfpathlineto{\pgfqpoint{3.869111in}{1.806451in}}%
\pgfpathlineto{\pgfqpoint{3.871618in}{1.808513in}}%
\pgfpathlineto{\pgfqpoint{3.874124in}{1.814481in}}%
\pgfpathlineto{\pgfqpoint{3.879138in}{1.816038in}}%
\pgfpathlineto{\pgfqpoint{3.881644in}{1.822938in}}%
\pgfpathlineto{\pgfqpoint{3.884151in}{1.823371in}}%
\pgfpathlineto{\pgfqpoint{3.889164in}{1.826235in}}%
\pgfpathlineto{\pgfqpoint{3.896684in}{1.827689in}}%
\pgfpathlineto{\pgfqpoint{3.899191in}{1.830908in}}%
\pgfpathlineto{\pgfqpoint{3.901698in}{1.838354in}}%
\pgfpathlineto{\pgfqpoint{3.904204in}{1.840676in}}%
\pgfpathlineto{\pgfqpoint{3.906711in}{1.846664in}}%
\pgfpathlineto{\pgfqpoint{3.914231in}{1.853926in}}%
\pgfpathlineto{\pgfqpoint{3.919244in}{1.857262in}}%
\pgfpathlineto{\pgfqpoint{3.924257in}{1.872868in}}%
\pgfpathlineto{\pgfqpoint{3.929271in}{1.873553in}}%
\pgfpathlineto{\pgfqpoint{3.941804in}{1.889712in}}%
\pgfpathlineto{\pgfqpoint{3.944311in}{1.896122in}}%
\pgfpathlineto{\pgfqpoint{3.946817in}{1.898925in}}%
\pgfpathlineto{\pgfqpoint{3.949324in}{1.904166in}}%
\pgfpathlineto{\pgfqpoint{3.954337in}{1.905690in}}%
\pgfpathlineto{\pgfqpoint{3.956844in}{1.906112in}}%
\pgfpathlineto{\pgfqpoint{3.959351in}{1.911529in}}%
\pgfpathlineto{\pgfqpoint{3.964364in}{1.912843in}}%
\pgfpathlineto{\pgfqpoint{3.966871in}{1.916834in}}%
\pgfpathlineto{\pgfqpoint{3.969377in}{1.916911in}}%
\pgfpathlineto{\pgfqpoint{3.974391in}{1.918342in}}%
\pgfpathlineto{\pgfqpoint{3.976897in}{1.940327in}}%
\pgfpathlineto{\pgfqpoint{3.981910in}{1.943285in}}%
\pgfpathlineto{\pgfqpoint{3.984417in}{1.950761in}}%
\pgfpathlineto{\pgfqpoint{3.986924in}{1.953105in}}%
\pgfpathlineto{\pgfqpoint{3.989430in}{1.953592in}}%
\pgfpathlineto{\pgfqpoint{3.991937in}{1.958803in}}%
\pgfpathlineto{\pgfqpoint{3.994444in}{1.968373in}}%
\pgfpathlineto{\pgfqpoint{4.004470in}{1.980769in}}%
\pgfpathlineto{\pgfqpoint{4.006977in}{1.990813in}}%
\pgfpathlineto{\pgfqpoint{4.009484in}{1.993876in}}%
\pgfpathlineto{\pgfqpoint{4.014497in}{2.004722in}}%
\pgfpathlineto{\pgfqpoint{4.017004in}{2.005002in}}%
\pgfpathlineto{\pgfqpoint{4.022017in}{2.017963in}}%
\pgfpathlineto{\pgfqpoint{4.027030in}{2.020950in}}%
\pgfpathlineto{\pgfqpoint{4.034550in}{2.035928in}}%
\pgfpathlineto{\pgfqpoint{4.044577in}{2.082350in}}%
\pgfpathlineto{\pgfqpoint{4.047083in}{2.082532in}}%
\pgfpathlineto{\pgfqpoint{4.052097in}{2.094897in}}%
\pgfpathlineto{\pgfqpoint{4.054603in}{2.098945in}}%
\pgfpathlineto{\pgfqpoint{4.064630in}{2.102602in}}%
\pgfpathlineto{\pgfqpoint{4.074657in}{2.128826in}}%
\pgfpathlineto{\pgfqpoint{4.077163in}{2.129833in}}%
\pgfpathlineto{\pgfqpoint{4.082177in}{2.135514in}}%
\pgfpathlineto{\pgfqpoint{4.084683in}{2.147673in}}%
\pgfpathlineto{\pgfqpoint{4.087190in}{2.148056in}}%
\pgfpathlineto{\pgfqpoint{4.089697in}{2.149839in}}%
\pgfpathlineto{\pgfqpoint{4.092203in}{2.153526in}}%
\pgfpathlineto{\pgfqpoint{4.094710in}{2.154342in}}%
\pgfpathlineto{\pgfqpoint{4.102230in}{2.181170in}}%
\pgfpathlineto{\pgfqpoint{4.107243in}{2.192505in}}%
\pgfpathlineto{\pgfqpoint{4.109750in}{2.202151in}}%
\pgfpathlineto{\pgfqpoint{4.112257in}{2.203512in}}%
\pgfpathlineto{\pgfqpoint{4.114763in}{2.212752in}}%
\pgfpathlineto{\pgfqpoint{4.117270in}{2.213686in}}%
\pgfpathlineto{\pgfqpoint{4.119776in}{2.222331in}}%
\pgfpathlineto{\pgfqpoint{4.122283in}{2.224860in}}%
\pgfpathlineto{\pgfqpoint{4.124790in}{2.224880in}}%
\pgfpathlineto{\pgfqpoint{4.129803in}{2.232349in}}%
\pgfpathlineto{\pgfqpoint{4.134816in}{2.233232in}}%
\pgfpathlineto{\pgfqpoint{4.137323in}{2.235269in}}%
\pgfpathlineto{\pgfqpoint{4.142336in}{2.242992in}}%
\pgfpathlineto{\pgfqpoint{4.144843in}{2.253007in}}%
\pgfpathlineto{\pgfqpoint{4.147350in}{2.253638in}}%
\pgfpathlineto{\pgfqpoint{4.149856in}{2.255688in}}%
\pgfpathlineto{\pgfqpoint{4.152363in}{2.275983in}}%
\pgfpathlineto{\pgfqpoint{4.154870in}{2.276311in}}%
\pgfpathlineto{\pgfqpoint{4.157376in}{2.283134in}}%
\pgfpathlineto{\pgfqpoint{4.159883in}{2.286543in}}%
\pgfpathlineto{\pgfqpoint{4.164896in}{2.287487in}}%
\pgfpathlineto{\pgfqpoint{4.167403in}{2.305586in}}%
\pgfpathlineto{\pgfqpoint{4.169910in}{2.305827in}}%
\pgfpathlineto{\pgfqpoint{4.172416in}{2.307742in}}%
\pgfpathlineto{\pgfqpoint{4.177430in}{2.315319in}}%
\pgfpathlineto{\pgfqpoint{4.182443in}{2.316429in}}%
\pgfpathlineto{\pgfqpoint{4.187456in}{2.327916in}}%
\pgfpathlineto{\pgfqpoint{4.192469in}{2.332968in}}%
\pgfpathlineto{\pgfqpoint{4.194976in}{2.335184in}}%
\pgfpathlineto{\pgfqpoint{4.197483in}{2.339252in}}%
\pgfpathlineto{\pgfqpoint{4.199989in}{2.340390in}}%
\pgfpathlineto{\pgfqpoint{4.202496in}{2.353540in}}%
\pgfpathlineto{\pgfqpoint{4.205003in}{2.359360in}}%
\pgfpathlineto{\pgfqpoint{4.210016in}{2.361037in}}%
\pgfpathlineto{\pgfqpoint{4.212523in}{2.372426in}}%
\pgfpathlineto{\pgfqpoint{4.222549in}{2.385581in}}%
\pgfpathlineto{\pgfqpoint{4.225056in}{2.393721in}}%
\pgfpathlineto{\pgfqpoint{4.227563in}{2.397467in}}%
\pgfpathlineto{\pgfqpoint{4.230069in}{2.398751in}}%
\pgfpathlineto{\pgfqpoint{4.237589in}{2.418552in}}%
\pgfpathlineto{\pgfqpoint{4.242603in}{2.451709in}}%
\pgfpathlineto{\pgfqpoint{4.247616in}{2.458292in}}%
\pgfpathlineto{\pgfqpoint{4.250122in}{2.491593in}}%
\pgfpathlineto{\pgfqpoint{4.257642in}{2.503103in}}%
\pgfpathlineto{\pgfqpoint{4.260149in}{2.508518in}}%
\pgfpathlineto{\pgfqpoint{4.262656in}{2.510266in}}%
\pgfpathlineto{\pgfqpoint{4.265162in}{2.513477in}}%
\pgfpathlineto{\pgfqpoint{4.267669in}{2.522842in}}%
\pgfpathlineto{\pgfqpoint{4.272682in}{2.530076in}}%
\pgfpathlineto{\pgfqpoint{4.275189in}{2.530260in}}%
\pgfpathlineto{\pgfqpoint{4.280202in}{2.546192in}}%
\pgfpathlineto{\pgfqpoint{4.285216in}{2.575609in}}%
\pgfpathlineto{\pgfqpoint{4.287722in}{2.575706in}}%
\pgfpathlineto{\pgfqpoint{4.290229in}{2.594663in}}%
\pgfpathlineto{\pgfqpoint{4.297749in}{2.610940in}}%
\pgfpathlineto{\pgfqpoint{4.300256in}{2.613538in}}%
\pgfpathlineto{\pgfqpoint{4.302762in}{2.620824in}}%
\pgfpathlineto{\pgfqpoint{4.305269in}{2.621283in}}%
\pgfpathlineto{\pgfqpoint{4.307776in}{2.630275in}}%
\pgfpathlineto{\pgfqpoint{4.307776in}{2.630275in}}%
\pgfusepath{stroke}%
\end{pgfscope}%
\begin{pgfscope}%
\pgfpathrectangle{\pgfqpoint{0.708220in}{0.535823in}}{\pgfqpoint{5.013309in}{2.094453in}}%
\pgfusepath{clip}%
\pgfsetbuttcap%
\pgfsetroundjoin%
\pgfsetlinewidth{1.003750pt}%
\definecolor{currentstroke}{rgb}{0.564706,0.564706,1.000000}%
\pgfsetstrokecolor{currentstroke}%
\pgfsetdash{{1.000000pt}{1.650000pt}}{0.000000pt}%
\pgfpathmoveto{\pgfqpoint{0.708220in}{0.693342in}}%
\pgfpathlineto{\pgfqpoint{0.710727in}{0.727066in}}%
\pgfpathlineto{\pgfqpoint{0.713233in}{0.738465in}}%
\pgfpathlineto{\pgfqpoint{0.718246in}{0.748959in}}%
\pgfpathlineto{\pgfqpoint{0.720753in}{0.764285in}}%
\pgfpathlineto{\pgfqpoint{0.723260in}{0.766711in}}%
\pgfpathlineto{\pgfqpoint{0.725766in}{0.767591in}}%
\pgfpathlineto{\pgfqpoint{0.728273in}{0.794265in}}%
\pgfpathlineto{\pgfqpoint{0.730780in}{0.800463in}}%
\pgfpathlineto{\pgfqpoint{0.733286in}{0.801132in}}%
\pgfpathlineto{\pgfqpoint{0.735793in}{0.807965in}}%
\pgfpathlineto{\pgfqpoint{0.740806in}{0.810458in}}%
\pgfpathlineto{\pgfqpoint{0.745820in}{0.813335in}}%
\pgfpathlineto{\pgfqpoint{0.755846in}{0.818113in}}%
\pgfpathlineto{\pgfqpoint{0.758353in}{0.820429in}}%
\pgfpathlineto{\pgfqpoint{0.760860in}{0.820809in}}%
\pgfpathlineto{\pgfqpoint{0.778406in}{0.837598in}}%
\pgfpathlineto{\pgfqpoint{0.783419in}{0.841966in}}%
\pgfpathlineto{\pgfqpoint{0.785926in}{0.845166in}}%
\pgfpathlineto{\pgfqpoint{0.788433in}{0.845528in}}%
\pgfpathlineto{\pgfqpoint{0.790939in}{0.849374in}}%
\pgfpathlineto{\pgfqpoint{0.813499in}{0.856511in}}%
\pgfpathlineto{\pgfqpoint{0.818513in}{0.857391in}}%
\pgfpathlineto{\pgfqpoint{0.833553in}{0.861231in}}%
\pgfpathlineto{\pgfqpoint{0.878672in}{0.867285in}}%
\pgfpathlineto{\pgfqpoint{0.921285in}{0.876742in}}%
\pgfpathlineto{\pgfqpoint{0.928805in}{0.877067in}}%
\pgfpathlineto{\pgfqpoint{0.933819in}{0.879295in}}%
\pgfpathlineto{\pgfqpoint{0.938832in}{0.879525in}}%
\pgfpathlineto{\pgfqpoint{0.943845in}{0.880740in}}%
\pgfpathlineto{\pgfqpoint{1.006512in}{0.890822in}}%
\pgfpathlineto{\pgfqpoint{1.014032in}{0.891478in}}%
\pgfpathlineto{\pgfqpoint{1.034085in}{0.893414in}}%
\pgfpathlineto{\pgfqpoint{1.044112in}{0.894956in}}%
\pgfpathlineto{\pgfqpoint{1.046618in}{0.894968in}}%
\pgfpathlineto{\pgfqpoint{1.051632in}{0.896577in}}%
\pgfpathlineto{\pgfqpoint{1.066671in}{0.897913in}}%
\pgfpathlineto{\pgfqpoint{1.079205in}{0.899607in}}%
\pgfpathlineto{\pgfqpoint{1.101765in}{0.901894in}}%
\pgfpathlineto{\pgfqpoint{1.106778in}{0.902253in}}%
\pgfpathlineto{\pgfqpoint{1.111791in}{0.904018in}}%
\pgfpathlineto{\pgfqpoint{1.116805in}{0.904159in}}%
\pgfpathlineto{\pgfqpoint{1.124324in}{0.907050in}}%
\pgfpathlineto{\pgfqpoint{1.136858in}{0.908229in}}%
\pgfpathlineto{\pgfqpoint{1.164431in}{0.911628in}}%
\pgfpathlineto{\pgfqpoint{1.166938in}{0.913914in}}%
\pgfpathlineto{\pgfqpoint{1.179471in}{0.915780in}}%
\pgfpathlineto{\pgfqpoint{1.186991in}{0.916582in}}%
\pgfpathlineto{\pgfqpoint{1.199524in}{0.919822in}}%
\pgfpathlineto{\pgfqpoint{1.237124in}{0.923603in}}%
\pgfpathlineto{\pgfqpoint{1.249657in}{0.926834in}}%
\pgfpathlineto{\pgfqpoint{1.287257in}{0.931131in}}%
\pgfpathlineto{\pgfqpoint{1.299790in}{0.932750in}}%
\pgfpathlineto{\pgfqpoint{1.312324in}{0.933501in}}%
\pgfpathlineto{\pgfqpoint{1.322350in}{0.935110in}}%
\pgfpathlineto{\pgfqpoint{1.334883in}{0.936311in}}%
\pgfpathlineto{\pgfqpoint{1.354937in}{0.939055in}}%
\pgfpathlineto{\pgfqpoint{1.359950in}{0.940863in}}%
\pgfpathlineto{\pgfqpoint{1.392537in}{0.944915in}}%
\pgfpathlineto{\pgfqpoint{1.425123in}{0.948150in}}%
\pgfpathlineto{\pgfqpoint{1.430136in}{0.949297in}}%
\pgfpathlineto{\pgfqpoint{1.445176in}{0.950413in}}%
\pgfpathlineto{\pgfqpoint{1.452696in}{0.951350in}}%
\pgfpathlineto{\pgfqpoint{1.472749in}{0.952905in}}%
\pgfpathlineto{\pgfqpoint{1.477763in}{0.954711in}}%
\pgfpathlineto{\pgfqpoint{1.482776in}{0.954876in}}%
\pgfpathlineto{\pgfqpoint{1.487789in}{0.956324in}}%
\pgfpathlineto{\pgfqpoint{1.505336in}{0.957242in}}%
\pgfpathlineto{\pgfqpoint{1.512856in}{0.959443in}}%
\pgfpathlineto{\pgfqpoint{1.537922in}{0.961932in}}%
\pgfpathlineto{\pgfqpoint{1.550456in}{0.963660in}}%
\pgfpathlineto{\pgfqpoint{1.562989in}{0.966475in}}%
\pgfpathlineto{\pgfqpoint{1.575522in}{0.968410in}}%
\pgfpathlineto{\pgfqpoint{1.583042in}{0.969356in}}%
\pgfpathlineto{\pgfqpoint{1.593069in}{0.971459in}}%
\pgfpathlineto{\pgfqpoint{1.610615in}{0.972883in}}%
\pgfpathlineto{\pgfqpoint{1.615629in}{0.976050in}}%
\pgfpathlineto{\pgfqpoint{1.618135in}{0.976161in}}%
\pgfpathlineto{\pgfqpoint{1.620642in}{0.978307in}}%
\pgfpathlineto{\pgfqpoint{1.633175in}{0.980175in}}%
\pgfpathlineto{\pgfqpoint{1.638189in}{0.981950in}}%
\pgfpathlineto{\pgfqpoint{1.645709in}{0.983482in}}%
\pgfpathlineto{\pgfqpoint{1.683308in}{0.990029in}}%
\pgfpathlineto{\pgfqpoint{1.688322in}{0.990117in}}%
\pgfpathlineto{\pgfqpoint{1.690828in}{0.991780in}}%
\pgfpathlineto{\pgfqpoint{1.710882in}{0.993524in}}%
\pgfpathlineto{\pgfqpoint{1.720908in}{0.996608in}}%
\pgfpathlineto{\pgfqpoint{1.725922in}{0.997900in}}%
\pgfpathlineto{\pgfqpoint{1.738455in}{0.999636in}}%
\pgfpathlineto{\pgfqpoint{1.750988in}{1.001401in}}%
\pgfpathlineto{\pgfqpoint{1.758508in}{1.003586in}}%
\pgfpathlineto{\pgfqpoint{1.766028in}{1.005042in}}%
\pgfpathlineto{\pgfqpoint{1.773548in}{1.006715in}}%
\pgfpathlineto{\pgfqpoint{1.806134in}{1.013805in}}%
\pgfpathlineto{\pgfqpoint{1.808641in}{1.015959in}}%
\pgfpathlineto{\pgfqpoint{1.821174in}{1.019808in}}%
\pgfpathlineto{\pgfqpoint{1.838721in}{1.027717in}}%
\pgfpathlineto{\pgfqpoint{1.841228in}{1.027785in}}%
\pgfpathlineto{\pgfqpoint{1.846241in}{1.029610in}}%
\pgfpathlineto{\pgfqpoint{1.858774in}{1.031639in}}%
\pgfpathlineto{\pgfqpoint{1.861281in}{1.034628in}}%
\pgfpathlineto{\pgfqpoint{1.863788in}{1.034708in}}%
\pgfpathlineto{\pgfqpoint{1.866294in}{1.036838in}}%
\pgfpathlineto{\pgfqpoint{1.881334in}{1.039245in}}%
\pgfpathlineto{\pgfqpoint{1.898881in}{1.045280in}}%
\pgfpathlineto{\pgfqpoint{1.901387in}{1.046797in}}%
\pgfpathlineto{\pgfqpoint{1.911414in}{1.048418in}}%
\pgfpathlineto{\pgfqpoint{1.913921in}{1.049046in}}%
\pgfpathlineto{\pgfqpoint{1.918934in}{1.052170in}}%
\pgfpathlineto{\pgfqpoint{1.944000in}{1.059418in}}%
\pgfpathlineto{\pgfqpoint{1.949014in}{1.061469in}}%
\pgfpathlineto{\pgfqpoint{1.954027in}{1.062446in}}%
\pgfpathlineto{\pgfqpoint{1.991627in}{1.068530in}}%
\pgfpathlineto{\pgfqpoint{1.999147in}{1.072514in}}%
\pgfpathlineto{\pgfqpoint{2.001654in}{1.072858in}}%
\pgfpathlineto{\pgfqpoint{2.004160in}{1.074976in}}%
\pgfpathlineto{\pgfqpoint{2.006667in}{1.075152in}}%
\pgfpathlineto{\pgfqpoint{2.011680in}{1.078223in}}%
\pgfpathlineto{\pgfqpoint{2.019200in}{1.081101in}}%
\pgfpathlineto{\pgfqpoint{2.024213in}{1.081810in}}%
\pgfpathlineto{\pgfqpoint{2.031733in}{1.085373in}}%
\pgfpathlineto{\pgfqpoint{2.056800in}{1.090915in}}%
\pgfpathlineto{\pgfqpoint{2.066827in}{1.092970in}}%
\pgfpathlineto{\pgfqpoint{2.071840in}{1.095257in}}%
\pgfpathlineto{\pgfqpoint{2.074346in}{1.095354in}}%
\pgfpathlineto{\pgfqpoint{2.076853in}{1.100188in}}%
\pgfpathlineto{\pgfqpoint{2.084373in}{1.103987in}}%
\pgfpathlineto{\pgfqpoint{2.086880in}{1.107664in}}%
\pgfpathlineto{\pgfqpoint{2.096906in}{1.109621in}}%
\pgfpathlineto{\pgfqpoint{2.099413in}{1.114559in}}%
\pgfpathlineto{\pgfqpoint{2.106933in}{1.115642in}}%
\pgfpathlineto{\pgfqpoint{2.114453in}{1.119169in}}%
\pgfpathlineto{\pgfqpoint{2.129493in}{1.121126in}}%
\pgfpathlineto{\pgfqpoint{2.147039in}{1.130768in}}%
\pgfpathlineto{\pgfqpoint{2.162079in}{1.132276in}}%
\pgfpathlineto{\pgfqpoint{2.164586in}{1.134167in}}%
\pgfpathlineto{\pgfqpoint{2.167093in}{1.134217in}}%
\pgfpathlineto{\pgfqpoint{2.189653in}{1.144542in}}%
\pgfpathlineto{\pgfqpoint{2.199679in}{1.145273in}}%
\pgfpathlineto{\pgfqpoint{2.207199in}{1.145890in}}%
\pgfpathlineto{\pgfqpoint{2.214719in}{1.149073in}}%
\pgfpathlineto{\pgfqpoint{2.222239in}{1.152050in}}%
\pgfpathlineto{\pgfqpoint{2.227252in}{1.154532in}}%
\pgfpathlineto{\pgfqpoint{2.232266in}{1.155376in}}%
\pgfpathlineto{\pgfqpoint{2.234772in}{1.158849in}}%
\pgfpathlineto{\pgfqpoint{2.239786in}{1.159472in}}%
\pgfpathlineto{\pgfqpoint{2.242292in}{1.162599in}}%
\pgfpathlineto{\pgfqpoint{2.247306in}{1.162804in}}%
\pgfpathlineto{\pgfqpoint{2.262346in}{1.168225in}}%
\pgfpathlineto{\pgfqpoint{2.274879in}{1.170289in}}%
\pgfpathlineto{\pgfqpoint{2.287412in}{1.174922in}}%
\pgfpathlineto{\pgfqpoint{2.294932in}{1.176198in}}%
\pgfpathlineto{\pgfqpoint{2.307465in}{1.177563in}}%
\pgfpathlineto{\pgfqpoint{2.314985in}{1.179937in}}%
\pgfpathlineto{\pgfqpoint{2.340052in}{1.186158in}}%
\pgfpathlineto{\pgfqpoint{2.352585in}{1.187822in}}%
\pgfpathlineto{\pgfqpoint{2.360105in}{1.189564in}}%
\pgfpathlineto{\pgfqpoint{2.380158in}{1.191505in}}%
\pgfpathlineto{\pgfqpoint{2.397705in}{1.196429in}}%
\pgfpathlineto{\pgfqpoint{2.402718in}{1.196553in}}%
\pgfpathlineto{\pgfqpoint{2.407732in}{1.198144in}}%
\pgfpathlineto{\pgfqpoint{2.412745in}{1.199894in}}%
\pgfpathlineto{\pgfqpoint{2.452851in}{1.214874in}}%
\pgfpathlineto{\pgfqpoint{2.457865in}{1.216034in}}%
\pgfpathlineto{\pgfqpoint{2.465385in}{1.219565in}}%
\pgfpathlineto{\pgfqpoint{2.475411in}{1.223048in}}%
\pgfpathlineto{\pgfqpoint{2.480425in}{1.225288in}}%
\pgfpathlineto{\pgfqpoint{2.497971in}{1.226892in}}%
\pgfpathlineto{\pgfqpoint{2.518024in}{1.233504in}}%
\pgfpathlineto{\pgfqpoint{2.520531in}{1.235820in}}%
\pgfpathlineto{\pgfqpoint{2.528051in}{1.237356in}}%
\pgfpathlineto{\pgfqpoint{2.530558in}{1.239075in}}%
\pgfpathlineto{\pgfqpoint{2.533064in}{1.239235in}}%
\pgfpathlineto{\pgfqpoint{2.535571in}{1.240725in}}%
\pgfpathlineto{\pgfqpoint{2.548104in}{1.241565in}}%
\pgfpathlineto{\pgfqpoint{2.553117in}{1.243505in}}%
\pgfpathlineto{\pgfqpoint{2.568157in}{1.246658in}}%
\pgfpathlineto{\pgfqpoint{2.570664in}{1.248808in}}%
\pgfpathlineto{\pgfqpoint{2.583197in}{1.249870in}}%
\pgfpathlineto{\pgfqpoint{2.588211in}{1.251454in}}%
\pgfpathlineto{\pgfqpoint{2.595731in}{1.252558in}}%
\pgfpathlineto{\pgfqpoint{2.653384in}{1.264559in}}%
\pgfpathlineto{\pgfqpoint{2.655890in}{1.266859in}}%
\pgfpathlineto{\pgfqpoint{2.660904in}{1.267824in}}%
\pgfpathlineto{\pgfqpoint{2.665917in}{1.269830in}}%
\pgfpathlineto{\pgfqpoint{2.673437in}{1.271139in}}%
\pgfpathlineto{\pgfqpoint{2.690983in}{1.274172in}}%
\pgfpathlineto{\pgfqpoint{2.695997in}{1.276088in}}%
\pgfpathlineto{\pgfqpoint{2.708530in}{1.279740in}}%
\pgfpathlineto{\pgfqpoint{2.711037in}{1.281140in}}%
\pgfpathlineto{\pgfqpoint{2.716050in}{1.281459in}}%
\pgfpathlineto{\pgfqpoint{2.721063in}{1.283709in}}%
\pgfpathlineto{\pgfqpoint{2.728583in}{1.284954in}}%
\pgfpathlineto{\pgfqpoint{2.736103in}{1.289601in}}%
\pgfpathlineto{\pgfqpoint{2.743623in}{1.291458in}}%
\pgfpathlineto{\pgfqpoint{2.746130in}{1.293696in}}%
\pgfpathlineto{\pgfqpoint{2.756156in}{1.294483in}}%
\pgfpathlineto{\pgfqpoint{2.763676in}{1.296755in}}%
\pgfpathlineto{\pgfqpoint{2.771196in}{1.297890in}}%
\pgfpathlineto{\pgfqpoint{2.773703in}{1.299410in}}%
\pgfpathlineto{\pgfqpoint{2.793756in}{1.302138in}}%
\pgfpathlineto{\pgfqpoint{2.826343in}{1.309471in}}%
\pgfpathlineto{\pgfqpoint{2.838876in}{1.310855in}}%
\pgfpathlineto{\pgfqpoint{2.841383in}{1.312972in}}%
\pgfpathlineto{\pgfqpoint{2.883996in}{1.321164in}}%
\pgfpathlineto{\pgfqpoint{2.889009in}{1.321958in}}%
\pgfpathlineto{\pgfqpoint{2.894022in}{1.324231in}}%
\pgfpathlineto{\pgfqpoint{2.911569in}{1.325820in}}%
\pgfpathlineto{\pgfqpoint{2.916582in}{1.327880in}}%
\pgfpathlineto{\pgfqpoint{2.934129in}{1.333574in}}%
\pgfpathlineto{\pgfqpoint{2.939142in}{1.334633in}}%
\pgfpathlineto{\pgfqpoint{2.956689in}{1.336840in}}%
\pgfpathlineto{\pgfqpoint{2.964209in}{1.339444in}}%
\pgfpathlineto{\pgfqpoint{2.966715in}{1.343396in}}%
\pgfpathlineto{\pgfqpoint{2.969222in}{1.343913in}}%
\pgfpathlineto{\pgfqpoint{2.971729in}{1.345842in}}%
\pgfpathlineto{\pgfqpoint{2.974235in}{1.346010in}}%
\pgfpathlineto{\pgfqpoint{2.976742in}{1.347996in}}%
\pgfpathlineto{\pgfqpoint{2.996795in}{1.351446in}}%
\pgfpathlineto{\pgfqpoint{3.001809in}{1.352956in}}%
\pgfpathlineto{\pgfqpoint{3.034395in}{1.358067in}}%
\pgfpathlineto{\pgfqpoint{3.036902in}{1.358372in}}%
\pgfpathlineto{\pgfqpoint{3.041915in}{1.360909in}}%
\pgfpathlineto{\pgfqpoint{3.046928in}{1.362256in}}%
\pgfpathlineto{\pgfqpoint{3.051942in}{1.363884in}}%
\pgfpathlineto{\pgfqpoint{3.064475in}{1.366592in}}%
\pgfpathlineto{\pgfqpoint{3.074502in}{1.369276in}}%
\pgfpathlineto{\pgfqpoint{3.079515in}{1.370925in}}%
\pgfpathlineto{\pgfqpoint{3.082022in}{1.375031in}}%
\pgfpathlineto{\pgfqpoint{3.084528in}{1.375467in}}%
\pgfpathlineto{\pgfqpoint{3.087035in}{1.378097in}}%
\pgfpathlineto{\pgfqpoint{3.089542in}{1.378945in}}%
\pgfpathlineto{\pgfqpoint{3.092048in}{1.381240in}}%
\pgfpathlineto{\pgfqpoint{3.104581in}{1.383365in}}%
\pgfpathlineto{\pgfqpoint{3.109595in}{1.384426in}}%
\pgfpathlineto{\pgfqpoint{3.114608in}{1.389622in}}%
\pgfpathlineto{\pgfqpoint{3.122128in}{1.390262in}}%
\pgfpathlineto{\pgfqpoint{3.144688in}{1.394831in}}%
\pgfpathlineto{\pgfqpoint{3.154715in}{1.396420in}}%
\pgfpathlineto{\pgfqpoint{3.162234in}{1.399476in}}%
\pgfpathlineto{\pgfqpoint{3.167248in}{1.400028in}}%
\pgfpathlineto{\pgfqpoint{3.172261in}{1.401317in}}%
\pgfpathlineto{\pgfqpoint{3.184794in}{1.404118in}}%
\pgfpathlineto{\pgfqpoint{3.189808in}{1.406291in}}%
\pgfpathlineto{\pgfqpoint{3.204848in}{1.409533in}}%
\pgfpathlineto{\pgfqpoint{3.207354in}{1.410041in}}%
\pgfpathlineto{\pgfqpoint{3.209861in}{1.412469in}}%
\pgfpathlineto{\pgfqpoint{3.214874in}{1.413654in}}%
\pgfpathlineto{\pgfqpoint{3.217381in}{1.415097in}}%
\pgfpathlineto{\pgfqpoint{3.222394in}{1.415982in}}%
\pgfpathlineto{\pgfqpoint{3.249967in}{1.421990in}}%
\pgfpathlineto{\pgfqpoint{3.265007in}{1.427318in}}%
\pgfpathlineto{\pgfqpoint{3.267514in}{1.429448in}}%
\pgfpathlineto{\pgfqpoint{3.275034in}{1.430111in}}%
\pgfpathlineto{\pgfqpoint{3.280047in}{1.432085in}}%
\pgfpathlineto{\pgfqpoint{3.287567in}{1.433144in}}%
\pgfpathlineto{\pgfqpoint{3.292581in}{1.437355in}}%
\pgfpathlineto{\pgfqpoint{3.300100in}{1.440442in}}%
\pgfpathlineto{\pgfqpoint{3.307620in}{1.441305in}}%
\pgfpathlineto{\pgfqpoint{3.310127in}{1.441692in}}%
\pgfpathlineto{\pgfqpoint{3.312634in}{1.444116in}}%
\pgfpathlineto{\pgfqpoint{3.337700in}{1.448110in}}%
\pgfpathlineto{\pgfqpoint{3.347727in}{1.450036in}}%
\pgfpathlineto{\pgfqpoint{3.352740in}{1.450760in}}%
\pgfpathlineto{\pgfqpoint{3.357754in}{1.452220in}}%
\pgfpathlineto{\pgfqpoint{3.375300in}{1.457078in}}%
\pgfpathlineto{\pgfqpoint{3.385327in}{1.464727in}}%
\pgfpathlineto{\pgfqpoint{3.395353in}{1.467912in}}%
\pgfpathlineto{\pgfqpoint{3.397860in}{1.470018in}}%
\pgfpathlineto{\pgfqpoint{3.402873in}{1.471603in}}%
\pgfpathlineto{\pgfqpoint{3.407887in}{1.472767in}}%
\pgfpathlineto{\pgfqpoint{3.412900in}{1.474136in}}%
\pgfpathlineto{\pgfqpoint{3.415407in}{1.476254in}}%
\pgfpathlineto{\pgfqpoint{3.422927in}{1.477236in}}%
\pgfpathlineto{\pgfqpoint{3.427940in}{1.478355in}}%
\pgfpathlineto{\pgfqpoint{3.435460in}{1.480816in}}%
\pgfpathlineto{\pgfqpoint{3.445486in}{1.482743in}}%
\pgfpathlineto{\pgfqpoint{3.468046in}{1.490967in}}%
\pgfpathlineto{\pgfqpoint{3.470553in}{1.492044in}}%
\pgfpathlineto{\pgfqpoint{3.478073in}{1.500653in}}%
\pgfpathlineto{\pgfqpoint{3.483086in}{1.502405in}}%
\pgfpathlineto{\pgfqpoint{3.493113in}{1.506964in}}%
\pgfpathlineto{\pgfqpoint{3.495620in}{1.507042in}}%
\pgfpathlineto{\pgfqpoint{3.505646in}{1.513049in}}%
\pgfpathlineto{\pgfqpoint{3.510659in}{1.514497in}}%
\pgfpathlineto{\pgfqpoint{3.513166in}{1.514738in}}%
\pgfpathlineto{\pgfqpoint{3.518179in}{1.517508in}}%
\pgfpathlineto{\pgfqpoint{3.520686in}{1.517565in}}%
\pgfpathlineto{\pgfqpoint{3.523193in}{1.519453in}}%
\pgfpathlineto{\pgfqpoint{3.535726in}{1.520640in}}%
\pgfpathlineto{\pgfqpoint{3.538233in}{1.522724in}}%
\pgfpathlineto{\pgfqpoint{3.545753in}{1.524717in}}%
\pgfpathlineto{\pgfqpoint{3.548259in}{1.525689in}}%
\pgfpathlineto{\pgfqpoint{3.555779in}{1.533275in}}%
\pgfpathlineto{\pgfqpoint{3.558286in}{1.533526in}}%
\pgfpathlineto{\pgfqpoint{3.565806in}{1.536933in}}%
\pgfpathlineto{\pgfqpoint{3.568313in}{1.537036in}}%
\pgfpathlineto{\pgfqpoint{3.573326in}{1.539341in}}%
\pgfpathlineto{\pgfqpoint{3.580846in}{1.540068in}}%
\pgfpathlineto{\pgfqpoint{3.588366in}{1.546837in}}%
\pgfpathlineto{\pgfqpoint{3.590872in}{1.548055in}}%
\pgfpathlineto{\pgfqpoint{3.595886in}{1.552630in}}%
\pgfpathlineto{\pgfqpoint{3.598392in}{1.555545in}}%
\pgfpathlineto{\pgfqpoint{3.605912in}{1.557897in}}%
\pgfpathlineto{\pgfqpoint{3.608419in}{1.559643in}}%
\pgfpathlineto{\pgfqpoint{3.610926in}{1.563016in}}%
\pgfpathlineto{\pgfqpoint{3.618446in}{1.563669in}}%
\pgfpathlineto{\pgfqpoint{3.623459in}{1.566725in}}%
\pgfpathlineto{\pgfqpoint{3.625966in}{1.571541in}}%
\pgfpathlineto{\pgfqpoint{3.630979in}{1.572201in}}%
\pgfpathlineto{\pgfqpoint{3.633486in}{1.574979in}}%
\pgfpathlineto{\pgfqpoint{3.635992in}{1.581364in}}%
\pgfpathlineto{\pgfqpoint{3.638499in}{1.582090in}}%
\pgfpathlineto{\pgfqpoint{3.643512in}{1.586391in}}%
\pgfpathlineto{\pgfqpoint{3.646019in}{1.586613in}}%
\pgfpathlineto{\pgfqpoint{3.648525in}{1.593625in}}%
\pgfpathlineto{\pgfqpoint{3.651032in}{1.593821in}}%
\pgfpathlineto{\pgfqpoint{3.656045in}{1.595685in}}%
\pgfpathlineto{\pgfqpoint{3.663565in}{1.609680in}}%
\pgfpathlineto{\pgfqpoint{3.666072in}{1.610405in}}%
\pgfpathlineto{\pgfqpoint{3.671085in}{1.619400in}}%
\pgfpathlineto{\pgfqpoint{3.673592in}{1.619431in}}%
\pgfpathlineto{\pgfqpoint{3.693645in}{1.632362in}}%
\pgfpathlineto{\pgfqpoint{3.696152in}{1.632400in}}%
\pgfpathlineto{\pgfqpoint{3.708685in}{1.643044in}}%
\pgfpathlineto{\pgfqpoint{3.716205in}{1.645505in}}%
\pgfpathlineto{\pgfqpoint{3.726232in}{1.651923in}}%
\pgfpathlineto{\pgfqpoint{3.728738in}{1.655251in}}%
\pgfpathlineto{\pgfqpoint{3.731245in}{1.655497in}}%
\pgfpathlineto{\pgfqpoint{3.738765in}{1.660233in}}%
\pgfpathlineto{\pgfqpoint{3.743778in}{1.660743in}}%
\pgfpathlineto{\pgfqpoint{3.746285in}{1.663124in}}%
\pgfpathlineto{\pgfqpoint{3.748792in}{1.667251in}}%
\pgfpathlineto{\pgfqpoint{3.751298in}{1.667878in}}%
\pgfpathlineto{\pgfqpoint{3.753805in}{1.673766in}}%
\pgfpathlineto{\pgfqpoint{3.756312in}{1.674161in}}%
\pgfpathlineto{\pgfqpoint{3.758818in}{1.676880in}}%
\pgfpathlineto{\pgfqpoint{3.761325in}{1.677108in}}%
\pgfpathlineto{\pgfqpoint{3.773858in}{1.686380in}}%
\pgfpathlineto{\pgfqpoint{3.778871in}{1.686718in}}%
\pgfpathlineto{\pgfqpoint{3.781378in}{1.690920in}}%
\pgfpathlineto{\pgfqpoint{3.786391in}{1.693718in}}%
\pgfpathlineto{\pgfqpoint{3.788898in}{1.705441in}}%
\pgfpathlineto{\pgfqpoint{3.796418in}{1.709247in}}%
\pgfpathlineto{\pgfqpoint{3.798925in}{1.711783in}}%
\pgfpathlineto{\pgfqpoint{3.801431in}{1.716936in}}%
\pgfpathlineto{\pgfqpoint{3.803938in}{1.725017in}}%
\pgfpathlineto{\pgfqpoint{3.806445in}{1.725211in}}%
\pgfpathlineto{\pgfqpoint{3.808951in}{1.729806in}}%
\pgfpathlineto{\pgfqpoint{3.811458in}{1.737957in}}%
\pgfpathlineto{\pgfqpoint{3.813965in}{1.738796in}}%
\pgfpathlineto{\pgfqpoint{3.816471in}{1.742651in}}%
\pgfpathlineto{\pgfqpoint{3.821485in}{1.743665in}}%
\pgfpathlineto{\pgfqpoint{3.831511in}{1.748846in}}%
\pgfpathlineto{\pgfqpoint{3.834018in}{1.748855in}}%
\pgfpathlineto{\pgfqpoint{3.841538in}{1.755636in}}%
\pgfpathlineto{\pgfqpoint{3.844044in}{1.756047in}}%
\pgfpathlineto{\pgfqpoint{3.849058in}{1.758989in}}%
\pgfpathlineto{\pgfqpoint{3.851564in}{1.759901in}}%
\pgfpathlineto{\pgfqpoint{3.856578in}{1.765492in}}%
\pgfpathlineto{\pgfqpoint{3.866604in}{1.769436in}}%
\pgfpathlineto{\pgfqpoint{3.869111in}{1.773321in}}%
\pgfpathlineto{\pgfqpoint{3.871618in}{1.773719in}}%
\pgfpathlineto{\pgfqpoint{3.876631in}{1.781428in}}%
\pgfpathlineto{\pgfqpoint{3.879138in}{1.783187in}}%
\pgfpathlineto{\pgfqpoint{3.881644in}{1.786335in}}%
\pgfpathlineto{\pgfqpoint{3.886658in}{1.787040in}}%
\pgfpathlineto{\pgfqpoint{3.889164in}{1.788613in}}%
\pgfpathlineto{\pgfqpoint{3.891671in}{1.795780in}}%
\pgfpathlineto{\pgfqpoint{3.894178in}{1.795865in}}%
\pgfpathlineto{\pgfqpoint{3.901698in}{1.803091in}}%
\pgfpathlineto{\pgfqpoint{3.904204in}{1.804482in}}%
\pgfpathlineto{\pgfqpoint{3.906711in}{1.807656in}}%
\pgfpathlineto{\pgfqpoint{3.909217in}{1.821948in}}%
\pgfpathlineto{\pgfqpoint{3.911724in}{1.827148in}}%
\pgfpathlineto{\pgfqpoint{3.914231in}{1.827345in}}%
\pgfpathlineto{\pgfqpoint{3.921751in}{1.832666in}}%
\pgfpathlineto{\pgfqpoint{3.926764in}{1.834337in}}%
\pgfpathlineto{\pgfqpoint{3.929271in}{1.841710in}}%
\pgfpathlineto{\pgfqpoint{3.931777in}{1.854430in}}%
\pgfpathlineto{\pgfqpoint{3.944311in}{1.864407in}}%
\pgfpathlineto{\pgfqpoint{3.946817in}{1.866473in}}%
\pgfpathlineto{\pgfqpoint{3.949324in}{1.870314in}}%
\pgfpathlineto{\pgfqpoint{3.951831in}{1.871734in}}%
\pgfpathlineto{\pgfqpoint{3.954337in}{1.871778in}}%
\pgfpathlineto{\pgfqpoint{3.959351in}{1.874452in}}%
\pgfpathlineto{\pgfqpoint{3.964364in}{1.881500in}}%
\pgfpathlineto{\pgfqpoint{3.969377in}{1.883433in}}%
\pgfpathlineto{\pgfqpoint{3.974391in}{1.886471in}}%
\pgfpathlineto{\pgfqpoint{3.979404in}{1.897992in}}%
\pgfpathlineto{\pgfqpoint{3.981910in}{1.898044in}}%
\pgfpathlineto{\pgfqpoint{3.989430in}{1.907124in}}%
\pgfpathlineto{\pgfqpoint{4.001964in}{1.913388in}}%
\pgfpathlineto{\pgfqpoint{4.009484in}{1.936603in}}%
\pgfpathlineto{\pgfqpoint{4.011990in}{1.936763in}}%
\pgfpathlineto{\pgfqpoint{4.014497in}{1.941434in}}%
\pgfpathlineto{\pgfqpoint{4.019510in}{1.942582in}}%
\pgfpathlineto{\pgfqpoint{4.022017in}{1.961781in}}%
\pgfpathlineto{\pgfqpoint{4.024524in}{1.965032in}}%
\pgfpathlineto{\pgfqpoint{4.027030in}{1.965093in}}%
\pgfpathlineto{\pgfqpoint{4.029537in}{1.969354in}}%
\pgfpathlineto{\pgfqpoint{4.034550in}{1.970859in}}%
\pgfpathlineto{\pgfqpoint{4.037057in}{1.972093in}}%
\pgfpathlineto{\pgfqpoint{4.042070in}{1.979891in}}%
\pgfpathlineto{\pgfqpoint{4.044577in}{1.983716in}}%
\pgfpathlineto{\pgfqpoint{4.049590in}{1.986469in}}%
\pgfpathlineto{\pgfqpoint{4.052097in}{1.995780in}}%
\pgfpathlineto{\pgfqpoint{4.054603in}{1.995925in}}%
\pgfpathlineto{\pgfqpoint{4.057110in}{2.000189in}}%
\pgfpathlineto{\pgfqpoint{4.059617in}{2.008940in}}%
\pgfpathlineto{\pgfqpoint{4.062123in}{2.012245in}}%
\pgfpathlineto{\pgfqpoint{4.064630in}{2.019526in}}%
\pgfpathlineto{\pgfqpoint{4.067137in}{2.019755in}}%
\pgfpathlineto{\pgfqpoint{4.069643in}{2.033337in}}%
\pgfpathlineto{\pgfqpoint{4.072150in}{2.039776in}}%
\pgfpathlineto{\pgfqpoint{4.074657in}{2.055527in}}%
\pgfpathlineto{\pgfqpoint{4.077163in}{2.057455in}}%
\pgfpathlineto{\pgfqpoint{4.082177in}{2.066748in}}%
\pgfpathlineto{\pgfqpoint{4.084683in}{2.068410in}}%
\pgfpathlineto{\pgfqpoint{4.099723in}{2.086107in}}%
\pgfpathlineto{\pgfqpoint{4.102230in}{2.087077in}}%
\pgfpathlineto{\pgfqpoint{4.109750in}{2.095286in}}%
\pgfpathlineto{\pgfqpoint{4.119776in}{2.099973in}}%
\pgfpathlineto{\pgfqpoint{4.127296in}{2.109921in}}%
\pgfpathlineto{\pgfqpoint{4.132310in}{2.120447in}}%
\pgfpathlineto{\pgfqpoint{4.134816in}{2.122575in}}%
\pgfpathlineto{\pgfqpoint{4.139830in}{2.132092in}}%
\pgfpathlineto{\pgfqpoint{4.142336in}{2.133680in}}%
\pgfpathlineto{\pgfqpoint{4.144843in}{2.146125in}}%
\pgfpathlineto{\pgfqpoint{4.147350in}{2.149996in}}%
\pgfpathlineto{\pgfqpoint{4.154870in}{2.185941in}}%
\pgfpathlineto{\pgfqpoint{4.157376in}{2.187396in}}%
\pgfpathlineto{\pgfqpoint{4.159883in}{2.190624in}}%
\pgfpathlineto{\pgfqpoint{4.162390in}{2.207625in}}%
\pgfpathlineto{\pgfqpoint{4.164896in}{2.209963in}}%
\pgfpathlineto{\pgfqpoint{4.172416in}{2.220477in}}%
\pgfpathlineto{\pgfqpoint{4.174923in}{2.220824in}}%
\pgfpathlineto{\pgfqpoint{4.179936in}{2.233107in}}%
\pgfpathlineto{\pgfqpoint{4.182443in}{2.234143in}}%
\pgfpathlineto{\pgfqpoint{4.189963in}{2.242362in}}%
\pgfpathlineto{\pgfqpoint{4.192469in}{2.253597in}}%
\pgfpathlineto{\pgfqpoint{4.194976in}{2.254561in}}%
\pgfpathlineto{\pgfqpoint{4.197483in}{2.265372in}}%
\pgfpathlineto{\pgfqpoint{4.202496in}{2.266969in}}%
\pgfpathlineto{\pgfqpoint{4.205003in}{2.267855in}}%
\pgfpathlineto{\pgfqpoint{4.207509in}{2.270606in}}%
\pgfpathlineto{\pgfqpoint{4.210016in}{2.271148in}}%
\pgfpathlineto{\pgfqpoint{4.220043in}{2.278016in}}%
\pgfpathlineto{\pgfqpoint{4.222549in}{2.291787in}}%
\pgfpathlineto{\pgfqpoint{4.225056in}{2.292946in}}%
\pgfpathlineto{\pgfqpoint{4.227563in}{2.302376in}}%
\pgfpathlineto{\pgfqpoint{4.230069in}{2.303231in}}%
\pgfpathlineto{\pgfqpoint{4.240096in}{2.312861in}}%
\pgfpathlineto{\pgfqpoint{4.242603in}{2.321117in}}%
\pgfpathlineto{\pgfqpoint{4.245109in}{2.337306in}}%
\pgfpathlineto{\pgfqpoint{4.247616in}{2.341800in}}%
\pgfpathlineto{\pgfqpoint{4.250122in}{2.343409in}}%
\pgfpathlineto{\pgfqpoint{4.252629in}{2.359084in}}%
\pgfpathlineto{\pgfqpoint{4.255136in}{2.361388in}}%
\pgfpathlineto{\pgfqpoint{4.260149in}{2.377893in}}%
\pgfpathlineto{\pgfqpoint{4.262656in}{2.383166in}}%
\pgfpathlineto{\pgfqpoint{4.265162in}{2.393780in}}%
\pgfpathlineto{\pgfqpoint{4.267669in}{2.396274in}}%
\pgfpathlineto{\pgfqpoint{4.270176in}{2.416751in}}%
\pgfpathlineto{\pgfqpoint{4.272682in}{2.421361in}}%
\pgfpathlineto{\pgfqpoint{4.280202in}{2.456602in}}%
\pgfpathlineto{\pgfqpoint{4.282709in}{2.459414in}}%
\pgfpathlineto{\pgfqpoint{4.287722in}{2.470450in}}%
\pgfpathlineto{\pgfqpoint{4.297749in}{2.492967in}}%
\pgfpathlineto{\pgfqpoint{4.300256in}{2.496857in}}%
\pgfpathlineto{\pgfqpoint{4.302762in}{2.498358in}}%
\pgfpathlineto{\pgfqpoint{4.307776in}{2.506197in}}%
\pgfpathlineto{\pgfqpoint{4.312789in}{2.534356in}}%
\pgfpathlineto{\pgfqpoint{4.315296in}{2.535899in}}%
\pgfpathlineto{\pgfqpoint{4.317802in}{2.539057in}}%
\pgfpathlineto{\pgfqpoint{4.322815in}{2.547283in}}%
\pgfpathlineto{\pgfqpoint{4.327829in}{2.557734in}}%
\pgfpathlineto{\pgfqpoint{4.330335in}{2.565434in}}%
\pgfpathlineto{\pgfqpoint{4.332842in}{2.577661in}}%
\pgfpathlineto{\pgfqpoint{4.335349in}{2.583904in}}%
\pgfpathlineto{\pgfqpoint{4.337855in}{2.586949in}}%
\pgfpathlineto{\pgfqpoint{4.340362in}{2.592221in}}%
\pgfpathlineto{\pgfqpoint{4.345375in}{2.621830in}}%
\pgfpathlineto{\pgfqpoint{4.347882in}{2.630275in}}%
\pgfpathlineto{\pgfqpoint{4.347882in}{2.630275in}}%
\pgfusepath{stroke}%
\end{pgfscope}%
\begin{pgfscope}%
\pgfpathrectangle{\pgfqpoint{0.708220in}{0.535823in}}{\pgfqpoint{5.013309in}{2.094453in}}%
\pgfusepath{clip}%
\pgfsetbuttcap%
\pgfsetroundjoin%
\pgfsetlinewidth{1.003750pt}%
\definecolor{currentstroke}{rgb}{0.564706,0.564706,1.000000}%
\pgfsetstrokecolor{currentstroke}%
\pgfsetdash{{3.700000pt}{1.600000pt}}{0.000000pt}%
\pgfpathmoveto{\pgfqpoint{0.708220in}{0.764047in}}%
\pgfpathlineto{\pgfqpoint{0.710727in}{0.768740in}}%
\pgfpathlineto{\pgfqpoint{0.723260in}{0.816868in}}%
\pgfpathlineto{\pgfqpoint{0.725766in}{0.818899in}}%
\pgfpathlineto{\pgfqpoint{0.728273in}{0.825324in}}%
\pgfpathlineto{\pgfqpoint{0.730780in}{0.828576in}}%
\pgfpathlineto{\pgfqpoint{0.735793in}{0.840043in}}%
\pgfpathlineto{\pgfqpoint{0.738300in}{0.854628in}}%
\pgfpathlineto{\pgfqpoint{0.743313in}{0.855552in}}%
\pgfpathlineto{\pgfqpoint{0.745820in}{0.862483in}}%
\pgfpathlineto{\pgfqpoint{0.750833in}{0.864732in}}%
\pgfpathlineto{\pgfqpoint{0.755846in}{0.866360in}}%
\pgfpathlineto{\pgfqpoint{0.758353in}{0.869150in}}%
\pgfpathlineto{\pgfqpoint{0.760860in}{0.869973in}}%
\pgfpathlineto{\pgfqpoint{0.765873in}{0.873474in}}%
\pgfpathlineto{\pgfqpoint{0.770886in}{0.875876in}}%
\pgfpathlineto{\pgfqpoint{0.775900in}{0.879543in}}%
\pgfpathlineto{\pgfqpoint{0.795953in}{0.887210in}}%
\pgfpathlineto{\pgfqpoint{0.798459in}{0.887352in}}%
\pgfpathlineto{\pgfqpoint{0.800966in}{0.889857in}}%
\pgfpathlineto{\pgfqpoint{0.805979in}{0.890517in}}%
\pgfpathlineto{\pgfqpoint{0.810993in}{0.891873in}}%
\pgfpathlineto{\pgfqpoint{0.833553in}{0.897758in}}%
\pgfpathlineto{\pgfqpoint{0.846086in}{0.899226in}}%
\pgfpathlineto{\pgfqpoint{0.858619in}{0.901438in}}%
\pgfpathlineto{\pgfqpoint{0.863632in}{0.903391in}}%
\pgfpathlineto{\pgfqpoint{0.883686in}{0.905049in}}%
\pgfpathlineto{\pgfqpoint{0.896219in}{0.906318in}}%
\pgfpathlineto{\pgfqpoint{0.898726in}{0.908868in}}%
\pgfpathlineto{\pgfqpoint{0.913766in}{0.912361in}}%
\pgfpathlineto{\pgfqpoint{0.933819in}{0.913488in}}%
\pgfpathlineto{\pgfqpoint{0.953872in}{0.916723in}}%
\pgfpathlineto{\pgfqpoint{1.011525in}{0.923723in}}%
\pgfpathlineto{\pgfqpoint{1.014032in}{0.925823in}}%
\pgfpathlineto{\pgfqpoint{1.059151in}{0.931377in}}%
\pgfpathlineto{\pgfqpoint{1.076698in}{0.933139in}}%
\pgfpathlineto{\pgfqpoint{1.086725in}{0.934222in}}%
\pgfpathlineto{\pgfqpoint{1.091738in}{0.935261in}}%
\pgfpathlineto{\pgfqpoint{1.111791in}{0.937159in}}%
\pgfpathlineto{\pgfqpoint{1.121818in}{0.938123in}}%
\pgfpathlineto{\pgfqpoint{1.134351in}{0.939590in}}%
\pgfpathlineto{\pgfqpoint{1.171951in}{0.944181in}}%
\pgfpathlineto{\pgfqpoint{1.174458in}{0.946012in}}%
\pgfpathlineto{\pgfqpoint{1.189498in}{0.947047in}}%
\pgfpathlineto{\pgfqpoint{1.197017in}{0.948395in}}%
\pgfpathlineto{\pgfqpoint{1.209551in}{0.949605in}}%
\pgfpathlineto{\pgfqpoint{1.217071in}{0.950383in}}%
\pgfpathlineto{\pgfqpoint{1.242137in}{0.951716in}}%
\pgfpathlineto{\pgfqpoint{1.277230in}{0.955339in}}%
\pgfpathlineto{\pgfqpoint{1.282244in}{0.956789in}}%
\pgfpathlineto{\pgfqpoint{1.297284in}{0.957329in}}%
\pgfpathlineto{\pgfqpoint{1.302297in}{0.958750in}}%
\pgfpathlineto{\pgfqpoint{1.322350in}{0.960076in}}%
\pgfpathlineto{\pgfqpoint{1.329870in}{0.961299in}}%
\pgfpathlineto{\pgfqpoint{1.342403in}{0.962427in}}%
\pgfpathlineto{\pgfqpoint{1.387523in}{0.966042in}}%
\pgfpathlineto{\pgfqpoint{1.392537in}{0.967078in}}%
\pgfpathlineto{\pgfqpoint{1.430136in}{0.968695in}}%
\pgfpathlineto{\pgfqpoint{1.457710in}{0.970850in}}%
\pgfpathlineto{\pgfqpoint{1.520376in}{0.976996in}}%
\pgfpathlineto{\pgfqpoint{1.525389in}{0.977627in}}%
\pgfpathlineto{\pgfqpoint{1.547949in}{0.982922in}}%
\pgfpathlineto{\pgfqpoint{1.562989in}{0.984729in}}%
\pgfpathlineto{\pgfqpoint{1.590562in}{0.987942in}}%
\pgfpathlineto{\pgfqpoint{1.595576in}{0.990496in}}%
\pgfpathlineto{\pgfqpoint{1.605602in}{0.991294in}}%
\pgfpathlineto{\pgfqpoint{1.618135in}{0.993396in}}%
\pgfpathlineto{\pgfqpoint{1.623149in}{0.995270in}}%
\pgfpathlineto{\pgfqpoint{1.635682in}{0.997200in}}%
\pgfpathlineto{\pgfqpoint{1.645709in}{0.998084in}}%
\pgfpathlineto{\pgfqpoint{1.660749in}{0.999207in}}%
\pgfpathlineto{\pgfqpoint{1.665762in}{1.000094in}}%
\pgfpathlineto{\pgfqpoint{1.670775in}{1.001483in}}%
\pgfpathlineto{\pgfqpoint{1.683308in}{1.002449in}}%
\pgfpathlineto{\pgfqpoint{1.728428in}{1.011015in}}%
\pgfpathlineto{\pgfqpoint{1.735948in}{1.011888in}}%
\pgfpathlineto{\pgfqpoint{1.740961in}{1.012921in}}%
\pgfpathlineto{\pgfqpoint{1.743468in}{1.013066in}}%
\pgfpathlineto{\pgfqpoint{1.748481in}{1.014900in}}%
\pgfpathlineto{\pgfqpoint{1.776055in}{1.017864in}}%
\pgfpathlineto{\pgfqpoint{1.788588in}{1.019659in}}%
\pgfpathlineto{\pgfqpoint{1.826188in}{1.025438in}}%
\pgfpathlineto{\pgfqpoint{1.851254in}{1.033399in}}%
\pgfpathlineto{\pgfqpoint{1.863788in}{1.034733in}}%
\pgfpathlineto{\pgfqpoint{1.868801in}{1.036011in}}%
\pgfpathlineto{\pgfqpoint{1.873814in}{1.036437in}}%
\pgfpathlineto{\pgfqpoint{1.883841in}{1.038950in}}%
\pgfpathlineto{\pgfqpoint{1.896374in}{1.040760in}}%
\pgfpathlineto{\pgfqpoint{1.898881in}{1.042078in}}%
\pgfpathlineto{\pgfqpoint{1.906401in}{1.043340in}}%
\pgfpathlineto{\pgfqpoint{1.911414in}{1.044673in}}%
\pgfpathlineto{\pgfqpoint{1.913921in}{1.045545in}}%
\pgfpathlineto{\pgfqpoint{1.916427in}{1.047633in}}%
\pgfpathlineto{\pgfqpoint{1.941494in}{1.053282in}}%
\pgfpathlineto{\pgfqpoint{1.949014in}{1.056644in}}%
\pgfpathlineto{\pgfqpoint{1.954027in}{1.057859in}}%
\pgfpathlineto{\pgfqpoint{1.971574in}{1.061483in}}%
\pgfpathlineto{\pgfqpoint{1.994134in}{1.066486in}}%
\pgfpathlineto{\pgfqpoint{2.001654in}{1.067599in}}%
\pgfpathlineto{\pgfqpoint{2.044267in}{1.083202in}}%
\pgfpathlineto{\pgfqpoint{2.049280in}{1.085380in}}%
\pgfpathlineto{\pgfqpoint{2.056800in}{1.089133in}}%
\pgfpathlineto{\pgfqpoint{2.076853in}{1.093615in}}%
\pgfpathlineto{\pgfqpoint{2.079360in}{1.095730in}}%
\pgfpathlineto{\pgfqpoint{2.084373in}{1.095856in}}%
\pgfpathlineto{\pgfqpoint{2.089386in}{1.097721in}}%
\pgfpathlineto{\pgfqpoint{2.091893in}{1.098107in}}%
\pgfpathlineto{\pgfqpoint{2.094400in}{1.102142in}}%
\pgfpathlineto{\pgfqpoint{2.101920in}{1.103614in}}%
\pgfpathlineto{\pgfqpoint{2.106933in}{1.104911in}}%
\pgfpathlineto{\pgfqpoint{2.116960in}{1.107433in}}%
\pgfpathlineto{\pgfqpoint{2.119466in}{1.109263in}}%
\pgfpathlineto{\pgfqpoint{2.144533in}{1.113887in}}%
\pgfpathlineto{\pgfqpoint{2.157066in}{1.120324in}}%
\pgfpathlineto{\pgfqpoint{2.159573in}{1.120354in}}%
\pgfpathlineto{\pgfqpoint{2.162079in}{1.121986in}}%
\pgfpathlineto{\pgfqpoint{2.174613in}{1.123979in}}%
\pgfpathlineto{\pgfqpoint{2.189653in}{1.125490in}}%
\pgfpathlineto{\pgfqpoint{2.202186in}{1.127409in}}%
\pgfpathlineto{\pgfqpoint{2.227252in}{1.135383in}}%
\pgfpathlineto{\pgfqpoint{2.229759in}{1.136663in}}%
\pgfpathlineto{\pgfqpoint{2.232266in}{1.139391in}}%
\pgfpathlineto{\pgfqpoint{2.239786in}{1.141512in}}%
\pgfpathlineto{\pgfqpoint{2.242292in}{1.141796in}}%
\pgfpathlineto{\pgfqpoint{2.249812in}{1.145403in}}%
\pgfpathlineto{\pgfqpoint{2.279892in}{1.151632in}}%
\pgfpathlineto{\pgfqpoint{2.284905in}{1.155457in}}%
\pgfpathlineto{\pgfqpoint{2.289919in}{1.158653in}}%
\pgfpathlineto{\pgfqpoint{2.292425in}{1.161420in}}%
\pgfpathlineto{\pgfqpoint{2.299945in}{1.162649in}}%
\pgfpathlineto{\pgfqpoint{2.307465in}{1.166491in}}%
\pgfpathlineto{\pgfqpoint{2.360105in}{1.173889in}}%
\pgfpathlineto{\pgfqpoint{2.365118in}{1.175723in}}%
\pgfpathlineto{\pgfqpoint{2.367625in}{1.176050in}}%
\pgfpathlineto{\pgfqpoint{2.370132in}{1.178341in}}%
\pgfpathlineto{\pgfqpoint{2.377652in}{1.179033in}}%
\pgfpathlineto{\pgfqpoint{2.385172in}{1.183064in}}%
\pgfpathlineto{\pgfqpoint{2.390185in}{1.183505in}}%
\pgfpathlineto{\pgfqpoint{2.395198in}{1.186416in}}%
\pgfpathlineto{\pgfqpoint{2.410238in}{1.189977in}}%
\pgfpathlineto{\pgfqpoint{2.412745in}{1.193439in}}%
\pgfpathlineto{\pgfqpoint{2.420265in}{1.194793in}}%
\pgfpathlineto{\pgfqpoint{2.430291in}{1.196848in}}%
\pgfpathlineto{\pgfqpoint{2.435305in}{1.197022in}}%
\pgfpathlineto{\pgfqpoint{2.437811in}{1.198970in}}%
\pgfpathlineto{\pgfqpoint{2.445331in}{1.200189in}}%
\pgfpathlineto{\pgfqpoint{2.450345in}{1.204963in}}%
\pgfpathlineto{\pgfqpoint{2.487944in}{1.218061in}}%
\pgfpathlineto{\pgfqpoint{2.495464in}{1.222357in}}%
\pgfpathlineto{\pgfqpoint{2.510504in}{1.225296in}}%
\pgfpathlineto{\pgfqpoint{2.515518in}{1.227451in}}%
\pgfpathlineto{\pgfqpoint{2.525544in}{1.234205in}}%
\pgfpathlineto{\pgfqpoint{2.545598in}{1.237797in}}%
\pgfpathlineto{\pgfqpoint{2.548104in}{1.239438in}}%
\pgfpathlineto{\pgfqpoint{2.555624in}{1.241303in}}%
\pgfpathlineto{\pgfqpoint{2.560637in}{1.243930in}}%
\pgfpathlineto{\pgfqpoint{2.565651in}{1.245948in}}%
\pgfpathlineto{\pgfqpoint{2.570664in}{1.246662in}}%
\pgfpathlineto{\pgfqpoint{2.573171in}{1.248718in}}%
\pgfpathlineto{\pgfqpoint{2.575677in}{1.249104in}}%
\pgfpathlineto{\pgfqpoint{2.578184in}{1.251872in}}%
\pgfpathlineto{\pgfqpoint{2.580691in}{1.252717in}}%
\pgfpathlineto{\pgfqpoint{2.583197in}{1.258811in}}%
\pgfpathlineto{\pgfqpoint{2.585704in}{1.260371in}}%
\pgfpathlineto{\pgfqpoint{2.590717in}{1.260923in}}%
\pgfpathlineto{\pgfqpoint{2.600744in}{1.270295in}}%
\pgfpathlineto{\pgfqpoint{2.608264in}{1.271893in}}%
\pgfpathlineto{\pgfqpoint{2.610771in}{1.276154in}}%
\pgfpathlineto{\pgfqpoint{2.613277in}{1.277274in}}%
\pgfpathlineto{\pgfqpoint{2.615784in}{1.282441in}}%
\pgfpathlineto{\pgfqpoint{2.618290in}{1.282495in}}%
\pgfpathlineto{\pgfqpoint{2.625810in}{1.286608in}}%
\pgfpathlineto{\pgfqpoint{2.628317in}{1.286778in}}%
\pgfpathlineto{\pgfqpoint{2.630824in}{1.290983in}}%
\pgfpathlineto{\pgfqpoint{2.635837in}{1.292737in}}%
\pgfpathlineto{\pgfqpoint{2.638344in}{1.294519in}}%
\pgfpathlineto{\pgfqpoint{2.640850in}{1.294687in}}%
\pgfpathlineto{\pgfqpoint{2.645864in}{1.297518in}}%
\pgfpathlineto{\pgfqpoint{2.648370in}{1.297655in}}%
\pgfpathlineto{\pgfqpoint{2.650877in}{1.300615in}}%
\pgfpathlineto{\pgfqpoint{2.655890in}{1.302171in}}%
\pgfpathlineto{\pgfqpoint{2.665917in}{1.306427in}}%
\pgfpathlineto{\pgfqpoint{2.670930in}{1.307184in}}%
\pgfpathlineto{\pgfqpoint{2.678450in}{1.309246in}}%
\pgfpathlineto{\pgfqpoint{2.680957in}{1.311338in}}%
\pgfpathlineto{\pgfqpoint{2.706023in}{1.313648in}}%
\pgfpathlineto{\pgfqpoint{2.708530in}{1.315386in}}%
\pgfpathlineto{\pgfqpoint{2.711037in}{1.319426in}}%
\pgfpathlineto{\pgfqpoint{2.723570in}{1.321111in}}%
\pgfpathlineto{\pgfqpoint{2.726077in}{1.325295in}}%
\pgfpathlineto{\pgfqpoint{2.751143in}{1.331568in}}%
\pgfpathlineto{\pgfqpoint{2.763676in}{1.334098in}}%
\pgfpathlineto{\pgfqpoint{2.773703in}{1.336883in}}%
\pgfpathlineto{\pgfqpoint{2.778716in}{1.342585in}}%
\pgfpathlineto{\pgfqpoint{2.781223in}{1.343725in}}%
\pgfpathlineto{\pgfqpoint{2.786236in}{1.347623in}}%
\pgfpathlineto{\pgfqpoint{2.791250in}{1.349695in}}%
\pgfpathlineto{\pgfqpoint{2.803783in}{1.352380in}}%
\pgfpathlineto{\pgfqpoint{2.808796in}{1.353674in}}%
\pgfpathlineto{\pgfqpoint{2.813810in}{1.354967in}}%
\pgfpathlineto{\pgfqpoint{2.826343in}{1.356492in}}%
\pgfpathlineto{\pgfqpoint{2.841383in}{1.361750in}}%
\pgfpathlineto{\pgfqpoint{2.851409in}{1.363518in}}%
\pgfpathlineto{\pgfqpoint{2.853916in}{1.366702in}}%
\pgfpathlineto{\pgfqpoint{2.863943in}{1.368532in}}%
\pgfpathlineto{\pgfqpoint{2.868956in}{1.369086in}}%
\pgfpathlineto{\pgfqpoint{2.871463in}{1.371839in}}%
\pgfpathlineto{\pgfqpoint{2.886503in}{1.373497in}}%
\pgfpathlineto{\pgfqpoint{2.909062in}{1.381875in}}%
\pgfpathlineto{\pgfqpoint{2.914076in}{1.383321in}}%
\pgfpathlineto{\pgfqpoint{2.919089in}{1.386148in}}%
\pgfpathlineto{\pgfqpoint{2.926609in}{1.390304in}}%
\pgfpathlineto{\pgfqpoint{2.934129in}{1.392214in}}%
\pgfpathlineto{\pgfqpoint{2.939142in}{1.399152in}}%
\pgfpathlineto{\pgfqpoint{2.954182in}{1.400989in}}%
\pgfpathlineto{\pgfqpoint{2.961702in}{1.406037in}}%
\pgfpathlineto{\pgfqpoint{2.974235in}{1.411054in}}%
\pgfpathlineto{\pgfqpoint{2.979249in}{1.412037in}}%
\pgfpathlineto{\pgfqpoint{2.981755in}{1.414299in}}%
\pgfpathlineto{\pgfqpoint{3.021862in}{1.421182in}}%
\pgfpathlineto{\pgfqpoint{3.026875in}{1.421993in}}%
\pgfpathlineto{\pgfqpoint{3.031888in}{1.423560in}}%
\pgfpathlineto{\pgfqpoint{3.036902in}{1.424072in}}%
\pgfpathlineto{\pgfqpoint{3.039408in}{1.428710in}}%
\pgfpathlineto{\pgfqpoint{3.041915in}{1.428837in}}%
\pgfpathlineto{\pgfqpoint{3.044422in}{1.430965in}}%
\pgfpathlineto{\pgfqpoint{3.051942in}{1.432210in}}%
\pgfpathlineto{\pgfqpoint{3.054448in}{1.432551in}}%
\pgfpathlineto{\pgfqpoint{3.056955in}{1.437992in}}%
\pgfpathlineto{\pgfqpoint{3.066982in}{1.439568in}}%
\pgfpathlineto{\pgfqpoint{3.069488in}{1.440987in}}%
\pgfpathlineto{\pgfqpoint{3.071995in}{1.445014in}}%
\pgfpathlineto{\pgfqpoint{3.082022in}{1.445311in}}%
\pgfpathlineto{\pgfqpoint{3.084528in}{1.447453in}}%
\pgfpathlineto{\pgfqpoint{3.087035in}{1.447720in}}%
\pgfpathlineto{\pgfqpoint{3.107088in}{1.460459in}}%
\pgfpathlineto{\pgfqpoint{3.109595in}{1.460466in}}%
\pgfpathlineto{\pgfqpoint{3.114608in}{1.464346in}}%
\pgfpathlineto{\pgfqpoint{3.117115in}{1.464629in}}%
\pgfpathlineto{\pgfqpoint{3.119621in}{1.468430in}}%
\pgfpathlineto{\pgfqpoint{3.124635in}{1.469759in}}%
\pgfpathlineto{\pgfqpoint{3.127141in}{1.474723in}}%
\pgfpathlineto{\pgfqpoint{3.129648in}{1.475563in}}%
\pgfpathlineto{\pgfqpoint{3.132155in}{1.479119in}}%
\pgfpathlineto{\pgfqpoint{3.139675in}{1.482106in}}%
\pgfpathlineto{\pgfqpoint{3.144688in}{1.487718in}}%
\pgfpathlineto{\pgfqpoint{3.147195in}{1.488021in}}%
\pgfpathlineto{\pgfqpoint{3.149701in}{1.493116in}}%
\pgfpathlineto{\pgfqpoint{3.157221in}{1.494677in}}%
\pgfpathlineto{\pgfqpoint{3.159728in}{1.497527in}}%
\pgfpathlineto{\pgfqpoint{3.162234in}{1.503612in}}%
\pgfpathlineto{\pgfqpoint{3.169754in}{1.509318in}}%
\pgfpathlineto{\pgfqpoint{3.182288in}{1.522362in}}%
\pgfpathlineto{\pgfqpoint{3.184794in}{1.522372in}}%
\pgfpathlineto{\pgfqpoint{3.187301in}{1.528866in}}%
\pgfpathlineto{\pgfqpoint{3.189808in}{1.528953in}}%
\pgfpathlineto{\pgfqpoint{3.192314in}{1.532180in}}%
\pgfpathlineto{\pgfqpoint{3.194821in}{1.532486in}}%
\pgfpathlineto{\pgfqpoint{3.197328in}{1.534881in}}%
\pgfpathlineto{\pgfqpoint{3.199834in}{1.534977in}}%
\pgfpathlineto{\pgfqpoint{3.204848in}{1.537019in}}%
\pgfpathlineto{\pgfqpoint{3.207354in}{1.537553in}}%
\pgfpathlineto{\pgfqpoint{3.209861in}{1.540718in}}%
\pgfpathlineto{\pgfqpoint{3.212368in}{1.540889in}}%
\pgfpathlineto{\pgfqpoint{3.214874in}{1.542466in}}%
\pgfpathlineto{\pgfqpoint{3.222394in}{1.553322in}}%
\pgfpathlineto{\pgfqpoint{3.227408in}{1.606048in}}%
\pgfpathlineto{\pgfqpoint{3.237434in}{1.614414in}}%
\pgfpathlineto{\pgfqpoint{3.285061in}{1.618291in}}%
\pgfpathlineto{\pgfqpoint{3.320154in}{1.619683in}}%
\pgfpathlineto{\pgfqpoint{3.325167in}{1.620295in}}%
\pgfpathlineto{\pgfqpoint{3.350234in}{1.621683in}}%
\pgfpathlineto{\pgfqpoint{3.417913in}{1.625220in}}%
\pgfpathlineto{\pgfqpoint{3.425433in}{1.626628in}}%
\pgfpathlineto{\pgfqpoint{3.435460in}{1.628008in}}%
\pgfpathlineto{\pgfqpoint{3.455513in}{1.631499in}}%
\pgfpathlineto{\pgfqpoint{3.460526in}{1.632491in}}%
\pgfpathlineto{\pgfqpoint{3.478073in}{1.634970in}}%
\pgfpathlineto{\pgfqpoint{3.485593in}{1.636019in}}%
\pgfpathlineto{\pgfqpoint{3.500633in}{1.637782in}}%
\pgfpathlineto{\pgfqpoint{3.505646in}{1.639352in}}%
\pgfpathlineto{\pgfqpoint{3.513166in}{1.640812in}}%
\pgfpathlineto{\pgfqpoint{3.515673in}{1.642247in}}%
\pgfpathlineto{\pgfqpoint{3.523193in}{1.642407in}}%
\pgfpathlineto{\pgfqpoint{3.525699in}{1.644687in}}%
\pgfpathlineto{\pgfqpoint{3.548259in}{1.649796in}}%
\pgfpathlineto{\pgfqpoint{3.575832in}{1.652905in}}%
\pgfpathlineto{\pgfqpoint{3.580846in}{1.653967in}}%
\pgfpathlineto{\pgfqpoint{3.598392in}{1.656800in}}%
\pgfpathlineto{\pgfqpoint{3.613432in}{1.658136in}}%
\pgfpathlineto{\pgfqpoint{3.643512in}{1.660763in}}%
\pgfpathlineto{\pgfqpoint{3.656045in}{1.661575in}}%
\pgfpathlineto{\pgfqpoint{3.688632in}{1.668035in}}%
\pgfpathlineto{\pgfqpoint{3.713698in}{1.671727in}}%
\pgfpathlineto{\pgfqpoint{3.721218in}{1.672955in}}%
\pgfpathlineto{\pgfqpoint{3.726232in}{1.673733in}}%
\pgfpathlineto{\pgfqpoint{3.733752in}{1.674860in}}%
\pgfpathlineto{\pgfqpoint{3.748792in}{1.676394in}}%
\pgfpathlineto{\pgfqpoint{3.758818in}{1.678171in}}%
\pgfpathlineto{\pgfqpoint{3.771352in}{1.681195in}}%
\pgfpathlineto{\pgfqpoint{3.783885in}{1.682218in}}%
\pgfpathlineto{\pgfqpoint{3.798925in}{1.685824in}}%
\pgfpathlineto{\pgfqpoint{3.823991in}{1.690988in}}%
\pgfpathlineto{\pgfqpoint{3.844044in}{1.697804in}}%
\pgfpathlineto{\pgfqpoint{3.854071in}{1.699270in}}%
\pgfpathlineto{\pgfqpoint{3.856578in}{1.700957in}}%
\pgfpathlineto{\pgfqpoint{3.861591in}{1.701810in}}%
\pgfpathlineto{\pgfqpoint{3.866604in}{1.703382in}}%
\pgfpathlineto{\pgfqpoint{3.876631in}{1.705506in}}%
\pgfpathlineto{\pgfqpoint{3.881644in}{1.706333in}}%
\pgfpathlineto{\pgfqpoint{3.884151in}{1.710188in}}%
\pgfpathlineto{\pgfqpoint{3.886658in}{1.711481in}}%
\pgfpathlineto{\pgfqpoint{3.889164in}{1.714404in}}%
\pgfpathlineto{\pgfqpoint{3.891671in}{1.720148in}}%
\pgfpathlineto{\pgfqpoint{3.899191in}{1.721330in}}%
\pgfpathlineto{\pgfqpoint{3.904204in}{1.723451in}}%
\pgfpathlineto{\pgfqpoint{3.926764in}{1.728308in}}%
\pgfpathlineto{\pgfqpoint{3.934284in}{1.732513in}}%
\pgfpathlineto{\pgfqpoint{3.939297in}{1.733384in}}%
\pgfpathlineto{\pgfqpoint{3.946817in}{1.734533in}}%
\pgfpathlineto{\pgfqpoint{3.954337in}{1.736543in}}%
\pgfpathlineto{\pgfqpoint{3.961857in}{1.738600in}}%
\pgfpathlineto{\pgfqpoint{3.969377in}{1.739942in}}%
\pgfpathlineto{\pgfqpoint{3.974391in}{1.742996in}}%
\pgfpathlineto{\pgfqpoint{3.976897in}{1.743130in}}%
\pgfpathlineto{\pgfqpoint{3.981910in}{1.747022in}}%
\pgfpathlineto{\pgfqpoint{3.989430in}{1.749254in}}%
\pgfpathlineto{\pgfqpoint{4.001964in}{1.750323in}}%
\pgfpathlineto{\pgfqpoint{4.004470in}{1.753107in}}%
\pgfpathlineto{\pgfqpoint{4.017004in}{1.754983in}}%
\pgfpathlineto{\pgfqpoint{4.024524in}{1.756890in}}%
\pgfpathlineto{\pgfqpoint{4.027030in}{1.760036in}}%
\pgfpathlineto{\pgfqpoint{4.032044in}{1.760278in}}%
\pgfpathlineto{\pgfqpoint{4.037057in}{1.763638in}}%
\pgfpathlineto{\pgfqpoint{4.042070in}{1.765342in}}%
\pgfpathlineto{\pgfqpoint{4.044577in}{1.768425in}}%
\pgfpathlineto{\pgfqpoint{4.047083in}{1.774851in}}%
\pgfpathlineto{\pgfqpoint{4.052097in}{1.777698in}}%
\pgfpathlineto{\pgfqpoint{4.054603in}{1.781795in}}%
\pgfpathlineto{\pgfqpoint{4.057110in}{1.783436in}}%
\pgfpathlineto{\pgfqpoint{4.059617in}{1.788603in}}%
\pgfpathlineto{\pgfqpoint{4.064630in}{1.789162in}}%
\pgfpathlineto{\pgfqpoint{4.067137in}{1.790172in}}%
\pgfpathlineto{\pgfqpoint{4.069643in}{1.794265in}}%
\pgfpathlineto{\pgfqpoint{4.072150in}{1.802676in}}%
\pgfpathlineto{\pgfqpoint{4.082177in}{1.806304in}}%
\pgfpathlineto{\pgfqpoint{4.087190in}{1.809885in}}%
\pgfpathlineto{\pgfqpoint{4.092203in}{1.810843in}}%
\pgfpathlineto{\pgfqpoint{4.102230in}{1.825331in}}%
\pgfpathlineto{\pgfqpoint{4.107243in}{1.827056in}}%
\pgfpathlineto{\pgfqpoint{4.112257in}{1.834696in}}%
\pgfpathlineto{\pgfqpoint{4.114763in}{1.836333in}}%
\pgfpathlineto{\pgfqpoint{4.117270in}{1.836501in}}%
\pgfpathlineto{\pgfqpoint{4.129803in}{1.842933in}}%
\pgfpathlineto{\pgfqpoint{4.134816in}{1.854853in}}%
\pgfpathlineto{\pgfqpoint{4.139830in}{1.857227in}}%
\pgfpathlineto{\pgfqpoint{4.142336in}{1.860053in}}%
\pgfpathlineto{\pgfqpoint{4.144843in}{1.860128in}}%
\pgfpathlineto{\pgfqpoint{4.147350in}{1.865186in}}%
\pgfpathlineto{\pgfqpoint{4.149856in}{1.873848in}}%
\pgfpathlineto{\pgfqpoint{4.152363in}{1.877607in}}%
\pgfpathlineto{\pgfqpoint{4.154870in}{1.887467in}}%
\pgfpathlineto{\pgfqpoint{4.159883in}{1.887925in}}%
\pgfpathlineto{\pgfqpoint{4.164896in}{1.896302in}}%
\pgfpathlineto{\pgfqpoint{4.167403in}{1.897425in}}%
\pgfpathlineto{\pgfqpoint{4.169910in}{1.901523in}}%
\pgfpathlineto{\pgfqpoint{4.172416in}{1.903416in}}%
\pgfpathlineto{\pgfqpoint{4.174923in}{1.907974in}}%
\pgfpathlineto{\pgfqpoint{4.179936in}{1.911971in}}%
\pgfpathlineto{\pgfqpoint{4.187456in}{1.935334in}}%
\pgfpathlineto{\pgfqpoint{4.189963in}{1.937542in}}%
\pgfpathlineto{\pgfqpoint{4.192469in}{1.946238in}}%
\pgfpathlineto{\pgfqpoint{4.197483in}{1.950747in}}%
\pgfpathlineto{\pgfqpoint{4.199989in}{1.950934in}}%
\pgfpathlineto{\pgfqpoint{4.202496in}{1.953360in}}%
\pgfpathlineto{\pgfqpoint{4.205003in}{1.958394in}}%
\pgfpathlineto{\pgfqpoint{4.207509in}{1.959346in}}%
\pgfpathlineto{\pgfqpoint{4.210016in}{1.968066in}}%
\pgfpathlineto{\pgfqpoint{4.212523in}{1.969220in}}%
\pgfpathlineto{\pgfqpoint{4.215029in}{1.976859in}}%
\pgfpathlineto{\pgfqpoint{4.220043in}{1.977512in}}%
\pgfpathlineto{\pgfqpoint{4.222549in}{1.982487in}}%
\pgfpathlineto{\pgfqpoint{4.225056in}{1.983496in}}%
\pgfpathlineto{\pgfqpoint{4.230069in}{1.988306in}}%
\pgfpathlineto{\pgfqpoint{4.232576in}{1.995718in}}%
\pgfpathlineto{\pgfqpoint{4.235083in}{1.995925in}}%
\pgfpathlineto{\pgfqpoint{4.237589in}{2.000481in}}%
\pgfpathlineto{\pgfqpoint{4.240096in}{2.002192in}}%
\pgfpathlineto{\pgfqpoint{4.245109in}{2.014258in}}%
\pgfpathlineto{\pgfqpoint{4.247616in}{2.038997in}}%
\pgfpathlineto{\pgfqpoint{4.250122in}{2.048097in}}%
\pgfpathlineto{\pgfqpoint{4.252629in}{2.048180in}}%
\pgfpathlineto{\pgfqpoint{4.255136in}{2.060054in}}%
\pgfpathlineto{\pgfqpoint{4.260149in}{2.066710in}}%
\pgfpathlineto{\pgfqpoint{4.265162in}{2.074308in}}%
\pgfpathlineto{\pgfqpoint{4.272682in}{2.112706in}}%
\pgfpathlineto{\pgfqpoint{4.275189in}{2.116744in}}%
\pgfpathlineto{\pgfqpoint{4.277696in}{2.139460in}}%
\pgfpathlineto{\pgfqpoint{4.280202in}{2.144178in}}%
\pgfpathlineto{\pgfqpoint{4.290229in}{2.203081in}}%
\pgfpathlineto{\pgfqpoint{4.292736in}{2.204885in}}%
\pgfpathlineto{\pgfqpoint{4.305269in}{2.250468in}}%
\pgfpathlineto{\pgfqpoint{4.307776in}{2.253410in}}%
\pgfpathlineto{\pgfqpoint{4.310282in}{2.266959in}}%
\pgfpathlineto{\pgfqpoint{4.312789in}{2.272256in}}%
\pgfpathlineto{\pgfqpoint{4.317802in}{2.286751in}}%
\pgfpathlineto{\pgfqpoint{4.320309in}{2.303031in}}%
\pgfpathlineto{\pgfqpoint{4.322815in}{2.310919in}}%
\pgfpathlineto{\pgfqpoint{4.327829in}{2.341431in}}%
\pgfpathlineto{\pgfqpoint{4.330335in}{2.369595in}}%
\pgfpathlineto{\pgfqpoint{4.332842in}{2.369729in}}%
\pgfpathlineto{\pgfqpoint{4.335349in}{2.378760in}}%
\pgfpathlineto{\pgfqpoint{4.337855in}{2.402063in}}%
\pgfpathlineto{\pgfqpoint{4.345375in}{2.532357in}}%
\pgfpathlineto{\pgfqpoint{4.347882in}{2.533614in}}%
\pgfpathlineto{\pgfqpoint{4.350389in}{2.586224in}}%
\pgfpathlineto{\pgfqpoint{4.352895in}{2.598325in}}%
\pgfpathlineto{\pgfqpoint{4.355402in}{2.598686in}}%
\pgfpathlineto{\pgfqpoint{4.357909in}{2.610935in}}%
\pgfpathlineto{\pgfqpoint{4.360415in}{2.612654in}}%
\pgfpathlineto{\pgfqpoint{4.362922in}{2.624826in}}%
\pgfpathlineto{\pgfqpoint{4.367935in}{2.630275in}}%
\pgfpathlineto{\pgfqpoint{4.367935in}{2.630275in}}%
\pgfusepath{stroke}%
\end{pgfscope}%
\begin{pgfscope}%
\pgfpathrectangle{\pgfqpoint{0.708220in}{0.535823in}}{\pgfqpoint{5.013309in}{2.094453in}}%
\pgfusepath{clip}%
\pgfsetrectcap%
\pgfsetroundjoin%
\pgfsetlinewidth{1.003750pt}%
\definecolor{currentstroke}{rgb}{0.811765,0.125490,0.125490}%
\pgfsetstrokecolor{currentstroke}%
\pgfsetdash{}{0pt}%
\pgfpathmoveto{\pgfqpoint{0.708220in}{0.973285in}}%
\pgfpathlineto{\pgfqpoint{0.710727in}{0.975945in}}%
\pgfpathlineto{\pgfqpoint{0.718246in}{0.975945in}}%
\pgfpathlineto{\pgfqpoint{0.720753in}{0.978565in}}%
\pgfpathlineto{\pgfqpoint{0.823526in}{0.978565in}}%
\pgfpathlineto{\pgfqpoint{0.826033in}{0.981145in}}%
\pgfpathlineto{\pgfqpoint{1.014032in}{0.981145in}}%
\pgfpathlineto{\pgfqpoint{1.016538in}{0.983687in}}%
\pgfpathlineto{\pgfqpoint{1.194511in}{0.983687in}}%
\pgfpathlineto{\pgfqpoint{1.197017in}{0.986192in}}%
\pgfpathlineto{\pgfqpoint{1.284750in}{0.986192in}}%
\pgfpathlineto{\pgfqpoint{1.287257in}{0.988661in}}%
\pgfpathlineto{\pgfqpoint{1.367470in}{0.988661in}}%
\pgfpathlineto{\pgfqpoint{1.369977in}{0.991095in}}%
\pgfpathlineto{\pgfqpoint{1.417603in}{0.991095in}}%
\pgfpathlineto{\pgfqpoint{1.420110in}{0.993495in}}%
\pgfpathlineto{\pgfqpoint{1.465229in}{0.993495in}}%
\pgfpathlineto{\pgfqpoint{1.467736in}{0.995862in}}%
\pgfpathlineto{\pgfqpoint{1.520376in}{0.995862in}}%
\pgfpathlineto{\pgfqpoint{1.522883in}{0.998196in}}%
\pgfpathlineto{\pgfqpoint{1.570509in}{0.998196in}}%
\pgfpathlineto{\pgfqpoint{1.573016in}{1.000500in}}%
\pgfpathlineto{\pgfqpoint{1.618135in}{1.000500in}}%
\pgfpathlineto{\pgfqpoint{1.620642in}{1.002773in}}%
\pgfpathlineto{\pgfqpoint{1.670775in}{1.002773in}}%
\pgfpathlineto{\pgfqpoint{1.673282in}{1.005016in}}%
\pgfpathlineto{\pgfqpoint{1.725922in}{1.005016in}}%
\pgfpathlineto{\pgfqpoint{1.728428in}{1.007230in}}%
\pgfpathlineto{\pgfqpoint{1.761015in}{1.007230in}}%
\pgfpathlineto{\pgfqpoint{1.763521in}{1.009416in}}%
\pgfpathlineto{\pgfqpoint{1.806134in}{1.009416in}}%
\pgfpathlineto{\pgfqpoint{1.808641in}{1.011574in}}%
\pgfpathlineto{\pgfqpoint{1.833708in}{1.011574in}}%
\pgfpathlineto{\pgfqpoint{1.836214in}{1.013706in}}%
\pgfpathlineto{\pgfqpoint{1.858774in}{1.013706in}}%
\pgfpathlineto{\pgfqpoint{1.861281in}{1.015811in}}%
\pgfpathlineto{\pgfqpoint{1.881334in}{1.015811in}}%
\pgfpathlineto{\pgfqpoint{1.883841in}{1.017891in}}%
\pgfpathlineto{\pgfqpoint{1.921441in}{1.017891in}}%
\pgfpathlineto{\pgfqpoint{1.923947in}{1.019946in}}%
\pgfpathlineto{\pgfqpoint{1.959040in}{1.019946in}}%
\pgfpathlineto{\pgfqpoint{1.961547in}{1.021977in}}%
\pgfpathlineto{\pgfqpoint{1.986614in}{1.021977in}}%
\pgfpathlineto{\pgfqpoint{1.989120in}{1.023984in}}%
\pgfpathlineto{\pgfqpoint{2.014187in}{1.023984in}}%
\pgfpathlineto{\pgfqpoint{2.016693in}{1.025967in}}%
\pgfpathlineto{\pgfqpoint{2.036747in}{1.025967in}}%
\pgfpathlineto{\pgfqpoint{2.039253in}{1.027928in}}%
\pgfpathlineto{\pgfqpoint{2.059307in}{1.027928in}}%
\pgfpathlineto{\pgfqpoint{2.061813in}{1.029867in}}%
\pgfpathlineto{\pgfqpoint{2.099413in}{1.029867in}}%
\pgfpathlineto{\pgfqpoint{2.101920in}{1.031785in}}%
\pgfpathlineto{\pgfqpoint{2.116960in}{1.031785in}}%
\pgfpathlineto{\pgfqpoint{2.119466in}{1.033681in}}%
\pgfpathlineto{\pgfqpoint{2.132000in}{1.033681in}}%
\pgfpathlineto{\pgfqpoint{2.134506in}{1.035556in}}%
\pgfpathlineto{\pgfqpoint{2.149546in}{1.035556in}}%
\pgfpathlineto{\pgfqpoint{2.152053in}{1.037411in}}%
\pgfpathlineto{\pgfqpoint{2.159573in}{1.037411in}}%
\pgfpathlineto{\pgfqpoint{2.162079in}{1.039246in}}%
\pgfpathlineto{\pgfqpoint{2.179626in}{1.039246in}}%
\pgfpathlineto{\pgfqpoint{2.182133in}{1.041062in}}%
\pgfpathlineto{\pgfqpoint{2.189653in}{1.041062in}}%
\pgfpathlineto{\pgfqpoint{2.192159in}{1.042859in}}%
\pgfpathlineto{\pgfqpoint{2.207199in}{1.042859in}}%
\pgfpathlineto{\pgfqpoint{2.209706in}{1.044637in}}%
\pgfpathlineto{\pgfqpoint{2.222239in}{1.044637in}}%
\pgfpathlineto{\pgfqpoint{2.224746in}{1.046397in}}%
\pgfpathlineto{\pgfqpoint{2.232266in}{1.046397in}}%
\pgfpathlineto{\pgfqpoint{2.234772in}{1.048139in}}%
\pgfpathlineto{\pgfqpoint{2.252319in}{1.048139in}}%
\pgfpathlineto{\pgfqpoint{2.254826in}{1.049863in}}%
\pgfpathlineto{\pgfqpoint{2.262346in}{1.049863in}}%
\pgfpathlineto{\pgfqpoint{2.264852in}{1.051571in}}%
\pgfpathlineto{\pgfqpoint{2.267359in}{1.051571in}}%
\pgfpathlineto{\pgfqpoint{2.269866in}{1.053261in}}%
\pgfpathlineto{\pgfqpoint{2.274879in}{1.053261in}}%
\pgfpathlineto{\pgfqpoint{2.277385in}{1.054935in}}%
\pgfpathlineto{\pgfqpoint{2.279892in}{1.054935in}}%
\pgfpathlineto{\pgfqpoint{2.282399in}{1.056593in}}%
\pgfpathlineto{\pgfqpoint{2.287412in}{1.056593in}}%
\pgfpathlineto{\pgfqpoint{2.289919in}{1.058235in}}%
\pgfpathlineto{\pgfqpoint{2.302452in}{1.058235in}}%
\pgfpathlineto{\pgfqpoint{2.304959in}{1.059862in}}%
\pgfpathlineto{\pgfqpoint{2.314985in}{1.059862in}}%
\pgfpathlineto{\pgfqpoint{2.319999in}{1.063069in}}%
\pgfpathlineto{\pgfqpoint{2.325012in}{1.063069in}}%
\pgfpathlineto{\pgfqpoint{2.327519in}{1.064651in}}%
\pgfpathlineto{\pgfqpoint{2.337545in}{1.064651in}}%
\pgfpathlineto{\pgfqpoint{2.340052in}{1.066218in}}%
\pgfpathlineto{\pgfqpoint{2.342559in}{1.066218in}}%
\pgfpathlineto{\pgfqpoint{2.345065in}{1.067771in}}%
\pgfpathlineto{\pgfqpoint{2.355092in}{1.067771in}}%
\pgfpathlineto{\pgfqpoint{2.357598in}{1.069310in}}%
\pgfpathlineto{\pgfqpoint{2.362612in}{1.069310in}}%
\pgfpathlineto{\pgfqpoint{2.365118in}{1.070835in}}%
\pgfpathlineto{\pgfqpoint{2.380158in}{1.070835in}}%
\pgfpathlineto{\pgfqpoint{2.382665in}{1.072347in}}%
\pgfpathlineto{\pgfqpoint{2.392692in}{1.072347in}}%
\pgfpathlineto{\pgfqpoint{2.395198in}{1.073845in}}%
\pgfpathlineto{\pgfqpoint{2.400212in}{1.073845in}}%
\pgfpathlineto{\pgfqpoint{2.402718in}{1.075331in}}%
\pgfpathlineto{\pgfqpoint{2.415251in}{1.075331in}}%
\pgfpathlineto{\pgfqpoint{2.417758in}{1.076804in}}%
\pgfpathlineto{\pgfqpoint{2.432798in}{1.076804in}}%
\pgfpathlineto{\pgfqpoint{2.435305in}{1.078264in}}%
\pgfpathlineto{\pgfqpoint{2.445331in}{1.078264in}}%
\pgfpathlineto{\pgfqpoint{2.447838in}{1.079712in}}%
\pgfpathlineto{\pgfqpoint{2.450345in}{1.079712in}}%
\pgfpathlineto{\pgfqpoint{2.452851in}{1.081148in}}%
\pgfpathlineto{\pgfqpoint{2.460371in}{1.081148in}}%
\pgfpathlineto{\pgfqpoint{2.462878in}{1.082572in}}%
\pgfpathlineto{\pgfqpoint{2.470398in}{1.082572in}}%
\pgfpathlineto{\pgfqpoint{2.475411in}{1.085385in}}%
\pgfpathlineto{\pgfqpoint{2.485438in}{1.085385in}}%
\pgfpathlineto{\pgfqpoint{2.487944in}{1.086775in}}%
\pgfpathlineto{\pgfqpoint{2.502984in}{1.086775in}}%
\pgfpathlineto{\pgfqpoint{2.505491in}{1.088153in}}%
\pgfpathlineto{\pgfqpoint{2.518024in}{1.088153in}}%
\pgfpathlineto{\pgfqpoint{2.520531in}{1.089520in}}%
\pgfpathlineto{\pgfqpoint{2.523038in}{1.093558in}}%
\pgfpathlineto{\pgfqpoint{2.525544in}{1.093558in}}%
\pgfpathlineto{\pgfqpoint{2.528051in}{1.094883in}}%
\pgfpathlineto{\pgfqpoint{2.533064in}{1.094883in}}%
\pgfpathlineto{\pgfqpoint{2.535571in}{1.096198in}}%
\pgfpathlineto{\pgfqpoint{2.543091in}{1.096198in}}%
\pgfpathlineto{\pgfqpoint{2.545598in}{1.097503in}}%
\pgfpathlineto{\pgfqpoint{2.548104in}{1.097503in}}%
\pgfpathlineto{\pgfqpoint{2.550611in}{1.098798in}}%
\pgfpathlineto{\pgfqpoint{2.555624in}{1.098798in}}%
\pgfpathlineto{\pgfqpoint{2.558131in}{1.100083in}}%
\pgfpathlineto{\pgfqpoint{2.560637in}{1.100083in}}%
\pgfpathlineto{\pgfqpoint{2.565651in}{1.102625in}}%
\pgfpathlineto{\pgfqpoint{2.568157in}{1.102625in}}%
\pgfpathlineto{\pgfqpoint{2.570664in}{1.106369in}}%
\pgfpathlineto{\pgfqpoint{2.573171in}{1.106369in}}%
\pgfpathlineto{\pgfqpoint{2.578184in}{1.107599in}}%
\pgfpathlineto{\pgfqpoint{2.590717in}{1.107599in}}%
\pgfpathlineto{\pgfqpoint{2.595731in}{1.108820in}}%
\pgfpathlineto{\pgfqpoint{2.598237in}{1.108820in}}%
\pgfpathlineto{\pgfqpoint{2.605757in}{1.111237in}}%
\pgfpathlineto{\pgfqpoint{2.613277in}{1.111237in}}%
\pgfpathlineto{\pgfqpoint{2.615784in}{1.113620in}}%
\pgfpathlineto{\pgfqpoint{2.618290in}{1.113620in}}%
\pgfpathlineto{\pgfqpoint{2.620797in}{1.115971in}}%
\pgfpathlineto{\pgfqpoint{2.633330in}{1.115971in}}%
\pgfpathlineto{\pgfqpoint{2.643357in}{1.119438in}}%
\pgfpathlineto{\pgfqpoint{2.645864in}{1.119438in}}%
\pgfpathlineto{\pgfqpoint{2.658397in}{1.123953in}}%
\pgfpathlineto{\pgfqpoint{2.663410in}{1.125064in}}%
\pgfpathlineto{\pgfqpoint{2.670930in}{1.127264in}}%
\pgfpathlineto{\pgfqpoint{2.678450in}{1.128353in}}%
\pgfpathlineto{\pgfqpoint{2.685970in}{1.130512in}}%
\pgfpathlineto{\pgfqpoint{2.690983in}{1.131581in}}%
\pgfpathlineto{\pgfqpoint{2.701010in}{1.134749in}}%
\pgfpathlineto{\pgfqpoint{2.703517in}{1.134749in}}%
\pgfpathlineto{\pgfqpoint{2.706023in}{1.136829in}}%
\pgfpathlineto{\pgfqpoint{2.718557in}{1.137859in}}%
\pgfpathlineto{\pgfqpoint{2.721063in}{1.139902in}}%
\pgfpathlineto{\pgfqpoint{2.726077in}{1.139902in}}%
\pgfpathlineto{\pgfqpoint{2.731090in}{1.145888in}}%
\pgfpathlineto{\pgfqpoint{2.736103in}{1.146866in}}%
\pgfpathlineto{\pgfqpoint{2.761170in}{1.155424in}}%
\pgfpathlineto{\pgfqpoint{2.768690in}{1.157269in}}%
\pgfpathlineto{\pgfqpoint{2.776210in}{1.158184in}}%
\pgfpathlineto{\pgfqpoint{2.781223in}{1.159094in}}%
\pgfpathlineto{\pgfqpoint{2.788743in}{1.160000in}}%
\pgfpathlineto{\pgfqpoint{2.796263in}{1.163575in}}%
\pgfpathlineto{\pgfqpoint{2.801276in}{1.167941in}}%
\pgfpathlineto{\pgfqpoint{2.816316in}{1.177173in}}%
\pgfpathlineto{\pgfqpoint{2.821329in}{1.177173in}}%
\pgfpathlineto{\pgfqpoint{2.826343in}{1.179607in}}%
\pgfpathlineto{\pgfqpoint{2.828849in}{1.179607in}}%
\pgfpathlineto{\pgfqpoint{2.831356in}{1.185934in}}%
\pgfpathlineto{\pgfqpoint{2.833863in}{1.185934in}}%
\pgfpathlineto{\pgfqpoint{2.838876in}{1.188247in}}%
\pgfpathlineto{\pgfqpoint{2.841383in}{1.188247in}}%
\pgfpathlineto{\pgfqpoint{2.851409in}{1.193528in}}%
\pgfpathlineto{\pgfqpoint{2.856423in}{1.194269in}}%
\pgfpathlineto{\pgfqpoint{2.871463in}{1.197928in}}%
\pgfpathlineto{\pgfqpoint{2.876476in}{1.198650in}}%
\pgfpathlineto{\pgfqpoint{2.878983in}{1.199370in}}%
\pgfpathlineto{\pgfqpoint{2.881489in}{1.202922in}}%
\pgfpathlineto{\pgfqpoint{2.883996in}{1.204323in}}%
\pgfpathlineto{\pgfqpoint{2.886503in}{1.204323in}}%
\pgfpathlineto{\pgfqpoint{2.896529in}{1.212496in}}%
\pgfpathlineto{\pgfqpoint{2.914076in}{1.216440in}}%
\pgfpathlineto{\pgfqpoint{2.916582in}{1.216440in}}%
\pgfpathlineto{\pgfqpoint{2.921596in}{1.218379in}}%
\pgfpathlineto{\pgfqpoint{2.924102in}{1.222820in}}%
\pgfpathlineto{\pgfqpoint{2.934129in}{1.225307in}}%
\pgfpathlineto{\pgfqpoint{2.936636in}{1.230175in}}%
\pgfpathlineto{\pgfqpoint{2.941649in}{1.230774in}}%
\pgfpathlineto{\pgfqpoint{2.949169in}{1.231966in}}%
\pgfpathlineto{\pgfqpoint{2.951676in}{1.231966in}}%
\pgfpathlineto{\pgfqpoint{2.954182in}{1.236651in}}%
\pgfpathlineto{\pgfqpoint{2.976742in}{1.242891in}}%
\pgfpathlineto{\pgfqpoint{2.989275in}{1.249450in}}%
\pgfpathlineto{\pgfqpoint{2.994289in}{1.251581in}}%
\pgfpathlineto{\pgfqpoint{2.996795in}{1.257822in}}%
\pgfpathlineto{\pgfqpoint{2.999302in}{1.257822in}}%
\pgfpathlineto{\pgfqpoint{3.001809in}{1.259852in}}%
\pgfpathlineto{\pgfqpoint{3.009329in}{1.260356in}}%
\pgfpathlineto{\pgfqpoint{3.011835in}{1.263349in}}%
\pgfpathlineto{\pgfqpoint{3.021862in}{1.265804in}}%
\pgfpathlineto{\pgfqpoint{3.029382in}{1.273432in}}%
\pgfpathlineto{\pgfqpoint{3.034395in}{1.273897in}}%
\pgfpathlineto{\pgfqpoint{3.039408in}{1.274825in}}%
\pgfpathlineto{\pgfqpoint{3.054448in}{1.278032in}}%
\pgfpathlineto{\pgfqpoint{3.056955in}{1.280287in}}%
\pgfpathlineto{\pgfqpoint{3.061968in}{1.282070in}}%
\pgfpathlineto{\pgfqpoint{3.064475in}{1.282954in}}%
\pgfpathlineto{\pgfqpoint{3.066982in}{1.286879in}}%
\pgfpathlineto{\pgfqpoint{3.082022in}{1.290294in}}%
\pgfpathlineto{\pgfqpoint{3.087035in}{1.294056in}}%
\pgfpathlineto{\pgfqpoint{3.094555in}{1.294881in}}%
\pgfpathlineto{\pgfqpoint{3.107088in}{1.299349in}}%
\pgfpathlineto{\pgfqpoint{3.112101in}{1.305259in}}%
\pgfpathlineto{\pgfqpoint{3.117115in}{1.307568in}}%
\pgfpathlineto{\pgfqpoint{3.119621in}{1.307950in}}%
\pgfpathlineto{\pgfqpoint{3.122128in}{1.309468in}}%
\pgfpathlineto{\pgfqpoint{3.124635in}{1.309468in}}%
\pgfpathlineto{\pgfqpoint{3.127141in}{1.312094in}}%
\pgfpathlineto{\pgfqpoint{3.129648in}{1.318307in}}%
\pgfpathlineto{\pgfqpoint{3.132155in}{1.318666in}}%
\pgfpathlineto{\pgfqpoint{3.134661in}{1.322211in}}%
\pgfpathlineto{\pgfqpoint{3.142181in}{1.326029in}}%
\pgfpathlineto{\pgfqpoint{3.144688in}{1.328752in}}%
\pgfpathlineto{\pgfqpoint{3.162234in}{1.333746in}}%
\pgfpathlineto{\pgfqpoint{3.164741in}{1.338278in}}%
\pgfpathlineto{\pgfqpoint{3.177274in}{1.340816in}}%
\pgfpathlineto{\pgfqpoint{3.179781in}{1.344861in}}%
\pgfpathlineto{\pgfqpoint{3.187301in}{1.347909in}}%
\pgfpathlineto{\pgfqpoint{3.189808in}{1.348512in}}%
\pgfpathlineto{\pgfqpoint{3.192314in}{1.351496in}}%
\pgfpathlineto{\pgfqpoint{3.197328in}{1.352087in}}%
\pgfpathlineto{\pgfqpoint{3.202341in}{1.354429in}}%
\pgfpathlineto{\pgfqpoint{3.209861in}{1.356166in}}%
\pgfpathlineto{\pgfqpoint{3.217381in}{1.357884in}}%
\pgfpathlineto{\pgfqpoint{3.222394in}{1.359586in}}%
\pgfpathlineto{\pgfqpoint{3.227408in}{1.360431in}}%
\pgfpathlineto{\pgfqpoint{3.237434in}{1.362663in}}%
\pgfpathlineto{\pgfqpoint{3.239941in}{1.363216in}}%
\pgfpathlineto{\pgfqpoint{3.242447in}{1.369190in}}%
\pgfpathlineto{\pgfqpoint{3.252474in}{1.373927in}}%
\pgfpathlineto{\pgfqpoint{3.257487in}{1.375478in}}%
\pgfpathlineto{\pgfqpoint{3.262501in}{1.378031in}}%
\pgfpathlineto{\pgfqpoint{3.270021in}{1.379042in}}%
\pgfpathlineto{\pgfqpoint{3.275034in}{1.380547in}}%
\pgfpathlineto{\pgfqpoint{3.280047in}{1.382040in}}%
\pgfpathlineto{\pgfqpoint{3.282554in}{1.383027in}}%
\pgfpathlineto{\pgfqpoint{3.287567in}{1.387882in}}%
\pgfpathlineto{\pgfqpoint{3.305114in}{1.393531in}}%
\pgfpathlineto{\pgfqpoint{3.307620in}{1.394224in}}%
\pgfpathlineto{\pgfqpoint{3.310127in}{1.398101in}}%
\pgfpathlineto{\pgfqpoint{3.312634in}{1.398326in}}%
\pgfpathlineto{\pgfqpoint{3.330180in}{1.407319in}}%
\pgfpathlineto{\pgfqpoint{3.337700in}{1.409020in}}%
\pgfpathlineto{\pgfqpoint{3.342714in}{1.412994in}}%
\pgfpathlineto{\pgfqpoint{3.347727in}{1.414435in}}%
\pgfpathlineto{\pgfqpoint{3.350234in}{1.414435in}}%
\pgfpathlineto{\pgfqpoint{3.352740in}{1.417080in}}%
\pgfpathlineto{\pgfqpoint{3.357754in}{1.417885in}}%
\pgfpathlineto{\pgfqpoint{3.365273in}{1.424003in}}%
\pgfpathlineto{\pgfqpoint{3.370287in}{1.424777in}}%
\pgfpathlineto{\pgfqpoint{3.375300in}{1.427458in}}%
\pgfpathlineto{\pgfqpoint{3.382820in}{1.429911in}}%
\pgfpathlineto{\pgfqpoint{3.387833in}{1.431959in}}%
\pgfpathlineto{\pgfqpoint{3.395353in}{1.433067in}}%
\pgfpathlineto{\pgfqpoint{3.402873in}{1.437962in}}%
\pgfpathlineto{\pgfqpoint{3.405380in}{1.440446in}}%
\pgfpathlineto{\pgfqpoint{3.407887in}{1.440974in}}%
\pgfpathlineto{\pgfqpoint{3.412900in}{1.444451in}}%
\pgfpathlineto{\pgfqpoint{3.420420in}{1.447859in}}%
\pgfpathlineto{\pgfqpoint{3.422927in}{1.451696in}}%
\pgfpathlineto{\pgfqpoint{3.430447in}{1.454965in}}%
\pgfpathlineto{\pgfqpoint{3.432953in}{1.457854in}}%
\pgfpathlineto{\pgfqpoint{3.435460in}{1.457854in}}%
\pgfpathlineto{\pgfqpoint{3.440473in}{1.461008in}}%
\pgfpathlineto{\pgfqpoint{3.442980in}{1.461321in}}%
\pgfpathlineto{\pgfqpoint{3.447993in}{1.463952in}}%
\pgfpathlineto{\pgfqpoint{3.463033in}{1.473658in}}%
\pgfpathlineto{\pgfqpoint{3.468046in}{1.474527in}}%
\pgfpathlineto{\pgfqpoint{3.473060in}{1.475822in}}%
\pgfpathlineto{\pgfqpoint{3.480580in}{1.476680in}}%
\pgfpathlineto{\pgfqpoint{3.483086in}{1.480069in}}%
\pgfpathlineto{\pgfqpoint{3.490606in}{1.481878in}}%
\pgfpathlineto{\pgfqpoint{3.500633in}{1.484077in}}%
\pgfpathlineto{\pgfqpoint{3.503139in}{1.486654in}}%
\pgfpathlineto{\pgfqpoint{3.513166in}{1.487727in}}%
\pgfpathlineto{\pgfqpoint{3.530713in}{1.503069in}}%
\pgfpathlineto{\pgfqpoint{3.535726in}{1.503558in}}%
\pgfpathlineto{\pgfqpoint{3.538233in}{1.506460in}}%
\pgfpathlineto{\pgfqpoint{3.550766in}{1.510840in}}%
\pgfpathlineto{\pgfqpoint{3.553273in}{1.513162in}}%
\pgfpathlineto{\pgfqpoint{3.560793in}{1.514312in}}%
\pgfpathlineto{\pgfqpoint{3.563299in}{1.521270in}}%
\pgfpathlineto{\pgfqpoint{3.565806in}{1.523130in}}%
\pgfpathlineto{\pgfqpoint{3.568313in}{1.523130in}}%
\pgfpathlineto{\pgfqpoint{3.570819in}{1.525293in}}%
\pgfpathlineto{\pgfqpoint{3.573326in}{1.533065in}}%
\pgfpathlineto{\pgfqpoint{3.578339in}{1.535208in}}%
\pgfpathlineto{\pgfqpoint{3.583352in}{1.539514in}}%
\pgfpathlineto{\pgfqpoint{3.585859in}{1.539811in}}%
\pgfpathlineto{\pgfqpoint{3.588366in}{1.541774in}}%
\pgfpathlineto{\pgfqpoint{3.590872in}{1.542067in}}%
\pgfpathlineto{\pgfqpoint{3.593379in}{1.546111in}}%
\pgfpathlineto{\pgfqpoint{3.600899in}{1.547438in}}%
\pgfpathlineto{\pgfqpoint{3.603406in}{1.548474in}}%
\pgfpathlineto{\pgfqpoint{3.610926in}{1.554651in}}%
\pgfpathlineto{\pgfqpoint{3.615939in}{1.555373in}}%
\pgfpathlineto{\pgfqpoint{3.625966in}{1.564333in}}%
\pgfpathlineto{\pgfqpoint{3.641005in}{1.571867in}}%
\pgfpathlineto{\pgfqpoint{3.643512in}{1.572358in}}%
\pgfpathlineto{\pgfqpoint{3.651032in}{1.576473in}}%
\pgfpathlineto{\pgfqpoint{3.661059in}{1.579712in}}%
\pgfpathlineto{\pgfqpoint{3.668579in}{1.586688in}}%
\pgfpathlineto{\pgfqpoint{3.671085in}{1.587363in}}%
\pgfpathlineto{\pgfqpoint{3.676099in}{1.592232in}}%
\pgfpathlineto{\pgfqpoint{3.681112in}{1.594616in}}%
\pgfpathlineto{\pgfqpoint{3.683619in}{1.599077in}}%
\pgfpathlineto{\pgfqpoint{3.688632in}{1.600954in}}%
\pgfpathlineto{\pgfqpoint{3.691139in}{1.601230in}}%
\pgfpathlineto{\pgfqpoint{3.696152in}{1.604579in}}%
\pgfpathlineto{\pgfqpoint{3.701165in}{1.615907in}}%
\pgfpathlineto{\pgfqpoint{3.716205in}{1.625218in}}%
\pgfpathlineto{\pgfqpoint{3.721218in}{1.626474in}}%
\pgfpathlineto{\pgfqpoint{3.738765in}{1.640648in}}%
\pgfpathlineto{\pgfqpoint{3.741272in}{1.644231in}}%
\pgfpathlineto{\pgfqpoint{3.748792in}{1.648003in}}%
\pgfpathlineto{\pgfqpoint{3.751298in}{1.651952in}}%
\pgfpathlineto{\pgfqpoint{3.753805in}{1.652157in}}%
\pgfpathlineto{\pgfqpoint{3.756312in}{1.653638in}}%
\pgfpathlineto{\pgfqpoint{3.768845in}{1.667553in}}%
\pgfpathlineto{\pgfqpoint{3.778871in}{1.670666in}}%
\pgfpathlineto{\pgfqpoint{3.781378in}{1.674627in}}%
\pgfpathlineto{\pgfqpoint{3.786391in}{1.675882in}}%
\pgfpathlineto{\pgfqpoint{3.788898in}{1.679638in}}%
\pgfpathlineto{\pgfqpoint{3.796418in}{1.681766in}}%
\pgfpathlineto{\pgfqpoint{3.798925in}{1.687455in}}%
\pgfpathlineto{\pgfqpoint{3.813965in}{1.695809in}}%
\pgfpathlineto{\pgfqpoint{3.816471in}{1.699664in}}%
\pgfpathlineto{\pgfqpoint{3.829005in}{1.701982in}}%
\pgfpathlineto{\pgfqpoint{3.831511in}{1.704003in}}%
\pgfpathlineto{\pgfqpoint{3.834018in}{1.704382in}}%
\pgfpathlineto{\pgfqpoint{3.841538in}{1.709011in}}%
\pgfpathlineto{\pgfqpoint{3.846551in}{1.710222in}}%
\pgfpathlineto{\pgfqpoint{3.851564in}{1.710478in}}%
\pgfpathlineto{\pgfqpoint{3.854071in}{1.713984in}}%
\pgfpathlineto{\pgfqpoint{3.859084in}{1.715763in}}%
\pgfpathlineto{\pgfqpoint{3.861591in}{1.719060in}}%
\pgfpathlineto{\pgfqpoint{3.876631in}{1.722703in}}%
\pgfpathlineto{\pgfqpoint{3.891671in}{1.733477in}}%
\pgfpathlineto{\pgfqpoint{3.894178in}{1.736045in}}%
\pgfpathlineto{\pgfqpoint{3.896684in}{1.736170in}}%
\pgfpathlineto{\pgfqpoint{3.899191in}{1.741281in}}%
\pgfpathlineto{\pgfqpoint{3.901698in}{1.741829in}}%
\pgfpathlineto{\pgfqpoint{3.906711in}{1.744006in}}%
\pgfpathlineto{\pgfqpoint{3.909217in}{1.745352in}}%
\pgfpathlineto{\pgfqpoint{3.911724in}{1.748365in}}%
\pgfpathlineto{\pgfqpoint{3.916737in}{1.748979in}}%
\pgfpathlineto{\pgfqpoint{3.929271in}{1.760379in}}%
\pgfpathlineto{\pgfqpoint{3.931777in}{1.764666in}}%
\pgfpathlineto{\pgfqpoint{3.941804in}{1.769004in}}%
\pgfpathlineto{\pgfqpoint{3.944311in}{1.773084in}}%
\pgfpathlineto{\pgfqpoint{3.949324in}{1.774749in}}%
\pgfpathlineto{\pgfqpoint{3.954337in}{1.776696in}}%
\pgfpathlineto{\pgfqpoint{3.956844in}{1.777192in}}%
\pgfpathlineto{\pgfqpoint{3.961857in}{1.780696in}}%
\pgfpathlineto{\pgfqpoint{3.981910in}{1.789512in}}%
\pgfpathlineto{\pgfqpoint{3.984417in}{1.805031in}}%
\pgfpathlineto{\pgfqpoint{3.986924in}{1.805808in}}%
\pgfpathlineto{\pgfqpoint{3.991937in}{1.809927in}}%
\pgfpathlineto{\pgfqpoint{3.994444in}{1.810478in}}%
\pgfpathlineto{\pgfqpoint{4.001964in}{1.815917in}}%
\pgfpathlineto{\pgfqpoint{4.006977in}{1.817646in}}%
\pgfpathlineto{\pgfqpoint{4.009484in}{1.818076in}}%
\pgfpathlineto{\pgfqpoint{4.011990in}{1.821417in}}%
\pgfpathlineto{\pgfqpoint{4.017004in}{1.823074in}}%
\pgfpathlineto{\pgfqpoint{4.024524in}{1.823981in}}%
\pgfpathlineto{\pgfqpoint{4.029537in}{1.829871in}}%
\pgfpathlineto{\pgfqpoint{4.032044in}{1.835567in}}%
\pgfpathlineto{\pgfqpoint{4.034550in}{1.836270in}}%
\pgfpathlineto{\pgfqpoint{4.037057in}{1.839674in}}%
\pgfpathlineto{\pgfqpoint{4.044577in}{1.840908in}}%
\pgfpathlineto{\pgfqpoint{4.054603in}{1.848996in}}%
\pgfpathlineto{\pgfqpoint{4.057110in}{1.849338in}}%
\pgfpathlineto{\pgfqpoint{4.062123in}{1.854589in}}%
\pgfpathlineto{\pgfqpoint{4.067137in}{1.855108in}}%
\pgfpathlineto{\pgfqpoint{4.069643in}{1.858655in}}%
\pgfpathlineto{\pgfqpoint{4.074657in}{1.860889in}}%
\pgfpathlineto{\pgfqpoint{4.084683in}{1.863423in}}%
\pgfpathlineto{\pgfqpoint{4.092203in}{1.866231in}}%
\pgfpathlineto{\pgfqpoint{4.094710in}{1.872309in}}%
\pgfpathlineto{\pgfqpoint{4.104737in}{1.876565in}}%
\pgfpathlineto{\pgfqpoint{4.109750in}{1.889197in}}%
\pgfpathlineto{\pgfqpoint{4.114763in}{1.891094in}}%
\pgfpathlineto{\pgfqpoint{4.122283in}{1.898965in}}%
\pgfpathlineto{\pgfqpoint{4.124790in}{1.901958in}}%
\pgfpathlineto{\pgfqpoint{4.127296in}{1.902484in}}%
\pgfpathlineto{\pgfqpoint{4.132310in}{1.904712in}}%
\pgfpathlineto{\pgfqpoint{4.142336in}{1.908852in}}%
\pgfpathlineto{\pgfqpoint{4.144843in}{1.916153in}}%
\pgfpathlineto{\pgfqpoint{4.147350in}{1.917656in}}%
\pgfpathlineto{\pgfqpoint{4.149856in}{1.923937in}}%
\pgfpathlineto{\pgfqpoint{4.152363in}{1.924840in}}%
\pgfpathlineto{\pgfqpoint{4.154870in}{1.929742in}}%
\pgfpathlineto{\pgfqpoint{4.157376in}{1.930868in}}%
\pgfpathlineto{\pgfqpoint{4.159883in}{1.933396in}}%
\pgfpathlineto{\pgfqpoint{4.162390in}{1.938499in}}%
\pgfpathlineto{\pgfqpoint{4.164896in}{1.946979in}}%
\pgfpathlineto{\pgfqpoint{4.169910in}{1.950476in}}%
\pgfpathlineto{\pgfqpoint{4.177430in}{1.956840in}}%
\pgfpathlineto{\pgfqpoint{4.179936in}{1.964470in}}%
\pgfpathlineto{\pgfqpoint{4.184949in}{1.965787in}}%
\pgfpathlineto{\pgfqpoint{4.187456in}{1.966075in}}%
\pgfpathlineto{\pgfqpoint{4.199989in}{1.980592in}}%
\pgfpathlineto{\pgfqpoint{4.205003in}{1.981445in}}%
\pgfpathlineto{\pgfqpoint{4.207509in}{1.985331in}}%
\pgfpathlineto{\pgfqpoint{4.210016in}{1.991645in}}%
\pgfpathlineto{\pgfqpoint{4.217536in}{1.999235in}}%
\pgfpathlineto{\pgfqpoint{4.225056in}{2.003789in}}%
\pgfpathlineto{\pgfqpoint{4.227563in}{2.004198in}}%
\pgfpathlineto{\pgfqpoint{4.232576in}{2.009220in}}%
\pgfpathlineto{\pgfqpoint{4.237589in}{2.010337in}}%
\pgfpathlineto{\pgfqpoint{4.245109in}{2.019134in}}%
\pgfpathlineto{\pgfqpoint{4.247616in}{2.019231in}}%
\pgfpathlineto{\pgfqpoint{4.250122in}{2.025552in}}%
\pgfpathlineto{\pgfqpoint{4.262656in}{2.031727in}}%
\pgfpathlineto{\pgfqpoint{4.265162in}{2.035602in}}%
\pgfpathlineto{\pgfqpoint{4.270176in}{2.038239in}}%
\pgfpathlineto{\pgfqpoint{4.277696in}{2.046818in}}%
\pgfpathlineto{\pgfqpoint{4.285216in}{2.051526in}}%
\pgfpathlineto{\pgfqpoint{4.292736in}{2.065377in}}%
\pgfpathlineto{\pgfqpoint{4.295242in}{2.066468in}}%
\pgfpathlineto{\pgfqpoint{4.297749in}{2.069026in}}%
\pgfpathlineto{\pgfqpoint{4.300256in}{2.075400in}}%
\pgfpathlineto{\pgfqpoint{4.307776in}{2.075630in}}%
\pgfpathlineto{\pgfqpoint{4.310282in}{2.081809in}}%
\pgfpathlineto{\pgfqpoint{4.312789in}{2.082061in}}%
\pgfpathlineto{\pgfqpoint{4.315296in}{2.085260in}}%
\pgfpathlineto{\pgfqpoint{4.320309in}{2.095991in}}%
\pgfpathlineto{\pgfqpoint{4.322815in}{2.096654in}}%
\pgfpathlineto{\pgfqpoint{4.330335in}{2.101561in}}%
\pgfpathlineto{\pgfqpoint{4.332842in}{2.102154in}}%
\pgfpathlineto{\pgfqpoint{4.337855in}{2.106948in}}%
\pgfpathlineto{\pgfqpoint{4.340362in}{2.111823in}}%
\pgfpathlineto{\pgfqpoint{4.342869in}{2.112003in}}%
\pgfpathlineto{\pgfqpoint{4.345375in}{2.123291in}}%
\pgfpathlineto{\pgfqpoint{4.347882in}{2.125031in}}%
\pgfpathlineto{\pgfqpoint{4.350389in}{2.130952in}}%
\pgfpathlineto{\pgfqpoint{4.362922in}{2.138078in}}%
\pgfpathlineto{\pgfqpoint{4.365429in}{2.138453in}}%
\pgfpathlineto{\pgfqpoint{4.370442in}{2.141986in}}%
\pgfpathlineto{\pgfqpoint{4.372949in}{2.151175in}}%
\pgfpathlineto{\pgfqpoint{4.380469in}{2.159417in}}%
\pgfpathlineto{\pgfqpoint{4.385482in}{2.160791in}}%
\pgfpathlineto{\pgfqpoint{4.387988in}{2.170788in}}%
\pgfpathlineto{\pgfqpoint{4.393002in}{2.171891in}}%
\pgfpathlineto{\pgfqpoint{4.398015in}{2.173238in}}%
\pgfpathlineto{\pgfqpoint{4.400522in}{2.177913in}}%
\pgfpathlineto{\pgfqpoint{4.403028in}{2.185449in}}%
\pgfpathlineto{\pgfqpoint{4.405535in}{2.188960in}}%
\pgfpathlineto{\pgfqpoint{4.410548in}{2.189954in}}%
\pgfpathlineto{\pgfqpoint{4.413055in}{2.192457in}}%
\pgfpathlineto{\pgfqpoint{4.418068in}{2.194470in}}%
\pgfpathlineto{\pgfqpoint{4.420575in}{2.204956in}}%
\pgfpathlineto{\pgfqpoint{4.423082in}{2.205034in}}%
\pgfpathlineto{\pgfqpoint{4.425588in}{2.206990in}}%
\pgfpathlineto{\pgfqpoint{4.433108in}{2.216226in}}%
\pgfpathlineto{\pgfqpoint{4.435615in}{2.217788in}}%
\pgfpathlineto{\pgfqpoint{4.438122in}{2.226737in}}%
\pgfpathlineto{\pgfqpoint{4.443135in}{2.229747in}}%
\pgfpathlineto{\pgfqpoint{4.445642in}{2.231811in}}%
\pgfpathlineto{\pgfqpoint{4.448148in}{2.242054in}}%
\pgfpathlineto{\pgfqpoint{4.453161in}{2.245410in}}%
\pgfpathlineto{\pgfqpoint{4.458175in}{2.259561in}}%
\pgfpathlineto{\pgfqpoint{4.460681in}{2.259872in}}%
\pgfpathlineto{\pgfqpoint{4.468201in}{2.279172in}}%
\pgfpathlineto{\pgfqpoint{4.470708in}{2.279461in}}%
\pgfpathlineto{\pgfqpoint{4.475721in}{2.283333in}}%
\pgfpathlineto{\pgfqpoint{4.480735in}{2.293108in}}%
\pgfpathlineto{\pgfqpoint{4.483241in}{2.303202in}}%
\pgfpathlineto{\pgfqpoint{4.485748in}{2.303886in}}%
\pgfpathlineto{\pgfqpoint{4.488255in}{2.311345in}}%
\pgfpathlineto{\pgfqpoint{4.490761in}{2.311576in}}%
\pgfpathlineto{\pgfqpoint{4.495775in}{2.318026in}}%
\pgfpathlineto{\pgfqpoint{4.498281in}{2.318482in}}%
\pgfpathlineto{\pgfqpoint{4.500788in}{2.326703in}}%
\pgfpathlineto{\pgfqpoint{4.503295in}{2.330974in}}%
\pgfpathlineto{\pgfqpoint{4.508308in}{2.332908in}}%
\pgfpathlineto{\pgfqpoint{4.510815in}{2.340563in}}%
\pgfpathlineto{\pgfqpoint{4.515828in}{2.341374in}}%
\pgfpathlineto{\pgfqpoint{4.520841in}{2.344309in}}%
\pgfpathlineto{\pgfqpoint{4.523348in}{2.345539in}}%
\pgfpathlineto{\pgfqpoint{4.525854in}{2.348490in}}%
\pgfpathlineto{\pgfqpoint{4.528361in}{2.348861in}}%
\pgfpathlineto{\pgfqpoint{4.530868in}{2.362013in}}%
\pgfpathlineto{\pgfqpoint{4.535881in}{2.363146in}}%
\pgfpathlineto{\pgfqpoint{4.538388in}{2.368188in}}%
\pgfpathlineto{\pgfqpoint{4.540894in}{2.369736in}}%
\pgfpathlineto{\pgfqpoint{4.543401in}{2.376526in}}%
\pgfpathlineto{\pgfqpoint{4.548414in}{2.378137in}}%
\pgfpathlineto{\pgfqpoint{4.550921in}{2.383032in}}%
\pgfpathlineto{\pgfqpoint{4.553428in}{2.405738in}}%
\pgfpathlineto{\pgfqpoint{4.555934in}{2.410426in}}%
\pgfpathlineto{\pgfqpoint{4.558441in}{2.426385in}}%
\pgfpathlineto{\pgfqpoint{4.563454in}{2.434140in}}%
\pgfpathlineto{\pgfqpoint{4.568468in}{2.462530in}}%
\pgfpathlineto{\pgfqpoint{4.570974in}{2.472812in}}%
\pgfpathlineto{\pgfqpoint{4.573481in}{2.473873in}}%
\pgfpathlineto{\pgfqpoint{4.578494in}{2.482691in}}%
\pgfpathlineto{\pgfqpoint{4.581001in}{2.493829in}}%
\pgfpathlineto{\pgfqpoint{4.583508in}{2.495378in}}%
\pgfpathlineto{\pgfqpoint{4.586014in}{2.495531in}}%
\pgfpathlineto{\pgfqpoint{4.591027in}{2.501287in}}%
\pgfpathlineto{\pgfqpoint{4.596041in}{2.502459in}}%
\pgfpathlineto{\pgfqpoint{4.601054in}{2.509539in}}%
\pgfpathlineto{\pgfqpoint{4.603561in}{2.510311in}}%
\pgfpathlineto{\pgfqpoint{4.606067in}{2.513764in}}%
\pgfpathlineto{\pgfqpoint{4.608574in}{2.515063in}}%
\pgfpathlineto{\pgfqpoint{4.611081in}{2.518636in}}%
\pgfpathlineto{\pgfqpoint{4.613587in}{2.519858in}}%
\pgfpathlineto{\pgfqpoint{4.616094in}{2.528100in}}%
\pgfpathlineto{\pgfqpoint{4.618601in}{2.548198in}}%
\pgfpathlineto{\pgfqpoint{4.626121in}{2.554282in}}%
\pgfpathlineto{\pgfqpoint{4.628627in}{2.563460in}}%
\pgfpathlineto{\pgfqpoint{4.631134in}{2.563699in}}%
\pgfpathlineto{\pgfqpoint{4.633641in}{2.569685in}}%
\pgfpathlineto{\pgfqpoint{4.638654in}{2.570745in}}%
\pgfpathlineto{\pgfqpoint{4.643667in}{2.580127in}}%
\pgfpathlineto{\pgfqpoint{4.648681in}{2.582682in}}%
\pgfpathlineto{\pgfqpoint{4.651187in}{2.592049in}}%
\pgfpathlineto{\pgfqpoint{4.653694in}{2.596620in}}%
\pgfpathlineto{\pgfqpoint{4.656201in}{2.610134in}}%
\pgfpathlineto{\pgfqpoint{4.658707in}{2.610232in}}%
\pgfpathlineto{\pgfqpoint{4.663720in}{2.620673in}}%
\pgfpathlineto{\pgfqpoint{4.666227in}{2.620777in}}%
\pgfpathlineto{\pgfqpoint{4.668734in}{2.623142in}}%
\pgfpathlineto{\pgfqpoint{4.673747in}{2.630275in}}%
\pgfpathlineto{\pgfqpoint{4.673747in}{2.630275in}}%
\pgfusepath{stroke}%
\end{pgfscope}%
\begin{pgfscope}%
\pgfpathrectangle{\pgfqpoint{0.708220in}{0.535823in}}{\pgfqpoint{5.013309in}{2.094453in}}%
\pgfusepath{clip}%
\pgfsetbuttcap%
\pgfsetroundjoin%
\pgfsetlinewidth{1.003750pt}%
\definecolor{currentstroke}{rgb}{0.811765,0.125490,0.125490}%
\pgfsetstrokecolor{currentstroke}%
\pgfsetdash{{3.700000pt}{1.600000pt}}{0.000000pt}%
\pgfpathmoveto{\pgfqpoint{0.757730in}{0.525823in}}%
\pgfpathlineto{\pgfqpoint{0.758353in}{0.527778in}}%
\pgfpathlineto{\pgfqpoint{0.763366in}{0.531830in}}%
\pgfpathlineto{\pgfqpoint{0.770886in}{0.532811in}}%
\pgfpathlineto{\pgfqpoint{0.773393in}{0.536473in}}%
\pgfpathlineto{\pgfqpoint{0.790939in}{0.539557in}}%
\pgfpathlineto{\pgfqpoint{0.793446in}{0.542882in}}%
\pgfpathlineto{\pgfqpoint{0.803473in}{0.545400in}}%
\pgfpathlineto{\pgfqpoint{0.805979in}{0.548168in}}%
\pgfpathlineto{\pgfqpoint{0.808486in}{0.556125in}}%
\pgfpathlineto{\pgfqpoint{0.813499in}{0.560371in}}%
\pgfpathlineto{\pgfqpoint{0.821019in}{0.562440in}}%
\pgfpathlineto{\pgfqpoint{0.876166in}{0.567307in}}%
\pgfpathlineto{\pgfqpoint{0.893712in}{0.568616in}}%
\pgfpathlineto{\pgfqpoint{0.901232in}{0.570365in}}%
\pgfpathlineto{\pgfqpoint{0.921285in}{0.572429in}}%
\pgfpathlineto{\pgfqpoint{0.926299in}{0.575261in}}%
\pgfpathlineto{\pgfqpoint{0.931312in}{0.576863in}}%
\pgfpathlineto{\pgfqpoint{0.933819in}{0.578717in}}%
\pgfpathlineto{\pgfqpoint{0.936325in}{0.578984in}}%
\pgfpathlineto{\pgfqpoint{0.938832in}{0.581290in}}%
\pgfpathlineto{\pgfqpoint{0.943845in}{0.582104in}}%
\pgfpathlineto{\pgfqpoint{0.948859in}{0.585401in}}%
\pgfpathlineto{\pgfqpoint{0.951365in}{0.585811in}}%
\pgfpathlineto{\pgfqpoint{0.953872in}{0.588584in}}%
\pgfpathlineto{\pgfqpoint{0.963899in}{0.591908in}}%
\pgfpathlineto{\pgfqpoint{0.968912in}{0.593092in}}%
\pgfpathlineto{\pgfqpoint{0.973925in}{0.594121in}}%
\pgfpathlineto{\pgfqpoint{0.983952in}{0.595144in}}%
\pgfpathlineto{\pgfqpoint{0.993978in}{0.598130in}}%
\pgfpathlineto{\pgfqpoint{1.011525in}{0.600288in}}%
\pgfpathlineto{\pgfqpoint{1.014032in}{0.602838in}}%
\pgfpathlineto{\pgfqpoint{1.016538in}{0.603395in}}%
\pgfpathlineto{\pgfqpoint{1.019045in}{0.605259in}}%
\pgfpathlineto{\pgfqpoint{1.021552in}{0.605420in}}%
\pgfpathlineto{\pgfqpoint{1.026565in}{0.607850in}}%
\pgfpathlineto{\pgfqpoint{1.031578in}{0.609556in}}%
\pgfpathlineto{\pgfqpoint{1.041605in}{0.614551in}}%
\pgfpathlineto{\pgfqpoint{1.044112in}{0.614746in}}%
\pgfpathlineto{\pgfqpoint{1.046618in}{0.616128in}}%
\pgfpathlineto{\pgfqpoint{1.049125in}{0.616170in}}%
\pgfpathlineto{\pgfqpoint{1.051632in}{0.618221in}}%
\pgfpathlineto{\pgfqpoint{1.061658in}{0.620121in}}%
\pgfpathlineto{\pgfqpoint{1.066671in}{0.621543in}}%
\pgfpathlineto{\pgfqpoint{1.074191in}{0.622788in}}%
\pgfpathlineto{\pgfqpoint{1.076698in}{0.623736in}}%
\pgfpathlineto{\pgfqpoint{1.079205in}{0.626510in}}%
\pgfpathlineto{\pgfqpoint{1.084218in}{0.626853in}}%
\pgfpathlineto{\pgfqpoint{1.086725in}{0.628481in}}%
\pgfpathlineto{\pgfqpoint{1.096751in}{0.629637in}}%
\pgfpathlineto{\pgfqpoint{1.104271in}{0.632102in}}%
\pgfpathlineto{\pgfqpoint{1.179471in}{0.642934in}}%
\pgfpathlineto{\pgfqpoint{1.181978in}{0.644654in}}%
\pgfpathlineto{\pgfqpoint{1.184484in}{0.644672in}}%
\pgfpathlineto{\pgfqpoint{1.186991in}{0.647273in}}%
\pgfpathlineto{\pgfqpoint{1.194511in}{0.648914in}}%
\pgfpathlineto{\pgfqpoint{1.204537in}{0.651171in}}%
\pgfpathlineto{\pgfqpoint{1.212057in}{0.652132in}}%
\pgfpathlineto{\pgfqpoint{1.232111in}{0.654778in}}%
\pgfpathlineto{\pgfqpoint{1.239631in}{0.655343in}}%
\pgfpathlineto{\pgfqpoint{1.254671in}{0.656383in}}%
\pgfpathlineto{\pgfqpoint{1.267204in}{0.658713in}}%
\pgfpathlineto{\pgfqpoint{1.277230in}{0.661869in}}%
\pgfpathlineto{\pgfqpoint{1.282244in}{0.663801in}}%
\pgfpathlineto{\pgfqpoint{1.289764in}{0.664662in}}%
\pgfpathlineto{\pgfqpoint{1.292270in}{0.667457in}}%
\pgfpathlineto{\pgfqpoint{1.304804in}{0.668173in}}%
\pgfpathlineto{\pgfqpoint{1.317337in}{0.669909in}}%
\pgfpathlineto{\pgfqpoint{1.332377in}{0.671177in}}%
\pgfpathlineto{\pgfqpoint{1.362457in}{0.676020in}}%
\pgfpathlineto{\pgfqpoint{1.364963in}{0.677785in}}%
\pgfpathlineto{\pgfqpoint{1.372483in}{0.679502in}}%
\pgfpathlineto{\pgfqpoint{1.374990in}{0.680405in}}%
\pgfpathlineto{\pgfqpoint{1.377497in}{0.685027in}}%
\pgfpathlineto{\pgfqpoint{1.382510in}{0.686274in}}%
\pgfpathlineto{\pgfqpoint{1.392537in}{0.687299in}}%
\pgfpathlineto{\pgfqpoint{1.417603in}{0.691464in}}%
\pgfpathlineto{\pgfqpoint{1.425123in}{0.696017in}}%
\pgfpathlineto{\pgfqpoint{1.427630in}{0.696044in}}%
\pgfpathlineto{\pgfqpoint{1.430136in}{0.697735in}}%
\pgfpathlineto{\pgfqpoint{1.450190in}{0.699476in}}%
\pgfpathlineto{\pgfqpoint{1.460216in}{0.700360in}}%
\pgfpathlineto{\pgfqpoint{1.462723in}{0.706129in}}%
\pgfpathlineto{\pgfqpoint{1.472749in}{0.710288in}}%
\pgfpathlineto{\pgfqpoint{1.487789in}{0.712729in}}%
\pgfpathlineto{\pgfqpoint{1.497816in}{0.713815in}}%
\pgfpathlineto{\pgfqpoint{1.512856in}{0.717068in}}%
\pgfpathlineto{\pgfqpoint{1.520376in}{0.722182in}}%
\pgfpathlineto{\pgfqpoint{1.535416in}{0.725645in}}%
\pgfpathlineto{\pgfqpoint{1.537922in}{0.727800in}}%
\pgfpathlineto{\pgfqpoint{1.545442in}{0.729917in}}%
\pgfpathlineto{\pgfqpoint{1.547949in}{0.732216in}}%
\pgfpathlineto{\pgfqpoint{1.560482in}{0.733641in}}%
\pgfpathlineto{\pgfqpoint{1.570509in}{0.736989in}}%
\pgfpathlineto{\pgfqpoint{1.573016in}{0.738912in}}%
\pgfpathlineto{\pgfqpoint{1.575522in}{0.742304in}}%
\pgfpathlineto{\pgfqpoint{1.588056in}{0.745640in}}%
\pgfpathlineto{\pgfqpoint{1.598082in}{0.746667in}}%
\pgfpathlineto{\pgfqpoint{1.608109in}{0.749209in}}%
\pgfpathlineto{\pgfqpoint{1.610615in}{0.749318in}}%
\pgfpathlineto{\pgfqpoint{1.613122in}{0.752881in}}%
\pgfpathlineto{\pgfqpoint{1.623149in}{0.754289in}}%
\pgfpathlineto{\pgfqpoint{1.638189in}{0.757091in}}%
\pgfpathlineto{\pgfqpoint{1.643202in}{0.760033in}}%
\pgfpathlineto{\pgfqpoint{1.648215in}{0.761283in}}%
\pgfpathlineto{\pgfqpoint{1.655735in}{0.764752in}}%
\pgfpathlineto{\pgfqpoint{1.668268in}{0.769670in}}%
\pgfpathlineto{\pgfqpoint{1.670775in}{0.772769in}}%
\pgfpathlineto{\pgfqpoint{1.688322in}{0.775584in}}%
\pgfpathlineto{\pgfqpoint{1.695842in}{0.780791in}}%
\pgfpathlineto{\pgfqpoint{1.705868in}{0.781874in}}%
\pgfpathlineto{\pgfqpoint{1.715895in}{0.782332in}}%
\pgfpathlineto{\pgfqpoint{1.723415in}{0.789240in}}%
\pgfpathlineto{\pgfqpoint{1.728428in}{0.790582in}}%
\pgfpathlineto{\pgfqpoint{1.730935in}{0.791513in}}%
\pgfpathlineto{\pgfqpoint{1.733442in}{0.796039in}}%
\pgfpathlineto{\pgfqpoint{1.748481in}{0.800169in}}%
\pgfpathlineto{\pgfqpoint{1.753495in}{0.802491in}}%
\pgfpathlineto{\pgfqpoint{1.758508in}{0.802890in}}%
\pgfpathlineto{\pgfqpoint{1.761015in}{0.804589in}}%
\pgfpathlineto{\pgfqpoint{1.766028in}{0.805910in}}%
\pgfpathlineto{\pgfqpoint{1.771041in}{0.807425in}}%
\pgfpathlineto{\pgfqpoint{1.773548in}{0.810713in}}%
\pgfpathlineto{\pgfqpoint{1.781068in}{0.812707in}}%
\pgfpathlineto{\pgfqpoint{1.783575in}{0.812850in}}%
\pgfpathlineto{\pgfqpoint{1.788588in}{0.817000in}}%
\pgfpathlineto{\pgfqpoint{1.801121in}{0.818301in}}%
\pgfpathlineto{\pgfqpoint{1.803628in}{0.824442in}}%
\pgfpathlineto{\pgfqpoint{1.811148in}{0.825321in}}%
\pgfpathlineto{\pgfqpoint{1.816161in}{0.830549in}}%
\pgfpathlineto{\pgfqpoint{1.826188in}{0.832418in}}%
\pgfpathlineto{\pgfqpoint{1.831201in}{0.834952in}}%
\pgfpathlineto{\pgfqpoint{1.838721in}{0.839987in}}%
\pgfpathlineto{\pgfqpoint{1.841228in}{0.841935in}}%
\pgfpathlineto{\pgfqpoint{1.846241in}{0.842963in}}%
\pgfpathlineto{\pgfqpoint{1.853761in}{0.844714in}}%
\pgfpathlineto{\pgfqpoint{1.856268in}{0.847627in}}%
\pgfpathlineto{\pgfqpoint{1.861281in}{0.848589in}}%
\pgfpathlineto{\pgfqpoint{1.873814in}{0.852351in}}%
\pgfpathlineto{\pgfqpoint{1.878827in}{0.859231in}}%
\pgfpathlineto{\pgfqpoint{1.881334in}{0.859830in}}%
\pgfpathlineto{\pgfqpoint{1.883841in}{0.863101in}}%
\pgfpathlineto{\pgfqpoint{1.891361in}{0.864244in}}%
\pgfpathlineto{\pgfqpoint{1.903894in}{0.869164in}}%
\pgfpathlineto{\pgfqpoint{1.908907in}{0.870174in}}%
\pgfpathlineto{\pgfqpoint{1.913921in}{0.873265in}}%
\pgfpathlineto{\pgfqpoint{1.921441in}{0.874895in}}%
\pgfpathlineto{\pgfqpoint{1.928961in}{0.875598in}}%
\pgfpathlineto{\pgfqpoint{1.938987in}{0.879527in}}%
\pgfpathlineto{\pgfqpoint{1.944000in}{0.887126in}}%
\pgfpathlineto{\pgfqpoint{1.951520in}{0.889003in}}%
\pgfpathlineto{\pgfqpoint{1.954027in}{0.892885in}}%
\pgfpathlineto{\pgfqpoint{1.961547in}{0.893859in}}%
\pgfpathlineto{\pgfqpoint{1.969067in}{0.895600in}}%
\pgfpathlineto{\pgfqpoint{1.986614in}{0.905452in}}%
\pgfpathlineto{\pgfqpoint{1.994134in}{0.906282in}}%
\pgfpathlineto{\pgfqpoint{2.001654in}{0.913052in}}%
\pgfpathlineto{\pgfqpoint{2.019200in}{0.917518in}}%
\pgfpathlineto{\pgfqpoint{2.021707in}{0.922196in}}%
\pgfpathlineto{\pgfqpoint{2.029227in}{0.923427in}}%
\pgfpathlineto{\pgfqpoint{2.039253in}{0.928339in}}%
\pgfpathlineto{\pgfqpoint{2.041760in}{0.930898in}}%
\pgfpathlineto{\pgfqpoint{2.046773in}{0.931077in}}%
\pgfpathlineto{\pgfqpoint{2.049280in}{0.933521in}}%
\pgfpathlineto{\pgfqpoint{2.051787in}{0.937830in}}%
\pgfpathlineto{\pgfqpoint{2.054293in}{0.938508in}}%
\pgfpathlineto{\pgfqpoint{2.056800in}{0.941154in}}%
\pgfpathlineto{\pgfqpoint{2.071840in}{0.944592in}}%
\pgfpathlineto{\pgfqpoint{2.074346in}{0.945848in}}%
\pgfpathlineto{\pgfqpoint{2.079360in}{0.950861in}}%
\pgfpathlineto{\pgfqpoint{2.081866in}{0.953291in}}%
\pgfpathlineto{\pgfqpoint{2.086880in}{0.954244in}}%
\pgfpathlineto{\pgfqpoint{2.094400in}{0.956940in}}%
\pgfpathlineto{\pgfqpoint{2.096906in}{0.958086in}}%
\pgfpathlineto{\pgfqpoint{2.101920in}{0.962898in}}%
\pgfpathlineto{\pgfqpoint{2.104426in}{0.963447in}}%
\pgfpathlineto{\pgfqpoint{2.109440in}{0.970173in}}%
\pgfpathlineto{\pgfqpoint{2.111946in}{0.971768in}}%
\pgfpathlineto{\pgfqpoint{2.114453in}{0.977207in}}%
\pgfpathlineto{\pgfqpoint{2.119466in}{0.979255in}}%
\pgfpathlineto{\pgfqpoint{2.121973in}{0.981155in}}%
\pgfpathlineto{\pgfqpoint{2.124480in}{0.981429in}}%
\pgfpathlineto{\pgfqpoint{2.129493in}{0.985347in}}%
\pgfpathlineto{\pgfqpoint{2.132000in}{0.985662in}}%
\pgfpathlineto{\pgfqpoint{2.137013in}{0.993881in}}%
\pgfpathlineto{\pgfqpoint{2.139520in}{0.994865in}}%
\pgfpathlineto{\pgfqpoint{2.144533in}{0.999522in}}%
\pgfpathlineto{\pgfqpoint{2.149546in}{1.000301in}}%
\pgfpathlineto{\pgfqpoint{2.154559in}{1.000349in}}%
\pgfpathlineto{\pgfqpoint{2.162079in}{1.004920in}}%
\pgfpathlineto{\pgfqpoint{2.174613in}{1.006299in}}%
\pgfpathlineto{\pgfqpoint{2.177119in}{1.009296in}}%
\pgfpathlineto{\pgfqpoint{2.182133in}{1.010429in}}%
\pgfpathlineto{\pgfqpoint{2.187146in}{1.013103in}}%
\pgfpathlineto{\pgfqpoint{2.204693in}{1.017887in}}%
\pgfpathlineto{\pgfqpoint{2.207199in}{1.019809in}}%
\pgfpathlineto{\pgfqpoint{2.209706in}{1.019917in}}%
\pgfpathlineto{\pgfqpoint{2.212212in}{1.023572in}}%
\pgfpathlineto{\pgfqpoint{2.217226in}{1.025414in}}%
\pgfpathlineto{\pgfqpoint{2.219732in}{1.027640in}}%
\pgfpathlineto{\pgfqpoint{2.222239in}{1.027969in}}%
\pgfpathlineto{\pgfqpoint{2.224746in}{1.030737in}}%
\pgfpathlineto{\pgfqpoint{2.229759in}{1.031785in}}%
\pgfpathlineto{\pgfqpoint{2.237279in}{1.035629in}}%
\pgfpathlineto{\pgfqpoint{2.249812in}{1.036777in}}%
\pgfpathlineto{\pgfqpoint{2.252319in}{1.037509in}}%
\pgfpathlineto{\pgfqpoint{2.254826in}{1.039509in}}%
\pgfpathlineto{\pgfqpoint{2.257332in}{1.039600in}}%
\pgfpathlineto{\pgfqpoint{2.262346in}{1.041196in}}%
\pgfpathlineto{\pgfqpoint{2.264852in}{1.046523in}}%
\pgfpathlineto{\pgfqpoint{2.274879in}{1.051292in}}%
\pgfpathlineto{\pgfqpoint{2.279892in}{1.051717in}}%
\pgfpathlineto{\pgfqpoint{2.282399in}{1.054952in}}%
\pgfpathlineto{\pgfqpoint{2.287412in}{1.055751in}}%
\pgfpathlineto{\pgfqpoint{2.292425in}{1.057180in}}%
\pgfpathlineto{\pgfqpoint{2.294932in}{1.061518in}}%
\pgfpathlineto{\pgfqpoint{2.297439in}{1.069349in}}%
\pgfpathlineto{\pgfqpoint{2.299945in}{1.071411in}}%
\pgfpathlineto{\pgfqpoint{2.304959in}{1.073111in}}%
\pgfpathlineto{\pgfqpoint{2.309972in}{1.076582in}}%
\pgfpathlineto{\pgfqpoint{2.314985in}{1.085103in}}%
\pgfpathlineto{\pgfqpoint{2.317492in}{1.086930in}}%
\pgfpathlineto{\pgfqpoint{2.322505in}{1.088044in}}%
\pgfpathlineto{\pgfqpoint{2.325012in}{1.090339in}}%
\pgfpathlineto{\pgfqpoint{2.330025in}{1.092005in}}%
\pgfpathlineto{\pgfqpoint{2.345065in}{1.097205in}}%
\pgfpathlineto{\pgfqpoint{2.352585in}{1.101993in}}%
\pgfpathlineto{\pgfqpoint{2.357598in}{1.103025in}}%
\pgfpathlineto{\pgfqpoint{2.365118in}{1.105050in}}%
\pgfpathlineto{\pgfqpoint{2.367625in}{1.105070in}}%
\pgfpathlineto{\pgfqpoint{2.370132in}{1.107778in}}%
\pgfpathlineto{\pgfqpoint{2.375145in}{1.108756in}}%
\pgfpathlineto{\pgfqpoint{2.377652in}{1.113579in}}%
\pgfpathlineto{\pgfqpoint{2.390185in}{1.118338in}}%
\pgfpathlineto{\pgfqpoint{2.392692in}{1.121215in}}%
\pgfpathlineto{\pgfqpoint{2.427785in}{1.136608in}}%
\pgfpathlineto{\pgfqpoint{2.437811in}{1.137729in}}%
\pgfpathlineto{\pgfqpoint{2.440318in}{1.143082in}}%
\pgfpathlineto{\pgfqpoint{2.442825in}{1.144904in}}%
\pgfpathlineto{\pgfqpoint{2.450345in}{1.146324in}}%
\pgfpathlineto{\pgfqpoint{2.457865in}{1.149231in}}%
\pgfpathlineto{\pgfqpoint{2.460371in}{1.149624in}}%
\pgfpathlineto{\pgfqpoint{2.462878in}{1.151405in}}%
\pgfpathlineto{\pgfqpoint{2.465385in}{1.151562in}}%
\pgfpathlineto{\pgfqpoint{2.470398in}{1.154806in}}%
\pgfpathlineto{\pgfqpoint{2.472905in}{1.154813in}}%
\pgfpathlineto{\pgfqpoint{2.475411in}{1.161637in}}%
\pgfpathlineto{\pgfqpoint{2.480425in}{1.163255in}}%
\pgfpathlineto{\pgfqpoint{2.482931in}{1.163548in}}%
\pgfpathlineto{\pgfqpoint{2.487944in}{1.167212in}}%
\pgfpathlineto{\pgfqpoint{2.490451in}{1.172514in}}%
\pgfpathlineto{\pgfqpoint{2.497971in}{1.174594in}}%
\pgfpathlineto{\pgfqpoint{2.502984in}{1.176368in}}%
\pgfpathlineto{\pgfqpoint{2.510504in}{1.177343in}}%
\pgfpathlineto{\pgfqpoint{2.520531in}{1.185796in}}%
\pgfpathlineto{\pgfqpoint{2.523038in}{1.189516in}}%
\pgfpathlineto{\pgfqpoint{2.538078in}{1.195020in}}%
\pgfpathlineto{\pgfqpoint{2.540584in}{1.197672in}}%
\pgfpathlineto{\pgfqpoint{2.543091in}{1.197939in}}%
\pgfpathlineto{\pgfqpoint{2.548104in}{1.200347in}}%
\pgfpathlineto{\pgfqpoint{2.550611in}{1.201048in}}%
\pgfpathlineto{\pgfqpoint{2.555624in}{1.205976in}}%
\pgfpathlineto{\pgfqpoint{2.560637in}{1.209153in}}%
\pgfpathlineto{\pgfqpoint{2.583197in}{1.214427in}}%
\pgfpathlineto{\pgfqpoint{2.585704in}{1.216349in}}%
\pgfpathlineto{\pgfqpoint{2.598237in}{1.219331in}}%
\pgfpathlineto{\pgfqpoint{2.600744in}{1.223434in}}%
\pgfpathlineto{\pgfqpoint{2.605757in}{1.226438in}}%
\pgfpathlineto{\pgfqpoint{2.608264in}{1.226549in}}%
\pgfpathlineto{\pgfqpoint{2.620797in}{1.232069in}}%
\pgfpathlineto{\pgfqpoint{2.625810in}{1.232846in}}%
\pgfpathlineto{\pgfqpoint{2.630824in}{1.234871in}}%
\pgfpathlineto{\pgfqpoint{2.640850in}{1.236915in}}%
\pgfpathlineto{\pgfqpoint{2.648370in}{1.239920in}}%
\pgfpathlineto{\pgfqpoint{2.650877in}{1.240400in}}%
\pgfpathlineto{\pgfqpoint{2.653384in}{1.242465in}}%
\pgfpathlineto{\pgfqpoint{2.660904in}{1.244051in}}%
\pgfpathlineto{\pgfqpoint{2.663410in}{1.250051in}}%
\pgfpathlineto{\pgfqpoint{2.668424in}{1.252187in}}%
\pgfpathlineto{\pgfqpoint{2.670930in}{1.255874in}}%
\pgfpathlineto{\pgfqpoint{2.680957in}{1.257134in}}%
\pgfpathlineto{\pgfqpoint{2.688477in}{1.264037in}}%
\pgfpathlineto{\pgfqpoint{2.693490in}{1.265108in}}%
\pgfpathlineto{\pgfqpoint{2.698503in}{1.265662in}}%
\pgfpathlineto{\pgfqpoint{2.706023in}{1.271568in}}%
\pgfpathlineto{\pgfqpoint{2.711037in}{1.280975in}}%
\pgfpathlineto{\pgfqpoint{2.713543in}{1.281284in}}%
\pgfpathlineto{\pgfqpoint{2.718557in}{1.285781in}}%
\pgfpathlineto{\pgfqpoint{2.746130in}{1.295689in}}%
\pgfpathlineto{\pgfqpoint{2.748637in}{1.299647in}}%
\pgfpathlineto{\pgfqpoint{2.758663in}{1.305239in}}%
\pgfpathlineto{\pgfqpoint{2.761170in}{1.305363in}}%
\pgfpathlineto{\pgfqpoint{2.763676in}{1.306820in}}%
\pgfpathlineto{\pgfqpoint{2.766183in}{1.317734in}}%
\pgfpathlineto{\pgfqpoint{2.768690in}{1.322885in}}%
\pgfpathlineto{\pgfqpoint{2.778716in}{1.325861in}}%
\pgfpathlineto{\pgfqpoint{2.781223in}{1.330655in}}%
\pgfpathlineto{\pgfqpoint{2.783730in}{1.330961in}}%
\pgfpathlineto{\pgfqpoint{2.786236in}{1.332969in}}%
\pgfpathlineto{\pgfqpoint{2.788743in}{1.333056in}}%
\pgfpathlineto{\pgfqpoint{2.793756in}{1.335829in}}%
\pgfpathlineto{\pgfqpoint{2.798770in}{1.343640in}}%
\pgfpathlineto{\pgfqpoint{2.803783in}{1.345075in}}%
\pgfpathlineto{\pgfqpoint{2.816316in}{1.349642in}}%
\pgfpathlineto{\pgfqpoint{2.823836in}{1.354124in}}%
\pgfpathlineto{\pgfqpoint{2.826343in}{1.354406in}}%
\pgfpathlineto{\pgfqpoint{2.828849in}{1.356825in}}%
\pgfpathlineto{\pgfqpoint{2.833863in}{1.357897in}}%
\pgfpathlineto{\pgfqpoint{2.836369in}{1.364466in}}%
\pgfpathlineto{\pgfqpoint{2.843889in}{1.368232in}}%
\pgfpathlineto{\pgfqpoint{2.851409in}{1.375745in}}%
\pgfpathlineto{\pgfqpoint{2.853916in}{1.375921in}}%
\pgfpathlineto{\pgfqpoint{2.856423in}{1.378965in}}%
\pgfpathlineto{\pgfqpoint{2.861436in}{1.380304in}}%
\pgfpathlineto{\pgfqpoint{2.863943in}{1.382488in}}%
\pgfpathlineto{\pgfqpoint{2.866449in}{1.382703in}}%
\pgfpathlineto{\pgfqpoint{2.868956in}{1.384473in}}%
\pgfpathlineto{\pgfqpoint{2.873969in}{1.384557in}}%
\pgfpathlineto{\pgfqpoint{2.876476in}{1.389512in}}%
\pgfpathlineto{\pgfqpoint{2.881489in}{1.390847in}}%
\pgfpathlineto{\pgfqpoint{2.886503in}{1.393112in}}%
\pgfpathlineto{\pgfqpoint{2.889009in}{1.393306in}}%
\pgfpathlineto{\pgfqpoint{2.891516in}{1.396259in}}%
\pgfpathlineto{\pgfqpoint{2.894022in}{1.408141in}}%
\pgfpathlineto{\pgfqpoint{2.906556in}{1.414531in}}%
\pgfpathlineto{\pgfqpoint{2.909062in}{1.419227in}}%
\pgfpathlineto{\pgfqpoint{2.916582in}{1.420032in}}%
\pgfpathlineto{\pgfqpoint{2.934129in}{1.424028in}}%
\pgfpathlineto{\pgfqpoint{2.944156in}{1.431623in}}%
\pgfpathlineto{\pgfqpoint{2.946662in}{1.432710in}}%
\pgfpathlineto{\pgfqpoint{2.951676in}{1.436259in}}%
\pgfpathlineto{\pgfqpoint{2.961702in}{1.438540in}}%
\pgfpathlineto{\pgfqpoint{2.966715in}{1.445818in}}%
\pgfpathlineto{\pgfqpoint{2.971729in}{1.446882in}}%
\pgfpathlineto{\pgfqpoint{2.976742in}{1.448058in}}%
\pgfpathlineto{\pgfqpoint{2.981755in}{1.456027in}}%
\pgfpathlineto{\pgfqpoint{2.984262in}{1.456347in}}%
\pgfpathlineto{\pgfqpoint{2.986769in}{1.458314in}}%
\pgfpathlineto{\pgfqpoint{2.989275in}{1.458480in}}%
\pgfpathlineto{\pgfqpoint{2.991782in}{1.459948in}}%
\pgfpathlineto{\pgfqpoint{2.994289in}{1.460120in}}%
\pgfpathlineto{\pgfqpoint{2.996795in}{1.470190in}}%
\pgfpathlineto{\pgfqpoint{2.999302in}{1.470269in}}%
\pgfpathlineto{\pgfqpoint{3.006822in}{1.475844in}}%
\pgfpathlineto{\pgfqpoint{3.016849in}{1.476720in}}%
\pgfpathlineto{\pgfqpoint{3.019355in}{1.478794in}}%
\pgfpathlineto{\pgfqpoint{3.021862in}{1.479075in}}%
\pgfpathlineto{\pgfqpoint{3.029382in}{1.486254in}}%
\pgfpathlineto{\pgfqpoint{3.031888in}{1.489615in}}%
\pgfpathlineto{\pgfqpoint{3.036902in}{1.491180in}}%
\pgfpathlineto{\pgfqpoint{3.039408in}{1.493189in}}%
\pgfpathlineto{\pgfqpoint{3.041915in}{1.493473in}}%
\pgfpathlineto{\pgfqpoint{3.044422in}{1.497013in}}%
\pgfpathlineto{\pgfqpoint{3.049435in}{1.497735in}}%
\pgfpathlineto{\pgfqpoint{3.054448in}{1.501845in}}%
\pgfpathlineto{\pgfqpoint{3.059462in}{1.502200in}}%
\pgfpathlineto{\pgfqpoint{3.064475in}{1.505959in}}%
\pgfpathlineto{\pgfqpoint{3.069488in}{1.507831in}}%
\pgfpathlineto{\pgfqpoint{3.074502in}{1.511513in}}%
\pgfpathlineto{\pgfqpoint{3.077008in}{1.513172in}}%
\pgfpathlineto{\pgfqpoint{3.082022in}{1.519434in}}%
\pgfpathlineto{\pgfqpoint{3.084528in}{1.519829in}}%
\pgfpathlineto{\pgfqpoint{3.087035in}{1.521744in}}%
\pgfpathlineto{\pgfqpoint{3.089542in}{1.525236in}}%
\pgfpathlineto{\pgfqpoint{3.092048in}{1.525373in}}%
\pgfpathlineto{\pgfqpoint{3.097061in}{1.530624in}}%
\pgfpathlineto{\pgfqpoint{3.099568in}{1.532130in}}%
\pgfpathlineto{\pgfqpoint{3.104581in}{1.533172in}}%
\pgfpathlineto{\pgfqpoint{3.119621in}{1.542425in}}%
\pgfpathlineto{\pgfqpoint{3.122128in}{1.549139in}}%
\pgfpathlineto{\pgfqpoint{3.124635in}{1.551673in}}%
\pgfpathlineto{\pgfqpoint{3.127141in}{1.552217in}}%
\pgfpathlineto{\pgfqpoint{3.129648in}{1.559961in}}%
\pgfpathlineto{\pgfqpoint{3.132155in}{1.560804in}}%
\pgfpathlineto{\pgfqpoint{3.134661in}{1.565078in}}%
\pgfpathlineto{\pgfqpoint{3.139675in}{1.566138in}}%
\pgfpathlineto{\pgfqpoint{3.144688in}{1.570258in}}%
\pgfpathlineto{\pgfqpoint{3.154715in}{1.572335in}}%
\pgfpathlineto{\pgfqpoint{3.167248in}{1.582186in}}%
\pgfpathlineto{\pgfqpoint{3.174768in}{1.582488in}}%
\pgfpathlineto{\pgfqpoint{3.177274in}{1.584600in}}%
\pgfpathlineto{\pgfqpoint{3.182288in}{1.585599in}}%
\pgfpathlineto{\pgfqpoint{3.184794in}{1.589953in}}%
\pgfpathlineto{\pgfqpoint{3.189808in}{1.591216in}}%
\pgfpathlineto{\pgfqpoint{3.197328in}{1.597077in}}%
\pgfpathlineto{\pgfqpoint{3.199834in}{1.598648in}}%
\pgfpathlineto{\pgfqpoint{3.204848in}{1.603963in}}%
\pgfpathlineto{\pgfqpoint{3.217381in}{1.606186in}}%
\pgfpathlineto{\pgfqpoint{3.224901in}{1.607434in}}%
\pgfpathlineto{\pgfqpoint{3.239941in}{1.615058in}}%
\pgfpathlineto{\pgfqpoint{3.244954in}{1.620292in}}%
\pgfpathlineto{\pgfqpoint{3.247461in}{1.625982in}}%
\pgfpathlineto{\pgfqpoint{3.249967in}{1.626543in}}%
\pgfpathlineto{\pgfqpoint{3.252474in}{1.628753in}}%
\pgfpathlineto{\pgfqpoint{3.257487in}{1.629523in}}%
\pgfpathlineto{\pgfqpoint{3.259994in}{1.637195in}}%
\pgfpathlineto{\pgfqpoint{3.262501in}{1.639671in}}%
\pgfpathlineto{\pgfqpoint{3.267514in}{1.641312in}}%
\pgfpathlineto{\pgfqpoint{3.272527in}{1.644208in}}%
\pgfpathlineto{\pgfqpoint{3.282554in}{1.651362in}}%
\pgfpathlineto{\pgfqpoint{3.287567in}{1.652488in}}%
\pgfpathlineto{\pgfqpoint{3.292581in}{1.657317in}}%
\pgfpathlineto{\pgfqpoint{3.300100in}{1.657937in}}%
\pgfpathlineto{\pgfqpoint{3.307620in}{1.659633in}}%
\pgfpathlineto{\pgfqpoint{3.315140in}{1.669427in}}%
\pgfpathlineto{\pgfqpoint{3.322660in}{1.671380in}}%
\pgfpathlineto{\pgfqpoint{3.325167in}{1.671932in}}%
\pgfpathlineto{\pgfqpoint{3.327674in}{1.673846in}}%
\pgfpathlineto{\pgfqpoint{3.332687in}{1.682351in}}%
\pgfpathlineto{\pgfqpoint{3.337700in}{1.685122in}}%
\pgfpathlineto{\pgfqpoint{3.340207in}{1.685236in}}%
\pgfpathlineto{\pgfqpoint{3.342714in}{1.686933in}}%
\pgfpathlineto{\pgfqpoint{3.350234in}{1.695799in}}%
\pgfpathlineto{\pgfqpoint{3.352740in}{1.698078in}}%
\pgfpathlineto{\pgfqpoint{3.360260in}{1.700567in}}%
\pgfpathlineto{\pgfqpoint{3.362767in}{1.702619in}}%
\pgfpathlineto{\pgfqpoint{3.370287in}{1.712474in}}%
\pgfpathlineto{\pgfqpoint{3.375300in}{1.715500in}}%
\pgfpathlineto{\pgfqpoint{3.377807in}{1.719950in}}%
\pgfpathlineto{\pgfqpoint{3.392847in}{1.724253in}}%
\pgfpathlineto{\pgfqpoint{3.395353in}{1.725196in}}%
\pgfpathlineto{\pgfqpoint{3.397860in}{1.728971in}}%
\pgfpathlineto{\pgfqpoint{3.400367in}{1.737827in}}%
\pgfpathlineto{\pgfqpoint{3.407887in}{1.743735in}}%
\pgfpathlineto{\pgfqpoint{3.410393in}{1.744728in}}%
\pgfpathlineto{\pgfqpoint{3.415407in}{1.748537in}}%
\pgfpathlineto{\pgfqpoint{3.420420in}{1.749169in}}%
\pgfpathlineto{\pgfqpoint{3.425433in}{1.755532in}}%
\pgfpathlineto{\pgfqpoint{3.427940in}{1.758270in}}%
\pgfpathlineto{\pgfqpoint{3.430447in}{1.767207in}}%
\pgfpathlineto{\pgfqpoint{3.442980in}{1.782766in}}%
\pgfpathlineto{\pgfqpoint{3.445486in}{1.787727in}}%
\pgfpathlineto{\pgfqpoint{3.450500in}{1.788540in}}%
\pgfpathlineto{\pgfqpoint{3.453006in}{1.792102in}}%
\pgfpathlineto{\pgfqpoint{3.460526in}{1.792950in}}%
\pgfpathlineto{\pgfqpoint{3.465540in}{1.796666in}}%
\pgfpathlineto{\pgfqpoint{3.470553in}{1.797506in}}%
\pgfpathlineto{\pgfqpoint{3.475566in}{1.801708in}}%
\pgfpathlineto{\pgfqpoint{3.490606in}{1.811337in}}%
\pgfpathlineto{\pgfqpoint{3.493113in}{1.812322in}}%
\pgfpathlineto{\pgfqpoint{3.498126in}{1.821762in}}%
\pgfpathlineto{\pgfqpoint{3.500633in}{1.829504in}}%
\pgfpathlineto{\pgfqpoint{3.503139in}{1.829674in}}%
\pgfpathlineto{\pgfqpoint{3.505646in}{1.832061in}}%
\pgfpathlineto{\pgfqpoint{3.513166in}{1.835175in}}%
\pgfpathlineto{\pgfqpoint{3.518179in}{1.840328in}}%
\pgfpathlineto{\pgfqpoint{3.528206in}{1.845522in}}%
\pgfpathlineto{\pgfqpoint{3.533219in}{1.846623in}}%
\pgfpathlineto{\pgfqpoint{3.540739in}{1.858416in}}%
\pgfpathlineto{\pgfqpoint{3.543246in}{1.858673in}}%
\pgfpathlineto{\pgfqpoint{3.548259in}{1.860678in}}%
\pgfpathlineto{\pgfqpoint{3.560793in}{1.866046in}}%
\pgfpathlineto{\pgfqpoint{3.585859in}{1.869359in}}%
\pgfpathlineto{\pgfqpoint{3.590872in}{1.870944in}}%
\pgfpathlineto{\pgfqpoint{3.593379in}{1.871276in}}%
\pgfpathlineto{\pgfqpoint{3.595886in}{1.874523in}}%
\pgfpathlineto{\pgfqpoint{3.605912in}{1.876533in}}%
\pgfpathlineto{\pgfqpoint{3.608419in}{1.880976in}}%
\pgfpathlineto{\pgfqpoint{3.620952in}{1.890901in}}%
\pgfpathlineto{\pgfqpoint{3.625966in}{1.905699in}}%
\pgfpathlineto{\pgfqpoint{3.638499in}{1.919455in}}%
\pgfpathlineto{\pgfqpoint{3.643512in}{1.919966in}}%
\pgfpathlineto{\pgfqpoint{3.646019in}{1.922925in}}%
\pgfpathlineto{\pgfqpoint{3.668579in}{1.930248in}}%
\pgfpathlineto{\pgfqpoint{3.673592in}{1.934233in}}%
\pgfpathlineto{\pgfqpoint{3.676099in}{1.941746in}}%
\pgfpathlineto{\pgfqpoint{3.678605in}{1.941984in}}%
\pgfpathlineto{\pgfqpoint{3.683619in}{1.951827in}}%
\pgfpathlineto{\pgfqpoint{3.688632in}{1.954511in}}%
\pgfpathlineto{\pgfqpoint{3.691139in}{1.954901in}}%
\pgfpathlineto{\pgfqpoint{3.698659in}{1.959703in}}%
\pgfpathlineto{\pgfqpoint{3.703672in}{1.967007in}}%
\pgfpathlineto{\pgfqpoint{3.706178in}{1.967071in}}%
\pgfpathlineto{\pgfqpoint{3.716205in}{1.971502in}}%
\pgfpathlineto{\pgfqpoint{3.718712in}{1.975063in}}%
\pgfpathlineto{\pgfqpoint{3.721218in}{1.975827in}}%
\pgfpathlineto{\pgfqpoint{3.723725in}{1.980105in}}%
\pgfpathlineto{\pgfqpoint{3.726232in}{1.981368in}}%
\pgfpathlineto{\pgfqpoint{3.728738in}{1.991007in}}%
\pgfpathlineto{\pgfqpoint{3.736258in}{1.993430in}}%
\pgfpathlineto{\pgfqpoint{3.738765in}{1.994195in}}%
\pgfpathlineto{\pgfqpoint{3.743778in}{1.998738in}}%
\pgfpathlineto{\pgfqpoint{3.746285in}{2.004604in}}%
\pgfpathlineto{\pgfqpoint{3.748792in}{2.004737in}}%
\pgfpathlineto{\pgfqpoint{3.753805in}{2.015763in}}%
\pgfpathlineto{\pgfqpoint{3.756312in}{2.016090in}}%
\pgfpathlineto{\pgfqpoint{3.761325in}{2.019325in}}%
\pgfpathlineto{\pgfqpoint{3.768845in}{2.028701in}}%
\pgfpathlineto{\pgfqpoint{3.773858in}{2.030780in}}%
\pgfpathlineto{\pgfqpoint{3.778871in}{2.033136in}}%
\pgfpathlineto{\pgfqpoint{3.781378in}{2.038218in}}%
\pgfpathlineto{\pgfqpoint{3.791405in}{2.041762in}}%
\pgfpathlineto{\pgfqpoint{3.793911in}{2.047196in}}%
\pgfpathlineto{\pgfqpoint{3.796418in}{2.049529in}}%
\pgfpathlineto{\pgfqpoint{3.798925in}{2.050063in}}%
\pgfpathlineto{\pgfqpoint{3.801431in}{2.057101in}}%
\pgfpathlineto{\pgfqpoint{3.803938in}{2.058746in}}%
\pgfpathlineto{\pgfqpoint{3.808951in}{2.067788in}}%
\pgfpathlineto{\pgfqpoint{3.813965in}{2.071624in}}%
\pgfpathlineto{\pgfqpoint{3.823991in}{2.084144in}}%
\pgfpathlineto{\pgfqpoint{3.826498in}{2.085971in}}%
\pgfpathlineto{\pgfqpoint{3.829005in}{2.086257in}}%
\pgfpathlineto{\pgfqpoint{3.831511in}{2.088994in}}%
\pgfpathlineto{\pgfqpoint{3.839031in}{2.090426in}}%
\pgfpathlineto{\pgfqpoint{3.846551in}{2.092581in}}%
\pgfpathlineto{\pgfqpoint{3.849058in}{2.092780in}}%
\pgfpathlineto{\pgfqpoint{3.851564in}{2.096860in}}%
\pgfpathlineto{\pgfqpoint{3.859084in}{2.097739in}}%
\pgfpathlineto{\pgfqpoint{3.864098in}{2.102329in}}%
\pgfpathlineto{\pgfqpoint{3.871618in}{2.113799in}}%
\pgfpathlineto{\pgfqpoint{3.884151in}{2.116217in}}%
\pgfpathlineto{\pgfqpoint{3.889164in}{2.120118in}}%
\pgfpathlineto{\pgfqpoint{3.891671in}{2.121086in}}%
\pgfpathlineto{\pgfqpoint{3.894178in}{2.124397in}}%
\pgfpathlineto{\pgfqpoint{3.901698in}{2.125773in}}%
\pgfpathlineto{\pgfqpoint{3.911724in}{2.129017in}}%
\pgfpathlineto{\pgfqpoint{3.929271in}{2.132679in}}%
\pgfpathlineto{\pgfqpoint{3.931777in}{2.137878in}}%
\pgfpathlineto{\pgfqpoint{3.936791in}{2.140024in}}%
\pgfpathlineto{\pgfqpoint{3.939297in}{2.141135in}}%
\pgfpathlineto{\pgfqpoint{3.941804in}{2.146270in}}%
\pgfpathlineto{\pgfqpoint{3.949324in}{2.148973in}}%
\pgfpathlineto{\pgfqpoint{3.951831in}{2.152177in}}%
\pgfpathlineto{\pgfqpoint{3.954337in}{2.153323in}}%
\pgfpathlineto{\pgfqpoint{3.956844in}{2.159126in}}%
\pgfpathlineto{\pgfqpoint{3.959351in}{2.161660in}}%
\pgfpathlineto{\pgfqpoint{3.971884in}{2.163963in}}%
\pgfpathlineto{\pgfqpoint{3.976897in}{2.165304in}}%
\pgfpathlineto{\pgfqpoint{3.981910in}{2.166421in}}%
\pgfpathlineto{\pgfqpoint{3.986924in}{2.170600in}}%
\pgfpathlineto{\pgfqpoint{3.989430in}{2.172637in}}%
\pgfpathlineto{\pgfqpoint{3.991937in}{2.176129in}}%
\pgfpathlineto{\pgfqpoint{3.999457in}{2.176353in}}%
\pgfpathlineto{\pgfqpoint{4.004470in}{2.178591in}}%
\pgfpathlineto{\pgfqpoint{4.009484in}{2.183264in}}%
\pgfpathlineto{\pgfqpoint{4.017004in}{2.186250in}}%
\pgfpathlineto{\pgfqpoint{4.019510in}{2.191246in}}%
\pgfpathlineto{\pgfqpoint{4.024524in}{2.192762in}}%
\pgfpathlineto{\pgfqpoint{4.027030in}{2.197573in}}%
\pgfpathlineto{\pgfqpoint{4.042070in}{2.206300in}}%
\pgfpathlineto{\pgfqpoint{4.044577in}{2.206532in}}%
\pgfpathlineto{\pgfqpoint{4.047083in}{2.208995in}}%
\pgfpathlineto{\pgfqpoint{4.052097in}{2.209212in}}%
\pgfpathlineto{\pgfqpoint{4.054603in}{2.211730in}}%
\pgfpathlineto{\pgfqpoint{4.057110in}{2.217334in}}%
\pgfpathlineto{\pgfqpoint{4.062123in}{2.219840in}}%
\pgfpathlineto{\pgfqpoint{4.064630in}{2.220312in}}%
\pgfpathlineto{\pgfqpoint{4.067137in}{2.223916in}}%
\pgfpathlineto{\pgfqpoint{4.069643in}{2.224433in}}%
\pgfpathlineto{\pgfqpoint{4.072150in}{2.228587in}}%
\pgfpathlineto{\pgfqpoint{4.074657in}{2.229229in}}%
\pgfpathlineto{\pgfqpoint{4.077163in}{2.234119in}}%
\pgfpathlineto{\pgfqpoint{4.097217in}{2.240151in}}%
\pgfpathlineto{\pgfqpoint{4.102230in}{2.244829in}}%
\pgfpathlineto{\pgfqpoint{4.107243in}{2.250053in}}%
\pgfpathlineto{\pgfqpoint{4.109750in}{2.252494in}}%
\pgfpathlineto{\pgfqpoint{4.112257in}{2.252684in}}%
\pgfpathlineto{\pgfqpoint{4.114763in}{2.255296in}}%
\pgfpathlineto{\pgfqpoint{4.119776in}{2.256220in}}%
\pgfpathlineto{\pgfqpoint{4.122283in}{2.257413in}}%
\pgfpathlineto{\pgfqpoint{4.124790in}{2.264882in}}%
\pgfpathlineto{\pgfqpoint{4.132310in}{2.270504in}}%
\pgfpathlineto{\pgfqpoint{4.134816in}{2.270717in}}%
\pgfpathlineto{\pgfqpoint{4.137323in}{2.273902in}}%
\pgfpathlineto{\pgfqpoint{4.142336in}{2.274855in}}%
\pgfpathlineto{\pgfqpoint{4.144843in}{2.279559in}}%
\pgfpathlineto{\pgfqpoint{4.147350in}{2.279669in}}%
\pgfpathlineto{\pgfqpoint{4.149856in}{2.284811in}}%
\pgfpathlineto{\pgfqpoint{4.157376in}{2.289859in}}%
\pgfpathlineto{\pgfqpoint{4.159883in}{2.292770in}}%
\pgfpathlineto{\pgfqpoint{4.164896in}{2.306457in}}%
\pgfpathlineto{\pgfqpoint{4.167403in}{2.306759in}}%
\pgfpathlineto{\pgfqpoint{4.172416in}{2.310456in}}%
\pgfpathlineto{\pgfqpoint{4.174923in}{2.314100in}}%
\pgfpathlineto{\pgfqpoint{4.182443in}{2.316222in}}%
\pgfpathlineto{\pgfqpoint{4.189963in}{2.321537in}}%
\pgfpathlineto{\pgfqpoint{4.192469in}{2.324700in}}%
\pgfpathlineto{\pgfqpoint{4.194976in}{2.324724in}}%
\pgfpathlineto{\pgfqpoint{4.197483in}{2.327461in}}%
\pgfpathlineto{\pgfqpoint{4.199989in}{2.327532in}}%
\pgfpathlineto{\pgfqpoint{4.205003in}{2.330614in}}%
\pgfpathlineto{\pgfqpoint{4.207509in}{2.336239in}}%
\pgfpathlineto{\pgfqpoint{4.212523in}{2.336797in}}%
\pgfpathlineto{\pgfqpoint{4.215029in}{2.341123in}}%
\pgfpathlineto{\pgfqpoint{4.217536in}{2.342354in}}%
\pgfpathlineto{\pgfqpoint{4.220043in}{2.345005in}}%
\pgfpathlineto{\pgfqpoint{4.225056in}{2.346391in}}%
\pgfpathlineto{\pgfqpoint{4.227563in}{2.349474in}}%
\pgfpathlineto{\pgfqpoint{4.230069in}{2.349846in}}%
\pgfpathlineto{\pgfqpoint{4.232576in}{2.352985in}}%
\pgfpathlineto{\pgfqpoint{4.237589in}{2.361292in}}%
\pgfpathlineto{\pgfqpoint{4.240096in}{2.363263in}}%
\pgfpathlineto{\pgfqpoint{4.245109in}{2.363606in}}%
\pgfpathlineto{\pgfqpoint{4.250122in}{2.366452in}}%
\pgfpathlineto{\pgfqpoint{4.252629in}{2.366810in}}%
\pgfpathlineto{\pgfqpoint{4.257642in}{2.371803in}}%
\pgfpathlineto{\pgfqpoint{4.260149in}{2.374844in}}%
\pgfpathlineto{\pgfqpoint{4.262656in}{2.375115in}}%
\pgfpathlineto{\pgfqpoint{4.267669in}{2.383948in}}%
\pgfpathlineto{\pgfqpoint{4.270176in}{2.384931in}}%
\pgfpathlineto{\pgfqpoint{4.272682in}{2.394811in}}%
\pgfpathlineto{\pgfqpoint{4.277696in}{2.398431in}}%
\pgfpathlineto{\pgfqpoint{4.285216in}{2.407180in}}%
\pgfpathlineto{\pgfqpoint{4.287722in}{2.407270in}}%
\pgfpathlineto{\pgfqpoint{4.290229in}{2.411396in}}%
\pgfpathlineto{\pgfqpoint{4.292736in}{2.411432in}}%
\pgfpathlineto{\pgfqpoint{4.295242in}{2.420636in}}%
\pgfpathlineto{\pgfqpoint{4.300256in}{2.421796in}}%
\pgfpathlineto{\pgfqpoint{4.305269in}{2.423262in}}%
\pgfpathlineto{\pgfqpoint{4.310282in}{2.425824in}}%
\pgfpathlineto{\pgfqpoint{4.312789in}{2.431803in}}%
\pgfpathlineto{\pgfqpoint{4.317802in}{2.433123in}}%
\pgfpathlineto{\pgfqpoint{4.320309in}{2.437218in}}%
\pgfpathlineto{\pgfqpoint{4.322815in}{2.445699in}}%
\pgfpathlineto{\pgfqpoint{4.325322in}{2.447035in}}%
\pgfpathlineto{\pgfqpoint{4.327829in}{2.450483in}}%
\pgfpathlineto{\pgfqpoint{4.335349in}{2.451776in}}%
\pgfpathlineto{\pgfqpoint{4.352895in}{2.461786in}}%
\pgfpathlineto{\pgfqpoint{4.355402in}{2.462026in}}%
\pgfpathlineto{\pgfqpoint{4.357909in}{2.466444in}}%
\pgfpathlineto{\pgfqpoint{4.360415in}{2.476736in}}%
\pgfpathlineto{\pgfqpoint{4.372949in}{2.485127in}}%
\pgfpathlineto{\pgfqpoint{4.377962in}{2.490365in}}%
\pgfpathlineto{\pgfqpoint{4.382975in}{2.491512in}}%
\pgfpathlineto{\pgfqpoint{4.385482in}{2.492468in}}%
\pgfpathlineto{\pgfqpoint{4.393002in}{2.499031in}}%
\pgfpathlineto{\pgfqpoint{4.403028in}{2.505149in}}%
\pgfpathlineto{\pgfqpoint{4.405535in}{2.509470in}}%
\pgfpathlineto{\pgfqpoint{4.408042in}{2.511312in}}%
\pgfpathlineto{\pgfqpoint{4.410548in}{2.515341in}}%
\pgfpathlineto{\pgfqpoint{4.413055in}{2.516632in}}%
\pgfpathlineto{\pgfqpoint{4.415562in}{2.520795in}}%
\pgfpathlineto{\pgfqpoint{4.420575in}{2.522138in}}%
\pgfpathlineto{\pgfqpoint{4.423082in}{2.530603in}}%
\pgfpathlineto{\pgfqpoint{4.425588in}{2.533678in}}%
\pgfpathlineto{\pgfqpoint{4.428095in}{2.541089in}}%
\pgfpathlineto{\pgfqpoint{4.430602in}{2.541654in}}%
\pgfpathlineto{\pgfqpoint{4.433108in}{2.546076in}}%
\pgfpathlineto{\pgfqpoint{4.450655in}{2.551307in}}%
\pgfpathlineto{\pgfqpoint{4.455668in}{2.557475in}}%
\pgfpathlineto{\pgfqpoint{4.458175in}{2.558548in}}%
\pgfpathlineto{\pgfqpoint{4.463188in}{2.565130in}}%
\pgfpathlineto{\pgfqpoint{4.465695in}{2.565342in}}%
\pgfpathlineto{\pgfqpoint{4.470708in}{2.571016in}}%
\pgfpathlineto{\pgfqpoint{4.473215in}{2.571827in}}%
\pgfpathlineto{\pgfqpoint{4.475721in}{2.577294in}}%
\pgfpathlineto{\pgfqpoint{4.478228in}{2.577913in}}%
\pgfpathlineto{\pgfqpoint{4.480735in}{2.580565in}}%
\pgfpathlineto{\pgfqpoint{4.485748in}{2.582606in}}%
\pgfpathlineto{\pgfqpoint{4.490761in}{2.599013in}}%
\pgfpathlineto{\pgfqpoint{4.493268in}{2.608139in}}%
\pgfpathlineto{\pgfqpoint{4.495775in}{2.610450in}}%
\pgfpathlineto{\pgfqpoint{4.498281in}{2.620203in}}%
\pgfpathlineto{\pgfqpoint{4.500788in}{2.620599in}}%
\pgfpathlineto{\pgfqpoint{4.503295in}{2.630275in}}%
\pgfpathlineto{\pgfqpoint{4.503295in}{2.630275in}}%
\pgfusepath{stroke}%
\end{pgfscope}%
\begin{pgfscope}%
\pgfpathrectangle{\pgfqpoint{0.708220in}{0.535823in}}{\pgfqpoint{5.013309in}{2.094453in}}%
\pgfusepath{clip}%
\pgfsetbuttcap%
\pgfsetroundjoin%
\pgfsetlinewidth{1.003750pt}%
\definecolor{currentstroke}{rgb}{0.811765,0.125490,0.125490}%
\pgfsetstrokecolor{currentstroke}%
\pgfsetdash{{1.000000pt}{1.650000pt}}{0.000000pt}%
\pgfpathmoveto{\pgfqpoint{0.708220in}{0.801509in}}%
\pgfpathlineto{\pgfqpoint{0.728273in}{0.802665in}}%
\pgfpathlineto{\pgfqpoint{0.753340in}{0.804259in}}%
\pgfpathlineto{\pgfqpoint{0.958885in}{0.808458in}}%
\pgfpathlineto{\pgfqpoint{1.109285in}{0.810755in}}%
\pgfpathlineto{\pgfqpoint{1.151898in}{0.811775in}}%
\pgfpathlineto{\pgfqpoint{1.440163in}{0.823886in}}%
\pgfpathlineto{\pgfqpoint{1.610615in}{0.837514in}}%
\pgfpathlineto{\pgfqpoint{1.648215in}{0.841507in}}%
\pgfpathlineto{\pgfqpoint{1.650722in}{0.844151in}}%
\pgfpathlineto{\pgfqpoint{1.695842in}{0.847744in}}%
\pgfpathlineto{\pgfqpoint{1.710882in}{0.849010in}}%
\pgfpathlineto{\pgfqpoint{1.733442in}{0.850957in}}%
\pgfpathlineto{\pgfqpoint{1.803628in}{0.856473in}}%
\pgfpathlineto{\pgfqpoint{1.806134in}{0.858704in}}%
\pgfpathlineto{\pgfqpoint{1.833708in}{0.863183in}}%
\pgfpathlineto{\pgfqpoint{1.838721in}{0.863930in}}%
\pgfpathlineto{\pgfqpoint{1.856268in}{0.864875in}}%
\pgfpathlineto{\pgfqpoint{1.866294in}{0.866666in}}%
\pgfpathlineto{\pgfqpoint{1.878827in}{0.868469in}}%
\pgfpathlineto{\pgfqpoint{1.908907in}{0.873477in}}%
\pgfpathlineto{\pgfqpoint{1.913921in}{0.874924in}}%
\pgfpathlineto{\pgfqpoint{1.916427in}{0.875271in}}%
\pgfpathlineto{\pgfqpoint{1.918934in}{0.877314in}}%
\pgfpathlineto{\pgfqpoint{1.926454in}{0.878548in}}%
\pgfpathlineto{\pgfqpoint{1.928961in}{0.878957in}}%
\pgfpathlineto{\pgfqpoint{1.931467in}{0.881909in}}%
\pgfpathlineto{\pgfqpoint{1.938987in}{0.884241in}}%
\pgfpathlineto{\pgfqpoint{1.951520in}{0.885698in}}%
\pgfpathlineto{\pgfqpoint{1.981600in}{0.891310in}}%
\pgfpathlineto{\pgfqpoint{1.986614in}{0.892602in}}%
\pgfpathlineto{\pgfqpoint{1.991627in}{0.892970in}}%
\pgfpathlineto{\pgfqpoint{2.006667in}{0.899885in}}%
\pgfpathlineto{\pgfqpoint{2.039253in}{0.906627in}}%
\pgfpathlineto{\pgfqpoint{2.044267in}{0.909248in}}%
\pgfpathlineto{\pgfqpoint{2.051787in}{0.909870in}}%
\pgfpathlineto{\pgfqpoint{2.061813in}{0.914150in}}%
\pgfpathlineto{\pgfqpoint{2.069333in}{0.917596in}}%
\pgfpathlineto{\pgfqpoint{2.074346in}{0.919588in}}%
\pgfpathlineto{\pgfqpoint{2.086880in}{0.921070in}}%
\pgfpathlineto{\pgfqpoint{2.111946in}{0.927634in}}%
\pgfpathlineto{\pgfqpoint{2.114453in}{0.930307in}}%
\pgfpathlineto{\pgfqpoint{2.119466in}{0.931196in}}%
\pgfpathlineto{\pgfqpoint{2.124480in}{0.932490in}}%
\pgfpathlineto{\pgfqpoint{2.126986in}{0.937051in}}%
\pgfpathlineto{\pgfqpoint{2.134506in}{0.938830in}}%
\pgfpathlineto{\pgfqpoint{2.139520in}{0.941854in}}%
\pgfpathlineto{\pgfqpoint{2.144533in}{0.942788in}}%
\pgfpathlineto{\pgfqpoint{2.147039in}{0.944611in}}%
\pgfpathlineto{\pgfqpoint{2.154559in}{0.945429in}}%
\pgfpathlineto{\pgfqpoint{2.159573in}{0.947887in}}%
\pgfpathlineto{\pgfqpoint{2.174613in}{0.948935in}}%
\pgfpathlineto{\pgfqpoint{2.179626in}{0.949188in}}%
\pgfpathlineto{\pgfqpoint{2.182133in}{0.955360in}}%
\pgfpathlineto{\pgfqpoint{2.189653in}{0.958633in}}%
\pgfpathlineto{\pgfqpoint{2.197173in}{0.963424in}}%
\pgfpathlineto{\pgfqpoint{2.207199in}{0.965485in}}%
\pgfpathlineto{\pgfqpoint{2.212212in}{0.968083in}}%
\pgfpathlineto{\pgfqpoint{2.214719in}{0.970884in}}%
\pgfpathlineto{\pgfqpoint{2.219732in}{0.972128in}}%
\pgfpathlineto{\pgfqpoint{2.222239in}{0.972465in}}%
\pgfpathlineto{\pgfqpoint{2.224746in}{0.975342in}}%
\pgfpathlineto{\pgfqpoint{2.237279in}{0.977851in}}%
\pgfpathlineto{\pgfqpoint{2.239786in}{0.980306in}}%
\pgfpathlineto{\pgfqpoint{2.247306in}{0.982642in}}%
\pgfpathlineto{\pgfqpoint{2.252319in}{0.985247in}}%
\pgfpathlineto{\pgfqpoint{2.262346in}{0.987690in}}%
\pgfpathlineto{\pgfqpoint{2.264852in}{0.990584in}}%
\pgfpathlineto{\pgfqpoint{2.267359in}{0.996589in}}%
\pgfpathlineto{\pgfqpoint{2.272372in}{0.997476in}}%
\pgfpathlineto{\pgfqpoint{2.279892in}{0.999442in}}%
\pgfpathlineto{\pgfqpoint{2.284905in}{1.003842in}}%
\pgfpathlineto{\pgfqpoint{2.294932in}{1.005738in}}%
\pgfpathlineto{\pgfqpoint{2.302452in}{1.009810in}}%
\pgfpathlineto{\pgfqpoint{2.307465in}{1.016162in}}%
\pgfpathlineto{\pgfqpoint{2.314985in}{1.017166in}}%
\pgfpathlineto{\pgfqpoint{2.317492in}{1.018465in}}%
\pgfpathlineto{\pgfqpoint{2.319999in}{1.022251in}}%
\pgfpathlineto{\pgfqpoint{2.325012in}{1.024002in}}%
\pgfpathlineto{\pgfqpoint{2.330025in}{1.026587in}}%
\pgfpathlineto{\pgfqpoint{2.335039in}{1.028006in}}%
\pgfpathlineto{\pgfqpoint{2.340052in}{1.028790in}}%
\pgfpathlineto{\pgfqpoint{2.360105in}{1.040545in}}%
\pgfpathlineto{\pgfqpoint{2.362612in}{1.045079in}}%
\pgfpathlineto{\pgfqpoint{2.377652in}{1.051202in}}%
\pgfpathlineto{\pgfqpoint{2.382665in}{1.051442in}}%
\pgfpathlineto{\pgfqpoint{2.387678in}{1.054705in}}%
\pgfpathlineto{\pgfqpoint{2.400212in}{1.057208in}}%
\pgfpathlineto{\pgfqpoint{2.407732in}{1.063269in}}%
\pgfpathlineto{\pgfqpoint{2.412745in}{1.064457in}}%
\pgfpathlineto{\pgfqpoint{2.415251in}{1.066897in}}%
\pgfpathlineto{\pgfqpoint{2.417758in}{1.067447in}}%
\pgfpathlineto{\pgfqpoint{2.422771in}{1.073544in}}%
\pgfpathlineto{\pgfqpoint{2.432798in}{1.079198in}}%
\pgfpathlineto{\pgfqpoint{2.437811in}{1.080323in}}%
\pgfpathlineto{\pgfqpoint{2.442825in}{1.081715in}}%
\pgfpathlineto{\pgfqpoint{2.450345in}{1.085015in}}%
\pgfpathlineto{\pgfqpoint{2.457865in}{1.088472in}}%
\pgfpathlineto{\pgfqpoint{2.462878in}{1.089711in}}%
\pgfpathlineto{\pgfqpoint{2.465385in}{1.092758in}}%
\pgfpathlineto{\pgfqpoint{2.467891in}{1.093731in}}%
\pgfpathlineto{\pgfqpoint{2.472905in}{1.097359in}}%
\pgfpathlineto{\pgfqpoint{2.480425in}{1.101201in}}%
\pgfpathlineto{\pgfqpoint{2.487944in}{1.108943in}}%
\pgfpathlineto{\pgfqpoint{2.497971in}{1.110741in}}%
\pgfpathlineto{\pgfqpoint{2.502984in}{1.112323in}}%
\pgfpathlineto{\pgfqpoint{2.507998in}{1.118202in}}%
\pgfpathlineto{\pgfqpoint{2.513011in}{1.119248in}}%
\pgfpathlineto{\pgfqpoint{2.523038in}{1.125179in}}%
\pgfpathlineto{\pgfqpoint{2.530558in}{1.127017in}}%
\pgfpathlineto{\pgfqpoint{2.533064in}{1.130016in}}%
\pgfpathlineto{\pgfqpoint{2.535571in}{1.130215in}}%
\pgfpathlineto{\pgfqpoint{2.540584in}{1.133599in}}%
\pgfpathlineto{\pgfqpoint{2.545598in}{1.135226in}}%
\pgfpathlineto{\pgfqpoint{2.548104in}{1.141364in}}%
\pgfpathlineto{\pgfqpoint{2.555624in}{1.145724in}}%
\pgfpathlineto{\pgfqpoint{2.558131in}{1.148464in}}%
\pgfpathlineto{\pgfqpoint{2.560637in}{1.148919in}}%
\pgfpathlineto{\pgfqpoint{2.563144in}{1.152435in}}%
\pgfpathlineto{\pgfqpoint{2.570664in}{1.154378in}}%
\pgfpathlineto{\pgfqpoint{2.575677in}{1.154553in}}%
\pgfpathlineto{\pgfqpoint{2.580691in}{1.158008in}}%
\pgfpathlineto{\pgfqpoint{2.583197in}{1.158439in}}%
\pgfpathlineto{\pgfqpoint{2.585704in}{1.164795in}}%
\pgfpathlineto{\pgfqpoint{2.593224in}{1.166720in}}%
\pgfpathlineto{\pgfqpoint{2.595731in}{1.168090in}}%
\pgfpathlineto{\pgfqpoint{2.598237in}{1.170916in}}%
\pgfpathlineto{\pgfqpoint{2.600744in}{1.171613in}}%
\pgfpathlineto{\pgfqpoint{2.603251in}{1.174659in}}%
\pgfpathlineto{\pgfqpoint{2.610771in}{1.177633in}}%
\pgfpathlineto{\pgfqpoint{2.613277in}{1.180253in}}%
\pgfpathlineto{\pgfqpoint{2.615784in}{1.180795in}}%
\pgfpathlineto{\pgfqpoint{2.618290in}{1.184170in}}%
\pgfpathlineto{\pgfqpoint{2.623304in}{1.187273in}}%
\pgfpathlineto{\pgfqpoint{2.625810in}{1.190385in}}%
\pgfpathlineto{\pgfqpoint{2.638344in}{1.196096in}}%
\pgfpathlineto{\pgfqpoint{2.640850in}{1.196383in}}%
\pgfpathlineto{\pgfqpoint{2.643357in}{1.199487in}}%
\pgfpathlineto{\pgfqpoint{2.655890in}{1.203742in}}%
\pgfpathlineto{\pgfqpoint{2.660904in}{1.207077in}}%
\pgfpathlineto{\pgfqpoint{2.663410in}{1.207126in}}%
\pgfpathlineto{\pgfqpoint{2.668424in}{1.213220in}}%
\pgfpathlineto{\pgfqpoint{2.670930in}{1.215079in}}%
\pgfpathlineto{\pgfqpoint{2.680957in}{1.229493in}}%
\pgfpathlineto{\pgfqpoint{2.683464in}{1.230884in}}%
\pgfpathlineto{\pgfqpoint{2.688477in}{1.234856in}}%
\pgfpathlineto{\pgfqpoint{2.698503in}{1.240643in}}%
\pgfpathlineto{\pgfqpoint{2.701010in}{1.241193in}}%
\pgfpathlineto{\pgfqpoint{2.706023in}{1.249614in}}%
\pgfpathlineto{\pgfqpoint{2.708530in}{1.250066in}}%
\pgfpathlineto{\pgfqpoint{2.711037in}{1.252468in}}%
\pgfpathlineto{\pgfqpoint{2.716050in}{1.253154in}}%
\pgfpathlineto{\pgfqpoint{2.726077in}{1.262524in}}%
\pgfpathlineto{\pgfqpoint{2.738610in}{1.269913in}}%
\pgfpathlineto{\pgfqpoint{2.741117in}{1.269964in}}%
\pgfpathlineto{\pgfqpoint{2.743623in}{1.272408in}}%
\pgfpathlineto{\pgfqpoint{2.746130in}{1.276953in}}%
\pgfpathlineto{\pgfqpoint{2.756156in}{1.283277in}}%
\pgfpathlineto{\pgfqpoint{2.758663in}{1.283512in}}%
\pgfpathlineto{\pgfqpoint{2.761170in}{1.285400in}}%
\pgfpathlineto{\pgfqpoint{2.766183in}{1.286041in}}%
\pgfpathlineto{\pgfqpoint{2.768690in}{1.288545in}}%
\pgfpathlineto{\pgfqpoint{2.771196in}{1.289290in}}%
\pgfpathlineto{\pgfqpoint{2.773703in}{1.291938in}}%
\pgfpathlineto{\pgfqpoint{2.778716in}{1.301059in}}%
\pgfpathlineto{\pgfqpoint{2.788743in}{1.302984in}}%
\pgfpathlineto{\pgfqpoint{2.793756in}{1.306504in}}%
\pgfpathlineto{\pgfqpoint{2.796263in}{1.310577in}}%
\pgfpathlineto{\pgfqpoint{2.806290in}{1.317762in}}%
\pgfpathlineto{\pgfqpoint{2.808796in}{1.318040in}}%
\pgfpathlineto{\pgfqpoint{2.813810in}{1.322717in}}%
\pgfpathlineto{\pgfqpoint{2.816316in}{1.330454in}}%
\pgfpathlineto{\pgfqpoint{2.818823in}{1.331280in}}%
\pgfpathlineto{\pgfqpoint{2.823836in}{1.338664in}}%
\pgfpathlineto{\pgfqpoint{2.826343in}{1.340511in}}%
\pgfpathlineto{\pgfqpoint{2.828849in}{1.340612in}}%
\pgfpathlineto{\pgfqpoint{2.831356in}{1.342326in}}%
\pgfpathlineto{\pgfqpoint{2.833863in}{1.345685in}}%
\pgfpathlineto{\pgfqpoint{2.843889in}{1.349677in}}%
\pgfpathlineto{\pgfqpoint{2.846396in}{1.352752in}}%
\pgfpathlineto{\pgfqpoint{2.851409in}{1.354595in}}%
\pgfpathlineto{\pgfqpoint{2.853916in}{1.356502in}}%
\pgfpathlineto{\pgfqpoint{2.856423in}{1.363475in}}%
\pgfpathlineto{\pgfqpoint{2.868956in}{1.367213in}}%
\pgfpathlineto{\pgfqpoint{2.873969in}{1.369603in}}%
\pgfpathlineto{\pgfqpoint{2.876476in}{1.370513in}}%
\pgfpathlineto{\pgfqpoint{2.878983in}{1.377351in}}%
\pgfpathlineto{\pgfqpoint{2.886503in}{1.382298in}}%
\pgfpathlineto{\pgfqpoint{2.891516in}{1.397167in}}%
\pgfpathlineto{\pgfqpoint{2.894022in}{1.400863in}}%
\pgfpathlineto{\pgfqpoint{2.899036in}{1.404028in}}%
\pgfpathlineto{\pgfqpoint{2.901542in}{1.409697in}}%
\pgfpathlineto{\pgfqpoint{2.904049in}{1.409868in}}%
\pgfpathlineto{\pgfqpoint{2.909062in}{1.412760in}}%
\pgfpathlineto{\pgfqpoint{2.911569in}{1.420130in}}%
\pgfpathlineto{\pgfqpoint{2.921596in}{1.423353in}}%
\pgfpathlineto{\pgfqpoint{2.929116in}{1.429802in}}%
\pgfpathlineto{\pgfqpoint{2.931622in}{1.430266in}}%
\pgfpathlineto{\pgfqpoint{2.934129in}{1.437612in}}%
\pgfpathlineto{\pgfqpoint{2.936636in}{1.437925in}}%
\pgfpathlineto{\pgfqpoint{2.939142in}{1.442234in}}%
\pgfpathlineto{\pgfqpoint{2.941649in}{1.442347in}}%
\pgfpathlineto{\pgfqpoint{2.944156in}{1.446093in}}%
\pgfpathlineto{\pgfqpoint{2.946662in}{1.447018in}}%
\pgfpathlineto{\pgfqpoint{2.949169in}{1.449813in}}%
\pgfpathlineto{\pgfqpoint{2.951676in}{1.454766in}}%
\pgfpathlineto{\pgfqpoint{2.956689in}{1.456650in}}%
\pgfpathlineto{\pgfqpoint{2.959195in}{1.456797in}}%
\pgfpathlineto{\pgfqpoint{2.974235in}{1.469734in}}%
\pgfpathlineto{\pgfqpoint{2.976742in}{1.471635in}}%
\pgfpathlineto{\pgfqpoint{2.979249in}{1.471902in}}%
\pgfpathlineto{\pgfqpoint{2.984262in}{1.474912in}}%
\pgfpathlineto{\pgfqpoint{2.991782in}{1.478285in}}%
\pgfpathlineto{\pgfqpoint{2.994289in}{1.486314in}}%
\pgfpathlineto{\pgfqpoint{2.999302in}{1.491812in}}%
\pgfpathlineto{\pgfqpoint{3.004315in}{1.496434in}}%
\pgfpathlineto{\pgfqpoint{3.006822in}{1.496922in}}%
\pgfpathlineto{\pgfqpoint{3.009329in}{1.504605in}}%
\pgfpathlineto{\pgfqpoint{3.011835in}{1.505085in}}%
\pgfpathlineto{\pgfqpoint{3.014342in}{1.513908in}}%
\pgfpathlineto{\pgfqpoint{3.016849in}{1.518127in}}%
\pgfpathlineto{\pgfqpoint{3.021862in}{1.521782in}}%
\pgfpathlineto{\pgfqpoint{3.024369in}{1.525366in}}%
\pgfpathlineto{\pgfqpoint{3.031888in}{1.528119in}}%
\pgfpathlineto{\pgfqpoint{3.034395in}{1.533388in}}%
\pgfpathlineto{\pgfqpoint{3.036902in}{1.535218in}}%
\pgfpathlineto{\pgfqpoint{3.041915in}{1.542440in}}%
\pgfpathlineto{\pgfqpoint{3.044422in}{1.545267in}}%
\pgfpathlineto{\pgfqpoint{3.049435in}{1.545500in}}%
\pgfpathlineto{\pgfqpoint{3.051942in}{1.553820in}}%
\pgfpathlineto{\pgfqpoint{3.059462in}{1.562516in}}%
\pgfpathlineto{\pgfqpoint{3.066982in}{1.565054in}}%
\pgfpathlineto{\pgfqpoint{3.069488in}{1.570168in}}%
\pgfpathlineto{\pgfqpoint{3.087035in}{1.577134in}}%
\pgfpathlineto{\pgfqpoint{3.089542in}{1.582371in}}%
\pgfpathlineto{\pgfqpoint{3.092048in}{1.582476in}}%
\pgfpathlineto{\pgfqpoint{3.097061in}{1.585956in}}%
\pgfpathlineto{\pgfqpoint{3.099568in}{1.588066in}}%
\pgfpathlineto{\pgfqpoint{3.109595in}{1.591500in}}%
\pgfpathlineto{\pgfqpoint{3.114608in}{1.597823in}}%
\pgfpathlineto{\pgfqpoint{3.117115in}{1.599300in}}%
\pgfpathlineto{\pgfqpoint{3.119621in}{1.605668in}}%
\pgfpathlineto{\pgfqpoint{3.122128in}{1.605699in}}%
\pgfpathlineto{\pgfqpoint{3.127141in}{1.607316in}}%
\pgfpathlineto{\pgfqpoint{3.129648in}{1.613063in}}%
\pgfpathlineto{\pgfqpoint{3.137168in}{1.619027in}}%
\pgfpathlineto{\pgfqpoint{3.142181in}{1.621756in}}%
\pgfpathlineto{\pgfqpoint{3.152208in}{1.623706in}}%
\pgfpathlineto{\pgfqpoint{3.154715in}{1.633930in}}%
\pgfpathlineto{\pgfqpoint{3.159728in}{1.642022in}}%
\pgfpathlineto{\pgfqpoint{3.164741in}{1.642461in}}%
\pgfpathlineto{\pgfqpoint{3.167248in}{1.643728in}}%
\pgfpathlineto{\pgfqpoint{3.172261in}{1.647946in}}%
\pgfpathlineto{\pgfqpoint{3.174768in}{1.661538in}}%
\pgfpathlineto{\pgfqpoint{3.179781in}{1.662160in}}%
\pgfpathlineto{\pgfqpoint{3.184794in}{1.665395in}}%
\pgfpathlineto{\pgfqpoint{3.189808in}{1.669851in}}%
\pgfpathlineto{\pgfqpoint{3.192314in}{1.672365in}}%
\pgfpathlineto{\pgfqpoint{3.197328in}{1.673223in}}%
\pgfpathlineto{\pgfqpoint{3.199834in}{1.677243in}}%
\pgfpathlineto{\pgfqpoint{3.202341in}{1.688470in}}%
\pgfpathlineto{\pgfqpoint{3.207354in}{1.690726in}}%
\pgfpathlineto{\pgfqpoint{3.209861in}{1.698036in}}%
\pgfpathlineto{\pgfqpoint{3.212368in}{1.699589in}}%
\pgfpathlineto{\pgfqpoint{3.217381in}{1.711289in}}%
\pgfpathlineto{\pgfqpoint{3.222394in}{1.712105in}}%
\pgfpathlineto{\pgfqpoint{3.224901in}{1.717576in}}%
\pgfpathlineto{\pgfqpoint{3.227408in}{1.718010in}}%
\pgfpathlineto{\pgfqpoint{3.239941in}{1.729345in}}%
\pgfpathlineto{\pgfqpoint{3.242447in}{1.735785in}}%
\pgfpathlineto{\pgfqpoint{3.252474in}{1.742891in}}%
\pgfpathlineto{\pgfqpoint{3.254981in}{1.743087in}}%
\pgfpathlineto{\pgfqpoint{3.257487in}{1.745637in}}%
\pgfpathlineto{\pgfqpoint{3.259994in}{1.750899in}}%
\pgfpathlineto{\pgfqpoint{3.272527in}{1.757936in}}%
\pgfpathlineto{\pgfqpoint{3.277541in}{1.765735in}}%
\pgfpathlineto{\pgfqpoint{3.285061in}{1.766362in}}%
\pgfpathlineto{\pgfqpoint{3.290074in}{1.767207in}}%
\pgfpathlineto{\pgfqpoint{3.292581in}{1.770747in}}%
\pgfpathlineto{\pgfqpoint{3.295087in}{1.777301in}}%
\pgfpathlineto{\pgfqpoint{3.297594in}{1.778153in}}%
\pgfpathlineto{\pgfqpoint{3.305114in}{1.784826in}}%
\pgfpathlineto{\pgfqpoint{3.307620in}{1.789166in}}%
\pgfpathlineto{\pgfqpoint{3.312634in}{1.790104in}}%
\pgfpathlineto{\pgfqpoint{3.315140in}{1.793893in}}%
\pgfpathlineto{\pgfqpoint{3.317647in}{1.794853in}}%
\pgfpathlineto{\pgfqpoint{3.325167in}{1.804943in}}%
\pgfpathlineto{\pgfqpoint{3.330180in}{1.807436in}}%
\pgfpathlineto{\pgfqpoint{3.332687in}{1.807662in}}%
\pgfpathlineto{\pgfqpoint{3.335194in}{1.815867in}}%
\pgfpathlineto{\pgfqpoint{3.337700in}{1.817417in}}%
\pgfpathlineto{\pgfqpoint{3.340207in}{1.821350in}}%
\pgfpathlineto{\pgfqpoint{3.345220in}{1.823239in}}%
\pgfpathlineto{\pgfqpoint{3.347727in}{1.837020in}}%
\pgfpathlineto{\pgfqpoint{3.355247in}{1.841431in}}%
\pgfpathlineto{\pgfqpoint{3.357754in}{1.844136in}}%
\pgfpathlineto{\pgfqpoint{3.365273in}{1.845622in}}%
\pgfpathlineto{\pgfqpoint{3.367780in}{1.847653in}}%
\pgfpathlineto{\pgfqpoint{3.370287in}{1.847868in}}%
\pgfpathlineto{\pgfqpoint{3.372793in}{1.851133in}}%
\pgfpathlineto{\pgfqpoint{3.385327in}{1.855741in}}%
\pgfpathlineto{\pgfqpoint{3.390340in}{1.861481in}}%
\pgfpathlineto{\pgfqpoint{3.392847in}{1.866502in}}%
\pgfpathlineto{\pgfqpoint{3.400367in}{1.872022in}}%
\pgfpathlineto{\pgfqpoint{3.402873in}{1.872046in}}%
\pgfpathlineto{\pgfqpoint{3.405380in}{1.874745in}}%
\pgfpathlineto{\pgfqpoint{3.407887in}{1.880845in}}%
\pgfpathlineto{\pgfqpoint{3.410393in}{1.882821in}}%
\pgfpathlineto{\pgfqpoint{3.417913in}{1.884226in}}%
\pgfpathlineto{\pgfqpoint{3.422927in}{1.885101in}}%
\pgfpathlineto{\pgfqpoint{3.425433in}{1.886942in}}%
\pgfpathlineto{\pgfqpoint{3.427940in}{1.890953in}}%
\pgfpathlineto{\pgfqpoint{3.430447in}{1.892427in}}%
\pgfpathlineto{\pgfqpoint{3.435460in}{1.898919in}}%
\pgfpathlineto{\pgfqpoint{3.442980in}{1.900923in}}%
\pgfpathlineto{\pgfqpoint{3.450500in}{1.905431in}}%
\pgfpathlineto{\pgfqpoint{3.453006in}{1.908612in}}%
\pgfpathlineto{\pgfqpoint{3.455513in}{1.909358in}}%
\pgfpathlineto{\pgfqpoint{3.458020in}{1.916684in}}%
\pgfpathlineto{\pgfqpoint{3.460526in}{1.917530in}}%
\pgfpathlineto{\pgfqpoint{3.463033in}{1.922520in}}%
\pgfpathlineto{\pgfqpoint{3.468046in}{1.925460in}}%
\pgfpathlineto{\pgfqpoint{3.470553in}{1.928389in}}%
\pgfpathlineto{\pgfqpoint{3.475566in}{1.930281in}}%
\pgfpathlineto{\pgfqpoint{3.483086in}{1.941153in}}%
\pgfpathlineto{\pgfqpoint{3.488100in}{1.948318in}}%
\pgfpathlineto{\pgfqpoint{3.490606in}{1.948408in}}%
\pgfpathlineto{\pgfqpoint{3.493113in}{1.950053in}}%
\pgfpathlineto{\pgfqpoint{3.503139in}{1.963124in}}%
\pgfpathlineto{\pgfqpoint{3.505646in}{1.963301in}}%
\pgfpathlineto{\pgfqpoint{3.508153in}{1.969599in}}%
\pgfpathlineto{\pgfqpoint{3.510659in}{1.971102in}}%
\pgfpathlineto{\pgfqpoint{3.513166in}{1.976871in}}%
\pgfpathlineto{\pgfqpoint{3.515673in}{1.977626in}}%
\pgfpathlineto{\pgfqpoint{3.523193in}{1.988766in}}%
\pgfpathlineto{\pgfqpoint{3.530713in}{1.996751in}}%
\pgfpathlineto{\pgfqpoint{3.535726in}{1.997840in}}%
\pgfpathlineto{\pgfqpoint{3.540739in}{2.003653in}}%
\pgfpathlineto{\pgfqpoint{3.545753in}{2.004790in}}%
\pgfpathlineto{\pgfqpoint{3.548259in}{2.018691in}}%
\pgfpathlineto{\pgfqpoint{3.550766in}{2.020455in}}%
\pgfpathlineto{\pgfqpoint{3.553273in}{2.020634in}}%
\pgfpathlineto{\pgfqpoint{3.560793in}{2.025476in}}%
\pgfpathlineto{\pgfqpoint{3.563299in}{2.031576in}}%
\pgfpathlineto{\pgfqpoint{3.565806in}{2.031987in}}%
\pgfpathlineto{\pgfqpoint{3.568313in}{2.034531in}}%
\pgfpathlineto{\pgfqpoint{3.570819in}{2.042552in}}%
\pgfpathlineto{\pgfqpoint{3.573326in}{2.044201in}}%
\pgfpathlineto{\pgfqpoint{3.575832in}{2.044376in}}%
\pgfpathlineto{\pgfqpoint{3.578339in}{2.046060in}}%
\pgfpathlineto{\pgfqpoint{3.588366in}{2.059096in}}%
\pgfpathlineto{\pgfqpoint{3.593379in}{2.061231in}}%
\pgfpathlineto{\pgfqpoint{3.595886in}{2.061448in}}%
\pgfpathlineto{\pgfqpoint{3.598392in}{2.068123in}}%
\pgfpathlineto{\pgfqpoint{3.600899in}{2.070383in}}%
\pgfpathlineto{\pgfqpoint{3.603406in}{2.076823in}}%
\pgfpathlineto{\pgfqpoint{3.605912in}{2.076968in}}%
\pgfpathlineto{\pgfqpoint{3.608419in}{2.082234in}}%
\pgfpathlineto{\pgfqpoint{3.613432in}{2.082958in}}%
\pgfpathlineto{\pgfqpoint{3.615939in}{2.084674in}}%
\pgfpathlineto{\pgfqpoint{3.620952in}{2.093117in}}%
\pgfpathlineto{\pgfqpoint{3.625966in}{2.110270in}}%
\pgfpathlineto{\pgfqpoint{3.628472in}{2.111097in}}%
\pgfpathlineto{\pgfqpoint{3.630979in}{2.116406in}}%
\pgfpathlineto{\pgfqpoint{3.633486in}{2.117585in}}%
\pgfpathlineto{\pgfqpoint{3.635992in}{2.120408in}}%
\pgfpathlineto{\pgfqpoint{3.641005in}{2.121415in}}%
\pgfpathlineto{\pgfqpoint{3.643512in}{2.121528in}}%
\pgfpathlineto{\pgfqpoint{3.648525in}{2.125061in}}%
\pgfpathlineto{\pgfqpoint{3.653539in}{2.127202in}}%
\pgfpathlineto{\pgfqpoint{3.656045in}{2.129724in}}%
\pgfpathlineto{\pgfqpoint{3.658552in}{2.130337in}}%
\pgfpathlineto{\pgfqpoint{3.666072in}{2.136903in}}%
\pgfpathlineto{\pgfqpoint{3.668579in}{2.143534in}}%
\pgfpathlineto{\pgfqpoint{3.671085in}{2.144647in}}%
\pgfpathlineto{\pgfqpoint{3.676099in}{2.153173in}}%
\pgfpathlineto{\pgfqpoint{3.683619in}{2.156972in}}%
\pgfpathlineto{\pgfqpoint{3.686125in}{2.163571in}}%
\pgfpathlineto{\pgfqpoint{3.691139in}{2.166413in}}%
\pgfpathlineto{\pgfqpoint{3.693645in}{2.175888in}}%
\pgfpathlineto{\pgfqpoint{3.696152in}{2.180043in}}%
\pgfpathlineto{\pgfqpoint{3.701165in}{2.180701in}}%
\pgfpathlineto{\pgfqpoint{3.706178in}{2.192882in}}%
\pgfpathlineto{\pgfqpoint{3.711192in}{2.194510in}}%
\pgfpathlineto{\pgfqpoint{3.713698in}{2.194841in}}%
\pgfpathlineto{\pgfqpoint{3.716205in}{2.202066in}}%
\pgfpathlineto{\pgfqpoint{3.718712in}{2.202471in}}%
\pgfpathlineto{\pgfqpoint{3.721218in}{2.207140in}}%
\pgfpathlineto{\pgfqpoint{3.726232in}{2.209854in}}%
\pgfpathlineto{\pgfqpoint{3.731245in}{2.211693in}}%
\pgfpathlineto{\pgfqpoint{3.733752in}{2.218297in}}%
\pgfpathlineto{\pgfqpoint{3.738765in}{2.223058in}}%
\pgfpathlineto{\pgfqpoint{3.741272in}{2.228882in}}%
\pgfpathlineto{\pgfqpoint{3.743778in}{2.229095in}}%
\pgfpathlineto{\pgfqpoint{3.746285in}{2.230771in}}%
\pgfpathlineto{\pgfqpoint{3.748792in}{2.236327in}}%
\pgfpathlineto{\pgfqpoint{3.751298in}{2.238228in}}%
\pgfpathlineto{\pgfqpoint{3.753805in}{2.244335in}}%
\pgfpathlineto{\pgfqpoint{3.756312in}{2.246223in}}%
\pgfpathlineto{\pgfqpoint{3.758818in}{2.255931in}}%
\pgfpathlineto{\pgfqpoint{3.761325in}{2.259130in}}%
\pgfpathlineto{\pgfqpoint{3.763832in}{2.264426in}}%
\pgfpathlineto{\pgfqpoint{3.766338in}{2.265919in}}%
\pgfpathlineto{\pgfqpoint{3.768845in}{2.271411in}}%
\pgfpathlineto{\pgfqpoint{3.773858in}{2.275300in}}%
\pgfpathlineto{\pgfqpoint{3.776365in}{2.283194in}}%
\pgfpathlineto{\pgfqpoint{3.781378in}{2.285728in}}%
\pgfpathlineto{\pgfqpoint{3.788898in}{2.286487in}}%
\pgfpathlineto{\pgfqpoint{3.791405in}{2.287571in}}%
\pgfpathlineto{\pgfqpoint{3.796418in}{2.293944in}}%
\pgfpathlineto{\pgfqpoint{3.798925in}{2.294194in}}%
\pgfpathlineto{\pgfqpoint{3.801431in}{2.296342in}}%
\pgfpathlineto{\pgfqpoint{3.803938in}{2.300784in}}%
\pgfpathlineto{\pgfqpoint{3.811458in}{2.302625in}}%
\pgfpathlineto{\pgfqpoint{3.813965in}{2.310302in}}%
\pgfpathlineto{\pgfqpoint{3.826498in}{2.317069in}}%
\pgfpathlineto{\pgfqpoint{3.829005in}{2.328237in}}%
\pgfpathlineto{\pgfqpoint{3.834018in}{2.329931in}}%
\pgfpathlineto{\pgfqpoint{3.836525in}{2.335188in}}%
\pgfpathlineto{\pgfqpoint{3.841538in}{2.336120in}}%
\pgfpathlineto{\pgfqpoint{3.844044in}{2.346145in}}%
\pgfpathlineto{\pgfqpoint{3.846551in}{2.346868in}}%
\pgfpathlineto{\pgfqpoint{3.851564in}{2.351883in}}%
\pgfpathlineto{\pgfqpoint{3.859084in}{2.354084in}}%
\pgfpathlineto{\pgfqpoint{3.861591in}{2.370190in}}%
\pgfpathlineto{\pgfqpoint{3.864098in}{2.371510in}}%
\pgfpathlineto{\pgfqpoint{3.866604in}{2.375560in}}%
\pgfpathlineto{\pgfqpoint{3.869111in}{2.382315in}}%
\pgfpathlineto{\pgfqpoint{3.874124in}{2.388175in}}%
\pgfpathlineto{\pgfqpoint{3.879138in}{2.403624in}}%
\pgfpathlineto{\pgfqpoint{3.881644in}{2.416310in}}%
\pgfpathlineto{\pgfqpoint{3.884151in}{2.417565in}}%
\pgfpathlineto{\pgfqpoint{3.886658in}{2.420775in}}%
\pgfpathlineto{\pgfqpoint{3.894178in}{2.423669in}}%
\pgfpathlineto{\pgfqpoint{3.896684in}{2.434409in}}%
\pgfpathlineto{\pgfqpoint{3.899191in}{2.434646in}}%
\pgfpathlineto{\pgfqpoint{3.901698in}{2.436124in}}%
\pgfpathlineto{\pgfqpoint{3.904204in}{2.441042in}}%
\pgfpathlineto{\pgfqpoint{3.906711in}{2.451393in}}%
\pgfpathlineto{\pgfqpoint{3.911724in}{2.460263in}}%
\pgfpathlineto{\pgfqpoint{3.914231in}{2.460968in}}%
\pgfpathlineto{\pgfqpoint{3.916737in}{2.467917in}}%
\pgfpathlineto{\pgfqpoint{3.919244in}{2.471369in}}%
\pgfpathlineto{\pgfqpoint{3.929271in}{2.476008in}}%
\pgfpathlineto{\pgfqpoint{3.931777in}{2.476211in}}%
\pgfpathlineto{\pgfqpoint{3.934284in}{2.480835in}}%
\pgfpathlineto{\pgfqpoint{3.936791in}{2.481240in}}%
\pgfpathlineto{\pgfqpoint{3.941804in}{2.484978in}}%
\pgfpathlineto{\pgfqpoint{3.946817in}{2.506139in}}%
\pgfpathlineto{\pgfqpoint{3.956844in}{2.508673in}}%
\pgfpathlineto{\pgfqpoint{3.959351in}{2.517868in}}%
\pgfpathlineto{\pgfqpoint{3.961857in}{2.519712in}}%
\pgfpathlineto{\pgfqpoint{3.966871in}{2.533972in}}%
\pgfpathlineto{\pgfqpoint{3.971884in}{2.535801in}}%
\pgfpathlineto{\pgfqpoint{3.974391in}{2.536669in}}%
\pgfpathlineto{\pgfqpoint{3.976897in}{2.545915in}}%
\pgfpathlineto{\pgfqpoint{3.984417in}{2.548650in}}%
\pgfpathlineto{\pgfqpoint{3.991937in}{2.569114in}}%
\pgfpathlineto{\pgfqpoint{3.994444in}{2.570644in}}%
\pgfpathlineto{\pgfqpoint{3.996950in}{2.576270in}}%
\pgfpathlineto{\pgfqpoint{4.001964in}{2.577809in}}%
\pgfpathlineto{\pgfqpoint{4.004470in}{2.579562in}}%
\pgfpathlineto{\pgfqpoint{4.009484in}{2.588586in}}%
\pgfpathlineto{\pgfqpoint{4.011990in}{2.588781in}}%
\pgfpathlineto{\pgfqpoint{4.017004in}{2.592349in}}%
\pgfpathlineto{\pgfqpoint{4.022017in}{2.599236in}}%
\pgfpathlineto{\pgfqpoint{4.024524in}{2.599829in}}%
\pgfpathlineto{\pgfqpoint{4.032044in}{2.624961in}}%
\pgfpathlineto{\pgfqpoint{4.039564in}{2.629967in}}%
\pgfpathlineto{\pgfqpoint{4.042070in}{2.630275in}}%
\pgfpathlineto{\pgfqpoint{4.042070in}{2.630275in}}%
\pgfusepath{stroke}%
\end{pgfscope}%
\begin{pgfscope}%
\pgfpathrectangle{\pgfqpoint{0.708220in}{0.535823in}}{\pgfqpoint{5.013309in}{2.094453in}}%
\pgfusepath{clip}%
\pgfsetrectcap%
\pgfsetroundjoin%
\pgfsetlinewidth{1.003750pt}%
\definecolor{currentstroke}{rgb}{0.062745,0.000000,0.062745}%
\pgfsetstrokecolor{currentstroke}%
\pgfsetdash{}{0pt}%
\pgfpathmoveto{\pgfqpoint{0.708220in}{0.593558in}}%
\pgfpathlineto{\pgfqpoint{0.713233in}{0.593558in}}%
\pgfpathlineto{\pgfqpoint{0.715740in}{0.616471in}}%
\pgfpathlineto{\pgfqpoint{0.740806in}{0.616471in}}%
\pgfpathlineto{\pgfqpoint{0.743313in}{0.636682in}}%
\pgfpathlineto{\pgfqpoint{0.745820in}{0.636682in}}%
\pgfpathlineto{\pgfqpoint{0.750833in}{0.671115in}}%
\pgfpathlineto{\pgfqpoint{0.760860in}{0.671115in}}%
\pgfpathlineto{\pgfqpoint{0.763366in}{0.686045in}}%
\pgfpathlineto{\pgfqpoint{0.800966in}{0.686045in}}%
\pgfpathlineto{\pgfqpoint{0.803473in}{0.699780in}}%
\pgfpathlineto{\pgfqpoint{0.866139in}{0.699780in}}%
\pgfpathlineto{\pgfqpoint{0.868646in}{0.712496in}}%
\pgfpathlineto{\pgfqpoint{0.936325in}{0.712496in}}%
\pgfpathlineto{\pgfqpoint{0.938832in}{0.724335in}}%
\pgfpathlineto{\pgfqpoint{1.021552in}{0.724335in}}%
\pgfpathlineto{\pgfqpoint{1.024058in}{0.735409in}}%
\pgfpathlineto{\pgfqpoint{1.126831in}{0.735409in}}%
\pgfpathlineto{\pgfqpoint{1.129338in}{0.745812in}}%
\pgfpathlineto{\pgfqpoint{1.222084in}{0.745812in}}%
\pgfpathlineto{\pgfqpoint{1.224591in}{0.755619in}}%
\pgfpathlineto{\pgfqpoint{1.297284in}{0.755619in}}%
\pgfpathlineto{\pgfqpoint{1.299790in}{0.764897in}}%
\pgfpathlineto{\pgfqpoint{1.362457in}{0.764897in}}%
\pgfpathlineto{\pgfqpoint{1.364963in}{0.773698in}}%
\pgfpathlineto{\pgfqpoint{1.437656in}{0.773698in}}%
\pgfpathlineto{\pgfqpoint{1.440163in}{0.782070in}}%
\pgfpathlineto{\pgfqpoint{1.490296in}{0.782070in}}%
\pgfpathlineto{\pgfqpoint{1.492803in}{0.790053in}}%
\pgfpathlineto{\pgfqpoint{1.547949in}{0.790053in}}%
\pgfpathlineto{\pgfqpoint{1.550456in}{0.797680in}}%
\pgfpathlineto{\pgfqpoint{1.605602in}{0.797680in}}%
\pgfpathlineto{\pgfqpoint{1.608109in}{0.804983in}}%
\pgfpathlineto{\pgfqpoint{1.650722in}{0.804983in}}%
\pgfpathlineto{\pgfqpoint{1.653229in}{0.811988in}}%
\pgfpathlineto{\pgfqpoint{1.740961in}{0.811988in}}%
\pgfpathlineto{\pgfqpoint{1.743468in}{0.818718in}}%
\pgfpathlineto{\pgfqpoint{1.818668in}{0.818718in}}%
\pgfpathlineto{\pgfqpoint{1.821174in}{0.825194in}}%
\pgfpathlineto{\pgfqpoint{1.871307in}{0.825194in}}%
\pgfpathlineto{\pgfqpoint{1.873814in}{0.831434in}}%
\pgfpathlineto{\pgfqpoint{1.901387in}{0.831434in}}%
\pgfpathlineto{\pgfqpoint{1.903894in}{0.837455in}}%
\pgfpathlineto{\pgfqpoint{1.938987in}{0.837455in}}%
\pgfpathlineto{\pgfqpoint{1.941494in}{0.843272in}}%
\pgfpathlineto{\pgfqpoint{2.001654in}{0.843272in}}%
\pgfpathlineto{\pgfqpoint{2.004160in}{0.848899in}}%
\pgfpathlineto{\pgfqpoint{2.041760in}{0.848899in}}%
\pgfpathlineto{\pgfqpoint{2.044267in}{0.854347in}}%
\pgfpathlineto{\pgfqpoint{2.081866in}{0.854347in}}%
\pgfpathlineto{\pgfqpoint{2.084373in}{0.859627in}}%
\pgfpathlineto{\pgfqpoint{2.116960in}{0.859627in}}%
\pgfpathlineto{\pgfqpoint{2.119466in}{0.864749in}}%
\pgfpathlineto{\pgfqpoint{2.142026in}{0.864749in}}%
\pgfpathlineto{\pgfqpoint{2.144533in}{0.869723in}}%
\pgfpathlineto{\pgfqpoint{2.177119in}{0.869723in}}%
\pgfpathlineto{\pgfqpoint{2.179626in}{0.874557in}}%
\pgfpathlineto{\pgfqpoint{2.204693in}{0.874557in}}%
\pgfpathlineto{\pgfqpoint{2.207199in}{0.879259in}}%
\pgfpathlineto{\pgfqpoint{2.237279in}{0.879259in}}%
\pgfpathlineto{\pgfqpoint{2.239786in}{0.883835in}}%
\pgfpathlineto{\pgfqpoint{2.259839in}{0.883835in}}%
\pgfpathlineto{\pgfqpoint{2.262346in}{0.888292in}}%
\pgfpathlineto{\pgfqpoint{2.272372in}{0.888292in}}%
\pgfpathlineto{\pgfqpoint{2.274879in}{0.892636in}}%
\pgfpathlineto{\pgfqpoint{2.289919in}{0.892636in}}%
\pgfpathlineto{\pgfqpoint{2.292425in}{0.896873in}}%
\pgfpathlineto{\pgfqpoint{2.302452in}{0.896873in}}%
\pgfpathlineto{\pgfqpoint{2.304959in}{0.901008in}}%
\pgfpathlineto{\pgfqpoint{2.322505in}{0.901008in}}%
\pgfpathlineto{\pgfqpoint{2.325012in}{0.905046in}}%
\pgfpathlineto{\pgfqpoint{2.332532in}{0.905046in}}%
\pgfpathlineto{\pgfqpoint{2.335039in}{0.908991in}}%
\pgfpathlineto{\pgfqpoint{2.350078in}{0.908991in}}%
\pgfpathlineto{\pgfqpoint{2.352585in}{0.912847in}}%
\pgfpathlineto{\pgfqpoint{2.360105in}{0.912847in}}%
\pgfpathlineto{\pgfqpoint{2.362612in}{0.916618in}}%
\pgfpathlineto{\pgfqpoint{2.377652in}{0.916618in}}%
\pgfpathlineto{\pgfqpoint{2.380158in}{0.920308in}}%
\pgfpathlineto{\pgfqpoint{2.392692in}{0.920308in}}%
\pgfpathlineto{\pgfqpoint{2.395198in}{0.923921in}}%
\pgfpathlineto{\pgfqpoint{2.400212in}{0.923921in}}%
\pgfpathlineto{\pgfqpoint{2.402718in}{0.927459in}}%
\pgfpathlineto{\pgfqpoint{2.407732in}{0.927459in}}%
\pgfpathlineto{\pgfqpoint{2.410238in}{0.930926in}}%
\pgfpathlineto{\pgfqpoint{2.415251in}{0.930926in}}%
\pgfpathlineto{\pgfqpoint{2.417758in}{0.934324in}}%
\pgfpathlineto{\pgfqpoint{2.422771in}{0.934324in}}%
\pgfpathlineto{\pgfqpoint{2.425278in}{0.937656in}}%
\pgfpathlineto{\pgfqpoint{2.435305in}{0.937656in}}%
\pgfpathlineto{\pgfqpoint{2.437811in}{0.940924in}}%
\pgfpathlineto{\pgfqpoint{2.442825in}{0.940924in}}%
\pgfpathlineto{\pgfqpoint{2.445331in}{0.944131in}}%
\pgfpathlineto{\pgfqpoint{2.465385in}{0.944131in}}%
\pgfpathlineto{\pgfqpoint{2.467891in}{0.947280in}}%
\pgfpathlineto{\pgfqpoint{2.495464in}{0.947280in}}%
\pgfpathlineto{\pgfqpoint{2.497971in}{0.950372in}}%
\pgfpathlineto{\pgfqpoint{2.515518in}{0.950372in}}%
\pgfpathlineto{\pgfqpoint{2.518024in}{0.953409in}}%
\pgfpathlineto{\pgfqpoint{2.525544in}{0.953409in}}%
\pgfpathlineto{\pgfqpoint{2.528051in}{0.956393in}}%
\pgfpathlineto{\pgfqpoint{2.530558in}{0.956393in}}%
\pgfpathlineto{\pgfqpoint{2.533064in}{0.959326in}}%
\pgfpathlineto{\pgfqpoint{2.535571in}{0.959326in}}%
\pgfpathlineto{\pgfqpoint{2.538078in}{0.962210in}}%
\pgfpathlineto{\pgfqpoint{2.543091in}{0.962210in}}%
\pgfpathlineto{\pgfqpoint{2.545598in}{0.965047in}}%
\pgfpathlineto{\pgfqpoint{2.548104in}{0.965047in}}%
\pgfpathlineto{\pgfqpoint{2.550611in}{0.967837in}}%
\pgfpathlineto{\pgfqpoint{2.563144in}{0.967837in}}%
\pgfpathlineto{\pgfqpoint{2.565651in}{0.970582in}}%
\pgfpathlineto{\pgfqpoint{2.568157in}{0.970582in}}%
\pgfpathlineto{\pgfqpoint{2.570664in}{0.973285in}}%
\pgfpathlineto{\pgfqpoint{2.578184in}{0.973285in}}%
\pgfpathlineto{\pgfqpoint{2.583197in}{0.978565in}}%
\pgfpathlineto{\pgfqpoint{2.588211in}{0.978565in}}%
\pgfpathlineto{\pgfqpoint{2.593224in}{0.983687in}}%
\pgfpathlineto{\pgfqpoint{2.595731in}{0.983687in}}%
\pgfpathlineto{\pgfqpoint{2.598237in}{0.991095in}}%
\pgfpathlineto{\pgfqpoint{2.600744in}{0.993495in}}%
\pgfpathlineto{\pgfqpoint{2.605757in}{0.993495in}}%
\pgfpathlineto{\pgfqpoint{2.608264in}{0.998196in}}%
\pgfpathlineto{\pgfqpoint{2.613277in}{0.998196in}}%
\pgfpathlineto{\pgfqpoint{2.615784in}{1.000500in}}%
\pgfpathlineto{\pgfqpoint{2.618290in}{1.005016in}}%
\pgfpathlineto{\pgfqpoint{2.620797in}{1.007230in}}%
\pgfpathlineto{\pgfqpoint{2.623304in}{1.007230in}}%
\pgfpathlineto{\pgfqpoint{2.625810in}{1.009416in}}%
\pgfpathlineto{\pgfqpoint{2.630824in}{1.017891in}}%
\pgfpathlineto{\pgfqpoint{2.633330in}{1.017891in}}%
\pgfpathlineto{\pgfqpoint{2.638344in}{1.023984in}}%
\pgfpathlineto{\pgfqpoint{2.640850in}{1.027928in}}%
\pgfpathlineto{\pgfqpoint{2.643357in}{1.027928in}}%
\pgfpathlineto{\pgfqpoint{2.645864in}{1.031785in}}%
\pgfpathlineto{\pgfqpoint{2.648370in}{1.031785in}}%
\pgfpathlineto{\pgfqpoint{2.650877in}{1.033681in}}%
\pgfpathlineto{\pgfqpoint{2.653384in}{1.037411in}}%
\pgfpathlineto{\pgfqpoint{2.655890in}{1.037411in}}%
\pgfpathlineto{\pgfqpoint{2.658397in}{1.042859in}}%
\pgfpathlineto{\pgfqpoint{2.660904in}{1.044637in}}%
\pgfpathlineto{\pgfqpoint{2.663410in}{1.049863in}}%
\pgfpathlineto{\pgfqpoint{2.665917in}{1.049863in}}%
\pgfpathlineto{\pgfqpoint{2.670930in}{1.059862in}}%
\pgfpathlineto{\pgfqpoint{2.673437in}{1.059862in}}%
\pgfpathlineto{\pgfqpoint{2.675944in}{1.066218in}}%
\pgfpathlineto{\pgfqpoint{2.678450in}{1.069310in}}%
\pgfpathlineto{\pgfqpoint{2.680957in}{1.075331in}}%
\pgfpathlineto{\pgfqpoint{2.690983in}{1.081148in}}%
\pgfpathlineto{\pgfqpoint{2.695997in}{1.085385in}}%
\pgfpathlineto{\pgfqpoint{2.701010in}{1.085385in}}%
\pgfpathlineto{\pgfqpoint{2.706023in}{1.089520in}}%
\pgfpathlineto{\pgfqpoint{2.708530in}{1.089520in}}%
\pgfpathlineto{\pgfqpoint{2.711037in}{1.090877in}}%
\pgfpathlineto{\pgfqpoint{2.713543in}{1.094883in}}%
\pgfpathlineto{\pgfqpoint{2.718557in}{1.097503in}}%
\pgfpathlineto{\pgfqpoint{2.721063in}{1.097503in}}%
\pgfpathlineto{\pgfqpoint{2.726077in}{1.100083in}}%
\pgfpathlineto{\pgfqpoint{2.728583in}{1.112433in}}%
\pgfpathlineto{\pgfqpoint{2.731090in}{1.112433in}}%
\pgfpathlineto{\pgfqpoint{2.733597in}{1.117134in}}%
\pgfpathlineto{\pgfqpoint{2.746130in}{1.128353in}}%
\pgfpathlineto{\pgfqpoint{2.751143in}{1.130512in}}%
\pgfpathlineto{\pgfqpoint{2.753650in}{1.133699in}}%
\pgfpathlineto{\pgfqpoint{2.758663in}{1.134749in}}%
\pgfpathlineto{\pgfqpoint{2.763676in}{1.135792in}}%
\pgfpathlineto{\pgfqpoint{2.766183in}{1.140914in}}%
\pgfpathlineto{\pgfqpoint{2.771196in}{1.143916in}}%
\pgfpathlineto{\pgfqpoint{2.776210in}{1.144905in}}%
\pgfpathlineto{\pgfqpoint{2.778716in}{1.149766in}}%
\pgfpathlineto{\pgfqpoint{2.783730in}{1.153559in}}%
\pgfpathlineto{\pgfqpoint{2.788743in}{1.153559in}}%
\pgfpathlineto{\pgfqpoint{2.796263in}{1.158184in}}%
\pgfpathlineto{\pgfqpoint{2.798770in}{1.158184in}}%
\pgfpathlineto{\pgfqpoint{2.803783in}{1.163575in}}%
\pgfpathlineto{\pgfqpoint{2.811303in}{1.164457in}}%
\pgfpathlineto{\pgfqpoint{2.816316in}{1.170509in}}%
\pgfpathlineto{\pgfqpoint{2.821329in}{1.173038in}}%
\pgfpathlineto{\pgfqpoint{2.823836in}{1.174704in}}%
\pgfpathlineto{\pgfqpoint{2.828849in}{1.175531in}}%
\pgfpathlineto{\pgfqpoint{2.831356in}{1.177173in}}%
\pgfpathlineto{\pgfqpoint{2.836369in}{1.192035in}}%
\pgfpathlineto{\pgfqpoint{2.841383in}{1.195007in}}%
\pgfpathlineto{\pgfqpoint{2.843889in}{1.195007in}}%
\pgfpathlineto{\pgfqpoint{2.846396in}{1.196473in}}%
\pgfpathlineto{\pgfqpoint{2.851409in}{1.203624in}}%
\pgfpathlineto{\pgfqpoint{2.853916in}{1.207776in}}%
\pgfpathlineto{\pgfqpoint{2.858929in}{1.209814in}}%
\pgfpathlineto{\pgfqpoint{2.861436in}{1.214479in}}%
\pgfpathlineto{\pgfqpoint{2.868956in}{1.218379in}}%
\pgfpathlineto{\pgfqpoint{2.871463in}{1.223445in}}%
\pgfpathlineto{\pgfqpoint{2.873969in}{1.224688in}}%
\pgfpathlineto{\pgfqpoint{2.876476in}{1.224688in}}%
\pgfpathlineto{\pgfqpoint{2.878983in}{1.227149in}}%
\pgfpathlineto{\pgfqpoint{2.881489in}{1.227149in}}%
\pgfpathlineto{\pgfqpoint{2.886503in}{1.231966in}}%
\pgfpathlineto{\pgfqpoint{2.889009in}{1.233738in}}%
\pgfpathlineto{\pgfqpoint{2.891516in}{1.240648in}}%
\pgfpathlineto{\pgfqpoint{2.896529in}{1.241773in}}%
\pgfpathlineto{\pgfqpoint{2.904049in}{1.243447in}}%
\pgfpathlineto{\pgfqpoint{2.909062in}{1.250519in}}%
\pgfpathlineto{\pgfqpoint{2.916582in}{1.252637in}}%
\pgfpathlineto{\pgfqpoint{2.921596in}{1.256797in}}%
\pgfpathlineto{\pgfqpoint{2.924102in}{1.258332in}}%
\pgfpathlineto{\pgfqpoint{2.926609in}{1.262357in}}%
\pgfpathlineto{\pgfqpoint{2.929116in}{1.262854in}}%
\pgfpathlineto{\pgfqpoint{2.931622in}{1.266776in}}%
\pgfpathlineto{\pgfqpoint{2.939142in}{1.270611in}}%
\pgfpathlineto{\pgfqpoint{2.944156in}{1.273432in}}%
\pgfpathlineto{\pgfqpoint{2.946662in}{1.276665in}}%
\pgfpathlineto{\pgfqpoint{2.949169in}{1.284272in}}%
\pgfpathlineto{\pgfqpoint{2.951676in}{1.285581in}}%
\pgfpathlineto{\pgfqpoint{2.954182in}{1.285581in}}%
\pgfpathlineto{\pgfqpoint{2.956689in}{1.291137in}}%
\pgfpathlineto{\pgfqpoint{2.959195in}{1.292394in}}%
\pgfpathlineto{\pgfqpoint{2.961702in}{1.296111in}}%
\pgfpathlineto{\pgfqpoint{2.976742in}{1.305646in}}%
\pgfpathlineto{\pgfqpoint{2.981755in}{1.306418in}}%
\pgfpathlineto{\pgfqpoint{2.986769in}{1.310973in}}%
\pgfpathlineto{\pgfqpoint{2.991782in}{1.318307in}}%
\pgfpathlineto{\pgfqpoint{2.996795in}{1.319381in}}%
\pgfpathlineto{\pgfqpoint{3.001809in}{1.327396in}}%
\pgfpathlineto{\pgfqpoint{3.006822in}{1.333088in}}%
\pgfpathlineto{\pgfqpoint{3.009329in}{1.334727in}}%
\pgfpathlineto{\pgfqpoint{3.016849in}{1.344245in}}%
\pgfpathlineto{\pgfqpoint{3.019355in}{1.344553in}}%
\pgfpathlineto{\pgfqpoint{3.024369in}{1.352969in}}%
\pgfpathlineto{\pgfqpoint{3.031888in}{1.365957in}}%
\pgfpathlineto{\pgfqpoint{3.036902in}{1.366500in}}%
\pgfpathlineto{\pgfqpoint{3.041915in}{1.368655in}}%
\pgfpathlineto{\pgfqpoint{3.049435in}{1.372886in}}%
\pgfpathlineto{\pgfqpoint{3.051942in}{1.372886in}}%
\pgfpathlineto{\pgfqpoint{3.054448in}{1.374187in}}%
\pgfpathlineto{\pgfqpoint{3.059462in}{1.384742in}}%
\pgfpathlineto{\pgfqpoint{3.061968in}{1.386922in}}%
\pgfpathlineto{\pgfqpoint{3.064475in}{1.386922in}}%
\pgfpathlineto{\pgfqpoint{3.066982in}{1.388598in}}%
\pgfpathlineto{\pgfqpoint{3.074502in}{1.401229in}}%
\pgfpathlineto{\pgfqpoint{3.077008in}{1.414230in}}%
\pgfpathlineto{\pgfqpoint{3.079515in}{1.417080in}}%
\pgfpathlineto{\pgfqpoint{3.084528in}{1.428972in}}%
\pgfpathlineto{\pgfqpoint{3.089542in}{1.431217in}}%
\pgfpathlineto{\pgfqpoint{3.102075in}{1.438319in}}%
\pgfpathlineto{\pgfqpoint{3.107088in}{1.438853in}}%
\pgfpathlineto{\pgfqpoint{3.112101in}{1.452026in}}%
\pgfpathlineto{\pgfqpoint{3.114608in}{1.452191in}}%
\pgfpathlineto{\pgfqpoint{3.117115in}{1.454803in}}%
\pgfpathlineto{\pgfqpoint{3.127141in}{1.457376in}}%
\pgfpathlineto{\pgfqpoint{3.129648in}{1.459123in}}%
\pgfpathlineto{\pgfqpoint{3.132155in}{1.459281in}}%
\pgfpathlineto{\pgfqpoint{3.134661in}{1.462409in}}%
\pgfpathlineto{\pgfqpoint{3.139675in}{1.465177in}}%
\pgfpathlineto{\pgfqpoint{3.142181in}{1.478383in}}%
\pgfpathlineto{\pgfqpoint{3.144688in}{1.479087in}}%
\pgfpathlineto{\pgfqpoint{3.147195in}{1.483393in}}%
\pgfpathlineto{\pgfqpoint{3.149701in}{1.484214in}}%
\pgfpathlineto{\pgfqpoint{3.152208in}{1.491301in}}%
\pgfpathlineto{\pgfqpoint{3.154715in}{1.504166in}}%
\pgfpathlineto{\pgfqpoint{3.157221in}{1.505980in}}%
\pgfpathlineto{\pgfqpoint{3.162234in}{1.518387in}}%
\pgfpathlineto{\pgfqpoint{3.164741in}{1.518387in}}%
\pgfpathlineto{\pgfqpoint{3.169754in}{1.532242in}}%
\pgfpathlineto{\pgfqpoint{3.172261in}{1.532448in}}%
\pgfpathlineto{\pgfqpoint{3.177274in}{1.535613in}}%
\pgfpathlineto{\pgfqpoint{3.182288in}{1.539712in}}%
\pgfpathlineto{\pgfqpoint{3.184794in}{1.547533in}}%
\pgfpathlineto{\pgfqpoint{3.187301in}{1.551175in}}%
\pgfpathlineto{\pgfqpoint{3.189808in}{1.558678in}}%
\pgfpathlineto{\pgfqpoint{3.194821in}{1.561659in}}%
\pgfpathlineto{\pgfqpoint{3.199834in}{1.565866in}}%
\pgfpathlineto{\pgfqpoint{3.202341in}{1.566797in}}%
\pgfpathlineto{\pgfqpoint{3.207354in}{1.574871in}}%
\pgfpathlineto{\pgfqpoint{3.209861in}{1.575032in}}%
\pgfpathlineto{\pgfqpoint{3.214874in}{1.579163in}}%
\pgfpathlineto{\pgfqpoint{3.222394in}{1.582352in}}%
\pgfpathlineto{\pgfqpoint{3.224901in}{1.586312in}}%
\pgfpathlineto{\pgfqpoint{3.227408in}{1.586838in}}%
\pgfpathlineto{\pgfqpoint{3.229914in}{1.593031in}}%
\pgfpathlineto{\pgfqpoint{3.234927in}{1.593969in}}%
\pgfpathlineto{\pgfqpoint{3.237434in}{1.596471in}}%
\pgfpathlineto{\pgfqpoint{3.239941in}{1.596542in}}%
\pgfpathlineto{\pgfqpoint{3.247461in}{1.603288in}}%
\pgfpathlineto{\pgfqpoint{3.249967in}{1.609451in}}%
\pgfpathlineto{\pgfqpoint{3.254981in}{1.613032in}}%
\pgfpathlineto{\pgfqpoint{3.259994in}{1.616666in}}%
\pgfpathlineto{\pgfqpoint{3.265007in}{1.629310in}}%
\pgfpathlineto{\pgfqpoint{3.272527in}{1.635072in}}%
\pgfpathlineto{\pgfqpoint{3.275034in}{1.636033in}}%
\pgfpathlineto{\pgfqpoint{3.280047in}{1.639933in}}%
\pgfpathlineto{\pgfqpoint{3.282554in}{1.640318in}}%
\pgfpathlineto{\pgfqpoint{3.290074in}{1.650404in}}%
\pgfpathlineto{\pgfqpoint{3.292581in}{1.651952in}}%
\pgfpathlineto{\pgfqpoint{3.295087in}{1.659635in}}%
\pgfpathlineto{\pgfqpoint{3.297594in}{1.660027in}}%
\pgfpathlineto{\pgfqpoint{3.300100in}{1.666376in}}%
\pgfpathlineto{\pgfqpoint{3.307620in}{1.670482in}}%
\pgfpathlineto{\pgfqpoint{3.310127in}{1.674896in}}%
\pgfpathlineto{\pgfqpoint{3.312634in}{1.676995in}}%
\pgfpathlineto{\pgfqpoint{3.315140in}{1.680814in}}%
\pgfpathlineto{\pgfqpoint{3.322660in}{1.684719in}}%
\pgfpathlineto{\pgfqpoint{3.325167in}{1.687956in}}%
\pgfpathlineto{\pgfqpoint{3.327674in}{1.704949in}}%
\pgfpathlineto{\pgfqpoint{3.332687in}{1.706413in}}%
\pgfpathlineto{\pgfqpoint{3.335194in}{1.720271in}}%
\pgfpathlineto{\pgfqpoint{3.340207in}{1.727266in}}%
\pgfpathlineto{\pgfqpoint{3.342714in}{1.733701in}}%
\pgfpathlineto{\pgfqpoint{3.345220in}{1.736735in}}%
\pgfpathlineto{\pgfqpoint{3.347727in}{1.744366in}}%
\pgfpathlineto{\pgfqpoint{3.350234in}{1.745709in}}%
\pgfpathlineto{\pgfqpoint{3.352740in}{1.761006in}}%
\pgfpathlineto{\pgfqpoint{3.357754in}{1.761847in}}%
\pgfpathlineto{\pgfqpoint{3.360260in}{1.762226in}}%
\pgfpathlineto{\pgfqpoint{3.370287in}{1.773590in}}%
\pgfpathlineto{\pgfqpoint{3.372793in}{1.774170in}}%
\pgfpathlineto{\pgfqpoint{3.375300in}{1.779577in}}%
\pgfpathlineto{\pgfqpoint{3.380313in}{1.781976in}}%
\pgfpathlineto{\pgfqpoint{3.385327in}{1.783462in}}%
\pgfpathlineto{\pgfqpoint{3.387833in}{1.789627in}}%
\pgfpathlineto{\pgfqpoint{3.397860in}{1.791505in}}%
\pgfpathlineto{\pgfqpoint{3.407887in}{1.793924in}}%
\pgfpathlineto{\pgfqpoint{3.410393in}{1.794529in}}%
\pgfpathlineto{\pgfqpoint{3.412900in}{1.800272in}}%
\pgfpathlineto{\pgfqpoint{3.415407in}{1.801028in}}%
\pgfpathlineto{\pgfqpoint{3.420420in}{1.807206in}}%
\pgfpathlineto{\pgfqpoint{3.422927in}{1.819996in}}%
\pgfpathlineto{\pgfqpoint{3.432953in}{1.821092in}}%
\pgfpathlineto{\pgfqpoint{3.435460in}{1.822333in}}%
\pgfpathlineto{\pgfqpoint{3.437966in}{1.825855in}}%
\pgfpathlineto{\pgfqpoint{3.440473in}{1.826395in}}%
\pgfpathlineto{\pgfqpoint{3.442980in}{1.830362in}}%
\pgfpathlineto{\pgfqpoint{3.447993in}{1.831033in}}%
\pgfpathlineto{\pgfqpoint{3.453006in}{1.839106in}}%
\pgfpathlineto{\pgfqpoint{3.455513in}{1.846105in}}%
\pgfpathlineto{\pgfqpoint{3.458020in}{1.848963in}}%
\pgfpathlineto{\pgfqpoint{3.460526in}{1.863334in}}%
\pgfpathlineto{\pgfqpoint{3.463033in}{1.863693in}}%
\pgfpathlineto{\pgfqpoint{3.465540in}{1.877864in}}%
\pgfpathlineto{\pgfqpoint{3.470553in}{1.878880in}}%
\pgfpathlineto{\pgfqpoint{3.473060in}{1.882870in}}%
\pgfpathlineto{\pgfqpoint{3.480580in}{1.883431in}}%
\pgfpathlineto{\pgfqpoint{3.485593in}{1.889300in}}%
\pgfpathlineto{\pgfqpoint{3.490606in}{1.893963in}}%
\pgfpathlineto{\pgfqpoint{3.493113in}{1.903376in}}%
\pgfpathlineto{\pgfqpoint{3.495620in}{1.907595in}}%
\pgfpathlineto{\pgfqpoint{3.498126in}{1.915414in}}%
\pgfpathlineto{\pgfqpoint{3.500633in}{1.918244in}}%
\pgfpathlineto{\pgfqpoint{3.503139in}{1.922902in}}%
\pgfpathlineto{\pgfqpoint{3.505646in}{1.930645in}}%
\pgfpathlineto{\pgfqpoint{3.508153in}{1.930858in}}%
\pgfpathlineto{\pgfqpoint{3.513166in}{1.937160in}}%
\pgfpathlineto{\pgfqpoint{3.515673in}{1.944578in}}%
\pgfpathlineto{\pgfqpoint{3.520686in}{1.950341in}}%
\pgfpathlineto{\pgfqpoint{3.523193in}{1.953984in}}%
\pgfpathlineto{\pgfqpoint{3.525699in}{1.954152in}}%
\pgfpathlineto{\pgfqpoint{3.528206in}{1.960196in}}%
\pgfpathlineto{\pgfqpoint{3.530713in}{1.962472in}}%
\pgfpathlineto{\pgfqpoint{3.533219in}{1.966954in}}%
\pgfpathlineto{\pgfqpoint{3.543246in}{1.974140in}}%
\pgfpathlineto{\pgfqpoint{3.545753in}{1.988524in}}%
\pgfpathlineto{\pgfqpoint{3.553273in}{1.998732in}}%
\pgfpathlineto{\pgfqpoint{3.555779in}{1.998773in}}%
\pgfpathlineto{\pgfqpoint{3.560793in}{2.002635in}}%
\pgfpathlineto{\pgfqpoint{3.565806in}{2.015130in}}%
\pgfpathlineto{\pgfqpoint{3.568313in}{2.018874in}}%
\pgfpathlineto{\pgfqpoint{3.573326in}{2.020387in}}%
\pgfpathlineto{\pgfqpoint{3.575832in}{2.023949in}}%
\pgfpathlineto{\pgfqpoint{3.580846in}{2.050053in}}%
\pgfpathlineto{\pgfqpoint{3.585859in}{2.058996in}}%
\pgfpathlineto{\pgfqpoint{3.588366in}{2.059351in}}%
\pgfpathlineto{\pgfqpoint{3.593379in}{2.064224in}}%
\pgfpathlineto{\pgfqpoint{3.595886in}{2.064730in}}%
\pgfpathlineto{\pgfqpoint{3.598392in}{2.070282in}}%
\pgfpathlineto{\pgfqpoint{3.600899in}{2.072564in}}%
\pgfpathlineto{\pgfqpoint{3.603406in}{2.082216in}}%
\pgfpathlineto{\pgfqpoint{3.605912in}{2.084642in}}%
\pgfpathlineto{\pgfqpoint{3.608419in}{2.093529in}}%
\pgfpathlineto{\pgfqpoint{3.613432in}{2.095810in}}%
\pgfpathlineto{\pgfqpoint{3.615939in}{2.099068in}}%
\pgfpathlineto{\pgfqpoint{3.623459in}{2.102586in}}%
\pgfpathlineto{\pgfqpoint{3.628472in}{2.116426in}}%
\pgfpathlineto{\pgfqpoint{3.635992in}{2.126508in}}%
\pgfpathlineto{\pgfqpoint{3.638499in}{2.139451in}}%
\pgfpathlineto{\pgfqpoint{3.641005in}{2.141409in}}%
\pgfpathlineto{\pgfqpoint{3.648525in}{2.170243in}}%
\pgfpathlineto{\pgfqpoint{3.651032in}{2.180464in}}%
\pgfpathlineto{\pgfqpoint{3.653539in}{2.183313in}}%
\pgfpathlineto{\pgfqpoint{3.656045in}{2.188359in}}%
\pgfpathlineto{\pgfqpoint{3.661059in}{2.189216in}}%
\pgfpathlineto{\pgfqpoint{3.663565in}{2.191588in}}%
\pgfpathlineto{\pgfqpoint{3.673592in}{2.211560in}}%
\pgfpathlineto{\pgfqpoint{3.676099in}{2.218786in}}%
\pgfpathlineto{\pgfqpoint{3.678605in}{2.240787in}}%
\pgfpathlineto{\pgfqpoint{3.683619in}{2.246036in}}%
\pgfpathlineto{\pgfqpoint{3.686125in}{2.246536in}}%
\pgfpathlineto{\pgfqpoint{3.688632in}{2.252277in}}%
\pgfpathlineto{\pgfqpoint{3.693645in}{2.252948in}}%
\pgfpathlineto{\pgfqpoint{3.696152in}{2.262230in}}%
\pgfpathlineto{\pgfqpoint{3.698659in}{2.263561in}}%
\pgfpathlineto{\pgfqpoint{3.703672in}{2.272470in}}%
\pgfpathlineto{\pgfqpoint{3.706178in}{2.274777in}}%
\pgfpathlineto{\pgfqpoint{3.708685in}{2.275350in}}%
\pgfpathlineto{\pgfqpoint{3.711192in}{2.279122in}}%
\pgfpathlineto{\pgfqpoint{3.716205in}{2.280060in}}%
\pgfpathlineto{\pgfqpoint{3.726232in}{2.294553in}}%
\pgfpathlineto{\pgfqpoint{3.731245in}{2.300230in}}%
\pgfpathlineto{\pgfqpoint{3.736258in}{2.315195in}}%
\pgfpathlineto{\pgfqpoint{3.738765in}{2.321998in}}%
\pgfpathlineto{\pgfqpoint{3.758818in}{2.340537in}}%
\pgfpathlineto{\pgfqpoint{3.761325in}{2.340817in}}%
\pgfpathlineto{\pgfqpoint{3.763832in}{2.350950in}}%
\pgfpathlineto{\pgfqpoint{3.773858in}{2.359101in}}%
\pgfpathlineto{\pgfqpoint{3.778871in}{2.368920in}}%
\pgfpathlineto{\pgfqpoint{3.781378in}{2.382902in}}%
\pgfpathlineto{\pgfqpoint{3.786391in}{2.385362in}}%
\pgfpathlineto{\pgfqpoint{3.788898in}{2.399800in}}%
\pgfpathlineto{\pgfqpoint{3.803938in}{2.427107in}}%
\pgfpathlineto{\pgfqpoint{3.806445in}{2.434441in}}%
\pgfpathlineto{\pgfqpoint{3.808951in}{2.435942in}}%
\pgfpathlineto{\pgfqpoint{3.811458in}{2.442220in}}%
\pgfpathlineto{\pgfqpoint{3.816471in}{2.444784in}}%
\pgfpathlineto{\pgfqpoint{3.821485in}{2.452785in}}%
\pgfpathlineto{\pgfqpoint{3.834018in}{2.456582in}}%
\pgfpathlineto{\pgfqpoint{3.839031in}{2.461889in}}%
\pgfpathlineto{\pgfqpoint{3.841538in}{2.462008in}}%
\pgfpathlineto{\pgfqpoint{3.846551in}{2.478880in}}%
\pgfpathlineto{\pgfqpoint{3.856578in}{2.487697in}}%
\pgfpathlineto{\pgfqpoint{3.864098in}{2.500309in}}%
\pgfpathlineto{\pgfqpoint{3.866604in}{2.500523in}}%
\pgfpathlineto{\pgfqpoint{3.879138in}{2.523319in}}%
\pgfpathlineto{\pgfqpoint{3.884151in}{2.544532in}}%
\pgfpathlineto{\pgfqpoint{3.889164in}{2.544843in}}%
\pgfpathlineto{\pgfqpoint{3.891671in}{2.546312in}}%
\pgfpathlineto{\pgfqpoint{3.894178in}{2.557695in}}%
\pgfpathlineto{\pgfqpoint{3.896684in}{2.559319in}}%
\pgfpathlineto{\pgfqpoint{3.899191in}{2.564961in}}%
\pgfpathlineto{\pgfqpoint{3.901698in}{2.579441in}}%
\pgfpathlineto{\pgfqpoint{3.906711in}{2.580723in}}%
\pgfpathlineto{\pgfqpoint{3.914231in}{2.595522in}}%
\pgfpathlineto{\pgfqpoint{3.916737in}{2.597761in}}%
\pgfpathlineto{\pgfqpoint{3.924257in}{2.599712in}}%
\pgfpathlineto{\pgfqpoint{3.929271in}{2.618545in}}%
\pgfpathlineto{\pgfqpoint{3.931777in}{2.618736in}}%
\pgfpathlineto{\pgfqpoint{3.941804in}{2.630275in}}%
\pgfpathlineto{\pgfqpoint{3.941804in}{2.630275in}}%
\pgfusepath{stroke}%
\end{pgfscope}%
\begin{pgfscope}%
\pgfpathrectangle{\pgfqpoint{0.708220in}{0.535823in}}{\pgfqpoint{5.013309in}{2.094453in}}%
\pgfusepath{clip}%
\pgfsetbuttcap%
\pgfsetroundjoin%
\pgfsetlinewidth{1.003750pt}%
\definecolor{currentstroke}{rgb}{0.000000,0.000000,0.000000}%
\pgfsetstrokecolor{currentstroke}%
\pgfsetdash{{3.700000pt}{1.600000pt}}{0.000000pt}%
\pgfpathmoveto{\pgfqpoint{0.708220in}{1.108820in}}%
\pgfpathlineto{\pgfqpoint{0.710727in}{1.108820in}}%
\pgfpathlineto{\pgfqpoint{0.713233in}{1.115971in}}%
\pgfpathlineto{\pgfqpoint{0.715740in}{1.117134in}}%
\pgfpathlineto{\pgfqpoint{0.718246in}{1.121710in}}%
\pgfpathlineto{\pgfqpoint{0.730780in}{1.128353in}}%
\pgfpathlineto{\pgfqpoint{0.733286in}{1.131581in}}%
\pgfpathlineto{\pgfqpoint{0.743313in}{1.134749in}}%
\pgfpathlineto{\pgfqpoint{0.745820in}{1.134749in}}%
\pgfpathlineto{\pgfqpoint{0.753340in}{1.139902in}}%
\pgfpathlineto{\pgfqpoint{0.760860in}{1.140914in}}%
\pgfpathlineto{\pgfqpoint{0.775900in}{1.147838in}}%
\pgfpathlineto{\pgfqpoint{0.778406in}{1.149766in}}%
\pgfpathlineto{\pgfqpoint{0.780913in}{1.149766in}}%
\pgfpathlineto{\pgfqpoint{0.790939in}{1.156349in}}%
\pgfpathlineto{\pgfqpoint{0.800966in}{1.160000in}}%
\pgfpathlineto{\pgfqpoint{0.803473in}{1.163575in}}%
\pgfpathlineto{\pgfqpoint{0.805979in}{1.163575in}}%
\pgfpathlineto{\pgfqpoint{0.810993in}{1.168801in}}%
\pgfpathlineto{\pgfqpoint{0.818513in}{1.174704in}}%
\pgfpathlineto{\pgfqpoint{0.821019in}{1.177173in}}%
\pgfpathlineto{\pgfqpoint{0.823526in}{1.177988in}}%
\pgfpathlineto{\pgfqpoint{0.826033in}{1.182800in}}%
\pgfpathlineto{\pgfqpoint{0.833553in}{1.184374in}}%
\pgfpathlineto{\pgfqpoint{0.836059in}{1.188247in}}%
\pgfpathlineto{\pgfqpoint{0.838566in}{1.188247in}}%
\pgfpathlineto{\pgfqpoint{0.858619in}{1.201510in}}%
\pgfpathlineto{\pgfqpoint{0.861126in}{1.210489in}}%
\pgfpathlineto{\pgfqpoint{0.863632in}{1.210489in}}%
\pgfpathlineto{\pgfqpoint{0.868646in}{1.216440in}}%
\pgfpathlineto{\pgfqpoint{0.871152in}{1.216440in}}%
\pgfpathlineto{\pgfqpoint{0.873659in}{1.217735in}}%
\pgfpathlineto{\pgfqpoint{0.876166in}{1.222820in}}%
\pgfpathlineto{\pgfqpoint{0.881179in}{1.224068in}}%
\pgfpathlineto{\pgfqpoint{0.883686in}{1.252110in}}%
\pgfpathlineto{\pgfqpoint{0.886192in}{1.253687in}}%
\pgfpathlineto{\pgfqpoint{0.888699in}{1.269183in}}%
\pgfpathlineto{\pgfqpoint{0.893712in}{1.272496in}}%
\pgfpathlineto{\pgfqpoint{0.896219in}{1.272496in}}%
\pgfpathlineto{\pgfqpoint{0.898726in}{1.286879in}}%
\pgfpathlineto{\pgfqpoint{0.901232in}{1.291137in}}%
\pgfpathlineto{\pgfqpoint{0.911259in}{1.345475in}}%
\pgfpathlineto{\pgfqpoint{0.918779in}{1.353847in}}%
\pgfpathlineto{\pgfqpoint{0.921285in}{1.374446in}}%
\pgfpathlineto{\pgfqpoint{0.931312in}{1.386681in}}%
\pgfpathlineto{\pgfqpoint{0.933819in}{1.387162in}}%
\pgfpathlineto{\pgfqpoint{0.936325in}{1.389311in}}%
\pgfpathlineto{\pgfqpoint{0.938832in}{1.393067in}}%
\pgfpathlineto{\pgfqpoint{0.943845in}{1.394915in}}%
\pgfpathlineto{\pgfqpoint{0.956379in}{1.399672in}}%
\pgfpathlineto{\pgfqpoint{0.961392in}{1.399672in}}%
\pgfpathlineto{\pgfqpoint{0.966405in}{1.401671in}}%
\pgfpathlineto{\pgfqpoint{0.976432in}{1.404301in}}%
\pgfpathlineto{\pgfqpoint{0.981445in}{1.406032in}}%
\pgfpathlineto{\pgfqpoint{0.986458in}{1.406247in}}%
\pgfpathlineto{\pgfqpoint{0.991472in}{1.407959in}}%
\pgfpathlineto{\pgfqpoint{1.006512in}{1.409443in}}%
\pgfpathlineto{\pgfqpoint{1.079205in}{1.414640in}}%
\pgfpathlineto{\pgfqpoint{1.094245in}{1.415253in}}%
\pgfpathlineto{\pgfqpoint{1.099258in}{1.415864in}}%
\pgfpathlineto{\pgfqpoint{1.126831in}{1.416675in}}%
\pgfpathlineto{\pgfqpoint{1.139364in}{1.417684in}}%
\pgfpathlineto{\pgfqpoint{1.174458in}{1.418687in}}%
\pgfpathlineto{\pgfqpoint{1.204537in}{1.420081in}}%
\pgfpathlineto{\pgfqpoint{1.257177in}{1.421070in}}%
\pgfpathlineto{\pgfqpoint{1.267204in}{1.421661in}}%
\pgfpathlineto{\pgfqpoint{1.287257in}{1.422250in}}%
\pgfpathlineto{\pgfqpoint{1.334883in}{1.423226in}}%
\pgfpathlineto{\pgfqpoint{1.354937in}{1.423810in}}%
\pgfpathlineto{\pgfqpoint{1.392537in}{1.424777in}}%
\pgfpathlineto{\pgfqpoint{1.417603in}{1.425740in}}%
\pgfpathlineto{\pgfqpoint{1.442670in}{1.426697in}}%
\pgfpathlineto{\pgfqpoint{1.457710in}{1.426887in}}%
\pgfpathlineto{\pgfqpoint{1.500323in}{1.427838in}}%
\pgfpathlineto{\pgfqpoint{1.520376in}{1.428406in}}%
\pgfpathlineto{\pgfqpoint{1.555469in}{1.429160in}}%
\pgfpathlineto{\pgfqpoint{1.565496in}{1.429724in}}%
\pgfpathlineto{\pgfqpoint{1.608109in}{1.430472in}}%
\pgfpathlineto{\pgfqpoint{1.618135in}{1.430845in}}%
\pgfpathlineto{\pgfqpoint{1.673282in}{1.431774in}}%
\pgfpathlineto{\pgfqpoint{1.700855in}{1.432698in}}%
\pgfpathlineto{\pgfqpoint{1.733442in}{1.433801in}}%
\pgfpathlineto{\pgfqpoint{1.758508in}{1.434532in}}%
\pgfpathlineto{\pgfqpoint{1.788588in}{1.435622in}}%
\pgfpathlineto{\pgfqpoint{1.821174in}{1.436886in}}%
\pgfpathlineto{\pgfqpoint{1.873814in}{1.437962in}}%
\pgfpathlineto{\pgfqpoint{1.901387in}{1.438853in}}%
\pgfpathlineto{\pgfqpoint{1.931467in}{1.439917in}}%
\pgfpathlineto{\pgfqpoint{1.961547in}{1.440974in}}%
\pgfpathlineto{\pgfqpoint{1.974080in}{1.441500in}}%
\pgfpathlineto{\pgfqpoint{1.979094in}{1.442547in}}%
\pgfpathlineto{\pgfqpoint{2.011680in}{1.443761in}}%
\pgfpathlineto{\pgfqpoint{2.036747in}{1.444795in}}%
\pgfpathlineto{\pgfqpoint{2.132000in}{1.450205in}}%
\pgfpathlineto{\pgfqpoint{2.147039in}{1.451201in}}%
\pgfpathlineto{\pgfqpoint{2.149546in}{1.452683in}}%
\pgfpathlineto{\pgfqpoint{2.174613in}{1.454641in}}%
\pgfpathlineto{\pgfqpoint{2.179626in}{1.455933in}}%
\pgfpathlineto{\pgfqpoint{2.202186in}{1.457854in}}%
\pgfpathlineto{\pgfqpoint{2.222239in}{1.458807in}}%
\pgfpathlineto{\pgfqpoint{2.229759in}{1.459281in}}%
\pgfpathlineto{\pgfqpoint{2.247306in}{1.460225in}}%
\pgfpathlineto{\pgfqpoint{2.262346in}{1.461165in}}%
\pgfpathlineto{\pgfqpoint{2.272372in}{1.462099in}}%
\pgfpathlineto{\pgfqpoint{2.287412in}{1.463799in}}%
\pgfpathlineto{\pgfqpoint{2.294932in}{1.465634in}}%
\pgfpathlineto{\pgfqpoint{2.304959in}{1.467450in}}%
\pgfpathlineto{\pgfqpoint{2.327519in}{1.468350in}}%
\pgfpathlineto{\pgfqpoint{2.342559in}{1.470138in}}%
\pgfpathlineto{\pgfqpoint{2.380158in}{1.473076in}}%
\pgfpathlineto{\pgfqpoint{2.392692in}{1.474959in}}%
\pgfpathlineto{\pgfqpoint{2.460371in}{1.479649in}}%
\pgfpathlineto{\pgfqpoint{2.470398in}{1.480906in}}%
\pgfpathlineto{\pgfqpoint{2.485438in}{1.481601in}}%
\pgfpathlineto{\pgfqpoint{2.490451in}{1.482568in}}%
\pgfpathlineto{\pgfqpoint{2.500478in}{1.483393in}}%
\pgfpathlineto{\pgfqpoint{2.505491in}{1.484487in}}%
\pgfpathlineto{\pgfqpoint{2.558131in}{1.489324in}}%
\pgfpathlineto{\pgfqpoint{2.578184in}{1.490250in}}%
\pgfpathlineto{\pgfqpoint{2.593224in}{1.491562in}}%
\pgfpathlineto{\pgfqpoint{2.625810in}{1.493901in}}%
\pgfpathlineto{\pgfqpoint{2.643357in}{1.494930in}}%
\pgfpathlineto{\pgfqpoint{2.658397in}{1.496334in}}%
\pgfpathlineto{\pgfqpoint{2.675944in}{1.497349in}}%
\pgfpathlineto{\pgfqpoint{2.685970in}{1.498358in}}%
\pgfpathlineto{\pgfqpoint{2.738610in}{1.501842in}}%
\pgfpathlineto{\pgfqpoint{2.746130in}{1.502457in}}%
\pgfpathlineto{\pgfqpoint{2.766183in}{1.503192in}}%
\pgfpathlineto{\pgfqpoint{2.778716in}{1.504652in}}%
\pgfpathlineto{\pgfqpoint{3.029382in}{1.529747in}}%
\pgfpathlineto{\pgfqpoint{3.034395in}{1.530895in}}%
\pgfpathlineto{\pgfqpoint{3.046928in}{1.531932in}}%
\pgfpathlineto{\pgfqpoint{3.059462in}{1.533680in}}%
\pgfpathlineto{\pgfqpoint{3.084528in}{1.535916in}}%
\pgfpathlineto{\pgfqpoint{3.089542in}{1.537124in}}%
\pgfpathlineto{\pgfqpoint{3.109595in}{1.539811in}}%
\pgfpathlineto{\pgfqpoint{3.127141in}{1.543038in}}%
\pgfpathlineto{\pgfqpoint{3.149701in}{1.546586in}}%
\pgfpathlineto{\pgfqpoint{3.167248in}{1.550990in}}%
\pgfpathlineto{\pgfqpoint{3.295087in}{1.570634in}}%
\pgfpathlineto{\pgfqpoint{3.310127in}{1.575032in}}%
\pgfpathlineto{\pgfqpoint{3.312634in}{1.575112in}}%
\pgfpathlineto{\pgfqpoint{3.315140in}{1.576792in}}%
\pgfpathlineto{\pgfqpoint{3.340207in}{1.578613in}}%
\pgfpathlineto{\pgfqpoint{3.355247in}{1.581425in}}%
\pgfpathlineto{\pgfqpoint{3.357754in}{1.584191in}}%
\pgfpathlineto{\pgfqpoint{3.362767in}{1.586463in}}%
\pgfpathlineto{\pgfqpoint{3.370287in}{1.589593in}}%
\pgfpathlineto{\pgfqpoint{3.400367in}{1.593248in}}%
\pgfpathlineto{\pgfqpoint{3.405380in}{1.594759in}}%
\pgfpathlineto{\pgfqpoint{3.407887in}{1.597391in}}%
\pgfpathlineto{\pgfqpoint{3.425433in}{1.598797in}}%
\pgfpathlineto{\pgfqpoint{3.427940in}{1.600954in}}%
\pgfpathlineto{\pgfqpoint{3.430447in}{1.601023in}}%
\pgfpathlineto{\pgfqpoint{3.432953in}{1.605255in}}%
\pgfpathlineto{\pgfqpoint{3.435460in}{1.605524in}}%
\pgfpathlineto{\pgfqpoint{3.440473in}{1.607399in}}%
\pgfpathlineto{\pgfqpoint{3.445486in}{1.608792in}}%
\pgfpathlineto{\pgfqpoint{3.447993in}{1.612451in}}%
\pgfpathlineto{\pgfqpoint{3.450500in}{1.613096in}}%
\pgfpathlineto{\pgfqpoint{3.453006in}{1.616350in}}%
\pgfpathlineto{\pgfqpoint{3.458020in}{1.617107in}}%
\pgfpathlineto{\pgfqpoint{3.460526in}{1.617672in}}%
\pgfpathlineto{\pgfqpoint{3.465540in}{1.621272in}}%
\pgfpathlineto{\pgfqpoint{3.470553in}{1.621885in}}%
\pgfpathlineto{\pgfqpoint{3.473060in}{1.623165in}}%
\pgfpathlineto{\pgfqpoint{3.480580in}{1.624195in}}%
\pgfpathlineto{\pgfqpoint{3.483086in}{1.624496in}}%
\pgfpathlineto{\pgfqpoint{3.485593in}{1.627246in}}%
\pgfpathlineto{\pgfqpoint{3.493113in}{1.628546in}}%
\pgfpathlineto{\pgfqpoint{3.495620in}{1.631869in}}%
\pgfpathlineto{\pgfqpoint{3.513166in}{1.634562in}}%
\pgfpathlineto{\pgfqpoint{3.520686in}{1.636652in}}%
\pgfpathlineto{\pgfqpoint{3.525699in}{1.639105in}}%
\pgfpathlineto{\pgfqpoint{3.528206in}{1.639381in}}%
\pgfpathlineto{\pgfqpoint{3.530713in}{1.642068in}}%
\pgfpathlineto{\pgfqpoint{3.535726in}{1.643639in}}%
\pgfpathlineto{\pgfqpoint{3.538233in}{1.646896in}}%
\pgfpathlineto{\pgfqpoint{3.553273in}{1.649833in}}%
\pgfpathlineto{\pgfqpoint{3.568313in}{1.652516in}}%
\pgfpathlineto{\pgfqpoint{3.583352in}{1.657560in}}%
\pgfpathlineto{\pgfqpoint{3.585859in}{1.661053in}}%
\pgfpathlineto{\pgfqpoint{3.600899in}{1.667083in}}%
\pgfpathlineto{\pgfqpoint{3.603406in}{1.670343in}}%
\pgfpathlineto{\pgfqpoint{3.605912in}{1.670528in}}%
\pgfpathlineto{\pgfqpoint{3.613432in}{1.675390in}}%
\pgfpathlineto{\pgfqpoint{3.615939in}{1.675793in}}%
\pgfpathlineto{\pgfqpoint{3.618446in}{1.678322in}}%
\pgfpathlineto{\pgfqpoint{3.623459in}{1.680684in}}%
\pgfpathlineto{\pgfqpoint{3.625966in}{1.683698in}}%
\pgfpathlineto{\pgfqpoint{3.635992in}{1.688621in}}%
\pgfpathlineto{\pgfqpoint{3.638499in}{1.694131in}}%
\pgfpathlineto{\pgfqpoint{3.648525in}{1.703167in}}%
\pgfpathlineto{\pgfqpoint{3.663565in}{1.711751in}}%
\pgfpathlineto{\pgfqpoint{3.668579in}{1.713876in}}%
\pgfpathlineto{\pgfqpoint{3.671085in}{1.713912in}}%
\pgfpathlineto{\pgfqpoint{3.673592in}{1.719857in}}%
\pgfpathlineto{\pgfqpoint{3.681112in}{1.724058in}}%
\pgfpathlineto{\pgfqpoint{3.686125in}{1.737299in}}%
\pgfpathlineto{\pgfqpoint{3.688632in}{1.739964in}}%
\pgfpathlineto{\pgfqpoint{3.691139in}{1.740639in}}%
\pgfpathlineto{\pgfqpoint{3.693645in}{1.743163in}}%
\pgfpathlineto{\pgfqpoint{3.708685in}{1.747308in}}%
\pgfpathlineto{\pgfqpoint{3.713698in}{1.750836in}}%
\pgfpathlineto{\pgfqpoint{3.723725in}{1.777662in}}%
\pgfpathlineto{\pgfqpoint{3.726232in}{1.782385in}}%
\pgfpathlineto{\pgfqpoint{3.728738in}{1.783056in}}%
\pgfpathlineto{\pgfqpoint{3.731245in}{1.785125in}}%
\pgfpathlineto{\pgfqpoint{3.733752in}{1.789466in}}%
\pgfpathlineto{\pgfqpoint{3.741272in}{1.794731in}}%
\pgfpathlineto{\pgfqpoint{3.743778in}{1.795043in}}%
\pgfpathlineto{\pgfqpoint{3.746285in}{1.797171in}}%
\pgfpathlineto{\pgfqpoint{3.753805in}{1.797897in}}%
\pgfpathlineto{\pgfqpoint{3.756312in}{1.799925in}}%
\pgfpathlineto{\pgfqpoint{3.758818in}{1.800402in}}%
\pgfpathlineto{\pgfqpoint{3.761325in}{1.805220in}}%
\pgfpathlineto{\pgfqpoint{3.763832in}{1.805535in}}%
\pgfpathlineto{\pgfqpoint{3.766338in}{1.821398in}}%
\pgfpathlineto{\pgfqpoint{3.768845in}{1.824075in}}%
\pgfpathlineto{\pgfqpoint{3.773858in}{1.825407in}}%
\pgfpathlineto{\pgfqpoint{3.776365in}{1.827525in}}%
\pgfpathlineto{\pgfqpoint{3.781378in}{1.828390in}}%
\pgfpathlineto{\pgfqpoint{3.783885in}{1.854178in}}%
\pgfpathlineto{\pgfqpoint{3.791405in}{1.858717in}}%
\pgfpathlineto{\pgfqpoint{3.793911in}{1.871526in}}%
\pgfpathlineto{\pgfqpoint{3.796418in}{1.874808in}}%
\pgfpathlineto{\pgfqpoint{3.798925in}{1.884745in}}%
\pgfpathlineto{\pgfqpoint{3.803938in}{1.888059in}}%
\pgfpathlineto{\pgfqpoint{3.808951in}{1.895510in}}%
\pgfpathlineto{\pgfqpoint{3.811458in}{1.900830in}}%
\pgfpathlineto{\pgfqpoint{3.813965in}{1.902734in}}%
\pgfpathlineto{\pgfqpoint{3.818978in}{1.921499in}}%
\pgfpathlineto{\pgfqpoint{3.821485in}{1.923093in}}%
\pgfpathlineto{\pgfqpoint{3.823991in}{1.928763in}}%
\pgfpathlineto{\pgfqpoint{3.829005in}{1.931715in}}%
\pgfpathlineto{\pgfqpoint{3.831511in}{1.953542in}}%
\pgfpathlineto{\pgfqpoint{3.834018in}{1.956788in}}%
\pgfpathlineto{\pgfqpoint{3.836525in}{1.957083in}}%
\pgfpathlineto{\pgfqpoint{3.839031in}{1.962842in}}%
\pgfpathlineto{\pgfqpoint{3.846551in}{1.966478in}}%
\pgfpathlineto{\pgfqpoint{3.849058in}{1.970825in}}%
\pgfpathlineto{\pgfqpoint{3.851564in}{1.970841in}}%
\pgfpathlineto{\pgfqpoint{3.854071in}{1.976078in}}%
\pgfpathlineto{\pgfqpoint{3.856578in}{1.977947in}}%
\pgfpathlineto{\pgfqpoint{3.859084in}{1.989374in}}%
\pgfpathlineto{\pgfqpoint{3.861591in}{2.010254in}}%
\pgfpathlineto{\pgfqpoint{3.864098in}{2.013925in}}%
\pgfpathlineto{\pgfqpoint{3.866604in}{2.023083in}}%
\pgfpathlineto{\pgfqpoint{3.869111in}{2.046014in}}%
\pgfpathlineto{\pgfqpoint{3.874124in}{2.047393in}}%
\pgfpathlineto{\pgfqpoint{3.876631in}{2.049583in}}%
\pgfpathlineto{\pgfqpoint{3.879138in}{2.053901in}}%
\pgfpathlineto{\pgfqpoint{3.881644in}{2.069689in}}%
\pgfpathlineto{\pgfqpoint{3.884151in}{2.076173in}}%
\pgfpathlineto{\pgfqpoint{3.889164in}{2.082825in}}%
\pgfpathlineto{\pgfqpoint{3.891671in}{2.083454in}}%
\pgfpathlineto{\pgfqpoint{3.894178in}{2.106418in}}%
\pgfpathlineto{\pgfqpoint{3.896684in}{2.110540in}}%
\pgfpathlineto{\pgfqpoint{3.899191in}{2.110593in}}%
\pgfpathlineto{\pgfqpoint{3.904204in}{2.122793in}}%
\pgfpathlineto{\pgfqpoint{3.906711in}{2.140562in}}%
\pgfpathlineto{\pgfqpoint{3.911724in}{2.156109in}}%
\pgfpathlineto{\pgfqpoint{3.914231in}{2.172619in}}%
\pgfpathlineto{\pgfqpoint{3.919244in}{2.186051in}}%
\pgfpathlineto{\pgfqpoint{3.924257in}{2.188850in}}%
\pgfpathlineto{\pgfqpoint{3.926764in}{2.194748in}}%
\pgfpathlineto{\pgfqpoint{3.931777in}{2.195947in}}%
\pgfpathlineto{\pgfqpoint{3.934284in}{2.199010in}}%
\pgfpathlineto{\pgfqpoint{3.936791in}{2.199952in}}%
\pgfpathlineto{\pgfqpoint{3.939297in}{2.204154in}}%
\pgfpathlineto{\pgfqpoint{3.941804in}{2.204411in}}%
\pgfpathlineto{\pgfqpoint{3.944311in}{2.207178in}}%
\pgfpathlineto{\pgfqpoint{3.946817in}{2.213206in}}%
\pgfpathlineto{\pgfqpoint{3.949324in}{2.215097in}}%
\pgfpathlineto{\pgfqpoint{3.951831in}{2.219594in}}%
\pgfpathlineto{\pgfqpoint{3.954337in}{2.236519in}}%
\pgfpathlineto{\pgfqpoint{3.961857in}{2.258924in}}%
\pgfpathlineto{\pgfqpoint{3.966871in}{2.260924in}}%
\pgfpathlineto{\pgfqpoint{3.969377in}{2.276156in}}%
\pgfpathlineto{\pgfqpoint{3.971884in}{2.277578in}}%
\pgfpathlineto{\pgfqpoint{3.974391in}{2.281259in}}%
\pgfpathlineto{\pgfqpoint{3.976897in}{2.282279in}}%
\pgfpathlineto{\pgfqpoint{3.979404in}{2.290093in}}%
\pgfpathlineto{\pgfqpoint{3.981910in}{2.291725in}}%
\pgfpathlineto{\pgfqpoint{3.984417in}{2.303897in}}%
\pgfpathlineto{\pgfqpoint{3.986924in}{2.330529in}}%
\pgfpathlineto{\pgfqpoint{3.991937in}{2.338517in}}%
\pgfpathlineto{\pgfqpoint{3.994444in}{2.344233in}}%
\pgfpathlineto{\pgfqpoint{3.996950in}{2.346438in}}%
\pgfpathlineto{\pgfqpoint{3.999457in}{2.355849in}}%
\pgfpathlineto{\pgfqpoint{4.001964in}{2.355896in}}%
\pgfpathlineto{\pgfqpoint{4.004470in}{2.357293in}}%
\pgfpathlineto{\pgfqpoint{4.006977in}{2.363744in}}%
\pgfpathlineto{\pgfqpoint{4.009484in}{2.377604in}}%
\pgfpathlineto{\pgfqpoint{4.011990in}{2.399197in}}%
\pgfpathlineto{\pgfqpoint{4.014497in}{2.400320in}}%
\pgfpathlineto{\pgfqpoint{4.017004in}{2.410002in}}%
\pgfpathlineto{\pgfqpoint{4.019510in}{2.412797in}}%
\pgfpathlineto{\pgfqpoint{4.027030in}{2.436834in}}%
\pgfpathlineto{\pgfqpoint{4.037057in}{2.479001in}}%
\pgfpathlineto{\pgfqpoint{4.039564in}{2.502938in}}%
\pgfpathlineto{\pgfqpoint{4.042070in}{2.510473in}}%
\pgfpathlineto{\pgfqpoint{4.044577in}{2.514017in}}%
\pgfpathlineto{\pgfqpoint{4.047083in}{2.521197in}}%
\pgfpathlineto{\pgfqpoint{4.049590in}{2.600417in}}%
\pgfpathlineto{\pgfqpoint{4.052097in}{2.630275in}}%
\pgfpathlineto{\pgfqpoint{4.052097in}{2.630275in}}%
\pgfusepath{stroke}%
\end{pgfscope}%
\begin{pgfscope}%
\pgfpathrectangle{\pgfqpoint{0.708220in}{0.535823in}}{\pgfqpoint{5.013309in}{2.094453in}}%
\pgfusepath{clip}%
\pgfsetbuttcap%
\pgfsetroundjoin%
\pgfsetlinewidth{1.003750pt}%
\definecolor{currentstroke}{rgb}{0.000000,0.000000,0.000000}%
\pgfsetstrokecolor{currentstroke}%
\pgfsetdash{{1.000000pt}{1.650000pt}}{0.000000pt}%
\pgfpathmoveto{\pgfqpoint{0.708220in}{0.636682in}}%
\pgfpathlineto{\pgfqpoint{0.728273in}{0.636682in}}%
\pgfpathlineto{\pgfqpoint{0.730780in}{0.654760in}}%
\pgfpathlineto{\pgfqpoint{0.748326in}{0.654760in}}%
\pgfpathlineto{\pgfqpoint{0.750833in}{0.671115in}}%
\pgfpathlineto{\pgfqpoint{0.818513in}{0.671115in}}%
\pgfpathlineto{\pgfqpoint{0.821019in}{0.686045in}}%
\pgfpathlineto{\pgfqpoint{0.938832in}{0.686045in}}%
\pgfpathlineto{\pgfqpoint{0.941339in}{0.699780in}}%
\pgfpathlineto{\pgfqpoint{1.139364in}{0.699780in}}%
\pgfpathlineto{\pgfqpoint{1.141871in}{0.712496in}}%
\pgfpathlineto{\pgfqpoint{1.372483in}{0.712496in}}%
\pgfpathlineto{\pgfqpoint{1.374990in}{0.724335in}}%
\pgfpathlineto{\pgfqpoint{1.583042in}{0.724335in}}%
\pgfpathlineto{\pgfqpoint{1.585549in}{0.735409in}}%
\pgfpathlineto{\pgfqpoint{1.793601in}{0.735409in}}%
\pgfpathlineto{\pgfqpoint{1.796108in}{0.745812in}}%
\pgfpathlineto{\pgfqpoint{1.936481in}{0.745812in}}%
\pgfpathlineto{\pgfqpoint{1.938987in}{0.755619in}}%
\pgfpathlineto{\pgfqpoint{2.061813in}{0.755619in}}%
\pgfpathlineto{\pgfqpoint{2.064320in}{0.764897in}}%
\pgfpathlineto{\pgfqpoint{2.154559in}{0.764897in}}%
\pgfpathlineto{\pgfqpoint{2.157066in}{0.773698in}}%
\pgfpathlineto{\pgfqpoint{2.219732in}{0.773698in}}%
\pgfpathlineto{\pgfqpoint{2.222239in}{0.782070in}}%
\pgfpathlineto{\pgfqpoint{2.299945in}{0.782070in}}%
\pgfpathlineto{\pgfqpoint{2.302452in}{0.790053in}}%
\pgfpathlineto{\pgfqpoint{2.390185in}{0.790053in}}%
\pgfpathlineto{\pgfqpoint{2.392692in}{0.797680in}}%
\pgfpathlineto{\pgfqpoint{2.435305in}{0.797680in}}%
\pgfpathlineto{\pgfqpoint{2.437811in}{0.804983in}}%
\pgfpathlineto{\pgfqpoint{2.467891in}{0.804983in}}%
\pgfpathlineto{\pgfqpoint{2.470398in}{0.811988in}}%
\pgfpathlineto{\pgfqpoint{2.495464in}{0.811988in}}%
\pgfpathlineto{\pgfqpoint{2.497971in}{0.818718in}}%
\pgfpathlineto{\pgfqpoint{2.530558in}{0.818718in}}%
\pgfpathlineto{\pgfqpoint{2.533064in}{0.825194in}}%
\pgfpathlineto{\pgfqpoint{2.553117in}{0.825194in}}%
\pgfpathlineto{\pgfqpoint{2.555624in}{0.831434in}}%
\pgfpathlineto{\pgfqpoint{2.565651in}{0.831434in}}%
\pgfpathlineto{\pgfqpoint{2.568157in}{0.837455in}}%
\pgfpathlineto{\pgfqpoint{2.593224in}{0.837455in}}%
\pgfpathlineto{\pgfqpoint{2.595731in}{0.843272in}}%
\pgfpathlineto{\pgfqpoint{2.615784in}{0.843272in}}%
\pgfpathlineto{\pgfqpoint{2.618290in}{0.848899in}}%
\pgfpathlineto{\pgfqpoint{2.625810in}{0.848899in}}%
\pgfpathlineto{\pgfqpoint{2.628317in}{0.854347in}}%
\pgfpathlineto{\pgfqpoint{2.640850in}{0.854347in}}%
\pgfpathlineto{\pgfqpoint{2.643357in}{0.859627in}}%
\pgfpathlineto{\pgfqpoint{2.645864in}{0.859627in}}%
\pgfpathlineto{\pgfqpoint{2.648370in}{0.864749in}}%
\pgfpathlineto{\pgfqpoint{2.668424in}{0.864749in}}%
\pgfpathlineto{\pgfqpoint{2.670930in}{0.869723in}}%
\pgfpathlineto{\pgfqpoint{2.680957in}{0.869723in}}%
\pgfpathlineto{\pgfqpoint{2.683464in}{0.874557in}}%
\pgfpathlineto{\pgfqpoint{2.695997in}{0.874557in}}%
\pgfpathlineto{\pgfqpoint{2.698503in}{0.879259in}}%
\pgfpathlineto{\pgfqpoint{2.713543in}{0.879259in}}%
\pgfpathlineto{\pgfqpoint{2.716050in}{0.883835in}}%
\pgfpathlineto{\pgfqpoint{2.733597in}{0.883835in}}%
\pgfpathlineto{\pgfqpoint{2.736103in}{0.888292in}}%
\pgfpathlineto{\pgfqpoint{2.743623in}{0.888292in}}%
\pgfpathlineto{\pgfqpoint{2.746130in}{0.892636in}}%
\pgfpathlineto{\pgfqpoint{2.753650in}{0.892636in}}%
\pgfpathlineto{\pgfqpoint{2.756156in}{0.896873in}}%
\pgfpathlineto{\pgfqpoint{2.771196in}{0.896873in}}%
\pgfpathlineto{\pgfqpoint{2.773703in}{0.901008in}}%
\pgfpathlineto{\pgfqpoint{2.776210in}{0.901008in}}%
\pgfpathlineto{\pgfqpoint{2.781223in}{0.908991in}}%
\pgfpathlineto{\pgfqpoint{2.788743in}{0.908991in}}%
\pgfpathlineto{\pgfqpoint{2.791250in}{0.912847in}}%
\pgfpathlineto{\pgfqpoint{2.801276in}{0.912847in}}%
\pgfpathlineto{\pgfqpoint{2.803783in}{0.916618in}}%
\pgfpathlineto{\pgfqpoint{2.808796in}{0.916618in}}%
\pgfpathlineto{\pgfqpoint{2.811303in}{0.920308in}}%
\pgfpathlineto{\pgfqpoint{2.813810in}{0.920308in}}%
\pgfpathlineto{\pgfqpoint{2.816316in}{0.923921in}}%
\pgfpathlineto{\pgfqpoint{2.826343in}{0.923921in}}%
\pgfpathlineto{\pgfqpoint{2.828849in}{0.927459in}}%
\pgfpathlineto{\pgfqpoint{2.836369in}{0.927459in}}%
\pgfpathlineto{\pgfqpoint{2.838876in}{0.934324in}}%
\pgfpathlineto{\pgfqpoint{2.848903in}{0.934324in}}%
\pgfpathlineto{\pgfqpoint{2.851409in}{0.937656in}}%
\pgfpathlineto{\pgfqpoint{2.861436in}{0.937656in}}%
\pgfpathlineto{\pgfqpoint{2.863943in}{0.940924in}}%
\pgfpathlineto{\pgfqpoint{2.868956in}{0.940924in}}%
\pgfpathlineto{\pgfqpoint{2.871463in}{0.944131in}}%
\pgfpathlineto{\pgfqpoint{2.886503in}{0.944131in}}%
\pgfpathlineto{\pgfqpoint{2.889009in}{0.947280in}}%
\pgfpathlineto{\pgfqpoint{2.891516in}{0.947280in}}%
\pgfpathlineto{\pgfqpoint{2.896529in}{0.953409in}}%
\pgfpathlineto{\pgfqpoint{2.899036in}{0.953409in}}%
\pgfpathlineto{\pgfqpoint{2.901542in}{0.956393in}}%
\pgfpathlineto{\pgfqpoint{2.906556in}{0.956393in}}%
\pgfpathlineto{\pgfqpoint{2.911569in}{0.962210in}}%
\pgfpathlineto{\pgfqpoint{2.919089in}{0.962210in}}%
\pgfpathlineto{\pgfqpoint{2.921596in}{0.965047in}}%
\pgfpathlineto{\pgfqpoint{2.924102in}{0.965047in}}%
\pgfpathlineto{\pgfqpoint{2.929116in}{0.970582in}}%
\pgfpathlineto{\pgfqpoint{2.936636in}{0.970582in}}%
\pgfpathlineto{\pgfqpoint{2.939142in}{0.973285in}}%
\pgfpathlineto{\pgfqpoint{2.941649in}{0.973285in}}%
\pgfpathlineto{\pgfqpoint{2.951676in}{0.983687in}}%
\pgfpathlineto{\pgfqpoint{2.954182in}{0.983687in}}%
\pgfpathlineto{\pgfqpoint{2.956689in}{0.986192in}}%
\pgfpathlineto{\pgfqpoint{2.959195in}{0.986192in}}%
\pgfpathlineto{\pgfqpoint{2.961702in}{0.988661in}}%
\pgfpathlineto{\pgfqpoint{2.964209in}{0.988661in}}%
\pgfpathlineto{\pgfqpoint{2.969222in}{0.993495in}}%
\pgfpathlineto{\pgfqpoint{2.971729in}{0.993495in}}%
\pgfpathlineto{\pgfqpoint{2.974235in}{0.995862in}}%
\pgfpathlineto{\pgfqpoint{2.979249in}{0.995862in}}%
\pgfpathlineto{\pgfqpoint{2.981755in}{0.998196in}}%
\pgfpathlineto{\pgfqpoint{2.986769in}{0.998196in}}%
\pgfpathlineto{\pgfqpoint{2.989275in}{1.002773in}}%
\pgfpathlineto{\pgfqpoint{2.991782in}{1.005016in}}%
\pgfpathlineto{\pgfqpoint{2.996795in}{1.005016in}}%
\pgfpathlineto{\pgfqpoint{3.001809in}{1.009416in}}%
\pgfpathlineto{\pgfqpoint{3.009329in}{1.009416in}}%
\pgfpathlineto{\pgfqpoint{3.011835in}{1.011574in}}%
\pgfpathlineto{\pgfqpoint{3.014342in}{1.011574in}}%
\pgfpathlineto{\pgfqpoint{3.016849in}{1.013706in}}%
\pgfpathlineto{\pgfqpoint{3.019355in}{1.013706in}}%
\pgfpathlineto{\pgfqpoint{3.029382in}{1.021977in}}%
\pgfpathlineto{\pgfqpoint{3.031888in}{1.021977in}}%
\pgfpathlineto{\pgfqpoint{3.039408in}{1.033681in}}%
\pgfpathlineto{\pgfqpoint{3.044422in}{1.033681in}}%
\pgfpathlineto{\pgfqpoint{3.056955in}{1.042859in}}%
\pgfpathlineto{\pgfqpoint{3.059462in}{1.042859in}}%
\pgfpathlineto{\pgfqpoint{3.061968in}{1.044637in}}%
\pgfpathlineto{\pgfqpoint{3.064475in}{1.044637in}}%
\pgfpathlineto{\pgfqpoint{3.066982in}{1.046397in}}%
\pgfpathlineto{\pgfqpoint{3.074502in}{1.046397in}}%
\pgfpathlineto{\pgfqpoint{3.077008in}{1.048139in}}%
\pgfpathlineto{\pgfqpoint{3.079515in}{1.048139in}}%
\pgfpathlineto{\pgfqpoint{3.082022in}{1.049863in}}%
\pgfpathlineto{\pgfqpoint{3.084528in}{1.049863in}}%
\pgfpathlineto{\pgfqpoint{3.089542in}{1.053261in}}%
\pgfpathlineto{\pgfqpoint{3.092048in}{1.053261in}}%
\pgfpathlineto{\pgfqpoint{3.097061in}{1.056593in}}%
\pgfpathlineto{\pgfqpoint{3.099568in}{1.056593in}}%
\pgfpathlineto{\pgfqpoint{3.102075in}{1.058235in}}%
\pgfpathlineto{\pgfqpoint{3.107088in}{1.058235in}}%
\pgfpathlineto{\pgfqpoint{3.109595in}{1.059862in}}%
\pgfpathlineto{\pgfqpoint{3.114608in}{1.059862in}}%
\pgfpathlineto{\pgfqpoint{3.122128in}{1.069310in}}%
\pgfpathlineto{\pgfqpoint{3.124635in}{1.070835in}}%
\pgfpathlineto{\pgfqpoint{3.127141in}{1.070835in}}%
\pgfpathlineto{\pgfqpoint{3.129648in}{1.073845in}}%
\pgfpathlineto{\pgfqpoint{3.132155in}{1.073845in}}%
\pgfpathlineto{\pgfqpoint{3.134661in}{1.075331in}}%
\pgfpathlineto{\pgfqpoint{3.139675in}{1.079712in}}%
\pgfpathlineto{\pgfqpoint{3.142181in}{1.079712in}}%
\pgfpathlineto{\pgfqpoint{3.144688in}{1.083984in}}%
\pgfpathlineto{\pgfqpoint{3.147195in}{1.085385in}}%
\pgfpathlineto{\pgfqpoint{3.152208in}{1.085385in}}%
\pgfpathlineto{\pgfqpoint{3.157221in}{1.089520in}}%
\pgfpathlineto{\pgfqpoint{3.159728in}{1.090877in}}%
\pgfpathlineto{\pgfqpoint{3.164741in}{1.090877in}}%
\pgfpathlineto{\pgfqpoint{3.167248in}{1.092222in}}%
\pgfpathlineto{\pgfqpoint{3.169754in}{1.092222in}}%
\pgfpathlineto{\pgfqpoint{3.172261in}{1.096198in}}%
\pgfpathlineto{\pgfqpoint{3.174768in}{1.096198in}}%
\pgfpathlineto{\pgfqpoint{3.187301in}{1.107599in}}%
\pgfpathlineto{\pgfqpoint{3.189808in}{1.107599in}}%
\pgfpathlineto{\pgfqpoint{3.194821in}{1.112433in}}%
\pgfpathlineto{\pgfqpoint{3.197328in}{1.117134in}}%
\pgfpathlineto{\pgfqpoint{3.202341in}{1.120578in}}%
\pgfpathlineto{\pgfqpoint{3.209861in}{1.132643in}}%
\pgfpathlineto{\pgfqpoint{3.212368in}{1.132643in}}%
\pgfpathlineto{\pgfqpoint{3.214874in}{1.137859in}}%
\pgfpathlineto{\pgfqpoint{3.217381in}{1.139902in}}%
\pgfpathlineto{\pgfqpoint{3.219888in}{1.139902in}}%
\pgfpathlineto{\pgfqpoint{3.232421in}{1.149766in}}%
\pgfpathlineto{\pgfqpoint{3.234927in}{1.153559in}}%
\pgfpathlineto{\pgfqpoint{3.237434in}{1.153559in}}%
\pgfpathlineto{\pgfqpoint{3.242447in}{1.158184in}}%
\pgfpathlineto{\pgfqpoint{3.244954in}{1.163575in}}%
\pgfpathlineto{\pgfqpoint{3.247461in}{1.163575in}}%
\pgfpathlineto{\pgfqpoint{3.249967in}{1.167941in}}%
\pgfpathlineto{\pgfqpoint{3.252474in}{1.167941in}}%
\pgfpathlineto{\pgfqpoint{3.259994in}{1.177173in}}%
\pgfpathlineto{\pgfqpoint{3.262501in}{1.182800in}}%
\pgfpathlineto{\pgfqpoint{3.267514in}{1.185934in}}%
\pgfpathlineto{\pgfqpoint{3.270021in}{1.189773in}}%
\pgfpathlineto{\pgfqpoint{3.275034in}{1.190530in}}%
\pgfpathlineto{\pgfqpoint{3.277541in}{1.190530in}}%
\pgfpathlineto{\pgfqpoint{3.280047in}{1.192035in}}%
\pgfpathlineto{\pgfqpoint{3.282554in}{1.195742in}}%
\pgfpathlineto{\pgfqpoint{3.285061in}{1.195742in}}%
\pgfpathlineto{\pgfqpoint{3.292581in}{1.199370in}}%
\pgfpathlineto{\pgfqpoint{3.295087in}{1.200086in}}%
\pgfpathlineto{\pgfqpoint{3.297594in}{1.206403in}}%
\pgfpathlineto{\pgfqpoint{3.310127in}{1.219660in}}%
\pgfpathlineto{\pgfqpoint{3.320154in}{1.222820in}}%
\pgfpathlineto{\pgfqpoint{3.322660in}{1.227149in}}%
\pgfpathlineto{\pgfqpoint{3.325167in}{1.228366in}}%
\pgfpathlineto{\pgfqpoint{3.327674in}{1.234324in}}%
\pgfpathlineto{\pgfqpoint{3.335194in}{1.236651in}}%
\pgfpathlineto{\pgfqpoint{3.337700in}{1.241773in}}%
\pgfpathlineto{\pgfqpoint{3.340207in}{1.243447in}}%
\pgfpathlineto{\pgfqpoint{3.350234in}{1.267260in}}%
\pgfpathlineto{\pgfqpoint{3.355247in}{1.268224in}}%
\pgfpathlineto{\pgfqpoint{3.360260in}{1.270136in}}%
\pgfpathlineto{\pgfqpoint{3.367780in}{1.274825in}}%
\pgfpathlineto{\pgfqpoint{3.372793in}{1.285146in}}%
\pgfpathlineto{\pgfqpoint{3.377807in}{1.286015in}}%
\pgfpathlineto{\pgfqpoint{3.382820in}{1.286447in}}%
\pgfpathlineto{\pgfqpoint{3.385327in}{1.296111in}}%
\pgfpathlineto{\pgfqpoint{3.395353in}{1.299349in}}%
\pgfpathlineto{\pgfqpoint{3.397860in}{1.306418in}}%
\pgfpathlineto{\pgfqpoint{3.402873in}{1.312465in}}%
\pgfpathlineto{\pgfqpoint{3.405380in}{1.322562in}}%
\pgfpathlineto{\pgfqpoint{3.407887in}{1.325341in}}%
\pgfpathlineto{\pgfqpoint{3.410393in}{1.332428in}}%
\pgfpathlineto{\pgfqpoint{3.412900in}{1.335378in}}%
\pgfpathlineto{\pgfqpoint{3.415407in}{1.335378in}}%
\pgfpathlineto{\pgfqpoint{3.417913in}{1.337638in}}%
\pgfpathlineto{\pgfqpoint{3.420420in}{1.337959in}}%
\pgfpathlineto{\pgfqpoint{3.422927in}{1.347303in}}%
\pgfpathlineto{\pgfqpoint{3.425433in}{1.351792in}}%
\pgfpathlineto{\pgfqpoint{3.432953in}{1.355300in}}%
\pgfpathlineto{\pgfqpoint{3.440473in}{1.358737in}}%
\pgfpathlineto{\pgfqpoint{3.447993in}{1.363492in}}%
\pgfpathlineto{\pgfqpoint{3.450500in}{1.367850in}}%
\pgfpathlineto{\pgfqpoint{3.453006in}{1.378285in}}%
\pgfpathlineto{\pgfqpoint{3.458020in}{1.378538in}}%
\pgfpathlineto{\pgfqpoint{3.460526in}{1.389311in}}%
\pgfpathlineto{\pgfqpoint{3.465540in}{1.390965in}}%
\pgfpathlineto{\pgfqpoint{3.468046in}{1.392835in}}%
\pgfpathlineto{\pgfqpoint{3.470553in}{1.393299in}}%
\pgfpathlineto{\pgfqpoint{3.473060in}{1.399225in}}%
\pgfpathlineto{\pgfqpoint{3.475566in}{1.399449in}}%
\pgfpathlineto{\pgfqpoint{3.480580in}{1.409020in}}%
\pgfpathlineto{\pgfqpoint{3.483086in}{1.422054in}}%
\pgfpathlineto{\pgfqpoint{3.485593in}{1.425932in}}%
\pgfpathlineto{\pgfqpoint{3.488100in}{1.426697in}}%
\pgfpathlineto{\pgfqpoint{3.495620in}{1.435259in}}%
\pgfpathlineto{\pgfqpoint{3.503139in}{1.436706in}}%
\pgfpathlineto{\pgfqpoint{3.510659in}{1.441325in}}%
\pgfpathlineto{\pgfqpoint{3.513166in}{1.453502in}}%
\pgfpathlineto{\pgfqpoint{3.515673in}{1.453665in}}%
\pgfpathlineto{\pgfqpoint{3.520686in}{1.462099in}}%
\pgfpathlineto{\pgfqpoint{3.533219in}{1.468799in}}%
\pgfpathlineto{\pgfqpoint{3.535726in}{1.471025in}}%
\pgfpathlineto{\pgfqpoint{3.538233in}{1.476537in}}%
\pgfpathlineto{\pgfqpoint{3.540739in}{1.477107in}}%
\pgfpathlineto{\pgfqpoint{3.543246in}{1.483393in}}%
\pgfpathlineto{\pgfqpoint{3.545753in}{1.495058in}}%
\pgfpathlineto{\pgfqpoint{3.550766in}{1.496334in}}%
\pgfpathlineto{\pgfqpoint{3.553273in}{1.499610in}}%
\pgfpathlineto{\pgfqpoint{3.555779in}{1.506220in}}%
\pgfpathlineto{\pgfqpoint{3.560793in}{1.509667in}}%
\pgfpathlineto{\pgfqpoint{3.573326in}{1.514197in}}%
\pgfpathlineto{\pgfqpoint{3.575832in}{1.523565in}}%
\pgfpathlineto{\pgfqpoint{3.578339in}{1.523890in}}%
\pgfpathlineto{\pgfqpoint{3.580846in}{1.527322in}}%
\pgfpathlineto{\pgfqpoint{3.583352in}{1.527640in}}%
\pgfpathlineto{\pgfqpoint{3.585859in}{1.530583in}}%
\pgfpathlineto{\pgfqpoint{3.588366in}{1.530791in}}%
\pgfpathlineto{\pgfqpoint{3.595886in}{1.534700in}}%
\pgfpathlineto{\pgfqpoint{3.598392in}{1.539909in}}%
\pgfpathlineto{\pgfqpoint{3.600899in}{1.540304in}}%
\pgfpathlineto{\pgfqpoint{3.608419in}{1.545444in}}%
\pgfpathlineto{\pgfqpoint{3.610926in}{1.546206in}}%
\pgfpathlineto{\pgfqpoint{3.613432in}{1.562180in}}%
\pgfpathlineto{\pgfqpoint{3.618446in}{1.562612in}}%
\pgfpathlineto{\pgfqpoint{3.620952in}{1.568476in}}%
\pgfpathlineto{\pgfqpoint{3.630979in}{1.570304in}}%
\pgfpathlineto{\pgfqpoint{3.633486in}{1.574388in}}%
\pgfpathlineto{\pgfqpoint{3.638499in}{1.576712in}}%
\pgfpathlineto{\pgfqpoint{3.641005in}{1.581192in}}%
\pgfpathlineto{\pgfqpoint{3.643512in}{1.581657in}}%
\pgfpathlineto{\pgfqpoint{3.648525in}{1.583657in}}%
\pgfpathlineto{\pgfqpoint{3.651032in}{1.591357in}}%
\pgfpathlineto{\pgfqpoint{3.656045in}{1.592958in}}%
\pgfpathlineto{\pgfqpoint{3.663565in}{1.597673in}}%
\pgfpathlineto{\pgfqpoint{3.666072in}{1.604376in}}%
\pgfpathlineto{\pgfqpoint{3.668579in}{1.604579in}}%
\pgfpathlineto{\pgfqpoint{3.671085in}{1.609253in}}%
\pgfpathlineto{\pgfqpoint{3.673592in}{1.617923in}}%
\pgfpathlineto{\pgfqpoint{3.676099in}{1.618922in}}%
\pgfpathlineto{\pgfqpoint{3.678605in}{1.622129in}}%
\pgfpathlineto{\pgfqpoint{3.688632in}{1.624376in}}%
\pgfpathlineto{\pgfqpoint{3.693645in}{1.631985in}}%
\pgfpathlineto{\pgfqpoint{3.698659in}{1.633593in}}%
\pgfpathlineto{\pgfqpoint{3.703672in}{1.645728in}}%
\pgfpathlineto{\pgfqpoint{3.721218in}{1.655660in}}%
\pgfpathlineto{\pgfqpoint{3.723725in}{1.658304in}}%
\pgfpathlineto{\pgfqpoint{3.728738in}{1.667224in}}%
\pgfpathlineto{\pgfqpoint{3.731245in}{1.670850in}}%
\pgfpathlineto{\pgfqpoint{3.733752in}{1.671034in}}%
\pgfpathlineto{\pgfqpoint{3.736258in}{1.673090in}}%
\pgfpathlineto{\pgfqpoint{3.738765in}{1.677660in}}%
\pgfpathlineto{\pgfqpoint{3.741272in}{1.679288in}}%
\pgfpathlineto{\pgfqpoint{3.743778in}{1.695650in}}%
\pgfpathlineto{\pgfqpoint{3.746285in}{1.699196in}}%
\pgfpathlineto{\pgfqpoint{3.751298in}{1.719268in}}%
\pgfpathlineto{\pgfqpoint{3.758818in}{1.721028in}}%
\pgfpathlineto{\pgfqpoint{3.771352in}{1.758374in}}%
\pgfpathlineto{\pgfqpoint{3.773858in}{1.759146in}}%
\pgfpathlineto{\pgfqpoint{3.776365in}{1.769186in}}%
\pgfpathlineto{\pgfqpoint{3.781378in}{1.771556in}}%
\pgfpathlineto{\pgfqpoint{3.783885in}{1.777464in}}%
\pgfpathlineto{\pgfqpoint{3.788898in}{1.777538in}}%
\pgfpathlineto{\pgfqpoint{3.791405in}{1.780672in}}%
\pgfpathlineto{\pgfqpoint{3.796418in}{1.782553in}}%
\pgfpathlineto{\pgfqpoint{3.798925in}{1.785243in}}%
\pgfpathlineto{\pgfqpoint{3.806445in}{1.785880in}}%
\pgfpathlineto{\pgfqpoint{3.813965in}{1.799121in}}%
\pgfpathlineto{\pgfqpoint{3.816471in}{1.805535in}}%
\pgfpathlineto{\pgfqpoint{3.818978in}{1.822105in}}%
\pgfpathlineto{\pgfqpoint{3.823991in}{1.823149in}}%
\pgfpathlineto{\pgfqpoint{3.826498in}{1.828702in}}%
\pgfpathlineto{\pgfqpoint{3.829005in}{1.830126in}}%
\pgfpathlineto{\pgfqpoint{3.831511in}{1.838154in}}%
\pgfpathlineto{\pgfqpoint{3.851564in}{1.849029in}}%
\pgfpathlineto{\pgfqpoint{3.854071in}{1.856221in}}%
\pgfpathlineto{\pgfqpoint{3.866604in}{1.858193in}}%
\pgfpathlineto{\pgfqpoint{3.869111in}{1.868470in}}%
\pgfpathlineto{\pgfqpoint{3.874124in}{1.874794in}}%
\pgfpathlineto{\pgfqpoint{3.876631in}{1.878304in}}%
\pgfpathlineto{\pgfqpoint{3.879138in}{1.878674in}}%
\pgfpathlineto{\pgfqpoint{3.884151in}{1.884467in}}%
\pgfpathlineto{\pgfqpoint{3.889164in}{1.887083in}}%
\pgfpathlineto{\pgfqpoint{3.891671in}{1.892717in}}%
\pgfpathlineto{\pgfqpoint{3.904204in}{1.904228in}}%
\pgfpathlineto{\pgfqpoint{3.906711in}{1.904464in}}%
\pgfpathlineto{\pgfqpoint{3.911724in}{1.909792in}}%
\pgfpathlineto{\pgfqpoint{3.919244in}{1.918798in}}%
\pgfpathlineto{\pgfqpoint{3.921751in}{1.923420in}}%
\pgfpathlineto{\pgfqpoint{3.924257in}{1.924095in}}%
\pgfpathlineto{\pgfqpoint{3.926764in}{1.926838in}}%
\pgfpathlineto{\pgfqpoint{3.929271in}{1.926849in}}%
\pgfpathlineto{\pgfqpoint{3.931777in}{1.931926in}}%
\pgfpathlineto{\pgfqpoint{3.936791in}{1.932026in}}%
\pgfpathlineto{\pgfqpoint{3.941804in}{1.934380in}}%
\pgfpathlineto{\pgfqpoint{3.946817in}{1.939107in}}%
\pgfpathlineto{\pgfqpoint{3.951831in}{1.940269in}}%
\pgfpathlineto{\pgfqpoint{3.956844in}{1.940613in}}%
\pgfpathlineto{\pgfqpoint{3.964364in}{1.947447in}}%
\pgfpathlineto{\pgfqpoint{3.971884in}{1.958724in}}%
\pgfpathlineto{\pgfqpoint{3.976897in}{1.960374in}}%
\pgfpathlineto{\pgfqpoint{3.979404in}{1.962909in}}%
\pgfpathlineto{\pgfqpoint{3.981910in}{1.963119in}}%
\pgfpathlineto{\pgfqpoint{3.984417in}{1.971145in}}%
\pgfpathlineto{\pgfqpoint{3.991937in}{1.976054in}}%
\pgfpathlineto{\pgfqpoint{3.994444in}{1.979439in}}%
\pgfpathlineto{\pgfqpoint{3.996950in}{1.979827in}}%
\pgfpathlineto{\pgfqpoint{3.999457in}{1.982758in}}%
\pgfpathlineto{\pgfqpoint{4.001964in}{1.990628in}}%
\pgfpathlineto{\pgfqpoint{4.006977in}{1.993662in}}%
\pgfpathlineto{\pgfqpoint{4.011990in}{2.007224in}}%
\pgfpathlineto{\pgfqpoint{4.014497in}{2.009355in}}%
\pgfpathlineto{\pgfqpoint{4.017004in}{2.013332in}}%
\pgfpathlineto{\pgfqpoint{4.024524in}{2.014317in}}%
\pgfpathlineto{\pgfqpoint{4.027030in}{2.015297in}}%
\pgfpathlineto{\pgfqpoint{4.029537in}{2.026654in}}%
\pgfpathlineto{\pgfqpoint{4.032044in}{2.032790in}}%
\pgfpathlineto{\pgfqpoint{4.042070in}{2.036424in}}%
\pgfpathlineto{\pgfqpoint{4.044577in}{2.043662in}}%
\pgfpathlineto{\pgfqpoint{4.047083in}{2.046458in}}%
\pgfpathlineto{\pgfqpoint{4.049590in}{2.046550in}}%
\pgfpathlineto{\pgfqpoint{4.054603in}{2.053886in}}%
\pgfpathlineto{\pgfqpoint{4.057110in}{2.054290in}}%
\pgfpathlineto{\pgfqpoint{4.062123in}{2.058300in}}%
\pgfpathlineto{\pgfqpoint{4.064630in}{2.060594in}}%
\pgfpathlineto{\pgfqpoint{4.067137in}{2.066289in}}%
\pgfpathlineto{\pgfqpoint{4.069643in}{2.068637in}}%
\pgfpathlineto{\pgfqpoint{4.074657in}{2.069513in}}%
\pgfpathlineto{\pgfqpoint{4.077163in}{2.072431in}}%
\pgfpathlineto{\pgfqpoint{4.079670in}{2.078976in}}%
\pgfpathlineto{\pgfqpoint{4.082177in}{2.081179in}}%
\pgfpathlineto{\pgfqpoint{4.089697in}{2.082792in}}%
\pgfpathlineto{\pgfqpoint{4.099723in}{2.108671in}}%
\pgfpathlineto{\pgfqpoint{4.102230in}{2.112925in}}%
\pgfpathlineto{\pgfqpoint{4.104737in}{2.124476in}}%
\pgfpathlineto{\pgfqpoint{4.109750in}{2.127324in}}%
\pgfpathlineto{\pgfqpoint{4.112257in}{2.133590in}}%
\pgfpathlineto{\pgfqpoint{4.114763in}{2.136640in}}%
\pgfpathlineto{\pgfqpoint{4.117270in}{2.143424in}}%
\pgfpathlineto{\pgfqpoint{4.119776in}{2.143544in}}%
\pgfpathlineto{\pgfqpoint{4.122283in}{2.147181in}}%
\pgfpathlineto{\pgfqpoint{4.124790in}{2.148543in}}%
\pgfpathlineto{\pgfqpoint{4.134816in}{2.158056in}}%
\pgfpathlineto{\pgfqpoint{4.137323in}{2.158658in}}%
\pgfpathlineto{\pgfqpoint{4.147350in}{2.188926in}}%
\pgfpathlineto{\pgfqpoint{4.152363in}{2.194239in}}%
\pgfpathlineto{\pgfqpoint{4.154870in}{2.195184in}}%
\pgfpathlineto{\pgfqpoint{4.162390in}{2.207920in}}%
\pgfpathlineto{\pgfqpoint{4.164896in}{2.211611in}}%
\pgfpathlineto{\pgfqpoint{4.167403in}{2.213075in}}%
\pgfpathlineto{\pgfqpoint{4.187456in}{2.288049in}}%
\pgfpathlineto{\pgfqpoint{4.189963in}{2.307310in}}%
\pgfpathlineto{\pgfqpoint{4.192469in}{2.315110in}}%
\pgfpathlineto{\pgfqpoint{4.194976in}{2.332196in}}%
\pgfpathlineto{\pgfqpoint{4.197483in}{2.333250in}}%
\pgfpathlineto{\pgfqpoint{4.199989in}{2.340201in}}%
\pgfpathlineto{\pgfqpoint{4.205003in}{2.340907in}}%
\pgfpathlineto{\pgfqpoint{4.210016in}{2.374247in}}%
\pgfpathlineto{\pgfqpoint{4.212523in}{2.375129in}}%
\pgfpathlineto{\pgfqpoint{4.215029in}{2.380919in}}%
\pgfpathlineto{\pgfqpoint{4.222549in}{2.413309in}}%
\pgfpathlineto{\pgfqpoint{4.225056in}{2.414577in}}%
\pgfpathlineto{\pgfqpoint{4.227563in}{2.417698in}}%
\pgfpathlineto{\pgfqpoint{4.230069in}{2.424475in}}%
\pgfpathlineto{\pgfqpoint{4.232576in}{2.425510in}}%
\pgfpathlineto{\pgfqpoint{4.235083in}{2.430320in}}%
\pgfpathlineto{\pgfqpoint{4.237589in}{2.431403in}}%
\pgfpathlineto{\pgfqpoint{4.247616in}{2.463316in}}%
\pgfpathlineto{\pgfqpoint{4.257642in}{2.478641in}}%
\pgfpathlineto{\pgfqpoint{4.260149in}{2.479314in}}%
\pgfpathlineto{\pgfqpoint{4.262656in}{2.486822in}}%
\pgfpathlineto{\pgfqpoint{4.265162in}{2.488799in}}%
\pgfpathlineto{\pgfqpoint{4.272682in}{2.517498in}}%
\pgfpathlineto{\pgfqpoint{4.275189in}{2.535881in}}%
\pgfpathlineto{\pgfqpoint{4.280202in}{2.592450in}}%
\pgfpathlineto{\pgfqpoint{4.282709in}{2.618042in}}%
\pgfpathlineto{\pgfqpoint{4.285216in}{2.630275in}}%
\pgfpathlineto{\pgfqpoint{4.285216in}{2.630275in}}%
\pgfusepath{stroke}%
\end{pgfscope}%
\begin{pgfscope}%
\pgfsetrectcap%
\pgfsetmiterjoin%
\pgfsetlinewidth{0.803000pt}%
\definecolor{currentstroke}{rgb}{0.000000,0.000000,0.000000}%
\pgfsetstrokecolor{currentstroke}%
\pgfsetdash{}{0pt}%
\pgfpathmoveto{\pgfqpoint{0.708220in}{0.535823in}}%
\pgfpathlineto{\pgfqpoint{0.708220in}{2.630275in}}%
\pgfusepath{stroke}%
\end{pgfscope}%
\begin{pgfscope}%
\pgfsetrectcap%
\pgfsetmiterjoin%
\pgfsetlinewidth{0.803000pt}%
\definecolor{currentstroke}{rgb}{0.000000,0.000000,0.000000}%
\pgfsetstrokecolor{currentstroke}%
\pgfsetdash{}{0pt}%
\pgfpathmoveto{\pgfqpoint{5.721529in}{0.535823in}}%
\pgfpathlineto{\pgfqpoint{5.721529in}{2.630275in}}%
\pgfusepath{stroke}%
\end{pgfscope}%
\begin{pgfscope}%
\pgfsetrectcap%
\pgfsetmiterjoin%
\pgfsetlinewidth{0.803000pt}%
\definecolor{currentstroke}{rgb}{0.000000,0.000000,0.000000}%
\pgfsetstrokecolor{currentstroke}%
\pgfsetdash{}{0pt}%
\pgfpathmoveto{\pgfqpoint{0.708220in}{0.535823in}}%
\pgfpathlineto{\pgfqpoint{5.721529in}{0.535823in}}%
\pgfusepath{stroke}%
\end{pgfscope}%
\begin{pgfscope}%
\pgfsetrectcap%
\pgfsetmiterjoin%
\pgfsetlinewidth{0.803000pt}%
\definecolor{currentstroke}{rgb}{0.000000,0.000000,0.000000}%
\pgfsetstrokecolor{currentstroke}%
\pgfsetdash{}{0pt}%
\pgfpathmoveto{\pgfqpoint{0.708220in}{2.630275in}}%
\pgfpathlineto{\pgfqpoint{5.721529in}{2.630275in}}%
\pgfusepath{stroke}%
\end{pgfscope}%
\begin{pgfscope}%
\pgfsetrectcap%
\pgfsetroundjoin%
\pgfsetlinewidth{1.003750pt}%
\definecolor{currentstroke}{rgb}{0.878431,0.878431,0.815686}%
\pgfsetstrokecolor{currentstroke}%
\pgfsetdash{}{0pt}%
\pgfpathmoveto{\pgfqpoint{4.804312in}{2.110068in}}%
\pgfpathlineto{\pgfqpoint{5.054312in}{2.110068in}}%
\pgfusepath{stroke}%
\end{pgfscope}%
\begin{pgfscope}%
\definecolor{textcolor}{rgb}{0.000000,0.000000,0.000000}%
\pgfsetstrokecolor{textcolor}%
\pgfsetfillcolor{textcolor}%
\pgftext[x=5.079312in,y=2.066318in,left,base]{\color{textcolor}\rmfamily\fontsize{9.000000}{10.800000}\selectfont T.+CPU1}%
\end{pgfscope}%
\begin{pgfscope}%
\pgfsetrectcap%
\pgfsetroundjoin%
\pgfsetlinewidth{1.003750pt}%
\definecolor{currentstroke}{rgb}{0.564706,0.564706,1.000000}%
\pgfsetstrokecolor{currentstroke}%
\pgfsetdash{}{0pt}%
\pgfpathmoveto{\pgfqpoint{4.804312in}{1.948269in}}%
\pgfpathlineto{\pgfqpoint{5.054312in}{1.948269in}}%
\pgfusepath{stroke}%
\end{pgfscope}%
\begin{pgfscope}%
\definecolor{textcolor}{rgb}{0.000000,0.000000,0.000000}%
\pgfsetstrokecolor{textcolor}%
\pgfsetfillcolor{textcolor}%
\pgftext[x=5.079312in,y=1.904519in,left,base]{\color{textcolor}\rmfamily\fontsize{9.000000}{10.800000}\selectfont P4+CPU1}%
\end{pgfscope}%
\begin{pgfscope}%
\pgfsetbuttcap%
\pgfsetroundjoin%
\pgfsetlinewidth{1.003750pt}%
\definecolor{currentstroke}{rgb}{0.564706,0.564706,1.000000}%
\pgfsetstrokecolor{currentstroke}%
\pgfsetdash{{1.000000pt}{1.650000pt}}{0.000000pt}%
\pgfpathmoveto{\pgfqpoint{4.804312in}{1.786469in}}%
\pgfpathlineto{\pgfqpoint{5.054312in}{1.786469in}}%
\pgfusepath{stroke}%
\end{pgfscope}%
\begin{pgfscope}%
\definecolor{textcolor}{rgb}{0.000000,0.000000,0.000000}%
\pgfsetstrokecolor{textcolor}%
\pgfsetfillcolor{textcolor}%
\pgftext[x=5.079312in,y=1.742719in,left,base]{\color{textcolor}\rmfamily\fontsize{9.000000}{10.800000}\selectfont P4+CPU8}%
\end{pgfscope}%
\begin{pgfscope}%
\pgfsetbuttcap%
\pgfsetroundjoin%
\pgfsetlinewidth{1.003750pt}%
\definecolor{currentstroke}{rgb}{0.564706,0.564706,1.000000}%
\pgfsetstrokecolor{currentstroke}%
\pgfsetdash{{3.700000pt}{1.600000pt}}{0.000000pt}%
\pgfpathmoveto{\pgfqpoint{4.804312in}{1.624670in}}%
\pgfpathlineto{\pgfqpoint{5.054312in}{1.624670in}}%
\pgfusepath{stroke}%
\end{pgfscope}%
\begin{pgfscope}%
\definecolor{textcolor}{rgb}{0.000000,0.000000,0.000000}%
\pgfsetstrokecolor{textcolor}%
\pgfsetfillcolor{textcolor}%
\pgftext[x=5.079312in,y=1.580920in,left,base]{\color{textcolor}\rmfamily\fontsize{9.000000}{10.800000}\selectfont P4+GPU}%
\end{pgfscope}%
\begin{pgfscope}%
\pgfsetrectcap%
\pgfsetroundjoin%
\pgfsetlinewidth{1.003750pt}%
\definecolor{currentstroke}{rgb}{0.811765,0.125490,0.125490}%
\pgfsetstrokecolor{currentstroke}%
\pgfsetdash{}{0pt}%
\pgfpathmoveto{\pgfqpoint{4.804312in}{1.462870in}}%
\pgfpathlineto{\pgfqpoint{5.054312in}{1.462870in}}%
\pgfusepath{stroke}%
\end{pgfscope}%
\begin{pgfscope}%
\definecolor{textcolor}{rgb}{0.000000,0.000000,0.000000}%
\pgfsetstrokecolor{textcolor}%
\pgfsetfillcolor{textcolor}%
\pgftext[x=5.079312in,y=1.419120in,left,base]{\color{textcolor}\rmfamily\fontsize{9.000000}{10.800000}\selectfont miniC2D}%
\end{pgfscope}%
\begin{pgfscope}%
\pgfsetbuttcap%
\pgfsetroundjoin%
\pgfsetlinewidth{1.003750pt}%
\definecolor{currentstroke}{rgb}{0.811765,0.125490,0.125490}%
\pgfsetstrokecolor{currentstroke}%
\pgfsetdash{{3.700000pt}{1.600000pt}}{0.000000pt}%
\pgfpathmoveto{\pgfqpoint{4.804312in}{1.301071in}}%
\pgfpathlineto{\pgfqpoint{5.054312in}{1.301071in}}%
\pgfusepath{stroke}%
\end{pgfscope}%
\begin{pgfscope}%
\definecolor{textcolor}{rgb}{0.000000,0.000000,0.000000}%
\pgfsetstrokecolor{textcolor}%
\pgfsetfillcolor{textcolor}%
\pgftext[x=5.079312in,y=1.257321in,left,base]{\color{textcolor}\rmfamily\fontsize{9.000000}{10.800000}\selectfont d4}%
\end{pgfscope}%
\begin{pgfscope}%
\pgfsetbuttcap%
\pgfsetroundjoin%
\pgfsetlinewidth{1.003750pt}%
\definecolor{currentstroke}{rgb}{0.811765,0.125490,0.125490}%
\pgfsetstrokecolor{currentstroke}%
\pgfsetdash{{1.000000pt}{1.650000pt}}{0.000000pt}%
\pgfpathmoveto{\pgfqpoint{4.804312in}{1.139271in}}%
\pgfpathlineto{\pgfqpoint{5.054312in}{1.139271in}}%
\pgfusepath{stroke}%
\end{pgfscope}%
\begin{pgfscope}%
\definecolor{textcolor}{rgb}{0.000000,0.000000,0.000000}%
\pgfsetstrokecolor{textcolor}%
\pgfsetfillcolor{textcolor}%
\pgftext[x=5.079312in,y=1.095521in,left,base]{\color{textcolor}\rmfamily\fontsize{9.000000}{10.800000}\selectfont cachet}%
\end{pgfscope}%
\begin{pgfscope}%
\pgfsetrectcap%
\pgfsetroundjoin%
\pgfsetlinewidth{1.003750pt}%
\definecolor{currentstroke}{rgb}{0.062745,0.000000,0.062745}%
\pgfsetstrokecolor{currentstroke}%
\pgfsetdash{}{0pt}%
\pgfpathmoveto{\pgfqpoint{4.804312in}{0.977471in}}%
\pgfpathlineto{\pgfqpoint{5.054312in}{0.977471in}}%
\pgfusepath{stroke}%
\end{pgfscope}%
\begin{pgfscope}%
\definecolor{textcolor}{rgb}{0.000000,0.000000,0.000000}%
\pgfsetstrokecolor{textcolor}%
\pgfsetfillcolor{textcolor}%
\pgftext[x=5.079312in,y=0.933721in,left,base]{\color{textcolor}\rmfamily\fontsize{9.000000}{10.800000}\selectfont ADDMC}%
\end{pgfscope}%
\begin{pgfscope}%
\pgfsetbuttcap%
\pgfsetroundjoin%
\pgfsetlinewidth{1.003750pt}%
\definecolor{currentstroke}{rgb}{0.000000,0.000000,0.000000}%
\pgfsetstrokecolor{currentstroke}%
\pgfsetdash{{3.700000pt}{1.600000pt}}{0.000000pt}%
\pgfpathmoveto{\pgfqpoint{4.804312in}{0.815672in}}%
\pgfpathlineto{\pgfqpoint{5.054312in}{0.815672in}}%
\pgfusepath{stroke}%
\end{pgfscope}%
\begin{pgfscope}%
\definecolor{textcolor}{rgb}{0.000000,0.000000,0.000000}%
\pgfsetstrokecolor{textcolor}%
\pgfsetfillcolor{textcolor}%
\pgftext[x=5.079312in,y=0.771922in,left,base]{\color{textcolor}\rmfamily\fontsize{9.000000}{10.800000}\selectfont gpusat2}%
\end{pgfscope}%
\begin{pgfscope}%
\pgfsetbuttcap%
\pgfsetroundjoin%
\pgfsetlinewidth{1.003750pt}%
\definecolor{currentstroke}{rgb}{0.000000,0.000000,0.000000}%
\pgfsetstrokecolor{currentstroke}%
\pgfsetdash{{1.000000pt}{1.650000pt}}{0.000000pt}%
\pgfpathmoveto{\pgfqpoint{4.804312in}{0.653872in}}%
\pgfpathlineto{\pgfqpoint{5.054312in}{0.653872in}}%
\pgfusepath{stroke}%
\end{pgfscope}%
\begin{pgfscope}%
\definecolor{textcolor}{rgb}{0.000000,0.000000,0.000000}%
\pgfsetstrokecolor{textcolor}%
\pgfsetfillcolor{textcolor}%
\pgftext[x=5.079312in,y=0.610122in,left,base]{\color{textcolor}\rmfamily\fontsize{9.000000}{10.800000}\selectfont DPMC}%
\end{pgfscope}%
\end{pgfpicture}%
\makeatother%
\endgroup%

%\input{figures/comparison_pmc_eq.pgf}
\vspace*{-0.5cm}
\caption{\label{fig:parallel:comparison} A cactus plot of the number of benchmarks solved by various counters, without (above) and with (below) the \tool{pmc-eq}  preprocessor.}
\end{center}
\vspace*{-0.8cm}
\end{figure}

We observe that the performance of \tool{TensorOrder2} is improved by the portfolio planner and, on hard benchmarks, by executing on a multi-core CPU and on a GPU. The flat line at 3 seconds for $\pkg{P4}+\pkg{GPU}$ is caused by overhead from initializing the GPU.


% Moreover, by removing the benchmarks where slicing was needed (see the line labeled ``\pkg{P4}+\pkg{GPU} no slicing''), we see that index slicing significantly boosts the \pkg{GPU} executor. 

Comparing \tool{TensorOrder2} with the other counters, \tool{TensorOrder2} is competitive without preprocessing but solves fewer benchmarks than all other counters, although \tool{TensorOrder2} (with some configuration) is faster than all other counters on 158 benchmarks before preprocessing. 
We observe that preprocessing significantly boosts \tool{TensorOrder2} relative to other counters, similar to prior observations with \tool{gpusat2} \cite{FHZ19}. \tool{TensorOrder2} solves the third-most preprocessed benchmarks of any solver and has the second-lowest PAR-2 score (notably, outperforming \tool{gpusat2} in both measures). \tool{TensorOrder2} (with some configuration) is faster than all other counters on 200 benchmarks with preprocessing. Since \tool{TensorOrder2} improves the virtual best solver on 158 benchmarks without preprocessing and on 200 benchmarks with preprocessing, we conclude that \tool{TensorOrder2} is useful as part of a portfolio of counters.

\begin{table}[t]
  \caption{\label{tab:comparison} The numbers of benchmarks solved by each counter fastest and in total after 1000 seconds, and the PAR-2 score.}
  \centering
  \begin{tabular}{l||r|r|r||r|r|r|}
  & \multicolumn{3}{c||}{Without preprocessing} & \multicolumn{3}{c|}{With \tool{pmc-eq} preprocessing} \\
 & \# Fastest & \# Solved & PAR-2 & \# Fastest & \# Solved & PAR-2\\ \hline 
\pkg{T.}+\pkg{CPU1} & 0 & 1151 & 1640803. & 0 & 1514 & 834301.\\ 
\pkg{P4}+\pkg{CPU1} & 45 & 1164 & 1562474. & 83 & 1526 & 805547.\\ 
\pkg{P4}+\pkg{CPU8} & 50 & 1185 & 1500968. & 67 & 1542 & 771821.\\ 
\pkg{P4}+\pkg{GPU} & 63 & 1210 & 1436949. & 50 & 1549 & 745659.\\ \hline 
\tool{miniC2D} & 50 & 1381 & 1131457. & 221 & 1643 & 585908.\\ 
\tool{d4} & 615 & 1508 & 883829. & 550 & 1575 & 747318.\\ 
\tool{cachet} & 264 & 1363 & 1156309. & 221 & 1391 & 1099003.\\ 
\tool{ADDMC} & 640 & 1415 & 1032903. & 491 & 1436 & 1008326.\\  
\tool{gpusat2} & 37 & 1258 & 1342646. & 25 & 1497 & 854828.\\ \hline 
\end{tabular}
\end{table}