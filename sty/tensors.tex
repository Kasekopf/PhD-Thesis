% Operations on Dimensions
\newcommand{\domain}[1]{[#1]}				% Get the domain of a dimension or set of dimensions
\newcommand{\dlabel}[1]{\ell(#1)} 			% Get the label attached to each dimension

% Operations on Assignments
\newcommand{\restrict}[1]{\big|_{#1}}		% Given an assignment to E, and D \subseteq E, get the corresponding assignment to D

% Operations on Tensors
\newcommand{\tdim}[1]{\mathcal{I}(#1)}		% Get the set of indices of a tensor
\newcommand{\rank}[1]{rank(#1)}				% Get the rank of a tensor
	
\newcommand{\contract}[0]{\otimes}
\newcommand{\bigcontract}[0]{\bigotimes}

% Operations on Tensor Networks
\newcommand{\tntensor}[1]{\mathcal{T}(#1)}	% Get the tensor identified by the network
\newcommand{\tnfree}[1]{\mathcal{F}(#1)}	% Get the set of free dimensions
\newcommand{\tnbound}[1]{\mathcal{B}(#1)}	% Get the set of bound dimensions
\newcommand{\struct}[1]{\text{struct}(#1)}	% Get the structure graph

% Operations on Graphs
\newcommand{\edge}[1]{\{#1\}}				% Construct an edge incident to the given vertices
\newcommand{\vinc}[2]{\delta_{#1}(#2)}		% Get all edges incident to a vertex
\newcommand{\vincf}[1]{\delta_{#1}}		    % Get the function that returns all edges incident to a vertex
\newcommand{\vincs}[2]{\delta_{#1}[#2]}		% Get all edges incident to a set of vertices
\newcommand{\einc}[2]{\epsilon_{#1}(#2)}	% Get all vertices incident to an edge 
\newcommand{\eincf}[1]{\epsilon_{#1}}	    % Get the function that returns all vertices incident to an edge
\newcommand{\eincs}[2]{\epsilon_{#1}[#2]}	% Get all vertices incident to a set of edges
\newcommand{\E}[1]{\mathcal{E}(#1)}			% Get the set of edges of a graph
\newcommand{\V}[1]{\mathcal{V}(#1)}			% Get the set of vertices of a graph
\newcommand{\Lv}[1]{\mathcal{L}(#1)}		% Get the set of leafs of a tree
\newcommand{\C}[3]{\mathcal{C}_{#1,#2}(#3)}	% Get the children of a node of a rooted tree
\newcommand{\paritions}[1]{\mathcal{P}(#1)}
\newcommand{\Line}[1]{Line(#1)}

\newcommand{\blocks}[1]{\mathcal{B}(#1)}
\newcommand{\Ind}[0]{\mathbf{Ind}}

\newcommand{\fv}[0]{z}
\newcommand{\copyt}[0]{\text{COPY}}

\newcommand{\support}[1]{\vars(#1)}
\newcommand{\depend}[1]{\text{dep}(#1)}
\newcommand{\shortcite}[1]{\cite{#1}} % In this bib style, no need for shortcite