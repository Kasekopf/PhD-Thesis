\section{Related Work}

Our proposed graded project-join trees can be seen as a specialization of \emph{structure trees} \cite{stearns2002exploiting} to the case of projected model counting. Sterns and Hunt \cite{stearns2002exploiting} suggest constructing structure trees by manually modifying tree decomposers to consider only structure trees respecting the variable quantification order (i.e., to consider gradedness directly). In this work, we take a different approach by using existing tools for standard project-join trees (in particular, tree decomposers) in a black-box way. This is crucial for the practical success of our tool, as we can leverage continual progress in tree decomposition. 

It is worth comparing our theoretical results to a different algorithm for (unweighted) projected counting \cite{fichte2018exploiting}, which runs on a formula $\varphi$ in time $2^{2^{O(k)}} \cdot p(\varphi)$, where $p$ scales polynomially in the size of $\varphi$ and \emph{primal treewidth} \cite{SS10} $k$ of $\varphi$. Assuming the Exponential-Time Hypothesis \cite{impagliazzo2001problems}, all FPT algorithms parameterized by primal treewidth must be double-exponential \cite{fichte2018exploiting}. 
On the other hand, by Theorem \ref{thm_td_to_join} and Theorem \ref{thm:graded_from_virtual} our algorithm, based on graded project-join trees, runs in time $2^{O(k')}$, where $k'$ is the primal treewidth of $\psi$ (which we call the $\{X,Y\}$-graded treewidth of $\phi$). While $k'$ is larger than $k$, on many problems $k'$ is significantly smaller than $2^k$. 

Projected model counting is also a special case of the \emph{functional aggregate query (FAQ)} problem \cite{KNR16}. Our graded project-join trees are a special case of FAQ variable orders. \textbf{Theorem 7.5} of \cite{KNR16} gives an algorithm for constructing a FAQ variable orders from a sequence of tree decompositions, which, in the context of projected model counting, is equivalent to the technique we discussed in Section \ref{sec:planning:free} of ignoring relevant variables while planning to project irrelevant variables. In contrast, our approach can potentially find lower-width graded project-join trees by incorporating relevant variables even when planning to project irrelevant variables. It may be possible to lift this improvement to the FAQ framework in future work.

In some sense, projected model counting on propositional formulas is a dual problem of \emph{maximum a posteriori hypothesis (MAP)} \cite{murphy2012machine,maua2015complexity,xue2016solving} on Bayesian networks \cite{pearl1985bayesian}: a projected model count has the form $\sum_X \max_Y f(X, Y)$, while a MAP probability has the form $\max_Y \sum_X f(X, Y)$.
% The decision version of MAP is NP\textsuperscript{PP}-complete \cite{park2004complexity}.
Both problems can be solved using variable elimination, but an elimination order may not freely interleave $X$ variables with $Y$ variables.
A valid variable order induces an \emph{evaluation tree} (similar to a project-join tree%
% in which each internal node projects out exactly one variable%
) \cite{park2004complexity}.
% The time complexity of exact algorithms is exponential in the width of a valid variable order.
As mentioned in \cite{park2004complexity}, exact MAP algorithms construct evaluation trees using constraint-satisfaction heuristics (similar to our planner \htb).
Our work goes further by constructing low-width graded project-join trees using tree-decomposition techniques (with our planner \Lg) and by performing efficient computations using compact ADDs (with our executor \Dmc).
% To adapt our framework for MAP, we can simply swap $\Sigma$-variables with $\exists$-variables. 
