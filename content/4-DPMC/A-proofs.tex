\section{Proofs}
\label{sec_proofs}

%%%%%%%%%%%%%%%%%%%%%%%%%%%%%%%%%%%%%%%%%%%%%%%%%%%%%%%%%%%%%%%%%%%%%%%%%%%%%%%%

\subsection{Proof of Theorem \ref{thm_csp_jt}}

In this section, we assume the antecedents of Theorem \ref{thm_csp_jt} regarding the functions $\clusterVarOrder$, $\clauseRank$, and $\chosenCluster$.
Notice that for each $i = 1, 2, \ldots, m$ in Algorithm \ref{alg_csp_jt}, we have the following:
\begin{itemize}
    \item $\Gamma_i$ is a set of clauses
    \item $\kappa_i$ is a set of nodes that includes leaves $l$ \st{} $\gamma(l) \in \Gamma_i$
    \item $n_i$ is an internal node
    \item $n_i$'s children include the leaves in $\kappa_i$
    \item $\pi(n_i) = X_i$
\end{itemize}

We show that the first property in Definition \ref{def_jointree} holds:
\begin{lemma}[Property \ref{prop1}]
\label{lemma_prop1}
    The set $\set{\pi(n) : n \in \V T \setminus \Lv T}$ is a partition of $X$.
\end{lemma}
\begin{proof}
    For each $i = 1, 2, \ldots, m$, Algorithm \ref{alg_csp_jt} constructs an internal nodes $n_i$ with $\pi(n_i) = X_i$.
    Recall that $\set{X_i}_{i = 1}^m$ is a partition of $X$.
    Then $\set{\pi(n_i)}_{i = 1}^m$ is the same partition of $X$.
\qed
\end{proof}

We show that the second property in Definition \ref{def_jointree} holds through the following lemmas.

\begin{lemma}
\label{lemma_unprojected}
    Let $1 \le p < q \le m$.
    Assume some $x \in \vars(\Gamma_p) \cap X_q$.
    Then $x \in \vars(n_p)$.
\end{lemma}
\begin{proof}
    Notice $x \in X_q = \pi(n_q)$.
    Then $x$ is projected in $n_q$.
    Since $p < q$, we know $x$ is projected in neither $n_p$ nor a descendants of $n_p$.
    Since $x \in \vars(\Gamma_p)$, we know $x$ remains in $n_p$.
\qed
\end{proof}

\begin{lemma}
\label{lemma_internal_descedant}
    Let $1 \le p_0 < q \le m$.
    Assume $\vars(\Gamma_{p_0}) \cap X_q \ne \emptyset$.
    Then the internal node $n_{p_0}$ is a descendant of the node $n_q$.
\end{lemma}
\begin{proof}
    Let $n_{p_1}, n_{p_2}, \ldots$ be the parent, grandparent,\ldots{} of $n_{p_0}$.
    By way of contradiction, assume every $p_i \ne q$.
    Let $x$ be a variable in $\vars(\Gamma_{p_0}) \cap X_q \ne \emptyset$.
    By Lemma \ref{lemma_unprojected}, we know $x \in \vars(n_{p_0})$.
    Notice that for all $i = 0, 1, 2, \ldots$, we have:
    \begin{itemize}
        \item $x \notin X_{p_i}$, as $x \in X_q$ already
        \item $x \in \vars(n_{p_i})$, as $x$ remains from $n_{p_0}$ without being projected according to $\pi(n_{p_i}) = X_{p_i}$
        \item $p_i < p_{i + 1} = \chosenCluster(n_{p_i}) \le q$ by Condition \ref{cond3} of Theorem \ref{thm_csp_jt}, as $x \in \pi(n_q) \cap \vars(n_{p_{i+1}}) = X_q \cap \vars(n_{p_{i+1}}) \ne \emptyset$
    \end{itemize}
    So the strictly increasing sequence $\seq{p_i}_i$ must contain $q$, contradiction.
\qed
\end{proof}

\begin{lemma}
\label{lemma_earlier_cluster}
    Let $1 \le p, q \le m$.
    Assume $\vars(\Gamma_p) \cap X_q \ne \emptyset$.
    Then $p \le q$.
\end{lemma}
\begin{proof}
    To the contrary, assume $p > q$.
    Then by construction, $X_q \cap \vars(\Gamma_p) = \emptyset$, contradiction.
\qed
\end{proof}

\begin{lemma}[Property \ref{prop2}]
\label{lemma_prop2}
    Let $1 \le q \le m$ and variable $x \in \pi(n_q)$.
    Take an arbitrary clause $c \in \phi$ \st{} $x \in \vars(c)$.
    Then the leaf $\gamma^{-1}(c)$ is a descendant of $n_q$.
\end{lemma}
\begin{proof}
    Notice that $c \in \Gamma_p$ for some $1 \le p \le m$.
    Then $x \in \vars(c) \subseteq \vars(\Gamma_p)$.
    Note that $x \in \pi(n_q) = X_q$.
    Thus $p \le q$ by Lemma \ref{lemma_earlier_cluster}.

    Let $l = \gamma^{-1}(c)$.
    Notice that $l \in \kappa_p$ (as $c \in \Gamma_p$).
    So $l$ is a child of the node $n_p$.
    \begin{itemize}
        \item If $p = q$, then $l$ is a child of $n_q$, and we are done.
        \item If $p < q$, by Lemma \ref{lemma_internal_descedant}, we know $n_p$ is a descendant of $n_q$, as $x \in \vars(\Gamma_p) \cap \pi(n_q) = \vars(\Gamma_p) \cap X_q \ne \emptyset$.
        Then $l$ is a descendant of $n_q$.
    \end{itemize}
\qed
\end{proof}

Now we can prove Theorem \ref{thm_csp_jt}
\begin{proof}
    Algorithm \ref{alg_csp_jt} returns a tree $T$ with root $n_m$, bijection $\gamma : \Lv T \to \phi$, and labeling function $\pi : \V T \setminus \Lv T \to 2^X$.
    The project-join tree properties are satisfied, by Lemma \ref{lemma_prop1} and Lemma \ref{lemma_prop2}
\qed
\end{proof}
