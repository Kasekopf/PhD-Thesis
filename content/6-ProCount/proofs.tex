



\section{Proofs}
\label{appendix:proofs}

\subsection{Proof of Theorem \ref{thm:proj_valuation}}
When performing a product followed by a projection, it is often possible to perform the projection first.
This is known as \emph{early projection} \cite{MPPV04}, which forms the core of the proof of Theorem \ref{thm:proj_valuation}.
\begin{theorem}[Early Projection]
\label{thm:early_projection}
    Let $X$ and $Y$ be sets of variables.
    For all functions $f: 2^X \to \R$ and $g: 2^Y \to \R$, if $x \in X \setminus Y$, then $\Sigma_x (f \mult g) = \pars{\Sigma_x f} \mult g$ and $\exists_x (f \mult g) = \pars{\exists_x f} \mult g.$
    % As a corollary, for all $X' \subseteq X \setminus Y$,
    % $$\exists_{X'} (A \mult B) = \pars{\exists_{X'} A} \mult B.$$
\end{theorem}

We ultimately prove Theorem \ref{thm:proj_valuation} by structural induction. 
It is therefore helpful to have some additional notations for subtrees of project-join trees.
Let $(T, r, \gamma, \pi)$ be a project-join tree for $\phi$, and let $n \in \V{T}$. Denote by $S(n) \subseteq \V{T}$ the set of all descendants of $n$ in $T$ (including $n$ itself). 
Let $P(n) = \bigcup_{o \in S(n) \setminus \Lv{T}} \pi(o)$ be the set of all variables projected in $S(n)$, and let $\Phi(n) = \{ \gamma(\ell) : \ell \in \Lv{T} \cap S(n) \}$ be the set of all clauses that appear as leaves in $S(n)$.

The key property of project-join trees is that variables projected in one branch of the tree cannot appear in sibling branches of the tree. Formally:
\begin{lemma}
\label{lemma:disjoint_P}
    In a project-join tree $(T, r, \gamma, \pi)$, let $n$ be an internal node with children $o \ne q$.
    Then $P(o) \cap \vars \pars{\Phi(q)} = \emptyset$.
\end{lemma}
\begin{proof}
    Let variable $x \in P(o)$.
    Notice that $x \in \pi(s)$ for some internal node $s$ that is a descendant of $o$.
    Assume there is an arbitrary clause $c \in \phi$ \st{} $x$ appears in $c$.
    By the last property in Definition \ref{def_jointree_old}, the corresponding leaf $\gamma^{-1}(c)$ is a descendant of $s$ and thus a descendant of $o$.
    So $x$ appears in no descendant leaf of $q$ (as $q$ is a sibling of $o$ in the tree $T$).
    Thus $x \notin \vars(\Phi(q))$. Since $x \in P(o)$ is arbitrary, we conclude that $P(o) \cap \vars \pars{\Phi(q)} = \emptyset$.
\end{proof}

Using Lemma \ref{lemma:disjoint_P}, we can prove that projected-valuations have an equivalent, non-recursive definition:
\begin{lemma}
\label{lemma:proj_valuation}
Let $(X, Y, \phi, W)$ be an instance of weighted projected model counting, and let $\T$ be an $(X,Y)$-graded project-join tree for $\phi$.

Define, for every node $n$ of $T$,
\begin{equation*}
    h^W_n = \left( \prod_{C \in \Phi(n)} [C] \right) \mult \left( \prod_{x \in P(n) \cap X} W_x \right).
\end{equation*}

Then for every node $n$ of $T$,
\begin{equation}
\label{eqn:proj_valuation_lemma}
    g^W_n = \sum_{P(n) \cap X} \displaystyle\exist_{P(n) \cap Y} h^W_n.
\end{equation}
\end{lemma}
\begin{proof}
We employ structural induction on $n \in \V{T}$.
In the base case, $n$ is a leaf. Then $P(n) = \emptyset$ and $\Phi(n) = \{ \gamma(n) \}$. Thus $h^W_n = \left( \prod_{C \in \{ \gamma(n) \} } [C] \right) \mult \left( \prod_{x \in \emptyset} W_x \right) = [\gamma(n)]$, so the right-hand side of Equation \eqref{eqn:proj_valuation_lemma} is
$\sum_\emptyset \exist_\emptyset h^W_n = h^W_n = [\gamma(n)]$, which is exactly $g^W_n$.

In the inductive case, $n$ is an internal node of $T$ and, for each $o \in \C{T}{r}{n}$, we have
$g^W_o = \sum_{P(o) \cap X} \exist_{P(o) \cap Y} h^W_o$.

Consider the product
\begin{equation}
\label{eqn:proj_valuation_child_product}
    \prod_{o \in \C T r n} g^W_o = \prod_{o \in \C T r n} \sum_{P(o) \cap X} \displaystyle\exist_{P(o) \cap Y} h^W_o.
\end{equation}

By Lemma \ref{lemma:disjoint_P}, for distinct $o, q \in \C{T}{r}{n}$, we know $P(o) \cap \vars(\Phi(q)) = \emptyset$. Thus $P(o) \cap \vars(h^W_q) = \emptyset$ as well. We can therefore apply Theorem \ref{thm:early_projection} to Equation \eqref{eqn:proj_valuation_child_product} to get that
\begin{equation}
\prod_{o \in \C{T}{r}{n}} g^W_o = \sum_{A \cap X} \prod_{o \in \C{T}{r}{n}} \displaystyle\exist_{P(o) \cap Y} h^W_o = \sum_{A \cap X} \displaystyle\exist_{A \cap Y} \prod_{o \in \C{T}{r}{n}} h^W_o \label{eq_child_prod}
\end{equation}
where $A = \bigcup_{o \in \C{T}{r}{n}} P(o)$.

Let $\mathcal{I}_X$ and $\mathcal{I}_Y$ be the grades of $\T$. By Definition \ref{def:graded}, either $n \in \mathcal{I}_X$ or $n \in \mathcal{I}_Y$. We divide the inductive case further into these two cases.

\paragraph{Case: $n \in \mathcal{I}_Y$.} Then for each $p \in S(n)$, by Definition \ref{def:graded}, we have $p \in \mathcal{I}_Y$, so $\pi(p) \subseteq Y$. Thus $A \subseteq Y$. 
By Definition \ref{def:graded_valuation} and Equation \eqref{eq_child_prod}, we have
\begin{equation*}
    g^W_n = \displaystyle\exist_{\pi(n)} \prod_{o \in \C{T}{r}{n}} g^W_o = \displaystyle\exist_{\pi(n)} \displaystyle\exist_{A} \prod_{o \in \C{T}{r}{n}} h^W_o = \displaystyle\exist_{P(n)} \prod_{o \in \C{T}{r}{n}} h^W_o.
\end{equation*}

We therefore conclude that $$g^W_n = \displaystyle\exist_{P(n)} \prod_{o \in \C{T}{r}{n}} \prod_{C \in \Phi(o)} [C] = \displaystyle\exist_{P(n)} \prod_{C \in \Phi(n)} [C] = \displaystyle\exist_{P(n)} h^W_n.$$

\paragraph{Case: $n \in \mathcal{I}_X$.} Thus $\pi(n) \subseteq X$. 
By Definition \ref{def:graded_valuation} and Equation \eqref{eq_child_prod}, we have
\begin{align*}
    g^W_n 
    &= \displaystyle\sum_{\pi(n)} \pars{ \prod_{o \in \C T r n} g^W_o \cdot \prod_{x \in \pi(n)} W_x }\\ 
    &= \sum_{\pi(n)} \pars{ \pars{ \sum_{A \cap X} \displaystyle\exist_{A \cap Y} \prod_{o \in \C{T}{r}{n}} h^W_o } \cdot \prod_{x \in \pi(n)} W_x }.
\end{align*}
Since $\pi(n) \cap A = \emptyset$, we can apply Theorem \ref{thm:early_projection} (in the other direction, which undoes early projection) to get that
\begin{equation*}
    g^W_n = \sum_{\pi(n)} \pars{ \sum_{A \cap X} \displaystyle\exist_{A \cap Y} \left( \prod_{o \in \C{T}{r}{n}} h^W_o \cdot \prod_{x \in \pi(n)} W_x \right) }.
\end{equation*}
Finally, observe that $\pi(n) \cup A = P(n)$ and that $h^W_n = \prod_{o \in \C{T}{r}{n}} h^W_o \cdot \prod_{x \in \pi(n)} W_x$. We therefore conclude that
\begin{equation*}
    g^W_n = \sum_{P(n) \cap X} \displaystyle\exist_{P(n) \cap Y} \left( \prod_{o \in \C{T}{r}{n}} h^W_o \cdot \prod_{x \in \pi(n)} W_x \right) =  \sum_{P(n) \cap X} \displaystyle\exist_{P(n) \cap Y} h^W_n.
\end{equation*}
\end{proof}

Moreover, this non-recursive definition is equivalent to the weighted projected model count at the root node.
\begin{theorem}{\ref{thm:proj_valuation}}
Let $(X, Y, \phi, W)$ be an instance of weighted projected model counting, and let $\T$ be a project-join tree for $\phi$ with root $r$. 
If $\T$ is $(X, Y)$-graded, then $g^W_r(\emptyset) = \func{WPMC}(\phi, W, Y).$
\end{theorem}
\begin{proof}
As $r$ is the root of the project-join tree, $P(r) = X \cup Y$ and $\Phi(r) = \phi$. By Lemma \ref{lemma:proj_valuation}, $$g^W_r = \sum_{X} \displaystyle\exist_{Y} h^W_n = \sum_{X} \displaystyle\exist_{Y} \left( \prod_{C \in \phi} [C] \right) \mult \left( \prod_{x \in X} W_x \right) = \sum_{X} \displaystyle\exist_{Y} [\phi] \mult W.$$
Thus $g^W_r(\emptyset)$ is exactly the $W$-weighted $Y$-projected model count of $\phi$.
\end{proof}
