\section{Decision Diagrams}
\label{sec:background:dd}
An \emph{algebraic decision diagram (ADD)} is a compact representation of a pseudo-Boolean function as a directed acyclic graph \cite{bahar1997algebraic}.
For functions with logical structure, an ADD representation can be exponentially smaller than the explicit representation.
Originally designed for matrix multiplication and shortest path algorithms, ADDs have also been used for Bayesian inference \cite{chavira2007compiling,gogate2011approximation}, stochastic planning \cite{hoey1999spudd}, model checking \cite{kwiatkowska2007stochastic}, and model counting \cite{fargier2014knowledge,DPV20}. Formally:

\begin{definition}
\label{def:add}[Algebraic decision diagram]
An \emph{ADD} is a tuple $(X, S, \sigma, G)$, where $X$ is a set of Boolean variables, $S$ is an arbitrary set (called the \textdef{carrier set}), $\sigma: X \to \N$ is an injection (called the \textdef{diagram variable order}), and $G$ is a rooted directed acyclic graph satisfying the following three properties:
\begin{enumerate}
    \item Every leaf node of $G$ is labeled with an element of $S$,
    \item Every internal node of $G$ is labeled with an element of $X$ and has two outgoing edges, labeled 0 and 1, and
    \item For every path in $G$, the labels of internal nodes must occur in increasing order under $\sigma$.
\end{enumerate}
\end{definition}
In this work, we only need to consider ADDs with the carrier set $S = \mathbb{R}$. 
An ADD $(X, S, \sigma, G)$ is a compact representation of a function $f: 2^X \to S$.
Although there are many ADDs representing $f$, for each injection $\sigma: X \to \N$, there is a unique minimal ADD that represents $f$ with $\sigma$ as the diagram variable order, called the \textdef{canonical ADD}.
ADDs can be minimized in polynomial time, so it is typical to only work with canonical ADDs.

One challenge in using ADDs is choosing the diagram variable order.
The choice of diagram variable order can have a dramatic impact on the size of the ADD; some variable orders may produce ADDs that are exponentially smaller than others for the same real-valued function.
A variety of techniques exist in prior work to heuristically find diagram variable orders \cite{tarjan1984simple,koster2001treewidth,dechter03}.
Moreover, since binary decision diagrams (BDDs) \cite{bryant1986graph} can be seen as ADDs with carrier set $S = \set{0, 1}$, there is significant overlap with the techniques to find variable orders for BDDs.

Several packages exist for efficiently manipulating ADDs.
For example, \cudd{} \cite{somenzi2015cudd} implements product, $\Sigma$-projection and $\exists$-projection on ADDs in polynomial time (in the size of the ADD representation).