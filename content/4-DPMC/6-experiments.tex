\section{Implementation and Evaluation}
\label{sec_experiments}

We aim to answer the following experimental research questions:
\begin{itemize}
    \item[(RQ1)] In the planning phase, how do constraint-satisfaction heuristics compare to tree-decomposition solvers?
    \item[(RQ2)] In the execution phase, how do ADDs compare to tensors as the underlying data structure?
    \item[(RQ3)] Are project-join-tree-based weighted model counters competitive with state-of-the-art tools?
\end{itemize}

To answer RQ1, we build two implementations of the planning phase: \Htb{} (Heuristic Tree Builder, based on \cite{DPV20}) and \Lg{} (Line Graph).
\Htb{} implements Algorithm \ref{alg_csp_jt} and so is representative of the constraint-satisfaction approach.
\Htb{} contains implementations of four clustering heuristics (\Be-\ListH, \Be-\TreeH, \Bm-\ListH, and \Bm-\TreeH) and nine cluster-variable-order heuristics (\Random, \Mcs, \Invmcs, \Lexp, \Invlexp, \Lexm, \Invlexm, \Minfill, and \Invminfill).
\Lg{} implements Algorithm \ref{alg_td_to_join} and so is representative of the tree-decomposition approach.
In order to find tree decompositions, \Lg{} leverages three state-of-the-art heuristic tree-decomposition solvers: \Flowcutter{} \cite{strasser2017computing}, \Htd{} \cite{AMW17}, and \Tamaki{} \cite{Tamaki17}.
These solvers are all \emph{anytime}, meaning that \Lg{} never halts but continues to produce better and better project-join trees when given additional time.
On the other hand, \Htb{} produces a single project-join tree.
We compare these implementations on the planning phase in Section \ref{sec_experiments_planning}.

To answer RQ2, we build two implementations of the execution phase: \Dmc{} (for Diagram Model Counter, based on \cite{DPV20}) and \Tensor{}.
\Dmc{} uses ADDs as the underlying data structure with \cudd{} \cite{somenzi2015cudd}.
\Tensor{} uses tensors as the underlying data structure with \Numpy{} \cite{numpy}.
We compare these implementations on the execution phase in Section \ref{sec_experiments_execution}.
Since \Lg{} is an anytime tool, each execution tool must additionally determine the best time to terminate \Lg{} and begin performing the valuation.
We explore options for this in Section \ref{sec_experiments_execution}.

To answer RQ3, we combine each implementation of the planning phase and each implementation of the execution phase to produce model counters that use project-join trees.
We then compare these model counters with the state-of-the-art tools \cachet{} \cite{sang2004combining}, \ctd{} \cite{darwiche2004new}, \df{} \cite{LM17}, and \minictd{} \cite{OD15} in Section \ref{sec_experiments_wmc}.

We use a set of \benchmarkCountAltogether{} literal-weighted model counting benchmarks from \cite{DPV20}.
These benchmarks were gathered from two sources.
First, the \classBayes{} class%
\footnote{\urlBenchmarksBayes}
consists of \benchmarkCountBayes{} CNF benchmarks%
\footnote{excluding 11 benchmarks double-counted by \cite{DPV20}}
that encode Bayesian inference problems \cite{sang2005performing}.
% This benchmark class is subdivided into three families: \famDqmr, \famGrid, and \famPlanRec.
All literal weights in this class are between 0 and 1. % JD: [0, 1] looks like references; let's avoid it
Second, the \classOther{} class%
\footnote{\urlBenchmarksOther}
consists of \benchmarkCountOther{} CNF benchmarks%
\footnote{including 73 benchmarks missed by \cite{DPV20}}
that are divided into eight families: \famBmc, \famCircuit, \famConfig, \famHandmade, \famPlanning, \famQif, \famRandom, and \famSchedule{} \cite{clarke2001bounded,sinz2003formal,palacios2009compiling,klebanov2013sat}.
All \classOther{} benchmarks are originally unweighted.
As we focus in this work on weighted model counting, we generate weights for these benchmarks.
Each variable $x$ is randomly assigned literal weights: either $W_x(\set{x}) = 0.5$ and $W_x(\emptyset) = 1.5$, or $W_x(\set{x}) = 1.5$ and $W_x(\emptyset) = 0.5$.
% \footnote{
%   For each variable $x$, \cachet{} requires $W(\set{x}) + W(\emptyset) = 1$ unless $W(\set{x}) = W(\emptyset) = 1$.
%   So we use weights 0.25 and 0.75 for \cachet{} and multiply the model count produced by \cachet{} on a formula $\phi$ by $2^{\size{\vars(\phi)}}$ as a postprocessing step.
% }
Generating weights in this particular fashion results in a reasonably low amount of floating-point underflow and overflow for all model counters.

We ran all experiments on single CPU cores of a Linux cluster with Xeon E5-2650v2 processors (2.60-GHz) and 30 GB of memory.
All code, benchmarks, and experimental data are available at \url{https://github.com/vardigroup/DPMC}.

%%%%%%%%%%%%%%%%%%%%%%%%%%%%%%%%%%%%%%%%%%%%%%%%%%%%%%%%%%%%%%%%%%%%%%%%%%%%%%%%

\subsection{Experiment 1: Comparing Project-Join Planners}
\label{sec_experiments_planning}

\begin{figure}[t]
	\centering
	%% Creator: Matplotlib, PGF backend
%%
%% To include the figure in your LaTeX document, write
%%   \input{<filename>.pgf}
%%
%% Make sure the required packages are loaded in your preamble
%%   \usepackage{pgf}
%%
%% and, on pdftex
%%   \usepackage[utf8]{inputenc}\DeclareUnicodeCharacter{2212}{-}
%%
%% or, on luatex and xetex
%%   \usepackage{unicode-math}
%%
%% Figures using additional raster images can only be included by \input if
%% they are in the same directory as the main LaTeX file. For loading figures
%% from other directories you can use the `import` package
%%   \usepackage{import}
%%
%% and then include the figures with
%%   \import{<path to file>}{<filename>.pgf}
%%
%% Matplotlib used the following preamble
%%   \usepackage[utf8x]{inputenc}
%%   \usepackage[T1]{fontenc}
%%
\begingroup%
\makeatletter%
\begin{pgfpicture}%
\pgfpathrectangle{\pgfpointorigin}{\pgfqpoint{4.803148in}{2.021259in}}%
\pgfusepath{use as bounding box, clip}%
\begin{pgfscope}%
\pgfsetbuttcap%
\pgfsetmiterjoin%
\definecolor{currentfill}{rgb}{1.000000,1.000000,1.000000}%
\pgfsetfillcolor{currentfill}%
\pgfsetlinewidth{0.000000pt}%
\definecolor{currentstroke}{rgb}{1.000000,1.000000,1.000000}%
\pgfsetstrokecolor{currentstroke}%
\pgfsetdash{}{0pt}%
\pgfpathmoveto{\pgfqpoint{0.000000in}{0.000000in}}%
\pgfpathlineto{\pgfqpoint{4.803148in}{0.000000in}}%
\pgfpathlineto{\pgfqpoint{4.803148in}{2.021259in}}%
\pgfpathlineto{\pgfqpoint{0.000000in}{2.021259in}}%
\pgfpathclose%
\pgfusepath{fill}%
\end{pgfscope}%
\begin{pgfscope}%
\pgfsetbuttcap%
\pgfsetmiterjoin%
\definecolor{currentfill}{rgb}{1.000000,1.000000,1.000000}%
\pgfsetfillcolor{currentfill}%
\pgfsetlinewidth{0.000000pt}%
\definecolor{currentstroke}{rgb}{0.000000,0.000000,0.000000}%
\pgfsetstrokecolor{currentstroke}%
\pgfsetstrokeopacity{0.000000}%
\pgfsetdash{}{0pt}%
\pgfpathmoveto{\pgfqpoint{0.694334in}{0.523557in}}%
\pgfpathlineto{\pgfqpoint{4.524677in}{0.523557in}}%
\pgfpathlineto{\pgfqpoint{4.524677in}{1.826535in}}%
\pgfpathlineto{\pgfqpoint{0.694334in}{1.826535in}}%
\pgfpathclose%
\pgfusepath{fill}%
\end{pgfscope}%
\begin{pgfscope}%
\pgfsetbuttcap%
\pgfsetroundjoin%
\definecolor{currentfill}{rgb}{0.000000,0.000000,0.000000}%
\pgfsetfillcolor{currentfill}%
\pgfsetlinewidth{0.803000pt}%
\definecolor{currentstroke}{rgb}{0.000000,0.000000,0.000000}%
\pgfsetstrokecolor{currentstroke}%
\pgfsetdash{}{0pt}%
\pgfsys@defobject{currentmarker}{\pgfqpoint{0.000000in}{-0.048611in}}{\pgfqpoint{0.000000in}{0.000000in}}{%
\pgfpathmoveto{\pgfqpoint{0.000000in}{0.000000in}}%
\pgfpathlineto{\pgfqpoint{0.000000in}{-0.048611in}}%
\pgfusepath{stroke,fill}%
}%
\begin{pgfscope}%
\pgfsys@transformshift{0.694334in}{0.523557in}%
\pgfsys@useobject{currentmarker}{}%
\end{pgfscope}%
\end{pgfscope}%
\begin{pgfscope}%
\definecolor{textcolor}{rgb}{0.000000,0.000000,0.000000}%
\pgfsetstrokecolor{textcolor}%
\pgfsetfillcolor{textcolor}%
\pgftext[x=0.694334in,y=0.426335in,,top]{\color{textcolor}\rmfamily\fontsize{9.000000}{10.800000}\selectfont \(\displaystyle 0\)}%
\end{pgfscope}%
\begin{pgfscope}%
\pgfsetbuttcap%
\pgfsetroundjoin%
\definecolor{currentfill}{rgb}{0.000000,0.000000,0.000000}%
\pgfsetfillcolor{currentfill}%
\pgfsetlinewidth{0.803000pt}%
\definecolor{currentstroke}{rgb}{0.000000,0.000000,0.000000}%
\pgfsetstrokecolor{currentstroke}%
\pgfsetdash{}{0pt}%
\pgfsys@defobject{currentmarker}{\pgfqpoint{0.000000in}{-0.048611in}}{\pgfqpoint{0.000000in}{0.000000in}}{%
\pgfpathmoveto{\pgfqpoint{0.000000in}{0.000000in}}%
\pgfpathlineto{\pgfqpoint{0.000000in}{-0.048611in}}%
\pgfusepath{stroke,fill}%
}%
\begin{pgfscope}%
\pgfsys@transformshift{1.173127in}{0.523557in}%
\pgfsys@useobject{currentmarker}{}%
\end{pgfscope}%
\end{pgfscope}%
\begin{pgfscope}%
\definecolor{textcolor}{rgb}{0.000000,0.000000,0.000000}%
\pgfsetstrokecolor{textcolor}%
\pgfsetfillcolor{textcolor}%
\pgftext[x=1.173127in,y=0.426335in,,top]{\color{textcolor}\rmfamily\fontsize{9.000000}{10.800000}\selectfont \(\displaystyle 250\)}%
\end{pgfscope}%
\begin{pgfscope}%
\pgfsetbuttcap%
\pgfsetroundjoin%
\definecolor{currentfill}{rgb}{0.000000,0.000000,0.000000}%
\pgfsetfillcolor{currentfill}%
\pgfsetlinewidth{0.803000pt}%
\definecolor{currentstroke}{rgb}{0.000000,0.000000,0.000000}%
\pgfsetstrokecolor{currentstroke}%
\pgfsetdash{}{0pt}%
\pgfsys@defobject{currentmarker}{\pgfqpoint{0.000000in}{-0.048611in}}{\pgfqpoint{0.000000in}{0.000000in}}{%
\pgfpathmoveto{\pgfqpoint{0.000000in}{0.000000in}}%
\pgfpathlineto{\pgfqpoint{0.000000in}{-0.048611in}}%
\pgfusepath{stroke,fill}%
}%
\begin{pgfscope}%
\pgfsys@transformshift{1.651920in}{0.523557in}%
\pgfsys@useobject{currentmarker}{}%
\end{pgfscope}%
\end{pgfscope}%
\begin{pgfscope}%
\definecolor{textcolor}{rgb}{0.000000,0.000000,0.000000}%
\pgfsetstrokecolor{textcolor}%
\pgfsetfillcolor{textcolor}%
\pgftext[x=1.651920in,y=0.426335in,,top]{\color{textcolor}\rmfamily\fontsize{9.000000}{10.800000}\selectfont \(\displaystyle 500\)}%
\end{pgfscope}%
\begin{pgfscope}%
\pgfsetbuttcap%
\pgfsetroundjoin%
\definecolor{currentfill}{rgb}{0.000000,0.000000,0.000000}%
\pgfsetfillcolor{currentfill}%
\pgfsetlinewidth{0.803000pt}%
\definecolor{currentstroke}{rgb}{0.000000,0.000000,0.000000}%
\pgfsetstrokecolor{currentstroke}%
\pgfsetdash{}{0pt}%
\pgfsys@defobject{currentmarker}{\pgfqpoint{0.000000in}{-0.048611in}}{\pgfqpoint{0.000000in}{0.000000in}}{%
\pgfpathmoveto{\pgfqpoint{0.000000in}{0.000000in}}%
\pgfpathlineto{\pgfqpoint{0.000000in}{-0.048611in}}%
\pgfusepath{stroke,fill}%
}%
\begin{pgfscope}%
\pgfsys@transformshift{2.130713in}{0.523557in}%
\pgfsys@useobject{currentmarker}{}%
\end{pgfscope}%
\end{pgfscope}%
\begin{pgfscope}%
\definecolor{textcolor}{rgb}{0.000000,0.000000,0.000000}%
\pgfsetstrokecolor{textcolor}%
\pgfsetfillcolor{textcolor}%
\pgftext[x=2.130713in,y=0.426335in,,top]{\color{textcolor}\rmfamily\fontsize{9.000000}{10.800000}\selectfont \(\displaystyle 750\)}%
\end{pgfscope}%
\begin{pgfscope}%
\pgfsetbuttcap%
\pgfsetroundjoin%
\definecolor{currentfill}{rgb}{0.000000,0.000000,0.000000}%
\pgfsetfillcolor{currentfill}%
\pgfsetlinewidth{0.803000pt}%
\definecolor{currentstroke}{rgb}{0.000000,0.000000,0.000000}%
\pgfsetstrokecolor{currentstroke}%
\pgfsetdash{}{0pt}%
\pgfsys@defobject{currentmarker}{\pgfqpoint{0.000000in}{-0.048611in}}{\pgfqpoint{0.000000in}{0.000000in}}{%
\pgfpathmoveto{\pgfqpoint{0.000000in}{0.000000in}}%
\pgfpathlineto{\pgfqpoint{0.000000in}{-0.048611in}}%
\pgfusepath{stroke,fill}%
}%
\begin{pgfscope}%
\pgfsys@transformshift{2.609506in}{0.523557in}%
\pgfsys@useobject{currentmarker}{}%
\end{pgfscope}%
\end{pgfscope}%
\begin{pgfscope}%
\definecolor{textcolor}{rgb}{0.000000,0.000000,0.000000}%
\pgfsetstrokecolor{textcolor}%
\pgfsetfillcolor{textcolor}%
\pgftext[x=2.609506in,y=0.426335in,,top]{\color{textcolor}\rmfamily\fontsize{9.000000}{10.800000}\selectfont \(\displaystyle 1000\)}%
\end{pgfscope}%
\begin{pgfscope}%
\pgfsetbuttcap%
\pgfsetroundjoin%
\definecolor{currentfill}{rgb}{0.000000,0.000000,0.000000}%
\pgfsetfillcolor{currentfill}%
\pgfsetlinewidth{0.803000pt}%
\definecolor{currentstroke}{rgb}{0.000000,0.000000,0.000000}%
\pgfsetstrokecolor{currentstroke}%
\pgfsetdash{}{0pt}%
\pgfsys@defobject{currentmarker}{\pgfqpoint{0.000000in}{-0.048611in}}{\pgfqpoint{0.000000in}{0.000000in}}{%
\pgfpathmoveto{\pgfqpoint{0.000000in}{0.000000in}}%
\pgfpathlineto{\pgfqpoint{0.000000in}{-0.048611in}}%
\pgfusepath{stroke,fill}%
}%
\begin{pgfscope}%
\pgfsys@transformshift{3.088299in}{0.523557in}%
\pgfsys@useobject{currentmarker}{}%
\end{pgfscope}%
\end{pgfscope}%
\begin{pgfscope}%
\definecolor{textcolor}{rgb}{0.000000,0.000000,0.000000}%
\pgfsetstrokecolor{textcolor}%
\pgfsetfillcolor{textcolor}%
\pgftext[x=3.088299in,y=0.426335in,,top]{\color{textcolor}\rmfamily\fontsize{9.000000}{10.800000}\selectfont \(\displaystyle 1250\)}%
\end{pgfscope}%
\begin{pgfscope}%
\pgfsetbuttcap%
\pgfsetroundjoin%
\definecolor{currentfill}{rgb}{0.000000,0.000000,0.000000}%
\pgfsetfillcolor{currentfill}%
\pgfsetlinewidth{0.803000pt}%
\definecolor{currentstroke}{rgb}{0.000000,0.000000,0.000000}%
\pgfsetstrokecolor{currentstroke}%
\pgfsetdash{}{0pt}%
\pgfsys@defobject{currentmarker}{\pgfqpoint{0.000000in}{-0.048611in}}{\pgfqpoint{0.000000in}{0.000000in}}{%
\pgfpathmoveto{\pgfqpoint{0.000000in}{0.000000in}}%
\pgfpathlineto{\pgfqpoint{0.000000in}{-0.048611in}}%
\pgfusepath{stroke,fill}%
}%
\begin{pgfscope}%
\pgfsys@transformshift{3.567091in}{0.523557in}%
\pgfsys@useobject{currentmarker}{}%
\end{pgfscope}%
\end{pgfscope}%
\begin{pgfscope}%
\definecolor{textcolor}{rgb}{0.000000,0.000000,0.000000}%
\pgfsetstrokecolor{textcolor}%
\pgfsetfillcolor{textcolor}%
\pgftext[x=3.567091in,y=0.426335in,,top]{\color{textcolor}\rmfamily\fontsize{9.000000}{10.800000}\selectfont \(\displaystyle 1500\)}%
\end{pgfscope}%
\begin{pgfscope}%
\pgfsetbuttcap%
\pgfsetroundjoin%
\definecolor{currentfill}{rgb}{0.000000,0.000000,0.000000}%
\pgfsetfillcolor{currentfill}%
\pgfsetlinewidth{0.803000pt}%
\definecolor{currentstroke}{rgb}{0.000000,0.000000,0.000000}%
\pgfsetstrokecolor{currentstroke}%
\pgfsetdash{}{0pt}%
\pgfsys@defobject{currentmarker}{\pgfqpoint{0.000000in}{-0.048611in}}{\pgfqpoint{0.000000in}{0.000000in}}{%
\pgfpathmoveto{\pgfqpoint{0.000000in}{0.000000in}}%
\pgfpathlineto{\pgfqpoint{0.000000in}{-0.048611in}}%
\pgfusepath{stroke,fill}%
}%
\begin{pgfscope}%
\pgfsys@transformshift{4.045884in}{0.523557in}%
\pgfsys@useobject{currentmarker}{}%
\end{pgfscope}%
\end{pgfscope}%
\begin{pgfscope}%
\definecolor{textcolor}{rgb}{0.000000,0.000000,0.000000}%
\pgfsetstrokecolor{textcolor}%
\pgfsetfillcolor{textcolor}%
\pgftext[x=4.045884in,y=0.426335in,,top]{\color{textcolor}\rmfamily\fontsize{9.000000}{10.800000}\selectfont \(\displaystyle 1750\)}%
\end{pgfscope}%
\begin{pgfscope}%
\pgfsetbuttcap%
\pgfsetroundjoin%
\definecolor{currentfill}{rgb}{0.000000,0.000000,0.000000}%
\pgfsetfillcolor{currentfill}%
\pgfsetlinewidth{0.803000pt}%
\definecolor{currentstroke}{rgb}{0.000000,0.000000,0.000000}%
\pgfsetstrokecolor{currentstroke}%
\pgfsetdash{}{0pt}%
\pgfsys@defobject{currentmarker}{\pgfqpoint{0.000000in}{-0.048611in}}{\pgfqpoint{0.000000in}{0.000000in}}{%
\pgfpathmoveto{\pgfqpoint{0.000000in}{0.000000in}}%
\pgfpathlineto{\pgfqpoint{0.000000in}{-0.048611in}}%
\pgfusepath{stroke,fill}%
}%
\begin{pgfscope}%
\pgfsys@transformshift{4.524677in}{0.523557in}%
\pgfsys@useobject{currentmarker}{}%
\end{pgfscope}%
\end{pgfscope}%
\begin{pgfscope}%
\definecolor{textcolor}{rgb}{0.000000,0.000000,0.000000}%
\pgfsetstrokecolor{textcolor}%
\pgfsetfillcolor{textcolor}%
\pgftext[x=4.524677in,y=0.426335in,,top]{\color{textcolor}\rmfamily\fontsize{9.000000}{10.800000}\selectfont \(\displaystyle 2000\)}%
\end{pgfscope}%
\begin{pgfscope}%
\definecolor{textcolor}{rgb}{0.000000,0.000000,0.000000}%
\pgfsetstrokecolor{textcolor}%
\pgfsetfillcolor{textcolor}%
\pgftext[x=2.609506in,y=0.260390in,,top]{\color{textcolor}\rmfamily\fontsize{9.000000}{10.800000}\selectfont Number of benchmarks solved}%
\end{pgfscope}%
\begin{pgfscope}%
\pgfsetbuttcap%
\pgfsetroundjoin%
\definecolor{currentfill}{rgb}{0.000000,0.000000,0.000000}%
\pgfsetfillcolor{currentfill}%
\pgfsetlinewidth{0.803000pt}%
\definecolor{currentstroke}{rgb}{0.000000,0.000000,0.000000}%
\pgfsetstrokecolor{currentstroke}%
\pgfsetdash{}{0pt}%
\pgfsys@defobject{currentmarker}{\pgfqpoint{-0.048611in}{0.000000in}}{\pgfqpoint{0.000000in}{0.000000in}}{%
\pgfpathmoveto{\pgfqpoint{0.000000in}{0.000000in}}%
\pgfpathlineto{\pgfqpoint{-0.048611in}{0.000000in}}%
\pgfusepath{stroke,fill}%
}%
\begin{pgfscope}%
\pgfsys@transformshift{0.694334in}{0.793606in}%
\pgfsys@useobject{currentmarker}{}%
\end{pgfscope}%
\end{pgfscope}%
\begin{pgfscope}%
\definecolor{textcolor}{rgb}{0.000000,0.000000,0.000000}%
\pgfsetstrokecolor{textcolor}%
\pgfsetfillcolor{textcolor}%
\pgftext[x=0.330525in, y=0.748881in, left, base]{\color{textcolor}\rmfamily\fontsize{9.000000}{10.800000}\selectfont \(\displaystyle 10^{-2}\)}%
\end{pgfscope}%
\begin{pgfscope}%
\pgfsetbuttcap%
\pgfsetroundjoin%
\definecolor{currentfill}{rgb}{0.000000,0.000000,0.000000}%
\pgfsetfillcolor{currentfill}%
\pgfsetlinewidth{0.803000pt}%
\definecolor{currentstroke}{rgb}{0.000000,0.000000,0.000000}%
\pgfsetstrokecolor{currentstroke}%
\pgfsetdash{}{0pt}%
\pgfsys@defobject{currentmarker}{\pgfqpoint{-0.048611in}{0.000000in}}{\pgfqpoint{0.000000in}{0.000000in}}{%
\pgfpathmoveto{\pgfqpoint{0.000000in}{0.000000in}}%
\pgfpathlineto{\pgfqpoint{-0.048611in}{0.000000in}}%
\pgfusepath{stroke,fill}%
}%
\begin{pgfscope}%
\pgfsys@transformshift{0.694334in}{1.310070in}%
\pgfsys@useobject{currentmarker}{}%
\end{pgfscope}%
\end{pgfscope}%
\begin{pgfscope}%
\definecolor{textcolor}{rgb}{0.000000,0.000000,0.000000}%
\pgfsetstrokecolor{textcolor}%
\pgfsetfillcolor{textcolor}%
\pgftext[x=0.410771in, y=1.265345in, left, base]{\color{textcolor}\rmfamily\fontsize{9.000000}{10.800000}\selectfont \(\displaystyle 10^{0}\)}%
\end{pgfscope}%
\begin{pgfscope}%
\pgfsetbuttcap%
\pgfsetroundjoin%
\definecolor{currentfill}{rgb}{0.000000,0.000000,0.000000}%
\pgfsetfillcolor{currentfill}%
\pgfsetlinewidth{0.803000pt}%
\definecolor{currentstroke}{rgb}{0.000000,0.000000,0.000000}%
\pgfsetstrokecolor{currentstroke}%
\pgfsetdash{}{0pt}%
\pgfsys@defobject{currentmarker}{\pgfqpoint{-0.048611in}{0.000000in}}{\pgfqpoint{0.000000in}{0.000000in}}{%
\pgfpathmoveto{\pgfqpoint{0.000000in}{0.000000in}}%
\pgfpathlineto{\pgfqpoint{-0.048611in}{0.000000in}}%
\pgfusepath{stroke,fill}%
}%
\begin{pgfscope}%
\pgfsys@transformshift{0.694334in}{1.826535in}%
\pgfsys@useobject{currentmarker}{}%
\end{pgfscope}%
\end{pgfscope}%
\begin{pgfscope}%
\definecolor{textcolor}{rgb}{0.000000,0.000000,0.000000}%
\pgfsetstrokecolor{textcolor}%
\pgfsetfillcolor{textcolor}%
\pgftext[x=0.410771in, y=1.781810in, left, base]{\color{textcolor}\rmfamily\fontsize{9.000000}{10.800000}\selectfont \(\displaystyle 10^{2}\)}%
\end{pgfscope}%
\begin{pgfscope}%
\definecolor{textcolor}{rgb}{0.000000,0.000000,0.000000}%
\pgfsetstrokecolor{textcolor}%
\pgfsetfillcolor{textcolor}%
\pgftext[x=0.274969in,y=1.175046in,,bottom,rotate=90.000000]{\color{textcolor}\rmfamily\fontsize{9.000000}{10.800000}\selectfont Longest solving time (s)}%
\end{pgfscope}%
\begin{pgfscope}%
\pgfpathrectangle{\pgfqpoint{0.694334in}{0.523557in}}{\pgfqpoint{3.830343in}{1.302977in}}%
\pgfusepath{clip}%
\pgfsetrectcap%
\pgfsetroundjoin%
\pgfsetlinewidth{1.003750pt}%
\definecolor{currentstroke}{rgb}{0.752941,0.752941,1.000000}%
\pgfsetstrokecolor{currentstroke}%
\pgfsetdash{}{0pt}%
\pgfpathmoveto{\pgfqpoint{0.694334in}{0.736317in}}%
\pgfpathlineto{\pgfqpoint{0.696249in}{0.753605in}}%
\pgfpathlineto{\pgfqpoint{0.698165in}{0.753605in}}%
\pgfpathlineto{\pgfqpoint{0.700080in}{0.753605in}}%
\pgfpathlineto{\pgfqpoint{0.701995in}{0.753605in}}%
\pgfpathlineto{\pgfqpoint{0.703910in}{0.753605in}}%
\pgfpathlineto{\pgfqpoint{0.705825in}{0.753605in}}%
\pgfpathlineto{\pgfqpoint{0.707741in}{0.753605in}}%
\pgfpathlineto{\pgfqpoint{0.709656in}{0.753605in}}%
\pgfpathlineto{\pgfqpoint{0.711571in}{0.753605in}}%
\pgfpathlineto{\pgfqpoint{0.713486in}{0.753605in}}%
\pgfpathlineto{\pgfqpoint{0.715401in}{0.753605in}}%
\pgfpathlineto{\pgfqpoint{0.717316in}{0.753605in}}%
\pgfpathlineto{\pgfqpoint{0.719232in}{0.753605in}}%
\pgfpathlineto{\pgfqpoint{0.721147in}{0.753605in}}%
\pgfpathlineto{\pgfqpoint{0.723062in}{0.753605in}}%
\pgfpathlineto{\pgfqpoint{0.724977in}{0.753605in}}%
\pgfpathlineto{\pgfqpoint{0.726892in}{0.753605in}}%
\pgfpathlineto{\pgfqpoint{0.728807in}{0.753605in}}%
\pgfpathlineto{\pgfqpoint{0.730723in}{0.753605in}}%
\pgfpathlineto{\pgfqpoint{0.732638in}{0.768580in}}%
\pgfpathlineto{\pgfqpoint{0.734553in}{0.768580in}}%
\pgfpathlineto{\pgfqpoint{0.736468in}{0.768580in}}%
\pgfpathlineto{\pgfqpoint{0.738383in}{0.768580in}}%
\pgfpathlineto{\pgfqpoint{0.740298in}{0.768580in}}%
\pgfpathlineto{\pgfqpoint{0.742214in}{0.768580in}}%
\pgfpathlineto{\pgfqpoint{0.744129in}{0.768580in}}%
\pgfpathlineto{\pgfqpoint{0.746044in}{0.768580in}}%
\pgfpathlineto{\pgfqpoint{0.747959in}{0.853115in}}%
\pgfpathlineto{\pgfqpoint{0.749874in}{0.882030in}}%
\pgfpathlineto{\pgfqpoint{0.751789in}{0.887015in}}%
\pgfpathlineto{\pgfqpoint{0.753705in}{1.826535in}}%
\pgfusepath{stroke}%
\end{pgfscope}%
\begin{pgfscope}%
\pgfpathrectangle{\pgfqpoint{0.694334in}{0.523557in}}{\pgfqpoint{3.830343in}{1.302977in}}%
\pgfusepath{clip}%
\pgfsetrectcap%
\pgfsetroundjoin%
\pgfsetlinewidth{1.003750pt}%
\definecolor{currentstroke}{rgb}{0.752941,0.752941,1.000000}%
\pgfsetstrokecolor{currentstroke}%
\pgfsetdash{}{0pt}%
\pgfpathmoveto{\pgfqpoint{0.694334in}{0.753605in}}%
\pgfpathlineto{\pgfqpoint{0.709656in}{0.753605in}}%
\pgfpathlineto{\pgfqpoint{0.711571in}{0.768580in}}%
\pgfpathlineto{\pgfqpoint{0.723062in}{0.768580in}}%
\pgfpathlineto{\pgfqpoint{0.724977in}{0.781789in}}%
\pgfpathlineto{\pgfqpoint{0.734553in}{0.781789in}}%
\pgfpathlineto{\pgfqpoint{0.738383in}{0.804294in}}%
\pgfpathlineto{\pgfqpoint{0.740298in}{0.804294in}}%
\pgfpathlineto{\pgfqpoint{0.742214in}{0.814053in}}%
\pgfpathlineto{\pgfqpoint{0.746044in}{0.814053in}}%
\pgfpathlineto{\pgfqpoint{0.747959in}{0.823029in}}%
\pgfpathlineto{\pgfqpoint{0.749874in}{0.823029in}}%
\pgfpathlineto{\pgfqpoint{0.751789in}{0.831341in}}%
\pgfpathlineto{\pgfqpoint{0.757535in}{0.831341in}}%
\pgfpathlineto{\pgfqpoint{0.759450in}{0.839078in}}%
\pgfpathlineto{\pgfqpoint{0.763280in}{0.839078in}}%
\pgfpathlineto{\pgfqpoint{0.765196in}{0.846316in}}%
\pgfpathlineto{\pgfqpoint{0.780517in}{0.846316in}}%
\pgfpathlineto{\pgfqpoint{0.782432in}{0.853115in}}%
\pgfpathlineto{\pgfqpoint{0.799669in}{0.853115in}}%
\pgfpathlineto{\pgfqpoint{0.801584in}{0.859525in}}%
\pgfpathlineto{\pgfqpoint{0.824566in}{0.859525in}}%
\pgfpathlineto{\pgfqpoint{0.826481in}{0.865589in}}%
\pgfpathlineto{\pgfqpoint{0.832227in}{0.865589in}}%
\pgfpathlineto{\pgfqpoint{0.834142in}{0.871341in}}%
\pgfpathlineto{\pgfqpoint{0.839887in}{0.871341in}}%
\pgfpathlineto{\pgfqpoint{0.841803in}{0.876813in}}%
\pgfpathlineto{\pgfqpoint{0.853294in}{0.876813in}}%
\pgfpathlineto{\pgfqpoint{0.855209in}{0.882030in}}%
\pgfpathlineto{\pgfqpoint{0.880106in}{0.882030in}}%
\pgfpathlineto{\pgfqpoint{0.882021in}{0.887015in}}%
\pgfpathlineto{\pgfqpoint{0.903088in}{0.887015in}}%
\pgfpathlineto{\pgfqpoint{0.905003in}{0.891788in}}%
\pgfpathlineto{\pgfqpoint{0.918409in}{0.891788in}}%
\pgfpathlineto{\pgfqpoint{0.920325in}{0.896366in}}%
\pgfpathlineto{\pgfqpoint{0.947137in}{0.896366in}}%
\pgfpathlineto{\pgfqpoint{0.949052in}{0.900765in}}%
\pgfpathlineto{\pgfqpoint{0.962458in}{0.900765in}}%
\pgfpathlineto{\pgfqpoint{0.964373in}{0.904998in}}%
\pgfpathlineto{\pgfqpoint{0.981610in}{0.904998in}}%
\pgfpathlineto{\pgfqpoint{0.983525in}{0.909076in}}%
\pgfpathlineto{\pgfqpoint{1.014168in}{0.909076in}}%
\pgfpathlineto{\pgfqpoint{1.016083in}{0.913012in}}%
\pgfpathlineto{\pgfqpoint{1.033320in}{0.913012in}}%
\pgfpathlineto{\pgfqpoint{1.035235in}{0.916814in}}%
\pgfpathlineto{\pgfqpoint{1.052471in}{0.916814in}}%
\pgfpathlineto{\pgfqpoint{1.054387in}{0.920491in}}%
\pgfpathlineto{\pgfqpoint{1.079284in}{0.920491in}}%
\pgfpathlineto{\pgfqpoint{1.081199in}{0.924052in}}%
\pgfpathlineto{\pgfqpoint{1.083114in}{0.924052in}}%
\pgfpathlineto{\pgfqpoint{1.085029in}{0.927503in}}%
\pgfpathlineto{\pgfqpoint{1.088860in}{0.927503in}}%
\pgfpathlineto{\pgfqpoint{1.090775in}{0.930851in}}%
\pgfpathlineto{\pgfqpoint{1.102266in}{0.930851in}}%
\pgfpathlineto{\pgfqpoint{1.104181in}{0.934101in}}%
\pgfpathlineto{\pgfqpoint{1.108011in}{0.934101in}}%
\pgfpathlineto{\pgfqpoint{1.109926in}{0.937261in}}%
\pgfpathlineto{\pgfqpoint{1.113757in}{0.937261in}}%
\pgfpathlineto{\pgfqpoint{1.115672in}{0.940334in}}%
\pgfpathlineto{\pgfqpoint{1.129078in}{0.940334in}}%
\pgfpathlineto{\pgfqpoint{1.130993in}{0.943324in}}%
\pgfpathlineto{\pgfqpoint{1.142484in}{0.943324in}}%
\pgfpathlineto{\pgfqpoint{1.144400in}{0.946237in}}%
\pgfpathlineto{\pgfqpoint{1.150145in}{0.946237in}}%
\pgfpathlineto{\pgfqpoint{1.152060in}{0.949077in}}%
\pgfpathlineto{\pgfqpoint{1.159721in}{0.949077in}}%
\pgfpathlineto{\pgfqpoint{1.161636in}{0.951846in}}%
\pgfpathlineto{\pgfqpoint{1.165466in}{0.962286in}}%
\pgfpathlineto{\pgfqpoint{1.167382in}{1.003526in}}%
\pgfpathlineto{\pgfqpoint{1.169297in}{1.016543in}}%
\pgfpathlineto{\pgfqpoint{1.171212in}{1.826535in}}%
\pgfpathlineto{\pgfqpoint{1.171212in}{1.826535in}}%
\pgfusepath{stroke}%
\end{pgfscope}%
\begin{pgfscope}%
\pgfpathrectangle{\pgfqpoint{0.694334in}{0.523557in}}{\pgfqpoint{3.830343in}{1.302977in}}%
\pgfusepath{clip}%
\pgfsetrectcap%
\pgfsetroundjoin%
\pgfsetlinewidth{1.003750pt}%
\definecolor{currentstroke}{rgb}{0.752941,0.752941,1.000000}%
\pgfsetstrokecolor{currentstroke}%
\pgfsetdash{}{0pt}%
\pgfpathmoveto{\pgfqpoint{0.694334in}{0.736317in}}%
\pgfpathlineto{\pgfqpoint{0.696249in}{0.753605in}}%
\pgfpathlineto{\pgfqpoint{0.736468in}{0.753605in}}%
\pgfpathlineto{\pgfqpoint{0.738383in}{0.768580in}}%
\pgfpathlineto{\pgfqpoint{0.765196in}{0.768580in}}%
\pgfpathlineto{\pgfqpoint{0.767111in}{0.781789in}}%
\pgfpathlineto{\pgfqpoint{0.784347in}{0.781789in}}%
\pgfpathlineto{\pgfqpoint{0.786263in}{0.793606in}}%
\pgfpathlineto{\pgfqpoint{0.807329in}{0.793606in}}%
\pgfpathlineto{\pgfqpoint{0.809245in}{0.804294in}}%
\pgfpathlineto{\pgfqpoint{0.828396in}{0.804294in}}%
\pgfpathlineto{\pgfqpoint{0.830311in}{0.814053in}}%
\pgfpathlineto{\pgfqpoint{0.853294in}{0.814053in}}%
\pgfpathlineto{\pgfqpoint{0.855209in}{0.823029in}}%
\pgfpathlineto{\pgfqpoint{0.878191in}{0.823029in}}%
\pgfpathlineto{\pgfqpoint{0.880106in}{0.831341in}}%
\pgfpathlineto{\pgfqpoint{0.893512in}{0.831341in}}%
\pgfpathlineto{\pgfqpoint{0.895427in}{0.839078in}}%
\pgfpathlineto{\pgfqpoint{0.924155in}{0.839078in}}%
\pgfpathlineto{\pgfqpoint{0.926070in}{0.846316in}}%
\pgfpathlineto{\pgfqpoint{0.960543in}{0.846316in}}%
\pgfpathlineto{\pgfqpoint{0.962458in}{0.853115in}}%
\pgfpathlineto{\pgfqpoint{0.993101in}{0.853115in}}%
\pgfpathlineto{\pgfqpoint{0.995016in}{0.859525in}}%
\pgfpathlineto{\pgfqpoint{1.014168in}{0.859525in}}%
\pgfpathlineto{\pgfqpoint{1.016083in}{0.865589in}}%
\pgfpathlineto{\pgfqpoint{1.046726in}{0.865589in}}%
\pgfpathlineto{\pgfqpoint{1.048641in}{0.871341in}}%
\pgfpathlineto{\pgfqpoint{1.065878in}{0.871341in}}%
\pgfpathlineto{\pgfqpoint{1.067793in}{0.876813in}}%
\pgfpathlineto{\pgfqpoint{1.085029in}{0.876813in}}%
\pgfpathlineto{\pgfqpoint{1.086944in}{0.882030in}}%
\pgfpathlineto{\pgfqpoint{1.115672in}{0.882030in}}%
\pgfpathlineto{\pgfqpoint{1.117587in}{0.887015in}}%
\pgfpathlineto{\pgfqpoint{1.142484in}{0.887015in}}%
\pgfpathlineto{\pgfqpoint{1.144400in}{0.891788in}}%
\pgfpathlineto{\pgfqpoint{1.155891in}{0.891788in}}%
\pgfpathlineto{\pgfqpoint{1.157806in}{0.896366in}}%
\pgfpathlineto{\pgfqpoint{1.171212in}{0.896366in}}%
\pgfpathlineto{\pgfqpoint{1.173127in}{0.900765in}}%
\pgfpathlineto{\pgfqpoint{1.186533in}{0.900765in}}%
\pgfpathlineto{\pgfqpoint{1.188449in}{0.904998in}}%
\pgfpathlineto{\pgfqpoint{1.199940in}{0.904998in}}%
\pgfpathlineto{\pgfqpoint{1.201855in}{0.909076in}}%
\pgfpathlineto{\pgfqpoint{1.222922in}{0.909076in}}%
\pgfpathlineto{\pgfqpoint{1.224837in}{0.913012in}}%
\pgfpathlineto{\pgfqpoint{1.236328in}{0.913012in}}%
\pgfpathlineto{\pgfqpoint{1.238243in}{0.916814in}}%
\pgfpathlineto{\pgfqpoint{1.261225in}{0.916814in}}%
\pgfpathlineto{\pgfqpoint{1.263140in}{0.920491in}}%
\pgfpathlineto{\pgfqpoint{1.282292in}{0.920491in}}%
\pgfpathlineto{\pgfqpoint{1.284207in}{0.924052in}}%
\pgfpathlineto{\pgfqpoint{1.305274in}{0.924052in}}%
\pgfpathlineto{\pgfqpoint{1.307189in}{0.927503in}}%
\pgfpathlineto{\pgfqpoint{1.322511in}{0.927503in}}%
\pgfpathlineto{\pgfqpoint{1.324426in}{0.930851in}}%
\pgfpathlineto{\pgfqpoint{1.335917in}{0.930851in}}%
\pgfpathlineto{\pgfqpoint{1.337832in}{0.934101in}}%
\pgfpathlineto{\pgfqpoint{1.341662in}{0.934101in}}%
\pgfpathlineto{\pgfqpoint{1.343577in}{0.937261in}}%
\pgfpathlineto{\pgfqpoint{1.351238in}{0.937261in}}%
\pgfpathlineto{\pgfqpoint{1.353153in}{0.940334in}}%
\pgfpathlineto{\pgfqpoint{1.360814in}{0.940334in}}%
\pgfpathlineto{\pgfqpoint{1.362729in}{0.943324in}}%
\pgfpathlineto{\pgfqpoint{1.370390in}{0.943324in}}%
\pgfpathlineto{\pgfqpoint{1.372305in}{0.946237in}}%
\pgfpathlineto{\pgfqpoint{1.378050in}{0.946237in}}%
\pgfpathlineto{\pgfqpoint{1.379966in}{0.949077in}}%
\pgfpathlineto{\pgfqpoint{1.383796in}{0.949077in}}%
\pgfpathlineto{\pgfqpoint{1.385711in}{0.951846in}}%
\pgfpathlineto{\pgfqpoint{1.391457in}{0.951846in}}%
\pgfpathlineto{\pgfqpoint{1.393372in}{0.954549in}}%
\pgfpathlineto{\pgfqpoint{1.395287in}{0.954549in}}%
\pgfpathlineto{\pgfqpoint{1.397202in}{0.957188in}}%
\pgfpathlineto{\pgfqpoint{1.401033in}{0.957188in}}%
\pgfpathlineto{\pgfqpoint{1.402948in}{0.959766in}}%
\pgfpathlineto{\pgfqpoint{1.404863in}{0.959766in}}%
\pgfpathlineto{\pgfqpoint{1.406778in}{0.962286in}}%
\pgfpathlineto{\pgfqpoint{1.418269in}{0.962286in}}%
\pgfpathlineto{\pgfqpoint{1.420184in}{0.964751in}}%
\pgfpathlineto{\pgfqpoint{1.422099in}{0.971836in}}%
\pgfpathlineto{\pgfqpoint{1.424015in}{0.971836in}}%
\pgfpathlineto{\pgfqpoint{1.425930in}{0.974102in}}%
\pgfpathlineto{\pgfqpoint{1.427845in}{0.974102in}}%
\pgfpathlineto{\pgfqpoint{1.429760in}{0.980637in}}%
\pgfpathlineto{\pgfqpoint{1.431675in}{1.826535in}}%
\pgfpathlineto{\pgfqpoint{1.431675in}{1.826535in}}%
\pgfusepath{stroke}%
\end{pgfscope}%
\begin{pgfscope}%
\pgfpathrectangle{\pgfqpoint{0.694334in}{0.523557in}}{\pgfqpoint{3.830343in}{1.302977in}}%
\pgfusepath{clip}%
\pgfsetrectcap%
\pgfsetroundjoin%
\pgfsetlinewidth{1.003750pt}%
\definecolor{currentstroke}{rgb}{0.752941,0.752941,1.000000}%
\pgfsetstrokecolor{currentstroke}%
\pgfsetdash{}{0pt}%
\pgfpathmoveto{\pgfqpoint{0.694334in}{0.753605in}}%
\pgfpathlineto{\pgfqpoint{0.696249in}{0.753605in}}%
\pgfpathlineto{\pgfqpoint{0.698165in}{0.753605in}}%
\pgfpathlineto{\pgfqpoint{0.700080in}{0.753605in}}%
\pgfpathlineto{\pgfqpoint{0.701995in}{0.753605in}}%
\pgfpathlineto{\pgfqpoint{0.703910in}{0.753605in}}%
\pgfpathlineto{\pgfqpoint{0.705825in}{0.768580in}}%
\pgfpathlineto{\pgfqpoint{0.707741in}{0.768580in}}%
\pgfpathlineto{\pgfqpoint{0.709656in}{0.768580in}}%
\pgfpathlineto{\pgfqpoint{0.711571in}{0.768580in}}%
\pgfpathlineto{\pgfqpoint{0.713486in}{0.768580in}}%
\pgfpathlineto{\pgfqpoint{0.715401in}{0.768580in}}%
\pgfpathlineto{\pgfqpoint{0.717316in}{0.781789in}}%
\pgfpathlineto{\pgfqpoint{0.719232in}{0.781789in}}%
\pgfpathlineto{\pgfqpoint{0.721147in}{0.781789in}}%
\pgfpathlineto{\pgfqpoint{0.723062in}{0.781789in}}%
\pgfpathlineto{\pgfqpoint{0.724977in}{0.781789in}}%
\pgfpathlineto{\pgfqpoint{0.726892in}{0.781789in}}%
\pgfpathlineto{\pgfqpoint{0.728807in}{0.793606in}}%
\pgfpathlineto{\pgfqpoint{0.730723in}{0.793606in}}%
\pgfpathlineto{\pgfqpoint{0.732638in}{0.793606in}}%
\pgfpathlineto{\pgfqpoint{0.734553in}{0.804294in}}%
\pgfpathlineto{\pgfqpoint{0.736468in}{0.804294in}}%
\pgfpathlineto{\pgfqpoint{0.738383in}{0.804294in}}%
\pgfpathlineto{\pgfqpoint{0.740298in}{0.804294in}}%
\pgfpathlineto{\pgfqpoint{0.742214in}{0.804294in}}%
\pgfpathlineto{\pgfqpoint{0.744129in}{0.814053in}}%
\pgfpathlineto{\pgfqpoint{0.746044in}{0.814053in}}%
\pgfpathlineto{\pgfqpoint{0.747959in}{0.823029in}}%
\pgfpathlineto{\pgfqpoint{0.749874in}{0.839078in}}%
\pgfpathlineto{\pgfqpoint{0.751789in}{0.846316in}}%
\pgfpathlineto{\pgfqpoint{0.753705in}{0.853115in}}%
\pgfpathlineto{\pgfqpoint{0.755620in}{0.853115in}}%
\pgfpathlineto{\pgfqpoint{0.757535in}{0.859525in}}%
\pgfpathlineto{\pgfqpoint{0.759450in}{0.871341in}}%
\pgfpathlineto{\pgfqpoint{0.761365in}{0.876813in}}%
\pgfpathlineto{\pgfqpoint{0.763280in}{0.876813in}}%
\pgfpathlineto{\pgfqpoint{0.765196in}{0.887015in}}%
\pgfpathlineto{\pgfqpoint{0.767111in}{0.896366in}}%
\pgfpathlineto{\pgfqpoint{0.769026in}{0.896366in}}%
\pgfpathlineto{\pgfqpoint{0.770941in}{1.826535in}}%
\pgfusepath{stroke}%
\end{pgfscope}%
\begin{pgfscope}%
\pgfpathrectangle{\pgfqpoint{0.694334in}{0.523557in}}{\pgfqpoint{3.830343in}{1.302977in}}%
\pgfusepath{clip}%
\pgfsetrectcap%
\pgfsetroundjoin%
\pgfsetlinewidth{1.003750pt}%
\definecolor{currentstroke}{rgb}{0.752941,0.752941,1.000000}%
\pgfsetstrokecolor{currentstroke}%
\pgfsetdash{}{0pt}%
\pgfpathmoveto{\pgfqpoint{0.694334in}{0.753605in}}%
\pgfpathlineto{\pgfqpoint{0.696249in}{0.753605in}}%
\pgfpathlineto{\pgfqpoint{0.698165in}{0.753605in}}%
\pgfpathlineto{\pgfqpoint{0.700080in}{0.753605in}}%
\pgfpathlineto{\pgfqpoint{0.701995in}{0.753605in}}%
\pgfpathlineto{\pgfqpoint{0.703910in}{0.753605in}}%
\pgfpathlineto{\pgfqpoint{0.705825in}{0.753605in}}%
\pgfpathlineto{\pgfqpoint{0.707741in}{0.768580in}}%
\pgfpathlineto{\pgfqpoint{0.709656in}{0.768580in}}%
\pgfpathlineto{\pgfqpoint{0.711571in}{0.768580in}}%
\pgfpathlineto{\pgfqpoint{0.713486in}{0.768580in}}%
\pgfpathlineto{\pgfqpoint{0.715401in}{0.768580in}}%
\pgfpathlineto{\pgfqpoint{0.717316in}{0.768580in}}%
\pgfpathlineto{\pgfqpoint{0.719232in}{0.768580in}}%
\pgfpathlineto{\pgfqpoint{0.721147in}{0.768580in}}%
\pgfpathlineto{\pgfqpoint{0.723062in}{0.768580in}}%
\pgfpathlineto{\pgfqpoint{0.724977in}{0.768580in}}%
\pgfpathlineto{\pgfqpoint{0.726892in}{0.781789in}}%
\pgfpathlineto{\pgfqpoint{0.728807in}{0.793606in}}%
\pgfpathlineto{\pgfqpoint{0.730723in}{0.793606in}}%
\pgfpathlineto{\pgfqpoint{0.732638in}{0.793606in}}%
\pgfpathlineto{\pgfqpoint{0.734553in}{0.793606in}}%
\pgfpathlineto{\pgfqpoint{0.736468in}{0.804294in}}%
\pgfpathlineto{\pgfqpoint{0.738383in}{0.804294in}}%
\pgfpathlineto{\pgfqpoint{0.740298in}{0.804294in}}%
\pgfpathlineto{\pgfqpoint{0.742214in}{0.814053in}}%
\pgfpathlineto{\pgfqpoint{0.744129in}{0.823029in}}%
\pgfpathlineto{\pgfqpoint{0.746044in}{0.823029in}}%
\pgfpathlineto{\pgfqpoint{0.747959in}{0.831341in}}%
\pgfpathlineto{\pgfqpoint{0.749874in}{0.853115in}}%
\pgfpathlineto{\pgfqpoint{0.751789in}{0.913012in}}%
\pgfpathlineto{\pgfqpoint{0.753705in}{0.943324in}}%
\pgfpathlineto{\pgfqpoint{0.755620in}{0.954549in}}%
\pgfpathlineto{\pgfqpoint{0.757535in}{0.954549in}}%
\pgfpathlineto{\pgfqpoint{0.759450in}{0.988797in}}%
\pgfpathlineto{\pgfqpoint{0.761365in}{0.988797in}}%
\pgfpathlineto{\pgfqpoint{0.763280in}{1.826535in}}%
\pgfusepath{stroke}%
\end{pgfscope}%
\begin{pgfscope}%
\pgfpathrectangle{\pgfqpoint{0.694334in}{0.523557in}}{\pgfqpoint{3.830343in}{1.302977in}}%
\pgfusepath{clip}%
\pgfsetrectcap%
\pgfsetroundjoin%
\pgfsetlinewidth{1.003750pt}%
\definecolor{currentstroke}{rgb}{0.752941,0.752941,1.000000}%
\pgfsetstrokecolor{currentstroke}%
\pgfsetdash{}{0pt}%
\pgfpathmoveto{\pgfqpoint{0.694334in}{0.753605in}}%
\pgfpathlineto{\pgfqpoint{0.724977in}{0.753605in}}%
\pgfpathlineto{\pgfqpoint{0.726892in}{0.768580in}}%
\pgfpathlineto{\pgfqpoint{0.772856in}{0.768580in}}%
\pgfpathlineto{\pgfqpoint{0.774772in}{0.781789in}}%
\pgfpathlineto{\pgfqpoint{0.807329in}{0.781789in}}%
\pgfpathlineto{\pgfqpoint{0.809245in}{0.793606in}}%
\pgfpathlineto{\pgfqpoint{0.837972in}{0.793606in}}%
\pgfpathlineto{\pgfqpoint{0.839887in}{0.804294in}}%
\pgfpathlineto{\pgfqpoint{0.876276in}{0.804294in}}%
\pgfpathlineto{\pgfqpoint{0.878191in}{0.814053in}}%
\pgfpathlineto{\pgfqpoint{0.901173in}{0.814053in}}%
\pgfpathlineto{\pgfqpoint{0.903088in}{0.823029in}}%
\pgfpathlineto{\pgfqpoint{0.937561in}{0.823029in}}%
\pgfpathlineto{\pgfqpoint{0.939476in}{0.831341in}}%
\pgfpathlineto{\pgfqpoint{0.968204in}{0.831341in}}%
\pgfpathlineto{\pgfqpoint{0.970119in}{0.839078in}}%
\pgfpathlineto{\pgfqpoint{0.998847in}{0.839078in}}%
\pgfpathlineto{\pgfqpoint{1.000762in}{0.846316in}}%
\pgfpathlineto{\pgfqpoint{1.040980in}{0.846316in}}%
\pgfpathlineto{\pgfqpoint{1.042895in}{0.853115in}}%
\pgfpathlineto{\pgfqpoint{1.069708in}{0.853115in}}%
\pgfpathlineto{\pgfqpoint{1.071623in}{0.859525in}}%
\pgfpathlineto{\pgfqpoint{1.098435in}{0.859525in}}%
\pgfpathlineto{\pgfqpoint{1.100351in}{0.865589in}}%
\pgfpathlineto{\pgfqpoint{1.136739in}{0.865589in}}%
\pgfpathlineto{\pgfqpoint{1.138654in}{0.871341in}}%
\pgfpathlineto{\pgfqpoint{1.159721in}{0.871341in}}%
\pgfpathlineto{\pgfqpoint{1.161636in}{0.876813in}}%
\pgfpathlineto{\pgfqpoint{1.182703in}{0.876813in}}%
\pgfpathlineto{\pgfqpoint{1.184618in}{0.882030in}}%
\pgfpathlineto{\pgfqpoint{1.192279in}{0.882030in}}%
\pgfpathlineto{\pgfqpoint{1.194194in}{0.887015in}}%
\pgfpathlineto{\pgfqpoint{1.207600in}{0.887015in}}%
\pgfpathlineto{\pgfqpoint{1.209515in}{0.891788in}}%
\pgfpathlineto{\pgfqpoint{1.222922in}{0.891788in}}%
\pgfpathlineto{\pgfqpoint{1.224837in}{0.896366in}}%
\pgfpathlineto{\pgfqpoint{1.238243in}{0.896366in}}%
\pgfpathlineto{\pgfqpoint{1.240158in}{0.900765in}}%
\pgfpathlineto{\pgfqpoint{1.251649in}{0.900765in}}%
\pgfpathlineto{\pgfqpoint{1.253564in}{0.904998in}}%
\pgfpathlineto{\pgfqpoint{1.268886in}{0.904998in}}%
\pgfpathlineto{\pgfqpoint{1.270801in}{0.909076in}}%
\pgfpathlineto{\pgfqpoint{1.284207in}{0.909076in}}%
\pgfpathlineto{\pgfqpoint{1.286122in}{0.913012in}}%
\pgfpathlineto{\pgfqpoint{1.293783in}{0.913012in}}%
\pgfpathlineto{\pgfqpoint{1.297613in}{0.927503in}}%
\pgfpathlineto{\pgfqpoint{1.299528in}{0.927503in}}%
\pgfpathlineto{\pgfqpoint{1.303359in}{0.976323in}}%
\pgfpathlineto{\pgfqpoint{1.305274in}{0.978501in}}%
\pgfpathlineto{\pgfqpoint{1.307189in}{0.986812in}}%
\pgfpathlineto{\pgfqpoint{1.309104in}{1.076863in}}%
\pgfpathlineto{\pgfqpoint{1.311019in}{1.078643in}}%
\pgfpathlineto{\pgfqpoint{1.312935in}{1.078643in}}%
\pgfpathlineto{\pgfqpoint{1.314850in}{1.082121in}}%
\pgfpathlineto{\pgfqpoint{1.316765in}{1.095805in}}%
\pgfpathlineto{\pgfqpoint{1.320595in}{1.103137in}}%
\pgfpathlineto{\pgfqpoint{1.322511in}{1.150951in}}%
\pgfpathlineto{\pgfqpoint{1.324426in}{1.163644in}}%
\pgfpathlineto{\pgfqpoint{1.326341in}{1.189412in}}%
\pgfpathlineto{\pgfqpoint{1.328256in}{1.826535in}}%
\pgfpathlineto{\pgfqpoint{1.328256in}{1.826535in}}%
\pgfusepath{stroke}%
\end{pgfscope}%
\begin{pgfscope}%
\pgfpathrectangle{\pgfqpoint{0.694334in}{0.523557in}}{\pgfqpoint{3.830343in}{1.302977in}}%
\pgfusepath{clip}%
\pgfsetrectcap%
\pgfsetroundjoin%
\pgfsetlinewidth{1.003750pt}%
\definecolor{currentstroke}{rgb}{0.752941,0.752941,1.000000}%
\pgfsetstrokecolor{currentstroke}%
\pgfsetdash{}{0pt}%
\pgfpathmoveto{\pgfqpoint{0.694334in}{0.736317in}}%
\pgfpathlineto{\pgfqpoint{0.696249in}{0.753605in}}%
\pgfpathlineto{\pgfqpoint{0.698165in}{0.753605in}}%
\pgfpathlineto{\pgfqpoint{0.700080in}{0.753605in}}%
\pgfpathlineto{\pgfqpoint{0.701995in}{0.753605in}}%
\pgfpathlineto{\pgfqpoint{0.703910in}{0.768580in}}%
\pgfpathlineto{\pgfqpoint{0.705825in}{0.768580in}}%
\pgfpathlineto{\pgfqpoint{0.707741in}{0.768580in}}%
\pgfpathlineto{\pgfqpoint{0.709656in}{0.768580in}}%
\pgfpathlineto{\pgfqpoint{0.711571in}{0.768580in}}%
\pgfpathlineto{\pgfqpoint{0.713486in}{0.768580in}}%
\pgfpathlineto{\pgfqpoint{0.715401in}{0.768580in}}%
\pgfpathlineto{\pgfqpoint{0.717316in}{0.781789in}}%
\pgfpathlineto{\pgfqpoint{0.719232in}{0.781789in}}%
\pgfpathlineto{\pgfqpoint{0.721147in}{0.781789in}}%
\pgfpathlineto{\pgfqpoint{0.723062in}{0.781789in}}%
\pgfpathlineto{\pgfqpoint{0.724977in}{0.781789in}}%
\pgfpathlineto{\pgfqpoint{0.726892in}{0.781789in}}%
\pgfpathlineto{\pgfqpoint{0.728807in}{0.781789in}}%
\pgfpathlineto{\pgfqpoint{0.730723in}{0.793606in}}%
\pgfpathlineto{\pgfqpoint{0.732638in}{0.793606in}}%
\pgfpathlineto{\pgfqpoint{0.734553in}{0.793606in}}%
\pgfpathlineto{\pgfqpoint{0.736468in}{0.804294in}}%
\pgfpathlineto{\pgfqpoint{0.738383in}{0.804294in}}%
\pgfpathlineto{\pgfqpoint{0.740298in}{0.804294in}}%
\pgfpathlineto{\pgfqpoint{0.742214in}{0.804294in}}%
\pgfpathlineto{\pgfqpoint{0.744129in}{0.804294in}}%
\pgfpathlineto{\pgfqpoint{0.746044in}{0.804294in}}%
\pgfpathlineto{\pgfqpoint{0.747959in}{0.823029in}}%
\pgfpathlineto{\pgfqpoint{0.749874in}{0.823029in}}%
\pgfpathlineto{\pgfqpoint{0.751789in}{0.823029in}}%
\pgfpathlineto{\pgfqpoint{0.753705in}{0.823029in}}%
\pgfpathlineto{\pgfqpoint{0.755620in}{0.859525in}}%
\pgfpathlineto{\pgfqpoint{0.757535in}{0.871341in}}%
\pgfpathlineto{\pgfqpoint{0.759450in}{0.871341in}}%
\pgfpathlineto{\pgfqpoint{0.761365in}{0.876813in}}%
\pgfpathlineto{\pgfqpoint{0.763280in}{1.032284in}}%
\pgfpathlineto{\pgfqpoint{0.765196in}{1.054059in}}%
\pgfpathlineto{\pgfqpoint{0.767111in}{1.826535in}}%
\pgfusepath{stroke}%
\end{pgfscope}%
\begin{pgfscope}%
\pgfpathrectangle{\pgfqpoint{0.694334in}{0.523557in}}{\pgfqpoint{3.830343in}{1.302977in}}%
\pgfusepath{clip}%
\pgfsetrectcap%
\pgfsetroundjoin%
\pgfsetlinewidth{1.003750pt}%
\definecolor{currentstroke}{rgb}{0.752941,0.752941,1.000000}%
\pgfsetstrokecolor{currentstroke}%
\pgfsetdash{}{0pt}%
\pgfpathmoveto{\pgfqpoint{0.694334in}{0.753605in}}%
\pgfpathlineto{\pgfqpoint{0.696249in}{0.753605in}}%
\pgfpathlineto{\pgfqpoint{0.698165in}{0.753605in}}%
\pgfpathlineto{\pgfqpoint{0.700080in}{0.753605in}}%
\pgfpathlineto{\pgfqpoint{0.701995in}{0.753605in}}%
\pgfpathlineto{\pgfqpoint{0.703910in}{0.753605in}}%
\pgfpathlineto{\pgfqpoint{0.705825in}{0.753605in}}%
\pgfpathlineto{\pgfqpoint{0.707741in}{0.753605in}}%
\pgfpathlineto{\pgfqpoint{0.709656in}{0.753605in}}%
\pgfpathlineto{\pgfqpoint{0.711571in}{0.753605in}}%
\pgfpathlineto{\pgfqpoint{0.713486in}{0.768580in}}%
\pgfpathlineto{\pgfqpoint{0.715401in}{0.768580in}}%
\pgfpathlineto{\pgfqpoint{0.717316in}{0.768580in}}%
\pgfpathlineto{\pgfqpoint{0.719232in}{0.831341in}}%
\pgfpathlineto{\pgfqpoint{0.721147in}{0.831341in}}%
\pgfpathlineto{\pgfqpoint{0.723062in}{0.853115in}}%
\pgfpathlineto{\pgfqpoint{0.724977in}{0.853115in}}%
\pgfpathlineto{\pgfqpoint{0.726892in}{0.853115in}}%
\pgfpathlineto{\pgfqpoint{0.728807in}{0.853115in}}%
\pgfpathlineto{\pgfqpoint{0.730723in}{0.876813in}}%
\pgfpathlineto{\pgfqpoint{0.732638in}{0.887015in}}%
\pgfpathlineto{\pgfqpoint{0.734553in}{0.891788in}}%
\pgfpathlineto{\pgfqpoint{0.736468in}{0.896366in}}%
\pgfpathlineto{\pgfqpoint{0.738383in}{0.904998in}}%
\pgfpathlineto{\pgfqpoint{0.740298in}{0.920491in}}%
\pgfpathlineto{\pgfqpoint{0.742214in}{1.826535in}}%
\pgfusepath{stroke}%
\end{pgfscope}%
\begin{pgfscope}%
\pgfpathrectangle{\pgfqpoint{0.694334in}{0.523557in}}{\pgfqpoint{3.830343in}{1.302977in}}%
\pgfusepath{clip}%
\pgfsetrectcap%
\pgfsetroundjoin%
\pgfsetlinewidth{1.003750pt}%
\definecolor{currentstroke}{rgb}{0.752941,0.752941,1.000000}%
\pgfsetstrokecolor{currentstroke}%
\pgfsetdash{}{0pt}%
\pgfpathmoveto{\pgfqpoint{0.694334in}{0.753605in}}%
\pgfpathlineto{\pgfqpoint{0.726892in}{0.753605in}}%
\pgfpathlineto{\pgfqpoint{0.728807in}{0.768580in}}%
\pgfpathlineto{\pgfqpoint{0.776687in}{0.768580in}}%
\pgfpathlineto{\pgfqpoint{0.778602in}{0.781789in}}%
\pgfpathlineto{\pgfqpoint{0.813075in}{0.781789in}}%
\pgfpathlineto{\pgfqpoint{0.814990in}{0.793606in}}%
\pgfpathlineto{\pgfqpoint{0.845633in}{0.793606in}}%
\pgfpathlineto{\pgfqpoint{0.847548in}{0.804294in}}%
\pgfpathlineto{\pgfqpoint{0.885851in}{0.804294in}}%
\pgfpathlineto{\pgfqpoint{0.887767in}{0.814053in}}%
\pgfpathlineto{\pgfqpoint{0.918409in}{0.814053in}}%
\pgfpathlineto{\pgfqpoint{0.920325in}{0.823029in}}%
\pgfpathlineto{\pgfqpoint{0.943307in}{0.823029in}}%
\pgfpathlineto{\pgfqpoint{0.945222in}{0.831341in}}%
\pgfpathlineto{\pgfqpoint{0.975864in}{0.831341in}}%
\pgfpathlineto{\pgfqpoint{0.977780in}{0.839078in}}%
\pgfpathlineto{\pgfqpoint{1.006507in}{0.839078in}}%
\pgfpathlineto{\pgfqpoint{1.008422in}{0.846316in}}%
\pgfpathlineto{\pgfqpoint{1.033320in}{0.846316in}}%
\pgfpathlineto{\pgfqpoint{1.035235in}{0.853115in}}%
\pgfpathlineto{\pgfqpoint{1.067793in}{0.853115in}}%
\pgfpathlineto{\pgfqpoint{1.069708in}{0.859525in}}%
\pgfpathlineto{\pgfqpoint{1.090775in}{0.859525in}}%
\pgfpathlineto{\pgfqpoint{1.092690in}{0.865589in}}%
\pgfpathlineto{\pgfqpoint{1.119502in}{0.865589in}}%
\pgfpathlineto{\pgfqpoint{1.121418in}{0.871341in}}%
\pgfpathlineto{\pgfqpoint{1.146315in}{0.871341in}}%
\pgfpathlineto{\pgfqpoint{1.148230in}{0.876813in}}%
\pgfpathlineto{\pgfqpoint{1.167382in}{0.876813in}}%
\pgfpathlineto{\pgfqpoint{1.169297in}{0.882030in}}%
\pgfpathlineto{\pgfqpoint{1.188449in}{0.882030in}}%
\pgfpathlineto{\pgfqpoint{1.190364in}{0.887015in}}%
\pgfpathlineto{\pgfqpoint{1.213346in}{0.887015in}}%
\pgfpathlineto{\pgfqpoint{1.215261in}{0.891788in}}%
\pgfpathlineto{\pgfqpoint{1.234413in}{0.891788in}}%
\pgfpathlineto{\pgfqpoint{1.236328in}{0.896366in}}%
\pgfpathlineto{\pgfqpoint{1.255480in}{0.896366in}}%
\pgfpathlineto{\pgfqpoint{1.257395in}{0.900765in}}%
\pgfpathlineto{\pgfqpoint{1.276546in}{0.900765in}}%
\pgfpathlineto{\pgfqpoint{1.278462in}{0.904998in}}%
\pgfpathlineto{\pgfqpoint{1.288037in}{0.904998in}}%
\pgfpathlineto{\pgfqpoint{1.289953in}{0.909076in}}%
\pgfpathlineto{\pgfqpoint{1.299528in}{0.909076in}}%
\pgfpathlineto{\pgfqpoint{1.301444in}{0.913012in}}%
\pgfpathlineto{\pgfqpoint{1.324426in}{0.913012in}}%
\pgfpathlineto{\pgfqpoint{1.326341in}{0.916814in}}%
\pgfpathlineto{\pgfqpoint{1.335917in}{0.916814in}}%
\pgfpathlineto{\pgfqpoint{1.337832in}{0.920491in}}%
\pgfpathlineto{\pgfqpoint{1.347408in}{0.920491in}}%
\pgfpathlineto{\pgfqpoint{1.349323in}{0.924052in}}%
\pgfpathlineto{\pgfqpoint{1.358899in}{0.924052in}}%
\pgfpathlineto{\pgfqpoint{1.360814in}{0.927503in}}%
\pgfpathlineto{\pgfqpoint{1.376135in}{0.927503in}}%
\pgfpathlineto{\pgfqpoint{1.378050in}{0.930851in}}%
\pgfpathlineto{\pgfqpoint{1.395287in}{0.930851in}}%
\pgfpathlineto{\pgfqpoint{1.397202in}{0.934101in}}%
\pgfpathlineto{\pgfqpoint{1.414439in}{0.934101in}}%
\pgfpathlineto{\pgfqpoint{1.416354in}{0.937261in}}%
\pgfpathlineto{\pgfqpoint{1.424015in}{0.937261in}}%
\pgfpathlineto{\pgfqpoint{1.425930in}{0.940334in}}%
\pgfpathlineto{\pgfqpoint{1.437421in}{0.940334in}}%
\pgfpathlineto{\pgfqpoint{1.439336in}{0.943324in}}%
\pgfpathlineto{\pgfqpoint{1.452742in}{0.943324in}}%
\pgfpathlineto{\pgfqpoint{1.454657in}{0.946237in}}%
\pgfpathlineto{\pgfqpoint{1.464233in}{0.946237in}}%
\pgfpathlineto{\pgfqpoint{1.468064in}{0.951846in}}%
\pgfpathlineto{\pgfqpoint{1.489130in}{0.951846in}}%
\pgfpathlineto{\pgfqpoint{1.491046in}{0.954549in}}%
\pgfpathlineto{\pgfqpoint{1.498706in}{0.954549in}}%
\pgfpathlineto{\pgfqpoint{1.500621in}{0.957188in}}%
\pgfpathlineto{\pgfqpoint{1.508282in}{0.957188in}}%
\pgfpathlineto{\pgfqpoint{1.510197in}{0.959766in}}%
\pgfpathlineto{\pgfqpoint{1.521688in}{0.959766in}}%
\pgfpathlineto{\pgfqpoint{1.523604in}{0.962286in}}%
\pgfpathlineto{\pgfqpoint{1.527434in}{0.962286in}}%
\pgfpathlineto{\pgfqpoint{1.529349in}{0.964751in}}%
\pgfpathlineto{\pgfqpoint{1.535095in}{0.964751in}}%
\pgfpathlineto{\pgfqpoint{1.537010in}{0.967163in}}%
\pgfpathlineto{\pgfqpoint{1.542755in}{0.967163in}}%
\pgfpathlineto{\pgfqpoint{1.544670in}{0.969524in}}%
\pgfpathlineto{\pgfqpoint{1.554246in}{0.969524in}}%
\pgfpathlineto{\pgfqpoint{1.556161in}{0.971836in}}%
\pgfpathlineto{\pgfqpoint{1.558077in}{0.971836in}}%
\pgfpathlineto{\pgfqpoint{1.559992in}{0.974102in}}%
\pgfpathlineto{\pgfqpoint{1.563822in}{0.974102in}}%
\pgfpathlineto{\pgfqpoint{1.565737in}{0.976323in}}%
\pgfpathlineto{\pgfqpoint{1.567652in}{0.976323in}}%
\pgfpathlineto{\pgfqpoint{1.569568in}{0.978501in}}%
\pgfpathlineto{\pgfqpoint{1.573398in}{0.978501in}}%
\pgfpathlineto{\pgfqpoint{1.575313in}{0.980637in}}%
\pgfpathlineto{\pgfqpoint{1.577228in}{0.984791in}}%
\pgfpathlineto{\pgfqpoint{1.582974in}{0.984791in}}%
\pgfpathlineto{\pgfqpoint{1.584889in}{0.988797in}}%
\pgfpathlineto{\pgfqpoint{1.592550in}{0.988797in}}%
\pgfpathlineto{\pgfqpoint{1.594465in}{0.990747in}}%
\pgfpathlineto{\pgfqpoint{1.598295in}{0.990747in}}%
\pgfpathlineto{\pgfqpoint{1.600210in}{0.992664in}}%
\pgfpathlineto{\pgfqpoint{1.604041in}{0.992664in}}%
\pgfpathlineto{\pgfqpoint{1.605956in}{0.994549in}}%
\pgfpathlineto{\pgfqpoint{1.607871in}{0.994549in}}%
\pgfpathlineto{\pgfqpoint{1.611701in}{1.000021in}}%
\pgfpathlineto{\pgfqpoint{1.613617in}{1.001787in}}%
\pgfpathlineto{\pgfqpoint{1.615532in}{1.005238in}}%
\pgfpathlineto{\pgfqpoint{1.617447in}{1.005238in}}%
\pgfpathlineto{\pgfqpoint{1.625108in}{1.016543in}}%
\pgfpathlineto{\pgfqpoint{1.627023in}{1.025402in}}%
\pgfpathlineto{\pgfqpoint{1.628938in}{1.025402in}}%
\pgfpathlineto{\pgfqpoint{1.630853in}{1.064547in}}%
\pgfpathlineto{\pgfqpoint{1.634683in}{1.068483in}}%
\pgfpathlineto{\pgfqpoint{1.636599in}{1.074139in}}%
\pgfpathlineto{\pgfqpoint{1.638514in}{1.087143in}}%
\pgfpathlineto{\pgfqpoint{1.640429in}{1.110686in}}%
\pgfpathlineto{\pgfqpoint{1.642344in}{1.169686in}}%
\pgfpathlineto{\pgfqpoint{1.646174in}{1.203023in}}%
\pgfpathlineto{\pgfqpoint{1.650005in}{1.220767in}}%
\pgfpathlineto{\pgfqpoint{1.651920in}{1.826535in}}%
\pgfpathlineto{\pgfqpoint{1.651920in}{1.826535in}}%
\pgfusepath{stroke}%
\end{pgfscope}%
\begin{pgfscope}%
\pgfpathrectangle{\pgfqpoint{0.694334in}{0.523557in}}{\pgfqpoint{3.830343in}{1.302977in}}%
\pgfusepath{clip}%
\pgfsetrectcap%
\pgfsetroundjoin%
\pgfsetlinewidth{1.003750pt}%
\definecolor{currentstroke}{rgb}{0.752941,0.752941,1.000000}%
\pgfsetstrokecolor{currentstroke}%
\pgfsetdash{}{0pt}%
\pgfpathmoveto{\pgfqpoint{0.694334in}{0.753605in}}%
\pgfpathlineto{\pgfqpoint{0.696249in}{0.753605in}}%
\pgfpathlineto{\pgfqpoint{0.698165in}{0.753605in}}%
\pgfpathlineto{\pgfqpoint{0.700080in}{0.753605in}}%
\pgfpathlineto{\pgfqpoint{0.701995in}{0.753605in}}%
\pgfpathlineto{\pgfqpoint{0.703910in}{0.753605in}}%
\pgfpathlineto{\pgfqpoint{0.705825in}{0.753605in}}%
\pgfpathlineto{\pgfqpoint{0.707741in}{0.753605in}}%
\pgfpathlineto{\pgfqpoint{0.709656in}{0.753605in}}%
\pgfpathlineto{\pgfqpoint{0.711571in}{0.753605in}}%
\pgfpathlineto{\pgfqpoint{0.713486in}{0.753605in}}%
\pgfpathlineto{\pgfqpoint{0.715401in}{0.753605in}}%
\pgfpathlineto{\pgfqpoint{0.717316in}{0.753605in}}%
\pgfpathlineto{\pgfqpoint{0.719232in}{0.753605in}}%
\pgfpathlineto{\pgfqpoint{0.721147in}{0.753605in}}%
\pgfpathlineto{\pgfqpoint{0.723062in}{0.753605in}}%
\pgfpathlineto{\pgfqpoint{0.724977in}{0.753605in}}%
\pgfpathlineto{\pgfqpoint{0.726892in}{0.753605in}}%
\pgfpathlineto{\pgfqpoint{0.728807in}{0.753605in}}%
\pgfpathlineto{\pgfqpoint{0.730723in}{0.753605in}}%
\pgfpathlineto{\pgfqpoint{0.732638in}{0.753605in}}%
\pgfpathlineto{\pgfqpoint{0.734553in}{0.753605in}}%
\pgfpathlineto{\pgfqpoint{0.736468in}{0.768580in}}%
\pgfpathlineto{\pgfqpoint{0.738383in}{0.768580in}}%
\pgfpathlineto{\pgfqpoint{0.740298in}{0.768580in}}%
\pgfpathlineto{\pgfqpoint{0.742214in}{0.768580in}}%
\pgfpathlineto{\pgfqpoint{0.744129in}{0.768580in}}%
\pgfpathlineto{\pgfqpoint{0.746044in}{0.768580in}}%
\pgfpathlineto{\pgfqpoint{0.747959in}{0.768580in}}%
\pgfpathlineto{\pgfqpoint{0.749874in}{0.768580in}}%
\pgfpathlineto{\pgfqpoint{0.751789in}{1.826535in}}%
\pgfusepath{stroke}%
\end{pgfscope}%
\begin{pgfscope}%
\pgfpathrectangle{\pgfqpoint{0.694334in}{0.523557in}}{\pgfqpoint{3.830343in}{1.302977in}}%
\pgfusepath{clip}%
\pgfsetrectcap%
\pgfsetroundjoin%
\pgfsetlinewidth{1.003750pt}%
\definecolor{currentstroke}{rgb}{0.752941,0.752941,1.000000}%
\pgfsetstrokecolor{currentstroke}%
\pgfsetdash{}{0pt}%
\pgfpathmoveto{\pgfqpoint{0.694334in}{0.736317in}}%
\pgfpathlineto{\pgfqpoint{0.696249in}{0.753605in}}%
\pgfpathlineto{\pgfqpoint{0.730723in}{0.753605in}}%
\pgfpathlineto{\pgfqpoint{0.732638in}{0.768580in}}%
\pgfpathlineto{\pgfqpoint{0.782432in}{0.768580in}}%
\pgfpathlineto{\pgfqpoint{0.784347in}{0.781789in}}%
\pgfpathlineto{\pgfqpoint{0.814990in}{0.781789in}}%
\pgfpathlineto{\pgfqpoint{0.816905in}{0.793606in}}%
\pgfpathlineto{\pgfqpoint{0.837972in}{0.793606in}}%
\pgfpathlineto{\pgfqpoint{0.839887in}{0.804294in}}%
\pgfpathlineto{\pgfqpoint{0.880106in}{0.804294in}}%
\pgfpathlineto{\pgfqpoint{0.882021in}{0.814053in}}%
\pgfpathlineto{\pgfqpoint{0.908834in}{0.814053in}}%
\pgfpathlineto{\pgfqpoint{0.910749in}{0.823029in}}%
\pgfpathlineto{\pgfqpoint{0.939476in}{0.823029in}}%
\pgfpathlineto{\pgfqpoint{0.941391in}{0.831341in}}%
\pgfpathlineto{\pgfqpoint{0.972034in}{0.831341in}}%
\pgfpathlineto{\pgfqpoint{0.973949in}{0.839078in}}%
\pgfpathlineto{\pgfqpoint{1.012253in}{0.839078in}}%
\pgfpathlineto{\pgfqpoint{1.014168in}{0.846316in}}%
\pgfpathlineto{\pgfqpoint{1.056302in}{0.846316in}}%
\pgfpathlineto{\pgfqpoint{1.058217in}{0.853115in}}%
\pgfpathlineto{\pgfqpoint{1.083114in}{0.853115in}}%
\pgfpathlineto{\pgfqpoint{1.085029in}{0.859525in}}%
\pgfpathlineto{\pgfqpoint{1.108011in}{0.859525in}}%
\pgfpathlineto{\pgfqpoint{1.109926in}{0.865589in}}%
\pgfpathlineto{\pgfqpoint{1.148230in}{0.865589in}}%
\pgfpathlineto{\pgfqpoint{1.150145in}{0.871341in}}%
\pgfpathlineto{\pgfqpoint{1.175042in}{0.871341in}}%
\pgfpathlineto{\pgfqpoint{1.176957in}{0.876813in}}%
\pgfpathlineto{\pgfqpoint{1.194194in}{0.876813in}}%
\pgfpathlineto{\pgfqpoint{1.196109in}{0.882030in}}%
\pgfpathlineto{\pgfqpoint{1.205685in}{0.882030in}}%
\pgfpathlineto{\pgfqpoint{1.207600in}{0.887015in}}%
\pgfpathlineto{\pgfqpoint{1.215261in}{0.887015in}}%
\pgfpathlineto{\pgfqpoint{1.217176in}{0.891788in}}%
\pgfpathlineto{\pgfqpoint{1.228667in}{0.891788in}}%
\pgfpathlineto{\pgfqpoint{1.230582in}{0.896366in}}%
\pgfpathlineto{\pgfqpoint{1.249734in}{0.896366in}}%
\pgfpathlineto{\pgfqpoint{1.251649in}{0.900765in}}%
\pgfpathlineto{\pgfqpoint{1.261225in}{0.900765in}}%
\pgfpathlineto{\pgfqpoint{1.263140in}{0.904998in}}%
\pgfpathlineto{\pgfqpoint{1.278462in}{0.904998in}}%
\pgfpathlineto{\pgfqpoint{1.282292in}{0.913012in}}%
\pgfpathlineto{\pgfqpoint{1.288037in}{0.913012in}}%
\pgfpathlineto{\pgfqpoint{1.291868in}{0.920491in}}%
\pgfpathlineto{\pgfqpoint{1.295698in}{0.930851in}}%
\pgfpathlineto{\pgfqpoint{1.297613in}{0.934101in}}%
\pgfpathlineto{\pgfqpoint{1.299528in}{0.940334in}}%
\pgfpathlineto{\pgfqpoint{1.307189in}{0.990747in}}%
\pgfpathlineto{\pgfqpoint{1.309104in}{1.091164in}}%
\pgfpathlineto{\pgfqpoint{1.311019in}{1.092732in}}%
\pgfpathlineto{\pgfqpoint{1.314850in}{1.100262in}}%
\pgfpathlineto{\pgfqpoint{1.318680in}{1.105941in}}%
\pgfpathlineto{\pgfqpoint{1.320595in}{1.112659in}}%
\pgfpathlineto{\pgfqpoint{1.326341in}{1.186750in}}%
\pgfpathlineto{\pgfqpoint{1.328256in}{1.826535in}}%
\pgfpathlineto{\pgfqpoint{1.328256in}{1.826535in}}%
\pgfusepath{stroke}%
\end{pgfscope}%
\begin{pgfscope}%
\pgfpathrectangle{\pgfqpoint{0.694334in}{0.523557in}}{\pgfqpoint{3.830343in}{1.302977in}}%
\pgfusepath{clip}%
\pgfsetrectcap%
\pgfsetroundjoin%
\pgfsetlinewidth{1.003750pt}%
\definecolor{currentstroke}{rgb}{0.752941,0.752941,1.000000}%
\pgfsetstrokecolor{currentstroke}%
\pgfsetdash{}{0pt}%
\pgfpathmoveto{\pgfqpoint{0.694334in}{0.736317in}}%
\pgfpathlineto{\pgfqpoint{0.696249in}{0.736317in}}%
\pgfpathlineto{\pgfqpoint{0.698165in}{0.753605in}}%
\pgfpathlineto{\pgfqpoint{0.728807in}{0.753605in}}%
\pgfpathlineto{\pgfqpoint{0.730723in}{0.768580in}}%
\pgfpathlineto{\pgfqpoint{0.772856in}{0.768580in}}%
\pgfpathlineto{\pgfqpoint{0.774772in}{0.781789in}}%
\pgfpathlineto{\pgfqpoint{0.824566in}{0.781789in}}%
\pgfpathlineto{\pgfqpoint{0.826481in}{0.793606in}}%
\pgfpathlineto{\pgfqpoint{0.853294in}{0.793606in}}%
\pgfpathlineto{\pgfqpoint{0.855209in}{0.804294in}}%
\pgfpathlineto{\pgfqpoint{0.883936in}{0.804294in}}%
\pgfpathlineto{\pgfqpoint{0.885851in}{0.814053in}}%
\pgfpathlineto{\pgfqpoint{0.924155in}{0.814053in}}%
\pgfpathlineto{\pgfqpoint{0.926070in}{0.823029in}}%
\pgfpathlineto{\pgfqpoint{0.947137in}{0.823029in}}%
\pgfpathlineto{\pgfqpoint{0.949052in}{0.831341in}}%
\pgfpathlineto{\pgfqpoint{0.979695in}{0.831341in}}%
\pgfpathlineto{\pgfqpoint{0.981610in}{0.839078in}}%
\pgfpathlineto{\pgfqpoint{1.006507in}{0.839078in}}%
\pgfpathlineto{\pgfqpoint{1.008422in}{0.846316in}}%
\pgfpathlineto{\pgfqpoint{1.027574in}{0.846316in}}%
\pgfpathlineto{\pgfqpoint{1.029489in}{0.853115in}}%
\pgfpathlineto{\pgfqpoint{1.060132in}{0.853115in}}%
\pgfpathlineto{\pgfqpoint{1.062047in}{0.859525in}}%
\pgfpathlineto{\pgfqpoint{1.086944in}{0.859525in}}%
\pgfpathlineto{\pgfqpoint{1.088860in}{0.865589in}}%
\pgfpathlineto{\pgfqpoint{1.113757in}{0.865589in}}%
\pgfpathlineto{\pgfqpoint{1.115672in}{0.871341in}}%
\pgfpathlineto{\pgfqpoint{1.140569in}{0.871341in}}%
\pgfpathlineto{\pgfqpoint{1.142484in}{0.876813in}}%
\pgfpathlineto{\pgfqpoint{1.157806in}{0.876813in}}%
\pgfpathlineto{\pgfqpoint{1.159721in}{0.882030in}}%
\pgfpathlineto{\pgfqpoint{1.178873in}{0.882030in}}%
\pgfpathlineto{\pgfqpoint{1.180788in}{0.887015in}}%
\pgfpathlineto{\pgfqpoint{1.198024in}{0.887015in}}%
\pgfpathlineto{\pgfqpoint{1.199940in}{0.891788in}}%
\pgfpathlineto{\pgfqpoint{1.228667in}{0.891788in}}%
\pgfpathlineto{\pgfqpoint{1.230582in}{0.896366in}}%
\pgfpathlineto{\pgfqpoint{1.243988in}{0.896366in}}%
\pgfpathlineto{\pgfqpoint{1.245904in}{0.900765in}}%
\pgfpathlineto{\pgfqpoint{1.259310in}{0.900765in}}%
\pgfpathlineto{\pgfqpoint{1.261225in}{0.904998in}}%
\pgfpathlineto{\pgfqpoint{1.274631in}{0.904998in}}%
\pgfpathlineto{\pgfqpoint{1.276546in}{0.909076in}}%
\pgfpathlineto{\pgfqpoint{1.286122in}{0.909076in}}%
\pgfpathlineto{\pgfqpoint{1.288037in}{0.913012in}}%
\pgfpathlineto{\pgfqpoint{1.293783in}{0.913012in}}%
\pgfpathlineto{\pgfqpoint{1.295698in}{0.916814in}}%
\pgfpathlineto{\pgfqpoint{1.305274in}{0.916814in}}%
\pgfpathlineto{\pgfqpoint{1.307189in}{0.920491in}}%
\pgfpathlineto{\pgfqpoint{1.332086in}{0.920491in}}%
\pgfpathlineto{\pgfqpoint{1.334002in}{0.924052in}}%
\pgfpathlineto{\pgfqpoint{1.345493in}{0.924052in}}%
\pgfpathlineto{\pgfqpoint{1.347408in}{0.927503in}}%
\pgfpathlineto{\pgfqpoint{1.358899in}{0.927503in}}%
\pgfpathlineto{\pgfqpoint{1.360814in}{0.930851in}}%
\pgfpathlineto{\pgfqpoint{1.368475in}{0.930851in}}%
\pgfpathlineto{\pgfqpoint{1.370390in}{0.934101in}}%
\pgfpathlineto{\pgfqpoint{1.381881in}{0.934101in}}%
\pgfpathlineto{\pgfqpoint{1.383796in}{0.937261in}}%
\pgfpathlineto{\pgfqpoint{1.395287in}{0.937261in}}%
\pgfpathlineto{\pgfqpoint{1.397202in}{0.940334in}}%
\pgfpathlineto{\pgfqpoint{1.410608in}{0.940334in}}%
\pgfpathlineto{\pgfqpoint{1.412524in}{0.943324in}}%
\pgfpathlineto{\pgfqpoint{1.418269in}{0.943324in}}%
\pgfpathlineto{\pgfqpoint{1.420184in}{0.946237in}}%
\pgfpathlineto{\pgfqpoint{1.433590in}{0.946237in}}%
\pgfpathlineto{\pgfqpoint{1.435506in}{0.949077in}}%
\pgfpathlineto{\pgfqpoint{1.446997in}{0.949077in}}%
\pgfpathlineto{\pgfqpoint{1.448912in}{0.951846in}}%
\pgfpathlineto{\pgfqpoint{1.454657in}{0.951846in}}%
\pgfpathlineto{\pgfqpoint{1.456573in}{0.954549in}}%
\pgfpathlineto{\pgfqpoint{1.462318in}{0.954549in}}%
\pgfpathlineto{\pgfqpoint{1.464233in}{0.957188in}}%
\pgfpathlineto{\pgfqpoint{1.468064in}{0.957188in}}%
\pgfpathlineto{\pgfqpoint{1.469979in}{0.959766in}}%
\pgfpathlineto{\pgfqpoint{1.477639in}{0.959766in}}%
\pgfpathlineto{\pgfqpoint{1.479555in}{0.962286in}}%
\pgfpathlineto{\pgfqpoint{1.491046in}{0.962286in}}%
\pgfpathlineto{\pgfqpoint{1.492961in}{0.964751in}}%
\pgfpathlineto{\pgfqpoint{1.502537in}{0.964751in}}%
\pgfpathlineto{\pgfqpoint{1.504452in}{0.967163in}}%
\pgfpathlineto{\pgfqpoint{1.512112in}{0.967163in}}%
\pgfpathlineto{\pgfqpoint{1.514028in}{0.969524in}}%
\pgfpathlineto{\pgfqpoint{1.523604in}{0.969524in}}%
\pgfpathlineto{\pgfqpoint{1.525519in}{0.971836in}}%
\pgfpathlineto{\pgfqpoint{1.529349in}{0.971836in}}%
\pgfpathlineto{\pgfqpoint{1.531264in}{0.974102in}}%
\pgfpathlineto{\pgfqpoint{1.535095in}{0.974102in}}%
\pgfpathlineto{\pgfqpoint{1.537010in}{0.976323in}}%
\pgfpathlineto{\pgfqpoint{1.540840in}{0.976323in}}%
\pgfpathlineto{\pgfqpoint{1.542755in}{0.978501in}}%
\pgfpathlineto{\pgfqpoint{1.544670in}{0.978501in}}%
\pgfpathlineto{\pgfqpoint{1.546586in}{0.980637in}}%
\pgfpathlineto{\pgfqpoint{1.548501in}{0.980637in}}%
\pgfpathlineto{\pgfqpoint{1.550416in}{0.982733in}}%
\pgfpathlineto{\pgfqpoint{1.558077in}{0.982733in}}%
\pgfpathlineto{\pgfqpoint{1.559992in}{0.984791in}}%
\pgfpathlineto{\pgfqpoint{1.563822in}{0.984791in}}%
\pgfpathlineto{\pgfqpoint{1.565737in}{0.986812in}}%
\pgfpathlineto{\pgfqpoint{1.571483in}{0.986812in}}%
\pgfpathlineto{\pgfqpoint{1.575313in}{0.990747in}}%
\pgfpathlineto{\pgfqpoint{1.577228in}{0.990747in}}%
\pgfpathlineto{\pgfqpoint{1.581059in}{0.994549in}}%
\pgfpathlineto{\pgfqpoint{1.584889in}{0.994549in}}%
\pgfpathlineto{\pgfqpoint{1.586804in}{0.996403in}}%
\pgfpathlineto{\pgfqpoint{1.588719in}{0.996403in}}%
\pgfpathlineto{\pgfqpoint{1.594465in}{1.001787in}}%
\pgfpathlineto{\pgfqpoint{1.598295in}{1.001787in}}%
\pgfpathlineto{\pgfqpoint{1.602126in}{1.005238in}}%
\pgfpathlineto{\pgfqpoint{1.604041in}{1.005238in}}%
\pgfpathlineto{\pgfqpoint{1.607871in}{1.008586in}}%
\pgfpathlineto{\pgfqpoint{1.609786in}{1.008586in}}%
\pgfpathlineto{\pgfqpoint{1.611701in}{1.010223in}}%
\pgfpathlineto{\pgfqpoint{1.613617in}{1.010223in}}%
\pgfpathlineto{\pgfqpoint{1.615532in}{1.011837in}}%
\pgfpathlineto{\pgfqpoint{1.617447in}{1.011837in}}%
\pgfpathlineto{\pgfqpoint{1.619362in}{1.013428in}}%
\pgfpathlineto{\pgfqpoint{1.623192in}{1.022526in}}%
\pgfpathlineto{\pgfqpoint{1.627023in}{1.032284in}}%
\pgfpathlineto{\pgfqpoint{1.632768in}{1.087959in}}%
\pgfpathlineto{\pgfqpoint{1.636599in}{1.092732in}}%
\pgfpathlineto{\pgfqpoint{1.638514in}{1.105941in}}%
\pgfpathlineto{\pgfqpoint{1.650005in}{1.258963in}}%
\pgfpathlineto{\pgfqpoint{1.651920in}{1.826535in}}%
\pgfpathlineto{\pgfqpoint{1.651920in}{1.826535in}}%
\pgfusepath{stroke}%
\end{pgfscope}%
\begin{pgfscope}%
\pgfpathrectangle{\pgfqpoint{0.694334in}{0.523557in}}{\pgfqpoint{3.830343in}{1.302977in}}%
\pgfusepath{clip}%
\pgfsetrectcap%
\pgfsetroundjoin%
\pgfsetlinewidth{1.003750pt}%
\definecolor{currentstroke}{rgb}{0.752941,0.752941,1.000000}%
\pgfsetstrokecolor{currentstroke}%
\pgfsetdash{}{0pt}%
\pgfpathmoveto{\pgfqpoint{0.694334in}{0.753605in}}%
\pgfpathlineto{\pgfqpoint{0.696249in}{0.753605in}}%
\pgfpathlineto{\pgfqpoint{0.698165in}{0.753605in}}%
\pgfpathlineto{\pgfqpoint{0.700080in}{0.753605in}}%
\pgfpathlineto{\pgfqpoint{0.701995in}{0.768580in}}%
\pgfpathlineto{\pgfqpoint{0.703910in}{0.768580in}}%
\pgfpathlineto{\pgfqpoint{0.705825in}{0.768580in}}%
\pgfpathlineto{\pgfqpoint{0.707741in}{0.768580in}}%
\pgfpathlineto{\pgfqpoint{0.709656in}{0.768580in}}%
\pgfpathlineto{\pgfqpoint{0.711571in}{0.768580in}}%
\pgfpathlineto{\pgfqpoint{0.713486in}{0.768580in}}%
\pgfpathlineto{\pgfqpoint{0.715401in}{0.768580in}}%
\pgfpathlineto{\pgfqpoint{0.717316in}{0.768580in}}%
\pgfpathlineto{\pgfqpoint{0.719232in}{0.768580in}}%
\pgfpathlineto{\pgfqpoint{0.721147in}{0.768580in}}%
\pgfpathlineto{\pgfqpoint{0.723062in}{0.781789in}}%
\pgfpathlineto{\pgfqpoint{0.724977in}{0.781789in}}%
\pgfpathlineto{\pgfqpoint{0.726892in}{0.793606in}}%
\pgfpathlineto{\pgfqpoint{0.728807in}{0.793606in}}%
\pgfpathlineto{\pgfqpoint{0.730723in}{0.793606in}}%
\pgfpathlineto{\pgfqpoint{0.732638in}{0.793606in}}%
\pgfpathlineto{\pgfqpoint{0.734553in}{0.793606in}}%
\pgfpathlineto{\pgfqpoint{0.736468in}{0.804294in}}%
\pgfpathlineto{\pgfqpoint{0.738383in}{0.804294in}}%
\pgfpathlineto{\pgfqpoint{0.740298in}{0.804294in}}%
\pgfpathlineto{\pgfqpoint{0.742214in}{0.804294in}}%
\pgfpathlineto{\pgfqpoint{0.744129in}{0.814053in}}%
\pgfpathlineto{\pgfqpoint{0.746044in}{0.814053in}}%
\pgfpathlineto{\pgfqpoint{0.747959in}{0.823029in}}%
\pgfpathlineto{\pgfqpoint{0.749874in}{0.823029in}}%
\pgfpathlineto{\pgfqpoint{0.751789in}{0.831341in}}%
\pgfpathlineto{\pgfqpoint{0.753705in}{0.831341in}}%
\pgfpathlineto{\pgfqpoint{0.755620in}{0.865589in}}%
\pgfpathlineto{\pgfqpoint{0.757535in}{0.865589in}}%
\pgfpathlineto{\pgfqpoint{0.759450in}{0.882030in}}%
\pgfpathlineto{\pgfqpoint{0.761365in}{0.896366in}}%
\pgfpathlineto{\pgfqpoint{0.763280in}{1.023973in}}%
\pgfpathlineto{\pgfqpoint{0.765196in}{1.062527in}}%
\pgfpathlineto{\pgfqpoint{0.767111in}{1.826535in}}%
\pgfusepath{stroke}%
\end{pgfscope}%
\begin{pgfscope}%
\pgfpathrectangle{\pgfqpoint{0.694334in}{0.523557in}}{\pgfqpoint{3.830343in}{1.302977in}}%
\pgfusepath{clip}%
\pgfsetrectcap%
\pgfsetroundjoin%
\pgfsetlinewidth{1.003750pt}%
\definecolor{currentstroke}{rgb}{0.752941,0.752941,1.000000}%
\pgfsetstrokecolor{currentstroke}%
\pgfsetdash{}{0pt}%
\pgfpathmoveto{\pgfqpoint{0.694334in}{0.823029in}}%
\pgfpathlineto{\pgfqpoint{0.696249in}{0.831341in}}%
\pgfpathlineto{\pgfqpoint{0.698165in}{0.839078in}}%
\pgfpathlineto{\pgfqpoint{0.700080in}{0.846316in}}%
\pgfpathlineto{\pgfqpoint{0.701995in}{0.846316in}}%
\pgfpathlineto{\pgfqpoint{0.703910in}{0.846316in}}%
\pgfpathlineto{\pgfqpoint{0.705825in}{0.853115in}}%
\pgfpathlineto{\pgfqpoint{0.707741in}{0.853115in}}%
\pgfpathlineto{\pgfqpoint{0.709656in}{0.853115in}}%
\pgfpathlineto{\pgfqpoint{0.711571in}{0.859525in}}%
\pgfpathlineto{\pgfqpoint{0.713486in}{0.859525in}}%
\pgfpathlineto{\pgfqpoint{0.715401in}{0.859525in}}%
\pgfpathlineto{\pgfqpoint{0.717316in}{0.859525in}}%
\pgfpathlineto{\pgfqpoint{0.719232in}{0.859525in}}%
\pgfpathlineto{\pgfqpoint{0.721147in}{0.865589in}}%
\pgfpathlineto{\pgfqpoint{0.723062in}{0.865589in}}%
\pgfpathlineto{\pgfqpoint{0.724977in}{0.871341in}}%
\pgfpathlineto{\pgfqpoint{0.726892in}{0.887015in}}%
\pgfpathlineto{\pgfqpoint{0.728807in}{0.891788in}}%
\pgfpathlineto{\pgfqpoint{0.730723in}{0.896366in}}%
\pgfpathlineto{\pgfqpoint{0.732638in}{0.896366in}}%
\pgfpathlineto{\pgfqpoint{0.734553in}{0.896366in}}%
\pgfpathlineto{\pgfqpoint{0.736468in}{0.900765in}}%
\pgfpathlineto{\pgfqpoint{0.738383in}{0.904998in}}%
\pgfpathlineto{\pgfqpoint{0.740298in}{1.826535in}}%
\pgfusepath{stroke}%
\end{pgfscope}%
\begin{pgfscope}%
\pgfpathrectangle{\pgfqpoint{0.694334in}{0.523557in}}{\pgfqpoint{3.830343in}{1.302977in}}%
\pgfusepath{clip}%
\pgfsetrectcap%
\pgfsetroundjoin%
\pgfsetlinewidth{1.003750pt}%
\definecolor{currentstroke}{rgb}{0.752941,0.752941,1.000000}%
\pgfsetstrokecolor{currentstroke}%
\pgfsetdash{}{0pt}%
\pgfpathmoveto{\pgfqpoint{0.694334in}{0.736317in}}%
\pgfpathlineto{\pgfqpoint{0.696249in}{0.753605in}}%
\pgfpathlineto{\pgfqpoint{0.740298in}{0.753605in}}%
\pgfpathlineto{\pgfqpoint{0.742214in}{0.768580in}}%
\pgfpathlineto{\pgfqpoint{0.770941in}{0.768580in}}%
\pgfpathlineto{\pgfqpoint{0.772856in}{0.781789in}}%
\pgfpathlineto{\pgfqpoint{0.776687in}{0.781789in}}%
\pgfpathlineto{\pgfqpoint{0.778602in}{0.793606in}}%
\pgfpathlineto{\pgfqpoint{0.813075in}{0.793606in}}%
\pgfpathlineto{\pgfqpoint{0.814990in}{0.804294in}}%
\pgfpathlineto{\pgfqpoint{0.857124in}{0.804294in}}%
\pgfpathlineto{\pgfqpoint{0.859039in}{0.814053in}}%
\pgfpathlineto{\pgfqpoint{0.885851in}{0.814053in}}%
\pgfpathlineto{\pgfqpoint{0.887767in}{0.823029in}}%
\pgfpathlineto{\pgfqpoint{0.908834in}{0.823029in}}%
\pgfpathlineto{\pgfqpoint{0.910749in}{0.831341in}}%
\pgfpathlineto{\pgfqpoint{0.954798in}{0.831341in}}%
\pgfpathlineto{\pgfqpoint{0.956713in}{0.839078in}}%
\pgfpathlineto{\pgfqpoint{0.989271in}{0.839078in}}%
\pgfpathlineto{\pgfqpoint{0.991186in}{0.846316in}}%
\pgfpathlineto{\pgfqpoint{1.027574in}{0.846316in}}%
\pgfpathlineto{\pgfqpoint{1.029489in}{0.853115in}}%
\pgfpathlineto{\pgfqpoint{1.056302in}{0.853115in}}%
\pgfpathlineto{\pgfqpoint{1.058217in}{0.859525in}}%
\pgfpathlineto{\pgfqpoint{1.069708in}{0.859525in}}%
\pgfpathlineto{\pgfqpoint{1.071623in}{0.865589in}}%
\pgfpathlineto{\pgfqpoint{1.086944in}{0.865589in}}%
\pgfpathlineto{\pgfqpoint{1.088860in}{0.871341in}}%
\pgfpathlineto{\pgfqpoint{1.096520in}{0.871341in}}%
\pgfpathlineto{\pgfqpoint{1.098435in}{0.876813in}}%
\pgfpathlineto{\pgfqpoint{1.104181in}{0.876813in}}%
\pgfpathlineto{\pgfqpoint{1.106096in}{0.882030in}}%
\pgfpathlineto{\pgfqpoint{1.109926in}{0.882030in}}%
\pgfpathlineto{\pgfqpoint{1.111842in}{0.887015in}}%
\pgfpathlineto{\pgfqpoint{1.127163in}{0.887015in}}%
\pgfpathlineto{\pgfqpoint{1.129078in}{0.891788in}}%
\pgfpathlineto{\pgfqpoint{1.140569in}{0.891788in}}%
\pgfpathlineto{\pgfqpoint{1.142484in}{0.896366in}}%
\pgfpathlineto{\pgfqpoint{1.152060in}{0.896366in}}%
\pgfpathlineto{\pgfqpoint{1.153975in}{0.900765in}}%
\pgfpathlineto{\pgfqpoint{1.157806in}{0.900765in}}%
\pgfpathlineto{\pgfqpoint{1.159721in}{0.904998in}}%
\pgfpathlineto{\pgfqpoint{1.161636in}{0.904998in}}%
\pgfpathlineto{\pgfqpoint{1.163551in}{0.916814in}}%
\pgfpathlineto{\pgfqpoint{1.165466in}{0.916814in}}%
\pgfpathlineto{\pgfqpoint{1.167382in}{1.826535in}}%
\pgfpathlineto{\pgfqpoint{1.167382in}{1.826535in}}%
\pgfusepath{stroke}%
\end{pgfscope}%
\begin{pgfscope}%
\pgfpathrectangle{\pgfqpoint{0.694334in}{0.523557in}}{\pgfqpoint{3.830343in}{1.302977in}}%
\pgfusepath{clip}%
\pgfsetrectcap%
\pgfsetroundjoin%
\pgfsetlinewidth{1.003750pt}%
\definecolor{currentstroke}{rgb}{0.752941,0.752941,1.000000}%
\pgfsetstrokecolor{currentstroke}%
\pgfsetdash{}{0pt}%
\pgfpathmoveto{\pgfqpoint{0.694334in}{0.753605in}}%
\pgfpathlineto{\pgfqpoint{0.696249in}{0.753605in}}%
\pgfpathlineto{\pgfqpoint{0.698165in}{0.753605in}}%
\pgfpathlineto{\pgfqpoint{0.700080in}{0.768580in}}%
\pgfpathlineto{\pgfqpoint{0.701995in}{0.768580in}}%
\pgfpathlineto{\pgfqpoint{0.703910in}{0.768580in}}%
\pgfpathlineto{\pgfqpoint{0.705825in}{0.768580in}}%
\pgfpathlineto{\pgfqpoint{0.707741in}{0.768580in}}%
\pgfpathlineto{\pgfqpoint{0.709656in}{0.768580in}}%
\pgfpathlineto{\pgfqpoint{0.711571in}{0.768580in}}%
\pgfpathlineto{\pgfqpoint{0.713486in}{0.768580in}}%
\pgfpathlineto{\pgfqpoint{0.715401in}{0.768580in}}%
\pgfpathlineto{\pgfqpoint{0.717316in}{0.768580in}}%
\pgfpathlineto{\pgfqpoint{0.719232in}{0.781789in}}%
\pgfpathlineto{\pgfqpoint{0.721147in}{0.781789in}}%
\pgfpathlineto{\pgfqpoint{0.723062in}{0.781789in}}%
\pgfpathlineto{\pgfqpoint{0.724977in}{0.793606in}}%
\pgfpathlineto{\pgfqpoint{0.726892in}{0.793606in}}%
\pgfpathlineto{\pgfqpoint{0.728807in}{0.793606in}}%
\pgfpathlineto{\pgfqpoint{0.730723in}{0.793606in}}%
\pgfpathlineto{\pgfqpoint{0.732638in}{0.793606in}}%
\pgfpathlineto{\pgfqpoint{0.734553in}{0.804294in}}%
\pgfpathlineto{\pgfqpoint{0.736468in}{0.804294in}}%
\pgfpathlineto{\pgfqpoint{0.738383in}{0.804294in}}%
\pgfpathlineto{\pgfqpoint{0.740298in}{0.804294in}}%
\pgfpathlineto{\pgfqpoint{0.742214in}{0.804294in}}%
\pgfpathlineto{\pgfqpoint{0.744129in}{0.814053in}}%
\pgfpathlineto{\pgfqpoint{0.746044in}{0.814053in}}%
\pgfpathlineto{\pgfqpoint{0.747959in}{0.823029in}}%
\pgfpathlineto{\pgfqpoint{0.749874in}{0.831341in}}%
\pgfpathlineto{\pgfqpoint{0.751789in}{0.846316in}}%
\pgfpathlineto{\pgfqpoint{0.753705in}{0.859525in}}%
\pgfpathlineto{\pgfqpoint{0.755620in}{0.865589in}}%
\pgfpathlineto{\pgfqpoint{0.757535in}{0.865589in}}%
\pgfpathlineto{\pgfqpoint{0.759450in}{0.876813in}}%
\pgfpathlineto{\pgfqpoint{0.761365in}{0.876813in}}%
\pgfpathlineto{\pgfqpoint{0.763280in}{0.876813in}}%
\pgfpathlineto{\pgfqpoint{0.765196in}{0.882030in}}%
\pgfpathlineto{\pgfqpoint{0.767111in}{0.891788in}}%
\pgfpathlineto{\pgfqpoint{0.769026in}{0.896366in}}%
\pgfpathlineto{\pgfqpoint{0.770941in}{1.826535in}}%
\pgfusepath{stroke}%
\end{pgfscope}%
\begin{pgfscope}%
\pgfpathrectangle{\pgfqpoint{0.694334in}{0.523557in}}{\pgfqpoint{3.830343in}{1.302977in}}%
\pgfusepath{clip}%
\pgfsetrectcap%
\pgfsetroundjoin%
\pgfsetlinewidth{1.003750pt}%
\definecolor{currentstroke}{rgb}{0.752941,0.752941,1.000000}%
\pgfsetstrokecolor{currentstroke}%
\pgfsetdash{}{0pt}%
\pgfpathmoveto{\pgfqpoint{0.694334in}{0.804294in}}%
\pgfpathlineto{\pgfqpoint{0.696249in}{0.823029in}}%
\pgfpathlineto{\pgfqpoint{0.698165in}{0.823029in}}%
\pgfpathlineto{\pgfqpoint{0.700080in}{0.823029in}}%
\pgfpathlineto{\pgfqpoint{0.701995in}{0.823029in}}%
\pgfpathlineto{\pgfqpoint{0.703910in}{0.831341in}}%
\pgfpathlineto{\pgfqpoint{0.705825in}{0.831341in}}%
\pgfpathlineto{\pgfqpoint{0.707741in}{0.839078in}}%
\pgfpathlineto{\pgfqpoint{0.709656in}{0.839078in}}%
\pgfpathlineto{\pgfqpoint{0.711571in}{0.846316in}}%
\pgfpathlineto{\pgfqpoint{0.713486in}{0.846316in}}%
\pgfpathlineto{\pgfqpoint{0.715401in}{0.846316in}}%
\pgfpathlineto{\pgfqpoint{0.717316in}{0.853115in}}%
\pgfpathlineto{\pgfqpoint{0.719232in}{0.853115in}}%
\pgfpathlineto{\pgfqpoint{0.721147in}{0.859525in}}%
\pgfpathlineto{\pgfqpoint{0.723062in}{0.859525in}}%
\pgfpathlineto{\pgfqpoint{0.724977in}{0.859525in}}%
\pgfpathlineto{\pgfqpoint{0.726892in}{0.865589in}}%
\pgfpathlineto{\pgfqpoint{0.728807in}{0.865589in}}%
\pgfpathlineto{\pgfqpoint{0.730723in}{0.865589in}}%
\pgfpathlineto{\pgfqpoint{0.732638in}{0.865589in}}%
\pgfpathlineto{\pgfqpoint{0.734553in}{0.865589in}}%
\pgfpathlineto{\pgfqpoint{0.736468in}{0.865589in}}%
\pgfpathlineto{\pgfqpoint{0.738383in}{0.871341in}}%
\pgfpathlineto{\pgfqpoint{0.740298in}{0.871341in}}%
\pgfpathlineto{\pgfqpoint{0.742214in}{0.882030in}}%
\pgfpathlineto{\pgfqpoint{0.744129in}{0.887015in}}%
\pgfpathlineto{\pgfqpoint{0.746044in}{0.887015in}}%
\pgfpathlineto{\pgfqpoint{0.747959in}{0.891788in}}%
\pgfpathlineto{\pgfqpoint{0.749874in}{0.896366in}}%
\pgfpathlineto{\pgfqpoint{0.751789in}{0.904998in}}%
\pgfpathlineto{\pgfqpoint{0.753705in}{0.916814in}}%
\pgfpathlineto{\pgfqpoint{0.755620in}{0.927503in}}%
\pgfpathlineto{\pgfqpoint{0.757535in}{0.937261in}}%
\pgfpathlineto{\pgfqpoint{0.759450in}{0.959766in}}%
\pgfpathlineto{\pgfqpoint{0.761365in}{0.988797in}}%
\pgfpathlineto{\pgfqpoint{0.763280in}{1.003526in}}%
\pgfpathlineto{\pgfqpoint{0.765196in}{1.022526in}}%
\pgfpathlineto{\pgfqpoint{0.767111in}{1.826535in}}%
\pgfusepath{stroke}%
\end{pgfscope}%
\begin{pgfscope}%
\pgfpathrectangle{\pgfqpoint{0.694334in}{0.523557in}}{\pgfqpoint{3.830343in}{1.302977in}}%
\pgfusepath{clip}%
\pgfsetrectcap%
\pgfsetroundjoin%
\pgfsetlinewidth{1.003750pt}%
\definecolor{currentstroke}{rgb}{0.752941,0.752941,1.000000}%
\pgfsetstrokecolor{currentstroke}%
\pgfsetdash{}{0pt}%
\pgfpathmoveto{\pgfqpoint{0.694334in}{0.753605in}}%
\pgfpathlineto{\pgfqpoint{0.709656in}{0.753605in}}%
\pgfpathlineto{\pgfqpoint{0.711571in}{0.768580in}}%
\pgfpathlineto{\pgfqpoint{0.749874in}{0.768580in}}%
\pgfpathlineto{\pgfqpoint{0.751789in}{0.781789in}}%
\pgfpathlineto{\pgfqpoint{0.776687in}{0.781789in}}%
\pgfpathlineto{\pgfqpoint{0.778602in}{0.793606in}}%
\pgfpathlineto{\pgfqpoint{0.795838in}{0.793606in}}%
\pgfpathlineto{\pgfqpoint{0.797754in}{0.804294in}}%
\pgfpathlineto{\pgfqpoint{0.820736in}{0.804294in}}%
\pgfpathlineto{\pgfqpoint{0.822651in}{0.814053in}}%
\pgfpathlineto{\pgfqpoint{0.859039in}{0.814053in}}%
\pgfpathlineto{\pgfqpoint{0.860954in}{0.823029in}}%
\pgfpathlineto{\pgfqpoint{0.870530in}{0.823029in}}%
\pgfpathlineto{\pgfqpoint{0.872445in}{0.831341in}}%
\pgfpathlineto{\pgfqpoint{0.895427in}{0.831341in}}%
\pgfpathlineto{\pgfqpoint{0.897342in}{0.839078in}}%
\pgfpathlineto{\pgfqpoint{0.926070in}{0.839078in}}%
\pgfpathlineto{\pgfqpoint{0.927985in}{0.846316in}}%
\pgfpathlineto{\pgfqpoint{0.960543in}{0.846316in}}%
\pgfpathlineto{\pgfqpoint{0.962458in}{0.853115in}}%
\pgfpathlineto{\pgfqpoint{0.983525in}{0.853115in}}%
\pgfpathlineto{\pgfqpoint{0.985440in}{0.859525in}}%
\pgfpathlineto{\pgfqpoint{1.008422in}{0.859525in}}%
\pgfpathlineto{\pgfqpoint{1.010338in}{0.865589in}}%
\pgfpathlineto{\pgfqpoint{1.039065in}{0.865589in}}%
\pgfpathlineto{\pgfqpoint{1.040980in}{0.871341in}}%
\pgfpathlineto{\pgfqpoint{1.065878in}{0.871341in}}%
\pgfpathlineto{\pgfqpoint{1.067793in}{0.876813in}}%
\pgfpathlineto{\pgfqpoint{1.086944in}{0.876813in}}%
\pgfpathlineto{\pgfqpoint{1.088860in}{0.882030in}}%
\pgfpathlineto{\pgfqpoint{1.106096in}{0.882030in}}%
\pgfpathlineto{\pgfqpoint{1.108011in}{0.887015in}}%
\pgfpathlineto{\pgfqpoint{1.130993in}{0.887015in}}%
\pgfpathlineto{\pgfqpoint{1.132909in}{0.891788in}}%
\pgfpathlineto{\pgfqpoint{1.142484in}{0.891788in}}%
\pgfpathlineto{\pgfqpoint{1.144400in}{0.896366in}}%
\pgfpathlineto{\pgfqpoint{1.161636in}{0.896366in}}%
\pgfpathlineto{\pgfqpoint{1.163551in}{0.900765in}}%
\pgfpathlineto{\pgfqpoint{1.176957in}{0.900765in}}%
\pgfpathlineto{\pgfqpoint{1.178873in}{0.904998in}}%
\pgfpathlineto{\pgfqpoint{1.192279in}{0.904998in}}%
\pgfpathlineto{\pgfqpoint{1.194194in}{0.909076in}}%
\pgfpathlineto{\pgfqpoint{1.211431in}{0.909076in}}%
\pgfpathlineto{\pgfqpoint{1.213346in}{0.913012in}}%
\pgfpathlineto{\pgfqpoint{1.226752in}{0.913012in}}%
\pgfpathlineto{\pgfqpoint{1.228667in}{0.916814in}}%
\pgfpathlineto{\pgfqpoint{1.245904in}{0.916814in}}%
\pgfpathlineto{\pgfqpoint{1.247819in}{0.920491in}}%
\pgfpathlineto{\pgfqpoint{1.257395in}{0.920491in}}%
\pgfpathlineto{\pgfqpoint{1.259310in}{0.924052in}}%
\pgfpathlineto{\pgfqpoint{1.274631in}{0.924052in}}%
\pgfpathlineto{\pgfqpoint{1.276546in}{0.927503in}}%
\pgfpathlineto{\pgfqpoint{1.295698in}{0.927503in}}%
\pgfpathlineto{\pgfqpoint{1.297613in}{0.930851in}}%
\pgfpathlineto{\pgfqpoint{1.303359in}{0.930851in}}%
\pgfpathlineto{\pgfqpoint{1.305274in}{0.934101in}}%
\pgfpathlineto{\pgfqpoint{1.320595in}{0.934101in}}%
\pgfpathlineto{\pgfqpoint{1.322511in}{0.937261in}}%
\pgfpathlineto{\pgfqpoint{1.328256in}{0.937261in}}%
\pgfpathlineto{\pgfqpoint{1.330171in}{0.940334in}}%
\pgfpathlineto{\pgfqpoint{1.335917in}{0.940334in}}%
\pgfpathlineto{\pgfqpoint{1.337832in}{0.943324in}}%
\pgfpathlineto{\pgfqpoint{1.341662in}{0.943324in}}%
\pgfpathlineto{\pgfqpoint{1.343577in}{0.946237in}}%
\pgfpathlineto{\pgfqpoint{1.351238in}{0.946237in}}%
\pgfpathlineto{\pgfqpoint{1.353153in}{0.949077in}}%
\pgfpathlineto{\pgfqpoint{1.356984in}{0.949077in}}%
\pgfpathlineto{\pgfqpoint{1.358899in}{0.951846in}}%
\pgfpathlineto{\pgfqpoint{1.364644in}{0.951846in}}%
\pgfpathlineto{\pgfqpoint{1.366559in}{0.954549in}}%
\pgfpathlineto{\pgfqpoint{1.374220in}{0.954549in}}%
\pgfpathlineto{\pgfqpoint{1.376135in}{0.957188in}}%
\pgfpathlineto{\pgfqpoint{1.379966in}{0.957188in}}%
\pgfpathlineto{\pgfqpoint{1.381881in}{0.959766in}}%
\pgfpathlineto{\pgfqpoint{1.393372in}{0.959766in}}%
\pgfpathlineto{\pgfqpoint{1.395287in}{0.962286in}}%
\pgfpathlineto{\pgfqpoint{1.397202in}{0.962286in}}%
\pgfpathlineto{\pgfqpoint{1.401033in}{0.967163in}}%
\pgfpathlineto{\pgfqpoint{1.402948in}{0.967163in}}%
\pgfpathlineto{\pgfqpoint{1.404863in}{0.969524in}}%
\pgfpathlineto{\pgfqpoint{1.406778in}{0.976323in}}%
\pgfpathlineto{\pgfqpoint{1.408693in}{0.976323in}}%
\pgfpathlineto{\pgfqpoint{1.410608in}{0.980637in}}%
\pgfpathlineto{\pgfqpoint{1.412524in}{0.988797in}}%
\pgfpathlineto{\pgfqpoint{1.414439in}{1.826535in}}%
\pgfpathlineto{\pgfqpoint{1.414439in}{1.826535in}}%
\pgfusepath{stroke}%
\end{pgfscope}%
\begin{pgfscope}%
\pgfpathrectangle{\pgfqpoint{0.694334in}{0.523557in}}{\pgfqpoint{3.830343in}{1.302977in}}%
\pgfusepath{clip}%
\pgfsetrectcap%
\pgfsetroundjoin%
\pgfsetlinewidth{1.003750pt}%
\definecolor{currentstroke}{rgb}{0.752941,0.752941,1.000000}%
\pgfsetstrokecolor{currentstroke}%
\pgfsetdash{}{0pt}%
\pgfpathmoveto{\pgfqpoint{0.694334in}{0.753605in}}%
\pgfpathlineto{\pgfqpoint{0.696249in}{0.753605in}}%
\pgfpathlineto{\pgfqpoint{0.698165in}{0.753605in}}%
\pgfpathlineto{\pgfqpoint{0.700080in}{0.753605in}}%
\pgfpathlineto{\pgfqpoint{0.701995in}{0.753605in}}%
\pgfpathlineto{\pgfqpoint{0.703910in}{0.753605in}}%
\pgfpathlineto{\pgfqpoint{0.705825in}{0.753605in}}%
\pgfpathlineto{\pgfqpoint{0.707741in}{0.753605in}}%
\pgfpathlineto{\pgfqpoint{0.709656in}{0.753605in}}%
\pgfpathlineto{\pgfqpoint{0.711571in}{0.753605in}}%
\pgfpathlineto{\pgfqpoint{0.713486in}{0.753605in}}%
\pgfpathlineto{\pgfqpoint{0.715401in}{0.753605in}}%
\pgfpathlineto{\pgfqpoint{0.717316in}{0.753605in}}%
\pgfpathlineto{\pgfqpoint{0.719232in}{0.753605in}}%
\pgfpathlineto{\pgfqpoint{0.721147in}{0.753605in}}%
\pgfpathlineto{\pgfqpoint{0.723062in}{0.753605in}}%
\pgfpathlineto{\pgfqpoint{0.724977in}{0.753605in}}%
\pgfpathlineto{\pgfqpoint{0.726892in}{0.753605in}}%
\pgfpathlineto{\pgfqpoint{0.728807in}{0.753605in}}%
\pgfpathlineto{\pgfqpoint{0.730723in}{0.753605in}}%
\pgfpathlineto{\pgfqpoint{0.732638in}{0.753605in}}%
\pgfpathlineto{\pgfqpoint{0.734553in}{0.753605in}}%
\pgfpathlineto{\pgfqpoint{0.736468in}{0.768580in}}%
\pgfpathlineto{\pgfqpoint{0.738383in}{0.768580in}}%
\pgfpathlineto{\pgfqpoint{0.740298in}{0.768580in}}%
\pgfpathlineto{\pgfqpoint{0.742214in}{0.768580in}}%
\pgfpathlineto{\pgfqpoint{0.744129in}{0.768580in}}%
\pgfpathlineto{\pgfqpoint{0.746044in}{0.768580in}}%
\pgfpathlineto{\pgfqpoint{0.747959in}{0.768580in}}%
\pgfpathlineto{\pgfqpoint{0.749874in}{0.768580in}}%
\pgfpathlineto{\pgfqpoint{0.751789in}{0.768580in}}%
\pgfpathlineto{\pgfqpoint{0.753705in}{0.768580in}}%
\pgfpathlineto{\pgfqpoint{0.755620in}{0.768580in}}%
\pgfpathlineto{\pgfqpoint{0.757535in}{0.768580in}}%
\pgfpathlineto{\pgfqpoint{0.759450in}{0.781789in}}%
\pgfpathlineto{\pgfqpoint{0.761365in}{0.804294in}}%
\pgfpathlineto{\pgfqpoint{0.763280in}{1.826535in}}%
\pgfusepath{stroke}%
\end{pgfscope}%
\begin{pgfscope}%
\pgfpathrectangle{\pgfqpoint{0.694334in}{0.523557in}}{\pgfqpoint{3.830343in}{1.302977in}}%
\pgfusepath{clip}%
\pgfsetrectcap%
\pgfsetroundjoin%
\pgfsetlinewidth{1.003750pt}%
\definecolor{currentstroke}{rgb}{0.752941,0.752941,1.000000}%
\pgfsetstrokecolor{currentstroke}%
\pgfsetdash{}{0pt}%
\pgfpathmoveto{\pgfqpoint{0.694334in}{0.753605in}}%
\pgfpathlineto{\pgfqpoint{0.730723in}{0.753605in}}%
\pgfpathlineto{\pgfqpoint{0.732638in}{0.768580in}}%
\pgfpathlineto{\pgfqpoint{0.767111in}{0.768580in}}%
\pgfpathlineto{\pgfqpoint{0.769026in}{0.781789in}}%
\pgfpathlineto{\pgfqpoint{0.805414in}{0.781789in}}%
\pgfpathlineto{\pgfqpoint{0.807329in}{0.793606in}}%
\pgfpathlineto{\pgfqpoint{0.836057in}{0.793606in}}%
\pgfpathlineto{\pgfqpoint{0.837972in}{0.804294in}}%
\pgfpathlineto{\pgfqpoint{0.870530in}{0.804294in}}%
\pgfpathlineto{\pgfqpoint{0.872445in}{0.814053in}}%
\pgfpathlineto{\pgfqpoint{0.893512in}{0.814053in}}%
\pgfpathlineto{\pgfqpoint{0.895427in}{0.823029in}}%
\pgfpathlineto{\pgfqpoint{0.918409in}{0.823029in}}%
\pgfpathlineto{\pgfqpoint{0.920325in}{0.831341in}}%
\pgfpathlineto{\pgfqpoint{0.958628in}{0.831341in}}%
\pgfpathlineto{\pgfqpoint{0.960543in}{0.839078in}}%
\pgfpathlineto{\pgfqpoint{0.972034in}{0.839078in}}%
\pgfpathlineto{\pgfqpoint{0.973949in}{0.846316in}}%
\pgfpathlineto{\pgfqpoint{1.014168in}{0.846316in}}%
\pgfpathlineto{\pgfqpoint{1.016083in}{0.853115in}}%
\pgfpathlineto{\pgfqpoint{1.039065in}{0.853115in}}%
\pgfpathlineto{\pgfqpoint{1.040980in}{0.859525in}}%
\pgfpathlineto{\pgfqpoint{1.065878in}{0.859525in}}%
\pgfpathlineto{\pgfqpoint{1.067793in}{0.865589in}}%
\pgfpathlineto{\pgfqpoint{1.094605in}{0.865589in}}%
\pgfpathlineto{\pgfqpoint{1.096520in}{0.871341in}}%
\pgfpathlineto{\pgfqpoint{1.123333in}{0.871341in}}%
\pgfpathlineto{\pgfqpoint{1.125248in}{0.876813in}}%
\pgfpathlineto{\pgfqpoint{1.142484in}{0.876813in}}%
\pgfpathlineto{\pgfqpoint{1.144400in}{0.882030in}}%
\pgfpathlineto{\pgfqpoint{1.171212in}{0.882030in}}%
\pgfpathlineto{\pgfqpoint{1.173127in}{0.887015in}}%
\pgfpathlineto{\pgfqpoint{1.198024in}{0.887015in}}%
\pgfpathlineto{\pgfqpoint{1.199940in}{0.891788in}}%
\pgfpathlineto{\pgfqpoint{1.205685in}{0.891788in}}%
\pgfpathlineto{\pgfqpoint{1.207600in}{0.896366in}}%
\pgfpathlineto{\pgfqpoint{1.213346in}{0.896366in}}%
\pgfpathlineto{\pgfqpoint{1.215261in}{0.900765in}}%
\pgfpathlineto{\pgfqpoint{1.226752in}{0.900765in}}%
\pgfpathlineto{\pgfqpoint{1.228667in}{0.904998in}}%
\pgfpathlineto{\pgfqpoint{1.238243in}{0.904998in}}%
\pgfpathlineto{\pgfqpoint{1.240158in}{0.909076in}}%
\pgfpathlineto{\pgfqpoint{1.257395in}{0.909076in}}%
\pgfpathlineto{\pgfqpoint{1.259310in}{0.913012in}}%
\pgfpathlineto{\pgfqpoint{1.266971in}{0.913012in}}%
\pgfpathlineto{\pgfqpoint{1.268886in}{0.916814in}}%
\pgfpathlineto{\pgfqpoint{1.274631in}{0.916814in}}%
\pgfpathlineto{\pgfqpoint{1.276546in}{0.920491in}}%
\pgfpathlineto{\pgfqpoint{1.286122in}{0.920491in}}%
\pgfpathlineto{\pgfqpoint{1.288037in}{0.924052in}}%
\pgfpathlineto{\pgfqpoint{1.291868in}{0.924052in}}%
\pgfpathlineto{\pgfqpoint{1.293783in}{0.927503in}}%
\pgfpathlineto{\pgfqpoint{1.297613in}{0.937261in}}%
\pgfpathlineto{\pgfqpoint{1.299528in}{0.940334in}}%
\pgfpathlineto{\pgfqpoint{1.301444in}{0.951846in}}%
\pgfpathlineto{\pgfqpoint{1.303359in}{0.980637in}}%
\pgfpathlineto{\pgfqpoint{1.307189in}{0.992664in}}%
\pgfpathlineto{\pgfqpoint{1.309104in}{1.092732in}}%
\pgfpathlineto{\pgfqpoint{1.311019in}{1.093508in}}%
\pgfpathlineto{\pgfqpoint{1.312935in}{1.098796in}}%
\pgfpathlineto{\pgfqpoint{1.318680in}{1.129573in}}%
\pgfpathlineto{\pgfqpoint{1.322511in}{1.143280in}}%
\pgfpathlineto{\pgfqpoint{1.326341in}{1.195181in}}%
\pgfpathlineto{\pgfqpoint{1.328256in}{1.826535in}}%
\pgfpathlineto{\pgfqpoint{1.328256in}{1.826535in}}%
\pgfusepath{stroke}%
\end{pgfscope}%
\begin{pgfscope}%
\pgfpathrectangle{\pgfqpoint{0.694334in}{0.523557in}}{\pgfqpoint{3.830343in}{1.302977in}}%
\pgfusepath{clip}%
\pgfsetrectcap%
\pgfsetroundjoin%
\pgfsetlinewidth{1.003750pt}%
\definecolor{currentstroke}{rgb}{0.752941,0.752941,1.000000}%
\pgfsetstrokecolor{currentstroke}%
\pgfsetdash{}{0pt}%
\pgfpathmoveto{\pgfqpoint{0.694334in}{0.753605in}}%
\pgfpathlineto{\pgfqpoint{0.728807in}{0.753605in}}%
\pgfpathlineto{\pgfqpoint{0.730723in}{0.768580in}}%
\pgfpathlineto{\pgfqpoint{0.770941in}{0.768580in}}%
\pgfpathlineto{\pgfqpoint{0.772856in}{0.781789in}}%
\pgfpathlineto{\pgfqpoint{0.813075in}{0.781789in}}%
\pgfpathlineto{\pgfqpoint{0.814990in}{0.793606in}}%
\pgfpathlineto{\pgfqpoint{0.839887in}{0.793606in}}%
\pgfpathlineto{\pgfqpoint{0.841803in}{0.804294in}}%
\pgfpathlineto{\pgfqpoint{0.878191in}{0.804294in}}%
\pgfpathlineto{\pgfqpoint{0.880106in}{0.814053in}}%
\pgfpathlineto{\pgfqpoint{0.914579in}{0.814053in}}%
\pgfpathlineto{\pgfqpoint{0.916494in}{0.823029in}}%
\pgfpathlineto{\pgfqpoint{0.939476in}{0.823029in}}%
\pgfpathlineto{\pgfqpoint{0.941391in}{0.831341in}}%
\pgfpathlineto{\pgfqpoint{0.968204in}{0.831341in}}%
\pgfpathlineto{\pgfqpoint{0.970119in}{0.839078in}}%
\pgfpathlineto{\pgfqpoint{1.000762in}{0.839078in}}%
\pgfpathlineto{\pgfqpoint{1.002677in}{0.846316in}}%
\pgfpathlineto{\pgfqpoint{1.025659in}{0.846316in}}%
\pgfpathlineto{\pgfqpoint{1.027574in}{0.853115in}}%
\pgfpathlineto{\pgfqpoint{1.044811in}{0.853115in}}%
\pgfpathlineto{\pgfqpoint{1.046726in}{0.859525in}}%
\pgfpathlineto{\pgfqpoint{1.073538in}{0.859525in}}%
\pgfpathlineto{\pgfqpoint{1.075453in}{0.865589in}}%
\pgfpathlineto{\pgfqpoint{1.092690in}{0.865589in}}%
\pgfpathlineto{\pgfqpoint{1.094605in}{0.871341in}}%
\pgfpathlineto{\pgfqpoint{1.119502in}{0.871341in}}%
\pgfpathlineto{\pgfqpoint{1.121418in}{0.876813in}}%
\pgfpathlineto{\pgfqpoint{1.142484in}{0.876813in}}%
\pgfpathlineto{\pgfqpoint{1.144400in}{0.882030in}}%
\pgfpathlineto{\pgfqpoint{1.157806in}{0.882030in}}%
\pgfpathlineto{\pgfqpoint{1.159721in}{0.887015in}}%
\pgfpathlineto{\pgfqpoint{1.184618in}{0.887015in}}%
\pgfpathlineto{\pgfqpoint{1.186533in}{0.891788in}}%
\pgfpathlineto{\pgfqpoint{1.196109in}{0.891788in}}%
\pgfpathlineto{\pgfqpoint{1.198024in}{0.896366in}}%
\pgfpathlineto{\pgfqpoint{1.215261in}{0.896366in}}%
\pgfpathlineto{\pgfqpoint{1.217176in}{0.900765in}}%
\pgfpathlineto{\pgfqpoint{1.234413in}{0.900765in}}%
\pgfpathlineto{\pgfqpoint{1.236328in}{0.904998in}}%
\pgfpathlineto{\pgfqpoint{1.257395in}{0.904998in}}%
\pgfpathlineto{\pgfqpoint{1.259310in}{0.909076in}}%
\pgfpathlineto{\pgfqpoint{1.274631in}{0.909076in}}%
\pgfpathlineto{\pgfqpoint{1.276546in}{0.913012in}}%
\pgfpathlineto{\pgfqpoint{1.278462in}{0.913012in}}%
\pgfpathlineto{\pgfqpoint{1.280377in}{0.916814in}}%
\pgfpathlineto{\pgfqpoint{1.295698in}{0.916814in}}%
\pgfpathlineto{\pgfqpoint{1.297613in}{0.920491in}}%
\pgfpathlineto{\pgfqpoint{1.307189in}{0.920491in}}%
\pgfpathlineto{\pgfqpoint{1.309104in}{0.924052in}}%
\pgfpathlineto{\pgfqpoint{1.314850in}{0.924052in}}%
\pgfpathlineto{\pgfqpoint{1.316765in}{0.927503in}}%
\pgfpathlineto{\pgfqpoint{1.328256in}{0.927503in}}%
\pgfpathlineto{\pgfqpoint{1.330171in}{0.930851in}}%
\pgfpathlineto{\pgfqpoint{1.347408in}{0.930851in}}%
\pgfpathlineto{\pgfqpoint{1.349323in}{0.934101in}}%
\pgfpathlineto{\pgfqpoint{1.355068in}{0.934101in}}%
\pgfpathlineto{\pgfqpoint{1.356984in}{0.937261in}}%
\pgfpathlineto{\pgfqpoint{1.370390in}{0.937261in}}%
\pgfpathlineto{\pgfqpoint{1.372305in}{0.940334in}}%
\pgfpathlineto{\pgfqpoint{1.379966in}{0.940334in}}%
\pgfpathlineto{\pgfqpoint{1.381881in}{0.943324in}}%
\pgfpathlineto{\pgfqpoint{1.391457in}{0.943324in}}%
\pgfpathlineto{\pgfqpoint{1.393372in}{0.946237in}}%
\pgfpathlineto{\pgfqpoint{1.401033in}{0.946237in}}%
\pgfpathlineto{\pgfqpoint{1.402948in}{0.949077in}}%
\pgfpathlineto{\pgfqpoint{1.412524in}{0.949077in}}%
\pgfpathlineto{\pgfqpoint{1.414439in}{0.951846in}}%
\pgfpathlineto{\pgfqpoint{1.425930in}{0.951846in}}%
\pgfpathlineto{\pgfqpoint{1.427845in}{0.954549in}}%
\pgfpathlineto{\pgfqpoint{1.433590in}{0.954549in}}%
\pgfpathlineto{\pgfqpoint{1.435506in}{0.957188in}}%
\pgfpathlineto{\pgfqpoint{1.445081in}{0.957188in}}%
\pgfpathlineto{\pgfqpoint{1.446997in}{0.959766in}}%
\pgfpathlineto{\pgfqpoint{1.452742in}{0.959766in}}%
\pgfpathlineto{\pgfqpoint{1.454657in}{0.962286in}}%
\pgfpathlineto{\pgfqpoint{1.466148in}{0.962286in}}%
\pgfpathlineto{\pgfqpoint{1.468064in}{0.964751in}}%
\pgfpathlineto{\pgfqpoint{1.471894in}{0.964751in}}%
\pgfpathlineto{\pgfqpoint{1.473809in}{0.967163in}}%
\pgfpathlineto{\pgfqpoint{1.489130in}{0.967163in}}%
\pgfpathlineto{\pgfqpoint{1.491046in}{0.969524in}}%
\pgfpathlineto{\pgfqpoint{1.498706in}{0.969524in}}%
\pgfpathlineto{\pgfqpoint{1.500621in}{0.971836in}}%
\pgfpathlineto{\pgfqpoint{1.506367in}{0.971836in}}%
\pgfpathlineto{\pgfqpoint{1.508282in}{0.974102in}}%
\pgfpathlineto{\pgfqpoint{1.515943in}{0.974102in}}%
\pgfpathlineto{\pgfqpoint{1.517858in}{0.976323in}}%
\pgfpathlineto{\pgfqpoint{1.527434in}{0.976323in}}%
\pgfpathlineto{\pgfqpoint{1.529349in}{0.978501in}}%
\pgfpathlineto{\pgfqpoint{1.533179in}{0.978501in}}%
\pgfpathlineto{\pgfqpoint{1.535095in}{0.980637in}}%
\pgfpathlineto{\pgfqpoint{1.537010in}{0.980637in}}%
\pgfpathlineto{\pgfqpoint{1.538925in}{0.982733in}}%
\pgfpathlineto{\pgfqpoint{1.540840in}{0.982733in}}%
\pgfpathlineto{\pgfqpoint{1.544670in}{0.986812in}}%
\pgfpathlineto{\pgfqpoint{1.550416in}{0.986812in}}%
\pgfpathlineto{\pgfqpoint{1.552331in}{0.992664in}}%
\pgfpathlineto{\pgfqpoint{1.559992in}{0.992664in}}%
\pgfpathlineto{\pgfqpoint{1.561907in}{0.994549in}}%
\pgfpathlineto{\pgfqpoint{1.563822in}{0.994549in}}%
\pgfpathlineto{\pgfqpoint{1.565737in}{0.996403in}}%
\pgfpathlineto{\pgfqpoint{1.571483in}{0.996403in}}%
\pgfpathlineto{\pgfqpoint{1.573398in}{0.998227in}}%
\pgfpathlineto{\pgfqpoint{1.575313in}{0.998227in}}%
\pgfpathlineto{\pgfqpoint{1.579143in}{1.001787in}}%
\pgfpathlineto{\pgfqpoint{1.584889in}{1.001787in}}%
\pgfpathlineto{\pgfqpoint{1.586804in}{1.003526in}}%
\pgfpathlineto{\pgfqpoint{1.588719in}{1.003526in}}%
\pgfpathlineto{\pgfqpoint{1.592550in}{1.006925in}}%
\pgfpathlineto{\pgfqpoint{1.594465in}{1.006925in}}%
\pgfpathlineto{\pgfqpoint{1.596380in}{1.008586in}}%
\pgfpathlineto{\pgfqpoint{1.600210in}{1.008586in}}%
\pgfpathlineto{\pgfqpoint{1.602126in}{1.010223in}}%
\pgfpathlineto{\pgfqpoint{1.604041in}{1.010223in}}%
\pgfpathlineto{\pgfqpoint{1.605956in}{1.011837in}}%
\pgfpathlineto{\pgfqpoint{1.607871in}{1.011837in}}%
\pgfpathlineto{\pgfqpoint{1.613617in}{1.016543in}}%
\pgfpathlineto{\pgfqpoint{1.615532in}{1.016543in}}%
\pgfpathlineto{\pgfqpoint{1.621277in}{1.021060in}}%
\pgfpathlineto{\pgfqpoint{1.623192in}{1.025402in}}%
\pgfpathlineto{\pgfqpoint{1.625108in}{1.025402in}}%
\pgfpathlineto{\pgfqpoint{1.628938in}{1.050711in}}%
\pgfpathlineto{\pgfqpoint{1.630853in}{1.087959in}}%
\pgfpathlineto{\pgfqpoint{1.634683in}{1.092732in}}%
\pgfpathlineto{\pgfqpoint{1.638514in}{1.120830in}}%
\pgfpathlineto{\pgfqpoint{1.640429in}{1.125578in}}%
\pgfpathlineto{\pgfqpoint{1.644259in}{1.192013in}}%
\pgfpathlineto{\pgfqpoint{1.646174in}{1.217231in}}%
\pgfpathlineto{\pgfqpoint{1.648090in}{1.266488in}}%
\pgfpathlineto{\pgfqpoint{1.650005in}{1.269265in}}%
\pgfpathlineto{\pgfqpoint{1.651920in}{1.826535in}}%
\pgfpathlineto{\pgfqpoint{1.651920in}{1.826535in}}%
\pgfusepath{stroke}%
\end{pgfscope}%
\begin{pgfscope}%
\pgfpathrectangle{\pgfqpoint{0.694334in}{0.523557in}}{\pgfqpoint{3.830343in}{1.302977in}}%
\pgfusepath{clip}%
\pgfsetrectcap%
\pgfsetroundjoin%
\pgfsetlinewidth{1.003750pt}%
\definecolor{currentstroke}{rgb}{0.752941,0.752941,1.000000}%
\pgfsetstrokecolor{currentstroke}%
\pgfsetdash{}{0pt}%
\pgfpathmoveto{\pgfqpoint{0.694334in}{0.753605in}}%
\pgfpathlineto{\pgfqpoint{0.696249in}{0.753605in}}%
\pgfpathlineto{\pgfqpoint{0.698165in}{0.753605in}}%
\pgfpathlineto{\pgfqpoint{0.700080in}{0.768580in}}%
\pgfpathlineto{\pgfqpoint{0.701995in}{0.768580in}}%
\pgfpathlineto{\pgfqpoint{0.703910in}{0.768580in}}%
\pgfpathlineto{\pgfqpoint{0.705825in}{0.768580in}}%
\pgfpathlineto{\pgfqpoint{0.707741in}{0.768580in}}%
\pgfpathlineto{\pgfqpoint{0.709656in}{0.768580in}}%
\pgfpathlineto{\pgfqpoint{0.711571in}{0.768580in}}%
\pgfpathlineto{\pgfqpoint{0.713486in}{0.768580in}}%
\pgfpathlineto{\pgfqpoint{0.715401in}{0.768580in}}%
\pgfpathlineto{\pgfqpoint{0.717316in}{0.768580in}}%
\pgfpathlineto{\pgfqpoint{0.719232in}{0.781789in}}%
\pgfpathlineto{\pgfqpoint{0.721147in}{0.781789in}}%
\pgfpathlineto{\pgfqpoint{0.723062in}{0.781789in}}%
\pgfpathlineto{\pgfqpoint{0.724977in}{0.793606in}}%
\pgfpathlineto{\pgfqpoint{0.726892in}{0.793606in}}%
\pgfpathlineto{\pgfqpoint{0.728807in}{0.793606in}}%
\pgfpathlineto{\pgfqpoint{0.730723in}{0.793606in}}%
\pgfpathlineto{\pgfqpoint{0.732638in}{0.804294in}}%
\pgfpathlineto{\pgfqpoint{0.734553in}{0.804294in}}%
\pgfpathlineto{\pgfqpoint{0.736468in}{0.804294in}}%
\pgfpathlineto{\pgfqpoint{0.738383in}{0.804294in}}%
\pgfpathlineto{\pgfqpoint{0.740298in}{0.804294in}}%
\pgfpathlineto{\pgfqpoint{0.742214in}{0.804294in}}%
\pgfpathlineto{\pgfqpoint{0.744129in}{0.814053in}}%
\pgfpathlineto{\pgfqpoint{0.746044in}{0.814053in}}%
\pgfpathlineto{\pgfqpoint{0.747959in}{0.823029in}}%
\pgfpathlineto{\pgfqpoint{0.749874in}{0.831341in}}%
\pgfpathlineto{\pgfqpoint{0.751789in}{0.831341in}}%
\pgfpathlineto{\pgfqpoint{0.753705in}{0.831341in}}%
\pgfpathlineto{\pgfqpoint{0.755620in}{0.865589in}}%
\pgfpathlineto{\pgfqpoint{0.757535in}{0.871341in}}%
\pgfpathlineto{\pgfqpoint{0.759450in}{0.882030in}}%
\pgfpathlineto{\pgfqpoint{0.761365in}{0.900765in}}%
\pgfpathlineto{\pgfqpoint{0.763280in}{0.913012in}}%
\pgfpathlineto{\pgfqpoint{0.765196in}{0.957188in}}%
\pgfpathlineto{\pgfqpoint{0.767111in}{1.032284in}}%
\pgfpathlineto{\pgfqpoint{0.769026in}{1.033611in}}%
\pgfpathlineto{\pgfqpoint{0.770941in}{1.036220in}}%
\pgfpathlineto{\pgfqpoint{0.772856in}{1.041261in}}%
\pgfpathlineto{\pgfqpoint{0.774772in}{1.049572in}}%
\pgfpathlineto{\pgfqpoint{0.776687in}{1.052954in}}%
\pgfpathlineto{\pgfqpoint{0.778602in}{1.066532in}}%
\pgfpathlineto{\pgfqpoint{0.780517in}{1.068483in}}%
\pgfpathlineto{\pgfqpoint{0.782432in}{1.069446in}}%
\pgfpathlineto{\pgfqpoint{0.784347in}{1.075054in}}%
\pgfpathlineto{\pgfqpoint{0.786263in}{1.154149in}}%
\pgfpathlineto{\pgfqpoint{0.788178in}{1.158131in}}%
\pgfpathlineto{\pgfqpoint{0.790093in}{1.186074in}}%
\pgfpathlineto{\pgfqpoint{0.792008in}{1.206182in}}%
\pgfpathlineto{\pgfqpoint{0.793923in}{1.246635in}}%
\pgfpathlineto{\pgfqpoint{0.795838in}{1.260369in}}%
\pgfpathlineto{\pgfqpoint{0.797754in}{1.265324in}}%
\pgfpathlineto{\pgfqpoint{0.799669in}{1.276453in}}%
\pgfpathlineto{\pgfqpoint{0.801584in}{1.323479in}}%
\pgfpathlineto{\pgfqpoint{0.803499in}{1.410931in}}%
\pgfpathlineto{\pgfqpoint{0.805414in}{1.826535in}}%
\pgfusepath{stroke}%
\end{pgfscope}%
\begin{pgfscope}%
\pgfpathrectangle{\pgfqpoint{0.694334in}{0.523557in}}{\pgfqpoint{3.830343in}{1.302977in}}%
\pgfusepath{clip}%
\pgfsetrectcap%
\pgfsetroundjoin%
\pgfsetlinewidth{1.003750pt}%
\definecolor{currentstroke}{rgb}{0.752941,0.752941,1.000000}%
\pgfsetstrokecolor{currentstroke}%
\pgfsetdash{}{0pt}%
\pgfpathmoveto{\pgfqpoint{0.694334in}{0.736317in}}%
\pgfpathlineto{\pgfqpoint{0.696249in}{0.753605in}}%
\pgfpathlineto{\pgfqpoint{0.698165in}{0.753605in}}%
\pgfpathlineto{\pgfqpoint{0.700080in}{0.753605in}}%
\pgfpathlineto{\pgfqpoint{0.701995in}{0.753605in}}%
\pgfpathlineto{\pgfqpoint{0.703910in}{0.753605in}}%
\pgfpathlineto{\pgfqpoint{0.705825in}{0.753605in}}%
\pgfpathlineto{\pgfqpoint{0.707741in}{0.753605in}}%
\pgfpathlineto{\pgfqpoint{0.709656in}{0.753605in}}%
\pgfpathlineto{\pgfqpoint{0.711571in}{0.753605in}}%
\pgfpathlineto{\pgfqpoint{0.713486in}{0.753605in}}%
\pgfpathlineto{\pgfqpoint{0.715401in}{0.753605in}}%
\pgfpathlineto{\pgfqpoint{0.717316in}{0.768580in}}%
\pgfpathlineto{\pgfqpoint{0.719232in}{0.768580in}}%
\pgfpathlineto{\pgfqpoint{0.721147in}{0.768580in}}%
\pgfpathlineto{\pgfqpoint{0.723062in}{0.768580in}}%
\pgfpathlineto{\pgfqpoint{0.724977in}{0.768580in}}%
\pgfpathlineto{\pgfqpoint{0.726892in}{0.768580in}}%
\pgfpathlineto{\pgfqpoint{0.728807in}{0.768580in}}%
\pgfpathlineto{\pgfqpoint{0.730723in}{0.781789in}}%
\pgfpathlineto{\pgfqpoint{0.732638in}{0.781789in}}%
\pgfpathlineto{\pgfqpoint{0.734553in}{0.781789in}}%
\pgfpathlineto{\pgfqpoint{0.736468in}{0.781789in}}%
\pgfpathlineto{\pgfqpoint{0.738383in}{0.793606in}}%
\pgfpathlineto{\pgfqpoint{0.740298in}{0.793606in}}%
\pgfpathlineto{\pgfqpoint{0.742214in}{0.804294in}}%
\pgfpathlineto{\pgfqpoint{0.744129in}{0.814053in}}%
\pgfpathlineto{\pgfqpoint{0.746044in}{0.823029in}}%
\pgfpathlineto{\pgfqpoint{0.747959in}{0.823029in}}%
\pgfpathlineto{\pgfqpoint{0.749874in}{0.823029in}}%
\pgfpathlineto{\pgfqpoint{0.751789in}{0.831341in}}%
\pgfpathlineto{\pgfqpoint{0.753705in}{0.865589in}}%
\pgfpathlineto{\pgfqpoint{0.755620in}{0.871341in}}%
\pgfpathlineto{\pgfqpoint{0.757535in}{0.964751in}}%
\pgfpathlineto{\pgfqpoint{0.759450in}{0.967163in}}%
\pgfpathlineto{\pgfqpoint{0.761365in}{1.049572in}}%
\pgfpathlineto{\pgfqpoint{0.763280in}{1.055153in}}%
\pgfpathlineto{\pgfqpoint{0.765196in}{1.122634in}}%
\pgfpathlineto{\pgfqpoint{0.767111in}{1.213314in}}%
\pgfpathlineto{\pgfqpoint{0.769026in}{1.256820in}}%
\pgfpathlineto{\pgfqpoint{0.770941in}{1.259493in}}%
\pgfpathlineto{\pgfqpoint{0.772856in}{1.275694in}}%
\pgfpathlineto{\pgfqpoint{0.774772in}{1.291976in}}%
\pgfpathlineto{\pgfqpoint{0.776687in}{1.293676in}}%
\pgfpathlineto{\pgfqpoint{0.778602in}{1.308716in}}%
\pgfpathlineto{\pgfqpoint{0.780517in}{1.309056in}}%
\pgfpathlineto{\pgfqpoint{0.782432in}{1.318285in}}%
\pgfpathlineto{\pgfqpoint{0.784347in}{1.345376in}}%
\pgfpathlineto{\pgfqpoint{0.786263in}{1.396385in}}%
\pgfpathlineto{\pgfqpoint{0.788178in}{1.397058in}}%
\pgfpathlineto{\pgfqpoint{0.790093in}{1.423520in}}%
\pgfpathlineto{\pgfqpoint{0.792008in}{1.428466in}}%
\pgfpathlineto{\pgfqpoint{0.793923in}{1.432641in}}%
\pgfpathlineto{\pgfqpoint{0.795838in}{1.454563in}}%
\pgfpathlineto{\pgfqpoint{0.797754in}{1.457583in}}%
\pgfpathlineto{\pgfqpoint{0.799669in}{1.458004in}}%
\pgfpathlineto{\pgfqpoint{0.801584in}{1.484056in}}%
\pgfpathlineto{\pgfqpoint{0.803499in}{1.491549in}}%
\pgfpathlineto{\pgfqpoint{0.805414in}{1.515088in}}%
\pgfpathlineto{\pgfqpoint{0.807329in}{1.535214in}}%
\pgfpathlineto{\pgfqpoint{0.809245in}{1.560549in}}%
\pgfpathlineto{\pgfqpoint{0.811160in}{1.561816in}}%
\pgfpathlineto{\pgfqpoint{0.813075in}{1.576569in}}%
\pgfpathlineto{\pgfqpoint{0.814990in}{1.592282in}}%
\pgfpathlineto{\pgfqpoint{0.816905in}{1.826535in}}%
\pgfusepath{stroke}%
\end{pgfscope}%
\begin{pgfscope}%
\pgfpathrectangle{\pgfqpoint{0.694334in}{0.523557in}}{\pgfqpoint{3.830343in}{1.302977in}}%
\pgfusepath{clip}%
\pgfsetrectcap%
\pgfsetroundjoin%
\pgfsetlinewidth{1.003750pt}%
\definecolor{currentstroke}{rgb}{0.752941,0.752941,1.000000}%
\pgfsetstrokecolor{currentstroke}%
\pgfsetdash{}{0pt}%
\pgfpathmoveto{\pgfqpoint{0.694334in}{0.753605in}}%
\pgfpathlineto{\pgfqpoint{0.730723in}{0.753605in}}%
\pgfpathlineto{\pgfqpoint{0.732638in}{0.768580in}}%
\pgfpathlineto{\pgfqpoint{0.767111in}{0.768580in}}%
\pgfpathlineto{\pgfqpoint{0.769026in}{0.781789in}}%
\pgfpathlineto{\pgfqpoint{0.776687in}{0.781789in}}%
\pgfpathlineto{\pgfqpoint{0.778602in}{0.793606in}}%
\pgfpathlineto{\pgfqpoint{0.807329in}{0.793606in}}%
\pgfpathlineto{\pgfqpoint{0.809245in}{0.804294in}}%
\pgfpathlineto{\pgfqpoint{0.839887in}{0.804294in}}%
\pgfpathlineto{\pgfqpoint{0.841803in}{0.814053in}}%
\pgfpathlineto{\pgfqpoint{0.868615in}{0.814053in}}%
\pgfpathlineto{\pgfqpoint{0.870530in}{0.823029in}}%
\pgfpathlineto{\pgfqpoint{0.899258in}{0.823029in}}%
\pgfpathlineto{\pgfqpoint{0.901173in}{0.831341in}}%
\pgfpathlineto{\pgfqpoint{0.929900in}{0.831341in}}%
\pgfpathlineto{\pgfqpoint{0.931816in}{0.839078in}}%
\pgfpathlineto{\pgfqpoint{0.972034in}{0.839078in}}%
\pgfpathlineto{\pgfqpoint{0.973949in}{0.846316in}}%
\pgfpathlineto{\pgfqpoint{1.000762in}{0.846316in}}%
\pgfpathlineto{\pgfqpoint{1.002677in}{0.853115in}}%
\pgfpathlineto{\pgfqpoint{1.037150in}{0.853115in}}%
\pgfpathlineto{\pgfqpoint{1.039065in}{0.859525in}}%
\pgfpathlineto{\pgfqpoint{1.058217in}{0.859525in}}%
\pgfpathlineto{\pgfqpoint{1.060132in}{0.865589in}}%
\pgfpathlineto{\pgfqpoint{1.073538in}{0.865589in}}%
\pgfpathlineto{\pgfqpoint{1.075453in}{0.871341in}}%
\pgfpathlineto{\pgfqpoint{1.100351in}{0.871341in}}%
\pgfpathlineto{\pgfqpoint{1.102266in}{0.876813in}}%
\pgfpathlineto{\pgfqpoint{1.111842in}{0.876813in}}%
\pgfpathlineto{\pgfqpoint{1.113757in}{0.882030in}}%
\pgfpathlineto{\pgfqpoint{1.115672in}{0.882030in}}%
\pgfpathlineto{\pgfqpoint{1.117587in}{0.887015in}}%
\pgfpathlineto{\pgfqpoint{1.119502in}{0.887015in}}%
\pgfpathlineto{\pgfqpoint{1.121418in}{0.891788in}}%
\pgfpathlineto{\pgfqpoint{1.134824in}{0.891788in}}%
\pgfpathlineto{\pgfqpoint{1.136739in}{0.896366in}}%
\pgfpathlineto{\pgfqpoint{1.146315in}{0.896366in}}%
\pgfpathlineto{\pgfqpoint{1.148230in}{0.900765in}}%
\pgfpathlineto{\pgfqpoint{1.169297in}{0.900765in}}%
\pgfpathlineto{\pgfqpoint{1.171212in}{0.904998in}}%
\pgfpathlineto{\pgfqpoint{1.178873in}{0.904998in}}%
\pgfpathlineto{\pgfqpoint{1.180788in}{0.909076in}}%
\pgfpathlineto{\pgfqpoint{1.184618in}{0.909076in}}%
\pgfpathlineto{\pgfqpoint{1.186533in}{0.916814in}}%
\pgfpathlineto{\pgfqpoint{1.188449in}{0.916814in}}%
\pgfpathlineto{\pgfqpoint{1.190364in}{0.920491in}}%
\pgfpathlineto{\pgfqpoint{1.192279in}{1.826535in}}%
\pgfpathlineto{\pgfqpoint{1.192279in}{1.826535in}}%
\pgfusepath{stroke}%
\end{pgfscope}%
\begin{pgfscope}%
\pgfpathrectangle{\pgfqpoint{0.694334in}{0.523557in}}{\pgfqpoint{3.830343in}{1.302977in}}%
\pgfusepath{clip}%
\pgfsetrectcap%
\pgfsetroundjoin%
\pgfsetlinewidth{1.003750pt}%
\definecolor{currentstroke}{rgb}{0.752941,0.752941,1.000000}%
\pgfsetstrokecolor{currentstroke}%
\pgfsetdash{}{0pt}%
\pgfpathmoveto{\pgfqpoint{0.694334in}{0.753605in}}%
\pgfpathlineto{\pgfqpoint{0.696249in}{0.753605in}}%
\pgfpathlineto{\pgfqpoint{0.698165in}{0.753605in}}%
\pgfpathlineto{\pgfqpoint{0.700080in}{0.753605in}}%
\pgfpathlineto{\pgfqpoint{0.701995in}{0.753605in}}%
\pgfpathlineto{\pgfqpoint{0.703910in}{0.768580in}}%
\pgfpathlineto{\pgfqpoint{0.705825in}{0.768580in}}%
\pgfpathlineto{\pgfqpoint{0.707741in}{0.768580in}}%
\pgfpathlineto{\pgfqpoint{0.709656in}{0.768580in}}%
\pgfpathlineto{\pgfqpoint{0.711571in}{0.768580in}}%
\pgfpathlineto{\pgfqpoint{0.713486in}{0.768580in}}%
\pgfpathlineto{\pgfqpoint{0.715401in}{0.768580in}}%
\pgfpathlineto{\pgfqpoint{0.717316in}{0.768580in}}%
\pgfpathlineto{\pgfqpoint{0.719232in}{0.781789in}}%
\pgfpathlineto{\pgfqpoint{0.721147in}{0.781789in}}%
\pgfpathlineto{\pgfqpoint{0.723062in}{0.781789in}}%
\pgfpathlineto{\pgfqpoint{0.724977in}{0.793606in}}%
\pgfpathlineto{\pgfqpoint{0.726892in}{0.793606in}}%
\pgfpathlineto{\pgfqpoint{0.728807in}{0.793606in}}%
\pgfpathlineto{\pgfqpoint{0.730723in}{0.793606in}}%
\pgfpathlineto{\pgfqpoint{0.732638in}{0.793606in}}%
\pgfpathlineto{\pgfqpoint{0.734553in}{0.793606in}}%
\pgfpathlineto{\pgfqpoint{0.736468in}{0.793606in}}%
\pgfpathlineto{\pgfqpoint{0.738383in}{0.804294in}}%
\pgfpathlineto{\pgfqpoint{0.740298in}{0.804294in}}%
\pgfpathlineto{\pgfqpoint{0.742214in}{0.804294in}}%
\pgfpathlineto{\pgfqpoint{0.744129in}{0.814053in}}%
\pgfpathlineto{\pgfqpoint{0.746044in}{0.814053in}}%
\pgfpathlineto{\pgfqpoint{0.747959in}{0.823029in}}%
\pgfpathlineto{\pgfqpoint{0.749874in}{0.831341in}}%
\pgfpathlineto{\pgfqpoint{0.751789in}{0.846316in}}%
\pgfpathlineto{\pgfqpoint{0.753705in}{0.859525in}}%
\pgfpathlineto{\pgfqpoint{0.755620in}{0.865589in}}%
\pgfpathlineto{\pgfqpoint{0.757535in}{0.871341in}}%
\pgfpathlineto{\pgfqpoint{0.759450in}{0.876813in}}%
\pgfpathlineto{\pgfqpoint{0.761365in}{0.882030in}}%
\pgfpathlineto{\pgfqpoint{0.763280in}{0.882030in}}%
\pgfpathlineto{\pgfqpoint{0.765196in}{0.882030in}}%
\pgfpathlineto{\pgfqpoint{0.767111in}{0.891788in}}%
\pgfpathlineto{\pgfqpoint{0.769026in}{0.896366in}}%
\pgfpathlineto{\pgfqpoint{0.770941in}{1.026813in}}%
\pgfpathlineto{\pgfqpoint{0.772856in}{1.095045in}}%
\pgfpathlineto{\pgfqpoint{0.774772in}{1.209805in}}%
\pgfpathlineto{\pgfqpoint{0.776687in}{1.273695in}}%
\pgfpathlineto{\pgfqpoint{0.778602in}{1.324863in}}%
\pgfpathlineto{\pgfqpoint{0.780517in}{1.410244in}}%
\pgfpathlineto{\pgfqpoint{0.782432in}{1.477195in}}%
\pgfpathlineto{\pgfqpoint{0.784347in}{1.497355in}}%
\pgfpathlineto{\pgfqpoint{0.786263in}{1.826535in}}%
\pgfusepath{stroke}%
\end{pgfscope}%
\begin{pgfscope}%
\pgfpathrectangle{\pgfqpoint{0.694334in}{0.523557in}}{\pgfqpoint{3.830343in}{1.302977in}}%
\pgfusepath{clip}%
\pgfsetrectcap%
\pgfsetroundjoin%
\pgfsetlinewidth{1.003750pt}%
\definecolor{currentstroke}{rgb}{0.752941,0.752941,1.000000}%
\pgfsetstrokecolor{currentstroke}%
\pgfsetdash{}{0pt}%
\pgfpathmoveto{\pgfqpoint{0.694334in}{0.736317in}}%
\pgfpathlineto{\pgfqpoint{0.696249in}{0.753605in}}%
\pgfpathlineto{\pgfqpoint{0.698165in}{0.753605in}}%
\pgfpathlineto{\pgfqpoint{0.700080in}{0.753605in}}%
\pgfpathlineto{\pgfqpoint{0.701995in}{0.753605in}}%
\pgfpathlineto{\pgfqpoint{0.703910in}{0.753605in}}%
\pgfpathlineto{\pgfqpoint{0.705825in}{0.753605in}}%
\pgfpathlineto{\pgfqpoint{0.707741in}{0.753605in}}%
\pgfpathlineto{\pgfqpoint{0.709656in}{0.753605in}}%
\pgfpathlineto{\pgfqpoint{0.711571in}{0.768580in}}%
\pgfpathlineto{\pgfqpoint{0.713486in}{0.768580in}}%
\pgfpathlineto{\pgfqpoint{0.715401in}{0.768580in}}%
\pgfpathlineto{\pgfqpoint{0.717316in}{0.768580in}}%
\pgfpathlineto{\pgfqpoint{0.719232in}{0.768580in}}%
\pgfpathlineto{\pgfqpoint{0.721147in}{0.768580in}}%
\pgfpathlineto{\pgfqpoint{0.723062in}{0.768580in}}%
\pgfpathlineto{\pgfqpoint{0.724977in}{0.768580in}}%
\pgfpathlineto{\pgfqpoint{0.726892in}{0.768580in}}%
\pgfpathlineto{\pgfqpoint{0.728807in}{0.768580in}}%
\pgfpathlineto{\pgfqpoint{0.730723in}{0.768580in}}%
\pgfpathlineto{\pgfqpoint{0.732638in}{0.793606in}}%
\pgfpathlineto{\pgfqpoint{0.734553in}{0.793606in}}%
\pgfpathlineto{\pgfqpoint{0.736468in}{0.793606in}}%
\pgfpathlineto{\pgfqpoint{0.738383in}{0.793606in}}%
\pgfpathlineto{\pgfqpoint{0.740298in}{0.793606in}}%
\pgfpathlineto{\pgfqpoint{0.742214in}{0.804294in}}%
\pgfpathlineto{\pgfqpoint{0.744129in}{0.804294in}}%
\pgfpathlineto{\pgfqpoint{0.746044in}{0.814053in}}%
\pgfpathlineto{\pgfqpoint{0.747959in}{0.823029in}}%
\pgfpathlineto{\pgfqpoint{0.749874in}{0.823029in}}%
\pgfpathlineto{\pgfqpoint{0.751789in}{0.831341in}}%
\pgfpathlineto{\pgfqpoint{0.753705in}{0.846316in}}%
\pgfpathlineto{\pgfqpoint{0.755620in}{0.859525in}}%
\pgfpathlineto{\pgfqpoint{0.757535in}{0.865589in}}%
\pgfpathlineto{\pgfqpoint{0.759450in}{0.896366in}}%
\pgfpathlineto{\pgfqpoint{0.761365in}{0.904998in}}%
\pgfpathlineto{\pgfqpoint{0.763280in}{0.930851in}}%
\pgfpathlineto{\pgfqpoint{0.765196in}{0.946237in}}%
\pgfpathlineto{\pgfqpoint{0.767111in}{0.951846in}}%
\pgfpathlineto{\pgfqpoint{0.769026in}{0.962286in}}%
\pgfpathlineto{\pgfqpoint{0.770941in}{0.971836in}}%
\pgfpathlineto{\pgfqpoint{0.772856in}{0.986812in}}%
\pgfpathlineto{\pgfqpoint{0.774772in}{0.988797in}}%
\pgfpathlineto{\pgfqpoint{0.776687in}{1.051838in}}%
\pgfpathlineto{\pgfqpoint{0.778602in}{1.060469in}}%
\pgfpathlineto{\pgfqpoint{0.780517in}{1.084660in}}%
\pgfpathlineto{\pgfqpoint{0.782432in}{1.119611in}}%
\pgfpathlineto{\pgfqpoint{0.784347in}{1.224917in}}%
\pgfpathlineto{\pgfqpoint{0.786263in}{1.281917in}}%
\pgfpathlineto{\pgfqpoint{0.788178in}{1.826535in}}%
\pgfusepath{stroke}%
\end{pgfscope}%
\begin{pgfscope}%
\pgfpathrectangle{\pgfqpoint{0.694334in}{0.523557in}}{\pgfqpoint{3.830343in}{1.302977in}}%
\pgfusepath{clip}%
\pgfsetrectcap%
\pgfsetroundjoin%
\pgfsetlinewidth{1.003750pt}%
\definecolor{currentstroke}{rgb}{0.752941,0.752941,1.000000}%
\pgfsetstrokecolor{currentstroke}%
\pgfsetdash{}{0pt}%
\pgfpathmoveto{\pgfqpoint{0.694334in}{0.753605in}}%
\pgfpathlineto{\pgfqpoint{0.726892in}{0.753605in}}%
\pgfpathlineto{\pgfqpoint{0.728807in}{0.768580in}}%
\pgfpathlineto{\pgfqpoint{0.763280in}{0.768580in}}%
\pgfpathlineto{\pgfqpoint{0.765196in}{0.781789in}}%
\pgfpathlineto{\pgfqpoint{0.780517in}{0.781789in}}%
\pgfpathlineto{\pgfqpoint{0.782432in}{0.793606in}}%
\pgfpathlineto{\pgfqpoint{0.803499in}{0.793606in}}%
\pgfpathlineto{\pgfqpoint{0.805414in}{0.804294in}}%
\pgfpathlineto{\pgfqpoint{0.826481in}{0.804294in}}%
\pgfpathlineto{\pgfqpoint{0.828396in}{0.814053in}}%
\pgfpathlineto{\pgfqpoint{0.843718in}{0.814053in}}%
\pgfpathlineto{\pgfqpoint{0.845633in}{0.823029in}}%
\pgfpathlineto{\pgfqpoint{0.868615in}{0.823029in}}%
\pgfpathlineto{\pgfqpoint{0.870530in}{0.831341in}}%
\pgfpathlineto{\pgfqpoint{0.891597in}{0.831341in}}%
\pgfpathlineto{\pgfqpoint{0.893512in}{0.839078in}}%
\pgfpathlineto{\pgfqpoint{0.914579in}{0.839078in}}%
\pgfpathlineto{\pgfqpoint{0.916494in}{0.846316in}}%
\pgfpathlineto{\pgfqpoint{0.939476in}{0.846316in}}%
\pgfpathlineto{\pgfqpoint{0.941391in}{0.853115in}}%
\pgfpathlineto{\pgfqpoint{0.970119in}{0.853115in}}%
\pgfpathlineto{\pgfqpoint{0.972034in}{0.859525in}}%
\pgfpathlineto{\pgfqpoint{0.995016in}{0.859525in}}%
\pgfpathlineto{\pgfqpoint{0.996931in}{0.865589in}}%
\pgfpathlineto{\pgfqpoint{1.016083in}{0.865589in}}%
\pgfpathlineto{\pgfqpoint{1.017998in}{0.871341in}}%
\pgfpathlineto{\pgfqpoint{1.033320in}{0.871341in}}%
\pgfpathlineto{\pgfqpoint{1.035235in}{0.876813in}}%
\pgfpathlineto{\pgfqpoint{1.067793in}{0.876813in}}%
\pgfpathlineto{\pgfqpoint{1.069708in}{0.882030in}}%
\pgfpathlineto{\pgfqpoint{1.085029in}{0.882030in}}%
\pgfpathlineto{\pgfqpoint{1.086944in}{0.887015in}}%
\pgfpathlineto{\pgfqpoint{1.104181in}{0.887015in}}%
\pgfpathlineto{\pgfqpoint{1.106096in}{0.891788in}}%
\pgfpathlineto{\pgfqpoint{1.129078in}{0.891788in}}%
\pgfpathlineto{\pgfqpoint{1.130993in}{0.896366in}}%
\pgfpathlineto{\pgfqpoint{1.138654in}{0.896366in}}%
\pgfpathlineto{\pgfqpoint{1.140569in}{0.900765in}}%
\pgfpathlineto{\pgfqpoint{1.155891in}{0.900765in}}%
\pgfpathlineto{\pgfqpoint{1.157806in}{0.904998in}}%
\pgfpathlineto{\pgfqpoint{1.169297in}{0.904998in}}%
\pgfpathlineto{\pgfqpoint{1.171212in}{0.909076in}}%
\pgfpathlineto{\pgfqpoint{1.188449in}{0.909076in}}%
\pgfpathlineto{\pgfqpoint{1.190364in}{0.913012in}}%
\pgfpathlineto{\pgfqpoint{1.199940in}{0.913012in}}%
\pgfpathlineto{\pgfqpoint{1.201855in}{0.916814in}}%
\pgfpathlineto{\pgfqpoint{1.213346in}{0.916814in}}%
\pgfpathlineto{\pgfqpoint{1.215261in}{0.920491in}}%
\pgfpathlineto{\pgfqpoint{1.224837in}{0.920491in}}%
\pgfpathlineto{\pgfqpoint{1.226752in}{0.924052in}}%
\pgfpathlineto{\pgfqpoint{1.240158in}{0.924052in}}%
\pgfpathlineto{\pgfqpoint{1.242073in}{0.927503in}}%
\pgfpathlineto{\pgfqpoint{1.245904in}{0.927503in}}%
\pgfpathlineto{\pgfqpoint{1.247819in}{0.930851in}}%
\pgfpathlineto{\pgfqpoint{1.263140in}{0.930851in}}%
\pgfpathlineto{\pgfqpoint{1.265055in}{0.934101in}}%
\pgfpathlineto{\pgfqpoint{1.276546in}{0.934101in}}%
\pgfpathlineto{\pgfqpoint{1.278462in}{0.937261in}}%
\pgfpathlineto{\pgfqpoint{1.289953in}{0.937261in}}%
\pgfpathlineto{\pgfqpoint{1.291868in}{0.940334in}}%
\pgfpathlineto{\pgfqpoint{1.305274in}{0.940334in}}%
\pgfpathlineto{\pgfqpoint{1.307189in}{0.943324in}}%
\pgfpathlineto{\pgfqpoint{1.322511in}{0.943324in}}%
\pgfpathlineto{\pgfqpoint{1.324426in}{0.946237in}}%
\pgfpathlineto{\pgfqpoint{1.337832in}{0.946237in}}%
\pgfpathlineto{\pgfqpoint{1.339747in}{0.949077in}}%
\pgfpathlineto{\pgfqpoint{1.343577in}{0.949077in}}%
\pgfpathlineto{\pgfqpoint{1.345493in}{0.951846in}}%
\pgfpathlineto{\pgfqpoint{1.351238in}{0.951846in}}%
\pgfpathlineto{\pgfqpoint{1.353153in}{0.954549in}}%
\pgfpathlineto{\pgfqpoint{1.355068in}{0.954549in}}%
\pgfpathlineto{\pgfqpoint{1.356984in}{0.959766in}}%
\pgfpathlineto{\pgfqpoint{1.358899in}{0.959766in}}%
\pgfpathlineto{\pgfqpoint{1.360814in}{0.962286in}}%
\pgfpathlineto{\pgfqpoint{1.368475in}{0.962286in}}%
\pgfpathlineto{\pgfqpoint{1.370390in}{0.964751in}}%
\pgfpathlineto{\pgfqpoint{1.374220in}{0.964751in}}%
\pgfpathlineto{\pgfqpoint{1.376135in}{0.967163in}}%
\pgfpathlineto{\pgfqpoint{1.381881in}{0.967163in}}%
\pgfpathlineto{\pgfqpoint{1.383796in}{0.969524in}}%
\pgfpathlineto{\pgfqpoint{1.387626in}{0.969524in}}%
\pgfpathlineto{\pgfqpoint{1.391457in}{0.974102in}}%
\pgfpathlineto{\pgfqpoint{1.397202in}{0.974102in}}%
\pgfpathlineto{\pgfqpoint{1.399117in}{0.976323in}}%
\pgfpathlineto{\pgfqpoint{1.401033in}{0.976323in}}%
\pgfpathlineto{\pgfqpoint{1.402948in}{0.978501in}}%
\pgfpathlineto{\pgfqpoint{1.408693in}{0.978501in}}%
\pgfpathlineto{\pgfqpoint{1.410608in}{0.980637in}}%
\pgfpathlineto{\pgfqpoint{1.412524in}{0.980637in}}%
\pgfpathlineto{\pgfqpoint{1.414439in}{0.982733in}}%
\pgfpathlineto{\pgfqpoint{1.420184in}{0.982733in}}%
\pgfpathlineto{\pgfqpoint{1.424015in}{0.994549in}}%
\pgfpathlineto{\pgfqpoint{1.425930in}{1.090371in}}%
\pgfpathlineto{\pgfqpoint{1.427845in}{1.826535in}}%
\pgfpathlineto{\pgfqpoint{1.427845in}{1.826535in}}%
\pgfusepath{stroke}%
\end{pgfscope}%
\begin{pgfscope}%
\pgfpathrectangle{\pgfqpoint{0.694334in}{0.523557in}}{\pgfqpoint{3.830343in}{1.302977in}}%
\pgfusepath{clip}%
\pgfsetrectcap%
\pgfsetroundjoin%
\pgfsetlinewidth{1.003750pt}%
\definecolor{currentstroke}{rgb}{0.752941,0.752941,1.000000}%
\pgfsetstrokecolor{currentstroke}%
\pgfsetdash{}{0pt}%
\pgfpathmoveto{\pgfqpoint{0.694334in}{0.715870in}}%
\pgfpathlineto{\pgfqpoint{0.696249in}{0.753605in}}%
\pgfpathlineto{\pgfqpoint{0.698165in}{0.753605in}}%
\pgfpathlineto{\pgfqpoint{0.700080in}{0.753605in}}%
\pgfpathlineto{\pgfqpoint{0.701995in}{0.753605in}}%
\pgfpathlineto{\pgfqpoint{0.703910in}{0.753605in}}%
\pgfpathlineto{\pgfqpoint{0.705825in}{0.753605in}}%
\pgfpathlineto{\pgfqpoint{0.707741in}{0.753605in}}%
\pgfpathlineto{\pgfqpoint{0.709656in}{0.753605in}}%
\pgfpathlineto{\pgfqpoint{0.711571in}{0.753605in}}%
\pgfpathlineto{\pgfqpoint{0.713486in}{0.753605in}}%
\pgfpathlineto{\pgfqpoint{0.715401in}{0.753605in}}%
\pgfpathlineto{\pgfqpoint{0.717316in}{0.753605in}}%
\pgfpathlineto{\pgfqpoint{0.719232in}{0.768580in}}%
\pgfpathlineto{\pgfqpoint{0.721147in}{0.768580in}}%
\pgfpathlineto{\pgfqpoint{0.723062in}{0.768580in}}%
\pgfpathlineto{\pgfqpoint{0.724977in}{0.768580in}}%
\pgfpathlineto{\pgfqpoint{0.726892in}{0.768580in}}%
\pgfpathlineto{\pgfqpoint{0.728807in}{0.768580in}}%
\pgfpathlineto{\pgfqpoint{0.730723in}{0.768580in}}%
\pgfpathlineto{\pgfqpoint{0.732638in}{0.768580in}}%
\pgfpathlineto{\pgfqpoint{0.734553in}{0.768580in}}%
\pgfpathlineto{\pgfqpoint{0.736468in}{0.768580in}}%
\pgfpathlineto{\pgfqpoint{0.738383in}{0.768580in}}%
\pgfpathlineto{\pgfqpoint{0.740298in}{0.768580in}}%
\pgfpathlineto{\pgfqpoint{0.742214in}{0.768580in}}%
\pgfpathlineto{\pgfqpoint{0.744129in}{0.781789in}}%
\pgfpathlineto{\pgfqpoint{0.746044in}{0.793606in}}%
\pgfpathlineto{\pgfqpoint{0.747959in}{0.804294in}}%
\pgfpathlineto{\pgfqpoint{0.749874in}{0.804294in}}%
\pgfpathlineto{\pgfqpoint{0.751789in}{0.814053in}}%
\pgfpathlineto{\pgfqpoint{0.753705in}{0.814053in}}%
\pgfpathlineto{\pgfqpoint{0.755620in}{0.814053in}}%
\pgfpathlineto{\pgfqpoint{0.757535in}{0.823029in}}%
\pgfpathlineto{\pgfqpoint{0.759450in}{0.831341in}}%
\pgfpathlineto{\pgfqpoint{0.761365in}{0.865589in}}%
\pgfpathlineto{\pgfqpoint{0.763280in}{1.826535in}}%
\pgfusepath{stroke}%
\end{pgfscope}%
\begin{pgfscope}%
\pgfpathrectangle{\pgfqpoint{0.694334in}{0.523557in}}{\pgfqpoint{3.830343in}{1.302977in}}%
\pgfusepath{clip}%
\pgfsetrectcap%
\pgfsetroundjoin%
\pgfsetlinewidth{1.003750pt}%
\definecolor{currentstroke}{rgb}{0.752941,0.752941,1.000000}%
\pgfsetstrokecolor{currentstroke}%
\pgfsetdash{}{0pt}%
\pgfpathmoveto{\pgfqpoint{0.694334in}{0.753605in}}%
\pgfpathlineto{\pgfqpoint{0.705825in}{0.753605in}}%
\pgfpathlineto{\pgfqpoint{0.707741in}{0.768580in}}%
\pgfpathlineto{\pgfqpoint{0.715401in}{0.768580in}}%
\pgfpathlineto{\pgfqpoint{0.717316in}{0.781789in}}%
\pgfpathlineto{\pgfqpoint{0.723062in}{0.781789in}}%
\pgfpathlineto{\pgfqpoint{0.724977in}{0.793606in}}%
\pgfpathlineto{\pgfqpoint{0.732638in}{0.793606in}}%
\pgfpathlineto{\pgfqpoint{0.734553in}{0.804294in}}%
\pgfpathlineto{\pgfqpoint{0.751789in}{0.804294in}}%
\pgfpathlineto{\pgfqpoint{0.753705in}{0.814053in}}%
\pgfpathlineto{\pgfqpoint{0.761365in}{0.814053in}}%
\pgfpathlineto{\pgfqpoint{0.763280in}{0.823029in}}%
\pgfpathlineto{\pgfqpoint{0.772856in}{0.823029in}}%
\pgfpathlineto{\pgfqpoint{0.774772in}{0.831341in}}%
\pgfpathlineto{\pgfqpoint{0.792008in}{0.831341in}}%
\pgfpathlineto{\pgfqpoint{0.793923in}{0.839078in}}%
\pgfpathlineto{\pgfqpoint{0.816905in}{0.839078in}}%
\pgfpathlineto{\pgfqpoint{0.818820in}{0.846316in}}%
\pgfpathlineto{\pgfqpoint{0.839887in}{0.846316in}}%
\pgfpathlineto{\pgfqpoint{0.841803in}{0.853115in}}%
\pgfpathlineto{\pgfqpoint{0.866700in}{0.853115in}}%
\pgfpathlineto{\pgfqpoint{0.868615in}{0.859525in}}%
\pgfpathlineto{\pgfqpoint{0.905003in}{0.859525in}}%
\pgfpathlineto{\pgfqpoint{0.906918in}{0.865589in}}%
\pgfpathlineto{\pgfqpoint{0.929900in}{0.865589in}}%
\pgfpathlineto{\pgfqpoint{0.931816in}{0.871341in}}%
\pgfpathlineto{\pgfqpoint{0.960543in}{0.871341in}}%
\pgfpathlineto{\pgfqpoint{0.962458in}{0.876813in}}%
\pgfpathlineto{\pgfqpoint{0.968204in}{0.876813in}}%
\pgfpathlineto{\pgfqpoint{0.970119in}{0.882030in}}%
\pgfpathlineto{\pgfqpoint{0.989271in}{0.882030in}}%
\pgfpathlineto{\pgfqpoint{0.991186in}{0.887015in}}%
\pgfpathlineto{\pgfqpoint{0.998847in}{0.887015in}}%
\pgfpathlineto{\pgfqpoint{1.000762in}{0.891788in}}%
\pgfpathlineto{\pgfqpoint{1.014168in}{0.891788in}}%
\pgfpathlineto{\pgfqpoint{1.016083in}{0.896366in}}%
\pgfpathlineto{\pgfqpoint{1.037150in}{0.896366in}}%
\pgfpathlineto{\pgfqpoint{1.039065in}{0.900765in}}%
\pgfpathlineto{\pgfqpoint{1.046726in}{0.900765in}}%
\pgfpathlineto{\pgfqpoint{1.048641in}{0.904998in}}%
\pgfpathlineto{\pgfqpoint{1.058217in}{0.904998in}}%
\pgfpathlineto{\pgfqpoint{1.060132in}{0.909076in}}%
\pgfpathlineto{\pgfqpoint{1.071623in}{0.909076in}}%
\pgfpathlineto{\pgfqpoint{1.073538in}{0.913012in}}%
\pgfpathlineto{\pgfqpoint{1.083114in}{0.913012in}}%
\pgfpathlineto{\pgfqpoint{1.085029in}{0.916814in}}%
\pgfpathlineto{\pgfqpoint{1.092690in}{0.916814in}}%
\pgfpathlineto{\pgfqpoint{1.094605in}{0.920491in}}%
\pgfpathlineto{\pgfqpoint{1.100351in}{0.920491in}}%
\pgfpathlineto{\pgfqpoint{1.102266in}{0.924052in}}%
\pgfpathlineto{\pgfqpoint{1.117587in}{0.924052in}}%
\pgfpathlineto{\pgfqpoint{1.119502in}{0.927503in}}%
\pgfpathlineto{\pgfqpoint{1.130993in}{0.927503in}}%
\pgfpathlineto{\pgfqpoint{1.132909in}{0.930851in}}%
\pgfpathlineto{\pgfqpoint{1.140569in}{0.930851in}}%
\pgfpathlineto{\pgfqpoint{1.142484in}{0.934101in}}%
\pgfpathlineto{\pgfqpoint{1.153975in}{0.934101in}}%
\pgfpathlineto{\pgfqpoint{1.159721in}{0.943324in}}%
\pgfpathlineto{\pgfqpoint{1.161636in}{0.943324in}}%
\pgfpathlineto{\pgfqpoint{1.163551in}{0.946237in}}%
\pgfpathlineto{\pgfqpoint{1.167382in}{0.946237in}}%
\pgfpathlineto{\pgfqpoint{1.173127in}{0.962286in}}%
\pgfpathlineto{\pgfqpoint{1.175042in}{0.964751in}}%
\pgfpathlineto{\pgfqpoint{1.176957in}{0.978501in}}%
\pgfpathlineto{\pgfqpoint{1.178873in}{1.826535in}}%
\pgfpathlineto{\pgfqpoint{1.178873in}{1.826535in}}%
\pgfusepath{stroke}%
\end{pgfscope}%
\begin{pgfscope}%
\pgfpathrectangle{\pgfqpoint{0.694334in}{0.523557in}}{\pgfqpoint{3.830343in}{1.302977in}}%
\pgfusepath{clip}%
\pgfsetrectcap%
\pgfsetroundjoin%
\pgfsetlinewidth{1.003750pt}%
\definecolor{currentstroke}{rgb}{0.752941,0.752941,1.000000}%
\pgfsetstrokecolor{currentstroke}%
\pgfsetdash{}{0pt}%
\pgfpathmoveto{\pgfqpoint{0.694334in}{0.753605in}}%
\pgfpathlineto{\pgfqpoint{0.717316in}{0.753605in}}%
\pgfpathlineto{\pgfqpoint{0.719232in}{0.768580in}}%
\pgfpathlineto{\pgfqpoint{0.740298in}{0.768580in}}%
\pgfpathlineto{\pgfqpoint{0.742214in}{0.781789in}}%
\pgfpathlineto{\pgfqpoint{0.755620in}{0.781789in}}%
\pgfpathlineto{\pgfqpoint{0.757535in}{0.793606in}}%
\pgfpathlineto{\pgfqpoint{0.767111in}{0.793606in}}%
\pgfpathlineto{\pgfqpoint{0.769026in}{0.804294in}}%
\pgfpathlineto{\pgfqpoint{0.790093in}{0.804294in}}%
\pgfpathlineto{\pgfqpoint{0.792008in}{0.814053in}}%
\pgfpathlineto{\pgfqpoint{0.814990in}{0.814053in}}%
\pgfpathlineto{\pgfqpoint{0.816905in}{0.823029in}}%
\pgfpathlineto{\pgfqpoint{0.834142in}{0.823029in}}%
\pgfpathlineto{\pgfqpoint{0.836057in}{0.831341in}}%
\pgfpathlineto{\pgfqpoint{0.851378in}{0.831341in}}%
\pgfpathlineto{\pgfqpoint{0.853294in}{0.839078in}}%
\pgfpathlineto{\pgfqpoint{0.876276in}{0.839078in}}%
\pgfpathlineto{\pgfqpoint{0.878191in}{0.846316in}}%
\pgfpathlineto{\pgfqpoint{0.897342in}{0.846316in}}%
\pgfpathlineto{\pgfqpoint{0.899258in}{0.853115in}}%
\pgfpathlineto{\pgfqpoint{0.920325in}{0.853115in}}%
\pgfpathlineto{\pgfqpoint{0.922240in}{0.859525in}}%
\pgfpathlineto{\pgfqpoint{0.950967in}{0.859525in}}%
\pgfpathlineto{\pgfqpoint{0.952882in}{0.865589in}}%
\pgfpathlineto{\pgfqpoint{0.968204in}{0.865589in}}%
\pgfpathlineto{\pgfqpoint{0.970119in}{0.871341in}}%
\pgfpathlineto{\pgfqpoint{0.989271in}{0.871341in}}%
\pgfpathlineto{\pgfqpoint{0.991186in}{0.876813in}}%
\pgfpathlineto{\pgfqpoint{1.017998in}{0.876813in}}%
\pgfpathlineto{\pgfqpoint{1.019913in}{0.882030in}}%
\pgfpathlineto{\pgfqpoint{1.040980in}{0.882030in}}%
\pgfpathlineto{\pgfqpoint{1.042895in}{0.887015in}}%
\pgfpathlineto{\pgfqpoint{1.056302in}{0.887015in}}%
\pgfpathlineto{\pgfqpoint{1.058217in}{0.891788in}}%
\pgfpathlineto{\pgfqpoint{1.073538in}{0.891788in}}%
\pgfpathlineto{\pgfqpoint{1.075453in}{0.896366in}}%
\pgfpathlineto{\pgfqpoint{1.094605in}{0.896366in}}%
\pgfpathlineto{\pgfqpoint{1.096520in}{0.900765in}}%
\pgfpathlineto{\pgfqpoint{1.113757in}{0.900765in}}%
\pgfpathlineto{\pgfqpoint{1.115672in}{0.904998in}}%
\pgfpathlineto{\pgfqpoint{1.132909in}{0.904998in}}%
\pgfpathlineto{\pgfqpoint{1.134824in}{0.909076in}}%
\pgfpathlineto{\pgfqpoint{1.152060in}{0.909076in}}%
\pgfpathlineto{\pgfqpoint{1.153975in}{0.913012in}}%
\pgfpathlineto{\pgfqpoint{1.175042in}{0.913012in}}%
\pgfpathlineto{\pgfqpoint{1.176957in}{0.916814in}}%
\pgfpathlineto{\pgfqpoint{1.196109in}{0.916814in}}%
\pgfpathlineto{\pgfqpoint{1.198024in}{0.920491in}}%
\pgfpathlineto{\pgfqpoint{1.207600in}{0.920491in}}%
\pgfpathlineto{\pgfqpoint{1.209515in}{0.924052in}}%
\pgfpathlineto{\pgfqpoint{1.224837in}{0.924052in}}%
\pgfpathlineto{\pgfqpoint{1.226752in}{0.927503in}}%
\pgfpathlineto{\pgfqpoint{1.243988in}{0.927503in}}%
\pgfpathlineto{\pgfqpoint{1.245904in}{0.930851in}}%
\pgfpathlineto{\pgfqpoint{1.255480in}{0.930851in}}%
\pgfpathlineto{\pgfqpoint{1.257395in}{0.934101in}}%
\pgfpathlineto{\pgfqpoint{1.265055in}{0.934101in}}%
\pgfpathlineto{\pgfqpoint{1.266971in}{0.937261in}}%
\pgfpathlineto{\pgfqpoint{1.280377in}{0.937261in}}%
\pgfpathlineto{\pgfqpoint{1.282292in}{0.940334in}}%
\pgfpathlineto{\pgfqpoint{1.291868in}{0.940334in}}%
\pgfpathlineto{\pgfqpoint{1.293783in}{0.943324in}}%
\pgfpathlineto{\pgfqpoint{1.303359in}{0.943324in}}%
\pgfpathlineto{\pgfqpoint{1.305274in}{0.946237in}}%
\pgfpathlineto{\pgfqpoint{1.316765in}{0.946237in}}%
\pgfpathlineto{\pgfqpoint{1.318680in}{0.949077in}}%
\pgfpathlineto{\pgfqpoint{1.324426in}{0.949077in}}%
\pgfpathlineto{\pgfqpoint{1.326341in}{0.951846in}}%
\pgfpathlineto{\pgfqpoint{1.332086in}{0.951846in}}%
\pgfpathlineto{\pgfqpoint{1.334002in}{0.954549in}}%
\pgfpathlineto{\pgfqpoint{1.345493in}{0.954549in}}%
\pgfpathlineto{\pgfqpoint{1.347408in}{0.957188in}}%
\pgfpathlineto{\pgfqpoint{1.349323in}{0.957188in}}%
\pgfpathlineto{\pgfqpoint{1.351238in}{0.959766in}}%
\pgfpathlineto{\pgfqpoint{1.360814in}{0.959766in}}%
\pgfpathlineto{\pgfqpoint{1.362729in}{0.962286in}}%
\pgfpathlineto{\pgfqpoint{1.364644in}{0.962286in}}%
\pgfpathlineto{\pgfqpoint{1.366559in}{0.964751in}}%
\pgfpathlineto{\pgfqpoint{1.368475in}{0.964751in}}%
\pgfpathlineto{\pgfqpoint{1.370390in}{0.967163in}}%
\pgfpathlineto{\pgfqpoint{1.374220in}{0.967163in}}%
\pgfpathlineto{\pgfqpoint{1.376135in}{0.969524in}}%
\pgfpathlineto{\pgfqpoint{1.381881in}{0.969524in}}%
\pgfpathlineto{\pgfqpoint{1.383796in}{0.971836in}}%
\pgfpathlineto{\pgfqpoint{1.385711in}{0.971836in}}%
\pgfpathlineto{\pgfqpoint{1.387626in}{0.974102in}}%
\pgfpathlineto{\pgfqpoint{1.391457in}{0.974102in}}%
\pgfpathlineto{\pgfqpoint{1.393372in}{0.976323in}}%
\pgfpathlineto{\pgfqpoint{1.395287in}{0.976323in}}%
\pgfpathlineto{\pgfqpoint{1.399117in}{0.980637in}}%
\pgfpathlineto{\pgfqpoint{1.401033in}{0.980637in}}%
\pgfpathlineto{\pgfqpoint{1.402948in}{0.982733in}}%
\pgfpathlineto{\pgfqpoint{1.404863in}{0.982733in}}%
\pgfpathlineto{\pgfqpoint{1.406778in}{0.984791in}}%
\pgfpathlineto{\pgfqpoint{1.412524in}{0.984791in}}%
\pgfpathlineto{\pgfqpoint{1.414439in}{0.986812in}}%
\pgfpathlineto{\pgfqpoint{1.418269in}{0.986812in}}%
\pgfpathlineto{\pgfqpoint{1.422099in}{0.992664in}}%
\pgfpathlineto{\pgfqpoint{1.424015in}{0.992664in}}%
\pgfpathlineto{\pgfqpoint{1.425930in}{0.994549in}}%
\pgfpathlineto{\pgfqpoint{1.427845in}{0.994549in}}%
\pgfpathlineto{\pgfqpoint{1.429760in}{0.998227in}}%
\pgfpathlineto{\pgfqpoint{1.431675in}{1.826535in}}%
\pgfpathlineto{\pgfqpoint{1.431675in}{1.826535in}}%
\pgfusepath{stroke}%
\end{pgfscope}%
\begin{pgfscope}%
\pgfpathrectangle{\pgfqpoint{0.694334in}{0.523557in}}{\pgfqpoint{3.830343in}{1.302977in}}%
\pgfusepath{clip}%
\pgfsetrectcap%
\pgfsetroundjoin%
\pgfsetlinewidth{1.003750pt}%
\definecolor{currentstroke}{rgb}{0.752941,0.752941,1.000000}%
\pgfsetstrokecolor{currentstroke}%
\pgfsetdash{}{0pt}%
\pgfpathmoveto{\pgfqpoint{0.694334in}{0.753605in}}%
\pgfpathlineto{\pgfqpoint{0.696249in}{0.753605in}}%
\pgfpathlineto{\pgfqpoint{0.698165in}{0.753605in}}%
\pgfpathlineto{\pgfqpoint{0.700080in}{0.753605in}}%
\pgfpathlineto{\pgfqpoint{0.701995in}{0.768580in}}%
\pgfpathlineto{\pgfqpoint{0.703910in}{0.768580in}}%
\pgfpathlineto{\pgfqpoint{0.705825in}{0.768580in}}%
\pgfpathlineto{\pgfqpoint{0.707741in}{0.768580in}}%
\pgfpathlineto{\pgfqpoint{0.709656in}{0.768580in}}%
\pgfpathlineto{\pgfqpoint{0.711571in}{0.781789in}}%
\pgfpathlineto{\pgfqpoint{0.713486in}{0.781789in}}%
\pgfpathlineto{\pgfqpoint{0.715401in}{0.781789in}}%
\pgfpathlineto{\pgfqpoint{0.717316in}{0.781789in}}%
\pgfpathlineto{\pgfqpoint{0.719232in}{0.781789in}}%
\pgfpathlineto{\pgfqpoint{0.721147in}{0.793606in}}%
\pgfpathlineto{\pgfqpoint{0.723062in}{0.793606in}}%
\pgfpathlineto{\pgfqpoint{0.724977in}{0.793606in}}%
\pgfpathlineto{\pgfqpoint{0.726892in}{0.793606in}}%
\pgfpathlineto{\pgfqpoint{0.728807in}{0.793606in}}%
\pgfpathlineto{\pgfqpoint{0.730723in}{0.793606in}}%
\pgfpathlineto{\pgfqpoint{0.732638in}{0.793606in}}%
\pgfpathlineto{\pgfqpoint{0.734553in}{0.804294in}}%
\pgfpathlineto{\pgfqpoint{0.736468in}{0.804294in}}%
\pgfpathlineto{\pgfqpoint{0.738383in}{0.804294in}}%
\pgfpathlineto{\pgfqpoint{0.740298in}{0.814053in}}%
\pgfpathlineto{\pgfqpoint{0.742214in}{0.814053in}}%
\pgfpathlineto{\pgfqpoint{0.744129in}{0.814053in}}%
\pgfpathlineto{\pgfqpoint{0.746044in}{0.814053in}}%
\pgfpathlineto{\pgfqpoint{0.747959in}{0.823029in}}%
\pgfpathlineto{\pgfqpoint{0.749874in}{0.831341in}}%
\pgfpathlineto{\pgfqpoint{0.751789in}{0.853115in}}%
\pgfpathlineto{\pgfqpoint{0.753705in}{0.865589in}}%
\pgfpathlineto{\pgfqpoint{0.755620in}{0.865589in}}%
\pgfpathlineto{\pgfqpoint{0.757535in}{0.871341in}}%
\pgfpathlineto{\pgfqpoint{0.759450in}{0.871341in}}%
\pgfpathlineto{\pgfqpoint{0.761365in}{0.871341in}}%
\pgfpathlineto{\pgfqpoint{0.763280in}{0.876813in}}%
\pgfpathlineto{\pgfqpoint{0.765196in}{0.882030in}}%
\pgfpathlineto{\pgfqpoint{0.767111in}{0.896366in}}%
\pgfpathlineto{\pgfqpoint{0.769026in}{0.900765in}}%
\pgfpathlineto{\pgfqpoint{0.770941in}{1.022526in}}%
\pgfpathlineto{\pgfqpoint{0.772856in}{1.034923in}}%
\pgfpathlineto{\pgfqpoint{0.774772in}{1.060469in}}%
\pgfpathlineto{\pgfqpoint{0.776687in}{1.062527in}}%
\pgfpathlineto{\pgfqpoint{0.778602in}{1.063542in}}%
\pgfpathlineto{\pgfqpoint{0.780517in}{1.063542in}}%
\pgfpathlineto{\pgfqpoint{0.782432in}{1.130689in}}%
\pgfpathlineto{\pgfqpoint{0.784347in}{1.139751in}}%
\pgfpathlineto{\pgfqpoint{0.786263in}{1.204182in}}%
\pgfpathlineto{\pgfqpoint{0.788178in}{1.237163in}}%
\pgfpathlineto{\pgfqpoint{0.790093in}{1.261066in}}%
\pgfpathlineto{\pgfqpoint{0.792008in}{1.264654in}}%
\pgfpathlineto{\pgfqpoint{0.793923in}{1.270389in}}%
\pgfpathlineto{\pgfqpoint{0.795838in}{1.275541in}}%
\pgfpathlineto{\pgfqpoint{0.797754in}{1.284342in}}%
\pgfpathlineto{\pgfqpoint{0.799669in}{1.477245in}}%
\pgfpathlineto{\pgfqpoint{0.801584in}{1.497355in}}%
\pgfpathlineto{\pgfqpoint{0.803499in}{1.826535in}}%
\pgfusepath{stroke}%
\end{pgfscope}%
\begin{pgfscope}%
\pgfpathrectangle{\pgfqpoint{0.694334in}{0.523557in}}{\pgfqpoint{3.830343in}{1.302977in}}%
\pgfusepath{clip}%
\pgfsetrectcap%
\pgfsetroundjoin%
\pgfsetlinewidth{1.003750pt}%
\definecolor{currentstroke}{rgb}{0.752941,0.752941,1.000000}%
\pgfsetstrokecolor{currentstroke}%
\pgfsetdash{}{0pt}%
\pgfpathmoveto{\pgfqpoint{0.694334in}{0.753605in}}%
\pgfpathlineto{\pgfqpoint{0.696249in}{0.753605in}}%
\pgfpathlineto{\pgfqpoint{0.698165in}{0.753605in}}%
\pgfpathlineto{\pgfqpoint{0.700080in}{0.753605in}}%
\pgfpathlineto{\pgfqpoint{0.701995in}{0.753605in}}%
\pgfpathlineto{\pgfqpoint{0.703910in}{0.753605in}}%
\pgfpathlineto{\pgfqpoint{0.705825in}{0.753605in}}%
\pgfpathlineto{\pgfqpoint{0.707741in}{0.768580in}}%
\pgfpathlineto{\pgfqpoint{0.709656in}{0.768580in}}%
\pgfpathlineto{\pgfqpoint{0.711571in}{0.768580in}}%
\pgfpathlineto{\pgfqpoint{0.713486in}{0.768580in}}%
\pgfpathlineto{\pgfqpoint{0.715401in}{0.768580in}}%
\pgfpathlineto{\pgfqpoint{0.717316in}{0.768580in}}%
\pgfpathlineto{\pgfqpoint{0.719232in}{0.768580in}}%
\pgfpathlineto{\pgfqpoint{0.721147in}{0.768580in}}%
\pgfpathlineto{\pgfqpoint{0.723062in}{0.768580in}}%
\pgfpathlineto{\pgfqpoint{0.724977in}{0.781789in}}%
\pgfpathlineto{\pgfqpoint{0.726892in}{0.781789in}}%
\pgfpathlineto{\pgfqpoint{0.728807in}{0.781789in}}%
\pgfpathlineto{\pgfqpoint{0.730723in}{0.793606in}}%
\pgfpathlineto{\pgfqpoint{0.732638in}{0.793606in}}%
\pgfpathlineto{\pgfqpoint{0.734553in}{0.793606in}}%
\pgfpathlineto{\pgfqpoint{0.736468in}{0.793606in}}%
\pgfpathlineto{\pgfqpoint{0.738383in}{0.804294in}}%
\pgfpathlineto{\pgfqpoint{0.740298in}{0.804294in}}%
\pgfpathlineto{\pgfqpoint{0.742214in}{0.814053in}}%
\pgfpathlineto{\pgfqpoint{0.744129in}{0.814053in}}%
\pgfpathlineto{\pgfqpoint{0.746044in}{0.823029in}}%
\pgfpathlineto{\pgfqpoint{0.747959in}{0.823029in}}%
\pgfpathlineto{\pgfqpoint{0.749874in}{0.831341in}}%
\pgfpathlineto{\pgfqpoint{0.751789in}{0.839078in}}%
\pgfpathlineto{\pgfqpoint{0.753705in}{0.846316in}}%
\pgfpathlineto{\pgfqpoint{0.755620in}{0.853115in}}%
\pgfpathlineto{\pgfqpoint{0.757535in}{0.853115in}}%
\pgfpathlineto{\pgfqpoint{0.759450in}{0.871341in}}%
\pgfpathlineto{\pgfqpoint{0.761365in}{0.871341in}}%
\pgfpathlineto{\pgfqpoint{0.763280in}{0.871341in}}%
\pgfpathlineto{\pgfqpoint{0.765196in}{0.913012in}}%
\pgfpathlineto{\pgfqpoint{0.767111in}{0.930851in}}%
\pgfpathlineto{\pgfqpoint{0.769026in}{0.934101in}}%
\pgfpathlineto{\pgfqpoint{0.770941in}{0.951846in}}%
\pgfpathlineto{\pgfqpoint{0.772856in}{0.951846in}}%
\pgfpathlineto{\pgfqpoint{0.774772in}{0.964751in}}%
\pgfpathlineto{\pgfqpoint{0.776687in}{0.969524in}}%
\pgfpathlineto{\pgfqpoint{0.778602in}{0.988797in}}%
\pgfpathlineto{\pgfqpoint{0.780517in}{0.990747in}}%
\pgfpathlineto{\pgfqpoint{0.782432in}{1.055153in}}%
\pgfpathlineto{\pgfqpoint{0.784347in}{1.057310in}}%
\pgfpathlineto{\pgfqpoint{0.786263in}{1.083820in}}%
\pgfpathlineto{\pgfqpoint{0.788178in}{1.119611in}}%
\pgfpathlineto{\pgfqpoint{0.790093in}{1.224677in}}%
\pgfpathlineto{\pgfqpoint{0.792008in}{1.283776in}}%
\pgfpathlineto{\pgfqpoint{0.793923in}{1.826535in}}%
\pgfusepath{stroke}%
\end{pgfscope}%
\begin{pgfscope}%
\pgfpathrectangle{\pgfqpoint{0.694334in}{0.523557in}}{\pgfqpoint{3.830343in}{1.302977in}}%
\pgfusepath{clip}%
\pgfsetrectcap%
\pgfsetroundjoin%
\pgfsetlinewidth{1.003750pt}%
\definecolor{currentstroke}{rgb}{0.752941,0.752941,1.000000}%
\pgfsetstrokecolor{currentstroke}%
\pgfsetdash{}{0pt}%
\pgfpathmoveto{\pgfqpoint{0.694334in}{0.753605in}}%
\pgfpathlineto{\pgfqpoint{0.724977in}{0.753605in}}%
\pgfpathlineto{\pgfqpoint{0.726892in}{0.768580in}}%
\pgfpathlineto{\pgfqpoint{0.772856in}{0.768580in}}%
\pgfpathlineto{\pgfqpoint{0.774772in}{0.781789in}}%
\pgfpathlineto{\pgfqpoint{0.809245in}{0.781789in}}%
\pgfpathlineto{\pgfqpoint{0.811160in}{0.793606in}}%
\pgfpathlineto{\pgfqpoint{0.830311in}{0.793606in}}%
\pgfpathlineto{\pgfqpoint{0.832227in}{0.804294in}}%
\pgfpathlineto{\pgfqpoint{0.870530in}{0.804294in}}%
\pgfpathlineto{\pgfqpoint{0.872445in}{0.814053in}}%
\pgfpathlineto{\pgfqpoint{0.901173in}{0.814053in}}%
\pgfpathlineto{\pgfqpoint{0.903088in}{0.823029in}}%
\pgfpathlineto{\pgfqpoint{0.920325in}{0.823029in}}%
\pgfpathlineto{\pgfqpoint{0.922240in}{0.831341in}}%
\pgfpathlineto{\pgfqpoint{0.945222in}{0.831341in}}%
\pgfpathlineto{\pgfqpoint{0.947137in}{0.839078in}}%
\pgfpathlineto{\pgfqpoint{0.972034in}{0.839078in}}%
\pgfpathlineto{\pgfqpoint{0.973949in}{0.846316in}}%
\pgfpathlineto{\pgfqpoint{0.993101in}{0.846316in}}%
\pgfpathlineto{\pgfqpoint{0.995016in}{0.853115in}}%
\pgfpathlineto{\pgfqpoint{1.033320in}{0.853115in}}%
\pgfpathlineto{\pgfqpoint{1.035235in}{0.859525in}}%
\pgfpathlineto{\pgfqpoint{1.056302in}{0.859525in}}%
\pgfpathlineto{\pgfqpoint{1.058217in}{0.865589in}}%
\pgfpathlineto{\pgfqpoint{1.085029in}{0.865589in}}%
\pgfpathlineto{\pgfqpoint{1.086944in}{0.871341in}}%
\pgfpathlineto{\pgfqpoint{1.111842in}{0.871341in}}%
\pgfpathlineto{\pgfqpoint{1.113757in}{0.876813in}}%
\pgfpathlineto{\pgfqpoint{1.132909in}{0.876813in}}%
\pgfpathlineto{\pgfqpoint{1.134824in}{0.882030in}}%
\pgfpathlineto{\pgfqpoint{1.153975in}{0.882030in}}%
\pgfpathlineto{\pgfqpoint{1.155891in}{0.887015in}}%
\pgfpathlineto{\pgfqpoint{1.173127in}{0.887015in}}%
\pgfpathlineto{\pgfqpoint{1.175042in}{0.891788in}}%
\pgfpathlineto{\pgfqpoint{1.199940in}{0.891788in}}%
\pgfpathlineto{\pgfqpoint{1.201855in}{0.896366in}}%
\pgfpathlineto{\pgfqpoint{1.207600in}{0.896366in}}%
\pgfpathlineto{\pgfqpoint{1.209515in}{0.900765in}}%
\pgfpathlineto{\pgfqpoint{1.221006in}{0.900765in}}%
\pgfpathlineto{\pgfqpoint{1.222922in}{0.904998in}}%
\pgfpathlineto{\pgfqpoint{1.234413in}{0.904998in}}%
\pgfpathlineto{\pgfqpoint{1.236328in}{0.909076in}}%
\pgfpathlineto{\pgfqpoint{1.257395in}{0.909076in}}%
\pgfpathlineto{\pgfqpoint{1.259310in}{0.913012in}}%
\pgfpathlineto{\pgfqpoint{1.274631in}{0.913012in}}%
\pgfpathlineto{\pgfqpoint{1.276546in}{0.916814in}}%
\pgfpathlineto{\pgfqpoint{1.284207in}{0.916814in}}%
\pgfpathlineto{\pgfqpoint{1.286122in}{0.920491in}}%
\pgfpathlineto{\pgfqpoint{1.299528in}{0.920491in}}%
\pgfpathlineto{\pgfqpoint{1.301444in}{0.924052in}}%
\pgfpathlineto{\pgfqpoint{1.322511in}{0.924052in}}%
\pgfpathlineto{\pgfqpoint{1.324426in}{0.927503in}}%
\pgfpathlineto{\pgfqpoint{1.328256in}{0.927503in}}%
\pgfpathlineto{\pgfqpoint{1.330171in}{0.930851in}}%
\pgfpathlineto{\pgfqpoint{1.347408in}{0.930851in}}%
\pgfpathlineto{\pgfqpoint{1.349323in}{0.934101in}}%
\pgfpathlineto{\pgfqpoint{1.356984in}{0.934101in}}%
\pgfpathlineto{\pgfqpoint{1.358899in}{0.937261in}}%
\pgfpathlineto{\pgfqpoint{1.360814in}{0.937261in}}%
\pgfpathlineto{\pgfqpoint{1.362729in}{0.940334in}}%
\pgfpathlineto{\pgfqpoint{1.364644in}{0.940334in}}%
\pgfpathlineto{\pgfqpoint{1.366559in}{0.943324in}}%
\pgfpathlineto{\pgfqpoint{1.370390in}{0.943324in}}%
\pgfpathlineto{\pgfqpoint{1.372305in}{0.946237in}}%
\pgfpathlineto{\pgfqpoint{1.374220in}{0.951846in}}%
\pgfpathlineto{\pgfqpoint{1.376135in}{0.951846in}}%
\pgfpathlineto{\pgfqpoint{1.378050in}{0.954549in}}%
\pgfpathlineto{\pgfqpoint{1.381881in}{0.954549in}}%
\pgfpathlineto{\pgfqpoint{1.383796in}{0.957188in}}%
\pgfpathlineto{\pgfqpoint{1.385711in}{0.962286in}}%
\pgfpathlineto{\pgfqpoint{1.387626in}{0.962286in}}%
\pgfpathlineto{\pgfqpoint{1.389542in}{0.964751in}}%
\pgfpathlineto{\pgfqpoint{1.397202in}{1.021060in}}%
\pgfpathlineto{\pgfqpoint{1.399117in}{1.087143in}}%
\pgfpathlineto{\pgfqpoint{1.401033in}{1.091164in}}%
\pgfpathlineto{\pgfqpoint{1.402948in}{1.099531in}}%
\pgfpathlineto{\pgfqpoint{1.404863in}{1.100262in}}%
\pgfpathlineto{\pgfqpoint{1.406778in}{1.107317in}}%
\pgfpathlineto{\pgfqpoint{1.408693in}{1.109350in}}%
\pgfpathlineto{\pgfqpoint{1.410608in}{1.119611in}}%
\pgfpathlineto{\pgfqpoint{1.416354in}{1.198566in}}%
\pgfpathlineto{\pgfqpoint{1.418269in}{1.826535in}}%
\pgfpathlineto{\pgfqpoint{1.418269in}{1.826535in}}%
\pgfusepath{stroke}%
\end{pgfscope}%
\begin{pgfscope}%
\pgfpathrectangle{\pgfqpoint{0.694334in}{0.523557in}}{\pgfqpoint{3.830343in}{1.302977in}}%
\pgfusepath{clip}%
\pgfsetrectcap%
\pgfsetroundjoin%
\pgfsetlinewidth{1.003750pt}%
\definecolor{currentstroke}{rgb}{0.752941,0.752941,1.000000}%
\pgfsetstrokecolor{currentstroke}%
\pgfsetdash{}{0pt}%
\pgfpathmoveto{\pgfqpoint{0.694334in}{0.753605in}}%
\pgfpathlineto{\pgfqpoint{0.696249in}{0.753605in}}%
\pgfpathlineto{\pgfqpoint{0.698165in}{0.753605in}}%
\pgfpathlineto{\pgfqpoint{0.700080in}{0.753605in}}%
\pgfpathlineto{\pgfqpoint{0.701995in}{0.753605in}}%
\pgfpathlineto{\pgfqpoint{0.703910in}{0.768580in}}%
\pgfpathlineto{\pgfqpoint{0.705825in}{0.768580in}}%
\pgfpathlineto{\pgfqpoint{0.707741in}{0.768580in}}%
\pgfpathlineto{\pgfqpoint{0.709656in}{0.768580in}}%
\pgfpathlineto{\pgfqpoint{0.711571in}{0.768580in}}%
\pgfpathlineto{\pgfqpoint{0.713486in}{0.768580in}}%
\pgfpathlineto{\pgfqpoint{0.715401in}{0.768580in}}%
\pgfpathlineto{\pgfqpoint{0.717316in}{0.768580in}}%
\pgfpathlineto{\pgfqpoint{0.719232in}{0.768580in}}%
\pgfpathlineto{\pgfqpoint{0.721147in}{0.781789in}}%
\pgfpathlineto{\pgfqpoint{0.723062in}{0.781789in}}%
\pgfpathlineto{\pgfqpoint{0.724977in}{0.793606in}}%
\pgfpathlineto{\pgfqpoint{0.726892in}{0.793606in}}%
\pgfpathlineto{\pgfqpoint{0.728807in}{0.793606in}}%
\pgfpathlineto{\pgfqpoint{0.730723in}{0.793606in}}%
\pgfpathlineto{\pgfqpoint{0.732638in}{0.793606in}}%
\pgfpathlineto{\pgfqpoint{0.734553in}{0.804294in}}%
\pgfpathlineto{\pgfqpoint{0.736468in}{0.804294in}}%
\pgfpathlineto{\pgfqpoint{0.738383in}{0.804294in}}%
\pgfpathlineto{\pgfqpoint{0.740298in}{0.804294in}}%
\pgfpathlineto{\pgfqpoint{0.742214in}{0.814053in}}%
\pgfpathlineto{\pgfqpoint{0.744129in}{0.814053in}}%
\pgfpathlineto{\pgfqpoint{0.746044in}{0.814053in}}%
\pgfpathlineto{\pgfqpoint{0.747959in}{0.823029in}}%
\pgfpathlineto{\pgfqpoint{0.749874in}{0.823029in}}%
\pgfpathlineto{\pgfqpoint{0.751789in}{0.831341in}}%
\pgfpathlineto{\pgfqpoint{0.753705in}{0.831341in}}%
\pgfpathlineto{\pgfqpoint{0.755620in}{0.865589in}}%
\pgfpathlineto{\pgfqpoint{0.757535in}{0.871341in}}%
\pgfpathlineto{\pgfqpoint{0.759450in}{0.882030in}}%
\pgfpathlineto{\pgfqpoint{0.761365in}{0.896366in}}%
\pgfpathlineto{\pgfqpoint{0.763280in}{0.913012in}}%
\pgfpathlineto{\pgfqpoint{0.765196in}{0.949077in}}%
\pgfpathlineto{\pgfqpoint{0.767111in}{1.022526in}}%
\pgfpathlineto{\pgfqpoint{0.769026in}{1.043699in}}%
\pgfpathlineto{\pgfqpoint{0.770941in}{1.054059in}}%
\pgfpathlineto{\pgfqpoint{0.772856in}{1.061503in}}%
\pgfpathlineto{\pgfqpoint{0.774772in}{1.062527in}}%
\pgfpathlineto{\pgfqpoint{0.776687in}{1.063542in}}%
\pgfpathlineto{\pgfqpoint{0.778602in}{1.075054in}}%
\pgfpathlineto{\pgfqpoint{0.780517in}{1.126734in}}%
\pgfpathlineto{\pgfqpoint{0.782432in}{1.208702in}}%
\pgfpathlineto{\pgfqpoint{0.784347in}{1.325158in}}%
\pgfpathlineto{\pgfqpoint{0.786263in}{1.826535in}}%
\pgfusepath{stroke}%
\end{pgfscope}%
\begin{pgfscope}%
\pgfpathrectangle{\pgfqpoint{0.694334in}{0.523557in}}{\pgfqpoint{3.830343in}{1.302977in}}%
\pgfusepath{clip}%
\pgfsetrectcap%
\pgfsetroundjoin%
\pgfsetlinewidth{1.003750pt}%
\definecolor{currentstroke}{rgb}{0.752941,0.752941,1.000000}%
\pgfsetstrokecolor{currentstroke}%
\pgfsetdash{}{0pt}%
\pgfpathmoveto{\pgfqpoint{0.694334in}{0.736317in}}%
\pgfpathlineto{\pgfqpoint{0.696249in}{0.753605in}}%
\pgfpathlineto{\pgfqpoint{0.698165in}{0.753605in}}%
\pgfpathlineto{\pgfqpoint{0.700080in}{0.753605in}}%
\pgfpathlineto{\pgfqpoint{0.701995in}{0.753605in}}%
\pgfpathlineto{\pgfqpoint{0.703910in}{0.753605in}}%
\pgfpathlineto{\pgfqpoint{0.705825in}{0.753605in}}%
\pgfpathlineto{\pgfqpoint{0.707741in}{0.753605in}}%
\pgfpathlineto{\pgfqpoint{0.709656in}{0.753605in}}%
\pgfpathlineto{\pgfqpoint{0.711571in}{0.753605in}}%
\pgfpathlineto{\pgfqpoint{0.713486in}{0.753605in}}%
\pgfpathlineto{\pgfqpoint{0.715401in}{0.753605in}}%
\pgfpathlineto{\pgfqpoint{0.717316in}{0.768580in}}%
\pgfpathlineto{\pgfqpoint{0.719232in}{0.768580in}}%
\pgfpathlineto{\pgfqpoint{0.721147in}{0.768580in}}%
\pgfpathlineto{\pgfqpoint{0.723062in}{0.768580in}}%
\pgfpathlineto{\pgfqpoint{0.724977in}{0.768580in}}%
\pgfpathlineto{\pgfqpoint{0.726892in}{0.793606in}}%
\pgfpathlineto{\pgfqpoint{0.728807in}{0.793606in}}%
\pgfpathlineto{\pgfqpoint{0.730723in}{0.793606in}}%
\pgfpathlineto{\pgfqpoint{0.732638in}{0.793606in}}%
\pgfpathlineto{\pgfqpoint{0.734553in}{0.814053in}}%
\pgfpathlineto{\pgfqpoint{0.736468in}{0.823029in}}%
\pgfpathlineto{\pgfqpoint{0.738383in}{0.823029in}}%
\pgfpathlineto{\pgfqpoint{0.740298in}{0.831341in}}%
\pgfpathlineto{\pgfqpoint{0.742214in}{0.839078in}}%
\pgfpathlineto{\pgfqpoint{0.744129in}{0.962286in}}%
\pgfpathlineto{\pgfqpoint{0.746044in}{1.051838in}}%
\pgfpathlineto{\pgfqpoint{0.747959in}{1.826535in}}%
\pgfusepath{stroke}%
\end{pgfscope}%
\begin{pgfscope}%
\pgfpathrectangle{\pgfqpoint{0.694334in}{0.523557in}}{\pgfqpoint{3.830343in}{1.302977in}}%
\pgfusepath{clip}%
\pgfsetrectcap%
\pgfsetroundjoin%
\pgfsetlinewidth{1.003750pt}%
\definecolor{currentstroke}{rgb}{0.125490,0.662745,0.705882}%
\pgfsetstrokecolor{currentstroke}%
\pgfsetdash{}{0pt}%
\pgfpathmoveto{\pgfqpoint{0.694334in}{0.753605in}}%
\pgfpathlineto{\pgfqpoint{0.728807in}{0.753605in}}%
\pgfpathlineto{\pgfqpoint{0.730723in}{0.768580in}}%
\pgfpathlineto{\pgfqpoint{0.767111in}{0.768580in}}%
\pgfpathlineto{\pgfqpoint{0.769026in}{0.781789in}}%
\pgfpathlineto{\pgfqpoint{0.816905in}{0.781789in}}%
\pgfpathlineto{\pgfqpoint{0.818820in}{0.793606in}}%
\pgfpathlineto{\pgfqpoint{0.834142in}{0.793606in}}%
\pgfpathlineto{\pgfqpoint{0.836057in}{0.804294in}}%
\pgfpathlineto{\pgfqpoint{0.880106in}{0.804294in}}%
\pgfpathlineto{\pgfqpoint{0.882021in}{0.814053in}}%
\pgfpathlineto{\pgfqpoint{0.926070in}{0.814053in}}%
\pgfpathlineto{\pgfqpoint{0.927985in}{0.823029in}}%
\pgfpathlineto{\pgfqpoint{0.954798in}{0.823029in}}%
\pgfpathlineto{\pgfqpoint{0.956713in}{0.831341in}}%
\pgfpathlineto{\pgfqpoint{0.987356in}{0.831341in}}%
\pgfpathlineto{\pgfqpoint{0.989271in}{0.839078in}}%
\pgfpathlineto{\pgfqpoint{1.021829in}{0.839078in}}%
\pgfpathlineto{\pgfqpoint{1.023744in}{0.846316in}}%
\pgfpathlineto{\pgfqpoint{1.052471in}{0.846316in}}%
\pgfpathlineto{\pgfqpoint{1.054387in}{0.853115in}}%
\pgfpathlineto{\pgfqpoint{1.077369in}{0.853115in}}%
\pgfpathlineto{\pgfqpoint{1.079284in}{0.859525in}}%
\pgfpathlineto{\pgfqpoint{1.104181in}{0.859525in}}%
\pgfpathlineto{\pgfqpoint{1.106096in}{0.865589in}}%
\pgfpathlineto{\pgfqpoint{1.138654in}{0.865589in}}%
\pgfpathlineto{\pgfqpoint{1.140569in}{0.871341in}}%
\pgfpathlineto{\pgfqpoint{1.171212in}{0.871341in}}%
\pgfpathlineto{\pgfqpoint{1.173127in}{0.876813in}}%
\pgfpathlineto{\pgfqpoint{1.180788in}{0.876813in}}%
\pgfpathlineto{\pgfqpoint{1.182703in}{0.882030in}}%
\pgfpathlineto{\pgfqpoint{1.207600in}{0.882030in}}%
\pgfpathlineto{\pgfqpoint{1.209515in}{0.887015in}}%
\pgfpathlineto{\pgfqpoint{1.224837in}{0.887015in}}%
\pgfpathlineto{\pgfqpoint{1.226752in}{0.891788in}}%
\pgfpathlineto{\pgfqpoint{1.253564in}{0.891788in}}%
\pgfpathlineto{\pgfqpoint{1.255480in}{0.896366in}}%
\pgfpathlineto{\pgfqpoint{1.280377in}{0.896366in}}%
\pgfpathlineto{\pgfqpoint{1.282292in}{0.900765in}}%
\pgfpathlineto{\pgfqpoint{1.303359in}{0.900765in}}%
\pgfpathlineto{\pgfqpoint{1.305274in}{0.904998in}}%
\pgfpathlineto{\pgfqpoint{1.312935in}{0.904998in}}%
\pgfpathlineto{\pgfqpoint{1.314850in}{0.909076in}}%
\pgfpathlineto{\pgfqpoint{1.322511in}{0.909076in}}%
\pgfpathlineto{\pgfqpoint{1.324426in}{0.913012in}}%
\pgfpathlineto{\pgfqpoint{1.334002in}{0.913012in}}%
\pgfpathlineto{\pgfqpoint{1.335917in}{0.916814in}}%
\pgfpathlineto{\pgfqpoint{1.347408in}{0.916814in}}%
\pgfpathlineto{\pgfqpoint{1.349323in}{0.920491in}}%
\pgfpathlineto{\pgfqpoint{1.362729in}{0.920491in}}%
\pgfpathlineto{\pgfqpoint{1.364644in}{0.924052in}}%
\pgfpathlineto{\pgfqpoint{1.387626in}{0.924052in}}%
\pgfpathlineto{\pgfqpoint{1.389542in}{0.927503in}}%
\pgfpathlineto{\pgfqpoint{1.401033in}{0.927503in}}%
\pgfpathlineto{\pgfqpoint{1.402948in}{0.930851in}}%
\pgfpathlineto{\pgfqpoint{1.422099in}{0.930851in}}%
\pgfpathlineto{\pgfqpoint{1.424015in}{0.934101in}}%
\pgfpathlineto{\pgfqpoint{1.435506in}{0.934101in}}%
\pgfpathlineto{\pgfqpoint{1.437421in}{0.937261in}}%
\pgfpathlineto{\pgfqpoint{1.443166in}{0.937261in}}%
\pgfpathlineto{\pgfqpoint{1.445081in}{0.940334in}}%
\pgfpathlineto{\pgfqpoint{1.456573in}{0.940334in}}%
\pgfpathlineto{\pgfqpoint{1.458488in}{0.943324in}}%
\pgfpathlineto{\pgfqpoint{1.469979in}{0.943324in}}%
\pgfpathlineto{\pgfqpoint{1.471894in}{0.946237in}}%
\pgfpathlineto{\pgfqpoint{1.492961in}{0.946237in}}%
\pgfpathlineto{\pgfqpoint{1.494876in}{0.949077in}}%
\pgfpathlineto{\pgfqpoint{1.498706in}{0.949077in}}%
\pgfpathlineto{\pgfqpoint{1.500621in}{0.951846in}}%
\pgfpathlineto{\pgfqpoint{1.508282in}{0.951846in}}%
\pgfpathlineto{\pgfqpoint{1.510197in}{0.954549in}}%
\pgfpathlineto{\pgfqpoint{1.512112in}{0.954549in}}%
\pgfpathlineto{\pgfqpoint{1.514028in}{0.957188in}}%
\pgfpathlineto{\pgfqpoint{1.523604in}{0.957188in}}%
\pgfpathlineto{\pgfqpoint{1.525519in}{0.959766in}}%
\pgfpathlineto{\pgfqpoint{1.537010in}{0.959766in}}%
\pgfpathlineto{\pgfqpoint{1.538925in}{0.962286in}}%
\pgfpathlineto{\pgfqpoint{1.552331in}{0.962286in}}%
\pgfpathlineto{\pgfqpoint{1.554246in}{0.964751in}}%
\pgfpathlineto{\pgfqpoint{1.561907in}{0.964751in}}%
\pgfpathlineto{\pgfqpoint{1.563822in}{0.967163in}}%
\pgfpathlineto{\pgfqpoint{1.567652in}{0.967163in}}%
\pgfpathlineto{\pgfqpoint{1.569568in}{0.969524in}}%
\pgfpathlineto{\pgfqpoint{1.575313in}{0.969524in}}%
\pgfpathlineto{\pgfqpoint{1.577228in}{0.971836in}}%
\pgfpathlineto{\pgfqpoint{1.581059in}{0.971836in}}%
\pgfpathlineto{\pgfqpoint{1.582974in}{0.974102in}}%
\pgfpathlineto{\pgfqpoint{1.590635in}{0.974102in}}%
\pgfpathlineto{\pgfqpoint{1.592550in}{0.976323in}}%
\pgfpathlineto{\pgfqpoint{1.598295in}{0.976323in}}%
\pgfpathlineto{\pgfqpoint{1.600210in}{0.978501in}}%
\pgfpathlineto{\pgfqpoint{1.613617in}{0.978501in}}%
\pgfpathlineto{\pgfqpoint{1.619362in}{0.984791in}}%
\pgfpathlineto{\pgfqpoint{1.625108in}{0.984791in}}%
\pgfpathlineto{\pgfqpoint{1.627023in}{0.986812in}}%
\pgfpathlineto{\pgfqpoint{1.632768in}{0.986812in}}%
\pgfpathlineto{\pgfqpoint{1.636599in}{0.994549in}}%
\pgfpathlineto{\pgfqpoint{1.644259in}{0.994549in}}%
\pgfpathlineto{\pgfqpoint{1.646174in}{0.996403in}}%
\pgfpathlineto{\pgfqpoint{1.648090in}{0.996403in}}%
\pgfpathlineto{\pgfqpoint{1.657666in}{1.005238in}}%
\pgfpathlineto{\pgfqpoint{1.659581in}{1.005238in}}%
\pgfpathlineto{\pgfqpoint{1.661496in}{1.006925in}}%
\pgfpathlineto{\pgfqpoint{1.663411in}{1.006925in}}%
\pgfpathlineto{\pgfqpoint{1.667241in}{1.011837in}}%
\pgfpathlineto{\pgfqpoint{1.669157in}{1.011837in}}%
\pgfpathlineto{\pgfqpoint{1.671072in}{1.016543in}}%
\pgfpathlineto{\pgfqpoint{1.672987in}{1.018069in}}%
\pgfpathlineto{\pgfqpoint{1.674902in}{1.025402in}}%
\pgfpathlineto{\pgfqpoint{1.676817in}{1.057310in}}%
\pgfpathlineto{\pgfqpoint{1.678732in}{1.065544in}}%
\pgfpathlineto{\pgfqpoint{1.680648in}{1.068483in}}%
\pgfpathlineto{\pgfqpoint{1.684478in}{1.069446in}}%
\pgfpathlineto{\pgfqpoint{1.686393in}{1.070400in}}%
\pgfpathlineto{\pgfqpoint{1.690223in}{1.095045in}}%
\pgfpathlineto{\pgfqpoint{1.694054in}{1.157696in}}%
\pgfpathlineto{\pgfqpoint{1.699799in}{1.222495in}}%
\pgfpathlineto{\pgfqpoint{1.701714in}{1.826535in}}%
\pgfpathlineto{\pgfqpoint{1.701714in}{1.826535in}}%
\pgfusepath{stroke}%
\end{pgfscope}%
\begin{pgfscope}%
\pgfpathrectangle{\pgfqpoint{0.694334in}{0.523557in}}{\pgfqpoint{3.830343in}{1.302977in}}%
\pgfusepath{clip}%
\pgfsetbuttcap%
\pgfsetroundjoin%
\pgfsetlinewidth{1.003750pt}%
\definecolor{currentstroke}{rgb}{1.000000,0.843137,0.000000}%
\pgfsetstrokecolor{currentstroke}%
\pgfsetdash{{3.700000pt}{1.600000pt}}{0.000000pt}%
\pgfpathmoveto{\pgfqpoint{0.694334in}{0.882028in}}%
\pgfpathlineto{\pgfqpoint{0.703910in}{0.884212in}}%
\pgfpathlineto{\pgfqpoint{0.713486in}{0.885735in}}%
\pgfpathlineto{\pgfqpoint{0.717316in}{0.887046in}}%
\pgfpathlineto{\pgfqpoint{0.818820in}{0.892097in}}%
\pgfpathlineto{\pgfqpoint{0.891597in}{0.894605in}}%
\pgfpathlineto{\pgfqpoint{0.937561in}{0.896013in}}%
\pgfpathlineto{\pgfqpoint{1.129078in}{0.903525in}}%
\pgfpathlineto{\pgfqpoint{1.146315in}{0.904460in}}%
\pgfpathlineto{\pgfqpoint{1.236328in}{0.908844in}}%
\pgfpathlineto{\pgfqpoint{1.249734in}{0.910171in}}%
\pgfpathlineto{\pgfqpoint{1.266971in}{0.911130in}}%
\pgfpathlineto{\pgfqpoint{1.278462in}{0.911911in}}%
\pgfpathlineto{\pgfqpoint{1.303359in}{0.913041in}}%
\pgfpathlineto{\pgfqpoint{1.334002in}{0.914483in}}%
\pgfpathlineto{\pgfqpoint{1.364644in}{0.916374in}}%
\pgfpathlineto{\pgfqpoint{1.389542in}{0.918116in}}%
\pgfpathlineto{\pgfqpoint{1.399117in}{0.919774in}}%
\pgfpathlineto{\pgfqpoint{1.441251in}{0.922543in}}%
\pgfpathlineto{\pgfqpoint{1.491046in}{0.927948in}}%
\pgfpathlineto{\pgfqpoint{1.496791in}{0.928841in}}%
\pgfpathlineto{\pgfqpoint{1.517858in}{0.930345in}}%
\pgfpathlineto{\pgfqpoint{1.558077in}{0.933711in}}%
\pgfpathlineto{\pgfqpoint{1.563822in}{0.934483in}}%
\pgfpathlineto{\pgfqpoint{1.596380in}{0.936587in}}%
\pgfpathlineto{\pgfqpoint{1.621277in}{0.938492in}}%
\pgfpathlineto{\pgfqpoint{1.636599in}{0.939654in}}%
\pgfpathlineto{\pgfqpoint{1.648090in}{0.940783in}}%
\pgfpathlineto{\pgfqpoint{1.728527in}{0.947627in}}%
\pgfpathlineto{\pgfqpoint{1.732357in}{0.949008in}}%
\pgfpathlineto{\pgfqpoint{1.747679in}{0.950929in}}%
\pgfpathlineto{\pgfqpoint{1.778321in}{0.955106in}}%
\pgfpathlineto{\pgfqpoint{1.789812in}{0.956718in}}%
\pgfpathlineto{\pgfqpoint{1.801303in}{0.957989in}}%
\pgfpathlineto{\pgfqpoint{1.805134in}{0.958897in}}%
\pgfpathlineto{\pgfqpoint{1.820455in}{0.961406in}}%
\pgfpathlineto{\pgfqpoint{1.826201in}{0.963438in}}%
\pgfpathlineto{\pgfqpoint{1.835776in}{0.964287in}}%
\pgfpathlineto{\pgfqpoint{1.839607in}{0.966225in}}%
\pgfpathlineto{\pgfqpoint{1.849183in}{0.967849in}}%
\pgfpathlineto{\pgfqpoint{1.851098in}{0.970103in}}%
\pgfpathlineto{\pgfqpoint{1.856843in}{0.971845in}}%
\pgfpathlineto{\pgfqpoint{1.862589in}{0.973543in}}%
\pgfpathlineto{\pgfqpoint{1.866419in}{0.974095in}}%
\pgfpathlineto{\pgfqpoint{1.870250in}{0.977298in}}%
\pgfpathlineto{\pgfqpoint{1.875995in}{0.978811in}}%
\pgfpathlineto{\pgfqpoint{1.883656in}{0.979792in}}%
\pgfpathlineto{\pgfqpoint{1.889401in}{0.983198in}}%
\pgfpathlineto{\pgfqpoint{1.898977in}{0.985122in}}%
\pgfpathlineto{\pgfqpoint{1.902807in}{0.986897in}}%
\pgfpathlineto{\pgfqpoint{1.908553in}{0.987987in}}%
\pgfpathlineto{\pgfqpoint{1.912383in}{0.989557in}}%
\pgfpathlineto{\pgfqpoint{1.914298in}{0.993236in}}%
\pgfpathlineto{\pgfqpoint{1.925790in}{0.995706in}}%
\pgfpathlineto{\pgfqpoint{1.931535in}{1.001037in}}%
\pgfpathlineto{\pgfqpoint{1.954517in}{1.010821in}}%
\pgfpathlineto{\pgfqpoint{1.962178in}{1.018685in}}%
\pgfpathlineto{\pgfqpoint{1.966008in}{1.019784in}}%
\pgfpathlineto{\pgfqpoint{1.977499in}{1.027197in}}%
\pgfpathlineto{\pgfqpoint{1.979414in}{1.027415in}}%
\pgfpathlineto{\pgfqpoint{1.983245in}{1.038273in}}%
\pgfpathlineto{\pgfqpoint{1.985160in}{1.047583in}}%
\pgfpathlineto{\pgfqpoint{1.987075in}{1.048916in}}%
\pgfpathlineto{\pgfqpoint{1.988990in}{1.051786in}}%
\pgfpathlineto{\pgfqpoint{1.990905in}{1.052177in}}%
\pgfpathlineto{\pgfqpoint{1.992821in}{1.053955in}}%
\pgfpathlineto{\pgfqpoint{1.994736in}{1.062425in}}%
\pgfpathlineto{\pgfqpoint{1.996651in}{1.076216in}}%
\pgfpathlineto{\pgfqpoint{2.000481in}{1.079092in}}%
\pgfpathlineto{\pgfqpoint{2.010057in}{1.082878in}}%
\pgfpathlineto{\pgfqpoint{2.044530in}{1.090249in}}%
\pgfpathlineto{\pgfqpoint{2.069427in}{1.094336in}}%
\pgfpathlineto{\pgfqpoint{2.073258in}{1.098161in}}%
\pgfpathlineto{\pgfqpoint{2.077088in}{1.100425in}}%
\pgfpathlineto{\pgfqpoint{2.082834in}{1.104769in}}%
\pgfpathlineto{\pgfqpoint{2.084749in}{1.104850in}}%
\pgfpathlineto{\pgfqpoint{2.090494in}{1.108439in}}%
\pgfpathlineto{\pgfqpoint{2.096240in}{1.110306in}}%
\pgfpathlineto{\pgfqpoint{2.100070in}{1.111998in}}%
\pgfpathlineto{\pgfqpoint{2.109646in}{1.114417in}}%
\pgfpathlineto{\pgfqpoint{2.111561in}{1.115560in}}%
\pgfpathlineto{\pgfqpoint{2.113476in}{1.121529in}}%
\pgfpathlineto{\pgfqpoint{2.115391in}{1.121602in}}%
\pgfpathlineto{\pgfqpoint{2.121137in}{1.126257in}}%
\pgfpathlineto{\pgfqpoint{2.128798in}{1.144421in}}%
\pgfpathlineto{\pgfqpoint{2.134543in}{1.146307in}}%
\pgfpathlineto{\pgfqpoint{2.136458in}{1.152987in}}%
\pgfpathlineto{\pgfqpoint{2.142204in}{1.159914in}}%
\pgfpathlineto{\pgfqpoint{2.146034in}{1.160839in}}%
\pgfpathlineto{\pgfqpoint{2.147949in}{1.163439in}}%
\pgfpathlineto{\pgfqpoint{2.149865in}{1.163845in}}%
\pgfpathlineto{\pgfqpoint{2.165186in}{1.210971in}}%
\pgfpathlineto{\pgfqpoint{2.167101in}{1.227547in}}%
\pgfpathlineto{\pgfqpoint{2.172847in}{1.231472in}}%
\pgfpathlineto{\pgfqpoint{2.174762in}{1.245158in}}%
\pgfpathlineto{\pgfqpoint{2.180507in}{1.259827in}}%
\pgfpathlineto{\pgfqpoint{2.182422in}{1.283880in}}%
\pgfpathlineto{\pgfqpoint{2.184338in}{1.287156in}}%
\pgfpathlineto{\pgfqpoint{2.186253in}{1.312284in}}%
\pgfpathlineto{\pgfqpoint{2.188168in}{1.313245in}}%
\pgfpathlineto{\pgfqpoint{2.191998in}{1.326712in}}%
\pgfpathlineto{\pgfqpoint{2.193914in}{1.406715in}}%
\pgfpathlineto{\pgfqpoint{2.197744in}{1.410779in}}%
\pgfpathlineto{\pgfqpoint{2.199659in}{1.425334in}}%
\pgfpathlineto{\pgfqpoint{2.203489in}{1.477665in}}%
\pgfpathlineto{\pgfqpoint{2.209235in}{1.520093in}}%
\pgfpathlineto{\pgfqpoint{2.214980in}{1.521665in}}%
\pgfpathlineto{\pgfqpoint{2.220726in}{1.606227in}}%
\pgfpathlineto{\pgfqpoint{2.224556in}{1.607179in}}%
\pgfpathlineto{\pgfqpoint{2.228387in}{1.826535in}}%
\pgfpathlineto{\pgfqpoint{2.228387in}{1.826535in}}%
\pgfusepath{stroke}%
\end{pgfscope}%
\begin{pgfscope}%
\pgfpathrectangle{\pgfqpoint{0.694334in}{0.523557in}}{\pgfqpoint{3.830343in}{1.302977in}}%
\pgfusepath{clip}%
\pgfsetbuttcap%
\pgfsetroundjoin%
\pgfsetlinewidth{1.003750pt}%
\definecolor{currentstroke}{rgb}{1.000000,0.694118,0.305882}%
\pgfsetstrokecolor{currentstroke}%
\pgfsetdash{{1.000000pt}{1.650000pt}}{0.000000pt}%
\pgfpathmoveto{\pgfqpoint{0.694334in}{0.552252in}}%
\pgfpathlineto{\pgfqpoint{0.698165in}{0.562855in}}%
\pgfpathlineto{\pgfqpoint{0.700080in}{0.571642in}}%
\pgfpathlineto{\pgfqpoint{0.703910in}{0.576494in}}%
\pgfpathlineto{\pgfqpoint{0.707741in}{0.576733in}}%
\pgfpathlineto{\pgfqpoint{0.721147in}{0.585015in}}%
\pgfpathlineto{\pgfqpoint{0.723062in}{0.594821in}}%
\pgfpathlineto{\pgfqpoint{0.734553in}{0.603763in}}%
\pgfpathlineto{\pgfqpoint{0.738383in}{0.608804in}}%
\pgfpathlineto{\pgfqpoint{0.740298in}{0.610772in}}%
\pgfpathlineto{\pgfqpoint{0.742214in}{0.610929in}}%
\pgfpathlineto{\pgfqpoint{0.747959in}{0.614894in}}%
\pgfpathlineto{\pgfqpoint{0.751789in}{0.615883in}}%
\pgfpathlineto{\pgfqpoint{0.753705in}{0.619214in}}%
\pgfpathlineto{\pgfqpoint{0.755620in}{0.619334in}}%
\pgfpathlineto{\pgfqpoint{0.759450in}{0.630416in}}%
\pgfpathlineto{\pgfqpoint{0.763280in}{0.631656in}}%
\pgfpathlineto{\pgfqpoint{0.769026in}{0.640025in}}%
\pgfpathlineto{\pgfqpoint{0.770941in}{0.641192in}}%
\pgfpathlineto{\pgfqpoint{0.772856in}{0.646082in}}%
\pgfpathlineto{\pgfqpoint{0.774772in}{0.646285in}}%
\pgfpathlineto{\pgfqpoint{0.776687in}{0.649454in}}%
\pgfpathlineto{\pgfqpoint{0.784347in}{0.652255in}}%
\pgfpathlineto{\pgfqpoint{0.786263in}{0.652902in}}%
\pgfpathlineto{\pgfqpoint{0.788178in}{0.659116in}}%
\pgfpathlineto{\pgfqpoint{0.793923in}{0.663099in}}%
\pgfpathlineto{\pgfqpoint{0.797754in}{0.666749in}}%
\pgfpathlineto{\pgfqpoint{0.814990in}{0.671268in}}%
\pgfpathlineto{\pgfqpoint{0.822651in}{0.672369in}}%
\pgfpathlineto{\pgfqpoint{0.837972in}{0.673935in}}%
\pgfpathlineto{\pgfqpoint{0.859039in}{0.674812in}}%
\pgfpathlineto{\pgfqpoint{0.864785in}{0.675803in}}%
\pgfpathlineto{\pgfqpoint{0.922240in}{0.677797in}}%
\pgfpathlineto{\pgfqpoint{0.956713in}{0.679532in}}%
\pgfpathlineto{\pgfqpoint{0.981610in}{0.680811in}}%
\pgfpathlineto{\pgfqpoint{1.014168in}{0.681984in}}%
\pgfpathlineto{\pgfqpoint{1.046726in}{0.683221in}}%
\pgfpathlineto{\pgfqpoint{1.067793in}{0.684391in}}%
\pgfpathlineto{\pgfqpoint{1.081199in}{0.685498in}}%
\pgfpathlineto{\pgfqpoint{1.161636in}{0.696139in}}%
\pgfpathlineto{\pgfqpoint{1.286122in}{0.702811in}}%
\pgfpathlineto{\pgfqpoint{1.297613in}{0.705081in}}%
\pgfpathlineto{\pgfqpoint{1.314850in}{0.706613in}}%
\pgfpathlineto{\pgfqpoint{1.328256in}{0.711818in}}%
\pgfpathlineto{\pgfqpoint{1.332086in}{0.712916in}}%
\pgfpathlineto{\pgfqpoint{1.337832in}{0.717717in}}%
\pgfpathlineto{\pgfqpoint{1.341662in}{0.718455in}}%
\pgfpathlineto{\pgfqpoint{1.343577in}{0.722831in}}%
\pgfpathlineto{\pgfqpoint{1.349323in}{0.724656in}}%
\pgfpathlineto{\pgfqpoint{1.356984in}{0.729778in}}%
\pgfpathlineto{\pgfqpoint{1.358899in}{0.729946in}}%
\pgfpathlineto{\pgfqpoint{1.360814in}{0.733112in}}%
\pgfpathlineto{\pgfqpoint{1.362729in}{0.733400in}}%
\pgfpathlineto{\pgfqpoint{1.364644in}{0.736119in}}%
\pgfpathlineto{\pgfqpoint{1.376135in}{0.739866in}}%
\pgfpathlineto{\pgfqpoint{1.383796in}{0.740642in}}%
\pgfpathlineto{\pgfqpoint{1.385711in}{0.743134in}}%
\pgfpathlineto{\pgfqpoint{1.389542in}{0.743816in}}%
\pgfpathlineto{\pgfqpoint{1.391457in}{0.745532in}}%
\pgfpathlineto{\pgfqpoint{1.395287in}{0.746418in}}%
\pgfpathlineto{\pgfqpoint{1.401033in}{0.752134in}}%
\pgfpathlineto{\pgfqpoint{1.404863in}{0.753147in}}%
\pgfpathlineto{\pgfqpoint{1.408693in}{0.756746in}}%
\pgfpathlineto{\pgfqpoint{1.412524in}{0.758303in}}%
\pgfpathlineto{\pgfqpoint{1.414439in}{0.760941in}}%
\pgfpathlineto{\pgfqpoint{1.422099in}{0.762711in}}%
\pgfpathlineto{\pgfqpoint{1.437421in}{0.768111in}}%
\pgfpathlineto{\pgfqpoint{1.439336in}{0.774566in}}%
\pgfpathlineto{\pgfqpoint{1.443166in}{0.775346in}}%
\pgfpathlineto{\pgfqpoint{1.445081in}{0.777703in}}%
\pgfpathlineto{\pgfqpoint{1.450827in}{0.778814in}}%
\pgfpathlineto{\pgfqpoint{1.456573in}{0.780437in}}%
\pgfpathlineto{\pgfqpoint{1.466148in}{0.782045in}}%
\pgfpathlineto{\pgfqpoint{1.468064in}{0.783586in}}%
\pgfpathlineto{\pgfqpoint{1.475724in}{0.784884in}}%
\pgfpathlineto{\pgfqpoint{1.485300in}{0.786112in}}%
\pgfpathlineto{\pgfqpoint{1.489130in}{0.787649in}}%
\pgfpathlineto{\pgfqpoint{1.502537in}{0.792016in}}%
\pgfpathlineto{\pgfqpoint{1.506367in}{0.793213in}}%
\pgfpathlineto{\pgfqpoint{1.508282in}{0.793741in}}%
\pgfpathlineto{\pgfqpoint{1.510197in}{0.796560in}}%
\pgfpathlineto{\pgfqpoint{1.531264in}{0.802024in}}%
\pgfpathlineto{\pgfqpoint{1.538925in}{0.803039in}}%
\pgfpathlineto{\pgfqpoint{1.552331in}{0.805905in}}%
\pgfpathlineto{\pgfqpoint{1.556161in}{0.810747in}}%
\pgfpathlineto{\pgfqpoint{1.561907in}{0.812245in}}%
\pgfpathlineto{\pgfqpoint{1.573398in}{0.816787in}}%
\pgfpathlineto{\pgfqpoint{1.577228in}{0.816807in}}%
\pgfpathlineto{\pgfqpoint{1.581059in}{0.818287in}}%
\pgfpathlineto{\pgfqpoint{1.594465in}{0.822272in}}%
\pgfpathlineto{\pgfqpoint{1.598295in}{0.824669in}}%
\pgfpathlineto{\pgfqpoint{1.600210in}{0.826752in}}%
\pgfpathlineto{\pgfqpoint{1.607871in}{0.827995in}}%
\pgfpathlineto{\pgfqpoint{1.613617in}{0.828779in}}%
\pgfpathlineto{\pgfqpoint{1.617447in}{0.831236in}}%
\pgfpathlineto{\pgfqpoint{1.625108in}{0.832523in}}%
\pgfpathlineto{\pgfqpoint{1.628938in}{0.834777in}}%
\pgfpathlineto{\pgfqpoint{1.632768in}{0.835505in}}%
\pgfpathlineto{\pgfqpoint{1.636599in}{0.837340in}}%
\pgfpathlineto{\pgfqpoint{1.661496in}{0.842613in}}%
\pgfpathlineto{\pgfqpoint{1.667241in}{0.842807in}}%
\pgfpathlineto{\pgfqpoint{1.671072in}{0.844404in}}%
\pgfpathlineto{\pgfqpoint{1.674902in}{0.845259in}}%
\pgfpathlineto{\pgfqpoint{1.692139in}{0.847124in}}%
\pgfpathlineto{\pgfqpoint{1.697884in}{0.848213in}}%
\pgfpathlineto{\pgfqpoint{1.709375in}{0.849867in}}%
\pgfpathlineto{\pgfqpoint{1.713205in}{0.850536in}}%
\pgfpathlineto{\pgfqpoint{1.718951in}{0.853526in}}%
\pgfpathlineto{\pgfqpoint{1.732357in}{0.855114in}}%
\pgfpathlineto{\pgfqpoint{1.743848in}{0.860327in}}%
\pgfpathlineto{\pgfqpoint{1.751509in}{0.861398in}}%
\pgfpathlineto{\pgfqpoint{1.757254in}{0.862643in}}%
\pgfpathlineto{\pgfqpoint{1.759170in}{0.862925in}}%
\pgfpathlineto{\pgfqpoint{1.761085in}{0.865868in}}%
\pgfpathlineto{\pgfqpoint{1.774491in}{0.868403in}}%
\pgfpathlineto{\pgfqpoint{1.789812in}{0.871377in}}%
\pgfpathlineto{\pgfqpoint{1.795558in}{0.874037in}}%
\pgfpathlineto{\pgfqpoint{1.856843in}{0.883408in}}%
\pgfpathlineto{\pgfqpoint{1.864504in}{0.885938in}}%
\pgfpathlineto{\pgfqpoint{1.879825in}{0.889184in}}%
\pgfpathlineto{\pgfqpoint{1.895147in}{0.890384in}}%
\pgfpathlineto{\pgfqpoint{1.918129in}{0.895598in}}%
\pgfpathlineto{\pgfqpoint{1.923874in}{0.901014in}}%
\pgfpathlineto{\pgfqpoint{1.925790in}{0.904030in}}%
\pgfpathlineto{\pgfqpoint{1.931535in}{0.904831in}}%
\pgfpathlineto{\pgfqpoint{1.935365in}{0.905195in}}%
\pgfpathlineto{\pgfqpoint{1.937281in}{0.907698in}}%
\pgfpathlineto{\pgfqpoint{1.941111in}{0.908317in}}%
\pgfpathlineto{\pgfqpoint{1.950687in}{0.911931in}}%
\pgfpathlineto{\pgfqpoint{1.954517in}{0.912443in}}%
\pgfpathlineto{\pgfqpoint{1.958347in}{0.913855in}}%
\pgfpathlineto{\pgfqpoint{1.960263in}{0.913921in}}%
\pgfpathlineto{\pgfqpoint{1.964093in}{0.916764in}}%
\pgfpathlineto{\pgfqpoint{1.969838in}{0.918159in}}%
\pgfpathlineto{\pgfqpoint{1.975584in}{0.921266in}}%
\pgfpathlineto{\pgfqpoint{1.977499in}{0.924901in}}%
\pgfpathlineto{\pgfqpoint{1.979414in}{0.925692in}}%
\pgfpathlineto{\pgfqpoint{1.981329in}{0.931105in}}%
\pgfpathlineto{\pgfqpoint{1.983245in}{0.931557in}}%
\pgfpathlineto{\pgfqpoint{1.987075in}{0.942243in}}%
\pgfpathlineto{\pgfqpoint{1.990905in}{0.943554in}}%
\pgfpathlineto{\pgfqpoint{1.994736in}{0.944384in}}%
\pgfpathlineto{\pgfqpoint{1.996651in}{0.948260in}}%
\pgfpathlineto{\pgfqpoint{1.998566in}{0.948622in}}%
\pgfpathlineto{\pgfqpoint{2.004312in}{0.953228in}}%
\pgfpathlineto{\pgfqpoint{2.006227in}{0.957442in}}%
\pgfpathlineto{\pgfqpoint{2.008142in}{0.958369in}}%
\pgfpathlineto{\pgfqpoint{2.011972in}{0.967476in}}%
\pgfpathlineto{\pgfqpoint{2.019633in}{0.977553in}}%
\pgfpathlineto{\pgfqpoint{2.025378in}{0.977929in}}%
\pgfpathlineto{\pgfqpoint{2.029209in}{0.985736in}}%
\pgfpathlineto{\pgfqpoint{2.031124in}{0.987240in}}%
\pgfpathlineto{\pgfqpoint{2.033039in}{0.991900in}}%
\pgfpathlineto{\pgfqpoint{2.038785in}{0.992909in}}%
\pgfpathlineto{\pgfqpoint{2.042615in}{0.996392in}}%
\pgfpathlineto{\pgfqpoint{2.044530in}{1.002336in}}%
\pgfpathlineto{\pgfqpoint{2.048360in}{1.003895in}}%
\pgfpathlineto{\pgfqpoint{2.050276in}{1.006786in}}%
\pgfpathlineto{\pgfqpoint{2.052191in}{1.006877in}}%
\pgfpathlineto{\pgfqpoint{2.054106in}{1.012262in}}%
\pgfpathlineto{\pgfqpoint{2.056021in}{1.031235in}}%
\pgfpathlineto{\pgfqpoint{2.057936in}{1.036547in}}%
\pgfpathlineto{\pgfqpoint{2.059852in}{1.046677in}}%
\pgfpathlineto{\pgfqpoint{2.061767in}{1.047326in}}%
\pgfpathlineto{\pgfqpoint{2.063682in}{1.065524in}}%
\pgfpathlineto{\pgfqpoint{2.065597in}{1.066270in}}%
\pgfpathlineto{\pgfqpoint{2.067512in}{1.069560in}}%
\pgfpathlineto{\pgfqpoint{2.069427in}{1.069958in}}%
\pgfpathlineto{\pgfqpoint{2.079003in}{1.081430in}}%
\pgfpathlineto{\pgfqpoint{2.080918in}{1.090634in}}%
\pgfpathlineto{\pgfqpoint{2.090494in}{1.092429in}}%
\pgfpathlineto{\pgfqpoint{2.092409in}{1.094737in}}%
\pgfpathlineto{\pgfqpoint{2.094325in}{1.100361in}}%
\pgfpathlineto{\pgfqpoint{2.098155in}{1.101341in}}%
\pgfpathlineto{\pgfqpoint{2.100070in}{1.107397in}}%
\pgfpathlineto{\pgfqpoint{2.103900in}{1.110461in}}%
\pgfpathlineto{\pgfqpoint{2.113476in}{1.112977in}}%
\pgfpathlineto{\pgfqpoint{2.117307in}{1.122379in}}%
\pgfpathlineto{\pgfqpoint{2.119222in}{1.123239in}}%
\pgfpathlineto{\pgfqpoint{2.121137in}{1.127200in}}%
\pgfpathlineto{\pgfqpoint{2.126883in}{1.128962in}}%
\pgfpathlineto{\pgfqpoint{2.128798in}{1.136597in}}%
\pgfpathlineto{\pgfqpoint{2.132628in}{1.137673in}}%
\pgfpathlineto{\pgfqpoint{2.149865in}{1.163392in}}%
\pgfpathlineto{\pgfqpoint{2.151780in}{1.165269in}}%
\pgfpathlineto{\pgfqpoint{2.153695in}{1.177814in}}%
\pgfpathlineto{\pgfqpoint{2.157525in}{1.185281in}}%
\pgfpathlineto{\pgfqpoint{2.161356in}{1.202387in}}%
\pgfpathlineto{\pgfqpoint{2.165186in}{1.206804in}}%
\pgfpathlineto{\pgfqpoint{2.167101in}{1.207327in}}%
\pgfpathlineto{\pgfqpoint{2.170931in}{1.212004in}}%
\pgfpathlineto{\pgfqpoint{2.172847in}{1.213238in}}%
\pgfpathlineto{\pgfqpoint{2.174762in}{1.218024in}}%
\pgfpathlineto{\pgfqpoint{2.176677in}{1.226466in}}%
\pgfpathlineto{\pgfqpoint{2.178592in}{1.227689in}}%
\pgfpathlineto{\pgfqpoint{2.186253in}{1.242292in}}%
\pgfpathlineto{\pgfqpoint{2.188168in}{1.243114in}}%
\pgfpathlineto{\pgfqpoint{2.190083in}{1.246235in}}%
\pgfpathlineto{\pgfqpoint{2.193914in}{1.248122in}}%
\pgfpathlineto{\pgfqpoint{2.197744in}{1.251938in}}%
\pgfpathlineto{\pgfqpoint{2.199659in}{1.263409in}}%
\pgfpathlineto{\pgfqpoint{2.203489in}{1.264934in}}%
\pgfpathlineto{\pgfqpoint{2.207320in}{1.279467in}}%
\pgfpathlineto{\pgfqpoint{2.209235in}{1.282216in}}%
\pgfpathlineto{\pgfqpoint{2.220726in}{1.312953in}}%
\pgfpathlineto{\pgfqpoint{2.224556in}{1.325838in}}%
\pgfpathlineto{\pgfqpoint{2.226471in}{1.330466in}}%
\pgfpathlineto{\pgfqpoint{2.230302in}{1.333134in}}%
\pgfpathlineto{\pgfqpoint{2.232217in}{1.337532in}}%
\pgfpathlineto{\pgfqpoint{2.234132in}{1.346652in}}%
\pgfpathlineto{\pgfqpoint{2.237962in}{1.436806in}}%
\pgfpathlineto{\pgfqpoint{2.239878in}{1.474914in}}%
\pgfpathlineto{\pgfqpoint{2.241793in}{1.485545in}}%
\pgfpathlineto{\pgfqpoint{2.243708in}{1.489341in}}%
\pgfpathlineto{\pgfqpoint{2.245623in}{1.498307in}}%
\pgfpathlineto{\pgfqpoint{2.253284in}{1.570475in}}%
\pgfpathlineto{\pgfqpoint{2.255199in}{1.571964in}}%
\pgfpathlineto{\pgfqpoint{2.259029in}{1.614520in}}%
\pgfpathlineto{\pgfqpoint{2.262860in}{1.651116in}}%
\pgfpathlineto{\pgfqpoint{2.266690in}{1.826535in}}%
\pgfpathlineto{\pgfqpoint{2.266690in}{1.826535in}}%
\pgfusepath{stroke}%
\end{pgfscope}%
\begin{pgfscope}%
\pgfpathrectangle{\pgfqpoint{0.694334in}{0.523557in}}{\pgfqpoint{3.830343in}{1.302977in}}%
\pgfusepath{clip}%
\pgfsetrectcap%
\pgfsetroundjoin%
\pgfsetlinewidth{1.003750pt}%
\definecolor{currentstroke}{rgb}{0.917647,0.372549,0.580392}%
\pgfsetstrokecolor{currentstroke}%
\pgfsetdash{}{0pt}%
\pgfpathmoveto{\pgfqpoint{0.694334in}{1.195055in}}%
\pgfpathlineto{\pgfqpoint{0.696249in}{1.197705in}}%
\pgfpathlineto{\pgfqpoint{0.715401in}{1.200155in}}%
\pgfpathlineto{\pgfqpoint{0.724977in}{1.203581in}}%
\pgfpathlineto{\pgfqpoint{0.738383in}{1.206387in}}%
\pgfpathlineto{\pgfqpoint{0.740298in}{1.206450in}}%
\pgfpathlineto{\pgfqpoint{0.744129in}{1.208421in}}%
\pgfpathlineto{\pgfqpoint{0.746044in}{1.208589in}}%
\pgfpathlineto{\pgfqpoint{0.747959in}{1.211340in}}%
\pgfpathlineto{\pgfqpoint{0.749874in}{1.211345in}}%
\pgfpathlineto{\pgfqpoint{0.751789in}{1.214079in}}%
\pgfpathlineto{\pgfqpoint{0.765196in}{1.217706in}}%
\pgfpathlineto{\pgfqpoint{0.772856in}{1.219156in}}%
\pgfpathlineto{\pgfqpoint{0.774772in}{1.219431in}}%
\pgfpathlineto{\pgfqpoint{0.776687in}{1.221373in}}%
\pgfpathlineto{\pgfqpoint{0.837972in}{1.224335in}}%
\pgfpathlineto{\pgfqpoint{0.927985in}{1.226461in}}%
\pgfpathlineto{\pgfqpoint{1.016083in}{1.230414in}}%
\pgfpathlineto{\pgfqpoint{1.025659in}{1.231067in}}%
\pgfpathlineto{\pgfqpoint{1.033320in}{1.232451in}}%
\pgfpathlineto{\pgfqpoint{1.050556in}{1.233804in}}%
\pgfpathlineto{\pgfqpoint{1.090775in}{1.235918in}}%
\pgfpathlineto{\pgfqpoint{1.270801in}{1.252818in}}%
\pgfpathlineto{\pgfqpoint{1.276546in}{1.253533in}}%
\pgfpathlineto{\pgfqpoint{1.293783in}{1.255034in}}%
\pgfpathlineto{\pgfqpoint{1.462318in}{1.270940in}}%
\pgfpathlineto{\pgfqpoint{1.471894in}{1.272025in}}%
\pgfpathlineto{\pgfqpoint{1.483385in}{1.273414in}}%
\pgfpathlineto{\pgfqpoint{1.491046in}{1.275019in}}%
\pgfpathlineto{\pgfqpoint{1.494876in}{1.276556in}}%
\pgfpathlineto{\pgfqpoint{1.502537in}{1.278999in}}%
\pgfpathlineto{\pgfqpoint{1.537010in}{1.282257in}}%
\pgfpathlineto{\pgfqpoint{1.558077in}{1.283564in}}%
\pgfpathlineto{\pgfqpoint{1.575313in}{1.284786in}}%
\pgfpathlineto{\pgfqpoint{1.579143in}{1.285764in}}%
\pgfpathlineto{\pgfqpoint{1.594465in}{1.287291in}}%
\pgfpathlineto{\pgfqpoint{1.604041in}{1.289240in}}%
\pgfpathlineto{\pgfqpoint{1.607871in}{1.290324in}}%
\pgfpathlineto{\pgfqpoint{1.623192in}{1.293321in}}%
\pgfpathlineto{\pgfqpoint{1.630853in}{1.294901in}}%
\pgfpathlineto{\pgfqpoint{1.648090in}{1.296826in}}%
\pgfpathlineto{\pgfqpoint{1.661496in}{1.300628in}}%
\pgfpathlineto{\pgfqpoint{1.671072in}{1.301832in}}%
\pgfpathlineto{\pgfqpoint{1.684478in}{1.303336in}}%
\pgfpathlineto{\pgfqpoint{1.699799in}{1.304607in}}%
\pgfpathlineto{\pgfqpoint{1.715121in}{1.307975in}}%
\pgfpathlineto{\pgfqpoint{1.740018in}{1.311880in}}%
\pgfpathlineto{\pgfqpoint{1.743848in}{1.313828in}}%
\pgfpathlineto{\pgfqpoint{1.749594in}{1.316971in}}%
\pgfpathlineto{\pgfqpoint{1.755339in}{1.322149in}}%
\pgfpathlineto{\pgfqpoint{1.759170in}{1.323558in}}%
\pgfpathlineto{\pgfqpoint{1.768745in}{1.324538in}}%
\pgfpathlineto{\pgfqpoint{1.774491in}{1.326468in}}%
\pgfpathlineto{\pgfqpoint{1.782152in}{1.327961in}}%
\pgfpathlineto{\pgfqpoint{1.791728in}{1.330843in}}%
\pgfpathlineto{\pgfqpoint{1.797473in}{1.333205in}}%
\pgfpathlineto{\pgfqpoint{1.803219in}{1.335420in}}%
\pgfpathlineto{\pgfqpoint{1.807049in}{1.336170in}}%
\pgfpathlineto{\pgfqpoint{1.808964in}{1.340543in}}%
\pgfpathlineto{\pgfqpoint{1.824285in}{1.343062in}}%
\pgfpathlineto{\pgfqpoint{1.831946in}{1.344244in}}%
\pgfpathlineto{\pgfqpoint{1.843437in}{1.351022in}}%
\pgfpathlineto{\pgfqpoint{1.856843in}{1.353266in}}%
\pgfpathlineto{\pgfqpoint{1.860674in}{1.355311in}}%
\pgfpathlineto{\pgfqpoint{1.868334in}{1.356197in}}%
\pgfpathlineto{\pgfqpoint{1.875995in}{1.358003in}}%
\pgfpathlineto{\pgfqpoint{1.881741in}{1.359114in}}%
\pgfpathlineto{\pgfqpoint{1.898977in}{1.362084in}}%
\pgfpathlineto{\pgfqpoint{1.906638in}{1.364194in}}%
\pgfpathlineto{\pgfqpoint{1.908553in}{1.367435in}}%
\pgfpathlineto{\pgfqpoint{1.920044in}{1.369591in}}%
\pgfpathlineto{\pgfqpoint{1.943026in}{1.377823in}}%
\pgfpathlineto{\pgfqpoint{1.952602in}{1.379297in}}%
\pgfpathlineto{\pgfqpoint{1.973669in}{1.385414in}}%
\pgfpathlineto{\pgfqpoint{1.975584in}{1.388690in}}%
\pgfpathlineto{\pgfqpoint{1.979414in}{1.390188in}}%
\pgfpathlineto{\pgfqpoint{1.987075in}{1.391978in}}%
\pgfpathlineto{\pgfqpoint{1.988990in}{1.393523in}}%
\pgfpathlineto{\pgfqpoint{1.990905in}{1.397591in}}%
\pgfpathlineto{\pgfqpoint{1.996651in}{1.400255in}}%
\pgfpathlineto{\pgfqpoint{1.998566in}{1.401797in}}%
\pgfpathlineto{\pgfqpoint{2.002396in}{1.402765in}}%
\pgfpathlineto{\pgfqpoint{2.006227in}{1.404106in}}%
\pgfpathlineto{\pgfqpoint{2.013887in}{1.407482in}}%
\pgfpathlineto{\pgfqpoint{2.017718in}{1.412427in}}%
\pgfpathlineto{\pgfqpoint{2.021548in}{1.412739in}}%
\pgfpathlineto{\pgfqpoint{2.023463in}{1.415211in}}%
\pgfpathlineto{\pgfqpoint{2.027294in}{1.415683in}}%
\pgfpathlineto{\pgfqpoint{2.029209in}{1.418506in}}%
\pgfpathlineto{\pgfqpoint{2.034954in}{1.419221in}}%
\pgfpathlineto{\pgfqpoint{2.036869in}{1.419990in}}%
\pgfpathlineto{\pgfqpoint{2.040700in}{1.422827in}}%
\pgfpathlineto{\pgfqpoint{2.044530in}{1.424988in}}%
\pgfpathlineto{\pgfqpoint{2.048360in}{1.428281in}}%
\pgfpathlineto{\pgfqpoint{2.052191in}{1.429040in}}%
\pgfpathlineto{\pgfqpoint{2.054106in}{1.436249in}}%
\pgfpathlineto{\pgfqpoint{2.057936in}{1.437096in}}%
\pgfpathlineto{\pgfqpoint{2.061767in}{1.437734in}}%
\pgfpathlineto{\pgfqpoint{2.065597in}{1.439421in}}%
\pgfpathlineto{\pgfqpoint{2.075173in}{1.451462in}}%
\pgfpathlineto{\pgfqpoint{2.077088in}{1.452035in}}%
\pgfpathlineto{\pgfqpoint{2.079003in}{1.459892in}}%
\pgfpathlineto{\pgfqpoint{2.082834in}{1.461317in}}%
\pgfpathlineto{\pgfqpoint{2.084749in}{1.470957in}}%
\pgfpathlineto{\pgfqpoint{2.086664in}{1.471310in}}%
\pgfpathlineto{\pgfqpoint{2.088579in}{1.479453in}}%
\pgfpathlineto{\pgfqpoint{2.092409in}{1.481516in}}%
\pgfpathlineto{\pgfqpoint{2.098155in}{1.487379in}}%
\pgfpathlineto{\pgfqpoint{2.100070in}{1.493951in}}%
\pgfpathlineto{\pgfqpoint{2.107731in}{1.497051in}}%
\pgfpathlineto{\pgfqpoint{2.111561in}{1.502227in}}%
\pgfpathlineto{\pgfqpoint{2.113476in}{1.507295in}}%
\pgfpathlineto{\pgfqpoint{2.117307in}{1.508090in}}%
\pgfpathlineto{\pgfqpoint{2.121137in}{1.510253in}}%
\pgfpathlineto{\pgfqpoint{2.123052in}{1.518337in}}%
\pgfpathlineto{\pgfqpoint{2.124967in}{1.519550in}}%
\pgfpathlineto{\pgfqpoint{2.130713in}{1.530636in}}%
\pgfpathlineto{\pgfqpoint{2.151780in}{1.546147in}}%
\pgfpathlineto{\pgfqpoint{2.155610in}{1.547742in}}%
\pgfpathlineto{\pgfqpoint{2.157525in}{1.548744in}}%
\pgfpathlineto{\pgfqpoint{2.159440in}{1.553957in}}%
\pgfpathlineto{\pgfqpoint{2.165186in}{1.556432in}}%
\pgfpathlineto{\pgfqpoint{2.170931in}{1.560348in}}%
\pgfpathlineto{\pgfqpoint{2.174762in}{1.565710in}}%
\pgfpathlineto{\pgfqpoint{2.178592in}{1.582376in}}%
\pgfpathlineto{\pgfqpoint{2.182422in}{1.583808in}}%
\pgfpathlineto{\pgfqpoint{2.188168in}{1.586606in}}%
\pgfpathlineto{\pgfqpoint{2.190083in}{1.591262in}}%
\pgfpathlineto{\pgfqpoint{2.191998in}{1.591599in}}%
\pgfpathlineto{\pgfqpoint{2.195829in}{1.608375in}}%
\pgfpathlineto{\pgfqpoint{2.197744in}{1.624503in}}%
\pgfpathlineto{\pgfqpoint{2.201574in}{1.628559in}}%
\pgfpathlineto{\pgfqpoint{2.203489in}{1.646610in}}%
\pgfpathlineto{\pgfqpoint{2.205405in}{1.649403in}}%
\pgfpathlineto{\pgfqpoint{2.209235in}{1.669327in}}%
\pgfpathlineto{\pgfqpoint{2.213065in}{1.680243in}}%
\pgfpathlineto{\pgfqpoint{2.222641in}{1.687698in}}%
\pgfpathlineto{\pgfqpoint{2.226471in}{1.691013in}}%
\pgfpathlineto{\pgfqpoint{2.228387in}{1.693230in}}%
\pgfpathlineto{\pgfqpoint{2.230302in}{1.693514in}}%
\pgfpathlineto{\pgfqpoint{2.232217in}{1.696116in}}%
\pgfpathlineto{\pgfqpoint{2.236047in}{1.697231in}}%
\pgfpathlineto{\pgfqpoint{2.241793in}{1.700835in}}%
\pgfpathlineto{\pgfqpoint{2.243708in}{1.704929in}}%
\pgfpathlineto{\pgfqpoint{2.245623in}{1.704992in}}%
\pgfpathlineto{\pgfqpoint{2.247538in}{1.707394in}}%
\pgfpathlineto{\pgfqpoint{2.249453in}{1.722035in}}%
\pgfpathlineto{\pgfqpoint{2.253284in}{1.723348in}}%
\pgfpathlineto{\pgfqpoint{2.255199in}{1.726623in}}%
\pgfpathlineto{\pgfqpoint{2.257114in}{1.727001in}}%
\pgfpathlineto{\pgfqpoint{2.264775in}{1.742296in}}%
\pgfpathlineto{\pgfqpoint{2.266690in}{1.743453in}}%
\pgfpathlineto{\pgfqpoint{2.268605in}{1.748524in}}%
\pgfpathlineto{\pgfqpoint{2.285842in}{1.752811in}}%
\pgfpathlineto{\pgfqpoint{2.293502in}{1.754020in}}%
\pgfpathlineto{\pgfqpoint{2.297333in}{1.759564in}}%
\pgfpathlineto{\pgfqpoint{2.299248in}{1.760116in}}%
\pgfpathlineto{\pgfqpoint{2.301163in}{1.762866in}}%
\pgfpathlineto{\pgfqpoint{2.303078in}{1.770227in}}%
\pgfpathlineto{\pgfqpoint{2.306909in}{1.772620in}}%
\pgfpathlineto{\pgfqpoint{2.310739in}{1.778454in}}%
\pgfpathlineto{\pgfqpoint{2.316484in}{1.786908in}}%
\pgfpathlineto{\pgfqpoint{2.322230in}{1.789890in}}%
\pgfpathlineto{\pgfqpoint{2.326060in}{1.793868in}}%
\pgfpathlineto{\pgfqpoint{2.339467in}{1.814021in}}%
\pgfpathlineto{\pgfqpoint{2.343297in}{1.816231in}}%
\pgfpathlineto{\pgfqpoint{2.347127in}{1.826535in}}%
\pgfpathlineto{\pgfqpoint{2.347127in}{1.826535in}}%
\pgfusepath{stroke}%
\end{pgfscope}%
\begin{pgfscope}%
\pgfsetrectcap%
\pgfsetmiterjoin%
\pgfsetlinewidth{0.803000pt}%
\definecolor{currentstroke}{rgb}{0.000000,0.000000,0.000000}%
\pgfsetstrokecolor{currentstroke}%
\pgfsetdash{}{0pt}%
\pgfpathmoveto{\pgfqpoint{0.694334in}{0.523557in}}%
\pgfpathlineto{\pgfqpoint{0.694334in}{1.826535in}}%
\pgfusepath{stroke}%
\end{pgfscope}%
\begin{pgfscope}%
\pgfsetrectcap%
\pgfsetmiterjoin%
\pgfsetlinewidth{0.803000pt}%
\definecolor{currentstroke}{rgb}{0.000000,0.000000,0.000000}%
\pgfsetstrokecolor{currentstroke}%
\pgfsetdash{}{0pt}%
\pgfpathmoveto{\pgfqpoint{4.524677in}{0.523557in}}%
\pgfpathlineto{\pgfqpoint{4.524677in}{1.826535in}}%
\pgfusepath{stroke}%
\end{pgfscope}%
\begin{pgfscope}%
\pgfsetrectcap%
\pgfsetmiterjoin%
\pgfsetlinewidth{0.803000pt}%
\definecolor{currentstroke}{rgb}{0.000000,0.000000,0.000000}%
\pgfsetstrokecolor{currentstroke}%
\pgfsetdash{}{0pt}%
\pgfpathmoveto{\pgfqpoint{0.694334in}{0.523557in}}%
\pgfpathlineto{\pgfqpoint{4.524677in}{0.523557in}}%
\pgfusepath{stroke}%
\end{pgfscope}%
\begin{pgfscope}%
\pgfsetrectcap%
\pgfsetmiterjoin%
\pgfsetlinewidth{0.803000pt}%
\definecolor{currentstroke}{rgb}{0.000000,0.000000,0.000000}%
\pgfsetstrokecolor{currentstroke}%
\pgfsetdash{}{0pt}%
\pgfpathmoveto{\pgfqpoint{0.694334in}{1.826535in}}%
\pgfpathlineto{\pgfqpoint{4.524677in}{1.826535in}}%
\pgfusepath{stroke}%
\end{pgfscope}%
\begin{pgfscope}%
\pgfsetrectcap%
\pgfsetroundjoin%
\pgfsetlinewidth{1.003750pt}%
\definecolor{currentstroke}{rgb}{0.752941,0.752941,1.000000}%
\pgfsetstrokecolor{currentstroke}%
\pgfsetdash{}{0pt}%
\pgfpathmoveto{\pgfqpoint{3.385880in}{1.070445in}}%
\pgfpathlineto{\pgfqpoint{3.608102in}{1.070445in}}%
\pgfusepath{stroke}%
\end{pgfscope}%
\begin{pgfscope}%
\definecolor{textcolor}{rgb}{0.000000,0.000000,0.000000}%
\pgfsetstrokecolor{textcolor}%
\pgfsetfillcolor{textcolor}%
\pgftext[x=3.630324in,y=1.031556in,left,base]{\color{textcolor}\rmfamily\fontsize{8.000000}{9.600000}\selectfont Non-best HTB}%
\end{pgfscope}%
\begin{pgfscope}%
\pgfsetrectcap%
\pgfsetroundjoin%
\pgfsetlinewidth{1.003750pt}%
\definecolor{currentstroke}{rgb}{0.125490,0.662745,0.705882}%
\pgfsetstrokecolor{currentstroke}%
\pgfsetdash{}{0pt}%
\pgfpathmoveto{\pgfqpoint{3.385880in}{0.959956in}}%
\pgfpathlineto{\pgfqpoint{3.608102in}{0.959956in}}%
\pgfusepath{stroke}%
\end{pgfscope}%
\begin{pgfscope}%
\definecolor{textcolor}{rgb}{0.000000,0.000000,0.000000}%
\pgfsetstrokecolor{textcolor}%
\pgfsetfillcolor{textcolor}%
\pgftext[x=3.630324in,y=0.921067in,left,base]{\color{textcolor}\rmfamily\fontsize{8.000000}{9.600000}\selectfont Best HTB}%
\end{pgfscope}%
\begin{pgfscope}%
\pgfsetbuttcap%
\pgfsetroundjoin%
\pgfsetlinewidth{1.003750pt}%
\definecolor{currentstroke}{rgb}{1.000000,0.843137,0.000000}%
\pgfsetstrokecolor{currentstroke}%
\pgfsetdash{{3.700000pt}{1.600000pt}}{0.000000pt}%
\pgfpathmoveto{\pgfqpoint{3.385880in}{0.849467in}}%
\pgfpathlineto{\pgfqpoint{3.608102in}{0.849467in}}%
\pgfusepath{stroke}%
\end{pgfscope}%
\begin{pgfscope}%
\definecolor{textcolor}{rgb}{0.000000,0.000000,0.000000}%
\pgfsetstrokecolor{textcolor}%
\pgfsetfillcolor{textcolor}%
\pgftext[x=3.630324in,y=0.810579in,left,base]{\color{textcolor}\rmfamily\fontsize{8.000000}{9.600000}\selectfont LG+htd}%
\end{pgfscope}%
\begin{pgfscope}%
\pgfsetbuttcap%
\pgfsetroundjoin%
\pgfsetlinewidth{1.003750pt}%
\definecolor{currentstroke}{rgb}{1.000000,0.694118,0.305882}%
\pgfsetstrokecolor{currentstroke}%
\pgfsetdash{{1.000000pt}{1.650000pt}}{0.000000pt}%
\pgfpathmoveto{\pgfqpoint{3.385880in}{0.738979in}}%
\pgfpathlineto{\pgfqpoint{3.608102in}{0.738979in}}%
\pgfusepath{stroke}%
\end{pgfscope}%
\begin{pgfscope}%
\definecolor{textcolor}{rgb}{0.000000,0.000000,0.000000}%
\pgfsetstrokecolor{textcolor}%
\pgfsetfillcolor{textcolor}%
\pgftext[x=3.630324in,y=0.700090in,left,base]{\color{textcolor}\rmfamily\fontsize{8.000000}{9.600000}\selectfont LG+FlowCutter}%
\end{pgfscope}%
\begin{pgfscope}%
\pgfsetrectcap%
\pgfsetroundjoin%
\pgfsetlinewidth{1.003750pt}%
\definecolor{currentstroke}{rgb}{0.917647,0.372549,0.580392}%
\pgfsetstrokecolor{currentstroke}%
\pgfsetdash{}{0pt}%
\pgfpathmoveto{\pgfqpoint{3.385880in}{0.628490in}}%
\pgfpathlineto{\pgfqpoint{3.608102in}{0.628490in}}%
\pgfusepath{stroke}%
\end{pgfscope}%
\begin{pgfscope}%
\definecolor{textcolor}{rgb}{0.000000,0.000000,0.000000}%
\pgfsetstrokecolor{textcolor}%
\pgfsetfillcolor{textcolor}%
\pgftext[x=3.630324in,y=0.589601in,left,base]{\color{textcolor}\rmfamily\fontsize{8.000000}{9.600000}\selectfont LG+Tamaki}%
\end{pgfscope}%
\end{pgfpicture}%
\makeatother%
\endgroup%

    \vspace*{-1cm}
	\caption{\label{fig:planning} A cactus plot of the performance of various planners.
	A planner ``solves'' a benchmark when it finds a project-join tree of width 30 or lower.}
\end{figure}

We first compare constraint-satisfaction heuristics (\Htb) and tree-decomposition tools (\Lg) at building project-join trees of CNF formulas.
To do this, we ran all 36 configurations of \Htb{} (combining four clustering heuristics with nine cluster-variable-order heuristics) and all three configurations of \Lg{} (choosing a tree-decomposition solver) once on each benchmark with a 100-second timeout.
In Figure \ref{fig:planning}, we compare how long it takes various methods to find a high-quality (meaning width at most 30) project-join tree of each benchmark.
We chose 30 for Figure \ref{fig:planning} since we observed in Chapter \ref{ch:tensors} that tensor-based approaches were unable to handle trees whose widths are above 30, but Figure \ref{fig:planning} is qualitatively similar for other choices of widths.
We observe that \Lg{} is generally able to find project-join trees of lower widths than those \Htb{} is able to find.
We therefore conclude that tree-decomposition solvers outperform constraint-satisfaction heuristics in this case.
We observe that \Be-\TreeH{} as the clustering heuristic and \Invlexp{} as the cluster-variable-order heuristic make up the best-performing heuristic configuration from \Htb.
This was previously observed to be the second-best heuristic configuration for weighted model counting in \cite{DPV20}.
We therefore choose \Be-\TreeH{} with \Invlexp{} as the representative heuristic configuration for \Htb{} in the remaining experiments.
For \Lg{}, we choose \Flowcutter{} as the representative tree-decomposition tool in the remaining experiments.

%%%%%%%%%%%%%%%%%%%%%%%%%%%%%%%%%%%%%%%%%%%%%%%%%%%%%%%%%%%%%%%%%%%%%%%%%%%%%%%%

\subsection{Experiment 2: Comparing Execution Environments}
\label{sec_experiments_execution}

\begin{figure}[t]
	\centering
	%% Creator: Matplotlib, PGF backend
%%
%% To include the figure in your LaTeX document, write
%%   \input{<filename>.pgf}
%%
%% Make sure the required packages are loaded in your preamble
%%   \usepackage{pgf}
%%
%% and, on pdftex
%%   \usepackage[utf8]{inputenc}\DeclareUnicodeCharacter{2212}{-}
%%
%% or, on luatex and xetex
%%   \usepackage{unicode-math}
%%
%% Figures using additional raster images can only be included by \input if
%% they are in the same directory as the main LaTeX file. For loading figures
%% from other directories you can use the `import` package
%%   \usepackage{import}
%%
%% and then include the figures with
%%   \import{<path to file>}{<filename>.pgf}
%%
%% Matplotlib used the following preamble
%%   \usepackage[utf8x]{inputenc}
%%   \usepackage[T1]{fontenc}
%%
\begingroup%
\makeatletter%
\begin{pgfpicture}%
\pgfpathrectangle{\pgfpointorigin}{\pgfqpoint{6.000000in}{2.500000in}}%
\pgfusepath{use as bounding box, clip}%
\begin{pgfscope}%
\pgfsetbuttcap%
\pgfsetmiterjoin%
\definecolor{currentfill}{rgb}{1.000000,1.000000,1.000000}%
\pgfsetfillcolor{currentfill}%
\pgfsetlinewidth{0.000000pt}%
\definecolor{currentstroke}{rgb}{1.000000,1.000000,1.000000}%
\pgfsetstrokecolor{currentstroke}%
\pgfsetdash{}{0pt}%
\pgfpathmoveto{\pgfqpoint{0.000000in}{0.000000in}}%
\pgfpathlineto{\pgfqpoint{6.000000in}{0.000000in}}%
\pgfpathlineto{\pgfqpoint{6.000000in}{2.500000in}}%
\pgfpathlineto{\pgfqpoint{0.000000in}{2.500000in}}%
\pgfpathclose%
\pgfusepath{fill}%
\end{pgfscope}%
\begin{pgfscope}%
\pgfsetbuttcap%
\pgfsetmiterjoin%
\definecolor{currentfill}{rgb}{1.000000,1.000000,1.000000}%
\pgfsetfillcolor{currentfill}%
\pgfsetlinewidth{0.000000pt}%
\definecolor{currentstroke}{rgb}{0.000000,0.000000,0.000000}%
\pgfsetstrokecolor{currentstroke}%
\pgfsetstrokeopacity{0.000000}%
\pgfsetdash{}{0pt}%
\pgfpathmoveto{\pgfqpoint{0.708220in}{0.535823in}}%
\pgfpathlineto{\pgfqpoint{5.721529in}{0.535823in}}%
\pgfpathlineto{\pgfqpoint{5.721529in}{2.305275in}}%
\pgfpathlineto{\pgfqpoint{0.708220in}{2.305275in}}%
\pgfpathclose%
\pgfusepath{fill}%
\end{pgfscope}%
\begin{pgfscope}%
\pgfsetbuttcap%
\pgfsetroundjoin%
\definecolor{currentfill}{rgb}{0.000000,0.000000,0.000000}%
\pgfsetfillcolor{currentfill}%
\pgfsetlinewidth{0.803000pt}%
\definecolor{currentstroke}{rgb}{0.000000,0.000000,0.000000}%
\pgfsetstrokecolor{currentstroke}%
\pgfsetdash{}{0pt}%
\pgfsys@defobject{currentmarker}{\pgfqpoint{0.000000in}{-0.048611in}}{\pgfqpoint{0.000000in}{0.000000in}}{%
\pgfpathmoveto{\pgfqpoint{0.000000in}{0.000000in}}%
\pgfpathlineto{\pgfqpoint{0.000000in}{-0.048611in}}%
\pgfusepath{stroke,fill}%
}%
\begin{pgfscope}%
\pgfsys@transformshift{0.708220in}{0.535823in}%
\pgfsys@useobject{currentmarker}{}%
\end{pgfscope}%
\end{pgfscope}%
\begin{pgfscope}%
\definecolor{textcolor}{rgb}{0.000000,0.000000,0.000000}%
\pgfsetstrokecolor{textcolor}%
\pgfsetfillcolor{textcolor}%
\pgftext[x=0.708220in,y=0.438600in,,top]{\color{textcolor}\rmfamily\fontsize{9.000000}{10.800000}\selectfont \(\displaystyle {0}\)}%
\end{pgfscope}%
\begin{pgfscope}%
\pgfsetbuttcap%
\pgfsetroundjoin%
\definecolor{currentfill}{rgb}{0.000000,0.000000,0.000000}%
\pgfsetfillcolor{currentfill}%
\pgfsetlinewidth{0.803000pt}%
\definecolor{currentstroke}{rgb}{0.000000,0.000000,0.000000}%
\pgfsetstrokecolor{currentstroke}%
\pgfsetdash{}{0pt}%
\pgfsys@defobject{currentmarker}{\pgfqpoint{0.000000in}{-0.048611in}}{\pgfqpoint{0.000000in}{0.000000in}}{%
\pgfpathmoveto{\pgfqpoint{0.000000in}{0.000000in}}%
\pgfpathlineto{\pgfqpoint{0.000000in}{-0.048611in}}%
\pgfusepath{stroke,fill}%
}%
\begin{pgfscope}%
\pgfsys@transformshift{1.334883in}{0.535823in}%
\pgfsys@useobject{currentmarker}{}%
\end{pgfscope}%
\end{pgfscope}%
\begin{pgfscope}%
\definecolor{textcolor}{rgb}{0.000000,0.000000,0.000000}%
\pgfsetstrokecolor{textcolor}%
\pgfsetfillcolor{textcolor}%
\pgftext[x=1.334883in,y=0.438600in,,top]{\color{textcolor}\rmfamily\fontsize{9.000000}{10.800000}\selectfont \(\displaystyle {250}\)}%
\end{pgfscope}%
\begin{pgfscope}%
\pgfsetbuttcap%
\pgfsetroundjoin%
\definecolor{currentfill}{rgb}{0.000000,0.000000,0.000000}%
\pgfsetfillcolor{currentfill}%
\pgfsetlinewidth{0.803000pt}%
\definecolor{currentstroke}{rgb}{0.000000,0.000000,0.000000}%
\pgfsetstrokecolor{currentstroke}%
\pgfsetdash{}{0pt}%
\pgfsys@defobject{currentmarker}{\pgfqpoint{0.000000in}{-0.048611in}}{\pgfqpoint{0.000000in}{0.000000in}}{%
\pgfpathmoveto{\pgfqpoint{0.000000in}{0.000000in}}%
\pgfpathlineto{\pgfqpoint{0.000000in}{-0.048611in}}%
\pgfusepath{stroke,fill}%
}%
\begin{pgfscope}%
\pgfsys@transformshift{1.961547in}{0.535823in}%
\pgfsys@useobject{currentmarker}{}%
\end{pgfscope}%
\end{pgfscope}%
\begin{pgfscope}%
\definecolor{textcolor}{rgb}{0.000000,0.000000,0.000000}%
\pgfsetstrokecolor{textcolor}%
\pgfsetfillcolor{textcolor}%
\pgftext[x=1.961547in,y=0.438600in,,top]{\color{textcolor}\rmfamily\fontsize{9.000000}{10.800000}\selectfont \(\displaystyle {500}\)}%
\end{pgfscope}%
\begin{pgfscope}%
\pgfsetbuttcap%
\pgfsetroundjoin%
\definecolor{currentfill}{rgb}{0.000000,0.000000,0.000000}%
\pgfsetfillcolor{currentfill}%
\pgfsetlinewidth{0.803000pt}%
\definecolor{currentstroke}{rgb}{0.000000,0.000000,0.000000}%
\pgfsetstrokecolor{currentstroke}%
\pgfsetdash{}{0pt}%
\pgfsys@defobject{currentmarker}{\pgfqpoint{0.000000in}{-0.048611in}}{\pgfqpoint{0.000000in}{0.000000in}}{%
\pgfpathmoveto{\pgfqpoint{0.000000in}{0.000000in}}%
\pgfpathlineto{\pgfqpoint{0.000000in}{-0.048611in}}%
\pgfusepath{stroke,fill}%
}%
\begin{pgfscope}%
\pgfsys@transformshift{2.588211in}{0.535823in}%
\pgfsys@useobject{currentmarker}{}%
\end{pgfscope}%
\end{pgfscope}%
\begin{pgfscope}%
\definecolor{textcolor}{rgb}{0.000000,0.000000,0.000000}%
\pgfsetstrokecolor{textcolor}%
\pgfsetfillcolor{textcolor}%
\pgftext[x=2.588211in,y=0.438600in,,top]{\color{textcolor}\rmfamily\fontsize{9.000000}{10.800000}\selectfont \(\displaystyle {750}\)}%
\end{pgfscope}%
\begin{pgfscope}%
\pgfsetbuttcap%
\pgfsetroundjoin%
\definecolor{currentfill}{rgb}{0.000000,0.000000,0.000000}%
\pgfsetfillcolor{currentfill}%
\pgfsetlinewidth{0.803000pt}%
\definecolor{currentstroke}{rgb}{0.000000,0.000000,0.000000}%
\pgfsetstrokecolor{currentstroke}%
\pgfsetdash{}{0pt}%
\pgfsys@defobject{currentmarker}{\pgfqpoint{0.000000in}{-0.048611in}}{\pgfqpoint{0.000000in}{0.000000in}}{%
\pgfpathmoveto{\pgfqpoint{0.000000in}{0.000000in}}%
\pgfpathlineto{\pgfqpoint{0.000000in}{-0.048611in}}%
\pgfusepath{stroke,fill}%
}%
\begin{pgfscope}%
\pgfsys@transformshift{3.214874in}{0.535823in}%
\pgfsys@useobject{currentmarker}{}%
\end{pgfscope}%
\end{pgfscope}%
\begin{pgfscope}%
\definecolor{textcolor}{rgb}{0.000000,0.000000,0.000000}%
\pgfsetstrokecolor{textcolor}%
\pgfsetfillcolor{textcolor}%
\pgftext[x=3.214874in,y=0.438600in,,top]{\color{textcolor}\rmfamily\fontsize{9.000000}{10.800000}\selectfont \(\displaystyle {1000}\)}%
\end{pgfscope}%
\begin{pgfscope}%
\pgfsetbuttcap%
\pgfsetroundjoin%
\definecolor{currentfill}{rgb}{0.000000,0.000000,0.000000}%
\pgfsetfillcolor{currentfill}%
\pgfsetlinewidth{0.803000pt}%
\definecolor{currentstroke}{rgb}{0.000000,0.000000,0.000000}%
\pgfsetstrokecolor{currentstroke}%
\pgfsetdash{}{0pt}%
\pgfsys@defobject{currentmarker}{\pgfqpoint{0.000000in}{-0.048611in}}{\pgfqpoint{0.000000in}{0.000000in}}{%
\pgfpathmoveto{\pgfqpoint{0.000000in}{0.000000in}}%
\pgfpathlineto{\pgfqpoint{0.000000in}{-0.048611in}}%
\pgfusepath{stroke,fill}%
}%
\begin{pgfscope}%
\pgfsys@transformshift{3.841538in}{0.535823in}%
\pgfsys@useobject{currentmarker}{}%
\end{pgfscope}%
\end{pgfscope}%
\begin{pgfscope}%
\definecolor{textcolor}{rgb}{0.000000,0.000000,0.000000}%
\pgfsetstrokecolor{textcolor}%
\pgfsetfillcolor{textcolor}%
\pgftext[x=3.841538in,y=0.438600in,,top]{\color{textcolor}\rmfamily\fontsize{9.000000}{10.800000}\selectfont \(\displaystyle {1250}\)}%
\end{pgfscope}%
\begin{pgfscope}%
\pgfsetbuttcap%
\pgfsetroundjoin%
\definecolor{currentfill}{rgb}{0.000000,0.000000,0.000000}%
\pgfsetfillcolor{currentfill}%
\pgfsetlinewidth{0.803000pt}%
\definecolor{currentstroke}{rgb}{0.000000,0.000000,0.000000}%
\pgfsetstrokecolor{currentstroke}%
\pgfsetdash{}{0pt}%
\pgfsys@defobject{currentmarker}{\pgfqpoint{0.000000in}{-0.048611in}}{\pgfqpoint{0.000000in}{0.000000in}}{%
\pgfpathmoveto{\pgfqpoint{0.000000in}{0.000000in}}%
\pgfpathlineto{\pgfqpoint{0.000000in}{-0.048611in}}%
\pgfusepath{stroke,fill}%
}%
\begin{pgfscope}%
\pgfsys@transformshift{4.468201in}{0.535823in}%
\pgfsys@useobject{currentmarker}{}%
\end{pgfscope}%
\end{pgfscope}%
\begin{pgfscope}%
\definecolor{textcolor}{rgb}{0.000000,0.000000,0.000000}%
\pgfsetstrokecolor{textcolor}%
\pgfsetfillcolor{textcolor}%
\pgftext[x=4.468201in,y=0.438600in,,top]{\color{textcolor}\rmfamily\fontsize{9.000000}{10.800000}\selectfont \(\displaystyle {1500}\)}%
\end{pgfscope}%
\begin{pgfscope}%
\pgfsetbuttcap%
\pgfsetroundjoin%
\definecolor{currentfill}{rgb}{0.000000,0.000000,0.000000}%
\pgfsetfillcolor{currentfill}%
\pgfsetlinewidth{0.803000pt}%
\definecolor{currentstroke}{rgb}{0.000000,0.000000,0.000000}%
\pgfsetstrokecolor{currentstroke}%
\pgfsetdash{}{0pt}%
\pgfsys@defobject{currentmarker}{\pgfqpoint{0.000000in}{-0.048611in}}{\pgfqpoint{0.000000in}{0.000000in}}{%
\pgfpathmoveto{\pgfqpoint{0.000000in}{0.000000in}}%
\pgfpathlineto{\pgfqpoint{0.000000in}{-0.048611in}}%
\pgfusepath{stroke,fill}%
}%
\begin{pgfscope}%
\pgfsys@transformshift{5.094865in}{0.535823in}%
\pgfsys@useobject{currentmarker}{}%
\end{pgfscope}%
\end{pgfscope}%
\begin{pgfscope}%
\definecolor{textcolor}{rgb}{0.000000,0.000000,0.000000}%
\pgfsetstrokecolor{textcolor}%
\pgfsetfillcolor{textcolor}%
\pgftext[x=5.094865in,y=0.438600in,,top]{\color{textcolor}\rmfamily\fontsize{9.000000}{10.800000}\selectfont \(\displaystyle {1750}\)}%
\end{pgfscope}%
\begin{pgfscope}%
\pgfsetbuttcap%
\pgfsetroundjoin%
\definecolor{currentfill}{rgb}{0.000000,0.000000,0.000000}%
\pgfsetfillcolor{currentfill}%
\pgfsetlinewidth{0.803000pt}%
\definecolor{currentstroke}{rgb}{0.000000,0.000000,0.000000}%
\pgfsetstrokecolor{currentstroke}%
\pgfsetdash{}{0pt}%
\pgfsys@defobject{currentmarker}{\pgfqpoint{0.000000in}{-0.048611in}}{\pgfqpoint{0.000000in}{0.000000in}}{%
\pgfpathmoveto{\pgfqpoint{0.000000in}{0.000000in}}%
\pgfpathlineto{\pgfqpoint{0.000000in}{-0.048611in}}%
\pgfusepath{stroke,fill}%
}%
\begin{pgfscope}%
\pgfsys@transformshift{5.721529in}{0.535823in}%
\pgfsys@useobject{currentmarker}{}%
\end{pgfscope}%
\end{pgfscope}%
\begin{pgfscope}%
\definecolor{textcolor}{rgb}{0.000000,0.000000,0.000000}%
\pgfsetstrokecolor{textcolor}%
\pgfsetfillcolor{textcolor}%
\pgftext[x=5.721529in,y=0.438600in,,top]{\color{textcolor}\rmfamily\fontsize{9.000000}{10.800000}\selectfont \(\displaystyle {2000}\)}%
\end{pgfscope}%
\begin{pgfscope}%
\definecolor{textcolor}{rgb}{0.000000,0.000000,0.000000}%
\pgfsetstrokecolor{textcolor}%
\pgfsetfillcolor{textcolor}%
\pgftext[x=3.214874in,y=0.272655in,,top]{\color{textcolor}\rmfamily\fontsize{10.000000}{12.000000}\selectfont Number of benchmarks solved}%
\end{pgfscope}%
\begin{pgfscope}%
\pgfsetbuttcap%
\pgfsetroundjoin%
\definecolor{currentfill}{rgb}{0.000000,0.000000,0.000000}%
\pgfsetfillcolor{currentfill}%
\pgfsetlinewidth{0.803000pt}%
\definecolor{currentstroke}{rgb}{0.000000,0.000000,0.000000}%
\pgfsetstrokecolor{currentstroke}%
\pgfsetdash{}{0pt}%
\pgfsys@defobject{currentmarker}{\pgfqpoint{-0.048611in}{0.000000in}}{\pgfqpoint{-0.000000in}{0.000000in}}{%
\pgfpathmoveto{\pgfqpoint{-0.000000in}{0.000000in}}%
\pgfpathlineto{\pgfqpoint{-0.048611in}{0.000000in}}%
\pgfusepath{stroke,fill}%
}%
\begin{pgfscope}%
\pgfsys@transformshift{0.708220in}{0.659667in}%
\pgfsys@useobject{currentmarker}{}%
\end{pgfscope}%
\end{pgfscope}%
\begin{pgfscope}%
\definecolor{textcolor}{rgb}{0.000000,0.000000,0.000000}%
\pgfsetstrokecolor{textcolor}%
\pgfsetfillcolor{textcolor}%
\pgftext[x=0.344411in, y=0.614942in, left, base]{\color{textcolor}\rmfamily\fontsize{9.000000}{10.800000}\selectfont \(\displaystyle {10^{-2}}\)}%
\end{pgfscope}%
\begin{pgfscope}%
\pgfsetbuttcap%
\pgfsetroundjoin%
\definecolor{currentfill}{rgb}{0.000000,0.000000,0.000000}%
\pgfsetfillcolor{currentfill}%
\pgfsetlinewidth{0.803000pt}%
\definecolor{currentstroke}{rgb}{0.000000,0.000000,0.000000}%
\pgfsetstrokecolor{currentstroke}%
\pgfsetdash{}{0pt}%
\pgfsys@defobject{currentmarker}{\pgfqpoint{-0.048611in}{0.000000in}}{\pgfqpoint{-0.000000in}{0.000000in}}{%
\pgfpathmoveto{\pgfqpoint{-0.000000in}{0.000000in}}%
\pgfpathlineto{\pgfqpoint{-0.048611in}{0.000000in}}%
\pgfusepath{stroke,fill}%
}%
\begin{pgfscope}%
\pgfsys@transformshift{0.708220in}{1.071069in}%
\pgfsys@useobject{currentmarker}{}%
\end{pgfscope}%
\end{pgfscope}%
\begin{pgfscope}%
\definecolor{textcolor}{rgb}{0.000000,0.000000,0.000000}%
\pgfsetstrokecolor{textcolor}%
\pgfsetfillcolor{textcolor}%
\pgftext[x=0.344411in, y=1.026344in, left, base]{\color{textcolor}\rmfamily\fontsize{9.000000}{10.800000}\selectfont \(\displaystyle {10^{-1}}\)}%
\end{pgfscope}%
\begin{pgfscope}%
\pgfsetbuttcap%
\pgfsetroundjoin%
\definecolor{currentfill}{rgb}{0.000000,0.000000,0.000000}%
\pgfsetfillcolor{currentfill}%
\pgfsetlinewidth{0.803000pt}%
\definecolor{currentstroke}{rgb}{0.000000,0.000000,0.000000}%
\pgfsetstrokecolor{currentstroke}%
\pgfsetdash{}{0pt}%
\pgfsys@defobject{currentmarker}{\pgfqpoint{-0.048611in}{0.000000in}}{\pgfqpoint{-0.000000in}{0.000000in}}{%
\pgfpathmoveto{\pgfqpoint{-0.000000in}{0.000000in}}%
\pgfpathlineto{\pgfqpoint{-0.048611in}{0.000000in}}%
\pgfusepath{stroke,fill}%
}%
\begin{pgfscope}%
\pgfsys@transformshift{0.708220in}{1.482471in}%
\pgfsys@useobject{currentmarker}{}%
\end{pgfscope}%
\end{pgfscope}%
\begin{pgfscope}%
\definecolor{textcolor}{rgb}{0.000000,0.000000,0.000000}%
\pgfsetstrokecolor{textcolor}%
\pgfsetfillcolor{textcolor}%
\pgftext[x=0.424657in, y=1.437746in, left, base]{\color{textcolor}\rmfamily\fontsize{9.000000}{10.800000}\selectfont \(\displaystyle {10^{0}}\)}%
\end{pgfscope}%
\begin{pgfscope}%
\pgfsetbuttcap%
\pgfsetroundjoin%
\definecolor{currentfill}{rgb}{0.000000,0.000000,0.000000}%
\pgfsetfillcolor{currentfill}%
\pgfsetlinewidth{0.803000pt}%
\definecolor{currentstroke}{rgb}{0.000000,0.000000,0.000000}%
\pgfsetstrokecolor{currentstroke}%
\pgfsetdash{}{0pt}%
\pgfsys@defobject{currentmarker}{\pgfqpoint{-0.048611in}{0.000000in}}{\pgfqpoint{-0.000000in}{0.000000in}}{%
\pgfpathmoveto{\pgfqpoint{-0.000000in}{0.000000in}}%
\pgfpathlineto{\pgfqpoint{-0.048611in}{0.000000in}}%
\pgfusepath{stroke,fill}%
}%
\begin{pgfscope}%
\pgfsys@transformshift{0.708220in}{1.893873in}%
\pgfsys@useobject{currentmarker}{}%
\end{pgfscope}%
\end{pgfscope}%
\begin{pgfscope}%
\definecolor{textcolor}{rgb}{0.000000,0.000000,0.000000}%
\pgfsetstrokecolor{textcolor}%
\pgfsetfillcolor{textcolor}%
\pgftext[x=0.424657in, y=1.849148in, left, base]{\color{textcolor}\rmfamily\fontsize{9.000000}{10.800000}\selectfont \(\displaystyle {10^{1}}\)}%
\end{pgfscope}%
\begin{pgfscope}%
\pgfsetbuttcap%
\pgfsetroundjoin%
\definecolor{currentfill}{rgb}{0.000000,0.000000,0.000000}%
\pgfsetfillcolor{currentfill}%
\pgfsetlinewidth{0.803000pt}%
\definecolor{currentstroke}{rgb}{0.000000,0.000000,0.000000}%
\pgfsetstrokecolor{currentstroke}%
\pgfsetdash{}{0pt}%
\pgfsys@defobject{currentmarker}{\pgfqpoint{-0.048611in}{0.000000in}}{\pgfqpoint{-0.000000in}{0.000000in}}{%
\pgfpathmoveto{\pgfqpoint{-0.000000in}{0.000000in}}%
\pgfpathlineto{\pgfqpoint{-0.048611in}{0.000000in}}%
\pgfusepath{stroke,fill}%
}%
\begin{pgfscope}%
\pgfsys@transformshift{0.708220in}{2.305275in}%
\pgfsys@useobject{currentmarker}{}%
\end{pgfscope}%
\end{pgfscope}%
\begin{pgfscope}%
\definecolor{textcolor}{rgb}{0.000000,0.000000,0.000000}%
\pgfsetstrokecolor{textcolor}%
\pgfsetfillcolor{textcolor}%
\pgftext[x=0.424657in, y=2.260550in, left, base]{\color{textcolor}\rmfamily\fontsize{9.000000}{10.800000}\selectfont \(\displaystyle {10^{2}}\)}%
\end{pgfscope}%
\begin{pgfscope}%
\pgfsetbuttcap%
\pgfsetroundjoin%
\definecolor{currentfill}{rgb}{0.000000,0.000000,0.000000}%
\pgfsetfillcolor{currentfill}%
\pgfsetlinewidth{0.602250pt}%
\definecolor{currentstroke}{rgb}{0.000000,0.000000,0.000000}%
\pgfsetstrokecolor{currentstroke}%
\pgfsetdash{}{0pt}%
\pgfsys@defobject{currentmarker}{\pgfqpoint{-0.027778in}{0.000000in}}{\pgfqpoint{-0.000000in}{0.000000in}}{%
\pgfpathmoveto{\pgfqpoint{-0.000000in}{0.000000in}}%
\pgfpathlineto{\pgfqpoint{-0.027778in}{0.000000in}}%
\pgfusepath{stroke,fill}%
}%
\begin{pgfscope}%
\pgfsys@transformshift{0.708220in}{0.535823in}%
\pgfsys@useobject{currentmarker}{}%
\end{pgfscope}%
\end{pgfscope}%
\begin{pgfscope}%
\pgfsetbuttcap%
\pgfsetroundjoin%
\definecolor{currentfill}{rgb}{0.000000,0.000000,0.000000}%
\pgfsetfillcolor{currentfill}%
\pgfsetlinewidth{0.602250pt}%
\definecolor{currentstroke}{rgb}{0.000000,0.000000,0.000000}%
\pgfsetstrokecolor{currentstroke}%
\pgfsetdash{}{0pt}%
\pgfsys@defobject{currentmarker}{\pgfqpoint{-0.027778in}{0.000000in}}{\pgfqpoint{-0.000000in}{0.000000in}}{%
\pgfpathmoveto{\pgfqpoint{-0.000000in}{0.000000in}}%
\pgfpathlineto{\pgfqpoint{-0.027778in}{0.000000in}}%
\pgfusepath{stroke,fill}%
}%
\begin{pgfscope}%
\pgfsys@transformshift{0.708220in}{0.568398in}%
\pgfsys@useobject{currentmarker}{}%
\end{pgfscope}%
\end{pgfscope}%
\begin{pgfscope}%
\pgfsetbuttcap%
\pgfsetroundjoin%
\definecolor{currentfill}{rgb}{0.000000,0.000000,0.000000}%
\pgfsetfillcolor{currentfill}%
\pgfsetlinewidth{0.602250pt}%
\definecolor{currentstroke}{rgb}{0.000000,0.000000,0.000000}%
\pgfsetstrokecolor{currentstroke}%
\pgfsetdash{}{0pt}%
\pgfsys@defobject{currentmarker}{\pgfqpoint{-0.027778in}{0.000000in}}{\pgfqpoint{-0.000000in}{0.000000in}}{%
\pgfpathmoveto{\pgfqpoint{-0.000000in}{0.000000in}}%
\pgfpathlineto{\pgfqpoint{-0.027778in}{0.000000in}}%
\pgfusepath{stroke,fill}%
}%
\begin{pgfscope}%
\pgfsys@transformshift{0.708220in}{0.595940in}%
\pgfsys@useobject{currentmarker}{}%
\end{pgfscope}%
\end{pgfscope}%
\begin{pgfscope}%
\pgfsetbuttcap%
\pgfsetroundjoin%
\definecolor{currentfill}{rgb}{0.000000,0.000000,0.000000}%
\pgfsetfillcolor{currentfill}%
\pgfsetlinewidth{0.602250pt}%
\definecolor{currentstroke}{rgb}{0.000000,0.000000,0.000000}%
\pgfsetstrokecolor{currentstroke}%
\pgfsetdash{}{0pt}%
\pgfsys@defobject{currentmarker}{\pgfqpoint{-0.027778in}{0.000000in}}{\pgfqpoint{-0.000000in}{0.000000in}}{%
\pgfpathmoveto{\pgfqpoint{-0.000000in}{0.000000in}}%
\pgfpathlineto{\pgfqpoint{-0.027778in}{0.000000in}}%
\pgfusepath{stroke,fill}%
}%
\begin{pgfscope}%
\pgfsys@transformshift{0.708220in}{0.619798in}%
\pgfsys@useobject{currentmarker}{}%
\end{pgfscope}%
\end{pgfscope}%
\begin{pgfscope}%
\pgfsetbuttcap%
\pgfsetroundjoin%
\definecolor{currentfill}{rgb}{0.000000,0.000000,0.000000}%
\pgfsetfillcolor{currentfill}%
\pgfsetlinewidth{0.602250pt}%
\definecolor{currentstroke}{rgb}{0.000000,0.000000,0.000000}%
\pgfsetstrokecolor{currentstroke}%
\pgfsetdash{}{0pt}%
\pgfsys@defobject{currentmarker}{\pgfqpoint{-0.027778in}{0.000000in}}{\pgfqpoint{-0.000000in}{0.000000in}}{%
\pgfpathmoveto{\pgfqpoint{-0.000000in}{0.000000in}}%
\pgfpathlineto{\pgfqpoint{-0.027778in}{0.000000in}}%
\pgfusepath{stroke,fill}%
}%
\begin{pgfscope}%
\pgfsys@transformshift{0.708220in}{0.640842in}%
\pgfsys@useobject{currentmarker}{}%
\end{pgfscope}%
\end{pgfscope}%
\begin{pgfscope}%
\pgfsetbuttcap%
\pgfsetroundjoin%
\definecolor{currentfill}{rgb}{0.000000,0.000000,0.000000}%
\pgfsetfillcolor{currentfill}%
\pgfsetlinewidth{0.602250pt}%
\definecolor{currentstroke}{rgb}{0.000000,0.000000,0.000000}%
\pgfsetstrokecolor{currentstroke}%
\pgfsetdash{}{0pt}%
\pgfsys@defobject{currentmarker}{\pgfqpoint{-0.027778in}{0.000000in}}{\pgfqpoint{-0.000000in}{0.000000in}}{%
\pgfpathmoveto{\pgfqpoint{-0.000000in}{0.000000in}}%
\pgfpathlineto{\pgfqpoint{-0.027778in}{0.000000in}}%
\pgfusepath{stroke,fill}%
}%
\begin{pgfscope}%
\pgfsys@transformshift{0.708220in}{0.783511in}%
\pgfsys@useobject{currentmarker}{}%
\end{pgfscope}%
\end{pgfscope}%
\begin{pgfscope}%
\pgfsetbuttcap%
\pgfsetroundjoin%
\definecolor{currentfill}{rgb}{0.000000,0.000000,0.000000}%
\pgfsetfillcolor{currentfill}%
\pgfsetlinewidth{0.602250pt}%
\definecolor{currentstroke}{rgb}{0.000000,0.000000,0.000000}%
\pgfsetstrokecolor{currentstroke}%
\pgfsetdash{}{0pt}%
\pgfsys@defobject{currentmarker}{\pgfqpoint{-0.027778in}{0.000000in}}{\pgfqpoint{-0.000000in}{0.000000in}}{%
\pgfpathmoveto{\pgfqpoint{-0.000000in}{0.000000in}}%
\pgfpathlineto{\pgfqpoint{-0.027778in}{0.000000in}}%
\pgfusepath{stroke,fill}%
}%
\begin{pgfscope}%
\pgfsys@transformshift{0.708220in}{0.855956in}%
\pgfsys@useobject{currentmarker}{}%
\end{pgfscope}%
\end{pgfscope}%
\begin{pgfscope}%
\pgfsetbuttcap%
\pgfsetroundjoin%
\definecolor{currentfill}{rgb}{0.000000,0.000000,0.000000}%
\pgfsetfillcolor{currentfill}%
\pgfsetlinewidth{0.602250pt}%
\definecolor{currentstroke}{rgb}{0.000000,0.000000,0.000000}%
\pgfsetstrokecolor{currentstroke}%
\pgfsetdash{}{0pt}%
\pgfsys@defobject{currentmarker}{\pgfqpoint{-0.027778in}{0.000000in}}{\pgfqpoint{-0.000000in}{0.000000in}}{%
\pgfpathmoveto{\pgfqpoint{-0.000000in}{0.000000in}}%
\pgfpathlineto{\pgfqpoint{-0.027778in}{0.000000in}}%
\pgfusepath{stroke,fill}%
}%
\begin{pgfscope}%
\pgfsys@transformshift{0.708220in}{0.907356in}%
\pgfsys@useobject{currentmarker}{}%
\end{pgfscope}%
\end{pgfscope}%
\begin{pgfscope}%
\pgfsetbuttcap%
\pgfsetroundjoin%
\definecolor{currentfill}{rgb}{0.000000,0.000000,0.000000}%
\pgfsetfillcolor{currentfill}%
\pgfsetlinewidth{0.602250pt}%
\definecolor{currentstroke}{rgb}{0.000000,0.000000,0.000000}%
\pgfsetstrokecolor{currentstroke}%
\pgfsetdash{}{0pt}%
\pgfsys@defobject{currentmarker}{\pgfqpoint{-0.027778in}{0.000000in}}{\pgfqpoint{-0.000000in}{0.000000in}}{%
\pgfpathmoveto{\pgfqpoint{-0.000000in}{0.000000in}}%
\pgfpathlineto{\pgfqpoint{-0.027778in}{0.000000in}}%
\pgfusepath{stroke,fill}%
}%
\begin{pgfscope}%
\pgfsys@transformshift{0.708220in}{0.947225in}%
\pgfsys@useobject{currentmarker}{}%
\end{pgfscope}%
\end{pgfscope}%
\begin{pgfscope}%
\pgfsetbuttcap%
\pgfsetroundjoin%
\definecolor{currentfill}{rgb}{0.000000,0.000000,0.000000}%
\pgfsetfillcolor{currentfill}%
\pgfsetlinewidth{0.602250pt}%
\definecolor{currentstroke}{rgb}{0.000000,0.000000,0.000000}%
\pgfsetstrokecolor{currentstroke}%
\pgfsetdash{}{0pt}%
\pgfsys@defobject{currentmarker}{\pgfqpoint{-0.027778in}{0.000000in}}{\pgfqpoint{-0.000000in}{0.000000in}}{%
\pgfpathmoveto{\pgfqpoint{-0.000000in}{0.000000in}}%
\pgfpathlineto{\pgfqpoint{-0.027778in}{0.000000in}}%
\pgfusepath{stroke,fill}%
}%
\begin{pgfscope}%
\pgfsys@transformshift{0.708220in}{0.979800in}%
\pgfsys@useobject{currentmarker}{}%
\end{pgfscope}%
\end{pgfscope}%
\begin{pgfscope}%
\pgfsetbuttcap%
\pgfsetroundjoin%
\definecolor{currentfill}{rgb}{0.000000,0.000000,0.000000}%
\pgfsetfillcolor{currentfill}%
\pgfsetlinewidth{0.602250pt}%
\definecolor{currentstroke}{rgb}{0.000000,0.000000,0.000000}%
\pgfsetstrokecolor{currentstroke}%
\pgfsetdash{}{0pt}%
\pgfsys@defobject{currentmarker}{\pgfqpoint{-0.027778in}{0.000000in}}{\pgfqpoint{-0.000000in}{0.000000in}}{%
\pgfpathmoveto{\pgfqpoint{-0.000000in}{0.000000in}}%
\pgfpathlineto{\pgfqpoint{-0.027778in}{0.000000in}}%
\pgfusepath{stroke,fill}%
}%
\begin{pgfscope}%
\pgfsys@transformshift{0.708220in}{1.007342in}%
\pgfsys@useobject{currentmarker}{}%
\end{pgfscope}%
\end{pgfscope}%
\begin{pgfscope}%
\pgfsetbuttcap%
\pgfsetroundjoin%
\definecolor{currentfill}{rgb}{0.000000,0.000000,0.000000}%
\pgfsetfillcolor{currentfill}%
\pgfsetlinewidth{0.602250pt}%
\definecolor{currentstroke}{rgb}{0.000000,0.000000,0.000000}%
\pgfsetstrokecolor{currentstroke}%
\pgfsetdash{}{0pt}%
\pgfsys@defobject{currentmarker}{\pgfqpoint{-0.027778in}{0.000000in}}{\pgfqpoint{-0.000000in}{0.000000in}}{%
\pgfpathmoveto{\pgfqpoint{-0.000000in}{0.000000in}}%
\pgfpathlineto{\pgfqpoint{-0.027778in}{0.000000in}}%
\pgfusepath{stroke,fill}%
}%
\begin{pgfscope}%
\pgfsys@transformshift{0.708220in}{1.031200in}%
\pgfsys@useobject{currentmarker}{}%
\end{pgfscope}%
\end{pgfscope}%
\begin{pgfscope}%
\pgfsetbuttcap%
\pgfsetroundjoin%
\definecolor{currentfill}{rgb}{0.000000,0.000000,0.000000}%
\pgfsetfillcolor{currentfill}%
\pgfsetlinewidth{0.602250pt}%
\definecolor{currentstroke}{rgb}{0.000000,0.000000,0.000000}%
\pgfsetstrokecolor{currentstroke}%
\pgfsetdash{}{0pt}%
\pgfsys@defobject{currentmarker}{\pgfqpoint{-0.027778in}{0.000000in}}{\pgfqpoint{-0.000000in}{0.000000in}}{%
\pgfpathmoveto{\pgfqpoint{-0.000000in}{0.000000in}}%
\pgfpathlineto{\pgfqpoint{-0.027778in}{0.000000in}}%
\pgfusepath{stroke,fill}%
}%
\begin{pgfscope}%
\pgfsys@transformshift{0.708220in}{1.052244in}%
\pgfsys@useobject{currentmarker}{}%
\end{pgfscope}%
\end{pgfscope}%
\begin{pgfscope}%
\pgfsetbuttcap%
\pgfsetroundjoin%
\definecolor{currentfill}{rgb}{0.000000,0.000000,0.000000}%
\pgfsetfillcolor{currentfill}%
\pgfsetlinewidth{0.602250pt}%
\definecolor{currentstroke}{rgb}{0.000000,0.000000,0.000000}%
\pgfsetstrokecolor{currentstroke}%
\pgfsetdash{}{0pt}%
\pgfsys@defobject{currentmarker}{\pgfqpoint{-0.027778in}{0.000000in}}{\pgfqpoint{-0.000000in}{0.000000in}}{%
\pgfpathmoveto{\pgfqpoint{-0.000000in}{0.000000in}}%
\pgfpathlineto{\pgfqpoint{-0.027778in}{0.000000in}}%
\pgfusepath{stroke,fill}%
}%
\begin{pgfscope}%
\pgfsys@transformshift{0.708220in}{1.194913in}%
\pgfsys@useobject{currentmarker}{}%
\end{pgfscope}%
\end{pgfscope}%
\begin{pgfscope}%
\pgfsetbuttcap%
\pgfsetroundjoin%
\definecolor{currentfill}{rgb}{0.000000,0.000000,0.000000}%
\pgfsetfillcolor{currentfill}%
\pgfsetlinewidth{0.602250pt}%
\definecolor{currentstroke}{rgb}{0.000000,0.000000,0.000000}%
\pgfsetstrokecolor{currentstroke}%
\pgfsetdash{}{0pt}%
\pgfsys@defobject{currentmarker}{\pgfqpoint{-0.027778in}{0.000000in}}{\pgfqpoint{-0.000000in}{0.000000in}}{%
\pgfpathmoveto{\pgfqpoint{-0.000000in}{0.000000in}}%
\pgfpathlineto{\pgfqpoint{-0.027778in}{0.000000in}}%
\pgfusepath{stroke,fill}%
}%
\begin{pgfscope}%
\pgfsys@transformshift{0.708220in}{1.267358in}%
\pgfsys@useobject{currentmarker}{}%
\end{pgfscope}%
\end{pgfscope}%
\begin{pgfscope}%
\pgfsetbuttcap%
\pgfsetroundjoin%
\definecolor{currentfill}{rgb}{0.000000,0.000000,0.000000}%
\pgfsetfillcolor{currentfill}%
\pgfsetlinewidth{0.602250pt}%
\definecolor{currentstroke}{rgb}{0.000000,0.000000,0.000000}%
\pgfsetstrokecolor{currentstroke}%
\pgfsetdash{}{0pt}%
\pgfsys@defobject{currentmarker}{\pgfqpoint{-0.027778in}{0.000000in}}{\pgfqpoint{-0.000000in}{0.000000in}}{%
\pgfpathmoveto{\pgfqpoint{-0.000000in}{0.000000in}}%
\pgfpathlineto{\pgfqpoint{-0.027778in}{0.000000in}}%
\pgfusepath{stroke,fill}%
}%
\begin{pgfscope}%
\pgfsys@transformshift{0.708220in}{1.318758in}%
\pgfsys@useobject{currentmarker}{}%
\end{pgfscope}%
\end{pgfscope}%
\begin{pgfscope}%
\pgfsetbuttcap%
\pgfsetroundjoin%
\definecolor{currentfill}{rgb}{0.000000,0.000000,0.000000}%
\pgfsetfillcolor{currentfill}%
\pgfsetlinewidth{0.602250pt}%
\definecolor{currentstroke}{rgb}{0.000000,0.000000,0.000000}%
\pgfsetstrokecolor{currentstroke}%
\pgfsetdash{}{0pt}%
\pgfsys@defobject{currentmarker}{\pgfqpoint{-0.027778in}{0.000000in}}{\pgfqpoint{-0.000000in}{0.000000in}}{%
\pgfpathmoveto{\pgfqpoint{-0.000000in}{0.000000in}}%
\pgfpathlineto{\pgfqpoint{-0.027778in}{0.000000in}}%
\pgfusepath{stroke,fill}%
}%
\begin{pgfscope}%
\pgfsys@transformshift{0.708220in}{1.358627in}%
\pgfsys@useobject{currentmarker}{}%
\end{pgfscope}%
\end{pgfscope}%
\begin{pgfscope}%
\pgfsetbuttcap%
\pgfsetroundjoin%
\definecolor{currentfill}{rgb}{0.000000,0.000000,0.000000}%
\pgfsetfillcolor{currentfill}%
\pgfsetlinewidth{0.602250pt}%
\definecolor{currentstroke}{rgb}{0.000000,0.000000,0.000000}%
\pgfsetstrokecolor{currentstroke}%
\pgfsetdash{}{0pt}%
\pgfsys@defobject{currentmarker}{\pgfqpoint{-0.027778in}{0.000000in}}{\pgfqpoint{-0.000000in}{0.000000in}}{%
\pgfpathmoveto{\pgfqpoint{-0.000000in}{0.000000in}}%
\pgfpathlineto{\pgfqpoint{-0.027778in}{0.000000in}}%
\pgfusepath{stroke,fill}%
}%
\begin{pgfscope}%
\pgfsys@transformshift{0.708220in}{1.391202in}%
\pgfsys@useobject{currentmarker}{}%
\end{pgfscope}%
\end{pgfscope}%
\begin{pgfscope}%
\pgfsetbuttcap%
\pgfsetroundjoin%
\definecolor{currentfill}{rgb}{0.000000,0.000000,0.000000}%
\pgfsetfillcolor{currentfill}%
\pgfsetlinewidth{0.602250pt}%
\definecolor{currentstroke}{rgb}{0.000000,0.000000,0.000000}%
\pgfsetstrokecolor{currentstroke}%
\pgfsetdash{}{0pt}%
\pgfsys@defobject{currentmarker}{\pgfqpoint{-0.027778in}{0.000000in}}{\pgfqpoint{-0.000000in}{0.000000in}}{%
\pgfpathmoveto{\pgfqpoint{-0.000000in}{0.000000in}}%
\pgfpathlineto{\pgfqpoint{-0.027778in}{0.000000in}}%
\pgfusepath{stroke,fill}%
}%
\begin{pgfscope}%
\pgfsys@transformshift{0.708220in}{1.418744in}%
\pgfsys@useobject{currentmarker}{}%
\end{pgfscope}%
\end{pgfscope}%
\begin{pgfscope}%
\pgfsetbuttcap%
\pgfsetroundjoin%
\definecolor{currentfill}{rgb}{0.000000,0.000000,0.000000}%
\pgfsetfillcolor{currentfill}%
\pgfsetlinewidth{0.602250pt}%
\definecolor{currentstroke}{rgb}{0.000000,0.000000,0.000000}%
\pgfsetstrokecolor{currentstroke}%
\pgfsetdash{}{0pt}%
\pgfsys@defobject{currentmarker}{\pgfqpoint{-0.027778in}{0.000000in}}{\pgfqpoint{-0.000000in}{0.000000in}}{%
\pgfpathmoveto{\pgfqpoint{-0.000000in}{0.000000in}}%
\pgfpathlineto{\pgfqpoint{-0.027778in}{0.000000in}}%
\pgfusepath{stroke,fill}%
}%
\begin{pgfscope}%
\pgfsys@transformshift{0.708220in}{1.442602in}%
\pgfsys@useobject{currentmarker}{}%
\end{pgfscope}%
\end{pgfscope}%
\begin{pgfscope}%
\pgfsetbuttcap%
\pgfsetroundjoin%
\definecolor{currentfill}{rgb}{0.000000,0.000000,0.000000}%
\pgfsetfillcolor{currentfill}%
\pgfsetlinewidth{0.602250pt}%
\definecolor{currentstroke}{rgb}{0.000000,0.000000,0.000000}%
\pgfsetstrokecolor{currentstroke}%
\pgfsetdash{}{0pt}%
\pgfsys@defobject{currentmarker}{\pgfqpoint{-0.027778in}{0.000000in}}{\pgfqpoint{-0.000000in}{0.000000in}}{%
\pgfpathmoveto{\pgfqpoint{-0.000000in}{0.000000in}}%
\pgfpathlineto{\pgfqpoint{-0.027778in}{0.000000in}}%
\pgfusepath{stroke,fill}%
}%
\begin{pgfscope}%
\pgfsys@transformshift{0.708220in}{1.463646in}%
\pgfsys@useobject{currentmarker}{}%
\end{pgfscope}%
\end{pgfscope}%
\begin{pgfscope}%
\pgfsetbuttcap%
\pgfsetroundjoin%
\definecolor{currentfill}{rgb}{0.000000,0.000000,0.000000}%
\pgfsetfillcolor{currentfill}%
\pgfsetlinewidth{0.602250pt}%
\definecolor{currentstroke}{rgb}{0.000000,0.000000,0.000000}%
\pgfsetstrokecolor{currentstroke}%
\pgfsetdash{}{0pt}%
\pgfsys@defobject{currentmarker}{\pgfqpoint{-0.027778in}{0.000000in}}{\pgfqpoint{-0.000000in}{0.000000in}}{%
\pgfpathmoveto{\pgfqpoint{-0.000000in}{0.000000in}}%
\pgfpathlineto{\pgfqpoint{-0.027778in}{0.000000in}}%
\pgfusepath{stroke,fill}%
}%
\begin{pgfscope}%
\pgfsys@transformshift{0.708220in}{1.606315in}%
\pgfsys@useobject{currentmarker}{}%
\end{pgfscope}%
\end{pgfscope}%
\begin{pgfscope}%
\pgfsetbuttcap%
\pgfsetroundjoin%
\definecolor{currentfill}{rgb}{0.000000,0.000000,0.000000}%
\pgfsetfillcolor{currentfill}%
\pgfsetlinewidth{0.602250pt}%
\definecolor{currentstroke}{rgb}{0.000000,0.000000,0.000000}%
\pgfsetstrokecolor{currentstroke}%
\pgfsetdash{}{0pt}%
\pgfsys@defobject{currentmarker}{\pgfqpoint{-0.027778in}{0.000000in}}{\pgfqpoint{-0.000000in}{0.000000in}}{%
\pgfpathmoveto{\pgfqpoint{-0.000000in}{0.000000in}}%
\pgfpathlineto{\pgfqpoint{-0.027778in}{0.000000in}}%
\pgfusepath{stroke,fill}%
}%
\begin{pgfscope}%
\pgfsys@transformshift{0.708220in}{1.678760in}%
\pgfsys@useobject{currentmarker}{}%
\end{pgfscope}%
\end{pgfscope}%
\begin{pgfscope}%
\pgfsetbuttcap%
\pgfsetroundjoin%
\definecolor{currentfill}{rgb}{0.000000,0.000000,0.000000}%
\pgfsetfillcolor{currentfill}%
\pgfsetlinewidth{0.602250pt}%
\definecolor{currentstroke}{rgb}{0.000000,0.000000,0.000000}%
\pgfsetstrokecolor{currentstroke}%
\pgfsetdash{}{0pt}%
\pgfsys@defobject{currentmarker}{\pgfqpoint{-0.027778in}{0.000000in}}{\pgfqpoint{-0.000000in}{0.000000in}}{%
\pgfpathmoveto{\pgfqpoint{-0.000000in}{0.000000in}}%
\pgfpathlineto{\pgfqpoint{-0.027778in}{0.000000in}}%
\pgfusepath{stroke,fill}%
}%
\begin{pgfscope}%
\pgfsys@transformshift{0.708220in}{1.730160in}%
\pgfsys@useobject{currentmarker}{}%
\end{pgfscope}%
\end{pgfscope}%
\begin{pgfscope}%
\pgfsetbuttcap%
\pgfsetroundjoin%
\definecolor{currentfill}{rgb}{0.000000,0.000000,0.000000}%
\pgfsetfillcolor{currentfill}%
\pgfsetlinewidth{0.602250pt}%
\definecolor{currentstroke}{rgb}{0.000000,0.000000,0.000000}%
\pgfsetstrokecolor{currentstroke}%
\pgfsetdash{}{0pt}%
\pgfsys@defobject{currentmarker}{\pgfqpoint{-0.027778in}{0.000000in}}{\pgfqpoint{-0.000000in}{0.000000in}}{%
\pgfpathmoveto{\pgfqpoint{-0.000000in}{0.000000in}}%
\pgfpathlineto{\pgfqpoint{-0.027778in}{0.000000in}}%
\pgfusepath{stroke,fill}%
}%
\begin{pgfscope}%
\pgfsys@transformshift{0.708220in}{1.770029in}%
\pgfsys@useobject{currentmarker}{}%
\end{pgfscope}%
\end{pgfscope}%
\begin{pgfscope}%
\pgfsetbuttcap%
\pgfsetroundjoin%
\definecolor{currentfill}{rgb}{0.000000,0.000000,0.000000}%
\pgfsetfillcolor{currentfill}%
\pgfsetlinewidth{0.602250pt}%
\definecolor{currentstroke}{rgb}{0.000000,0.000000,0.000000}%
\pgfsetstrokecolor{currentstroke}%
\pgfsetdash{}{0pt}%
\pgfsys@defobject{currentmarker}{\pgfqpoint{-0.027778in}{0.000000in}}{\pgfqpoint{-0.000000in}{0.000000in}}{%
\pgfpathmoveto{\pgfqpoint{-0.000000in}{0.000000in}}%
\pgfpathlineto{\pgfqpoint{-0.027778in}{0.000000in}}%
\pgfusepath{stroke,fill}%
}%
\begin{pgfscope}%
\pgfsys@transformshift{0.708220in}{1.802604in}%
\pgfsys@useobject{currentmarker}{}%
\end{pgfscope}%
\end{pgfscope}%
\begin{pgfscope}%
\pgfsetbuttcap%
\pgfsetroundjoin%
\definecolor{currentfill}{rgb}{0.000000,0.000000,0.000000}%
\pgfsetfillcolor{currentfill}%
\pgfsetlinewidth{0.602250pt}%
\definecolor{currentstroke}{rgb}{0.000000,0.000000,0.000000}%
\pgfsetstrokecolor{currentstroke}%
\pgfsetdash{}{0pt}%
\pgfsys@defobject{currentmarker}{\pgfqpoint{-0.027778in}{0.000000in}}{\pgfqpoint{-0.000000in}{0.000000in}}{%
\pgfpathmoveto{\pgfqpoint{-0.000000in}{0.000000in}}%
\pgfpathlineto{\pgfqpoint{-0.027778in}{0.000000in}}%
\pgfusepath{stroke,fill}%
}%
\begin{pgfscope}%
\pgfsys@transformshift{0.708220in}{1.830146in}%
\pgfsys@useobject{currentmarker}{}%
\end{pgfscope}%
\end{pgfscope}%
\begin{pgfscope}%
\pgfsetbuttcap%
\pgfsetroundjoin%
\definecolor{currentfill}{rgb}{0.000000,0.000000,0.000000}%
\pgfsetfillcolor{currentfill}%
\pgfsetlinewidth{0.602250pt}%
\definecolor{currentstroke}{rgb}{0.000000,0.000000,0.000000}%
\pgfsetstrokecolor{currentstroke}%
\pgfsetdash{}{0pt}%
\pgfsys@defobject{currentmarker}{\pgfqpoint{-0.027778in}{0.000000in}}{\pgfqpoint{-0.000000in}{0.000000in}}{%
\pgfpathmoveto{\pgfqpoint{-0.000000in}{0.000000in}}%
\pgfpathlineto{\pgfqpoint{-0.027778in}{0.000000in}}%
\pgfusepath{stroke,fill}%
}%
\begin{pgfscope}%
\pgfsys@transformshift{0.708220in}{1.854004in}%
\pgfsys@useobject{currentmarker}{}%
\end{pgfscope}%
\end{pgfscope}%
\begin{pgfscope}%
\pgfsetbuttcap%
\pgfsetroundjoin%
\definecolor{currentfill}{rgb}{0.000000,0.000000,0.000000}%
\pgfsetfillcolor{currentfill}%
\pgfsetlinewidth{0.602250pt}%
\definecolor{currentstroke}{rgb}{0.000000,0.000000,0.000000}%
\pgfsetstrokecolor{currentstroke}%
\pgfsetdash{}{0pt}%
\pgfsys@defobject{currentmarker}{\pgfqpoint{-0.027778in}{0.000000in}}{\pgfqpoint{-0.000000in}{0.000000in}}{%
\pgfpathmoveto{\pgfqpoint{-0.000000in}{0.000000in}}%
\pgfpathlineto{\pgfqpoint{-0.027778in}{0.000000in}}%
\pgfusepath{stroke,fill}%
}%
\begin{pgfscope}%
\pgfsys@transformshift{0.708220in}{1.875048in}%
\pgfsys@useobject{currentmarker}{}%
\end{pgfscope}%
\end{pgfscope}%
\begin{pgfscope}%
\pgfsetbuttcap%
\pgfsetroundjoin%
\definecolor{currentfill}{rgb}{0.000000,0.000000,0.000000}%
\pgfsetfillcolor{currentfill}%
\pgfsetlinewidth{0.602250pt}%
\definecolor{currentstroke}{rgb}{0.000000,0.000000,0.000000}%
\pgfsetstrokecolor{currentstroke}%
\pgfsetdash{}{0pt}%
\pgfsys@defobject{currentmarker}{\pgfqpoint{-0.027778in}{0.000000in}}{\pgfqpoint{-0.000000in}{0.000000in}}{%
\pgfpathmoveto{\pgfqpoint{-0.000000in}{0.000000in}}%
\pgfpathlineto{\pgfqpoint{-0.027778in}{0.000000in}}%
\pgfusepath{stroke,fill}%
}%
\begin{pgfscope}%
\pgfsys@transformshift{0.708220in}{2.017718in}%
\pgfsys@useobject{currentmarker}{}%
\end{pgfscope}%
\end{pgfscope}%
\begin{pgfscope}%
\pgfsetbuttcap%
\pgfsetroundjoin%
\definecolor{currentfill}{rgb}{0.000000,0.000000,0.000000}%
\pgfsetfillcolor{currentfill}%
\pgfsetlinewidth{0.602250pt}%
\definecolor{currentstroke}{rgb}{0.000000,0.000000,0.000000}%
\pgfsetstrokecolor{currentstroke}%
\pgfsetdash{}{0pt}%
\pgfsys@defobject{currentmarker}{\pgfqpoint{-0.027778in}{0.000000in}}{\pgfqpoint{-0.000000in}{0.000000in}}{%
\pgfpathmoveto{\pgfqpoint{-0.000000in}{0.000000in}}%
\pgfpathlineto{\pgfqpoint{-0.027778in}{0.000000in}}%
\pgfusepath{stroke,fill}%
}%
\begin{pgfscope}%
\pgfsys@transformshift{0.708220in}{2.090162in}%
\pgfsys@useobject{currentmarker}{}%
\end{pgfscope}%
\end{pgfscope}%
\begin{pgfscope}%
\pgfsetbuttcap%
\pgfsetroundjoin%
\definecolor{currentfill}{rgb}{0.000000,0.000000,0.000000}%
\pgfsetfillcolor{currentfill}%
\pgfsetlinewidth{0.602250pt}%
\definecolor{currentstroke}{rgb}{0.000000,0.000000,0.000000}%
\pgfsetstrokecolor{currentstroke}%
\pgfsetdash{}{0pt}%
\pgfsys@defobject{currentmarker}{\pgfqpoint{-0.027778in}{0.000000in}}{\pgfqpoint{-0.000000in}{0.000000in}}{%
\pgfpathmoveto{\pgfqpoint{-0.000000in}{0.000000in}}%
\pgfpathlineto{\pgfqpoint{-0.027778in}{0.000000in}}%
\pgfusepath{stroke,fill}%
}%
\begin{pgfscope}%
\pgfsys@transformshift{0.708220in}{2.141562in}%
\pgfsys@useobject{currentmarker}{}%
\end{pgfscope}%
\end{pgfscope}%
\begin{pgfscope}%
\pgfsetbuttcap%
\pgfsetroundjoin%
\definecolor{currentfill}{rgb}{0.000000,0.000000,0.000000}%
\pgfsetfillcolor{currentfill}%
\pgfsetlinewidth{0.602250pt}%
\definecolor{currentstroke}{rgb}{0.000000,0.000000,0.000000}%
\pgfsetstrokecolor{currentstroke}%
\pgfsetdash{}{0pt}%
\pgfsys@defobject{currentmarker}{\pgfqpoint{-0.027778in}{0.000000in}}{\pgfqpoint{-0.000000in}{0.000000in}}{%
\pgfpathmoveto{\pgfqpoint{-0.000000in}{0.000000in}}%
\pgfpathlineto{\pgfqpoint{-0.027778in}{0.000000in}}%
\pgfusepath{stroke,fill}%
}%
\begin{pgfscope}%
\pgfsys@transformshift{0.708220in}{2.181431in}%
\pgfsys@useobject{currentmarker}{}%
\end{pgfscope}%
\end{pgfscope}%
\begin{pgfscope}%
\pgfsetbuttcap%
\pgfsetroundjoin%
\definecolor{currentfill}{rgb}{0.000000,0.000000,0.000000}%
\pgfsetfillcolor{currentfill}%
\pgfsetlinewidth{0.602250pt}%
\definecolor{currentstroke}{rgb}{0.000000,0.000000,0.000000}%
\pgfsetstrokecolor{currentstroke}%
\pgfsetdash{}{0pt}%
\pgfsys@defobject{currentmarker}{\pgfqpoint{-0.027778in}{0.000000in}}{\pgfqpoint{-0.000000in}{0.000000in}}{%
\pgfpathmoveto{\pgfqpoint{-0.000000in}{0.000000in}}%
\pgfpathlineto{\pgfqpoint{-0.027778in}{0.000000in}}%
\pgfusepath{stroke,fill}%
}%
\begin{pgfscope}%
\pgfsys@transformshift{0.708220in}{2.214006in}%
\pgfsys@useobject{currentmarker}{}%
\end{pgfscope}%
\end{pgfscope}%
\begin{pgfscope}%
\pgfsetbuttcap%
\pgfsetroundjoin%
\definecolor{currentfill}{rgb}{0.000000,0.000000,0.000000}%
\pgfsetfillcolor{currentfill}%
\pgfsetlinewidth{0.602250pt}%
\definecolor{currentstroke}{rgb}{0.000000,0.000000,0.000000}%
\pgfsetstrokecolor{currentstroke}%
\pgfsetdash{}{0pt}%
\pgfsys@defobject{currentmarker}{\pgfqpoint{-0.027778in}{0.000000in}}{\pgfqpoint{-0.000000in}{0.000000in}}{%
\pgfpathmoveto{\pgfqpoint{-0.000000in}{0.000000in}}%
\pgfpathlineto{\pgfqpoint{-0.027778in}{0.000000in}}%
\pgfusepath{stroke,fill}%
}%
\begin{pgfscope}%
\pgfsys@transformshift{0.708220in}{2.241548in}%
\pgfsys@useobject{currentmarker}{}%
\end{pgfscope}%
\end{pgfscope}%
\begin{pgfscope}%
\pgfsetbuttcap%
\pgfsetroundjoin%
\definecolor{currentfill}{rgb}{0.000000,0.000000,0.000000}%
\pgfsetfillcolor{currentfill}%
\pgfsetlinewidth{0.602250pt}%
\definecolor{currentstroke}{rgb}{0.000000,0.000000,0.000000}%
\pgfsetstrokecolor{currentstroke}%
\pgfsetdash{}{0pt}%
\pgfsys@defobject{currentmarker}{\pgfqpoint{-0.027778in}{0.000000in}}{\pgfqpoint{-0.000000in}{0.000000in}}{%
\pgfpathmoveto{\pgfqpoint{-0.000000in}{0.000000in}}%
\pgfpathlineto{\pgfqpoint{-0.027778in}{0.000000in}}%
\pgfusepath{stroke,fill}%
}%
\begin{pgfscope}%
\pgfsys@transformshift{0.708220in}{2.265406in}%
\pgfsys@useobject{currentmarker}{}%
\end{pgfscope}%
\end{pgfscope}%
\begin{pgfscope}%
\pgfsetbuttcap%
\pgfsetroundjoin%
\definecolor{currentfill}{rgb}{0.000000,0.000000,0.000000}%
\pgfsetfillcolor{currentfill}%
\pgfsetlinewidth{0.602250pt}%
\definecolor{currentstroke}{rgb}{0.000000,0.000000,0.000000}%
\pgfsetstrokecolor{currentstroke}%
\pgfsetdash{}{0pt}%
\pgfsys@defobject{currentmarker}{\pgfqpoint{-0.027778in}{0.000000in}}{\pgfqpoint{-0.000000in}{0.000000in}}{%
\pgfpathmoveto{\pgfqpoint{-0.000000in}{0.000000in}}%
\pgfpathlineto{\pgfqpoint{-0.027778in}{0.000000in}}%
\pgfusepath{stroke,fill}%
}%
\begin{pgfscope}%
\pgfsys@transformshift{0.708220in}{2.286450in}%
\pgfsys@useobject{currentmarker}{}%
\end{pgfscope}%
\end{pgfscope}%
\begin{pgfscope}%
\definecolor{textcolor}{rgb}{0.000000,0.000000,0.000000}%
\pgfsetstrokecolor{textcolor}%
\pgfsetfillcolor{textcolor}%
\pgftext[x=0.288855in,y=1.420549in,,bottom,rotate=90.000000]{\color{textcolor}\rmfamily\fontsize{10.000000}{12.000000}\selectfont Longest solving time (s)}%
\end{pgfscope}%
\begin{pgfscope}%
\pgfpathrectangle{\pgfqpoint{0.708220in}{0.535823in}}{\pgfqpoint{5.013309in}{1.769453in}}%
\pgfusepath{clip}%
\pgfsetrectcap%
\pgfsetroundjoin%
\pgfsetlinewidth{1.003750pt}%
\definecolor{currentstroke}{rgb}{0.000000,0.000000,1.000000}%
\pgfsetstrokecolor{currentstroke}%
\pgfsetdash{}{0pt}%
\pgfpathmoveto{\pgfqpoint{0.708220in}{0.726221in}}%
\pgfpathlineto{\pgfqpoint{0.710727in}{0.742915in}}%
\pgfpathlineto{\pgfqpoint{0.715740in}{0.755474in}}%
\pgfpathlineto{\pgfqpoint{0.718246in}{0.786804in}}%
\pgfpathlineto{\pgfqpoint{0.720753in}{0.792526in}}%
\pgfpathlineto{\pgfqpoint{0.723260in}{0.819427in}}%
\pgfpathlineto{\pgfqpoint{0.725766in}{0.819906in}}%
\pgfpathlineto{\pgfqpoint{0.728273in}{0.827456in}}%
\pgfpathlineto{\pgfqpoint{0.730780in}{0.828939in}}%
\pgfpathlineto{\pgfqpoint{0.738300in}{0.858407in}}%
\pgfpathlineto{\pgfqpoint{0.740806in}{0.861493in}}%
\pgfpathlineto{\pgfqpoint{0.743313in}{0.862721in}}%
\pgfpathlineto{\pgfqpoint{0.745820in}{0.868685in}}%
\pgfpathlineto{\pgfqpoint{0.755846in}{0.872674in}}%
\pgfpathlineto{\pgfqpoint{0.758353in}{0.876274in}}%
\pgfpathlineto{\pgfqpoint{0.760860in}{0.877160in}}%
\pgfpathlineto{\pgfqpoint{0.763366in}{0.879541in}}%
\pgfpathlineto{\pgfqpoint{0.765873in}{0.890272in}}%
\pgfpathlineto{\pgfqpoint{0.773393in}{0.894401in}}%
\pgfpathlineto{\pgfqpoint{0.775900in}{0.896340in}}%
\pgfpathlineto{\pgfqpoint{0.778406in}{0.904942in}}%
\pgfpathlineto{\pgfqpoint{0.780913in}{0.905222in}}%
\pgfpathlineto{\pgfqpoint{0.783419in}{0.913961in}}%
\pgfpathlineto{\pgfqpoint{0.788433in}{0.915647in}}%
\pgfpathlineto{\pgfqpoint{0.793446in}{0.925391in}}%
\pgfpathlineto{\pgfqpoint{0.795953in}{0.927659in}}%
\pgfpathlineto{\pgfqpoint{0.798459in}{0.936507in}}%
\pgfpathlineto{\pgfqpoint{0.803473in}{0.939881in}}%
\pgfpathlineto{\pgfqpoint{0.805979in}{0.946513in}}%
\pgfpathlineto{\pgfqpoint{0.821019in}{0.958043in}}%
\pgfpathlineto{\pgfqpoint{0.823526in}{0.962119in}}%
\pgfpathlineto{\pgfqpoint{0.826033in}{0.962951in}}%
\pgfpathlineto{\pgfqpoint{0.828539in}{0.968860in}}%
\pgfpathlineto{\pgfqpoint{0.831046in}{0.968975in}}%
\pgfpathlineto{\pgfqpoint{0.836059in}{0.980264in}}%
\pgfpathlineto{\pgfqpoint{0.841073in}{0.981798in}}%
\pgfpathlineto{\pgfqpoint{0.846086in}{0.994781in}}%
\pgfpathlineto{\pgfqpoint{0.856112in}{0.998734in}}%
\pgfpathlineto{\pgfqpoint{0.861126in}{1.017070in}}%
\pgfpathlineto{\pgfqpoint{0.866139in}{1.017164in}}%
\pgfpathlineto{\pgfqpoint{0.873659in}{1.024123in}}%
\pgfpathlineto{\pgfqpoint{0.878672in}{1.025013in}}%
\pgfpathlineto{\pgfqpoint{0.881179in}{1.025707in}}%
\pgfpathlineto{\pgfqpoint{0.886192in}{1.032331in}}%
\pgfpathlineto{\pgfqpoint{0.891206in}{1.033839in}}%
\pgfpathlineto{\pgfqpoint{0.898726in}{1.039200in}}%
\pgfpathlineto{\pgfqpoint{0.901232in}{1.039720in}}%
\pgfpathlineto{\pgfqpoint{0.906246in}{1.049565in}}%
\pgfpathlineto{\pgfqpoint{0.911259in}{1.053770in}}%
\pgfpathlineto{\pgfqpoint{0.913766in}{1.061056in}}%
\pgfpathlineto{\pgfqpoint{0.918779in}{1.061861in}}%
\pgfpathlineto{\pgfqpoint{0.926299in}{1.064413in}}%
\pgfpathlineto{\pgfqpoint{0.928805in}{1.064425in}}%
\pgfpathlineto{\pgfqpoint{0.931312in}{1.069690in}}%
\pgfpathlineto{\pgfqpoint{0.933819in}{1.070007in}}%
\pgfpathlineto{\pgfqpoint{0.941339in}{1.076758in}}%
\pgfpathlineto{\pgfqpoint{0.943845in}{1.076877in}}%
\pgfpathlineto{\pgfqpoint{0.948859in}{1.078616in}}%
\pgfpathlineto{\pgfqpoint{0.953872in}{1.088230in}}%
\pgfpathlineto{\pgfqpoint{0.956379in}{1.088900in}}%
\pgfpathlineto{\pgfqpoint{0.963899in}{1.103397in}}%
\pgfpathlineto{\pgfqpoint{0.966405in}{1.114994in}}%
\pgfpathlineto{\pgfqpoint{0.971419in}{1.115663in}}%
\pgfpathlineto{\pgfqpoint{0.976432in}{1.118592in}}%
\pgfpathlineto{\pgfqpoint{0.986458in}{1.122553in}}%
\pgfpathlineto{\pgfqpoint{0.988965in}{1.125989in}}%
\pgfpathlineto{\pgfqpoint{0.991472in}{1.126042in}}%
\pgfpathlineto{\pgfqpoint{0.993978in}{1.133787in}}%
\pgfpathlineto{\pgfqpoint{1.001498in}{1.138204in}}%
\pgfpathlineto{\pgfqpoint{1.004005in}{1.143847in}}%
\pgfpathlineto{\pgfqpoint{1.009018in}{1.148391in}}%
\pgfpathlineto{\pgfqpoint{1.011525in}{1.148625in}}%
\pgfpathlineto{\pgfqpoint{1.014032in}{1.158270in}}%
\pgfpathlineto{\pgfqpoint{1.021552in}{1.159843in}}%
\pgfpathlineto{\pgfqpoint{1.026565in}{1.162576in}}%
\pgfpathlineto{\pgfqpoint{1.029072in}{1.162773in}}%
\pgfpathlineto{\pgfqpoint{1.036592in}{1.167128in}}%
\pgfpathlineto{\pgfqpoint{1.041605in}{1.167657in}}%
\pgfpathlineto{\pgfqpoint{1.049125in}{1.173741in}}%
\pgfpathlineto{\pgfqpoint{1.051632in}{1.174061in}}%
\pgfpathlineto{\pgfqpoint{1.064165in}{1.182833in}}%
\pgfpathlineto{\pgfqpoint{1.066671in}{1.183363in}}%
\pgfpathlineto{\pgfqpoint{1.074191in}{1.188597in}}%
\pgfpathlineto{\pgfqpoint{1.079205in}{1.191484in}}%
\pgfpathlineto{\pgfqpoint{1.081711in}{1.194148in}}%
\pgfpathlineto{\pgfqpoint{1.086725in}{1.195226in}}%
\pgfpathlineto{\pgfqpoint{1.091738in}{1.197483in}}%
\pgfpathlineto{\pgfqpoint{1.106778in}{1.201469in}}%
\pgfpathlineto{\pgfqpoint{1.111791in}{1.203527in}}%
\pgfpathlineto{\pgfqpoint{1.114298in}{1.207950in}}%
\pgfpathlineto{\pgfqpoint{1.119311in}{1.210396in}}%
\pgfpathlineto{\pgfqpoint{1.124324in}{1.212034in}}%
\pgfpathlineto{\pgfqpoint{1.136858in}{1.225497in}}%
\pgfpathlineto{\pgfqpoint{1.139364in}{1.231466in}}%
\pgfpathlineto{\pgfqpoint{1.141871in}{1.232761in}}%
\pgfpathlineto{\pgfqpoint{1.144378in}{1.236787in}}%
\pgfpathlineto{\pgfqpoint{1.156911in}{1.243735in}}%
\pgfpathlineto{\pgfqpoint{1.159418in}{1.243857in}}%
\pgfpathlineto{\pgfqpoint{1.161924in}{1.248516in}}%
\pgfpathlineto{\pgfqpoint{1.166938in}{1.249079in}}%
\pgfpathlineto{\pgfqpoint{1.179471in}{1.254675in}}%
\pgfpathlineto{\pgfqpoint{1.181978in}{1.255397in}}%
\pgfpathlineto{\pgfqpoint{1.184484in}{1.259747in}}%
\pgfpathlineto{\pgfqpoint{1.186991in}{1.260194in}}%
\pgfpathlineto{\pgfqpoint{1.189498in}{1.268521in}}%
\pgfpathlineto{\pgfqpoint{1.194511in}{1.271054in}}%
\pgfpathlineto{\pgfqpoint{1.199524in}{1.272320in}}%
\pgfpathlineto{\pgfqpoint{1.204537in}{1.275996in}}%
\pgfpathlineto{\pgfqpoint{1.212057in}{1.279913in}}%
\pgfpathlineto{\pgfqpoint{1.214564in}{1.282391in}}%
\pgfpathlineto{\pgfqpoint{1.222084in}{1.294994in}}%
\pgfpathlineto{\pgfqpoint{1.224591in}{1.296664in}}%
\pgfpathlineto{\pgfqpoint{1.229604in}{1.307309in}}%
\pgfpathlineto{\pgfqpoint{1.237124in}{1.309811in}}%
\pgfpathlineto{\pgfqpoint{1.239631in}{1.317088in}}%
\pgfpathlineto{\pgfqpoint{1.247151in}{1.323369in}}%
\pgfpathlineto{\pgfqpoint{1.249657in}{1.326865in}}%
\pgfpathlineto{\pgfqpoint{1.252164in}{1.327204in}}%
\pgfpathlineto{\pgfqpoint{1.254671in}{1.330877in}}%
\pgfpathlineto{\pgfqpoint{1.257177in}{1.332311in}}%
\pgfpathlineto{\pgfqpoint{1.259684in}{1.340555in}}%
\pgfpathlineto{\pgfqpoint{1.262190in}{1.342067in}}%
\pgfpathlineto{\pgfqpoint{1.264697in}{1.342148in}}%
\pgfpathlineto{\pgfqpoint{1.267204in}{1.347855in}}%
\pgfpathlineto{\pgfqpoint{1.277230in}{1.356859in}}%
\pgfpathlineto{\pgfqpoint{1.279737in}{1.360735in}}%
\pgfpathlineto{\pgfqpoint{1.287257in}{1.362291in}}%
\pgfpathlineto{\pgfqpoint{1.292270in}{1.368190in}}%
\pgfpathlineto{\pgfqpoint{1.294777in}{1.369146in}}%
\pgfpathlineto{\pgfqpoint{1.299790in}{1.379586in}}%
\pgfpathlineto{\pgfqpoint{1.302297in}{1.379697in}}%
\pgfpathlineto{\pgfqpoint{1.309817in}{1.384561in}}%
\pgfpathlineto{\pgfqpoint{1.314830in}{1.390231in}}%
\pgfpathlineto{\pgfqpoint{1.317337in}{1.396125in}}%
\pgfpathlineto{\pgfqpoint{1.322350in}{1.398827in}}%
\pgfpathlineto{\pgfqpoint{1.327363in}{1.400201in}}%
\pgfpathlineto{\pgfqpoint{1.332377in}{1.402497in}}%
\pgfpathlineto{\pgfqpoint{1.337390in}{1.409348in}}%
\pgfpathlineto{\pgfqpoint{1.339897in}{1.410724in}}%
\pgfpathlineto{\pgfqpoint{1.342403in}{1.413910in}}%
\pgfpathlineto{\pgfqpoint{1.349923in}{1.414764in}}%
\pgfpathlineto{\pgfqpoint{1.359950in}{1.420044in}}%
\pgfpathlineto{\pgfqpoint{1.362457in}{1.433925in}}%
\pgfpathlineto{\pgfqpoint{1.364963in}{1.436193in}}%
\pgfpathlineto{\pgfqpoint{1.367470in}{1.443883in}}%
\pgfpathlineto{\pgfqpoint{1.369977in}{1.446244in}}%
\pgfpathlineto{\pgfqpoint{1.372483in}{1.446690in}}%
\pgfpathlineto{\pgfqpoint{1.380003in}{1.462397in}}%
\pgfpathlineto{\pgfqpoint{1.382510in}{1.463200in}}%
\pgfpathlineto{\pgfqpoint{1.387523in}{1.468714in}}%
\pgfpathlineto{\pgfqpoint{1.395043in}{1.487137in}}%
\pgfpathlineto{\pgfqpoint{1.397550in}{1.487165in}}%
\pgfpathlineto{\pgfqpoint{1.400056in}{1.489278in}}%
\pgfpathlineto{\pgfqpoint{1.405070in}{1.490113in}}%
\pgfpathlineto{\pgfqpoint{1.412590in}{1.498568in}}%
\pgfpathlineto{\pgfqpoint{1.415096in}{1.511098in}}%
\pgfpathlineto{\pgfqpoint{1.417603in}{1.512319in}}%
\pgfpathlineto{\pgfqpoint{1.420110in}{1.522039in}}%
\pgfpathlineto{\pgfqpoint{1.425123in}{1.531117in}}%
\pgfpathlineto{\pgfqpoint{1.427630in}{1.532084in}}%
\pgfpathlineto{\pgfqpoint{1.435150in}{1.546312in}}%
\pgfpathlineto{\pgfqpoint{1.437656in}{1.549440in}}%
\pgfpathlineto{\pgfqpoint{1.440163in}{1.557901in}}%
\pgfpathlineto{\pgfqpoint{1.442670in}{1.559280in}}%
\pgfpathlineto{\pgfqpoint{1.445176in}{1.562880in}}%
\pgfpathlineto{\pgfqpoint{1.450190in}{1.564109in}}%
\pgfpathlineto{\pgfqpoint{1.452696in}{1.568290in}}%
\pgfpathlineto{\pgfqpoint{1.455203in}{1.568886in}}%
\pgfpathlineto{\pgfqpoint{1.460216in}{1.574664in}}%
\pgfpathlineto{\pgfqpoint{1.462723in}{1.575270in}}%
\pgfpathlineto{\pgfqpoint{1.470243in}{1.584434in}}%
\pgfpathlineto{\pgfqpoint{1.472749in}{1.585551in}}%
\pgfpathlineto{\pgfqpoint{1.477763in}{1.590785in}}%
\pgfpathlineto{\pgfqpoint{1.480269in}{1.596329in}}%
\pgfpathlineto{\pgfqpoint{1.482776in}{1.597691in}}%
\pgfpathlineto{\pgfqpoint{1.485283in}{1.600336in}}%
\pgfpathlineto{\pgfqpoint{1.487789in}{1.606812in}}%
\pgfpathlineto{\pgfqpoint{1.490296in}{1.607061in}}%
\pgfpathlineto{\pgfqpoint{1.495309in}{1.610719in}}%
\pgfpathlineto{\pgfqpoint{1.505336in}{1.619278in}}%
\pgfpathlineto{\pgfqpoint{1.507843in}{1.619344in}}%
\pgfpathlineto{\pgfqpoint{1.510349in}{1.622423in}}%
\pgfpathlineto{\pgfqpoint{1.512856in}{1.623658in}}%
\pgfpathlineto{\pgfqpoint{1.520376in}{1.646680in}}%
\pgfpathlineto{\pgfqpoint{1.525389in}{1.649956in}}%
\pgfpathlineto{\pgfqpoint{1.527896in}{1.655487in}}%
\pgfpathlineto{\pgfqpoint{1.530402in}{1.670733in}}%
\pgfpathlineto{\pgfqpoint{1.532909in}{1.672782in}}%
\pgfpathlineto{\pgfqpoint{1.535416in}{1.677972in}}%
\pgfpathlineto{\pgfqpoint{1.537922in}{1.679383in}}%
\pgfpathlineto{\pgfqpoint{1.540429in}{1.691150in}}%
\pgfpathlineto{\pgfqpoint{1.542936in}{1.693977in}}%
\pgfpathlineto{\pgfqpoint{1.547949in}{1.694751in}}%
\pgfpathlineto{\pgfqpoint{1.550456in}{1.696404in}}%
\pgfpathlineto{\pgfqpoint{1.552962in}{1.704652in}}%
\pgfpathlineto{\pgfqpoint{1.555469in}{1.705207in}}%
\pgfpathlineto{\pgfqpoint{1.557976in}{1.717169in}}%
\pgfpathlineto{\pgfqpoint{1.562989in}{1.720046in}}%
\pgfpathlineto{\pgfqpoint{1.568002in}{1.736233in}}%
\pgfpathlineto{\pgfqpoint{1.575522in}{1.739745in}}%
\pgfpathlineto{\pgfqpoint{1.578029in}{1.748218in}}%
\pgfpathlineto{\pgfqpoint{1.588056in}{1.758339in}}%
\pgfpathlineto{\pgfqpoint{1.590562in}{1.759523in}}%
\pgfpathlineto{\pgfqpoint{1.593069in}{1.762584in}}%
\pgfpathlineto{\pgfqpoint{1.595576in}{1.773352in}}%
\pgfpathlineto{\pgfqpoint{1.598082in}{1.774989in}}%
\pgfpathlineto{\pgfqpoint{1.600589in}{1.786907in}}%
\pgfpathlineto{\pgfqpoint{1.603095in}{1.793364in}}%
\pgfpathlineto{\pgfqpoint{1.608109in}{1.810075in}}%
\pgfpathlineto{\pgfqpoint{1.610615in}{1.810520in}}%
\pgfpathlineto{\pgfqpoint{1.613122in}{1.818667in}}%
\pgfpathlineto{\pgfqpoint{1.618135in}{1.822790in}}%
\pgfpathlineto{\pgfqpoint{1.623149in}{1.824207in}}%
\pgfpathlineto{\pgfqpoint{1.625655in}{1.839257in}}%
\pgfpathlineto{\pgfqpoint{1.628162in}{1.840394in}}%
\pgfpathlineto{\pgfqpoint{1.633175in}{1.845476in}}%
\pgfpathlineto{\pgfqpoint{1.638189in}{1.848359in}}%
\pgfpathlineto{\pgfqpoint{1.640695in}{1.856033in}}%
\pgfpathlineto{\pgfqpoint{1.643202in}{1.857552in}}%
\pgfpathlineto{\pgfqpoint{1.645709in}{1.868785in}}%
\pgfpathlineto{\pgfqpoint{1.648215in}{1.869193in}}%
\pgfpathlineto{\pgfqpoint{1.655735in}{1.876794in}}%
\pgfpathlineto{\pgfqpoint{1.658242in}{1.876979in}}%
\pgfpathlineto{\pgfqpoint{1.660749in}{1.887606in}}%
\pgfpathlineto{\pgfqpoint{1.663255in}{1.890237in}}%
\pgfpathlineto{\pgfqpoint{1.668268in}{1.908169in}}%
\pgfpathlineto{\pgfqpoint{1.675788in}{1.912667in}}%
\pgfpathlineto{\pgfqpoint{1.683308in}{1.943752in}}%
\pgfpathlineto{\pgfqpoint{1.685815in}{1.946038in}}%
\pgfpathlineto{\pgfqpoint{1.688322in}{1.961567in}}%
\pgfpathlineto{\pgfqpoint{1.695842in}{1.980273in}}%
\pgfpathlineto{\pgfqpoint{1.700855in}{1.983096in}}%
\pgfpathlineto{\pgfqpoint{1.713388in}{1.987084in}}%
\pgfpathlineto{\pgfqpoint{1.720908in}{1.992341in}}%
\pgfpathlineto{\pgfqpoint{1.723415in}{2.000954in}}%
\pgfpathlineto{\pgfqpoint{1.725922in}{2.001082in}}%
\pgfpathlineto{\pgfqpoint{1.728428in}{2.002445in}}%
\pgfpathlineto{\pgfqpoint{1.730935in}{2.006385in}}%
\pgfpathlineto{\pgfqpoint{1.738455in}{2.008678in}}%
\pgfpathlineto{\pgfqpoint{1.740961in}{2.009919in}}%
\pgfpathlineto{\pgfqpoint{1.745975in}{2.013937in}}%
\pgfpathlineto{\pgfqpoint{1.748481in}{2.014409in}}%
\pgfpathlineto{\pgfqpoint{1.753495in}{2.018616in}}%
\pgfpathlineto{\pgfqpoint{1.756001in}{2.025689in}}%
\pgfpathlineto{\pgfqpoint{1.761015in}{2.028397in}}%
\pgfpathlineto{\pgfqpoint{1.766028in}{2.036388in}}%
\pgfpathlineto{\pgfqpoint{1.768535in}{2.039336in}}%
\pgfpathlineto{\pgfqpoint{1.771041in}{2.047599in}}%
\pgfpathlineto{\pgfqpoint{1.776055in}{2.086075in}}%
\pgfpathlineto{\pgfqpoint{1.778561in}{2.090856in}}%
\pgfpathlineto{\pgfqpoint{1.781068in}{2.091598in}}%
\pgfpathlineto{\pgfqpoint{1.783575in}{2.093836in}}%
\pgfpathlineto{\pgfqpoint{1.786081in}{2.098487in}}%
\pgfpathlineto{\pgfqpoint{1.791095in}{2.111862in}}%
\pgfpathlineto{\pgfqpoint{1.793601in}{2.113436in}}%
\pgfpathlineto{\pgfqpoint{1.796108in}{2.117260in}}%
\pgfpathlineto{\pgfqpoint{1.798615in}{2.128638in}}%
\pgfpathlineto{\pgfqpoint{1.801121in}{2.129021in}}%
\pgfpathlineto{\pgfqpoint{1.803628in}{2.131090in}}%
\pgfpathlineto{\pgfqpoint{1.806134in}{2.138060in}}%
\pgfpathlineto{\pgfqpoint{1.808641in}{2.148600in}}%
\pgfpathlineto{\pgfqpoint{1.811148in}{2.151916in}}%
\pgfpathlineto{\pgfqpoint{1.816161in}{2.163361in}}%
\pgfpathlineto{\pgfqpoint{1.818668in}{2.167445in}}%
\pgfpathlineto{\pgfqpoint{1.821174in}{2.181416in}}%
\pgfpathlineto{\pgfqpoint{1.823681in}{2.182857in}}%
\pgfpathlineto{\pgfqpoint{1.826188in}{2.186594in}}%
\pgfpathlineto{\pgfqpoint{1.833708in}{2.189371in}}%
\pgfpathlineto{\pgfqpoint{1.841228in}{2.202307in}}%
\pgfpathlineto{\pgfqpoint{1.846241in}{2.217176in}}%
\pgfpathlineto{\pgfqpoint{1.856268in}{2.218711in}}%
\pgfpathlineto{\pgfqpoint{1.858774in}{2.218996in}}%
\pgfpathlineto{\pgfqpoint{1.861281in}{2.225442in}}%
\pgfpathlineto{\pgfqpoint{1.863788in}{2.226384in}}%
\pgfpathlineto{\pgfqpoint{1.871307in}{2.237413in}}%
\pgfpathlineto{\pgfqpoint{1.873814in}{2.237888in}}%
\pgfpathlineto{\pgfqpoint{1.876321in}{2.245028in}}%
\pgfpathlineto{\pgfqpoint{1.883841in}{2.246728in}}%
\pgfpathlineto{\pgfqpoint{1.886347in}{2.253066in}}%
\pgfpathlineto{\pgfqpoint{1.888854in}{2.255580in}}%
\pgfpathlineto{\pgfqpoint{1.891361in}{2.256288in}}%
\pgfpathlineto{\pgfqpoint{1.893867in}{2.259014in}}%
\pgfpathlineto{\pgfqpoint{1.896374in}{2.266646in}}%
\pgfpathlineto{\pgfqpoint{1.898881in}{2.266685in}}%
\pgfpathlineto{\pgfqpoint{1.901387in}{2.271904in}}%
\pgfpathlineto{\pgfqpoint{1.906401in}{2.275193in}}%
\pgfpathlineto{\pgfqpoint{1.908907in}{2.291911in}}%
\pgfpathlineto{\pgfqpoint{1.911414in}{2.295595in}}%
\pgfpathlineto{\pgfqpoint{1.913921in}{2.302175in}}%
\pgfpathlineto{\pgfqpoint{1.916427in}{2.304511in}}%
\pgfpathlineto{\pgfqpoint{1.928961in}{2.305275in}}%
\pgfpathlineto{\pgfqpoint{5.400677in}{2.306381in}}%
\pgfpathlineto{\pgfqpoint{5.495930in}{2.307743in}}%
\pgfpathlineto{\pgfqpoint{5.571129in}{2.310756in}}%
\pgfpathlineto{\pgfqpoint{5.583663in}{2.313640in}}%
\pgfpathlineto{\pgfqpoint{5.588676in}{2.314242in}}%
\pgfpathlineto{\pgfqpoint{5.593313in}{2.315275in}}%
\pgfpathlineto{\pgfqpoint{5.593313in}{2.315275in}}%
\pgfusepath{stroke}%
\end{pgfscope}%
\begin{pgfscope}%
\pgfpathrectangle{\pgfqpoint{0.708220in}{0.535823in}}{\pgfqpoint{5.013309in}{1.769453in}}%
\pgfusepath{clip}%
\pgfsetbuttcap%
\pgfsetroundjoin%
\pgfsetlinewidth{1.003750pt}%
\definecolor{currentstroke}{rgb}{0.000000,0.501961,0.000000}%
\pgfsetstrokecolor{currentstroke}%
\pgfsetdash{{3.700000pt}{1.600000pt}}{0.000000pt}%
\pgfpathmoveto{\pgfqpoint{0.708220in}{0.633641in}}%
\pgfpathlineto{\pgfqpoint{0.713233in}{0.663878in}}%
\pgfpathlineto{\pgfqpoint{0.715740in}{0.710153in}}%
\pgfpathlineto{\pgfqpoint{0.718246in}{0.717904in}}%
\pgfpathlineto{\pgfqpoint{0.720753in}{0.729608in}}%
\pgfpathlineto{\pgfqpoint{0.723260in}{0.759376in}}%
\pgfpathlineto{\pgfqpoint{0.728273in}{0.771595in}}%
\pgfpathlineto{\pgfqpoint{0.730780in}{0.772358in}}%
\pgfpathlineto{\pgfqpoint{0.733286in}{0.777053in}}%
\pgfpathlineto{\pgfqpoint{0.738300in}{0.779288in}}%
\pgfpathlineto{\pgfqpoint{0.740806in}{0.786929in}}%
\pgfpathlineto{\pgfqpoint{0.743313in}{0.787435in}}%
\pgfpathlineto{\pgfqpoint{0.748326in}{0.795380in}}%
\pgfpathlineto{\pgfqpoint{0.750833in}{0.795572in}}%
\pgfpathlineto{\pgfqpoint{0.753340in}{0.796956in}}%
\pgfpathlineto{\pgfqpoint{0.755846in}{0.800236in}}%
\pgfpathlineto{\pgfqpoint{0.758353in}{0.810851in}}%
\pgfpathlineto{\pgfqpoint{0.760860in}{0.811014in}}%
\pgfpathlineto{\pgfqpoint{0.768380in}{0.825995in}}%
\pgfpathlineto{\pgfqpoint{0.770886in}{0.826424in}}%
\pgfpathlineto{\pgfqpoint{0.773393in}{0.830131in}}%
\pgfpathlineto{\pgfqpoint{0.775900in}{0.835905in}}%
\pgfpathlineto{\pgfqpoint{0.783419in}{0.843517in}}%
\pgfpathlineto{\pgfqpoint{0.788433in}{0.849619in}}%
\pgfpathlineto{\pgfqpoint{0.790939in}{0.849861in}}%
\pgfpathlineto{\pgfqpoint{0.795953in}{0.853868in}}%
\pgfpathlineto{\pgfqpoint{0.798459in}{0.854337in}}%
\pgfpathlineto{\pgfqpoint{0.800966in}{0.860058in}}%
\pgfpathlineto{\pgfqpoint{0.803473in}{0.861684in}}%
\pgfpathlineto{\pgfqpoint{0.805979in}{0.864798in}}%
\pgfpathlineto{\pgfqpoint{0.821019in}{0.872164in}}%
\pgfpathlineto{\pgfqpoint{0.826033in}{0.876649in}}%
\pgfpathlineto{\pgfqpoint{0.828539in}{0.883782in}}%
\pgfpathlineto{\pgfqpoint{0.831046in}{0.884960in}}%
\pgfpathlineto{\pgfqpoint{0.838566in}{0.894577in}}%
\pgfpathlineto{\pgfqpoint{0.846086in}{0.897696in}}%
\pgfpathlineto{\pgfqpoint{0.848593in}{0.900186in}}%
\pgfpathlineto{\pgfqpoint{0.851099in}{0.905653in}}%
\pgfpathlineto{\pgfqpoint{0.856112in}{0.907111in}}%
\pgfpathlineto{\pgfqpoint{0.858619in}{0.913184in}}%
\pgfpathlineto{\pgfqpoint{0.861126in}{0.915920in}}%
\pgfpathlineto{\pgfqpoint{0.863632in}{0.921263in}}%
\pgfpathlineto{\pgfqpoint{0.866139in}{0.922728in}}%
\pgfpathlineto{\pgfqpoint{0.868646in}{0.927229in}}%
\pgfpathlineto{\pgfqpoint{0.871152in}{0.928658in}}%
\pgfpathlineto{\pgfqpoint{0.873659in}{0.928781in}}%
\pgfpathlineto{\pgfqpoint{0.876166in}{0.933329in}}%
\pgfpathlineto{\pgfqpoint{0.883686in}{0.938424in}}%
\pgfpathlineto{\pgfqpoint{0.888699in}{0.945966in}}%
\pgfpathlineto{\pgfqpoint{0.891206in}{0.946976in}}%
\pgfpathlineto{\pgfqpoint{0.896219in}{0.953389in}}%
\pgfpathlineto{\pgfqpoint{0.901232in}{0.954976in}}%
\pgfpathlineto{\pgfqpoint{0.903739in}{0.959098in}}%
\pgfpathlineto{\pgfqpoint{0.906246in}{0.959500in}}%
\pgfpathlineto{\pgfqpoint{0.908752in}{0.963676in}}%
\pgfpathlineto{\pgfqpoint{0.913766in}{0.964509in}}%
\pgfpathlineto{\pgfqpoint{0.918779in}{0.971297in}}%
\pgfpathlineto{\pgfqpoint{0.933819in}{0.976278in}}%
\pgfpathlineto{\pgfqpoint{0.936325in}{0.979983in}}%
\pgfpathlineto{\pgfqpoint{0.938832in}{0.980856in}}%
\pgfpathlineto{\pgfqpoint{0.943845in}{0.985585in}}%
\pgfpathlineto{\pgfqpoint{0.946352in}{0.986501in}}%
\pgfpathlineto{\pgfqpoint{0.948859in}{0.991060in}}%
\pgfpathlineto{\pgfqpoint{0.951365in}{1.000286in}}%
\pgfpathlineto{\pgfqpoint{0.953872in}{1.001069in}}%
\pgfpathlineto{\pgfqpoint{0.956379in}{1.003774in}}%
\pgfpathlineto{\pgfqpoint{0.961392in}{1.005230in}}%
\pgfpathlineto{\pgfqpoint{0.966405in}{1.006684in}}%
\pgfpathlineto{\pgfqpoint{0.968912in}{1.010576in}}%
\pgfpathlineto{\pgfqpoint{0.971419in}{1.010837in}}%
\pgfpathlineto{\pgfqpoint{0.973925in}{1.012665in}}%
\pgfpathlineto{\pgfqpoint{0.976432in}{1.012676in}}%
\pgfpathlineto{\pgfqpoint{0.978939in}{1.018670in}}%
\pgfpathlineto{\pgfqpoint{0.981445in}{1.018957in}}%
\pgfpathlineto{\pgfqpoint{0.988965in}{1.024644in}}%
\pgfpathlineto{\pgfqpoint{0.993978in}{1.025457in}}%
\pgfpathlineto{\pgfqpoint{0.998992in}{1.028746in}}%
\pgfpathlineto{\pgfqpoint{1.004005in}{1.031422in}}%
\pgfpathlineto{\pgfqpoint{1.014032in}{1.041335in}}%
\pgfpathlineto{\pgfqpoint{1.021552in}{1.043329in}}%
\pgfpathlineto{\pgfqpoint{1.024058in}{1.045429in}}%
\pgfpathlineto{\pgfqpoint{1.026565in}{1.045803in}}%
\pgfpathlineto{\pgfqpoint{1.031578in}{1.049982in}}%
\pgfpathlineto{\pgfqpoint{1.036592in}{1.052321in}}%
\pgfpathlineto{\pgfqpoint{1.039098in}{1.056761in}}%
\pgfpathlineto{\pgfqpoint{1.044112in}{1.059645in}}%
\pgfpathlineto{\pgfqpoint{1.046618in}{1.059854in}}%
\pgfpathlineto{\pgfqpoint{1.049125in}{1.064503in}}%
\pgfpathlineto{\pgfqpoint{1.051632in}{1.065319in}}%
\pgfpathlineto{\pgfqpoint{1.054138in}{1.070549in}}%
\pgfpathlineto{\pgfqpoint{1.059151in}{1.072413in}}%
\pgfpathlineto{\pgfqpoint{1.061658in}{1.072872in}}%
\pgfpathlineto{\pgfqpoint{1.064165in}{1.081669in}}%
\pgfpathlineto{\pgfqpoint{1.066671in}{1.082064in}}%
\pgfpathlineto{\pgfqpoint{1.069178in}{1.091493in}}%
\pgfpathlineto{\pgfqpoint{1.076698in}{1.098563in}}%
\pgfpathlineto{\pgfqpoint{1.081711in}{1.100701in}}%
\pgfpathlineto{\pgfqpoint{1.084218in}{1.106537in}}%
\pgfpathlineto{\pgfqpoint{1.086725in}{1.106549in}}%
\pgfpathlineto{\pgfqpoint{1.089231in}{1.109874in}}%
\pgfpathlineto{\pgfqpoint{1.094245in}{1.111842in}}%
\pgfpathlineto{\pgfqpoint{1.096751in}{1.114554in}}%
\pgfpathlineto{\pgfqpoint{1.101765in}{1.115421in}}%
\pgfpathlineto{\pgfqpoint{1.104271in}{1.117062in}}%
\pgfpathlineto{\pgfqpoint{1.111791in}{1.126625in}}%
\pgfpathlineto{\pgfqpoint{1.116805in}{1.127540in}}%
\pgfpathlineto{\pgfqpoint{1.119311in}{1.134012in}}%
\pgfpathlineto{\pgfqpoint{1.121818in}{1.136436in}}%
\pgfpathlineto{\pgfqpoint{1.124324in}{1.141749in}}%
\pgfpathlineto{\pgfqpoint{1.126831in}{1.142392in}}%
\pgfpathlineto{\pgfqpoint{1.136858in}{1.154282in}}%
\pgfpathlineto{\pgfqpoint{1.141871in}{1.156659in}}%
\pgfpathlineto{\pgfqpoint{1.154404in}{1.168854in}}%
\pgfpathlineto{\pgfqpoint{1.156911in}{1.177667in}}%
\pgfpathlineto{\pgfqpoint{1.164431in}{1.180382in}}%
\pgfpathlineto{\pgfqpoint{1.171951in}{1.181284in}}%
\pgfpathlineto{\pgfqpoint{1.176964in}{1.184475in}}%
\pgfpathlineto{\pgfqpoint{1.179471in}{1.185518in}}%
\pgfpathlineto{\pgfqpoint{1.181978in}{1.192882in}}%
\pgfpathlineto{\pgfqpoint{1.186991in}{1.194271in}}%
\pgfpathlineto{\pgfqpoint{1.192004in}{1.195829in}}%
\pgfpathlineto{\pgfqpoint{1.194511in}{1.199438in}}%
\pgfpathlineto{\pgfqpoint{1.202031in}{1.201473in}}%
\pgfpathlineto{\pgfqpoint{1.204537in}{1.211809in}}%
\pgfpathlineto{\pgfqpoint{1.209551in}{1.216272in}}%
\pgfpathlineto{\pgfqpoint{1.219577in}{1.219742in}}%
\pgfpathlineto{\pgfqpoint{1.222084in}{1.222944in}}%
\pgfpathlineto{\pgfqpoint{1.224591in}{1.230924in}}%
\pgfpathlineto{\pgfqpoint{1.227097in}{1.231038in}}%
\pgfpathlineto{\pgfqpoint{1.229604in}{1.232741in}}%
\pgfpathlineto{\pgfqpoint{1.232111in}{1.232966in}}%
\pgfpathlineto{\pgfqpoint{1.234617in}{1.240527in}}%
\pgfpathlineto{\pgfqpoint{1.237124in}{1.242534in}}%
\pgfpathlineto{\pgfqpoint{1.239631in}{1.243038in}}%
\pgfpathlineto{\pgfqpoint{1.242137in}{1.245247in}}%
\pgfpathlineto{\pgfqpoint{1.252164in}{1.248125in}}%
\pgfpathlineto{\pgfqpoint{1.254671in}{1.251939in}}%
\pgfpathlineto{\pgfqpoint{1.262190in}{1.254334in}}%
\pgfpathlineto{\pgfqpoint{1.267204in}{1.260066in}}%
\pgfpathlineto{\pgfqpoint{1.269710in}{1.262275in}}%
\pgfpathlineto{\pgfqpoint{1.274724in}{1.276237in}}%
\pgfpathlineto{\pgfqpoint{1.277230in}{1.276419in}}%
\pgfpathlineto{\pgfqpoint{1.282244in}{1.291562in}}%
\pgfpathlineto{\pgfqpoint{1.289764in}{1.293453in}}%
\pgfpathlineto{\pgfqpoint{1.292270in}{1.305140in}}%
\pgfpathlineto{\pgfqpoint{1.294777in}{1.310490in}}%
\pgfpathlineto{\pgfqpoint{1.302297in}{1.316907in}}%
\pgfpathlineto{\pgfqpoint{1.304804in}{1.316929in}}%
\pgfpathlineto{\pgfqpoint{1.309817in}{1.318936in}}%
\pgfpathlineto{\pgfqpoint{1.314830in}{1.320624in}}%
\pgfpathlineto{\pgfqpoint{1.319844in}{1.326455in}}%
\pgfpathlineto{\pgfqpoint{1.322350in}{1.330987in}}%
\pgfpathlineto{\pgfqpoint{1.324857in}{1.332404in}}%
\pgfpathlineto{\pgfqpoint{1.327363in}{1.345867in}}%
\pgfpathlineto{\pgfqpoint{1.329870in}{1.351247in}}%
\pgfpathlineto{\pgfqpoint{1.337390in}{1.354186in}}%
\pgfpathlineto{\pgfqpoint{1.342403in}{1.354980in}}%
\pgfpathlineto{\pgfqpoint{1.344910in}{1.356947in}}%
\pgfpathlineto{\pgfqpoint{1.349923in}{1.357653in}}%
\pgfpathlineto{\pgfqpoint{1.352430in}{1.360458in}}%
\pgfpathlineto{\pgfqpoint{1.354937in}{1.361043in}}%
\pgfpathlineto{\pgfqpoint{1.357443in}{1.367259in}}%
\pgfpathlineto{\pgfqpoint{1.369977in}{1.377631in}}%
\pgfpathlineto{\pgfqpoint{1.380003in}{1.379223in}}%
\pgfpathlineto{\pgfqpoint{1.382510in}{1.379792in}}%
\pgfpathlineto{\pgfqpoint{1.387523in}{1.393855in}}%
\pgfpathlineto{\pgfqpoint{1.392537in}{1.394952in}}%
\pgfpathlineto{\pgfqpoint{1.395043in}{1.399001in}}%
\pgfpathlineto{\pgfqpoint{1.402563in}{1.402245in}}%
\pgfpathlineto{\pgfqpoint{1.405070in}{1.407863in}}%
\pgfpathlineto{\pgfqpoint{1.410083in}{1.408274in}}%
\pgfpathlineto{\pgfqpoint{1.412590in}{1.410117in}}%
\pgfpathlineto{\pgfqpoint{1.415096in}{1.421159in}}%
\pgfpathlineto{\pgfqpoint{1.417603in}{1.422756in}}%
\pgfpathlineto{\pgfqpoint{1.420110in}{1.426230in}}%
\pgfpathlineto{\pgfqpoint{1.427630in}{1.429035in}}%
\pgfpathlineto{\pgfqpoint{1.430136in}{1.429553in}}%
\pgfpathlineto{\pgfqpoint{1.447683in}{1.448840in}}%
\pgfpathlineto{\pgfqpoint{1.455203in}{1.453613in}}%
\pgfpathlineto{\pgfqpoint{1.460216in}{1.466736in}}%
\pgfpathlineto{\pgfqpoint{1.462723in}{1.467103in}}%
\pgfpathlineto{\pgfqpoint{1.467736in}{1.470494in}}%
\pgfpathlineto{\pgfqpoint{1.475256in}{1.484762in}}%
\pgfpathlineto{\pgfqpoint{1.480269in}{1.487832in}}%
\pgfpathlineto{\pgfqpoint{1.482776in}{1.492677in}}%
\pgfpathlineto{\pgfqpoint{1.487789in}{1.495105in}}%
\pgfpathlineto{\pgfqpoint{1.492803in}{1.517496in}}%
\pgfpathlineto{\pgfqpoint{1.495309in}{1.518406in}}%
\pgfpathlineto{\pgfqpoint{1.497816in}{1.526777in}}%
\pgfpathlineto{\pgfqpoint{1.500323in}{1.527865in}}%
\pgfpathlineto{\pgfqpoint{1.502829in}{1.530657in}}%
\pgfpathlineto{\pgfqpoint{1.507843in}{1.532679in}}%
\pgfpathlineto{\pgfqpoint{1.510349in}{1.544689in}}%
\pgfpathlineto{\pgfqpoint{1.512856in}{1.548745in}}%
\pgfpathlineto{\pgfqpoint{1.515363in}{1.560635in}}%
\pgfpathlineto{\pgfqpoint{1.517869in}{1.565934in}}%
\pgfpathlineto{\pgfqpoint{1.522883in}{1.566489in}}%
\pgfpathlineto{\pgfqpoint{1.525389in}{1.570415in}}%
\pgfpathlineto{\pgfqpoint{1.527896in}{1.577585in}}%
\pgfpathlineto{\pgfqpoint{1.530402in}{1.581177in}}%
\pgfpathlineto{\pgfqpoint{1.535416in}{1.597643in}}%
\pgfpathlineto{\pgfqpoint{1.537922in}{1.597692in}}%
\pgfpathlineto{\pgfqpoint{1.550456in}{1.611544in}}%
\pgfpathlineto{\pgfqpoint{1.555469in}{1.614875in}}%
\pgfpathlineto{\pgfqpoint{1.557976in}{1.619386in}}%
\pgfpathlineto{\pgfqpoint{1.562989in}{1.621592in}}%
\pgfpathlineto{\pgfqpoint{1.565496in}{1.621607in}}%
\pgfpathlineto{\pgfqpoint{1.568002in}{1.623512in}}%
\pgfpathlineto{\pgfqpoint{1.583042in}{1.627027in}}%
\pgfpathlineto{\pgfqpoint{1.590562in}{1.628857in}}%
\pgfpathlineto{\pgfqpoint{1.593069in}{1.628943in}}%
\pgfpathlineto{\pgfqpoint{1.595576in}{1.631070in}}%
\pgfpathlineto{\pgfqpoint{1.598082in}{1.631708in}}%
\pgfpathlineto{\pgfqpoint{1.600589in}{1.634252in}}%
\pgfpathlineto{\pgfqpoint{1.603095in}{1.640406in}}%
\pgfpathlineto{\pgfqpoint{1.605602in}{1.640630in}}%
\pgfpathlineto{\pgfqpoint{1.608109in}{1.644887in}}%
\pgfpathlineto{\pgfqpoint{1.615629in}{1.645857in}}%
\pgfpathlineto{\pgfqpoint{1.618135in}{1.647783in}}%
\pgfpathlineto{\pgfqpoint{1.620642in}{1.652127in}}%
\pgfpathlineto{\pgfqpoint{1.638189in}{1.656450in}}%
\pgfpathlineto{\pgfqpoint{1.643202in}{1.662163in}}%
\pgfpathlineto{\pgfqpoint{1.645709in}{1.666515in}}%
\pgfpathlineto{\pgfqpoint{1.650722in}{1.667192in}}%
\pgfpathlineto{\pgfqpoint{1.663255in}{1.670855in}}%
\pgfpathlineto{\pgfqpoint{1.668268in}{1.676208in}}%
\pgfpathlineto{\pgfqpoint{1.673282in}{1.677505in}}%
\pgfpathlineto{\pgfqpoint{1.678295in}{1.680943in}}%
\pgfpathlineto{\pgfqpoint{1.680802in}{1.687881in}}%
\pgfpathlineto{\pgfqpoint{1.685815in}{1.692942in}}%
\pgfpathlineto{\pgfqpoint{1.688322in}{1.693092in}}%
\pgfpathlineto{\pgfqpoint{1.690828in}{1.694960in}}%
\pgfpathlineto{\pgfqpoint{1.695842in}{1.700569in}}%
\pgfpathlineto{\pgfqpoint{1.700855in}{1.701112in}}%
\pgfpathlineto{\pgfqpoint{1.703362in}{1.704115in}}%
\pgfpathlineto{\pgfqpoint{1.713388in}{1.706715in}}%
\pgfpathlineto{\pgfqpoint{1.715895in}{1.711130in}}%
\pgfpathlineto{\pgfqpoint{1.718402in}{1.713050in}}%
\pgfpathlineto{\pgfqpoint{1.720908in}{1.713450in}}%
\pgfpathlineto{\pgfqpoint{1.728428in}{1.724579in}}%
\pgfpathlineto{\pgfqpoint{1.730935in}{1.733898in}}%
\pgfpathlineto{\pgfqpoint{1.733442in}{1.735378in}}%
\pgfpathlineto{\pgfqpoint{1.735948in}{1.739893in}}%
\pgfpathlineto{\pgfqpoint{1.738455in}{1.748389in}}%
\pgfpathlineto{\pgfqpoint{1.740961in}{1.749839in}}%
\pgfpathlineto{\pgfqpoint{1.745975in}{1.762602in}}%
\pgfpathlineto{\pgfqpoint{1.748481in}{1.764431in}}%
\pgfpathlineto{\pgfqpoint{1.750988in}{1.764804in}}%
\pgfpathlineto{\pgfqpoint{1.756001in}{1.768816in}}%
\pgfpathlineto{\pgfqpoint{1.758508in}{1.777677in}}%
\pgfpathlineto{\pgfqpoint{1.761015in}{1.778348in}}%
\pgfpathlineto{\pgfqpoint{1.766028in}{1.784987in}}%
\pgfpathlineto{\pgfqpoint{1.768535in}{1.786237in}}%
\pgfpathlineto{\pgfqpoint{1.776055in}{1.799256in}}%
\pgfpathlineto{\pgfqpoint{1.778561in}{1.801596in}}%
\pgfpathlineto{\pgfqpoint{1.781068in}{1.801918in}}%
\pgfpathlineto{\pgfqpoint{1.786081in}{1.817037in}}%
\pgfpathlineto{\pgfqpoint{1.788588in}{1.819693in}}%
\pgfpathlineto{\pgfqpoint{1.793601in}{1.820911in}}%
\pgfpathlineto{\pgfqpoint{1.796108in}{1.826570in}}%
\pgfpathlineto{\pgfqpoint{1.798615in}{1.828554in}}%
\pgfpathlineto{\pgfqpoint{1.803628in}{1.829557in}}%
\pgfpathlineto{\pgfqpoint{1.806134in}{1.831478in}}%
\pgfpathlineto{\pgfqpoint{1.811148in}{1.838983in}}%
\pgfpathlineto{\pgfqpoint{1.813654in}{1.839070in}}%
\pgfpathlineto{\pgfqpoint{1.816161in}{1.842594in}}%
\pgfpathlineto{\pgfqpoint{1.821174in}{1.843652in}}%
\pgfpathlineto{\pgfqpoint{1.823681in}{1.845398in}}%
\pgfpathlineto{\pgfqpoint{1.826188in}{1.853883in}}%
\pgfpathlineto{\pgfqpoint{1.831201in}{1.857884in}}%
\pgfpathlineto{\pgfqpoint{1.833708in}{1.858864in}}%
\pgfpathlineto{\pgfqpoint{1.836214in}{1.861181in}}%
\pgfpathlineto{\pgfqpoint{1.838721in}{1.870165in}}%
\pgfpathlineto{\pgfqpoint{1.841228in}{1.873838in}}%
\pgfpathlineto{\pgfqpoint{1.856268in}{1.881552in}}%
\pgfpathlineto{\pgfqpoint{1.868801in}{1.894994in}}%
\pgfpathlineto{\pgfqpoint{1.873814in}{1.896655in}}%
\pgfpathlineto{\pgfqpoint{1.876321in}{1.897376in}}%
\pgfpathlineto{\pgfqpoint{1.881334in}{1.906622in}}%
\pgfpathlineto{\pgfqpoint{1.883841in}{1.907992in}}%
\pgfpathlineto{\pgfqpoint{1.891361in}{1.908712in}}%
\pgfpathlineto{\pgfqpoint{1.893867in}{1.913809in}}%
\pgfpathlineto{\pgfqpoint{1.896374in}{1.913993in}}%
\pgfpathlineto{\pgfqpoint{1.898881in}{1.916146in}}%
\pgfpathlineto{\pgfqpoint{1.901387in}{1.916592in}}%
\pgfpathlineto{\pgfqpoint{1.903894in}{1.918248in}}%
\pgfpathlineto{\pgfqpoint{1.906401in}{1.922503in}}%
\pgfpathlineto{\pgfqpoint{1.913921in}{1.926615in}}%
\pgfpathlineto{\pgfqpoint{1.916427in}{1.932171in}}%
\pgfpathlineto{\pgfqpoint{1.926454in}{1.940861in}}%
\pgfpathlineto{\pgfqpoint{1.933974in}{1.941291in}}%
\pgfpathlineto{\pgfqpoint{1.936481in}{1.942869in}}%
\pgfpathlineto{\pgfqpoint{1.941494in}{1.949030in}}%
\pgfpathlineto{\pgfqpoint{1.954027in}{1.953215in}}%
\pgfpathlineto{\pgfqpoint{1.956534in}{1.958628in}}%
\pgfpathlineto{\pgfqpoint{1.959040in}{1.959255in}}%
\pgfpathlineto{\pgfqpoint{1.961547in}{1.968523in}}%
\pgfpathlineto{\pgfqpoint{1.964054in}{1.969999in}}%
\pgfpathlineto{\pgfqpoint{1.966560in}{1.977539in}}%
\pgfpathlineto{\pgfqpoint{1.971574in}{1.979291in}}%
\pgfpathlineto{\pgfqpoint{1.979094in}{1.990788in}}%
\pgfpathlineto{\pgfqpoint{1.984107in}{1.992121in}}%
\pgfpathlineto{\pgfqpoint{1.989120in}{1.992550in}}%
\pgfpathlineto{\pgfqpoint{2.001654in}{1.998436in}}%
\pgfpathlineto{\pgfqpoint{2.004160in}{2.002438in}}%
\pgfpathlineto{\pgfqpoint{2.009173in}{2.003684in}}%
\pgfpathlineto{\pgfqpoint{2.014187in}{2.015927in}}%
\pgfpathlineto{\pgfqpoint{2.016693in}{2.019124in}}%
\pgfpathlineto{\pgfqpoint{2.019200in}{2.030374in}}%
\pgfpathlineto{\pgfqpoint{2.021707in}{2.032115in}}%
\pgfpathlineto{\pgfqpoint{2.024213in}{2.038484in}}%
\pgfpathlineto{\pgfqpoint{2.031733in}{2.040284in}}%
\pgfpathlineto{\pgfqpoint{2.034240in}{2.043925in}}%
\pgfpathlineto{\pgfqpoint{2.036747in}{2.044585in}}%
\pgfpathlineto{\pgfqpoint{2.044267in}{2.052034in}}%
\pgfpathlineto{\pgfqpoint{2.046773in}{2.053418in}}%
\pgfpathlineto{\pgfqpoint{2.051787in}{2.064583in}}%
\pgfpathlineto{\pgfqpoint{2.054293in}{2.076998in}}%
\pgfpathlineto{\pgfqpoint{2.069333in}{2.086381in}}%
\pgfpathlineto{\pgfqpoint{2.071840in}{2.101455in}}%
\pgfpathlineto{\pgfqpoint{2.076853in}{2.103181in}}%
\pgfpathlineto{\pgfqpoint{2.079360in}{2.107014in}}%
\pgfpathlineto{\pgfqpoint{2.081866in}{2.107975in}}%
\pgfpathlineto{\pgfqpoint{2.084373in}{2.116026in}}%
\pgfpathlineto{\pgfqpoint{2.086880in}{2.130011in}}%
\pgfpathlineto{\pgfqpoint{2.089386in}{2.133989in}}%
\pgfpathlineto{\pgfqpoint{2.091893in}{2.142287in}}%
\pgfpathlineto{\pgfqpoint{2.096906in}{2.146390in}}%
\pgfpathlineto{\pgfqpoint{2.101920in}{2.146876in}}%
\pgfpathlineto{\pgfqpoint{2.104426in}{2.148965in}}%
\pgfpathlineto{\pgfqpoint{2.109440in}{2.149894in}}%
\pgfpathlineto{\pgfqpoint{2.111946in}{2.157208in}}%
\pgfpathlineto{\pgfqpoint{2.114453in}{2.157887in}}%
\pgfpathlineto{\pgfqpoint{2.116960in}{2.160799in}}%
\pgfpathlineto{\pgfqpoint{2.126986in}{2.163571in}}%
\pgfpathlineto{\pgfqpoint{2.129493in}{2.169561in}}%
\pgfpathlineto{\pgfqpoint{2.132000in}{2.172247in}}%
\pgfpathlineto{\pgfqpoint{2.134506in}{2.172638in}}%
\pgfpathlineto{\pgfqpoint{2.139520in}{2.176192in}}%
\pgfpathlineto{\pgfqpoint{2.142026in}{2.186689in}}%
\pgfpathlineto{\pgfqpoint{2.147039in}{2.191746in}}%
\pgfpathlineto{\pgfqpoint{2.152053in}{2.195785in}}%
\pgfpathlineto{\pgfqpoint{2.154559in}{2.198361in}}%
\pgfpathlineto{\pgfqpoint{2.157066in}{2.198541in}}%
\pgfpathlineto{\pgfqpoint{2.159573in}{2.201998in}}%
\pgfpathlineto{\pgfqpoint{2.164586in}{2.204086in}}%
\pgfpathlineto{\pgfqpoint{2.169599in}{2.206594in}}%
\pgfpathlineto{\pgfqpoint{2.172106in}{2.210461in}}%
\pgfpathlineto{\pgfqpoint{2.174613in}{2.220490in}}%
\pgfpathlineto{\pgfqpoint{2.179626in}{2.222669in}}%
\pgfpathlineto{\pgfqpoint{2.182133in}{2.230952in}}%
\pgfpathlineto{\pgfqpoint{2.184639in}{2.231903in}}%
\pgfpathlineto{\pgfqpoint{2.187146in}{2.234940in}}%
\pgfpathlineto{\pgfqpoint{2.194666in}{2.236710in}}%
\pgfpathlineto{\pgfqpoint{2.197173in}{2.244181in}}%
\pgfpathlineto{\pgfqpoint{2.199679in}{2.247748in}}%
\pgfpathlineto{\pgfqpoint{2.204693in}{2.249700in}}%
\pgfpathlineto{\pgfqpoint{2.209706in}{2.254780in}}%
\pgfpathlineto{\pgfqpoint{2.212212in}{2.256778in}}%
\pgfpathlineto{\pgfqpoint{2.214719in}{2.257187in}}%
\pgfpathlineto{\pgfqpoint{2.232266in}{2.276157in}}%
\pgfpathlineto{\pgfqpoint{2.234772in}{2.281499in}}%
\pgfpathlineto{\pgfqpoint{2.244799in}{2.283535in}}%
\pgfpathlineto{\pgfqpoint{2.247306in}{2.289452in}}%
\pgfpathlineto{\pgfqpoint{2.249812in}{2.290445in}}%
\pgfpathlineto{\pgfqpoint{2.252319in}{2.303613in}}%
\pgfpathlineto{\pgfqpoint{2.257332in}{2.305275in}}%
\pgfpathlineto{\pgfqpoint{5.536036in}{2.306326in}}%
\pgfpathlineto{\pgfqpoint{5.618756in}{2.309247in}}%
\pgfpathlineto{\pgfqpoint{5.626276in}{2.310928in}}%
\pgfpathlineto{\pgfqpoint{5.627232in}{2.315275in}}%
\pgfpathlineto{\pgfqpoint{5.627232in}{2.315275in}}%
\pgfusepath{stroke}%
\end{pgfscope}%
\begin{pgfscope}%
\pgfpathrectangle{\pgfqpoint{0.708220in}{0.535823in}}{\pgfqpoint{5.013309in}{1.769453in}}%
\pgfusepath{clip}%
\pgfsetrectcap%
\pgfsetroundjoin%
\pgfsetlinewidth{1.003750pt}%
\definecolor{currentstroke}{rgb}{0.000000,0.501961,0.000000}%
\pgfsetstrokecolor{currentstroke}%
\pgfsetdash{}{0pt}%
\pgfpathmoveto{\pgfqpoint{0.708220in}{0.633641in}}%
\pgfpathlineto{\pgfqpoint{0.713233in}{0.663878in}}%
\pgfpathlineto{\pgfqpoint{0.715740in}{0.710153in}}%
\pgfpathlineto{\pgfqpoint{0.718246in}{0.717904in}}%
\pgfpathlineto{\pgfqpoint{0.720753in}{0.729608in}}%
\pgfpathlineto{\pgfqpoint{0.723260in}{0.759376in}}%
\pgfpathlineto{\pgfqpoint{0.728273in}{0.771595in}}%
\pgfpathlineto{\pgfqpoint{0.730780in}{0.772358in}}%
\pgfpathlineto{\pgfqpoint{0.733286in}{0.777053in}}%
\pgfpathlineto{\pgfqpoint{0.738300in}{0.779288in}}%
\pgfpathlineto{\pgfqpoint{0.740806in}{0.786929in}}%
\pgfpathlineto{\pgfqpoint{0.743313in}{0.787435in}}%
\pgfpathlineto{\pgfqpoint{0.748326in}{0.795380in}}%
\pgfpathlineto{\pgfqpoint{0.750833in}{0.795572in}}%
\pgfpathlineto{\pgfqpoint{0.753340in}{0.796956in}}%
\pgfpathlineto{\pgfqpoint{0.755846in}{0.800236in}}%
\pgfpathlineto{\pgfqpoint{0.758353in}{0.810851in}}%
\pgfpathlineto{\pgfqpoint{0.760860in}{0.811014in}}%
\pgfpathlineto{\pgfqpoint{0.768380in}{0.825995in}}%
\pgfpathlineto{\pgfqpoint{0.770886in}{0.826424in}}%
\pgfpathlineto{\pgfqpoint{0.773393in}{0.830131in}}%
\pgfpathlineto{\pgfqpoint{0.775900in}{0.835905in}}%
\pgfpathlineto{\pgfqpoint{0.783419in}{0.843517in}}%
\pgfpathlineto{\pgfqpoint{0.788433in}{0.849619in}}%
\pgfpathlineto{\pgfqpoint{0.790939in}{0.849861in}}%
\pgfpathlineto{\pgfqpoint{0.795953in}{0.853868in}}%
\pgfpathlineto{\pgfqpoint{0.798459in}{0.854337in}}%
\pgfpathlineto{\pgfqpoint{0.800966in}{0.860058in}}%
\pgfpathlineto{\pgfqpoint{0.803473in}{0.861684in}}%
\pgfpathlineto{\pgfqpoint{0.805979in}{0.864798in}}%
\pgfpathlineto{\pgfqpoint{0.821019in}{0.872164in}}%
\pgfpathlineto{\pgfqpoint{0.826033in}{0.876649in}}%
\pgfpathlineto{\pgfqpoint{0.828539in}{0.883782in}}%
\pgfpathlineto{\pgfqpoint{0.831046in}{0.884960in}}%
\pgfpathlineto{\pgfqpoint{0.838566in}{0.894577in}}%
\pgfpathlineto{\pgfqpoint{0.846086in}{0.897696in}}%
\pgfpathlineto{\pgfqpoint{0.848593in}{0.900186in}}%
\pgfpathlineto{\pgfqpoint{0.851099in}{0.905653in}}%
\pgfpathlineto{\pgfqpoint{0.856112in}{0.907111in}}%
\pgfpathlineto{\pgfqpoint{0.858619in}{0.913184in}}%
\pgfpathlineto{\pgfqpoint{0.861126in}{0.915920in}}%
\pgfpathlineto{\pgfqpoint{0.863632in}{0.921263in}}%
\pgfpathlineto{\pgfqpoint{0.866139in}{0.922728in}}%
\pgfpathlineto{\pgfqpoint{0.868646in}{0.927229in}}%
\pgfpathlineto{\pgfqpoint{0.871152in}{0.928658in}}%
\pgfpathlineto{\pgfqpoint{0.873659in}{0.928781in}}%
\pgfpathlineto{\pgfqpoint{0.876166in}{0.933329in}}%
\pgfpathlineto{\pgfqpoint{0.883686in}{0.938424in}}%
\pgfpathlineto{\pgfqpoint{0.888699in}{0.945966in}}%
\pgfpathlineto{\pgfqpoint{0.891206in}{0.946976in}}%
\pgfpathlineto{\pgfqpoint{0.896219in}{0.953389in}}%
\pgfpathlineto{\pgfqpoint{0.901232in}{0.954976in}}%
\pgfpathlineto{\pgfqpoint{0.903739in}{0.959098in}}%
\pgfpathlineto{\pgfqpoint{0.906246in}{0.959500in}}%
\pgfpathlineto{\pgfqpoint{0.908752in}{0.963676in}}%
\pgfpathlineto{\pgfqpoint{0.913766in}{0.964509in}}%
\pgfpathlineto{\pgfqpoint{0.918779in}{0.971297in}}%
\pgfpathlineto{\pgfqpoint{0.933819in}{0.976278in}}%
\pgfpathlineto{\pgfqpoint{0.936325in}{0.979983in}}%
\pgfpathlineto{\pgfqpoint{0.938832in}{0.980856in}}%
\pgfpathlineto{\pgfqpoint{0.943845in}{0.985585in}}%
\pgfpathlineto{\pgfqpoint{0.946352in}{0.986501in}}%
\pgfpathlineto{\pgfqpoint{0.948859in}{0.991060in}}%
\pgfpathlineto{\pgfqpoint{0.951365in}{1.000286in}}%
\pgfpathlineto{\pgfqpoint{0.953872in}{1.001069in}}%
\pgfpathlineto{\pgfqpoint{0.956379in}{1.003774in}}%
\pgfpathlineto{\pgfqpoint{0.961392in}{1.005230in}}%
\pgfpathlineto{\pgfqpoint{0.966405in}{1.006684in}}%
\pgfpathlineto{\pgfqpoint{0.968912in}{1.010576in}}%
\pgfpathlineto{\pgfqpoint{0.971419in}{1.010837in}}%
\pgfpathlineto{\pgfqpoint{0.973925in}{1.012665in}}%
\pgfpathlineto{\pgfqpoint{0.976432in}{1.012676in}}%
\pgfpathlineto{\pgfqpoint{0.978939in}{1.018670in}}%
\pgfpathlineto{\pgfqpoint{0.981445in}{1.018957in}}%
\pgfpathlineto{\pgfqpoint{0.988965in}{1.024644in}}%
\pgfpathlineto{\pgfqpoint{0.993978in}{1.025457in}}%
\pgfpathlineto{\pgfqpoint{0.998992in}{1.028746in}}%
\pgfpathlineto{\pgfqpoint{1.004005in}{1.031422in}}%
\pgfpathlineto{\pgfqpoint{1.014032in}{1.041335in}}%
\pgfpathlineto{\pgfqpoint{1.021552in}{1.043329in}}%
\pgfpathlineto{\pgfqpoint{1.024058in}{1.045429in}}%
\pgfpathlineto{\pgfqpoint{1.026565in}{1.045803in}}%
\pgfpathlineto{\pgfqpoint{1.031578in}{1.049982in}}%
\pgfpathlineto{\pgfqpoint{1.036592in}{1.052321in}}%
\pgfpathlineto{\pgfqpoint{1.039098in}{1.056761in}}%
\pgfpathlineto{\pgfqpoint{1.044112in}{1.059645in}}%
\pgfpathlineto{\pgfqpoint{1.046618in}{1.059854in}}%
\pgfpathlineto{\pgfqpoint{1.049125in}{1.064503in}}%
\pgfpathlineto{\pgfqpoint{1.051632in}{1.065319in}}%
\pgfpathlineto{\pgfqpoint{1.054138in}{1.070549in}}%
\pgfpathlineto{\pgfqpoint{1.059151in}{1.072413in}}%
\pgfpathlineto{\pgfqpoint{1.061658in}{1.072872in}}%
\pgfpathlineto{\pgfqpoint{1.064165in}{1.081669in}}%
\pgfpathlineto{\pgfqpoint{1.066671in}{1.082064in}}%
\pgfpathlineto{\pgfqpoint{1.069178in}{1.091493in}}%
\pgfpathlineto{\pgfqpoint{1.076698in}{1.098563in}}%
\pgfpathlineto{\pgfqpoint{1.081711in}{1.100701in}}%
\pgfpathlineto{\pgfqpoint{1.084218in}{1.106537in}}%
\pgfpathlineto{\pgfqpoint{1.086725in}{1.106549in}}%
\pgfpathlineto{\pgfqpoint{1.089231in}{1.109874in}}%
\pgfpathlineto{\pgfqpoint{1.094245in}{1.111842in}}%
\pgfpathlineto{\pgfqpoint{1.096751in}{1.114554in}}%
\pgfpathlineto{\pgfqpoint{1.101765in}{1.115421in}}%
\pgfpathlineto{\pgfqpoint{1.104271in}{1.117062in}}%
\pgfpathlineto{\pgfqpoint{1.106778in}{1.120097in}}%
\pgfpathlineto{\pgfqpoint{1.111791in}{1.122250in}}%
\pgfpathlineto{\pgfqpoint{1.114298in}{1.126625in}}%
\pgfpathlineto{\pgfqpoint{1.119311in}{1.127540in}}%
\pgfpathlineto{\pgfqpoint{1.121818in}{1.132804in}}%
\pgfpathlineto{\pgfqpoint{1.126831in}{1.136436in}}%
\pgfpathlineto{\pgfqpoint{1.129338in}{1.141749in}}%
\pgfpathlineto{\pgfqpoint{1.134351in}{1.142932in}}%
\pgfpathlineto{\pgfqpoint{1.136858in}{1.145675in}}%
\pgfpathlineto{\pgfqpoint{1.139364in}{1.150456in}}%
\pgfpathlineto{\pgfqpoint{1.151898in}{1.159368in}}%
\pgfpathlineto{\pgfqpoint{1.156911in}{1.165629in}}%
\pgfpathlineto{\pgfqpoint{1.161924in}{1.166519in}}%
\pgfpathlineto{\pgfqpoint{1.164431in}{1.168854in}}%
\pgfpathlineto{\pgfqpoint{1.166938in}{1.177667in}}%
\pgfpathlineto{\pgfqpoint{1.174458in}{1.180382in}}%
\pgfpathlineto{\pgfqpoint{1.184484in}{1.181284in}}%
\pgfpathlineto{\pgfqpoint{1.189498in}{1.184475in}}%
\pgfpathlineto{\pgfqpoint{1.192004in}{1.185518in}}%
\pgfpathlineto{\pgfqpoint{1.194511in}{1.192882in}}%
\pgfpathlineto{\pgfqpoint{1.199524in}{1.194271in}}%
\pgfpathlineto{\pgfqpoint{1.204537in}{1.195829in}}%
\pgfpathlineto{\pgfqpoint{1.209551in}{1.199438in}}%
\pgfpathlineto{\pgfqpoint{1.217071in}{1.201473in}}%
\pgfpathlineto{\pgfqpoint{1.219577in}{1.211809in}}%
\pgfpathlineto{\pgfqpoint{1.224591in}{1.216272in}}%
\pgfpathlineto{\pgfqpoint{1.234617in}{1.219742in}}%
\pgfpathlineto{\pgfqpoint{1.237124in}{1.222944in}}%
\pgfpathlineto{\pgfqpoint{1.239631in}{1.230924in}}%
\pgfpathlineto{\pgfqpoint{1.242137in}{1.231038in}}%
\pgfpathlineto{\pgfqpoint{1.244644in}{1.232741in}}%
\pgfpathlineto{\pgfqpoint{1.247151in}{1.232966in}}%
\pgfpathlineto{\pgfqpoint{1.249657in}{1.240527in}}%
\pgfpathlineto{\pgfqpoint{1.252164in}{1.242534in}}%
\pgfpathlineto{\pgfqpoint{1.254671in}{1.243038in}}%
\pgfpathlineto{\pgfqpoint{1.257177in}{1.245247in}}%
\pgfpathlineto{\pgfqpoint{1.269710in}{1.248125in}}%
\pgfpathlineto{\pgfqpoint{1.272217in}{1.251939in}}%
\pgfpathlineto{\pgfqpoint{1.279737in}{1.254334in}}%
\pgfpathlineto{\pgfqpoint{1.282244in}{1.257990in}}%
\pgfpathlineto{\pgfqpoint{1.289764in}{1.260066in}}%
\pgfpathlineto{\pgfqpoint{1.292270in}{1.262275in}}%
\pgfpathlineto{\pgfqpoint{1.294777in}{1.269337in}}%
\pgfpathlineto{\pgfqpoint{1.297284in}{1.270455in}}%
\pgfpathlineto{\pgfqpoint{1.299790in}{1.276237in}}%
\pgfpathlineto{\pgfqpoint{1.302297in}{1.276419in}}%
\pgfpathlineto{\pgfqpoint{1.307310in}{1.291562in}}%
\pgfpathlineto{\pgfqpoint{1.317337in}{1.294463in}}%
\pgfpathlineto{\pgfqpoint{1.322350in}{1.295782in}}%
\pgfpathlineto{\pgfqpoint{1.327363in}{1.299247in}}%
\pgfpathlineto{\pgfqpoint{1.332377in}{1.310490in}}%
\pgfpathlineto{\pgfqpoint{1.334883in}{1.312889in}}%
\pgfpathlineto{\pgfqpoint{1.337390in}{1.316907in}}%
\pgfpathlineto{\pgfqpoint{1.339897in}{1.316929in}}%
\pgfpathlineto{\pgfqpoint{1.344910in}{1.318936in}}%
\pgfpathlineto{\pgfqpoint{1.349923in}{1.320624in}}%
\pgfpathlineto{\pgfqpoint{1.352430in}{1.321833in}}%
\pgfpathlineto{\pgfqpoint{1.354937in}{1.325298in}}%
\pgfpathlineto{\pgfqpoint{1.364963in}{1.330987in}}%
\pgfpathlineto{\pgfqpoint{1.367470in}{1.332404in}}%
\pgfpathlineto{\pgfqpoint{1.369977in}{1.341588in}}%
\pgfpathlineto{\pgfqpoint{1.372483in}{1.344919in}}%
\pgfpathlineto{\pgfqpoint{1.380003in}{1.346971in}}%
\pgfpathlineto{\pgfqpoint{1.385017in}{1.354164in}}%
\pgfpathlineto{\pgfqpoint{1.390030in}{1.354414in}}%
\pgfpathlineto{\pgfqpoint{1.392537in}{1.356947in}}%
\pgfpathlineto{\pgfqpoint{1.397550in}{1.357784in}}%
\pgfpathlineto{\pgfqpoint{1.400056in}{1.360039in}}%
\pgfpathlineto{\pgfqpoint{1.405070in}{1.361043in}}%
\pgfpathlineto{\pgfqpoint{1.412590in}{1.369553in}}%
\pgfpathlineto{\pgfqpoint{1.422616in}{1.373805in}}%
\pgfpathlineto{\pgfqpoint{1.427630in}{1.374810in}}%
\pgfpathlineto{\pgfqpoint{1.432643in}{1.377660in}}%
\pgfpathlineto{\pgfqpoint{1.445176in}{1.379792in}}%
\pgfpathlineto{\pgfqpoint{1.452696in}{1.393855in}}%
\pgfpathlineto{\pgfqpoint{1.457710in}{1.394952in}}%
\pgfpathlineto{\pgfqpoint{1.460216in}{1.399001in}}%
\pgfpathlineto{\pgfqpoint{1.467736in}{1.403354in}}%
\pgfpathlineto{\pgfqpoint{1.470243in}{1.408274in}}%
\pgfpathlineto{\pgfqpoint{1.472749in}{1.408365in}}%
\pgfpathlineto{\pgfqpoint{1.475256in}{1.412972in}}%
\pgfpathlineto{\pgfqpoint{1.477763in}{1.420141in}}%
\pgfpathlineto{\pgfqpoint{1.485283in}{1.422756in}}%
\pgfpathlineto{\pgfqpoint{1.487789in}{1.425444in}}%
\pgfpathlineto{\pgfqpoint{1.497816in}{1.429035in}}%
\pgfpathlineto{\pgfqpoint{1.500323in}{1.434339in}}%
\pgfpathlineto{\pgfqpoint{1.502829in}{1.435758in}}%
\pgfpathlineto{\pgfqpoint{1.507843in}{1.442503in}}%
\pgfpathlineto{\pgfqpoint{1.512856in}{1.444121in}}%
\pgfpathlineto{\pgfqpoint{1.515363in}{1.448840in}}%
\pgfpathlineto{\pgfqpoint{1.520376in}{1.450218in}}%
\pgfpathlineto{\pgfqpoint{1.522883in}{1.450683in}}%
\pgfpathlineto{\pgfqpoint{1.525389in}{1.453613in}}%
\pgfpathlineto{\pgfqpoint{1.527896in}{1.459503in}}%
\pgfpathlineto{\pgfqpoint{1.530402in}{1.461051in}}%
\pgfpathlineto{\pgfqpoint{1.535416in}{1.461933in}}%
\pgfpathlineto{\pgfqpoint{1.540429in}{1.466731in}}%
\pgfpathlineto{\pgfqpoint{1.542936in}{1.467103in}}%
\pgfpathlineto{\pgfqpoint{1.547949in}{1.470494in}}%
\pgfpathlineto{\pgfqpoint{1.552962in}{1.479771in}}%
\pgfpathlineto{\pgfqpoint{1.557976in}{1.480845in}}%
\pgfpathlineto{\pgfqpoint{1.562989in}{1.487090in}}%
\pgfpathlineto{\pgfqpoint{1.565496in}{1.488522in}}%
\pgfpathlineto{\pgfqpoint{1.570509in}{1.493316in}}%
\pgfpathlineto{\pgfqpoint{1.578029in}{1.495743in}}%
\pgfpathlineto{\pgfqpoint{1.580536in}{1.498748in}}%
\pgfpathlineto{\pgfqpoint{1.583042in}{1.499634in}}%
\pgfpathlineto{\pgfqpoint{1.585549in}{1.509725in}}%
\pgfpathlineto{\pgfqpoint{1.590562in}{1.516767in}}%
\pgfpathlineto{\pgfqpoint{1.593069in}{1.517496in}}%
\pgfpathlineto{\pgfqpoint{1.598082in}{1.526777in}}%
\pgfpathlineto{\pgfqpoint{1.600589in}{1.531606in}}%
\pgfpathlineto{\pgfqpoint{1.603095in}{1.531856in}}%
\pgfpathlineto{\pgfqpoint{1.605602in}{1.534760in}}%
\pgfpathlineto{\pgfqpoint{1.608109in}{1.535713in}}%
\pgfpathlineto{\pgfqpoint{1.615629in}{1.547853in}}%
\pgfpathlineto{\pgfqpoint{1.618135in}{1.548006in}}%
\pgfpathlineto{\pgfqpoint{1.625655in}{1.560635in}}%
\pgfpathlineto{\pgfqpoint{1.628162in}{1.562914in}}%
\pgfpathlineto{\pgfqpoint{1.630669in}{1.562925in}}%
\pgfpathlineto{\pgfqpoint{1.640695in}{1.577585in}}%
\pgfpathlineto{\pgfqpoint{1.643202in}{1.577963in}}%
\pgfpathlineto{\pgfqpoint{1.650722in}{1.582849in}}%
\pgfpathlineto{\pgfqpoint{1.653229in}{1.588610in}}%
\pgfpathlineto{\pgfqpoint{1.660749in}{1.590127in}}%
\pgfpathlineto{\pgfqpoint{1.670775in}{1.592172in}}%
\pgfpathlineto{\pgfqpoint{1.685815in}{1.597705in}}%
\pgfpathlineto{\pgfqpoint{1.693335in}{1.598589in}}%
\pgfpathlineto{\pgfqpoint{1.695842in}{1.598616in}}%
\pgfpathlineto{\pgfqpoint{1.698348in}{1.603123in}}%
\pgfpathlineto{\pgfqpoint{1.700855in}{1.603906in}}%
\pgfpathlineto{\pgfqpoint{1.708375in}{1.614135in}}%
\pgfpathlineto{\pgfqpoint{1.713388in}{1.615414in}}%
\pgfpathlineto{\pgfqpoint{1.715895in}{1.616161in}}%
\pgfpathlineto{\pgfqpoint{1.718402in}{1.618906in}}%
\pgfpathlineto{\pgfqpoint{1.720908in}{1.619386in}}%
\pgfpathlineto{\pgfqpoint{1.723415in}{1.622394in}}%
\pgfpathlineto{\pgfqpoint{1.728428in}{1.624023in}}%
\pgfpathlineto{\pgfqpoint{1.733442in}{1.626563in}}%
\pgfpathlineto{\pgfqpoint{1.753495in}{1.629360in}}%
\pgfpathlineto{\pgfqpoint{1.768535in}{1.633192in}}%
\pgfpathlineto{\pgfqpoint{1.773548in}{1.633652in}}%
\pgfpathlineto{\pgfqpoint{1.776055in}{1.634942in}}%
\pgfpathlineto{\pgfqpoint{1.778561in}{1.645857in}}%
\pgfpathlineto{\pgfqpoint{1.781068in}{1.646020in}}%
\pgfpathlineto{\pgfqpoint{1.783575in}{1.650274in}}%
\pgfpathlineto{\pgfqpoint{1.821174in}{1.657851in}}%
\pgfpathlineto{\pgfqpoint{1.823681in}{1.666646in}}%
\pgfpathlineto{\pgfqpoint{1.833708in}{1.668516in}}%
\pgfpathlineto{\pgfqpoint{1.836214in}{1.671561in}}%
\pgfpathlineto{\pgfqpoint{1.846241in}{1.672331in}}%
\pgfpathlineto{\pgfqpoint{1.851254in}{1.674874in}}%
\pgfpathlineto{\pgfqpoint{1.853761in}{1.675059in}}%
\pgfpathlineto{\pgfqpoint{1.858774in}{1.687880in}}%
\pgfpathlineto{\pgfqpoint{1.868801in}{1.689863in}}%
\pgfpathlineto{\pgfqpoint{1.871307in}{1.694144in}}%
\pgfpathlineto{\pgfqpoint{1.878827in}{1.694998in}}%
\pgfpathlineto{\pgfqpoint{1.883841in}{1.698851in}}%
\pgfpathlineto{\pgfqpoint{1.886347in}{1.701237in}}%
\pgfpathlineto{\pgfqpoint{1.891361in}{1.701921in}}%
\pgfpathlineto{\pgfqpoint{1.896374in}{1.705375in}}%
\pgfpathlineto{\pgfqpoint{1.901387in}{1.705664in}}%
\pgfpathlineto{\pgfqpoint{1.906401in}{1.707572in}}%
\pgfpathlineto{\pgfqpoint{1.911414in}{1.716639in}}%
\pgfpathlineto{\pgfqpoint{1.913921in}{1.719017in}}%
\pgfpathlineto{\pgfqpoint{1.916427in}{1.719686in}}%
\pgfpathlineto{\pgfqpoint{1.921441in}{1.730660in}}%
\pgfpathlineto{\pgfqpoint{1.931467in}{1.734017in}}%
\pgfpathlineto{\pgfqpoint{1.936481in}{1.735518in}}%
\pgfpathlineto{\pgfqpoint{1.938987in}{1.738435in}}%
\pgfpathlineto{\pgfqpoint{1.941494in}{1.738485in}}%
\pgfpathlineto{\pgfqpoint{1.944000in}{1.742574in}}%
\pgfpathlineto{\pgfqpoint{1.949014in}{1.743067in}}%
\pgfpathlineto{\pgfqpoint{1.956534in}{1.748762in}}%
\pgfpathlineto{\pgfqpoint{1.961547in}{1.751132in}}%
\pgfpathlineto{\pgfqpoint{1.966560in}{1.756976in}}%
\pgfpathlineto{\pgfqpoint{1.969067in}{1.758997in}}%
\pgfpathlineto{\pgfqpoint{1.971574in}{1.764290in}}%
\pgfpathlineto{\pgfqpoint{1.979094in}{1.765591in}}%
\pgfpathlineto{\pgfqpoint{1.984107in}{1.766510in}}%
\pgfpathlineto{\pgfqpoint{1.986614in}{1.766523in}}%
\pgfpathlineto{\pgfqpoint{1.991627in}{1.772317in}}%
\pgfpathlineto{\pgfqpoint{1.996640in}{1.774504in}}%
\pgfpathlineto{\pgfqpoint{1.999147in}{1.779271in}}%
\pgfpathlineto{\pgfqpoint{2.006667in}{1.780227in}}%
\pgfpathlineto{\pgfqpoint{2.011680in}{1.781907in}}%
\pgfpathlineto{\pgfqpoint{2.019200in}{1.782939in}}%
\pgfpathlineto{\pgfqpoint{2.026720in}{1.795030in}}%
\pgfpathlineto{\pgfqpoint{2.034240in}{1.795816in}}%
\pgfpathlineto{\pgfqpoint{2.039253in}{1.800092in}}%
\pgfpathlineto{\pgfqpoint{2.044267in}{1.801466in}}%
\pgfpathlineto{\pgfqpoint{2.049280in}{1.802001in}}%
\pgfpathlineto{\pgfqpoint{2.051787in}{1.806537in}}%
\pgfpathlineto{\pgfqpoint{2.054293in}{1.814224in}}%
\pgfpathlineto{\pgfqpoint{2.064320in}{1.821067in}}%
\pgfpathlineto{\pgfqpoint{2.081866in}{1.825220in}}%
\pgfpathlineto{\pgfqpoint{2.084373in}{1.827041in}}%
\pgfpathlineto{\pgfqpoint{2.089386in}{1.834028in}}%
\pgfpathlineto{\pgfqpoint{2.094400in}{1.836199in}}%
\pgfpathlineto{\pgfqpoint{2.101920in}{1.839655in}}%
\pgfpathlineto{\pgfqpoint{2.104426in}{1.846837in}}%
\pgfpathlineto{\pgfqpoint{2.106933in}{1.846854in}}%
\pgfpathlineto{\pgfqpoint{2.109440in}{1.851814in}}%
\pgfpathlineto{\pgfqpoint{2.111946in}{1.859620in}}%
\pgfpathlineto{\pgfqpoint{2.119466in}{1.865528in}}%
\pgfpathlineto{\pgfqpoint{2.121973in}{1.869023in}}%
\pgfpathlineto{\pgfqpoint{2.124480in}{1.875060in}}%
\pgfpathlineto{\pgfqpoint{2.126986in}{1.875211in}}%
\pgfpathlineto{\pgfqpoint{2.137013in}{1.879715in}}%
\pgfpathlineto{\pgfqpoint{2.139520in}{1.880030in}}%
\pgfpathlineto{\pgfqpoint{2.142026in}{1.881624in}}%
\pgfpathlineto{\pgfqpoint{2.149546in}{1.895155in}}%
\pgfpathlineto{\pgfqpoint{2.159573in}{1.902269in}}%
\pgfpathlineto{\pgfqpoint{2.162079in}{1.906587in}}%
\pgfpathlineto{\pgfqpoint{2.164586in}{1.907518in}}%
\pgfpathlineto{\pgfqpoint{2.167093in}{1.916564in}}%
\pgfpathlineto{\pgfqpoint{2.169599in}{1.918424in}}%
\pgfpathlineto{\pgfqpoint{2.177119in}{1.920061in}}%
\pgfpathlineto{\pgfqpoint{2.187146in}{1.925422in}}%
\pgfpathlineto{\pgfqpoint{2.189653in}{1.927020in}}%
\pgfpathlineto{\pgfqpoint{2.194666in}{1.928016in}}%
\pgfpathlineto{\pgfqpoint{2.197173in}{1.932915in}}%
\pgfpathlineto{\pgfqpoint{2.209706in}{1.934971in}}%
\pgfpathlineto{\pgfqpoint{2.212212in}{1.935561in}}%
\pgfpathlineto{\pgfqpoint{2.219732in}{1.941733in}}%
\pgfpathlineto{\pgfqpoint{2.222239in}{1.942022in}}%
\pgfpathlineto{\pgfqpoint{2.224746in}{1.949368in}}%
\pgfpathlineto{\pgfqpoint{2.227252in}{1.950592in}}%
\pgfpathlineto{\pgfqpoint{2.229759in}{1.953292in}}%
\pgfpathlineto{\pgfqpoint{2.232266in}{1.953724in}}%
\pgfpathlineto{\pgfqpoint{2.237279in}{1.960613in}}%
\pgfpathlineto{\pgfqpoint{2.242292in}{1.961817in}}%
\pgfpathlineto{\pgfqpoint{2.247306in}{1.979325in}}%
\pgfpathlineto{\pgfqpoint{2.252319in}{1.986143in}}%
\pgfpathlineto{\pgfqpoint{2.254826in}{1.986214in}}%
\pgfpathlineto{\pgfqpoint{2.257332in}{1.991081in}}%
\pgfpathlineto{\pgfqpoint{2.259839in}{1.993310in}}%
\pgfpathlineto{\pgfqpoint{2.267359in}{1.994342in}}%
\pgfpathlineto{\pgfqpoint{2.277385in}{1.996183in}}%
\pgfpathlineto{\pgfqpoint{2.287412in}{2.000450in}}%
\pgfpathlineto{\pgfqpoint{2.294932in}{2.002653in}}%
\pgfpathlineto{\pgfqpoint{2.297439in}{2.005431in}}%
\pgfpathlineto{\pgfqpoint{2.299945in}{2.005494in}}%
\pgfpathlineto{\pgfqpoint{2.307465in}{2.009484in}}%
\pgfpathlineto{\pgfqpoint{2.309972in}{2.011579in}}%
\pgfpathlineto{\pgfqpoint{2.312479in}{2.016696in}}%
\pgfpathlineto{\pgfqpoint{2.314985in}{2.017208in}}%
\pgfpathlineto{\pgfqpoint{2.317492in}{2.020279in}}%
\pgfpathlineto{\pgfqpoint{2.319999in}{2.021111in}}%
\pgfpathlineto{\pgfqpoint{2.322505in}{2.032989in}}%
\pgfpathlineto{\pgfqpoint{2.325012in}{2.033251in}}%
\pgfpathlineto{\pgfqpoint{2.327519in}{2.034707in}}%
\pgfpathlineto{\pgfqpoint{2.330025in}{2.042544in}}%
\pgfpathlineto{\pgfqpoint{2.342559in}{2.047236in}}%
\pgfpathlineto{\pgfqpoint{2.345065in}{2.047751in}}%
\pgfpathlineto{\pgfqpoint{2.352585in}{2.054220in}}%
\pgfpathlineto{\pgfqpoint{2.355092in}{2.060447in}}%
\pgfpathlineto{\pgfqpoint{2.357598in}{2.062932in}}%
\pgfpathlineto{\pgfqpoint{2.362612in}{2.064859in}}%
\pgfpathlineto{\pgfqpoint{2.365118in}{2.069975in}}%
\pgfpathlineto{\pgfqpoint{2.367625in}{2.078419in}}%
\pgfpathlineto{\pgfqpoint{2.370132in}{2.078489in}}%
\pgfpathlineto{\pgfqpoint{2.377652in}{2.082714in}}%
\pgfpathlineto{\pgfqpoint{2.390185in}{2.088274in}}%
\pgfpathlineto{\pgfqpoint{2.392692in}{2.095207in}}%
\pgfpathlineto{\pgfqpoint{2.395198in}{2.098613in}}%
\pgfpathlineto{\pgfqpoint{2.400212in}{2.099587in}}%
\pgfpathlineto{\pgfqpoint{2.402718in}{2.102833in}}%
\pgfpathlineto{\pgfqpoint{2.405225in}{2.103198in}}%
\pgfpathlineto{\pgfqpoint{2.407732in}{2.104782in}}%
\pgfpathlineto{\pgfqpoint{2.410238in}{2.108704in}}%
\pgfpathlineto{\pgfqpoint{2.412745in}{2.116709in}}%
\pgfpathlineto{\pgfqpoint{2.415251in}{2.119029in}}%
\pgfpathlineto{\pgfqpoint{2.420265in}{2.120870in}}%
\pgfpathlineto{\pgfqpoint{2.422771in}{2.133156in}}%
\pgfpathlineto{\pgfqpoint{2.425278in}{2.137142in}}%
\pgfpathlineto{\pgfqpoint{2.427785in}{2.138752in}}%
\pgfpathlineto{\pgfqpoint{2.430291in}{2.138840in}}%
\pgfpathlineto{\pgfqpoint{2.432798in}{2.142155in}}%
\pgfpathlineto{\pgfqpoint{2.437811in}{2.143564in}}%
\pgfpathlineto{\pgfqpoint{2.440318in}{2.148830in}}%
\pgfpathlineto{\pgfqpoint{2.447838in}{2.151255in}}%
\pgfpathlineto{\pgfqpoint{2.450345in}{2.153324in}}%
\pgfpathlineto{\pgfqpoint{2.455358in}{2.154093in}}%
\pgfpathlineto{\pgfqpoint{2.465385in}{2.162079in}}%
\pgfpathlineto{\pgfqpoint{2.467891in}{2.164831in}}%
\pgfpathlineto{\pgfqpoint{2.472905in}{2.184686in}}%
\pgfpathlineto{\pgfqpoint{2.477918in}{2.188399in}}%
\pgfpathlineto{\pgfqpoint{2.480425in}{2.188738in}}%
\pgfpathlineto{\pgfqpoint{2.482931in}{2.191576in}}%
\pgfpathlineto{\pgfqpoint{2.485438in}{2.198853in}}%
\pgfpathlineto{\pgfqpoint{2.495464in}{2.209493in}}%
\pgfpathlineto{\pgfqpoint{2.497971in}{2.237690in}}%
\pgfpathlineto{\pgfqpoint{2.500478in}{2.244352in}}%
\pgfpathlineto{\pgfqpoint{2.502984in}{2.247393in}}%
\pgfpathlineto{\pgfqpoint{2.505491in}{2.253235in}}%
\pgfpathlineto{\pgfqpoint{2.507998in}{2.253988in}}%
\pgfpathlineto{\pgfqpoint{2.513011in}{2.260258in}}%
\pgfpathlineto{\pgfqpoint{2.518024in}{2.266140in}}%
\pgfpathlineto{\pgfqpoint{2.520531in}{2.285994in}}%
\pgfpathlineto{\pgfqpoint{2.528051in}{2.287527in}}%
\pgfpathlineto{\pgfqpoint{2.530558in}{2.305275in}}%
\pgfpathlineto{\pgfqpoint{3.292581in}{2.306372in}}%
\pgfpathlineto{\pgfqpoint{3.322660in}{2.306694in}}%
\pgfpathlineto{\pgfqpoint{3.397860in}{2.307778in}}%
\pgfpathlineto{\pgfqpoint{3.488100in}{2.311573in}}%
\pgfpathlineto{\pgfqpoint{3.513166in}{2.314030in}}%
\pgfpathlineto{\pgfqpoint{3.576921in}{2.315275in}}%
\pgfpathlineto{\pgfqpoint{3.576921in}{2.315275in}}%
\pgfusepath{stroke}%
\end{pgfscope}%
\begin{pgfscope}%
\pgfpathrectangle{\pgfqpoint{0.708220in}{0.535823in}}{\pgfqpoint{5.013309in}{1.769453in}}%
\pgfusepath{clip}%
\pgfsetbuttcap%
\pgfsetroundjoin%
\pgfsetlinewidth{1.003750pt}%
\definecolor{currentstroke}{rgb}{0.000000,0.501961,0.000000}%
\pgfsetstrokecolor{currentstroke}%
\pgfsetdash{{1.000000pt}{1.650000pt}}{0.000000pt}%
\pgfpathmoveto{\pgfqpoint{0.708220in}{0.633641in}}%
\pgfpathlineto{\pgfqpoint{0.713233in}{0.663878in}}%
\pgfpathlineto{\pgfqpoint{0.715740in}{0.710153in}}%
\pgfpathlineto{\pgfqpoint{0.718246in}{0.717904in}}%
\pgfpathlineto{\pgfqpoint{0.720753in}{0.729608in}}%
\pgfpathlineto{\pgfqpoint{0.723260in}{0.759376in}}%
\pgfpathlineto{\pgfqpoint{0.728273in}{0.771595in}}%
\pgfpathlineto{\pgfqpoint{0.730780in}{0.772358in}}%
\pgfpathlineto{\pgfqpoint{0.733286in}{0.777053in}}%
\pgfpathlineto{\pgfqpoint{0.738300in}{0.779288in}}%
\pgfpathlineto{\pgfqpoint{0.740806in}{0.786929in}}%
\pgfpathlineto{\pgfqpoint{0.743313in}{0.787435in}}%
\pgfpathlineto{\pgfqpoint{0.748326in}{0.795380in}}%
\pgfpathlineto{\pgfqpoint{0.750833in}{0.795572in}}%
\pgfpathlineto{\pgfqpoint{0.753340in}{0.796956in}}%
\pgfpathlineto{\pgfqpoint{0.755846in}{0.800236in}}%
\pgfpathlineto{\pgfqpoint{0.758353in}{0.810851in}}%
\pgfpathlineto{\pgfqpoint{0.760860in}{0.811014in}}%
\pgfpathlineto{\pgfqpoint{0.765873in}{0.816470in}}%
\pgfpathlineto{\pgfqpoint{0.768380in}{0.823976in}}%
\pgfpathlineto{\pgfqpoint{0.770886in}{0.825995in}}%
\pgfpathlineto{\pgfqpoint{0.773393in}{0.826424in}}%
\pgfpathlineto{\pgfqpoint{0.775900in}{0.830131in}}%
\pgfpathlineto{\pgfqpoint{0.778406in}{0.835905in}}%
\pgfpathlineto{\pgfqpoint{0.785926in}{0.843517in}}%
\pgfpathlineto{\pgfqpoint{0.790939in}{0.849619in}}%
\pgfpathlineto{\pgfqpoint{0.793446in}{0.849861in}}%
\pgfpathlineto{\pgfqpoint{0.798459in}{0.853868in}}%
\pgfpathlineto{\pgfqpoint{0.800966in}{0.854337in}}%
\pgfpathlineto{\pgfqpoint{0.803473in}{0.860058in}}%
\pgfpathlineto{\pgfqpoint{0.805979in}{0.861684in}}%
\pgfpathlineto{\pgfqpoint{0.808486in}{0.864798in}}%
\pgfpathlineto{\pgfqpoint{0.818513in}{0.869903in}}%
\pgfpathlineto{\pgfqpoint{0.826033in}{0.876649in}}%
\pgfpathlineto{\pgfqpoint{0.828539in}{0.881623in}}%
\pgfpathlineto{\pgfqpoint{0.833553in}{0.884960in}}%
\pgfpathlineto{\pgfqpoint{0.836059in}{0.885640in}}%
\pgfpathlineto{\pgfqpoint{0.843579in}{0.892788in}}%
\pgfpathlineto{\pgfqpoint{0.851099in}{0.897571in}}%
\pgfpathlineto{\pgfqpoint{0.853606in}{0.897696in}}%
\pgfpathlineto{\pgfqpoint{0.856112in}{0.900186in}}%
\pgfpathlineto{\pgfqpoint{0.858619in}{0.905653in}}%
\pgfpathlineto{\pgfqpoint{0.863632in}{0.907111in}}%
\pgfpathlineto{\pgfqpoint{0.866139in}{0.914570in}}%
\pgfpathlineto{\pgfqpoint{0.871152in}{0.918014in}}%
\pgfpathlineto{\pgfqpoint{0.873659in}{0.920875in}}%
\pgfpathlineto{\pgfqpoint{0.876166in}{0.921263in}}%
\pgfpathlineto{\pgfqpoint{0.878672in}{0.928658in}}%
\pgfpathlineto{\pgfqpoint{0.883686in}{0.929193in}}%
\pgfpathlineto{\pgfqpoint{0.886192in}{0.933329in}}%
\pgfpathlineto{\pgfqpoint{0.891206in}{0.937977in}}%
\pgfpathlineto{\pgfqpoint{0.896219in}{0.945966in}}%
\pgfpathlineto{\pgfqpoint{0.898726in}{0.946976in}}%
\pgfpathlineto{\pgfqpoint{0.903739in}{0.953389in}}%
\pgfpathlineto{\pgfqpoint{0.906246in}{0.954480in}}%
\pgfpathlineto{\pgfqpoint{0.908752in}{0.958220in}}%
\pgfpathlineto{\pgfqpoint{0.913766in}{0.959500in}}%
\pgfpathlineto{\pgfqpoint{0.916272in}{0.963676in}}%
\pgfpathlineto{\pgfqpoint{0.918779in}{0.964164in}}%
\pgfpathlineto{\pgfqpoint{0.921285in}{0.971297in}}%
\pgfpathlineto{\pgfqpoint{0.936325in}{0.976278in}}%
\pgfpathlineto{\pgfqpoint{0.938832in}{0.978973in}}%
\pgfpathlineto{\pgfqpoint{0.943845in}{0.980856in}}%
\pgfpathlineto{\pgfqpoint{0.946352in}{0.983819in}}%
\pgfpathlineto{\pgfqpoint{0.948859in}{0.983853in}}%
\pgfpathlineto{\pgfqpoint{0.951365in}{0.986501in}}%
\pgfpathlineto{\pgfqpoint{0.953872in}{0.991060in}}%
\pgfpathlineto{\pgfqpoint{0.956379in}{1.000286in}}%
\pgfpathlineto{\pgfqpoint{0.958885in}{1.001069in}}%
\pgfpathlineto{\pgfqpoint{0.961392in}{1.003774in}}%
\pgfpathlineto{\pgfqpoint{0.966405in}{1.005230in}}%
\pgfpathlineto{\pgfqpoint{0.973925in}{1.007482in}}%
\pgfpathlineto{\pgfqpoint{0.978939in}{1.012665in}}%
\pgfpathlineto{\pgfqpoint{0.981445in}{1.012676in}}%
\pgfpathlineto{\pgfqpoint{0.983952in}{1.018670in}}%
\pgfpathlineto{\pgfqpoint{0.986458in}{1.018957in}}%
\pgfpathlineto{\pgfqpoint{0.988965in}{1.020866in}}%
\pgfpathlineto{\pgfqpoint{0.991472in}{1.024644in}}%
\pgfpathlineto{\pgfqpoint{0.996485in}{1.025457in}}%
\pgfpathlineto{\pgfqpoint{1.001498in}{1.028746in}}%
\pgfpathlineto{\pgfqpoint{1.006512in}{1.031422in}}%
\pgfpathlineto{\pgfqpoint{1.016538in}{1.040329in}}%
\pgfpathlineto{\pgfqpoint{1.024058in}{1.042883in}}%
\pgfpathlineto{\pgfqpoint{1.029072in}{1.043329in}}%
\pgfpathlineto{\pgfqpoint{1.036592in}{1.049982in}}%
\pgfpathlineto{\pgfqpoint{1.041605in}{1.052321in}}%
\pgfpathlineto{\pgfqpoint{1.044112in}{1.058451in}}%
\pgfpathlineto{\pgfqpoint{1.049125in}{1.059854in}}%
\pgfpathlineto{\pgfqpoint{1.051632in}{1.064503in}}%
\pgfpathlineto{\pgfqpoint{1.054138in}{1.065319in}}%
\pgfpathlineto{\pgfqpoint{1.061658in}{1.071964in}}%
\pgfpathlineto{\pgfqpoint{1.066671in}{1.072872in}}%
\pgfpathlineto{\pgfqpoint{1.069178in}{1.074828in}}%
\pgfpathlineto{\pgfqpoint{1.074191in}{1.087119in}}%
\pgfpathlineto{\pgfqpoint{1.076698in}{1.087994in}}%
\pgfpathlineto{\pgfqpoint{1.084218in}{1.097332in}}%
\pgfpathlineto{\pgfqpoint{1.091738in}{1.100701in}}%
\pgfpathlineto{\pgfqpoint{1.096751in}{1.109874in}}%
\pgfpathlineto{\pgfqpoint{1.101765in}{1.111842in}}%
\pgfpathlineto{\pgfqpoint{1.104271in}{1.114554in}}%
\pgfpathlineto{\pgfqpoint{1.109285in}{1.115421in}}%
\pgfpathlineto{\pgfqpoint{1.114298in}{1.117062in}}%
\pgfpathlineto{\pgfqpoint{1.116805in}{1.120966in}}%
\pgfpathlineto{\pgfqpoint{1.119311in}{1.122250in}}%
\pgfpathlineto{\pgfqpoint{1.121818in}{1.126625in}}%
\pgfpathlineto{\pgfqpoint{1.126831in}{1.127540in}}%
\pgfpathlineto{\pgfqpoint{1.129338in}{1.132804in}}%
\pgfpathlineto{\pgfqpoint{1.134351in}{1.136436in}}%
\pgfpathlineto{\pgfqpoint{1.136858in}{1.141749in}}%
\pgfpathlineto{\pgfqpoint{1.141871in}{1.142932in}}%
\pgfpathlineto{\pgfqpoint{1.144378in}{1.145675in}}%
\pgfpathlineto{\pgfqpoint{1.146884in}{1.150456in}}%
\pgfpathlineto{\pgfqpoint{1.159418in}{1.159368in}}%
\pgfpathlineto{\pgfqpoint{1.164431in}{1.165892in}}%
\pgfpathlineto{\pgfqpoint{1.166938in}{1.166519in}}%
\pgfpathlineto{\pgfqpoint{1.169444in}{1.168854in}}%
\pgfpathlineto{\pgfqpoint{1.171951in}{1.177667in}}%
\pgfpathlineto{\pgfqpoint{1.179471in}{1.180382in}}%
\pgfpathlineto{\pgfqpoint{1.192004in}{1.182009in}}%
\pgfpathlineto{\pgfqpoint{1.199524in}{1.185518in}}%
\pgfpathlineto{\pgfqpoint{1.202031in}{1.194271in}}%
\pgfpathlineto{\pgfqpoint{1.207044in}{1.194843in}}%
\pgfpathlineto{\pgfqpoint{1.209551in}{1.195829in}}%
\pgfpathlineto{\pgfqpoint{1.212057in}{1.198065in}}%
\pgfpathlineto{\pgfqpoint{1.227097in}{1.201473in}}%
\pgfpathlineto{\pgfqpoint{1.229604in}{1.211809in}}%
\pgfpathlineto{\pgfqpoint{1.244644in}{1.218594in}}%
\pgfpathlineto{\pgfqpoint{1.249657in}{1.219742in}}%
\pgfpathlineto{\pgfqpoint{1.252164in}{1.222944in}}%
\pgfpathlineto{\pgfqpoint{1.254671in}{1.229459in}}%
\pgfpathlineto{\pgfqpoint{1.257177in}{1.229606in}}%
\pgfpathlineto{\pgfqpoint{1.264697in}{1.232706in}}%
\pgfpathlineto{\pgfqpoint{1.267204in}{1.232966in}}%
\pgfpathlineto{\pgfqpoint{1.269710in}{1.240527in}}%
\pgfpathlineto{\pgfqpoint{1.272217in}{1.242267in}}%
\pgfpathlineto{\pgfqpoint{1.279737in}{1.243038in}}%
\pgfpathlineto{\pgfqpoint{1.284750in}{1.245247in}}%
\pgfpathlineto{\pgfqpoint{1.294777in}{1.248125in}}%
\pgfpathlineto{\pgfqpoint{1.297284in}{1.251939in}}%
\pgfpathlineto{\pgfqpoint{1.304804in}{1.254334in}}%
\pgfpathlineto{\pgfqpoint{1.307310in}{1.257990in}}%
\pgfpathlineto{\pgfqpoint{1.314830in}{1.260066in}}%
\pgfpathlineto{\pgfqpoint{1.317337in}{1.262275in}}%
\pgfpathlineto{\pgfqpoint{1.319844in}{1.269337in}}%
\pgfpathlineto{\pgfqpoint{1.322350in}{1.270455in}}%
\pgfpathlineto{\pgfqpoint{1.324857in}{1.276419in}}%
\pgfpathlineto{\pgfqpoint{1.332377in}{1.285216in}}%
\pgfpathlineto{\pgfqpoint{1.334883in}{1.291562in}}%
\pgfpathlineto{\pgfqpoint{1.339897in}{1.293018in}}%
\pgfpathlineto{\pgfqpoint{1.344910in}{1.293453in}}%
\pgfpathlineto{\pgfqpoint{1.359950in}{1.300051in}}%
\pgfpathlineto{\pgfqpoint{1.362457in}{1.304104in}}%
\pgfpathlineto{\pgfqpoint{1.367470in}{1.307150in}}%
\pgfpathlineto{\pgfqpoint{1.372483in}{1.312889in}}%
\pgfpathlineto{\pgfqpoint{1.395043in}{1.327620in}}%
\pgfpathlineto{\pgfqpoint{1.397550in}{1.330987in}}%
\pgfpathlineto{\pgfqpoint{1.400056in}{1.332404in}}%
\pgfpathlineto{\pgfqpoint{1.402563in}{1.341588in}}%
\pgfpathlineto{\pgfqpoint{1.405070in}{1.345397in}}%
\pgfpathlineto{\pgfqpoint{1.410083in}{1.346971in}}%
\pgfpathlineto{\pgfqpoint{1.412590in}{1.350331in}}%
\pgfpathlineto{\pgfqpoint{1.440163in}{1.361043in}}%
\pgfpathlineto{\pgfqpoint{1.442670in}{1.366573in}}%
\pgfpathlineto{\pgfqpoint{1.462723in}{1.371198in}}%
\pgfpathlineto{\pgfqpoint{1.470243in}{1.372311in}}%
\pgfpathlineto{\pgfqpoint{1.472749in}{1.373805in}}%
\pgfpathlineto{\pgfqpoint{1.477763in}{1.374810in}}%
\pgfpathlineto{\pgfqpoint{1.485283in}{1.377631in}}%
\pgfpathlineto{\pgfqpoint{1.492803in}{1.378773in}}%
\pgfpathlineto{\pgfqpoint{1.497816in}{1.379792in}}%
\pgfpathlineto{\pgfqpoint{1.500323in}{1.384896in}}%
\pgfpathlineto{\pgfqpoint{1.502829in}{1.385704in}}%
\pgfpathlineto{\pgfqpoint{1.510349in}{1.393855in}}%
\pgfpathlineto{\pgfqpoint{1.515363in}{1.394952in}}%
\pgfpathlineto{\pgfqpoint{1.517869in}{1.399001in}}%
\pgfpathlineto{\pgfqpoint{1.527896in}{1.403354in}}%
\pgfpathlineto{\pgfqpoint{1.532909in}{1.407863in}}%
\pgfpathlineto{\pgfqpoint{1.535416in}{1.407864in}}%
\pgfpathlineto{\pgfqpoint{1.540429in}{1.412644in}}%
\pgfpathlineto{\pgfqpoint{1.542936in}{1.412972in}}%
\pgfpathlineto{\pgfqpoint{1.545442in}{1.418310in}}%
\pgfpathlineto{\pgfqpoint{1.550456in}{1.421159in}}%
\pgfpathlineto{\pgfqpoint{1.555469in}{1.422756in}}%
\pgfpathlineto{\pgfqpoint{1.557976in}{1.427441in}}%
\pgfpathlineto{\pgfqpoint{1.562989in}{1.429035in}}%
\pgfpathlineto{\pgfqpoint{1.565496in}{1.429553in}}%
\pgfpathlineto{\pgfqpoint{1.570509in}{1.434339in}}%
\pgfpathlineto{\pgfqpoint{1.588056in}{1.444121in}}%
\pgfpathlineto{\pgfqpoint{1.590562in}{1.448840in}}%
\pgfpathlineto{\pgfqpoint{1.593069in}{1.450218in}}%
\pgfpathlineto{\pgfqpoint{1.600589in}{1.460927in}}%
\pgfpathlineto{\pgfqpoint{1.605602in}{1.461933in}}%
\pgfpathlineto{\pgfqpoint{1.608109in}{1.464776in}}%
\pgfpathlineto{\pgfqpoint{1.610615in}{1.465133in}}%
\pgfpathlineto{\pgfqpoint{1.613122in}{1.466731in}}%
\pgfpathlineto{\pgfqpoint{1.615629in}{1.466736in}}%
\pgfpathlineto{\pgfqpoint{1.620642in}{1.470610in}}%
\pgfpathlineto{\pgfqpoint{1.623149in}{1.474778in}}%
\pgfpathlineto{\pgfqpoint{1.628162in}{1.475677in}}%
\pgfpathlineto{\pgfqpoint{1.630669in}{1.479771in}}%
\pgfpathlineto{\pgfqpoint{1.633175in}{1.491666in}}%
\pgfpathlineto{\pgfqpoint{1.640695in}{1.499634in}}%
\pgfpathlineto{\pgfqpoint{1.643202in}{1.507692in}}%
\pgfpathlineto{\pgfqpoint{1.648215in}{1.509725in}}%
\pgfpathlineto{\pgfqpoint{1.650722in}{1.510372in}}%
\pgfpathlineto{\pgfqpoint{1.653229in}{1.516767in}}%
\pgfpathlineto{\pgfqpoint{1.658242in}{1.519114in}}%
\pgfpathlineto{\pgfqpoint{1.668268in}{1.535716in}}%
\pgfpathlineto{\pgfqpoint{1.670775in}{1.536872in}}%
\pgfpathlineto{\pgfqpoint{1.673282in}{1.544695in}}%
\pgfpathlineto{\pgfqpoint{1.675788in}{1.547853in}}%
\pgfpathlineto{\pgfqpoint{1.680802in}{1.548661in}}%
\pgfpathlineto{\pgfqpoint{1.685815in}{1.555814in}}%
\pgfpathlineto{\pgfqpoint{1.688322in}{1.562914in}}%
\pgfpathlineto{\pgfqpoint{1.690828in}{1.562925in}}%
\pgfpathlineto{\pgfqpoint{1.693335in}{1.566788in}}%
\pgfpathlineto{\pgfqpoint{1.698348in}{1.569895in}}%
\pgfpathlineto{\pgfqpoint{1.703362in}{1.577882in}}%
\pgfpathlineto{\pgfqpoint{1.705868in}{1.577963in}}%
\pgfpathlineto{\pgfqpoint{1.713388in}{1.584950in}}%
\pgfpathlineto{\pgfqpoint{1.715895in}{1.588610in}}%
\pgfpathlineto{\pgfqpoint{1.723415in}{1.590127in}}%
\pgfpathlineto{\pgfqpoint{1.733442in}{1.592172in}}%
\pgfpathlineto{\pgfqpoint{1.748481in}{1.597616in}}%
\pgfpathlineto{\pgfqpoint{1.758508in}{1.598457in}}%
\pgfpathlineto{\pgfqpoint{1.763521in}{1.602199in}}%
\pgfpathlineto{\pgfqpoint{1.768535in}{1.612501in}}%
\pgfpathlineto{\pgfqpoint{1.771041in}{1.614135in}}%
\pgfpathlineto{\pgfqpoint{1.778561in}{1.614875in}}%
\pgfpathlineto{\pgfqpoint{1.798615in}{1.623919in}}%
\pgfpathlineto{\pgfqpoint{1.808641in}{1.626104in}}%
\pgfpathlineto{\pgfqpoint{1.816161in}{1.627914in}}%
\pgfpathlineto{\pgfqpoint{1.826188in}{1.629044in}}%
\pgfpathlineto{\pgfqpoint{1.828694in}{1.631022in}}%
\pgfpathlineto{\pgfqpoint{1.831201in}{1.631070in}}%
\pgfpathlineto{\pgfqpoint{1.833708in}{1.633192in}}%
\pgfpathlineto{\pgfqpoint{1.843734in}{1.635397in}}%
\pgfpathlineto{\pgfqpoint{1.848748in}{1.640406in}}%
\pgfpathlineto{\pgfqpoint{1.851254in}{1.641235in}}%
\pgfpathlineto{\pgfqpoint{1.856268in}{1.649863in}}%
\pgfpathlineto{\pgfqpoint{1.858774in}{1.650274in}}%
\pgfpathlineto{\pgfqpoint{1.861281in}{1.651988in}}%
\pgfpathlineto{\pgfqpoint{1.866294in}{1.653447in}}%
\pgfpathlineto{\pgfqpoint{1.871307in}{1.654979in}}%
\pgfpathlineto{\pgfqpoint{1.881334in}{1.655903in}}%
\pgfpathlineto{\pgfqpoint{1.888854in}{1.657851in}}%
\pgfpathlineto{\pgfqpoint{1.896374in}{1.666588in}}%
\pgfpathlineto{\pgfqpoint{1.903894in}{1.667192in}}%
\pgfpathlineto{\pgfqpoint{1.916427in}{1.670855in}}%
\pgfpathlineto{\pgfqpoint{1.923947in}{1.671805in}}%
\pgfpathlineto{\pgfqpoint{1.928961in}{1.674294in}}%
\pgfpathlineto{\pgfqpoint{1.933974in}{1.674522in}}%
\pgfpathlineto{\pgfqpoint{1.938987in}{1.676627in}}%
\pgfpathlineto{\pgfqpoint{1.941494in}{1.677505in}}%
\pgfpathlineto{\pgfqpoint{1.946507in}{1.682522in}}%
\pgfpathlineto{\pgfqpoint{1.949014in}{1.687880in}}%
\pgfpathlineto{\pgfqpoint{1.954027in}{1.688211in}}%
\pgfpathlineto{\pgfqpoint{1.969067in}{1.694960in}}%
\pgfpathlineto{\pgfqpoint{1.971574in}{1.694998in}}%
\pgfpathlineto{\pgfqpoint{1.974080in}{1.700569in}}%
\pgfpathlineto{\pgfqpoint{1.981600in}{1.702102in}}%
\pgfpathlineto{\pgfqpoint{1.984107in}{1.704115in}}%
\pgfpathlineto{\pgfqpoint{1.991627in}{1.705375in}}%
\pgfpathlineto{\pgfqpoint{1.996640in}{1.707572in}}%
\pgfpathlineto{\pgfqpoint{2.004160in}{1.716639in}}%
\pgfpathlineto{\pgfqpoint{2.011680in}{1.723602in}}%
\pgfpathlineto{\pgfqpoint{2.014187in}{1.724579in}}%
\pgfpathlineto{\pgfqpoint{2.016693in}{1.730660in}}%
\pgfpathlineto{\pgfqpoint{2.026720in}{1.734737in}}%
\pgfpathlineto{\pgfqpoint{2.029227in}{1.738435in}}%
\pgfpathlineto{\pgfqpoint{2.031733in}{1.738485in}}%
\pgfpathlineto{\pgfqpoint{2.034240in}{1.742574in}}%
\pgfpathlineto{\pgfqpoint{2.039253in}{1.743067in}}%
\pgfpathlineto{\pgfqpoint{2.041760in}{1.744950in}}%
\pgfpathlineto{\pgfqpoint{2.044267in}{1.745353in}}%
\pgfpathlineto{\pgfqpoint{2.046773in}{1.748145in}}%
\pgfpathlineto{\pgfqpoint{2.051787in}{1.748762in}}%
\pgfpathlineto{\pgfqpoint{2.054293in}{1.754628in}}%
\pgfpathlineto{\pgfqpoint{2.056800in}{1.756976in}}%
\pgfpathlineto{\pgfqpoint{2.061813in}{1.758359in}}%
\pgfpathlineto{\pgfqpoint{2.064320in}{1.759566in}}%
\pgfpathlineto{\pgfqpoint{2.066827in}{1.762602in}}%
\pgfpathlineto{\pgfqpoint{2.074346in}{1.764804in}}%
\pgfpathlineto{\pgfqpoint{2.076853in}{1.765057in}}%
\pgfpathlineto{\pgfqpoint{2.081866in}{1.768816in}}%
\pgfpathlineto{\pgfqpoint{2.084373in}{1.772317in}}%
\pgfpathlineto{\pgfqpoint{2.089386in}{1.774504in}}%
\pgfpathlineto{\pgfqpoint{2.091893in}{1.777677in}}%
\pgfpathlineto{\pgfqpoint{2.109440in}{1.782939in}}%
\pgfpathlineto{\pgfqpoint{2.111946in}{1.794226in}}%
\pgfpathlineto{\pgfqpoint{2.116960in}{1.795461in}}%
\pgfpathlineto{\pgfqpoint{2.119466in}{1.795816in}}%
\pgfpathlineto{\pgfqpoint{2.121973in}{1.798775in}}%
\pgfpathlineto{\pgfqpoint{2.137013in}{1.801596in}}%
\pgfpathlineto{\pgfqpoint{2.139520in}{1.814224in}}%
\pgfpathlineto{\pgfqpoint{2.147039in}{1.819582in}}%
\pgfpathlineto{\pgfqpoint{2.154559in}{1.820911in}}%
\pgfpathlineto{\pgfqpoint{2.169599in}{1.825220in}}%
\pgfpathlineto{\pgfqpoint{2.174613in}{1.827041in}}%
\pgfpathlineto{\pgfqpoint{2.177119in}{1.827657in}}%
\pgfpathlineto{\pgfqpoint{2.187146in}{1.837557in}}%
\pgfpathlineto{\pgfqpoint{2.197173in}{1.843652in}}%
\pgfpathlineto{\pgfqpoint{2.209706in}{1.857884in}}%
\pgfpathlineto{\pgfqpoint{2.212212in}{1.864690in}}%
\pgfpathlineto{\pgfqpoint{2.217226in}{1.868218in}}%
\pgfpathlineto{\pgfqpoint{2.219732in}{1.875211in}}%
\pgfpathlineto{\pgfqpoint{2.222239in}{1.875505in}}%
\pgfpathlineto{\pgfqpoint{2.224746in}{1.879715in}}%
\pgfpathlineto{\pgfqpoint{2.229759in}{1.880324in}}%
\pgfpathlineto{\pgfqpoint{2.232266in}{1.886798in}}%
\pgfpathlineto{\pgfqpoint{2.244799in}{1.896353in}}%
\pgfpathlineto{\pgfqpoint{2.247306in}{1.896655in}}%
\pgfpathlineto{\pgfqpoint{2.249812in}{1.906587in}}%
\pgfpathlineto{\pgfqpoint{2.252319in}{1.906622in}}%
\pgfpathlineto{\pgfqpoint{2.254826in}{1.918048in}}%
\pgfpathlineto{\pgfqpoint{2.262346in}{1.920061in}}%
\pgfpathlineto{\pgfqpoint{2.269866in}{1.922503in}}%
\pgfpathlineto{\pgfqpoint{2.272372in}{1.925633in}}%
\pgfpathlineto{\pgfqpoint{2.274879in}{1.932171in}}%
\pgfpathlineto{\pgfqpoint{2.282399in}{1.937763in}}%
\pgfpathlineto{\pgfqpoint{2.284905in}{1.938054in}}%
\pgfpathlineto{\pgfqpoint{2.287412in}{1.941733in}}%
\pgfpathlineto{\pgfqpoint{2.289919in}{1.949543in}}%
\pgfpathlineto{\pgfqpoint{2.292425in}{1.952260in}}%
\pgfpathlineto{\pgfqpoint{2.294932in}{1.952697in}}%
\pgfpathlineto{\pgfqpoint{2.299945in}{1.958628in}}%
\pgfpathlineto{\pgfqpoint{2.304959in}{1.960827in}}%
\pgfpathlineto{\pgfqpoint{2.307465in}{1.971907in}}%
\pgfpathlineto{\pgfqpoint{2.312479in}{1.983214in}}%
\pgfpathlineto{\pgfqpoint{2.317492in}{1.983828in}}%
\pgfpathlineto{\pgfqpoint{2.319999in}{1.990788in}}%
\pgfpathlineto{\pgfqpoint{2.345065in}{2.000450in}}%
\pgfpathlineto{\pgfqpoint{2.350078in}{2.000986in}}%
\pgfpathlineto{\pgfqpoint{2.355092in}{2.003655in}}%
\pgfpathlineto{\pgfqpoint{2.357598in}{2.003684in}}%
\pgfpathlineto{\pgfqpoint{2.360105in}{2.009484in}}%
\pgfpathlineto{\pgfqpoint{2.362612in}{2.010089in}}%
\pgfpathlineto{\pgfqpoint{2.367625in}{2.020279in}}%
\pgfpathlineto{\pgfqpoint{2.370132in}{2.020327in}}%
\pgfpathlineto{\pgfqpoint{2.372638in}{2.030374in}}%
\pgfpathlineto{\pgfqpoint{2.377652in}{2.033251in}}%
\pgfpathlineto{\pgfqpoint{2.380158in}{2.038484in}}%
\pgfpathlineto{\pgfqpoint{2.387678in}{2.040284in}}%
\pgfpathlineto{\pgfqpoint{2.390185in}{2.043925in}}%
\pgfpathlineto{\pgfqpoint{2.397705in}{2.045135in}}%
\pgfpathlineto{\pgfqpoint{2.400212in}{2.050396in}}%
\pgfpathlineto{\pgfqpoint{2.407732in}{2.053418in}}%
\pgfpathlineto{\pgfqpoint{2.412745in}{2.061353in}}%
\pgfpathlineto{\pgfqpoint{2.415251in}{2.063751in}}%
\pgfpathlineto{\pgfqpoint{2.420265in}{2.077515in}}%
\pgfpathlineto{\pgfqpoint{2.425278in}{2.078489in}}%
\pgfpathlineto{\pgfqpoint{2.427785in}{2.078531in}}%
\pgfpathlineto{\pgfqpoint{2.435305in}{2.083467in}}%
\pgfpathlineto{\pgfqpoint{2.440318in}{2.086381in}}%
\pgfpathlineto{\pgfqpoint{2.445331in}{2.095832in}}%
\pgfpathlineto{\pgfqpoint{2.447838in}{2.096043in}}%
\pgfpathlineto{\pgfqpoint{2.450345in}{2.097724in}}%
\pgfpathlineto{\pgfqpoint{2.457865in}{2.098708in}}%
\pgfpathlineto{\pgfqpoint{2.460371in}{2.101023in}}%
\pgfpathlineto{\pgfqpoint{2.467891in}{2.103181in}}%
\pgfpathlineto{\pgfqpoint{2.475411in}{2.107975in}}%
\pgfpathlineto{\pgfqpoint{2.477918in}{2.113505in}}%
\pgfpathlineto{\pgfqpoint{2.485438in}{2.120167in}}%
\pgfpathlineto{\pgfqpoint{2.487944in}{2.130011in}}%
\pgfpathlineto{\pgfqpoint{2.490451in}{2.130904in}}%
\pgfpathlineto{\pgfqpoint{2.492958in}{2.136828in}}%
\pgfpathlineto{\pgfqpoint{2.495464in}{2.138437in}}%
\pgfpathlineto{\pgfqpoint{2.500478in}{2.139376in}}%
\pgfpathlineto{\pgfqpoint{2.502984in}{2.140896in}}%
\pgfpathlineto{\pgfqpoint{2.505491in}{2.146155in}}%
\pgfpathlineto{\pgfqpoint{2.510504in}{2.146876in}}%
\pgfpathlineto{\pgfqpoint{2.515518in}{2.149894in}}%
\pgfpathlineto{\pgfqpoint{2.518024in}{2.150981in}}%
\pgfpathlineto{\pgfqpoint{2.520531in}{2.153324in}}%
\pgfpathlineto{\pgfqpoint{2.525544in}{2.154416in}}%
\pgfpathlineto{\pgfqpoint{2.535571in}{2.165251in}}%
\pgfpathlineto{\pgfqpoint{2.538078in}{2.184686in}}%
\pgfpathlineto{\pgfqpoint{2.543091in}{2.188227in}}%
\pgfpathlineto{\pgfqpoint{2.545598in}{2.188258in}}%
\pgfpathlineto{\pgfqpoint{2.548104in}{2.191102in}}%
\pgfpathlineto{\pgfqpoint{2.553117in}{2.193312in}}%
\pgfpathlineto{\pgfqpoint{2.558131in}{2.198361in}}%
\pgfpathlineto{\pgfqpoint{2.560637in}{2.198626in}}%
\pgfpathlineto{\pgfqpoint{2.563144in}{2.203163in}}%
\pgfpathlineto{\pgfqpoint{2.565651in}{2.205401in}}%
\pgfpathlineto{\pgfqpoint{2.568157in}{2.237690in}}%
\pgfpathlineto{\pgfqpoint{2.570664in}{2.243895in}}%
\pgfpathlineto{\pgfqpoint{2.573171in}{2.244181in}}%
\pgfpathlineto{\pgfqpoint{2.580691in}{2.254780in}}%
\pgfpathlineto{\pgfqpoint{2.585704in}{2.258572in}}%
\pgfpathlineto{\pgfqpoint{2.588211in}{2.259702in}}%
\pgfpathlineto{\pgfqpoint{2.590717in}{2.262641in}}%
\pgfpathlineto{\pgfqpoint{2.593224in}{2.263169in}}%
\pgfpathlineto{\pgfqpoint{2.595731in}{2.265694in}}%
\pgfpathlineto{\pgfqpoint{2.598237in}{2.278780in}}%
\pgfpathlineto{\pgfqpoint{2.600744in}{2.281968in}}%
\pgfpathlineto{\pgfqpoint{2.608264in}{2.283535in}}%
\pgfpathlineto{\pgfqpoint{2.610771in}{2.289417in}}%
\pgfpathlineto{\pgfqpoint{2.613277in}{2.303613in}}%
\pgfpathlineto{\pgfqpoint{2.618290in}{2.304668in}}%
\pgfpathlineto{\pgfqpoint{2.623304in}{2.305275in}}%
\pgfpathlineto{\pgfqpoint{5.658862in}{2.305275in}}%
\pgfpathlineto{\pgfqpoint{5.658862in}{2.305275in}}%
\pgfusepath{stroke}%
\end{pgfscope}%
\begin{pgfscope}%
\pgfpathrectangle{\pgfqpoint{0.708220in}{0.535823in}}{\pgfqpoint{5.013309in}{1.769453in}}%
\pgfusepath{clip}%
\pgfsetrectcap%
\pgfsetroundjoin%
\pgfsetlinewidth{1.003750pt}%
\definecolor{currentstroke}{rgb}{0.000000,0.000000,0.000000}%
\pgfsetstrokecolor{currentstroke}%
\pgfsetdash{}{0pt}%
\pgfpathmoveto{\pgfqpoint{0.708220in}{0.719784in}}%
\pgfpathlineto{\pgfqpoint{0.713233in}{0.719784in}}%
\pgfpathlineto{\pgfqpoint{0.715740in}{0.732111in}}%
\pgfpathlineto{\pgfqpoint{0.723260in}{0.732111in}}%
\pgfpathlineto{\pgfqpoint{0.725766in}{0.743642in}}%
\pgfpathlineto{\pgfqpoint{0.740806in}{0.743642in}}%
\pgfpathlineto{\pgfqpoint{0.743313in}{0.754474in}}%
\pgfpathlineto{\pgfqpoint{0.755846in}{0.754474in}}%
\pgfpathlineto{\pgfqpoint{0.760860in}{0.774347in}}%
\pgfpathlineto{\pgfqpoint{0.763366in}{0.774347in}}%
\pgfpathlineto{\pgfqpoint{0.765873in}{0.783511in}}%
\pgfpathlineto{\pgfqpoint{0.780913in}{0.783511in}}%
\pgfpathlineto{\pgfqpoint{0.783419in}{0.792229in}}%
\pgfpathlineto{\pgfqpoint{0.795953in}{0.792229in}}%
\pgfpathlineto{\pgfqpoint{0.798459in}{0.800540in}}%
\pgfpathlineto{\pgfqpoint{0.803473in}{0.800540in}}%
\pgfpathlineto{\pgfqpoint{0.805979in}{0.808483in}}%
\pgfpathlineto{\pgfqpoint{0.836059in}{0.808483in}}%
\pgfpathlineto{\pgfqpoint{0.838566in}{0.816087in}}%
\pgfpathlineto{\pgfqpoint{0.898726in}{0.816087in}}%
\pgfpathlineto{\pgfqpoint{0.901232in}{0.823380in}}%
\pgfpathlineto{\pgfqpoint{0.941339in}{0.823380in}}%
\pgfpathlineto{\pgfqpoint{0.943845in}{0.830388in}}%
\pgfpathlineto{\pgfqpoint{1.021552in}{0.830388in}}%
\pgfpathlineto{\pgfqpoint{1.024058in}{0.837131in}}%
\pgfpathlineto{\pgfqpoint{1.096751in}{0.837131in}}%
\pgfpathlineto{\pgfqpoint{1.099258in}{0.843629in}}%
\pgfpathlineto{\pgfqpoint{1.179471in}{0.843629in}}%
\pgfpathlineto{\pgfqpoint{1.181978in}{0.849898in}}%
\pgfpathlineto{\pgfqpoint{1.257177in}{0.849898in}}%
\pgfpathlineto{\pgfqpoint{1.259684in}{0.855956in}}%
\pgfpathlineto{\pgfqpoint{1.405070in}{0.855956in}}%
\pgfpathlineto{\pgfqpoint{1.407576in}{0.861814in}}%
\pgfpathlineto{\pgfqpoint{1.527896in}{0.861814in}}%
\pgfpathlineto{\pgfqpoint{1.530402in}{0.867487in}}%
\pgfpathlineto{\pgfqpoint{1.630669in}{0.867487in}}%
\pgfpathlineto{\pgfqpoint{1.633175in}{0.872985in}}%
\pgfpathlineto{\pgfqpoint{1.653229in}{0.872985in}}%
\pgfpathlineto{\pgfqpoint{1.655735in}{0.878319in}}%
\pgfpathlineto{\pgfqpoint{1.730935in}{0.878319in}}%
\pgfpathlineto{\pgfqpoint{1.733442in}{0.883498in}}%
\pgfpathlineto{\pgfqpoint{1.788588in}{0.883498in}}%
\pgfpathlineto{\pgfqpoint{1.791095in}{0.888531in}}%
\pgfpathlineto{\pgfqpoint{1.838721in}{0.888531in}}%
\pgfpathlineto{\pgfqpoint{1.841228in}{0.893426in}}%
\pgfpathlineto{\pgfqpoint{1.878827in}{0.893426in}}%
\pgfpathlineto{\pgfqpoint{1.881334in}{0.898191in}}%
\pgfpathlineto{\pgfqpoint{1.906401in}{0.898191in}}%
\pgfpathlineto{\pgfqpoint{1.908907in}{0.902832in}}%
\pgfpathlineto{\pgfqpoint{1.926454in}{0.902832in}}%
\pgfpathlineto{\pgfqpoint{1.931467in}{0.911768in}}%
\pgfpathlineto{\pgfqpoint{1.946507in}{0.911768in}}%
\pgfpathlineto{\pgfqpoint{1.949014in}{0.916073in}}%
\pgfpathlineto{\pgfqpoint{1.951520in}{0.916073in}}%
\pgfpathlineto{\pgfqpoint{1.956534in}{0.924385in}}%
\pgfpathlineto{\pgfqpoint{1.966560in}{0.924385in}}%
\pgfpathlineto{\pgfqpoint{1.969067in}{0.928400in}}%
\pgfpathlineto{\pgfqpoint{1.986614in}{0.928400in}}%
\pgfpathlineto{\pgfqpoint{1.989120in}{0.932327in}}%
\pgfpathlineto{\pgfqpoint{2.006667in}{0.932327in}}%
\pgfpathlineto{\pgfqpoint{2.009173in}{0.936169in}}%
\pgfpathlineto{\pgfqpoint{2.024213in}{0.936169in}}%
\pgfpathlineto{\pgfqpoint{2.026720in}{0.939931in}}%
\pgfpathlineto{\pgfqpoint{2.039253in}{0.939931in}}%
\pgfpathlineto{\pgfqpoint{2.041760in}{0.943615in}}%
\pgfpathlineto{\pgfqpoint{2.059307in}{0.943615in}}%
\pgfpathlineto{\pgfqpoint{2.061813in}{0.947225in}}%
\pgfpathlineto{\pgfqpoint{2.121973in}{0.947225in}}%
\pgfpathlineto{\pgfqpoint{2.124480in}{0.950763in}}%
\pgfpathlineto{\pgfqpoint{2.169599in}{0.950763in}}%
\pgfpathlineto{\pgfqpoint{2.172106in}{0.954232in}}%
\pgfpathlineto{\pgfqpoint{2.214719in}{0.954232in}}%
\pgfpathlineto{\pgfqpoint{2.217226in}{0.957636in}}%
\pgfpathlineto{\pgfqpoint{2.249812in}{0.957636in}}%
\pgfpathlineto{\pgfqpoint{2.252319in}{0.960975in}}%
\pgfpathlineto{\pgfqpoint{2.272372in}{0.960975in}}%
\pgfpathlineto{\pgfqpoint{2.274879in}{0.964254in}}%
\pgfpathlineto{\pgfqpoint{2.294932in}{0.964254in}}%
\pgfpathlineto{\pgfqpoint{2.297439in}{0.967473in}}%
\pgfpathlineto{\pgfqpoint{2.314985in}{0.967473in}}%
\pgfpathlineto{\pgfqpoint{2.317492in}{0.970635in}}%
\pgfpathlineto{\pgfqpoint{2.332532in}{0.970635in}}%
\pgfpathlineto{\pgfqpoint{2.335039in}{0.973743in}}%
\pgfpathlineto{\pgfqpoint{2.347572in}{0.973743in}}%
\pgfpathlineto{\pgfqpoint{2.350078in}{0.976797in}}%
\pgfpathlineto{\pgfqpoint{2.362612in}{0.976797in}}%
\pgfpathlineto{\pgfqpoint{2.365118in}{0.979800in}}%
\pgfpathlineto{\pgfqpoint{2.377652in}{0.979800in}}%
\pgfpathlineto{\pgfqpoint{2.380158in}{0.982753in}}%
\pgfpathlineto{\pgfqpoint{2.382665in}{0.982753in}}%
\pgfpathlineto{\pgfqpoint{2.385172in}{0.988517in}}%
\pgfpathlineto{\pgfqpoint{2.402718in}{0.988517in}}%
\pgfpathlineto{\pgfqpoint{2.405225in}{0.991331in}}%
\pgfpathlineto{\pgfqpoint{2.412745in}{0.991331in}}%
\pgfpathlineto{\pgfqpoint{2.415251in}{0.994101in}}%
\pgfpathlineto{\pgfqpoint{2.420265in}{0.994101in}}%
\pgfpathlineto{\pgfqpoint{2.422771in}{0.996829in}}%
\pgfpathlineto{\pgfqpoint{2.425278in}{0.996829in}}%
\pgfpathlineto{\pgfqpoint{2.427785in}{0.999516in}}%
\pgfpathlineto{\pgfqpoint{2.432798in}{0.999516in}}%
\pgfpathlineto{\pgfqpoint{2.435305in}{1.002163in}}%
\pgfpathlineto{\pgfqpoint{2.440318in}{1.002163in}}%
\pgfpathlineto{\pgfqpoint{2.445331in}{1.007342in}}%
\pgfpathlineto{\pgfqpoint{2.450345in}{1.007342in}}%
\pgfpathlineto{\pgfqpoint{2.452851in}{1.009876in}}%
\pgfpathlineto{\pgfqpoint{2.457865in}{1.009876in}}%
\pgfpathlineto{\pgfqpoint{2.462878in}{1.014840in}}%
\pgfpathlineto{\pgfqpoint{2.465385in}{1.014840in}}%
\pgfpathlineto{\pgfqpoint{2.467891in}{1.019669in}}%
\pgfpathlineto{\pgfqpoint{2.472905in}{1.024371in}}%
\pgfpathlineto{\pgfqpoint{2.475411in}{1.024371in}}%
\pgfpathlineto{\pgfqpoint{2.477918in}{1.028953in}}%
\pgfpathlineto{\pgfqpoint{2.480425in}{1.031200in}}%
\pgfpathlineto{\pgfqpoint{2.482931in}{1.031200in}}%
\pgfpathlineto{\pgfqpoint{2.487944in}{1.035612in}}%
\pgfpathlineto{\pgfqpoint{2.495464in}{1.035612in}}%
\pgfpathlineto{\pgfqpoint{2.497971in}{1.039917in}}%
\pgfpathlineto{\pgfqpoint{2.502984in}{1.039917in}}%
\pgfpathlineto{\pgfqpoint{2.505491in}{1.042032in}}%
\pgfpathlineto{\pgfqpoint{2.510504in}{1.042032in}}%
\pgfpathlineto{\pgfqpoint{2.513011in}{1.054219in}}%
\pgfpathlineto{\pgfqpoint{2.518024in}{1.060014in}}%
\pgfpathlineto{\pgfqpoint{2.520531in}{1.060014in}}%
\pgfpathlineto{\pgfqpoint{2.523038in}{1.061904in}}%
\pgfpathlineto{\pgfqpoint{2.525544in}{1.067459in}}%
\pgfpathlineto{\pgfqpoint{2.528051in}{1.069273in}}%
\pgfpathlineto{\pgfqpoint{2.535571in}{1.083158in}}%
\pgfpathlineto{\pgfqpoint{2.538078in}{1.084820in}}%
\pgfpathlineto{\pgfqpoint{2.543091in}{1.089715in}}%
\pgfpathlineto{\pgfqpoint{2.545598in}{1.091317in}}%
\pgfpathlineto{\pgfqpoint{2.548104in}{1.091317in}}%
\pgfpathlineto{\pgfqpoint{2.550611in}{1.092906in}}%
\pgfpathlineto{\pgfqpoint{2.553117in}{1.096040in}}%
\pgfpathlineto{\pgfqpoint{2.558131in}{1.096040in}}%
\pgfpathlineto{\pgfqpoint{2.563144in}{1.106598in}}%
\pgfpathlineto{\pgfqpoint{2.568157in}{1.117946in}}%
\pgfpathlineto{\pgfqpoint{2.570664in}{1.117946in}}%
\pgfpathlineto{\pgfqpoint{2.573171in}{1.120673in}}%
\pgfpathlineto{\pgfqpoint{2.575677in}{1.127316in}}%
\pgfpathlineto{\pgfqpoint{2.578184in}{1.128616in}}%
\pgfpathlineto{\pgfqpoint{2.583197in}{1.128616in}}%
\pgfpathlineto{\pgfqpoint{2.590717in}{1.133721in}}%
\pgfpathlineto{\pgfqpoint{2.593224in}{1.136220in}}%
\pgfpathlineto{\pgfqpoint{2.598237in}{1.136220in}}%
\pgfpathlineto{\pgfqpoint{2.600744in}{1.138684in}}%
\pgfpathlineto{\pgfqpoint{2.603251in}{1.138684in}}%
\pgfpathlineto{\pgfqpoint{2.613277in}{1.147051in}}%
\pgfpathlineto{\pgfqpoint{2.625810in}{1.151663in}}%
\pgfpathlineto{\pgfqpoint{2.633330in}{1.153924in}}%
\pgfpathlineto{\pgfqpoint{2.635837in}{1.153924in}}%
\pgfpathlineto{\pgfqpoint{2.643357in}{1.163762in}}%
\pgfpathlineto{\pgfqpoint{2.645864in}{1.164822in}}%
\pgfpathlineto{\pgfqpoint{2.648370in}{1.169002in}}%
\pgfpathlineto{\pgfqpoint{2.653384in}{1.172073in}}%
\pgfpathlineto{\pgfqpoint{2.655890in}{1.177079in}}%
\pgfpathlineto{\pgfqpoint{2.658397in}{1.177079in}}%
\pgfpathlineto{\pgfqpoint{2.660904in}{1.179042in}}%
\pgfpathlineto{\pgfqpoint{2.665917in}{1.180016in}}%
\pgfpathlineto{\pgfqpoint{2.670930in}{1.185749in}}%
\pgfpathlineto{\pgfqpoint{2.675944in}{1.188548in}}%
\pgfpathlineto{\pgfqpoint{2.680957in}{1.191304in}}%
\pgfpathlineto{\pgfqpoint{2.683464in}{1.191304in}}%
\pgfpathlineto{\pgfqpoint{2.688477in}{1.195805in}}%
\pgfpathlineto{\pgfqpoint{2.693490in}{1.201060in}}%
\pgfpathlineto{\pgfqpoint{2.695997in}{1.201060in}}%
\pgfpathlineto{\pgfqpoint{2.698503in}{1.204480in}}%
\pgfpathlineto{\pgfqpoint{2.701010in}{1.204480in}}%
\pgfpathlineto{\pgfqpoint{2.703517in}{1.206165in}}%
\pgfpathlineto{\pgfqpoint{2.706023in}{1.209489in}}%
\pgfpathlineto{\pgfqpoint{2.718557in}{1.212753in}}%
\pgfpathlineto{\pgfqpoint{2.721063in}{1.212753in}}%
\pgfpathlineto{\pgfqpoint{2.723570in}{1.216750in}}%
\pgfpathlineto{\pgfqpoint{2.726077in}{1.216750in}}%
\pgfpathlineto{\pgfqpoint{2.728583in}{1.222965in}}%
\pgfpathlineto{\pgfqpoint{2.733597in}{1.225241in}}%
\pgfpathlineto{\pgfqpoint{2.736103in}{1.225994in}}%
\pgfpathlineto{\pgfqpoint{2.743623in}{1.236914in}}%
\pgfpathlineto{\pgfqpoint{2.746130in}{1.236914in}}%
\pgfpathlineto{\pgfqpoint{2.753650in}{1.244518in}}%
\pgfpathlineto{\pgfqpoint{2.756156in}{1.245866in}}%
\pgfpathlineto{\pgfqpoint{2.758663in}{1.245866in}}%
\pgfpathlineto{\pgfqpoint{2.763676in}{1.248533in}}%
\pgfpathlineto{\pgfqpoint{2.766183in}{1.252460in}}%
\pgfpathlineto{\pgfqpoint{2.768690in}{1.252460in}}%
\pgfpathlineto{\pgfqpoint{2.776210in}{1.256935in}}%
\pgfpathlineto{\pgfqpoint{2.781223in}{1.257565in}}%
\pgfpathlineto{\pgfqpoint{2.783730in}{1.259443in}}%
\pgfpathlineto{\pgfqpoint{2.793756in}{1.283845in}}%
\pgfpathlineto{\pgfqpoint{2.798770in}{1.290245in}}%
\pgfpathlineto{\pgfqpoint{2.801276in}{1.300429in}}%
\pgfpathlineto{\pgfqpoint{2.803783in}{1.304345in}}%
\pgfpathlineto{\pgfqpoint{2.806290in}{1.304345in}}%
\pgfpathlineto{\pgfqpoint{2.808796in}{1.309122in}}%
\pgfpathlineto{\pgfqpoint{2.823836in}{1.319204in}}%
\pgfpathlineto{\pgfqpoint{2.826343in}{1.320536in}}%
\pgfpathlineto{\pgfqpoint{2.831356in}{1.321418in}}%
\pgfpathlineto{\pgfqpoint{2.833863in}{1.323605in}}%
\pgfpathlineto{\pgfqpoint{2.836369in}{1.324039in}}%
\pgfpathlineto{\pgfqpoint{2.838876in}{1.326622in}}%
\pgfpathlineto{\pgfqpoint{2.841383in}{1.327049in}}%
\pgfpathlineto{\pgfqpoint{2.843889in}{1.329169in}}%
\pgfpathlineto{\pgfqpoint{2.846396in}{1.329590in}}%
\pgfpathlineto{\pgfqpoint{2.851409in}{1.335787in}}%
\pgfpathlineto{\pgfqpoint{2.853916in}{1.335787in}}%
\pgfpathlineto{\pgfqpoint{2.858929in}{1.337806in}}%
\pgfpathlineto{\pgfqpoint{2.861436in}{1.338607in}}%
\pgfpathlineto{\pgfqpoint{2.863943in}{1.347951in}}%
\pgfpathlineto{\pgfqpoint{2.873969in}{1.352076in}}%
\pgfpathlineto{\pgfqpoint{2.876476in}{1.353553in}}%
\pgfpathlineto{\pgfqpoint{2.881489in}{1.360405in}}%
\pgfpathlineto{\pgfqpoint{2.883996in}{1.365634in}}%
\pgfpathlineto{\pgfqpoint{2.886503in}{1.367003in}}%
\pgfpathlineto{\pgfqpoint{2.889009in}{1.369711in}}%
\pgfpathlineto{\pgfqpoint{2.891516in}{1.370046in}}%
\pgfpathlineto{\pgfqpoint{2.894022in}{1.376950in}}%
\pgfpathlineto{\pgfqpoint{2.899036in}{1.377594in}}%
\pgfpathlineto{\pgfqpoint{2.901542in}{1.380463in}}%
\pgfpathlineto{\pgfqpoint{2.904049in}{1.381410in}}%
\pgfpathlineto{\pgfqpoint{2.906556in}{1.384218in}}%
\pgfpathlineto{\pgfqpoint{2.909062in}{1.384218in}}%
\pgfpathlineto{\pgfqpoint{2.919089in}{1.388199in}}%
\pgfpathlineto{\pgfqpoint{2.929116in}{1.390904in}}%
\pgfpathlineto{\pgfqpoint{2.931622in}{1.393274in}}%
\pgfpathlineto{\pgfqpoint{2.934129in}{1.397923in}}%
\pgfpathlineto{\pgfqpoint{2.936636in}{1.398781in}}%
\pgfpathlineto{\pgfqpoint{2.939142in}{1.401050in}}%
\pgfpathlineto{\pgfqpoint{2.944156in}{1.406872in}}%
\pgfpathlineto{\pgfqpoint{2.946662in}{1.409580in}}%
\pgfpathlineto{\pgfqpoint{2.949169in}{1.419508in}}%
\pgfpathlineto{\pgfqpoint{2.951676in}{1.420016in}}%
\pgfpathlineto{\pgfqpoint{2.954182in}{1.425014in}}%
\pgfpathlineto{\pgfqpoint{2.959195in}{1.425997in}}%
\pgfpathlineto{\pgfqpoint{2.964209in}{1.430116in}}%
\pgfpathlineto{\pgfqpoint{2.969222in}{1.435773in}}%
\pgfpathlineto{\pgfqpoint{2.976742in}{1.437160in}}%
\pgfpathlineto{\pgfqpoint{2.979249in}{1.437390in}}%
\pgfpathlineto{\pgfqpoint{2.981755in}{1.439902in}}%
\pgfpathlineto{\pgfqpoint{2.984262in}{1.440128in}}%
\pgfpathlineto{\pgfqpoint{2.994289in}{1.448533in}}%
\pgfpathlineto{\pgfqpoint{2.999302in}{1.451107in}}%
\pgfpathlineto{\pgfqpoint{3.004315in}{1.451744in}}%
\pgfpathlineto{\pgfqpoint{3.006822in}{1.455731in}}%
\pgfpathlineto{\pgfqpoint{3.011835in}{1.459428in}}%
\pgfpathlineto{\pgfqpoint{3.014342in}{1.467379in}}%
\pgfpathlineto{\pgfqpoint{3.021862in}{1.472741in}}%
\pgfpathlineto{\pgfqpoint{3.024369in}{1.475735in}}%
\pgfpathlineto{\pgfqpoint{3.026875in}{1.487405in}}%
\pgfpathlineto{\pgfqpoint{3.029382in}{1.488445in}}%
\pgfpathlineto{\pgfqpoint{3.034395in}{1.492882in}}%
\pgfpathlineto{\pgfqpoint{3.039408in}{1.497376in}}%
\pgfpathlineto{\pgfqpoint{3.044422in}{1.499338in}}%
\pgfpathlineto{\pgfqpoint{3.046928in}{1.505097in}}%
\pgfpathlineto{\pgfqpoint{3.059462in}{1.518730in}}%
\pgfpathlineto{\pgfqpoint{3.061968in}{1.520038in}}%
\pgfpathlineto{\pgfqpoint{3.064475in}{1.525598in}}%
\pgfpathlineto{\pgfqpoint{3.074502in}{1.527691in}}%
\pgfpathlineto{\pgfqpoint{3.077008in}{1.530034in}}%
\pgfpathlineto{\pgfqpoint{3.082022in}{1.530989in}}%
\pgfpathlineto{\pgfqpoint{3.084528in}{1.535693in}}%
\pgfpathlineto{\pgfqpoint{3.089542in}{1.536487in}}%
\pgfpathlineto{\pgfqpoint{3.092048in}{1.543733in}}%
\pgfpathlineto{\pgfqpoint{3.097061in}{1.548735in}}%
\pgfpathlineto{\pgfqpoint{3.102075in}{1.551427in}}%
\pgfpathlineto{\pgfqpoint{3.104581in}{1.557399in}}%
\pgfpathlineto{\pgfqpoint{3.119621in}{1.562951in}}%
\pgfpathlineto{\pgfqpoint{3.122128in}{1.565775in}}%
\pgfpathlineto{\pgfqpoint{3.127141in}{1.567004in}}%
\pgfpathlineto{\pgfqpoint{3.129648in}{1.570422in}}%
\pgfpathlineto{\pgfqpoint{3.139675in}{1.571511in}}%
\pgfpathlineto{\pgfqpoint{3.142181in}{1.572161in}}%
\pgfpathlineto{\pgfqpoint{3.149701in}{1.579056in}}%
\pgfpathlineto{\pgfqpoint{3.152208in}{1.584589in}}%
\pgfpathlineto{\pgfqpoint{3.154715in}{1.584790in}}%
\pgfpathlineto{\pgfqpoint{3.157221in}{1.588776in}}%
\pgfpathlineto{\pgfqpoint{3.159728in}{1.589465in}}%
\pgfpathlineto{\pgfqpoint{3.164741in}{1.594689in}}%
\pgfpathlineto{\pgfqpoint{3.167248in}{1.599301in}}%
\pgfpathlineto{\pgfqpoint{3.169754in}{1.599672in}}%
\pgfpathlineto{\pgfqpoint{3.172261in}{1.604339in}}%
\pgfpathlineto{\pgfqpoint{3.174768in}{1.604880in}}%
\pgfpathlineto{\pgfqpoint{3.177274in}{1.613151in}}%
\pgfpathlineto{\pgfqpoint{3.182288in}{1.614436in}}%
\pgfpathlineto{\pgfqpoint{3.184794in}{1.623019in}}%
\pgfpathlineto{\pgfqpoint{3.187301in}{1.639485in}}%
\pgfpathlineto{\pgfqpoint{3.189808in}{1.640963in}}%
\pgfpathlineto{\pgfqpoint{3.192314in}{1.644677in}}%
\pgfpathlineto{\pgfqpoint{3.194821in}{1.646256in}}%
\pgfpathlineto{\pgfqpoint{3.199834in}{1.654356in}}%
\pgfpathlineto{\pgfqpoint{3.204848in}{1.657067in}}%
\pgfpathlineto{\pgfqpoint{3.207354in}{1.660661in}}%
\pgfpathlineto{\pgfqpoint{3.209861in}{1.668337in}}%
\pgfpathlineto{\pgfqpoint{3.217381in}{1.673686in}}%
\pgfpathlineto{\pgfqpoint{3.219888in}{1.677084in}}%
\pgfpathlineto{\pgfqpoint{3.224901in}{1.677804in}}%
\pgfpathlineto{\pgfqpoint{3.227408in}{1.682006in}}%
\pgfpathlineto{\pgfqpoint{3.229914in}{1.682473in}}%
\pgfpathlineto{\pgfqpoint{3.232421in}{1.686567in}}%
\pgfpathlineto{\pgfqpoint{3.234927in}{1.686909in}}%
\pgfpathlineto{\pgfqpoint{3.237434in}{1.688664in}}%
\pgfpathlineto{\pgfqpoint{3.242447in}{1.689114in}}%
\pgfpathlineto{\pgfqpoint{3.247461in}{1.704918in}}%
\pgfpathlineto{\pgfqpoint{3.249967in}{1.705175in}}%
\pgfpathlineto{\pgfqpoint{3.254981in}{1.714435in}}%
\pgfpathlineto{\pgfqpoint{3.259994in}{1.716424in}}%
\pgfpathlineto{\pgfqpoint{3.262501in}{1.720194in}}%
\pgfpathlineto{\pgfqpoint{3.265007in}{1.733084in}}%
\pgfpathlineto{\pgfqpoint{3.267514in}{1.738494in}}%
\pgfpathlineto{\pgfqpoint{3.270021in}{1.758631in}}%
\pgfpathlineto{\pgfqpoint{3.272527in}{1.764476in}}%
\pgfpathlineto{\pgfqpoint{3.275034in}{1.766492in}}%
\pgfpathlineto{\pgfqpoint{3.277541in}{1.766893in}}%
\pgfpathlineto{\pgfqpoint{3.280047in}{1.769778in}}%
\pgfpathlineto{\pgfqpoint{3.282554in}{1.774371in}}%
\pgfpathlineto{\pgfqpoint{3.285061in}{1.775206in}}%
\pgfpathlineto{\pgfqpoint{3.287567in}{1.779188in}}%
\pgfpathlineto{\pgfqpoint{3.290074in}{1.790437in}}%
\pgfpathlineto{\pgfqpoint{3.295087in}{1.800959in}}%
\pgfpathlineto{\pgfqpoint{3.297594in}{1.811350in}}%
\pgfpathlineto{\pgfqpoint{3.300100in}{1.814553in}}%
\pgfpathlineto{\pgfqpoint{3.302607in}{1.821223in}}%
\pgfpathlineto{\pgfqpoint{3.307620in}{1.822533in}}%
\pgfpathlineto{\pgfqpoint{3.310127in}{1.833809in}}%
\pgfpathlineto{\pgfqpoint{3.315140in}{1.840027in}}%
\pgfpathlineto{\pgfqpoint{3.317647in}{1.851621in}}%
\pgfpathlineto{\pgfqpoint{3.320154in}{1.851825in}}%
\pgfpathlineto{\pgfqpoint{3.322660in}{1.855073in}}%
\pgfpathlineto{\pgfqpoint{3.325167in}{1.862381in}}%
\pgfpathlineto{\pgfqpoint{3.327674in}{1.883766in}}%
\pgfpathlineto{\pgfqpoint{3.332687in}{1.886673in}}%
\pgfpathlineto{\pgfqpoint{3.335194in}{1.887600in}}%
\pgfpathlineto{\pgfqpoint{3.347727in}{1.906529in}}%
\pgfpathlineto{\pgfqpoint{3.350234in}{1.906811in}}%
\pgfpathlineto{\pgfqpoint{3.355247in}{1.913098in}}%
\pgfpathlineto{\pgfqpoint{3.357754in}{1.920714in}}%
\pgfpathlineto{\pgfqpoint{3.362767in}{1.925702in}}%
\pgfpathlineto{\pgfqpoint{3.365273in}{1.926270in}}%
\pgfpathlineto{\pgfqpoint{3.367780in}{1.929841in}}%
\pgfpathlineto{\pgfqpoint{3.370287in}{1.930030in}}%
\pgfpathlineto{\pgfqpoint{3.372793in}{1.940997in}}%
\pgfpathlineto{\pgfqpoint{3.377807in}{1.949585in}}%
\pgfpathlineto{\pgfqpoint{3.382820in}{1.950810in}}%
\pgfpathlineto{\pgfqpoint{3.385327in}{1.958166in}}%
\pgfpathlineto{\pgfqpoint{3.392847in}{1.963992in}}%
\pgfpathlineto{\pgfqpoint{3.397860in}{1.965110in}}%
\pgfpathlineto{\pgfqpoint{3.400367in}{1.965685in}}%
\pgfpathlineto{\pgfqpoint{3.402873in}{1.968731in}}%
\pgfpathlineto{\pgfqpoint{3.405380in}{1.975748in}}%
\pgfpathlineto{\pgfqpoint{3.407887in}{1.977647in}}%
\pgfpathlineto{\pgfqpoint{3.410393in}{1.977782in}}%
\pgfpathlineto{\pgfqpoint{3.412900in}{1.981003in}}%
\pgfpathlineto{\pgfqpoint{3.417913in}{1.984200in}}%
\pgfpathlineto{\pgfqpoint{3.422927in}{1.993113in}}%
\pgfpathlineto{\pgfqpoint{3.425433in}{1.995060in}}%
\pgfpathlineto{\pgfqpoint{3.427940in}{2.005308in}}%
\pgfpathlineto{\pgfqpoint{3.430447in}{2.007023in}}%
\pgfpathlineto{\pgfqpoint{3.432953in}{2.015949in}}%
\pgfpathlineto{\pgfqpoint{3.437966in}{2.018226in}}%
\pgfpathlineto{\pgfqpoint{3.440473in}{2.021588in}}%
\pgfpathlineto{\pgfqpoint{3.442980in}{2.036765in}}%
\pgfpathlineto{\pgfqpoint{3.445486in}{2.039522in}}%
\pgfpathlineto{\pgfqpoint{3.447993in}{2.045364in}}%
\pgfpathlineto{\pgfqpoint{3.450500in}{2.046036in}}%
\pgfpathlineto{\pgfqpoint{3.455513in}{2.059088in}}%
\pgfpathlineto{\pgfqpoint{3.458020in}{2.072498in}}%
\pgfpathlineto{\pgfqpoint{3.460526in}{2.078897in}}%
\pgfpathlineto{\pgfqpoint{3.463033in}{2.080577in}}%
\pgfpathlineto{\pgfqpoint{3.465540in}{2.092511in}}%
\pgfpathlineto{\pgfqpoint{3.468046in}{2.094620in}}%
\pgfpathlineto{\pgfqpoint{3.470553in}{2.094736in}}%
\pgfpathlineto{\pgfqpoint{3.473060in}{2.099174in}}%
\pgfpathlineto{\pgfqpoint{3.475566in}{2.106828in}}%
\pgfpathlineto{\pgfqpoint{3.478073in}{2.109872in}}%
\pgfpathlineto{\pgfqpoint{3.483086in}{2.111386in}}%
\pgfpathlineto{\pgfqpoint{3.485593in}{2.114318in}}%
\pgfpathlineto{\pgfqpoint{3.490606in}{2.127115in}}%
\pgfpathlineto{\pgfqpoint{3.493113in}{2.128533in}}%
\pgfpathlineto{\pgfqpoint{3.495620in}{2.133363in}}%
\pgfpathlineto{\pgfqpoint{3.500633in}{2.150713in}}%
\pgfpathlineto{\pgfqpoint{3.508153in}{2.159337in}}%
\pgfpathlineto{\pgfqpoint{3.515673in}{2.160409in}}%
\pgfpathlineto{\pgfqpoint{3.518179in}{2.161519in}}%
\pgfpathlineto{\pgfqpoint{3.520686in}{2.164749in}}%
\pgfpathlineto{\pgfqpoint{3.525699in}{2.166214in}}%
\pgfpathlineto{\pgfqpoint{3.533219in}{2.187498in}}%
\pgfpathlineto{\pgfqpoint{3.535726in}{2.188015in}}%
\pgfpathlineto{\pgfqpoint{3.543246in}{2.203309in}}%
\pgfpathlineto{\pgfqpoint{3.548259in}{2.207835in}}%
\pgfpathlineto{\pgfqpoint{3.553273in}{2.225361in}}%
\pgfpathlineto{\pgfqpoint{3.555779in}{2.225660in}}%
\pgfpathlineto{\pgfqpoint{3.563299in}{2.242373in}}%
\pgfpathlineto{\pgfqpoint{3.565806in}{2.242924in}}%
\pgfpathlineto{\pgfqpoint{3.575832in}{2.252349in}}%
\pgfpathlineto{\pgfqpoint{3.578339in}{2.254550in}}%
\pgfpathlineto{\pgfqpoint{3.580846in}{2.259754in}}%
\pgfpathlineto{\pgfqpoint{3.583352in}{2.260965in}}%
\pgfpathlineto{\pgfqpoint{3.585859in}{2.263615in}}%
\pgfpathlineto{\pgfqpoint{3.590872in}{2.272255in}}%
\pgfpathlineto{\pgfqpoint{3.593379in}{2.281098in}}%
\pgfpathlineto{\pgfqpoint{3.598392in}{2.283758in}}%
\pgfpathlineto{\pgfqpoint{3.600899in}{2.284747in}}%
\pgfpathlineto{\pgfqpoint{3.603406in}{2.296848in}}%
\pgfpathlineto{\pgfqpoint{3.605912in}{2.299341in}}%
\pgfpathlineto{\pgfqpoint{3.608419in}{2.303997in}}%
\pgfpathlineto{\pgfqpoint{3.610926in}{2.305275in}}%
\pgfpathlineto{\pgfqpoint{5.400677in}{2.306381in}}%
\pgfpathlineto{\pgfqpoint{5.495930in}{2.307743in}}%
\pgfpathlineto{\pgfqpoint{5.571129in}{2.310756in}}%
\pgfpathlineto{\pgfqpoint{5.583663in}{2.313640in}}%
\pgfpathlineto{\pgfqpoint{5.588676in}{2.314242in}}%
\pgfpathlineto{\pgfqpoint{5.593313in}{2.315275in}}%
\pgfpathlineto{\pgfqpoint{5.593313in}{2.315275in}}%
\pgfusepath{stroke}%
\end{pgfscope}%
\begin{pgfscope}%
\pgfpathrectangle{\pgfqpoint{0.708220in}{0.535823in}}{\pgfqpoint{5.013309in}{1.769453in}}%
\pgfusepath{clip}%
\pgfsetbuttcap%
\pgfsetroundjoin%
\pgfsetlinewidth{1.003750pt}%
\definecolor{currentstroke}{rgb}{1.000000,0.647059,0.000000}%
\pgfsetstrokecolor{currentstroke}%
\pgfsetdash{{3.700000pt}{1.600000pt}}{0.000000pt}%
\pgfpathmoveto{\pgfqpoint{0.708220in}{0.606902in}}%
\pgfpathlineto{\pgfqpoint{0.710727in}{0.606909in}}%
\pgfpathlineto{\pgfqpoint{0.713233in}{0.623389in}}%
\pgfpathlineto{\pgfqpoint{0.718246in}{0.625894in}}%
\pgfpathlineto{\pgfqpoint{0.720753in}{0.628848in}}%
\pgfpathlineto{\pgfqpoint{0.723260in}{0.629962in}}%
\pgfpathlineto{\pgfqpoint{0.725766in}{0.634766in}}%
\pgfpathlineto{\pgfqpoint{0.728273in}{0.635676in}}%
\pgfpathlineto{\pgfqpoint{0.730780in}{0.649484in}}%
\pgfpathlineto{\pgfqpoint{0.735793in}{0.651568in}}%
\pgfpathlineto{\pgfqpoint{0.740806in}{0.666362in}}%
\pgfpathlineto{\pgfqpoint{0.743313in}{0.667704in}}%
\pgfpathlineto{\pgfqpoint{0.745820in}{0.672109in}}%
\pgfpathlineto{\pgfqpoint{0.750833in}{0.673462in}}%
\pgfpathlineto{\pgfqpoint{0.753340in}{0.684636in}}%
\pgfpathlineto{\pgfqpoint{0.755846in}{0.684678in}}%
\pgfpathlineto{\pgfqpoint{0.758353in}{0.689845in}}%
\pgfpathlineto{\pgfqpoint{0.763366in}{0.695312in}}%
\pgfpathlineto{\pgfqpoint{0.765873in}{0.700207in}}%
\pgfpathlineto{\pgfqpoint{0.768380in}{0.700521in}}%
\pgfpathlineto{\pgfqpoint{0.770886in}{0.705081in}}%
\pgfpathlineto{\pgfqpoint{0.780913in}{0.708075in}}%
\pgfpathlineto{\pgfqpoint{0.783419in}{0.708106in}}%
\pgfpathlineto{\pgfqpoint{0.788433in}{0.729832in}}%
\pgfpathlineto{\pgfqpoint{0.790939in}{0.732282in}}%
\pgfpathlineto{\pgfqpoint{0.793446in}{0.732707in}}%
\pgfpathlineto{\pgfqpoint{0.795953in}{0.735926in}}%
\pgfpathlineto{\pgfqpoint{0.798459in}{0.736115in}}%
\pgfpathlineto{\pgfqpoint{0.800966in}{0.738305in}}%
\pgfpathlineto{\pgfqpoint{0.803473in}{0.738459in}}%
\pgfpathlineto{\pgfqpoint{0.810993in}{0.747331in}}%
\pgfpathlineto{\pgfqpoint{0.816006in}{0.748383in}}%
\pgfpathlineto{\pgfqpoint{0.818513in}{0.755766in}}%
\pgfpathlineto{\pgfqpoint{0.821019in}{0.758654in}}%
\pgfpathlineto{\pgfqpoint{0.831046in}{0.760097in}}%
\pgfpathlineto{\pgfqpoint{0.838566in}{0.761299in}}%
\pgfpathlineto{\pgfqpoint{0.863632in}{0.762483in}}%
\pgfpathlineto{\pgfqpoint{0.871152in}{0.765658in}}%
\pgfpathlineto{\pgfqpoint{0.873659in}{0.767678in}}%
\pgfpathlineto{\pgfqpoint{0.938832in}{0.772826in}}%
\pgfpathlineto{\pgfqpoint{0.948859in}{0.776772in}}%
\pgfpathlineto{\pgfqpoint{1.076698in}{0.786923in}}%
\pgfpathlineto{\pgfqpoint{1.106778in}{0.788155in}}%
\pgfpathlineto{\pgfqpoint{1.164431in}{0.790977in}}%
\pgfpathlineto{\pgfqpoint{1.171951in}{0.793695in}}%
\pgfpathlineto{\pgfqpoint{1.214564in}{0.795350in}}%
\pgfpathlineto{\pgfqpoint{1.237124in}{0.796477in}}%
\pgfpathlineto{\pgfqpoint{1.269710in}{0.797400in}}%
\pgfpathlineto{\pgfqpoint{1.282244in}{0.798785in}}%
\pgfpathlineto{\pgfqpoint{1.287257in}{0.798855in}}%
\pgfpathlineto{\pgfqpoint{1.289764in}{0.800419in}}%
\pgfpathlineto{\pgfqpoint{1.294777in}{0.801123in}}%
\pgfpathlineto{\pgfqpoint{1.299790in}{0.802657in}}%
\pgfpathlineto{\pgfqpoint{1.312324in}{0.803438in}}%
\pgfpathlineto{\pgfqpoint{1.374990in}{0.807104in}}%
\pgfpathlineto{\pgfqpoint{1.390030in}{0.808679in}}%
\pgfpathlineto{\pgfqpoint{1.412590in}{0.809712in}}%
\pgfpathlineto{\pgfqpoint{1.480269in}{0.811762in}}%
\pgfpathlineto{\pgfqpoint{1.495309in}{0.813161in}}%
\pgfpathlineto{\pgfqpoint{1.502829in}{0.814422in}}%
\pgfpathlineto{\pgfqpoint{1.573016in}{0.817704in}}%
\pgfpathlineto{\pgfqpoint{1.618135in}{0.818989in}}%
\pgfpathlineto{\pgfqpoint{1.700855in}{0.824968in}}%
\pgfpathlineto{\pgfqpoint{1.708375in}{0.826512in}}%
\pgfpathlineto{\pgfqpoint{1.718402in}{0.827303in}}%
\pgfpathlineto{\pgfqpoint{1.720908in}{0.830188in}}%
\pgfpathlineto{\pgfqpoint{1.730935in}{0.831369in}}%
\pgfpathlineto{\pgfqpoint{1.733442in}{0.831758in}}%
\pgfpathlineto{\pgfqpoint{1.735948in}{0.834437in}}%
\pgfpathlineto{\pgfqpoint{1.745975in}{0.836267in}}%
\pgfpathlineto{\pgfqpoint{1.768535in}{0.837878in}}%
\pgfpathlineto{\pgfqpoint{1.786081in}{0.840893in}}%
\pgfpathlineto{\pgfqpoint{1.896374in}{0.844206in}}%
\pgfpathlineto{\pgfqpoint{1.903894in}{0.845030in}}%
\pgfpathlineto{\pgfqpoint{1.913921in}{0.846116in}}%
\pgfpathlineto{\pgfqpoint{1.918934in}{0.847616in}}%
\pgfpathlineto{\pgfqpoint{1.936481in}{0.848968in}}%
\pgfpathlineto{\pgfqpoint{1.964054in}{0.851392in}}%
\pgfpathlineto{\pgfqpoint{1.976587in}{0.854872in}}%
\pgfpathlineto{\pgfqpoint{1.991627in}{0.855780in}}%
\pgfpathlineto{\pgfqpoint{2.009173in}{0.861813in}}%
\pgfpathlineto{\pgfqpoint{2.021707in}{0.871676in}}%
\pgfpathlineto{\pgfqpoint{2.024213in}{0.872123in}}%
\pgfpathlineto{\pgfqpoint{2.029227in}{0.875762in}}%
\pgfpathlineto{\pgfqpoint{2.034240in}{0.878426in}}%
\pgfpathlineto{\pgfqpoint{2.039253in}{0.878652in}}%
\pgfpathlineto{\pgfqpoint{2.041760in}{0.884097in}}%
\pgfpathlineto{\pgfqpoint{2.046773in}{0.884449in}}%
\pgfpathlineto{\pgfqpoint{2.051787in}{0.886785in}}%
\pgfpathlineto{\pgfqpoint{2.056800in}{0.887370in}}%
\pgfpathlineto{\pgfqpoint{2.059307in}{0.888922in}}%
\pgfpathlineto{\pgfqpoint{2.094400in}{0.891261in}}%
\pgfpathlineto{\pgfqpoint{2.111946in}{0.892459in}}%
\pgfpathlineto{\pgfqpoint{2.119466in}{0.894182in}}%
\pgfpathlineto{\pgfqpoint{2.159573in}{0.897190in}}%
\pgfpathlineto{\pgfqpoint{2.164586in}{0.898667in}}%
\pgfpathlineto{\pgfqpoint{2.172106in}{0.900630in}}%
\pgfpathlineto{\pgfqpoint{2.192159in}{0.901691in}}%
\pgfpathlineto{\pgfqpoint{2.204693in}{0.903592in}}%
\pgfpathlineto{\pgfqpoint{2.219732in}{0.906363in}}%
\pgfpathlineto{\pgfqpoint{2.224746in}{0.907700in}}%
\pgfpathlineto{\pgfqpoint{2.229759in}{0.910115in}}%
\pgfpathlineto{\pgfqpoint{2.234772in}{0.913313in}}%
\pgfpathlineto{\pgfqpoint{2.237279in}{0.913336in}}%
\pgfpathlineto{\pgfqpoint{2.242292in}{0.915705in}}%
\pgfpathlineto{\pgfqpoint{2.252319in}{0.916695in}}%
\pgfpathlineto{\pgfqpoint{2.264852in}{0.917996in}}%
\pgfpathlineto{\pgfqpoint{2.284905in}{0.918832in}}%
\pgfpathlineto{\pgfqpoint{2.294932in}{0.919705in}}%
\pgfpathlineto{\pgfqpoint{2.322505in}{0.921625in}}%
\pgfpathlineto{\pgfqpoint{2.385172in}{0.924357in}}%
\pgfpathlineto{\pgfqpoint{2.405225in}{0.925795in}}%
\pgfpathlineto{\pgfqpoint{2.442825in}{0.927859in}}%
\pgfpathlineto{\pgfqpoint{2.447838in}{0.928864in}}%
\pgfpathlineto{\pgfqpoint{2.470398in}{0.930185in}}%
\pgfpathlineto{\pgfqpoint{2.513011in}{0.936767in}}%
\pgfpathlineto{\pgfqpoint{2.523038in}{0.938551in}}%
\pgfpathlineto{\pgfqpoint{2.530558in}{0.940691in}}%
\pgfpathlineto{\pgfqpoint{2.543091in}{0.949277in}}%
\pgfpathlineto{\pgfqpoint{2.555624in}{0.950307in}}%
\pgfpathlineto{\pgfqpoint{2.560637in}{0.955989in}}%
\pgfpathlineto{\pgfqpoint{2.563144in}{0.963035in}}%
\pgfpathlineto{\pgfqpoint{2.570664in}{0.966167in}}%
\pgfpathlineto{\pgfqpoint{2.578184in}{0.971319in}}%
\pgfpathlineto{\pgfqpoint{2.580691in}{0.971994in}}%
\pgfpathlineto{\pgfqpoint{2.583197in}{0.975704in}}%
\pgfpathlineto{\pgfqpoint{2.588211in}{0.977702in}}%
\pgfpathlineto{\pgfqpoint{2.590717in}{0.981972in}}%
\pgfpathlineto{\pgfqpoint{2.593224in}{0.982082in}}%
\pgfpathlineto{\pgfqpoint{2.595731in}{0.986214in}}%
\pgfpathlineto{\pgfqpoint{2.600744in}{0.987280in}}%
\pgfpathlineto{\pgfqpoint{2.603251in}{0.989854in}}%
\pgfpathlineto{\pgfqpoint{2.605757in}{0.990509in}}%
\pgfpathlineto{\pgfqpoint{2.613277in}{0.996470in}}%
\pgfpathlineto{\pgfqpoint{2.620797in}{1.001474in}}%
\pgfpathlineto{\pgfqpoint{2.625810in}{1.002805in}}%
\pgfpathlineto{\pgfqpoint{2.630824in}{1.004169in}}%
\pgfpathlineto{\pgfqpoint{2.633330in}{1.004505in}}%
\pgfpathlineto{\pgfqpoint{2.638344in}{1.006732in}}%
\pgfpathlineto{\pgfqpoint{2.640850in}{1.006982in}}%
\pgfpathlineto{\pgfqpoint{2.658397in}{1.019427in}}%
\pgfpathlineto{\pgfqpoint{2.663410in}{1.024413in}}%
\pgfpathlineto{\pgfqpoint{2.670930in}{1.029560in}}%
\pgfpathlineto{\pgfqpoint{2.673437in}{1.033079in}}%
\pgfpathlineto{\pgfqpoint{2.680957in}{1.036603in}}%
\pgfpathlineto{\pgfqpoint{2.685970in}{1.038812in}}%
\pgfpathlineto{\pgfqpoint{2.695997in}{1.049706in}}%
\pgfpathlineto{\pgfqpoint{2.698503in}{1.050485in}}%
\pgfpathlineto{\pgfqpoint{2.701010in}{1.054153in}}%
\pgfpathlineto{\pgfqpoint{2.703517in}{1.055007in}}%
\pgfpathlineto{\pgfqpoint{2.706023in}{1.061587in}}%
\pgfpathlineto{\pgfqpoint{2.708530in}{1.063162in}}%
\pgfpathlineto{\pgfqpoint{2.711037in}{1.067308in}}%
\pgfpathlineto{\pgfqpoint{2.713543in}{1.074221in}}%
\pgfpathlineto{\pgfqpoint{2.723570in}{1.080742in}}%
\pgfpathlineto{\pgfqpoint{2.733597in}{1.083439in}}%
\pgfpathlineto{\pgfqpoint{2.746130in}{1.089748in}}%
\pgfpathlineto{\pgfqpoint{2.748637in}{1.105850in}}%
\pgfpathlineto{\pgfqpoint{2.751143in}{1.107111in}}%
\pgfpathlineto{\pgfqpoint{2.753650in}{1.111284in}}%
\pgfpathlineto{\pgfqpoint{2.756156in}{1.112633in}}%
\pgfpathlineto{\pgfqpoint{2.758663in}{1.116855in}}%
\pgfpathlineto{\pgfqpoint{2.766183in}{1.119921in}}%
\pgfpathlineto{\pgfqpoint{2.768690in}{1.123483in}}%
\pgfpathlineto{\pgfqpoint{2.773703in}{1.124121in}}%
\pgfpathlineto{\pgfqpoint{2.776210in}{1.127324in}}%
\pgfpathlineto{\pgfqpoint{2.778716in}{1.127360in}}%
\pgfpathlineto{\pgfqpoint{2.783730in}{1.130248in}}%
\pgfpathlineto{\pgfqpoint{2.786236in}{1.130496in}}%
\pgfpathlineto{\pgfqpoint{2.788743in}{1.134242in}}%
\pgfpathlineto{\pgfqpoint{2.791250in}{1.135853in}}%
\pgfpathlineto{\pgfqpoint{2.793756in}{1.140941in}}%
\pgfpathlineto{\pgfqpoint{2.798770in}{1.144363in}}%
\pgfpathlineto{\pgfqpoint{2.803783in}{1.153777in}}%
\pgfpathlineto{\pgfqpoint{2.806290in}{1.154689in}}%
\pgfpathlineto{\pgfqpoint{2.811303in}{1.158268in}}%
\pgfpathlineto{\pgfqpoint{2.826343in}{1.165359in}}%
\pgfpathlineto{\pgfqpoint{2.831356in}{1.173471in}}%
\pgfpathlineto{\pgfqpoint{2.833863in}{1.176434in}}%
\pgfpathlineto{\pgfqpoint{2.836369in}{1.186028in}}%
\pgfpathlineto{\pgfqpoint{2.843889in}{1.191436in}}%
\pgfpathlineto{\pgfqpoint{2.846396in}{1.193920in}}%
\pgfpathlineto{\pgfqpoint{2.848903in}{1.194453in}}%
\pgfpathlineto{\pgfqpoint{2.853916in}{1.199497in}}%
\pgfpathlineto{\pgfqpoint{2.856423in}{1.199960in}}%
\pgfpathlineto{\pgfqpoint{2.858929in}{1.201786in}}%
\pgfpathlineto{\pgfqpoint{2.863943in}{1.201979in}}%
\pgfpathlineto{\pgfqpoint{2.866449in}{1.204028in}}%
\pgfpathlineto{\pgfqpoint{2.873969in}{1.205730in}}%
\pgfpathlineto{\pgfqpoint{2.878983in}{1.213597in}}%
\pgfpathlineto{\pgfqpoint{2.881489in}{1.214455in}}%
\pgfpathlineto{\pgfqpoint{2.886503in}{1.217824in}}%
\pgfpathlineto{\pgfqpoint{2.891516in}{1.220204in}}%
\pgfpathlineto{\pgfqpoint{2.894022in}{1.223956in}}%
\pgfpathlineto{\pgfqpoint{2.899036in}{1.225419in}}%
\pgfpathlineto{\pgfqpoint{2.901542in}{1.227871in}}%
\pgfpathlineto{\pgfqpoint{2.904049in}{1.228509in}}%
\pgfpathlineto{\pgfqpoint{2.909062in}{1.234004in}}%
\pgfpathlineto{\pgfqpoint{2.911569in}{1.235270in}}%
\pgfpathlineto{\pgfqpoint{2.914076in}{1.239231in}}%
\pgfpathlineto{\pgfqpoint{2.919089in}{1.240781in}}%
\pgfpathlineto{\pgfqpoint{2.921596in}{1.248672in}}%
\pgfpathlineto{\pgfqpoint{2.924102in}{1.249039in}}%
\pgfpathlineto{\pgfqpoint{2.934129in}{1.257240in}}%
\pgfpathlineto{\pgfqpoint{2.939142in}{1.257748in}}%
\pgfpathlineto{\pgfqpoint{2.944156in}{1.266031in}}%
\pgfpathlineto{\pgfqpoint{2.949169in}{1.268757in}}%
\pgfpathlineto{\pgfqpoint{2.954182in}{1.278450in}}%
\pgfpathlineto{\pgfqpoint{2.956689in}{1.279516in}}%
\pgfpathlineto{\pgfqpoint{2.961702in}{1.283587in}}%
\pgfpathlineto{\pgfqpoint{2.966715in}{1.286276in}}%
\pgfpathlineto{\pgfqpoint{2.974235in}{1.294391in}}%
\pgfpathlineto{\pgfqpoint{2.976742in}{1.294656in}}%
\pgfpathlineto{\pgfqpoint{2.986769in}{1.299659in}}%
\pgfpathlineto{\pgfqpoint{2.991782in}{1.302285in}}%
\pgfpathlineto{\pgfqpoint{2.994289in}{1.305040in}}%
\pgfpathlineto{\pgfqpoint{2.996795in}{1.305673in}}%
\pgfpathlineto{\pgfqpoint{2.999302in}{1.308077in}}%
\pgfpathlineto{\pgfqpoint{3.001809in}{1.308156in}}%
\pgfpathlineto{\pgfqpoint{3.004315in}{1.310311in}}%
\pgfpathlineto{\pgfqpoint{3.006822in}{1.310591in}}%
\pgfpathlineto{\pgfqpoint{3.014342in}{1.316923in}}%
\pgfpathlineto{\pgfqpoint{3.016849in}{1.317219in}}%
\pgfpathlineto{\pgfqpoint{3.019355in}{1.319142in}}%
\pgfpathlineto{\pgfqpoint{3.021862in}{1.324504in}}%
\pgfpathlineto{\pgfqpoint{3.026875in}{1.327436in}}%
\pgfpathlineto{\pgfqpoint{3.031888in}{1.328490in}}%
\pgfpathlineto{\pgfqpoint{3.039408in}{1.330195in}}%
\pgfpathlineto{\pgfqpoint{3.044422in}{1.343448in}}%
\pgfpathlineto{\pgfqpoint{3.049435in}{1.345149in}}%
\pgfpathlineto{\pgfqpoint{3.051942in}{1.345272in}}%
\pgfpathlineto{\pgfqpoint{3.059462in}{1.350051in}}%
\pgfpathlineto{\pgfqpoint{3.064475in}{1.361325in}}%
\pgfpathlineto{\pgfqpoint{3.071995in}{1.364369in}}%
\pgfpathlineto{\pgfqpoint{3.077008in}{1.368083in}}%
\pgfpathlineto{\pgfqpoint{3.079515in}{1.383517in}}%
\pgfpathlineto{\pgfqpoint{3.087035in}{1.385118in}}%
\pgfpathlineto{\pgfqpoint{3.089542in}{1.387049in}}%
\pgfpathlineto{\pgfqpoint{3.092048in}{1.387308in}}%
\pgfpathlineto{\pgfqpoint{3.094555in}{1.391124in}}%
\pgfpathlineto{\pgfqpoint{3.112101in}{1.397304in}}%
\pgfpathlineto{\pgfqpoint{3.114608in}{1.403113in}}%
\pgfpathlineto{\pgfqpoint{3.119621in}{1.403528in}}%
\pgfpathlineto{\pgfqpoint{3.127141in}{1.406804in}}%
\pgfpathlineto{\pgfqpoint{3.134661in}{1.417952in}}%
\pgfpathlineto{\pgfqpoint{3.137168in}{1.418165in}}%
\pgfpathlineto{\pgfqpoint{3.139675in}{1.420347in}}%
\pgfpathlineto{\pgfqpoint{3.142181in}{1.427682in}}%
\pgfpathlineto{\pgfqpoint{3.144688in}{1.427951in}}%
\pgfpathlineto{\pgfqpoint{3.147195in}{1.431343in}}%
\pgfpathlineto{\pgfqpoint{3.149701in}{1.438917in}}%
\pgfpathlineto{\pgfqpoint{3.157221in}{1.440082in}}%
\pgfpathlineto{\pgfqpoint{3.159728in}{1.445789in}}%
\pgfpathlineto{\pgfqpoint{3.167248in}{1.447890in}}%
\pgfpathlineto{\pgfqpoint{3.172261in}{1.456473in}}%
\pgfpathlineto{\pgfqpoint{3.174768in}{1.456887in}}%
\pgfpathlineto{\pgfqpoint{3.177274in}{1.460044in}}%
\pgfpathlineto{\pgfqpoint{3.179781in}{1.468980in}}%
\pgfpathlineto{\pgfqpoint{3.194821in}{1.477520in}}%
\pgfpathlineto{\pgfqpoint{3.197328in}{1.483046in}}%
\pgfpathlineto{\pgfqpoint{3.199834in}{1.496745in}}%
\pgfpathlineto{\pgfqpoint{3.202341in}{1.497302in}}%
\pgfpathlineto{\pgfqpoint{3.204848in}{1.501145in}}%
\pgfpathlineto{\pgfqpoint{3.207354in}{1.501698in}}%
\pgfpathlineto{\pgfqpoint{3.209861in}{1.506115in}}%
\pgfpathlineto{\pgfqpoint{3.214874in}{1.509898in}}%
\pgfpathlineto{\pgfqpoint{3.222394in}{1.518104in}}%
\pgfpathlineto{\pgfqpoint{3.224901in}{1.523608in}}%
\pgfpathlineto{\pgfqpoint{3.227408in}{1.524923in}}%
\pgfpathlineto{\pgfqpoint{3.229914in}{1.533622in}}%
\pgfpathlineto{\pgfqpoint{3.234927in}{1.535342in}}%
\pgfpathlineto{\pgfqpoint{3.237434in}{1.535812in}}%
\pgfpathlineto{\pgfqpoint{3.239941in}{1.539156in}}%
\pgfpathlineto{\pgfqpoint{3.244954in}{1.540341in}}%
\pgfpathlineto{\pgfqpoint{3.249967in}{1.541527in}}%
\pgfpathlineto{\pgfqpoint{3.257487in}{1.546267in}}%
\pgfpathlineto{\pgfqpoint{3.259994in}{1.546660in}}%
\pgfpathlineto{\pgfqpoint{3.262501in}{1.551703in}}%
\pgfpathlineto{\pgfqpoint{3.265007in}{1.551892in}}%
\pgfpathlineto{\pgfqpoint{3.272527in}{1.558392in}}%
\pgfpathlineto{\pgfqpoint{3.275034in}{1.563761in}}%
\pgfpathlineto{\pgfqpoint{3.277541in}{1.563770in}}%
\pgfpathlineto{\pgfqpoint{3.287567in}{1.570445in}}%
\pgfpathlineto{\pgfqpoint{3.290074in}{1.575855in}}%
\pgfpathlineto{\pgfqpoint{3.297594in}{1.577923in}}%
\pgfpathlineto{\pgfqpoint{3.300100in}{1.578594in}}%
\pgfpathlineto{\pgfqpoint{3.302607in}{1.581242in}}%
\pgfpathlineto{\pgfqpoint{3.305114in}{1.587577in}}%
\pgfpathlineto{\pgfqpoint{3.307620in}{1.590414in}}%
\pgfpathlineto{\pgfqpoint{3.312634in}{1.611414in}}%
\pgfpathlineto{\pgfqpoint{3.315140in}{1.612427in}}%
\pgfpathlineto{\pgfqpoint{3.317647in}{1.615207in}}%
\pgfpathlineto{\pgfqpoint{3.320154in}{1.623116in}}%
\pgfpathlineto{\pgfqpoint{3.322660in}{1.625214in}}%
\pgfpathlineto{\pgfqpoint{3.325167in}{1.635283in}}%
\pgfpathlineto{\pgfqpoint{3.327674in}{1.637206in}}%
\pgfpathlineto{\pgfqpoint{3.330180in}{1.646379in}}%
\pgfpathlineto{\pgfqpoint{3.337700in}{1.652315in}}%
\pgfpathlineto{\pgfqpoint{3.340207in}{1.664933in}}%
\pgfpathlineto{\pgfqpoint{3.342714in}{1.671129in}}%
\pgfpathlineto{\pgfqpoint{3.347727in}{1.674493in}}%
\pgfpathlineto{\pgfqpoint{3.357754in}{1.686468in}}%
\pgfpathlineto{\pgfqpoint{3.360260in}{1.686968in}}%
\pgfpathlineto{\pgfqpoint{3.365273in}{1.690671in}}%
\pgfpathlineto{\pgfqpoint{3.367780in}{1.696529in}}%
\pgfpathlineto{\pgfqpoint{3.370287in}{1.699035in}}%
\pgfpathlineto{\pgfqpoint{3.372793in}{1.711430in}}%
\pgfpathlineto{\pgfqpoint{3.375300in}{1.713682in}}%
\pgfpathlineto{\pgfqpoint{3.382820in}{1.727748in}}%
\pgfpathlineto{\pgfqpoint{3.385327in}{1.727819in}}%
\pgfpathlineto{\pgfqpoint{3.387833in}{1.730885in}}%
\pgfpathlineto{\pgfqpoint{3.390340in}{1.739475in}}%
\pgfpathlineto{\pgfqpoint{3.395353in}{1.746675in}}%
\pgfpathlineto{\pgfqpoint{3.402873in}{1.748233in}}%
\pgfpathlineto{\pgfqpoint{3.415407in}{1.754713in}}%
\pgfpathlineto{\pgfqpoint{3.420420in}{1.759154in}}%
\pgfpathlineto{\pgfqpoint{3.422927in}{1.760686in}}%
\pgfpathlineto{\pgfqpoint{3.425433in}{1.760762in}}%
\pgfpathlineto{\pgfqpoint{3.427940in}{1.770908in}}%
\pgfpathlineto{\pgfqpoint{3.432953in}{1.773925in}}%
\pgfpathlineto{\pgfqpoint{3.435460in}{1.776579in}}%
\pgfpathlineto{\pgfqpoint{3.437966in}{1.781034in}}%
\pgfpathlineto{\pgfqpoint{3.442980in}{1.784661in}}%
\pgfpathlineto{\pgfqpoint{3.445486in}{1.787402in}}%
\pgfpathlineto{\pgfqpoint{3.455513in}{1.819570in}}%
\pgfpathlineto{\pgfqpoint{3.463033in}{1.827194in}}%
\pgfpathlineto{\pgfqpoint{3.465540in}{1.832921in}}%
\pgfpathlineto{\pgfqpoint{3.470553in}{1.834019in}}%
\pgfpathlineto{\pgfqpoint{3.473060in}{1.840662in}}%
\pgfpathlineto{\pgfqpoint{3.475566in}{1.868170in}}%
\pgfpathlineto{\pgfqpoint{3.480580in}{1.876227in}}%
\pgfpathlineto{\pgfqpoint{3.483086in}{1.889862in}}%
\pgfpathlineto{\pgfqpoint{3.485593in}{1.893734in}}%
\pgfpathlineto{\pgfqpoint{3.488100in}{1.894724in}}%
\pgfpathlineto{\pgfqpoint{3.490606in}{1.899501in}}%
\pgfpathlineto{\pgfqpoint{3.498126in}{1.906979in}}%
\pgfpathlineto{\pgfqpoint{3.500633in}{1.912130in}}%
\pgfpathlineto{\pgfqpoint{3.503139in}{1.928372in}}%
\pgfpathlineto{\pgfqpoint{3.505646in}{1.929975in}}%
\pgfpathlineto{\pgfqpoint{3.508153in}{1.930083in}}%
\pgfpathlineto{\pgfqpoint{3.510659in}{1.939935in}}%
\pgfpathlineto{\pgfqpoint{3.513166in}{1.944103in}}%
\pgfpathlineto{\pgfqpoint{3.518179in}{1.945438in}}%
\pgfpathlineto{\pgfqpoint{3.520686in}{1.949316in}}%
\pgfpathlineto{\pgfqpoint{3.523193in}{1.950979in}}%
\pgfpathlineto{\pgfqpoint{3.525699in}{1.956086in}}%
\pgfpathlineto{\pgfqpoint{3.528206in}{1.957016in}}%
\pgfpathlineto{\pgfqpoint{3.533219in}{1.975024in}}%
\pgfpathlineto{\pgfqpoint{3.538233in}{1.976067in}}%
\pgfpathlineto{\pgfqpoint{3.543246in}{1.982485in}}%
\pgfpathlineto{\pgfqpoint{3.545753in}{1.983875in}}%
\pgfpathlineto{\pgfqpoint{3.548259in}{1.986544in}}%
\pgfpathlineto{\pgfqpoint{3.550766in}{1.992480in}}%
\pgfpathlineto{\pgfqpoint{3.555779in}{1.996282in}}%
\pgfpathlineto{\pgfqpoint{3.558286in}{2.001655in}}%
\pgfpathlineto{\pgfqpoint{3.560793in}{2.004047in}}%
\pgfpathlineto{\pgfqpoint{3.563299in}{2.012774in}}%
\pgfpathlineto{\pgfqpoint{3.568313in}{2.015416in}}%
\pgfpathlineto{\pgfqpoint{3.573326in}{2.022415in}}%
\pgfpathlineto{\pgfqpoint{3.575832in}{2.029682in}}%
\pgfpathlineto{\pgfqpoint{3.583352in}{2.035383in}}%
\pgfpathlineto{\pgfqpoint{3.585859in}{2.045072in}}%
\pgfpathlineto{\pgfqpoint{3.590872in}{2.047654in}}%
\pgfpathlineto{\pgfqpoint{3.593379in}{2.051799in}}%
\pgfpathlineto{\pgfqpoint{3.595886in}{2.061275in}}%
\pgfpathlineto{\pgfqpoint{3.598392in}{2.062852in}}%
\pgfpathlineto{\pgfqpoint{3.603406in}{2.075086in}}%
\pgfpathlineto{\pgfqpoint{3.608419in}{2.075719in}}%
\pgfpathlineto{\pgfqpoint{3.613432in}{2.082420in}}%
\pgfpathlineto{\pgfqpoint{3.623459in}{2.086586in}}%
\pgfpathlineto{\pgfqpoint{3.625966in}{2.093362in}}%
\pgfpathlineto{\pgfqpoint{3.628472in}{2.094047in}}%
\pgfpathlineto{\pgfqpoint{3.635992in}{2.102309in}}%
\pgfpathlineto{\pgfqpoint{3.641005in}{2.107130in}}%
\pgfpathlineto{\pgfqpoint{3.643512in}{2.109395in}}%
\pgfpathlineto{\pgfqpoint{3.648525in}{2.118889in}}%
\pgfpathlineto{\pgfqpoint{3.653539in}{2.124013in}}%
\pgfpathlineto{\pgfqpoint{3.656045in}{2.124934in}}%
\pgfpathlineto{\pgfqpoint{3.661059in}{2.141591in}}%
\pgfpathlineto{\pgfqpoint{3.671085in}{2.149464in}}%
\pgfpathlineto{\pgfqpoint{3.673592in}{2.154987in}}%
\pgfpathlineto{\pgfqpoint{3.676099in}{2.157512in}}%
\pgfpathlineto{\pgfqpoint{3.678605in}{2.157521in}}%
\pgfpathlineto{\pgfqpoint{3.683619in}{2.163370in}}%
\pgfpathlineto{\pgfqpoint{3.688632in}{2.166070in}}%
\pgfpathlineto{\pgfqpoint{3.691139in}{2.170255in}}%
\pgfpathlineto{\pgfqpoint{3.693645in}{2.172107in}}%
\pgfpathlineto{\pgfqpoint{3.696152in}{2.172335in}}%
\pgfpathlineto{\pgfqpoint{3.698659in}{2.181933in}}%
\pgfpathlineto{\pgfqpoint{3.701165in}{2.181953in}}%
\pgfpathlineto{\pgfqpoint{3.703672in}{2.186088in}}%
\pgfpathlineto{\pgfqpoint{3.706178in}{2.198963in}}%
\pgfpathlineto{\pgfqpoint{3.711192in}{2.201464in}}%
\pgfpathlineto{\pgfqpoint{3.713698in}{2.204152in}}%
\pgfpathlineto{\pgfqpoint{3.716205in}{2.210423in}}%
\pgfpathlineto{\pgfqpoint{3.718712in}{2.221988in}}%
\pgfpathlineto{\pgfqpoint{3.721218in}{2.226623in}}%
\pgfpathlineto{\pgfqpoint{3.723725in}{2.235836in}}%
\pgfpathlineto{\pgfqpoint{3.726232in}{2.235892in}}%
\pgfpathlineto{\pgfqpoint{3.728738in}{2.239751in}}%
\pgfpathlineto{\pgfqpoint{3.731245in}{2.247495in}}%
\pgfpathlineto{\pgfqpoint{3.736258in}{2.267193in}}%
\pgfpathlineto{\pgfqpoint{3.746285in}{2.305275in}}%
\pgfpathlineto{\pgfqpoint{5.541049in}{2.306325in}}%
\pgfpathlineto{\pgfqpoint{5.618756in}{2.309247in}}%
\pgfpathlineto{\pgfqpoint{5.626276in}{2.310928in}}%
\pgfpathlineto{\pgfqpoint{5.627232in}{2.315275in}}%
\pgfpathlineto{\pgfqpoint{5.627232in}{2.315275in}}%
\pgfusepath{stroke}%
\end{pgfscope}%
\begin{pgfscope}%
\pgfpathrectangle{\pgfqpoint{0.708220in}{0.535823in}}{\pgfqpoint{5.013309in}{1.769453in}}%
\pgfusepath{clip}%
\pgfsetrectcap%
\pgfsetroundjoin%
\pgfsetlinewidth{1.003750pt}%
\definecolor{currentstroke}{rgb}{1.000000,0.647059,0.000000}%
\pgfsetstrokecolor{currentstroke}%
\pgfsetdash{}{0pt}%
\pgfpathmoveto{\pgfqpoint{0.708220in}{0.606902in}}%
\pgfpathlineto{\pgfqpoint{0.710727in}{0.606909in}}%
\pgfpathlineto{\pgfqpoint{0.713233in}{0.623389in}}%
\pgfpathlineto{\pgfqpoint{0.718246in}{0.625894in}}%
\pgfpathlineto{\pgfqpoint{0.720753in}{0.628848in}}%
\pgfpathlineto{\pgfqpoint{0.723260in}{0.629962in}}%
\pgfpathlineto{\pgfqpoint{0.725766in}{0.634766in}}%
\pgfpathlineto{\pgfqpoint{0.728273in}{0.635676in}}%
\pgfpathlineto{\pgfqpoint{0.730780in}{0.649484in}}%
\pgfpathlineto{\pgfqpoint{0.735793in}{0.651568in}}%
\pgfpathlineto{\pgfqpoint{0.740806in}{0.666362in}}%
\pgfpathlineto{\pgfqpoint{0.743313in}{0.667704in}}%
\pgfpathlineto{\pgfqpoint{0.745820in}{0.672109in}}%
\pgfpathlineto{\pgfqpoint{0.750833in}{0.673462in}}%
\pgfpathlineto{\pgfqpoint{0.753340in}{0.684636in}}%
\pgfpathlineto{\pgfqpoint{0.755846in}{0.684678in}}%
\pgfpathlineto{\pgfqpoint{0.758353in}{0.689845in}}%
\pgfpathlineto{\pgfqpoint{0.763366in}{0.695312in}}%
\pgfpathlineto{\pgfqpoint{0.765873in}{0.700207in}}%
\pgfpathlineto{\pgfqpoint{0.768380in}{0.700521in}}%
\pgfpathlineto{\pgfqpoint{0.770886in}{0.705081in}}%
\pgfpathlineto{\pgfqpoint{0.780913in}{0.708075in}}%
\pgfpathlineto{\pgfqpoint{0.783419in}{0.708106in}}%
\pgfpathlineto{\pgfqpoint{0.788433in}{0.729832in}}%
\pgfpathlineto{\pgfqpoint{0.790939in}{0.732282in}}%
\pgfpathlineto{\pgfqpoint{0.793446in}{0.732707in}}%
\pgfpathlineto{\pgfqpoint{0.795953in}{0.735926in}}%
\pgfpathlineto{\pgfqpoint{0.798459in}{0.736115in}}%
\pgfpathlineto{\pgfqpoint{0.800966in}{0.738305in}}%
\pgfpathlineto{\pgfqpoint{0.803473in}{0.738459in}}%
\pgfpathlineto{\pgfqpoint{0.810993in}{0.747331in}}%
\pgfpathlineto{\pgfqpoint{0.816006in}{0.748383in}}%
\pgfpathlineto{\pgfqpoint{0.818513in}{0.755766in}}%
\pgfpathlineto{\pgfqpoint{0.821019in}{0.758654in}}%
\pgfpathlineto{\pgfqpoint{0.831046in}{0.760097in}}%
\pgfpathlineto{\pgfqpoint{0.838566in}{0.761299in}}%
\pgfpathlineto{\pgfqpoint{0.863632in}{0.762483in}}%
\pgfpathlineto{\pgfqpoint{0.871152in}{0.765658in}}%
\pgfpathlineto{\pgfqpoint{0.873659in}{0.767678in}}%
\pgfpathlineto{\pgfqpoint{0.938832in}{0.772826in}}%
\pgfpathlineto{\pgfqpoint{0.948859in}{0.776772in}}%
\pgfpathlineto{\pgfqpoint{1.076698in}{0.786923in}}%
\pgfpathlineto{\pgfqpoint{1.106778in}{0.788155in}}%
\pgfpathlineto{\pgfqpoint{1.164431in}{0.790977in}}%
\pgfpathlineto{\pgfqpoint{1.171951in}{0.793695in}}%
\pgfpathlineto{\pgfqpoint{1.214564in}{0.795350in}}%
\pgfpathlineto{\pgfqpoint{1.237124in}{0.796477in}}%
\pgfpathlineto{\pgfqpoint{1.269710in}{0.797400in}}%
\pgfpathlineto{\pgfqpoint{1.282244in}{0.798785in}}%
\pgfpathlineto{\pgfqpoint{1.287257in}{0.798855in}}%
\pgfpathlineto{\pgfqpoint{1.289764in}{0.800419in}}%
\pgfpathlineto{\pgfqpoint{1.294777in}{0.801123in}}%
\pgfpathlineto{\pgfqpoint{1.299790in}{0.802657in}}%
\pgfpathlineto{\pgfqpoint{1.312324in}{0.803438in}}%
\pgfpathlineto{\pgfqpoint{1.374990in}{0.807104in}}%
\pgfpathlineto{\pgfqpoint{1.390030in}{0.808679in}}%
\pgfpathlineto{\pgfqpoint{1.412590in}{0.809712in}}%
\pgfpathlineto{\pgfqpoint{1.480269in}{0.811762in}}%
\pgfpathlineto{\pgfqpoint{1.495309in}{0.813161in}}%
\pgfpathlineto{\pgfqpoint{1.502829in}{0.814422in}}%
\pgfpathlineto{\pgfqpoint{1.573016in}{0.817704in}}%
\pgfpathlineto{\pgfqpoint{1.618135in}{0.818989in}}%
\pgfpathlineto{\pgfqpoint{1.700855in}{0.824968in}}%
\pgfpathlineto{\pgfqpoint{1.708375in}{0.826512in}}%
\pgfpathlineto{\pgfqpoint{1.718402in}{0.827303in}}%
\pgfpathlineto{\pgfqpoint{1.720908in}{0.830188in}}%
\pgfpathlineto{\pgfqpoint{1.730935in}{0.831369in}}%
\pgfpathlineto{\pgfqpoint{1.733442in}{0.831758in}}%
\pgfpathlineto{\pgfqpoint{1.735948in}{0.834437in}}%
\pgfpathlineto{\pgfqpoint{1.745975in}{0.836267in}}%
\pgfpathlineto{\pgfqpoint{1.768535in}{0.837878in}}%
\pgfpathlineto{\pgfqpoint{1.786081in}{0.840893in}}%
\pgfpathlineto{\pgfqpoint{1.896374in}{0.844206in}}%
\pgfpathlineto{\pgfqpoint{1.903894in}{0.845030in}}%
\pgfpathlineto{\pgfqpoint{1.913921in}{0.846116in}}%
\pgfpathlineto{\pgfqpoint{1.918934in}{0.847616in}}%
\pgfpathlineto{\pgfqpoint{1.936481in}{0.848968in}}%
\pgfpathlineto{\pgfqpoint{1.964054in}{0.851392in}}%
\pgfpathlineto{\pgfqpoint{1.976587in}{0.854872in}}%
\pgfpathlineto{\pgfqpoint{1.991627in}{0.855780in}}%
\pgfpathlineto{\pgfqpoint{2.009173in}{0.861813in}}%
\pgfpathlineto{\pgfqpoint{2.021707in}{0.871676in}}%
\pgfpathlineto{\pgfqpoint{2.024213in}{0.872123in}}%
\pgfpathlineto{\pgfqpoint{2.029227in}{0.875762in}}%
\pgfpathlineto{\pgfqpoint{2.034240in}{0.878426in}}%
\pgfpathlineto{\pgfqpoint{2.039253in}{0.878652in}}%
\pgfpathlineto{\pgfqpoint{2.041760in}{0.884163in}}%
\pgfpathlineto{\pgfqpoint{2.044267in}{0.884449in}}%
\pgfpathlineto{\pgfqpoint{2.054293in}{0.889875in}}%
\pgfpathlineto{\pgfqpoint{2.076853in}{0.891575in}}%
\pgfpathlineto{\pgfqpoint{2.099413in}{0.896842in}}%
\pgfpathlineto{\pgfqpoint{2.101920in}{0.898414in}}%
\pgfpathlineto{\pgfqpoint{2.106933in}{0.898667in}}%
\pgfpathlineto{\pgfqpoint{2.109440in}{0.900634in}}%
\pgfpathlineto{\pgfqpoint{2.129493in}{0.904179in}}%
\pgfpathlineto{\pgfqpoint{2.139520in}{0.907678in}}%
\pgfpathlineto{\pgfqpoint{2.144533in}{0.907678in}}%
\pgfpathlineto{\pgfqpoint{2.149546in}{0.912064in}}%
\pgfpathlineto{\pgfqpoint{2.152053in}{0.912082in}}%
\pgfpathlineto{\pgfqpoint{2.157066in}{0.915705in}}%
\pgfpathlineto{\pgfqpoint{2.159573in}{0.917996in}}%
\pgfpathlineto{\pgfqpoint{2.169599in}{0.918832in}}%
\pgfpathlineto{\pgfqpoint{2.219732in}{0.929019in}}%
\pgfpathlineto{\pgfqpoint{2.232266in}{0.931189in}}%
\pgfpathlineto{\pgfqpoint{2.237279in}{0.932607in}}%
\pgfpathlineto{\pgfqpoint{2.247306in}{0.933156in}}%
\pgfpathlineto{\pgfqpoint{2.262346in}{0.936718in}}%
\pgfpathlineto{\pgfqpoint{2.267359in}{0.936963in}}%
\pgfpathlineto{\pgfqpoint{2.269866in}{0.940468in}}%
\pgfpathlineto{\pgfqpoint{2.274879in}{0.940691in}}%
\pgfpathlineto{\pgfqpoint{2.279892in}{0.943878in}}%
\pgfpathlineto{\pgfqpoint{2.294932in}{0.945338in}}%
\pgfpathlineto{\pgfqpoint{2.302452in}{0.949308in}}%
\pgfpathlineto{\pgfqpoint{2.307465in}{0.950307in}}%
\pgfpathlineto{\pgfqpoint{2.309972in}{0.954728in}}%
\pgfpathlineto{\pgfqpoint{2.317492in}{0.958122in}}%
\pgfpathlineto{\pgfqpoint{2.319999in}{0.958122in}}%
\pgfpathlineto{\pgfqpoint{2.322505in}{0.961452in}}%
\pgfpathlineto{\pgfqpoint{2.347572in}{0.961452in}}%
\pgfpathlineto{\pgfqpoint{2.352585in}{0.964722in}}%
\pgfpathlineto{\pgfqpoint{2.382665in}{0.964722in}}%
\pgfpathlineto{\pgfqpoint{2.390185in}{0.967933in}}%
\pgfpathlineto{\pgfqpoint{2.397705in}{0.968106in}}%
\pgfpathlineto{\pgfqpoint{2.400212in}{0.970862in}}%
\pgfpathlineto{\pgfqpoint{2.402718in}{0.971088in}}%
\pgfpathlineto{\pgfqpoint{2.405225in}{0.974187in}}%
\pgfpathlineto{\pgfqpoint{2.412745in}{0.974187in}}%
\pgfpathlineto{\pgfqpoint{2.417758in}{0.976648in}}%
\pgfpathlineto{\pgfqpoint{2.422771in}{0.977702in}}%
\pgfpathlineto{\pgfqpoint{2.425278in}{0.981972in}}%
\pgfpathlineto{\pgfqpoint{2.430291in}{0.983176in}}%
\pgfpathlineto{\pgfqpoint{2.435305in}{0.989854in}}%
\pgfpathlineto{\pgfqpoint{2.440318in}{0.992136in}}%
\pgfpathlineto{\pgfqpoint{2.442825in}{0.994893in}}%
\pgfpathlineto{\pgfqpoint{2.450345in}{0.994893in}}%
\pgfpathlineto{\pgfqpoint{2.462878in}{1.000284in}}%
\pgfpathlineto{\pgfqpoint{2.465385in}{1.000284in}}%
\pgfpathlineto{\pgfqpoint{2.470398in}{1.002352in}}%
\pgfpathlineto{\pgfqpoint{2.485438in}{1.006159in}}%
\pgfpathlineto{\pgfqpoint{2.507998in}{1.010602in}}%
\pgfpathlineto{\pgfqpoint{2.513011in}{1.010602in}}%
\pgfpathlineto{\pgfqpoint{2.515518in}{1.013067in}}%
\pgfpathlineto{\pgfqpoint{2.530558in}{1.013307in}}%
\pgfpathlineto{\pgfqpoint{2.533064in}{1.015310in}}%
\pgfpathlineto{\pgfqpoint{2.540584in}{1.016406in}}%
\pgfpathlineto{\pgfqpoint{2.543091in}{1.017967in}}%
\pgfpathlineto{\pgfqpoint{2.545598in}{1.017967in}}%
\pgfpathlineto{\pgfqpoint{2.565651in}{1.029560in}}%
\pgfpathlineto{\pgfqpoint{2.568157in}{1.033079in}}%
\pgfpathlineto{\pgfqpoint{2.575677in}{1.036603in}}%
\pgfpathlineto{\pgfqpoint{2.578184in}{1.037702in}}%
\pgfpathlineto{\pgfqpoint{2.583197in}{1.045563in}}%
\pgfpathlineto{\pgfqpoint{2.585704in}{1.046600in}}%
\pgfpathlineto{\pgfqpoint{2.588211in}{1.049706in}}%
\pgfpathlineto{\pgfqpoint{2.590717in}{1.050485in}}%
\pgfpathlineto{\pgfqpoint{2.593224in}{1.054153in}}%
\pgfpathlineto{\pgfqpoint{2.595731in}{1.055007in}}%
\pgfpathlineto{\pgfqpoint{2.598237in}{1.061587in}}%
\pgfpathlineto{\pgfqpoint{2.600744in}{1.061599in}}%
\pgfpathlineto{\pgfqpoint{2.605757in}{1.063162in}}%
\pgfpathlineto{\pgfqpoint{2.608264in}{1.067308in}}%
\pgfpathlineto{\pgfqpoint{2.615784in}{1.068445in}}%
\pgfpathlineto{\pgfqpoint{2.618290in}{1.071560in}}%
\pgfpathlineto{\pgfqpoint{2.625810in}{1.074033in}}%
\pgfpathlineto{\pgfqpoint{2.630824in}{1.074221in}}%
\pgfpathlineto{\pgfqpoint{2.633330in}{1.075913in}}%
\pgfpathlineto{\pgfqpoint{2.663410in}{1.078809in}}%
\pgfpathlineto{\pgfqpoint{2.683464in}{1.082242in}}%
\pgfpathlineto{\pgfqpoint{2.698503in}{1.088592in}}%
\pgfpathlineto{\pgfqpoint{2.706023in}{1.089748in}}%
\pgfpathlineto{\pgfqpoint{2.708530in}{1.090170in}}%
\pgfpathlineto{\pgfqpoint{2.711037in}{1.095288in}}%
\pgfpathlineto{\pgfqpoint{2.713543in}{1.095383in}}%
\pgfpathlineto{\pgfqpoint{2.716050in}{1.100262in}}%
\pgfpathlineto{\pgfqpoint{2.718557in}{1.100333in}}%
\pgfpathlineto{\pgfqpoint{2.721063in}{1.105850in}}%
\pgfpathlineto{\pgfqpoint{2.723570in}{1.107111in}}%
\pgfpathlineto{\pgfqpoint{2.726077in}{1.111284in}}%
\pgfpathlineto{\pgfqpoint{2.728583in}{1.112633in}}%
\pgfpathlineto{\pgfqpoint{2.731090in}{1.116855in}}%
\pgfpathlineto{\pgfqpoint{2.738610in}{1.119921in}}%
\pgfpathlineto{\pgfqpoint{2.741117in}{1.123483in}}%
\pgfpathlineto{\pgfqpoint{2.746130in}{1.124121in}}%
\pgfpathlineto{\pgfqpoint{2.751143in}{1.129477in}}%
\pgfpathlineto{\pgfqpoint{2.756156in}{1.130496in}}%
\pgfpathlineto{\pgfqpoint{2.758663in}{1.134242in}}%
\pgfpathlineto{\pgfqpoint{2.761170in}{1.135853in}}%
\pgfpathlineto{\pgfqpoint{2.763676in}{1.140941in}}%
\pgfpathlineto{\pgfqpoint{2.768690in}{1.144363in}}%
\pgfpathlineto{\pgfqpoint{2.773703in}{1.152192in}}%
\pgfpathlineto{\pgfqpoint{2.788743in}{1.160552in}}%
\pgfpathlineto{\pgfqpoint{2.796263in}{1.163824in}}%
\pgfpathlineto{\pgfqpoint{2.798770in}{1.165359in}}%
\pgfpathlineto{\pgfqpoint{2.801276in}{1.170727in}}%
\pgfpathlineto{\pgfqpoint{2.803783in}{1.171026in}}%
\pgfpathlineto{\pgfqpoint{2.808796in}{1.176434in}}%
\pgfpathlineto{\pgfqpoint{2.813810in}{1.186028in}}%
\pgfpathlineto{\pgfqpoint{2.821329in}{1.191436in}}%
\pgfpathlineto{\pgfqpoint{2.823836in}{1.193920in}}%
\pgfpathlineto{\pgfqpoint{2.826343in}{1.194453in}}%
\pgfpathlineto{\pgfqpoint{2.831356in}{1.199497in}}%
\pgfpathlineto{\pgfqpoint{2.833863in}{1.199960in}}%
\pgfpathlineto{\pgfqpoint{2.836369in}{1.201786in}}%
\pgfpathlineto{\pgfqpoint{2.841383in}{1.201979in}}%
\pgfpathlineto{\pgfqpoint{2.843889in}{1.204028in}}%
\pgfpathlineto{\pgfqpoint{2.851409in}{1.205730in}}%
\pgfpathlineto{\pgfqpoint{2.856423in}{1.212429in}}%
\pgfpathlineto{\pgfqpoint{2.861436in}{1.214455in}}%
\pgfpathlineto{\pgfqpoint{2.863943in}{1.217824in}}%
\pgfpathlineto{\pgfqpoint{2.868956in}{1.220204in}}%
\pgfpathlineto{\pgfqpoint{2.871463in}{1.223956in}}%
\pgfpathlineto{\pgfqpoint{2.876476in}{1.225419in}}%
\pgfpathlineto{\pgfqpoint{2.878983in}{1.227871in}}%
\pgfpathlineto{\pgfqpoint{2.881489in}{1.228509in}}%
\pgfpathlineto{\pgfqpoint{2.886503in}{1.234004in}}%
\pgfpathlineto{\pgfqpoint{2.889009in}{1.235270in}}%
\pgfpathlineto{\pgfqpoint{2.891516in}{1.239231in}}%
\pgfpathlineto{\pgfqpoint{2.894022in}{1.240781in}}%
\pgfpathlineto{\pgfqpoint{2.896529in}{1.248672in}}%
\pgfpathlineto{\pgfqpoint{2.899036in}{1.249039in}}%
\pgfpathlineto{\pgfqpoint{2.909062in}{1.257240in}}%
\pgfpathlineto{\pgfqpoint{2.914076in}{1.257748in}}%
\pgfpathlineto{\pgfqpoint{2.919089in}{1.266031in}}%
\pgfpathlineto{\pgfqpoint{2.924102in}{1.268757in}}%
\pgfpathlineto{\pgfqpoint{2.929116in}{1.277943in}}%
\pgfpathlineto{\pgfqpoint{2.939142in}{1.279062in}}%
\pgfpathlineto{\pgfqpoint{2.944156in}{1.279516in}}%
\pgfpathlineto{\pgfqpoint{2.949169in}{1.283470in}}%
\pgfpathlineto{\pgfqpoint{2.954182in}{1.283587in}}%
\pgfpathlineto{\pgfqpoint{2.959195in}{1.286276in}}%
\pgfpathlineto{\pgfqpoint{2.966715in}{1.294391in}}%
\pgfpathlineto{\pgfqpoint{2.969222in}{1.294656in}}%
\pgfpathlineto{\pgfqpoint{2.979249in}{1.299659in}}%
\pgfpathlineto{\pgfqpoint{2.984262in}{1.302285in}}%
\pgfpathlineto{\pgfqpoint{2.986769in}{1.305040in}}%
\pgfpathlineto{\pgfqpoint{2.989275in}{1.305673in}}%
\pgfpathlineto{\pgfqpoint{2.991782in}{1.308077in}}%
\pgfpathlineto{\pgfqpoint{2.994289in}{1.308156in}}%
\pgfpathlineto{\pgfqpoint{2.996795in}{1.310311in}}%
\pgfpathlineto{\pgfqpoint{2.999302in}{1.310591in}}%
\pgfpathlineto{\pgfqpoint{3.006822in}{1.316923in}}%
\pgfpathlineto{\pgfqpoint{3.009329in}{1.317219in}}%
\pgfpathlineto{\pgfqpoint{3.011835in}{1.319142in}}%
\pgfpathlineto{\pgfqpoint{3.014342in}{1.324504in}}%
\pgfpathlineto{\pgfqpoint{3.019355in}{1.327436in}}%
\pgfpathlineto{\pgfqpoint{3.024369in}{1.328490in}}%
\pgfpathlineto{\pgfqpoint{3.031888in}{1.330195in}}%
\pgfpathlineto{\pgfqpoint{3.036902in}{1.343448in}}%
\pgfpathlineto{\pgfqpoint{3.041915in}{1.345149in}}%
\pgfpathlineto{\pgfqpoint{3.044422in}{1.345272in}}%
\pgfpathlineto{\pgfqpoint{3.051942in}{1.350051in}}%
\pgfpathlineto{\pgfqpoint{3.056955in}{1.361325in}}%
\pgfpathlineto{\pgfqpoint{3.064475in}{1.364369in}}%
\pgfpathlineto{\pgfqpoint{3.069488in}{1.368083in}}%
\pgfpathlineto{\pgfqpoint{3.071995in}{1.383517in}}%
\pgfpathlineto{\pgfqpoint{3.079515in}{1.385118in}}%
\pgfpathlineto{\pgfqpoint{3.082022in}{1.387049in}}%
\pgfpathlineto{\pgfqpoint{3.084528in}{1.387308in}}%
\pgfpathlineto{\pgfqpoint{3.087035in}{1.391124in}}%
\pgfpathlineto{\pgfqpoint{3.104581in}{1.397304in}}%
\pgfpathlineto{\pgfqpoint{3.107088in}{1.403113in}}%
\pgfpathlineto{\pgfqpoint{3.112101in}{1.403528in}}%
\pgfpathlineto{\pgfqpoint{3.119621in}{1.406804in}}%
\pgfpathlineto{\pgfqpoint{3.127141in}{1.417952in}}%
\pgfpathlineto{\pgfqpoint{3.129648in}{1.418165in}}%
\pgfpathlineto{\pgfqpoint{3.132155in}{1.420347in}}%
\pgfpathlineto{\pgfqpoint{3.134661in}{1.427682in}}%
\pgfpathlineto{\pgfqpoint{3.137168in}{1.427951in}}%
\pgfpathlineto{\pgfqpoint{3.139675in}{1.431343in}}%
\pgfpathlineto{\pgfqpoint{3.142181in}{1.438917in}}%
\pgfpathlineto{\pgfqpoint{3.149701in}{1.440082in}}%
\pgfpathlineto{\pgfqpoint{3.152208in}{1.445789in}}%
\pgfpathlineto{\pgfqpoint{3.159728in}{1.447890in}}%
\pgfpathlineto{\pgfqpoint{3.162234in}{1.456473in}}%
\pgfpathlineto{\pgfqpoint{3.164741in}{1.456887in}}%
\pgfpathlineto{\pgfqpoint{3.167248in}{1.460044in}}%
\pgfpathlineto{\pgfqpoint{3.169754in}{1.468980in}}%
\pgfpathlineto{\pgfqpoint{3.184794in}{1.477520in}}%
\pgfpathlineto{\pgfqpoint{3.187301in}{1.483046in}}%
\pgfpathlineto{\pgfqpoint{3.189808in}{1.496745in}}%
\pgfpathlineto{\pgfqpoint{3.194821in}{1.506115in}}%
\pgfpathlineto{\pgfqpoint{3.197328in}{1.507384in}}%
\pgfpathlineto{\pgfqpoint{3.202341in}{1.516441in}}%
\pgfpathlineto{\pgfqpoint{3.204848in}{1.518104in}}%
\pgfpathlineto{\pgfqpoint{3.207354in}{1.523608in}}%
\pgfpathlineto{\pgfqpoint{3.209861in}{1.524923in}}%
\pgfpathlineto{\pgfqpoint{3.212368in}{1.533622in}}%
\pgfpathlineto{\pgfqpoint{3.217381in}{1.535342in}}%
\pgfpathlineto{\pgfqpoint{3.219888in}{1.535812in}}%
\pgfpathlineto{\pgfqpoint{3.222394in}{1.539156in}}%
\pgfpathlineto{\pgfqpoint{3.227408in}{1.540341in}}%
\pgfpathlineto{\pgfqpoint{3.232421in}{1.541527in}}%
\pgfpathlineto{\pgfqpoint{3.239941in}{1.546267in}}%
\pgfpathlineto{\pgfqpoint{3.242447in}{1.546660in}}%
\pgfpathlineto{\pgfqpoint{3.244954in}{1.551703in}}%
\pgfpathlineto{\pgfqpoint{3.247461in}{1.551892in}}%
\pgfpathlineto{\pgfqpoint{3.254981in}{1.558392in}}%
\pgfpathlineto{\pgfqpoint{3.257487in}{1.563761in}}%
\pgfpathlineto{\pgfqpoint{3.259994in}{1.565489in}}%
\pgfpathlineto{\pgfqpoint{3.265007in}{1.566197in}}%
\pgfpathlineto{\pgfqpoint{3.267514in}{1.568955in}}%
\pgfpathlineto{\pgfqpoint{3.272527in}{1.570445in}}%
\pgfpathlineto{\pgfqpoint{3.275034in}{1.576852in}}%
\pgfpathlineto{\pgfqpoint{3.280047in}{1.577923in}}%
\pgfpathlineto{\pgfqpoint{3.282554in}{1.578594in}}%
\pgfpathlineto{\pgfqpoint{3.292581in}{1.590414in}}%
\pgfpathlineto{\pgfqpoint{3.295087in}{1.611414in}}%
\pgfpathlineto{\pgfqpoint{3.297594in}{1.612427in}}%
\pgfpathlineto{\pgfqpoint{3.300100in}{1.615207in}}%
\pgfpathlineto{\pgfqpoint{3.302607in}{1.623116in}}%
\pgfpathlineto{\pgfqpoint{3.305114in}{1.625214in}}%
\pgfpathlineto{\pgfqpoint{3.307620in}{1.635283in}}%
\pgfpathlineto{\pgfqpoint{3.310127in}{1.637206in}}%
\pgfpathlineto{\pgfqpoint{3.315140in}{1.646379in}}%
\pgfpathlineto{\pgfqpoint{3.320154in}{1.650634in}}%
\pgfpathlineto{\pgfqpoint{3.322660in}{1.664933in}}%
\pgfpathlineto{\pgfqpoint{3.325167in}{1.671129in}}%
\pgfpathlineto{\pgfqpoint{3.327674in}{1.672359in}}%
\pgfpathlineto{\pgfqpoint{3.342714in}{1.688901in}}%
\pgfpathlineto{\pgfqpoint{3.345220in}{1.690671in}}%
\pgfpathlineto{\pgfqpoint{3.347727in}{1.696529in}}%
\pgfpathlineto{\pgfqpoint{3.350234in}{1.699035in}}%
\pgfpathlineto{\pgfqpoint{3.355247in}{1.709010in}}%
\pgfpathlineto{\pgfqpoint{3.362767in}{1.716229in}}%
\pgfpathlineto{\pgfqpoint{3.365273in}{1.716344in}}%
\pgfpathlineto{\pgfqpoint{3.367780in}{1.718658in}}%
\pgfpathlineto{\pgfqpoint{3.370287in}{1.719155in}}%
\pgfpathlineto{\pgfqpoint{3.372793in}{1.722704in}}%
\pgfpathlineto{\pgfqpoint{3.375300in}{1.723865in}}%
\pgfpathlineto{\pgfqpoint{3.377807in}{1.727748in}}%
\pgfpathlineto{\pgfqpoint{3.380313in}{1.727819in}}%
\pgfpathlineto{\pgfqpoint{3.382820in}{1.730885in}}%
\pgfpathlineto{\pgfqpoint{3.385327in}{1.740384in}}%
\pgfpathlineto{\pgfqpoint{3.392847in}{1.746879in}}%
\pgfpathlineto{\pgfqpoint{3.397860in}{1.748233in}}%
\pgfpathlineto{\pgfqpoint{3.405380in}{1.751907in}}%
\pgfpathlineto{\pgfqpoint{3.407887in}{1.754713in}}%
\pgfpathlineto{\pgfqpoint{3.410393in}{1.755192in}}%
\pgfpathlineto{\pgfqpoint{3.412900in}{1.759154in}}%
\pgfpathlineto{\pgfqpoint{3.415407in}{1.760686in}}%
\pgfpathlineto{\pgfqpoint{3.417913in}{1.760762in}}%
\pgfpathlineto{\pgfqpoint{3.420420in}{1.770908in}}%
\pgfpathlineto{\pgfqpoint{3.425433in}{1.773925in}}%
\pgfpathlineto{\pgfqpoint{3.427940in}{1.779408in}}%
\pgfpathlineto{\pgfqpoint{3.430447in}{1.781034in}}%
\pgfpathlineto{\pgfqpoint{3.432953in}{1.784661in}}%
\pgfpathlineto{\pgfqpoint{3.435460in}{1.784714in}}%
\pgfpathlineto{\pgfqpoint{3.437966in}{1.787402in}}%
\pgfpathlineto{\pgfqpoint{3.440473in}{1.787511in}}%
\pgfpathlineto{\pgfqpoint{3.442980in}{1.790096in}}%
\pgfpathlineto{\pgfqpoint{3.455513in}{1.819570in}}%
\pgfpathlineto{\pgfqpoint{3.458020in}{1.824313in}}%
\pgfpathlineto{\pgfqpoint{3.460526in}{1.825823in}}%
\pgfpathlineto{\pgfqpoint{3.465540in}{1.839623in}}%
\pgfpathlineto{\pgfqpoint{3.468046in}{1.840662in}}%
\pgfpathlineto{\pgfqpoint{3.470553in}{1.868170in}}%
\pgfpathlineto{\pgfqpoint{3.478073in}{1.879677in}}%
\pgfpathlineto{\pgfqpoint{3.480580in}{1.881326in}}%
\pgfpathlineto{\pgfqpoint{3.483086in}{1.889862in}}%
\pgfpathlineto{\pgfqpoint{3.485593in}{1.893734in}}%
\pgfpathlineto{\pgfqpoint{3.488100in}{1.894724in}}%
\pgfpathlineto{\pgfqpoint{3.490606in}{1.905931in}}%
\pgfpathlineto{\pgfqpoint{3.495620in}{1.907778in}}%
\pgfpathlineto{\pgfqpoint{3.498126in}{1.910910in}}%
\pgfpathlineto{\pgfqpoint{3.500633in}{1.928372in}}%
\pgfpathlineto{\pgfqpoint{3.503139in}{1.930083in}}%
\pgfpathlineto{\pgfqpoint{3.505646in}{1.944999in}}%
\pgfpathlineto{\pgfqpoint{3.513166in}{1.950979in}}%
\pgfpathlineto{\pgfqpoint{3.515673in}{1.957016in}}%
\pgfpathlineto{\pgfqpoint{3.518179in}{1.960113in}}%
\pgfpathlineto{\pgfqpoint{3.520686in}{1.966126in}}%
\pgfpathlineto{\pgfqpoint{3.525699in}{1.968936in}}%
\pgfpathlineto{\pgfqpoint{3.528206in}{1.975024in}}%
\pgfpathlineto{\pgfqpoint{3.530713in}{1.976067in}}%
\pgfpathlineto{\pgfqpoint{3.548259in}{2.001655in}}%
\pgfpathlineto{\pgfqpoint{3.550766in}{2.003555in}}%
\pgfpathlineto{\pgfqpoint{3.553273in}{2.004047in}}%
\pgfpathlineto{\pgfqpoint{3.555779in}{2.012774in}}%
\pgfpathlineto{\pgfqpoint{3.560793in}{2.018596in}}%
\pgfpathlineto{\pgfqpoint{3.563299in}{2.022415in}}%
\pgfpathlineto{\pgfqpoint{3.565806in}{2.029682in}}%
\pgfpathlineto{\pgfqpoint{3.570819in}{2.032768in}}%
\pgfpathlineto{\pgfqpoint{3.573326in}{2.033485in}}%
\pgfpathlineto{\pgfqpoint{3.578339in}{2.045072in}}%
\pgfpathlineto{\pgfqpoint{3.580846in}{2.048744in}}%
\pgfpathlineto{\pgfqpoint{3.583352in}{2.049347in}}%
\pgfpathlineto{\pgfqpoint{3.588366in}{2.053380in}}%
\pgfpathlineto{\pgfqpoint{3.593379in}{2.068845in}}%
\pgfpathlineto{\pgfqpoint{3.595886in}{2.070009in}}%
\pgfpathlineto{\pgfqpoint{3.598392in}{2.075086in}}%
\pgfpathlineto{\pgfqpoint{3.605912in}{2.076634in}}%
\pgfpathlineto{\pgfqpoint{3.613432in}{2.083334in}}%
\pgfpathlineto{\pgfqpoint{3.620952in}{2.086586in}}%
\pgfpathlineto{\pgfqpoint{3.623459in}{2.091566in}}%
\pgfpathlineto{\pgfqpoint{3.628472in}{2.094047in}}%
\pgfpathlineto{\pgfqpoint{3.630979in}{2.102004in}}%
\pgfpathlineto{\pgfqpoint{3.633486in}{2.102309in}}%
\pgfpathlineto{\pgfqpoint{3.638499in}{2.107130in}}%
\pgfpathlineto{\pgfqpoint{3.641005in}{2.109395in}}%
\pgfpathlineto{\pgfqpoint{3.643512in}{2.118889in}}%
\pgfpathlineto{\pgfqpoint{3.648525in}{2.121624in}}%
\pgfpathlineto{\pgfqpoint{3.653539in}{2.124934in}}%
\pgfpathlineto{\pgfqpoint{3.656045in}{2.135601in}}%
\pgfpathlineto{\pgfqpoint{3.658552in}{2.135617in}}%
\pgfpathlineto{\pgfqpoint{3.661059in}{2.141591in}}%
\pgfpathlineto{\pgfqpoint{3.663565in}{2.143873in}}%
\pgfpathlineto{\pgfqpoint{3.668579in}{2.144908in}}%
\pgfpathlineto{\pgfqpoint{3.671085in}{2.149405in}}%
\pgfpathlineto{\pgfqpoint{3.673592in}{2.150425in}}%
\pgfpathlineto{\pgfqpoint{3.676099in}{2.157224in}}%
\pgfpathlineto{\pgfqpoint{3.681112in}{2.157521in}}%
\pgfpathlineto{\pgfqpoint{3.686125in}{2.162566in}}%
\pgfpathlineto{\pgfqpoint{3.698659in}{2.167539in}}%
\pgfpathlineto{\pgfqpoint{3.703672in}{2.172107in}}%
\pgfpathlineto{\pgfqpoint{3.706178in}{2.180807in}}%
\pgfpathlineto{\pgfqpoint{3.708685in}{2.197677in}}%
\pgfpathlineto{\pgfqpoint{3.713698in}{2.200203in}}%
\pgfpathlineto{\pgfqpoint{3.716205in}{2.204152in}}%
\pgfpathlineto{\pgfqpoint{3.718712in}{2.205024in}}%
\pgfpathlineto{\pgfqpoint{3.721218in}{2.209046in}}%
\pgfpathlineto{\pgfqpoint{3.723725in}{2.210423in}}%
\pgfpathlineto{\pgfqpoint{3.726232in}{2.221988in}}%
\pgfpathlineto{\pgfqpoint{3.728738in}{2.226623in}}%
\pgfpathlineto{\pgfqpoint{3.731245in}{2.227043in}}%
\pgfpathlineto{\pgfqpoint{3.733752in}{2.229226in}}%
\pgfpathlineto{\pgfqpoint{3.736258in}{2.229682in}}%
\pgfpathlineto{\pgfqpoint{3.738765in}{2.235836in}}%
\pgfpathlineto{\pgfqpoint{3.741272in}{2.235892in}}%
\pgfpathlineto{\pgfqpoint{3.748792in}{2.247495in}}%
\pgfpathlineto{\pgfqpoint{3.751298in}{2.256492in}}%
\pgfpathlineto{\pgfqpoint{3.753805in}{2.277171in}}%
\pgfpathlineto{\pgfqpoint{3.756312in}{2.277918in}}%
\pgfpathlineto{\pgfqpoint{3.763832in}{2.305275in}}%
\pgfpathlineto{\pgfqpoint{5.134971in}{2.306310in}}%
\pgfpathlineto{\pgfqpoint{5.237744in}{2.310784in}}%
\pgfpathlineto{\pgfqpoint{5.272837in}{2.313563in}}%
\pgfpathlineto{\pgfqpoint{5.277851in}{2.314990in}}%
\pgfpathlineto{\pgfqpoint{5.279490in}{2.315275in}}%
\pgfpathlineto{\pgfqpoint{5.279490in}{2.315275in}}%
\pgfusepath{stroke}%
\end{pgfscope}%
\begin{pgfscope}%
\pgfpathrectangle{\pgfqpoint{0.708220in}{0.535823in}}{\pgfqpoint{5.013309in}{1.769453in}}%
\pgfusepath{clip}%
\pgfsetbuttcap%
\pgfsetroundjoin%
\pgfsetlinewidth{1.003750pt}%
\definecolor{currentstroke}{rgb}{1.000000,0.647059,0.000000}%
\pgfsetstrokecolor{currentstroke}%
\pgfsetdash{{1.000000pt}{1.650000pt}}{0.000000pt}%
\pgfpathmoveto{\pgfqpoint{0.708220in}{0.606902in}}%
\pgfpathlineto{\pgfqpoint{0.710727in}{0.606909in}}%
\pgfpathlineto{\pgfqpoint{0.713233in}{0.623389in}}%
\pgfpathlineto{\pgfqpoint{0.718246in}{0.625894in}}%
\pgfpathlineto{\pgfqpoint{0.720753in}{0.628848in}}%
\pgfpathlineto{\pgfqpoint{0.723260in}{0.629962in}}%
\pgfpathlineto{\pgfqpoint{0.725766in}{0.634766in}}%
\pgfpathlineto{\pgfqpoint{0.728273in}{0.635676in}}%
\pgfpathlineto{\pgfqpoint{0.730780in}{0.649484in}}%
\pgfpathlineto{\pgfqpoint{0.735793in}{0.651568in}}%
\pgfpathlineto{\pgfqpoint{0.740806in}{0.666362in}}%
\pgfpathlineto{\pgfqpoint{0.743313in}{0.667704in}}%
\pgfpathlineto{\pgfqpoint{0.745820in}{0.672109in}}%
\pgfpathlineto{\pgfqpoint{0.750833in}{0.673462in}}%
\pgfpathlineto{\pgfqpoint{0.753340in}{0.684636in}}%
\pgfpathlineto{\pgfqpoint{0.755846in}{0.684678in}}%
\pgfpathlineto{\pgfqpoint{0.758353in}{0.689845in}}%
\pgfpathlineto{\pgfqpoint{0.763366in}{0.695312in}}%
\pgfpathlineto{\pgfqpoint{0.765873in}{0.700207in}}%
\pgfpathlineto{\pgfqpoint{0.770886in}{0.700887in}}%
\pgfpathlineto{\pgfqpoint{0.773393in}{0.702709in}}%
\pgfpathlineto{\pgfqpoint{0.775900in}{0.702958in}}%
\pgfpathlineto{\pgfqpoint{0.778406in}{0.706984in}}%
\pgfpathlineto{\pgfqpoint{0.783419in}{0.708075in}}%
\pgfpathlineto{\pgfqpoint{0.788433in}{0.729832in}}%
\pgfpathlineto{\pgfqpoint{0.790939in}{0.732282in}}%
\pgfpathlineto{\pgfqpoint{0.793446in}{0.732707in}}%
\pgfpathlineto{\pgfqpoint{0.795953in}{0.735926in}}%
\pgfpathlineto{\pgfqpoint{0.798459in}{0.736115in}}%
\pgfpathlineto{\pgfqpoint{0.800966in}{0.738305in}}%
\pgfpathlineto{\pgfqpoint{0.803473in}{0.738459in}}%
\pgfpathlineto{\pgfqpoint{0.810993in}{0.747331in}}%
\pgfpathlineto{\pgfqpoint{0.816006in}{0.748383in}}%
\pgfpathlineto{\pgfqpoint{0.818513in}{0.755766in}}%
\pgfpathlineto{\pgfqpoint{0.821019in}{0.758654in}}%
\pgfpathlineto{\pgfqpoint{0.831046in}{0.760097in}}%
\pgfpathlineto{\pgfqpoint{0.838566in}{0.761299in}}%
\pgfpathlineto{\pgfqpoint{0.863632in}{0.762483in}}%
\pgfpathlineto{\pgfqpoint{0.871152in}{0.765658in}}%
\pgfpathlineto{\pgfqpoint{0.873659in}{0.767678in}}%
\pgfpathlineto{\pgfqpoint{0.938832in}{0.772826in}}%
\pgfpathlineto{\pgfqpoint{0.948859in}{0.776772in}}%
\pgfpathlineto{\pgfqpoint{1.076698in}{0.786923in}}%
\pgfpathlineto{\pgfqpoint{1.106778in}{0.788155in}}%
\pgfpathlineto{\pgfqpoint{1.164431in}{0.790977in}}%
\pgfpathlineto{\pgfqpoint{1.171951in}{0.793695in}}%
\pgfpathlineto{\pgfqpoint{1.214564in}{0.795350in}}%
\pgfpathlineto{\pgfqpoint{1.237124in}{0.796477in}}%
\pgfpathlineto{\pgfqpoint{1.269710in}{0.797400in}}%
\pgfpathlineto{\pgfqpoint{1.282244in}{0.798785in}}%
\pgfpathlineto{\pgfqpoint{1.287257in}{0.798855in}}%
\pgfpathlineto{\pgfqpoint{1.289764in}{0.800419in}}%
\pgfpathlineto{\pgfqpoint{1.294777in}{0.801123in}}%
\pgfpathlineto{\pgfqpoint{1.299790in}{0.802657in}}%
\pgfpathlineto{\pgfqpoint{1.312324in}{0.803438in}}%
\pgfpathlineto{\pgfqpoint{1.374990in}{0.807104in}}%
\pgfpathlineto{\pgfqpoint{1.390030in}{0.808679in}}%
\pgfpathlineto{\pgfqpoint{1.412590in}{0.809712in}}%
\pgfpathlineto{\pgfqpoint{1.480269in}{0.811762in}}%
\pgfpathlineto{\pgfqpoint{1.495309in}{0.813161in}}%
\pgfpathlineto{\pgfqpoint{1.502829in}{0.814422in}}%
\pgfpathlineto{\pgfqpoint{1.573016in}{0.817704in}}%
\pgfpathlineto{\pgfqpoint{1.618135in}{0.818989in}}%
\pgfpathlineto{\pgfqpoint{1.700855in}{0.824968in}}%
\pgfpathlineto{\pgfqpoint{1.708375in}{0.826512in}}%
\pgfpathlineto{\pgfqpoint{1.718402in}{0.827303in}}%
\pgfpathlineto{\pgfqpoint{1.720908in}{0.830188in}}%
\pgfpathlineto{\pgfqpoint{1.730935in}{0.831369in}}%
\pgfpathlineto{\pgfqpoint{1.733442in}{0.831758in}}%
\pgfpathlineto{\pgfqpoint{1.735948in}{0.834437in}}%
\pgfpathlineto{\pgfqpoint{1.745975in}{0.836267in}}%
\pgfpathlineto{\pgfqpoint{1.768535in}{0.837878in}}%
\pgfpathlineto{\pgfqpoint{1.786081in}{0.840893in}}%
\pgfpathlineto{\pgfqpoint{1.896374in}{0.844206in}}%
\pgfpathlineto{\pgfqpoint{1.926454in}{0.847695in}}%
\pgfpathlineto{\pgfqpoint{1.931467in}{0.848821in}}%
\pgfpathlineto{\pgfqpoint{1.961547in}{0.850374in}}%
\pgfpathlineto{\pgfqpoint{1.966560in}{0.851392in}}%
\pgfpathlineto{\pgfqpoint{1.979094in}{0.854872in}}%
\pgfpathlineto{\pgfqpoint{1.994134in}{0.855780in}}%
\pgfpathlineto{\pgfqpoint{2.009173in}{0.861813in}}%
\pgfpathlineto{\pgfqpoint{2.021707in}{0.871676in}}%
\pgfpathlineto{\pgfqpoint{2.024213in}{0.872123in}}%
\pgfpathlineto{\pgfqpoint{2.029227in}{0.875145in}}%
\pgfpathlineto{\pgfqpoint{2.034240in}{0.877384in}}%
\pgfpathlineto{\pgfqpoint{2.039253in}{0.878600in}}%
\pgfpathlineto{\pgfqpoint{2.041760in}{0.878652in}}%
\pgfpathlineto{\pgfqpoint{2.044267in}{0.884097in}}%
\pgfpathlineto{\pgfqpoint{2.049280in}{0.884449in}}%
\pgfpathlineto{\pgfqpoint{2.054293in}{0.886785in}}%
\pgfpathlineto{\pgfqpoint{2.061813in}{0.887370in}}%
\pgfpathlineto{\pgfqpoint{2.064320in}{0.888922in}}%
\pgfpathlineto{\pgfqpoint{2.111946in}{0.892309in}}%
\pgfpathlineto{\pgfqpoint{2.119466in}{0.893827in}}%
\pgfpathlineto{\pgfqpoint{2.142026in}{0.896133in}}%
\pgfpathlineto{\pgfqpoint{2.159573in}{0.897190in}}%
\pgfpathlineto{\pgfqpoint{2.164586in}{0.898667in}}%
\pgfpathlineto{\pgfqpoint{2.172106in}{0.900630in}}%
\pgfpathlineto{\pgfqpoint{2.192159in}{0.901691in}}%
\pgfpathlineto{\pgfqpoint{2.204693in}{0.903592in}}%
\pgfpathlineto{\pgfqpoint{2.219732in}{0.906363in}}%
\pgfpathlineto{\pgfqpoint{2.224746in}{0.907700in}}%
\pgfpathlineto{\pgfqpoint{2.229759in}{0.910115in}}%
\pgfpathlineto{\pgfqpoint{2.234772in}{0.913313in}}%
\pgfpathlineto{\pgfqpoint{2.237279in}{0.913336in}}%
\pgfpathlineto{\pgfqpoint{2.242292in}{0.915705in}}%
\pgfpathlineto{\pgfqpoint{2.252319in}{0.916695in}}%
\pgfpathlineto{\pgfqpoint{2.264852in}{0.917996in}}%
\pgfpathlineto{\pgfqpoint{2.284905in}{0.918832in}}%
\pgfpathlineto{\pgfqpoint{2.294932in}{0.919705in}}%
\pgfpathlineto{\pgfqpoint{2.322505in}{0.921625in}}%
\pgfpathlineto{\pgfqpoint{2.385172in}{0.924357in}}%
\pgfpathlineto{\pgfqpoint{2.405225in}{0.925795in}}%
\pgfpathlineto{\pgfqpoint{2.442825in}{0.927859in}}%
\pgfpathlineto{\pgfqpoint{2.447838in}{0.928864in}}%
\pgfpathlineto{\pgfqpoint{2.470398in}{0.930185in}}%
\pgfpathlineto{\pgfqpoint{2.513011in}{0.936767in}}%
\pgfpathlineto{\pgfqpoint{2.530558in}{0.939086in}}%
\pgfpathlineto{\pgfqpoint{2.535571in}{0.942403in}}%
\pgfpathlineto{\pgfqpoint{2.538078in}{0.945338in}}%
\pgfpathlineto{\pgfqpoint{2.540584in}{0.945726in}}%
\pgfpathlineto{\pgfqpoint{2.543091in}{0.949277in}}%
\pgfpathlineto{\pgfqpoint{2.555624in}{0.950307in}}%
\pgfpathlineto{\pgfqpoint{2.560637in}{0.955989in}}%
\pgfpathlineto{\pgfqpoint{2.563144in}{0.963035in}}%
\pgfpathlineto{\pgfqpoint{2.575677in}{0.968106in}}%
\pgfpathlineto{\pgfqpoint{2.590717in}{0.977702in}}%
\pgfpathlineto{\pgfqpoint{2.593224in}{0.981972in}}%
\pgfpathlineto{\pgfqpoint{2.595731in}{0.982082in}}%
\pgfpathlineto{\pgfqpoint{2.598237in}{0.986214in}}%
\pgfpathlineto{\pgfqpoint{2.603251in}{0.987280in}}%
\pgfpathlineto{\pgfqpoint{2.605757in}{0.989854in}}%
\pgfpathlineto{\pgfqpoint{2.608264in}{0.990509in}}%
\pgfpathlineto{\pgfqpoint{2.610771in}{0.992795in}}%
\pgfpathlineto{\pgfqpoint{2.613277in}{0.992858in}}%
\pgfpathlineto{\pgfqpoint{2.615784in}{0.996160in}}%
\pgfpathlineto{\pgfqpoint{2.618290in}{0.996470in}}%
\pgfpathlineto{\pgfqpoint{2.625810in}{1.001474in}}%
\pgfpathlineto{\pgfqpoint{2.630824in}{1.002805in}}%
\pgfpathlineto{\pgfqpoint{2.635837in}{1.004169in}}%
\pgfpathlineto{\pgfqpoint{2.638344in}{1.004505in}}%
\pgfpathlineto{\pgfqpoint{2.643357in}{1.006732in}}%
\pgfpathlineto{\pgfqpoint{2.645864in}{1.006982in}}%
\pgfpathlineto{\pgfqpoint{2.658397in}{1.015310in}}%
\pgfpathlineto{\pgfqpoint{2.663410in}{1.022626in}}%
\pgfpathlineto{\pgfqpoint{2.673437in}{1.029560in}}%
\pgfpathlineto{\pgfqpoint{2.675944in}{1.033079in}}%
\pgfpathlineto{\pgfqpoint{2.683464in}{1.036603in}}%
\pgfpathlineto{\pgfqpoint{2.688477in}{1.038812in}}%
\pgfpathlineto{\pgfqpoint{2.698503in}{1.049706in}}%
\pgfpathlineto{\pgfqpoint{2.701010in}{1.050485in}}%
\pgfpathlineto{\pgfqpoint{2.703517in}{1.054153in}}%
\pgfpathlineto{\pgfqpoint{2.708530in}{1.055007in}}%
\pgfpathlineto{\pgfqpoint{2.711037in}{1.061549in}}%
\pgfpathlineto{\pgfqpoint{2.713543in}{1.061587in}}%
\pgfpathlineto{\pgfqpoint{2.716050in}{1.063162in}}%
\pgfpathlineto{\pgfqpoint{2.718557in}{1.067308in}}%
\pgfpathlineto{\pgfqpoint{2.721063in}{1.074221in}}%
\pgfpathlineto{\pgfqpoint{2.731090in}{1.080742in}}%
\pgfpathlineto{\pgfqpoint{2.741117in}{1.083439in}}%
\pgfpathlineto{\pgfqpoint{2.746130in}{1.088096in}}%
\pgfpathlineto{\pgfqpoint{2.748637in}{1.089748in}}%
\pgfpathlineto{\pgfqpoint{2.751143in}{1.093141in}}%
\pgfpathlineto{\pgfqpoint{2.753650in}{1.107111in}}%
\pgfpathlineto{\pgfqpoint{2.756156in}{1.111284in}}%
\pgfpathlineto{\pgfqpoint{2.758663in}{1.112633in}}%
\pgfpathlineto{\pgfqpoint{2.761170in}{1.116855in}}%
\pgfpathlineto{\pgfqpoint{2.766183in}{1.118117in}}%
\pgfpathlineto{\pgfqpoint{2.768690in}{1.118181in}}%
\pgfpathlineto{\pgfqpoint{2.771196in}{1.119921in}}%
\pgfpathlineto{\pgfqpoint{2.773703in}{1.123483in}}%
\pgfpathlineto{\pgfqpoint{2.778716in}{1.124121in}}%
\pgfpathlineto{\pgfqpoint{2.781223in}{1.127324in}}%
\pgfpathlineto{\pgfqpoint{2.783730in}{1.127360in}}%
\pgfpathlineto{\pgfqpoint{2.788743in}{1.130248in}}%
\pgfpathlineto{\pgfqpoint{2.791250in}{1.130496in}}%
\pgfpathlineto{\pgfqpoint{2.793756in}{1.134242in}}%
\pgfpathlineto{\pgfqpoint{2.796263in}{1.135853in}}%
\pgfpathlineto{\pgfqpoint{2.798770in}{1.140941in}}%
\pgfpathlineto{\pgfqpoint{2.803783in}{1.144363in}}%
\pgfpathlineto{\pgfqpoint{2.808796in}{1.153777in}}%
\pgfpathlineto{\pgfqpoint{2.811303in}{1.154689in}}%
\pgfpathlineto{\pgfqpoint{2.813810in}{1.157212in}}%
\pgfpathlineto{\pgfqpoint{2.818823in}{1.158268in}}%
\pgfpathlineto{\pgfqpoint{2.821329in}{1.160552in}}%
\pgfpathlineto{\pgfqpoint{2.828849in}{1.163824in}}%
\pgfpathlineto{\pgfqpoint{2.831356in}{1.165359in}}%
\pgfpathlineto{\pgfqpoint{2.836369in}{1.173471in}}%
\pgfpathlineto{\pgfqpoint{2.838876in}{1.176434in}}%
\pgfpathlineto{\pgfqpoint{2.841383in}{1.176949in}}%
\pgfpathlineto{\pgfqpoint{2.843889in}{1.187606in}}%
\pgfpathlineto{\pgfqpoint{2.846396in}{1.188751in}}%
\pgfpathlineto{\pgfqpoint{2.851409in}{1.193920in}}%
\pgfpathlineto{\pgfqpoint{2.853916in}{1.194453in}}%
\pgfpathlineto{\pgfqpoint{2.858929in}{1.199240in}}%
\pgfpathlineto{\pgfqpoint{2.863943in}{1.199960in}}%
\pgfpathlineto{\pgfqpoint{2.866449in}{1.201786in}}%
\pgfpathlineto{\pgfqpoint{2.876476in}{1.203121in}}%
\pgfpathlineto{\pgfqpoint{2.886503in}{1.205730in}}%
\pgfpathlineto{\pgfqpoint{2.889009in}{1.206277in}}%
\pgfpathlineto{\pgfqpoint{2.894022in}{1.211179in}}%
\pgfpathlineto{\pgfqpoint{2.896529in}{1.213597in}}%
\pgfpathlineto{\pgfqpoint{2.899036in}{1.214455in}}%
\pgfpathlineto{\pgfqpoint{2.904049in}{1.217824in}}%
\pgfpathlineto{\pgfqpoint{2.911569in}{1.220204in}}%
\pgfpathlineto{\pgfqpoint{2.914076in}{1.223956in}}%
\pgfpathlineto{\pgfqpoint{2.926609in}{1.228509in}}%
\pgfpathlineto{\pgfqpoint{2.929116in}{1.231948in}}%
\pgfpathlineto{\pgfqpoint{2.931622in}{1.232155in}}%
\pgfpathlineto{\pgfqpoint{2.936636in}{1.235270in}}%
\pgfpathlineto{\pgfqpoint{2.939142in}{1.239231in}}%
\pgfpathlineto{\pgfqpoint{2.944156in}{1.240781in}}%
\pgfpathlineto{\pgfqpoint{2.946662in}{1.246250in}}%
\pgfpathlineto{\pgfqpoint{2.949169in}{1.248672in}}%
\pgfpathlineto{\pgfqpoint{2.951676in}{1.249039in}}%
\pgfpathlineto{\pgfqpoint{2.961702in}{1.257240in}}%
\pgfpathlineto{\pgfqpoint{2.964209in}{1.263131in}}%
\pgfpathlineto{\pgfqpoint{2.969222in}{1.268757in}}%
\pgfpathlineto{\pgfqpoint{2.971729in}{1.269391in}}%
\pgfpathlineto{\pgfqpoint{2.974235in}{1.278450in}}%
\pgfpathlineto{\pgfqpoint{2.994289in}{1.286276in}}%
\pgfpathlineto{\pgfqpoint{2.996795in}{1.292243in}}%
\pgfpathlineto{\pgfqpoint{3.001809in}{1.296165in}}%
\pgfpathlineto{\pgfqpoint{3.009329in}{1.298203in}}%
\pgfpathlineto{\pgfqpoint{3.014342in}{1.300502in}}%
\pgfpathlineto{\pgfqpoint{3.019355in}{1.302285in}}%
\pgfpathlineto{\pgfqpoint{3.021862in}{1.305040in}}%
\pgfpathlineto{\pgfqpoint{3.031888in}{1.308156in}}%
\pgfpathlineto{\pgfqpoint{3.034395in}{1.310311in}}%
\pgfpathlineto{\pgfqpoint{3.036902in}{1.310591in}}%
\pgfpathlineto{\pgfqpoint{3.039408in}{1.315936in}}%
\pgfpathlineto{\pgfqpoint{3.044422in}{1.317219in}}%
\pgfpathlineto{\pgfqpoint{3.049435in}{1.324504in}}%
\pgfpathlineto{\pgfqpoint{3.056955in}{1.328479in}}%
\pgfpathlineto{\pgfqpoint{3.061968in}{1.328820in}}%
\pgfpathlineto{\pgfqpoint{3.064475in}{1.329905in}}%
\pgfpathlineto{\pgfqpoint{3.066982in}{1.332410in}}%
\pgfpathlineto{\pgfqpoint{3.069488in}{1.337842in}}%
\pgfpathlineto{\pgfqpoint{3.071995in}{1.339441in}}%
\pgfpathlineto{\pgfqpoint{3.074502in}{1.343448in}}%
\pgfpathlineto{\pgfqpoint{3.079515in}{1.345149in}}%
\pgfpathlineto{\pgfqpoint{3.084528in}{1.345609in}}%
\pgfpathlineto{\pgfqpoint{3.094555in}{1.350051in}}%
\pgfpathlineto{\pgfqpoint{3.099568in}{1.361325in}}%
\pgfpathlineto{\pgfqpoint{3.112101in}{1.367008in}}%
\pgfpathlineto{\pgfqpoint{3.114608in}{1.368083in}}%
\pgfpathlineto{\pgfqpoint{3.119621in}{1.383517in}}%
\pgfpathlineto{\pgfqpoint{3.127141in}{1.385118in}}%
\pgfpathlineto{\pgfqpoint{3.129648in}{1.387049in}}%
\pgfpathlineto{\pgfqpoint{3.132155in}{1.392034in}}%
\pgfpathlineto{\pgfqpoint{3.139675in}{1.396513in}}%
\pgfpathlineto{\pgfqpoint{3.144688in}{1.398758in}}%
\pgfpathlineto{\pgfqpoint{3.147195in}{1.399885in}}%
\pgfpathlineto{\pgfqpoint{3.149701in}{1.403113in}}%
\pgfpathlineto{\pgfqpoint{3.152208in}{1.403128in}}%
\pgfpathlineto{\pgfqpoint{3.159728in}{1.406804in}}%
\pgfpathlineto{\pgfqpoint{3.167248in}{1.417952in}}%
\pgfpathlineto{\pgfqpoint{3.169754in}{1.418165in}}%
\pgfpathlineto{\pgfqpoint{3.172261in}{1.420347in}}%
\pgfpathlineto{\pgfqpoint{3.174768in}{1.420572in}}%
\pgfpathlineto{\pgfqpoint{3.177274in}{1.422296in}}%
\pgfpathlineto{\pgfqpoint{3.179781in}{1.422572in}}%
\pgfpathlineto{\pgfqpoint{3.182288in}{1.427682in}}%
\pgfpathlineto{\pgfqpoint{3.187301in}{1.428826in}}%
\pgfpathlineto{\pgfqpoint{3.192314in}{1.434887in}}%
\pgfpathlineto{\pgfqpoint{3.197328in}{1.443137in}}%
\pgfpathlineto{\pgfqpoint{3.199834in}{1.445789in}}%
\pgfpathlineto{\pgfqpoint{3.204848in}{1.447890in}}%
\pgfpathlineto{\pgfqpoint{3.207354in}{1.450680in}}%
\pgfpathlineto{\pgfqpoint{3.212368in}{1.452234in}}%
\pgfpathlineto{\pgfqpoint{3.214874in}{1.455605in}}%
\pgfpathlineto{\pgfqpoint{3.219888in}{1.458109in}}%
\pgfpathlineto{\pgfqpoint{3.227408in}{1.463010in}}%
\pgfpathlineto{\pgfqpoint{3.229914in}{1.468980in}}%
\pgfpathlineto{\pgfqpoint{3.247461in}{1.478645in}}%
\pgfpathlineto{\pgfqpoint{3.249967in}{1.482790in}}%
\pgfpathlineto{\pgfqpoint{3.252474in}{1.483046in}}%
\pgfpathlineto{\pgfqpoint{3.254981in}{1.493071in}}%
\pgfpathlineto{\pgfqpoint{3.257487in}{1.496745in}}%
\pgfpathlineto{\pgfqpoint{3.259994in}{1.497302in}}%
\pgfpathlineto{\pgfqpoint{3.262501in}{1.501145in}}%
\pgfpathlineto{\pgfqpoint{3.265007in}{1.501698in}}%
\pgfpathlineto{\pgfqpoint{3.267514in}{1.506115in}}%
\pgfpathlineto{\pgfqpoint{3.272527in}{1.509898in}}%
\pgfpathlineto{\pgfqpoint{3.277541in}{1.516441in}}%
\pgfpathlineto{\pgfqpoint{3.280047in}{1.518104in}}%
\pgfpathlineto{\pgfqpoint{3.282554in}{1.523608in}}%
\pgfpathlineto{\pgfqpoint{3.285061in}{1.524923in}}%
\pgfpathlineto{\pgfqpoint{3.287567in}{1.533622in}}%
\pgfpathlineto{\pgfqpoint{3.297594in}{1.535982in}}%
\pgfpathlineto{\pgfqpoint{3.300100in}{1.539156in}}%
\pgfpathlineto{\pgfqpoint{3.305114in}{1.540341in}}%
\pgfpathlineto{\pgfqpoint{3.310127in}{1.541527in}}%
\pgfpathlineto{\pgfqpoint{3.312634in}{1.546173in}}%
\pgfpathlineto{\pgfqpoint{3.322660in}{1.547396in}}%
\pgfpathlineto{\pgfqpoint{3.327674in}{1.549471in}}%
\pgfpathlineto{\pgfqpoint{3.335194in}{1.549794in}}%
\pgfpathlineto{\pgfqpoint{3.345220in}{1.557741in}}%
\pgfpathlineto{\pgfqpoint{3.350234in}{1.560007in}}%
\pgfpathlineto{\pgfqpoint{3.362767in}{1.566197in}}%
\pgfpathlineto{\pgfqpoint{3.365273in}{1.569370in}}%
\pgfpathlineto{\pgfqpoint{3.367780in}{1.570445in}}%
\pgfpathlineto{\pgfqpoint{3.372793in}{1.575855in}}%
\pgfpathlineto{\pgfqpoint{3.377807in}{1.576939in}}%
\pgfpathlineto{\pgfqpoint{3.380313in}{1.577057in}}%
\pgfpathlineto{\pgfqpoint{3.395353in}{1.587577in}}%
\pgfpathlineto{\pgfqpoint{3.402873in}{1.596071in}}%
\pgfpathlineto{\pgfqpoint{3.405380in}{1.596452in}}%
\pgfpathlineto{\pgfqpoint{3.407887in}{1.600344in}}%
\pgfpathlineto{\pgfqpoint{3.410393in}{1.611414in}}%
\pgfpathlineto{\pgfqpoint{3.415407in}{1.613024in}}%
\pgfpathlineto{\pgfqpoint{3.417913in}{1.618231in}}%
\pgfpathlineto{\pgfqpoint{3.420420in}{1.619263in}}%
\pgfpathlineto{\pgfqpoint{3.425433in}{1.623116in}}%
\pgfpathlineto{\pgfqpoint{3.427940in}{1.624930in}}%
\pgfpathlineto{\pgfqpoint{3.432953in}{1.631315in}}%
\pgfpathlineto{\pgfqpoint{3.435460in}{1.632202in}}%
\pgfpathlineto{\pgfqpoint{3.437966in}{1.635079in}}%
\pgfpathlineto{\pgfqpoint{3.440473in}{1.635469in}}%
\pgfpathlineto{\pgfqpoint{3.445486in}{1.637614in}}%
\pgfpathlineto{\pgfqpoint{3.447993in}{1.648451in}}%
\pgfpathlineto{\pgfqpoint{3.453006in}{1.649559in}}%
\pgfpathlineto{\pgfqpoint{3.458020in}{1.650855in}}%
\pgfpathlineto{\pgfqpoint{3.468046in}{1.653919in}}%
\pgfpathlineto{\pgfqpoint{3.475566in}{1.664727in}}%
\pgfpathlineto{\pgfqpoint{3.480580in}{1.664933in}}%
\pgfpathlineto{\pgfqpoint{3.493113in}{1.674493in}}%
\pgfpathlineto{\pgfqpoint{3.495620in}{1.680875in}}%
\pgfpathlineto{\pgfqpoint{3.498126in}{1.681747in}}%
\pgfpathlineto{\pgfqpoint{3.500633in}{1.685484in}}%
\pgfpathlineto{\pgfqpoint{3.505646in}{1.687882in}}%
\pgfpathlineto{\pgfqpoint{3.508153in}{1.689927in}}%
\pgfpathlineto{\pgfqpoint{3.515673in}{1.691417in}}%
\pgfpathlineto{\pgfqpoint{3.518179in}{1.694354in}}%
\pgfpathlineto{\pgfqpoint{3.520686in}{1.699471in}}%
\pgfpathlineto{\pgfqpoint{3.523193in}{1.700057in}}%
\pgfpathlineto{\pgfqpoint{3.528206in}{1.709010in}}%
\pgfpathlineto{\pgfqpoint{3.530713in}{1.711430in}}%
\pgfpathlineto{\pgfqpoint{3.533219in}{1.716229in}}%
\pgfpathlineto{\pgfqpoint{3.538233in}{1.719155in}}%
\pgfpathlineto{\pgfqpoint{3.540739in}{1.719236in}}%
\pgfpathlineto{\pgfqpoint{3.543246in}{1.722204in}}%
\pgfpathlineto{\pgfqpoint{3.545753in}{1.722704in}}%
\pgfpathlineto{\pgfqpoint{3.548259in}{1.727740in}}%
\pgfpathlineto{\pgfqpoint{3.550766in}{1.727748in}}%
\pgfpathlineto{\pgfqpoint{3.553273in}{1.731307in}}%
\pgfpathlineto{\pgfqpoint{3.558286in}{1.746675in}}%
\pgfpathlineto{\pgfqpoint{3.560793in}{1.746677in}}%
\pgfpathlineto{\pgfqpoint{3.565806in}{1.753190in}}%
\pgfpathlineto{\pgfqpoint{3.570819in}{1.760686in}}%
\pgfpathlineto{\pgfqpoint{3.573326in}{1.760762in}}%
\pgfpathlineto{\pgfqpoint{3.575832in}{1.764333in}}%
\pgfpathlineto{\pgfqpoint{3.578339in}{1.779408in}}%
\pgfpathlineto{\pgfqpoint{3.580846in}{1.787402in}}%
\pgfpathlineto{\pgfqpoint{3.583352in}{1.803297in}}%
\pgfpathlineto{\pgfqpoint{3.590872in}{1.806421in}}%
\pgfpathlineto{\pgfqpoint{3.593379in}{1.810900in}}%
\pgfpathlineto{\pgfqpoint{3.598392in}{1.822076in}}%
\pgfpathlineto{\pgfqpoint{3.600899in}{1.825823in}}%
\pgfpathlineto{\pgfqpoint{3.605912in}{1.836748in}}%
\pgfpathlineto{\pgfqpoint{3.608419in}{1.838636in}}%
\pgfpathlineto{\pgfqpoint{3.610926in}{1.838959in}}%
\pgfpathlineto{\pgfqpoint{3.613432in}{1.840662in}}%
\pgfpathlineto{\pgfqpoint{3.615939in}{1.854455in}}%
\pgfpathlineto{\pgfqpoint{3.618446in}{1.858870in}}%
\pgfpathlineto{\pgfqpoint{3.623459in}{1.876097in}}%
\pgfpathlineto{\pgfqpoint{3.625966in}{1.876227in}}%
\pgfpathlineto{\pgfqpoint{3.630979in}{1.881326in}}%
\pgfpathlineto{\pgfqpoint{3.633486in}{1.893734in}}%
\pgfpathlineto{\pgfqpoint{3.635992in}{1.894724in}}%
\pgfpathlineto{\pgfqpoint{3.643512in}{1.906979in}}%
\pgfpathlineto{\pgfqpoint{3.651032in}{1.908257in}}%
\pgfpathlineto{\pgfqpoint{3.653539in}{1.910910in}}%
\pgfpathlineto{\pgfqpoint{3.658552in}{1.920457in}}%
\pgfpathlineto{\pgfqpoint{3.661059in}{1.921341in}}%
\pgfpathlineto{\pgfqpoint{3.663565in}{1.928372in}}%
\pgfpathlineto{\pgfqpoint{3.666072in}{1.931399in}}%
\pgfpathlineto{\pgfqpoint{3.668579in}{1.939935in}}%
\pgfpathlineto{\pgfqpoint{3.671085in}{1.944103in}}%
\pgfpathlineto{\pgfqpoint{3.673592in}{1.945438in}}%
\pgfpathlineto{\pgfqpoint{3.676099in}{1.956075in}}%
\pgfpathlineto{\pgfqpoint{3.678605in}{1.957143in}}%
\pgfpathlineto{\pgfqpoint{3.681112in}{1.960113in}}%
\pgfpathlineto{\pgfqpoint{3.683619in}{1.970448in}}%
\pgfpathlineto{\pgfqpoint{3.688632in}{1.975091in}}%
\pgfpathlineto{\pgfqpoint{3.696152in}{1.978396in}}%
\pgfpathlineto{\pgfqpoint{3.698659in}{1.986544in}}%
\pgfpathlineto{\pgfqpoint{3.706178in}{1.992480in}}%
\pgfpathlineto{\pgfqpoint{3.708685in}{1.996282in}}%
\pgfpathlineto{\pgfqpoint{3.711192in}{1.997521in}}%
\pgfpathlineto{\pgfqpoint{3.713698in}{2.001655in}}%
\pgfpathlineto{\pgfqpoint{3.716205in}{2.009988in}}%
\pgfpathlineto{\pgfqpoint{3.721218in}{2.013981in}}%
\pgfpathlineto{\pgfqpoint{3.726232in}{2.016427in}}%
\pgfpathlineto{\pgfqpoint{3.731245in}{2.023844in}}%
\pgfpathlineto{\pgfqpoint{3.733752in}{2.024042in}}%
\pgfpathlineto{\pgfqpoint{3.736258in}{2.026604in}}%
\pgfpathlineto{\pgfqpoint{3.738765in}{2.033485in}}%
\pgfpathlineto{\pgfqpoint{3.741272in}{2.036068in}}%
\pgfpathlineto{\pgfqpoint{3.748792in}{2.048744in}}%
\pgfpathlineto{\pgfqpoint{3.751298in}{2.060429in}}%
\pgfpathlineto{\pgfqpoint{3.753805in}{2.061275in}}%
\pgfpathlineto{\pgfqpoint{3.756312in}{2.084424in}}%
\pgfpathlineto{\pgfqpoint{3.758818in}{2.085845in}}%
\pgfpathlineto{\pgfqpoint{3.763832in}{2.091566in}}%
\pgfpathlineto{\pgfqpoint{3.766338in}{2.109799in}}%
\pgfpathlineto{\pgfqpoint{3.768845in}{2.114413in}}%
\pgfpathlineto{\pgfqpoint{3.773858in}{2.116543in}}%
\pgfpathlineto{\pgfqpoint{3.776365in}{2.118780in}}%
\pgfpathlineto{\pgfqpoint{3.781378in}{2.119993in}}%
\pgfpathlineto{\pgfqpoint{3.783885in}{2.123038in}}%
\pgfpathlineto{\pgfqpoint{3.788898in}{2.124013in}}%
\pgfpathlineto{\pgfqpoint{3.793911in}{2.144520in}}%
\pgfpathlineto{\pgfqpoint{3.796418in}{2.149072in}}%
\pgfpathlineto{\pgfqpoint{3.798925in}{2.149821in}}%
\pgfpathlineto{\pgfqpoint{3.803938in}{2.162247in}}%
\pgfpathlineto{\pgfqpoint{3.808951in}{2.170255in}}%
\pgfpathlineto{\pgfqpoint{3.813965in}{2.171420in}}%
\pgfpathlineto{\pgfqpoint{3.816471in}{2.178739in}}%
\pgfpathlineto{\pgfqpoint{3.821485in}{2.180807in}}%
\pgfpathlineto{\pgfqpoint{3.823991in}{2.184106in}}%
\pgfpathlineto{\pgfqpoint{3.829005in}{2.197677in}}%
\pgfpathlineto{\pgfqpoint{3.834018in}{2.203297in}}%
\pgfpathlineto{\pgfqpoint{3.836525in}{2.203409in}}%
\pgfpathlineto{\pgfqpoint{3.839031in}{2.209046in}}%
\pgfpathlineto{\pgfqpoint{3.841538in}{2.210423in}}%
\pgfpathlineto{\pgfqpoint{3.844044in}{2.223173in}}%
\pgfpathlineto{\pgfqpoint{3.849058in}{2.225826in}}%
\pgfpathlineto{\pgfqpoint{3.851564in}{2.228564in}}%
\pgfpathlineto{\pgfqpoint{3.854071in}{2.229226in}}%
\pgfpathlineto{\pgfqpoint{3.859084in}{2.239751in}}%
\pgfpathlineto{\pgfqpoint{3.861591in}{2.241258in}}%
\pgfpathlineto{\pgfqpoint{3.864098in}{2.246472in}}%
\pgfpathlineto{\pgfqpoint{3.866604in}{2.256492in}}%
\pgfpathlineto{\pgfqpoint{3.869111in}{2.258735in}}%
\pgfpathlineto{\pgfqpoint{3.871618in}{2.258951in}}%
\pgfpathlineto{\pgfqpoint{3.874124in}{2.261109in}}%
\pgfpathlineto{\pgfqpoint{3.876631in}{2.266981in}}%
\pgfpathlineto{\pgfqpoint{3.879138in}{2.267193in}}%
\pgfpathlineto{\pgfqpoint{3.881644in}{2.272121in}}%
\pgfpathlineto{\pgfqpoint{3.884151in}{2.273422in}}%
\pgfpathlineto{\pgfqpoint{3.886658in}{2.284287in}}%
\pgfpathlineto{\pgfqpoint{3.889164in}{2.285026in}}%
\pgfpathlineto{\pgfqpoint{3.891671in}{2.290691in}}%
\pgfpathlineto{\pgfqpoint{3.894178in}{2.292378in}}%
\pgfpathlineto{\pgfqpoint{3.896684in}{2.295688in}}%
\pgfpathlineto{\pgfqpoint{3.899191in}{2.305275in}}%
\pgfpathlineto{\pgfqpoint{5.658862in}{2.305275in}}%
\pgfpathlineto{\pgfqpoint{5.658862in}{2.305275in}}%
\pgfusepath{stroke}%
\end{pgfscope}%
\begin{pgfscope}%
\pgfpathrectangle{\pgfqpoint{0.708220in}{0.535823in}}{\pgfqpoint{5.013309in}{1.769453in}}%
\pgfusepath{clip}%
\pgfsetbuttcap%
\pgfsetroundjoin%
\pgfsetlinewidth{1.003750pt}%
\definecolor{currentstroke}{rgb}{0.000000,0.000000,0.000000}%
\pgfsetstrokecolor{currentstroke}%
\pgfsetdash{{3.700000pt}{1.600000pt}}{0.000000pt}%
\pgfpathmoveto{\pgfqpoint{0.708220in}{0.606902in}}%
\pgfpathlineto{\pgfqpoint{0.710727in}{0.606909in}}%
\pgfpathlineto{\pgfqpoint{0.713233in}{0.623389in}}%
\pgfpathlineto{\pgfqpoint{0.718246in}{0.625894in}}%
\pgfpathlineto{\pgfqpoint{0.720753in}{0.628848in}}%
\pgfpathlineto{\pgfqpoint{0.723260in}{0.629962in}}%
\pgfpathlineto{\pgfqpoint{0.725766in}{0.634766in}}%
\pgfpathlineto{\pgfqpoint{0.728273in}{0.635676in}}%
\pgfpathlineto{\pgfqpoint{0.730780in}{0.649484in}}%
\pgfpathlineto{\pgfqpoint{0.735793in}{0.651568in}}%
\pgfpathlineto{\pgfqpoint{0.740806in}{0.666362in}}%
\pgfpathlineto{\pgfqpoint{0.743313in}{0.667704in}}%
\pgfpathlineto{\pgfqpoint{0.745820in}{0.672109in}}%
\pgfpathlineto{\pgfqpoint{0.750833in}{0.673462in}}%
\pgfpathlineto{\pgfqpoint{0.753340in}{0.684636in}}%
\pgfpathlineto{\pgfqpoint{0.755846in}{0.684678in}}%
\pgfpathlineto{\pgfqpoint{0.758353in}{0.689845in}}%
\pgfpathlineto{\pgfqpoint{0.763366in}{0.695312in}}%
\pgfpathlineto{\pgfqpoint{0.765873in}{0.700207in}}%
\pgfpathlineto{\pgfqpoint{0.768380in}{0.700521in}}%
\pgfpathlineto{\pgfqpoint{0.770886in}{0.705081in}}%
\pgfpathlineto{\pgfqpoint{0.780913in}{0.708075in}}%
\pgfpathlineto{\pgfqpoint{0.783419in}{0.708106in}}%
\pgfpathlineto{\pgfqpoint{0.788433in}{0.729832in}}%
\pgfpathlineto{\pgfqpoint{0.790939in}{0.732282in}}%
\pgfpathlineto{\pgfqpoint{0.793446in}{0.732707in}}%
\pgfpathlineto{\pgfqpoint{0.795953in}{0.735926in}}%
\pgfpathlineto{\pgfqpoint{0.798459in}{0.736115in}}%
\pgfpathlineto{\pgfqpoint{0.800966in}{0.738305in}}%
\pgfpathlineto{\pgfqpoint{0.803473in}{0.738459in}}%
\pgfpathlineto{\pgfqpoint{0.810993in}{0.747331in}}%
\pgfpathlineto{\pgfqpoint{0.816006in}{0.748383in}}%
\pgfpathlineto{\pgfqpoint{0.818513in}{0.755766in}}%
\pgfpathlineto{\pgfqpoint{0.821019in}{0.758654in}}%
\pgfpathlineto{\pgfqpoint{0.831046in}{0.760097in}}%
\pgfpathlineto{\pgfqpoint{0.838566in}{0.761299in}}%
\pgfpathlineto{\pgfqpoint{0.863632in}{0.762483in}}%
\pgfpathlineto{\pgfqpoint{0.871152in}{0.765658in}}%
\pgfpathlineto{\pgfqpoint{0.873659in}{0.767678in}}%
\pgfpathlineto{\pgfqpoint{0.938832in}{0.772826in}}%
\pgfpathlineto{\pgfqpoint{0.948859in}{0.776772in}}%
\pgfpathlineto{\pgfqpoint{1.076698in}{0.786913in}}%
\pgfpathlineto{\pgfqpoint{1.109285in}{0.788155in}}%
\pgfpathlineto{\pgfqpoint{1.166938in}{0.790977in}}%
\pgfpathlineto{\pgfqpoint{1.174458in}{0.793695in}}%
\pgfpathlineto{\pgfqpoint{1.217071in}{0.795350in}}%
\pgfpathlineto{\pgfqpoint{1.239631in}{0.796477in}}%
\pgfpathlineto{\pgfqpoint{1.272217in}{0.797400in}}%
\pgfpathlineto{\pgfqpoint{1.284750in}{0.798785in}}%
\pgfpathlineto{\pgfqpoint{1.289764in}{0.798855in}}%
\pgfpathlineto{\pgfqpoint{1.292270in}{0.800419in}}%
\pgfpathlineto{\pgfqpoint{1.297284in}{0.801123in}}%
\pgfpathlineto{\pgfqpoint{1.302297in}{0.802657in}}%
\pgfpathlineto{\pgfqpoint{1.314830in}{0.803438in}}%
\pgfpathlineto{\pgfqpoint{1.377497in}{0.807104in}}%
\pgfpathlineto{\pgfqpoint{1.392537in}{0.808679in}}%
\pgfpathlineto{\pgfqpoint{1.415096in}{0.809712in}}%
\pgfpathlineto{\pgfqpoint{1.482776in}{0.811762in}}%
\pgfpathlineto{\pgfqpoint{1.497816in}{0.813161in}}%
\pgfpathlineto{\pgfqpoint{1.505336in}{0.814422in}}%
\pgfpathlineto{\pgfqpoint{1.573016in}{0.817565in}}%
\pgfpathlineto{\pgfqpoint{1.618135in}{0.818463in}}%
\pgfpathlineto{\pgfqpoint{1.628162in}{0.819192in}}%
\pgfpathlineto{\pgfqpoint{1.650722in}{0.820452in}}%
\pgfpathlineto{\pgfqpoint{1.665762in}{0.822106in}}%
\pgfpathlineto{\pgfqpoint{1.720908in}{0.827303in}}%
\pgfpathlineto{\pgfqpoint{1.723415in}{0.830188in}}%
\pgfpathlineto{\pgfqpoint{1.733442in}{0.831369in}}%
\pgfpathlineto{\pgfqpoint{1.735948in}{0.831758in}}%
\pgfpathlineto{\pgfqpoint{1.738455in}{0.834437in}}%
\pgfpathlineto{\pgfqpoint{1.748481in}{0.836267in}}%
\pgfpathlineto{\pgfqpoint{1.771041in}{0.837878in}}%
\pgfpathlineto{\pgfqpoint{1.788588in}{0.840893in}}%
\pgfpathlineto{\pgfqpoint{1.898881in}{0.844206in}}%
\pgfpathlineto{\pgfqpoint{1.906401in}{0.845030in}}%
\pgfpathlineto{\pgfqpoint{1.916427in}{0.846116in}}%
\pgfpathlineto{\pgfqpoint{1.921441in}{0.847616in}}%
\pgfpathlineto{\pgfqpoint{1.938987in}{0.848968in}}%
\pgfpathlineto{\pgfqpoint{1.966560in}{0.851392in}}%
\pgfpathlineto{\pgfqpoint{1.979094in}{0.854872in}}%
\pgfpathlineto{\pgfqpoint{1.994134in}{0.855780in}}%
\pgfpathlineto{\pgfqpoint{2.011680in}{0.861813in}}%
\pgfpathlineto{\pgfqpoint{2.024213in}{0.871676in}}%
\pgfpathlineto{\pgfqpoint{2.026720in}{0.872123in}}%
\pgfpathlineto{\pgfqpoint{2.031733in}{0.875762in}}%
\pgfpathlineto{\pgfqpoint{2.036747in}{0.878426in}}%
\pgfpathlineto{\pgfqpoint{2.041760in}{0.878652in}}%
\pgfpathlineto{\pgfqpoint{2.044267in}{0.884163in}}%
\pgfpathlineto{\pgfqpoint{2.046773in}{0.884449in}}%
\pgfpathlineto{\pgfqpoint{2.056800in}{0.889875in}}%
\pgfpathlineto{\pgfqpoint{2.079360in}{0.891575in}}%
\pgfpathlineto{\pgfqpoint{2.101920in}{0.896842in}}%
\pgfpathlineto{\pgfqpoint{2.104426in}{0.898414in}}%
\pgfpathlineto{\pgfqpoint{2.109440in}{0.898667in}}%
\pgfpathlineto{\pgfqpoint{2.111946in}{0.900634in}}%
\pgfpathlineto{\pgfqpoint{2.132000in}{0.904179in}}%
\pgfpathlineto{\pgfqpoint{2.142026in}{0.907678in}}%
\pgfpathlineto{\pgfqpoint{2.147039in}{0.907678in}}%
\pgfpathlineto{\pgfqpoint{2.152053in}{0.912064in}}%
\pgfpathlineto{\pgfqpoint{2.154559in}{0.912082in}}%
\pgfpathlineto{\pgfqpoint{2.159573in}{0.915705in}}%
\pgfpathlineto{\pgfqpoint{2.162079in}{0.917996in}}%
\pgfpathlineto{\pgfqpoint{2.172106in}{0.918832in}}%
\pgfpathlineto{\pgfqpoint{2.202186in}{0.924385in}}%
\pgfpathlineto{\pgfqpoint{2.212212in}{0.926340in}}%
\pgfpathlineto{\pgfqpoint{2.214719in}{0.928400in}}%
\pgfpathlineto{\pgfqpoint{2.242292in}{0.929334in}}%
\pgfpathlineto{\pgfqpoint{2.247306in}{0.930631in}}%
\pgfpathlineto{\pgfqpoint{2.259839in}{0.932327in}}%
\pgfpathlineto{\pgfqpoint{2.274879in}{0.933156in}}%
\pgfpathlineto{\pgfqpoint{2.287412in}{0.936169in}}%
\pgfpathlineto{\pgfqpoint{2.304959in}{0.936963in}}%
\pgfpathlineto{\pgfqpoint{2.307465in}{0.939931in}}%
\pgfpathlineto{\pgfqpoint{2.319999in}{0.940691in}}%
\pgfpathlineto{\pgfqpoint{2.325012in}{0.943615in}}%
\pgfpathlineto{\pgfqpoint{2.345065in}{0.944141in}}%
\pgfpathlineto{\pgfqpoint{2.352585in}{0.947225in}}%
\pgfpathlineto{\pgfqpoint{2.382665in}{0.947740in}}%
\pgfpathlineto{\pgfqpoint{2.387678in}{0.950082in}}%
\pgfpathlineto{\pgfqpoint{2.405225in}{0.950763in}}%
\pgfpathlineto{\pgfqpoint{2.422771in}{0.950763in}}%
\pgfpathlineto{\pgfqpoint{2.425278in}{0.954232in}}%
\pgfpathlineto{\pgfqpoint{2.442825in}{0.954728in}}%
\pgfpathlineto{\pgfqpoint{2.450345in}{0.957636in}}%
\pgfpathlineto{\pgfqpoint{2.462878in}{0.958122in}}%
\pgfpathlineto{\pgfqpoint{2.465385in}{0.960975in}}%
\pgfpathlineto{\pgfqpoint{2.482931in}{0.961452in}}%
\pgfpathlineto{\pgfqpoint{2.485438in}{0.963854in}}%
\pgfpathlineto{\pgfqpoint{2.495464in}{0.964722in}}%
\pgfpathlineto{\pgfqpoint{2.500478in}{0.964722in}}%
\pgfpathlineto{\pgfqpoint{2.507998in}{0.967933in}}%
\pgfpathlineto{\pgfqpoint{2.510504in}{0.968106in}}%
\pgfpathlineto{\pgfqpoint{2.513011in}{0.970635in}}%
\pgfpathlineto{\pgfqpoint{2.523038in}{0.970862in}}%
\pgfpathlineto{\pgfqpoint{2.525544in}{0.973743in}}%
\pgfpathlineto{\pgfqpoint{2.538078in}{0.974187in}}%
\pgfpathlineto{\pgfqpoint{2.543091in}{0.976648in}}%
\pgfpathlineto{\pgfqpoint{2.550611in}{0.977702in}}%
\pgfpathlineto{\pgfqpoint{2.555624in}{0.981972in}}%
\pgfpathlineto{\pgfqpoint{2.560637in}{0.983176in}}%
\pgfpathlineto{\pgfqpoint{2.565651in}{0.988517in}}%
\pgfpathlineto{\pgfqpoint{2.568157in}{0.988517in}}%
\pgfpathlineto{\pgfqpoint{2.575677in}{0.991331in}}%
\pgfpathlineto{\pgfqpoint{2.578184in}{0.991331in}}%
\pgfpathlineto{\pgfqpoint{2.580691in}{0.994101in}}%
\pgfpathlineto{\pgfqpoint{2.583197in}{0.994101in}}%
\pgfpathlineto{\pgfqpoint{2.585704in}{0.996160in}}%
\pgfpathlineto{\pgfqpoint{2.590717in}{1.001648in}}%
\pgfpathlineto{\pgfqpoint{2.628317in}{1.009876in}}%
\pgfpathlineto{\pgfqpoint{2.630824in}{1.013067in}}%
\pgfpathlineto{\pgfqpoint{2.633330in}{1.013307in}}%
\pgfpathlineto{\pgfqpoint{2.638344in}{1.016406in}}%
\pgfpathlineto{\pgfqpoint{2.640850in}{1.022375in}}%
\pgfpathlineto{\pgfqpoint{2.643357in}{1.022626in}}%
\pgfpathlineto{\pgfqpoint{2.650877in}{1.027007in}}%
\pgfpathlineto{\pgfqpoint{2.658397in}{1.029560in}}%
\pgfpathlineto{\pgfqpoint{2.660904in}{1.033079in}}%
\pgfpathlineto{\pgfqpoint{2.665917in}{1.035612in}}%
\pgfpathlineto{\pgfqpoint{2.670930in}{1.036603in}}%
\pgfpathlineto{\pgfqpoint{2.675944in}{1.039917in}}%
\pgfpathlineto{\pgfqpoint{2.678450in}{1.041871in}}%
\pgfpathlineto{\pgfqpoint{2.683464in}{1.042032in}}%
\pgfpathlineto{\pgfqpoint{2.685970in}{1.045563in}}%
\pgfpathlineto{\pgfqpoint{2.688477in}{1.046600in}}%
\pgfpathlineto{\pgfqpoint{2.690983in}{1.049706in}}%
\pgfpathlineto{\pgfqpoint{2.695997in}{1.060014in}}%
\pgfpathlineto{\pgfqpoint{2.698503in}{1.060014in}}%
\pgfpathlineto{\pgfqpoint{2.703517in}{1.067308in}}%
\pgfpathlineto{\pgfqpoint{2.706023in}{1.074221in}}%
\pgfpathlineto{\pgfqpoint{2.711037in}{1.076960in}}%
\pgfpathlineto{\pgfqpoint{2.728583in}{1.083439in}}%
\pgfpathlineto{\pgfqpoint{2.733597in}{1.088096in}}%
\pgfpathlineto{\pgfqpoint{2.746130in}{1.091317in}}%
\pgfpathlineto{\pgfqpoint{2.748637in}{1.092906in}}%
\pgfpathlineto{\pgfqpoint{2.753650in}{1.100333in}}%
\pgfpathlineto{\pgfqpoint{2.756156in}{1.105850in}}%
\pgfpathlineto{\pgfqpoint{2.758663in}{1.107111in}}%
\pgfpathlineto{\pgfqpoint{2.761170in}{1.111284in}}%
\pgfpathlineto{\pgfqpoint{2.763676in}{1.112633in}}%
\pgfpathlineto{\pgfqpoint{2.766183in}{1.116855in}}%
\pgfpathlineto{\pgfqpoint{2.776210in}{1.120673in}}%
\pgfpathlineto{\pgfqpoint{2.778716in}{1.123483in}}%
\pgfpathlineto{\pgfqpoint{2.783730in}{1.124121in}}%
\pgfpathlineto{\pgfqpoint{2.786236in}{1.127316in}}%
\pgfpathlineto{\pgfqpoint{2.788743in}{1.127360in}}%
\pgfpathlineto{\pgfqpoint{2.793756in}{1.130248in}}%
\pgfpathlineto{\pgfqpoint{2.796263in}{1.130496in}}%
\pgfpathlineto{\pgfqpoint{2.803783in}{1.140941in}}%
\pgfpathlineto{\pgfqpoint{2.811303in}{1.145880in}}%
\pgfpathlineto{\pgfqpoint{2.821329in}{1.153777in}}%
\pgfpathlineto{\pgfqpoint{2.823836in}{1.154689in}}%
\pgfpathlineto{\pgfqpoint{2.826343in}{1.157212in}}%
\pgfpathlineto{\pgfqpoint{2.831356in}{1.158268in}}%
\pgfpathlineto{\pgfqpoint{2.846396in}{1.165359in}}%
\pgfpathlineto{\pgfqpoint{2.851409in}{1.170727in}}%
\pgfpathlineto{\pgfqpoint{2.853916in}{1.171026in}}%
\pgfpathlineto{\pgfqpoint{2.863943in}{1.181276in}}%
\pgfpathlineto{\pgfqpoint{2.866449in}{1.186028in}}%
\pgfpathlineto{\pgfqpoint{2.871463in}{1.187606in}}%
\pgfpathlineto{\pgfqpoint{2.873969in}{1.190390in}}%
\pgfpathlineto{\pgfqpoint{2.878983in}{1.191436in}}%
\pgfpathlineto{\pgfqpoint{2.889009in}{1.201786in}}%
\pgfpathlineto{\pgfqpoint{2.891516in}{1.201923in}}%
\pgfpathlineto{\pgfqpoint{2.894022in}{1.204028in}}%
\pgfpathlineto{\pgfqpoint{2.899036in}{1.205730in}}%
\pgfpathlineto{\pgfqpoint{2.904049in}{1.211128in}}%
\pgfpathlineto{\pgfqpoint{2.916582in}{1.220204in}}%
\pgfpathlineto{\pgfqpoint{2.919089in}{1.222965in}}%
\pgfpathlineto{\pgfqpoint{2.929116in}{1.225994in}}%
\pgfpathlineto{\pgfqpoint{2.934129in}{1.239231in}}%
\pgfpathlineto{\pgfqpoint{2.939142in}{1.240781in}}%
\pgfpathlineto{\pgfqpoint{2.941649in}{1.248533in}}%
\pgfpathlineto{\pgfqpoint{2.946662in}{1.249039in}}%
\pgfpathlineto{\pgfqpoint{2.949169in}{1.251456in}}%
\pgfpathlineto{\pgfqpoint{2.951676in}{1.256032in}}%
\pgfpathlineto{\pgfqpoint{2.956689in}{1.257565in}}%
\pgfpathlineto{\pgfqpoint{2.961702in}{1.257748in}}%
\pgfpathlineto{\pgfqpoint{2.966715in}{1.266031in}}%
\pgfpathlineto{\pgfqpoint{2.971729in}{1.268757in}}%
\pgfpathlineto{\pgfqpoint{2.976742in}{1.278450in}}%
\pgfpathlineto{\pgfqpoint{2.979249in}{1.279516in}}%
\pgfpathlineto{\pgfqpoint{2.986769in}{1.286276in}}%
\pgfpathlineto{\pgfqpoint{2.991782in}{1.288910in}}%
\pgfpathlineto{\pgfqpoint{2.996795in}{1.292243in}}%
\pgfpathlineto{\pgfqpoint{2.999302in}{1.294391in}}%
\pgfpathlineto{\pgfqpoint{3.001809in}{1.294656in}}%
\pgfpathlineto{\pgfqpoint{3.011835in}{1.299659in}}%
\pgfpathlineto{\pgfqpoint{3.019355in}{1.302285in}}%
\pgfpathlineto{\pgfqpoint{3.021862in}{1.305040in}}%
\pgfpathlineto{\pgfqpoint{3.024369in}{1.305673in}}%
\pgfpathlineto{\pgfqpoint{3.029382in}{1.310311in}}%
\pgfpathlineto{\pgfqpoint{3.034395in}{1.312652in}}%
\pgfpathlineto{\pgfqpoint{3.054448in}{1.321418in}}%
\pgfpathlineto{\pgfqpoint{3.056955in}{1.324039in}}%
\pgfpathlineto{\pgfqpoint{3.059462in}{1.324504in}}%
\pgfpathlineto{\pgfqpoint{3.061968in}{1.326605in}}%
\pgfpathlineto{\pgfqpoint{3.077008in}{1.330195in}}%
\pgfpathlineto{\pgfqpoint{3.079515in}{1.337404in}}%
\pgfpathlineto{\pgfqpoint{3.082022in}{1.337842in}}%
\pgfpathlineto{\pgfqpoint{3.084528in}{1.344687in}}%
\pgfpathlineto{\pgfqpoint{3.092048in}{1.346832in}}%
\pgfpathlineto{\pgfqpoint{3.094555in}{1.347940in}}%
\pgfpathlineto{\pgfqpoint{3.097061in}{1.355531in}}%
\pgfpathlineto{\pgfqpoint{3.099568in}{1.356831in}}%
\pgfpathlineto{\pgfqpoint{3.102075in}{1.362227in}}%
\pgfpathlineto{\pgfqpoint{3.107088in}{1.365634in}}%
\pgfpathlineto{\pgfqpoint{3.112101in}{1.368083in}}%
\pgfpathlineto{\pgfqpoint{3.114608in}{1.377594in}}%
\pgfpathlineto{\pgfqpoint{3.117115in}{1.380463in}}%
\pgfpathlineto{\pgfqpoint{3.137168in}{1.388199in}}%
\pgfpathlineto{\pgfqpoint{3.139675in}{1.392034in}}%
\pgfpathlineto{\pgfqpoint{3.154715in}{1.397923in}}%
\pgfpathlineto{\pgfqpoint{3.157221in}{1.403113in}}%
\pgfpathlineto{\pgfqpoint{3.162234in}{1.403528in}}%
\pgfpathlineto{\pgfqpoint{3.164741in}{1.404429in}}%
\pgfpathlineto{\pgfqpoint{3.167248in}{1.406804in}}%
\pgfpathlineto{\pgfqpoint{3.174768in}{1.417952in}}%
\pgfpathlineto{\pgfqpoint{3.177274in}{1.418165in}}%
\pgfpathlineto{\pgfqpoint{3.182288in}{1.420347in}}%
\pgfpathlineto{\pgfqpoint{3.184794in}{1.425997in}}%
\pgfpathlineto{\pgfqpoint{3.187301in}{1.427682in}}%
\pgfpathlineto{\pgfqpoint{3.192314in}{1.427951in}}%
\pgfpathlineto{\pgfqpoint{3.199834in}{1.436468in}}%
\pgfpathlineto{\pgfqpoint{3.202341in}{1.436930in}}%
\pgfpathlineto{\pgfqpoint{3.204848in}{1.439267in}}%
\pgfpathlineto{\pgfqpoint{3.212368in}{1.442602in}}%
\pgfpathlineto{\pgfqpoint{3.214874in}{1.445789in}}%
\pgfpathlineto{\pgfqpoint{3.222394in}{1.447890in}}%
\pgfpathlineto{\pgfqpoint{3.224901in}{1.456473in}}%
\pgfpathlineto{\pgfqpoint{3.227408in}{1.460044in}}%
\pgfpathlineto{\pgfqpoint{3.229914in}{1.467379in}}%
\pgfpathlineto{\pgfqpoint{3.242447in}{1.475360in}}%
\pgfpathlineto{\pgfqpoint{3.244954in}{1.491528in}}%
\pgfpathlineto{\pgfqpoint{3.247461in}{1.495559in}}%
\pgfpathlineto{\pgfqpoint{3.252474in}{1.498686in}}%
\pgfpathlineto{\pgfqpoint{3.259994in}{1.507384in}}%
\pgfpathlineto{\pgfqpoint{3.262501in}{1.507908in}}%
\pgfpathlineto{\pgfqpoint{3.265007in}{1.516441in}}%
\pgfpathlineto{\pgfqpoint{3.270021in}{1.520038in}}%
\pgfpathlineto{\pgfqpoint{3.272527in}{1.523608in}}%
\pgfpathlineto{\pgfqpoint{3.280047in}{1.527274in}}%
\pgfpathlineto{\pgfqpoint{3.282554in}{1.527552in}}%
\pgfpathlineto{\pgfqpoint{3.285061in}{1.530034in}}%
\pgfpathlineto{\pgfqpoint{3.287567in}{1.535342in}}%
\pgfpathlineto{\pgfqpoint{3.292581in}{1.535812in}}%
\pgfpathlineto{\pgfqpoint{3.295087in}{1.539156in}}%
\pgfpathlineto{\pgfqpoint{3.302607in}{1.541527in}}%
\pgfpathlineto{\pgfqpoint{3.305114in}{1.546267in}}%
\pgfpathlineto{\pgfqpoint{3.307620in}{1.546876in}}%
\pgfpathlineto{\pgfqpoint{3.315140in}{1.551427in}}%
\pgfpathlineto{\pgfqpoint{3.320154in}{1.551892in}}%
\pgfpathlineto{\pgfqpoint{3.325167in}{1.557399in}}%
\pgfpathlineto{\pgfqpoint{3.332687in}{1.559036in}}%
\pgfpathlineto{\pgfqpoint{3.337700in}{1.561923in}}%
\pgfpathlineto{\pgfqpoint{3.342714in}{1.565489in}}%
\pgfpathlineto{\pgfqpoint{3.347727in}{1.565775in}}%
\pgfpathlineto{\pgfqpoint{3.362767in}{1.572161in}}%
\pgfpathlineto{\pgfqpoint{3.365273in}{1.576939in}}%
\pgfpathlineto{\pgfqpoint{3.367780in}{1.578326in}}%
\pgfpathlineto{\pgfqpoint{3.377807in}{1.590414in}}%
\pgfpathlineto{\pgfqpoint{3.382820in}{1.599301in}}%
\pgfpathlineto{\pgfqpoint{3.385327in}{1.599672in}}%
\pgfpathlineto{\pgfqpoint{3.387833in}{1.604339in}}%
\pgfpathlineto{\pgfqpoint{3.390340in}{1.611414in}}%
\pgfpathlineto{\pgfqpoint{3.400367in}{1.615207in}}%
\pgfpathlineto{\pgfqpoint{3.402873in}{1.623019in}}%
\pgfpathlineto{\pgfqpoint{3.405380in}{1.623116in}}%
\pgfpathlineto{\pgfqpoint{3.407887in}{1.635283in}}%
\pgfpathlineto{\pgfqpoint{3.410393in}{1.637206in}}%
\pgfpathlineto{\pgfqpoint{3.412900in}{1.646379in}}%
\pgfpathlineto{\pgfqpoint{3.422927in}{1.664933in}}%
\pgfpathlineto{\pgfqpoint{3.425433in}{1.671129in}}%
\pgfpathlineto{\pgfqpoint{3.427940in}{1.671156in}}%
\pgfpathlineto{\pgfqpoint{3.432953in}{1.675367in}}%
\pgfpathlineto{\pgfqpoint{3.442980in}{1.682006in}}%
\pgfpathlineto{\pgfqpoint{3.445486in}{1.685484in}}%
\pgfpathlineto{\pgfqpoint{3.458020in}{1.690671in}}%
\pgfpathlineto{\pgfqpoint{3.463033in}{1.704722in}}%
\pgfpathlineto{\pgfqpoint{3.465540in}{1.705175in}}%
\pgfpathlineto{\pgfqpoint{3.468046in}{1.709010in}}%
\pgfpathlineto{\pgfqpoint{3.473060in}{1.711430in}}%
\pgfpathlineto{\pgfqpoint{3.475566in}{1.716229in}}%
\pgfpathlineto{\pgfqpoint{3.478073in}{1.716424in}}%
\pgfpathlineto{\pgfqpoint{3.480580in}{1.718658in}}%
\pgfpathlineto{\pgfqpoint{3.483086in}{1.719155in}}%
\pgfpathlineto{\pgfqpoint{3.488100in}{1.727819in}}%
\pgfpathlineto{\pgfqpoint{3.498126in}{1.746879in}}%
\pgfpathlineto{\pgfqpoint{3.505646in}{1.751907in}}%
\pgfpathlineto{\pgfqpoint{3.513166in}{1.760762in}}%
\pgfpathlineto{\pgfqpoint{3.518179in}{1.770908in}}%
\pgfpathlineto{\pgfqpoint{3.525699in}{1.774371in}}%
\pgfpathlineto{\pgfqpoint{3.528206in}{1.779408in}}%
\pgfpathlineto{\pgfqpoint{3.530713in}{1.781034in}}%
\pgfpathlineto{\pgfqpoint{3.535726in}{1.787402in}}%
\pgfpathlineto{\pgfqpoint{3.538233in}{1.787511in}}%
\pgfpathlineto{\pgfqpoint{3.540739in}{1.790437in}}%
\pgfpathlineto{\pgfqpoint{3.548259in}{1.803297in}}%
\pgfpathlineto{\pgfqpoint{3.550766in}{1.806421in}}%
\pgfpathlineto{\pgfqpoint{3.555779in}{1.814553in}}%
\pgfpathlineto{\pgfqpoint{3.560793in}{1.819570in}}%
\pgfpathlineto{\pgfqpoint{3.565806in}{1.824313in}}%
\pgfpathlineto{\pgfqpoint{3.568313in}{1.839623in}}%
\pgfpathlineto{\pgfqpoint{3.573326in}{1.840662in}}%
\pgfpathlineto{\pgfqpoint{3.575832in}{1.851621in}}%
\pgfpathlineto{\pgfqpoint{3.578339in}{1.851825in}}%
\pgfpathlineto{\pgfqpoint{3.580846in}{1.855073in}}%
\pgfpathlineto{\pgfqpoint{3.585859in}{1.876424in}}%
\pgfpathlineto{\pgfqpoint{3.590872in}{1.881326in}}%
\pgfpathlineto{\pgfqpoint{3.595886in}{1.885777in}}%
\pgfpathlineto{\pgfqpoint{3.598392in}{1.886673in}}%
\pgfpathlineto{\pgfqpoint{3.603406in}{1.893734in}}%
\pgfpathlineto{\pgfqpoint{3.608419in}{1.895137in}}%
\pgfpathlineto{\pgfqpoint{3.610926in}{1.897971in}}%
\pgfpathlineto{\pgfqpoint{3.613432in}{1.903507in}}%
\pgfpathlineto{\pgfqpoint{3.618446in}{1.906811in}}%
\pgfpathlineto{\pgfqpoint{3.620952in}{1.906979in}}%
\pgfpathlineto{\pgfqpoint{3.628472in}{1.913098in}}%
\pgfpathlineto{\pgfqpoint{3.633486in}{1.925702in}}%
\pgfpathlineto{\pgfqpoint{3.638499in}{1.929841in}}%
\pgfpathlineto{\pgfqpoint{3.641005in}{1.930030in}}%
\pgfpathlineto{\pgfqpoint{3.643512in}{1.940997in}}%
\pgfpathlineto{\pgfqpoint{3.646019in}{1.944999in}}%
\pgfpathlineto{\pgfqpoint{3.648525in}{1.945296in}}%
\pgfpathlineto{\pgfqpoint{3.651032in}{1.949872in}}%
\pgfpathlineto{\pgfqpoint{3.656045in}{1.950979in}}%
\pgfpathlineto{\pgfqpoint{3.658552in}{1.960113in}}%
\pgfpathlineto{\pgfqpoint{3.661059in}{1.963980in}}%
\pgfpathlineto{\pgfqpoint{3.666072in}{1.964112in}}%
\pgfpathlineto{\pgfqpoint{3.668579in}{1.965685in}}%
\pgfpathlineto{\pgfqpoint{3.671085in}{1.968936in}}%
\pgfpathlineto{\pgfqpoint{3.673592in}{1.975024in}}%
\pgfpathlineto{\pgfqpoint{3.678605in}{1.977782in}}%
\pgfpathlineto{\pgfqpoint{3.686125in}{1.984200in}}%
\pgfpathlineto{\pgfqpoint{3.691139in}{1.992480in}}%
\pgfpathlineto{\pgfqpoint{3.696152in}{1.997521in}}%
\pgfpathlineto{\pgfqpoint{3.698659in}{2.001655in}}%
\pgfpathlineto{\pgfqpoint{3.701165in}{2.003555in}}%
\pgfpathlineto{\pgfqpoint{3.703672in}{2.004047in}}%
\pgfpathlineto{\pgfqpoint{3.706178in}{2.017369in}}%
\pgfpathlineto{\pgfqpoint{3.711192in}{2.018596in}}%
\pgfpathlineto{\pgfqpoint{3.713698in}{2.022415in}}%
\pgfpathlineto{\pgfqpoint{3.716205in}{2.031710in}}%
\pgfpathlineto{\pgfqpoint{3.721218in}{2.033485in}}%
\pgfpathlineto{\pgfqpoint{3.726232in}{2.039522in}}%
\pgfpathlineto{\pgfqpoint{3.728738in}{2.040614in}}%
\pgfpathlineto{\pgfqpoint{3.731245in}{2.046036in}}%
\pgfpathlineto{\pgfqpoint{3.733752in}{2.048744in}}%
\pgfpathlineto{\pgfqpoint{3.738765in}{2.059088in}}%
\pgfpathlineto{\pgfqpoint{3.741272in}{2.068845in}}%
\pgfpathlineto{\pgfqpoint{3.743778in}{2.070009in}}%
\pgfpathlineto{\pgfqpoint{3.746285in}{2.075086in}}%
\pgfpathlineto{\pgfqpoint{3.751298in}{2.076634in}}%
\pgfpathlineto{\pgfqpoint{3.758818in}{2.085845in}}%
\pgfpathlineto{\pgfqpoint{3.761325in}{2.091566in}}%
\pgfpathlineto{\pgfqpoint{3.771352in}{2.094620in}}%
\pgfpathlineto{\pgfqpoint{3.773858in}{2.094736in}}%
\pgfpathlineto{\pgfqpoint{3.778871in}{2.106828in}}%
\pgfpathlineto{\pgfqpoint{3.781378in}{2.109395in}}%
\pgfpathlineto{\pgfqpoint{3.788898in}{2.111386in}}%
\pgfpathlineto{\pgfqpoint{3.798925in}{2.121624in}}%
\pgfpathlineto{\pgfqpoint{3.806445in}{2.128533in}}%
\pgfpathlineto{\pgfqpoint{3.808951in}{2.135601in}}%
\pgfpathlineto{\pgfqpoint{3.811458in}{2.135617in}}%
\pgfpathlineto{\pgfqpoint{3.813965in}{2.141591in}}%
\pgfpathlineto{\pgfqpoint{3.816471in}{2.143873in}}%
\pgfpathlineto{\pgfqpoint{3.821485in}{2.144908in}}%
\pgfpathlineto{\pgfqpoint{3.829005in}{2.157224in}}%
\pgfpathlineto{\pgfqpoint{3.831511in}{2.157521in}}%
\pgfpathlineto{\pgfqpoint{3.836525in}{2.160349in}}%
\pgfpathlineto{\pgfqpoint{3.841538in}{2.161519in}}%
\pgfpathlineto{\pgfqpoint{3.844044in}{2.165023in}}%
\pgfpathlineto{\pgfqpoint{3.851564in}{2.166070in}}%
\pgfpathlineto{\pgfqpoint{3.856578in}{2.170255in}}%
\pgfpathlineto{\pgfqpoint{3.861591in}{2.187498in}}%
\pgfpathlineto{\pgfqpoint{3.864098in}{2.188015in}}%
\pgfpathlineto{\pgfqpoint{3.871618in}{2.203309in}}%
\pgfpathlineto{\pgfqpoint{3.881644in}{2.210423in}}%
\pgfpathlineto{\pgfqpoint{3.884151in}{2.221988in}}%
\pgfpathlineto{\pgfqpoint{3.889164in}{2.229226in}}%
\pgfpathlineto{\pgfqpoint{3.891671in}{2.229682in}}%
\pgfpathlineto{\pgfqpoint{3.894178in}{2.235836in}}%
\pgfpathlineto{\pgfqpoint{3.896684in}{2.235892in}}%
\pgfpathlineto{\pgfqpoint{3.899191in}{2.239265in}}%
\pgfpathlineto{\pgfqpoint{3.901698in}{2.239751in}}%
\pgfpathlineto{\pgfqpoint{3.904204in}{2.242373in}}%
\pgfpathlineto{\pgfqpoint{3.906711in}{2.242924in}}%
\pgfpathlineto{\pgfqpoint{3.909217in}{2.244934in}}%
\pgfpathlineto{\pgfqpoint{3.914231in}{2.254550in}}%
\pgfpathlineto{\pgfqpoint{3.916737in}{2.256492in}}%
\pgfpathlineto{\pgfqpoint{3.921751in}{2.263615in}}%
\pgfpathlineto{\pgfqpoint{3.926764in}{2.277171in}}%
\pgfpathlineto{\pgfqpoint{3.931777in}{2.284747in}}%
\pgfpathlineto{\pgfqpoint{3.934284in}{2.296848in}}%
\pgfpathlineto{\pgfqpoint{3.936791in}{2.299341in}}%
\pgfpathlineto{\pgfqpoint{3.939297in}{2.303997in}}%
\pgfpathlineto{\pgfqpoint{3.941804in}{2.305275in}}%
\pgfpathlineto{\pgfqpoint{5.583663in}{2.306381in}}%
\pgfpathlineto{\pgfqpoint{5.633796in}{2.307602in}}%
\pgfpathlineto{\pgfqpoint{5.643822in}{2.310180in}}%
\pgfpathlineto{\pgfqpoint{5.653730in}{2.315275in}}%
\pgfpathlineto{\pgfqpoint{5.653730in}{2.315275in}}%
\pgfusepath{stroke}%
\end{pgfscope}%
\begin{pgfscope}%
\pgfpathrectangle{\pgfqpoint{0.708220in}{0.535823in}}{\pgfqpoint{5.013309in}{1.769453in}}%
\pgfusepath{clip}%
\pgfsetbuttcap%
\pgfsetroundjoin%
\pgfsetlinewidth{1.003750pt}%
\definecolor{currentstroke}{rgb}{0.000000,0.000000,0.000000}%
\pgfsetstrokecolor{currentstroke}%
\pgfsetdash{{1.000000pt}{1.650000pt}}{0.000000pt}%
\pgfpathmoveto{\pgfqpoint{0.708220in}{0.606902in}}%
\pgfpathlineto{\pgfqpoint{0.710727in}{0.606909in}}%
\pgfpathlineto{\pgfqpoint{0.713233in}{0.623389in}}%
\pgfpathlineto{\pgfqpoint{0.718246in}{0.625894in}}%
\pgfpathlineto{\pgfqpoint{0.720753in}{0.628848in}}%
\pgfpathlineto{\pgfqpoint{0.723260in}{0.629962in}}%
\pgfpathlineto{\pgfqpoint{0.725766in}{0.633641in}}%
\pgfpathlineto{\pgfqpoint{0.730780in}{0.635676in}}%
\pgfpathlineto{\pgfqpoint{0.733286in}{0.647111in}}%
\pgfpathlineto{\pgfqpoint{0.738300in}{0.650451in}}%
\pgfpathlineto{\pgfqpoint{0.740806in}{0.651568in}}%
\pgfpathlineto{\pgfqpoint{0.745820in}{0.666362in}}%
\pgfpathlineto{\pgfqpoint{0.748326in}{0.667704in}}%
\pgfpathlineto{\pgfqpoint{0.750833in}{0.672109in}}%
\pgfpathlineto{\pgfqpoint{0.755846in}{0.673462in}}%
\pgfpathlineto{\pgfqpoint{0.758353in}{0.684636in}}%
\pgfpathlineto{\pgfqpoint{0.760860in}{0.684678in}}%
\pgfpathlineto{\pgfqpoint{0.763366in}{0.689845in}}%
\pgfpathlineto{\pgfqpoint{0.768380in}{0.695312in}}%
\pgfpathlineto{\pgfqpoint{0.770886in}{0.700207in}}%
\pgfpathlineto{\pgfqpoint{0.773393in}{0.700521in}}%
\pgfpathlineto{\pgfqpoint{0.775900in}{0.705081in}}%
\pgfpathlineto{\pgfqpoint{0.785926in}{0.708075in}}%
\pgfpathlineto{\pgfqpoint{0.788433in}{0.708106in}}%
\pgfpathlineto{\pgfqpoint{0.793446in}{0.729832in}}%
\pgfpathlineto{\pgfqpoint{0.795953in}{0.732282in}}%
\pgfpathlineto{\pgfqpoint{0.798459in}{0.732707in}}%
\pgfpathlineto{\pgfqpoint{0.800966in}{0.735926in}}%
\pgfpathlineto{\pgfqpoint{0.803473in}{0.736115in}}%
\pgfpathlineto{\pgfqpoint{0.805979in}{0.738305in}}%
\pgfpathlineto{\pgfqpoint{0.808486in}{0.738459in}}%
\pgfpathlineto{\pgfqpoint{0.816006in}{0.747331in}}%
\pgfpathlineto{\pgfqpoint{0.821019in}{0.748383in}}%
\pgfpathlineto{\pgfqpoint{0.823526in}{0.755766in}}%
\pgfpathlineto{\pgfqpoint{0.826033in}{0.758654in}}%
\pgfpathlineto{\pgfqpoint{0.836059in}{0.760097in}}%
\pgfpathlineto{\pgfqpoint{0.843579in}{0.761299in}}%
\pgfpathlineto{\pgfqpoint{0.868646in}{0.762483in}}%
\pgfpathlineto{\pgfqpoint{0.886192in}{0.768868in}}%
\pgfpathlineto{\pgfqpoint{0.923792in}{0.771499in}}%
\pgfpathlineto{\pgfqpoint{0.948859in}{0.772826in}}%
\pgfpathlineto{\pgfqpoint{0.958885in}{0.776772in}}%
\pgfpathlineto{\pgfqpoint{1.086725in}{0.786923in}}%
\pgfpathlineto{\pgfqpoint{1.111791in}{0.788011in}}%
\pgfpathlineto{\pgfqpoint{1.179471in}{0.792117in}}%
\pgfpathlineto{\pgfqpoint{1.192004in}{0.793866in}}%
\pgfpathlineto{\pgfqpoint{1.229604in}{0.795585in}}%
\pgfpathlineto{\pgfqpoint{1.257177in}{0.796690in}}%
\pgfpathlineto{\pgfqpoint{1.299790in}{0.798855in}}%
\pgfpathlineto{\pgfqpoint{1.302297in}{0.800419in}}%
\pgfpathlineto{\pgfqpoint{1.307310in}{0.801123in}}%
\pgfpathlineto{\pgfqpoint{1.312324in}{0.802657in}}%
\pgfpathlineto{\pgfqpoint{1.324857in}{0.803438in}}%
\pgfpathlineto{\pgfqpoint{1.387523in}{0.807104in}}%
\pgfpathlineto{\pgfqpoint{1.402563in}{0.808679in}}%
\pgfpathlineto{\pgfqpoint{1.425123in}{0.809712in}}%
\pgfpathlineto{\pgfqpoint{1.492803in}{0.811762in}}%
\pgfpathlineto{\pgfqpoint{1.507843in}{0.813161in}}%
\pgfpathlineto{\pgfqpoint{1.515363in}{0.814422in}}%
\pgfpathlineto{\pgfqpoint{1.583042in}{0.817558in}}%
\pgfpathlineto{\pgfqpoint{1.630669in}{0.818463in}}%
\pgfpathlineto{\pgfqpoint{1.640695in}{0.819192in}}%
\pgfpathlineto{\pgfqpoint{1.663255in}{0.820452in}}%
\pgfpathlineto{\pgfqpoint{1.678295in}{0.822106in}}%
\pgfpathlineto{\pgfqpoint{1.733442in}{0.827303in}}%
\pgfpathlineto{\pgfqpoint{1.735948in}{0.830131in}}%
\pgfpathlineto{\pgfqpoint{1.748481in}{0.831369in}}%
\pgfpathlineto{\pgfqpoint{1.750988in}{0.831758in}}%
\pgfpathlineto{\pgfqpoint{1.753495in}{0.834437in}}%
\pgfpathlineto{\pgfqpoint{1.763521in}{0.836267in}}%
\pgfpathlineto{\pgfqpoint{1.786081in}{0.837878in}}%
\pgfpathlineto{\pgfqpoint{1.803628in}{0.840893in}}%
\pgfpathlineto{\pgfqpoint{1.913921in}{0.844206in}}%
\pgfpathlineto{\pgfqpoint{1.921441in}{0.845030in}}%
\pgfpathlineto{\pgfqpoint{1.931467in}{0.846116in}}%
\pgfpathlineto{\pgfqpoint{1.936481in}{0.847616in}}%
\pgfpathlineto{\pgfqpoint{1.954027in}{0.848968in}}%
\pgfpathlineto{\pgfqpoint{1.989120in}{0.852459in}}%
\pgfpathlineto{\pgfqpoint{1.994134in}{0.853520in}}%
\pgfpathlineto{\pgfqpoint{2.021707in}{0.857179in}}%
\pgfpathlineto{\pgfqpoint{2.024213in}{0.859502in}}%
\pgfpathlineto{\pgfqpoint{2.026720in}{0.859516in}}%
\pgfpathlineto{\pgfqpoint{2.039253in}{0.869596in}}%
\pgfpathlineto{\pgfqpoint{2.046773in}{0.872123in}}%
\pgfpathlineto{\pgfqpoint{2.049280in}{0.872164in}}%
\pgfpathlineto{\pgfqpoint{2.056800in}{0.877384in}}%
\pgfpathlineto{\pgfqpoint{2.061813in}{0.878600in}}%
\pgfpathlineto{\pgfqpoint{2.064320in}{0.878652in}}%
\pgfpathlineto{\pgfqpoint{2.066827in}{0.883782in}}%
\pgfpathlineto{\pgfqpoint{2.076853in}{0.885945in}}%
\pgfpathlineto{\pgfqpoint{2.084373in}{0.889875in}}%
\pgfpathlineto{\pgfqpoint{2.106933in}{0.891575in}}%
\pgfpathlineto{\pgfqpoint{2.129493in}{0.896842in}}%
\pgfpathlineto{\pgfqpoint{2.132000in}{0.898414in}}%
\pgfpathlineto{\pgfqpoint{2.137013in}{0.898667in}}%
\pgfpathlineto{\pgfqpoint{2.139520in}{0.900634in}}%
\pgfpathlineto{\pgfqpoint{2.159573in}{0.904179in}}%
\pgfpathlineto{\pgfqpoint{2.164586in}{0.905653in}}%
\pgfpathlineto{\pgfqpoint{2.167093in}{0.905743in}}%
\pgfpathlineto{\pgfqpoint{2.169599in}{0.907613in}}%
\pgfpathlineto{\pgfqpoint{2.177119in}{0.907678in}}%
\pgfpathlineto{\pgfqpoint{2.182133in}{0.912064in}}%
\pgfpathlineto{\pgfqpoint{2.184639in}{0.912082in}}%
\pgfpathlineto{\pgfqpoint{2.189653in}{0.915705in}}%
\pgfpathlineto{\pgfqpoint{2.192159in}{0.917996in}}%
\pgfpathlineto{\pgfqpoint{2.202186in}{0.918832in}}%
\pgfpathlineto{\pgfqpoint{2.232266in}{0.924385in}}%
\pgfpathlineto{\pgfqpoint{2.242292in}{0.926340in}}%
\pgfpathlineto{\pgfqpoint{2.244799in}{0.928400in}}%
\pgfpathlineto{\pgfqpoint{2.269866in}{0.929334in}}%
\pgfpathlineto{\pgfqpoint{2.274879in}{0.930631in}}%
\pgfpathlineto{\pgfqpoint{2.287412in}{0.932327in}}%
\pgfpathlineto{\pgfqpoint{2.302452in}{0.933156in}}%
\pgfpathlineto{\pgfqpoint{2.314985in}{0.936169in}}%
\pgfpathlineto{\pgfqpoint{2.332532in}{0.936963in}}%
\pgfpathlineto{\pgfqpoint{2.335039in}{0.939931in}}%
\pgfpathlineto{\pgfqpoint{2.347572in}{0.940691in}}%
\pgfpathlineto{\pgfqpoint{2.352585in}{0.943615in}}%
\pgfpathlineto{\pgfqpoint{2.372638in}{0.944141in}}%
\pgfpathlineto{\pgfqpoint{2.380158in}{0.947072in}}%
\pgfpathlineto{\pgfqpoint{2.400212in}{0.947225in}}%
\pgfpathlineto{\pgfqpoint{2.412745in}{0.947740in}}%
\pgfpathlineto{\pgfqpoint{2.417758in}{0.950082in}}%
\pgfpathlineto{\pgfqpoint{2.435305in}{0.950763in}}%
\pgfpathlineto{\pgfqpoint{2.452851in}{0.950763in}}%
\pgfpathlineto{\pgfqpoint{2.455358in}{0.953389in}}%
\pgfpathlineto{\pgfqpoint{2.460371in}{0.954232in}}%
\pgfpathlineto{\pgfqpoint{2.477918in}{0.955128in}}%
\pgfpathlineto{\pgfqpoint{2.487944in}{0.957636in}}%
\pgfpathlineto{\pgfqpoint{2.497971in}{0.958122in}}%
\pgfpathlineto{\pgfqpoint{2.500478in}{0.960975in}}%
\pgfpathlineto{\pgfqpoint{2.518024in}{0.961452in}}%
\pgfpathlineto{\pgfqpoint{2.520531in}{0.963854in}}%
\pgfpathlineto{\pgfqpoint{2.530558in}{0.964722in}}%
\pgfpathlineto{\pgfqpoint{2.535571in}{0.964722in}}%
\pgfpathlineto{\pgfqpoint{2.543091in}{0.967933in}}%
\pgfpathlineto{\pgfqpoint{2.553117in}{0.970635in}}%
\pgfpathlineto{\pgfqpoint{2.560637in}{0.971297in}}%
\pgfpathlineto{\pgfqpoint{2.573171in}{0.974187in}}%
\pgfpathlineto{\pgfqpoint{2.578184in}{0.974187in}}%
\pgfpathlineto{\pgfqpoint{2.583197in}{0.976648in}}%
\pgfpathlineto{\pgfqpoint{2.588211in}{0.977016in}}%
\pgfpathlineto{\pgfqpoint{2.590717in}{0.979800in}}%
\pgfpathlineto{\pgfqpoint{2.598237in}{0.981098in}}%
\pgfpathlineto{\pgfqpoint{2.605757in}{0.983176in}}%
\pgfpathlineto{\pgfqpoint{2.608264in}{0.986214in}}%
\pgfpathlineto{\pgfqpoint{2.610771in}{0.986501in}}%
\pgfpathlineto{\pgfqpoint{2.613277in}{0.988517in}}%
\pgfpathlineto{\pgfqpoint{2.615784in}{0.988517in}}%
\pgfpathlineto{\pgfqpoint{2.623304in}{0.991331in}}%
\pgfpathlineto{\pgfqpoint{2.625810in}{0.991331in}}%
\pgfpathlineto{\pgfqpoint{2.633330in}{0.999901in}}%
\pgfpathlineto{\pgfqpoint{2.635837in}{1.000286in}}%
\pgfpathlineto{\pgfqpoint{2.638344in}{1.002163in}}%
\pgfpathlineto{\pgfqpoint{2.645864in}{1.002805in}}%
\pgfpathlineto{\pgfqpoint{2.650877in}{1.004169in}}%
\pgfpathlineto{\pgfqpoint{2.663410in}{1.006982in}}%
\pgfpathlineto{\pgfqpoint{2.678450in}{1.010576in}}%
\pgfpathlineto{\pgfqpoint{2.680957in}{1.013067in}}%
\pgfpathlineto{\pgfqpoint{2.683464in}{1.013307in}}%
\pgfpathlineto{\pgfqpoint{2.688477in}{1.016406in}}%
\pgfpathlineto{\pgfqpoint{2.690983in}{1.017164in}}%
\pgfpathlineto{\pgfqpoint{2.693490in}{1.022375in}}%
\pgfpathlineto{\pgfqpoint{2.695997in}{1.022626in}}%
\pgfpathlineto{\pgfqpoint{2.701010in}{1.024854in}}%
\pgfpathlineto{\pgfqpoint{2.706023in}{1.026136in}}%
\pgfpathlineto{\pgfqpoint{2.716050in}{1.028953in}}%
\pgfpathlineto{\pgfqpoint{2.718557in}{1.032940in}}%
\pgfpathlineto{\pgfqpoint{2.723570in}{1.035612in}}%
\pgfpathlineto{\pgfqpoint{2.728583in}{1.036603in}}%
\pgfpathlineto{\pgfqpoint{2.736103in}{1.039720in}}%
\pgfpathlineto{\pgfqpoint{2.741117in}{1.041871in}}%
\pgfpathlineto{\pgfqpoint{2.746130in}{1.042032in}}%
\pgfpathlineto{\pgfqpoint{2.748637in}{1.045563in}}%
\pgfpathlineto{\pgfqpoint{2.753650in}{1.046600in}}%
\pgfpathlineto{\pgfqpoint{2.758663in}{1.049565in}}%
\pgfpathlineto{\pgfqpoint{2.761170in}{1.049982in}}%
\pgfpathlineto{\pgfqpoint{2.763676in}{1.053770in}}%
\pgfpathlineto{\pgfqpoint{2.766183in}{1.055007in}}%
\pgfpathlineto{\pgfqpoint{2.768690in}{1.060014in}}%
\pgfpathlineto{\pgfqpoint{2.778716in}{1.064413in}}%
\pgfpathlineto{\pgfqpoint{2.781223in}{1.069690in}}%
\pgfpathlineto{\pgfqpoint{2.791250in}{1.074221in}}%
\pgfpathlineto{\pgfqpoint{2.796263in}{1.076960in}}%
\pgfpathlineto{\pgfqpoint{2.801276in}{1.078174in}}%
\pgfpathlineto{\pgfqpoint{2.806290in}{1.078809in}}%
\pgfpathlineto{\pgfqpoint{2.808796in}{1.081192in}}%
\pgfpathlineto{\pgfqpoint{2.811303in}{1.081323in}}%
\pgfpathlineto{\pgfqpoint{2.813810in}{1.088096in}}%
\pgfpathlineto{\pgfqpoint{2.818823in}{1.089676in}}%
\pgfpathlineto{\pgfqpoint{2.821329in}{1.092906in}}%
\pgfpathlineto{\pgfqpoint{2.823836in}{1.099693in}}%
\pgfpathlineto{\pgfqpoint{2.828849in}{1.100333in}}%
\pgfpathlineto{\pgfqpoint{2.833863in}{1.111284in}}%
\pgfpathlineto{\pgfqpoint{2.836369in}{1.112633in}}%
\pgfpathlineto{\pgfqpoint{2.843889in}{1.119921in}}%
\pgfpathlineto{\pgfqpoint{2.851409in}{1.121181in}}%
\pgfpathlineto{\pgfqpoint{2.866449in}{1.127316in}}%
\pgfpathlineto{\pgfqpoint{2.871463in}{1.127540in}}%
\pgfpathlineto{\pgfqpoint{2.876476in}{1.130496in}}%
\pgfpathlineto{\pgfqpoint{2.878983in}{1.135023in}}%
\pgfpathlineto{\pgfqpoint{2.881489in}{1.135655in}}%
\pgfpathlineto{\pgfqpoint{2.883996in}{1.140941in}}%
\pgfpathlineto{\pgfqpoint{2.894022in}{1.144363in}}%
\pgfpathlineto{\pgfqpoint{2.896529in}{1.148215in}}%
\pgfpathlineto{\pgfqpoint{2.899036in}{1.148391in}}%
\pgfpathlineto{\pgfqpoint{2.904049in}{1.152797in}}%
\pgfpathlineto{\pgfqpoint{2.906556in}{1.157212in}}%
\pgfpathlineto{\pgfqpoint{2.921596in}{1.162576in}}%
\pgfpathlineto{\pgfqpoint{2.929116in}{1.163824in}}%
\pgfpathlineto{\pgfqpoint{2.946662in}{1.169002in}}%
\pgfpathlineto{\pgfqpoint{2.954182in}{1.173741in}}%
\pgfpathlineto{\pgfqpoint{2.956689in}{1.174061in}}%
\pgfpathlineto{\pgfqpoint{2.966715in}{1.181129in}}%
\pgfpathlineto{\pgfqpoint{2.971729in}{1.182141in}}%
\pgfpathlineto{\pgfqpoint{2.974235in}{1.188597in}}%
\pgfpathlineto{\pgfqpoint{2.979249in}{1.191304in}}%
\pgfpathlineto{\pgfqpoint{2.981755in}{1.191484in}}%
\pgfpathlineto{\pgfqpoint{2.984262in}{1.194148in}}%
\pgfpathlineto{\pgfqpoint{2.991782in}{1.194843in}}%
\pgfpathlineto{\pgfqpoint{2.996795in}{1.197998in}}%
\pgfpathlineto{\pgfqpoint{3.004315in}{1.201786in}}%
\pgfpathlineto{\pgfqpoint{3.009329in}{1.202803in}}%
\pgfpathlineto{\pgfqpoint{3.016849in}{1.204420in}}%
\pgfpathlineto{\pgfqpoint{3.019355in}{1.209582in}}%
\pgfpathlineto{\pgfqpoint{3.034395in}{1.213597in}}%
\pgfpathlineto{\pgfqpoint{3.036902in}{1.216750in}}%
\pgfpathlineto{\pgfqpoint{3.041915in}{1.218703in}}%
\pgfpathlineto{\pgfqpoint{3.046928in}{1.223956in}}%
\pgfpathlineto{\pgfqpoint{3.051942in}{1.224707in}}%
\pgfpathlineto{\pgfqpoint{3.056955in}{1.225497in}}%
\pgfpathlineto{\pgfqpoint{3.059462in}{1.230924in}}%
\pgfpathlineto{\pgfqpoint{3.061968in}{1.231466in}}%
\pgfpathlineto{\pgfqpoint{3.064475in}{1.236787in}}%
\pgfpathlineto{\pgfqpoint{3.069488in}{1.240527in}}%
\pgfpathlineto{\pgfqpoint{3.074502in}{1.240781in}}%
\pgfpathlineto{\pgfqpoint{3.084528in}{1.248516in}}%
\pgfpathlineto{\pgfqpoint{3.092048in}{1.249079in}}%
\pgfpathlineto{\pgfqpoint{3.097061in}{1.251446in}}%
\pgfpathlineto{\pgfqpoint{3.099568in}{1.251456in}}%
\pgfpathlineto{\pgfqpoint{3.102075in}{1.254675in}}%
\pgfpathlineto{\pgfqpoint{3.112101in}{1.257748in}}%
\pgfpathlineto{\pgfqpoint{3.114608in}{1.259747in}}%
\pgfpathlineto{\pgfqpoint{3.117115in}{1.260194in}}%
\pgfpathlineto{\pgfqpoint{3.119621in}{1.262275in}}%
\pgfpathlineto{\pgfqpoint{3.124635in}{1.268521in}}%
\pgfpathlineto{\pgfqpoint{3.129648in}{1.269337in}}%
\pgfpathlineto{\pgfqpoint{3.132155in}{1.271238in}}%
\pgfpathlineto{\pgfqpoint{3.134661in}{1.274604in}}%
\pgfpathlineto{\pgfqpoint{3.152208in}{1.285216in}}%
\pgfpathlineto{\pgfqpoint{3.159728in}{1.294994in}}%
\pgfpathlineto{\pgfqpoint{3.179781in}{1.308077in}}%
\pgfpathlineto{\pgfqpoint{3.182288in}{1.308328in}}%
\pgfpathlineto{\pgfqpoint{3.184794in}{1.309811in}}%
\pgfpathlineto{\pgfqpoint{3.192314in}{1.316907in}}%
\pgfpathlineto{\pgfqpoint{3.199834in}{1.317219in}}%
\pgfpathlineto{\pgfqpoint{3.204848in}{1.324504in}}%
\pgfpathlineto{\pgfqpoint{3.222394in}{1.330195in}}%
\pgfpathlineto{\pgfqpoint{3.224901in}{1.337404in}}%
\pgfpathlineto{\pgfqpoint{3.227408in}{1.337842in}}%
\pgfpathlineto{\pgfqpoint{3.229914in}{1.342067in}}%
\pgfpathlineto{\pgfqpoint{3.232421in}{1.342148in}}%
\pgfpathlineto{\pgfqpoint{3.234927in}{1.345149in}}%
\pgfpathlineto{\pgfqpoint{3.239941in}{1.345867in}}%
\pgfpathlineto{\pgfqpoint{3.247461in}{1.347940in}}%
\pgfpathlineto{\pgfqpoint{3.249967in}{1.354597in}}%
\pgfpathlineto{\pgfqpoint{3.265007in}{1.367008in}}%
\pgfpathlineto{\pgfqpoint{3.272527in}{1.368190in}}%
\pgfpathlineto{\pgfqpoint{3.277541in}{1.371809in}}%
\pgfpathlineto{\pgfqpoint{3.287567in}{1.379586in}}%
\pgfpathlineto{\pgfqpoint{3.315140in}{1.388634in}}%
\pgfpathlineto{\pgfqpoint{3.317647in}{1.392034in}}%
\pgfpathlineto{\pgfqpoint{3.322660in}{1.393652in}}%
\pgfpathlineto{\pgfqpoint{3.325167in}{1.396125in}}%
\pgfpathlineto{\pgfqpoint{3.332687in}{1.397925in}}%
\pgfpathlineto{\pgfqpoint{3.350234in}{1.404429in}}%
\pgfpathlineto{\pgfqpoint{3.352740in}{1.406550in}}%
\pgfpathlineto{\pgfqpoint{3.355247in}{1.406804in}}%
\pgfpathlineto{\pgfqpoint{3.362767in}{1.413913in}}%
\pgfpathlineto{\pgfqpoint{3.372793in}{1.417952in}}%
\pgfpathlineto{\pgfqpoint{3.375300in}{1.420044in}}%
\pgfpathlineto{\pgfqpoint{3.380313in}{1.420347in}}%
\pgfpathlineto{\pgfqpoint{3.382820in}{1.425997in}}%
\pgfpathlineto{\pgfqpoint{3.387833in}{1.427704in}}%
\pgfpathlineto{\pgfqpoint{3.390340in}{1.427951in}}%
\pgfpathlineto{\pgfqpoint{3.395353in}{1.434339in}}%
\pgfpathlineto{\pgfqpoint{3.397860in}{1.436193in}}%
\pgfpathlineto{\pgfqpoint{3.402873in}{1.436930in}}%
\pgfpathlineto{\pgfqpoint{3.405380in}{1.440055in}}%
\pgfpathlineto{\pgfqpoint{3.407887in}{1.440082in}}%
\pgfpathlineto{\pgfqpoint{3.410393in}{1.442602in}}%
\pgfpathlineto{\pgfqpoint{3.412900in}{1.446773in}}%
\pgfpathlineto{\pgfqpoint{3.417913in}{1.447890in}}%
\pgfpathlineto{\pgfqpoint{3.420420in}{1.450683in}}%
\pgfpathlineto{\pgfqpoint{3.425433in}{1.460044in}}%
\pgfpathlineto{\pgfqpoint{3.437966in}{1.465133in}}%
\pgfpathlineto{\pgfqpoint{3.440473in}{1.468714in}}%
\pgfpathlineto{\pgfqpoint{3.445486in}{1.468980in}}%
\pgfpathlineto{\pgfqpoint{3.447993in}{1.471606in}}%
\pgfpathlineto{\pgfqpoint{3.453006in}{1.472741in}}%
\pgfpathlineto{\pgfqpoint{3.460526in}{1.480368in}}%
\pgfpathlineto{\pgfqpoint{3.465540in}{1.491528in}}%
\pgfpathlineto{\pgfqpoint{3.468046in}{1.496745in}}%
\pgfpathlineto{\pgfqpoint{3.470553in}{1.498568in}}%
\pgfpathlineto{\pgfqpoint{3.473060in}{1.498686in}}%
\pgfpathlineto{\pgfqpoint{3.483086in}{1.509725in}}%
\pgfpathlineto{\pgfqpoint{3.485593in}{1.516441in}}%
\pgfpathlineto{\pgfqpoint{3.490606in}{1.518104in}}%
\pgfpathlineto{\pgfqpoint{3.493113in}{1.520038in}}%
\pgfpathlineto{\pgfqpoint{3.495620in}{1.523541in}}%
\pgfpathlineto{\pgfqpoint{3.503139in}{1.527552in}}%
\pgfpathlineto{\pgfqpoint{3.505646in}{1.531117in}}%
\pgfpathlineto{\pgfqpoint{3.513166in}{1.532084in}}%
\pgfpathlineto{\pgfqpoint{3.515673in}{1.534760in}}%
\pgfpathlineto{\pgfqpoint{3.520686in}{1.537585in}}%
\pgfpathlineto{\pgfqpoint{3.523193in}{1.540341in}}%
\pgfpathlineto{\pgfqpoint{3.525699in}{1.540616in}}%
\pgfpathlineto{\pgfqpoint{3.528206in}{1.544695in}}%
\pgfpathlineto{\pgfqpoint{3.540739in}{1.548006in}}%
\pgfpathlineto{\pgfqpoint{3.545753in}{1.551427in}}%
\pgfpathlineto{\pgfqpoint{3.550766in}{1.551892in}}%
\pgfpathlineto{\pgfqpoint{3.558286in}{1.558392in}}%
\pgfpathlineto{\pgfqpoint{3.560793in}{1.559280in}}%
\pgfpathlineto{\pgfqpoint{3.563299in}{1.561923in}}%
\pgfpathlineto{\pgfqpoint{3.575832in}{1.565775in}}%
\pgfpathlineto{\pgfqpoint{3.578339in}{1.568955in}}%
\pgfpathlineto{\pgfqpoint{3.580846in}{1.569370in}}%
\pgfpathlineto{\pgfqpoint{3.583352in}{1.571511in}}%
\pgfpathlineto{\pgfqpoint{3.588366in}{1.572161in}}%
\pgfpathlineto{\pgfqpoint{3.590872in}{1.574664in}}%
\pgfpathlineto{\pgfqpoint{3.595886in}{1.576939in}}%
\pgfpathlineto{\pgfqpoint{3.603406in}{1.578832in}}%
\pgfpathlineto{\pgfqpoint{3.605912in}{1.583399in}}%
\pgfpathlineto{\pgfqpoint{3.613432in}{1.587577in}}%
\pgfpathlineto{\pgfqpoint{3.615939in}{1.590007in}}%
\pgfpathlineto{\pgfqpoint{3.618446in}{1.590414in}}%
\pgfpathlineto{\pgfqpoint{3.620952in}{1.594689in}}%
\pgfpathlineto{\pgfqpoint{3.630979in}{1.600336in}}%
\pgfpathlineto{\pgfqpoint{3.635992in}{1.606812in}}%
\pgfpathlineto{\pgfqpoint{3.638499in}{1.607061in}}%
\pgfpathlineto{\pgfqpoint{3.648525in}{1.612427in}}%
\pgfpathlineto{\pgfqpoint{3.658552in}{1.616161in}}%
\pgfpathlineto{\pgfqpoint{3.661059in}{1.623116in}}%
\pgfpathlineto{\pgfqpoint{3.663565in}{1.623658in}}%
\pgfpathlineto{\pgfqpoint{3.666072in}{1.628394in}}%
\pgfpathlineto{\pgfqpoint{3.668579in}{1.629044in}}%
\pgfpathlineto{\pgfqpoint{3.676099in}{1.634942in}}%
\pgfpathlineto{\pgfqpoint{3.678605in}{1.635283in}}%
\pgfpathlineto{\pgfqpoint{3.681112in}{1.643855in}}%
\pgfpathlineto{\pgfqpoint{3.683619in}{1.646379in}}%
\pgfpathlineto{\pgfqpoint{3.691139in}{1.666646in}}%
\pgfpathlineto{\pgfqpoint{3.693645in}{1.667061in}}%
\pgfpathlineto{\pgfqpoint{3.696152in}{1.672782in}}%
\pgfpathlineto{\pgfqpoint{3.698659in}{1.675367in}}%
\pgfpathlineto{\pgfqpoint{3.703672in}{1.685484in}}%
\pgfpathlineto{\pgfqpoint{3.723725in}{1.694998in}}%
\pgfpathlineto{\pgfqpoint{3.731245in}{1.705207in}}%
\pgfpathlineto{\pgfqpoint{3.733752in}{1.705375in}}%
\pgfpathlineto{\pgfqpoint{3.738765in}{1.709010in}}%
\pgfpathlineto{\pgfqpoint{3.741272in}{1.716229in}}%
\pgfpathlineto{\pgfqpoint{3.746285in}{1.719155in}}%
\pgfpathlineto{\pgfqpoint{3.748792in}{1.720046in}}%
\pgfpathlineto{\pgfqpoint{3.751298in}{1.722704in}}%
\pgfpathlineto{\pgfqpoint{3.753805in}{1.732986in}}%
\pgfpathlineto{\pgfqpoint{3.758818in}{1.733084in}}%
\pgfpathlineto{\pgfqpoint{3.766338in}{1.737778in}}%
\pgfpathlineto{\pgfqpoint{3.771352in}{1.739745in}}%
\pgfpathlineto{\pgfqpoint{3.773858in}{1.742968in}}%
\pgfpathlineto{\pgfqpoint{3.776365in}{1.748145in}}%
\pgfpathlineto{\pgfqpoint{3.781378in}{1.748762in}}%
\pgfpathlineto{\pgfqpoint{3.786391in}{1.754628in}}%
\pgfpathlineto{\pgfqpoint{3.791405in}{1.758631in}}%
\pgfpathlineto{\pgfqpoint{3.793911in}{1.759523in}}%
\pgfpathlineto{\pgfqpoint{3.796418in}{1.765057in}}%
\pgfpathlineto{\pgfqpoint{3.798925in}{1.766510in}}%
\pgfpathlineto{\pgfqpoint{3.801431in}{1.766523in}}%
\pgfpathlineto{\pgfqpoint{3.803938in}{1.772317in}}%
\pgfpathlineto{\pgfqpoint{3.808951in}{1.774371in}}%
\pgfpathlineto{\pgfqpoint{3.813965in}{1.774989in}}%
\pgfpathlineto{\pgfqpoint{3.821485in}{1.786907in}}%
\pgfpathlineto{\pgfqpoint{3.826498in}{1.787511in}}%
\pgfpathlineto{\pgfqpoint{3.831511in}{1.795347in}}%
\pgfpathlineto{\pgfqpoint{3.834018in}{1.795816in}}%
\pgfpathlineto{\pgfqpoint{3.839031in}{1.800092in}}%
\pgfpathlineto{\pgfqpoint{3.844044in}{1.803297in}}%
\pgfpathlineto{\pgfqpoint{3.849058in}{1.810520in}}%
\pgfpathlineto{\pgfqpoint{3.851564in}{1.817907in}}%
\pgfpathlineto{\pgfqpoint{3.854071in}{1.821216in}}%
\pgfpathlineto{\pgfqpoint{3.864098in}{1.823747in}}%
\pgfpathlineto{\pgfqpoint{3.871618in}{1.825220in}}%
\pgfpathlineto{\pgfqpoint{3.874124in}{1.827041in}}%
\pgfpathlineto{\pgfqpoint{3.876631in}{1.831703in}}%
\pgfpathlineto{\pgfqpoint{3.879138in}{1.839257in}}%
\pgfpathlineto{\pgfqpoint{3.886658in}{1.840662in}}%
\pgfpathlineto{\pgfqpoint{3.891671in}{1.845476in}}%
\pgfpathlineto{\pgfqpoint{3.894178in}{1.846854in}}%
\pgfpathlineto{\pgfqpoint{3.896684in}{1.846955in}}%
\pgfpathlineto{\pgfqpoint{3.901698in}{1.868170in}}%
\pgfpathlineto{\pgfqpoint{3.904204in}{1.869193in}}%
\pgfpathlineto{\pgfqpoint{3.906711in}{1.871647in}}%
\pgfpathlineto{\pgfqpoint{3.909217in}{1.876424in}}%
\pgfpathlineto{\pgfqpoint{3.914231in}{1.876979in}}%
\pgfpathlineto{\pgfqpoint{3.919244in}{1.881326in}}%
\pgfpathlineto{\pgfqpoint{3.921751in}{1.886798in}}%
\pgfpathlineto{\pgfqpoint{3.924257in}{1.887606in}}%
\pgfpathlineto{\pgfqpoint{3.926764in}{1.890237in}}%
\pgfpathlineto{\pgfqpoint{3.929271in}{1.891209in}}%
\pgfpathlineto{\pgfqpoint{3.931777in}{1.893734in}}%
\pgfpathlineto{\pgfqpoint{3.934284in}{1.894498in}}%
\pgfpathlineto{\pgfqpoint{3.939297in}{1.905931in}}%
\pgfpathlineto{\pgfqpoint{3.941804in}{1.906587in}}%
\pgfpathlineto{\pgfqpoint{3.944311in}{1.910910in}}%
\pgfpathlineto{\pgfqpoint{3.946817in}{1.911334in}}%
\pgfpathlineto{\pgfqpoint{3.949324in}{1.913098in}}%
\pgfpathlineto{\pgfqpoint{3.954337in}{1.928372in}}%
\pgfpathlineto{\pgfqpoint{3.956844in}{1.933133in}}%
\pgfpathlineto{\pgfqpoint{3.959351in}{1.934154in}}%
\pgfpathlineto{\pgfqpoint{3.964364in}{1.960113in}}%
\pgfpathlineto{\pgfqpoint{3.966871in}{1.963980in}}%
\pgfpathlineto{\pgfqpoint{3.969377in}{1.964112in}}%
\pgfpathlineto{\pgfqpoint{3.974391in}{1.976067in}}%
\pgfpathlineto{\pgfqpoint{3.976897in}{1.981972in}}%
\pgfpathlineto{\pgfqpoint{3.984417in}{1.988492in}}%
\pgfpathlineto{\pgfqpoint{3.989430in}{1.995400in}}%
\pgfpathlineto{\pgfqpoint{3.991937in}{1.997503in}}%
\pgfpathlineto{\pgfqpoint{3.994444in}{1.997521in}}%
\pgfpathlineto{\pgfqpoint{3.996950in}{2.000954in}}%
\pgfpathlineto{\pgfqpoint{4.001964in}{2.001655in}}%
\pgfpathlineto{\pgfqpoint{4.009484in}{2.007102in}}%
\pgfpathlineto{\pgfqpoint{4.011990in}{2.012703in}}%
\pgfpathlineto{\pgfqpoint{4.017004in}{2.014409in}}%
\pgfpathlineto{\pgfqpoint{4.019510in}{2.018616in}}%
\pgfpathlineto{\pgfqpoint{4.022017in}{2.028397in}}%
\pgfpathlineto{\pgfqpoint{4.027030in}{2.036388in}}%
\pgfpathlineto{\pgfqpoint{4.029537in}{2.039522in}}%
\pgfpathlineto{\pgfqpoint{4.032044in}{2.040614in}}%
\pgfpathlineto{\pgfqpoint{4.034550in}{2.046537in}}%
\pgfpathlineto{\pgfqpoint{4.044577in}{2.054209in}}%
\pgfpathlineto{\pgfqpoint{4.052097in}{2.068845in}}%
\pgfpathlineto{\pgfqpoint{4.054603in}{2.076634in}}%
\pgfpathlineto{\pgfqpoint{4.062123in}{2.082335in}}%
\pgfpathlineto{\pgfqpoint{4.064630in}{2.091566in}}%
\pgfpathlineto{\pgfqpoint{4.069643in}{2.098487in}}%
\pgfpathlineto{\pgfqpoint{4.072150in}{2.104782in}}%
\pgfpathlineto{\pgfqpoint{4.082177in}{2.111862in}}%
\pgfpathlineto{\pgfqpoint{4.084683in}{2.113436in}}%
\pgfpathlineto{\pgfqpoint{4.089697in}{2.119993in}}%
\pgfpathlineto{\pgfqpoint{4.092203in}{2.128533in}}%
\pgfpathlineto{\pgfqpoint{4.094710in}{2.131090in}}%
\pgfpathlineto{\pgfqpoint{4.097217in}{2.135601in}}%
\pgfpathlineto{\pgfqpoint{4.102230in}{2.151916in}}%
\pgfpathlineto{\pgfqpoint{4.104737in}{2.153324in}}%
\pgfpathlineto{\pgfqpoint{4.107243in}{2.157224in}}%
\pgfpathlineto{\pgfqpoint{4.112257in}{2.159337in}}%
\pgfpathlineto{\pgfqpoint{4.117270in}{2.165636in}}%
\pgfpathlineto{\pgfqpoint{4.119776in}{2.167539in}}%
\pgfpathlineto{\pgfqpoint{4.122283in}{2.180807in}}%
\pgfpathlineto{\pgfqpoint{4.124790in}{2.187520in}}%
\pgfpathlineto{\pgfqpoint{4.129803in}{2.203309in}}%
\pgfpathlineto{\pgfqpoint{4.132310in}{2.205024in}}%
\pgfpathlineto{\pgfqpoint{4.134816in}{2.209046in}}%
\pgfpathlineto{\pgfqpoint{4.139830in}{2.211083in}}%
\pgfpathlineto{\pgfqpoint{4.144843in}{2.225361in}}%
\pgfpathlineto{\pgfqpoint{4.147350in}{2.225442in}}%
\pgfpathlineto{\pgfqpoint{4.149856in}{2.229226in}}%
\pgfpathlineto{\pgfqpoint{4.152363in}{2.229682in}}%
\pgfpathlineto{\pgfqpoint{4.157376in}{2.242373in}}%
\pgfpathlineto{\pgfqpoint{4.159883in}{2.242924in}}%
\pgfpathlineto{\pgfqpoint{4.164896in}{2.266646in}}%
\pgfpathlineto{\pgfqpoint{4.167403in}{2.271904in}}%
\pgfpathlineto{\pgfqpoint{4.172416in}{2.273047in}}%
\pgfpathlineto{\pgfqpoint{4.174923in}{2.277171in}}%
\pgfpathlineto{\pgfqpoint{4.177430in}{2.291911in}}%
\pgfpathlineto{\pgfqpoint{4.187456in}{2.304876in}}%
\pgfpathlineto{\pgfqpoint{4.194976in}{2.305275in}}%
\pgfpathlineto{\pgfqpoint{5.583663in}{2.306381in}}%
\pgfpathlineto{\pgfqpoint{5.633796in}{2.307602in}}%
\pgfpathlineto{\pgfqpoint{5.643822in}{2.310180in}}%
\pgfpathlineto{\pgfqpoint{5.653730in}{2.315275in}}%
\pgfpathlineto{\pgfqpoint{5.653730in}{2.315275in}}%
\pgfusepath{stroke}%
\end{pgfscope}%
\begin{pgfscope}%
\pgfsetrectcap%
\pgfsetmiterjoin%
\pgfsetlinewidth{0.803000pt}%
\definecolor{currentstroke}{rgb}{0.000000,0.000000,0.000000}%
\pgfsetstrokecolor{currentstroke}%
\pgfsetdash{}{0pt}%
\pgfpathmoveto{\pgfqpoint{0.708220in}{0.535823in}}%
\pgfpathlineto{\pgfqpoint{0.708220in}{2.305275in}}%
\pgfusepath{stroke}%
\end{pgfscope}%
\begin{pgfscope}%
\pgfsetrectcap%
\pgfsetmiterjoin%
\pgfsetlinewidth{0.803000pt}%
\definecolor{currentstroke}{rgb}{0.000000,0.000000,0.000000}%
\pgfsetstrokecolor{currentstroke}%
\pgfsetdash{}{0pt}%
\pgfpathmoveto{\pgfqpoint{5.721529in}{0.535823in}}%
\pgfpathlineto{\pgfqpoint{5.721529in}{2.305275in}}%
\pgfusepath{stroke}%
\end{pgfscope}%
\begin{pgfscope}%
\pgfsetrectcap%
\pgfsetmiterjoin%
\pgfsetlinewidth{0.803000pt}%
\definecolor{currentstroke}{rgb}{0.000000,0.000000,0.000000}%
\pgfsetstrokecolor{currentstroke}%
\pgfsetdash{}{0pt}%
\pgfpathmoveto{\pgfqpoint{0.708220in}{0.535823in}}%
\pgfpathlineto{\pgfqpoint{5.721529in}{0.535823in}}%
\pgfusepath{stroke}%
\end{pgfscope}%
\begin{pgfscope}%
\pgfsetrectcap%
\pgfsetmiterjoin%
\pgfsetlinewidth{0.803000pt}%
\definecolor{currentstroke}{rgb}{0.000000,0.000000,0.000000}%
\pgfsetstrokecolor{currentstroke}%
\pgfsetdash{}{0pt}%
\pgfpathmoveto{\pgfqpoint{0.708220in}{2.305275in}}%
\pgfpathlineto{\pgfqpoint{5.721529in}{2.305275in}}%
\pgfusepath{stroke}%
\end{pgfscope}%
\begin{pgfscope}%
\pgfsetrectcap%
\pgfsetroundjoin%
\pgfsetlinewidth{1.003750pt}%
\definecolor{currentstroke}{rgb}{0.000000,0.000000,1.000000}%
\pgfsetstrokecolor{currentstroke}%
\pgfsetdash{}{0pt}%
\pgfpathmoveto{\pgfqpoint{4.396639in}{2.189088in}}%
\pgfpathlineto{\pgfqpoint{4.646639in}{2.189088in}}%
\pgfusepath{stroke}%
\end{pgfscope}%
\begin{pgfscope}%
\definecolor{textcolor}{rgb}{0.000000,0.000000,0.000000}%
\pgfsetstrokecolor{textcolor}%
\pgfsetfillcolor{textcolor}%
\pgftext[x=4.671639in,y=2.145338in,left,base]{\color{textcolor}\rmfamily\fontsize{9.000000}{10.800000}\selectfont tensor+HTB}%
\end{pgfscope}%
\begin{pgfscope}%
\pgfsetbuttcap%
\pgfsetroundjoin%
\pgfsetlinewidth{1.003750pt}%
\definecolor{currentstroke}{rgb}{0.000000,0.501961,0.000000}%
\pgfsetstrokecolor{currentstroke}%
\pgfsetdash{{3.700000pt}{1.600000pt}}{0.000000pt}%
\pgfpathmoveto{\pgfqpoint{4.396639in}{2.021061in}}%
\pgfpathlineto{\pgfqpoint{4.646639in}{2.021061in}}%
\pgfusepath{stroke}%
\end{pgfscope}%
\begin{pgfscope}%
\definecolor{textcolor}{rgb}{0.000000,0.000000,0.000000}%
\pgfsetstrokecolor{textcolor}%
\pgfsetfillcolor{textcolor}%
\pgftext[x=4.671639in,y=1.977311in,left,base]{\color{textcolor}\rmfamily\fontsize{9.000000}{10.800000}\selectfont tensor+LG (first)}%
\end{pgfscope}%
\begin{pgfscope}%
\pgfsetrectcap%
\pgfsetroundjoin%
\pgfsetlinewidth{1.003750pt}%
\definecolor{currentstroke}{rgb}{0.000000,0.501961,0.000000}%
\pgfsetstrokecolor{currentstroke}%
\pgfsetdash{}{0pt}%
\pgfpathmoveto{\pgfqpoint{4.396639in}{1.846092in}}%
\pgfpathlineto{\pgfqpoint{4.646639in}{1.846092in}}%
\pgfusepath{stroke}%
\end{pgfscope}%
\begin{pgfscope}%
\definecolor{textcolor}{rgb}{0.000000,0.000000,0.000000}%
\pgfsetstrokecolor{textcolor}%
\pgfsetfillcolor{textcolor}%
\pgftext[x=4.671639in,y=1.802342in,left,base]{\color{textcolor}\rmfamily\fontsize{9.000000}{10.800000}\selectfont tensor+LG (cost)}%
\end{pgfscope}%
\begin{pgfscope}%
\pgfsetbuttcap%
\pgfsetroundjoin%
\pgfsetlinewidth{1.003750pt}%
\definecolor{currentstroke}{rgb}{0.000000,0.501961,0.000000}%
\pgfsetstrokecolor{currentstroke}%
\pgfsetdash{{1.000000pt}{1.650000pt}}{0.000000pt}%
\pgfpathmoveto{\pgfqpoint{4.396639in}{1.671122in}}%
\pgfpathlineto{\pgfqpoint{4.646639in}{1.671122in}}%
\pgfusepath{stroke}%
\end{pgfscope}%
\begin{pgfscope}%
\definecolor{textcolor}{rgb}{0.000000,0.000000,0.000000}%
\pgfsetstrokecolor{textcolor}%
\pgfsetfillcolor{textcolor}%
\pgftext[x=4.671639in,y=1.627372in,left,base]{\color{textcolor}\rmfamily\fontsize{9.000000}{10.800000}\selectfont tensor+LG (best)}%
\end{pgfscope}%
\begin{pgfscope}%
\pgfsetrectcap%
\pgfsetroundjoin%
\pgfsetlinewidth{1.003750pt}%
\definecolor{currentstroke}{rgb}{0.000000,0.000000,0.000000}%
\pgfsetstrokecolor{currentstroke}%
\pgfsetdash{}{0pt}%
\pgfpathmoveto{\pgfqpoint{4.396639in}{1.502380in}}%
\pgfpathlineto{\pgfqpoint{4.646639in}{1.502380in}}%
\pgfusepath{stroke}%
\end{pgfscope}%
\begin{pgfscope}%
\definecolor{textcolor}{rgb}{0.000000,0.000000,0.000000}%
\pgfsetstrokecolor{textcolor}%
\pgfsetfillcolor{textcolor}%
\pgftext[x=4.671639in,y=1.458630in,left,base]{\color{textcolor}\rmfamily\fontsize{9.000000}{10.800000}\selectfont DMC+HTB}%
\end{pgfscope}%
\begin{pgfscope}%
\pgfsetbuttcap%
\pgfsetroundjoin%
\pgfsetlinewidth{1.003750pt}%
\definecolor{currentstroke}{rgb}{1.000000,0.647059,0.000000}%
\pgfsetstrokecolor{currentstroke}%
\pgfsetdash{{3.700000pt}{1.600000pt}}{0.000000pt}%
\pgfpathmoveto{\pgfqpoint{4.396639in}{1.334353in}}%
\pgfpathlineto{\pgfqpoint{4.646639in}{1.334353in}}%
\pgfusepath{stroke}%
\end{pgfscope}%
\begin{pgfscope}%
\definecolor{textcolor}{rgb}{0.000000,0.000000,0.000000}%
\pgfsetstrokecolor{textcolor}%
\pgfsetfillcolor{textcolor}%
\pgftext[x=4.671639in,y=1.290603in,left,base]{\color{textcolor}\rmfamily\fontsize{9.000000}{10.800000}\selectfont DMC+LG (first)}%
\end{pgfscope}%
\begin{pgfscope}%
\pgfsetrectcap%
\pgfsetroundjoin%
\pgfsetlinewidth{1.003750pt}%
\definecolor{currentstroke}{rgb}{1.000000,0.647059,0.000000}%
\pgfsetstrokecolor{currentstroke}%
\pgfsetdash{}{0pt}%
\pgfpathmoveto{\pgfqpoint{4.396639in}{1.159384in}}%
\pgfpathlineto{\pgfqpoint{4.646639in}{1.159384in}}%
\pgfusepath{stroke}%
\end{pgfscope}%
\begin{pgfscope}%
\definecolor{textcolor}{rgb}{0.000000,0.000000,0.000000}%
\pgfsetstrokecolor{textcolor}%
\pgfsetfillcolor{textcolor}%
\pgftext[x=4.671639in,y=1.115634in,left,base]{\color{textcolor}\rmfamily\fontsize{9.000000}{10.800000}\selectfont DMC+LG (cost)}%
\end{pgfscope}%
\begin{pgfscope}%
\pgfsetbuttcap%
\pgfsetroundjoin%
\pgfsetlinewidth{1.003750pt}%
\definecolor{currentstroke}{rgb}{1.000000,0.647059,0.000000}%
\pgfsetstrokecolor{currentstroke}%
\pgfsetdash{{1.000000pt}{1.650000pt}}{0.000000pt}%
\pgfpathmoveto{\pgfqpoint{4.396639in}{0.984414in}}%
\pgfpathlineto{\pgfqpoint{4.646639in}{0.984414in}}%
\pgfusepath{stroke}%
\end{pgfscope}%
\begin{pgfscope}%
\definecolor{textcolor}{rgb}{0.000000,0.000000,0.000000}%
\pgfsetstrokecolor{textcolor}%
\pgfsetfillcolor{textcolor}%
\pgftext[x=4.671639in,y=0.940664in,left,base]{\color{textcolor}\rmfamily\fontsize{9.000000}{10.800000}\selectfont DMC+LG (best)}%
\end{pgfscope}%
\begin{pgfscope}%
\pgfsetbuttcap%
\pgfsetroundjoin%
\pgfsetlinewidth{1.003750pt}%
\definecolor{currentstroke}{rgb}{0.000000,0.000000,0.000000}%
\pgfsetstrokecolor{currentstroke}%
\pgfsetdash{{3.700000pt}{1.600000pt}}{0.000000pt}%
\pgfpathmoveto{\pgfqpoint{4.396639in}{0.815672in}}%
\pgfpathlineto{\pgfqpoint{4.646639in}{0.815672in}}%
\pgfusepath{stroke}%
\end{pgfscope}%
\begin{pgfscope}%
\definecolor{textcolor}{rgb}{0.000000,0.000000,0.000000}%
\pgfsetstrokecolor{textcolor}%
\pgfsetfillcolor{textcolor}%
\pgftext[x=4.671639in,y=0.771922in,left,base]{\color{textcolor}\rmfamily\fontsize{9.000000}{10.800000}\selectfont VBS*}%
\end{pgfscope}%
\begin{pgfscope}%
\pgfsetbuttcap%
\pgfsetroundjoin%
\pgfsetlinewidth{1.003750pt}%
\definecolor{currentstroke}{rgb}{0.000000,0.000000,0.000000}%
\pgfsetstrokecolor{currentstroke}%
\pgfsetdash{{1.000000pt}{1.650000pt}}{0.000000pt}%
\pgfpathmoveto{\pgfqpoint{4.396639in}{0.653872in}}%
\pgfpathlineto{\pgfqpoint{4.646639in}{0.653872in}}%
\pgfusepath{stroke}%
\end{pgfscope}%
\begin{pgfscope}%
\definecolor{textcolor}{rgb}{0.000000,0.000000,0.000000}%
\pgfsetstrokecolor{textcolor}%
\pgfsetfillcolor{textcolor}%
\pgftext[x=4.671639in,y=0.610122in,left,base]{\color{textcolor}\rmfamily\fontsize{9.000000}{10.800000}\selectfont VBS}%
\end{pgfscope}%
\end{pgfpicture}%
\makeatother%
\endgroup%

    \vspace*{-1cm}
	\caption{
	A cactus plot of the performance of various planners and executors for weighted model counting.
    Different strategies for stopping \Lg{} are considered.
    ``(first)'' indicates that \Lg{} was stopped after it produced the first project-join tree.
    ``(cost)'' indicates that the executor attempted to predict the cost of computing each project-join tree.
    ``(best)'' indicates a simulated case where the executor has perfect information on all project-join trees generated by \Lg{} and valuates the tree with the shortest total time.
    \tool{VBS*} is the virtual best solver of \Dmc{}+\Htb{} and \Dmc{}+\Lg{} (cost).
	\tool{VBS} is the virtual best solver of \Dmc{}+\Htb{}, \Dmc{}+\Lg{} (cost), \Tensor{}+\Htb{}, and \Tensor{}+\Lg{} (cost).}
	\label{fig:execution}
\end{figure}

Next, we compare ADDs (\Dmc) and tensors (\Tensor) as a data structure for valuating project-join trees.
To do this, we ran both \Dmc{} and \Tensor{} on all project-join trees generated by \Htb{} and \Lg{} (with their representative configurations) in Experiment 1, each with a 100-second timeout.
The total times recorded include both the planning phase and the execution phase.

Since \Lg{} is an anytime tool, it may have produced more than one project-join tree of each benchmark in Experiment 1.
We follow Chapter \ref{ch:tensors} by allowing \Tensor{} and \Dmc{} to stop \Lg{} at a time proportional to the estimated cost to valuate the best-seen project-join tree.
The constant of proportionality is chosen to minimize the PAR-2 score (\ie, the sum of the running times of all completed benchmarks plus twice the timeout for every uncompleted benchmark) of each executor.
\Tensor{} and \Dmc{} use different methods for estimating cost.
Tensors are a dense data structure, so the number of floating-point operations to valuate a project-join tree can be computed exactly as in Chapter \ref{ch:tensors}.
We use this as the cost estimator for \Tensor{}.
ADDs are a sparse data structure, and estimating the amount of sparsity is difficult.
It is thus hard to find a good cost estimator for \Dmc{}.
As a first step, we use $2^w$ as an estimate of the cost for \Dmc{} to valuate a project-join tree of width $w$.

We present results from this experiment in Figure \ref{fig:execution}.
We observe that the benefit of \Lg{} over \Htb{} seen in Experiment 1 is maintained once the full weighted model count is computed.
We also observe that \Dmc{} is able to solve significantly more benchmarks than \Tensor{}, even when using identical project-join trees.
We attribute this difference to the sparsity of ADDs over tensors.
Nevertheless, we observe that \Tensor{} still outperforms \Dmc{} on some benchmarks; compare \tool{VBS*} (which excludes \Tensor{}) with \tool{VBS} (which includes \Tensor{}).

Moreover, we observe significant differences based on the strategy used to stop \Lg{}.
The executor \Tensor{} performs significantly better when cost estimation is used than when only the first project-join tree of \Lg{} is used.
In fact, the performance of \Tensor{} is almost as good as the hypothetical performance if \Tensor{} is able to predict the planning and valuation times of all trees produced by \Lg{}.
On the other hand, \Dmc{} is not significantly improved by cost estimation.
It would be interesting in the future to find better cost estimators for \Dmc{}.

%%%%%%%%%%%%%%%%%%%%%%%%%%%%%%%%%%%%%%%%%%%%%%%%%%%%%%%%%%%%%%%%%%%%%%%%%%%%%%%%

\subsection{Experiment 3: Comparing Exact Weighted Model Counters}
\label{sec_experiments_wmc}

\begin{figure}[t]
	\centering
	%% Creator: Matplotlib, PGF backend
%%
%% To include the figure in your LaTeX document, write
%%   \input{<filename>.pgf}
%%
%% Make sure the required packages are loaded in your preamble
%%   \usepackage{pgf}
%%
%% and, on pdftex
%%   \usepackage[utf8]{inputenc}\DeclareUnicodeCharacter{2212}{-}
%%
%% or, on luatex and xetex
%%   \usepackage{unicode-math}
%%
%% Figures using additional raster images can only be included by \input if
%% they are in the same directory as the main LaTeX file. For loading figures
%% from other directories you can use the `import` package
%%   \usepackage{import}
%%
%% and then include the figures with
%%   \import{<path to file>}{<filename>.pgf}
%%
%% Matplotlib used the following preamble
%%   \usepackage[utf8x]{inputenc}
%%   \usepackage[T1]{fontenc}
%%
\begingroup%
\makeatletter%
\begin{pgfpicture}%
\pgfpathrectangle{\pgfpointorigin}{\pgfqpoint{6.000000in}{3.100000in}}%
\pgfusepath{use as bounding box, clip}%
\begin{pgfscope}%
\pgfsetbuttcap%
\pgfsetmiterjoin%
\definecolor{currentfill}{rgb}{1.000000,1.000000,1.000000}%
\pgfsetfillcolor{currentfill}%
\pgfsetlinewidth{0.000000pt}%
\definecolor{currentstroke}{rgb}{1.000000,1.000000,1.000000}%
\pgfsetstrokecolor{currentstroke}%
\pgfsetdash{}{0pt}%
\pgfpathmoveto{\pgfqpoint{0.000000in}{0.000000in}}%
\pgfpathlineto{\pgfqpoint{6.000000in}{0.000000in}}%
\pgfpathlineto{\pgfqpoint{6.000000in}{3.100000in}}%
\pgfpathlineto{\pgfqpoint{0.000000in}{3.100000in}}%
\pgfpathclose%
\pgfusepath{fill}%
\end{pgfscope}%
\begin{pgfscope}%
\pgfsetbuttcap%
\pgfsetmiterjoin%
\definecolor{currentfill}{rgb}{1.000000,1.000000,1.000000}%
\pgfsetfillcolor{currentfill}%
\pgfsetlinewidth{0.000000pt}%
\definecolor{currentstroke}{rgb}{0.000000,0.000000,0.000000}%
\pgfsetstrokecolor{currentstroke}%
\pgfsetstrokeopacity{0.000000}%
\pgfsetdash{}{0pt}%
\pgfpathmoveto{\pgfqpoint{0.708220in}{0.535823in}}%
\pgfpathlineto{\pgfqpoint{5.721529in}{0.535823in}}%
\pgfpathlineto{\pgfqpoint{5.721529in}{2.905275in}}%
\pgfpathlineto{\pgfqpoint{0.708220in}{2.905275in}}%
\pgfpathclose%
\pgfusepath{fill}%
\end{pgfscope}%
\begin{pgfscope}%
\pgfsetbuttcap%
\pgfsetroundjoin%
\definecolor{currentfill}{rgb}{0.000000,0.000000,0.000000}%
\pgfsetfillcolor{currentfill}%
\pgfsetlinewidth{0.803000pt}%
\definecolor{currentstroke}{rgb}{0.000000,0.000000,0.000000}%
\pgfsetstrokecolor{currentstroke}%
\pgfsetdash{}{0pt}%
\pgfsys@defobject{currentmarker}{\pgfqpoint{0.000000in}{-0.048611in}}{\pgfqpoint{0.000000in}{0.000000in}}{%
\pgfpathmoveto{\pgfqpoint{0.000000in}{0.000000in}}%
\pgfpathlineto{\pgfqpoint{0.000000in}{-0.048611in}}%
\pgfusepath{stroke,fill}%
}%
\begin{pgfscope}%
\pgfsys@transformshift{0.708220in}{0.535823in}%
\pgfsys@useobject{currentmarker}{}%
\end{pgfscope}%
\end{pgfscope}%
\begin{pgfscope}%
\definecolor{textcolor}{rgb}{0.000000,0.000000,0.000000}%
\pgfsetstrokecolor{textcolor}%
\pgfsetfillcolor{textcolor}%
\pgftext[x=0.708220in,y=0.438600in,,top]{\color{textcolor}\rmfamily\fontsize{9.000000}{10.800000}\selectfont \(\displaystyle {0}\)}%
\end{pgfscope}%
\begin{pgfscope}%
\pgfsetbuttcap%
\pgfsetroundjoin%
\definecolor{currentfill}{rgb}{0.000000,0.000000,0.000000}%
\pgfsetfillcolor{currentfill}%
\pgfsetlinewidth{0.803000pt}%
\definecolor{currentstroke}{rgb}{0.000000,0.000000,0.000000}%
\pgfsetstrokecolor{currentstroke}%
\pgfsetdash{}{0pt}%
\pgfsys@defobject{currentmarker}{\pgfqpoint{0.000000in}{-0.048611in}}{\pgfqpoint{0.000000in}{0.000000in}}{%
\pgfpathmoveto{\pgfqpoint{0.000000in}{0.000000in}}%
\pgfpathlineto{\pgfqpoint{0.000000in}{-0.048611in}}%
\pgfusepath{stroke,fill}%
}%
\begin{pgfscope}%
\pgfsys@transformshift{1.334883in}{0.535823in}%
\pgfsys@useobject{currentmarker}{}%
\end{pgfscope}%
\end{pgfscope}%
\begin{pgfscope}%
\definecolor{textcolor}{rgb}{0.000000,0.000000,0.000000}%
\pgfsetstrokecolor{textcolor}%
\pgfsetfillcolor{textcolor}%
\pgftext[x=1.334883in,y=0.438600in,,top]{\color{textcolor}\rmfamily\fontsize{9.000000}{10.800000}\selectfont \(\displaystyle {250}\)}%
\end{pgfscope}%
\begin{pgfscope}%
\pgfsetbuttcap%
\pgfsetroundjoin%
\definecolor{currentfill}{rgb}{0.000000,0.000000,0.000000}%
\pgfsetfillcolor{currentfill}%
\pgfsetlinewidth{0.803000pt}%
\definecolor{currentstroke}{rgb}{0.000000,0.000000,0.000000}%
\pgfsetstrokecolor{currentstroke}%
\pgfsetdash{}{0pt}%
\pgfsys@defobject{currentmarker}{\pgfqpoint{0.000000in}{-0.048611in}}{\pgfqpoint{0.000000in}{0.000000in}}{%
\pgfpathmoveto{\pgfqpoint{0.000000in}{0.000000in}}%
\pgfpathlineto{\pgfqpoint{0.000000in}{-0.048611in}}%
\pgfusepath{stroke,fill}%
}%
\begin{pgfscope}%
\pgfsys@transformshift{1.961547in}{0.535823in}%
\pgfsys@useobject{currentmarker}{}%
\end{pgfscope}%
\end{pgfscope}%
\begin{pgfscope}%
\definecolor{textcolor}{rgb}{0.000000,0.000000,0.000000}%
\pgfsetstrokecolor{textcolor}%
\pgfsetfillcolor{textcolor}%
\pgftext[x=1.961547in,y=0.438600in,,top]{\color{textcolor}\rmfamily\fontsize{9.000000}{10.800000}\selectfont \(\displaystyle {500}\)}%
\end{pgfscope}%
\begin{pgfscope}%
\pgfsetbuttcap%
\pgfsetroundjoin%
\definecolor{currentfill}{rgb}{0.000000,0.000000,0.000000}%
\pgfsetfillcolor{currentfill}%
\pgfsetlinewidth{0.803000pt}%
\definecolor{currentstroke}{rgb}{0.000000,0.000000,0.000000}%
\pgfsetstrokecolor{currentstroke}%
\pgfsetdash{}{0pt}%
\pgfsys@defobject{currentmarker}{\pgfqpoint{0.000000in}{-0.048611in}}{\pgfqpoint{0.000000in}{0.000000in}}{%
\pgfpathmoveto{\pgfqpoint{0.000000in}{0.000000in}}%
\pgfpathlineto{\pgfqpoint{0.000000in}{-0.048611in}}%
\pgfusepath{stroke,fill}%
}%
\begin{pgfscope}%
\pgfsys@transformshift{2.588211in}{0.535823in}%
\pgfsys@useobject{currentmarker}{}%
\end{pgfscope}%
\end{pgfscope}%
\begin{pgfscope}%
\definecolor{textcolor}{rgb}{0.000000,0.000000,0.000000}%
\pgfsetstrokecolor{textcolor}%
\pgfsetfillcolor{textcolor}%
\pgftext[x=2.588211in,y=0.438600in,,top]{\color{textcolor}\rmfamily\fontsize{9.000000}{10.800000}\selectfont \(\displaystyle {750}\)}%
\end{pgfscope}%
\begin{pgfscope}%
\pgfsetbuttcap%
\pgfsetroundjoin%
\definecolor{currentfill}{rgb}{0.000000,0.000000,0.000000}%
\pgfsetfillcolor{currentfill}%
\pgfsetlinewidth{0.803000pt}%
\definecolor{currentstroke}{rgb}{0.000000,0.000000,0.000000}%
\pgfsetstrokecolor{currentstroke}%
\pgfsetdash{}{0pt}%
\pgfsys@defobject{currentmarker}{\pgfqpoint{0.000000in}{-0.048611in}}{\pgfqpoint{0.000000in}{0.000000in}}{%
\pgfpathmoveto{\pgfqpoint{0.000000in}{0.000000in}}%
\pgfpathlineto{\pgfqpoint{0.000000in}{-0.048611in}}%
\pgfusepath{stroke,fill}%
}%
\begin{pgfscope}%
\pgfsys@transformshift{3.214874in}{0.535823in}%
\pgfsys@useobject{currentmarker}{}%
\end{pgfscope}%
\end{pgfscope}%
\begin{pgfscope}%
\definecolor{textcolor}{rgb}{0.000000,0.000000,0.000000}%
\pgfsetstrokecolor{textcolor}%
\pgfsetfillcolor{textcolor}%
\pgftext[x=3.214874in,y=0.438600in,,top]{\color{textcolor}\rmfamily\fontsize{9.000000}{10.800000}\selectfont \(\displaystyle {1000}\)}%
\end{pgfscope}%
\begin{pgfscope}%
\pgfsetbuttcap%
\pgfsetroundjoin%
\definecolor{currentfill}{rgb}{0.000000,0.000000,0.000000}%
\pgfsetfillcolor{currentfill}%
\pgfsetlinewidth{0.803000pt}%
\definecolor{currentstroke}{rgb}{0.000000,0.000000,0.000000}%
\pgfsetstrokecolor{currentstroke}%
\pgfsetdash{}{0pt}%
\pgfsys@defobject{currentmarker}{\pgfqpoint{0.000000in}{-0.048611in}}{\pgfqpoint{0.000000in}{0.000000in}}{%
\pgfpathmoveto{\pgfqpoint{0.000000in}{0.000000in}}%
\pgfpathlineto{\pgfqpoint{0.000000in}{-0.048611in}}%
\pgfusepath{stroke,fill}%
}%
\begin{pgfscope}%
\pgfsys@transformshift{3.841538in}{0.535823in}%
\pgfsys@useobject{currentmarker}{}%
\end{pgfscope}%
\end{pgfscope}%
\begin{pgfscope}%
\definecolor{textcolor}{rgb}{0.000000,0.000000,0.000000}%
\pgfsetstrokecolor{textcolor}%
\pgfsetfillcolor{textcolor}%
\pgftext[x=3.841538in,y=0.438600in,,top]{\color{textcolor}\rmfamily\fontsize{9.000000}{10.800000}\selectfont \(\displaystyle {1250}\)}%
\end{pgfscope}%
\begin{pgfscope}%
\pgfsetbuttcap%
\pgfsetroundjoin%
\definecolor{currentfill}{rgb}{0.000000,0.000000,0.000000}%
\pgfsetfillcolor{currentfill}%
\pgfsetlinewidth{0.803000pt}%
\definecolor{currentstroke}{rgb}{0.000000,0.000000,0.000000}%
\pgfsetstrokecolor{currentstroke}%
\pgfsetdash{}{0pt}%
\pgfsys@defobject{currentmarker}{\pgfqpoint{0.000000in}{-0.048611in}}{\pgfqpoint{0.000000in}{0.000000in}}{%
\pgfpathmoveto{\pgfqpoint{0.000000in}{0.000000in}}%
\pgfpathlineto{\pgfqpoint{0.000000in}{-0.048611in}}%
\pgfusepath{stroke,fill}%
}%
\begin{pgfscope}%
\pgfsys@transformshift{4.468201in}{0.535823in}%
\pgfsys@useobject{currentmarker}{}%
\end{pgfscope}%
\end{pgfscope}%
\begin{pgfscope}%
\definecolor{textcolor}{rgb}{0.000000,0.000000,0.000000}%
\pgfsetstrokecolor{textcolor}%
\pgfsetfillcolor{textcolor}%
\pgftext[x=4.468201in,y=0.438600in,,top]{\color{textcolor}\rmfamily\fontsize{9.000000}{10.800000}\selectfont \(\displaystyle {1500}\)}%
\end{pgfscope}%
\begin{pgfscope}%
\pgfsetbuttcap%
\pgfsetroundjoin%
\definecolor{currentfill}{rgb}{0.000000,0.000000,0.000000}%
\pgfsetfillcolor{currentfill}%
\pgfsetlinewidth{0.803000pt}%
\definecolor{currentstroke}{rgb}{0.000000,0.000000,0.000000}%
\pgfsetstrokecolor{currentstroke}%
\pgfsetdash{}{0pt}%
\pgfsys@defobject{currentmarker}{\pgfqpoint{0.000000in}{-0.048611in}}{\pgfqpoint{0.000000in}{0.000000in}}{%
\pgfpathmoveto{\pgfqpoint{0.000000in}{0.000000in}}%
\pgfpathlineto{\pgfqpoint{0.000000in}{-0.048611in}}%
\pgfusepath{stroke,fill}%
}%
\begin{pgfscope}%
\pgfsys@transformshift{5.094865in}{0.535823in}%
\pgfsys@useobject{currentmarker}{}%
\end{pgfscope}%
\end{pgfscope}%
\begin{pgfscope}%
\definecolor{textcolor}{rgb}{0.000000,0.000000,0.000000}%
\pgfsetstrokecolor{textcolor}%
\pgfsetfillcolor{textcolor}%
\pgftext[x=5.094865in,y=0.438600in,,top]{\color{textcolor}\rmfamily\fontsize{9.000000}{10.800000}\selectfont \(\displaystyle {1750}\)}%
\end{pgfscope}%
\begin{pgfscope}%
\pgfsetbuttcap%
\pgfsetroundjoin%
\definecolor{currentfill}{rgb}{0.000000,0.000000,0.000000}%
\pgfsetfillcolor{currentfill}%
\pgfsetlinewidth{0.803000pt}%
\definecolor{currentstroke}{rgb}{0.000000,0.000000,0.000000}%
\pgfsetstrokecolor{currentstroke}%
\pgfsetdash{}{0pt}%
\pgfsys@defobject{currentmarker}{\pgfqpoint{0.000000in}{-0.048611in}}{\pgfqpoint{0.000000in}{0.000000in}}{%
\pgfpathmoveto{\pgfqpoint{0.000000in}{0.000000in}}%
\pgfpathlineto{\pgfqpoint{0.000000in}{-0.048611in}}%
\pgfusepath{stroke,fill}%
}%
\begin{pgfscope}%
\pgfsys@transformshift{5.721529in}{0.535823in}%
\pgfsys@useobject{currentmarker}{}%
\end{pgfscope}%
\end{pgfscope}%
\begin{pgfscope}%
\definecolor{textcolor}{rgb}{0.000000,0.000000,0.000000}%
\pgfsetstrokecolor{textcolor}%
\pgfsetfillcolor{textcolor}%
\pgftext[x=5.721529in,y=0.438600in,,top]{\color{textcolor}\rmfamily\fontsize{9.000000}{10.800000}\selectfont \(\displaystyle {2000}\)}%
\end{pgfscope}%
\begin{pgfscope}%
\definecolor{textcolor}{rgb}{0.000000,0.000000,0.000000}%
\pgfsetstrokecolor{textcolor}%
\pgfsetfillcolor{textcolor}%
\pgftext[x=3.214874in,y=0.272655in,,top]{\color{textcolor}\rmfamily\fontsize{10.000000}{12.000000}\selectfont Number of benchmarks solved}%
\end{pgfscope}%
\begin{pgfscope}%
\pgfsetbuttcap%
\pgfsetroundjoin%
\definecolor{currentfill}{rgb}{0.000000,0.000000,0.000000}%
\pgfsetfillcolor{currentfill}%
\pgfsetlinewidth{0.803000pt}%
\definecolor{currentstroke}{rgb}{0.000000,0.000000,0.000000}%
\pgfsetstrokecolor{currentstroke}%
\pgfsetdash{}{0pt}%
\pgfsys@defobject{currentmarker}{\pgfqpoint{-0.048611in}{0.000000in}}{\pgfqpoint{-0.000000in}{0.000000in}}{%
\pgfpathmoveto{\pgfqpoint{-0.000000in}{0.000000in}}%
\pgfpathlineto{\pgfqpoint{-0.048611in}{0.000000in}}%
\pgfusepath{stroke,fill}%
}%
\begin{pgfscope}%
\pgfsys@transformshift{0.708220in}{0.535823in}%
\pgfsys@useobject{currentmarker}{}%
\end{pgfscope}%
\end{pgfscope}%
\begin{pgfscope}%
\definecolor{textcolor}{rgb}{0.000000,0.000000,0.000000}%
\pgfsetstrokecolor{textcolor}%
\pgfsetfillcolor{textcolor}%
\pgftext[x=0.344411in, y=0.491098in, left, base]{\color{textcolor}\rmfamily\fontsize{9.000000}{10.800000}\selectfont \(\displaystyle {10^{-3}}\)}%
\end{pgfscope}%
\begin{pgfscope}%
\pgfsetbuttcap%
\pgfsetroundjoin%
\definecolor{currentfill}{rgb}{0.000000,0.000000,0.000000}%
\pgfsetfillcolor{currentfill}%
\pgfsetlinewidth{0.803000pt}%
\definecolor{currentstroke}{rgb}{0.000000,0.000000,0.000000}%
\pgfsetstrokecolor{currentstroke}%
\pgfsetdash{}{0pt}%
\pgfsys@defobject{currentmarker}{\pgfqpoint{-0.048611in}{0.000000in}}{\pgfqpoint{-0.000000in}{0.000000in}}{%
\pgfpathmoveto{\pgfqpoint{-0.000000in}{0.000000in}}%
\pgfpathlineto{\pgfqpoint{-0.048611in}{0.000000in}}%
\pgfusepath{stroke,fill}%
}%
\begin{pgfscope}%
\pgfsys@transformshift{0.708220in}{0.930731in}%
\pgfsys@useobject{currentmarker}{}%
\end{pgfscope}%
\end{pgfscope}%
\begin{pgfscope}%
\definecolor{textcolor}{rgb}{0.000000,0.000000,0.000000}%
\pgfsetstrokecolor{textcolor}%
\pgfsetfillcolor{textcolor}%
\pgftext[x=0.344411in, y=0.886007in, left, base]{\color{textcolor}\rmfamily\fontsize{9.000000}{10.800000}\selectfont \(\displaystyle {10^{-2}}\)}%
\end{pgfscope}%
\begin{pgfscope}%
\pgfsetbuttcap%
\pgfsetroundjoin%
\definecolor{currentfill}{rgb}{0.000000,0.000000,0.000000}%
\pgfsetfillcolor{currentfill}%
\pgfsetlinewidth{0.803000pt}%
\definecolor{currentstroke}{rgb}{0.000000,0.000000,0.000000}%
\pgfsetstrokecolor{currentstroke}%
\pgfsetdash{}{0pt}%
\pgfsys@defobject{currentmarker}{\pgfqpoint{-0.048611in}{0.000000in}}{\pgfqpoint{-0.000000in}{0.000000in}}{%
\pgfpathmoveto{\pgfqpoint{-0.000000in}{0.000000in}}%
\pgfpathlineto{\pgfqpoint{-0.048611in}{0.000000in}}%
\pgfusepath{stroke,fill}%
}%
\begin{pgfscope}%
\pgfsys@transformshift{0.708220in}{1.325640in}%
\pgfsys@useobject{currentmarker}{}%
\end{pgfscope}%
\end{pgfscope}%
\begin{pgfscope}%
\definecolor{textcolor}{rgb}{0.000000,0.000000,0.000000}%
\pgfsetstrokecolor{textcolor}%
\pgfsetfillcolor{textcolor}%
\pgftext[x=0.344411in, y=1.280915in, left, base]{\color{textcolor}\rmfamily\fontsize{9.000000}{10.800000}\selectfont \(\displaystyle {10^{-1}}\)}%
\end{pgfscope}%
\begin{pgfscope}%
\pgfsetbuttcap%
\pgfsetroundjoin%
\definecolor{currentfill}{rgb}{0.000000,0.000000,0.000000}%
\pgfsetfillcolor{currentfill}%
\pgfsetlinewidth{0.803000pt}%
\definecolor{currentstroke}{rgb}{0.000000,0.000000,0.000000}%
\pgfsetstrokecolor{currentstroke}%
\pgfsetdash{}{0pt}%
\pgfsys@defobject{currentmarker}{\pgfqpoint{-0.048611in}{0.000000in}}{\pgfqpoint{-0.000000in}{0.000000in}}{%
\pgfpathmoveto{\pgfqpoint{-0.000000in}{0.000000in}}%
\pgfpathlineto{\pgfqpoint{-0.048611in}{0.000000in}}%
\pgfusepath{stroke,fill}%
}%
\begin{pgfscope}%
\pgfsys@transformshift{0.708220in}{1.720549in}%
\pgfsys@useobject{currentmarker}{}%
\end{pgfscope}%
\end{pgfscope}%
\begin{pgfscope}%
\definecolor{textcolor}{rgb}{0.000000,0.000000,0.000000}%
\pgfsetstrokecolor{textcolor}%
\pgfsetfillcolor{textcolor}%
\pgftext[x=0.424657in, y=1.675824in, left, base]{\color{textcolor}\rmfamily\fontsize{9.000000}{10.800000}\selectfont \(\displaystyle {10^{0}}\)}%
\end{pgfscope}%
\begin{pgfscope}%
\pgfsetbuttcap%
\pgfsetroundjoin%
\definecolor{currentfill}{rgb}{0.000000,0.000000,0.000000}%
\pgfsetfillcolor{currentfill}%
\pgfsetlinewidth{0.803000pt}%
\definecolor{currentstroke}{rgb}{0.000000,0.000000,0.000000}%
\pgfsetstrokecolor{currentstroke}%
\pgfsetdash{}{0pt}%
\pgfsys@defobject{currentmarker}{\pgfqpoint{-0.048611in}{0.000000in}}{\pgfqpoint{-0.000000in}{0.000000in}}{%
\pgfpathmoveto{\pgfqpoint{-0.000000in}{0.000000in}}%
\pgfpathlineto{\pgfqpoint{-0.048611in}{0.000000in}}%
\pgfusepath{stroke,fill}%
}%
\begin{pgfscope}%
\pgfsys@transformshift{0.708220in}{2.115458in}%
\pgfsys@useobject{currentmarker}{}%
\end{pgfscope}%
\end{pgfscope}%
\begin{pgfscope}%
\definecolor{textcolor}{rgb}{0.000000,0.000000,0.000000}%
\pgfsetstrokecolor{textcolor}%
\pgfsetfillcolor{textcolor}%
\pgftext[x=0.424657in, y=2.070733in, left, base]{\color{textcolor}\rmfamily\fontsize{9.000000}{10.800000}\selectfont \(\displaystyle {10^{1}}\)}%
\end{pgfscope}%
\begin{pgfscope}%
\pgfsetbuttcap%
\pgfsetroundjoin%
\definecolor{currentfill}{rgb}{0.000000,0.000000,0.000000}%
\pgfsetfillcolor{currentfill}%
\pgfsetlinewidth{0.803000pt}%
\definecolor{currentstroke}{rgb}{0.000000,0.000000,0.000000}%
\pgfsetstrokecolor{currentstroke}%
\pgfsetdash{}{0pt}%
\pgfsys@defobject{currentmarker}{\pgfqpoint{-0.048611in}{0.000000in}}{\pgfqpoint{-0.000000in}{0.000000in}}{%
\pgfpathmoveto{\pgfqpoint{-0.000000in}{0.000000in}}%
\pgfpathlineto{\pgfqpoint{-0.048611in}{0.000000in}}%
\pgfusepath{stroke,fill}%
}%
\begin{pgfscope}%
\pgfsys@transformshift{0.708220in}{2.510366in}%
\pgfsys@useobject{currentmarker}{}%
\end{pgfscope}%
\end{pgfscope}%
\begin{pgfscope}%
\definecolor{textcolor}{rgb}{0.000000,0.000000,0.000000}%
\pgfsetstrokecolor{textcolor}%
\pgfsetfillcolor{textcolor}%
\pgftext[x=0.424657in, y=2.465642in, left, base]{\color{textcolor}\rmfamily\fontsize{9.000000}{10.800000}\selectfont \(\displaystyle {10^{2}}\)}%
\end{pgfscope}%
\begin{pgfscope}%
\pgfsetbuttcap%
\pgfsetroundjoin%
\definecolor{currentfill}{rgb}{0.000000,0.000000,0.000000}%
\pgfsetfillcolor{currentfill}%
\pgfsetlinewidth{0.803000pt}%
\definecolor{currentstroke}{rgb}{0.000000,0.000000,0.000000}%
\pgfsetstrokecolor{currentstroke}%
\pgfsetdash{}{0pt}%
\pgfsys@defobject{currentmarker}{\pgfqpoint{-0.048611in}{0.000000in}}{\pgfqpoint{-0.000000in}{0.000000in}}{%
\pgfpathmoveto{\pgfqpoint{-0.000000in}{0.000000in}}%
\pgfpathlineto{\pgfqpoint{-0.048611in}{0.000000in}}%
\pgfusepath{stroke,fill}%
}%
\begin{pgfscope}%
\pgfsys@transformshift{0.708220in}{2.905275in}%
\pgfsys@useobject{currentmarker}{}%
\end{pgfscope}%
\end{pgfscope}%
\begin{pgfscope}%
\definecolor{textcolor}{rgb}{0.000000,0.000000,0.000000}%
\pgfsetstrokecolor{textcolor}%
\pgfsetfillcolor{textcolor}%
\pgftext[x=0.424657in, y=2.860550in, left, base]{\color{textcolor}\rmfamily\fontsize{9.000000}{10.800000}\selectfont \(\displaystyle {10^{3}}\)}%
\end{pgfscope}%
\begin{pgfscope}%
\pgfsetbuttcap%
\pgfsetroundjoin%
\definecolor{currentfill}{rgb}{0.000000,0.000000,0.000000}%
\pgfsetfillcolor{currentfill}%
\pgfsetlinewidth{0.602250pt}%
\definecolor{currentstroke}{rgb}{0.000000,0.000000,0.000000}%
\pgfsetstrokecolor{currentstroke}%
\pgfsetdash{}{0pt}%
\pgfsys@defobject{currentmarker}{\pgfqpoint{-0.027778in}{0.000000in}}{\pgfqpoint{-0.000000in}{0.000000in}}{%
\pgfpathmoveto{\pgfqpoint{-0.000000in}{0.000000in}}%
\pgfpathlineto{\pgfqpoint{-0.027778in}{0.000000in}}%
\pgfusepath{stroke,fill}%
}%
\begin{pgfscope}%
\pgfsys@transformshift{0.708220in}{0.654702in}%
\pgfsys@useobject{currentmarker}{}%
\end{pgfscope}%
\end{pgfscope}%
\begin{pgfscope}%
\pgfsetbuttcap%
\pgfsetroundjoin%
\definecolor{currentfill}{rgb}{0.000000,0.000000,0.000000}%
\pgfsetfillcolor{currentfill}%
\pgfsetlinewidth{0.602250pt}%
\definecolor{currentstroke}{rgb}{0.000000,0.000000,0.000000}%
\pgfsetstrokecolor{currentstroke}%
\pgfsetdash{}{0pt}%
\pgfsys@defobject{currentmarker}{\pgfqpoint{-0.027778in}{0.000000in}}{\pgfqpoint{-0.000000in}{0.000000in}}{%
\pgfpathmoveto{\pgfqpoint{-0.000000in}{0.000000in}}%
\pgfpathlineto{\pgfqpoint{-0.027778in}{0.000000in}}%
\pgfusepath{stroke,fill}%
}%
\begin{pgfscope}%
\pgfsys@transformshift{0.708220in}{0.724242in}%
\pgfsys@useobject{currentmarker}{}%
\end{pgfscope}%
\end{pgfscope}%
\begin{pgfscope}%
\pgfsetbuttcap%
\pgfsetroundjoin%
\definecolor{currentfill}{rgb}{0.000000,0.000000,0.000000}%
\pgfsetfillcolor{currentfill}%
\pgfsetlinewidth{0.602250pt}%
\definecolor{currentstroke}{rgb}{0.000000,0.000000,0.000000}%
\pgfsetstrokecolor{currentstroke}%
\pgfsetdash{}{0pt}%
\pgfsys@defobject{currentmarker}{\pgfqpoint{-0.027778in}{0.000000in}}{\pgfqpoint{-0.000000in}{0.000000in}}{%
\pgfpathmoveto{\pgfqpoint{-0.000000in}{0.000000in}}%
\pgfpathlineto{\pgfqpoint{-0.027778in}{0.000000in}}%
\pgfusepath{stroke,fill}%
}%
\begin{pgfscope}%
\pgfsys@transformshift{0.708220in}{0.773581in}%
\pgfsys@useobject{currentmarker}{}%
\end{pgfscope}%
\end{pgfscope}%
\begin{pgfscope}%
\pgfsetbuttcap%
\pgfsetroundjoin%
\definecolor{currentfill}{rgb}{0.000000,0.000000,0.000000}%
\pgfsetfillcolor{currentfill}%
\pgfsetlinewidth{0.602250pt}%
\definecolor{currentstroke}{rgb}{0.000000,0.000000,0.000000}%
\pgfsetstrokecolor{currentstroke}%
\pgfsetdash{}{0pt}%
\pgfsys@defobject{currentmarker}{\pgfqpoint{-0.027778in}{0.000000in}}{\pgfqpoint{-0.000000in}{0.000000in}}{%
\pgfpathmoveto{\pgfqpoint{-0.000000in}{0.000000in}}%
\pgfpathlineto{\pgfqpoint{-0.027778in}{0.000000in}}%
\pgfusepath{stroke,fill}%
}%
\begin{pgfscope}%
\pgfsys@transformshift{0.708220in}{0.811852in}%
\pgfsys@useobject{currentmarker}{}%
\end{pgfscope}%
\end{pgfscope}%
\begin{pgfscope}%
\pgfsetbuttcap%
\pgfsetroundjoin%
\definecolor{currentfill}{rgb}{0.000000,0.000000,0.000000}%
\pgfsetfillcolor{currentfill}%
\pgfsetlinewidth{0.602250pt}%
\definecolor{currentstroke}{rgb}{0.000000,0.000000,0.000000}%
\pgfsetstrokecolor{currentstroke}%
\pgfsetdash{}{0pt}%
\pgfsys@defobject{currentmarker}{\pgfqpoint{-0.027778in}{0.000000in}}{\pgfqpoint{-0.000000in}{0.000000in}}{%
\pgfpathmoveto{\pgfqpoint{-0.000000in}{0.000000in}}%
\pgfpathlineto{\pgfqpoint{-0.027778in}{0.000000in}}%
\pgfusepath{stroke,fill}%
}%
\begin{pgfscope}%
\pgfsys@transformshift{0.708220in}{0.843121in}%
\pgfsys@useobject{currentmarker}{}%
\end{pgfscope}%
\end{pgfscope}%
\begin{pgfscope}%
\pgfsetbuttcap%
\pgfsetroundjoin%
\definecolor{currentfill}{rgb}{0.000000,0.000000,0.000000}%
\pgfsetfillcolor{currentfill}%
\pgfsetlinewidth{0.602250pt}%
\definecolor{currentstroke}{rgb}{0.000000,0.000000,0.000000}%
\pgfsetstrokecolor{currentstroke}%
\pgfsetdash{}{0pt}%
\pgfsys@defobject{currentmarker}{\pgfqpoint{-0.027778in}{0.000000in}}{\pgfqpoint{-0.000000in}{0.000000in}}{%
\pgfpathmoveto{\pgfqpoint{-0.000000in}{0.000000in}}%
\pgfpathlineto{\pgfqpoint{-0.027778in}{0.000000in}}%
\pgfusepath{stroke,fill}%
}%
\begin{pgfscope}%
\pgfsys@transformshift{0.708220in}{0.869559in}%
\pgfsys@useobject{currentmarker}{}%
\end{pgfscope}%
\end{pgfscope}%
\begin{pgfscope}%
\pgfsetbuttcap%
\pgfsetroundjoin%
\definecolor{currentfill}{rgb}{0.000000,0.000000,0.000000}%
\pgfsetfillcolor{currentfill}%
\pgfsetlinewidth{0.602250pt}%
\definecolor{currentstroke}{rgb}{0.000000,0.000000,0.000000}%
\pgfsetstrokecolor{currentstroke}%
\pgfsetdash{}{0pt}%
\pgfsys@defobject{currentmarker}{\pgfqpoint{-0.027778in}{0.000000in}}{\pgfqpoint{-0.000000in}{0.000000in}}{%
\pgfpathmoveto{\pgfqpoint{-0.000000in}{0.000000in}}%
\pgfpathlineto{\pgfqpoint{-0.027778in}{0.000000in}}%
\pgfusepath{stroke,fill}%
}%
\begin{pgfscope}%
\pgfsys@transformshift{0.708220in}{0.892461in}%
\pgfsys@useobject{currentmarker}{}%
\end{pgfscope}%
\end{pgfscope}%
\begin{pgfscope}%
\pgfsetbuttcap%
\pgfsetroundjoin%
\definecolor{currentfill}{rgb}{0.000000,0.000000,0.000000}%
\pgfsetfillcolor{currentfill}%
\pgfsetlinewidth{0.602250pt}%
\definecolor{currentstroke}{rgb}{0.000000,0.000000,0.000000}%
\pgfsetstrokecolor{currentstroke}%
\pgfsetdash{}{0pt}%
\pgfsys@defobject{currentmarker}{\pgfqpoint{-0.027778in}{0.000000in}}{\pgfqpoint{-0.000000in}{0.000000in}}{%
\pgfpathmoveto{\pgfqpoint{-0.000000in}{0.000000in}}%
\pgfpathlineto{\pgfqpoint{-0.027778in}{0.000000in}}%
\pgfusepath{stroke,fill}%
}%
\begin{pgfscope}%
\pgfsys@transformshift{0.708220in}{0.912661in}%
\pgfsys@useobject{currentmarker}{}%
\end{pgfscope}%
\end{pgfscope}%
\begin{pgfscope}%
\pgfsetbuttcap%
\pgfsetroundjoin%
\definecolor{currentfill}{rgb}{0.000000,0.000000,0.000000}%
\pgfsetfillcolor{currentfill}%
\pgfsetlinewidth{0.602250pt}%
\definecolor{currentstroke}{rgb}{0.000000,0.000000,0.000000}%
\pgfsetstrokecolor{currentstroke}%
\pgfsetdash{}{0pt}%
\pgfsys@defobject{currentmarker}{\pgfqpoint{-0.027778in}{0.000000in}}{\pgfqpoint{-0.000000in}{0.000000in}}{%
\pgfpathmoveto{\pgfqpoint{-0.000000in}{0.000000in}}%
\pgfpathlineto{\pgfqpoint{-0.027778in}{0.000000in}}%
\pgfusepath{stroke,fill}%
}%
\begin{pgfscope}%
\pgfsys@transformshift{0.708220in}{1.049611in}%
\pgfsys@useobject{currentmarker}{}%
\end{pgfscope}%
\end{pgfscope}%
\begin{pgfscope}%
\pgfsetbuttcap%
\pgfsetroundjoin%
\definecolor{currentfill}{rgb}{0.000000,0.000000,0.000000}%
\pgfsetfillcolor{currentfill}%
\pgfsetlinewidth{0.602250pt}%
\definecolor{currentstroke}{rgb}{0.000000,0.000000,0.000000}%
\pgfsetstrokecolor{currentstroke}%
\pgfsetdash{}{0pt}%
\pgfsys@defobject{currentmarker}{\pgfqpoint{-0.027778in}{0.000000in}}{\pgfqpoint{-0.000000in}{0.000000in}}{%
\pgfpathmoveto{\pgfqpoint{-0.000000in}{0.000000in}}%
\pgfpathlineto{\pgfqpoint{-0.027778in}{0.000000in}}%
\pgfusepath{stroke,fill}%
}%
\begin{pgfscope}%
\pgfsys@transformshift{0.708220in}{1.119151in}%
\pgfsys@useobject{currentmarker}{}%
\end{pgfscope}%
\end{pgfscope}%
\begin{pgfscope}%
\pgfsetbuttcap%
\pgfsetroundjoin%
\definecolor{currentfill}{rgb}{0.000000,0.000000,0.000000}%
\pgfsetfillcolor{currentfill}%
\pgfsetlinewidth{0.602250pt}%
\definecolor{currentstroke}{rgb}{0.000000,0.000000,0.000000}%
\pgfsetstrokecolor{currentstroke}%
\pgfsetdash{}{0pt}%
\pgfsys@defobject{currentmarker}{\pgfqpoint{-0.027778in}{0.000000in}}{\pgfqpoint{-0.000000in}{0.000000in}}{%
\pgfpathmoveto{\pgfqpoint{-0.000000in}{0.000000in}}%
\pgfpathlineto{\pgfqpoint{-0.027778in}{0.000000in}}%
\pgfusepath{stroke,fill}%
}%
\begin{pgfscope}%
\pgfsys@transformshift{0.708220in}{1.168490in}%
\pgfsys@useobject{currentmarker}{}%
\end{pgfscope}%
\end{pgfscope}%
\begin{pgfscope}%
\pgfsetbuttcap%
\pgfsetroundjoin%
\definecolor{currentfill}{rgb}{0.000000,0.000000,0.000000}%
\pgfsetfillcolor{currentfill}%
\pgfsetlinewidth{0.602250pt}%
\definecolor{currentstroke}{rgb}{0.000000,0.000000,0.000000}%
\pgfsetstrokecolor{currentstroke}%
\pgfsetdash{}{0pt}%
\pgfsys@defobject{currentmarker}{\pgfqpoint{-0.027778in}{0.000000in}}{\pgfqpoint{-0.000000in}{0.000000in}}{%
\pgfpathmoveto{\pgfqpoint{-0.000000in}{0.000000in}}%
\pgfpathlineto{\pgfqpoint{-0.027778in}{0.000000in}}%
\pgfusepath{stroke,fill}%
}%
\begin{pgfscope}%
\pgfsys@transformshift{0.708220in}{1.206761in}%
\pgfsys@useobject{currentmarker}{}%
\end{pgfscope}%
\end{pgfscope}%
\begin{pgfscope}%
\pgfsetbuttcap%
\pgfsetroundjoin%
\definecolor{currentfill}{rgb}{0.000000,0.000000,0.000000}%
\pgfsetfillcolor{currentfill}%
\pgfsetlinewidth{0.602250pt}%
\definecolor{currentstroke}{rgb}{0.000000,0.000000,0.000000}%
\pgfsetstrokecolor{currentstroke}%
\pgfsetdash{}{0pt}%
\pgfsys@defobject{currentmarker}{\pgfqpoint{-0.027778in}{0.000000in}}{\pgfqpoint{-0.000000in}{0.000000in}}{%
\pgfpathmoveto{\pgfqpoint{-0.000000in}{0.000000in}}%
\pgfpathlineto{\pgfqpoint{-0.027778in}{0.000000in}}%
\pgfusepath{stroke,fill}%
}%
\begin{pgfscope}%
\pgfsys@transformshift{0.708220in}{1.238030in}%
\pgfsys@useobject{currentmarker}{}%
\end{pgfscope}%
\end{pgfscope}%
\begin{pgfscope}%
\pgfsetbuttcap%
\pgfsetroundjoin%
\definecolor{currentfill}{rgb}{0.000000,0.000000,0.000000}%
\pgfsetfillcolor{currentfill}%
\pgfsetlinewidth{0.602250pt}%
\definecolor{currentstroke}{rgb}{0.000000,0.000000,0.000000}%
\pgfsetstrokecolor{currentstroke}%
\pgfsetdash{}{0pt}%
\pgfsys@defobject{currentmarker}{\pgfqpoint{-0.027778in}{0.000000in}}{\pgfqpoint{-0.000000in}{0.000000in}}{%
\pgfpathmoveto{\pgfqpoint{-0.000000in}{0.000000in}}%
\pgfpathlineto{\pgfqpoint{-0.027778in}{0.000000in}}%
\pgfusepath{stroke,fill}%
}%
\begin{pgfscope}%
\pgfsys@transformshift{0.708220in}{1.264468in}%
\pgfsys@useobject{currentmarker}{}%
\end{pgfscope}%
\end{pgfscope}%
\begin{pgfscope}%
\pgfsetbuttcap%
\pgfsetroundjoin%
\definecolor{currentfill}{rgb}{0.000000,0.000000,0.000000}%
\pgfsetfillcolor{currentfill}%
\pgfsetlinewidth{0.602250pt}%
\definecolor{currentstroke}{rgb}{0.000000,0.000000,0.000000}%
\pgfsetstrokecolor{currentstroke}%
\pgfsetdash{}{0pt}%
\pgfsys@defobject{currentmarker}{\pgfqpoint{-0.027778in}{0.000000in}}{\pgfqpoint{-0.000000in}{0.000000in}}{%
\pgfpathmoveto{\pgfqpoint{-0.000000in}{0.000000in}}%
\pgfpathlineto{\pgfqpoint{-0.027778in}{0.000000in}}%
\pgfusepath{stroke,fill}%
}%
\begin{pgfscope}%
\pgfsys@transformshift{0.708220in}{1.287370in}%
\pgfsys@useobject{currentmarker}{}%
\end{pgfscope}%
\end{pgfscope}%
\begin{pgfscope}%
\pgfsetbuttcap%
\pgfsetroundjoin%
\definecolor{currentfill}{rgb}{0.000000,0.000000,0.000000}%
\pgfsetfillcolor{currentfill}%
\pgfsetlinewidth{0.602250pt}%
\definecolor{currentstroke}{rgb}{0.000000,0.000000,0.000000}%
\pgfsetstrokecolor{currentstroke}%
\pgfsetdash{}{0pt}%
\pgfsys@defobject{currentmarker}{\pgfqpoint{-0.027778in}{0.000000in}}{\pgfqpoint{-0.000000in}{0.000000in}}{%
\pgfpathmoveto{\pgfqpoint{-0.000000in}{0.000000in}}%
\pgfpathlineto{\pgfqpoint{-0.027778in}{0.000000in}}%
\pgfusepath{stroke,fill}%
}%
\begin{pgfscope}%
\pgfsys@transformshift{0.708220in}{1.307570in}%
\pgfsys@useobject{currentmarker}{}%
\end{pgfscope}%
\end{pgfscope}%
\begin{pgfscope}%
\pgfsetbuttcap%
\pgfsetroundjoin%
\definecolor{currentfill}{rgb}{0.000000,0.000000,0.000000}%
\pgfsetfillcolor{currentfill}%
\pgfsetlinewidth{0.602250pt}%
\definecolor{currentstroke}{rgb}{0.000000,0.000000,0.000000}%
\pgfsetstrokecolor{currentstroke}%
\pgfsetdash{}{0pt}%
\pgfsys@defobject{currentmarker}{\pgfqpoint{-0.027778in}{0.000000in}}{\pgfqpoint{-0.000000in}{0.000000in}}{%
\pgfpathmoveto{\pgfqpoint{-0.000000in}{0.000000in}}%
\pgfpathlineto{\pgfqpoint{-0.027778in}{0.000000in}}%
\pgfusepath{stroke,fill}%
}%
\begin{pgfscope}%
\pgfsys@transformshift{0.708220in}{1.444520in}%
\pgfsys@useobject{currentmarker}{}%
\end{pgfscope}%
\end{pgfscope}%
\begin{pgfscope}%
\pgfsetbuttcap%
\pgfsetroundjoin%
\definecolor{currentfill}{rgb}{0.000000,0.000000,0.000000}%
\pgfsetfillcolor{currentfill}%
\pgfsetlinewidth{0.602250pt}%
\definecolor{currentstroke}{rgb}{0.000000,0.000000,0.000000}%
\pgfsetstrokecolor{currentstroke}%
\pgfsetdash{}{0pt}%
\pgfsys@defobject{currentmarker}{\pgfqpoint{-0.027778in}{0.000000in}}{\pgfqpoint{-0.000000in}{0.000000in}}{%
\pgfpathmoveto{\pgfqpoint{-0.000000in}{0.000000in}}%
\pgfpathlineto{\pgfqpoint{-0.027778in}{0.000000in}}%
\pgfusepath{stroke,fill}%
}%
\begin{pgfscope}%
\pgfsys@transformshift{0.708220in}{1.514060in}%
\pgfsys@useobject{currentmarker}{}%
\end{pgfscope}%
\end{pgfscope}%
\begin{pgfscope}%
\pgfsetbuttcap%
\pgfsetroundjoin%
\definecolor{currentfill}{rgb}{0.000000,0.000000,0.000000}%
\pgfsetfillcolor{currentfill}%
\pgfsetlinewidth{0.602250pt}%
\definecolor{currentstroke}{rgb}{0.000000,0.000000,0.000000}%
\pgfsetstrokecolor{currentstroke}%
\pgfsetdash{}{0pt}%
\pgfsys@defobject{currentmarker}{\pgfqpoint{-0.027778in}{0.000000in}}{\pgfqpoint{-0.000000in}{0.000000in}}{%
\pgfpathmoveto{\pgfqpoint{-0.000000in}{0.000000in}}%
\pgfpathlineto{\pgfqpoint{-0.027778in}{0.000000in}}%
\pgfusepath{stroke,fill}%
}%
\begin{pgfscope}%
\pgfsys@transformshift{0.708220in}{1.563399in}%
\pgfsys@useobject{currentmarker}{}%
\end{pgfscope}%
\end{pgfscope}%
\begin{pgfscope}%
\pgfsetbuttcap%
\pgfsetroundjoin%
\definecolor{currentfill}{rgb}{0.000000,0.000000,0.000000}%
\pgfsetfillcolor{currentfill}%
\pgfsetlinewidth{0.602250pt}%
\definecolor{currentstroke}{rgb}{0.000000,0.000000,0.000000}%
\pgfsetstrokecolor{currentstroke}%
\pgfsetdash{}{0pt}%
\pgfsys@defobject{currentmarker}{\pgfqpoint{-0.027778in}{0.000000in}}{\pgfqpoint{-0.000000in}{0.000000in}}{%
\pgfpathmoveto{\pgfqpoint{-0.000000in}{0.000000in}}%
\pgfpathlineto{\pgfqpoint{-0.027778in}{0.000000in}}%
\pgfusepath{stroke,fill}%
}%
\begin{pgfscope}%
\pgfsys@transformshift{0.708220in}{1.601670in}%
\pgfsys@useobject{currentmarker}{}%
\end{pgfscope}%
\end{pgfscope}%
\begin{pgfscope}%
\pgfsetbuttcap%
\pgfsetroundjoin%
\definecolor{currentfill}{rgb}{0.000000,0.000000,0.000000}%
\pgfsetfillcolor{currentfill}%
\pgfsetlinewidth{0.602250pt}%
\definecolor{currentstroke}{rgb}{0.000000,0.000000,0.000000}%
\pgfsetstrokecolor{currentstroke}%
\pgfsetdash{}{0pt}%
\pgfsys@defobject{currentmarker}{\pgfqpoint{-0.027778in}{0.000000in}}{\pgfqpoint{-0.000000in}{0.000000in}}{%
\pgfpathmoveto{\pgfqpoint{-0.000000in}{0.000000in}}%
\pgfpathlineto{\pgfqpoint{-0.027778in}{0.000000in}}%
\pgfusepath{stroke,fill}%
}%
\begin{pgfscope}%
\pgfsys@transformshift{0.708220in}{1.632939in}%
\pgfsys@useobject{currentmarker}{}%
\end{pgfscope}%
\end{pgfscope}%
\begin{pgfscope}%
\pgfsetbuttcap%
\pgfsetroundjoin%
\definecolor{currentfill}{rgb}{0.000000,0.000000,0.000000}%
\pgfsetfillcolor{currentfill}%
\pgfsetlinewidth{0.602250pt}%
\definecolor{currentstroke}{rgb}{0.000000,0.000000,0.000000}%
\pgfsetstrokecolor{currentstroke}%
\pgfsetdash{}{0pt}%
\pgfsys@defobject{currentmarker}{\pgfqpoint{-0.027778in}{0.000000in}}{\pgfqpoint{-0.000000in}{0.000000in}}{%
\pgfpathmoveto{\pgfqpoint{-0.000000in}{0.000000in}}%
\pgfpathlineto{\pgfqpoint{-0.027778in}{0.000000in}}%
\pgfusepath{stroke,fill}%
}%
\begin{pgfscope}%
\pgfsys@transformshift{0.708220in}{1.659377in}%
\pgfsys@useobject{currentmarker}{}%
\end{pgfscope}%
\end{pgfscope}%
\begin{pgfscope}%
\pgfsetbuttcap%
\pgfsetroundjoin%
\definecolor{currentfill}{rgb}{0.000000,0.000000,0.000000}%
\pgfsetfillcolor{currentfill}%
\pgfsetlinewidth{0.602250pt}%
\definecolor{currentstroke}{rgb}{0.000000,0.000000,0.000000}%
\pgfsetstrokecolor{currentstroke}%
\pgfsetdash{}{0pt}%
\pgfsys@defobject{currentmarker}{\pgfqpoint{-0.027778in}{0.000000in}}{\pgfqpoint{-0.000000in}{0.000000in}}{%
\pgfpathmoveto{\pgfqpoint{-0.000000in}{0.000000in}}%
\pgfpathlineto{\pgfqpoint{-0.027778in}{0.000000in}}%
\pgfusepath{stroke,fill}%
}%
\begin{pgfscope}%
\pgfsys@transformshift{0.708220in}{1.682278in}%
\pgfsys@useobject{currentmarker}{}%
\end{pgfscope}%
\end{pgfscope}%
\begin{pgfscope}%
\pgfsetbuttcap%
\pgfsetroundjoin%
\definecolor{currentfill}{rgb}{0.000000,0.000000,0.000000}%
\pgfsetfillcolor{currentfill}%
\pgfsetlinewidth{0.602250pt}%
\definecolor{currentstroke}{rgb}{0.000000,0.000000,0.000000}%
\pgfsetstrokecolor{currentstroke}%
\pgfsetdash{}{0pt}%
\pgfsys@defobject{currentmarker}{\pgfqpoint{-0.027778in}{0.000000in}}{\pgfqpoint{-0.000000in}{0.000000in}}{%
\pgfpathmoveto{\pgfqpoint{-0.000000in}{0.000000in}}%
\pgfpathlineto{\pgfqpoint{-0.027778in}{0.000000in}}%
\pgfusepath{stroke,fill}%
}%
\begin{pgfscope}%
\pgfsys@transformshift{0.708220in}{1.702479in}%
\pgfsys@useobject{currentmarker}{}%
\end{pgfscope}%
\end{pgfscope}%
\begin{pgfscope}%
\pgfsetbuttcap%
\pgfsetroundjoin%
\definecolor{currentfill}{rgb}{0.000000,0.000000,0.000000}%
\pgfsetfillcolor{currentfill}%
\pgfsetlinewidth{0.602250pt}%
\definecolor{currentstroke}{rgb}{0.000000,0.000000,0.000000}%
\pgfsetstrokecolor{currentstroke}%
\pgfsetdash{}{0pt}%
\pgfsys@defobject{currentmarker}{\pgfqpoint{-0.027778in}{0.000000in}}{\pgfqpoint{-0.000000in}{0.000000in}}{%
\pgfpathmoveto{\pgfqpoint{-0.000000in}{0.000000in}}%
\pgfpathlineto{\pgfqpoint{-0.027778in}{0.000000in}}%
\pgfusepath{stroke,fill}%
}%
\begin{pgfscope}%
\pgfsys@transformshift{0.708220in}{1.839428in}%
\pgfsys@useobject{currentmarker}{}%
\end{pgfscope}%
\end{pgfscope}%
\begin{pgfscope}%
\pgfsetbuttcap%
\pgfsetroundjoin%
\definecolor{currentfill}{rgb}{0.000000,0.000000,0.000000}%
\pgfsetfillcolor{currentfill}%
\pgfsetlinewidth{0.602250pt}%
\definecolor{currentstroke}{rgb}{0.000000,0.000000,0.000000}%
\pgfsetstrokecolor{currentstroke}%
\pgfsetdash{}{0pt}%
\pgfsys@defobject{currentmarker}{\pgfqpoint{-0.027778in}{0.000000in}}{\pgfqpoint{-0.000000in}{0.000000in}}{%
\pgfpathmoveto{\pgfqpoint{-0.000000in}{0.000000in}}%
\pgfpathlineto{\pgfqpoint{-0.027778in}{0.000000in}}%
\pgfusepath{stroke,fill}%
}%
\begin{pgfscope}%
\pgfsys@transformshift{0.708220in}{1.908968in}%
\pgfsys@useobject{currentmarker}{}%
\end{pgfscope}%
\end{pgfscope}%
\begin{pgfscope}%
\pgfsetbuttcap%
\pgfsetroundjoin%
\definecolor{currentfill}{rgb}{0.000000,0.000000,0.000000}%
\pgfsetfillcolor{currentfill}%
\pgfsetlinewidth{0.602250pt}%
\definecolor{currentstroke}{rgb}{0.000000,0.000000,0.000000}%
\pgfsetstrokecolor{currentstroke}%
\pgfsetdash{}{0pt}%
\pgfsys@defobject{currentmarker}{\pgfqpoint{-0.027778in}{0.000000in}}{\pgfqpoint{-0.000000in}{0.000000in}}{%
\pgfpathmoveto{\pgfqpoint{-0.000000in}{0.000000in}}%
\pgfpathlineto{\pgfqpoint{-0.027778in}{0.000000in}}%
\pgfusepath{stroke,fill}%
}%
\begin{pgfscope}%
\pgfsys@transformshift{0.708220in}{1.958308in}%
\pgfsys@useobject{currentmarker}{}%
\end{pgfscope}%
\end{pgfscope}%
\begin{pgfscope}%
\pgfsetbuttcap%
\pgfsetroundjoin%
\definecolor{currentfill}{rgb}{0.000000,0.000000,0.000000}%
\pgfsetfillcolor{currentfill}%
\pgfsetlinewidth{0.602250pt}%
\definecolor{currentstroke}{rgb}{0.000000,0.000000,0.000000}%
\pgfsetstrokecolor{currentstroke}%
\pgfsetdash{}{0pt}%
\pgfsys@defobject{currentmarker}{\pgfqpoint{-0.027778in}{0.000000in}}{\pgfqpoint{-0.000000in}{0.000000in}}{%
\pgfpathmoveto{\pgfqpoint{-0.000000in}{0.000000in}}%
\pgfpathlineto{\pgfqpoint{-0.027778in}{0.000000in}}%
\pgfusepath{stroke,fill}%
}%
\begin{pgfscope}%
\pgfsys@transformshift{0.708220in}{1.996578in}%
\pgfsys@useobject{currentmarker}{}%
\end{pgfscope}%
\end{pgfscope}%
\begin{pgfscope}%
\pgfsetbuttcap%
\pgfsetroundjoin%
\definecolor{currentfill}{rgb}{0.000000,0.000000,0.000000}%
\pgfsetfillcolor{currentfill}%
\pgfsetlinewidth{0.602250pt}%
\definecolor{currentstroke}{rgb}{0.000000,0.000000,0.000000}%
\pgfsetstrokecolor{currentstroke}%
\pgfsetdash{}{0pt}%
\pgfsys@defobject{currentmarker}{\pgfqpoint{-0.027778in}{0.000000in}}{\pgfqpoint{-0.000000in}{0.000000in}}{%
\pgfpathmoveto{\pgfqpoint{-0.000000in}{0.000000in}}%
\pgfpathlineto{\pgfqpoint{-0.027778in}{0.000000in}}%
\pgfusepath{stroke,fill}%
}%
\begin{pgfscope}%
\pgfsys@transformshift{0.708220in}{2.027848in}%
\pgfsys@useobject{currentmarker}{}%
\end{pgfscope}%
\end{pgfscope}%
\begin{pgfscope}%
\pgfsetbuttcap%
\pgfsetroundjoin%
\definecolor{currentfill}{rgb}{0.000000,0.000000,0.000000}%
\pgfsetfillcolor{currentfill}%
\pgfsetlinewidth{0.602250pt}%
\definecolor{currentstroke}{rgb}{0.000000,0.000000,0.000000}%
\pgfsetstrokecolor{currentstroke}%
\pgfsetdash{}{0pt}%
\pgfsys@defobject{currentmarker}{\pgfqpoint{-0.027778in}{0.000000in}}{\pgfqpoint{-0.000000in}{0.000000in}}{%
\pgfpathmoveto{\pgfqpoint{-0.000000in}{0.000000in}}%
\pgfpathlineto{\pgfqpoint{-0.027778in}{0.000000in}}%
\pgfusepath{stroke,fill}%
}%
\begin{pgfscope}%
\pgfsys@transformshift{0.708220in}{2.054286in}%
\pgfsys@useobject{currentmarker}{}%
\end{pgfscope}%
\end{pgfscope}%
\begin{pgfscope}%
\pgfsetbuttcap%
\pgfsetroundjoin%
\definecolor{currentfill}{rgb}{0.000000,0.000000,0.000000}%
\pgfsetfillcolor{currentfill}%
\pgfsetlinewidth{0.602250pt}%
\definecolor{currentstroke}{rgb}{0.000000,0.000000,0.000000}%
\pgfsetstrokecolor{currentstroke}%
\pgfsetdash{}{0pt}%
\pgfsys@defobject{currentmarker}{\pgfqpoint{-0.027778in}{0.000000in}}{\pgfqpoint{-0.000000in}{0.000000in}}{%
\pgfpathmoveto{\pgfqpoint{-0.000000in}{0.000000in}}%
\pgfpathlineto{\pgfqpoint{-0.027778in}{0.000000in}}%
\pgfusepath{stroke,fill}%
}%
\begin{pgfscope}%
\pgfsys@transformshift{0.708220in}{2.077187in}%
\pgfsys@useobject{currentmarker}{}%
\end{pgfscope}%
\end{pgfscope}%
\begin{pgfscope}%
\pgfsetbuttcap%
\pgfsetroundjoin%
\definecolor{currentfill}{rgb}{0.000000,0.000000,0.000000}%
\pgfsetfillcolor{currentfill}%
\pgfsetlinewidth{0.602250pt}%
\definecolor{currentstroke}{rgb}{0.000000,0.000000,0.000000}%
\pgfsetstrokecolor{currentstroke}%
\pgfsetdash{}{0pt}%
\pgfsys@defobject{currentmarker}{\pgfqpoint{-0.027778in}{0.000000in}}{\pgfqpoint{-0.000000in}{0.000000in}}{%
\pgfpathmoveto{\pgfqpoint{-0.000000in}{0.000000in}}%
\pgfpathlineto{\pgfqpoint{-0.027778in}{0.000000in}}%
\pgfusepath{stroke,fill}%
}%
\begin{pgfscope}%
\pgfsys@transformshift{0.708220in}{2.097388in}%
\pgfsys@useobject{currentmarker}{}%
\end{pgfscope}%
\end{pgfscope}%
\begin{pgfscope}%
\pgfsetbuttcap%
\pgfsetroundjoin%
\definecolor{currentfill}{rgb}{0.000000,0.000000,0.000000}%
\pgfsetfillcolor{currentfill}%
\pgfsetlinewidth{0.602250pt}%
\definecolor{currentstroke}{rgb}{0.000000,0.000000,0.000000}%
\pgfsetstrokecolor{currentstroke}%
\pgfsetdash{}{0pt}%
\pgfsys@defobject{currentmarker}{\pgfqpoint{-0.027778in}{0.000000in}}{\pgfqpoint{-0.000000in}{0.000000in}}{%
\pgfpathmoveto{\pgfqpoint{-0.000000in}{0.000000in}}%
\pgfpathlineto{\pgfqpoint{-0.027778in}{0.000000in}}%
\pgfusepath{stroke,fill}%
}%
\begin{pgfscope}%
\pgfsys@transformshift{0.708220in}{2.234337in}%
\pgfsys@useobject{currentmarker}{}%
\end{pgfscope}%
\end{pgfscope}%
\begin{pgfscope}%
\pgfsetbuttcap%
\pgfsetroundjoin%
\definecolor{currentfill}{rgb}{0.000000,0.000000,0.000000}%
\pgfsetfillcolor{currentfill}%
\pgfsetlinewidth{0.602250pt}%
\definecolor{currentstroke}{rgb}{0.000000,0.000000,0.000000}%
\pgfsetstrokecolor{currentstroke}%
\pgfsetdash{}{0pt}%
\pgfsys@defobject{currentmarker}{\pgfqpoint{-0.027778in}{0.000000in}}{\pgfqpoint{-0.000000in}{0.000000in}}{%
\pgfpathmoveto{\pgfqpoint{-0.000000in}{0.000000in}}%
\pgfpathlineto{\pgfqpoint{-0.027778in}{0.000000in}}%
\pgfusepath{stroke,fill}%
}%
\begin{pgfscope}%
\pgfsys@transformshift{0.708220in}{2.303877in}%
\pgfsys@useobject{currentmarker}{}%
\end{pgfscope}%
\end{pgfscope}%
\begin{pgfscope}%
\pgfsetbuttcap%
\pgfsetroundjoin%
\definecolor{currentfill}{rgb}{0.000000,0.000000,0.000000}%
\pgfsetfillcolor{currentfill}%
\pgfsetlinewidth{0.602250pt}%
\definecolor{currentstroke}{rgb}{0.000000,0.000000,0.000000}%
\pgfsetstrokecolor{currentstroke}%
\pgfsetdash{}{0pt}%
\pgfsys@defobject{currentmarker}{\pgfqpoint{-0.027778in}{0.000000in}}{\pgfqpoint{-0.000000in}{0.000000in}}{%
\pgfpathmoveto{\pgfqpoint{-0.000000in}{0.000000in}}%
\pgfpathlineto{\pgfqpoint{-0.027778in}{0.000000in}}%
\pgfusepath{stroke,fill}%
}%
\begin{pgfscope}%
\pgfsys@transformshift{0.708220in}{2.353216in}%
\pgfsys@useobject{currentmarker}{}%
\end{pgfscope}%
\end{pgfscope}%
\begin{pgfscope}%
\pgfsetbuttcap%
\pgfsetroundjoin%
\definecolor{currentfill}{rgb}{0.000000,0.000000,0.000000}%
\pgfsetfillcolor{currentfill}%
\pgfsetlinewidth{0.602250pt}%
\definecolor{currentstroke}{rgb}{0.000000,0.000000,0.000000}%
\pgfsetstrokecolor{currentstroke}%
\pgfsetdash{}{0pt}%
\pgfsys@defobject{currentmarker}{\pgfqpoint{-0.027778in}{0.000000in}}{\pgfqpoint{-0.000000in}{0.000000in}}{%
\pgfpathmoveto{\pgfqpoint{-0.000000in}{0.000000in}}%
\pgfpathlineto{\pgfqpoint{-0.027778in}{0.000000in}}%
\pgfusepath{stroke,fill}%
}%
\begin{pgfscope}%
\pgfsys@transformshift{0.708220in}{2.391487in}%
\pgfsys@useobject{currentmarker}{}%
\end{pgfscope}%
\end{pgfscope}%
\begin{pgfscope}%
\pgfsetbuttcap%
\pgfsetroundjoin%
\definecolor{currentfill}{rgb}{0.000000,0.000000,0.000000}%
\pgfsetfillcolor{currentfill}%
\pgfsetlinewidth{0.602250pt}%
\definecolor{currentstroke}{rgb}{0.000000,0.000000,0.000000}%
\pgfsetstrokecolor{currentstroke}%
\pgfsetdash{}{0pt}%
\pgfsys@defobject{currentmarker}{\pgfqpoint{-0.027778in}{0.000000in}}{\pgfqpoint{-0.000000in}{0.000000in}}{%
\pgfpathmoveto{\pgfqpoint{-0.000000in}{0.000000in}}%
\pgfpathlineto{\pgfqpoint{-0.027778in}{0.000000in}}%
\pgfusepath{stroke,fill}%
}%
\begin{pgfscope}%
\pgfsys@transformshift{0.708220in}{2.422756in}%
\pgfsys@useobject{currentmarker}{}%
\end{pgfscope}%
\end{pgfscope}%
\begin{pgfscope}%
\pgfsetbuttcap%
\pgfsetroundjoin%
\definecolor{currentfill}{rgb}{0.000000,0.000000,0.000000}%
\pgfsetfillcolor{currentfill}%
\pgfsetlinewidth{0.602250pt}%
\definecolor{currentstroke}{rgb}{0.000000,0.000000,0.000000}%
\pgfsetstrokecolor{currentstroke}%
\pgfsetdash{}{0pt}%
\pgfsys@defobject{currentmarker}{\pgfqpoint{-0.027778in}{0.000000in}}{\pgfqpoint{-0.000000in}{0.000000in}}{%
\pgfpathmoveto{\pgfqpoint{-0.000000in}{0.000000in}}%
\pgfpathlineto{\pgfqpoint{-0.027778in}{0.000000in}}%
\pgfusepath{stroke,fill}%
}%
\begin{pgfscope}%
\pgfsys@transformshift{0.708220in}{2.449194in}%
\pgfsys@useobject{currentmarker}{}%
\end{pgfscope}%
\end{pgfscope}%
\begin{pgfscope}%
\pgfsetbuttcap%
\pgfsetroundjoin%
\definecolor{currentfill}{rgb}{0.000000,0.000000,0.000000}%
\pgfsetfillcolor{currentfill}%
\pgfsetlinewidth{0.602250pt}%
\definecolor{currentstroke}{rgb}{0.000000,0.000000,0.000000}%
\pgfsetstrokecolor{currentstroke}%
\pgfsetdash{}{0pt}%
\pgfsys@defobject{currentmarker}{\pgfqpoint{-0.027778in}{0.000000in}}{\pgfqpoint{-0.000000in}{0.000000in}}{%
\pgfpathmoveto{\pgfqpoint{-0.000000in}{0.000000in}}%
\pgfpathlineto{\pgfqpoint{-0.027778in}{0.000000in}}%
\pgfusepath{stroke,fill}%
}%
\begin{pgfscope}%
\pgfsys@transformshift{0.708220in}{2.472096in}%
\pgfsys@useobject{currentmarker}{}%
\end{pgfscope}%
\end{pgfscope}%
\begin{pgfscope}%
\pgfsetbuttcap%
\pgfsetroundjoin%
\definecolor{currentfill}{rgb}{0.000000,0.000000,0.000000}%
\pgfsetfillcolor{currentfill}%
\pgfsetlinewidth{0.602250pt}%
\definecolor{currentstroke}{rgb}{0.000000,0.000000,0.000000}%
\pgfsetstrokecolor{currentstroke}%
\pgfsetdash{}{0pt}%
\pgfsys@defobject{currentmarker}{\pgfqpoint{-0.027778in}{0.000000in}}{\pgfqpoint{-0.000000in}{0.000000in}}{%
\pgfpathmoveto{\pgfqpoint{-0.000000in}{0.000000in}}%
\pgfpathlineto{\pgfqpoint{-0.027778in}{0.000000in}}%
\pgfusepath{stroke,fill}%
}%
\begin{pgfscope}%
\pgfsys@transformshift{0.708220in}{2.492296in}%
\pgfsys@useobject{currentmarker}{}%
\end{pgfscope}%
\end{pgfscope}%
\begin{pgfscope}%
\pgfsetbuttcap%
\pgfsetroundjoin%
\definecolor{currentfill}{rgb}{0.000000,0.000000,0.000000}%
\pgfsetfillcolor{currentfill}%
\pgfsetlinewidth{0.602250pt}%
\definecolor{currentstroke}{rgb}{0.000000,0.000000,0.000000}%
\pgfsetstrokecolor{currentstroke}%
\pgfsetdash{}{0pt}%
\pgfsys@defobject{currentmarker}{\pgfqpoint{-0.027778in}{0.000000in}}{\pgfqpoint{-0.000000in}{0.000000in}}{%
\pgfpathmoveto{\pgfqpoint{-0.000000in}{0.000000in}}%
\pgfpathlineto{\pgfqpoint{-0.027778in}{0.000000in}}%
\pgfusepath{stroke,fill}%
}%
\begin{pgfscope}%
\pgfsys@transformshift{0.708220in}{2.629246in}%
\pgfsys@useobject{currentmarker}{}%
\end{pgfscope}%
\end{pgfscope}%
\begin{pgfscope}%
\pgfsetbuttcap%
\pgfsetroundjoin%
\definecolor{currentfill}{rgb}{0.000000,0.000000,0.000000}%
\pgfsetfillcolor{currentfill}%
\pgfsetlinewidth{0.602250pt}%
\definecolor{currentstroke}{rgb}{0.000000,0.000000,0.000000}%
\pgfsetstrokecolor{currentstroke}%
\pgfsetdash{}{0pt}%
\pgfsys@defobject{currentmarker}{\pgfqpoint{-0.027778in}{0.000000in}}{\pgfqpoint{-0.000000in}{0.000000in}}{%
\pgfpathmoveto{\pgfqpoint{-0.000000in}{0.000000in}}%
\pgfpathlineto{\pgfqpoint{-0.027778in}{0.000000in}}%
\pgfusepath{stroke,fill}%
}%
\begin{pgfscope}%
\pgfsys@transformshift{0.708220in}{2.698786in}%
\pgfsys@useobject{currentmarker}{}%
\end{pgfscope}%
\end{pgfscope}%
\begin{pgfscope}%
\pgfsetbuttcap%
\pgfsetroundjoin%
\definecolor{currentfill}{rgb}{0.000000,0.000000,0.000000}%
\pgfsetfillcolor{currentfill}%
\pgfsetlinewidth{0.602250pt}%
\definecolor{currentstroke}{rgb}{0.000000,0.000000,0.000000}%
\pgfsetstrokecolor{currentstroke}%
\pgfsetdash{}{0pt}%
\pgfsys@defobject{currentmarker}{\pgfqpoint{-0.027778in}{0.000000in}}{\pgfqpoint{-0.000000in}{0.000000in}}{%
\pgfpathmoveto{\pgfqpoint{-0.000000in}{0.000000in}}%
\pgfpathlineto{\pgfqpoint{-0.027778in}{0.000000in}}%
\pgfusepath{stroke,fill}%
}%
\begin{pgfscope}%
\pgfsys@transformshift{0.708220in}{2.748125in}%
\pgfsys@useobject{currentmarker}{}%
\end{pgfscope}%
\end{pgfscope}%
\begin{pgfscope}%
\pgfsetbuttcap%
\pgfsetroundjoin%
\definecolor{currentfill}{rgb}{0.000000,0.000000,0.000000}%
\pgfsetfillcolor{currentfill}%
\pgfsetlinewidth{0.602250pt}%
\definecolor{currentstroke}{rgb}{0.000000,0.000000,0.000000}%
\pgfsetstrokecolor{currentstroke}%
\pgfsetdash{}{0pt}%
\pgfsys@defobject{currentmarker}{\pgfqpoint{-0.027778in}{0.000000in}}{\pgfqpoint{-0.000000in}{0.000000in}}{%
\pgfpathmoveto{\pgfqpoint{-0.000000in}{0.000000in}}%
\pgfpathlineto{\pgfqpoint{-0.027778in}{0.000000in}}%
\pgfusepath{stroke,fill}%
}%
\begin{pgfscope}%
\pgfsys@transformshift{0.708220in}{2.786396in}%
\pgfsys@useobject{currentmarker}{}%
\end{pgfscope}%
\end{pgfscope}%
\begin{pgfscope}%
\pgfsetbuttcap%
\pgfsetroundjoin%
\definecolor{currentfill}{rgb}{0.000000,0.000000,0.000000}%
\pgfsetfillcolor{currentfill}%
\pgfsetlinewidth{0.602250pt}%
\definecolor{currentstroke}{rgb}{0.000000,0.000000,0.000000}%
\pgfsetstrokecolor{currentstroke}%
\pgfsetdash{}{0pt}%
\pgfsys@defobject{currentmarker}{\pgfqpoint{-0.027778in}{0.000000in}}{\pgfqpoint{-0.000000in}{0.000000in}}{%
\pgfpathmoveto{\pgfqpoint{-0.000000in}{0.000000in}}%
\pgfpathlineto{\pgfqpoint{-0.027778in}{0.000000in}}%
\pgfusepath{stroke,fill}%
}%
\begin{pgfscope}%
\pgfsys@transformshift{0.708220in}{2.817665in}%
\pgfsys@useobject{currentmarker}{}%
\end{pgfscope}%
\end{pgfscope}%
\begin{pgfscope}%
\pgfsetbuttcap%
\pgfsetroundjoin%
\definecolor{currentfill}{rgb}{0.000000,0.000000,0.000000}%
\pgfsetfillcolor{currentfill}%
\pgfsetlinewidth{0.602250pt}%
\definecolor{currentstroke}{rgb}{0.000000,0.000000,0.000000}%
\pgfsetstrokecolor{currentstroke}%
\pgfsetdash{}{0pt}%
\pgfsys@defobject{currentmarker}{\pgfqpoint{-0.027778in}{0.000000in}}{\pgfqpoint{-0.000000in}{0.000000in}}{%
\pgfpathmoveto{\pgfqpoint{-0.000000in}{0.000000in}}%
\pgfpathlineto{\pgfqpoint{-0.027778in}{0.000000in}}%
\pgfusepath{stroke,fill}%
}%
\begin{pgfscope}%
\pgfsys@transformshift{0.708220in}{2.844103in}%
\pgfsys@useobject{currentmarker}{}%
\end{pgfscope}%
\end{pgfscope}%
\begin{pgfscope}%
\pgfsetbuttcap%
\pgfsetroundjoin%
\definecolor{currentfill}{rgb}{0.000000,0.000000,0.000000}%
\pgfsetfillcolor{currentfill}%
\pgfsetlinewidth{0.602250pt}%
\definecolor{currentstroke}{rgb}{0.000000,0.000000,0.000000}%
\pgfsetstrokecolor{currentstroke}%
\pgfsetdash{}{0pt}%
\pgfsys@defobject{currentmarker}{\pgfqpoint{-0.027778in}{0.000000in}}{\pgfqpoint{-0.000000in}{0.000000in}}{%
\pgfpathmoveto{\pgfqpoint{-0.000000in}{0.000000in}}%
\pgfpathlineto{\pgfqpoint{-0.027778in}{0.000000in}}%
\pgfusepath{stroke,fill}%
}%
\begin{pgfscope}%
\pgfsys@transformshift{0.708220in}{2.867005in}%
\pgfsys@useobject{currentmarker}{}%
\end{pgfscope}%
\end{pgfscope}%
\begin{pgfscope}%
\pgfsetbuttcap%
\pgfsetroundjoin%
\definecolor{currentfill}{rgb}{0.000000,0.000000,0.000000}%
\pgfsetfillcolor{currentfill}%
\pgfsetlinewidth{0.602250pt}%
\definecolor{currentstroke}{rgb}{0.000000,0.000000,0.000000}%
\pgfsetstrokecolor{currentstroke}%
\pgfsetdash{}{0pt}%
\pgfsys@defobject{currentmarker}{\pgfqpoint{-0.027778in}{0.000000in}}{\pgfqpoint{-0.000000in}{0.000000in}}{%
\pgfpathmoveto{\pgfqpoint{-0.000000in}{0.000000in}}%
\pgfpathlineto{\pgfqpoint{-0.027778in}{0.000000in}}%
\pgfusepath{stroke,fill}%
}%
\begin{pgfscope}%
\pgfsys@transformshift{0.708220in}{2.887205in}%
\pgfsys@useobject{currentmarker}{}%
\end{pgfscope}%
\end{pgfscope}%
\begin{pgfscope}%
\definecolor{textcolor}{rgb}{0.000000,0.000000,0.000000}%
\pgfsetstrokecolor{textcolor}%
\pgfsetfillcolor{textcolor}%
\pgftext[x=0.288855in,y=1.720549in,,bottom,rotate=90.000000]{\color{textcolor}\rmfamily\fontsize{10.000000}{12.000000}\selectfont Longest solving time (s)}%
\end{pgfscope}%
\begin{pgfscope}%
\pgfpathrectangle{\pgfqpoint{0.708220in}{0.535823in}}{\pgfqpoint{5.013309in}{2.369453in}}%
\pgfusepath{clip}%
\pgfsetrectcap%
\pgfsetroundjoin%
\pgfsetlinewidth{1.003750pt}%
\definecolor{currentstroke}{rgb}{0.000000,0.000000,1.000000}%
\pgfsetstrokecolor{currentstroke}%
\pgfsetdash{}{0pt}%
\pgfpathmoveto{\pgfqpoint{0.708220in}{0.847768in}}%
\pgfpathlineto{\pgfqpoint{0.710727in}{0.859025in}}%
\pgfpathlineto{\pgfqpoint{0.713233in}{0.879918in}}%
\pgfpathlineto{\pgfqpoint{0.715740in}{0.921430in}}%
\pgfpathlineto{\pgfqpoint{0.718246in}{0.923652in}}%
\pgfpathlineto{\pgfqpoint{0.720753in}{0.935139in}}%
\pgfpathlineto{\pgfqpoint{0.725766in}{0.943360in}}%
\pgfpathlineto{\pgfqpoint{0.728273in}{0.948892in}}%
\pgfpathlineto{\pgfqpoint{0.733286in}{0.973965in}}%
\pgfpathlineto{\pgfqpoint{0.735793in}{0.985055in}}%
\pgfpathlineto{\pgfqpoint{0.738300in}{1.001644in}}%
\pgfpathlineto{\pgfqpoint{0.740806in}{1.002238in}}%
\pgfpathlineto{\pgfqpoint{0.748326in}{1.007876in}}%
\pgfpathlineto{\pgfqpoint{0.750833in}{1.013895in}}%
\pgfpathlineto{\pgfqpoint{0.753340in}{1.024263in}}%
\pgfpathlineto{\pgfqpoint{0.755846in}{1.028551in}}%
\pgfpathlineto{\pgfqpoint{0.765873in}{1.069512in}}%
\pgfpathlineto{\pgfqpoint{0.768380in}{1.084654in}}%
\pgfpathlineto{\pgfqpoint{0.770886in}{1.084671in}}%
\pgfpathlineto{\pgfqpoint{0.775900in}{1.090761in}}%
\pgfpathlineto{\pgfqpoint{0.778406in}{1.092868in}}%
\pgfpathlineto{\pgfqpoint{0.780913in}{1.119046in}}%
\pgfpathlineto{\pgfqpoint{0.783419in}{1.119859in}}%
\pgfpathlineto{\pgfqpoint{0.788433in}{1.125307in}}%
\pgfpathlineto{\pgfqpoint{0.790939in}{1.137370in}}%
\pgfpathlineto{\pgfqpoint{0.800966in}{1.154620in}}%
\pgfpathlineto{\pgfqpoint{0.803473in}{1.163398in}}%
\pgfpathlineto{\pgfqpoint{0.805979in}{1.167217in}}%
\pgfpathlineto{\pgfqpoint{0.808486in}{1.168455in}}%
\pgfpathlineto{\pgfqpoint{0.813499in}{1.182374in}}%
\pgfpathlineto{\pgfqpoint{0.818513in}{1.186224in}}%
\pgfpathlineto{\pgfqpoint{0.826033in}{1.205304in}}%
\pgfpathlineto{\pgfqpoint{0.828539in}{1.206726in}}%
\pgfpathlineto{\pgfqpoint{0.833553in}{1.221594in}}%
\pgfpathlineto{\pgfqpoint{0.836059in}{1.248796in}}%
\pgfpathlineto{\pgfqpoint{0.838566in}{1.251959in}}%
\pgfpathlineto{\pgfqpoint{0.851099in}{1.256701in}}%
\pgfpathlineto{\pgfqpoint{0.853606in}{1.261021in}}%
\pgfpathlineto{\pgfqpoint{0.858619in}{1.265116in}}%
\pgfpathlineto{\pgfqpoint{0.863632in}{1.265663in}}%
\pgfpathlineto{\pgfqpoint{0.866139in}{1.277119in}}%
\pgfpathlineto{\pgfqpoint{0.871152in}{1.281681in}}%
\pgfpathlineto{\pgfqpoint{0.873659in}{1.287815in}}%
\pgfpathlineto{\pgfqpoint{0.878672in}{1.290741in}}%
\pgfpathlineto{\pgfqpoint{0.888699in}{1.294546in}}%
\pgfpathlineto{\pgfqpoint{0.891206in}{1.305330in}}%
\pgfpathlineto{\pgfqpoint{0.893712in}{1.305761in}}%
\pgfpathlineto{\pgfqpoint{0.896219in}{1.309024in}}%
\pgfpathlineto{\pgfqpoint{0.901232in}{1.328126in}}%
\pgfpathlineto{\pgfqpoint{0.906246in}{1.328946in}}%
\pgfpathlineto{\pgfqpoint{0.913766in}{1.333023in}}%
\pgfpathlineto{\pgfqpoint{0.918779in}{1.342277in}}%
\pgfpathlineto{\pgfqpoint{0.921285in}{1.342374in}}%
\pgfpathlineto{\pgfqpoint{0.923792in}{1.347458in}}%
\pgfpathlineto{\pgfqpoint{0.926299in}{1.348299in}}%
\pgfpathlineto{\pgfqpoint{0.928805in}{1.350395in}}%
\pgfpathlineto{\pgfqpoint{0.943845in}{1.354366in}}%
\pgfpathlineto{\pgfqpoint{0.951365in}{1.357687in}}%
\pgfpathlineto{\pgfqpoint{0.953872in}{1.358207in}}%
\pgfpathlineto{\pgfqpoint{0.956379in}{1.362489in}}%
\pgfpathlineto{\pgfqpoint{0.958885in}{1.363857in}}%
\pgfpathlineto{\pgfqpoint{0.961392in}{1.370655in}}%
\pgfpathlineto{\pgfqpoint{0.963899in}{1.373466in}}%
\pgfpathlineto{\pgfqpoint{0.966405in}{1.378428in}}%
\pgfpathlineto{\pgfqpoint{0.968912in}{1.378476in}}%
\pgfpathlineto{\pgfqpoint{0.976432in}{1.383889in}}%
\pgfpathlineto{\pgfqpoint{0.978939in}{1.384445in}}%
\pgfpathlineto{\pgfqpoint{0.981445in}{1.392797in}}%
\pgfpathlineto{\pgfqpoint{0.986458in}{1.394120in}}%
\pgfpathlineto{\pgfqpoint{0.991472in}{1.399287in}}%
\pgfpathlineto{\pgfqpoint{0.996485in}{1.404345in}}%
\pgfpathlineto{\pgfqpoint{1.004005in}{1.424631in}}%
\pgfpathlineto{\pgfqpoint{1.006512in}{1.424965in}}%
\pgfpathlineto{\pgfqpoint{1.009018in}{1.427822in}}%
\pgfpathlineto{\pgfqpoint{1.019045in}{1.429943in}}%
\pgfpathlineto{\pgfqpoint{1.021552in}{1.433562in}}%
\pgfpathlineto{\pgfqpoint{1.026565in}{1.434059in}}%
\pgfpathlineto{\pgfqpoint{1.029072in}{1.437825in}}%
\pgfpathlineto{\pgfqpoint{1.034085in}{1.438985in}}%
\pgfpathlineto{\pgfqpoint{1.039098in}{1.442832in}}%
\pgfpathlineto{\pgfqpoint{1.046618in}{1.448038in}}%
\pgfpathlineto{\pgfqpoint{1.064165in}{1.456171in}}%
\pgfpathlineto{\pgfqpoint{1.066671in}{1.456272in}}%
\pgfpathlineto{\pgfqpoint{1.069178in}{1.457734in}}%
\pgfpathlineto{\pgfqpoint{1.076698in}{1.458695in}}%
\pgfpathlineto{\pgfqpoint{1.079205in}{1.461369in}}%
\pgfpathlineto{\pgfqpoint{1.089231in}{1.462923in}}%
\pgfpathlineto{\pgfqpoint{1.094245in}{1.468295in}}%
\pgfpathlineto{\pgfqpoint{1.101765in}{1.469018in}}%
\pgfpathlineto{\pgfqpoint{1.114298in}{1.479791in}}%
\pgfpathlineto{\pgfqpoint{1.116805in}{1.481072in}}%
\pgfpathlineto{\pgfqpoint{1.119311in}{1.487488in}}%
\pgfpathlineto{\pgfqpoint{1.121818in}{1.488799in}}%
\pgfpathlineto{\pgfqpoint{1.124324in}{1.492062in}}%
\pgfpathlineto{\pgfqpoint{1.126831in}{1.492909in}}%
\pgfpathlineto{\pgfqpoint{1.131844in}{1.500901in}}%
\pgfpathlineto{\pgfqpoint{1.134351in}{1.501107in}}%
\pgfpathlineto{\pgfqpoint{1.136858in}{1.506981in}}%
\pgfpathlineto{\pgfqpoint{1.141871in}{1.507348in}}%
\pgfpathlineto{\pgfqpoint{1.146884in}{1.517020in}}%
\pgfpathlineto{\pgfqpoint{1.156911in}{1.521461in}}%
\pgfpathlineto{\pgfqpoint{1.159418in}{1.522980in}}%
\pgfpathlineto{\pgfqpoint{1.164431in}{1.523713in}}%
\pgfpathlineto{\pgfqpoint{1.171951in}{1.526195in}}%
\pgfpathlineto{\pgfqpoint{1.174458in}{1.526311in}}%
\pgfpathlineto{\pgfqpoint{1.184484in}{1.531649in}}%
\pgfpathlineto{\pgfqpoint{1.186991in}{1.535660in}}%
\pgfpathlineto{\pgfqpoint{1.189498in}{1.535699in}}%
\pgfpathlineto{\pgfqpoint{1.214564in}{1.549654in}}%
\pgfpathlineto{\pgfqpoint{1.219577in}{1.555290in}}%
\pgfpathlineto{\pgfqpoint{1.222084in}{1.556622in}}%
\pgfpathlineto{\pgfqpoint{1.227097in}{1.562768in}}%
\pgfpathlineto{\pgfqpoint{1.232111in}{1.565473in}}%
\pgfpathlineto{\pgfqpoint{1.234617in}{1.570108in}}%
\pgfpathlineto{\pgfqpoint{1.242137in}{1.575076in}}%
\pgfpathlineto{\pgfqpoint{1.244644in}{1.579017in}}%
\pgfpathlineto{\pgfqpoint{1.247151in}{1.589099in}}%
\pgfpathlineto{\pgfqpoint{1.252164in}{1.589976in}}%
\pgfpathlineto{\pgfqpoint{1.257177in}{1.594285in}}%
\pgfpathlineto{\pgfqpoint{1.289764in}{1.620060in}}%
\pgfpathlineto{\pgfqpoint{1.292270in}{1.620780in}}%
\pgfpathlineto{\pgfqpoint{1.297284in}{1.626687in}}%
\pgfpathlineto{\pgfqpoint{1.304804in}{1.629397in}}%
\pgfpathlineto{\pgfqpoint{1.307310in}{1.633812in}}%
\pgfpathlineto{\pgfqpoint{1.309817in}{1.641834in}}%
\pgfpathlineto{\pgfqpoint{1.317337in}{1.643724in}}%
\pgfpathlineto{\pgfqpoint{1.322350in}{1.649358in}}%
\pgfpathlineto{\pgfqpoint{1.327363in}{1.650960in}}%
\pgfpathlineto{\pgfqpoint{1.332377in}{1.652036in}}%
\pgfpathlineto{\pgfqpoint{1.352430in}{1.663298in}}%
\pgfpathlineto{\pgfqpoint{1.357443in}{1.674892in}}%
\pgfpathlineto{\pgfqpoint{1.359950in}{1.675295in}}%
\pgfpathlineto{\pgfqpoint{1.364963in}{1.681509in}}%
\pgfpathlineto{\pgfqpoint{1.367470in}{1.685976in}}%
\pgfpathlineto{\pgfqpoint{1.369977in}{1.705596in}}%
\pgfpathlineto{\pgfqpoint{1.372483in}{1.706001in}}%
\pgfpathlineto{\pgfqpoint{1.374990in}{1.707705in}}%
\pgfpathlineto{\pgfqpoint{1.377497in}{1.711266in}}%
\pgfpathlineto{\pgfqpoint{1.382510in}{1.712135in}}%
\pgfpathlineto{\pgfqpoint{1.385017in}{1.716986in}}%
\pgfpathlineto{\pgfqpoint{1.387523in}{1.725486in}}%
\pgfpathlineto{\pgfqpoint{1.390030in}{1.726501in}}%
\pgfpathlineto{\pgfqpoint{1.392537in}{1.730780in}}%
\pgfpathlineto{\pgfqpoint{1.400056in}{1.732738in}}%
\pgfpathlineto{\pgfqpoint{1.410083in}{1.740241in}}%
\pgfpathlineto{\pgfqpoint{1.415096in}{1.752174in}}%
\pgfpathlineto{\pgfqpoint{1.417603in}{1.752911in}}%
\pgfpathlineto{\pgfqpoint{1.420110in}{1.754907in}}%
\pgfpathlineto{\pgfqpoint{1.422616in}{1.759665in}}%
\pgfpathlineto{\pgfqpoint{1.425123in}{1.760030in}}%
\pgfpathlineto{\pgfqpoint{1.427630in}{1.774519in}}%
\pgfpathlineto{\pgfqpoint{1.432643in}{1.777431in}}%
\pgfpathlineto{\pgfqpoint{1.435150in}{1.789616in}}%
\pgfpathlineto{\pgfqpoint{1.440163in}{1.792710in}}%
\pgfpathlineto{\pgfqpoint{1.442670in}{1.802282in}}%
\pgfpathlineto{\pgfqpoint{1.452696in}{1.806195in}}%
\pgfpathlineto{\pgfqpoint{1.455203in}{1.810710in}}%
\pgfpathlineto{\pgfqpoint{1.460216in}{1.813781in}}%
\pgfpathlineto{\pgfqpoint{1.470243in}{1.832696in}}%
\pgfpathlineto{\pgfqpoint{1.472749in}{1.833099in}}%
\pgfpathlineto{\pgfqpoint{1.480269in}{1.838668in}}%
\pgfpathlineto{\pgfqpoint{1.482776in}{1.839007in}}%
\pgfpathlineto{\pgfqpoint{1.487789in}{1.842281in}}%
\pgfpathlineto{\pgfqpoint{1.490296in}{1.842528in}}%
\pgfpathlineto{\pgfqpoint{1.492803in}{1.847774in}}%
\pgfpathlineto{\pgfqpoint{1.497816in}{1.851756in}}%
\pgfpathlineto{\pgfqpoint{1.500323in}{1.851845in}}%
\pgfpathlineto{\pgfqpoint{1.502829in}{1.854022in}}%
\pgfpathlineto{\pgfqpoint{1.505336in}{1.861322in}}%
\pgfpathlineto{\pgfqpoint{1.507843in}{1.861840in}}%
\pgfpathlineto{\pgfqpoint{1.517869in}{1.869276in}}%
\pgfpathlineto{\pgfqpoint{1.520376in}{1.869824in}}%
\pgfpathlineto{\pgfqpoint{1.522883in}{1.879362in}}%
\pgfpathlineto{\pgfqpoint{1.525389in}{1.896863in}}%
\pgfpathlineto{\pgfqpoint{1.532909in}{1.902133in}}%
\pgfpathlineto{\pgfqpoint{1.535416in}{1.920111in}}%
\pgfpathlineto{\pgfqpoint{1.540429in}{1.922717in}}%
\pgfpathlineto{\pgfqpoint{1.545442in}{1.932675in}}%
\pgfpathlineto{\pgfqpoint{1.547949in}{1.933070in}}%
\pgfpathlineto{\pgfqpoint{1.550456in}{1.946596in}}%
\pgfpathlineto{\pgfqpoint{1.557976in}{1.948282in}}%
\pgfpathlineto{\pgfqpoint{1.562989in}{1.952251in}}%
\pgfpathlineto{\pgfqpoint{1.568002in}{1.956350in}}%
\pgfpathlineto{\pgfqpoint{1.570509in}{1.974252in}}%
\pgfpathlineto{\pgfqpoint{1.575522in}{1.979515in}}%
\pgfpathlineto{\pgfqpoint{1.580536in}{1.980252in}}%
\pgfpathlineto{\pgfqpoint{1.588056in}{1.993135in}}%
\pgfpathlineto{\pgfqpoint{1.595576in}{1.997028in}}%
\pgfpathlineto{\pgfqpoint{1.598082in}{2.004256in}}%
\pgfpathlineto{\pgfqpoint{1.603095in}{2.031426in}}%
\pgfpathlineto{\pgfqpoint{1.608109in}{2.040070in}}%
\pgfpathlineto{\pgfqpoint{1.610615in}{2.053127in}}%
\pgfpathlineto{\pgfqpoint{1.613122in}{2.059788in}}%
\pgfpathlineto{\pgfqpoint{1.615629in}{2.062586in}}%
\pgfpathlineto{\pgfqpoint{1.625655in}{2.066047in}}%
\pgfpathlineto{\pgfqpoint{1.630669in}{2.078862in}}%
\pgfpathlineto{\pgfqpoint{1.635682in}{2.086294in}}%
\pgfpathlineto{\pgfqpoint{1.638189in}{2.086806in}}%
\pgfpathlineto{\pgfqpoint{1.643202in}{2.093283in}}%
\pgfpathlineto{\pgfqpoint{1.648215in}{2.104973in}}%
\pgfpathlineto{\pgfqpoint{1.653229in}{2.112180in}}%
\pgfpathlineto{\pgfqpoint{1.660749in}{2.114552in}}%
\pgfpathlineto{\pgfqpoint{1.663255in}{2.125687in}}%
\pgfpathlineto{\pgfqpoint{1.665762in}{2.130117in}}%
\pgfpathlineto{\pgfqpoint{1.668268in}{2.138060in}}%
\pgfpathlineto{\pgfqpoint{1.670775in}{2.151399in}}%
\pgfpathlineto{\pgfqpoint{1.673282in}{2.153275in}}%
\pgfpathlineto{\pgfqpoint{1.675788in}{2.153473in}}%
\pgfpathlineto{\pgfqpoint{1.678295in}{2.159054in}}%
\pgfpathlineto{\pgfqpoint{1.680802in}{2.160424in}}%
\pgfpathlineto{\pgfqpoint{1.685815in}{2.168312in}}%
\pgfpathlineto{\pgfqpoint{1.690828in}{2.179685in}}%
\pgfpathlineto{\pgfqpoint{1.693335in}{2.202569in}}%
\pgfpathlineto{\pgfqpoint{1.695842in}{2.215160in}}%
\pgfpathlineto{\pgfqpoint{1.703362in}{2.222181in}}%
\pgfpathlineto{\pgfqpoint{1.708375in}{2.222656in}}%
\pgfpathlineto{\pgfqpoint{1.710882in}{2.223458in}}%
\pgfpathlineto{\pgfqpoint{1.713388in}{2.227025in}}%
\pgfpathlineto{\pgfqpoint{1.715895in}{2.227245in}}%
\pgfpathlineto{\pgfqpoint{1.718402in}{2.230216in}}%
\pgfpathlineto{\pgfqpoint{1.720908in}{2.230798in}}%
\pgfpathlineto{\pgfqpoint{1.723415in}{2.233058in}}%
\pgfpathlineto{\pgfqpoint{1.730935in}{2.235771in}}%
\pgfpathlineto{\pgfqpoint{1.733442in}{2.238718in}}%
\pgfpathlineto{\pgfqpoint{1.735948in}{2.238970in}}%
\pgfpathlineto{\pgfqpoint{1.740961in}{2.242771in}}%
\pgfpathlineto{\pgfqpoint{1.745975in}{2.251373in}}%
\pgfpathlineto{\pgfqpoint{1.763521in}{2.256845in}}%
\pgfpathlineto{\pgfqpoint{1.771041in}{2.269062in}}%
\pgfpathlineto{\pgfqpoint{1.786081in}{2.318129in}}%
\pgfpathlineto{\pgfqpoint{1.793601in}{2.334114in}}%
\pgfpathlineto{\pgfqpoint{1.796108in}{2.339344in}}%
\pgfpathlineto{\pgfqpoint{1.798615in}{2.348952in}}%
\pgfpathlineto{\pgfqpoint{1.803628in}{2.352847in}}%
\pgfpathlineto{\pgfqpoint{1.806134in}{2.359272in}}%
\pgfpathlineto{\pgfqpoint{1.813654in}{2.367691in}}%
\pgfpathlineto{\pgfqpoint{1.816161in}{2.388561in}}%
\pgfpathlineto{\pgfqpoint{1.821174in}{2.402009in}}%
\pgfpathlineto{\pgfqpoint{1.826188in}{2.406737in}}%
\pgfpathlineto{\pgfqpoint{1.831201in}{2.430493in}}%
\pgfpathlineto{\pgfqpoint{1.836214in}{2.445869in}}%
\pgfpathlineto{\pgfqpoint{1.841228in}{2.447910in}}%
\pgfpathlineto{\pgfqpoint{1.843734in}{2.449111in}}%
\pgfpathlineto{\pgfqpoint{1.853761in}{2.464962in}}%
\pgfpathlineto{\pgfqpoint{1.856268in}{2.467474in}}%
\pgfpathlineto{\pgfqpoint{1.858774in}{2.467830in}}%
\pgfpathlineto{\pgfqpoint{1.861281in}{2.471891in}}%
\pgfpathlineto{\pgfqpoint{1.868801in}{2.475058in}}%
\pgfpathlineto{\pgfqpoint{1.873814in}{2.486651in}}%
\pgfpathlineto{\pgfqpoint{1.878827in}{2.487233in}}%
\pgfpathlineto{\pgfqpoint{1.881334in}{2.490927in}}%
\pgfpathlineto{\pgfqpoint{1.891361in}{2.495737in}}%
\pgfpathlineto{\pgfqpoint{1.893867in}{2.499855in}}%
\pgfpathlineto{\pgfqpoint{1.903894in}{2.503868in}}%
\pgfpathlineto{\pgfqpoint{1.906401in}{2.511899in}}%
\pgfpathlineto{\pgfqpoint{1.911414in}{2.513097in}}%
\pgfpathlineto{\pgfqpoint{1.918934in}{2.516707in}}%
\pgfpathlineto{\pgfqpoint{1.921441in}{2.532266in}}%
\pgfpathlineto{\pgfqpoint{1.923947in}{2.533797in}}%
\pgfpathlineto{\pgfqpoint{1.926454in}{2.539142in}}%
\pgfpathlineto{\pgfqpoint{1.928961in}{2.548540in}}%
\pgfpathlineto{\pgfqpoint{1.931467in}{2.570034in}}%
\pgfpathlineto{\pgfqpoint{1.933974in}{2.570440in}}%
\pgfpathlineto{\pgfqpoint{1.936481in}{2.594434in}}%
\pgfpathlineto{\pgfqpoint{1.938987in}{2.595104in}}%
\pgfpathlineto{\pgfqpoint{1.944000in}{2.608519in}}%
\pgfpathlineto{\pgfqpoint{1.946507in}{2.623749in}}%
\pgfpathlineto{\pgfqpoint{1.949014in}{2.627332in}}%
\pgfpathlineto{\pgfqpoint{1.951520in}{2.628429in}}%
\pgfpathlineto{\pgfqpoint{1.954027in}{2.631220in}}%
\pgfpathlineto{\pgfqpoint{1.956534in}{2.631228in}}%
\pgfpathlineto{\pgfqpoint{1.959040in}{2.633945in}}%
\pgfpathlineto{\pgfqpoint{1.964054in}{2.651827in}}%
\pgfpathlineto{\pgfqpoint{1.969067in}{2.653282in}}%
\pgfpathlineto{\pgfqpoint{1.971574in}{2.653948in}}%
\pgfpathlineto{\pgfqpoint{1.974080in}{2.661230in}}%
\pgfpathlineto{\pgfqpoint{1.976587in}{2.662284in}}%
\pgfpathlineto{\pgfqpoint{1.979094in}{2.675758in}}%
\pgfpathlineto{\pgfqpoint{1.981600in}{2.678473in}}%
\pgfpathlineto{\pgfqpoint{1.984107in}{2.714381in}}%
\pgfpathlineto{\pgfqpoint{1.989120in}{2.716416in}}%
\pgfpathlineto{\pgfqpoint{1.991627in}{2.721500in}}%
\pgfpathlineto{\pgfqpoint{1.994134in}{2.736778in}}%
\pgfpathlineto{\pgfqpoint{1.996640in}{2.744736in}}%
\pgfpathlineto{\pgfqpoint{1.999147in}{2.744869in}}%
\pgfpathlineto{\pgfqpoint{2.004160in}{2.762064in}}%
\pgfpathlineto{\pgfqpoint{2.006667in}{2.765827in}}%
\pgfpathlineto{\pgfqpoint{2.009173in}{2.766511in}}%
\pgfpathlineto{\pgfqpoint{2.011680in}{2.776054in}}%
\pgfpathlineto{\pgfqpoint{2.014187in}{2.905275in}}%
\pgfpathlineto{\pgfqpoint{2.014187in}{2.905275in}}%
\pgfusepath{stroke}%
\end{pgfscope}%
\begin{pgfscope}%
\pgfpathrectangle{\pgfqpoint{0.708220in}{0.535823in}}{\pgfqpoint{5.013309in}{2.369453in}}%
\pgfusepath{clip}%
\pgfsetrectcap%
\pgfsetroundjoin%
\pgfsetlinewidth{1.003750pt}%
\definecolor{currentstroke}{rgb}{0.000000,0.501961,0.000000}%
\pgfsetstrokecolor{currentstroke}%
\pgfsetdash{}{0pt}%
\pgfpathmoveto{\pgfqpoint{0.708220in}{0.869067in}}%
\pgfpathlineto{\pgfqpoint{0.710727in}{0.874997in}}%
\pgfpathlineto{\pgfqpoint{0.713233in}{0.893452in}}%
\pgfpathlineto{\pgfqpoint{0.715740in}{0.896565in}}%
\pgfpathlineto{\pgfqpoint{0.720753in}{0.910371in}}%
\pgfpathlineto{\pgfqpoint{0.723260in}{0.912321in}}%
\pgfpathlineto{\pgfqpoint{0.725766in}{0.916264in}}%
\pgfpathlineto{\pgfqpoint{0.728273in}{0.924663in}}%
\pgfpathlineto{\pgfqpoint{0.730780in}{0.926261in}}%
\pgfpathlineto{\pgfqpoint{0.733286in}{0.930437in}}%
\pgfpathlineto{\pgfqpoint{0.738300in}{0.944771in}}%
\pgfpathlineto{\pgfqpoint{0.743313in}{0.946876in}}%
\pgfpathlineto{\pgfqpoint{0.758353in}{0.960874in}}%
\pgfpathlineto{\pgfqpoint{0.760860in}{0.974953in}}%
\pgfpathlineto{\pgfqpoint{0.763366in}{0.980307in}}%
\pgfpathlineto{\pgfqpoint{0.765873in}{0.989666in}}%
\pgfpathlineto{\pgfqpoint{0.768380in}{0.991759in}}%
\pgfpathlineto{\pgfqpoint{0.773393in}{1.006859in}}%
\pgfpathlineto{\pgfqpoint{0.775900in}{1.012072in}}%
\pgfpathlineto{\pgfqpoint{0.778406in}{1.030889in}}%
\pgfpathlineto{\pgfqpoint{0.783419in}{1.042250in}}%
\pgfpathlineto{\pgfqpoint{0.785926in}{1.043159in}}%
\pgfpathlineto{\pgfqpoint{0.788433in}{1.049338in}}%
\pgfpathlineto{\pgfqpoint{0.790939in}{1.049962in}}%
\pgfpathlineto{\pgfqpoint{0.793446in}{1.058141in}}%
\pgfpathlineto{\pgfqpoint{0.798459in}{1.061867in}}%
\pgfpathlineto{\pgfqpoint{0.803473in}{1.063786in}}%
\pgfpathlineto{\pgfqpoint{0.805979in}{1.068109in}}%
\pgfpathlineto{\pgfqpoint{0.808486in}{1.069153in}}%
\pgfpathlineto{\pgfqpoint{0.810993in}{1.071738in}}%
\pgfpathlineto{\pgfqpoint{0.816006in}{1.079808in}}%
\pgfpathlineto{\pgfqpoint{0.823526in}{1.082041in}}%
\pgfpathlineto{\pgfqpoint{0.826033in}{1.084422in}}%
\pgfpathlineto{\pgfqpoint{0.828539in}{1.093120in}}%
\pgfpathlineto{\pgfqpoint{0.831046in}{1.096627in}}%
\pgfpathlineto{\pgfqpoint{0.833553in}{1.097600in}}%
\pgfpathlineto{\pgfqpoint{0.836059in}{1.100105in}}%
\pgfpathlineto{\pgfqpoint{0.838566in}{1.104478in}}%
\pgfpathlineto{\pgfqpoint{0.841073in}{1.118036in}}%
\pgfpathlineto{\pgfqpoint{0.843579in}{1.118232in}}%
\pgfpathlineto{\pgfqpoint{0.846086in}{1.119854in}}%
\pgfpathlineto{\pgfqpoint{0.851099in}{1.135212in}}%
\pgfpathlineto{\pgfqpoint{0.853606in}{1.142912in}}%
\pgfpathlineto{\pgfqpoint{0.858619in}{1.145337in}}%
\pgfpathlineto{\pgfqpoint{0.861126in}{1.153271in}}%
\pgfpathlineto{\pgfqpoint{0.863632in}{1.155284in}}%
\pgfpathlineto{\pgfqpoint{0.866139in}{1.155476in}}%
\pgfpathlineto{\pgfqpoint{0.873659in}{1.167226in}}%
\pgfpathlineto{\pgfqpoint{0.878672in}{1.172732in}}%
\pgfpathlineto{\pgfqpoint{0.881179in}{1.179161in}}%
\pgfpathlineto{\pgfqpoint{0.886192in}{1.180712in}}%
\pgfpathlineto{\pgfqpoint{0.888699in}{1.181625in}}%
\pgfpathlineto{\pgfqpoint{0.891206in}{1.186049in}}%
\pgfpathlineto{\pgfqpoint{0.893712in}{1.186600in}}%
\pgfpathlineto{\pgfqpoint{0.898726in}{1.193088in}}%
\pgfpathlineto{\pgfqpoint{0.901232in}{1.195838in}}%
\pgfpathlineto{\pgfqpoint{0.906246in}{1.204113in}}%
\pgfpathlineto{\pgfqpoint{0.908752in}{1.207774in}}%
\pgfpathlineto{\pgfqpoint{0.911259in}{1.214061in}}%
\pgfpathlineto{\pgfqpoint{0.916272in}{1.217766in}}%
\pgfpathlineto{\pgfqpoint{0.918779in}{1.223298in}}%
\pgfpathlineto{\pgfqpoint{0.921285in}{1.224991in}}%
\pgfpathlineto{\pgfqpoint{0.923792in}{1.236166in}}%
\pgfpathlineto{\pgfqpoint{0.926299in}{1.236684in}}%
\pgfpathlineto{\pgfqpoint{0.928805in}{1.245265in}}%
\pgfpathlineto{\pgfqpoint{0.941339in}{1.256502in}}%
\pgfpathlineto{\pgfqpoint{0.946352in}{1.267532in}}%
\pgfpathlineto{\pgfqpoint{0.948859in}{1.270636in}}%
\pgfpathlineto{\pgfqpoint{0.953872in}{1.271169in}}%
\pgfpathlineto{\pgfqpoint{0.958885in}{1.277881in}}%
\pgfpathlineto{\pgfqpoint{0.961392in}{1.279268in}}%
\pgfpathlineto{\pgfqpoint{0.963899in}{1.283096in}}%
\pgfpathlineto{\pgfqpoint{0.968912in}{1.285341in}}%
\pgfpathlineto{\pgfqpoint{0.971419in}{1.288807in}}%
\pgfpathlineto{\pgfqpoint{0.973925in}{1.290320in}}%
\pgfpathlineto{\pgfqpoint{0.976432in}{1.290352in}}%
\pgfpathlineto{\pgfqpoint{0.978939in}{1.293022in}}%
\pgfpathlineto{\pgfqpoint{0.983952in}{1.294090in}}%
\pgfpathlineto{\pgfqpoint{0.986458in}{1.295900in}}%
\pgfpathlineto{\pgfqpoint{0.988965in}{1.295995in}}%
\pgfpathlineto{\pgfqpoint{0.993978in}{1.299106in}}%
\pgfpathlineto{\pgfqpoint{0.996485in}{1.304381in}}%
\pgfpathlineto{\pgfqpoint{1.006512in}{1.307796in}}%
\pgfpathlineto{\pgfqpoint{1.009018in}{1.310907in}}%
\pgfpathlineto{\pgfqpoint{1.011525in}{1.311193in}}%
\pgfpathlineto{\pgfqpoint{1.014032in}{1.315163in}}%
\pgfpathlineto{\pgfqpoint{1.016538in}{1.316136in}}%
\pgfpathlineto{\pgfqpoint{1.021552in}{1.323800in}}%
\pgfpathlineto{\pgfqpoint{1.031578in}{1.331397in}}%
\pgfpathlineto{\pgfqpoint{1.034085in}{1.332116in}}%
\pgfpathlineto{\pgfqpoint{1.036592in}{1.336788in}}%
\pgfpathlineto{\pgfqpoint{1.039098in}{1.337331in}}%
\pgfpathlineto{\pgfqpoint{1.046618in}{1.343556in}}%
\pgfpathlineto{\pgfqpoint{1.049125in}{1.343567in}}%
\pgfpathlineto{\pgfqpoint{1.051632in}{1.347684in}}%
\pgfpathlineto{\pgfqpoint{1.056645in}{1.350584in}}%
\pgfpathlineto{\pgfqpoint{1.061658in}{1.351532in}}%
\pgfpathlineto{\pgfqpoint{1.066671in}{1.357187in}}%
\pgfpathlineto{\pgfqpoint{1.069178in}{1.361316in}}%
\pgfpathlineto{\pgfqpoint{1.071685in}{1.361804in}}%
\pgfpathlineto{\pgfqpoint{1.074191in}{1.366668in}}%
\pgfpathlineto{\pgfqpoint{1.081711in}{1.368280in}}%
\pgfpathlineto{\pgfqpoint{1.086725in}{1.373265in}}%
\pgfpathlineto{\pgfqpoint{1.089231in}{1.373391in}}%
\pgfpathlineto{\pgfqpoint{1.091738in}{1.380072in}}%
\pgfpathlineto{\pgfqpoint{1.094245in}{1.382734in}}%
\pgfpathlineto{\pgfqpoint{1.099258in}{1.383237in}}%
\pgfpathlineto{\pgfqpoint{1.104271in}{1.385597in}}%
\pgfpathlineto{\pgfqpoint{1.109285in}{1.386521in}}%
\pgfpathlineto{\pgfqpoint{1.111791in}{1.389446in}}%
\pgfpathlineto{\pgfqpoint{1.114298in}{1.395269in}}%
\pgfpathlineto{\pgfqpoint{1.116805in}{1.395300in}}%
\pgfpathlineto{\pgfqpoint{1.119311in}{1.397599in}}%
\pgfpathlineto{\pgfqpoint{1.121818in}{1.398234in}}%
\pgfpathlineto{\pgfqpoint{1.124324in}{1.403045in}}%
\pgfpathlineto{\pgfqpoint{1.129338in}{1.403832in}}%
\pgfpathlineto{\pgfqpoint{1.134351in}{1.411791in}}%
\pgfpathlineto{\pgfqpoint{1.136858in}{1.413412in}}%
\pgfpathlineto{\pgfqpoint{1.139364in}{1.416435in}}%
\pgfpathlineto{\pgfqpoint{1.141871in}{1.416470in}}%
\pgfpathlineto{\pgfqpoint{1.149391in}{1.423271in}}%
\pgfpathlineto{\pgfqpoint{1.164431in}{1.426351in}}%
\pgfpathlineto{\pgfqpoint{1.169444in}{1.430758in}}%
\pgfpathlineto{\pgfqpoint{1.171951in}{1.432120in}}%
\pgfpathlineto{\pgfqpoint{1.179471in}{1.442873in}}%
\pgfpathlineto{\pgfqpoint{1.192004in}{1.446207in}}%
\pgfpathlineto{\pgfqpoint{1.199524in}{1.449723in}}%
\pgfpathlineto{\pgfqpoint{1.202031in}{1.457232in}}%
\pgfpathlineto{\pgfqpoint{1.207044in}{1.459542in}}%
\pgfpathlineto{\pgfqpoint{1.209551in}{1.463355in}}%
\pgfpathlineto{\pgfqpoint{1.212057in}{1.463380in}}%
\pgfpathlineto{\pgfqpoint{1.214564in}{1.468703in}}%
\pgfpathlineto{\pgfqpoint{1.217071in}{1.470698in}}%
\pgfpathlineto{\pgfqpoint{1.219577in}{1.470994in}}%
\pgfpathlineto{\pgfqpoint{1.222084in}{1.473044in}}%
\pgfpathlineto{\pgfqpoint{1.224591in}{1.473344in}}%
\pgfpathlineto{\pgfqpoint{1.227097in}{1.477361in}}%
\pgfpathlineto{\pgfqpoint{1.229604in}{1.478078in}}%
\pgfpathlineto{\pgfqpoint{1.232111in}{1.486980in}}%
\pgfpathlineto{\pgfqpoint{1.234617in}{1.487079in}}%
\pgfpathlineto{\pgfqpoint{1.239631in}{1.492632in}}%
\pgfpathlineto{\pgfqpoint{1.249657in}{1.495144in}}%
\pgfpathlineto{\pgfqpoint{1.254671in}{1.499696in}}%
\pgfpathlineto{\pgfqpoint{1.262190in}{1.502625in}}%
\pgfpathlineto{\pgfqpoint{1.269710in}{1.504836in}}%
\pgfpathlineto{\pgfqpoint{1.274724in}{1.506810in}}%
\pgfpathlineto{\pgfqpoint{1.279737in}{1.507945in}}%
\pgfpathlineto{\pgfqpoint{1.282244in}{1.509860in}}%
\pgfpathlineto{\pgfqpoint{1.287257in}{1.511628in}}%
\pgfpathlineto{\pgfqpoint{1.292270in}{1.517040in}}%
\pgfpathlineto{\pgfqpoint{1.294777in}{1.521579in}}%
\pgfpathlineto{\pgfqpoint{1.304804in}{1.525064in}}%
\pgfpathlineto{\pgfqpoint{1.307310in}{1.529426in}}%
\pgfpathlineto{\pgfqpoint{1.309817in}{1.531205in}}%
\pgfpathlineto{\pgfqpoint{1.312324in}{1.531242in}}%
\pgfpathlineto{\pgfqpoint{1.314830in}{1.533943in}}%
\pgfpathlineto{\pgfqpoint{1.317337in}{1.534033in}}%
\pgfpathlineto{\pgfqpoint{1.322350in}{1.538570in}}%
\pgfpathlineto{\pgfqpoint{1.327363in}{1.546887in}}%
\pgfpathlineto{\pgfqpoint{1.332377in}{1.551309in}}%
\pgfpathlineto{\pgfqpoint{1.334883in}{1.551746in}}%
\pgfpathlineto{\pgfqpoint{1.339897in}{1.556063in}}%
\pgfpathlineto{\pgfqpoint{1.342403in}{1.561822in}}%
\pgfpathlineto{\pgfqpoint{1.357443in}{1.571538in}}%
\pgfpathlineto{\pgfqpoint{1.359950in}{1.574504in}}%
\pgfpathlineto{\pgfqpoint{1.369977in}{1.578208in}}%
\pgfpathlineto{\pgfqpoint{1.372483in}{1.581309in}}%
\pgfpathlineto{\pgfqpoint{1.387523in}{1.587045in}}%
\pgfpathlineto{\pgfqpoint{1.392537in}{1.589821in}}%
\pgfpathlineto{\pgfqpoint{1.395043in}{1.591073in}}%
\pgfpathlineto{\pgfqpoint{1.397550in}{1.594014in}}%
\pgfpathlineto{\pgfqpoint{1.400056in}{1.594375in}}%
\pgfpathlineto{\pgfqpoint{1.402563in}{1.597741in}}%
\pgfpathlineto{\pgfqpoint{1.407576in}{1.607609in}}%
\pgfpathlineto{\pgfqpoint{1.410083in}{1.612145in}}%
\pgfpathlineto{\pgfqpoint{1.432643in}{1.617360in}}%
\pgfpathlineto{\pgfqpoint{1.437656in}{1.619283in}}%
\pgfpathlineto{\pgfqpoint{1.442670in}{1.621209in}}%
\pgfpathlineto{\pgfqpoint{1.445176in}{1.621297in}}%
\pgfpathlineto{\pgfqpoint{1.450190in}{1.625592in}}%
\pgfpathlineto{\pgfqpoint{1.457710in}{1.630992in}}%
\pgfpathlineto{\pgfqpoint{1.462723in}{1.631893in}}%
\pgfpathlineto{\pgfqpoint{1.475256in}{1.637009in}}%
\pgfpathlineto{\pgfqpoint{1.477763in}{1.643268in}}%
\pgfpathlineto{\pgfqpoint{1.482776in}{1.645459in}}%
\pgfpathlineto{\pgfqpoint{1.487789in}{1.660672in}}%
\pgfpathlineto{\pgfqpoint{1.490296in}{1.660760in}}%
\pgfpathlineto{\pgfqpoint{1.492803in}{1.663629in}}%
\pgfpathlineto{\pgfqpoint{1.500323in}{1.666072in}}%
\pgfpathlineto{\pgfqpoint{1.502829in}{1.670132in}}%
\pgfpathlineto{\pgfqpoint{1.505336in}{1.670398in}}%
\pgfpathlineto{\pgfqpoint{1.507843in}{1.672456in}}%
\pgfpathlineto{\pgfqpoint{1.517869in}{1.673142in}}%
\pgfpathlineto{\pgfqpoint{1.522883in}{1.679381in}}%
\pgfpathlineto{\pgfqpoint{1.530402in}{1.685577in}}%
\pgfpathlineto{\pgfqpoint{1.540429in}{1.690805in}}%
\pgfpathlineto{\pgfqpoint{1.542936in}{1.694974in}}%
\pgfpathlineto{\pgfqpoint{1.547949in}{1.705907in}}%
\pgfpathlineto{\pgfqpoint{1.552962in}{1.710529in}}%
\pgfpathlineto{\pgfqpoint{1.557976in}{1.712537in}}%
\pgfpathlineto{\pgfqpoint{1.560482in}{1.714419in}}%
\pgfpathlineto{\pgfqpoint{1.562989in}{1.714518in}}%
\pgfpathlineto{\pgfqpoint{1.565496in}{1.720369in}}%
\pgfpathlineto{\pgfqpoint{1.568002in}{1.721325in}}%
\pgfpathlineto{\pgfqpoint{1.570509in}{1.724840in}}%
\pgfpathlineto{\pgfqpoint{1.580536in}{1.730252in}}%
\pgfpathlineto{\pgfqpoint{1.590562in}{1.732111in}}%
\pgfpathlineto{\pgfqpoint{1.593069in}{1.734569in}}%
\pgfpathlineto{\pgfqpoint{1.605602in}{1.738410in}}%
\pgfpathlineto{\pgfqpoint{1.608109in}{1.750951in}}%
\pgfpathlineto{\pgfqpoint{1.610615in}{1.753290in}}%
\pgfpathlineto{\pgfqpoint{1.613122in}{1.753891in}}%
\pgfpathlineto{\pgfqpoint{1.615629in}{1.757145in}}%
\pgfpathlineto{\pgfqpoint{1.618135in}{1.762507in}}%
\pgfpathlineto{\pgfqpoint{1.620642in}{1.762519in}}%
\pgfpathlineto{\pgfqpoint{1.623149in}{1.770595in}}%
\pgfpathlineto{\pgfqpoint{1.635682in}{1.775683in}}%
\pgfpathlineto{\pgfqpoint{1.638189in}{1.780630in}}%
\pgfpathlineto{\pgfqpoint{1.648215in}{1.782095in}}%
\pgfpathlineto{\pgfqpoint{1.650722in}{1.795075in}}%
\pgfpathlineto{\pgfqpoint{1.653229in}{1.795929in}}%
\pgfpathlineto{\pgfqpoint{1.658242in}{1.810597in}}%
\pgfpathlineto{\pgfqpoint{1.660749in}{1.812025in}}%
\pgfpathlineto{\pgfqpoint{1.663255in}{1.815064in}}%
\pgfpathlineto{\pgfqpoint{1.665762in}{1.820887in}}%
\pgfpathlineto{\pgfqpoint{1.668268in}{1.821260in}}%
\pgfpathlineto{\pgfqpoint{1.675788in}{1.825606in}}%
\pgfpathlineto{\pgfqpoint{1.678295in}{1.825845in}}%
\pgfpathlineto{\pgfqpoint{1.680802in}{1.828704in}}%
\pgfpathlineto{\pgfqpoint{1.685815in}{1.829839in}}%
\pgfpathlineto{\pgfqpoint{1.688322in}{1.838138in}}%
\pgfpathlineto{\pgfqpoint{1.695842in}{1.843489in}}%
\pgfpathlineto{\pgfqpoint{1.703362in}{1.846978in}}%
\pgfpathlineto{\pgfqpoint{1.710882in}{1.848247in}}%
\pgfpathlineto{\pgfqpoint{1.713388in}{1.851228in}}%
\pgfpathlineto{\pgfqpoint{1.715895in}{1.851623in}}%
\pgfpathlineto{\pgfqpoint{1.720908in}{1.854645in}}%
\pgfpathlineto{\pgfqpoint{1.725922in}{1.856017in}}%
\pgfpathlineto{\pgfqpoint{1.728428in}{1.859594in}}%
\pgfpathlineto{\pgfqpoint{1.730935in}{1.859648in}}%
\pgfpathlineto{\pgfqpoint{1.735948in}{1.861769in}}%
\pgfpathlineto{\pgfqpoint{1.738455in}{1.862082in}}%
\pgfpathlineto{\pgfqpoint{1.740961in}{1.864281in}}%
\pgfpathlineto{\pgfqpoint{1.745975in}{1.864535in}}%
\pgfpathlineto{\pgfqpoint{1.753495in}{1.868289in}}%
\pgfpathlineto{\pgfqpoint{1.761015in}{1.869872in}}%
\pgfpathlineto{\pgfqpoint{1.781068in}{1.874001in}}%
\pgfpathlineto{\pgfqpoint{1.783575in}{1.876470in}}%
\pgfpathlineto{\pgfqpoint{1.796108in}{1.878003in}}%
\pgfpathlineto{\pgfqpoint{1.813654in}{1.882462in}}%
\pgfpathlineto{\pgfqpoint{1.816161in}{1.896960in}}%
\pgfpathlineto{\pgfqpoint{1.823681in}{1.900139in}}%
\pgfpathlineto{\pgfqpoint{1.828694in}{1.900683in}}%
\pgfpathlineto{\pgfqpoint{1.838721in}{1.904371in}}%
\pgfpathlineto{\pgfqpoint{1.841228in}{1.904731in}}%
\pgfpathlineto{\pgfqpoint{1.843734in}{1.906401in}}%
\pgfpathlineto{\pgfqpoint{1.848748in}{1.906615in}}%
\pgfpathlineto{\pgfqpoint{1.851254in}{1.910249in}}%
\pgfpathlineto{\pgfqpoint{1.856268in}{1.911216in}}%
\pgfpathlineto{\pgfqpoint{1.863788in}{1.912083in}}%
\pgfpathlineto{\pgfqpoint{1.868801in}{1.913925in}}%
\pgfpathlineto{\pgfqpoint{1.876321in}{1.915371in}}%
\pgfpathlineto{\pgfqpoint{1.878827in}{1.917734in}}%
\pgfpathlineto{\pgfqpoint{1.886347in}{1.918954in}}%
\pgfpathlineto{\pgfqpoint{1.893867in}{1.920869in}}%
\pgfpathlineto{\pgfqpoint{1.901387in}{1.922084in}}%
\pgfpathlineto{\pgfqpoint{1.908907in}{1.926305in}}%
\pgfpathlineto{\pgfqpoint{1.911414in}{1.926748in}}%
\pgfpathlineto{\pgfqpoint{1.918934in}{1.933039in}}%
\pgfpathlineto{\pgfqpoint{1.923947in}{1.933574in}}%
\pgfpathlineto{\pgfqpoint{1.933974in}{1.937123in}}%
\pgfpathlineto{\pgfqpoint{1.936481in}{1.937308in}}%
\pgfpathlineto{\pgfqpoint{1.938987in}{1.938668in}}%
\pgfpathlineto{\pgfqpoint{1.941494in}{1.942087in}}%
\pgfpathlineto{\pgfqpoint{1.944000in}{1.949059in}}%
\pgfpathlineto{\pgfqpoint{1.951520in}{1.951923in}}%
\pgfpathlineto{\pgfqpoint{1.966560in}{1.957751in}}%
\pgfpathlineto{\pgfqpoint{1.969067in}{1.961504in}}%
\pgfpathlineto{\pgfqpoint{1.971574in}{1.961582in}}%
\pgfpathlineto{\pgfqpoint{1.981600in}{1.968676in}}%
\pgfpathlineto{\pgfqpoint{1.984107in}{1.981976in}}%
\pgfpathlineto{\pgfqpoint{1.986614in}{1.985786in}}%
\pgfpathlineto{\pgfqpoint{1.989120in}{1.985819in}}%
\pgfpathlineto{\pgfqpoint{1.994134in}{1.989228in}}%
\pgfpathlineto{\pgfqpoint{2.016693in}{1.995786in}}%
\pgfpathlineto{\pgfqpoint{2.021707in}{2.000299in}}%
\pgfpathlineto{\pgfqpoint{2.029227in}{2.005747in}}%
\pgfpathlineto{\pgfqpoint{2.036747in}{2.007584in}}%
\pgfpathlineto{\pgfqpoint{2.044267in}{2.008901in}}%
\pgfpathlineto{\pgfqpoint{2.054293in}{2.010458in}}%
\pgfpathlineto{\pgfqpoint{2.056800in}{2.011445in}}%
\pgfpathlineto{\pgfqpoint{2.059307in}{2.015936in}}%
\pgfpathlineto{\pgfqpoint{2.064320in}{2.017075in}}%
\pgfpathlineto{\pgfqpoint{2.066827in}{2.020332in}}%
\pgfpathlineto{\pgfqpoint{2.069333in}{2.021077in}}%
\pgfpathlineto{\pgfqpoint{2.071840in}{2.023206in}}%
\pgfpathlineto{\pgfqpoint{2.074346in}{2.028200in}}%
\pgfpathlineto{\pgfqpoint{2.079360in}{2.028755in}}%
\pgfpathlineto{\pgfqpoint{2.081866in}{2.030894in}}%
\pgfpathlineto{\pgfqpoint{2.086880in}{2.038609in}}%
\pgfpathlineto{\pgfqpoint{2.091893in}{2.041450in}}%
\pgfpathlineto{\pgfqpoint{2.096906in}{2.043445in}}%
\pgfpathlineto{\pgfqpoint{2.099413in}{2.044730in}}%
\pgfpathlineto{\pgfqpoint{2.106933in}{2.051784in}}%
\pgfpathlineto{\pgfqpoint{2.111946in}{2.052561in}}%
\pgfpathlineto{\pgfqpoint{2.116960in}{2.058127in}}%
\pgfpathlineto{\pgfqpoint{2.119466in}{2.058500in}}%
\pgfpathlineto{\pgfqpoint{2.121973in}{2.060444in}}%
\pgfpathlineto{\pgfqpoint{2.124480in}{2.060764in}}%
\pgfpathlineto{\pgfqpoint{2.126986in}{2.062499in}}%
\pgfpathlineto{\pgfqpoint{2.129493in}{2.067672in}}%
\pgfpathlineto{\pgfqpoint{2.134506in}{2.073237in}}%
\pgfpathlineto{\pgfqpoint{2.137013in}{2.079967in}}%
\pgfpathlineto{\pgfqpoint{2.139520in}{2.080004in}}%
\pgfpathlineto{\pgfqpoint{2.142026in}{2.085935in}}%
\pgfpathlineto{\pgfqpoint{2.144533in}{2.086154in}}%
\pgfpathlineto{\pgfqpoint{2.147039in}{2.088373in}}%
\pgfpathlineto{\pgfqpoint{2.149546in}{2.095235in}}%
\pgfpathlineto{\pgfqpoint{2.164586in}{2.107960in}}%
\pgfpathlineto{\pgfqpoint{2.167093in}{2.107970in}}%
\pgfpathlineto{\pgfqpoint{2.172106in}{2.115657in}}%
\pgfpathlineto{\pgfqpoint{2.174613in}{2.116645in}}%
\pgfpathlineto{\pgfqpoint{2.182133in}{2.123577in}}%
\pgfpathlineto{\pgfqpoint{2.184639in}{2.125294in}}%
\pgfpathlineto{\pgfqpoint{2.187146in}{2.134459in}}%
\pgfpathlineto{\pgfqpoint{2.189653in}{2.137728in}}%
\pgfpathlineto{\pgfqpoint{2.202186in}{2.141917in}}%
\pgfpathlineto{\pgfqpoint{2.204693in}{2.146167in}}%
\pgfpathlineto{\pgfqpoint{2.209706in}{2.147921in}}%
\pgfpathlineto{\pgfqpoint{2.217226in}{2.149694in}}%
\pgfpathlineto{\pgfqpoint{2.219732in}{2.154639in}}%
\pgfpathlineto{\pgfqpoint{2.222239in}{2.154692in}}%
\pgfpathlineto{\pgfqpoint{2.227252in}{2.159212in}}%
\pgfpathlineto{\pgfqpoint{2.229759in}{2.159608in}}%
\pgfpathlineto{\pgfqpoint{2.232266in}{2.162329in}}%
\pgfpathlineto{\pgfqpoint{2.234772in}{2.173325in}}%
\pgfpathlineto{\pgfqpoint{2.242292in}{2.180182in}}%
\pgfpathlineto{\pgfqpoint{2.244799in}{2.184166in}}%
\pgfpathlineto{\pgfqpoint{2.249812in}{2.185161in}}%
\pgfpathlineto{\pgfqpoint{2.257332in}{2.192847in}}%
\pgfpathlineto{\pgfqpoint{2.259839in}{2.193104in}}%
\pgfpathlineto{\pgfqpoint{2.262346in}{2.196018in}}%
\pgfpathlineto{\pgfqpoint{2.267359in}{2.197384in}}%
\pgfpathlineto{\pgfqpoint{2.269866in}{2.199815in}}%
\pgfpathlineto{\pgfqpoint{2.272372in}{2.199898in}}%
\pgfpathlineto{\pgfqpoint{2.274879in}{2.202270in}}%
\pgfpathlineto{\pgfqpoint{2.277385in}{2.202500in}}%
\pgfpathlineto{\pgfqpoint{2.282399in}{2.207172in}}%
\pgfpathlineto{\pgfqpoint{2.284905in}{2.214069in}}%
\pgfpathlineto{\pgfqpoint{2.287412in}{2.216474in}}%
\pgfpathlineto{\pgfqpoint{2.289919in}{2.217144in}}%
\pgfpathlineto{\pgfqpoint{2.292425in}{2.219644in}}%
\pgfpathlineto{\pgfqpoint{2.294932in}{2.228552in}}%
\pgfpathlineto{\pgfqpoint{2.297439in}{2.229736in}}%
\pgfpathlineto{\pgfqpoint{2.299945in}{2.232201in}}%
\pgfpathlineto{\pgfqpoint{2.302452in}{2.233046in}}%
\pgfpathlineto{\pgfqpoint{2.309972in}{2.243379in}}%
\pgfpathlineto{\pgfqpoint{2.317492in}{2.244228in}}%
\pgfpathlineto{\pgfqpoint{2.327519in}{2.251602in}}%
\pgfpathlineto{\pgfqpoint{2.330025in}{2.255174in}}%
\pgfpathlineto{\pgfqpoint{2.335039in}{2.256299in}}%
\pgfpathlineto{\pgfqpoint{2.340052in}{2.264727in}}%
\pgfpathlineto{\pgfqpoint{2.342559in}{2.265641in}}%
\pgfpathlineto{\pgfqpoint{2.345065in}{2.270409in}}%
\pgfpathlineto{\pgfqpoint{2.347572in}{2.270620in}}%
\pgfpathlineto{\pgfqpoint{2.350078in}{2.272256in}}%
\pgfpathlineto{\pgfqpoint{2.355092in}{2.287851in}}%
\pgfpathlineto{\pgfqpoint{2.357598in}{2.287885in}}%
\pgfpathlineto{\pgfqpoint{2.362612in}{2.289915in}}%
\pgfpathlineto{\pgfqpoint{2.365118in}{2.290081in}}%
\pgfpathlineto{\pgfqpoint{2.367625in}{2.297557in}}%
\pgfpathlineto{\pgfqpoint{2.375145in}{2.302936in}}%
\pgfpathlineto{\pgfqpoint{2.380158in}{2.305142in}}%
\pgfpathlineto{\pgfqpoint{2.382665in}{2.310523in}}%
\pgfpathlineto{\pgfqpoint{2.387678in}{2.311651in}}%
\pgfpathlineto{\pgfqpoint{2.390185in}{2.314545in}}%
\pgfpathlineto{\pgfqpoint{2.392692in}{2.319671in}}%
\pgfpathlineto{\pgfqpoint{2.397705in}{2.320867in}}%
\pgfpathlineto{\pgfqpoint{2.400212in}{2.329192in}}%
\pgfpathlineto{\pgfqpoint{2.407732in}{2.338144in}}%
\pgfpathlineto{\pgfqpoint{2.412745in}{2.340104in}}%
\pgfpathlineto{\pgfqpoint{2.415251in}{2.341828in}}%
\pgfpathlineto{\pgfqpoint{2.420265in}{2.348547in}}%
\pgfpathlineto{\pgfqpoint{2.422771in}{2.348726in}}%
\pgfpathlineto{\pgfqpoint{2.425278in}{2.350231in}}%
\pgfpathlineto{\pgfqpoint{2.427785in}{2.354038in}}%
\pgfpathlineto{\pgfqpoint{2.430291in}{2.355630in}}%
\pgfpathlineto{\pgfqpoint{2.432798in}{2.361826in}}%
\pgfpathlineto{\pgfqpoint{2.437811in}{2.362315in}}%
\pgfpathlineto{\pgfqpoint{2.440318in}{2.364449in}}%
\pgfpathlineto{\pgfqpoint{2.445331in}{2.366102in}}%
\pgfpathlineto{\pgfqpoint{2.450345in}{2.368344in}}%
\pgfpathlineto{\pgfqpoint{2.452851in}{2.368425in}}%
\pgfpathlineto{\pgfqpoint{2.460371in}{2.372008in}}%
\pgfpathlineto{\pgfqpoint{2.462878in}{2.372812in}}%
\pgfpathlineto{\pgfqpoint{2.467891in}{2.391113in}}%
\pgfpathlineto{\pgfqpoint{2.470398in}{2.391981in}}%
\pgfpathlineto{\pgfqpoint{2.472905in}{2.394310in}}%
\pgfpathlineto{\pgfqpoint{2.485438in}{2.396280in}}%
\pgfpathlineto{\pgfqpoint{2.487944in}{2.397247in}}%
\pgfpathlineto{\pgfqpoint{2.495464in}{2.403239in}}%
\pgfpathlineto{\pgfqpoint{2.497971in}{2.403579in}}%
\pgfpathlineto{\pgfqpoint{2.500478in}{2.409311in}}%
\pgfpathlineto{\pgfqpoint{2.502984in}{2.409899in}}%
\pgfpathlineto{\pgfqpoint{2.505491in}{2.415567in}}%
\pgfpathlineto{\pgfqpoint{2.507998in}{2.415594in}}%
\pgfpathlineto{\pgfqpoint{2.510504in}{2.417552in}}%
\pgfpathlineto{\pgfqpoint{2.515518in}{2.427335in}}%
\pgfpathlineto{\pgfqpoint{2.530558in}{2.433669in}}%
\pgfpathlineto{\pgfqpoint{2.533064in}{2.434077in}}%
\pgfpathlineto{\pgfqpoint{2.535571in}{2.437040in}}%
\pgfpathlineto{\pgfqpoint{2.538078in}{2.438280in}}%
\pgfpathlineto{\pgfqpoint{2.540584in}{2.445436in}}%
\pgfpathlineto{\pgfqpoint{2.545598in}{2.446788in}}%
\pgfpathlineto{\pgfqpoint{2.548104in}{2.457904in}}%
\pgfpathlineto{\pgfqpoint{2.550611in}{2.459081in}}%
\pgfpathlineto{\pgfqpoint{2.553117in}{2.473310in}}%
\pgfpathlineto{\pgfqpoint{2.560637in}{2.486574in}}%
\pgfpathlineto{\pgfqpoint{2.568157in}{2.489188in}}%
\pgfpathlineto{\pgfqpoint{2.570664in}{2.505697in}}%
\pgfpathlineto{\pgfqpoint{2.575677in}{2.510659in}}%
\pgfpathlineto{\pgfqpoint{2.580691in}{2.526295in}}%
\pgfpathlineto{\pgfqpoint{2.583197in}{2.533537in}}%
\pgfpathlineto{\pgfqpoint{2.585704in}{2.534044in}}%
\pgfpathlineto{\pgfqpoint{2.588211in}{2.561298in}}%
\pgfpathlineto{\pgfqpoint{2.590717in}{2.564550in}}%
\pgfpathlineto{\pgfqpoint{2.595731in}{2.565752in}}%
\pgfpathlineto{\pgfqpoint{2.600744in}{2.577498in}}%
\pgfpathlineto{\pgfqpoint{2.603251in}{2.577673in}}%
\pgfpathlineto{\pgfqpoint{2.605757in}{2.583529in}}%
\pgfpathlineto{\pgfqpoint{2.610771in}{2.584214in}}%
\pgfpathlineto{\pgfqpoint{2.615784in}{2.596206in}}%
\pgfpathlineto{\pgfqpoint{2.618290in}{2.603113in}}%
\pgfpathlineto{\pgfqpoint{2.620797in}{2.603609in}}%
\pgfpathlineto{\pgfqpoint{2.623304in}{2.613384in}}%
\pgfpathlineto{\pgfqpoint{2.625810in}{2.630363in}}%
\pgfpathlineto{\pgfqpoint{2.630824in}{2.634996in}}%
\pgfpathlineto{\pgfqpoint{2.638344in}{2.650243in}}%
\pgfpathlineto{\pgfqpoint{2.640850in}{2.651348in}}%
\pgfpathlineto{\pgfqpoint{2.645864in}{2.658237in}}%
\pgfpathlineto{\pgfqpoint{2.648370in}{2.661745in}}%
\pgfpathlineto{\pgfqpoint{2.650877in}{2.679879in}}%
\pgfpathlineto{\pgfqpoint{2.655890in}{2.691743in}}%
\pgfpathlineto{\pgfqpoint{2.658397in}{2.699832in}}%
\pgfpathlineto{\pgfqpoint{2.660904in}{2.704101in}}%
\pgfpathlineto{\pgfqpoint{2.663410in}{2.704940in}}%
\pgfpathlineto{\pgfqpoint{2.665917in}{2.708887in}}%
\pgfpathlineto{\pgfqpoint{2.668424in}{2.751254in}}%
\pgfpathlineto{\pgfqpoint{2.670930in}{2.772456in}}%
\pgfpathlineto{\pgfqpoint{2.678450in}{2.794979in}}%
\pgfpathlineto{\pgfqpoint{2.680957in}{2.821403in}}%
\pgfpathlineto{\pgfqpoint{2.685970in}{2.839959in}}%
\pgfpathlineto{\pgfqpoint{2.693490in}{2.905275in}}%
\pgfpathlineto{\pgfqpoint{2.693490in}{2.905275in}}%
\pgfusepath{stroke}%
\end{pgfscope}%
\begin{pgfscope}%
\pgfpathrectangle{\pgfqpoint{0.708220in}{0.535823in}}{\pgfqpoint{5.013309in}{2.369453in}}%
\pgfusepath{clip}%
\pgfsetrectcap%
\pgfsetroundjoin%
\pgfsetlinewidth{1.003750pt}%
\definecolor{currentstroke}{rgb}{0.000000,0.000000,0.000000}%
\pgfsetstrokecolor{currentstroke}%
\pgfsetdash{}{0pt}%
\pgfpathmoveto{\pgfqpoint{0.708220in}{0.869559in}}%
\pgfpathlineto{\pgfqpoint{0.710727in}{0.892461in}}%
\pgfpathlineto{\pgfqpoint{0.715740in}{0.892461in}}%
\pgfpathlineto{\pgfqpoint{0.718246in}{0.912661in}}%
\pgfpathlineto{\pgfqpoint{0.720753in}{0.912661in}}%
\pgfpathlineto{\pgfqpoint{0.725766in}{0.947078in}}%
\pgfpathlineto{\pgfqpoint{0.730780in}{0.947078in}}%
\pgfpathlineto{\pgfqpoint{0.733286in}{0.962001in}}%
\pgfpathlineto{\pgfqpoint{0.735793in}{0.962001in}}%
\pgfpathlineto{\pgfqpoint{0.738300in}{0.988439in}}%
\pgfpathlineto{\pgfqpoint{0.740806in}{0.988439in}}%
\pgfpathlineto{\pgfqpoint{0.745820in}{1.011340in}}%
\pgfpathlineto{\pgfqpoint{0.753340in}{1.011340in}}%
\pgfpathlineto{\pgfqpoint{0.755846in}{1.021738in}}%
\pgfpathlineto{\pgfqpoint{0.790939in}{1.021738in}}%
\pgfpathlineto{\pgfqpoint{0.793446in}{1.031541in}}%
\pgfpathlineto{\pgfqpoint{0.853606in}{1.031541in}}%
\pgfpathlineto{\pgfqpoint{0.856112in}{1.040814in}}%
\pgfpathlineto{\pgfqpoint{0.878672in}{1.040814in}}%
\pgfpathlineto{\pgfqpoint{0.881179in}{1.049611in}}%
\pgfpathlineto{\pgfqpoint{0.911259in}{1.049611in}}%
\pgfpathlineto{\pgfqpoint{0.913766in}{1.057979in}}%
\pgfpathlineto{\pgfqpoint{0.981445in}{1.057979in}}%
\pgfpathlineto{\pgfqpoint{0.983952in}{1.065957in}}%
\pgfpathlineto{\pgfqpoint{1.051632in}{1.065957in}}%
\pgfpathlineto{\pgfqpoint{1.054138in}{1.073581in}}%
\pgfpathlineto{\pgfqpoint{1.091738in}{1.073581in}}%
\pgfpathlineto{\pgfqpoint{1.094245in}{1.080880in}}%
\pgfpathlineto{\pgfqpoint{1.156911in}{1.080880in}}%
\pgfpathlineto{\pgfqpoint{1.159418in}{1.087881in}}%
\pgfpathlineto{\pgfqpoint{1.244644in}{1.087881in}}%
\pgfpathlineto{\pgfqpoint{1.247151in}{1.094608in}}%
\pgfpathlineto{\pgfqpoint{1.309817in}{1.094608in}}%
\pgfpathlineto{\pgfqpoint{1.312324in}{1.101081in}}%
\pgfpathlineto{\pgfqpoint{1.377497in}{1.101081in}}%
\pgfpathlineto{\pgfqpoint{1.380003in}{1.107318in}}%
\pgfpathlineto{\pgfqpoint{1.397550in}{1.107318in}}%
\pgfpathlineto{\pgfqpoint{1.400056in}{1.113336in}}%
\pgfpathlineto{\pgfqpoint{1.427630in}{1.113336in}}%
\pgfpathlineto{\pgfqpoint{1.430136in}{1.119151in}}%
\pgfpathlineto{\pgfqpoint{1.500323in}{1.119151in}}%
\pgfpathlineto{\pgfqpoint{1.502829in}{1.124774in}}%
\pgfpathlineto{\pgfqpoint{1.547949in}{1.124774in}}%
\pgfpathlineto{\pgfqpoint{1.550456in}{1.130220in}}%
\pgfpathlineto{\pgfqpoint{1.578029in}{1.130220in}}%
\pgfpathlineto{\pgfqpoint{1.580536in}{1.135497in}}%
\pgfpathlineto{\pgfqpoint{1.618135in}{1.135497in}}%
\pgfpathlineto{\pgfqpoint{1.620642in}{1.140617in}}%
\pgfpathlineto{\pgfqpoint{1.628162in}{1.140617in}}%
\pgfpathlineto{\pgfqpoint{1.630669in}{1.145589in}}%
\pgfpathlineto{\pgfqpoint{1.653229in}{1.145589in}}%
\pgfpathlineto{\pgfqpoint{1.655735in}{1.150420in}}%
\pgfpathlineto{\pgfqpoint{1.703362in}{1.150420in}}%
\pgfpathlineto{\pgfqpoint{1.705868in}{1.155119in}}%
\pgfpathlineto{\pgfqpoint{1.733442in}{1.155119in}}%
\pgfpathlineto{\pgfqpoint{1.735948in}{1.159693in}}%
\pgfpathlineto{\pgfqpoint{1.771041in}{1.159693in}}%
\pgfpathlineto{\pgfqpoint{1.773548in}{1.164148in}}%
\pgfpathlineto{\pgfqpoint{1.828694in}{1.164148in}}%
\pgfpathlineto{\pgfqpoint{1.831201in}{1.168490in}}%
\pgfpathlineto{\pgfqpoint{1.881334in}{1.168490in}}%
\pgfpathlineto{\pgfqpoint{1.883841in}{1.172725in}}%
\pgfpathlineto{\pgfqpoint{1.926454in}{1.172725in}}%
\pgfpathlineto{\pgfqpoint{1.928961in}{1.176858in}}%
\pgfpathlineto{\pgfqpoint{1.964054in}{1.176858in}}%
\pgfpathlineto{\pgfqpoint{1.966560in}{1.180894in}}%
\pgfpathlineto{\pgfqpoint{1.994134in}{1.180894in}}%
\pgfpathlineto{\pgfqpoint{1.996640in}{1.184837in}}%
\pgfpathlineto{\pgfqpoint{2.014187in}{1.184837in}}%
\pgfpathlineto{\pgfqpoint{2.016693in}{1.188691in}}%
\pgfpathlineto{\pgfqpoint{2.036747in}{1.188691in}}%
\pgfpathlineto{\pgfqpoint{2.039253in}{1.192460in}}%
\pgfpathlineto{\pgfqpoint{2.059307in}{1.192460in}}%
\pgfpathlineto{\pgfqpoint{2.061813in}{1.196149in}}%
\pgfpathlineto{\pgfqpoint{2.081866in}{1.196149in}}%
\pgfpathlineto{\pgfqpoint{2.084373in}{1.199760in}}%
\pgfpathlineto{\pgfqpoint{2.114453in}{1.199760in}}%
\pgfpathlineto{\pgfqpoint{2.116960in}{1.203296in}}%
\pgfpathlineto{\pgfqpoint{2.129493in}{1.203296in}}%
\pgfpathlineto{\pgfqpoint{2.132000in}{1.206761in}}%
\pgfpathlineto{\pgfqpoint{2.149546in}{1.206761in}}%
\pgfpathlineto{\pgfqpoint{2.152053in}{1.210157in}}%
\pgfpathlineto{\pgfqpoint{2.172106in}{1.210157in}}%
\pgfpathlineto{\pgfqpoint{2.174613in}{1.213487in}}%
\pgfpathlineto{\pgfqpoint{2.207199in}{1.213487in}}%
\pgfpathlineto{\pgfqpoint{2.209706in}{1.216754in}}%
\pgfpathlineto{\pgfqpoint{2.227252in}{1.216754in}}%
\pgfpathlineto{\pgfqpoint{2.229759in}{1.219960in}}%
\pgfpathlineto{\pgfqpoint{2.242292in}{1.219960in}}%
\pgfpathlineto{\pgfqpoint{2.244799in}{1.223107in}}%
\pgfpathlineto{\pgfqpoint{2.257332in}{1.223107in}}%
\pgfpathlineto{\pgfqpoint{2.259839in}{1.226197in}}%
\pgfpathlineto{\pgfqpoint{2.272372in}{1.226197in}}%
\pgfpathlineto{\pgfqpoint{2.274879in}{1.229233in}}%
\pgfpathlineto{\pgfqpoint{2.282399in}{1.229233in}}%
\pgfpathlineto{\pgfqpoint{2.284905in}{1.232216in}}%
\pgfpathlineto{\pgfqpoint{2.297439in}{1.232216in}}%
\pgfpathlineto{\pgfqpoint{2.299945in}{1.235148in}}%
\pgfpathlineto{\pgfqpoint{2.319999in}{1.235148in}}%
\pgfpathlineto{\pgfqpoint{2.322505in}{1.238030in}}%
\pgfpathlineto{\pgfqpoint{2.332532in}{1.238030in}}%
\pgfpathlineto{\pgfqpoint{2.335039in}{1.240865in}}%
\pgfpathlineto{\pgfqpoint{2.340052in}{1.240865in}}%
\pgfpathlineto{\pgfqpoint{2.342559in}{1.243654in}}%
\pgfpathlineto{\pgfqpoint{2.355092in}{1.243654in}}%
\pgfpathlineto{\pgfqpoint{2.357598in}{1.246398in}}%
\pgfpathlineto{\pgfqpoint{2.377652in}{1.246398in}}%
\pgfpathlineto{\pgfqpoint{2.380158in}{1.249099in}}%
\pgfpathlineto{\pgfqpoint{2.390185in}{1.249099in}}%
\pgfpathlineto{\pgfqpoint{2.392692in}{1.251758in}}%
\pgfpathlineto{\pgfqpoint{2.395198in}{1.251758in}}%
\pgfpathlineto{\pgfqpoint{2.397705in}{1.254376in}}%
\pgfpathlineto{\pgfqpoint{2.405225in}{1.254376in}}%
\pgfpathlineto{\pgfqpoint{2.407732in}{1.256956in}}%
\pgfpathlineto{\pgfqpoint{2.410238in}{1.256956in}}%
\pgfpathlineto{\pgfqpoint{2.412745in}{1.259496in}}%
\pgfpathlineto{\pgfqpoint{2.415251in}{1.259496in}}%
\pgfpathlineto{\pgfqpoint{2.417758in}{1.262000in}}%
\pgfpathlineto{\pgfqpoint{2.430291in}{1.262000in}}%
\pgfpathlineto{\pgfqpoint{2.432798in}{1.264468in}}%
\pgfpathlineto{\pgfqpoint{2.440318in}{1.264468in}}%
\pgfpathlineto{\pgfqpoint{2.442825in}{1.266901in}}%
\pgfpathlineto{\pgfqpoint{2.445331in}{1.271665in}}%
\pgfpathlineto{\pgfqpoint{2.447838in}{1.273999in}}%
\pgfpathlineto{\pgfqpoint{2.450345in}{1.273999in}}%
\pgfpathlineto{\pgfqpoint{2.460371in}{1.283027in}}%
\pgfpathlineto{\pgfqpoint{2.462878in}{1.283027in}}%
\pgfpathlineto{\pgfqpoint{2.465385in}{1.285212in}}%
\pgfpathlineto{\pgfqpoint{2.467891in}{1.285212in}}%
\pgfpathlineto{\pgfqpoint{2.470398in}{1.287370in}}%
\pgfpathlineto{\pgfqpoint{2.472905in}{1.287370in}}%
\pgfpathlineto{\pgfqpoint{2.480425in}{1.293683in}}%
\pgfpathlineto{\pgfqpoint{2.487944in}{1.293683in}}%
\pgfpathlineto{\pgfqpoint{2.490451in}{1.295737in}}%
\pgfpathlineto{\pgfqpoint{2.495464in}{1.301756in}}%
\pgfpathlineto{\pgfqpoint{2.497971in}{1.303716in}}%
\pgfpathlineto{\pgfqpoint{2.500478in}{1.303716in}}%
\pgfpathlineto{\pgfqpoint{2.502984in}{1.309465in}}%
\pgfpathlineto{\pgfqpoint{2.505491in}{1.309465in}}%
\pgfpathlineto{\pgfqpoint{2.520531in}{1.320416in}}%
\pgfpathlineto{\pgfqpoint{2.523038in}{1.320416in}}%
\pgfpathlineto{\pgfqpoint{2.525544in}{1.322175in}}%
\pgfpathlineto{\pgfqpoint{2.530558in}{1.322175in}}%
\pgfpathlineto{\pgfqpoint{2.535571in}{1.325640in}}%
\pgfpathlineto{\pgfqpoint{2.540584in}{1.325640in}}%
\pgfpathlineto{\pgfqpoint{2.543091in}{1.327347in}}%
\pgfpathlineto{\pgfqpoint{2.545598in}{1.327347in}}%
\pgfpathlineto{\pgfqpoint{2.548104in}{1.329036in}}%
\pgfpathlineto{\pgfqpoint{2.550611in}{1.329036in}}%
\pgfpathlineto{\pgfqpoint{2.558131in}{1.334008in}}%
\pgfpathlineto{\pgfqpoint{2.560637in}{1.334008in}}%
\pgfpathlineto{\pgfqpoint{2.565651in}{1.346601in}}%
\pgfpathlineto{\pgfqpoint{2.568157in}{1.346601in}}%
\pgfpathlineto{\pgfqpoint{2.573171in}{1.351095in}}%
\pgfpathlineto{\pgfqpoint{2.575677in}{1.351095in}}%
\pgfpathlineto{\pgfqpoint{2.578184in}{1.365277in}}%
\pgfpathlineto{\pgfqpoint{2.580691in}{1.366633in}}%
\pgfpathlineto{\pgfqpoint{2.583197in}{1.369313in}}%
\pgfpathlineto{\pgfqpoint{2.585704in}{1.378376in}}%
\pgfpathlineto{\pgfqpoint{2.588211in}{1.378376in}}%
\pgfpathlineto{\pgfqpoint{2.593224in}{1.382118in}}%
\pgfpathlineto{\pgfqpoint{2.598237in}{1.382118in}}%
\pgfpathlineto{\pgfqpoint{2.608264in}{1.386984in}}%
\pgfpathlineto{\pgfqpoint{2.610771in}{1.390545in}}%
\pgfpathlineto{\pgfqpoint{2.623304in}{1.395180in}}%
\pgfpathlineto{\pgfqpoint{2.640850in}{1.403003in}}%
\pgfpathlineto{\pgfqpoint{2.643357in}{1.406249in}}%
\pgfpathlineto{\pgfqpoint{2.648370in}{1.408379in}}%
\pgfpathlineto{\pgfqpoint{2.650877in}{1.411526in}}%
\pgfpathlineto{\pgfqpoint{2.660904in}{1.415635in}}%
\pgfpathlineto{\pgfqpoint{2.663410in}{1.418652in}}%
\pgfpathlineto{\pgfqpoint{2.668424in}{1.419647in}}%
\pgfpathlineto{\pgfqpoint{2.673437in}{1.422595in}}%
\pgfpathlineto{\pgfqpoint{2.675944in}{1.425494in}}%
\pgfpathlineto{\pgfqpoint{2.683464in}{1.426450in}}%
\pgfpathlineto{\pgfqpoint{2.690983in}{1.430219in}}%
\pgfpathlineto{\pgfqpoint{2.693490in}{1.432073in}}%
\pgfpathlineto{\pgfqpoint{2.695997in}{1.432073in}}%
\pgfpathlineto{\pgfqpoint{2.698503in}{1.434817in}}%
\pgfpathlineto{\pgfqpoint{2.703517in}{1.435722in}}%
\pgfpathlineto{\pgfqpoint{2.711037in}{1.436623in}}%
\pgfpathlineto{\pgfqpoint{2.716050in}{1.440177in}}%
\pgfpathlineto{\pgfqpoint{2.718557in}{1.445375in}}%
\pgfpathlineto{\pgfqpoint{2.721063in}{1.447073in}}%
\pgfpathlineto{\pgfqpoint{2.723570in}{1.450420in}}%
\pgfpathlineto{\pgfqpoint{2.728583in}{1.451246in}}%
\pgfpathlineto{\pgfqpoint{2.731090in}{1.451246in}}%
\pgfpathlineto{\pgfqpoint{2.733597in}{1.453702in}}%
\pgfpathlineto{\pgfqpoint{2.738610in}{1.454513in}}%
\pgfpathlineto{\pgfqpoint{2.741117in}{1.457719in}}%
\pgfpathlineto{\pgfqpoint{2.743623in}{1.463189in}}%
\pgfpathlineto{\pgfqpoint{2.751143in}{1.465481in}}%
\pgfpathlineto{\pgfqpoint{2.753650in}{1.469975in}}%
\pgfpathlineto{\pgfqpoint{2.761170in}{1.474354in}}%
\pgfpathlineto{\pgfqpoint{2.766183in}{1.480024in}}%
\pgfpathlineto{\pgfqpoint{2.771196in}{1.480720in}}%
\pgfpathlineto{\pgfqpoint{2.776210in}{1.481413in}}%
\pgfpathlineto{\pgfqpoint{2.778716in}{1.481413in}}%
\pgfpathlineto{\pgfqpoint{2.781223in}{1.484836in}}%
\pgfpathlineto{\pgfqpoint{2.786236in}{1.486186in}}%
\pgfpathlineto{\pgfqpoint{2.791250in}{1.488856in}}%
\pgfpathlineto{\pgfqpoint{2.793756in}{1.490175in}}%
\pgfpathlineto{\pgfqpoint{2.796263in}{1.490175in}}%
\pgfpathlineto{\pgfqpoint{2.798770in}{1.492784in}}%
\pgfpathlineto{\pgfqpoint{2.801276in}{1.492784in}}%
\pgfpathlineto{\pgfqpoint{2.803783in}{1.496624in}}%
\pgfpathlineto{\pgfqpoint{2.813810in}{1.499759in}}%
\pgfpathlineto{\pgfqpoint{2.816316in}{1.505863in}}%
\pgfpathlineto{\pgfqpoint{2.818823in}{1.505863in}}%
\pgfpathlineto{\pgfqpoint{2.823836in}{1.508836in}}%
\pgfpathlineto{\pgfqpoint{2.826343in}{1.512912in}}%
\pgfpathlineto{\pgfqpoint{2.831356in}{1.514630in}}%
\pgfpathlineto{\pgfqpoint{2.833863in}{1.521882in}}%
\pgfpathlineto{\pgfqpoint{2.841383in}{1.527787in}}%
\pgfpathlineto{\pgfqpoint{2.843889in}{1.528840in}}%
\pgfpathlineto{\pgfqpoint{2.846396in}{1.540987in}}%
\pgfpathlineto{\pgfqpoint{2.848903in}{1.541475in}}%
\pgfpathlineto{\pgfqpoint{2.851409in}{1.543894in}}%
\pgfpathlineto{\pgfqpoint{2.853916in}{1.543894in}}%
\pgfpathlineto{\pgfqpoint{2.858929in}{1.547224in}}%
\pgfpathlineto{\pgfqpoint{2.883996in}{1.557289in}}%
\pgfpathlineto{\pgfqpoint{2.886503in}{1.560371in}}%
\pgfpathlineto{\pgfqpoint{2.891516in}{1.562539in}}%
\pgfpathlineto{\pgfqpoint{2.894022in}{1.566795in}}%
\pgfpathlineto{\pgfqpoint{2.899036in}{1.567634in}}%
\pgfpathlineto{\pgfqpoint{2.904049in}{1.569299in}}%
\pgfpathlineto{\pgfqpoint{2.909062in}{1.574200in}}%
\pgfpathlineto{\pgfqpoint{2.911569in}{1.575003in}}%
\pgfpathlineto{\pgfqpoint{2.914076in}{1.578179in}}%
\pgfpathlineto{\pgfqpoint{2.916582in}{1.578572in}}%
\pgfpathlineto{\pgfqpoint{2.919089in}{1.581297in}}%
\pgfpathlineto{\pgfqpoint{2.926609in}{1.583600in}}%
\pgfpathlineto{\pgfqpoint{2.931622in}{1.592149in}}%
\pgfpathlineto{\pgfqpoint{2.939142in}{1.602354in}}%
\pgfpathlineto{\pgfqpoint{2.944156in}{1.602696in}}%
\pgfpathlineto{\pgfqpoint{2.946662in}{1.606739in}}%
\pgfpathlineto{\pgfqpoint{2.951676in}{1.607072in}}%
\pgfpathlineto{\pgfqpoint{2.954182in}{1.610689in}}%
\pgfpathlineto{\pgfqpoint{2.956689in}{1.611986in}}%
\pgfpathlineto{\pgfqpoint{2.959195in}{1.615186in}}%
\pgfpathlineto{\pgfqpoint{2.964209in}{1.615819in}}%
\pgfpathlineto{\pgfqpoint{2.969222in}{1.616764in}}%
\pgfpathlineto{\pgfqpoint{2.974235in}{1.617078in}}%
\pgfpathlineto{\pgfqpoint{2.981755in}{1.622631in}}%
\pgfpathlineto{\pgfqpoint{2.994289in}{1.625042in}}%
\pgfpathlineto{\pgfqpoint{3.001809in}{1.632653in}}%
\pgfpathlineto{\pgfqpoint{3.009329in}{1.636335in}}%
\pgfpathlineto{\pgfqpoint{3.011835in}{1.638286in}}%
\pgfpathlineto{\pgfqpoint{3.014342in}{1.638286in}}%
\pgfpathlineto{\pgfqpoint{3.016849in}{1.641307in}}%
\pgfpathlineto{\pgfqpoint{3.019355in}{1.641850in}}%
\pgfpathlineto{\pgfqpoint{3.021862in}{1.645076in}}%
\pgfpathlineto{\pgfqpoint{3.024369in}{1.652885in}}%
\pgfpathlineto{\pgfqpoint{3.034395in}{1.654909in}}%
\pgfpathlineto{\pgfqpoint{3.036902in}{1.655160in}}%
\pgfpathlineto{\pgfqpoint{3.056955in}{1.666574in}}%
\pgfpathlineto{\pgfqpoint{3.059462in}{1.670752in}}%
\pgfpathlineto{\pgfqpoint{3.061968in}{1.670981in}}%
\pgfpathlineto{\pgfqpoint{3.074502in}{1.678813in}}%
\pgfpathlineto{\pgfqpoint{3.079515in}{1.687556in}}%
\pgfpathlineto{\pgfqpoint{3.084528in}{1.688592in}}%
\pgfpathlineto{\pgfqpoint{3.092048in}{1.691461in}}%
\pgfpathlineto{\pgfqpoint{3.094555in}{1.693682in}}%
\pgfpathlineto{\pgfqpoint{3.099568in}{1.694881in}}%
\pgfpathlineto{\pgfqpoint{3.117115in}{1.706248in}}%
\pgfpathlineto{\pgfqpoint{3.119621in}{1.706435in}}%
\pgfpathlineto{\pgfqpoint{3.122128in}{1.710665in}}%
\pgfpathlineto{\pgfqpoint{3.134661in}{1.716909in}}%
\pgfpathlineto{\pgfqpoint{3.137168in}{1.719517in}}%
\pgfpathlineto{\pgfqpoint{3.139675in}{1.720034in}}%
\pgfpathlineto{\pgfqpoint{3.144688in}{1.725285in}}%
\pgfpathlineto{\pgfqpoint{3.149701in}{1.727276in}}%
\pgfpathlineto{\pgfqpoint{3.152208in}{1.733112in}}%
\pgfpathlineto{\pgfqpoint{3.154715in}{1.735486in}}%
\pgfpathlineto{\pgfqpoint{3.157221in}{1.748207in}}%
\pgfpathlineto{\pgfqpoint{3.162234in}{1.748645in}}%
\pgfpathlineto{\pgfqpoint{3.172261in}{1.757580in}}%
\pgfpathlineto{\pgfqpoint{3.179781in}{1.760322in}}%
\pgfpathlineto{\pgfqpoint{3.202341in}{1.766729in}}%
\pgfpathlineto{\pgfqpoint{3.217381in}{1.770359in}}%
\pgfpathlineto{\pgfqpoint{3.219888in}{1.774541in}}%
\pgfpathlineto{\pgfqpoint{3.227408in}{1.777150in}}%
\pgfpathlineto{\pgfqpoint{3.229914in}{1.781652in}}%
\pgfpathlineto{\pgfqpoint{3.232421in}{1.782252in}}%
\pgfpathlineto{\pgfqpoint{3.234927in}{1.785453in}}%
\pgfpathlineto{\pgfqpoint{3.247461in}{1.791001in}}%
\pgfpathlineto{\pgfqpoint{3.252474in}{1.792811in}}%
\pgfpathlineto{\pgfqpoint{3.254981in}{1.793036in}}%
\pgfpathlineto{\pgfqpoint{3.259994in}{1.797802in}}%
\pgfpathlineto{\pgfqpoint{3.262501in}{1.797802in}}%
\pgfpathlineto{\pgfqpoint{3.270021in}{1.805810in}}%
\pgfpathlineto{\pgfqpoint{3.272527in}{1.806539in}}%
\pgfpathlineto{\pgfqpoint{3.275034in}{1.809933in}}%
\pgfpathlineto{\pgfqpoint{3.277541in}{1.810645in}}%
\pgfpathlineto{\pgfqpoint{3.282554in}{1.817893in}}%
\pgfpathlineto{\pgfqpoint{3.297594in}{1.824755in}}%
\pgfpathlineto{\pgfqpoint{3.307620in}{1.835262in}}%
\pgfpathlineto{\pgfqpoint{3.317647in}{1.839085in}}%
\pgfpathlineto{\pgfqpoint{3.320154in}{1.843998in}}%
\pgfpathlineto{\pgfqpoint{3.322660in}{1.844164in}}%
\pgfpathlineto{\pgfqpoint{3.325167in}{1.849987in}}%
\pgfpathlineto{\pgfqpoint{3.330180in}{1.855150in}}%
\pgfpathlineto{\pgfqpoint{3.332687in}{1.859171in}}%
\pgfpathlineto{\pgfqpoint{3.335194in}{1.860314in}}%
\pgfpathlineto{\pgfqpoint{3.337700in}{1.868033in}}%
\pgfpathlineto{\pgfqpoint{3.340207in}{1.871268in}}%
\pgfpathlineto{\pgfqpoint{3.342714in}{1.880421in}}%
\pgfpathlineto{\pgfqpoint{3.345220in}{1.881297in}}%
\pgfpathlineto{\pgfqpoint{3.350234in}{1.893170in}}%
\pgfpathlineto{\pgfqpoint{3.355247in}{1.896522in}}%
\pgfpathlineto{\pgfqpoint{3.360260in}{1.901191in}}%
\pgfpathlineto{\pgfqpoint{3.362767in}{1.904860in}}%
\pgfpathlineto{\pgfqpoint{3.367780in}{1.906202in}}%
\pgfpathlineto{\pgfqpoint{3.372793in}{1.909710in}}%
\pgfpathlineto{\pgfqpoint{3.377807in}{1.910845in}}%
\pgfpathlineto{\pgfqpoint{3.380313in}{1.914481in}}%
\pgfpathlineto{\pgfqpoint{3.382820in}{1.920144in}}%
\pgfpathlineto{\pgfqpoint{3.385327in}{1.920679in}}%
\pgfpathlineto{\pgfqpoint{3.390340in}{1.930435in}}%
\pgfpathlineto{\pgfqpoint{3.392847in}{1.930989in}}%
\pgfpathlineto{\pgfqpoint{3.395353in}{1.932938in}}%
\pgfpathlineto{\pgfqpoint{3.397860in}{1.932988in}}%
\pgfpathlineto{\pgfqpoint{3.400367in}{1.935112in}}%
\pgfpathlineto{\pgfqpoint{3.407887in}{1.936578in}}%
\pgfpathlineto{\pgfqpoint{3.410393in}{1.938418in}}%
\pgfpathlineto{\pgfqpoint{3.412900in}{1.942791in}}%
\pgfpathlineto{\pgfqpoint{3.415407in}{1.943260in}}%
\pgfpathlineto{\pgfqpoint{3.417913in}{1.951217in}}%
\pgfpathlineto{\pgfqpoint{3.420420in}{1.952951in}}%
\pgfpathlineto{\pgfqpoint{3.422927in}{1.976090in}}%
\pgfpathlineto{\pgfqpoint{3.427940in}{1.982687in}}%
\pgfpathlineto{\pgfqpoint{3.430447in}{1.987781in}}%
\pgfpathlineto{\pgfqpoint{3.432953in}{1.989756in}}%
\pgfpathlineto{\pgfqpoint{3.437966in}{1.997024in}}%
\pgfpathlineto{\pgfqpoint{3.442980in}{2.020697in}}%
\pgfpathlineto{\pgfqpoint{3.447993in}{2.028589in}}%
\pgfpathlineto{\pgfqpoint{3.458020in}{2.036134in}}%
\pgfpathlineto{\pgfqpoint{3.460526in}{2.042076in}}%
\pgfpathlineto{\pgfqpoint{3.470553in}{2.051445in}}%
\pgfpathlineto{\pgfqpoint{3.475566in}{2.052413in}}%
\pgfpathlineto{\pgfqpoint{3.480580in}{2.057826in}}%
\pgfpathlineto{\pgfqpoint{3.485593in}{2.069716in}}%
\pgfpathlineto{\pgfqpoint{3.490606in}{2.070810in}}%
\pgfpathlineto{\pgfqpoint{3.495620in}{2.074399in}}%
\pgfpathlineto{\pgfqpoint{3.498126in}{2.082423in}}%
\pgfpathlineto{\pgfqpoint{3.500633in}{2.083976in}}%
\pgfpathlineto{\pgfqpoint{3.505646in}{2.090367in}}%
\pgfpathlineto{\pgfqpoint{3.508153in}{2.092576in}}%
\pgfpathlineto{\pgfqpoint{3.510659in}{2.101585in}}%
\pgfpathlineto{\pgfqpoint{3.513166in}{2.104004in}}%
\pgfpathlineto{\pgfqpoint{3.515673in}{2.114977in}}%
\pgfpathlineto{\pgfqpoint{3.518179in}{2.115937in}}%
\pgfpathlineto{\pgfqpoint{3.520686in}{2.124152in}}%
\pgfpathlineto{\pgfqpoint{3.525699in}{2.129797in}}%
\pgfpathlineto{\pgfqpoint{3.528206in}{2.129860in}}%
\pgfpathlineto{\pgfqpoint{3.530713in}{2.131429in}}%
\pgfpathlineto{\pgfqpoint{3.533219in}{2.139040in}}%
\pgfpathlineto{\pgfqpoint{3.548259in}{2.144888in}}%
\pgfpathlineto{\pgfqpoint{3.553273in}{2.146111in}}%
\pgfpathlineto{\pgfqpoint{3.558286in}{2.150137in}}%
\pgfpathlineto{\pgfqpoint{3.563299in}{2.150823in}}%
\pgfpathlineto{\pgfqpoint{3.568313in}{2.156977in}}%
\pgfpathlineto{\pgfqpoint{3.570819in}{2.159423in}}%
\pgfpathlineto{\pgfqpoint{3.575832in}{2.168571in}}%
\pgfpathlineto{\pgfqpoint{3.580846in}{2.172478in}}%
\pgfpathlineto{\pgfqpoint{3.583352in}{2.179313in}}%
\pgfpathlineto{\pgfqpoint{3.588366in}{2.185546in}}%
\pgfpathlineto{\pgfqpoint{3.605912in}{2.194808in}}%
\pgfpathlineto{\pgfqpoint{3.608419in}{2.196591in}}%
\pgfpathlineto{\pgfqpoint{3.610926in}{2.200260in}}%
\pgfpathlineto{\pgfqpoint{3.618446in}{2.202762in}}%
\pgfpathlineto{\pgfqpoint{3.620952in}{2.202968in}}%
\pgfpathlineto{\pgfqpoint{3.623459in}{2.205198in}}%
\pgfpathlineto{\pgfqpoint{3.625966in}{2.216267in}}%
\pgfpathlineto{\pgfqpoint{3.630979in}{2.218651in}}%
\pgfpathlineto{\pgfqpoint{3.633486in}{2.220929in}}%
\pgfpathlineto{\pgfqpoint{3.638499in}{2.228040in}}%
\pgfpathlineto{\pgfqpoint{3.643512in}{2.228865in}}%
\pgfpathlineto{\pgfqpoint{3.648525in}{2.234697in}}%
\pgfpathlineto{\pgfqpoint{3.653539in}{2.242018in}}%
\pgfpathlineto{\pgfqpoint{3.658552in}{2.249267in}}%
\pgfpathlineto{\pgfqpoint{3.661059in}{2.252429in}}%
\pgfpathlineto{\pgfqpoint{3.666072in}{2.262418in}}%
\pgfpathlineto{\pgfqpoint{3.668579in}{2.263702in}}%
\pgfpathlineto{\pgfqpoint{3.671085in}{2.266256in}}%
\pgfpathlineto{\pgfqpoint{3.673592in}{2.276461in}}%
\pgfpathlineto{\pgfqpoint{3.676099in}{2.279999in}}%
\pgfpathlineto{\pgfqpoint{3.678605in}{2.280779in}}%
\pgfpathlineto{\pgfqpoint{3.681112in}{2.282957in}}%
\pgfpathlineto{\pgfqpoint{3.688632in}{2.301395in}}%
\pgfpathlineto{\pgfqpoint{3.691139in}{2.301511in}}%
\pgfpathlineto{\pgfqpoint{3.696152in}{2.303184in}}%
\pgfpathlineto{\pgfqpoint{3.698659in}{2.303247in}}%
\pgfpathlineto{\pgfqpoint{3.701165in}{2.306363in}}%
\pgfpathlineto{\pgfqpoint{3.703672in}{2.307637in}}%
\pgfpathlineto{\pgfqpoint{3.708685in}{2.313228in}}%
\pgfpathlineto{\pgfqpoint{3.711192in}{2.315433in}}%
\pgfpathlineto{\pgfqpoint{3.718712in}{2.337371in}}%
\pgfpathlineto{\pgfqpoint{3.721218in}{2.337432in}}%
\pgfpathlineto{\pgfqpoint{3.726232in}{2.349843in}}%
\pgfpathlineto{\pgfqpoint{3.728738in}{2.356793in}}%
\pgfpathlineto{\pgfqpoint{3.733752in}{2.360634in}}%
\pgfpathlineto{\pgfqpoint{3.736258in}{2.360647in}}%
\pgfpathlineto{\pgfqpoint{3.741272in}{2.364130in}}%
\pgfpathlineto{\pgfqpoint{3.743778in}{2.364218in}}%
\pgfpathlineto{\pgfqpoint{3.746285in}{2.366301in}}%
\pgfpathlineto{\pgfqpoint{3.753805in}{2.368012in}}%
\pgfpathlineto{\pgfqpoint{3.756312in}{2.370732in}}%
\pgfpathlineto{\pgfqpoint{3.758818in}{2.370933in}}%
\pgfpathlineto{\pgfqpoint{3.766338in}{2.390624in}}%
\pgfpathlineto{\pgfqpoint{3.768845in}{2.391604in}}%
\pgfpathlineto{\pgfqpoint{3.776365in}{2.402918in}}%
\pgfpathlineto{\pgfqpoint{3.781378in}{2.404499in}}%
\pgfpathlineto{\pgfqpoint{3.783885in}{2.414248in}}%
\pgfpathlineto{\pgfqpoint{3.786391in}{2.415227in}}%
\pgfpathlineto{\pgfqpoint{3.788898in}{2.421010in}}%
\pgfpathlineto{\pgfqpoint{3.791405in}{2.430797in}}%
\pgfpathlineto{\pgfqpoint{3.796418in}{2.436461in}}%
\pgfpathlineto{\pgfqpoint{3.801431in}{2.441889in}}%
\pgfpathlineto{\pgfqpoint{3.803938in}{2.442160in}}%
\pgfpathlineto{\pgfqpoint{3.811458in}{2.452687in}}%
\pgfpathlineto{\pgfqpoint{3.816471in}{2.453787in}}%
\pgfpathlineto{\pgfqpoint{3.818978in}{2.454687in}}%
\pgfpathlineto{\pgfqpoint{3.821485in}{2.457499in}}%
\pgfpathlineto{\pgfqpoint{3.826498in}{2.459239in}}%
\pgfpathlineto{\pgfqpoint{3.829005in}{2.469077in}}%
\pgfpathlineto{\pgfqpoint{3.831511in}{2.470958in}}%
\pgfpathlineto{\pgfqpoint{3.834018in}{2.471376in}}%
\pgfpathlineto{\pgfqpoint{3.836525in}{2.473669in}}%
\pgfpathlineto{\pgfqpoint{3.839031in}{2.473796in}}%
\pgfpathlineto{\pgfqpoint{3.844044in}{2.476046in}}%
\pgfpathlineto{\pgfqpoint{3.846551in}{2.476182in}}%
\pgfpathlineto{\pgfqpoint{3.851564in}{2.479526in}}%
\pgfpathlineto{\pgfqpoint{3.866604in}{2.484056in}}%
\pgfpathlineto{\pgfqpoint{3.869111in}{2.488171in}}%
\pgfpathlineto{\pgfqpoint{3.874124in}{2.489168in}}%
\pgfpathlineto{\pgfqpoint{3.876631in}{2.495090in}}%
\pgfpathlineto{\pgfqpoint{3.881644in}{2.500124in}}%
\pgfpathlineto{\pgfqpoint{3.884151in}{2.501072in}}%
\pgfpathlineto{\pgfqpoint{3.886658in}{2.504562in}}%
\pgfpathlineto{\pgfqpoint{3.891671in}{2.506581in}}%
\pgfpathlineto{\pgfqpoint{3.896684in}{2.509791in}}%
\pgfpathlineto{\pgfqpoint{3.899191in}{2.511167in}}%
\pgfpathlineto{\pgfqpoint{3.901698in}{2.514735in}}%
\pgfpathlineto{\pgfqpoint{3.904204in}{2.530658in}}%
\pgfpathlineto{\pgfqpoint{3.906711in}{2.533019in}}%
\pgfpathlineto{\pgfqpoint{3.909217in}{2.533672in}}%
\pgfpathlineto{\pgfqpoint{3.919244in}{2.547674in}}%
\pgfpathlineto{\pgfqpoint{3.921751in}{2.547772in}}%
\pgfpathlineto{\pgfqpoint{3.929271in}{2.560830in}}%
\pgfpathlineto{\pgfqpoint{3.931777in}{2.560882in}}%
\pgfpathlineto{\pgfqpoint{3.934284in}{2.584828in}}%
\pgfpathlineto{\pgfqpoint{3.936791in}{2.586079in}}%
\pgfpathlineto{\pgfqpoint{3.939297in}{2.605569in}}%
\pgfpathlineto{\pgfqpoint{3.944311in}{2.609074in}}%
\pgfpathlineto{\pgfqpoint{3.946817in}{2.619082in}}%
\pgfpathlineto{\pgfqpoint{3.949324in}{2.619848in}}%
\pgfpathlineto{\pgfqpoint{3.956844in}{2.630937in}}%
\pgfpathlineto{\pgfqpoint{3.959351in}{2.639640in}}%
\pgfpathlineto{\pgfqpoint{3.961857in}{2.640076in}}%
\pgfpathlineto{\pgfqpoint{3.966871in}{2.645495in}}%
\pgfpathlineto{\pgfqpoint{3.969377in}{2.652225in}}%
\pgfpathlineto{\pgfqpoint{3.971884in}{2.664379in}}%
\pgfpathlineto{\pgfqpoint{3.984417in}{2.672227in}}%
\pgfpathlineto{\pgfqpoint{3.989430in}{2.678692in}}%
\pgfpathlineto{\pgfqpoint{3.991937in}{2.684683in}}%
\pgfpathlineto{\pgfqpoint{3.994444in}{2.686420in}}%
\pgfpathlineto{\pgfqpoint{3.996950in}{2.690481in}}%
\pgfpathlineto{\pgfqpoint{4.001964in}{2.691442in}}%
\pgfpathlineto{\pgfqpoint{4.004470in}{2.693622in}}%
\pgfpathlineto{\pgfqpoint{4.006977in}{2.693884in}}%
\pgfpathlineto{\pgfqpoint{4.011990in}{2.698637in}}%
\pgfpathlineto{\pgfqpoint{4.014497in}{2.700350in}}%
\pgfpathlineto{\pgfqpoint{4.022017in}{2.713668in}}%
\pgfpathlineto{\pgfqpoint{4.024524in}{2.715629in}}%
\pgfpathlineto{\pgfqpoint{4.027030in}{2.719790in}}%
\pgfpathlineto{\pgfqpoint{4.029537in}{2.720419in}}%
\pgfpathlineto{\pgfqpoint{4.032044in}{2.729812in}}%
\pgfpathlineto{\pgfqpoint{4.034550in}{2.729963in}}%
\pgfpathlineto{\pgfqpoint{4.044577in}{2.750516in}}%
\pgfpathlineto{\pgfqpoint{4.049590in}{2.754879in}}%
\pgfpathlineto{\pgfqpoint{4.054603in}{2.756477in}}%
\pgfpathlineto{\pgfqpoint{4.057110in}{2.757285in}}%
\pgfpathlineto{\pgfqpoint{4.059617in}{2.763435in}}%
\pgfpathlineto{\pgfqpoint{4.062123in}{2.766232in}}%
\pgfpathlineto{\pgfqpoint{4.067137in}{2.767844in}}%
\pgfpathlineto{\pgfqpoint{4.069643in}{2.768806in}}%
\pgfpathlineto{\pgfqpoint{4.074657in}{2.782796in}}%
\pgfpathlineto{\pgfqpoint{4.077163in}{2.786488in}}%
\pgfpathlineto{\pgfqpoint{4.084683in}{2.814498in}}%
\pgfpathlineto{\pgfqpoint{4.087190in}{2.822193in}}%
\pgfpathlineto{\pgfqpoint{4.089697in}{2.822619in}}%
\pgfpathlineto{\pgfqpoint{4.092203in}{2.834331in}}%
\pgfpathlineto{\pgfqpoint{4.094710in}{2.836176in}}%
\pgfpathlineto{\pgfqpoint{4.104737in}{2.852580in}}%
\pgfpathlineto{\pgfqpoint{4.107243in}{2.856974in}}%
\pgfpathlineto{\pgfqpoint{4.109750in}{2.857114in}}%
\pgfpathlineto{\pgfqpoint{4.114763in}{2.863527in}}%
\pgfpathlineto{\pgfqpoint{4.117270in}{2.864323in}}%
\pgfpathlineto{\pgfqpoint{4.119776in}{2.868448in}}%
\pgfpathlineto{\pgfqpoint{4.122283in}{2.880460in}}%
\pgfpathlineto{\pgfqpoint{4.127296in}{2.887045in}}%
\pgfpathlineto{\pgfqpoint{4.129803in}{2.888246in}}%
\pgfpathlineto{\pgfqpoint{4.134816in}{2.903514in}}%
\pgfpathlineto{\pgfqpoint{4.137323in}{2.905275in}}%
\pgfpathlineto{\pgfqpoint{4.137323in}{2.905275in}}%
\pgfusepath{stroke}%
\end{pgfscope}%
\begin{pgfscope}%
\pgfpathrectangle{\pgfqpoint{0.708220in}{0.535823in}}{\pgfqpoint{5.013309in}{2.369453in}}%
\pgfusepath{clip}%
\pgfsetrectcap%
\pgfsetroundjoin%
\pgfsetlinewidth{1.003750pt}%
\definecolor{currentstroke}{rgb}{1.000000,0.647059,0.000000}%
\pgfsetstrokecolor{currentstroke}%
\pgfsetdash{}{0pt}%
\pgfpathmoveto{\pgfqpoint{0.708220in}{0.947078in}}%
\pgfpathlineto{\pgfqpoint{0.710727in}{0.947078in}}%
\pgfpathlineto{\pgfqpoint{0.713233in}{0.962001in}}%
\pgfpathlineto{\pgfqpoint{0.718246in}{0.962001in}}%
\pgfpathlineto{\pgfqpoint{0.720753in}{0.975729in}}%
\pgfpathlineto{\pgfqpoint{0.728273in}{0.975729in}}%
\pgfpathlineto{\pgfqpoint{0.730780in}{0.988439in}}%
\pgfpathlineto{\pgfqpoint{0.740806in}{0.988439in}}%
\pgfpathlineto{\pgfqpoint{0.743313in}{1.000271in}}%
\pgfpathlineto{\pgfqpoint{0.755846in}{1.000271in}}%
\pgfpathlineto{\pgfqpoint{0.758353in}{1.011340in}}%
\pgfpathlineto{\pgfqpoint{0.803473in}{1.011340in}}%
\pgfpathlineto{\pgfqpoint{0.805979in}{1.021738in}}%
\pgfpathlineto{\pgfqpoint{0.898726in}{1.021738in}}%
\pgfpathlineto{\pgfqpoint{0.901232in}{1.031541in}}%
\pgfpathlineto{\pgfqpoint{0.986458in}{1.031541in}}%
\pgfpathlineto{\pgfqpoint{0.988965in}{1.040814in}}%
\pgfpathlineto{\pgfqpoint{1.084218in}{1.040814in}}%
\pgfpathlineto{\pgfqpoint{1.086725in}{1.049611in}}%
\pgfpathlineto{\pgfqpoint{1.239631in}{1.049611in}}%
\pgfpathlineto{\pgfqpoint{1.242137in}{1.057979in}}%
\pgfpathlineto{\pgfqpoint{1.400056in}{1.057979in}}%
\pgfpathlineto{\pgfqpoint{1.402563in}{1.065957in}}%
\pgfpathlineto{\pgfqpoint{1.598082in}{1.065957in}}%
\pgfpathlineto{\pgfqpoint{1.600589in}{1.073581in}}%
\pgfpathlineto{\pgfqpoint{1.713388in}{1.073581in}}%
\pgfpathlineto{\pgfqpoint{1.715895in}{1.080880in}}%
\pgfpathlineto{\pgfqpoint{1.756001in}{1.080880in}}%
\pgfpathlineto{\pgfqpoint{1.758508in}{1.087881in}}%
\pgfpathlineto{\pgfqpoint{1.803628in}{1.087881in}}%
\pgfpathlineto{\pgfqpoint{1.806134in}{1.094608in}}%
\pgfpathlineto{\pgfqpoint{1.826188in}{1.094608in}}%
\pgfpathlineto{\pgfqpoint{1.828694in}{1.101081in}}%
\pgfpathlineto{\pgfqpoint{1.866294in}{1.101081in}}%
\pgfpathlineto{\pgfqpoint{1.868801in}{1.107318in}}%
\pgfpathlineto{\pgfqpoint{1.896374in}{1.107318in}}%
\pgfpathlineto{\pgfqpoint{1.898881in}{1.113336in}}%
\pgfpathlineto{\pgfqpoint{1.931467in}{1.113336in}}%
\pgfpathlineto{\pgfqpoint{1.933974in}{1.119151in}}%
\pgfpathlineto{\pgfqpoint{1.959040in}{1.119151in}}%
\pgfpathlineto{\pgfqpoint{1.961547in}{1.124774in}}%
\pgfpathlineto{\pgfqpoint{1.979094in}{1.124774in}}%
\pgfpathlineto{\pgfqpoint{1.981600in}{1.130220in}}%
\pgfpathlineto{\pgfqpoint{2.009173in}{1.130220in}}%
\pgfpathlineto{\pgfqpoint{2.011680in}{1.135497in}}%
\pgfpathlineto{\pgfqpoint{2.056800in}{1.135497in}}%
\pgfpathlineto{\pgfqpoint{2.059307in}{1.140617in}}%
\pgfpathlineto{\pgfqpoint{2.064320in}{1.140617in}}%
\pgfpathlineto{\pgfqpoint{2.066827in}{1.145589in}}%
\pgfpathlineto{\pgfqpoint{2.081866in}{1.145589in}}%
\pgfpathlineto{\pgfqpoint{2.084373in}{1.150420in}}%
\pgfpathlineto{\pgfqpoint{2.111946in}{1.150420in}}%
\pgfpathlineto{\pgfqpoint{2.114453in}{1.155119in}}%
\pgfpathlineto{\pgfqpoint{2.134506in}{1.155119in}}%
\pgfpathlineto{\pgfqpoint{2.137013in}{1.159693in}}%
\pgfpathlineto{\pgfqpoint{2.164586in}{1.159693in}}%
\pgfpathlineto{\pgfqpoint{2.167093in}{1.164148in}}%
\pgfpathlineto{\pgfqpoint{2.184639in}{1.164148in}}%
\pgfpathlineto{\pgfqpoint{2.187146in}{1.168490in}}%
\pgfpathlineto{\pgfqpoint{2.199679in}{1.168490in}}%
\pgfpathlineto{\pgfqpoint{2.202186in}{1.172725in}}%
\pgfpathlineto{\pgfqpoint{2.224746in}{1.172725in}}%
\pgfpathlineto{\pgfqpoint{2.227252in}{1.176858in}}%
\pgfpathlineto{\pgfqpoint{2.242292in}{1.176858in}}%
\pgfpathlineto{\pgfqpoint{2.244799in}{1.180894in}}%
\pgfpathlineto{\pgfqpoint{2.282399in}{1.180894in}}%
\pgfpathlineto{\pgfqpoint{2.284905in}{1.184837in}}%
\pgfpathlineto{\pgfqpoint{2.319999in}{1.184837in}}%
\pgfpathlineto{\pgfqpoint{2.322505in}{1.188691in}}%
\pgfpathlineto{\pgfqpoint{2.335039in}{1.188691in}}%
\pgfpathlineto{\pgfqpoint{2.337545in}{1.192460in}}%
\pgfpathlineto{\pgfqpoint{2.367625in}{1.192460in}}%
\pgfpathlineto{\pgfqpoint{2.370132in}{1.196149in}}%
\pgfpathlineto{\pgfqpoint{2.382665in}{1.196149in}}%
\pgfpathlineto{\pgfqpoint{2.385172in}{1.199760in}}%
\pgfpathlineto{\pgfqpoint{2.412745in}{1.199760in}}%
\pgfpathlineto{\pgfqpoint{2.415251in}{1.203296in}}%
\pgfpathlineto{\pgfqpoint{2.452851in}{1.203296in}}%
\pgfpathlineto{\pgfqpoint{2.455358in}{1.206761in}}%
\pgfpathlineto{\pgfqpoint{2.482931in}{1.206761in}}%
\pgfpathlineto{\pgfqpoint{2.485438in}{1.210157in}}%
\pgfpathlineto{\pgfqpoint{2.495464in}{1.210157in}}%
\pgfpathlineto{\pgfqpoint{2.497971in}{1.213487in}}%
\pgfpathlineto{\pgfqpoint{2.505491in}{1.213487in}}%
\pgfpathlineto{\pgfqpoint{2.507998in}{1.216754in}}%
\pgfpathlineto{\pgfqpoint{2.510504in}{1.216754in}}%
\pgfpathlineto{\pgfqpoint{2.513011in}{1.219960in}}%
\pgfpathlineto{\pgfqpoint{2.518024in}{1.219960in}}%
\pgfpathlineto{\pgfqpoint{2.520531in}{1.223107in}}%
\pgfpathlineto{\pgfqpoint{2.533064in}{1.223107in}}%
\pgfpathlineto{\pgfqpoint{2.538078in}{1.229233in}}%
\pgfpathlineto{\pgfqpoint{2.550611in}{1.229233in}}%
\pgfpathlineto{\pgfqpoint{2.553117in}{1.232216in}}%
\pgfpathlineto{\pgfqpoint{2.563144in}{1.232216in}}%
\pgfpathlineto{\pgfqpoint{2.568157in}{1.238030in}}%
\pgfpathlineto{\pgfqpoint{2.580691in}{1.238030in}}%
\pgfpathlineto{\pgfqpoint{2.583197in}{1.240865in}}%
\pgfpathlineto{\pgfqpoint{2.608264in}{1.240865in}}%
\pgfpathlineto{\pgfqpoint{2.610771in}{1.243654in}}%
\pgfpathlineto{\pgfqpoint{2.623304in}{1.243654in}}%
\pgfpathlineto{\pgfqpoint{2.625810in}{1.246398in}}%
\pgfpathlineto{\pgfqpoint{2.640850in}{1.246398in}}%
\pgfpathlineto{\pgfqpoint{2.643357in}{1.249099in}}%
\pgfpathlineto{\pgfqpoint{2.645864in}{1.249099in}}%
\pgfpathlineto{\pgfqpoint{2.648370in}{1.251758in}}%
\pgfpathlineto{\pgfqpoint{2.658397in}{1.251758in}}%
\pgfpathlineto{\pgfqpoint{2.660904in}{1.254376in}}%
\pgfpathlineto{\pgfqpoint{2.673437in}{1.254376in}}%
\pgfpathlineto{\pgfqpoint{2.675944in}{1.259496in}}%
\pgfpathlineto{\pgfqpoint{2.685970in}{1.259496in}}%
\pgfpathlineto{\pgfqpoint{2.688477in}{1.262000in}}%
\pgfpathlineto{\pgfqpoint{2.693490in}{1.262000in}}%
\pgfpathlineto{\pgfqpoint{2.695997in}{1.264468in}}%
\pgfpathlineto{\pgfqpoint{2.701010in}{1.264468in}}%
\pgfpathlineto{\pgfqpoint{2.703517in}{1.266901in}}%
\pgfpathlineto{\pgfqpoint{2.708530in}{1.266901in}}%
\pgfpathlineto{\pgfqpoint{2.713543in}{1.271665in}}%
\pgfpathlineto{\pgfqpoint{2.718557in}{1.271665in}}%
\pgfpathlineto{\pgfqpoint{2.721063in}{1.273999in}}%
\pgfpathlineto{\pgfqpoint{2.723570in}{1.273999in}}%
\pgfpathlineto{\pgfqpoint{2.726077in}{1.276301in}}%
\pgfpathlineto{\pgfqpoint{2.733597in}{1.276301in}}%
\pgfpathlineto{\pgfqpoint{2.736103in}{1.278572in}}%
\pgfpathlineto{\pgfqpoint{2.738610in}{1.278572in}}%
\pgfpathlineto{\pgfqpoint{2.741117in}{1.283027in}}%
\pgfpathlineto{\pgfqpoint{2.746130in}{1.283027in}}%
\pgfpathlineto{\pgfqpoint{2.751143in}{1.287370in}}%
\pgfpathlineto{\pgfqpoint{2.753650in}{1.287370in}}%
\pgfpathlineto{\pgfqpoint{2.756156in}{1.289500in}}%
\pgfpathlineto{\pgfqpoint{2.761170in}{1.289500in}}%
\pgfpathlineto{\pgfqpoint{2.763676in}{1.291604in}}%
\pgfpathlineto{\pgfqpoint{2.768690in}{1.291604in}}%
\pgfpathlineto{\pgfqpoint{2.771196in}{1.293683in}}%
\pgfpathlineto{\pgfqpoint{2.773703in}{1.293683in}}%
\pgfpathlineto{\pgfqpoint{2.776210in}{1.295737in}}%
\pgfpathlineto{\pgfqpoint{2.783730in}{1.295737in}}%
\pgfpathlineto{\pgfqpoint{2.786236in}{1.297767in}}%
\pgfpathlineto{\pgfqpoint{2.791250in}{1.297767in}}%
\pgfpathlineto{\pgfqpoint{2.793756in}{1.299773in}}%
\pgfpathlineto{\pgfqpoint{2.796263in}{1.299773in}}%
\pgfpathlineto{\pgfqpoint{2.801276in}{1.303716in}}%
\pgfpathlineto{\pgfqpoint{2.803783in}{1.303716in}}%
\pgfpathlineto{\pgfqpoint{2.806290in}{1.305654in}}%
\pgfpathlineto{\pgfqpoint{2.808796in}{1.305654in}}%
\pgfpathlineto{\pgfqpoint{2.811303in}{1.307570in}}%
\pgfpathlineto{\pgfqpoint{2.816316in}{1.307570in}}%
\pgfpathlineto{\pgfqpoint{2.818823in}{1.309465in}}%
\pgfpathlineto{\pgfqpoint{2.823836in}{1.309465in}}%
\pgfpathlineto{\pgfqpoint{2.826343in}{1.311340in}}%
\pgfpathlineto{\pgfqpoint{2.828849in}{1.311340in}}%
\pgfpathlineto{\pgfqpoint{2.836369in}{1.316843in}}%
\pgfpathlineto{\pgfqpoint{2.838876in}{1.316843in}}%
\pgfpathlineto{\pgfqpoint{2.841383in}{1.318639in}}%
\pgfpathlineto{\pgfqpoint{2.843889in}{1.318639in}}%
\pgfpathlineto{\pgfqpoint{2.846396in}{1.322175in}}%
\pgfpathlineto{\pgfqpoint{2.848903in}{1.322175in}}%
\pgfpathlineto{\pgfqpoint{2.851409in}{1.323916in}}%
\pgfpathlineto{\pgfqpoint{2.856423in}{1.323916in}}%
\pgfpathlineto{\pgfqpoint{2.863943in}{1.332367in}}%
\pgfpathlineto{\pgfqpoint{2.866449in}{1.334008in}}%
\pgfpathlineto{\pgfqpoint{2.868956in}{1.334008in}}%
\pgfpathlineto{\pgfqpoint{2.871463in}{1.335634in}}%
\pgfpathlineto{\pgfqpoint{2.881489in}{1.335634in}}%
\pgfpathlineto{\pgfqpoint{2.883996in}{1.337244in}}%
\pgfpathlineto{\pgfqpoint{2.886503in}{1.337244in}}%
\pgfpathlineto{\pgfqpoint{2.889009in}{1.338839in}}%
\pgfpathlineto{\pgfqpoint{2.891516in}{1.338839in}}%
\pgfpathlineto{\pgfqpoint{2.896529in}{1.343539in}}%
\pgfpathlineto{\pgfqpoint{2.904049in}{1.343539in}}%
\pgfpathlineto{\pgfqpoint{2.906556in}{1.345077in}}%
\pgfpathlineto{\pgfqpoint{2.911569in}{1.345077in}}%
\pgfpathlineto{\pgfqpoint{2.914076in}{1.346601in}}%
\pgfpathlineto{\pgfqpoint{2.916582in}{1.346601in}}%
\pgfpathlineto{\pgfqpoint{2.919089in}{1.348112in}}%
\pgfpathlineto{\pgfqpoint{2.921596in}{1.348112in}}%
\pgfpathlineto{\pgfqpoint{2.924102in}{1.349610in}}%
\pgfpathlineto{\pgfqpoint{2.926609in}{1.349610in}}%
\pgfpathlineto{\pgfqpoint{2.929116in}{1.352567in}}%
\pgfpathlineto{\pgfqpoint{2.939142in}{1.352567in}}%
\pgfpathlineto{\pgfqpoint{2.941649in}{1.354027in}}%
\pgfpathlineto{\pgfqpoint{2.949169in}{1.354027in}}%
\pgfpathlineto{\pgfqpoint{2.951676in}{1.358333in}}%
\pgfpathlineto{\pgfqpoint{2.954182in}{1.359744in}}%
\pgfpathlineto{\pgfqpoint{2.961702in}{1.359744in}}%
\pgfpathlineto{\pgfqpoint{2.964209in}{1.361144in}}%
\pgfpathlineto{\pgfqpoint{2.969222in}{1.366633in}}%
\pgfpathlineto{\pgfqpoint{2.971729in}{1.366633in}}%
\pgfpathlineto{\pgfqpoint{2.974235in}{1.367978in}}%
\pgfpathlineto{\pgfqpoint{2.976742in}{1.367978in}}%
\pgfpathlineto{\pgfqpoint{2.979249in}{1.369313in}}%
\pgfpathlineto{\pgfqpoint{2.981755in}{1.374550in}}%
\pgfpathlineto{\pgfqpoint{2.984262in}{1.375835in}}%
\pgfpathlineto{\pgfqpoint{2.986769in}{1.378376in}}%
\pgfpathlineto{\pgfqpoint{2.989275in}{1.378376in}}%
\pgfpathlineto{\pgfqpoint{2.991782in}{1.382118in}}%
\pgfpathlineto{\pgfqpoint{2.996795in}{1.383347in}}%
\pgfpathlineto{\pgfqpoint{2.999302in}{1.397452in}}%
\pgfpathlineto{\pgfqpoint{3.001809in}{1.398576in}}%
\pgfpathlineto{\pgfqpoint{3.004315in}{1.401907in}}%
\pgfpathlineto{\pgfqpoint{3.011835in}{1.406249in}}%
\pgfpathlineto{\pgfqpoint{3.016849in}{1.410484in}}%
\pgfpathlineto{\pgfqpoint{3.021862in}{1.410484in}}%
\pgfpathlineto{\pgfqpoint{3.024369in}{1.412563in}}%
\pgfpathlineto{\pgfqpoint{3.026875in}{1.412563in}}%
\pgfpathlineto{\pgfqpoint{3.029382in}{1.417652in}}%
\pgfpathlineto{\pgfqpoint{3.031888in}{1.418652in}}%
\pgfpathlineto{\pgfqpoint{3.034395in}{1.421618in}}%
\pgfpathlineto{\pgfqpoint{3.036902in}{1.421618in}}%
\pgfpathlineto{\pgfqpoint{3.041915in}{1.426450in}}%
\pgfpathlineto{\pgfqpoint{3.049435in}{1.428345in}}%
\pgfpathlineto{\pgfqpoint{3.051942in}{1.433907in}}%
\pgfpathlineto{\pgfqpoint{3.056955in}{1.434817in}}%
\pgfpathlineto{\pgfqpoint{3.059462in}{1.437518in}}%
\pgfpathlineto{\pgfqpoint{3.066982in}{1.438409in}}%
\pgfpathlineto{\pgfqpoint{3.069488in}{1.439296in}}%
\pgfpathlineto{\pgfqpoint{3.071995in}{1.442796in}}%
\pgfpathlineto{\pgfqpoint{3.079515in}{1.446226in}}%
\pgfpathlineto{\pgfqpoint{3.082022in}{1.448754in}}%
\pgfpathlineto{\pgfqpoint{3.087035in}{1.449589in}}%
\pgfpathlineto{\pgfqpoint{3.089542in}{1.455320in}}%
\pgfpathlineto{\pgfqpoint{3.097061in}{1.456123in}}%
\pgfpathlineto{\pgfqpoint{3.099568in}{1.457719in}}%
\pgfpathlineto{\pgfqpoint{3.104581in}{1.457719in}}%
\pgfpathlineto{\pgfqpoint{3.109595in}{1.462418in}}%
\pgfpathlineto{\pgfqpoint{3.114608in}{1.463189in}}%
\pgfpathlineto{\pgfqpoint{3.127141in}{1.463956in}}%
\pgfpathlineto{\pgfqpoint{3.132155in}{1.468490in}}%
\pgfpathlineto{\pgfqpoint{3.134661in}{1.469234in}}%
\pgfpathlineto{\pgfqpoint{3.139675in}{1.472906in}}%
\pgfpathlineto{\pgfqpoint{3.142181in}{1.474354in}}%
\pgfpathlineto{\pgfqpoint{3.149701in}{1.475073in}}%
\pgfpathlineto{\pgfqpoint{3.159728in}{1.484836in}}%
\pgfpathlineto{\pgfqpoint{3.162234in}{1.484836in}}%
\pgfpathlineto{\pgfqpoint{3.164741in}{1.486186in}}%
\pgfpathlineto{\pgfqpoint{3.167248in}{1.486186in}}%
\pgfpathlineto{\pgfqpoint{3.169754in}{1.490175in}}%
\pgfpathlineto{\pgfqpoint{3.174768in}{1.492135in}}%
\pgfpathlineto{\pgfqpoint{3.182288in}{1.492784in}}%
\pgfpathlineto{\pgfqpoint{3.187301in}{1.494714in}}%
\pgfpathlineto{\pgfqpoint{3.189808in}{1.497885in}}%
\pgfpathlineto{\pgfqpoint{3.202341in}{1.501613in}}%
\pgfpathlineto{\pgfqpoint{3.207354in}{1.506462in}}%
\pgfpathlineto{\pgfqpoint{3.212368in}{1.508245in}}%
\pgfpathlineto{\pgfqpoint{3.214874in}{1.513487in}}%
\pgfpathlineto{\pgfqpoint{3.217381in}{1.515766in}}%
\pgfpathlineto{\pgfqpoint{3.232421in}{1.519683in}}%
\pgfpathlineto{\pgfqpoint{3.237434in}{1.527787in}}%
\pgfpathlineto{\pgfqpoint{3.252474in}{1.528840in}}%
\pgfpathlineto{\pgfqpoint{3.257487in}{1.530925in}}%
\pgfpathlineto{\pgfqpoint{3.259994in}{1.535021in}}%
\pgfpathlineto{\pgfqpoint{3.262501in}{1.536532in}}%
\pgfpathlineto{\pgfqpoint{3.267514in}{1.541961in}}%
\pgfpathlineto{\pgfqpoint{3.277541in}{1.547695in}}%
\pgfpathlineto{\pgfqpoint{3.280047in}{1.547695in}}%
\pgfpathlineto{\pgfqpoint{3.285061in}{1.556398in}}%
\pgfpathlineto{\pgfqpoint{3.295087in}{1.562108in}}%
\pgfpathlineto{\pgfqpoint{3.297594in}{1.571358in}}%
\pgfpathlineto{\pgfqpoint{3.300100in}{1.571767in}}%
\pgfpathlineto{\pgfqpoint{3.302607in}{1.576995in}}%
\pgfpathlineto{\pgfqpoint{3.305114in}{1.579355in}}%
\pgfpathlineto{\pgfqpoint{3.307620in}{1.579355in}}%
\pgfpathlineto{\pgfqpoint{3.312634in}{1.587741in}}%
\pgfpathlineto{\pgfqpoint{3.315140in}{1.588113in}}%
\pgfpathlineto{\pgfqpoint{3.322660in}{1.600982in}}%
\pgfpathlineto{\pgfqpoint{3.325167in}{1.602696in}}%
\pgfpathlineto{\pgfqpoint{3.327674in}{1.613913in}}%
\pgfpathlineto{\pgfqpoint{3.330180in}{1.614232in}}%
\pgfpathlineto{\pgfqpoint{3.332687in}{1.617391in}}%
\pgfpathlineto{\pgfqpoint{3.335194in}{1.618016in}}%
\pgfpathlineto{\pgfqpoint{3.340207in}{1.621106in}}%
\pgfpathlineto{\pgfqpoint{3.345220in}{1.622934in}}%
\pgfpathlineto{\pgfqpoint{3.350234in}{1.623841in}}%
\pgfpathlineto{\pgfqpoint{3.352740in}{1.627125in}}%
\pgfpathlineto{\pgfqpoint{3.362767in}{1.630347in}}%
\pgfpathlineto{\pgfqpoint{3.367780in}{1.631504in}}%
\pgfpathlineto{\pgfqpoint{3.370287in}{1.635774in}}%
\pgfpathlineto{\pgfqpoint{3.372793in}{1.637174in}}%
\pgfpathlineto{\pgfqpoint{3.375300in}{1.648765in}}%
\pgfpathlineto{\pgfqpoint{3.377807in}{1.650063in}}%
\pgfpathlineto{\pgfqpoint{3.382820in}{1.656411in}}%
\pgfpathlineto{\pgfqpoint{3.385327in}{1.657405in}}%
\pgfpathlineto{\pgfqpoint{3.392847in}{1.670522in}}%
\pgfpathlineto{\pgfqpoint{3.395353in}{1.673932in}}%
\pgfpathlineto{\pgfqpoint{3.400367in}{1.676833in}}%
\pgfpathlineto{\pgfqpoint{3.407887in}{1.680121in}}%
\pgfpathlineto{\pgfqpoint{3.410393in}{1.680555in}}%
\pgfpathlineto{\pgfqpoint{3.412900in}{1.683985in}}%
\pgfpathlineto{\pgfqpoint{3.415407in}{1.698430in}}%
\pgfpathlineto{\pgfqpoint{3.422927in}{1.703997in}}%
\pgfpathlineto{\pgfqpoint{3.427940in}{1.711571in}}%
\pgfpathlineto{\pgfqpoint{3.437966in}{1.723440in}}%
\pgfpathlineto{\pgfqpoint{3.440473in}{1.736739in}}%
\pgfpathlineto{\pgfqpoint{3.447993in}{1.740292in}}%
\pgfpathlineto{\pgfqpoint{3.450500in}{1.741813in}}%
\pgfpathlineto{\pgfqpoint{3.453006in}{1.745263in}}%
\pgfpathlineto{\pgfqpoint{3.455513in}{1.750959in}}%
\pgfpathlineto{\pgfqpoint{3.458020in}{1.750959in}}%
\pgfpathlineto{\pgfqpoint{3.463033in}{1.763689in}}%
\pgfpathlineto{\pgfqpoint{3.473060in}{1.769201in}}%
\pgfpathlineto{\pgfqpoint{3.475566in}{1.773914in}}%
\pgfpathlineto{\pgfqpoint{3.478073in}{1.774416in}}%
\pgfpathlineto{\pgfqpoint{3.480580in}{1.778745in}}%
\pgfpathlineto{\pgfqpoint{3.483086in}{1.779477in}}%
\pgfpathlineto{\pgfqpoint{3.485593in}{1.784156in}}%
\pgfpathlineto{\pgfqpoint{3.490606in}{1.788134in}}%
\pgfpathlineto{\pgfqpoint{3.498126in}{1.795269in}}%
\pgfpathlineto{\pgfqpoint{3.505646in}{1.802013in}}%
\pgfpathlineto{\pgfqpoint{3.513166in}{1.806227in}}%
\pgfpathlineto{\pgfqpoint{3.515673in}{1.814357in}}%
\pgfpathlineto{\pgfqpoint{3.520686in}{1.818573in}}%
\pgfpathlineto{\pgfqpoint{3.530713in}{1.832159in}}%
\pgfpathlineto{\pgfqpoint{3.533219in}{1.834116in}}%
\pgfpathlineto{\pgfqpoint{3.535726in}{1.838741in}}%
\pgfpathlineto{\pgfqpoint{3.538233in}{1.839857in}}%
\pgfpathlineto{\pgfqpoint{3.540739in}{1.851592in}}%
\pgfpathlineto{\pgfqpoint{3.543246in}{1.855072in}}%
\pgfpathlineto{\pgfqpoint{3.550766in}{1.858328in}}%
\pgfpathlineto{\pgfqpoint{3.553273in}{1.874444in}}%
\pgfpathlineto{\pgfqpoint{3.558286in}{1.880759in}}%
\pgfpathlineto{\pgfqpoint{3.563299in}{1.895040in}}%
\pgfpathlineto{\pgfqpoint{3.565806in}{1.896952in}}%
\pgfpathlineto{\pgfqpoint{3.568313in}{1.897319in}}%
\pgfpathlineto{\pgfqpoint{3.570819in}{1.899749in}}%
\pgfpathlineto{\pgfqpoint{3.573326in}{1.904860in}}%
\pgfpathlineto{\pgfqpoint{3.578339in}{1.905853in}}%
\pgfpathlineto{\pgfqpoint{3.583352in}{1.916079in}}%
\pgfpathlineto{\pgfqpoint{3.585859in}{1.920091in}}%
\pgfpathlineto{\pgfqpoint{3.595886in}{1.921637in}}%
\pgfpathlineto{\pgfqpoint{3.598392in}{1.928558in}}%
\pgfpathlineto{\pgfqpoint{3.605912in}{1.933979in}}%
\pgfpathlineto{\pgfqpoint{3.608419in}{1.939952in}}%
\pgfpathlineto{\pgfqpoint{3.615939in}{1.947147in}}%
\pgfpathlineto{\pgfqpoint{3.618446in}{1.947604in}}%
\pgfpathlineto{\pgfqpoint{3.620952in}{1.957964in}}%
\pgfpathlineto{\pgfqpoint{3.623459in}{1.958393in}}%
\pgfpathlineto{\pgfqpoint{3.625966in}{1.961114in}}%
\pgfpathlineto{\pgfqpoint{3.628472in}{1.961157in}}%
\pgfpathlineto{\pgfqpoint{3.630979in}{1.964952in}}%
\pgfpathlineto{\pgfqpoint{3.635992in}{1.965364in}}%
\pgfpathlineto{\pgfqpoint{3.638499in}{1.968423in}}%
\pgfpathlineto{\pgfqpoint{3.663565in}{1.975315in}}%
\pgfpathlineto{\pgfqpoint{3.666072in}{1.980818in}}%
\pgfpathlineto{\pgfqpoint{3.668579in}{1.981792in}}%
\pgfpathlineto{\pgfqpoint{3.671085in}{1.985784in}}%
\pgfpathlineto{\pgfqpoint{3.673592in}{1.986003in}}%
\pgfpathlineto{\pgfqpoint{3.678605in}{1.988106in}}%
\pgfpathlineto{\pgfqpoint{3.683619in}{1.989541in}}%
\pgfpathlineto{\pgfqpoint{3.691139in}{1.996852in}}%
\pgfpathlineto{\pgfqpoint{3.696152in}{1.997809in}}%
\pgfpathlineto{\pgfqpoint{3.706178in}{2.001548in}}%
\pgfpathlineto{\pgfqpoint{3.708685in}{2.008822in}}%
\pgfpathlineto{\pgfqpoint{3.711192in}{2.009109in}}%
\pgfpathlineto{\pgfqpoint{3.721218in}{2.018719in}}%
\pgfpathlineto{\pgfqpoint{3.726232in}{2.028732in}}%
\pgfpathlineto{\pgfqpoint{3.733752in}{2.032945in}}%
\pgfpathlineto{\pgfqpoint{3.736258in}{2.044350in}}%
\pgfpathlineto{\pgfqpoint{3.738765in}{2.045592in}}%
\pgfpathlineto{\pgfqpoint{3.741272in}{2.050295in}}%
\pgfpathlineto{\pgfqpoint{3.743778in}{2.051095in}}%
\pgfpathlineto{\pgfqpoint{3.748792in}{2.069716in}}%
\pgfpathlineto{\pgfqpoint{3.758818in}{2.080247in}}%
\pgfpathlineto{\pgfqpoint{3.761325in}{2.100522in}}%
\pgfpathlineto{\pgfqpoint{3.766338in}{2.104627in}}%
\pgfpathlineto{\pgfqpoint{3.768845in}{2.128752in}}%
\pgfpathlineto{\pgfqpoint{3.771352in}{2.129307in}}%
\pgfpathlineto{\pgfqpoint{3.773858in}{2.131820in}}%
\pgfpathlineto{\pgfqpoint{3.776365in}{2.132178in}}%
\pgfpathlineto{\pgfqpoint{3.781378in}{2.134710in}}%
\pgfpathlineto{\pgfqpoint{3.783885in}{2.134833in}}%
\pgfpathlineto{\pgfqpoint{3.786391in}{2.141046in}}%
\pgfpathlineto{\pgfqpoint{3.788898in}{2.141223in}}%
\pgfpathlineto{\pgfqpoint{3.791405in}{2.144077in}}%
\pgfpathlineto{\pgfqpoint{3.793911in}{2.148773in}}%
\pgfpathlineto{\pgfqpoint{3.796418in}{2.150823in}}%
\pgfpathlineto{\pgfqpoint{3.798925in}{2.150878in}}%
\pgfpathlineto{\pgfqpoint{3.803938in}{2.154044in}}%
\pgfpathlineto{\pgfqpoint{3.806445in}{2.154863in}}%
\pgfpathlineto{\pgfqpoint{3.808951in}{2.158705in}}%
\pgfpathlineto{\pgfqpoint{3.811458in}{2.159715in}}%
\pgfpathlineto{\pgfqpoint{3.816471in}{2.163294in}}%
\pgfpathlineto{\pgfqpoint{3.818978in}{2.168408in}}%
\pgfpathlineto{\pgfqpoint{3.821485in}{2.169036in}}%
\pgfpathlineto{\pgfqpoint{3.829005in}{2.176489in}}%
\pgfpathlineto{\pgfqpoint{3.831511in}{2.184586in}}%
\pgfpathlineto{\pgfqpoint{3.836525in}{2.185921in}}%
\pgfpathlineto{\pgfqpoint{3.839031in}{2.186239in}}%
\pgfpathlineto{\pgfqpoint{3.844044in}{2.193529in}}%
\pgfpathlineto{\pgfqpoint{3.846551in}{2.195906in}}%
\pgfpathlineto{\pgfqpoint{3.851564in}{2.198123in}}%
\pgfpathlineto{\pgfqpoint{3.854071in}{2.200186in}}%
\pgfpathlineto{\pgfqpoint{3.856578in}{2.205076in}}%
\pgfpathlineto{\pgfqpoint{3.861591in}{2.208011in}}%
\pgfpathlineto{\pgfqpoint{3.864098in}{2.208530in}}%
\pgfpathlineto{\pgfqpoint{3.866604in}{2.210226in}}%
\pgfpathlineto{\pgfqpoint{3.869111in}{2.220130in}}%
\pgfpathlineto{\pgfqpoint{3.871618in}{2.220920in}}%
\pgfpathlineto{\pgfqpoint{3.874124in}{2.227933in}}%
\pgfpathlineto{\pgfqpoint{3.881644in}{2.235916in}}%
\pgfpathlineto{\pgfqpoint{3.884151in}{2.236171in}}%
\pgfpathlineto{\pgfqpoint{3.889164in}{2.239914in}}%
\pgfpathlineto{\pgfqpoint{3.896684in}{2.240816in}}%
\pgfpathlineto{\pgfqpoint{3.901698in}{2.244848in}}%
\pgfpathlineto{\pgfqpoint{3.906711in}{2.254545in}}%
\pgfpathlineto{\pgfqpoint{3.911724in}{2.259910in}}%
\pgfpathlineto{\pgfqpoint{3.914231in}{2.259925in}}%
\pgfpathlineto{\pgfqpoint{3.916737in}{2.262608in}}%
\pgfpathlineto{\pgfqpoint{3.919244in}{2.269248in}}%
\pgfpathlineto{\pgfqpoint{3.921751in}{2.269744in}}%
\pgfpathlineto{\pgfqpoint{3.924257in}{2.273497in}}%
\pgfpathlineto{\pgfqpoint{3.929271in}{2.274056in}}%
\pgfpathlineto{\pgfqpoint{3.939297in}{2.283505in}}%
\pgfpathlineto{\pgfqpoint{3.946817in}{2.284935in}}%
\pgfpathlineto{\pgfqpoint{3.949324in}{2.297120in}}%
\pgfpathlineto{\pgfqpoint{3.954337in}{2.297814in}}%
\pgfpathlineto{\pgfqpoint{3.964364in}{2.302430in}}%
\pgfpathlineto{\pgfqpoint{3.969377in}{2.309606in}}%
\pgfpathlineto{\pgfqpoint{3.974391in}{2.310114in}}%
\pgfpathlineto{\pgfqpoint{3.979404in}{2.312234in}}%
\pgfpathlineto{\pgfqpoint{3.981910in}{2.312267in}}%
\pgfpathlineto{\pgfqpoint{3.984417in}{2.323828in}}%
\pgfpathlineto{\pgfqpoint{3.991937in}{2.328715in}}%
\pgfpathlineto{\pgfqpoint{3.996950in}{2.331297in}}%
\pgfpathlineto{\pgfqpoint{3.999457in}{2.334133in}}%
\pgfpathlineto{\pgfqpoint{4.001964in}{2.334253in}}%
\pgfpathlineto{\pgfqpoint{4.004470in}{2.345225in}}%
\pgfpathlineto{\pgfqpoint{4.006977in}{2.345804in}}%
\pgfpathlineto{\pgfqpoint{4.009484in}{2.348416in}}%
\pgfpathlineto{\pgfqpoint{4.011990in}{2.349243in}}%
\pgfpathlineto{\pgfqpoint{4.014497in}{2.356714in}}%
\pgfpathlineto{\pgfqpoint{4.017004in}{2.359030in}}%
\pgfpathlineto{\pgfqpoint{4.022017in}{2.361012in}}%
\pgfpathlineto{\pgfqpoint{4.024524in}{2.363392in}}%
\pgfpathlineto{\pgfqpoint{4.032044in}{2.365480in}}%
\pgfpathlineto{\pgfqpoint{4.034550in}{2.366098in}}%
\pgfpathlineto{\pgfqpoint{4.037057in}{2.370368in}}%
\pgfpathlineto{\pgfqpoint{4.054603in}{2.380714in}}%
\pgfpathlineto{\pgfqpoint{4.059617in}{2.391518in}}%
\pgfpathlineto{\pgfqpoint{4.062123in}{2.392179in}}%
\pgfpathlineto{\pgfqpoint{4.064630in}{2.397936in}}%
\pgfpathlineto{\pgfqpoint{4.067137in}{2.398967in}}%
\pgfpathlineto{\pgfqpoint{4.069643in}{2.402725in}}%
\pgfpathlineto{\pgfqpoint{4.072150in}{2.412557in}}%
\pgfpathlineto{\pgfqpoint{4.074657in}{2.413069in}}%
\pgfpathlineto{\pgfqpoint{4.087190in}{2.435834in}}%
\pgfpathlineto{\pgfqpoint{4.089697in}{2.437214in}}%
\pgfpathlineto{\pgfqpoint{4.092203in}{2.442295in}}%
\pgfpathlineto{\pgfqpoint{4.094710in}{2.444359in}}%
\pgfpathlineto{\pgfqpoint{4.097217in}{2.452165in}}%
\pgfpathlineto{\pgfqpoint{4.099723in}{2.453549in}}%
\pgfpathlineto{\pgfqpoint{4.102230in}{2.456959in}}%
\pgfpathlineto{\pgfqpoint{4.104737in}{2.457151in}}%
\pgfpathlineto{\pgfqpoint{4.109750in}{2.460988in}}%
\pgfpathlineto{\pgfqpoint{4.112257in}{2.470857in}}%
\pgfpathlineto{\pgfqpoint{4.117270in}{2.478527in}}%
\pgfpathlineto{\pgfqpoint{4.127296in}{2.484006in}}%
\pgfpathlineto{\pgfqpoint{4.134816in}{2.508515in}}%
\pgfpathlineto{\pgfqpoint{4.137323in}{2.515283in}}%
\pgfpathlineto{\pgfqpoint{4.139830in}{2.517781in}}%
\pgfpathlineto{\pgfqpoint{4.142336in}{2.529576in}}%
\pgfpathlineto{\pgfqpoint{4.147350in}{2.533139in}}%
\pgfpathlineto{\pgfqpoint{4.149856in}{2.548633in}}%
\pgfpathlineto{\pgfqpoint{4.157376in}{2.552817in}}%
\pgfpathlineto{\pgfqpoint{4.159883in}{2.555323in}}%
\pgfpathlineto{\pgfqpoint{4.162390in}{2.559593in}}%
\pgfpathlineto{\pgfqpoint{4.167403in}{2.562783in}}%
\pgfpathlineto{\pgfqpoint{4.169910in}{2.568180in}}%
\pgfpathlineto{\pgfqpoint{4.172416in}{2.570650in}}%
\pgfpathlineto{\pgfqpoint{4.174923in}{2.590977in}}%
\pgfpathlineto{\pgfqpoint{4.177430in}{2.594080in}}%
\pgfpathlineto{\pgfqpoint{4.179936in}{2.594686in}}%
\pgfpathlineto{\pgfqpoint{4.182443in}{2.599163in}}%
\pgfpathlineto{\pgfqpoint{4.184949in}{2.608249in}}%
\pgfpathlineto{\pgfqpoint{4.192469in}{2.617042in}}%
\pgfpathlineto{\pgfqpoint{4.194976in}{2.620994in}}%
\pgfpathlineto{\pgfqpoint{4.197483in}{2.621530in}}%
\pgfpathlineto{\pgfqpoint{4.199989in}{2.637508in}}%
\pgfpathlineto{\pgfqpoint{4.205003in}{2.639913in}}%
\pgfpathlineto{\pgfqpoint{4.207509in}{2.647267in}}%
\pgfpathlineto{\pgfqpoint{4.210016in}{2.647283in}}%
\pgfpathlineto{\pgfqpoint{4.215029in}{2.660669in}}%
\pgfpathlineto{\pgfqpoint{4.217536in}{2.665217in}}%
\pgfpathlineto{\pgfqpoint{4.220043in}{2.674503in}}%
\pgfpathlineto{\pgfqpoint{4.227563in}{2.681839in}}%
\pgfpathlineto{\pgfqpoint{4.230069in}{2.687145in}}%
\pgfpathlineto{\pgfqpoint{4.232576in}{2.707655in}}%
\pgfpathlineto{\pgfqpoint{4.235083in}{2.713455in}}%
\pgfpathlineto{\pgfqpoint{4.237589in}{2.714926in}}%
\pgfpathlineto{\pgfqpoint{4.240096in}{2.723043in}}%
\pgfpathlineto{\pgfqpoint{4.242603in}{2.725836in}}%
\pgfpathlineto{\pgfqpoint{4.252629in}{2.744281in}}%
\pgfpathlineto{\pgfqpoint{4.255136in}{2.744312in}}%
\pgfpathlineto{\pgfqpoint{4.257642in}{2.746505in}}%
\pgfpathlineto{\pgfqpoint{4.260149in}{2.746977in}}%
\pgfpathlineto{\pgfqpoint{4.262656in}{2.757430in}}%
\pgfpathlineto{\pgfqpoint{4.265162in}{2.762164in}}%
\pgfpathlineto{\pgfqpoint{4.267669in}{2.781996in}}%
\pgfpathlineto{\pgfqpoint{4.270176in}{2.787137in}}%
\pgfpathlineto{\pgfqpoint{4.272682in}{2.789302in}}%
\pgfpathlineto{\pgfqpoint{4.280202in}{2.800156in}}%
\pgfpathlineto{\pgfqpoint{4.285216in}{2.806979in}}%
\pgfpathlineto{\pgfqpoint{4.287722in}{2.809387in}}%
\pgfpathlineto{\pgfqpoint{4.290229in}{2.824568in}}%
\pgfpathlineto{\pgfqpoint{4.295242in}{2.829113in}}%
\pgfpathlineto{\pgfqpoint{4.297749in}{2.846058in}}%
\pgfpathlineto{\pgfqpoint{4.302762in}{2.846891in}}%
\pgfpathlineto{\pgfqpoint{4.307776in}{2.856064in}}%
\pgfpathlineto{\pgfqpoint{4.312789in}{2.858427in}}%
\pgfpathlineto{\pgfqpoint{4.315296in}{2.860524in}}%
\pgfpathlineto{\pgfqpoint{4.320309in}{2.860810in}}%
\pgfpathlineto{\pgfqpoint{4.322815in}{2.867237in}}%
\pgfpathlineto{\pgfqpoint{4.325322in}{2.877158in}}%
\pgfpathlineto{\pgfqpoint{4.330335in}{2.879351in}}%
\pgfpathlineto{\pgfqpoint{4.332842in}{2.894248in}}%
\pgfpathlineto{\pgfqpoint{4.342869in}{2.899596in}}%
\pgfpathlineto{\pgfqpoint{4.345375in}{2.905275in}}%
\pgfpathlineto{\pgfqpoint{4.345375in}{2.905275in}}%
\pgfusepath{stroke}%
\end{pgfscope}%
\begin{pgfscope}%
\pgfpathrectangle{\pgfqpoint{0.708220in}{0.535823in}}{\pgfqpoint{5.013309in}{2.369453in}}%
\pgfusepath{clip}%
\pgfsetbuttcap%
\pgfsetroundjoin%
\pgfsetlinewidth{1.003750pt}%
\definecolor{currentstroke}{rgb}{0.000000,0.000000,1.000000}%
\pgfsetstrokecolor{currentstroke}%
\pgfsetdash{{3.700000pt}{1.600000pt}}{0.000000pt}%
\pgfpathmoveto{\pgfqpoint{0.713233in}{0.531304in}}%
\pgfpathlineto{\pgfqpoint{0.720753in}{0.532358in}}%
\pgfpathlineto{\pgfqpoint{0.735793in}{0.536678in}}%
\pgfpathlineto{\pgfqpoint{0.738300in}{0.536849in}}%
\pgfpathlineto{\pgfqpoint{0.740806in}{0.538545in}}%
\pgfpathlineto{\pgfqpoint{0.745820in}{0.538882in}}%
\pgfpathlineto{\pgfqpoint{0.753340in}{0.540559in}}%
\pgfpathlineto{\pgfqpoint{0.755846in}{0.544027in}}%
\pgfpathlineto{\pgfqpoint{0.758353in}{0.647790in}}%
\pgfpathlineto{\pgfqpoint{0.760860in}{0.648592in}}%
\pgfpathlineto{\pgfqpoint{0.763366in}{0.650711in}}%
\pgfpathlineto{\pgfqpoint{0.785926in}{0.652978in}}%
\pgfpathlineto{\pgfqpoint{0.795953in}{0.653928in}}%
\pgfpathlineto{\pgfqpoint{0.826033in}{0.658182in}}%
\pgfpathlineto{\pgfqpoint{0.841073in}{0.660021in}}%
\pgfpathlineto{\pgfqpoint{0.851099in}{0.664453in}}%
\pgfpathlineto{\pgfqpoint{0.853606in}{0.669953in}}%
\pgfpathlineto{\pgfqpoint{0.856112in}{0.715565in}}%
\pgfpathlineto{\pgfqpoint{0.858619in}{0.718664in}}%
\pgfpathlineto{\pgfqpoint{0.861126in}{0.719371in}}%
\pgfpathlineto{\pgfqpoint{0.866139in}{0.722461in}}%
\pgfpathlineto{\pgfqpoint{0.878672in}{0.724242in}}%
\pgfpathlineto{\pgfqpoint{0.911259in}{0.730638in}}%
\pgfpathlineto{\pgfqpoint{0.913766in}{0.731955in}}%
\pgfpathlineto{\pgfqpoint{0.916272in}{0.738865in}}%
\pgfpathlineto{\pgfqpoint{0.918779in}{0.767471in}}%
\pgfpathlineto{\pgfqpoint{0.966405in}{0.774095in}}%
\pgfpathlineto{\pgfqpoint{0.988965in}{0.775585in}}%
\pgfpathlineto{\pgfqpoint{0.996485in}{0.776852in}}%
\pgfpathlineto{\pgfqpoint{1.006512in}{0.777900in}}%
\pgfpathlineto{\pgfqpoint{1.014032in}{0.780432in}}%
\pgfpathlineto{\pgfqpoint{1.029072in}{0.783008in}}%
\pgfpathlineto{\pgfqpoint{1.031578in}{0.806487in}}%
\pgfpathlineto{\pgfqpoint{1.046618in}{0.808562in}}%
\pgfpathlineto{\pgfqpoint{1.071685in}{0.811302in}}%
\pgfpathlineto{\pgfqpoint{1.101765in}{0.813116in}}%
\pgfpathlineto{\pgfqpoint{1.114298in}{0.814169in}}%
\pgfpathlineto{\pgfqpoint{1.134351in}{0.815752in}}%
\pgfpathlineto{\pgfqpoint{1.139364in}{0.815953in}}%
\pgfpathlineto{\pgfqpoint{1.146884in}{0.819368in}}%
\pgfpathlineto{\pgfqpoint{1.154404in}{0.820644in}}%
\pgfpathlineto{\pgfqpoint{1.156911in}{0.824892in}}%
\pgfpathlineto{\pgfqpoint{1.159418in}{0.826915in}}%
\pgfpathlineto{\pgfqpoint{1.161924in}{0.835882in}}%
\pgfpathlineto{\pgfqpoint{1.164431in}{0.837956in}}%
\pgfpathlineto{\pgfqpoint{1.174458in}{0.839423in}}%
\pgfpathlineto{\pgfqpoint{1.202031in}{0.841224in}}%
\pgfpathlineto{\pgfqpoint{1.217071in}{0.842520in}}%
\pgfpathlineto{\pgfqpoint{1.294777in}{0.850040in}}%
\pgfpathlineto{\pgfqpoint{1.299790in}{0.853707in}}%
\pgfpathlineto{\pgfqpoint{1.302297in}{0.863169in}}%
\pgfpathlineto{\pgfqpoint{1.307310in}{0.864386in}}%
\pgfpathlineto{\pgfqpoint{1.317337in}{0.867390in}}%
\pgfpathlineto{\pgfqpoint{1.342403in}{0.869682in}}%
\pgfpathlineto{\pgfqpoint{1.352430in}{0.870195in}}%
\pgfpathlineto{\pgfqpoint{1.374990in}{0.871750in}}%
\pgfpathlineto{\pgfqpoint{1.397550in}{0.874510in}}%
\pgfpathlineto{\pgfqpoint{1.400056in}{0.878602in}}%
\pgfpathlineto{\pgfqpoint{1.402563in}{0.887082in}}%
\pgfpathlineto{\pgfqpoint{1.437656in}{0.898485in}}%
\pgfpathlineto{\pgfqpoint{1.440163in}{0.909158in}}%
\pgfpathlineto{\pgfqpoint{1.447683in}{0.910205in}}%
\pgfpathlineto{\pgfqpoint{1.450190in}{0.910494in}}%
\pgfpathlineto{\pgfqpoint{1.452696in}{0.913175in}}%
\pgfpathlineto{\pgfqpoint{1.457710in}{0.913327in}}%
\pgfpathlineto{\pgfqpoint{1.462723in}{0.915665in}}%
\pgfpathlineto{\pgfqpoint{1.467736in}{0.916132in}}%
\pgfpathlineto{\pgfqpoint{1.470243in}{0.918948in}}%
\pgfpathlineto{\pgfqpoint{1.475256in}{0.919717in}}%
\pgfpathlineto{\pgfqpoint{1.477763in}{0.920957in}}%
\pgfpathlineto{\pgfqpoint{1.480269in}{0.926196in}}%
\pgfpathlineto{\pgfqpoint{1.510349in}{0.931638in}}%
\pgfpathlineto{\pgfqpoint{1.512856in}{0.931894in}}%
\pgfpathlineto{\pgfqpoint{1.517869in}{0.933504in}}%
\pgfpathlineto{\pgfqpoint{1.525389in}{0.934111in}}%
\pgfpathlineto{\pgfqpoint{1.527896in}{0.943071in}}%
\pgfpathlineto{\pgfqpoint{1.532909in}{0.944991in}}%
\pgfpathlineto{\pgfqpoint{1.535416in}{0.948321in}}%
\pgfpathlineto{\pgfqpoint{1.542936in}{0.949139in}}%
\pgfpathlineto{\pgfqpoint{1.547949in}{0.952948in}}%
\pgfpathlineto{\pgfqpoint{1.550456in}{0.958973in}}%
\pgfpathlineto{\pgfqpoint{1.555469in}{0.960190in}}%
\pgfpathlineto{\pgfqpoint{1.562989in}{0.962913in}}%
\pgfpathlineto{\pgfqpoint{1.580536in}{0.964723in}}%
\pgfpathlineto{\pgfqpoint{1.583042in}{0.974005in}}%
\pgfpathlineto{\pgfqpoint{1.590562in}{0.975253in}}%
\pgfpathlineto{\pgfqpoint{1.598082in}{0.976990in}}%
\pgfpathlineto{\pgfqpoint{1.600589in}{0.979422in}}%
\pgfpathlineto{\pgfqpoint{1.603095in}{0.985946in}}%
\pgfpathlineto{\pgfqpoint{1.608109in}{0.987209in}}%
\pgfpathlineto{\pgfqpoint{1.615629in}{0.989805in}}%
\pgfpathlineto{\pgfqpoint{1.620642in}{0.992590in}}%
\pgfpathlineto{\pgfqpoint{1.623149in}{0.997098in}}%
\pgfpathlineto{\pgfqpoint{1.628162in}{0.997505in}}%
\pgfpathlineto{\pgfqpoint{1.633175in}{1.000979in}}%
\pgfpathlineto{\pgfqpoint{1.638189in}{1.002701in}}%
\pgfpathlineto{\pgfqpoint{1.640695in}{1.007240in}}%
\pgfpathlineto{\pgfqpoint{1.643202in}{1.007273in}}%
\pgfpathlineto{\pgfqpoint{1.645709in}{1.009984in}}%
\pgfpathlineto{\pgfqpoint{1.658242in}{1.014126in}}%
\pgfpathlineto{\pgfqpoint{1.660749in}{1.018499in}}%
\pgfpathlineto{\pgfqpoint{1.665762in}{1.019146in}}%
\pgfpathlineto{\pgfqpoint{1.668268in}{1.020675in}}%
\pgfpathlineto{\pgfqpoint{1.673282in}{1.021949in}}%
\pgfpathlineto{\pgfqpoint{1.675788in}{1.023684in}}%
\pgfpathlineto{\pgfqpoint{1.678295in}{1.029692in}}%
\pgfpathlineto{\pgfqpoint{1.683308in}{1.031064in}}%
\pgfpathlineto{\pgfqpoint{1.685815in}{1.031455in}}%
\pgfpathlineto{\pgfqpoint{1.690828in}{1.033596in}}%
\pgfpathlineto{\pgfqpoint{1.693335in}{1.033624in}}%
\pgfpathlineto{\pgfqpoint{1.698348in}{1.038825in}}%
\pgfpathlineto{\pgfqpoint{1.705868in}{1.040054in}}%
\pgfpathlineto{\pgfqpoint{1.710882in}{1.040787in}}%
\pgfpathlineto{\pgfqpoint{1.715895in}{1.041012in}}%
\pgfpathlineto{\pgfqpoint{1.720908in}{1.044130in}}%
\pgfpathlineto{\pgfqpoint{1.723415in}{1.048380in}}%
\pgfpathlineto{\pgfqpoint{1.728428in}{1.049679in}}%
\pgfpathlineto{\pgfqpoint{1.740961in}{1.056922in}}%
\pgfpathlineto{\pgfqpoint{1.743468in}{1.057299in}}%
\pgfpathlineto{\pgfqpoint{1.745975in}{1.065590in}}%
\pgfpathlineto{\pgfqpoint{1.750988in}{1.067764in}}%
\pgfpathlineto{\pgfqpoint{1.753495in}{1.075494in}}%
\pgfpathlineto{\pgfqpoint{1.771041in}{1.081835in}}%
\pgfpathlineto{\pgfqpoint{1.773548in}{1.084633in}}%
\pgfpathlineto{\pgfqpoint{1.778561in}{1.086165in}}%
\pgfpathlineto{\pgfqpoint{1.786081in}{1.089411in}}%
\pgfpathlineto{\pgfqpoint{1.788588in}{1.094370in}}%
\pgfpathlineto{\pgfqpoint{1.796108in}{1.095700in}}%
\pgfpathlineto{\pgfqpoint{1.798615in}{1.096491in}}%
\pgfpathlineto{\pgfqpoint{1.801121in}{1.100023in}}%
\pgfpathlineto{\pgfqpoint{1.806134in}{1.100629in}}%
\pgfpathlineto{\pgfqpoint{1.808641in}{1.102995in}}%
\pgfpathlineto{\pgfqpoint{1.811148in}{1.103334in}}%
\pgfpathlineto{\pgfqpoint{1.813654in}{1.106864in}}%
\pgfpathlineto{\pgfqpoint{1.821174in}{1.107868in}}%
\pgfpathlineto{\pgfqpoint{1.823681in}{1.111248in}}%
\pgfpathlineto{\pgfqpoint{1.831201in}{1.114215in}}%
\pgfpathlineto{\pgfqpoint{1.836214in}{1.116988in}}%
\pgfpathlineto{\pgfqpoint{1.838721in}{1.118032in}}%
\pgfpathlineto{\pgfqpoint{1.841228in}{1.123603in}}%
\pgfpathlineto{\pgfqpoint{1.846241in}{1.125387in}}%
\pgfpathlineto{\pgfqpoint{1.848748in}{1.130482in}}%
\pgfpathlineto{\pgfqpoint{1.863788in}{1.140233in}}%
\pgfpathlineto{\pgfqpoint{1.868801in}{1.140698in}}%
\pgfpathlineto{\pgfqpoint{1.873814in}{1.144734in}}%
\pgfpathlineto{\pgfqpoint{1.878827in}{1.144896in}}%
\pgfpathlineto{\pgfqpoint{1.883841in}{1.147789in}}%
\pgfpathlineto{\pgfqpoint{1.886347in}{1.149460in}}%
\pgfpathlineto{\pgfqpoint{1.888854in}{1.149757in}}%
\pgfpathlineto{\pgfqpoint{1.891361in}{1.153756in}}%
\pgfpathlineto{\pgfqpoint{1.901387in}{1.156573in}}%
\pgfpathlineto{\pgfqpoint{1.906401in}{1.162091in}}%
\pgfpathlineto{\pgfqpoint{1.933974in}{1.169818in}}%
\pgfpathlineto{\pgfqpoint{1.938987in}{1.172687in}}%
\pgfpathlineto{\pgfqpoint{1.941494in}{1.172738in}}%
\pgfpathlineto{\pgfqpoint{1.949014in}{1.177217in}}%
\pgfpathlineto{\pgfqpoint{1.951520in}{1.177815in}}%
\pgfpathlineto{\pgfqpoint{1.956534in}{1.180358in}}%
\pgfpathlineto{\pgfqpoint{1.961547in}{1.182213in}}%
\pgfpathlineto{\pgfqpoint{1.964054in}{1.185148in}}%
\pgfpathlineto{\pgfqpoint{1.969067in}{1.185715in}}%
\pgfpathlineto{\pgfqpoint{1.974080in}{1.190767in}}%
\pgfpathlineto{\pgfqpoint{1.984107in}{1.191799in}}%
\pgfpathlineto{\pgfqpoint{1.986614in}{1.192121in}}%
\pgfpathlineto{\pgfqpoint{1.991627in}{1.199015in}}%
\pgfpathlineto{\pgfqpoint{2.001654in}{1.203040in}}%
\pgfpathlineto{\pgfqpoint{2.006667in}{1.204572in}}%
\pgfpathlineto{\pgfqpoint{2.011680in}{1.213299in}}%
\pgfpathlineto{\pgfqpoint{2.019200in}{1.214349in}}%
\pgfpathlineto{\pgfqpoint{2.024213in}{1.216151in}}%
\pgfpathlineto{\pgfqpoint{2.031733in}{1.217097in}}%
\pgfpathlineto{\pgfqpoint{2.034240in}{1.219311in}}%
\pgfpathlineto{\pgfqpoint{2.039253in}{1.220913in}}%
\pgfpathlineto{\pgfqpoint{2.054293in}{1.226084in}}%
\pgfpathlineto{\pgfqpoint{2.061813in}{1.229076in}}%
\pgfpathlineto{\pgfqpoint{2.064320in}{1.232629in}}%
\pgfpathlineto{\pgfqpoint{2.069333in}{1.234381in}}%
\pgfpathlineto{\pgfqpoint{2.074346in}{1.235081in}}%
\pgfpathlineto{\pgfqpoint{2.079360in}{1.236892in}}%
\pgfpathlineto{\pgfqpoint{2.084373in}{1.237595in}}%
\pgfpathlineto{\pgfqpoint{2.086880in}{1.240355in}}%
\pgfpathlineto{\pgfqpoint{2.091893in}{1.241743in}}%
\pgfpathlineto{\pgfqpoint{2.104426in}{1.252311in}}%
\pgfpathlineto{\pgfqpoint{2.106933in}{1.252387in}}%
\pgfpathlineto{\pgfqpoint{2.109440in}{1.255692in}}%
\pgfpathlineto{\pgfqpoint{2.111946in}{1.255766in}}%
\pgfpathlineto{\pgfqpoint{2.114453in}{1.258834in}}%
\pgfpathlineto{\pgfqpoint{2.124480in}{1.262628in}}%
\pgfpathlineto{\pgfqpoint{2.129493in}{1.264992in}}%
\pgfpathlineto{\pgfqpoint{2.132000in}{1.265053in}}%
\pgfpathlineto{\pgfqpoint{2.134506in}{1.266371in}}%
\pgfpathlineto{\pgfqpoint{2.139520in}{1.273911in}}%
\pgfpathlineto{\pgfqpoint{2.149546in}{1.277345in}}%
\pgfpathlineto{\pgfqpoint{2.167093in}{1.288240in}}%
\pgfpathlineto{\pgfqpoint{2.174613in}{1.289661in}}%
\pgfpathlineto{\pgfqpoint{2.177119in}{1.292125in}}%
\pgfpathlineto{\pgfqpoint{2.179626in}{1.292610in}}%
\pgfpathlineto{\pgfqpoint{2.187146in}{1.300227in}}%
\pgfpathlineto{\pgfqpoint{2.192159in}{1.302270in}}%
\pgfpathlineto{\pgfqpoint{2.194666in}{1.302395in}}%
\pgfpathlineto{\pgfqpoint{2.199679in}{1.308257in}}%
\pgfpathlineto{\pgfqpoint{2.204693in}{1.309588in}}%
\pgfpathlineto{\pgfqpoint{2.207199in}{1.314798in}}%
\pgfpathlineto{\pgfqpoint{2.209706in}{1.315167in}}%
\pgfpathlineto{\pgfqpoint{2.214719in}{1.318124in}}%
\pgfpathlineto{\pgfqpoint{2.227252in}{1.320593in}}%
\pgfpathlineto{\pgfqpoint{2.229759in}{1.323081in}}%
\pgfpathlineto{\pgfqpoint{2.232266in}{1.323825in}}%
\pgfpathlineto{\pgfqpoint{2.237279in}{1.331442in}}%
\pgfpathlineto{\pgfqpoint{2.239786in}{1.332946in}}%
\pgfpathlineto{\pgfqpoint{2.244799in}{1.333491in}}%
\pgfpathlineto{\pgfqpoint{2.249812in}{1.339152in}}%
\pgfpathlineto{\pgfqpoint{2.257332in}{1.347482in}}%
\pgfpathlineto{\pgfqpoint{2.259839in}{1.349574in}}%
\pgfpathlineto{\pgfqpoint{2.267359in}{1.350069in}}%
\pgfpathlineto{\pgfqpoint{2.269866in}{1.353039in}}%
\pgfpathlineto{\pgfqpoint{2.272372in}{1.353950in}}%
\pgfpathlineto{\pgfqpoint{2.274879in}{1.358100in}}%
\pgfpathlineto{\pgfqpoint{2.284905in}{1.358610in}}%
\pgfpathlineto{\pgfqpoint{2.297439in}{1.367576in}}%
\pgfpathlineto{\pgfqpoint{2.299945in}{1.367651in}}%
\pgfpathlineto{\pgfqpoint{2.304959in}{1.374403in}}%
\pgfpathlineto{\pgfqpoint{2.307465in}{1.375589in}}%
\pgfpathlineto{\pgfqpoint{2.309972in}{1.379800in}}%
\pgfpathlineto{\pgfqpoint{2.312479in}{1.380508in}}%
\pgfpathlineto{\pgfqpoint{2.314985in}{1.382442in}}%
\pgfpathlineto{\pgfqpoint{2.317492in}{1.382747in}}%
\pgfpathlineto{\pgfqpoint{2.325012in}{1.387246in}}%
\pgfpathlineto{\pgfqpoint{2.327519in}{1.387530in}}%
\pgfpathlineto{\pgfqpoint{2.330025in}{1.390495in}}%
\pgfpathlineto{\pgfqpoint{2.332532in}{1.390710in}}%
\pgfpathlineto{\pgfqpoint{2.337545in}{1.393508in}}%
\pgfpathlineto{\pgfqpoint{2.342559in}{1.395941in}}%
\pgfpathlineto{\pgfqpoint{2.347572in}{1.398316in}}%
\pgfpathlineto{\pgfqpoint{2.355092in}{1.399586in}}%
\pgfpathlineto{\pgfqpoint{2.357598in}{1.402204in}}%
\pgfpathlineto{\pgfqpoint{2.387678in}{1.410845in}}%
\pgfpathlineto{\pgfqpoint{2.390185in}{1.414193in}}%
\pgfpathlineto{\pgfqpoint{2.395198in}{1.416686in}}%
\pgfpathlineto{\pgfqpoint{2.397705in}{1.418782in}}%
\pgfpathlineto{\pgfqpoint{2.402718in}{1.419532in}}%
\pgfpathlineto{\pgfqpoint{2.405225in}{1.422686in}}%
\pgfpathlineto{\pgfqpoint{2.415251in}{1.427022in}}%
\pgfpathlineto{\pgfqpoint{2.417758in}{1.432543in}}%
\pgfpathlineto{\pgfqpoint{2.425278in}{1.435962in}}%
\pgfpathlineto{\pgfqpoint{2.430291in}{1.437777in}}%
\pgfpathlineto{\pgfqpoint{2.432798in}{1.440887in}}%
\pgfpathlineto{\pgfqpoint{2.435305in}{1.441846in}}%
\pgfpathlineto{\pgfqpoint{2.437811in}{1.448321in}}%
\pgfpathlineto{\pgfqpoint{2.440318in}{1.448803in}}%
\pgfpathlineto{\pgfqpoint{2.442825in}{1.452807in}}%
\pgfpathlineto{\pgfqpoint{2.457865in}{1.458104in}}%
\pgfpathlineto{\pgfqpoint{2.460371in}{1.469593in}}%
\pgfpathlineto{\pgfqpoint{2.462878in}{1.470845in}}%
\pgfpathlineto{\pgfqpoint{2.465385in}{1.475155in}}%
\pgfpathlineto{\pgfqpoint{2.477918in}{1.480142in}}%
\pgfpathlineto{\pgfqpoint{2.490451in}{1.486256in}}%
\pgfpathlineto{\pgfqpoint{2.495464in}{1.488179in}}%
\pgfpathlineto{\pgfqpoint{2.500478in}{1.495811in}}%
\pgfpathlineto{\pgfqpoint{2.502984in}{1.502271in}}%
\pgfpathlineto{\pgfqpoint{2.505491in}{1.505190in}}%
\pgfpathlineto{\pgfqpoint{2.533064in}{1.511648in}}%
\pgfpathlineto{\pgfqpoint{2.538078in}{1.513382in}}%
\pgfpathlineto{\pgfqpoint{2.540584in}{1.513623in}}%
\pgfpathlineto{\pgfqpoint{2.543091in}{1.515678in}}%
\pgfpathlineto{\pgfqpoint{2.548104in}{1.522480in}}%
\pgfpathlineto{\pgfqpoint{2.553117in}{1.525354in}}%
\pgfpathlineto{\pgfqpoint{2.560637in}{1.527481in}}%
\pgfpathlineto{\pgfqpoint{2.565651in}{1.531701in}}%
\pgfpathlineto{\pgfqpoint{2.568157in}{1.535022in}}%
\pgfpathlineto{\pgfqpoint{2.575677in}{1.535677in}}%
\pgfpathlineto{\pgfqpoint{2.578184in}{1.539398in}}%
\pgfpathlineto{\pgfqpoint{2.583197in}{1.542350in}}%
\pgfpathlineto{\pgfqpoint{2.588211in}{1.546444in}}%
\pgfpathlineto{\pgfqpoint{2.590717in}{1.546624in}}%
\pgfpathlineto{\pgfqpoint{2.598237in}{1.550717in}}%
\pgfpathlineto{\pgfqpoint{2.600744in}{1.556553in}}%
\pgfpathlineto{\pgfqpoint{2.608264in}{1.558923in}}%
\pgfpathlineto{\pgfqpoint{2.610771in}{1.560680in}}%
\pgfpathlineto{\pgfqpoint{2.615784in}{1.560935in}}%
\pgfpathlineto{\pgfqpoint{2.618290in}{1.562193in}}%
\pgfpathlineto{\pgfqpoint{2.620797in}{1.564927in}}%
\pgfpathlineto{\pgfqpoint{2.623304in}{1.565765in}}%
\pgfpathlineto{\pgfqpoint{2.628317in}{1.571186in}}%
\pgfpathlineto{\pgfqpoint{2.630824in}{1.571245in}}%
\pgfpathlineto{\pgfqpoint{2.633330in}{1.575050in}}%
\pgfpathlineto{\pgfqpoint{2.640850in}{1.575571in}}%
\pgfpathlineto{\pgfqpoint{2.643357in}{1.578120in}}%
\pgfpathlineto{\pgfqpoint{2.655890in}{1.579862in}}%
\pgfpathlineto{\pgfqpoint{2.668424in}{1.586042in}}%
\pgfpathlineto{\pgfqpoint{2.685970in}{1.601467in}}%
\pgfpathlineto{\pgfqpoint{2.688477in}{1.606808in}}%
\pgfpathlineto{\pgfqpoint{2.690983in}{1.607234in}}%
\pgfpathlineto{\pgfqpoint{2.693490in}{1.610342in}}%
\pgfpathlineto{\pgfqpoint{2.703517in}{1.611920in}}%
\pgfpathlineto{\pgfqpoint{2.708530in}{1.613996in}}%
\pgfpathlineto{\pgfqpoint{2.711037in}{1.614727in}}%
\pgfpathlineto{\pgfqpoint{2.716050in}{1.618500in}}%
\pgfpathlineto{\pgfqpoint{2.718557in}{1.619550in}}%
\pgfpathlineto{\pgfqpoint{2.721063in}{1.622465in}}%
\pgfpathlineto{\pgfqpoint{2.723570in}{1.623157in}}%
\pgfpathlineto{\pgfqpoint{2.726077in}{1.633234in}}%
\pgfpathlineto{\pgfqpoint{2.741117in}{1.636216in}}%
\pgfpathlineto{\pgfqpoint{2.746130in}{1.637490in}}%
\pgfpathlineto{\pgfqpoint{2.748637in}{1.639928in}}%
\pgfpathlineto{\pgfqpoint{2.751143in}{1.640262in}}%
\pgfpathlineto{\pgfqpoint{2.753650in}{1.644624in}}%
\pgfpathlineto{\pgfqpoint{2.756156in}{1.645231in}}%
\pgfpathlineto{\pgfqpoint{2.758663in}{1.647748in}}%
\pgfpathlineto{\pgfqpoint{2.766183in}{1.649784in}}%
\pgfpathlineto{\pgfqpoint{2.776210in}{1.652405in}}%
\pgfpathlineto{\pgfqpoint{2.783730in}{1.653516in}}%
\pgfpathlineto{\pgfqpoint{2.786236in}{1.660356in}}%
\pgfpathlineto{\pgfqpoint{2.788743in}{1.660548in}}%
\pgfpathlineto{\pgfqpoint{2.793756in}{1.662058in}}%
\pgfpathlineto{\pgfqpoint{2.798770in}{1.663608in}}%
\pgfpathlineto{\pgfqpoint{2.803783in}{1.664981in}}%
\pgfpathlineto{\pgfqpoint{2.806290in}{1.670233in}}%
\pgfpathlineto{\pgfqpoint{2.811303in}{1.674370in}}%
\pgfpathlineto{\pgfqpoint{2.816316in}{1.678106in}}%
\pgfpathlineto{\pgfqpoint{2.821329in}{1.683629in}}%
\pgfpathlineto{\pgfqpoint{2.823836in}{1.685397in}}%
\pgfpathlineto{\pgfqpoint{2.826343in}{1.691652in}}%
\pgfpathlineto{\pgfqpoint{2.828849in}{1.692650in}}%
\pgfpathlineto{\pgfqpoint{2.833863in}{1.697262in}}%
\pgfpathlineto{\pgfqpoint{2.836369in}{1.697672in}}%
\pgfpathlineto{\pgfqpoint{2.838876in}{1.700518in}}%
\pgfpathlineto{\pgfqpoint{2.841383in}{1.700841in}}%
\pgfpathlineto{\pgfqpoint{2.846396in}{1.704243in}}%
\pgfpathlineto{\pgfqpoint{2.848903in}{1.704628in}}%
\pgfpathlineto{\pgfqpoint{2.851409in}{1.713181in}}%
\pgfpathlineto{\pgfqpoint{2.853916in}{1.717221in}}%
\pgfpathlineto{\pgfqpoint{2.856423in}{1.718753in}}%
\pgfpathlineto{\pgfqpoint{2.861436in}{1.719702in}}%
\pgfpathlineto{\pgfqpoint{2.866449in}{1.721528in}}%
\pgfpathlineto{\pgfqpoint{2.868956in}{1.725606in}}%
\pgfpathlineto{\pgfqpoint{2.871463in}{1.726329in}}%
\pgfpathlineto{\pgfqpoint{2.876476in}{1.730773in}}%
\pgfpathlineto{\pgfqpoint{2.883996in}{1.733350in}}%
\pgfpathlineto{\pgfqpoint{2.886503in}{1.737081in}}%
\pgfpathlineto{\pgfqpoint{2.889009in}{1.737850in}}%
\pgfpathlineto{\pgfqpoint{2.891516in}{1.741674in}}%
\pgfpathlineto{\pgfqpoint{2.894022in}{1.749010in}}%
\pgfpathlineto{\pgfqpoint{2.901542in}{1.755017in}}%
\pgfpathlineto{\pgfqpoint{2.904049in}{1.756539in}}%
\pgfpathlineto{\pgfqpoint{2.906556in}{1.756622in}}%
\pgfpathlineto{\pgfqpoint{2.914076in}{1.759319in}}%
\pgfpathlineto{\pgfqpoint{2.916582in}{1.760308in}}%
\pgfpathlineto{\pgfqpoint{2.919089in}{1.763835in}}%
\pgfpathlineto{\pgfqpoint{2.926609in}{1.765681in}}%
\pgfpathlineto{\pgfqpoint{2.929116in}{1.768125in}}%
\pgfpathlineto{\pgfqpoint{2.931622in}{1.768783in}}%
\pgfpathlineto{\pgfqpoint{2.934129in}{1.771074in}}%
\pgfpathlineto{\pgfqpoint{2.941649in}{1.771636in}}%
\pgfpathlineto{\pgfqpoint{2.946662in}{1.780209in}}%
\pgfpathlineto{\pgfqpoint{2.951676in}{1.781462in}}%
\pgfpathlineto{\pgfqpoint{2.954182in}{1.784500in}}%
\pgfpathlineto{\pgfqpoint{2.956689in}{1.784774in}}%
\pgfpathlineto{\pgfqpoint{2.961702in}{1.787630in}}%
\pgfpathlineto{\pgfqpoint{2.964209in}{1.788665in}}%
\pgfpathlineto{\pgfqpoint{2.969222in}{1.792858in}}%
\pgfpathlineto{\pgfqpoint{2.971729in}{1.793296in}}%
\pgfpathlineto{\pgfqpoint{2.974235in}{1.795972in}}%
\pgfpathlineto{\pgfqpoint{2.979249in}{1.796467in}}%
\pgfpathlineto{\pgfqpoint{2.981755in}{1.798881in}}%
\pgfpathlineto{\pgfqpoint{2.984262in}{1.799689in}}%
\pgfpathlineto{\pgfqpoint{2.986769in}{1.802912in}}%
\pgfpathlineto{\pgfqpoint{2.989275in}{1.803096in}}%
\pgfpathlineto{\pgfqpoint{2.991782in}{1.805728in}}%
\pgfpathlineto{\pgfqpoint{2.996795in}{1.806780in}}%
\pgfpathlineto{\pgfqpoint{3.004315in}{1.818018in}}%
\pgfpathlineto{\pgfqpoint{3.009329in}{1.819035in}}%
\pgfpathlineto{\pgfqpoint{3.011835in}{1.825173in}}%
\pgfpathlineto{\pgfqpoint{3.014342in}{1.825596in}}%
\pgfpathlineto{\pgfqpoint{3.016849in}{1.827975in}}%
\pgfpathlineto{\pgfqpoint{3.019355in}{1.833689in}}%
\pgfpathlineto{\pgfqpoint{3.029382in}{1.839167in}}%
\pgfpathlineto{\pgfqpoint{3.031888in}{1.839508in}}%
\pgfpathlineto{\pgfqpoint{3.046928in}{1.848176in}}%
\pgfpathlineto{\pgfqpoint{3.051942in}{1.849326in}}%
\pgfpathlineto{\pgfqpoint{3.054448in}{1.850071in}}%
\pgfpathlineto{\pgfqpoint{3.056955in}{1.853127in}}%
\pgfpathlineto{\pgfqpoint{3.071995in}{1.856528in}}%
\pgfpathlineto{\pgfqpoint{3.077008in}{1.863330in}}%
\pgfpathlineto{\pgfqpoint{3.094555in}{1.869641in}}%
\pgfpathlineto{\pgfqpoint{3.097061in}{1.874532in}}%
\pgfpathlineto{\pgfqpoint{3.104581in}{1.878598in}}%
\pgfpathlineto{\pgfqpoint{3.119621in}{1.885201in}}%
\pgfpathlineto{\pgfqpoint{3.127141in}{1.890876in}}%
\pgfpathlineto{\pgfqpoint{3.134661in}{1.892440in}}%
\pgfpathlineto{\pgfqpoint{3.142181in}{1.895764in}}%
\pgfpathlineto{\pgfqpoint{3.159728in}{1.905058in}}%
\pgfpathlineto{\pgfqpoint{3.162234in}{1.908033in}}%
\pgfpathlineto{\pgfqpoint{3.167248in}{1.909565in}}%
\pgfpathlineto{\pgfqpoint{3.172261in}{1.911185in}}%
\pgfpathlineto{\pgfqpoint{3.174768in}{1.911361in}}%
\pgfpathlineto{\pgfqpoint{3.177274in}{1.914213in}}%
\pgfpathlineto{\pgfqpoint{3.179781in}{1.919623in}}%
\pgfpathlineto{\pgfqpoint{3.184794in}{1.920487in}}%
\pgfpathlineto{\pgfqpoint{3.189808in}{1.922640in}}%
\pgfpathlineto{\pgfqpoint{3.202341in}{1.923702in}}%
\pgfpathlineto{\pgfqpoint{3.209861in}{1.924758in}}%
\pgfpathlineto{\pgfqpoint{3.217381in}{1.926096in}}%
\pgfpathlineto{\pgfqpoint{3.229914in}{1.933177in}}%
\pgfpathlineto{\pgfqpoint{3.232421in}{1.936004in}}%
\pgfpathlineto{\pgfqpoint{3.234927in}{1.937069in}}%
\pgfpathlineto{\pgfqpoint{3.237434in}{1.943101in}}%
\pgfpathlineto{\pgfqpoint{3.244954in}{1.944364in}}%
\pgfpathlineto{\pgfqpoint{3.247461in}{1.947863in}}%
\pgfpathlineto{\pgfqpoint{3.252474in}{1.949384in}}%
\pgfpathlineto{\pgfqpoint{3.257487in}{1.949902in}}%
\pgfpathlineto{\pgfqpoint{3.262501in}{1.951775in}}%
\pgfpathlineto{\pgfqpoint{3.267514in}{1.952869in}}%
\pgfpathlineto{\pgfqpoint{3.272527in}{1.955687in}}%
\pgfpathlineto{\pgfqpoint{3.275034in}{1.959529in}}%
\pgfpathlineto{\pgfqpoint{3.285061in}{1.962485in}}%
\pgfpathlineto{\pgfqpoint{3.287567in}{1.968198in}}%
\pgfpathlineto{\pgfqpoint{3.307620in}{1.977405in}}%
\pgfpathlineto{\pgfqpoint{3.310127in}{1.978752in}}%
\pgfpathlineto{\pgfqpoint{3.312634in}{1.981909in}}%
\pgfpathlineto{\pgfqpoint{3.317647in}{1.982662in}}%
\pgfpathlineto{\pgfqpoint{3.322660in}{1.985321in}}%
\pgfpathlineto{\pgfqpoint{3.325167in}{1.987516in}}%
\pgfpathlineto{\pgfqpoint{3.337700in}{1.989213in}}%
\pgfpathlineto{\pgfqpoint{3.340207in}{1.989265in}}%
\pgfpathlineto{\pgfqpoint{3.345220in}{1.993104in}}%
\pgfpathlineto{\pgfqpoint{3.367780in}{1.998830in}}%
\pgfpathlineto{\pgfqpoint{3.375300in}{2.000428in}}%
\pgfpathlineto{\pgfqpoint{3.377807in}{2.000479in}}%
\pgfpathlineto{\pgfqpoint{3.382820in}{2.002102in}}%
\pgfpathlineto{\pgfqpoint{3.385327in}{2.002534in}}%
\pgfpathlineto{\pgfqpoint{3.387833in}{2.008283in}}%
\pgfpathlineto{\pgfqpoint{3.390340in}{2.008796in}}%
\pgfpathlineto{\pgfqpoint{3.395353in}{2.015091in}}%
\pgfpathlineto{\pgfqpoint{3.405380in}{2.016507in}}%
\pgfpathlineto{\pgfqpoint{3.407887in}{2.019182in}}%
\pgfpathlineto{\pgfqpoint{3.412900in}{2.020514in}}%
\pgfpathlineto{\pgfqpoint{3.417913in}{2.021502in}}%
\pgfpathlineto{\pgfqpoint{3.422927in}{2.024767in}}%
\pgfpathlineto{\pgfqpoint{3.435460in}{2.027202in}}%
\pgfpathlineto{\pgfqpoint{3.445486in}{2.033848in}}%
\pgfpathlineto{\pgfqpoint{3.447993in}{2.034293in}}%
\pgfpathlineto{\pgfqpoint{3.453006in}{2.037353in}}%
\pgfpathlineto{\pgfqpoint{3.455513in}{2.037738in}}%
\pgfpathlineto{\pgfqpoint{3.463033in}{2.044277in}}%
\pgfpathlineto{\pgfqpoint{3.468046in}{2.044489in}}%
\pgfpathlineto{\pgfqpoint{3.473060in}{2.047146in}}%
\pgfpathlineto{\pgfqpoint{3.475566in}{2.047843in}}%
\pgfpathlineto{\pgfqpoint{3.480580in}{2.052810in}}%
\pgfpathlineto{\pgfqpoint{3.483086in}{2.052865in}}%
\pgfpathlineto{\pgfqpoint{3.488100in}{2.054256in}}%
\pgfpathlineto{\pgfqpoint{3.493113in}{2.055884in}}%
\pgfpathlineto{\pgfqpoint{3.500633in}{2.059448in}}%
\pgfpathlineto{\pgfqpoint{3.508153in}{2.060752in}}%
\pgfpathlineto{\pgfqpoint{3.515673in}{2.062524in}}%
\pgfpathlineto{\pgfqpoint{3.523193in}{2.064699in}}%
\pgfpathlineto{\pgfqpoint{3.530713in}{2.071714in}}%
\pgfpathlineto{\pgfqpoint{3.560793in}{2.080507in}}%
\pgfpathlineto{\pgfqpoint{3.563299in}{2.080524in}}%
\pgfpathlineto{\pgfqpoint{3.573326in}{2.087408in}}%
\pgfpathlineto{\pgfqpoint{3.590872in}{2.092929in}}%
\pgfpathlineto{\pgfqpoint{3.593379in}{2.094983in}}%
\pgfpathlineto{\pgfqpoint{3.598392in}{2.096193in}}%
\pgfpathlineto{\pgfqpoint{3.600899in}{2.098935in}}%
\pgfpathlineto{\pgfqpoint{3.603406in}{2.099641in}}%
\pgfpathlineto{\pgfqpoint{3.605912in}{2.104439in}}%
\pgfpathlineto{\pgfqpoint{3.613432in}{2.106323in}}%
\pgfpathlineto{\pgfqpoint{3.615939in}{2.109593in}}%
\pgfpathlineto{\pgfqpoint{3.625966in}{2.110570in}}%
\pgfpathlineto{\pgfqpoint{3.630979in}{2.112069in}}%
\pgfpathlineto{\pgfqpoint{3.638499in}{2.116132in}}%
\pgfpathlineto{\pgfqpoint{3.641005in}{2.118626in}}%
\pgfpathlineto{\pgfqpoint{3.648525in}{2.119714in}}%
\pgfpathlineto{\pgfqpoint{3.651032in}{2.123861in}}%
\pgfpathlineto{\pgfqpoint{3.658552in}{2.127203in}}%
\pgfpathlineto{\pgfqpoint{3.663565in}{2.128902in}}%
\pgfpathlineto{\pgfqpoint{3.666072in}{2.131217in}}%
\pgfpathlineto{\pgfqpoint{3.668579in}{2.131434in}}%
\pgfpathlineto{\pgfqpoint{3.673592in}{2.133241in}}%
\pgfpathlineto{\pgfqpoint{3.676099in}{2.141683in}}%
\pgfpathlineto{\pgfqpoint{3.686125in}{2.143488in}}%
\pgfpathlineto{\pgfqpoint{3.696152in}{2.149969in}}%
\pgfpathlineto{\pgfqpoint{3.698659in}{2.153488in}}%
\pgfpathlineto{\pgfqpoint{3.701165in}{2.154860in}}%
\pgfpathlineto{\pgfqpoint{3.703672in}{2.158062in}}%
\pgfpathlineto{\pgfqpoint{3.708685in}{2.159869in}}%
\pgfpathlineto{\pgfqpoint{3.711192in}{2.161992in}}%
\pgfpathlineto{\pgfqpoint{3.738765in}{2.168255in}}%
\pgfpathlineto{\pgfqpoint{3.741272in}{2.173637in}}%
\pgfpathlineto{\pgfqpoint{3.746285in}{2.174925in}}%
\pgfpathlineto{\pgfqpoint{3.758818in}{2.178711in}}%
\pgfpathlineto{\pgfqpoint{3.761325in}{2.184037in}}%
\pgfpathlineto{\pgfqpoint{3.763832in}{2.184176in}}%
\pgfpathlineto{\pgfqpoint{3.766338in}{2.189888in}}%
\pgfpathlineto{\pgfqpoint{3.768845in}{2.191580in}}%
\pgfpathlineto{\pgfqpoint{3.771352in}{2.191708in}}%
\pgfpathlineto{\pgfqpoint{3.773858in}{2.193346in}}%
\pgfpathlineto{\pgfqpoint{3.776365in}{2.198611in}}%
\pgfpathlineto{\pgfqpoint{3.781378in}{2.199498in}}%
\pgfpathlineto{\pgfqpoint{3.788898in}{2.207724in}}%
\pgfpathlineto{\pgfqpoint{3.791405in}{2.208688in}}%
\pgfpathlineto{\pgfqpoint{3.793911in}{2.211793in}}%
\pgfpathlineto{\pgfqpoint{3.798925in}{2.212669in}}%
\pgfpathlineto{\pgfqpoint{3.801431in}{2.213838in}}%
\pgfpathlineto{\pgfqpoint{3.803938in}{2.217712in}}%
\pgfpathlineto{\pgfqpoint{3.811458in}{2.220156in}}%
\pgfpathlineto{\pgfqpoint{3.813965in}{2.220635in}}%
\pgfpathlineto{\pgfqpoint{3.816471in}{2.224034in}}%
\pgfpathlineto{\pgfqpoint{3.823991in}{2.226023in}}%
\pgfpathlineto{\pgfqpoint{3.834018in}{2.227062in}}%
\pgfpathlineto{\pgfqpoint{3.836525in}{2.231601in}}%
\pgfpathlineto{\pgfqpoint{3.844044in}{2.235684in}}%
\pgfpathlineto{\pgfqpoint{3.859084in}{2.245120in}}%
\pgfpathlineto{\pgfqpoint{3.864098in}{2.246856in}}%
\pgfpathlineto{\pgfqpoint{3.869111in}{2.250289in}}%
\pgfpathlineto{\pgfqpoint{3.871618in}{2.263457in}}%
\pgfpathlineto{\pgfqpoint{3.874124in}{2.264377in}}%
\pgfpathlineto{\pgfqpoint{3.876631in}{2.267263in}}%
\pgfpathlineto{\pgfqpoint{3.879138in}{2.267489in}}%
\pgfpathlineto{\pgfqpoint{3.884151in}{2.270839in}}%
\pgfpathlineto{\pgfqpoint{3.886658in}{2.272180in}}%
\pgfpathlineto{\pgfqpoint{3.889164in}{2.274799in}}%
\pgfpathlineto{\pgfqpoint{3.896684in}{2.275390in}}%
\pgfpathlineto{\pgfqpoint{3.899191in}{2.277732in}}%
\pgfpathlineto{\pgfqpoint{3.901698in}{2.282068in}}%
\pgfpathlineto{\pgfqpoint{3.906711in}{2.284005in}}%
\pgfpathlineto{\pgfqpoint{3.909217in}{2.288919in}}%
\pgfpathlineto{\pgfqpoint{3.914231in}{2.289625in}}%
\pgfpathlineto{\pgfqpoint{3.916737in}{2.295292in}}%
\pgfpathlineto{\pgfqpoint{3.929271in}{2.305363in}}%
\pgfpathlineto{\pgfqpoint{3.931777in}{2.310164in}}%
\pgfpathlineto{\pgfqpoint{3.936791in}{2.311598in}}%
\pgfpathlineto{\pgfqpoint{3.939297in}{2.315659in}}%
\pgfpathlineto{\pgfqpoint{3.949324in}{2.318098in}}%
\pgfpathlineto{\pgfqpoint{3.954337in}{2.318241in}}%
\pgfpathlineto{\pgfqpoint{3.956844in}{2.321503in}}%
\pgfpathlineto{\pgfqpoint{3.964364in}{2.324973in}}%
\pgfpathlineto{\pgfqpoint{3.979404in}{2.328021in}}%
\pgfpathlineto{\pgfqpoint{3.981910in}{2.332334in}}%
\pgfpathlineto{\pgfqpoint{3.986924in}{2.335737in}}%
\pgfpathlineto{\pgfqpoint{3.989430in}{2.338711in}}%
\pgfpathlineto{\pgfqpoint{3.996950in}{2.340354in}}%
\pgfpathlineto{\pgfqpoint{4.001964in}{2.341411in}}%
\pgfpathlineto{\pgfqpoint{4.004470in}{2.345372in}}%
\pgfpathlineto{\pgfqpoint{4.009484in}{2.348093in}}%
\pgfpathlineto{\pgfqpoint{4.011990in}{2.355209in}}%
\pgfpathlineto{\pgfqpoint{4.017004in}{2.359749in}}%
\pgfpathlineto{\pgfqpoint{4.019510in}{2.360192in}}%
\pgfpathlineto{\pgfqpoint{4.029537in}{2.371671in}}%
\pgfpathlineto{\pgfqpoint{4.034550in}{2.372229in}}%
\pgfpathlineto{\pgfqpoint{4.039564in}{2.379550in}}%
\pgfpathlineto{\pgfqpoint{4.052097in}{2.384639in}}%
\pgfpathlineto{\pgfqpoint{4.054603in}{2.388291in}}%
\pgfpathlineto{\pgfqpoint{4.057110in}{2.389769in}}%
\pgfpathlineto{\pgfqpoint{4.062123in}{2.390181in}}%
\pgfpathlineto{\pgfqpoint{4.064630in}{2.396259in}}%
\pgfpathlineto{\pgfqpoint{4.067137in}{2.398485in}}%
\pgfpathlineto{\pgfqpoint{4.074657in}{2.399255in}}%
\pgfpathlineto{\pgfqpoint{4.079670in}{2.403703in}}%
\pgfpathlineto{\pgfqpoint{4.082177in}{2.405147in}}%
\pgfpathlineto{\pgfqpoint{4.084683in}{2.410736in}}%
\pgfpathlineto{\pgfqpoint{4.092203in}{2.414103in}}%
\pgfpathlineto{\pgfqpoint{4.094710in}{2.414399in}}%
\pgfpathlineto{\pgfqpoint{4.097217in}{2.417777in}}%
\pgfpathlineto{\pgfqpoint{4.099723in}{2.417799in}}%
\pgfpathlineto{\pgfqpoint{4.104737in}{2.424203in}}%
\pgfpathlineto{\pgfqpoint{4.107243in}{2.424312in}}%
\pgfpathlineto{\pgfqpoint{4.109750in}{2.427123in}}%
\pgfpathlineto{\pgfqpoint{4.112257in}{2.427733in}}%
\pgfpathlineto{\pgfqpoint{4.114763in}{2.429578in}}%
\pgfpathlineto{\pgfqpoint{4.117270in}{2.430029in}}%
\pgfpathlineto{\pgfqpoint{4.119776in}{2.436742in}}%
\pgfpathlineto{\pgfqpoint{4.122283in}{2.437800in}}%
\pgfpathlineto{\pgfqpoint{4.124790in}{2.440523in}}%
\pgfpathlineto{\pgfqpoint{4.127296in}{2.440961in}}%
\pgfpathlineto{\pgfqpoint{4.134816in}{2.447141in}}%
\pgfpathlineto{\pgfqpoint{4.144843in}{2.453952in}}%
\pgfpathlineto{\pgfqpoint{4.152363in}{2.455951in}}%
\pgfpathlineto{\pgfqpoint{4.157376in}{2.458632in}}%
\pgfpathlineto{\pgfqpoint{4.159883in}{2.463867in}}%
\pgfpathlineto{\pgfqpoint{4.174923in}{2.467606in}}%
\pgfpathlineto{\pgfqpoint{4.184949in}{2.475305in}}%
\pgfpathlineto{\pgfqpoint{4.187456in}{2.478450in}}%
\pgfpathlineto{\pgfqpoint{4.189963in}{2.479519in}}%
\pgfpathlineto{\pgfqpoint{4.192469in}{2.487999in}}%
\pgfpathlineto{\pgfqpoint{4.194976in}{2.488401in}}%
\pgfpathlineto{\pgfqpoint{4.197483in}{2.495003in}}%
\pgfpathlineto{\pgfqpoint{4.202496in}{2.495656in}}%
\pgfpathlineto{\pgfqpoint{4.205003in}{2.497932in}}%
\pgfpathlineto{\pgfqpoint{4.207509in}{2.506688in}}%
\pgfpathlineto{\pgfqpoint{4.215029in}{2.510791in}}%
\pgfpathlineto{\pgfqpoint{4.220043in}{2.511375in}}%
\pgfpathlineto{\pgfqpoint{4.222549in}{2.512991in}}%
\pgfpathlineto{\pgfqpoint{4.225056in}{2.520640in}}%
\pgfpathlineto{\pgfqpoint{4.232576in}{2.530037in}}%
\pgfpathlineto{\pgfqpoint{4.237589in}{2.530950in}}%
\pgfpathlineto{\pgfqpoint{4.247616in}{2.549327in}}%
\pgfpathlineto{\pgfqpoint{4.252629in}{2.550587in}}%
\pgfpathlineto{\pgfqpoint{4.255136in}{2.551585in}}%
\pgfpathlineto{\pgfqpoint{4.260149in}{2.558434in}}%
\pgfpathlineto{\pgfqpoint{4.262656in}{2.559062in}}%
\pgfpathlineto{\pgfqpoint{4.265162in}{2.561264in}}%
\pgfpathlineto{\pgfqpoint{4.267669in}{2.561478in}}%
\pgfpathlineto{\pgfqpoint{4.270176in}{2.563601in}}%
\pgfpathlineto{\pgfqpoint{4.272682in}{2.571198in}}%
\pgfpathlineto{\pgfqpoint{4.275189in}{2.575008in}}%
\pgfpathlineto{\pgfqpoint{4.290229in}{2.582039in}}%
\pgfpathlineto{\pgfqpoint{4.292736in}{2.582336in}}%
\pgfpathlineto{\pgfqpoint{4.302762in}{2.597871in}}%
\pgfpathlineto{\pgfqpoint{4.305269in}{2.599869in}}%
\pgfpathlineto{\pgfqpoint{4.307776in}{2.604128in}}%
\pgfpathlineto{\pgfqpoint{4.310282in}{2.611903in}}%
\pgfpathlineto{\pgfqpoint{4.312789in}{2.612414in}}%
\pgfpathlineto{\pgfqpoint{4.317802in}{2.616056in}}%
\pgfpathlineto{\pgfqpoint{4.320309in}{2.616137in}}%
\pgfpathlineto{\pgfqpoint{4.322815in}{2.618083in}}%
\pgfpathlineto{\pgfqpoint{4.327829in}{2.619060in}}%
\pgfpathlineto{\pgfqpoint{4.342869in}{2.626079in}}%
\pgfpathlineto{\pgfqpoint{4.345375in}{2.626474in}}%
\pgfpathlineto{\pgfqpoint{4.347882in}{2.628945in}}%
\pgfpathlineto{\pgfqpoint{4.352895in}{2.629868in}}%
\pgfpathlineto{\pgfqpoint{4.355402in}{2.637548in}}%
\pgfpathlineto{\pgfqpoint{4.357909in}{2.638992in}}%
\pgfpathlineto{\pgfqpoint{4.360415in}{2.639013in}}%
\pgfpathlineto{\pgfqpoint{4.362922in}{2.643136in}}%
\pgfpathlineto{\pgfqpoint{4.367935in}{2.645521in}}%
\pgfpathlineto{\pgfqpoint{4.370442in}{2.651122in}}%
\pgfpathlineto{\pgfqpoint{4.377962in}{2.652105in}}%
\pgfpathlineto{\pgfqpoint{4.380469in}{2.654031in}}%
\pgfpathlineto{\pgfqpoint{4.385482in}{2.654689in}}%
\pgfpathlineto{\pgfqpoint{4.390495in}{2.657189in}}%
\pgfpathlineto{\pgfqpoint{4.395508in}{2.661290in}}%
\pgfpathlineto{\pgfqpoint{4.398015in}{2.667700in}}%
\pgfpathlineto{\pgfqpoint{4.400522in}{2.668069in}}%
\pgfpathlineto{\pgfqpoint{4.403028in}{2.671867in}}%
\pgfpathlineto{\pgfqpoint{4.408042in}{2.673050in}}%
\pgfpathlineto{\pgfqpoint{4.410548in}{2.681309in}}%
\pgfpathlineto{\pgfqpoint{4.413055in}{2.682748in}}%
\pgfpathlineto{\pgfqpoint{4.418068in}{2.694173in}}%
\pgfpathlineto{\pgfqpoint{4.423082in}{2.697286in}}%
\pgfpathlineto{\pgfqpoint{4.425588in}{2.701234in}}%
\pgfpathlineto{\pgfqpoint{4.435615in}{2.704475in}}%
\pgfpathlineto{\pgfqpoint{4.438122in}{2.704575in}}%
\pgfpathlineto{\pgfqpoint{4.440628in}{2.709553in}}%
\pgfpathlineto{\pgfqpoint{4.453161in}{2.713623in}}%
\pgfpathlineto{\pgfqpoint{4.458175in}{2.721867in}}%
\pgfpathlineto{\pgfqpoint{4.465695in}{2.726047in}}%
\pgfpathlineto{\pgfqpoint{4.468201in}{2.732562in}}%
\pgfpathlineto{\pgfqpoint{4.473215in}{2.734246in}}%
\pgfpathlineto{\pgfqpoint{4.475721in}{2.737535in}}%
\pgfpathlineto{\pgfqpoint{4.485748in}{2.739269in}}%
\pgfpathlineto{\pgfqpoint{4.488255in}{2.742553in}}%
\pgfpathlineto{\pgfqpoint{4.490761in}{2.742933in}}%
\pgfpathlineto{\pgfqpoint{4.493268in}{2.745145in}}%
\pgfpathlineto{\pgfqpoint{4.495775in}{2.749090in}}%
\pgfpathlineto{\pgfqpoint{4.503295in}{2.751584in}}%
\pgfpathlineto{\pgfqpoint{4.505801in}{2.751673in}}%
\pgfpathlineto{\pgfqpoint{4.508308in}{2.757246in}}%
\pgfpathlineto{\pgfqpoint{4.510815in}{2.757682in}}%
\pgfpathlineto{\pgfqpoint{4.513321in}{2.760409in}}%
\pgfpathlineto{\pgfqpoint{4.515828in}{2.768128in}}%
\pgfpathlineto{\pgfqpoint{4.518335in}{2.768750in}}%
\pgfpathlineto{\pgfqpoint{4.523348in}{2.772696in}}%
\pgfpathlineto{\pgfqpoint{4.530868in}{2.781791in}}%
\pgfpathlineto{\pgfqpoint{4.533374in}{2.790004in}}%
\pgfpathlineto{\pgfqpoint{4.538388in}{2.791892in}}%
\pgfpathlineto{\pgfqpoint{4.543401in}{2.791972in}}%
\pgfpathlineto{\pgfqpoint{4.548414in}{2.794645in}}%
\pgfpathlineto{\pgfqpoint{4.558441in}{2.798479in}}%
\pgfpathlineto{\pgfqpoint{4.563454in}{2.802411in}}%
\pgfpathlineto{\pgfqpoint{4.565961in}{2.803008in}}%
\pgfpathlineto{\pgfqpoint{4.568468in}{2.809782in}}%
\pgfpathlineto{\pgfqpoint{4.573481in}{2.817111in}}%
\pgfpathlineto{\pgfqpoint{4.575988in}{2.817909in}}%
\pgfpathlineto{\pgfqpoint{4.578494in}{2.823067in}}%
\pgfpathlineto{\pgfqpoint{4.581001in}{2.823168in}}%
\pgfpathlineto{\pgfqpoint{4.583508in}{2.829282in}}%
\pgfpathlineto{\pgfqpoint{4.588521in}{2.831614in}}%
\pgfpathlineto{\pgfqpoint{4.591027in}{2.836565in}}%
\pgfpathlineto{\pgfqpoint{4.598547in}{2.837206in}}%
\pgfpathlineto{\pgfqpoint{4.606067in}{2.840722in}}%
\pgfpathlineto{\pgfqpoint{4.611081in}{2.841616in}}%
\pgfpathlineto{\pgfqpoint{4.621107in}{2.853050in}}%
\pgfpathlineto{\pgfqpoint{4.623614in}{2.855359in}}%
\pgfpathlineto{\pgfqpoint{4.626121in}{2.862072in}}%
\pgfpathlineto{\pgfqpoint{4.628627in}{2.862170in}}%
\pgfpathlineto{\pgfqpoint{4.631134in}{2.866418in}}%
\pgfpathlineto{\pgfqpoint{4.633641in}{2.866687in}}%
\pgfpathlineto{\pgfqpoint{4.636147in}{2.875892in}}%
\pgfpathlineto{\pgfqpoint{4.641161in}{2.878417in}}%
\pgfpathlineto{\pgfqpoint{4.648681in}{2.891065in}}%
\pgfpathlineto{\pgfqpoint{4.651187in}{2.893014in}}%
\pgfpathlineto{\pgfqpoint{4.653694in}{2.897256in}}%
\pgfpathlineto{\pgfqpoint{4.656201in}{2.905275in}}%
\pgfpathlineto{\pgfqpoint{4.656201in}{2.905275in}}%
\pgfusepath{stroke}%
\end{pgfscope}%
\begin{pgfscope}%
\pgfpathrectangle{\pgfqpoint{0.708220in}{0.535823in}}{\pgfqpoint{5.013309in}{2.369453in}}%
\pgfusepath{clip}%
\pgfsetbuttcap%
\pgfsetroundjoin%
\pgfsetlinewidth{1.003750pt}%
\definecolor{currentstroke}{rgb}{0.000000,0.501961,0.000000}%
\pgfsetstrokecolor{currentstroke}%
\pgfsetdash{{3.700000pt}{1.600000pt}}{0.000000pt}%
\pgfpathmoveto{\pgfqpoint{0.708220in}{1.000271in}}%
\pgfpathlineto{\pgfqpoint{0.723260in}{1.000271in}}%
\pgfpathlineto{\pgfqpoint{0.725766in}{1.011340in}}%
\pgfpathlineto{\pgfqpoint{0.730780in}{1.011340in}}%
\pgfpathlineto{\pgfqpoint{0.733286in}{1.021738in}}%
\pgfpathlineto{\pgfqpoint{0.740806in}{1.021738in}}%
\pgfpathlineto{\pgfqpoint{0.743313in}{1.031541in}}%
\pgfpathlineto{\pgfqpoint{0.745820in}{1.031541in}}%
\pgfpathlineto{\pgfqpoint{0.748326in}{1.040814in}}%
\pgfpathlineto{\pgfqpoint{0.765873in}{1.040814in}}%
\pgfpathlineto{\pgfqpoint{0.768380in}{1.049611in}}%
\pgfpathlineto{\pgfqpoint{0.778406in}{1.049611in}}%
\pgfpathlineto{\pgfqpoint{0.780913in}{1.057979in}}%
\pgfpathlineto{\pgfqpoint{0.788433in}{1.057979in}}%
\pgfpathlineto{\pgfqpoint{0.790939in}{1.065957in}}%
\pgfpathlineto{\pgfqpoint{0.810993in}{1.065957in}}%
\pgfpathlineto{\pgfqpoint{0.813499in}{1.073581in}}%
\pgfpathlineto{\pgfqpoint{0.833553in}{1.073581in}}%
\pgfpathlineto{\pgfqpoint{0.836059in}{1.080880in}}%
\pgfpathlineto{\pgfqpoint{0.853606in}{1.080880in}}%
\pgfpathlineto{\pgfqpoint{0.856112in}{1.087881in}}%
\pgfpathlineto{\pgfqpoint{0.873659in}{1.087881in}}%
\pgfpathlineto{\pgfqpoint{0.876166in}{1.094608in}}%
\pgfpathlineto{\pgfqpoint{0.896219in}{1.094608in}}%
\pgfpathlineto{\pgfqpoint{0.898726in}{1.101081in}}%
\pgfpathlineto{\pgfqpoint{0.923792in}{1.101081in}}%
\pgfpathlineto{\pgfqpoint{0.926299in}{1.107318in}}%
\pgfpathlineto{\pgfqpoint{0.933819in}{1.107318in}}%
\pgfpathlineto{\pgfqpoint{0.936325in}{1.113336in}}%
\pgfpathlineto{\pgfqpoint{0.946352in}{1.113336in}}%
\pgfpathlineto{\pgfqpoint{0.948859in}{1.119151in}}%
\pgfpathlineto{\pgfqpoint{0.958885in}{1.119151in}}%
\pgfpathlineto{\pgfqpoint{0.961392in}{1.124774in}}%
\pgfpathlineto{\pgfqpoint{0.983952in}{1.124774in}}%
\pgfpathlineto{\pgfqpoint{0.986458in}{1.130220in}}%
\pgfpathlineto{\pgfqpoint{1.014032in}{1.130220in}}%
\pgfpathlineto{\pgfqpoint{1.016538in}{1.135497in}}%
\pgfpathlineto{\pgfqpoint{1.039098in}{1.135497in}}%
\pgfpathlineto{\pgfqpoint{1.041605in}{1.140617in}}%
\pgfpathlineto{\pgfqpoint{1.059151in}{1.140617in}}%
\pgfpathlineto{\pgfqpoint{1.061658in}{1.145589in}}%
\pgfpathlineto{\pgfqpoint{1.079205in}{1.145589in}}%
\pgfpathlineto{\pgfqpoint{1.081711in}{1.150420in}}%
\pgfpathlineto{\pgfqpoint{1.114298in}{1.150420in}}%
\pgfpathlineto{\pgfqpoint{1.116805in}{1.155119in}}%
\pgfpathlineto{\pgfqpoint{1.139364in}{1.155119in}}%
\pgfpathlineto{\pgfqpoint{1.141871in}{1.159693in}}%
\pgfpathlineto{\pgfqpoint{1.166938in}{1.159693in}}%
\pgfpathlineto{\pgfqpoint{1.169444in}{1.164148in}}%
\pgfpathlineto{\pgfqpoint{1.197017in}{1.164148in}}%
\pgfpathlineto{\pgfqpoint{1.199524in}{1.168490in}}%
\pgfpathlineto{\pgfqpoint{1.214564in}{1.168490in}}%
\pgfpathlineto{\pgfqpoint{1.217071in}{1.172725in}}%
\pgfpathlineto{\pgfqpoint{1.234617in}{1.172725in}}%
\pgfpathlineto{\pgfqpoint{1.237124in}{1.176858in}}%
\pgfpathlineto{\pgfqpoint{1.252164in}{1.176858in}}%
\pgfpathlineto{\pgfqpoint{1.254671in}{1.180894in}}%
\pgfpathlineto{\pgfqpoint{1.279737in}{1.180894in}}%
\pgfpathlineto{\pgfqpoint{1.282244in}{1.184837in}}%
\pgfpathlineto{\pgfqpoint{1.289764in}{1.184837in}}%
\pgfpathlineto{\pgfqpoint{1.292270in}{1.188691in}}%
\pgfpathlineto{\pgfqpoint{1.299790in}{1.188691in}}%
\pgfpathlineto{\pgfqpoint{1.302297in}{1.192460in}}%
\pgfpathlineto{\pgfqpoint{1.327363in}{1.192460in}}%
\pgfpathlineto{\pgfqpoint{1.329870in}{1.196149in}}%
\pgfpathlineto{\pgfqpoint{1.337390in}{1.196149in}}%
\pgfpathlineto{\pgfqpoint{1.339897in}{1.199760in}}%
\pgfpathlineto{\pgfqpoint{1.349923in}{1.199760in}}%
\pgfpathlineto{\pgfqpoint{1.352430in}{1.203296in}}%
\pgfpathlineto{\pgfqpoint{1.382510in}{1.203296in}}%
\pgfpathlineto{\pgfqpoint{1.385017in}{1.206761in}}%
\pgfpathlineto{\pgfqpoint{1.407576in}{1.206761in}}%
\pgfpathlineto{\pgfqpoint{1.410083in}{1.210157in}}%
\pgfpathlineto{\pgfqpoint{1.422616in}{1.210157in}}%
\pgfpathlineto{\pgfqpoint{1.425123in}{1.213487in}}%
\pgfpathlineto{\pgfqpoint{1.447683in}{1.213487in}}%
\pgfpathlineto{\pgfqpoint{1.450190in}{1.216754in}}%
\pgfpathlineto{\pgfqpoint{1.460216in}{1.216754in}}%
\pgfpathlineto{\pgfqpoint{1.462723in}{1.219960in}}%
\pgfpathlineto{\pgfqpoint{1.495309in}{1.219960in}}%
\pgfpathlineto{\pgfqpoint{1.497816in}{1.223107in}}%
\pgfpathlineto{\pgfqpoint{1.527896in}{1.223107in}}%
\pgfpathlineto{\pgfqpoint{1.530402in}{1.226197in}}%
\pgfpathlineto{\pgfqpoint{1.545442in}{1.226197in}}%
\pgfpathlineto{\pgfqpoint{1.547949in}{1.229233in}}%
\pgfpathlineto{\pgfqpoint{1.570509in}{1.229233in}}%
\pgfpathlineto{\pgfqpoint{1.573016in}{1.232216in}}%
\pgfpathlineto{\pgfqpoint{1.600589in}{1.232216in}}%
\pgfpathlineto{\pgfqpoint{1.603095in}{1.235148in}}%
\pgfpathlineto{\pgfqpoint{1.618135in}{1.235148in}}%
\pgfpathlineto{\pgfqpoint{1.620642in}{1.238030in}}%
\pgfpathlineto{\pgfqpoint{1.638189in}{1.238030in}}%
\pgfpathlineto{\pgfqpoint{1.640695in}{1.240865in}}%
\pgfpathlineto{\pgfqpoint{1.663255in}{1.240865in}}%
\pgfpathlineto{\pgfqpoint{1.665762in}{1.243654in}}%
\pgfpathlineto{\pgfqpoint{1.700855in}{1.243654in}}%
\pgfpathlineto{\pgfqpoint{1.703362in}{1.246398in}}%
\pgfpathlineto{\pgfqpoint{1.718402in}{1.246398in}}%
\pgfpathlineto{\pgfqpoint{1.720908in}{1.249099in}}%
\pgfpathlineto{\pgfqpoint{1.740961in}{1.249099in}}%
\pgfpathlineto{\pgfqpoint{1.743468in}{1.251758in}}%
\pgfpathlineto{\pgfqpoint{1.771041in}{1.251758in}}%
\pgfpathlineto{\pgfqpoint{1.773548in}{1.254376in}}%
\pgfpathlineto{\pgfqpoint{1.783575in}{1.254376in}}%
\pgfpathlineto{\pgfqpoint{1.786081in}{1.256956in}}%
\pgfpathlineto{\pgfqpoint{1.803628in}{1.256956in}}%
\pgfpathlineto{\pgfqpoint{1.806134in}{1.259496in}}%
\pgfpathlineto{\pgfqpoint{1.813654in}{1.259496in}}%
\pgfpathlineto{\pgfqpoint{1.816161in}{1.262000in}}%
\pgfpathlineto{\pgfqpoint{1.836214in}{1.262000in}}%
\pgfpathlineto{\pgfqpoint{1.838721in}{1.264468in}}%
\pgfpathlineto{\pgfqpoint{1.843734in}{1.264468in}}%
\pgfpathlineto{\pgfqpoint{1.846241in}{1.266901in}}%
\pgfpathlineto{\pgfqpoint{1.858774in}{1.266901in}}%
\pgfpathlineto{\pgfqpoint{1.861281in}{1.269300in}}%
\pgfpathlineto{\pgfqpoint{1.871307in}{1.269300in}}%
\pgfpathlineto{\pgfqpoint{1.873814in}{1.271665in}}%
\pgfpathlineto{\pgfqpoint{1.886347in}{1.271665in}}%
\pgfpathlineto{\pgfqpoint{1.888854in}{1.273999in}}%
\pgfpathlineto{\pgfqpoint{1.901387in}{1.273999in}}%
\pgfpathlineto{\pgfqpoint{1.903894in}{1.276301in}}%
\pgfpathlineto{\pgfqpoint{1.911414in}{1.276301in}}%
\pgfpathlineto{\pgfqpoint{1.913921in}{1.278572in}}%
\pgfpathlineto{\pgfqpoint{1.923947in}{1.278572in}}%
\pgfpathlineto{\pgfqpoint{1.926454in}{1.280814in}}%
\pgfpathlineto{\pgfqpoint{1.933974in}{1.280814in}}%
\pgfpathlineto{\pgfqpoint{1.936481in}{1.283027in}}%
\pgfpathlineto{\pgfqpoint{1.941494in}{1.283027in}}%
\pgfpathlineto{\pgfqpoint{1.944000in}{1.285212in}}%
\pgfpathlineto{\pgfqpoint{1.949014in}{1.285212in}}%
\pgfpathlineto{\pgfqpoint{1.951520in}{1.287370in}}%
\pgfpathlineto{\pgfqpoint{1.954027in}{1.287370in}}%
\pgfpathlineto{\pgfqpoint{1.956534in}{1.289500in}}%
\pgfpathlineto{\pgfqpoint{1.959040in}{1.289500in}}%
\pgfpathlineto{\pgfqpoint{1.961547in}{1.291604in}}%
\pgfpathlineto{\pgfqpoint{1.971574in}{1.291604in}}%
\pgfpathlineto{\pgfqpoint{1.974080in}{1.293683in}}%
\pgfpathlineto{\pgfqpoint{1.979094in}{1.293683in}}%
\pgfpathlineto{\pgfqpoint{1.981600in}{1.295737in}}%
\pgfpathlineto{\pgfqpoint{1.984107in}{1.295737in}}%
\pgfpathlineto{\pgfqpoint{1.986614in}{1.297767in}}%
\pgfpathlineto{\pgfqpoint{1.996640in}{1.297767in}}%
\pgfpathlineto{\pgfqpoint{1.999147in}{1.299773in}}%
\pgfpathlineto{\pgfqpoint{2.006667in}{1.299773in}}%
\pgfpathlineto{\pgfqpoint{2.009173in}{1.301756in}}%
\pgfpathlineto{\pgfqpoint{2.019200in}{1.301756in}}%
\pgfpathlineto{\pgfqpoint{2.024213in}{1.305654in}}%
\pgfpathlineto{\pgfqpoint{2.029227in}{1.305654in}}%
\pgfpathlineto{\pgfqpoint{2.031733in}{1.307570in}}%
\pgfpathlineto{\pgfqpoint{2.039253in}{1.307570in}}%
\pgfpathlineto{\pgfqpoint{2.044267in}{1.311340in}}%
\pgfpathlineto{\pgfqpoint{2.049280in}{1.311340in}}%
\pgfpathlineto{\pgfqpoint{2.051787in}{1.313194in}}%
\pgfpathlineto{\pgfqpoint{2.056800in}{1.313194in}}%
\pgfpathlineto{\pgfqpoint{2.059307in}{1.315028in}}%
\pgfpathlineto{\pgfqpoint{2.061813in}{1.315028in}}%
\pgfpathlineto{\pgfqpoint{2.064320in}{1.316843in}}%
\pgfpathlineto{\pgfqpoint{2.066827in}{1.316843in}}%
\pgfpathlineto{\pgfqpoint{2.069333in}{1.318639in}}%
\pgfpathlineto{\pgfqpoint{2.074346in}{1.318639in}}%
\pgfpathlineto{\pgfqpoint{2.076853in}{1.320416in}}%
\pgfpathlineto{\pgfqpoint{2.079360in}{1.320416in}}%
\pgfpathlineto{\pgfqpoint{2.081866in}{1.322175in}}%
\pgfpathlineto{\pgfqpoint{2.089386in}{1.322175in}}%
\pgfpathlineto{\pgfqpoint{2.091893in}{1.323916in}}%
\pgfpathlineto{\pgfqpoint{2.094400in}{1.323916in}}%
\pgfpathlineto{\pgfqpoint{2.096906in}{1.325640in}}%
\pgfpathlineto{\pgfqpoint{2.099413in}{1.325640in}}%
\pgfpathlineto{\pgfqpoint{2.101920in}{1.327347in}}%
\pgfpathlineto{\pgfqpoint{2.104426in}{1.327347in}}%
\pgfpathlineto{\pgfqpoint{2.106933in}{1.329036in}}%
\pgfpathlineto{\pgfqpoint{2.111946in}{1.329036in}}%
\pgfpathlineto{\pgfqpoint{2.114453in}{1.330710in}}%
\pgfpathlineto{\pgfqpoint{2.119466in}{1.330710in}}%
\pgfpathlineto{\pgfqpoint{2.121973in}{1.332367in}}%
\pgfpathlineto{\pgfqpoint{2.124480in}{1.335634in}}%
\pgfpathlineto{\pgfqpoint{2.132000in}{1.335634in}}%
\pgfpathlineto{\pgfqpoint{2.134506in}{1.337244in}}%
\pgfpathlineto{\pgfqpoint{2.139520in}{1.337244in}}%
\pgfpathlineto{\pgfqpoint{2.142026in}{1.338839in}}%
\pgfpathlineto{\pgfqpoint{2.144533in}{1.338839in}}%
\pgfpathlineto{\pgfqpoint{2.147039in}{1.340420in}}%
\pgfpathlineto{\pgfqpoint{2.149546in}{1.340420in}}%
\pgfpathlineto{\pgfqpoint{2.152053in}{1.341986in}}%
\pgfpathlineto{\pgfqpoint{2.157066in}{1.341986in}}%
\pgfpathlineto{\pgfqpoint{2.159573in}{1.343539in}}%
\pgfpathlineto{\pgfqpoint{2.164586in}{1.343539in}}%
\pgfpathlineto{\pgfqpoint{2.167093in}{1.345077in}}%
\pgfpathlineto{\pgfqpoint{2.174613in}{1.345077in}}%
\pgfpathlineto{\pgfqpoint{2.179626in}{1.348112in}}%
\pgfpathlineto{\pgfqpoint{2.182133in}{1.348112in}}%
\pgfpathlineto{\pgfqpoint{2.184639in}{1.349610in}}%
\pgfpathlineto{\pgfqpoint{2.187146in}{1.349610in}}%
\pgfpathlineto{\pgfqpoint{2.189653in}{1.351095in}}%
\pgfpathlineto{\pgfqpoint{2.192159in}{1.351095in}}%
\pgfpathlineto{\pgfqpoint{2.197173in}{1.355474in}}%
\pgfpathlineto{\pgfqpoint{2.202186in}{1.355474in}}%
\pgfpathlineto{\pgfqpoint{2.204693in}{1.356910in}}%
\pgfpathlineto{\pgfqpoint{2.209706in}{1.356910in}}%
\pgfpathlineto{\pgfqpoint{2.214719in}{1.361144in}}%
\pgfpathlineto{\pgfqpoint{2.224746in}{1.366633in}}%
\pgfpathlineto{\pgfqpoint{2.227252in}{1.371952in}}%
\pgfpathlineto{\pgfqpoint{2.229759in}{1.373256in}}%
\pgfpathlineto{\pgfqpoint{2.232266in}{1.373256in}}%
\pgfpathlineto{\pgfqpoint{2.234772in}{1.377110in}}%
\pgfpathlineto{\pgfqpoint{2.237279in}{1.378376in}}%
\pgfpathlineto{\pgfqpoint{2.244799in}{1.378376in}}%
\pgfpathlineto{\pgfqpoint{2.247306in}{1.379632in}}%
\pgfpathlineto{\pgfqpoint{2.249812in}{1.379632in}}%
\pgfpathlineto{\pgfqpoint{2.252319in}{1.380880in}}%
\pgfpathlineto{\pgfqpoint{2.257332in}{1.380880in}}%
\pgfpathlineto{\pgfqpoint{2.262346in}{1.383347in}}%
\pgfpathlineto{\pgfqpoint{2.264852in}{1.383347in}}%
\pgfpathlineto{\pgfqpoint{2.267359in}{1.386984in}}%
\pgfpathlineto{\pgfqpoint{2.272372in}{1.386984in}}%
\pgfpathlineto{\pgfqpoint{2.277385in}{1.390545in}}%
\pgfpathlineto{\pgfqpoint{2.282399in}{1.398576in}}%
\pgfpathlineto{\pgfqpoint{2.287412in}{1.399694in}}%
\pgfpathlineto{\pgfqpoint{2.292425in}{1.400804in}}%
\pgfpathlineto{\pgfqpoint{2.294932in}{1.403003in}}%
\pgfpathlineto{\pgfqpoint{2.297439in}{1.403003in}}%
\pgfpathlineto{\pgfqpoint{2.302452in}{1.406249in}}%
\pgfpathlineto{\pgfqpoint{2.314985in}{1.412563in}}%
\pgfpathlineto{\pgfqpoint{2.322505in}{1.420635in}}%
\pgfpathlineto{\pgfqpoint{2.330025in}{1.425494in}}%
\pgfpathlineto{\pgfqpoint{2.350078in}{1.433907in}}%
\pgfpathlineto{\pgfqpoint{2.352585in}{1.436623in}}%
\pgfpathlineto{\pgfqpoint{2.367625in}{1.441055in}}%
\pgfpathlineto{\pgfqpoint{2.370132in}{1.441055in}}%
\pgfpathlineto{\pgfqpoint{2.385172in}{1.450420in}}%
\pgfpathlineto{\pgfqpoint{2.390185in}{1.451246in}}%
\pgfpathlineto{\pgfqpoint{2.392692in}{1.455320in}}%
\pgfpathlineto{\pgfqpoint{2.395198in}{1.455320in}}%
\pgfpathlineto{\pgfqpoint{2.397705in}{1.457719in}}%
\pgfpathlineto{\pgfqpoint{2.400212in}{1.457719in}}%
\pgfpathlineto{\pgfqpoint{2.402718in}{1.460085in}}%
\pgfpathlineto{\pgfqpoint{2.415251in}{1.463189in}}%
\pgfpathlineto{\pgfqpoint{2.417758in}{1.466238in}}%
\pgfpathlineto{\pgfqpoint{2.422771in}{1.468490in}}%
\pgfpathlineto{\pgfqpoint{2.425278in}{1.469975in}}%
\pgfpathlineto{\pgfqpoint{2.430291in}{1.470712in}}%
\pgfpathlineto{\pgfqpoint{2.437811in}{1.472906in}}%
\pgfpathlineto{\pgfqpoint{2.442825in}{1.476502in}}%
\pgfpathlineto{\pgfqpoint{2.445331in}{1.477212in}}%
\pgfpathlineto{\pgfqpoint{2.447838in}{1.482103in}}%
\pgfpathlineto{\pgfqpoint{2.455358in}{1.482790in}}%
\pgfpathlineto{\pgfqpoint{2.460371in}{1.483475in}}%
\pgfpathlineto{\pgfqpoint{2.462878in}{1.488192in}}%
\pgfpathlineto{\pgfqpoint{2.467891in}{1.488856in}}%
\pgfpathlineto{\pgfqpoint{2.487944in}{1.494714in}}%
\pgfpathlineto{\pgfqpoint{2.495464in}{1.505863in}}%
\pgfpathlineto{\pgfqpoint{2.497971in}{1.505863in}}%
\pgfpathlineto{\pgfqpoint{2.500478in}{1.509424in}}%
\pgfpathlineto{\pgfqpoint{2.507998in}{1.510010in}}%
\pgfpathlineto{\pgfqpoint{2.510504in}{1.510595in}}%
\pgfpathlineto{\pgfqpoint{2.513011in}{1.512912in}}%
\pgfpathlineto{\pgfqpoint{2.518024in}{1.514630in}}%
\pgfpathlineto{\pgfqpoint{2.528051in}{1.522427in}}%
\pgfpathlineto{\pgfqpoint{2.540584in}{1.529885in}}%
\pgfpathlineto{\pgfqpoint{2.553117in}{1.533496in}}%
\pgfpathlineto{\pgfqpoint{2.563144in}{1.536532in}}%
\pgfpathlineto{\pgfqpoint{2.565651in}{1.540007in}}%
\pgfpathlineto{\pgfqpoint{2.575677in}{1.544373in}}%
\pgfpathlineto{\pgfqpoint{2.588211in}{1.553242in}}%
\pgfpathlineto{\pgfqpoint{2.590717in}{1.559057in}}%
\pgfpathlineto{\pgfqpoint{2.598237in}{1.560807in}}%
\pgfpathlineto{\pgfqpoint{2.603251in}{1.564680in}}%
\pgfpathlineto{\pgfqpoint{2.605757in}{1.569713in}}%
\pgfpathlineto{\pgfqpoint{2.618290in}{1.575003in}}%
\pgfpathlineto{\pgfqpoint{2.623304in}{1.575802in}}%
\pgfpathlineto{\pgfqpoint{2.628317in}{1.585495in}}%
\pgfpathlineto{\pgfqpoint{2.630824in}{1.588113in}}%
\pgfpathlineto{\pgfqpoint{2.638344in}{1.590326in}}%
\pgfpathlineto{\pgfqpoint{2.645864in}{1.595025in}}%
\pgfpathlineto{\pgfqpoint{2.655890in}{1.599251in}}%
\pgfpathlineto{\pgfqpoint{2.663410in}{1.609055in}}%
\pgfpathlineto{\pgfqpoint{2.665917in}{1.609710in}}%
\pgfpathlineto{\pgfqpoint{2.668424in}{1.611663in}}%
\pgfpathlineto{\pgfqpoint{2.670930in}{1.611986in}}%
\pgfpathlineto{\pgfqpoint{2.678450in}{1.619259in}}%
\pgfpathlineto{\pgfqpoint{2.683464in}{1.619877in}}%
\pgfpathlineto{\pgfqpoint{2.688477in}{1.626235in}}%
\pgfpathlineto{\pgfqpoint{2.690983in}{1.628890in}}%
\pgfpathlineto{\pgfqpoint{2.701010in}{1.629474in}}%
\pgfpathlineto{\pgfqpoint{2.703517in}{1.632366in}}%
\pgfpathlineto{\pgfqpoint{2.721063in}{1.636615in}}%
\pgfpathlineto{\pgfqpoint{2.723570in}{1.641850in}}%
\pgfpathlineto{\pgfqpoint{2.726077in}{1.643202in}}%
\pgfpathlineto{\pgfqpoint{2.733597in}{1.644008in}}%
\pgfpathlineto{\pgfqpoint{2.736103in}{1.644810in}}%
\pgfpathlineto{\pgfqpoint{2.741117in}{1.650580in}}%
\pgfpathlineto{\pgfqpoint{2.746130in}{1.651095in}}%
\pgfpathlineto{\pgfqpoint{2.751143in}{1.654657in}}%
\pgfpathlineto{\pgfqpoint{2.756156in}{1.661083in}}%
\pgfpathlineto{\pgfqpoint{2.758663in}{1.661810in}}%
\pgfpathlineto{\pgfqpoint{2.761170in}{1.664684in}}%
\pgfpathlineto{\pgfqpoint{2.766183in}{1.665868in}}%
\pgfpathlineto{\pgfqpoint{2.768690in}{1.666574in}}%
\pgfpathlineto{\pgfqpoint{2.773703in}{1.670062in}}%
\pgfpathlineto{\pgfqpoint{2.786236in}{1.674381in}}%
\pgfpathlineto{\pgfqpoint{2.791250in}{1.678813in}}%
\pgfpathlineto{\pgfqpoint{2.793756in}{1.684409in}}%
\pgfpathlineto{\pgfqpoint{2.798770in}{1.687971in}}%
\pgfpathlineto{\pgfqpoint{2.803783in}{1.689417in}}%
\pgfpathlineto{\pgfqpoint{2.808796in}{1.692474in}}%
\pgfpathlineto{\pgfqpoint{2.813810in}{1.695279in}}%
\pgfpathlineto{\pgfqpoint{2.816316in}{1.699596in}}%
\pgfpathlineto{\pgfqpoint{2.826343in}{1.703429in}}%
\pgfpathlineto{\pgfqpoint{2.833863in}{1.707178in}}%
\pgfpathlineto{\pgfqpoint{2.836369in}{1.711028in}}%
\pgfpathlineto{\pgfqpoint{2.838876in}{1.711752in}}%
\pgfpathlineto{\pgfqpoint{2.841383in}{1.714083in}}%
\pgfpathlineto{\pgfqpoint{2.843889in}{1.718478in}}%
\pgfpathlineto{\pgfqpoint{2.846396in}{1.719862in}}%
\pgfpathlineto{\pgfqpoint{2.851409in}{1.724449in}}%
\pgfpathlineto{\pgfqpoint{2.856423in}{1.728917in}}%
\pgfpathlineto{\pgfqpoint{2.858929in}{1.728917in}}%
\pgfpathlineto{\pgfqpoint{2.861436in}{1.730704in}}%
\pgfpathlineto{\pgfqpoint{2.871463in}{1.731510in}}%
\pgfpathlineto{\pgfqpoint{2.878983in}{1.732952in}}%
\pgfpathlineto{\pgfqpoint{2.883996in}{1.733589in}}%
\pgfpathlineto{\pgfqpoint{2.889009in}{1.736583in}}%
\pgfpathlineto{\pgfqpoint{2.891516in}{1.736895in}}%
\pgfpathlineto{\pgfqpoint{2.894022in}{1.740292in}}%
\pgfpathlineto{\pgfqpoint{2.899036in}{1.741510in}}%
\pgfpathlineto{\pgfqpoint{2.904049in}{1.746889in}}%
\pgfpathlineto{\pgfqpoint{2.909062in}{1.748207in}}%
\pgfpathlineto{\pgfqpoint{2.914076in}{1.751675in}}%
\pgfpathlineto{\pgfqpoint{2.919089in}{1.752674in}}%
\pgfpathlineto{\pgfqpoint{2.921596in}{1.754653in}}%
\pgfpathlineto{\pgfqpoint{2.924102in}{1.755215in}}%
\pgfpathlineto{\pgfqpoint{2.926609in}{1.759777in}}%
\pgfpathlineto{\pgfqpoint{2.931622in}{1.761407in}}%
\pgfpathlineto{\pgfqpoint{2.934129in}{1.761542in}}%
\pgfpathlineto{\pgfqpoint{2.936636in}{1.765282in}}%
\pgfpathlineto{\pgfqpoint{2.941649in}{1.765941in}}%
\pgfpathlineto{\pgfqpoint{2.946662in}{1.768295in}}%
\pgfpathlineto{\pgfqpoint{2.951676in}{1.769459in}}%
\pgfpathlineto{\pgfqpoint{2.961702in}{1.783445in}}%
\pgfpathlineto{\pgfqpoint{2.964209in}{1.784038in}}%
\pgfpathlineto{\pgfqpoint{2.969222in}{1.790887in}}%
\pgfpathlineto{\pgfqpoint{2.971729in}{1.791115in}}%
\pgfpathlineto{\pgfqpoint{2.976742in}{1.793933in}}%
\pgfpathlineto{\pgfqpoint{2.979249in}{1.794045in}}%
\pgfpathlineto{\pgfqpoint{2.999302in}{1.804659in}}%
\pgfpathlineto{\pgfqpoint{3.001809in}{1.808090in}}%
\pgfpathlineto{\pgfqpoint{3.004315in}{1.809117in}}%
\pgfpathlineto{\pgfqpoint{3.011835in}{1.818282in}}%
\pgfpathlineto{\pgfqpoint{3.019355in}{1.820019in}}%
\pgfpathlineto{\pgfqpoint{3.021862in}{1.821072in}}%
\pgfpathlineto{\pgfqpoint{3.024369in}{1.825872in}}%
\pgfpathlineto{\pgfqpoint{3.026875in}{1.827534in}}%
\pgfpathlineto{\pgfqpoint{3.029382in}{1.831262in}}%
\pgfpathlineto{\pgfqpoint{3.034395in}{1.831980in}}%
\pgfpathlineto{\pgfqpoint{3.039408in}{1.839685in}}%
\pgfpathlineto{\pgfqpoint{3.041915in}{1.839771in}}%
\pgfpathlineto{\pgfqpoint{3.046928in}{1.844081in}}%
\pgfpathlineto{\pgfqpoint{3.051942in}{1.844664in}}%
\pgfpathlineto{\pgfqpoint{3.054448in}{1.848285in}}%
\pgfpathlineto{\pgfqpoint{3.056955in}{1.848692in}}%
\pgfpathlineto{\pgfqpoint{3.059462in}{1.853657in}}%
\pgfpathlineto{\pgfqpoint{3.061968in}{1.853657in}}%
\pgfpathlineto{\pgfqpoint{3.064475in}{1.856086in}}%
\pgfpathlineto{\pgfqpoint{3.071995in}{1.858712in}}%
\pgfpathlineto{\pgfqpoint{3.077008in}{1.865768in}}%
\pgfpathlineto{\pgfqpoint{3.082022in}{1.867306in}}%
\pgfpathlineto{\pgfqpoint{3.092048in}{1.874234in}}%
\pgfpathlineto{\pgfqpoint{3.094555in}{1.874304in}}%
\pgfpathlineto{\pgfqpoint{3.102075in}{1.879473in}}%
\pgfpathlineto{\pgfqpoint{3.107088in}{1.884161in}}%
\pgfpathlineto{\pgfqpoint{3.112101in}{1.884623in}}%
\pgfpathlineto{\pgfqpoint{3.114608in}{1.885871in}}%
\pgfpathlineto{\pgfqpoint{3.117115in}{1.889559in}}%
\pgfpathlineto{\pgfqpoint{3.119621in}{1.889559in}}%
\pgfpathlineto{\pgfqpoint{3.122128in}{1.894916in}}%
\pgfpathlineto{\pgfqpoint{3.127141in}{1.895721in}}%
\pgfpathlineto{\pgfqpoint{3.132155in}{1.896645in}}%
\pgfpathlineto{\pgfqpoint{3.134661in}{1.904919in}}%
\pgfpathlineto{\pgfqpoint{3.139675in}{1.906376in}}%
\pgfpathlineto{\pgfqpoint{3.142181in}{1.906434in}}%
\pgfpathlineto{\pgfqpoint{3.144688in}{1.916244in}}%
\pgfpathlineto{\pgfqpoint{3.147195in}{1.919285in}}%
\pgfpathlineto{\pgfqpoint{3.149701in}{1.924585in}}%
\pgfpathlineto{\pgfqpoint{3.154715in}{1.926970in}}%
\pgfpathlineto{\pgfqpoint{3.162234in}{1.928201in}}%
\pgfpathlineto{\pgfqpoint{3.172261in}{1.936188in}}%
\pgfpathlineto{\pgfqpoint{3.179781in}{1.950186in}}%
\pgfpathlineto{\pgfqpoint{3.182288in}{1.950231in}}%
\pgfpathlineto{\pgfqpoint{3.184794in}{1.954229in}}%
\pgfpathlineto{\pgfqpoint{3.192314in}{1.958265in}}%
\pgfpathlineto{\pgfqpoint{3.194821in}{1.966226in}}%
\pgfpathlineto{\pgfqpoint{3.197328in}{1.966839in}}%
\pgfpathlineto{\pgfqpoint{3.199834in}{1.971189in}}%
\pgfpathlineto{\pgfqpoint{3.202341in}{1.971308in}}%
\pgfpathlineto{\pgfqpoint{3.207354in}{1.973873in}}%
\pgfpathlineto{\pgfqpoint{3.209861in}{1.980742in}}%
\pgfpathlineto{\pgfqpoint{3.212368in}{1.980818in}}%
\pgfpathlineto{\pgfqpoint{3.214874in}{1.984905in}}%
\pgfpathlineto{\pgfqpoint{3.222394in}{1.988070in}}%
\pgfpathlineto{\pgfqpoint{3.227408in}{1.996613in}}%
\pgfpathlineto{\pgfqpoint{3.234927in}{1.998895in}}%
\pgfpathlineto{\pgfqpoint{3.247461in}{2.005044in}}%
\pgfpathlineto{\pgfqpoint{3.249967in}{2.007572in}}%
\pgfpathlineto{\pgfqpoint{3.252474in}{2.007636in}}%
\pgfpathlineto{\pgfqpoint{3.262501in}{2.017509in}}%
\pgfpathlineto{\pgfqpoint{3.265007in}{2.017691in}}%
\pgfpathlineto{\pgfqpoint{3.267514in}{2.020339in}}%
\pgfpathlineto{\pgfqpoint{3.272527in}{2.021025in}}%
\pgfpathlineto{\pgfqpoint{3.277541in}{2.025575in}}%
\pgfpathlineto{\pgfqpoint{3.280047in}{2.027103in}}%
\pgfpathlineto{\pgfqpoint{3.282554in}{2.031020in}}%
\pgfpathlineto{\pgfqpoint{3.287567in}{2.032528in}}%
\pgfpathlineto{\pgfqpoint{3.292581in}{2.038353in}}%
\pgfpathlineto{\pgfqpoint{3.295087in}{2.038782in}}%
\pgfpathlineto{\pgfqpoint{3.297594in}{2.043726in}}%
\pgfpathlineto{\pgfqpoint{3.300100in}{2.043830in}}%
\pgfpathlineto{\pgfqpoint{3.302607in}{2.045049in}}%
\pgfpathlineto{\pgfqpoint{3.305114in}{2.051718in}}%
\pgfpathlineto{\pgfqpoint{3.310127in}{2.052364in}}%
\pgfpathlineto{\pgfqpoint{3.312634in}{2.053031in}}%
\pgfpathlineto{\pgfqpoint{3.315140in}{2.054995in}}%
\pgfpathlineto{\pgfqpoint{3.320154in}{2.056235in}}%
\pgfpathlineto{\pgfqpoint{3.327674in}{2.060328in}}%
\pgfpathlineto{\pgfqpoint{3.330180in}{2.060611in}}%
\pgfpathlineto{\pgfqpoint{3.332687in}{2.063236in}}%
\pgfpathlineto{\pgfqpoint{3.347727in}{2.068548in}}%
\pgfpathlineto{\pgfqpoint{3.352740in}{2.072603in}}%
\pgfpathlineto{\pgfqpoint{3.357754in}{2.073853in}}%
\pgfpathlineto{\pgfqpoint{3.362767in}{2.088189in}}%
\pgfpathlineto{\pgfqpoint{3.377807in}{2.101194in}}%
\pgfpathlineto{\pgfqpoint{3.382820in}{2.101827in}}%
\pgfpathlineto{\pgfqpoint{3.392847in}{2.107399in}}%
\pgfpathlineto{\pgfqpoint{3.397860in}{2.109507in}}%
\pgfpathlineto{\pgfqpoint{3.400367in}{2.109951in}}%
\pgfpathlineto{\pgfqpoint{3.405380in}{2.115097in}}%
\pgfpathlineto{\pgfqpoint{3.407887in}{2.115526in}}%
\pgfpathlineto{\pgfqpoint{3.410393in}{2.117504in}}%
\pgfpathlineto{\pgfqpoint{3.415407in}{2.124608in}}%
\pgfpathlineto{\pgfqpoint{3.417913in}{2.126516in}}%
\pgfpathlineto{\pgfqpoint{3.420420in}{2.126596in}}%
\pgfpathlineto{\pgfqpoint{3.422927in}{2.131195in}}%
\pgfpathlineto{\pgfqpoint{3.425433in}{2.133016in}}%
\pgfpathlineto{\pgfqpoint{3.430447in}{2.140602in}}%
\pgfpathlineto{\pgfqpoint{3.432953in}{2.144613in}}%
\pgfpathlineto{\pgfqpoint{3.440473in}{2.145982in}}%
\pgfpathlineto{\pgfqpoint{3.447993in}{2.149520in}}%
\pgfpathlineto{\pgfqpoint{3.453006in}{2.155122in}}%
\pgfpathlineto{\pgfqpoint{3.455513in}{2.156289in}}%
\pgfpathlineto{\pgfqpoint{3.458020in}{2.161599in}}%
\pgfpathlineto{\pgfqpoint{3.460526in}{2.164239in}}%
\pgfpathlineto{\pgfqpoint{3.463033in}{2.164458in}}%
\pgfpathlineto{\pgfqpoint{3.465540in}{2.170797in}}%
\pgfpathlineto{\pgfqpoint{3.475566in}{2.178080in}}%
\pgfpathlineto{\pgfqpoint{3.478073in}{2.183424in}}%
\pgfpathlineto{\pgfqpoint{3.485593in}{2.186194in}}%
\pgfpathlineto{\pgfqpoint{3.488100in}{2.189277in}}%
\pgfpathlineto{\pgfqpoint{3.490606in}{2.198726in}}%
\pgfpathlineto{\pgfqpoint{3.493113in}{2.198852in}}%
\pgfpathlineto{\pgfqpoint{3.500633in}{2.206474in}}%
\pgfpathlineto{\pgfqpoint{3.503139in}{2.211288in}}%
\pgfpathlineto{\pgfqpoint{3.513166in}{2.213103in}}%
\pgfpathlineto{\pgfqpoint{3.520686in}{2.218830in}}%
\pgfpathlineto{\pgfqpoint{3.525699in}{2.220938in}}%
\pgfpathlineto{\pgfqpoint{3.533219in}{2.221651in}}%
\pgfpathlineto{\pgfqpoint{3.535726in}{2.224671in}}%
\pgfpathlineto{\pgfqpoint{3.538233in}{2.225648in}}%
\pgfpathlineto{\pgfqpoint{3.540739in}{2.227978in}}%
\pgfpathlineto{\pgfqpoint{3.548259in}{2.229210in}}%
\pgfpathlineto{\pgfqpoint{3.550766in}{2.229537in}}%
\pgfpathlineto{\pgfqpoint{3.553273in}{2.232084in}}%
\pgfpathlineto{\pgfqpoint{3.558286in}{2.232232in}}%
\pgfpathlineto{\pgfqpoint{3.565806in}{2.238379in}}%
\pgfpathlineto{\pgfqpoint{3.568313in}{2.240535in}}%
\pgfpathlineto{\pgfqpoint{3.570819in}{2.244694in}}%
\pgfpathlineto{\pgfqpoint{3.578339in}{2.248423in}}%
\pgfpathlineto{\pgfqpoint{3.580846in}{2.248431in}}%
\pgfpathlineto{\pgfqpoint{3.585859in}{2.257986in}}%
\pgfpathlineto{\pgfqpoint{3.588366in}{2.258635in}}%
\pgfpathlineto{\pgfqpoint{3.590872in}{2.263369in}}%
\pgfpathlineto{\pgfqpoint{3.603406in}{2.272464in}}%
\pgfpathlineto{\pgfqpoint{3.605912in}{2.272909in}}%
\pgfpathlineto{\pgfqpoint{3.608419in}{2.275283in}}%
\pgfpathlineto{\pgfqpoint{3.610926in}{2.280190in}}%
\pgfpathlineto{\pgfqpoint{3.615939in}{2.280773in}}%
\pgfpathlineto{\pgfqpoint{3.618446in}{2.281393in}}%
\pgfpathlineto{\pgfqpoint{3.623459in}{2.285432in}}%
\pgfpathlineto{\pgfqpoint{3.625966in}{2.292210in}}%
\pgfpathlineto{\pgfqpoint{3.635992in}{2.293739in}}%
\pgfpathlineto{\pgfqpoint{3.638499in}{2.300803in}}%
\pgfpathlineto{\pgfqpoint{3.643512in}{2.301209in}}%
\pgfpathlineto{\pgfqpoint{3.646019in}{2.313390in}}%
\pgfpathlineto{\pgfqpoint{3.648525in}{2.318489in}}%
\pgfpathlineto{\pgfqpoint{3.653539in}{2.321564in}}%
\pgfpathlineto{\pgfqpoint{3.661059in}{2.324296in}}%
\pgfpathlineto{\pgfqpoint{3.666072in}{2.325323in}}%
\pgfpathlineto{\pgfqpoint{3.671085in}{2.330359in}}%
\pgfpathlineto{\pgfqpoint{3.676099in}{2.338533in}}%
\pgfpathlineto{\pgfqpoint{3.683619in}{2.340996in}}%
\pgfpathlineto{\pgfqpoint{3.691139in}{2.342317in}}%
\pgfpathlineto{\pgfqpoint{3.693645in}{2.347984in}}%
\pgfpathlineto{\pgfqpoint{3.698659in}{2.350276in}}%
\pgfpathlineto{\pgfqpoint{3.711192in}{2.353807in}}%
\pgfpathlineto{\pgfqpoint{3.718712in}{2.358806in}}%
\pgfpathlineto{\pgfqpoint{3.723725in}{2.359514in}}%
\pgfpathlineto{\pgfqpoint{3.726232in}{2.362094in}}%
\pgfpathlineto{\pgfqpoint{3.728738in}{2.362995in}}%
\pgfpathlineto{\pgfqpoint{3.731245in}{2.366348in}}%
\pgfpathlineto{\pgfqpoint{3.738765in}{2.368032in}}%
\pgfpathlineto{\pgfqpoint{3.741272in}{2.369859in}}%
\pgfpathlineto{\pgfqpoint{3.743778in}{2.373444in}}%
\pgfpathlineto{\pgfqpoint{3.751298in}{2.375312in}}%
\pgfpathlineto{\pgfqpoint{3.753805in}{2.376469in}}%
\pgfpathlineto{\pgfqpoint{3.761325in}{2.386655in}}%
\pgfpathlineto{\pgfqpoint{3.771352in}{2.389573in}}%
\pgfpathlineto{\pgfqpoint{3.773858in}{2.393139in}}%
\pgfpathlineto{\pgfqpoint{3.783885in}{2.397085in}}%
\pgfpathlineto{\pgfqpoint{3.786391in}{2.401079in}}%
\pgfpathlineto{\pgfqpoint{3.791405in}{2.403683in}}%
\pgfpathlineto{\pgfqpoint{3.793911in}{2.404861in}}%
\pgfpathlineto{\pgfqpoint{3.796418in}{2.407977in}}%
\pgfpathlineto{\pgfqpoint{3.798925in}{2.414527in}}%
\pgfpathlineto{\pgfqpoint{3.803938in}{2.415281in}}%
\pgfpathlineto{\pgfqpoint{3.808951in}{2.419219in}}%
\pgfpathlineto{\pgfqpoint{3.813965in}{2.420790in}}%
\pgfpathlineto{\pgfqpoint{3.821485in}{2.422086in}}%
\pgfpathlineto{\pgfqpoint{3.823991in}{2.423811in}}%
\pgfpathlineto{\pgfqpoint{3.826498in}{2.430680in}}%
\pgfpathlineto{\pgfqpoint{3.834018in}{2.440732in}}%
\pgfpathlineto{\pgfqpoint{3.836525in}{2.441069in}}%
\pgfpathlineto{\pgfqpoint{3.839031in}{2.445046in}}%
\pgfpathlineto{\pgfqpoint{3.841538in}{2.445712in}}%
\pgfpathlineto{\pgfqpoint{3.844044in}{2.447819in}}%
\pgfpathlineto{\pgfqpoint{3.849058in}{2.448839in}}%
\pgfpathlineto{\pgfqpoint{3.851564in}{2.452410in}}%
\pgfpathlineto{\pgfqpoint{3.856578in}{2.454616in}}%
\pgfpathlineto{\pgfqpoint{3.859084in}{2.455196in}}%
\pgfpathlineto{\pgfqpoint{3.864098in}{2.464571in}}%
\pgfpathlineto{\pgfqpoint{3.866604in}{2.467721in}}%
\pgfpathlineto{\pgfqpoint{3.869111in}{2.468950in}}%
\pgfpathlineto{\pgfqpoint{3.874124in}{2.473460in}}%
\pgfpathlineto{\pgfqpoint{3.876631in}{2.479534in}}%
\pgfpathlineto{\pgfqpoint{3.891671in}{2.483990in}}%
\pgfpathlineto{\pgfqpoint{3.896684in}{2.502653in}}%
\pgfpathlineto{\pgfqpoint{3.899191in}{2.502975in}}%
\pgfpathlineto{\pgfqpoint{3.901698in}{2.504590in}}%
\pgfpathlineto{\pgfqpoint{3.904204in}{2.504673in}}%
\pgfpathlineto{\pgfqpoint{3.906711in}{2.508221in}}%
\pgfpathlineto{\pgfqpoint{3.909217in}{2.508991in}}%
\pgfpathlineto{\pgfqpoint{3.911724in}{2.515491in}}%
\pgfpathlineto{\pgfqpoint{3.916737in}{2.517809in}}%
\pgfpathlineto{\pgfqpoint{3.921751in}{2.519095in}}%
\pgfpathlineto{\pgfqpoint{3.924257in}{2.524639in}}%
\pgfpathlineto{\pgfqpoint{3.929271in}{2.526046in}}%
\pgfpathlineto{\pgfqpoint{3.931777in}{2.531151in}}%
\pgfpathlineto{\pgfqpoint{3.934284in}{2.531291in}}%
\pgfpathlineto{\pgfqpoint{3.939297in}{2.534281in}}%
\pgfpathlineto{\pgfqpoint{3.949324in}{2.546180in}}%
\pgfpathlineto{\pgfqpoint{3.954337in}{2.550325in}}%
\pgfpathlineto{\pgfqpoint{3.961857in}{2.565279in}}%
\pgfpathlineto{\pgfqpoint{3.964364in}{2.565462in}}%
\pgfpathlineto{\pgfqpoint{3.969377in}{2.567365in}}%
\pgfpathlineto{\pgfqpoint{3.974391in}{2.568381in}}%
\pgfpathlineto{\pgfqpoint{3.984417in}{2.574556in}}%
\pgfpathlineto{\pgfqpoint{3.986924in}{2.574603in}}%
\pgfpathlineto{\pgfqpoint{3.989430in}{2.580029in}}%
\pgfpathlineto{\pgfqpoint{3.996950in}{2.582569in}}%
\pgfpathlineto{\pgfqpoint{3.999457in}{2.586189in}}%
\pgfpathlineto{\pgfqpoint{4.001964in}{2.591855in}}%
\pgfpathlineto{\pgfqpoint{4.009484in}{2.596650in}}%
\pgfpathlineto{\pgfqpoint{4.011990in}{2.604855in}}%
\pgfpathlineto{\pgfqpoint{4.027030in}{2.615135in}}%
\pgfpathlineto{\pgfqpoint{4.029537in}{2.615270in}}%
\pgfpathlineto{\pgfqpoint{4.032044in}{2.616536in}}%
\pgfpathlineto{\pgfqpoint{4.037057in}{2.620472in}}%
\pgfpathlineto{\pgfqpoint{4.042070in}{2.625405in}}%
\pgfpathlineto{\pgfqpoint{4.044577in}{2.633024in}}%
\pgfpathlineto{\pgfqpoint{4.047083in}{2.635005in}}%
\pgfpathlineto{\pgfqpoint{4.049590in}{2.639323in}}%
\pgfpathlineto{\pgfqpoint{4.052097in}{2.640205in}}%
\pgfpathlineto{\pgfqpoint{4.054603in}{2.646984in}}%
\pgfpathlineto{\pgfqpoint{4.057110in}{2.649412in}}%
\pgfpathlineto{\pgfqpoint{4.059617in}{2.656695in}}%
\pgfpathlineto{\pgfqpoint{4.062123in}{2.657652in}}%
\pgfpathlineto{\pgfqpoint{4.074657in}{2.693799in}}%
\pgfpathlineto{\pgfqpoint{4.077163in}{2.694699in}}%
\pgfpathlineto{\pgfqpoint{4.079670in}{2.698428in}}%
\pgfpathlineto{\pgfqpoint{4.082177in}{2.699363in}}%
\pgfpathlineto{\pgfqpoint{4.084683in}{2.702506in}}%
\pgfpathlineto{\pgfqpoint{4.087190in}{2.702874in}}%
\pgfpathlineto{\pgfqpoint{4.089697in}{2.706220in}}%
\pgfpathlineto{\pgfqpoint{4.102230in}{2.712655in}}%
\pgfpathlineto{\pgfqpoint{4.104737in}{2.715618in}}%
\pgfpathlineto{\pgfqpoint{4.107243in}{2.723519in}}%
\pgfpathlineto{\pgfqpoint{4.109750in}{2.723925in}}%
\pgfpathlineto{\pgfqpoint{4.117270in}{2.731344in}}%
\pgfpathlineto{\pgfqpoint{4.119776in}{2.732273in}}%
\pgfpathlineto{\pgfqpoint{4.122283in}{2.744056in}}%
\pgfpathlineto{\pgfqpoint{4.127296in}{2.751820in}}%
\pgfpathlineto{\pgfqpoint{4.129803in}{2.752205in}}%
\pgfpathlineto{\pgfqpoint{4.137323in}{2.757898in}}%
\pgfpathlineto{\pgfqpoint{4.139830in}{2.760792in}}%
\pgfpathlineto{\pgfqpoint{4.142336in}{2.766208in}}%
\pgfpathlineto{\pgfqpoint{4.144843in}{2.768920in}}%
\pgfpathlineto{\pgfqpoint{4.147350in}{2.769616in}}%
\pgfpathlineto{\pgfqpoint{4.154870in}{2.797722in}}%
\pgfpathlineto{\pgfqpoint{4.157376in}{2.800184in}}%
\pgfpathlineto{\pgfqpoint{4.159883in}{2.800374in}}%
\pgfpathlineto{\pgfqpoint{4.167403in}{2.807127in}}%
\pgfpathlineto{\pgfqpoint{4.172416in}{2.807406in}}%
\pgfpathlineto{\pgfqpoint{4.177430in}{2.812025in}}%
\pgfpathlineto{\pgfqpoint{4.179936in}{2.815651in}}%
\pgfpathlineto{\pgfqpoint{4.182443in}{2.821501in}}%
\pgfpathlineto{\pgfqpoint{4.184949in}{2.824433in}}%
\pgfpathlineto{\pgfqpoint{4.192469in}{2.827879in}}%
\pgfpathlineto{\pgfqpoint{4.194976in}{2.831444in}}%
\pgfpathlineto{\pgfqpoint{4.202496in}{2.852035in}}%
\pgfpathlineto{\pgfqpoint{4.207509in}{2.852755in}}%
\pgfpathlineto{\pgfqpoint{4.212523in}{2.856223in}}%
\pgfpathlineto{\pgfqpoint{4.215029in}{2.856457in}}%
\pgfpathlineto{\pgfqpoint{4.217536in}{2.867713in}}%
\pgfpathlineto{\pgfqpoint{4.220043in}{2.871830in}}%
\pgfpathlineto{\pgfqpoint{4.222549in}{2.872254in}}%
\pgfpathlineto{\pgfqpoint{4.227563in}{2.879374in}}%
\pgfpathlineto{\pgfqpoint{4.230069in}{2.886067in}}%
\pgfpathlineto{\pgfqpoint{4.232576in}{2.886290in}}%
\pgfpathlineto{\pgfqpoint{4.235083in}{2.887739in}}%
\pgfpathlineto{\pgfqpoint{4.237589in}{2.894110in}}%
\pgfpathlineto{\pgfqpoint{4.240096in}{2.895741in}}%
\pgfpathlineto{\pgfqpoint{4.242603in}{2.905275in}}%
\pgfpathlineto{\pgfqpoint{4.242603in}{2.905275in}}%
\pgfusepath{stroke}%
\end{pgfscope}%
\begin{pgfscope}%
\pgfpathrectangle{\pgfqpoint{0.708220in}{0.535823in}}{\pgfqpoint{5.013309in}{2.369453in}}%
\pgfusepath{clip}%
\pgfsetbuttcap%
\pgfsetroundjoin%
\pgfsetlinewidth{1.003750pt}%
\definecolor{currentstroke}{rgb}{0.000000,0.000000,0.000000}%
\pgfsetstrokecolor{currentstroke}%
\pgfsetdash{{3.700000pt}{1.600000pt}}{0.000000pt}%
\pgfpathmoveto{\pgfqpoint{0.708220in}{0.937573in}}%
\pgfpathlineto{\pgfqpoint{0.718246in}{0.940174in}}%
\pgfpathlineto{\pgfqpoint{0.755846in}{0.942959in}}%
\pgfpathlineto{\pgfqpoint{0.851099in}{0.950459in}}%
\pgfpathlineto{\pgfqpoint{0.858619in}{0.951541in}}%
\pgfpathlineto{\pgfqpoint{0.993978in}{0.956186in}}%
\pgfpathlineto{\pgfqpoint{1.039098in}{0.957204in}}%
\pgfpathlineto{\pgfqpoint{1.054138in}{0.957600in}}%
\pgfpathlineto{\pgfqpoint{1.104271in}{0.958900in}}%
\pgfpathlineto{\pgfqpoint{1.141871in}{0.960190in}}%
\pgfpathlineto{\pgfqpoint{1.212057in}{0.962813in}}%
\pgfpathlineto{\pgfqpoint{1.224591in}{0.963296in}}%
\pgfpathlineto{\pgfqpoint{1.254671in}{0.964639in}}%
\pgfpathlineto{\pgfqpoint{1.272217in}{0.965901in}}%
\pgfpathlineto{\pgfqpoint{1.342403in}{0.969194in}}%
\pgfpathlineto{\pgfqpoint{1.357443in}{0.971183in}}%
\pgfpathlineto{\pgfqpoint{1.372483in}{0.972761in}}%
\pgfpathlineto{\pgfqpoint{1.392537in}{0.975478in}}%
\pgfpathlineto{\pgfqpoint{1.407576in}{0.978126in}}%
\pgfpathlineto{\pgfqpoint{1.417603in}{0.979047in}}%
\pgfpathlineto{\pgfqpoint{1.442670in}{0.982671in}}%
\pgfpathlineto{\pgfqpoint{1.455203in}{0.983795in}}%
\pgfpathlineto{\pgfqpoint{1.462723in}{0.985386in}}%
\pgfpathlineto{\pgfqpoint{1.472749in}{0.986987in}}%
\pgfpathlineto{\pgfqpoint{1.490296in}{0.988034in}}%
\pgfpathlineto{\pgfqpoint{1.497816in}{0.989452in}}%
\pgfpathlineto{\pgfqpoint{1.505336in}{0.990315in}}%
\pgfpathlineto{\pgfqpoint{1.512856in}{0.991402in}}%
\pgfpathlineto{\pgfqpoint{1.517869in}{0.991655in}}%
\pgfpathlineto{\pgfqpoint{1.522883in}{0.993758in}}%
\pgfpathlineto{\pgfqpoint{1.530402in}{0.994977in}}%
\pgfpathlineto{\pgfqpoint{1.545442in}{0.997075in}}%
\pgfpathlineto{\pgfqpoint{1.557976in}{0.999572in}}%
\pgfpathlineto{\pgfqpoint{1.562989in}{1.001036in}}%
\pgfpathlineto{\pgfqpoint{1.568002in}{1.002068in}}%
\pgfpathlineto{\pgfqpoint{1.575522in}{1.003746in}}%
\pgfpathlineto{\pgfqpoint{1.583042in}{1.005607in}}%
\pgfpathlineto{\pgfqpoint{1.635682in}{1.015491in}}%
\pgfpathlineto{\pgfqpoint{1.638189in}{1.018468in}}%
\pgfpathlineto{\pgfqpoint{1.643202in}{1.019320in}}%
\pgfpathlineto{\pgfqpoint{1.648215in}{1.021354in}}%
\pgfpathlineto{\pgfqpoint{1.658242in}{1.021990in}}%
\pgfpathlineto{\pgfqpoint{1.678295in}{1.026012in}}%
\pgfpathlineto{\pgfqpoint{1.685815in}{1.026827in}}%
\pgfpathlineto{\pgfqpoint{1.720908in}{1.036416in}}%
\pgfpathlineto{\pgfqpoint{1.725922in}{1.038971in}}%
\pgfpathlineto{\pgfqpoint{1.730935in}{1.041273in}}%
\pgfpathlineto{\pgfqpoint{1.733442in}{1.042859in}}%
\pgfpathlineto{\pgfqpoint{1.738455in}{1.043162in}}%
\pgfpathlineto{\pgfqpoint{1.753495in}{1.048268in}}%
\pgfpathlineto{\pgfqpoint{1.758508in}{1.049405in}}%
\pgfpathlineto{\pgfqpoint{1.763521in}{1.051453in}}%
\pgfpathlineto{\pgfqpoint{1.771041in}{1.052485in}}%
\pgfpathlineto{\pgfqpoint{1.788588in}{1.060564in}}%
\pgfpathlineto{\pgfqpoint{1.793601in}{1.062897in}}%
\pgfpathlineto{\pgfqpoint{1.801121in}{1.064477in}}%
\pgfpathlineto{\pgfqpoint{1.806134in}{1.064666in}}%
\pgfpathlineto{\pgfqpoint{1.816161in}{1.069132in}}%
\pgfpathlineto{\pgfqpoint{1.826188in}{1.070192in}}%
\pgfpathlineto{\pgfqpoint{1.828694in}{1.070928in}}%
\pgfpathlineto{\pgfqpoint{1.833708in}{1.073439in}}%
\pgfpathlineto{\pgfqpoint{1.838721in}{1.075073in}}%
\pgfpathlineto{\pgfqpoint{1.841228in}{1.077707in}}%
\pgfpathlineto{\pgfqpoint{1.848748in}{1.080293in}}%
\pgfpathlineto{\pgfqpoint{1.851254in}{1.082700in}}%
\pgfpathlineto{\pgfqpoint{1.861281in}{1.085408in}}%
\pgfpathlineto{\pgfqpoint{1.866294in}{1.087476in}}%
\pgfpathlineto{\pgfqpoint{1.871307in}{1.089091in}}%
\pgfpathlineto{\pgfqpoint{1.873814in}{1.089173in}}%
\pgfpathlineto{\pgfqpoint{1.881334in}{1.091989in}}%
\pgfpathlineto{\pgfqpoint{1.891361in}{1.093264in}}%
\pgfpathlineto{\pgfqpoint{1.898881in}{1.095378in}}%
\pgfpathlineto{\pgfqpoint{1.901387in}{1.095831in}}%
\pgfpathlineto{\pgfqpoint{1.906401in}{1.100208in}}%
\pgfpathlineto{\pgfqpoint{1.911414in}{1.103553in}}%
\pgfpathlineto{\pgfqpoint{1.916427in}{1.104751in}}%
\pgfpathlineto{\pgfqpoint{1.918934in}{1.107177in}}%
\pgfpathlineto{\pgfqpoint{1.923947in}{1.108301in}}%
\pgfpathlineto{\pgfqpoint{1.926454in}{1.111356in}}%
\pgfpathlineto{\pgfqpoint{1.928961in}{1.111499in}}%
\pgfpathlineto{\pgfqpoint{1.933974in}{1.113094in}}%
\pgfpathlineto{\pgfqpoint{1.941494in}{1.114286in}}%
\pgfpathlineto{\pgfqpoint{1.946507in}{1.115511in}}%
\pgfpathlineto{\pgfqpoint{1.949014in}{1.116460in}}%
\pgfpathlineto{\pgfqpoint{1.954027in}{1.120931in}}%
\pgfpathlineto{\pgfqpoint{1.984107in}{1.139519in}}%
\pgfpathlineto{\pgfqpoint{1.989120in}{1.140990in}}%
\pgfpathlineto{\pgfqpoint{1.991627in}{1.141287in}}%
\pgfpathlineto{\pgfqpoint{1.994134in}{1.144931in}}%
\pgfpathlineto{\pgfqpoint{2.001654in}{1.146181in}}%
\pgfpathlineto{\pgfqpoint{2.004160in}{1.147465in}}%
\pgfpathlineto{\pgfqpoint{2.006667in}{1.150034in}}%
\pgfpathlineto{\pgfqpoint{2.011680in}{1.150725in}}%
\pgfpathlineto{\pgfqpoint{2.016693in}{1.153025in}}%
\pgfpathlineto{\pgfqpoint{2.019200in}{1.154697in}}%
\pgfpathlineto{\pgfqpoint{2.021707in}{1.154715in}}%
\pgfpathlineto{\pgfqpoint{2.024213in}{1.159155in}}%
\pgfpathlineto{\pgfqpoint{2.034240in}{1.161641in}}%
\pgfpathlineto{\pgfqpoint{2.036747in}{1.163690in}}%
\pgfpathlineto{\pgfqpoint{2.039253in}{1.163862in}}%
\pgfpathlineto{\pgfqpoint{2.041760in}{1.165353in}}%
\pgfpathlineto{\pgfqpoint{2.051787in}{1.167121in}}%
\pgfpathlineto{\pgfqpoint{2.054293in}{1.171373in}}%
\pgfpathlineto{\pgfqpoint{2.059307in}{1.173926in}}%
\pgfpathlineto{\pgfqpoint{2.071840in}{1.178338in}}%
\pgfpathlineto{\pgfqpoint{2.076853in}{1.182959in}}%
\pgfpathlineto{\pgfqpoint{2.079360in}{1.186925in}}%
\pgfpathlineto{\pgfqpoint{2.089386in}{1.188458in}}%
\pgfpathlineto{\pgfqpoint{2.091893in}{1.192617in}}%
\pgfpathlineto{\pgfqpoint{2.096906in}{1.193089in}}%
\pgfpathlineto{\pgfqpoint{2.109440in}{1.200704in}}%
\pgfpathlineto{\pgfqpoint{2.142026in}{1.209763in}}%
\pgfpathlineto{\pgfqpoint{2.152053in}{1.216641in}}%
\pgfpathlineto{\pgfqpoint{2.154559in}{1.218675in}}%
\pgfpathlineto{\pgfqpoint{2.159573in}{1.219081in}}%
\pgfpathlineto{\pgfqpoint{2.162079in}{1.221601in}}%
\pgfpathlineto{\pgfqpoint{2.167093in}{1.222942in}}%
\pgfpathlineto{\pgfqpoint{2.169599in}{1.226161in}}%
\pgfpathlineto{\pgfqpoint{2.179626in}{1.228452in}}%
\pgfpathlineto{\pgfqpoint{2.184639in}{1.231047in}}%
\pgfpathlineto{\pgfqpoint{2.187146in}{1.231202in}}%
\pgfpathlineto{\pgfqpoint{2.189653in}{1.237959in}}%
\pgfpathlineto{\pgfqpoint{2.192159in}{1.238578in}}%
\pgfpathlineto{\pgfqpoint{2.204693in}{1.247099in}}%
\pgfpathlineto{\pgfqpoint{2.207199in}{1.247956in}}%
\pgfpathlineto{\pgfqpoint{2.209706in}{1.250253in}}%
\pgfpathlineto{\pgfqpoint{2.212212in}{1.250282in}}%
\pgfpathlineto{\pgfqpoint{2.229759in}{1.259955in}}%
\pgfpathlineto{\pgfqpoint{2.239786in}{1.264179in}}%
\pgfpathlineto{\pgfqpoint{2.242292in}{1.266533in}}%
\pgfpathlineto{\pgfqpoint{2.249812in}{1.268677in}}%
\pgfpathlineto{\pgfqpoint{2.254826in}{1.273857in}}%
\pgfpathlineto{\pgfqpoint{2.259839in}{1.279093in}}%
\pgfpathlineto{\pgfqpoint{2.277385in}{1.280997in}}%
\pgfpathlineto{\pgfqpoint{2.287412in}{1.284250in}}%
\pgfpathlineto{\pgfqpoint{2.297439in}{1.292934in}}%
\pgfpathlineto{\pgfqpoint{2.299945in}{1.293723in}}%
\pgfpathlineto{\pgfqpoint{2.307465in}{1.299386in}}%
\pgfpathlineto{\pgfqpoint{2.314985in}{1.302183in}}%
\pgfpathlineto{\pgfqpoint{2.319999in}{1.303251in}}%
\pgfpathlineto{\pgfqpoint{2.325012in}{1.312529in}}%
\pgfpathlineto{\pgfqpoint{2.330025in}{1.312812in}}%
\pgfpathlineto{\pgfqpoint{2.345065in}{1.321074in}}%
\pgfpathlineto{\pgfqpoint{2.347572in}{1.321704in}}%
\pgfpathlineto{\pgfqpoint{2.350078in}{1.325445in}}%
\pgfpathlineto{\pgfqpoint{2.352585in}{1.326449in}}%
\pgfpathlineto{\pgfqpoint{2.355092in}{1.329994in}}%
\pgfpathlineto{\pgfqpoint{2.357598in}{1.330213in}}%
\pgfpathlineto{\pgfqpoint{2.360105in}{1.334178in}}%
\pgfpathlineto{\pgfqpoint{2.365118in}{1.334766in}}%
\pgfpathlineto{\pgfqpoint{2.367625in}{1.339891in}}%
\pgfpathlineto{\pgfqpoint{2.370132in}{1.340028in}}%
\pgfpathlineto{\pgfqpoint{2.375145in}{1.343692in}}%
\pgfpathlineto{\pgfqpoint{2.382665in}{1.345970in}}%
\pgfpathlineto{\pgfqpoint{2.385172in}{1.348120in}}%
\pgfpathlineto{\pgfqpoint{2.390185in}{1.348956in}}%
\pgfpathlineto{\pgfqpoint{2.395198in}{1.349962in}}%
\pgfpathlineto{\pgfqpoint{2.402718in}{1.351630in}}%
\pgfpathlineto{\pgfqpoint{2.410238in}{1.353350in}}%
\pgfpathlineto{\pgfqpoint{2.420265in}{1.354403in}}%
\pgfpathlineto{\pgfqpoint{2.427785in}{1.360909in}}%
\pgfpathlineto{\pgfqpoint{2.435305in}{1.364425in}}%
\pgfpathlineto{\pgfqpoint{2.437811in}{1.364849in}}%
\pgfpathlineto{\pgfqpoint{2.440318in}{1.366430in}}%
\pgfpathlineto{\pgfqpoint{2.442825in}{1.369817in}}%
\pgfpathlineto{\pgfqpoint{2.450345in}{1.371569in}}%
\pgfpathlineto{\pgfqpoint{2.452851in}{1.374993in}}%
\pgfpathlineto{\pgfqpoint{2.462878in}{1.377285in}}%
\pgfpathlineto{\pgfqpoint{2.467891in}{1.379451in}}%
\pgfpathlineto{\pgfqpoint{2.470398in}{1.384667in}}%
\pgfpathlineto{\pgfqpoint{2.480425in}{1.388199in}}%
\pgfpathlineto{\pgfqpoint{2.487944in}{1.389640in}}%
\pgfpathlineto{\pgfqpoint{2.490451in}{1.395246in}}%
\pgfpathlineto{\pgfqpoint{2.492958in}{1.396537in}}%
\pgfpathlineto{\pgfqpoint{2.495464in}{1.401085in}}%
\pgfpathlineto{\pgfqpoint{2.500478in}{1.401323in}}%
\pgfpathlineto{\pgfqpoint{2.502984in}{1.403930in}}%
\pgfpathlineto{\pgfqpoint{2.507998in}{1.404629in}}%
\pgfpathlineto{\pgfqpoint{2.510504in}{1.407208in}}%
\pgfpathlineto{\pgfqpoint{2.518024in}{1.408478in}}%
\pgfpathlineto{\pgfqpoint{2.520531in}{1.414285in}}%
\pgfpathlineto{\pgfqpoint{2.528051in}{1.419347in}}%
\pgfpathlineto{\pgfqpoint{2.530558in}{1.424446in}}%
\pgfpathlineto{\pgfqpoint{2.533064in}{1.434652in}}%
\pgfpathlineto{\pgfqpoint{2.535571in}{1.434712in}}%
\pgfpathlineto{\pgfqpoint{2.538078in}{1.438882in}}%
\pgfpathlineto{\pgfqpoint{2.540584in}{1.439335in}}%
\pgfpathlineto{\pgfqpoint{2.543091in}{1.442400in}}%
\pgfpathlineto{\pgfqpoint{2.553117in}{1.443169in}}%
\pgfpathlineto{\pgfqpoint{2.578184in}{1.452620in}}%
\pgfpathlineto{\pgfqpoint{2.580691in}{1.455929in}}%
\pgfpathlineto{\pgfqpoint{2.585704in}{1.456265in}}%
\pgfpathlineto{\pgfqpoint{2.588211in}{1.461634in}}%
\pgfpathlineto{\pgfqpoint{2.593224in}{1.462224in}}%
\pgfpathlineto{\pgfqpoint{2.600744in}{1.474109in}}%
\pgfpathlineto{\pgfqpoint{2.603251in}{1.474558in}}%
\pgfpathlineto{\pgfqpoint{2.608264in}{1.478251in}}%
\pgfpathlineto{\pgfqpoint{2.610771in}{1.482507in}}%
\pgfpathlineto{\pgfqpoint{2.615784in}{1.484276in}}%
\pgfpathlineto{\pgfqpoint{2.620797in}{1.484797in}}%
\pgfpathlineto{\pgfqpoint{2.623304in}{1.487816in}}%
\pgfpathlineto{\pgfqpoint{2.625810in}{1.488667in}}%
\pgfpathlineto{\pgfqpoint{2.628317in}{1.492058in}}%
\pgfpathlineto{\pgfqpoint{2.633330in}{1.492608in}}%
\pgfpathlineto{\pgfqpoint{2.635837in}{1.496418in}}%
\pgfpathlineto{\pgfqpoint{2.638344in}{1.496541in}}%
\pgfpathlineto{\pgfqpoint{2.643357in}{1.499479in}}%
\pgfpathlineto{\pgfqpoint{2.650877in}{1.514456in}}%
\pgfpathlineto{\pgfqpoint{2.658397in}{1.522106in}}%
\pgfpathlineto{\pgfqpoint{2.660904in}{1.528836in}}%
\pgfpathlineto{\pgfqpoint{2.665917in}{1.530750in}}%
\pgfpathlineto{\pgfqpoint{2.668424in}{1.532354in}}%
\pgfpathlineto{\pgfqpoint{2.670930in}{1.532512in}}%
\pgfpathlineto{\pgfqpoint{2.673437in}{1.536727in}}%
\pgfpathlineto{\pgfqpoint{2.675944in}{1.537168in}}%
\pgfpathlineto{\pgfqpoint{2.680957in}{1.548220in}}%
\pgfpathlineto{\pgfqpoint{2.685970in}{1.550220in}}%
\pgfpathlineto{\pgfqpoint{2.690983in}{1.554262in}}%
\pgfpathlineto{\pgfqpoint{2.693490in}{1.558758in}}%
\pgfpathlineto{\pgfqpoint{2.695997in}{1.558783in}}%
\pgfpathlineto{\pgfqpoint{2.718557in}{1.570864in}}%
\pgfpathlineto{\pgfqpoint{2.721063in}{1.572722in}}%
\pgfpathlineto{\pgfqpoint{2.723570in}{1.573103in}}%
\pgfpathlineto{\pgfqpoint{2.728583in}{1.577101in}}%
\pgfpathlineto{\pgfqpoint{2.731090in}{1.577123in}}%
\pgfpathlineto{\pgfqpoint{2.733597in}{1.580360in}}%
\pgfpathlineto{\pgfqpoint{2.738610in}{1.581558in}}%
\pgfpathlineto{\pgfqpoint{2.741117in}{1.586348in}}%
\pgfpathlineto{\pgfqpoint{2.746130in}{1.587460in}}%
\pgfpathlineto{\pgfqpoint{2.751143in}{1.591471in}}%
\pgfpathlineto{\pgfqpoint{2.753650in}{1.591613in}}%
\pgfpathlineto{\pgfqpoint{2.756156in}{1.592888in}}%
\pgfpathlineto{\pgfqpoint{2.758663in}{1.596115in}}%
\pgfpathlineto{\pgfqpoint{2.768690in}{1.598693in}}%
\pgfpathlineto{\pgfqpoint{2.771196in}{1.604083in}}%
\pgfpathlineto{\pgfqpoint{2.776210in}{1.605212in}}%
\pgfpathlineto{\pgfqpoint{2.778716in}{1.611845in}}%
\pgfpathlineto{\pgfqpoint{2.781223in}{1.612012in}}%
\pgfpathlineto{\pgfqpoint{2.783730in}{1.615517in}}%
\pgfpathlineto{\pgfqpoint{2.788743in}{1.616110in}}%
\pgfpathlineto{\pgfqpoint{2.791250in}{1.618842in}}%
\pgfpathlineto{\pgfqpoint{2.793756in}{1.624318in}}%
\pgfpathlineto{\pgfqpoint{2.803783in}{1.629128in}}%
\pgfpathlineto{\pgfqpoint{2.808796in}{1.635125in}}%
\pgfpathlineto{\pgfqpoint{2.811303in}{1.635776in}}%
\pgfpathlineto{\pgfqpoint{2.818823in}{1.641465in}}%
\pgfpathlineto{\pgfqpoint{2.821329in}{1.641710in}}%
\pgfpathlineto{\pgfqpoint{2.823836in}{1.646673in}}%
\pgfpathlineto{\pgfqpoint{2.828849in}{1.648122in}}%
\pgfpathlineto{\pgfqpoint{2.833863in}{1.662270in}}%
\pgfpathlineto{\pgfqpoint{2.836369in}{1.662320in}}%
\pgfpathlineto{\pgfqpoint{2.838876in}{1.663657in}}%
\pgfpathlineto{\pgfqpoint{2.841383in}{1.666754in}}%
\pgfpathlineto{\pgfqpoint{2.846396in}{1.667437in}}%
\pgfpathlineto{\pgfqpoint{2.848903in}{1.673870in}}%
\pgfpathlineto{\pgfqpoint{2.853916in}{1.674922in}}%
\pgfpathlineto{\pgfqpoint{2.856423in}{1.679525in}}%
\pgfpathlineto{\pgfqpoint{2.863943in}{1.681481in}}%
\pgfpathlineto{\pgfqpoint{2.868956in}{1.685235in}}%
\pgfpathlineto{\pgfqpoint{2.878983in}{1.689768in}}%
\pgfpathlineto{\pgfqpoint{2.881489in}{1.693976in}}%
\pgfpathlineto{\pgfqpoint{2.886503in}{1.695558in}}%
\pgfpathlineto{\pgfqpoint{2.889009in}{1.697955in}}%
\pgfpathlineto{\pgfqpoint{2.894022in}{1.699661in}}%
\pgfpathlineto{\pgfqpoint{2.896529in}{1.705318in}}%
\pgfpathlineto{\pgfqpoint{2.901542in}{1.706230in}}%
\pgfpathlineto{\pgfqpoint{2.904049in}{1.709958in}}%
\pgfpathlineto{\pgfqpoint{2.906556in}{1.711084in}}%
\pgfpathlineto{\pgfqpoint{2.911569in}{1.715546in}}%
\pgfpathlineto{\pgfqpoint{2.914076in}{1.715801in}}%
\pgfpathlineto{\pgfqpoint{2.921596in}{1.722823in}}%
\pgfpathlineto{\pgfqpoint{2.924102in}{1.729447in}}%
\pgfpathlineto{\pgfqpoint{2.926609in}{1.731472in}}%
\pgfpathlineto{\pgfqpoint{2.929116in}{1.738225in}}%
\pgfpathlineto{\pgfqpoint{2.936636in}{1.741113in}}%
\pgfpathlineto{\pgfqpoint{2.939142in}{1.744801in}}%
\pgfpathlineto{\pgfqpoint{2.949169in}{1.746855in}}%
\pgfpathlineto{\pgfqpoint{2.951676in}{1.749681in}}%
\pgfpathlineto{\pgfqpoint{2.954182in}{1.754203in}}%
\pgfpathlineto{\pgfqpoint{2.956689in}{1.755967in}}%
\pgfpathlineto{\pgfqpoint{2.959195in}{1.763413in}}%
\pgfpathlineto{\pgfqpoint{2.964209in}{1.764424in}}%
\pgfpathlineto{\pgfqpoint{2.966715in}{1.768052in}}%
\pgfpathlineto{\pgfqpoint{2.969222in}{1.768144in}}%
\pgfpathlineto{\pgfqpoint{2.974235in}{1.781525in}}%
\pgfpathlineto{\pgfqpoint{2.981755in}{1.784670in}}%
\pgfpathlineto{\pgfqpoint{2.989275in}{1.793314in}}%
\pgfpathlineto{\pgfqpoint{2.991782in}{1.800430in}}%
\pgfpathlineto{\pgfqpoint{2.994289in}{1.803689in}}%
\pgfpathlineto{\pgfqpoint{2.996795in}{1.804286in}}%
\pgfpathlineto{\pgfqpoint{3.001809in}{1.808708in}}%
\pgfpathlineto{\pgfqpoint{3.004315in}{1.813344in}}%
\pgfpathlineto{\pgfqpoint{3.011835in}{1.813700in}}%
\pgfpathlineto{\pgfqpoint{3.014342in}{1.819851in}}%
\pgfpathlineto{\pgfqpoint{3.016849in}{1.820261in}}%
\pgfpathlineto{\pgfqpoint{3.019355in}{1.823104in}}%
\pgfpathlineto{\pgfqpoint{3.021862in}{1.823728in}}%
\pgfpathlineto{\pgfqpoint{3.024369in}{1.825688in}}%
\pgfpathlineto{\pgfqpoint{3.031888in}{1.826786in}}%
\pgfpathlineto{\pgfqpoint{3.051942in}{1.846448in}}%
\pgfpathlineto{\pgfqpoint{3.056955in}{1.853643in}}%
\pgfpathlineto{\pgfqpoint{3.059462in}{1.856274in}}%
\pgfpathlineto{\pgfqpoint{3.061968in}{1.856625in}}%
\pgfpathlineto{\pgfqpoint{3.064475in}{1.859354in}}%
\pgfpathlineto{\pgfqpoint{3.071995in}{1.861701in}}%
\pgfpathlineto{\pgfqpoint{3.074502in}{1.863631in}}%
\pgfpathlineto{\pgfqpoint{3.079515in}{1.876829in}}%
\pgfpathlineto{\pgfqpoint{3.082022in}{1.879128in}}%
\pgfpathlineto{\pgfqpoint{3.087035in}{1.880044in}}%
\pgfpathlineto{\pgfqpoint{3.089542in}{1.887311in}}%
\pgfpathlineto{\pgfqpoint{3.094555in}{1.888243in}}%
\pgfpathlineto{\pgfqpoint{3.107088in}{1.889391in}}%
\pgfpathlineto{\pgfqpoint{3.109595in}{1.890097in}}%
\pgfpathlineto{\pgfqpoint{3.112101in}{1.896081in}}%
\pgfpathlineto{\pgfqpoint{3.114608in}{1.898351in}}%
\pgfpathlineto{\pgfqpoint{3.119621in}{1.908748in}}%
\pgfpathlineto{\pgfqpoint{3.122128in}{1.917814in}}%
\pgfpathlineto{\pgfqpoint{3.124635in}{1.919946in}}%
\pgfpathlineto{\pgfqpoint{3.127141in}{1.920446in}}%
\pgfpathlineto{\pgfqpoint{3.134661in}{1.930394in}}%
\pgfpathlineto{\pgfqpoint{3.139675in}{1.931396in}}%
\pgfpathlineto{\pgfqpoint{3.147195in}{1.937307in}}%
\pgfpathlineto{\pgfqpoint{3.149701in}{1.949493in}}%
\pgfpathlineto{\pgfqpoint{3.152208in}{1.952078in}}%
\pgfpathlineto{\pgfqpoint{3.167248in}{1.957435in}}%
\pgfpathlineto{\pgfqpoint{3.174768in}{1.969512in}}%
\pgfpathlineto{\pgfqpoint{3.177274in}{1.970385in}}%
\pgfpathlineto{\pgfqpoint{3.182288in}{1.982150in}}%
\pgfpathlineto{\pgfqpoint{3.192314in}{1.983935in}}%
\pgfpathlineto{\pgfqpoint{3.202341in}{1.985838in}}%
\pgfpathlineto{\pgfqpoint{3.207354in}{1.989393in}}%
\pgfpathlineto{\pgfqpoint{3.214874in}{2.003950in}}%
\pgfpathlineto{\pgfqpoint{3.222394in}{2.008657in}}%
\pgfpathlineto{\pgfqpoint{3.227408in}{2.011169in}}%
\pgfpathlineto{\pgfqpoint{3.229914in}{2.015734in}}%
\pgfpathlineto{\pgfqpoint{3.232421in}{2.015870in}}%
\pgfpathlineto{\pgfqpoint{3.234927in}{2.019278in}}%
\pgfpathlineto{\pgfqpoint{3.237434in}{2.025595in}}%
\pgfpathlineto{\pgfqpoint{3.242447in}{2.026405in}}%
\pgfpathlineto{\pgfqpoint{3.244954in}{2.030282in}}%
\pgfpathlineto{\pgfqpoint{3.249967in}{2.031196in}}%
\pgfpathlineto{\pgfqpoint{3.252474in}{2.035670in}}%
\pgfpathlineto{\pgfqpoint{3.254981in}{2.037621in}}%
\pgfpathlineto{\pgfqpoint{3.267514in}{2.041167in}}%
\pgfpathlineto{\pgfqpoint{3.270021in}{2.043494in}}%
\pgfpathlineto{\pgfqpoint{3.272527in}{2.043697in}}%
\pgfpathlineto{\pgfqpoint{3.280047in}{2.054642in}}%
\pgfpathlineto{\pgfqpoint{3.285061in}{2.055741in}}%
\pgfpathlineto{\pgfqpoint{3.287567in}{2.058520in}}%
\pgfpathlineto{\pgfqpoint{3.292581in}{2.059906in}}%
\pgfpathlineto{\pgfqpoint{3.300100in}{2.064571in}}%
\pgfpathlineto{\pgfqpoint{3.302607in}{2.076555in}}%
\pgfpathlineto{\pgfqpoint{3.305114in}{2.079751in}}%
\pgfpathlineto{\pgfqpoint{3.307620in}{2.080721in}}%
\pgfpathlineto{\pgfqpoint{3.312634in}{2.085615in}}%
\pgfpathlineto{\pgfqpoint{3.330180in}{2.094743in}}%
\pgfpathlineto{\pgfqpoint{3.332687in}{2.095541in}}%
\pgfpathlineto{\pgfqpoint{3.335194in}{2.099163in}}%
\pgfpathlineto{\pgfqpoint{3.340207in}{2.101097in}}%
\pgfpathlineto{\pgfqpoint{3.347727in}{2.103621in}}%
\pgfpathlineto{\pgfqpoint{3.350234in}{2.105860in}}%
\pgfpathlineto{\pgfqpoint{3.352740in}{2.105981in}}%
\pgfpathlineto{\pgfqpoint{3.357754in}{2.110869in}}%
\pgfpathlineto{\pgfqpoint{3.360260in}{2.113973in}}%
\pgfpathlineto{\pgfqpoint{3.365273in}{2.115068in}}%
\pgfpathlineto{\pgfqpoint{3.370287in}{2.118507in}}%
\pgfpathlineto{\pgfqpoint{3.372793in}{2.118687in}}%
\pgfpathlineto{\pgfqpoint{3.375300in}{2.120105in}}%
\pgfpathlineto{\pgfqpoint{3.377807in}{2.128108in}}%
\pgfpathlineto{\pgfqpoint{3.387833in}{2.132849in}}%
\pgfpathlineto{\pgfqpoint{3.390340in}{2.138169in}}%
\pgfpathlineto{\pgfqpoint{3.395353in}{2.138884in}}%
\pgfpathlineto{\pgfqpoint{3.407887in}{2.143449in}}%
\pgfpathlineto{\pgfqpoint{3.410393in}{2.145925in}}%
\pgfpathlineto{\pgfqpoint{3.420420in}{2.150312in}}%
\pgfpathlineto{\pgfqpoint{3.430447in}{2.162242in}}%
\pgfpathlineto{\pgfqpoint{3.432953in}{2.162515in}}%
\pgfpathlineto{\pgfqpoint{3.435460in}{2.164142in}}%
\pgfpathlineto{\pgfqpoint{3.437966in}{2.167340in}}%
\pgfpathlineto{\pgfqpoint{3.440473in}{2.167778in}}%
\pgfpathlineto{\pgfqpoint{3.442980in}{2.172682in}}%
\pgfpathlineto{\pgfqpoint{3.447993in}{2.174417in}}%
\pgfpathlineto{\pgfqpoint{3.453006in}{2.176336in}}%
\pgfpathlineto{\pgfqpoint{3.455513in}{2.179785in}}%
\pgfpathlineto{\pgfqpoint{3.458020in}{2.187186in}}%
\pgfpathlineto{\pgfqpoint{3.460526in}{2.187552in}}%
\pgfpathlineto{\pgfqpoint{3.465540in}{2.192257in}}%
\pgfpathlineto{\pgfqpoint{3.470553in}{2.194542in}}%
\pgfpathlineto{\pgfqpoint{3.473060in}{2.201091in}}%
\pgfpathlineto{\pgfqpoint{3.475566in}{2.201398in}}%
\pgfpathlineto{\pgfqpoint{3.480580in}{2.209073in}}%
\pgfpathlineto{\pgfqpoint{3.483086in}{2.210737in}}%
\pgfpathlineto{\pgfqpoint{3.493113in}{2.211772in}}%
\pgfpathlineto{\pgfqpoint{3.500633in}{2.213083in}}%
\pgfpathlineto{\pgfqpoint{3.505646in}{2.216249in}}%
\pgfpathlineto{\pgfqpoint{3.513166in}{2.220024in}}%
\pgfpathlineto{\pgfqpoint{3.520686in}{2.225628in}}%
\pgfpathlineto{\pgfqpoint{3.523193in}{2.225634in}}%
\pgfpathlineto{\pgfqpoint{3.528206in}{2.226976in}}%
\pgfpathlineto{\pgfqpoint{3.535726in}{2.228481in}}%
\pgfpathlineto{\pgfqpoint{3.553273in}{2.234551in}}%
\pgfpathlineto{\pgfqpoint{3.563299in}{2.236474in}}%
\pgfpathlineto{\pgfqpoint{3.570819in}{2.242831in}}%
\pgfpathlineto{\pgfqpoint{3.593379in}{2.249127in}}%
\pgfpathlineto{\pgfqpoint{3.595886in}{2.255440in}}%
\pgfpathlineto{\pgfqpoint{3.600899in}{2.256652in}}%
\pgfpathlineto{\pgfqpoint{3.603406in}{2.260554in}}%
\pgfpathlineto{\pgfqpoint{3.605912in}{2.260850in}}%
\pgfpathlineto{\pgfqpoint{3.610926in}{2.264901in}}%
\pgfpathlineto{\pgfqpoint{3.615939in}{2.265694in}}%
\pgfpathlineto{\pgfqpoint{3.618446in}{2.265731in}}%
\pgfpathlineto{\pgfqpoint{3.625966in}{2.272437in}}%
\pgfpathlineto{\pgfqpoint{3.630979in}{2.273268in}}%
\pgfpathlineto{\pgfqpoint{3.643512in}{2.278810in}}%
\pgfpathlineto{\pgfqpoint{3.656045in}{2.283576in}}%
\pgfpathlineto{\pgfqpoint{3.663565in}{2.286287in}}%
\pgfpathlineto{\pgfqpoint{3.671085in}{2.293580in}}%
\pgfpathlineto{\pgfqpoint{3.676099in}{2.294903in}}%
\pgfpathlineto{\pgfqpoint{3.678605in}{2.297889in}}%
\pgfpathlineto{\pgfqpoint{3.681112in}{2.298400in}}%
\pgfpathlineto{\pgfqpoint{3.686125in}{2.300965in}}%
\pgfpathlineto{\pgfqpoint{3.691139in}{2.301236in}}%
\pgfpathlineto{\pgfqpoint{3.693645in}{2.306192in}}%
\pgfpathlineto{\pgfqpoint{3.706178in}{2.311011in}}%
\pgfpathlineto{\pgfqpoint{3.713698in}{2.315629in}}%
\pgfpathlineto{\pgfqpoint{3.716205in}{2.316418in}}%
\pgfpathlineto{\pgfqpoint{3.721218in}{2.320238in}}%
\pgfpathlineto{\pgfqpoint{3.723725in}{2.320535in}}%
\pgfpathlineto{\pgfqpoint{3.726232in}{2.322870in}}%
\pgfpathlineto{\pgfqpoint{3.731245in}{2.324860in}}%
\pgfpathlineto{\pgfqpoint{3.736258in}{2.327145in}}%
\pgfpathlineto{\pgfqpoint{3.738765in}{2.333366in}}%
\pgfpathlineto{\pgfqpoint{3.746285in}{2.334355in}}%
\pgfpathlineto{\pgfqpoint{3.751298in}{2.336909in}}%
\pgfpathlineto{\pgfqpoint{3.756312in}{2.341039in}}%
\pgfpathlineto{\pgfqpoint{3.758818in}{2.341223in}}%
\pgfpathlineto{\pgfqpoint{3.761325in}{2.344390in}}%
\pgfpathlineto{\pgfqpoint{3.763832in}{2.344831in}}%
\pgfpathlineto{\pgfqpoint{3.771352in}{2.349879in}}%
\pgfpathlineto{\pgfqpoint{3.778871in}{2.352760in}}%
\pgfpathlineto{\pgfqpoint{3.788898in}{2.356686in}}%
\pgfpathlineto{\pgfqpoint{3.791405in}{2.357072in}}%
\pgfpathlineto{\pgfqpoint{3.798925in}{2.362450in}}%
\pgfpathlineto{\pgfqpoint{3.803938in}{2.364212in}}%
\pgfpathlineto{\pgfqpoint{3.806445in}{2.365915in}}%
\pgfpathlineto{\pgfqpoint{3.808951in}{2.369696in}}%
\pgfpathlineto{\pgfqpoint{3.811458in}{2.369805in}}%
\pgfpathlineto{\pgfqpoint{3.813965in}{2.371482in}}%
\pgfpathlineto{\pgfqpoint{3.818978in}{2.372087in}}%
\pgfpathlineto{\pgfqpoint{3.821485in}{2.374245in}}%
\pgfpathlineto{\pgfqpoint{3.823991in}{2.374609in}}%
\pgfpathlineto{\pgfqpoint{3.826498in}{2.378268in}}%
\pgfpathlineto{\pgfqpoint{3.831511in}{2.380413in}}%
\pgfpathlineto{\pgfqpoint{3.834018in}{2.381138in}}%
\pgfpathlineto{\pgfqpoint{3.836525in}{2.383452in}}%
\pgfpathlineto{\pgfqpoint{3.839031in}{2.383960in}}%
\pgfpathlineto{\pgfqpoint{3.841538in}{2.391255in}}%
\pgfpathlineto{\pgfqpoint{3.846551in}{2.396629in}}%
\pgfpathlineto{\pgfqpoint{3.849058in}{2.402563in}}%
\pgfpathlineto{\pgfqpoint{3.856578in}{2.404944in}}%
\pgfpathlineto{\pgfqpoint{3.859084in}{2.406267in}}%
\pgfpathlineto{\pgfqpoint{3.861591in}{2.406293in}}%
\pgfpathlineto{\pgfqpoint{3.866604in}{2.410470in}}%
\pgfpathlineto{\pgfqpoint{3.869111in}{2.412766in}}%
\pgfpathlineto{\pgfqpoint{3.871618in}{2.412814in}}%
\pgfpathlineto{\pgfqpoint{3.874124in}{2.417277in}}%
\pgfpathlineto{\pgfqpoint{3.884151in}{2.421950in}}%
\pgfpathlineto{\pgfqpoint{3.886658in}{2.424831in}}%
\pgfpathlineto{\pgfqpoint{3.894178in}{2.426428in}}%
\pgfpathlineto{\pgfqpoint{3.896684in}{2.430514in}}%
\pgfpathlineto{\pgfqpoint{3.901698in}{2.433145in}}%
\pgfpathlineto{\pgfqpoint{3.904204in}{2.434023in}}%
\pgfpathlineto{\pgfqpoint{3.906711in}{2.437643in}}%
\pgfpathlineto{\pgfqpoint{3.909217in}{2.438375in}}%
\pgfpathlineto{\pgfqpoint{3.914231in}{2.445386in}}%
\pgfpathlineto{\pgfqpoint{3.916737in}{2.449925in}}%
\pgfpathlineto{\pgfqpoint{3.921751in}{2.450073in}}%
\pgfpathlineto{\pgfqpoint{3.926764in}{2.453922in}}%
\pgfpathlineto{\pgfqpoint{3.929271in}{2.454193in}}%
\pgfpathlineto{\pgfqpoint{3.931777in}{2.460276in}}%
\pgfpathlineto{\pgfqpoint{3.934284in}{2.460712in}}%
\pgfpathlineto{\pgfqpoint{3.936791in}{2.463148in}}%
\pgfpathlineto{\pgfqpoint{3.941804in}{2.463527in}}%
\pgfpathlineto{\pgfqpoint{3.944311in}{2.472772in}}%
\pgfpathlineto{\pgfqpoint{3.956844in}{2.478313in}}%
\pgfpathlineto{\pgfqpoint{3.959351in}{2.481690in}}%
\pgfpathlineto{\pgfqpoint{3.964364in}{2.492142in}}%
\pgfpathlineto{\pgfqpoint{3.966871in}{2.492603in}}%
\pgfpathlineto{\pgfqpoint{3.971884in}{2.495255in}}%
\pgfpathlineto{\pgfqpoint{3.976897in}{2.496679in}}%
\pgfpathlineto{\pgfqpoint{3.994444in}{2.511888in}}%
\pgfpathlineto{\pgfqpoint{3.996950in}{2.511980in}}%
\pgfpathlineto{\pgfqpoint{3.999457in}{2.521212in}}%
\pgfpathlineto{\pgfqpoint{4.009484in}{2.525688in}}%
\pgfpathlineto{\pgfqpoint{4.011990in}{2.530462in}}%
\pgfpathlineto{\pgfqpoint{4.027030in}{2.533758in}}%
\pgfpathlineto{\pgfqpoint{4.037057in}{2.547697in}}%
\pgfpathlineto{\pgfqpoint{4.039564in}{2.557488in}}%
\pgfpathlineto{\pgfqpoint{4.042070in}{2.557509in}}%
\pgfpathlineto{\pgfqpoint{4.044577in}{2.560587in}}%
\pgfpathlineto{\pgfqpoint{4.049590in}{2.561909in}}%
\pgfpathlineto{\pgfqpoint{4.052097in}{2.567385in}}%
\pgfpathlineto{\pgfqpoint{4.062123in}{2.569935in}}%
\pgfpathlineto{\pgfqpoint{4.064630in}{2.575189in}}%
\pgfpathlineto{\pgfqpoint{4.067137in}{2.575453in}}%
\pgfpathlineto{\pgfqpoint{4.072150in}{2.580839in}}%
\pgfpathlineto{\pgfqpoint{4.074657in}{2.581646in}}%
\pgfpathlineto{\pgfqpoint{4.077163in}{2.585816in}}%
\pgfpathlineto{\pgfqpoint{4.082177in}{2.588582in}}%
\pgfpathlineto{\pgfqpoint{4.084683in}{2.591596in}}%
\pgfpathlineto{\pgfqpoint{4.087190in}{2.591912in}}%
\pgfpathlineto{\pgfqpoint{4.089697in}{2.603844in}}%
\pgfpathlineto{\pgfqpoint{4.092203in}{2.604906in}}%
\pgfpathlineto{\pgfqpoint{4.097217in}{2.611295in}}%
\pgfpathlineto{\pgfqpoint{4.102230in}{2.621147in}}%
\pgfpathlineto{\pgfqpoint{4.104737in}{2.621288in}}%
\pgfpathlineto{\pgfqpoint{4.107243in}{2.631346in}}%
\pgfpathlineto{\pgfqpoint{4.112257in}{2.634943in}}%
\pgfpathlineto{\pgfqpoint{4.114763in}{2.647212in}}%
\pgfpathlineto{\pgfqpoint{4.117270in}{2.652568in}}%
\pgfpathlineto{\pgfqpoint{4.119776in}{2.653214in}}%
\pgfpathlineto{\pgfqpoint{4.122283in}{2.657633in}}%
\pgfpathlineto{\pgfqpoint{4.124790in}{2.659195in}}%
\pgfpathlineto{\pgfqpoint{4.127296in}{2.659251in}}%
\pgfpathlineto{\pgfqpoint{4.129803in}{2.661220in}}%
\pgfpathlineto{\pgfqpoint{4.132310in}{2.672107in}}%
\pgfpathlineto{\pgfqpoint{4.134816in}{2.674829in}}%
\pgfpathlineto{\pgfqpoint{4.137323in}{2.685981in}}%
\pgfpathlineto{\pgfqpoint{4.139830in}{2.687545in}}%
\pgfpathlineto{\pgfqpoint{4.142336in}{2.694532in}}%
\pgfpathlineto{\pgfqpoint{4.149856in}{2.696396in}}%
\pgfpathlineto{\pgfqpoint{4.154870in}{2.699769in}}%
\pgfpathlineto{\pgfqpoint{4.159883in}{2.718195in}}%
\pgfpathlineto{\pgfqpoint{4.162390in}{2.719611in}}%
\pgfpathlineto{\pgfqpoint{4.164896in}{2.723140in}}%
\pgfpathlineto{\pgfqpoint{4.167403in}{2.730890in}}%
\pgfpathlineto{\pgfqpoint{4.172416in}{2.732795in}}%
\pgfpathlineto{\pgfqpoint{4.174923in}{2.741563in}}%
\pgfpathlineto{\pgfqpoint{4.179936in}{2.746557in}}%
\pgfpathlineto{\pgfqpoint{4.182443in}{2.756168in}}%
\pgfpathlineto{\pgfqpoint{4.184949in}{2.756476in}}%
\pgfpathlineto{\pgfqpoint{4.187456in}{2.761521in}}%
\pgfpathlineto{\pgfqpoint{4.189963in}{2.762386in}}%
\pgfpathlineto{\pgfqpoint{4.192469in}{2.767765in}}%
\pgfpathlineto{\pgfqpoint{4.207509in}{2.780016in}}%
\pgfpathlineto{\pgfqpoint{4.212523in}{2.780554in}}%
\pgfpathlineto{\pgfqpoint{4.215029in}{2.785910in}}%
\pgfpathlineto{\pgfqpoint{4.225056in}{2.793559in}}%
\pgfpathlineto{\pgfqpoint{4.230069in}{2.806723in}}%
\pgfpathlineto{\pgfqpoint{4.235083in}{2.809742in}}%
\pgfpathlineto{\pgfqpoint{4.237589in}{2.822411in}}%
\pgfpathlineto{\pgfqpoint{4.240096in}{2.823260in}}%
\pgfpathlineto{\pgfqpoint{4.242603in}{2.828247in}}%
\pgfpathlineto{\pgfqpoint{4.245109in}{2.828744in}}%
\pgfpathlineto{\pgfqpoint{4.247616in}{2.836311in}}%
\pgfpathlineto{\pgfqpoint{4.250122in}{2.850940in}}%
\pgfpathlineto{\pgfqpoint{4.255136in}{2.856043in}}%
\pgfpathlineto{\pgfqpoint{4.257642in}{2.857356in}}%
\pgfpathlineto{\pgfqpoint{4.260149in}{2.860134in}}%
\pgfpathlineto{\pgfqpoint{4.262656in}{2.866101in}}%
\pgfpathlineto{\pgfqpoint{4.267669in}{2.870956in}}%
\pgfpathlineto{\pgfqpoint{4.270176in}{2.874213in}}%
\pgfpathlineto{\pgfqpoint{4.275189in}{2.887961in}}%
\pgfpathlineto{\pgfqpoint{4.277696in}{2.888318in}}%
\pgfpathlineto{\pgfqpoint{4.280202in}{2.895440in}}%
\pgfpathlineto{\pgfqpoint{4.282709in}{2.898096in}}%
\pgfpathlineto{\pgfqpoint{4.285216in}{2.904151in}}%
\pgfpathlineto{\pgfqpoint{4.287722in}{2.905275in}}%
\pgfpathlineto{\pgfqpoint{4.287722in}{2.905275in}}%
\pgfusepath{stroke}%
\end{pgfscope}%
\begin{pgfscope}%
\pgfpathrectangle{\pgfqpoint{0.708220in}{0.535823in}}{\pgfqpoint{5.013309in}{2.369453in}}%
\pgfusepath{clip}%
\pgfsetbuttcap%
\pgfsetroundjoin%
\pgfsetlinewidth{1.003750pt}%
\definecolor{currentstroke}{rgb}{1.000000,0.647059,0.000000}%
\pgfsetstrokecolor{currentstroke}%
\pgfsetdash{{3.700000pt}{1.600000pt}}{0.000000pt}%
\pgfpathmoveto{\pgfqpoint{0.708220in}{1.429487in}}%
\pgfpathlineto{\pgfqpoint{0.710727in}{1.491428in}}%
\pgfpathlineto{\pgfqpoint{0.713233in}{1.507782in}}%
\pgfpathlineto{\pgfqpoint{0.715740in}{1.531610in}}%
\pgfpathlineto{\pgfqpoint{0.718246in}{1.535054in}}%
\pgfpathlineto{\pgfqpoint{0.720753in}{1.536256in}}%
\pgfpathlineto{\pgfqpoint{0.723260in}{1.539233in}}%
\pgfpathlineto{\pgfqpoint{0.725766in}{1.546604in}}%
\pgfpathlineto{\pgfqpoint{0.728273in}{1.548180in}}%
\pgfpathlineto{\pgfqpoint{0.738300in}{1.584463in}}%
\pgfpathlineto{\pgfqpoint{0.740806in}{1.584911in}}%
\pgfpathlineto{\pgfqpoint{0.743313in}{1.601110in}}%
\pgfpathlineto{\pgfqpoint{0.748326in}{1.605560in}}%
\pgfpathlineto{\pgfqpoint{0.750833in}{1.605715in}}%
\pgfpathlineto{\pgfqpoint{0.753340in}{1.608049in}}%
\pgfpathlineto{\pgfqpoint{0.755846in}{1.612764in}}%
\pgfpathlineto{\pgfqpoint{0.760860in}{1.627208in}}%
\pgfpathlineto{\pgfqpoint{0.765873in}{1.634400in}}%
\pgfpathlineto{\pgfqpoint{0.768380in}{1.637992in}}%
\pgfpathlineto{\pgfqpoint{0.773393in}{1.640040in}}%
\pgfpathlineto{\pgfqpoint{0.775900in}{1.641615in}}%
\pgfpathlineto{\pgfqpoint{0.778406in}{1.650347in}}%
\pgfpathlineto{\pgfqpoint{0.788433in}{1.655714in}}%
\pgfpathlineto{\pgfqpoint{0.793446in}{1.655825in}}%
\pgfpathlineto{\pgfqpoint{0.803473in}{1.661857in}}%
\pgfpathlineto{\pgfqpoint{0.805979in}{1.661927in}}%
\pgfpathlineto{\pgfqpoint{0.808486in}{1.663529in}}%
\pgfpathlineto{\pgfqpoint{0.816006in}{1.664486in}}%
\pgfpathlineto{\pgfqpoint{0.826033in}{1.667195in}}%
\pgfpathlineto{\pgfqpoint{0.836059in}{1.669320in}}%
\pgfpathlineto{\pgfqpoint{0.851099in}{1.674617in}}%
\pgfpathlineto{\pgfqpoint{0.863632in}{1.676644in}}%
\pgfpathlineto{\pgfqpoint{0.868646in}{1.681209in}}%
\pgfpathlineto{\pgfqpoint{0.913766in}{1.688801in}}%
\pgfpathlineto{\pgfqpoint{0.921285in}{1.689159in}}%
\pgfpathlineto{\pgfqpoint{0.926299in}{1.691116in}}%
\pgfpathlineto{\pgfqpoint{0.941339in}{1.693288in}}%
\pgfpathlineto{\pgfqpoint{0.946352in}{1.695351in}}%
\pgfpathlineto{\pgfqpoint{0.953872in}{1.696397in}}%
\pgfpathlineto{\pgfqpoint{0.961392in}{1.698120in}}%
\pgfpathlineto{\pgfqpoint{0.968912in}{1.699074in}}%
\pgfpathlineto{\pgfqpoint{0.981445in}{1.702621in}}%
\pgfpathlineto{\pgfqpoint{0.991472in}{1.704047in}}%
\pgfpathlineto{\pgfqpoint{0.998992in}{1.704567in}}%
\pgfpathlineto{\pgfqpoint{1.004005in}{1.706138in}}%
\pgfpathlineto{\pgfqpoint{1.036592in}{1.709403in}}%
\pgfpathlineto{\pgfqpoint{1.039098in}{1.710903in}}%
\pgfpathlineto{\pgfqpoint{1.066671in}{1.713032in}}%
\pgfpathlineto{\pgfqpoint{1.069178in}{1.714572in}}%
\pgfpathlineto{\pgfqpoint{1.096751in}{1.716999in}}%
\pgfpathlineto{\pgfqpoint{1.111791in}{1.718283in}}%
\pgfpathlineto{\pgfqpoint{1.124324in}{1.720052in}}%
\pgfpathlineto{\pgfqpoint{1.131844in}{1.721508in}}%
\pgfpathlineto{\pgfqpoint{1.149391in}{1.723503in}}%
\pgfpathlineto{\pgfqpoint{1.164431in}{1.724988in}}%
\pgfpathlineto{\pgfqpoint{1.186991in}{1.730136in}}%
\pgfpathlineto{\pgfqpoint{1.194511in}{1.730900in}}%
\pgfpathlineto{\pgfqpoint{1.197017in}{1.732194in}}%
\pgfpathlineto{\pgfqpoint{1.202031in}{1.732464in}}%
\pgfpathlineto{\pgfqpoint{1.209551in}{1.734698in}}%
\pgfpathlineto{\pgfqpoint{1.282244in}{1.743654in}}%
\pgfpathlineto{\pgfqpoint{1.287257in}{1.745659in}}%
\pgfpathlineto{\pgfqpoint{1.312324in}{1.748423in}}%
\pgfpathlineto{\pgfqpoint{1.349923in}{1.751183in}}%
\pgfpathlineto{\pgfqpoint{1.357443in}{1.752239in}}%
\pgfpathlineto{\pgfqpoint{1.402563in}{1.759452in}}%
\pgfpathlineto{\pgfqpoint{1.412590in}{1.760443in}}%
\pgfpathlineto{\pgfqpoint{1.420110in}{1.761250in}}%
\pgfpathlineto{\pgfqpoint{1.435150in}{1.762287in}}%
\pgfpathlineto{\pgfqpoint{1.467736in}{1.766502in}}%
\pgfpathlineto{\pgfqpoint{1.475256in}{1.767564in}}%
\pgfpathlineto{\pgfqpoint{1.477763in}{1.767907in}}%
\pgfpathlineto{\pgfqpoint{1.480269in}{1.769542in}}%
\pgfpathlineto{\pgfqpoint{1.490296in}{1.770719in}}%
\pgfpathlineto{\pgfqpoint{1.510349in}{1.773181in}}%
\pgfpathlineto{\pgfqpoint{1.520376in}{1.774863in}}%
\pgfpathlineto{\pgfqpoint{1.535416in}{1.776331in}}%
\pgfpathlineto{\pgfqpoint{1.557976in}{1.777440in}}%
\pgfpathlineto{\pgfqpoint{1.573016in}{1.780268in}}%
\pgfpathlineto{\pgfqpoint{1.600589in}{1.782052in}}%
\pgfpathlineto{\pgfqpoint{1.613122in}{1.782790in}}%
\pgfpathlineto{\pgfqpoint{1.630669in}{1.783824in}}%
\pgfpathlineto{\pgfqpoint{1.635682in}{1.785120in}}%
\pgfpathlineto{\pgfqpoint{1.650722in}{1.786487in}}%
\pgfpathlineto{\pgfqpoint{1.665762in}{1.788524in}}%
\pgfpathlineto{\pgfqpoint{1.670775in}{1.789905in}}%
\pgfpathlineto{\pgfqpoint{1.678295in}{1.791098in}}%
\pgfpathlineto{\pgfqpoint{1.690828in}{1.792742in}}%
\pgfpathlineto{\pgfqpoint{1.710882in}{1.793798in}}%
\pgfpathlineto{\pgfqpoint{1.730935in}{1.797042in}}%
\pgfpathlineto{\pgfqpoint{1.748481in}{1.798206in}}%
\pgfpathlineto{\pgfqpoint{1.753495in}{1.799497in}}%
\pgfpathlineto{\pgfqpoint{1.788588in}{1.800526in}}%
\pgfpathlineto{\pgfqpoint{1.831201in}{1.808169in}}%
\pgfpathlineto{\pgfqpoint{1.838721in}{1.808814in}}%
\pgfpathlineto{\pgfqpoint{1.871307in}{1.815271in}}%
\pgfpathlineto{\pgfqpoint{1.886347in}{1.815907in}}%
\pgfpathlineto{\pgfqpoint{1.891361in}{1.817523in}}%
\pgfpathlineto{\pgfqpoint{1.901387in}{1.819231in}}%
\pgfpathlineto{\pgfqpoint{1.911414in}{1.820535in}}%
\pgfpathlineto{\pgfqpoint{1.936481in}{1.822319in}}%
\pgfpathlineto{\pgfqpoint{1.949014in}{1.825304in}}%
\pgfpathlineto{\pgfqpoint{1.954027in}{1.826478in}}%
\pgfpathlineto{\pgfqpoint{1.966560in}{1.827698in}}%
\pgfpathlineto{\pgfqpoint{1.981600in}{1.830840in}}%
\pgfpathlineto{\pgfqpoint{1.986614in}{1.831534in}}%
\pgfpathlineto{\pgfqpoint{1.994134in}{1.833742in}}%
\pgfpathlineto{\pgfqpoint{2.011680in}{1.838542in}}%
\pgfpathlineto{\pgfqpoint{2.024213in}{1.839710in}}%
\pgfpathlineto{\pgfqpoint{2.026720in}{1.841856in}}%
\pgfpathlineto{\pgfqpoint{2.039253in}{1.842990in}}%
\pgfpathlineto{\pgfqpoint{2.044267in}{1.844877in}}%
\pgfpathlineto{\pgfqpoint{2.061813in}{1.846145in}}%
\pgfpathlineto{\pgfqpoint{2.106933in}{1.853887in}}%
\pgfpathlineto{\pgfqpoint{2.114453in}{1.856910in}}%
\pgfpathlineto{\pgfqpoint{2.121973in}{1.857633in}}%
\pgfpathlineto{\pgfqpoint{2.126986in}{1.859651in}}%
\pgfpathlineto{\pgfqpoint{2.134506in}{1.862030in}}%
\pgfpathlineto{\pgfqpoint{2.144533in}{1.862706in}}%
\pgfpathlineto{\pgfqpoint{2.149546in}{1.863727in}}%
\pgfpathlineto{\pgfqpoint{2.152053in}{1.863759in}}%
\pgfpathlineto{\pgfqpoint{2.154559in}{1.866133in}}%
\pgfpathlineto{\pgfqpoint{2.184639in}{1.869788in}}%
\pgfpathlineto{\pgfqpoint{2.194666in}{1.872253in}}%
\pgfpathlineto{\pgfqpoint{2.209706in}{1.873650in}}%
\pgfpathlineto{\pgfqpoint{2.224746in}{1.876033in}}%
\pgfpathlineto{\pgfqpoint{2.227252in}{1.878060in}}%
\pgfpathlineto{\pgfqpoint{2.237279in}{1.879668in}}%
\pgfpathlineto{\pgfqpoint{2.249812in}{1.882486in}}%
\pgfpathlineto{\pgfqpoint{2.267359in}{1.883403in}}%
\pgfpathlineto{\pgfqpoint{2.272372in}{1.884566in}}%
\pgfpathlineto{\pgfqpoint{2.294932in}{1.887909in}}%
\pgfpathlineto{\pgfqpoint{2.299945in}{1.888873in}}%
\pgfpathlineto{\pgfqpoint{2.302452in}{1.889020in}}%
\pgfpathlineto{\pgfqpoint{2.307465in}{1.890429in}}%
\pgfpathlineto{\pgfqpoint{2.314985in}{1.891649in}}%
\pgfpathlineto{\pgfqpoint{2.317492in}{1.892146in}}%
\pgfpathlineto{\pgfqpoint{2.319999in}{1.894394in}}%
\pgfpathlineto{\pgfqpoint{2.360105in}{1.900778in}}%
\pgfpathlineto{\pgfqpoint{2.370132in}{1.902338in}}%
\pgfpathlineto{\pgfqpoint{2.382665in}{1.904782in}}%
\pgfpathlineto{\pgfqpoint{2.395198in}{1.906394in}}%
\pgfpathlineto{\pgfqpoint{2.422771in}{1.912391in}}%
\pgfpathlineto{\pgfqpoint{2.427785in}{1.912812in}}%
\pgfpathlineto{\pgfqpoint{2.435305in}{1.916835in}}%
\pgfpathlineto{\pgfqpoint{2.465385in}{1.921115in}}%
\pgfpathlineto{\pgfqpoint{2.472905in}{1.924244in}}%
\pgfpathlineto{\pgfqpoint{2.477918in}{1.925570in}}%
\pgfpathlineto{\pgfqpoint{2.502984in}{1.930508in}}%
\pgfpathlineto{\pgfqpoint{2.510504in}{1.932984in}}%
\pgfpathlineto{\pgfqpoint{2.513011in}{1.933480in}}%
\pgfpathlineto{\pgfqpoint{2.515518in}{1.935315in}}%
\pgfpathlineto{\pgfqpoint{2.520531in}{1.936793in}}%
\pgfpathlineto{\pgfqpoint{2.530558in}{1.937396in}}%
\pgfpathlineto{\pgfqpoint{2.533064in}{1.939364in}}%
\pgfpathlineto{\pgfqpoint{2.545598in}{1.942605in}}%
\pgfpathlineto{\pgfqpoint{2.550611in}{1.944651in}}%
\pgfpathlineto{\pgfqpoint{2.555624in}{1.945726in}}%
\pgfpathlineto{\pgfqpoint{2.560637in}{1.947698in}}%
\pgfpathlineto{\pgfqpoint{2.583197in}{1.952311in}}%
\pgfpathlineto{\pgfqpoint{2.588211in}{1.954395in}}%
\pgfpathlineto{\pgfqpoint{2.590717in}{1.954961in}}%
\pgfpathlineto{\pgfqpoint{2.593224in}{1.957765in}}%
\pgfpathlineto{\pgfqpoint{2.605757in}{1.959355in}}%
\pgfpathlineto{\pgfqpoint{2.618290in}{1.962367in}}%
\pgfpathlineto{\pgfqpoint{2.620797in}{1.963215in}}%
\pgfpathlineto{\pgfqpoint{2.623304in}{1.965533in}}%
\pgfpathlineto{\pgfqpoint{2.630824in}{1.966630in}}%
\pgfpathlineto{\pgfqpoint{2.633330in}{1.968012in}}%
\pgfpathlineto{\pgfqpoint{2.643357in}{1.968885in}}%
\pgfpathlineto{\pgfqpoint{2.648370in}{1.971292in}}%
\pgfpathlineto{\pgfqpoint{2.655890in}{1.971982in}}%
\pgfpathlineto{\pgfqpoint{2.660904in}{1.972766in}}%
\pgfpathlineto{\pgfqpoint{2.668424in}{1.973440in}}%
\pgfpathlineto{\pgfqpoint{2.670930in}{1.975767in}}%
\pgfpathlineto{\pgfqpoint{2.698503in}{1.980796in}}%
\pgfpathlineto{\pgfqpoint{2.711037in}{1.985388in}}%
\pgfpathlineto{\pgfqpoint{2.713543in}{1.985399in}}%
\pgfpathlineto{\pgfqpoint{2.716050in}{1.987131in}}%
\pgfpathlineto{\pgfqpoint{2.718557in}{1.987252in}}%
\pgfpathlineto{\pgfqpoint{2.723570in}{1.988871in}}%
\pgfpathlineto{\pgfqpoint{2.731090in}{1.991061in}}%
\pgfpathlineto{\pgfqpoint{2.733597in}{1.992096in}}%
\pgfpathlineto{\pgfqpoint{2.736103in}{1.995315in}}%
\pgfpathlineto{\pgfqpoint{2.738610in}{1.995946in}}%
\pgfpathlineto{\pgfqpoint{2.741117in}{1.997819in}}%
\pgfpathlineto{\pgfqpoint{2.748637in}{1.998841in}}%
\pgfpathlineto{\pgfqpoint{2.753650in}{2.001360in}}%
\pgfpathlineto{\pgfqpoint{2.761170in}{2.003109in}}%
\pgfpathlineto{\pgfqpoint{2.763676in}{2.005719in}}%
\pgfpathlineto{\pgfqpoint{2.778716in}{2.009484in}}%
\pgfpathlineto{\pgfqpoint{2.786236in}{2.010204in}}%
\pgfpathlineto{\pgfqpoint{2.793756in}{2.012706in}}%
\pgfpathlineto{\pgfqpoint{2.798770in}{2.014425in}}%
\pgfpathlineto{\pgfqpoint{2.803783in}{2.016489in}}%
\pgfpathlineto{\pgfqpoint{2.808796in}{2.017690in}}%
\pgfpathlineto{\pgfqpoint{2.816316in}{2.018874in}}%
\pgfpathlineto{\pgfqpoint{2.823836in}{2.020946in}}%
\pgfpathlineto{\pgfqpoint{2.828849in}{2.026852in}}%
\pgfpathlineto{\pgfqpoint{2.841383in}{2.028036in}}%
\pgfpathlineto{\pgfqpoint{2.843889in}{2.029635in}}%
\pgfpathlineto{\pgfqpoint{2.848903in}{2.029912in}}%
\pgfpathlineto{\pgfqpoint{2.861436in}{2.033681in}}%
\pgfpathlineto{\pgfqpoint{2.871463in}{2.034782in}}%
\pgfpathlineto{\pgfqpoint{2.876476in}{2.037221in}}%
\pgfpathlineto{\pgfqpoint{2.886503in}{2.041425in}}%
\pgfpathlineto{\pgfqpoint{2.896529in}{2.044692in}}%
\pgfpathlineto{\pgfqpoint{2.901542in}{2.045007in}}%
\pgfpathlineto{\pgfqpoint{2.904049in}{2.046496in}}%
\pgfpathlineto{\pgfqpoint{2.911569in}{2.047350in}}%
\pgfpathlineto{\pgfqpoint{2.916582in}{2.048332in}}%
\pgfpathlineto{\pgfqpoint{2.924102in}{2.049112in}}%
\pgfpathlineto{\pgfqpoint{2.926609in}{2.049127in}}%
\pgfpathlineto{\pgfqpoint{2.929116in}{2.052891in}}%
\pgfpathlineto{\pgfqpoint{2.941649in}{2.054653in}}%
\pgfpathlineto{\pgfqpoint{2.949169in}{2.058781in}}%
\pgfpathlineto{\pgfqpoint{2.951676in}{2.059010in}}%
\pgfpathlineto{\pgfqpoint{2.956689in}{2.060778in}}%
\pgfpathlineto{\pgfqpoint{2.959195in}{2.060860in}}%
\pgfpathlineto{\pgfqpoint{2.961702in}{2.062158in}}%
\pgfpathlineto{\pgfqpoint{2.971729in}{2.062693in}}%
\pgfpathlineto{\pgfqpoint{2.974235in}{2.066461in}}%
\pgfpathlineto{\pgfqpoint{2.989275in}{2.071082in}}%
\pgfpathlineto{\pgfqpoint{2.991782in}{2.074160in}}%
\pgfpathlineto{\pgfqpoint{2.994289in}{2.075253in}}%
\pgfpathlineto{\pgfqpoint{2.996795in}{2.077994in}}%
\pgfpathlineto{\pgfqpoint{3.006822in}{2.079151in}}%
\pgfpathlineto{\pgfqpoint{3.011835in}{2.082183in}}%
\pgfpathlineto{\pgfqpoint{3.014342in}{2.082254in}}%
\pgfpathlineto{\pgfqpoint{3.019355in}{2.084804in}}%
\pgfpathlineto{\pgfqpoint{3.024369in}{2.085933in}}%
\pgfpathlineto{\pgfqpoint{3.026875in}{2.086220in}}%
\pgfpathlineto{\pgfqpoint{3.029382in}{2.088619in}}%
\pgfpathlineto{\pgfqpoint{3.036902in}{2.090187in}}%
\pgfpathlineto{\pgfqpoint{3.051942in}{2.093986in}}%
\pgfpathlineto{\pgfqpoint{3.084528in}{2.101583in}}%
\pgfpathlineto{\pgfqpoint{3.087035in}{2.104023in}}%
\pgfpathlineto{\pgfqpoint{3.097061in}{2.104673in}}%
\pgfpathlineto{\pgfqpoint{3.102075in}{2.108331in}}%
\pgfpathlineto{\pgfqpoint{3.114608in}{2.110853in}}%
\pgfpathlineto{\pgfqpoint{3.117115in}{2.112919in}}%
\pgfpathlineto{\pgfqpoint{3.119621in}{2.113293in}}%
\pgfpathlineto{\pgfqpoint{3.122128in}{2.115675in}}%
\pgfpathlineto{\pgfqpoint{3.127141in}{2.115805in}}%
\pgfpathlineto{\pgfqpoint{3.129648in}{2.123605in}}%
\pgfpathlineto{\pgfqpoint{3.137168in}{2.127134in}}%
\pgfpathlineto{\pgfqpoint{3.142181in}{2.127254in}}%
\pgfpathlineto{\pgfqpoint{3.147195in}{2.129205in}}%
\pgfpathlineto{\pgfqpoint{3.157221in}{2.129939in}}%
\pgfpathlineto{\pgfqpoint{3.162234in}{2.134044in}}%
\pgfpathlineto{\pgfqpoint{3.169754in}{2.135267in}}%
\pgfpathlineto{\pgfqpoint{3.184794in}{2.145534in}}%
\pgfpathlineto{\pgfqpoint{3.187301in}{2.146968in}}%
\pgfpathlineto{\pgfqpoint{3.189808in}{2.151923in}}%
\pgfpathlineto{\pgfqpoint{3.194821in}{2.153590in}}%
\pgfpathlineto{\pgfqpoint{3.197328in}{2.155392in}}%
\pgfpathlineto{\pgfqpoint{3.199834in}{2.155620in}}%
\pgfpathlineto{\pgfqpoint{3.204848in}{2.158177in}}%
\pgfpathlineto{\pgfqpoint{3.217381in}{2.159138in}}%
\pgfpathlineto{\pgfqpoint{3.234927in}{2.165827in}}%
\pgfpathlineto{\pgfqpoint{3.244954in}{2.167774in}}%
\pgfpathlineto{\pgfqpoint{3.247461in}{2.169935in}}%
\pgfpathlineto{\pgfqpoint{3.257487in}{2.172787in}}%
\pgfpathlineto{\pgfqpoint{3.259994in}{2.174818in}}%
\pgfpathlineto{\pgfqpoint{3.262501in}{2.174912in}}%
\pgfpathlineto{\pgfqpoint{3.267514in}{2.177635in}}%
\pgfpathlineto{\pgfqpoint{3.280047in}{2.179833in}}%
\pgfpathlineto{\pgfqpoint{3.287567in}{2.182843in}}%
\pgfpathlineto{\pgfqpoint{3.292581in}{2.184045in}}%
\pgfpathlineto{\pgfqpoint{3.297594in}{2.187662in}}%
\pgfpathlineto{\pgfqpoint{3.312634in}{2.190887in}}%
\pgfpathlineto{\pgfqpoint{3.317647in}{2.195712in}}%
\pgfpathlineto{\pgfqpoint{3.320154in}{2.195939in}}%
\pgfpathlineto{\pgfqpoint{3.327674in}{2.206897in}}%
\pgfpathlineto{\pgfqpoint{3.332687in}{2.209010in}}%
\pgfpathlineto{\pgfqpoint{3.340207in}{2.209610in}}%
\pgfpathlineto{\pgfqpoint{3.342714in}{2.211836in}}%
\pgfpathlineto{\pgfqpoint{3.345220in}{2.212478in}}%
\pgfpathlineto{\pgfqpoint{3.350234in}{2.216563in}}%
\pgfpathlineto{\pgfqpoint{3.352740in}{2.216645in}}%
\pgfpathlineto{\pgfqpoint{3.355247in}{2.219216in}}%
\pgfpathlineto{\pgfqpoint{3.362767in}{2.221206in}}%
\pgfpathlineto{\pgfqpoint{3.367780in}{2.222367in}}%
\pgfpathlineto{\pgfqpoint{3.370287in}{2.224630in}}%
\pgfpathlineto{\pgfqpoint{3.372793in}{2.224878in}}%
\pgfpathlineto{\pgfqpoint{3.377807in}{2.227363in}}%
\pgfpathlineto{\pgfqpoint{3.382820in}{2.229615in}}%
\pgfpathlineto{\pgfqpoint{3.400367in}{2.236969in}}%
\pgfpathlineto{\pgfqpoint{3.412900in}{2.240093in}}%
\pgfpathlineto{\pgfqpoint{3.417913in}{2.242416in}}%
\pgfpathlineto{\pgfqpoint{3.420420in}{2.242475in}}%
\pgfpathlineto{\pgfqpoint{3.425433in}{2.244179in}}%
\pgfpathlineto{\pgfqpoint{3.427940in}{2.244223in}}%
\pgfpathlineto{\pgfqpoint{3.430447in}{2.250836in}}%
\pgfpathlineto{\pgfqpoint{3.432953in}{2.253793in}}%
\pgfpathlineto{\pgfqpoint{3.435460in}{2.254454in}}%
\pgfpathlineto{\pgfqpoint{3.437966in}{2.258838in}}%
\pgfpathlineto{\pgfqpoint{3.440473in}{2.258936in}}%
\pgfpathlineto{\pgfqpoint{3.442980in}{2.262448in}}%
\pgfpathlineto{\pgfqpoint{3.447993in}{2.263927in}}%
\pgfpathlineto{\pgfqpoint{3.453006in}{2.265544in}}%
\pgfpathlineto{\pgfqpoint{3.455513in}{2.267924in}}%
\pgfpathlineto{\pgfqpoint{3.470553in}{2.271431in}}%
\pgfpathlineto{\pgfqpoint{3.473060in}{2.274384in}}%
\pgfpathlineto{\pgfqpoint{3.480580in}{2.278117in}}%
\pgfpathlineto{\pgfqpoint{3.483086in}{2.281057in}}%
\pgfpathlineto{\pgfqpoint{3.488100in}{2.282266in}}%
\pgfpathlineto{\pgfqpoint{3.490606in}{2.284611in}}%
\pgfpathlineto{\pgfqpoint{3.495620in}{2.287022in}}%
\pgfpathlineto{\pgfqpoint{3.500633in}{2.289273in}}%
\pgfpathlineto{\pgfqpoint{3.505646in}{2.290585in}}%
\pgfpathlineto{\pgfqpoint{3.510659in}{2.292518in}}%
\pgfpathlineto{\pgfqpoint{3.530713in}{2.300249in}}%
\pgfpathlineto{\pgfqpoint{3.535726in}{2.302302in}}%
\pgfpathlineto{\pgfqpoint{3.543246in}{2.303461in}}%
\pgfpathlineto{\pgfqpoint{3.548259in}{2.311391in}}%
\pgfpathlineto{\pgfqpoint{3.563299in}{2.314175in}}%
\pgfpathlineto{\pgfqpoint{3.565806in}{2.321126in}}%
\pgfpathlineto{\pgfqpoint{3.570819in}{2.322016in}}%
\pgfpathlineto{\pgfqpoint{3.573326in}{2.325207in}}%
\pgfpathlineto{\pgfqpoint{3.578339in}{2.326258in}}%
\pgfpathlineto{\pgfqpoint{3.580846in}{2.329540in}}%
\pgfpathlineto{\pgfqpoint{3.588366in}{2.332448in}}%
\pgfpathlineto{\pgfqpoint{3.593379in}{2.333321in}}%
\pgfpathlineto{\pgfqpoint{3.595886in}{2.335589in}}%
\pgfpathlineto{\pgfqpoint{3.605912in}{2.336634in}}%
\pgfpathlineto{\pgfqpoint{3.610926in}{2.338340in}}%
\pgfpathlineto{\pgfqpoint{3.615939in}{2.347855in}}%
\pgfpathlineto{\pgfqpoint{3.625966in}{2.349894in}}%
\pgfpathlineto{\pgfqpoint{3.628472in}{2.352598in}}%
\pgfpathlineto{\pgfqpoint{3.630979in}{2.353166in}}%
\pgfpathlineto{\pgfqpoint{3.635992in}{2.356063in}}%
\pgfpathlineto{\pgfqpoint{3.638499in}{2.356231in}}%
\pgfpathlineto{\pgfqpoint{3.651032in}{2.363271in}}%
\pgfpathlineto{\pgfqpoint{3.656045in}{2.364687in}}%
\pgfpathlineto{\pgfqpoint{3.666072in}{2.367044in}}%
\pgfpathlineto{\pgfqpoint{3.673592in}{2.370065in}}%
\pgfpathlineto{\pgfqpoint{3.678605in}{2.370871in}}%
\pgfpathlineto{\pgfqpoint{3.688632in}{2.372930in}}%
\pgfpathlineto{\pgfqpoint{3.691139in}{2.377712in}}%
\pgfpathlineto{\pgfqpoint{3.696152in}{2.378835in}}%
\pgfpathlineto{\pgfqpoint{3.706178in}{2.380518in}}%
\pgfpathlineto{\pgfqpoint{3.708685in}{2.382215in}}%
\pgfpathlineto{\pgfqpoint{3.711192in}{2.382338in}}%
\pgfpathlineto{\pgfqpoint{3.713698in}{2.385464in}}%
\pgfpathlineto{\pgfqpoint{3.718712in}{2.389199in}}%
\pgfpathlineto{\pgfqpoint{3.721218in}{2.392386in}}%
\pgfpathlineto{\pgfqpoint{3.723725in}{2.392637in}}%
\pgfpathlineto{\pgfqpoint{3.728738in}{2.395025in}}%
\pgfpathlineto{\pgfqpoint{3.738765in}{2.396387in}}%
\pgfpathlineto{\pgfqpoint{3.748792in}{2.399648in}}%
\pgfpathlineto{\pgfqpoint{3.753805in}{2.401023in}}%
\pgfpathlineto{\pgfqpoint{3.756312in}{2.403457in}}%
\pgfpathlineto{\pgfqpoint{3.758818in}{2.404215in}}%
\pgfpathlineto{\pgfqpoint{3.761325in}{2.411761in}}%
\pgfpathlineto{\pgfqpoint{3.763832in}{2.413291in}}%
\pgfpathlineto{\pgfqpoint{3.773858in}{2.414484in}}%
\pgfpathlineto{\pgfqpoint{3.778871in}{2.415492in}}%
\pgfpathlineto{\pgfqpoint{3.781378in}{2.418173in}}%
\pgfpathlineto{\pgfqpoint{3.786391in}{2.418390in}}%
\pgfpathlineto{\pgfqpoint{3.791405in}{2.424718in}}%
\pgfpathlineto{\pgfqpoint{3.793911in}{2.433576in}}%
\pgfpathlineto{\pgfqpoint{3.796418in}{2.437085in}}%
\pgfpathlineto{\pgfqpoint{3.801431in}{2.437434in}}%
\pgfpathlineto{\pgfqpoint{3.803938in}{2.440519in}}%
\pgfpathlineto{\pgfqpoint{3.808951in}{2.440840in}}%
\pgfpathlineto{\pgfqpoint{3.811458in}{2.445438in}}%
\pgfpathlineto{\pgfqpoint{3.818978in}{2.449347in}}%
\pgfpathlineto{\pgfqpoint{3.823991in}{2.449858in}}%
\pgfpathlineto{\pgfqpoint{3.829005in}{2.453474in}}%
\pgfpathlineto{\pgfqpoint{3.841538in}{2.459448in}}%
\pgfpathlineto{\pgfqpoint{3.849058in}{2.463832in}}%
\pgfpathlineto{\pgfqpoint{3.854071in}{2.464283in}}%
\pgfpathlineto{\pgfqpoint{3.864098in}{2.470969in}}%
\pgfpathlineto{\pgfqpoint{3.876631in}{2.478720in}}%
\pgfpathlineto{\pgfqpoint{3.879138in}{2.478808in}}%
\pgfpathlineto{\pgfqpoint{3.881644in}{2.481085in}}%
\pgfpathlineto{\pgfqpoint{3.886658in}{2.483045in}}%
\pgfpathlineto{\pgfqpoint{3.896684in}{2.486319in}}%
\pgfpathlineto{\pgfqpoint{3.899191in}{2.486834in}}%
\pgfpathlineto{\pgfqpoint{3.901698in}{2.489611in}}%
\pgfpathlineto{\pgfqpoint{3.906711in}{2.490501in}}%
\pgfpathlineto{\pgfqpoint{3.909217in}{2.492891in}}%
\pgfpathlineto{\pgfqpoint{3.931777in}{2.496529in}}%
\pgfpathlineto{\pgfqpoint{3.934284in}{2.499718in}}%
\pgfpathlineto{\pgfqpoint{3.949324in}{2.504516in}}%
\pgfpathlineto{\pgfqpoint{3.951831in}{2.508119in}}%
\pgfpathlineto{\pgfqpoint{3.969377in}{2.512630in}}%
\pgfpathlineto{\pgfqpoint{3.974391in}{2.514027in}}%
\pgfpathlineto{\pgfqpoint{3.976897in}{2.517335in}}%
\pgfpathlineto{\pgfqpoint{3.989430in}{2.518488in}}%
\pgfpathlineto{\pgfqpoint{3.994444in}{2.523128in}}%
\pgfpathlineto{\pgfqpoint{4.004470in}{2.525698in}}%
\pgfpathlineto{\pgfqpoint{4.006977in}{2.526826in}}%
\pgfpathlineto{\pgfqpoint{4.009484in}{2.531421in}}%
\pgfpathlineto{\pgfqpoint{4.011990in}{2.533472in}}%
\pgfpathlineto{\pgfqpoint{4.017004in}{2.534065in}}%
\pgfpathlineto{\pgfqpoint{4.019510in}{2.538461in}}%
\pgfpathlineto{\pgfqpoint{4.027030in}{2.539286in}}%
\pgfpathlineto{\pgfqpoint{4.034550in}{2.542516in}}%
\pgfpathlineto{\pgfqpoint{4.039564in}{2.543170in}}%
\pgfpathlineto{\pgfqpoint{4.044577in}{2.544479in}}%
\pgfpathlineto{\pgfqpoint{4.047083in}{2.549035in}}%
\pgfpathlineto{\pgfqpoint{4.052097in}{2.550316in}}%
\pgfpathlineto{\pgfqpoint{4.059617in}{2.551410in}}%
\pgfpathlineto{\pgfqpoint{4.067137in}{2.554656in}}%
\pgfpathlineto{\pgfqpoint{4.069643in}{2.557013in}}%
\pgfpathlineto{\pgfqpoint{4.074657in}{2.557611in}}%
\pgfpathlineto{\pgfqpoint{4.079670in}{2.560624in}}%
\pgfpathlineto{\pgfqpoint{4.082177in}{2.564207in}}%
\pgfpathlineto{\pgfqpoint{4.084683in}{2.564541in}}%
\pgfpathlineto{\pgfqpoint{4.087190in}{2.566897in}}%
\pgfpathlineto{\pgfqpoint{4.089697in}{2.567254in}}%
\pgfpathlineto{\pgfqpoint{4.094710in}{2.571985in}}%
\pgfpathlineto{\pgfqpoint{4.099723in}{2.573992in}}%
\pgfpathlineto{\pgfqpoint{4.104737in}{2.580784in}}%
\pgfpathlineto{\pgfqpoint{4.107243in}{2.581453in}}%
\pgfpathlineto{\pgfqpoint{4.109750in}{2.584637in}}%
\pgfpathlineto{\pgfqpoint{4.134816in}{2.592817in}}%
\pgfpathlineto{\pgfqpoint{4.139830in}{2.594442in}}%
\pgfpathlineto{\pgfqpoint{4.142336in}{2.595132in}}%
\pgfpathlineto{\pgfqpoint{4.144843in}{2.597110in}}%
\pgfpathlineto{\pgfqpoint{4.154870in}{2.598542in}}%
\pgfpathlineto{\pgfqpoint{4.159883in}{2.600937in}}%
\pgfpathlineto{\pgfqpoint{4.164896in}{2.603600in}}%
\pgfpathlineto{\pgfqpoint{4.169910in}{2.604396in}}%
\pgfpathlineto{\pgfqpoint{4.179936in}{2.605803in}}%
\pgfpathlineto{\pgfqpoint{4.184949in}{2.609037in}}%
\pgfpathlineto{\pgfqpoint{4.189963in}{2.610214in}}%
\pgfpathlineto{\pgfqpoint{4.192469in}{2.612576in}}%
\pgfpathlineto{\pgfqpoint{4.194976in}{2.612738in}}%
\pgfpathlineto{\pgfqpoint{4.199989in}{2.621908in}}%
\pgfpathlineto{\pgfqpoint{4.202496in}{2.622247in}}%
\pgfpathlineto{\pgfqpoint{4.205003in}{2.624796in}}%
\pgfpathlineto{\pgfqpoint{4.207509in}{2.628965in}}%
\pgfpathlineto{\pgfqpoint{4.212523in}{2.629509in}}%
\pgfpathlineto{\pgfqpoint{4.220043in}{2.634440in}}%
\pgfpathlineto{\pgfqpoint{4.227563in}{2.637473in}}%
\pgfpathlineto{\pgfqpoint{4.232576in}{2.639024in}}%
\pgfpathlineto{\pgfqpoint{4.235083in}{2.640182in}}%
\pgfpathlineto{\pgfqpoint{4.240096in}{2.647417in}}%
\pgfpathlineto{\pgfqpoint{4.245109in}{2.650060in}}%
\pgfpathlineto{\pgfqpoint{4.257642in}{2.653486in}}%
\pgfpathlineto{\pgfqpoint{4.260149in}{2.657985in}}%
\pgfpathlineto{\pgfqpoint{4.262656in}{2.658401in}}%
\pgfpathlineto{\pgfqpoint{4.265162in}{2.660857in}}%
\pgfpathlineto{\pgfqpoint{4.272682in}{2.664065in}}%
\pgfpathlineto{\pgfqpoint{4.285216in}{2.669605in}}%
\pgfpathlineto{\pgfqpoint{4.290229in}{2.674740in}}%
\pgfpathlineto{\pgfqpoint{4.292736in}{2.676562in}}%
\pgfpathlineto{\pgfqpoint{4.295242in}{2.680458in}}%
\pgfpathlineto{\pgfqpoint{4.297749in}{2.680603in}}%
\pgfpathlineto{\pgfqpoint{4.300256in}{2.684434in}}%
\pgfpathlineto{\pgfqpoint{4.305269in}{2.686357in}}%
\pgfpathlineto{\pgfqpoint{4.307776in}{2.687116in}}%
\pgfpathlineto{\pgfqpoint{4.315296in}{2.700495in}}%
\pgfpathlineto{\pgfqpoint{4.325322in}{2.705426in}}%
\pgfpathlineto{\pgfqpoint{4.327829in}{2.707879in}}%
\pgfpathlineto{\pgfqpoint{4.330335in}{2.708244in}}%
\pgfpathlineto{\pgfqpoint{4.335349in}{2.711380in}}%
\pgfpathlineto{\pgfqpoint{4.337855in}{2.719019in}}%
\pgfpathlineto{\pgfqpoint{4.340362in}{2.722984in}}%
\pgfpathlineto{\pgfqpoint{4.342869in}{2.723065in}}%
\pgfpathlineto{\pgfqpoint{4.347882in}{2.728732in}}%
\pgfpathlineto{\pgfqpoint{4.350389in}{2.735791in}}%
\pgfpathlineto{\pgfqpoint{4.352895in}{2.736643in}}%
\pgfpathlineto{\pgfqpoint{4.355402in}{2.742961in}}%
\pgfpathlineto{\pgfqpoint{4.357909in}{2.744004in}}%
\pgfpathlineto{\pgfqpoint{4.360415in}{2.746930in}}%
\pgfpathlineto{\pgfqpoint{4.362922in}{2.758600in}}%
\pgfpathlineto{\pgfqpoint{4.365429in}{2.759327in}}%
\pgfpathlineto{\pgfqpoint{4.367935in}{2.761621in}}%
\pgfpathlineto{\pgfqpoint{4.370442in}{2.783764in}}%
\pgfpathlineto{\pgfqpoint{4.375455in}{2.790157in}}%
\pgfpathlineto{\pgfqpoint{4.377962in}{2.800605in}}%
\pgfpathlineto{\pgfqpoint{4.380469in}{2.802388in}}%
\pgfpathlineto{\pgfqpoint{4.382975in}{2.805850in}}%
\pgfpathlineto{\pgfqpoint{4.385482in}{2.854621in}}%
\pgfpathlineto{\pgfqpoint{4.393002in}{2.878479in}}%
\pgfpathlineto{\pgfqpoint{4.395508in}{2.897577in}}%
\pgfpathlineto{\pgfqpoint{4.398015in}{2.905275in}}%
\pgfpathlineto{\pgfqpoint{4.398015in}{2.905275in}}%
\pgfusepath{stroke}%
\end{pgfscope}%
\begin{pgfscope}%
\pgfpathrectangle{\pgfqpoint{0.708220in}{0.535823in}}{\pgfqpoint{5.013309in}{2.369453in}}%
\pgfusepath{clip}%
\pgfsetbuttcap%
\pgfsetroundjoin%
\pgfsetlinewidth{1.003750pt}%
\definecolor{currentstroke}{rgb}{1.000000,0.000000,0.000000}%
\pgfsetstrokecolor{currentstroke}%
\pgfsetdash{{1.000000pt}{1.650000pt}}{0.000000pt}%
\pgfpathmoveto{\pgfqpoint{0.713233in}{0.531304in}}%
\pgfpathlineto{\pgfqpoint{0.720753in}{0.532358in}}%
\pgfpathlineto{\pgfqpoint{0.735793in}{0.536678in}}%
\pgfpathlineto{\pgfqpoint{0.738300in}{0.536849in}}%
\pgfpathlineto{\pgfqpoint{0.740806in}{0.538545in}}%
\pgfpathlineto{\pgfqpoint{0.745820in}{0.538882in}}%
\pgfpathlineto{\pgfqpoint{0.753340in}{0.540559in}}%
\pgfpathlineto{\pgfqpoint{0.755846in}{0.544027in}}%
\pgfpathlineto{\pgfqpoint{0.758353in}{0.647790in}}%
\pgfpathlineto{\pgfqpoint{0.760860in}{0.648592in}}%
\pgfpathlineto{\pgfqpoint{0.763366in}{0.650711in}}%
\pgfpathlineto{\pgfqpoint{0.785926in}{0.652978in}}%
\pgfpathlineto{\pgfqpoint{0.795953in}{0.653928in}}%
\pgfpathlineto{\pgfqpoint{0.826033in}{0.658182in}}%
\pgfpathlineto{\pgfqpoint{0.841073in}{0.660021in}}%
\pgfpathlineto{\pgfqpoint{0.851099in}{0.664453in}}%
\pgfpathlineto{\pgfqpoint{0.853606in}{0.669953in}}%
\pgfpathlineto{\pgfqpoint{0.856112in}{0.715565in}}%
\pgfpathlineto{\pgfqpoint{0.858619in}{0.718664in}}%
\pgfpathlineto{\pgfqpoint{0.861126in}{0.719371in}}%
\pgfpathlineto{\pgfqpoint{0.866139in}{0.722461in}}%
\pgfpathlineto{\pgfqpoint{0.878672in}{0.724242in}}%
\pgfpathlineto{\pgfqpoint{0.911259in}{0.730638in}}%
\pgfpathlineto{\pgfqpoint{0.913766in}{0.731955in}}%
\pgfpathlineto{\pgfqpoint{0.916272in}{0.738865in}}%
\pgfpathlineto{\pgfqpoint{0.918779in}{0.767471in}}%
\pgfpathlineto{\pgfqpoint{0.966405in}{0.774095in}}%
\pgfpathlineto{\pgfqpoint{0.988965in}{0.775585in}}%
\pgfpathlineto{\pgfqpoint{0.996485in}{0.776852in}}%
\pgfpathlineto{\pgfqpoint{1.006512in}{0.777900in}}%
\pgfpathlineto{\pgfqpoint{1.014032in}{0.780432in}}%
\pgfpathlineto{\pgfqpoint{1.029072in}{0.783008in}}%
\pgfpathlineto{\pgfqpoint{1.031578in}{0.806487in}}%
\pgfpathlineto{\pgfqpoint{1.046618in}{0.808562in}}%
\pgfpathlineto{\pgfqpoint{1.071685in}{0.811302in}}%
\pgfpathlineto{\pgfqpoint{1.101765in}{0.813116in}}%
\pgfpathlineto{\pgfqpoint{1.114298in}{0.814169in}}%
\pgfpathlineto{\pgfqpoint{1.134351in}{0.815752in}}%
\pgfpathlineto{\pgfqpoint{1.139364in}{0.815953in}}%
\pgfpathlineto{\pgfqpoint{1.146884in}{0.819368in}}%
\pgfpathlineto{\pgfqpoint{1.154404in}{0.820644in}}%
\pgfpathlineto{\pgfqpoint{1.156911in}{0.824892in}}%
\pgfpathlineto{\pgfqpoint{1.159418in}{0.826915in}}%
\pgfpathlineto{\pgfqpoint{1.161924in}{0.835882in}}%
\pgfpathlineto{\pgfqpoint{1.164431in}{0.837956in}}%
\pgfpathlineto{\pgfqpoint{1.174458in}{0.839423in}}%
\pgfpathlineto{\pgfqpoint{1.202031in}{0.841224in}}%
\pgfpathlineto{\pgfqpoint{1.217071in}{0.842520in}}%
\pgfpathlineto{\pgfqpoint{1.294777in}{0.850040in}}%
\pgfpathlineto{\pgfqpoint{1.299790in}{0.853707in}}%
\pgfpathlineto{\pgfqpoint{1.302297in}{0.863169in}}%
\pgfpathlineto{\pgfqpoint{1.307310in}{0.864386in}}%
\pgfpathlineto{\pgfqpoint{1.317337in}{0.867390in}}%
\pgfpathlineto{\pgfqpoint{1.342403in}{0.869682in}}%
\pgfpathlineto{\pgfqpoint{1.352430in}{0.870195in}}%
\pgfpathlineto{\pgfqpoint{1.374990in}{0.871750in}}%
\pgfpathlineto{\pgfqpoint{1.397550in}{0.874510in}}%
\pgfpathlineto{\pgfqpoint{1.400056in}{0.878602in}}%
\pgfpathlineto{\pgfqpoint{1.402563in}{0.887082in}}%
\pgfpathlineto{\pgfqpoint{1.437656in}{0.898485in}}%
\pgfpathlineto{\pgfqpoint{1.440163in}{0.909158in}}%
\pgfpathlineto{\pgfqpoint{1.447683in}{0.910205in}}%
\pgfpathlineto{\pgfqpoint{1.450190in}{0.910494in}}%
\pgfpathlineto{\pgfqpoint{1.452696in}{0.913175in}}%
\pgfpathlineto{\pgfqpoint{1.457710in}{0.913327in}}%
\pgfpathlineto{\pgfqpoint{1.462723in}{0.915665in}}%
\pgfpathlineto{\pgfqpoint{1.467736in}{0.916132in}}%
\pgfpathlineto{\pgfqpoint{1.470243in}{0.918948in}}%
\pgfpathlineto{\pgfqpoint{1.475256in}{0.919717in}}%
\pgfpathlineto{\pgfqpoint{1.477763in}{0.920957in}}%
\pgfpathlineto{\pgfqpoint{1.480269in}{0.926196in}}%
\pgfpathlineto{\pgfqpoint{1.510349in}{0.931638in}}%
\pgfpathlineto{\pgfqpoint{1.512856in}{0.931894in}}%
\pgfpathlineto{\pgfqpoint{1.517869in}{0.933504in}}%
\pgfpathlineto{\pgfqpoint{1.525389in}{0.934111in}}%
\pgfpathlineto{\pgfqpoint{1.527896in}{0.943071in}}%
\pgfpathlineto{\pgfqpoint{1.532909in}{0.944991in}}%
\pgfpathlineto{\pgfqpoint{1.535416in}{0.948321in}}%
\pgfpathlineto{\pgfqpoint{1.542936in}{0.949139in}}%
\pgfpathlineto{\pgfqpoint{1.552962in}{0.958973in}}%
\pgfpathlineto{\pgfqpoint{1.557976in}{0.960190in}}%
\pgfpathlineto{\pgfqpoint{1.562989in}{0.961686in}}%
\pgfpathlineto{\pgfqpoint{1.583042in}{0.964723in}}%
\pgfpathlineto{\pgfqpoint{1.585549in}{0.974005in}}%
\pgfpathlineto{\pgfqpoint{1.595576in}{0.975253in}}%
\pgfpathlineto{\pgfqpoint{1.600589in}{0.976990in}}%
\pgfpathlineto{\pgfqpoint{1.603095in}{0.979422in}}%
\pgfpathlineto{\pgfqpoint{1.605602in}{0.979989in}}%
\pgfpathlineto{\pgfqpoint{1.613122in}{0.984711in}}%
\pgfpathlineto{\pgfqpoint{1.618135in}{0.985946in}}%
\pgfpathlineto{\pgfqpoint{1.623149in}{0.987024in}}%
\pgfpathlineto{\pgfqpoint{1.635682in}{0.987813in}}%
\pgfpathlineto{\pgfqpoint{1.640695in}{0.989440in}}%
\pgfpathlineto{\pgfqpoint{1.645709in}{0.990315in}}%
\pgfpathlineto{\pgfqpoint{1.650722in}{0.991655in}}%
\pgfpathlineto{\pgfqpoint{1.653229in}{0.992590in}}%
\pgfpathlineto{\pgfqpoint{1.658242in}{0.996082in}}%
\pgfpathlineto{\pgfqpoint{1.665762in}{0.997505in}}%
\pgfpathlineto{\pgfqpoint{1.670775in}{1.000979in}}%
\pgfpathlineto{\pgfqpoint{1.675788in}{1.001400in}}%
\pgfpathlineto{\pgfqpoint{1.678295in}{1.004629in}}%
\pgfpathlineto{\pgfqpoint{1.683308in}{1.006039in}}%
\pgfpathlineto{\pgfqpoint{1.688322in}{1.007240in}}%
\pgfpathlineto{\pgfqpoint{1.693335in}{1.009129in}}%
\pgfpathlineto{\pgfqpoint{1.705868in}{1.013079in}}%
\pgfpathlineto{\pgfqpoint{1.708375in}{1.014126in}}%
\pgfpathlineto{\pgfqpoint{1.710882in}{1.018468in}}%
\pgfpathlineto{\pgfqpoint{1.723415in}{1.019320in}}%
\pgfpathlineto{\pgfqpoint{1.728428in}{1.020898in}}%
\pgfpathlineto{\pgfqpoint{1.738455in}{1.021990in}}%
\pgfpathlineto{\pgfqpoint{1.791095in}{1.036416in}}%
\pgfpathlineto{\pgfqpoint{1.796108in}{1.038825in}}%
\pgfpathlineto{\pgfqpoint{1.803628in}{1.040787in}}%
\pgfpathlineto{\pgfqpoint{1.806134in}{1.042868in}}%
\pgfpathlineto{\pgfqpoint{1.811148in}{1.044157in}}%
\pgfpathlineto{\pgfqpoint{1.816161in}{1.047097in}}%
\pgfpathlineto{\pgfqpoint{1.826188in}{1.049405in}}%
\pgfpathlineto{\pgfqpoint{1.831201in}{1.053502in}}%
\pgfpathlineto{\pgfqpoint{1.833708in}{1.055536in}}%
\pgfpathlineto{\pgfqpoint{1.841228in}{1.056897in}}%
\pgfpathlineto{\pgfqpoint{1.843734in}{1.057529in}}%
\pgfpathlineto{\pgfqpoint{1.846241in}{1.062174in}}%
\pgfpathlineto{\pgfqpoint{1.848748in}{1.063681in}}%
\pgfpathlineto{\pgfqpoint{1.858774in}{1.064666in}}%
\pgfpathlineto{\pgfqpoint{1.863788in}{1.066253in}}%
\pgfpathlineto{\pgfqpoint{1.883841in}{1.070928in}}%
\pgfpathlineto{\pgfqpoint{1.886347in}{1.074451in}}%
\pgfpathlineto{\pgfqpoint{1.893867in}{1.078600in}}%
\pgfpathlineto{\pgfqpoint{1.896374in}{1.080128in}}%
\pgfpathlineto{\pgfqpoint{1.898881in}{1.084304in}}%
\pgfpathlineto{\pgfqpoint{1.918934in}{1.089173in}}%
\pgfpathlineto{\pgfqpoint{1.921441in}{1.089411in}}%
\pgfpathlineto{\pgfqpoint{1.923947in}{1.091620in}}%
\pgfpathlineto{\pgfqpoint{1.946507in}{1.095831in}}%
\pgfpathlineto{\pgfqpoint{1.951520in}{1.100208in}}%
\pgfpathlineto{\pgfqpoint{1.954027in}{1.100629in}}%
\pgfpathlineto{\pgfqpoint{1.959040in}{1.103334in}}%
\pgfpathlineto{\pgfqpoint{1.964054in}{1.104751in}}%
\pgfpathlineto{\pgfqpoint{1.966560in}{1.107465in}}%
\pgfpathlineto{\pgfqpoint{1.971574in}{1.108301in}}%
\pgfpathlineto{\pgfqpoint{1.974080in}{1.111248in}}%
\pgfpathlineto{\pgfqpoint{1.976587in}{1.111356in}}%
\pgfpathlineto{\pgfqpoint{1.981600in}{1.113206in}}%
\pgfpathlineto{\pgfqpoint{1.986614in}{1.114286in}}%
\pgfpathlineto{\pgfqpoint{1.996640in}{1.118032in}}%
\pgfpathlineto{\pgfqpoint{2.009173in}{1.125387in}}%
\pgfpathlineto{\pgfqpoint{2.011680in}{1.128647in}}%
\pgfpathlineto{\pgfqpoint{2.016693in}{1.130327in}}%
\pgfpathlineto{\pgfqpoint{2.019200in}{1.132884in}}%
\pgfpathlineto{\pgfqpoint{2.021707in}{1.133011in}}%
\pgfpathlineto{\pgfqpoint{2.024213in}{1.135907in}}%
\pgfpathlineto{\pgfqpoint{2.026720in}{1.136817in}}%
\pgfpathlineto{\pgfqpoint{2.029227in}{1.139519in}}%
\pgfpathlineto{\pgfqpoint{2.039253in}{1.141287in}}%
\pgfpathlineto{\pgfqpoint{2.041760in}{1.144896in}}%
\pgfpathlineto{\pgfqpoint{2.046773in}{1.145995in}}%
\pgfpathlineto{\pgfqpoint{2.054293in}{1.147465in}}%
\pgfpathlineto{\pgfqpoint{2.056800in}{1.147789in}}%
\pgfpathlineto{\pgfqpoint{2.059307in}{1.150034in}}%
\pgfpathlineto{\pgfqpoint{2.064320in}{1.150725in}}%
\pgfpathlineto{\pgfqpoint{2.069333in}{1.153756in}}%
\pgfpathlineto{\pgfqpoint{2.071840in}{1.154697in}}%
\pgfpathlineto{\pgfqpoint{2.074346in}{1.159155in}}%
\pgfpathlineto{\pgfqpoint{2.089386in}{1.162091in}}%
\pgfpathlineto{\pgfqpoint{2.091893in}{1.163690in}}%
\pgfpathlineto{\pgfqpoint{2.094400in}{1.163862in}}%
\pgfpathlineto{\pgfqpoint{2.096906in}{1.165353in}}%
\pgfpathlineto{\pgfqpoint{2.106933in}{1.166545in}}%
\pgfpathlineto{\pgfqpoint{2.111946in}{1.168039in}}%
\pgfpathlineto{\pgfqpoint{2.114453in}{1.168580in}}%
\pgfpathlineto{\pgfqpoint{2.119466in}{1.172687in}}%
\pgfpathlineto{\pgfqpoint{2.126986in}{1.173926in}}%
\pgfpathlineto{\pgfqpoint{2.132000in}{1.175056in}}%
\pgfpathlineto{\pgfqpoint{2.134506in}{1.176858in}}%
\pgfpathlineto{\pgfqpoint{2.142026in}{1.178338in}}%
\pgfpathlineto{\pgfqpoint{2.144533in}{1.180374in}}%
\pgfpathlineto{\pgfqpoint{2.149546in}{1.182213in}}%
\pgfpathlineto{\pgfqpoint{2.152053in}{1.182959in}}%
\pgfpathlineto{\pgfqpoint{2.162079in}{1.191642in}}%
\pgfpathlineto{\pgfqpoint{2.172106in}{1.193089in}}%
\pgfpathlineto{\pgfqpoint{2.174613in}{1.195304in}}%
\pgfpathlineto{\pgfqpoint{2.177119in}{1.199398in}}%
\pgfpathlineto{\pgfqpoint{2.187146in}{1.201335in}}%
\pgfpathlineto{\pgfqpoint{2.197173in}{1.204572in}}%
\pgfpathlineto{\pgfqpoint{2.204693in}{1.205860in}}%
\pgfpathlineto{\pgfqpoint{2.207199in}{1.208087in}}%
\pgfpathlineto{\pgfqpoint{2.214719in}{1.209763in}}%
\pgfpathlineto{\pgfqpoint{2.227252in}{1.216576in}}%
\pgfpathlineto{\pgfqpoint{2.232266in}{1.217097in}}%
\pgfpathlineto{\pgfqpoint{2.237279in}{1.219677in}}%
\pgfpathlineto{\pgfqpoint{2.239786in}{1.221601in}}%
\pgfpathlineto{\pgfqpoint{2.244799in}{1.221899in}}%
\pgfpathlineto{\pgfqpoint{2.247306in}{1.222942in}}%
\pgfpathlineto{\pgfqpoint{2.249812in}{1.226161in}}%
\pgfpathlineto{\pgfqpoint{2.264852in}{1.229948in}}%
\pgfpathlineto{\pgfqpoint{2.267359in}{1.232629in}}%
\pgfpathlineto{\pgfqpoint{2.269866in}{1.233615in}}%
\pgfpathlineto{\pgfqpoint{2.272372in}{1.237463in}}%
\pgfpathlineto{\pgfqpoint{2.277385in}{1.238578in}}%
\pgfpathlineto{\pgfqpoint{2.287412in}{1.245989in}}%
\pgfpathlineto{\pgfqpoint{2.297439in}{1.247956in}}%
\pgfpathlineto{\pgfqpoint{2.302452in}{1.251758in}}%
\pgfpathlineto{\pgfqpoint{2.309972in}{1.252734in}}%
\pgfpathlineto{\pgfqpoint{2.312479in}{1.256248in}}%
\pgfpathlineto{\pgfqpoint{2.360105in}{1.270770in}}%
\pgfpathlineto{\pgfqpoint{2.362612in}{1.273911in}}%
\pgfpathlineto{\pgfqpoint{2.370132in}{1.276890in}}%
\pgfpathlineto{\pgfqpoint{2.372638in}{1.279093in}}%
\pgfpathlineto{\pgfqpoint{2.390185in}{1.280997in}}%
\pgfpathlineto{\pgfqpoint{2.410238in}{1.287179in}}%
\pgfpathlineto{\pgfqpoint{2.412745in}{1.289661in}}%
\pgfpathlineto{\pgfqpoint{2.425278in}{1.292610in}}%
\pgfpathlineto{\pgfqpoint{2.427785in}{1.299050in}}%
\pgfpathlineto{\pgfqpoint{2.450345in}{1.307570in}}%
\pgfpathlineto{\pgfqpoint{2.452851in}{1.307570in}}%
\pgfpathlineto{\pgfqpoint{2.460371in}{1.311340in}}%
\pgfpathlineto{\pgfqpoint{2.462878in}{1.311340in}}%
\pgfpathlineto{\pgfqpoint{2.467891in}{1.312599in}}%
\pgfpathlineto{\pgfqpoint{2.497971in}{1.320593in}}%
\pgfpathlineto{\pgfqpoint{2.500478in}{1.323081in}}%
\pgfpathlineto{\pgfqpoint{2.513011in}{1.327655in}}%
\pgfpathlineto{\pgfqpoint{2.515518in}{1.329994in}}%
\pgfpathlineto{\pgfqpoint{2.520531in}{1.330710in}}%
\pgfpathlineto{\pgfqpoint{2.528051in}{1.334740in}}%
\pgfpathlineto{\pgfqpoint{2.533064in}{1.335634in}}%
\pgfpathlineto{\pgfqpoint{2.535571in}{1.335634in}}%
\pgfpathlineto{\pgfqpoint{2.538078in}{1.338839in}}%
\pgfpathlineto{\pgfqpoint{2.543091in}{1.339891in}}%
\pgfpathlineto{\pgfqpoint{2.548104in}{1.340420in}}%
\pgfpathlineto{\pgfqpoint{2.550611in}{1.341986in}}%
\pgfpathlineto{\pgfqpoint{2.558131in}{1.342423in}}%
\pgfpathlineto{\pgfqpoint{2.563144in}{1.343692in}}%
\pgfpathlineto{\pgfqpoint{2.570664in}{1.345077in}}%
\pgfpathlineto{\pgfqpoint{2.573171in}{1.348112in}}%
\pgfpathlineto{\pgfqpoint{2.578184in}{1.348300in}}%
\pgfpathlineto{\pgfqpoint{2.580691in}{1.349610in}}%
\pgfpathlineto{\pgfqpoint{2.590717in}{1.350002in}}%
\pgfpathlineto{\pgfqpoint{2.600744in}{1.353950in}}%
\pgfpathlineto{\pgfqpoint{2.605757in}{1.354072in}}%
\pgfpathlineto{\pgfqpoint{2.608264in}{1.355474in}}%
\pgfpathlineto{\pgfqpoint{2.610771in}{1.355474in}}%
\pgfpathlineto{\pgfqpoint{2.613277in}{1.356910in}}%
\pgfpathlineto{\pgfqpoint{2.618290in}{1.357001in}}%
\pgfpathlineto{\pgfqpoint{2.635837in}{1.365277in}}%
\pgfpathlineto{\pgfqpoint{2.640850in}{1.366430in}}%
\pgfpathlineto{\pgfqpoint{2.643357in}{1.367651in}}%
\pgfpathlineto{\pgfqpoint{2.645864in}{1.371310in}}%
\pgfpathlineto{\pgfqpoint{2.648370in}{1.371569in}}%
\pgfpathlineto{\pgfqpoint{2.658397in}{1.377285in}}%
\pgfpathlineto{\pgfqpoint{2.660904in}{1.378376in}}%
\pgfpathlineto{\pgfqpoint{2.665917in}{1.382442in}}%
\pgfpathlineto{\pgfqpoint{2.668424in}{1.382747in}}%
\pgfpathlineto{\pgfqpoint{2.675944in}{1.387246in}}%
\pgfpathlineto{\pgfqpoint{2.683464in}{1.389206in}}%
\pgfpathlineto{\pgfqpoint{2.688477in}{1.393508in}}%
\pgfpathlineto{\pgfqpoint{2.695997in}{1.397663in}}%
\pgfpathlineto{\pgfqpoint{2.701010in}{1.399694in}}%
\pgfpathlineto{\pgfqpoint{2.703517in}{1.402943in}}%
\pgfpathlineto{\pgfqpoint{2.711037in}{1.403460in}}%
\pgfpathlineto{\pgfqpoint{2.721063in}{1.410484in}}%
\pgfpathlineto{\pgfqpoint{2.723570in}{1.410504in}}%
\pgfpathlineto{\pgfqpoint{2.726077in}{1.415232in}}%
\pgfpathlineto{\pgfqpoint{2.736103in}{1.420635in}}%
\pgfpathlineto{\pgfqpoint{2.738610in}{1.422595in}}%
\pgfpathlineto{\pgfqpoint{2.743623in}{1.423567in}}%
\pgfpathlineto{\pgfqpoint{2.746130in}{1.429284in}}%
\pgfpathlineto{\pgfqpoint{2.748637in}{1.432073in}}%
\pgfpathlineto{\pgfqpoint{2.761170in}{1.435049in}}%
\pgfpathlineto{\pgfqpoint{2.768690in}{1.438882in}}%
\pgfpathlineto{\pgfqpoint{2.771196in}{1.439335in}}%
\pgfpathlineto{\pgfqpoint{2.773703in}{1.441055in}}%
\pgfpathlineto{\pgfqpoint{2.778716in}{1.441846in}}%
\pgfpathlineto{\pgfqpoint{2.786236in}{1.445375in}}%
\pgfpathlineto{\pgfqpoint{2.788743in}{1.445571in}}%
\pgfpathlineto{\pgfqpoint{2.793756in}{1.448754in}}%
\pgfpathlineto{\pgfqpoint{2.798770in}{1.448983in}}%
\pgfpathlineto{\pgfqpoint{2.806290in}{1.453744in}}%
\pgfpathlineto{\pgfqpoint{2.808796in}{1.456265in}}%
\pgfpathlineto{\pgfqpoint{2.811303in}{1.456353in}}%
\pgfpathlineto{\pgfqpoint{2.813810in}{1.463189in}}%
\pgfpathlineto{\pgfqpoint{2.816316in}{1.463189in}}%
\pgfpathlineto{\pgfqpoint{2.821329in}{1.469607in}}%
\pgfpathlineto{\pgfqpoint{2.823836in}{1.470712in}}%
\pgfpathlineto{\pgfqpoint{2.831356in}{1.480142in}}%
\pgfpathlineto{\pgfqpoint{2.836369in}{1.482103in}}%
\pgfpathlineto{\pgfqpoint{2.843889in}{1.482507in}}%
\pgfpathlineto{\pgfqpoint{2.848903in}{1.483475in}}%
\pgfpathlineto{\pgfqpoint{2.851409in}{1.484019in}}%
\pgfpathlineto{\pgfqpoint{2.856423in}{1.487816in}}%
\pgfpathlineto{\pgfqpoint{2.863943in}{1.489517in}}%
\pgfpathlineto{\pgfqpoint{2.866449in}{1.491215in}}%
\pgfpathlineto{\pgfqpoint{2.873969in}{1.492784in}}%
\pgfpathlineto{\pgfqpoint{2.878983in}{1.496541in}}%
\pgfpathlineto{\pgfqpoint{2.881489in}{1.502271in}}%
\pgfpathlineto{\pgfqpoint{2.883996in}{1.505190in}}%
\pgfpathlineto{\pgfqpoint{2.894022in}{1.506190in}}%
\pgfpathlineto{\pgfqpoint{2.896529in}{1.507782in}}%
\pgfpathlineto{\pgfqpoint{2.899036in}{1.512912in}}%
\pgfpathlineto{\pgfqpoint{2.904049in}{1.513114in}}%
\pgfpathlineto{\pgfqpoint{2.909062in}{1.519671in}}%
\pgfpathlineto{\pgfqpoint{2.911569in}{1.519683in}}%
\pgfpathlineto{\pgfqpoint{2.916582in}{1.524053in}}%
\pgfpathlineto{\pgfqpoint{2.921596in}{1.525354in}}%
\pgfpathlineto{\pgfqpoint{2.924102in}{1.526729in}}%
\pgfpathlineto{\pgfqpoint{2.929116in}{1.531610in}}%
\pgfpathlineto{\pgfqpoint{2.949169in}{1.539233in}}%
\pgfpathlineto{\pgfqpoint{2.951676in}{1.541994in}}%
\pgfpathlineto{\pgfqpoint{2.954182in}{1.542350in}}%
\pgfpathlineto{\pgfqpoint{2.964209in}{1.547224in}}%
\pgfpathlineto{\pgfqpoint{2.979249in}{1.553242in}}%
\pgfpathlineto{\pgfqpoint{2.981755in}{1.557034in}}%
\pgfpathlineto{\pgfqpoint{2.984262in}{1.558636in}}%
\pgfpathlineto{\pgfqpoint{2.991782in}{1.559496in}}%
\pgfpathlineto{\pgfqpoint{2.994289in}{1.560792in}}%
\pgfpathlineto{\pgfqpoint{2.996795in}{1.560807in}}%
\pgfpathlineto{\pgfqpoint{2.999302in}{1.562970in}}%
\pgfpathlineto{\pgfqpoint{3.001809in}{1.568778in}}%
\pgfpathlineto{\pgfqpoint{3.004315in}{1.570948in}}%
\pgfpathlineto{\pgfqpoint{3.011835in}{1.572722in}}%
\pgfpathlineto{\pgfqpoint{3.019355in}{1.579862in}}%
\pgfpathlineto{\pgfqpoint{3.021862in}{1.580993in}}%
\pgfpathlineto{\pgfqpoint{3.026875in}{1.584911in}}%
\pgfpathlineto{\pgfqpoint{3.051942in}{1.596092in}}%
\pgfpathlineto{\pgfqpoint{3.054448in}{1.596115in}}%
\pgfpathlineto{\pgfqpoint{3.059462in}{1.599251in}}%
\pgfpathlineto{\pgfqpoint{3.069488in}{1.609055in}}%
\pgfpathlineto{\pgfqpoint{3.071995in}{1.611845in}}%
\pgfpathlineto{\pgfqpoint{3.077008in}{1.612012in}}%
\pgfpathlineto{\pgfqpoint{3.084528in}{1.616764in}}%
\pgfpathlineto{\pgfqpoint{3.089542in}{1.618500in}}%
\pgfpathlineto{\pgfqpoint{3.094555in}{1.622465in}}%
\pgfpathlineto{\pgfqpoint{3.097061in}{1.626086in}}%
\pgfpathlineto{\pgfqpoint{3.099568in}{1.626235in}}%
\pgfpathlineto{\pgfqpoint{3.102075in}{1.628890in}}%
\pgfpathlineto{\pgfqpoint{3.104581in}{1.629474in}}%
\pgfpathlineto{\pgfqpoint{3.107088in}{1.631990in}}%
\pgfpathlineto{\pgfqpoint{3.112101in}{1.632939in}}%
\pgfpathlineto{\pgfqpoint{3.114608in}{1.635125in}}%
\pgfpathlineto{\pgfqpoint{3.132155in}{1.640138in}}%
\pgfpathlineto{\pgfqpoint{3.134661in}{1.641710in}}%
\pgfpathlineto{\pgfqpoint{3.137168in}{1.641850in}}%
\pgfpathlineto{\pgfqpoint{3.154715in}{1.650580in}}%
\pgfpathlineto{\pgfqpoint{3.162234in}{1.652405in}}%
\pgfpathlineto{\pgfqpoint{3.169754in}{1.654657in}}%
\pgfpathlineto{\pgfqpoint{3.174768in}{1.655795in}}%
\pgfpathlineto{\pgfqpoint{3.177274in}{1.655825in}}%
\pgfpathlineto{\pgfqpoint{3.184794in}{1.659215in}}%
\pgfpathlineto{\pgfqpoint{3.189808in}{1.661810in}}%
\pgfpathlineto{\pgfqpoint{3.194821in}{1.661927in}}%
\pgfpathlineto{\pgfqpoint{3.197328in}{1.663529in}}%
\pgfpathlineto{\pgfqpoint{3.202341in}{1.664684in}}%
\pgfpathlineto{\pgfqpoint{3.207354in}{1.666248in}}%
\pgfpathlineto{\pgfqpoint{3.214874in}{1.667108in}}%
\pgfpathlineto{\pgfqpoint{3.217381in}{1.670062in}}%
\pgfpathlineto{\pgfqpoint{3.224901in}{1.670752in}}%
\pgfpathlineto{\pgfqpoint{3.234927in}{1.674358in}}%
\pgfpathlineto{\pgfqpoint{3.237434in}{1.674381in}}%
\pgfpathlineto{\pgfqpoint{3.239941in}{1.678106in}}%
\pgfpathlineto{\pgfqpoint{3.247461in}{1.680950in}}%
\pgfpathlineto{\pgfqpoint{3.252474in}{1.681481in}}%
\pgfpathlineto{\pgfqpoint{3.254981in}{1.683791in}}%
\pgfpathlineto{\pgfqpoint{3.265007in}{1.685397in}}%
\pgfpathlineto{\pgfqpoint{3.267514in}{1.687681in}}%
\pgfpathlineto{\pgfqpoint{3.275034in}{1.689211in}}%
\pgfpathlineto{\pgfqpoint{3.282554in}{1.692255in}}%
\pgfpathlineto{\pgfqpoint{3.297594in}{1.694619in}}%
\pgfpathlineto{\pgfqpoint{3.302607in}{1.695558in}}%
\pgfpathlineto{\pgfqpoint{3.307620in}{1.696397in}}%
\pgfpathlineto{\pgfqpoint{3.312634in}{1.698983in}}%
\pgfpathlineto{\pgfqpoint{3.315140in}{1.699074in}}%
\pgfpathlineto{\pgfqpoint{3.317647in}{1.700755in}}%
\pgfpathlineto{\pgfqpoint{3.322660in}{1.701502in}}%
\pgfpathlineto{\pgfqpoint{3.327674in}{1.703238in}}%
\pgfpathlineto{\pgfqpoint{3.335194in}{1.704327in}}%
\pgfpathlineto{\pgfqpoint{3.347727in}{1.706343in}}%
\pgfpathlineto{\pgfqpoint{3.352740in}{1.708031in}}%
\pgfpathlineto{\pgfqpoint{3.357754in}{1.709275in}}%
\pgfpathlineto{\pgfqpoint{3.360260in}{1.709311in}}%
\pgfpathlineto{\pgfqpoint{3.362767in}{1.710903in}}%
\pgfpathlineto{\pgfqpoint{3.372793in}{1.712479in}}%
\pgfpathlineto{\pgfqpoint{3.377807in}{1.714572in}}%
\pgfpathlineto{\pgfqpoint{3.380313in}{1.714705in}}%
\pgfpathlineto{\pgfqpoint{3.385327in}{1.716868in}}%
\pgfpathlineto{\pgfqpoint{3.400367in}{1.720052in}}%
\pgfpathlineto{\pgfqpoint{3.407887in}{1.722406in}}%
\pgfpathlineto{\pgfqpoint{3.427940in}{1.726615in}}%
\pgfpathlineto{\pgfqpoint{3.432953in}{1.728917in}}%
\pgfpathlineto{\pgfqpoint{3.440473in}{1.731357in}}%
\pgfpathlineto{\pgfqpoint{3.458020in}{1.732952in}}%
\pgfpathlineto{\pgfqpoint{3.460526in}{1.734663in}}%
\pgfpathlineto{\pgfqpoint{3.468046in}{1.735765in}}%
\pgfpathlineto{\pgfqpoint{3.473060in}{1.736625in}}%
\pgfpathlineto{\pgfqpoint{3.478073in}{1.736895in}}%
\pgfpathlineto{\pgfqpoint{3.483086in}{1.738206in}}%
\pgfpathlineto{\pgfqpoint{3.488100in}{1.739323in}}%
\pgfpathlineto{\pgfqpoint{3.500633in}{1.741510in}}%
\pgfpathlineto{\pgfqpoint{3.505646in}{1.744801in}}%
\pgfpathlineto{\pgfqpoint{3.515673in}{1.746206in}}%
\pgfpathlineto{\pgfqpoint{3.525699in}{1.748423in}}%
\pgfpathlineto{\pgfqpoint{3.540739in}{1.750291in}}%
\pgfpathlineto{\pgfqpoint{3.563299in}{1.756929in}}%
\pgfpathlineto{\pgfqpoint{3.568313in}{1.760389in}}%
\pgfpathlineto{\pgfqpoint{3.600899in}{1.770524in}}%
\pgfpathlineto{\pgfqpoint{3.603406in}{1.771511in}}%
\pgfpathlineto{\pgfqpoint{3.605912in}{1.775267in}}%
\pgfpathlineto{\pgfqpoint{3.613432in}{1.776331in}}%
\pgfpathlineto{\pgfqpoint{3.623459in}{1.777098in}}%
\pgfpathlineto{\pgfqpoint{3.635992in}{1.780931in}}%
\pgfpathlineto{\pgfqpoint{3.643512in}{1.781710in}}%
\pgfpathlineto{\pgfqpoint{3.653539in}{1.783605in}}%
\pgfpathlineto{\pgfqpoint{3.658552in}{1.784774in}}%
\pgfpathlineto{\pgfqpoint{3.666072in}{1.787717in}}%
\pgfpathlineto{\pgfqpoint{3.668579in}{1.787813in}}%
\pgfpathlineto{\pgfqpoint{3.673592in}{1.791098in}}%
\pgfpathlineto{\pgfqpoint{3.678605in}{1.791781in}}%
\pgfpathlineto{\pgfqpoint{3.686125in}{1.793713in}}%
\pgfpathlineto{\pgfqpoint{3.691139in}{1.794045in}}%
\pgfpathlineto{\pgfqpoint{3.696152in}{1.796306in}}%
\pgfpathlineto{\pgfqpoint{3.701165in}{1.797694in}}%
\pgfpathlineto{\pgfqpoint{3.718712in}{1.800150in}}%
\pgfpathlineto{\pgfqpoint{3.728738in}{1.800465in}}%
\pgfpathlineto{\pgfqpoint{3.731245in}{1.802912in}}%
\pgfpathlineto{\pgfqpoint{3.748792in}{1.808090in}}%
\pgfpathlineto{\pgfqpoint{3.751298in}{1.810626in}}%
\pgfpathlineto{\pgfqpoint{3.753805in}{1.810875in}}%
\pgfpathlineto{\pgfqpoint{3.756312in}{1.812637in}}%
\pgfpathlineto{\pgfqpoint{3.761325in}{1.813391in}}%
\pgfpathlineto{\pgfqpoint{3.766338in}{1.814754in}}%
\pgfpathlineto{\pgfqpoint{3.776365in}{1.817523in}}%
\pgfpathlineto{\pgfqpoint{3.783885in}{1.821661in}}%
\pgfpathlineto{\pgfqpoint{3.788898in}{1.825173in}}%
\pgfpathlineto{\pgfqpoint{3.808951in}{1.827698in}}%
\pgfpathlineto{\pgfqpoint{3.811458in}{1.830228in}}%
\pgfpathlineto{\pgfqpoint{3.816471in}{1.831262in}}%
\pgfpathlineto{\pgfqpoint{3.818978in}{1.833689in}}%
\pgfpathlineto{\pgfqpoint{3.839031in}{1.839512in}}%
\pgfpathlineto{\pgfqpoint{3.849058in}{1.841144in}}%
\pgfpathlineto{\pgfqpoint{3.861591in}{1.844081in}}%
\pgfpathlineto{\pgfqpoint{3.881644in}{1.846658in}}%
\pgfpathlineto{\pgfqpoint{3.884151in}{1.848176in}}%
\pgfpathlineto{\pgfqpoint{3.891671in}{1.848650in}}%
\pgfpathlineto{\pgfqpoint{3.899191in}{1.853127in}}%
\pgfpathlineto{\pgfqpoint{3.914231in}{1.857030in}}%
\pgfpathlineto{\pgfqpoint{3.919244in}{1.859354in}}%
\pgfpathlineto{\pgfqpoint{3.924257in}{1.860141in}}%
\pgfpathlineto{\pgfqpoint{3.929271in}{1.863606in}}%
\pgfpathlineto{\pgfqpoint{3.931777in}{1.865768in}}%
\pgfpathlineto{\pgfqpoint{3.939297in}{1.867660in}}%
\pgfpathlineto{\pgfqpoint{3.941804in}{1.869463in}}%
\pgfpathlineto{\pgfqpoint{3.944311in}{1.869515in}}%
\pgfpathlineto{\pgfqpoint{3.946817in}{1.871144in}}%
\pgfpathlineto{\pgfqpoint{3.949324in}{1.871149in}}%
\pgfpathlineto{\pgfqpoint{3.956844in}{1.874532in}}%
\pgfpathlineto{\pgfqpoint{3.959351in}{1.874563in}}%
\pgfpathlineto{\pgfqpoint{3.964364in}{1.876668in}}%
\pgfpathlineto{\pgfqpoint{3.969377in}{1.878227in}}%
\pgfpathlineto{\pgfqpoint{3.991937in}{1.880896in}}%
\pgfpathlineto{\pgfqpoint{4.039564in}{1.894785in}}%
\pgfpathlineto{\pgfqpoint{4.052097in}{1.897871in}}%
\pgfpathlineto{\pgfqpoint{4.054603in}{1.898351in}}%
\pgfpathlineto{\pgfqpoint{4.059617in}{1.900628in}}%
\pgfpathlineto{\pgfqpoint{4.062123in}{1.901090in}}%
\pgfpathlineto{\pgfqpoint{4.064630in}{1.903033in}}%
\pgfpathlineto{\pgfqpoint{4.069643in}{1.903268in}}%
\pgfpathlineto{\pgfqpoint{4.074657in}{1.905313in}}%
\pgfpathlineto{\pgfqpoint{4.089697in}{1.911185in}}%
\pgfpathlineto{\pgfqpoint{4.099723in}{1.914275in}}%
\pgfpathlineto{\pgfqpoint{4.107243in}{1.918812in}}%
\pgfpathlineto{\pgfqpoint{4.117270in}{1.920487in}}%
\pgfpathlineto{\pgfqpoint{4.119776in}{1.922654in}}%
\pgfpathlineto{\pgfqpoint{4.127296in}{1.924096in}}%
\pgfpathlineto{\pgfqpoint{4.132310in}{1.925382in}}%
\pgfpathlineto{\pgfqpoint{4.139830in}{1.926804in}}%
\pgfpathlineto{\pgfqpoint{4.147350in}{1.927444in}}%
\pgfpathlineto{\pgfqpoint{4.149856in}{1.928977in}}%
\pgfpathlineto{\pgfqpoint{4.152363in}{1.929152in}}%
\pgfpathlineto{\pgfqpoint{4.159883in}{1.932868in}}%
\pgfpathlineto{\pgfqpoint{4.172416in}{1.935506in}}%
\pgfpathlineto{\pgfqpoint{4.174923in}{1.936004in}}%
\pgfpathlineto{\pgfqpoint{4.177430in}{1.943511in}}%
\pgfpathlineto{\pgfqpoint{4.187456in}{1.948477in}}%
\pgfpathlineto{\pgfqpoint{4.194976in}{1.950361in}}%
\pgfpathlineto{\pgfqpoint{4.202496in}{1.956924in}}%
\pgfpathlineto{\pgfqpoint{4.215029in}{1.961449in}}%
\pgfpathlineto{\pgfqpoint{4.217536in}{1.965826in}}%
\pgfpathlineto{\pgfqpoint{4.230069in}{1.969890in}}%
\pgfpathlineto{\pgfqpoint{4.232576in}{1.972688in}}%
\pgfpathlineto{\pgfqpoint{4.237589in}{1.972828in}}%
\pgfpathlineto{\pgfqpoint{4.250122in}{1.978752in}}%
\pgfpathlineto{\pgfqpoint{4.252629in}{1.978948in}}%
\pgfpathlineto{\pgfqpoint{4.255136in}{1.981909in}}%
\pgfpathlineto{\pgfqpoint{4.265162in}{1.983439in}}%
\pgfpathlineto{\pgfqpoint{4.267669in}{1.985321in}}%
\pgfpathlineto{\pgfqpoint{4.270176in}{1.985382in}}%
\pgfpathlineto{\pgfqpoint{4.272682in}{1.987252in}}%
\pgfpathlineto{\pgfqpoint{4.280202in}{1.989086in}}%
\pgfpathlineto{\pgfqpoint{4.282709in}{1.991979in}}%
\pgfpathlineto{\pgfqpoint{4.287722in}{1.993104in}}%
\pgfpathlineto{\pgfqpoint{4.310282in}{1.998760in}}%
\pgfpathlineto{\pgfqpoint{4.315296in}{2.000400in}}%
\pgfpathlineto{\pgfqpoint{4.322815in}{2.001713in}}%
\pgfpathlineto{\pgfqpoint{4.325322in}{2.002102in}}%
\pgfpathlineto{\pgfqpoint{4.332842in}{2.008012in}}%
\pgfpathlineto{\pgfqpoint{4.342869in}{2.010145in}}%
\pgfpathlineto{\pgfqpoint{4.347882in}{2.014909in}}%
\pgfpathlineto{\pgfqpoint{4.355402in}{2.015922in}}%
\pgfpathlineto{\pgfqpoint{4.357909in}{2.017690in}}%
\pgfpathlineto{\pgfqpoint{4.362922in}{2.018859in}}%
\pgfpathlineto{\pgfqpoint{4.367935in}{2.020519in}}%
\pgfpathlineto{\pgfqpoint{4.372949in}{2.021502in}}%
\pgfpathlineto{\pgfqpoint{4.377962in}{2.024767in}}%
\pgfpathlineto{\pgfqpoint{4.387988in}{2.027103in}}%
\pgfpathlineto{\pgfqpoint{4.393002in}{2.028015in}}%
\pgfpathlineto{\pgfqpoint{4.403028in}{2.031449in}}%
\pgfpathlineto{\pgfqpoint{4.405535in}{2.033681in}}%
\pgfpathlineto{\pgfqpoint{4.408042in}{2.033848in}}%
\pgfpathlineto{\pgfqpoint{4.415562in}{2.038448in}}%
\pgfpathlineto{\pgfqpoint{4.423082in}{2.040496in}}%
\pgfpathlineto{\pgfqpoint{4.428095in}{2.044277in}}%
\pgfpathlineto{\pgfqpoint{4.430602in}{2.044692in}}%
\pgfpathlineto{\pgfqpoint{4.433108in}{2.046496in}}%
\pgfpathlineto{\pgfqpoint{4.435615in}{2.046498in}}%
\pgfpathlineto{\pgfqpoint{4.438122in}{2.047843in}}%
\pgfpathlineto{\pgfqpoint{4.440628in}{2.054156in}}%
\pgfpathlineto{\pgfqpoint{4.445642in}{2.054653in}}%
\pgfpathlineto{\pgfqpoint{4.448148in}{2.056226in}}%
\pgfpathlineto{\pgfqpoint{4.450655in}{2.060860in}}%
\pgfpathlineto{\pgfqpoint{4.455668in}{2.062163in}}%
\pgfpathlineto{\pgfqpoint{4.460681in}{2.062908in}}%
\pgfpathlineto{\pgfqpoint{4.463188in}{2.066461in}}%
\pgfpathlineto{\pgfqpoint{4.465695in}{2.067757in}}%
\pgfpathlineto{\pgfqpoint{4.468201in}{2.072530in}}%
\pgfpathlineto{\pgfqpoint{4.473215in}{2.073330in}}%
\pgfpathlineto{\pgfqpoint{4.475721in}{2.077994in}}%
\pgfpathlineto{\pgfqpoint{4.493268in}{2.082183in}}%
\pgfpathlineto{\pgfqpoint{4.510815in}{2.090978in}}%
\pgfpathlineto{\pgfqpoint{4.518335in}{2.094505in}}%
\pgfpathlineto{\pgfqpoint{4.525854in}{2.096653in}}%
\pgfpathlineto{\pgfqpoint{4.528361in}{2.097189in}}%
\pgfpathlineto{\pgfqpoint{4.530868in}{2.099641in}}%
\pgfpathlineto{\pgfqpoint{4.535881in}{2.100654in}}%
\pgfpathlineto{\pgfqpoint{4.545908in}{2.104023in}}%
\pgfpathlineto{\pgfqpoint{4.550921in}{2.104439in}}%
\pgfpathlineto{\pgfqpoint{4.555934in}{2.109507in}}%
\pgfpathlineto{\pgfqpoint{4.581001in}{2.116132in}}%
\pgfpathlineto{\pgfqpoint{4.583508in}{2.125291in}}%
\pgfpathlineto{\pgfqpoint{4.588521in}{2.126516in}}%
\pgfpathlineto{\pgfqpoint{4.593534in}{2.127140in}}%
\pgfpathlineto{\pgfqpoint{4.601054in}{2.134401in}}%
\pgfpathlineto{\pgfqpoint{4.606067in}{2.135267in}}%
\pgfpathlineto{\pgfqpoint{4.608574in}{2.141953in}}%
\pgfpathlineto{\pgfqpoint{4.618601in}{2.145534in}}%
\pgfpathlineto{\pgfqpoint{4.621107in}{2.146968in}}%
\pgfpathlineto{\pgfqpoint{4.626121in}{2.157121in}}%
\pgfpathlineto{\pgfqpoint{4.631134in}{2.158238in}}%
\pgfpathlineto{\pgfqpoint{4.636147in}{2.158929in}}%
\pgfpathlineto{\pgfqpoint{4.638654in}{2.161992in}}%
\pgfpathlineto{\pgfqpoint{4.643667in}{2.163374in}}%
\pgfpathlineto{\pgfqpoint{4.646174in}{2.167144in}}%
\pgfpathlineto{\pgfqpoint{4.648681in}{2.173852in}}%
\pgfpathlineto{\pgfqpoint{4.666227in}{2.181244in}}%
\pgfpathlineto{\pgfqpoint{4.668734in}{2.189981in}}%
\pgfpathlineto{\pgfqpoint{4.676254in}{2.195712in}}%
\pgfpathlineto{\pgfqpoint{4.678760in}{2.198852in}}%
\pgfpathlineto{\pgfqpoint{4.681267in}{2.199498in}}%
\pgfpathlineto{\pgfqpoint{4.686280in}{2.206474in}}%
\pgfpathlineto{\pgfqpoint{4.691294in}{2.207724in}}%
\pgfpathlineto{\pgfqpoint{4.696307in}{2.209200in}}%
\pgfpathlineto{\pgfqpoint{4.698814in}{2.209483in}}%
\pgfpathlineto{\pgfqpoint{4.703827in}{2.212478in}}%
\pgfpathlineto{\pgfqpoint{4.708840in}{2.213103in}}%
\pgfpathlineto{\pgfqpoint{4.711347in}{2.218830in}}%
\pgfpathlineto{\pgfqpoint{4.713854in}{2.219216in}}%
\pgfpathlineto{\pgfqpoint{4.716360in}{2.220938in}}%
\pgfpathlineto{\pgfqpoint{4.721374in}{2.222367in}}%
\pgfpathlineto{\pgfqpoint{4.723880in}{2.225453in}}%
\pgfpathlineto{\pgfqpoint{4.726387in}{2.225648in}}%
\pgfpathlineto{\pgfqpoint{4.728893in}{2.230817in}}%
\pgfpathlineto{\pgfqpoint{4.731400in}{2.231601in}}%
\pgfpathlineto{\pgfqpoint{4.738920in}{2.236745in}}%
\pgfpathlineto{\pgfqpoint{4.746440in}{2.237414in}}%
\pgfpathlineto{\pgfqpoint{4.751453in}{2.241435in}}%
\pgfpathlineto{\pgfqpoint{4.771507in}{2.262692in}}%
\pgfpathlineto{\pgfqpoint{4.791560in}{2.269946in}}%
\pgfpathlineto{\pgfqpoint{4.794066in}{2.274799in}}%
\pgfpathlineto{\pgfqpoint{4.801586in}{2.275557in}}%
\pgfpathlineto{\pgfqpoint{4.804093in}{2.282266in}}%
\pgfpathlineto{\pgfqpoint{4.811613in}{2.285432in}}%
\pgfpathlineto{\pgfqpoint{4.816626in}{2.287022in}}%
\pgfpathlineto{\pgfqpoint{4.824146in}{2.292735in}}%
\pgfpathlineto{\pgfqpoint{4.831666in}{2.308563in}}%
\pgfpathlineto{\pgfqpoint{4.834173in}{2.311134in}}%
\pgfpathlineto{\pgfqpoint{4.841693in}{2.313390in}}%
\pgfpathlineto{\pgfqpoint{4.844200in}{2.314175in}}%
\pgfpathlineto{\pgfqpoint{4.846706in}{2.316492in}}%
\pgfpathlineto{\pgfqpoint{4.849213in}{2.317156in}}%
\pgfpathlineto{\pgfqpoint{4.854226in}{2.324296in}}%
\pgfpathlineto{\pgfqpoint{4.856733in}{2.325323in}}%
\pgfpathlineto{\pgfqpoint{4.859240in}{2.327872in}}%
\pgfpathlineto{\pgfqpoint{4.861746in}{2.328021in}}%
\pgfpathlineto{\pgfqpoint{4.864253in}{2.332448in}}%
\pgfpathlineto{\pgfqpoint{4.869266in}{2.333321in}}%
\pgfpathlineto{\pgfqpoint{4.881799in}{2.346620in}}%
\pgfpathlineto{\pgfqpoint{4.884306in}{2.346647in}}%
\pgfpathlineto{\pgfqpoint{4.896839in}{2.352426in}}%
\pgfpathlineto{\pgfqpoint{4.904359in}{2.362995in}}%
\pgfpathlineto{\pgfqpoint{4.911879in}{2.364687in}}%
\pgfpathlineto{\pgfqpoint{4.924413in}{2.372463in}}%
\pgfpathlineto{\pgfqpoint{4.929426in}{2.379018in}}%
\pgfpathlineto{\pgfqpoint{4.931932in}{2.379554in}}%
\pgfpathlineto{\pgfqpoint{4.936946in}{2.388595in}}%
\pgfpathlineto{\pgfqpoint{4.946972in}{2.391255in}}%
\pgfpathlineto{\pgfqpoint{4.951986in}{2.395940in}}%
\pgfpathlineto{\pgfqpoint{4.954492in}{2.396006in}}%
\pgfpathlineto{\pgfqpoint{4.956999in}{2.398485in}}%
\pgfpathlineto{\pgfqpoint{4.969532in}{2.401852in}}%
\pgfpathlineto{\pgfqpoint{4.972039in}{2.403703in}}%
\pgfpathlineto{\pgfqpoint{4.977052in}{2.410470in}}%
\pgfpathlineto{\pgfqpoint{4.979559in}{2.413442in}}%
\pgfpathlineto{\pgfqpoint{4.982066in}{2.414399in}}%
\pgfpathlineto{\pgfqpoint{4.984572in}{2.420790in}}%
\pgfpathlineto{\pgfqpoint{4.987079in}{2.422086in}}%
\pgfpathlineto{\pgfqpoint{4.989586in}{2.424718in}}%
\pgfpathlineto{\pgfqpoint{4.992092in}{2.424831in}}%
\pgfpathlineto{\pgfqpoint{4.997105in}{2.429578in}}%
\pgfpathlineto{\pgfqpoint{4.999612in}{2.430680in}}%
\pgfpathlineto{\pgfqpoint{5.004625in}{2.437224in}}%
\pgfpathlineto{\pgfqpoint{5.007132in}{2.437434in}}%
\pgfpathlineto{\pgfqpoint{5.012145in}{2.445046in}}%
\pgfpathlineto{\pgfqpoint{5.014652in}{2.448608in}}%
\pgfpathlineto{\pgfqpoint{5.017159in}{2.448839in}}%
\pgfpathlineto{\pgfqpoint{5.019665in}{2.452410in}}%
\pgfpathlineto{\pgfqpoint{5.027185in}{2.455951in}}%
\pgfpathlineto{\pgfqpoint{5.029692in}{2.456342in}}%
\pgfpathlineto{\pgfqpoint{5.042225in}{2.466303in}}%
\pgfpathlineto{\pgfqpoint{5.044732in}{2.472230in}}%
\pgfpathlineto{\pgfqpoint{5.047239in}{2.472674in}}%
\pgfpathlineto{\pgfqpoint{5.049745in}{2.478720in}}%
\pgfpathlineto{\pgfqpoint{5.054759in}{2.481085in}}%
\pgfpathlineto{\pgfqpoint{5.062279in}{2.508221in}}%
\pgfpathlineto{\pgfqpoint{5.064785in}{2.511046in}}%
\pgfpathlineto{\pgfqpoint{5.072305in}{2.530302in}}%
\pgfpathlineto{\pgfqpoint{5.077318in}{2.531291in}}%
\pgfpathlineto{\pgfqpoint{5.087345in}{2.542890in}}%
\pgfpathlineto{\pgfqpoint{5.089852in}{2.549327in}}%
\pgfpathlineto{\pgfqpoint{5.099878in}{2.552251in}}%
\pgfpathlineto{\pgfqpoint{5.104892in}{2.556140in}}%
\pgfpathlineto{\pgfqpoint{5.107398in}{2.557611in}}%
\pgfpathlineto{\pgfqpoint{5.109905in}{2.560991in}}%
\pgfpathlineto{\pgfqpoint{5.112412in}{2.561478in}}%
\pgfpathlineto{\pgfqpoint{5.119932in}{2.577856in}}%
\pgfpathlineto{\pgfqpoint{5.122438in}{2.580029in}}%
\pgfpathlineto{\pgfqpoint{5.129958in}{2.612282in}}%
\pgfpathlineto{\pgfqpoint{5.134971in}{2.618083in}}%
\pgfpathlineto{\pgfqpoint{5.139985in}{2.620472in}}%
\pgfpathlineto{\pgfqpoint{5.142491in}{2.624230in}}%
\pgfpathlineto{\pgfqpoint{5.147505in}{2.626079in}}%
\pgfpathlineto{\pgfqpoint{5.150011in}{2.628945in}}%
\pgfpathlineto{\pgfqpoint{5.157531in}{2.651122in}}%
\pgfpathlineto{\pgfqpoint{5.160038in}{2.651219in}}%
\pgfpathlineto{\pgfqpoint{5.162545in}{2.654031in}}%
\pgfpathlineto{\pgfqpoint{5.167558in}{2.654689in}}%
\pgfpathlineto{\pgfqpoint{5.170065in}{2.657652in}}%
\pgfpathlineto{\pgfqpoint{5.172571in}{2.657985in}}%
\pgfpathlineto{\pgfqpoint{5.175078in}{2.659534in}}%
\pgfpathlineto{\pgfqpoint{5.177585in}{2.666649in}}%
\pgfpathlineto{\pgfqpoint{5.180091in}{2.669605in}}%
\pgfpathlineto{\pgfqpoint{5.182598in}{2.686827in}}%
\pgfpathlineto{\pgfqpoint{5.185105in}{2.694173in}}%
\pgfpathlineto{\pgfqpoint{5.187611in}{2.697286in}}%
\pgfpathlineto{\pgfqpoint{5.190118in}{2.703155in}}%
\pgfpathlineto{\pgfqpoint{5.192625in}{2.705426in}}%
\pgfpathlineto{\pgfqpoint{5.195131in}{2.721867in}}%
\pgfpathlineto{\pgfqpoint{5.197638in}{2.723482in}}%
\pgfpathlineto{\pgfqpoint{5.202651in}{2.749028in}}%
\pgfpathlineto{\pgfqpoint{5.205158in}{2.752205in}}%
\pgfpathlineto{\pgfqpoint{5.207664in}{2.768750in}}%
\pgfpathlineto{\pgfqpoint{5.210171in}{2.769852in}}%
\pgfpathlineto{\pgfqpoint{5.212678in}{2.776110in}}%
\pgfpathlineto{\pgfqpoint{5.217691in}{2.781791in}}%
\pgfpathlineto{\pgfqpoint{5.222704in}{2.801004in}}%
\pgfpathlineto{\pgfqpoint{5.225211in}{2.803008in}}%
\pgfpathlineto{\pgfqpoint{5.230224in}{2.809742in}}%
\pgfpathlineto{\pgfqpoint{5.232731in}{2.809782in}}%
\pgfpathlineto{\pgfqpoint{5.235238in}{2.821501in}}%
\pgfpathlineto{\pgfqpoint{5.237744in}{2.822411in}}%
\pgfpathlineto{\pgfqpoint{5.240251in}{2.837206in}}%
\pgfpathlineto{\pgfqpoint{5.245264in}{2.844916in}}%
\pgfpathlineto{\pgfqpoint{5.247771in}{2.846976in}}%
\pgfpathlineto{\pgfqpoint{5.250278in}{2.852035in}}%
\pgfpathlineto{\pgfqpoint{5.255291in}{2.856223in}}%
\pgfpathlineto{\pgfqpoint{5.257798in}{2.856457in}}%
\pgfpathlineto{\pgfqpoint{5.262811in}{2.886290in}}%
\pgfpathlineto{\pgfqpoint{5.275344in}{2.894110in}}%
\pgfpathlineto{\pgfqpoint{5.277851in}{2.905275in}}%
\pgfpathlineto{\pgfqpoint{5.277851in}{2.905275in}}%
\pgfusepath{stroke}%
\end{pgfscope}%
\begin{pgfscope}%
\pgfpathrectangle{\pgfqpoint{0.708220in}{0.535823in}}{\pgfqpoint{5.013309in}{2.369453in}}%
\pgfusepath{clip}%
\pgfsetbuttcap%
\pgfsetroundjoin%
\pgfsetlinewidth{1.003750pt}%
\definecolor{currentstroke}{rgb}{0.000000,0.000000,0.000000}%
\pgfsetstrokecolor{currentstroke}%
\pgfsetdash{{1.000000pt}{1.650000pt}}{0.000000pt}%
\pgfpathmoveto{\pgfqpoint{0.713233in}{0.531304in}}%
\pgfpathlineto{\pgfqpoint{0.720753in}{0.532358in}}%
\pgfpathlineto{\pgfqpoint{0.735793in}{0.536678in}}%
\pgfpathlineto{\pgfqpoint{0.738300in}{0.536849in}}%
\pgfpathlineto{\pgfqpoint{0.740806in}{0.538545in}}%
\pgfpathlineto{\pgfqpoint{0.745820in}{0.538882in}}%
\pgfpathlineto{\pgfqpoint{0.753340in}{0.540559in}}%
\pgfpathlineto{\pgfqpoint{0.755846in}{0.544027in}}%
\pgfpathlineto{\pgfqpoint{0.758353in}{0.647790in}}%
\pgfpathlineto{\pgfqpoint{0.760860in}{0.648592in}}%
\pgfpathlineto{\pgfqpoint{0.763366in}{0.650711in}}%
\pgfpathlineto{\pgfqpoint{0.785926in}{0.652978in}}%
\pgfpathlineto{\pgfqpoint{0.795953in}{0.653928in}}%
\pgfpathlineto{\pgfqpoint{0.826033in}{0.658182in}}%
\pgfpathlineto{\pgfqpoint{0.841073in}{0.660021in}}%
\pgfpathlineto{\pgfqpoint{0.851099in}{0.664453in}}%
\pgfpathlineto{\pgfqpoint{0.853606in}{0.669953in}}%
\pgfpathlineto{\pgfqpoint{0.856112in}{0.715565in}}%
\pgfpathlineto{\pgfqpoint{0.858619in}{0.718664in}}%
\pgfpathlineto{\pgfqpoint{0.861126in}{0.719371in}}%
\pgfpathlineto{\pgfqpoint{0.866139in}{0.722461in}}%
\pgfpathlineto{\pgfqpoint{0.878672in}{0.724242in}}%
\pgfpathlineto{\pgfqpoint{0.911259in}{0.730638in}}%
\pgfpathlineto{\pgfqpoint{0.913766in}{0.731955in}}%
\pgfpathlineto{\pgfqpoint{0.916272in}{0.738865in}}%
\pgfpathlineto{\pgfqpoint{0.918779in}{0.767471in}}%
\pgfpathlineto{\pgfqpoint{0.966405in}{0.774095in}}%
\pgfpathlineto{\pgfqpoint{0.988965in}{0.775585in}}%
\pgfpathlineto{\pgfqpoint{0.996485in}{0.776852in}}%
\pgfpathlineto{\pgfqpoint{1.006512in}{0.777900in}}%
\pgfpathlineto{\pgfqpoint{1.014032in}{0.780432in}}%
\pgfpathlineto{\pgfqpoint{1.029072in}{0.783008in}}%
\pgfpathlineto{\pgfqpoint{1.031578in}{0.806487in}}%
\pgfpathlineto{\pgfqpoint{1.046618in}{0.808562in}}%
\pgfpathlineto{\pgfqpoint{1.071685in}{0.811302in}}%
\pgfpathlineto{\pgfqpoint{1.101765in}{0.813116in}}%
\pgfpathlineto{\pgfqpoint{1.114298in}{0.814169in}}%
\pgfpathlineto{\pgfqpoint{1.134351in}{0.815752in}}%
\pgfpathlineto{\pgfqpoint{1.139364in}{0.815953in}}%
\pgfpathlineto{\pgfqpoint{1.146884in}{0.819368in}}%
\pgfpathlineto{\pgfqpoint{1.154404in}{0.820644in}}%
\pgfpathlineto{\pgfqpoint{1.156911in}{0.824892in}}%
\pgfpathlineto{\pgfqpoint{1.159418in}{0.826915in}}%
\pgfpathlineto{\pgfqpoint{1.161924in}{0.835882in}}%
\pgfpathlineto{\pgfqpoint{1.164431in}{0.837956in}}%
\pgfpathlineto{\pgfqpoint{1.174458in}{0.839423in}}%
\pgfpathlineto{\pgfqpoint{1.202031in}{0.841224in}}%
\pgfpathlineto{\pgfqpoint{1.217071in}{0.842520in}}%
\pgfpathlineto{\pgfqpoint{1.294777in}{0.850040in}}%
\pgfpathlineto{\pgfqpoint{1.299790in}{0.853707in}}%
\pgfpathlineto{\pgfqpoint{1.302297in}{0.863169in}}%
\pgfpathlineto{\pgfqpoint{1.307310in}{0.864386in}}%
\pgfpathlineto{\pgfqpoint{1.317337in}{0.867390in}}%
\pgfpathlineto{\pgfqpoint{1.342403in}{0.869682in}}%
\pgfpathlineto{\pgfqpoint{1.352430in}{0.870195in}}%
\pgfpathlineto{\pgfqpoint{1.374990in}{0.871750in}}%
\pgfpathlineto{\pgfqpoint{1.397550in}{0.874510in}}%
\pgfpathlineto{\pgfqpoint{1.400056in}{0.878602in}}%
\pgfpathlineto{\pgfqpoint{1.402563in}{0.887082in}}%
\pgfpathlineto{\pgfqpoint{1.427630in}{0.893976in}}%
\pgfpathlineto{\pgfqpoint{1.432643in}{0.895331in}}%
\pgfpathlineto{\pgfqpoint{1.437656in}{0.896565in}}%
\pgfpathlineto{\pgfqpoint{1.442670in}{0.898485in}}%
\pgfpathlineto{\pgfqpoint{1.445176in}{0.909158in}}%
\pgfpathlineto{\pgfqpoint{1.452696in}{0.910205in}}%
\pgfpathlineto{\pgfqpoint{1.455203in}{0.910494in}}%
\pgfpathlineto{\pgfqpoint{1.460216in}{0.913175in}}%
\pgfpathlineto{\pgfqpoint{1.465229in}{0.913327in}}%
\pgfpathlineto{\pgfqpoint{1.470243in}{0.915665in}}%
\pgfpathlineto{\pgfqpoint{1.477763in}{0.916264in}}%
\pgfpathlineto{\pgfqpoint{1.480269in}{0.918948in}}%
\pgfpathlineto{\pgfqpoint{1.485283in}{0.919717in}}%
\pgfpathlineto{\pgfqpoint{1.487789in}{0.920957in}}%
\pgfpathlineto{\pgfqpoint{1.490296in}{0.926196in}}%
\pgfpathlineto{\pgfqpoint{1.515363in}{0.930628in}}%
\pgfpathlineto{\pgfqpoint{1.525389in}{0.931894in}}%
\pgfpathlineto{\pgfqpoint{1.530402in}{0.933504in}}%
\pgfpathlineto{\pgfqpoint{1.537922in}{0.934111in}}%
\pgfpathlineto{\pgfqpoint{1.540429in}{0.943071in}}%
\pgfpathlineto{\pgfqpoint{1.545442in}{0.944991in}}%
\pgfpathlineto{\pgfqpoint{1.550456in}{0.946876in}}%
\pgfpathlineto{\pgfqpoint{1.552962in}{0.948321in}}%
\pgfpathlineto{\pgfqpoint{1.560482in}{0.949139in}}%
\pgfpathlineto{\pgfqpoint{1.562989in}{0.951415in}}%
\pgfpathlineto{\pgfqpoint{1.568002in}{0.952948in}}%
\pgfpathlineto{\pgfqpoint{1.570509in}{0.958332in}}%
\pgfpathlineto{\pgfqpoint{1.593069in}{0.963041in}}%
\pgfpathlineto{\pgfqpoint{1.605602in}{0.964723in}}%
\pgfpathlineto{\pgfqpoint{1.608109in}{0.974005in}}%
\pgfpathlineto{\pgfqpoint{1.620642in}{0.975253in}}%
\pgfpathlineto{\pgfqpoint{1.625655in}{0.976990in}}%
\pgfpathlineto{\pgfqpoint{1.628162in}{0.979422in}}%
\pgfpathlineto{\pgfqpoint{1.630669in}{0.979989in}}%
\pgfpathlineto{\pgfqpoint{1.638189in}{0.984711in}}%
\pgfpathlineto{\pgfqpoint{1.643202in}{0.985946in}}%
\pgfpathlineto{\pgfqpoint{1.648215in}{0.987024in}}%
\pgfpathlineto{\pgfqpoint{1.660749in}{0.987813in}}%
\pgfpathlineto{\pgfqpoint{1.665762in}{0.989440in}}%
\pgfpathlineto{\pgfqpoint{1.670775in}{0.990315in}}%
\pgfpathlineto{\pgfqpoint{1.675788in}{0.991655in}}%
\pgfpathlineto{\pgfqpoint{1.678295in}{0.992590in}}%
\pgfpathlineto{\pgfqpoint{1.683308in}{0.996082in}}%
\pgfpathlineto{\pgfqpoint{1.690828in}{0.997505in}}%
\pgfpathlineto{\pgfqpoint{1.693335in}{0.999882in}}%
\pgfpathlineto{\pgfqpoint{1.705868in}{1.001400in}}%
\pgfpathlineto{\pgfqpoint{1.708375in}{1.004629in}}%
\pgfpathlineto{\pgfqpoint{1.713388in}{1.006039in}}%
\pgfpathlineto{\pgfqpoint{1.718402in}{1.007240in}}%
\pgfpathlineto{\pgfqpoint{1.723415in}{1.009129in}}%
\pgfpathlineto{\pgfqpoint{1.735948in}{1.011340in}}%
\pgfpathlineto{\pgfqpoint{1.748481in}{1.012072in}}%
\pgfpathlineto{\pgfqpoint{1.753495in}{1.014126in}}%
\pgfpathlineto{\pgfqpoint{1.756001in}{1.018468in}}%
\pgfpathlineto{\pgfqpoint{1.768535in}{1.019320in}}%
\pgfpathlineto{\pgfqpoint{1.773548in}{1.020898in}}%
\pgfpathlineto{\pgfqpoint{1.841228in}{1.021738in}}%
\pgfpathlineto{\pgfqpoint{1.848748in}{1.022623in}}%
\pgfpathlineto{\pgfqpoint{1.853761in}{1.023684in}}%
\pgfpathlineto{\pgfqpoint{1.858774in}{1.025421in}}%
\pgfpathlineto{\pgfqpoint{1.888854in}{1.031541in}}%
\pgfpathlineto{\pgfqpoint{1.936481in}{1.032614in}}%
\pgfpathlineto{\pgfqpoint{1.944000in}{1.033906in}}%
\pgfpathlineto{\pgfqpoint{1.949014in}{1.038286in}}%
\pgfpathlineto{\pgfqpoint{1.964054in}{1.040814in}}%
\pgfpathlineto{\pgfqpoint{1.996640in}{1.040814in}}%
\pgfpathlineto{\pgfqpoint{1.999147in}{1.043159in}}%
\pgfpathlineto{\pgfqpoint{2.004160in}{1.044157in}}%
\pgfpathlineto{\pgfqpoint{2.009173in}{1.047097in}}%
\pgfpathlineto{\pgfqpoint{2.029227in}{1.049611in}}%
\pgfpathlineto{\pgfqpoint{2.109440in}{1.049962in}}%
\pgfpathlineto{\pgfqpoint{2.111946in}{1.050875in}}%
\pgfpathlineto{\pgfqpoint{2.116960in}{1.055536in}}%
\pgfpathlineto{\pgfqpoint{2.121973in}{1.056897in}}%
\pgfpathlineto{\pgfqpoint{2.129493in}{1.057979in}}%
\pgfpathlineto{\pgfqpoint{2.239786in}{1.057979in}}%
\pgfpathlineto{\pgfqpoint{2.242292in}{1.061867in}}%
\pgfpathlineto{\pgfqpoint{2.259839in}{1.064666in}}%
\pgfpathlineto{\pgfqpoint{2.264852in}{1.065957in}}%
\pgfpathlineto{\pgfqpoint{2.387678in}{1.066882in}}%
\pgfpathlineto{\pgfqpoint{2.400212in}{1.070124in}}%
\pgfpathlineto{\pgfqpoint{2.405225in}{1.070928in}}%
\pgfpathlineto{\pgfqpoint{2.407732in}{1.073581in}}%
\pgfpathlineto{\pgfqpoint{2.465385in}{1.073581in}}%
\pgfpathlineto{\pgfqpoint{2.475411in}{1.080128in}}%
\pgfpathlineto{\pgfqpoint{2.480425in}{1.080880in}}%
\pgfpathlineto{\pgfqpoint{2.500478in}{1.080880in}}%
\pgfpathlineto{\pgfqpoint{2.518024in}{1.087881in}}%
\pgfpathlineto{\pgfqpoint{2.535571in}{1.088484in}}%
\pgfpathlineto{\pgfqpoint{2.540584in}{1.089411in}}%
\pgfpathlineto{\pgfqpoint{2.543091in}{1.091989in}}%
\pgfpathlineto{\pgfqpoint{2.553117in}{1.093264in}}%
\pgfpathlineto{\pgfqpoint{2.558131in}{1.094608in}}%
\pgfpathlineto{\pgfqpoint{2.570664in}{1.095378in}}%
\pgfpathlineto{\pgfqpoint{2.573171in}{1.095831in}}%
\pgfpathlineto{\pgfqpoint{2.578184in}{1.100208in}}%
\pgfpathlineto{\pgfqpoint{2.583197in}{1.101081in}}%
\pgfpathlineto{\pgfqpoint{2.595731in}{1.101081in}}%
\pgfpathlineto{\pgfqpoint{2.598237in}{1.103334in}}%
\pgfpathlineto{\pgfqpoint{2.605757in}{1.104751in}}%
\pgfpathlineto{\pgfqpoint{2.608264in}{1.107318in}}%
\pgfpathlineto{\pgfqpoint{2.620797in}{1.108301in}}%
\pgfpathlineto{\pgfqpoint{2.623304in}{1.111248in}}%
\pgfpathlineto{\pgfqpoint{2.625810in}{1.111356in}}%
\pgfpathlineto{\pgfqpoint{2.628317in}{1.113336in}}%
\pgfpathlineto{\pgfqpoint{2.645864in}{1.114286in}}%
\pgfpathlineto{\pgfqpoint{2.650877in}{1.116460in}}%
\pgfpathlineto{\pgfqpoint{2.653384in}{1.119151in}}%
\pgfpathlineto{\pgfqpoint{2.665917in}{1.119854in}}%
\pgfpathlineto{\pgfqpoint{2.668424in}{1.124774in}}%
\pgfpathlineto{\pgfqpoint{2.678450in}{1.124774in}}%
\pgfpathlineto{\pgfqpoint{2.680957in}{1.128647in}}%
\pgfpathlineto{\pgfqpoint{2.685970in}{1.130220in}}%
\pgfpathlineto{\pgfqpoint{2.690983in}{1.135497in}}%
\pgfpathlineto{\pgfqpoint{2.693490in}{1.135497in}}%
\pgfpathlineto{\pgfqpoint{2.695997in}{1.136817in}}%
\pgfpathlineto{\pgfqpoint{2.698503in}{1.140233in}}%
\pgfpathlineto{\pgfqpoint{2.708530in}{1.141287in}}%
\pgfpathlineto{\pgfqpoint{2.711037in}{1.144896in}}%
\pgfpathlineto{\pgfqpoint{2.716050in}{1.145995in}}%
\pgfpathlineto{\pgfqpoint{2.721063in}{1.147465in}}%
\pgfpathlineto{\pgfqpoint{2.723570in}{1.147789in}}%
\pgfpathlineto{\pgfqpoint{2.726077in}{1.150320in}}%
\pgfpathlineto{\pgfqpoint{2.738610in}{1.150725in}}%
\pgfpathlineto{\pgfqpoint{2.746130in}{1.155119in}}%
\pgfpathlineto{\pgfqpoint{2.758663in}{1.155119in}}%
\pgfpathlineto{\pgfqpoint{2.761170in}{1.159521in}}%
\pgfpathlineto{\pgfqpoint{2.778716in}{1.159693in}}%
\pgfpathlineto{\pgfqpoint{2.793756in}{1.159878in}}%
\pgfpathlineto{\pgfqpoint{2.798770in}{1.161641in}}%
\pgfpathlineto{\pgfqpoint{2.801276in}{1.162091in}}%
\pgfpathlineto{\pgfqpoint{2.803783in}{1.164148in}}%
\pgfpathlineto{\pgfqpoint{2.833863in}{1.164148in}}%
\pgfpathlineto{\pgfqpoint{2.838876in}{1.165545in}}%
\pgfpathlineto{\pgfqpoint{2.861436in}{1.168580in}}%
\pgfpathlineto{\pgfqpoint{2.863943in}{1.172687in}}%
\pgfpathlineto{\pgfqpoint{2.883996in}{1.173926in}}%
\pgfpathlineto{\pgfqpoint{2.889009in}{1.175056in}}%
\pgfpathlineto{\pgfqpoint{2.891516in}{1.176858in}}%
\pgfpathlineto{\pgfqpoint{2.906556in}{1.177217in}}%
\pgfpathlineto{\pgfqpoint{2.909062in}{1.180374in}}%
\pgfpathlineto{\pgfqpoint{2.919089in}{1.180894in}}%
\pgfpathlineto{\pgfqpoint{2.929116in}{1.180894in}}%
\pgfpathlineto{\pgfqpoint{2.936636in}{1.184837in}}%
\pgfpathlineto{\pgfqpoint{2.956689in}{1.184837in}}%
\pgfpathlineto{\pgfqpoint{2.961702in}{1.188691in}}%
\pgfpathlineto{\pgfqpoint{2.964209in}{1.188691in}}%
\pgfpathlineto{\pgfqpoint{2.966715in}{1.191642in}}%
\pgfpathlineto{\pgfqpoint{2.989275in}{1.192933in}}%
\pgfpathlineto{\pgfqpoint{2.994289in}{1.196149in}}%
\pgfpathlineto{\pgfqpoint{3.001809in}{1.196149in}}%
\pgfpathlineto{\pgfqpoint{3.004315in}{1.199760in}}%
\pgfpathlineto{\pgfqpoint{3.014342in}{1.199760in}}%
\pgfpathlineto{\pgfqpoint{3.021862in}{1.203296in}}%
\pgfpathlineto{\pgfqpoint{3.031888in}{1.204113in}}%
\pgfpathlineto{\pgfqpoint{3.069488in}{1.211180in}}%
\pgfpathlineto{\pgfqpoint{3.071995in}{1.213487in}}%
\pgfpathlineto{\pgfqpoint{3.079515in}{1.214349in}}%
\pgfpathlineto{\pgfqpoint{3.082022in}{1.216576in}}%
\pgfpathlineto{\pgfqpoint{3.089542in}{1.217766in}}%
\pgfpathlineto{\pgfqpoint{3.094555in}{1.221601in}}%
\pgfpathlineto{\pgfqpoint{3.107088in}{1.223298in}}%
\pgfpathlineto{\pgfqpoint{3.109595in}{1.226197in}}%
\pgfpathlineto{\pgfqpoint{3.112101in}{1.226971in}}%
\pgfpathlineto{\pgfqpoint{3.114608in}{1.228971in}}%
\pgfpathlineto{\pgfqpoint{3.127141in}{1.229948in}}%
\pgfpathlineto{\pgfqpoint{3.129648in}{1.232216in}}%
\pgfpathlineto{\pgfqpoint{3.134661in}{1.233615in}}%
\pgfpathlineto{\pgfqpoint{3.142181in}{1.237959in}}%
\pgfpathlineto{\pgfqpoint{3.152208in}{1.238578in}}%
\pgfpathlineto{\pgfqpoint{3.154715in}{1.240747in}}%
\pgfpathlineto{\pgfqpoint{3.164741in}{1.242114in}}%
\pgfpathlineto{\pgfqpoint{3.167248in}{1.243654in}}%
\pgfpathlineto{\pgfqpoint{3.172261in}{1.243723in}}%
\pgfpathlineto{\pgfqpoint{3.174768in}{1.245989in}}%
\pgfpathlineto{\pgfqpoint{3.182288in}{1.246398in}}%
\pgfpathlineto{\pgfqpoint{3.189808in}{1.247099in}}%
\pgfpathlineto{\pgfqpoint{3.199834in}{1.254376in}}%
\pgfpathlineto{\pgfqpoint{3.202341in}{1.254376in}}%
\pgfpathlineto{\pgfqpoint{3.207354in}{1.256956in}}%
\pgfpathlineto{\pgfqpoint{3.239941in}{1.266901in}}%
\pgfpathlineto{\pgfqpoint{3.242447in}{1.266901in}}%
\pgfpathlineto{\pgfqpoint{3.257487in}{1.276301in}}%
\pgfpathlineto{\pgfqpoint{3.265007in}{1.276890in}}%
\pgfpathlineto{\pgfqpoint{3.267514in}{1.279093in}}%
\pgfpathlineto{\pgfqpoint{3.277541in}{1.280710in}}%
\pgfpathlineto{\pgfqpoint{3.282554in}{1.281807in}}%
\pgfpathlineto{\pgfqpoint{3.287567in}{1.283027in}}%
\pgfpathlineto{\pgfqpoint{3.300100in}{1.287179in}}%
\pgfpathlineto{\pgfqpoint{3.302607in}{1.289500in}}%
\pgfpathlineto{\pgfqpoint{3.305114in}{1.289661in}}%
\pgfpathlineto{\pgfqpoint{3.310127in}{1.291604in}}%
\pgfpathlineto{\pgfqpoint{3.317647in}{1.292610in}}%
\pgfpathlineto{\pgfqpoint{3.322660in}{1.297767in}}%
\pgfpathlineto{\pgfqpoint{3.325167in}{1.297767in}}%
\pgfpathlineto{\pgfqpoint{3.332687in}{1.303251in}}%
\pgfpathlineto{\pgfqpoint{3.335194in}{1.305339in}}%
\pgfpathlineto{\pgfqpoint{3.337700in}{1.305654in}}%
\pgfpathlineto{\pgfqpoint{3.345220in}{1.309465in}}%
\pgfpathlineto{\pgfqpoint{3.347727in}{1.309465in}}%
\pgfpathlineto{\pgfqpoint{3.352740in}{1.314339in}}%
\pgfpathlineto{\pgfqpoint{3.382820in}{1.320593in}}%
\pgfpathlineto{\pgfqpoint{3.387833in}{1.325445in}}%
\pgfpathlineto{\pgfqpoint{3.390340in}{1.325640in}}%
\pgfpathlineto{\pgfqpoint{3.392847in}{1.327151in}}%
\pgfpathlineto{\pgfqpoint{3.400367in}{1.327655in}}%
\pgfpathlineto{\pgfqpoint{3.402873in}{1.329632in}}%
\pgfpathlineto{\pgfqpoint{3.415407in}{1.332367in}}%
\pgfpathlineto{\pgfqpoint{3.422927in}{1.338839in}}%
\pgfpathlineto{\pgfqpoint{3.432953in}{1.342374in}}%
\pgfpathlineto{\pgfqpoint{3.437966in}{1.343539in}}%
\pgfpathlineto{\pgfqpoint{3.440473in}{1.343692in}}%
\pgfpathlineto{\pgfqpoint{3.442980in}{1.345077in}}%
\pgfpathlineto{\pgfqpoint{3.445486in}{1.349610in}}%
\pgfpathlineto{\pgfqpoint{3.450500in}{1.349962in}}%
\pgfpathlineto{\pgfqpoint{3.455513in}{1.353950in}}%
\pgfpathlineto{\pgfqpoint{3.460526in}{1.354072in}}%
\pgfpathlineto{\pgfqpoint{3.465540in}{1.358275in}}%
\pgfpathlineto{\pgfqpoint{3.473060in}{1.359744in}}%
\pgfpathlineto{\pgfqpoint{3.485593in}{1.365277in}}%
\pgfpathlineto{\pgfqpoint{3.490606in}{1.366430in}}%
\pgfpathlineto{\pgfqpoint{3.500633in}{1.371569in}}%
\pgfpathlineto{\pgfqpoint{3.503139in}{1.374550in}}%
\pgfpathlineto{\pgfqpoint{3.508153in}{1.375589in}}%
\pgfpathlineto{\pgfqpoint{3.515673in}{1.382442in}}%
\pgfpathlineto{\pgfqpoint{3.523193in}{1.388199in}}%
\pgfpathlineto{\pgfqpoint{3.525699in}{1.389206in}}%
\pgfpathlineto{\pgfqpoint{3.530713in}{1.396537in}}%
\pgfpathlineto{\pgfqpoint{3.535726in}{1.398316in}}%
\pgfpathlineto{\pgfqpoint{3.538233in}{1.398576in}}%
\pgfpathlineto{\pgfqpoint{3.540739in}{1.401907in}}%
\pgfpathlineto{\pgfqpoint{3.545753in}{1.403003in}}%
\pgfpathlineto{\pgfqpoint{3.548259in}{1.403460in}}%
\pgfpathlineto{\pgfqpoint{3.550766in}{1.409081in}}%
\pgfpathlineto{\pgfqpoint{3.553273in}{1.410504in}}%
\pgfpathlineto{\pgfqpoint{3.558286in}{1.417652in}}%
\pgfpathlineto{\pgfqpoint{3.560793in}{1.420635in}}%
\pgfpathlineto{\pgfqpoint{3.583352in}{1.428345in}}%
\pgfpathlineto{\pgfqpoint{3.588366in}{1.429943in}}%
\pgfpathlineto{\pgfqpoint{3.600899in}{1.441846in}}%
\pgfpathlineto{\pgfqpoint{3.608419in}{1.446226in}}%
\pgfpathlineto{\pgfqpoint{3.615939in}{1.448983in}}%
\pgfpathlineto{\pgfqpoint{3.623459in}{1.450420in}}%
\pgfpathlineto{\pgfqpoint{3.625966in}{1.452620in}}%
\pgfpathlineto{\pgfqpoint{3.630979in}{1.453935in}}%
\pgfpathlineto{\pgfqpoint{3.633486in}{1.456123in}}%
\pgfpathlineto{\pgfqpoint{3.638499in}{1.456353in}}%
\pgfpathlineto{\pgfqpoint{3.646019in}{1.460866in}}%
\pgfpathlineto{\pgfqpoint{3.648525in}{1.463189in}}%
\pgfpathlineto{\pgfqpoint{3.653539in}{1.463189in}}%
\pgfpathlineto{\pgfqpoint{3.656045in}{1.468295in}}%
\pgfpathlineto{\pgfqpoint{3.663565in}{1.469607in}}%
\pgfpathlineto{\pgfqpoint{3.666072in}{1.473829in}}%
\pgfpathlineto{\pgfqpoint{3.676099in}{1.474354in}}%
\pgfpathlineto{\pgfqpoint{3.678605in}{1.477354in}}%
\pgfpathlineto{\pgfqpoint{3.681112in}{1.477467in}}%
\pgfpathlineto{\pgfqpoint{3.686125in}{1.481603in}}%
\pgfpathlineto{\pgfqpoint{3.698659in}{1.484019in}}%
\pgfpathlineto{\pgfqpoint{3.701165in}{1.486186in}}%
\pgfpathlineto{\pgfqpoint{3.708685in}{1.486980in}}%
\pgfpathlineto{\pgfqpoint{3.721218in}{1.492784in}}%
\pgfpathlineto{\pgfqpoint{3.723725in}{1.496541in}}%
\pgfpathlineto{\pgfqpoint{3.728738in}{1.498512in}}%
\pgfpathlineto{\pgfqpoint{3.736258in}{1.505190in}}%
\pgfpathlineto{\pgfqpoint{3.748792in}{1.507348in}}%
\pgfpathlineto{\pgfqpoint{3.751298in}{1.513052in}}%
\pgfpathlineto{\pgfqpoint{3.753805in}{1.513114in}}%
\pgfpathlineto{\pgfqpoint{3.756312in}{1.516894in}}%
\pgfpathlineto{\pgfqpoint{3.758818in}{1.517456in}}%
\pgfpathlineto{\pgfqpoint{3.761325in}{1.519671in}}%
\pgfpathlineto{\pgfqpoint{3.763832in}{1.519683in}}%
\pgfpathlineto{\pgfqpoint{3.766338in}{1.522359in}}%
\pgfpathlineto{\pgfqpoint{3.773858in}{1.524287in}}%
\pgfpathlineto{\pgfqpoint{3.776365in}{1.527787in}}%
\pgfpathlineto{\pgfqpoint{3.778871in}{1.527787in}}%
\pgfpathlineto{\pgfqpoint{3.781378in}{1.529363in}}%
\pgfpathlineto{\pgfqpoint{3.783885in}{1.529426in}}%
\pgfpathlineto{\pgfqpoint{3.793911in}{1.535021in}}%
\pgfpathlineto{\pgfqpoint{3.798925in}{1.535660in}}%
\pgfpathlineto{\pgfqpoint{3.801431in}{1.539322in}}%
\pgfpathlineto{\pgfqpoint{3.829005in}{1.549654in}}%
\pgfpathlineto{\pgfqpoint{3.831511in}{1.556398in}}%
\pgfpathlineto{\pgfqpoint{3.836525in}{1.558636in}}%
\pgfpathlineto{\pgfqpoint{3.839031in}{1.559496in}}%
\pgfpathlineto{\pgfqpoint{3.841538in}{1.568778in}}%
\pgfpathlineto{\pgfqpoint{3.846551in}{1.571517in}}%
\pgfpathlineto{\pgfqpoint{3.849058in}{1.575076in}}%
\pgfpathlineto{\pgfqpoint{3.851564in}{1.575403in}}%
\pgfpathlineto{\pgfqpoint{3.856578in}{1.583352in}}%
\pgfpathlineto{\pgfqpoint{3.864098in}{1.589913in}}%
\pgfpathlineto{\pgfqpoint{3.866604in}{1.590326in}}%
\pgfpathlineto{\pgfqpoint{3.869111in}{1.592663in}}%
\pgfpathlineto{\pgfqpoint{3.879138in}{1.594511in}}%
\pgfpathlineto{\pgfqpoint{3.884151in}{1.599120in}}%
\pgfpathlineto{\pgfqpoint{3.886658in}{1.601467in}}%
\pgfpathlineto{\pgfqpoint{3.889164in}{1.601962in}}%
\pgfpathlineto{\pgfqpoint{3.901698in}{1.612012in}}%
\pgfpathlineto{\pgfqpoint{3.911724in}{1.616028in}}%
\pgfpathlineto{\pgfqpoint{3.919244in}{1.622465in}}%
\pgfpathlineto{\pgfqpoint{3.929271in}{1.627628in}}%
\pgfpathlineto{\pgfqpoint{3.934284in}{1.628299in}}%
\pgfpathlineto{\pgfqpoint{3.939297in}{1.630347in}}%
\pgfpathlineto{\pgfqpoint{3.941804in}{1.635125in}}%
\pgfpathlineto{\pgfqpoint{3.949324in}{1.636794in}}%
\pgfpathlineto{\pgfqpoint{3.954337in}{1.641710in}}%
\pgfpathlineto{\pgfqpoint{3.959351in}{1.643724in}}%
\pgfpathlineto{\pgfqpoint{3.966871in}{1.647252in}}%
\pgfpathlineto{\pgfqpoint{3.969377in}{1.649784in}}%
\pgfpathlineto{\pgfqpoint{3.981910in}{1.652036in}}%
\pgfpathlineto{\pgfqpoint{3.994444in}{1.655940in}}%
\pgfpathlineto{\pgfqpoint{3.996950in}{1.658640in}}%
\pgfpathlineto{\pgfqpoint{4.004470in}{1.660760in}}%
\pgfpathlineto{\pgfqpoint{4.009484in}{1.663298in}}%
\pgfpathlineto{\pgfqpoint{4.017004in}{1.664684in}}%
\pgfpathlineto{\pgfqpoint{4.027030in}{1.670264in}}%
\pgfpathlineto{\pgfqpoint{4.029537in}{1.670322in}}%
\pgfpathlineto{\pgfqpoint{4.037057in}{1.674358in}}%
\pgfpathlineto{\pgfqpoint{4.039564in}{1.674892in}}%
\pgfpathlineto{\pgfqpoint{4.042070in}{1.678106in}}%
\pgfpathlineto{\pgfqpoint{4.049590in}{1.681237in}}%
\pgfpathlineto{\pgfqpoint{4.052097in}{1.681481in}}%
\pgfpathlineto{\pgfqpoint{4.054603in}{1.683791in}}%
\pgfpathlineto{\pgfqpoint{4.062123in}{1.684628in}}%
\pgfpathlineto{\pgfqpoint{4.064630in}{1.687978in}}%
\pgfpathlineto{\pgfqpoint{4.067137in}{1.688436in}}%
\pgfpathlineto{\pgfqpoint{4.069643in}{1.690535in}}%
\pgfpathlineto{\pgfqpoint{4.094710in}{1.696397in}}%
\pgfpathlineto{\pgfqpoint{4.099723in}{1.699074in}}%
\pgfpathlineto{\pgfqpoint{4.104737in}{1.701502in}}%
\pgfpathlineto{\pgfqpoint{4.107243in}{1.703238in}}%
\pgfpathlineto{\pgfqpoint{4.114763in}{1.704327in}}%
\pgfpathlineto{\pgfqpoint{4.124790in}{1.706343in}}%
\pgfpathlineto{\pgfqpoint{4.129803in}{1.707705in}}%
\pgfpathlineto{\pgfqpoint{4.132310in}{1.709249in}}%
\pgfpathlineto{\pgfqpoint{4.137323in}{1.709311in}}%
\pgfpathlineto{\pgfqpoint{4.139830in}{1.710903in}}%
\pgfpathlineto{\pgfqpoint{4.149856in}{1.712479in}}%
\pgfpathlineto{\pgfqpoint{4.154870in}{1.714518in}}%
\pgfpathlineto{\pgfqpoint{4.157376in}{1.714572in}}%
\pgfpathlineto{\pgfqpoint{4.162390in}{1.716868in}}%
\pgfpathlineto{\pgfqpoint{4.172416in}{1.718283in}}%
\pgfpathlineto{\pgfqpoint{4.197483in}{1.724988in}}%
\pgfpathlineto{\pgfqpoint{4.210016in}{1.731357in}}%
\pgfpathlineto{\pgfqpoint{4.220043in}{1.732313in}}%
\pgfpathlineto{\pgfqpoint{4.222549in}{1.734663in}}%
\pgfpathlineto{\pgfqpoint{4.227563in}{1.735765in}}%
\pgfpathlineto{\pgfqpoint{4.232576in}{1.736625in}}%
\pgfpathlineto{\pgfqpoint{4.235083in}{1.736657in}}%
\pgfpathlineto{\pgfqpoint{4.237589in}{1.738206in}}%
\pgfpathlineto{\pgfqpoint{4.242603in}{1.739323in}}%
\pgfpathlineto{\pgfqpoint{4.252629in}{1.741510in}}%
\pgfpathlineto{\pgfqpoint{4.257642in}{1.745627in}}%
\pgfpathlineto{\pgfqpoint{4.280202in}{1.749318in}}%
\pgfpathlineto{\pgfqpoint{4.287722in}{1.753149in}}%
\pgfpathlineto{\pgfqpoint{4.297749in}{1.756929in}}%
\pgfpathlineto{\pgfqpoint{4.300256in}{1.759350in}}%
\pgfpathlineto{\pgfqpoint{4.317802in}{1.764534in}}%
\pgfpathlineto{\pgfqpoint{4.320309in}{1.766061in}}%
\pgfpathlineto{\pgfqpoint{4.322815in}{1.766251in}}%
\pgfpathlineto{\pgfqpoint{4.327829in}{1.770524in}}%
\pgfpathlineto{\pgfqpoint{4.330335in}{1.771511in}}%
\pgfpathlineto{\pgfqpoint{4.332842in}{1.774519in}}%
\pgfpathlineto{\pgfqpoint{4.340362in}{1.775998in}}%
\pgfpathlineto{\pgfqpoint{4.347882in}{1.777098in}}%
\pgfpathlineto{\pgfqpoint{4.350389in}{1.780100in}}%
\pgfpathlineto{\pgfqpoint{4.365429in}{1.782396in}}%
\pgfpathlineto{\pgfqpoint{4.385482in}{1.787813in}}%
\pgfpathlineto{\pgfqpoint{4.387988in}{1.791098in}}%
\pgfpathlineto{\pgfqpoint{4.420575in}{1.800234in}}%
\pgfpathlineto{\pgfqpoint{4.423082in}{1.802912in}}%
\pgfpathlineto{\pgfqpoint{4.428095in}{1.803663in}}%
\pgfpathlineto{\pgfqpoint{4.430602in}{1.806195in}}%
\pgfpathlineto{\pgfqpoint{4.433108in}{1.806669in}}%
\pgfpathlineto{\pgfqpoint{4.438122in}{1.812661in}}%
\pgfpathlineto{\pgfqpoint{4.455668in}{1.821661in}}%
\pgfpathlineto{\pgfqpoint{4.460681in}{1.825173in}}%
\pgfpathlineto{\pgfqpoint{4.463188in}{1.825596in}}%
\pgfpathlineto{\pgfqpoint{4.465695in}{1.827517in}}%
\pgfpathlineto{\pgfqpoint{4.473215in}{1.827821in}}%
\pgfpathlineto{\pgfqpoint{4.475721in}{1.830228in}}%
\pgfpathlineto{\pgfqpoint{4.478228in}{1.830864in}}%
\pgfpathlineto{\pgfqpoint{4.480735in}{1.832696in}}%
\pgfpathlineto{\pgfqpoint{4.485748in}{1.834116in}}%
\pgfpathlineto{\pgfqpoint{4.495775in}{1.838092in}}%
\pgfpathlineto{\pgfqpoint{4.505801in}{1.839508in}}%
\pgfpathlineto{\pgfqpoint{4.508308in}{1.839512in}}%
\pgfpathlineto{\pgfqpoint{4.513321in}{1.842281in}}%
\pgfpathlineto{\pgfqpoint{4.518335in}{1.842826in}}%
\pgfpathlineto{\pgfqpoint{4.523348in}{1.844629in}}%
\pgfpathlineto{\pgfqpoint{4.528361in}{1.845780in}}%
\pgfpathlineto{\pgfqpoint{4.538388in}{1.848370in}}%
\pgfpathlineto{\pgfqpoint{4.540894in}{1.851756in}}%
\pgfpathlineto{\pgfqpoint{4.545908in}{1.853725in}}%
\pgfpathlineto{\pgfqpoint{4.560948in}{1.856528in}}%
\pgfpathlineto{\pgfqpoint{4.563454in}{1.857030in}}%
\pgfpathlineto{\pgfqpoint{4.565961in}{1.859354in}}%
\pgfpathlineto{\pgfqpoint{4.583508in}{1.863768in}}%
\pgfpathlineto{\pgfqpoint{4.586014in}{1.866439in}}%
\pgfpathlineto{\pgfqpoint{4.596041in}{1.869515in}}%
\pgfpathlineto{\pgfqpoint{4.598547in}{1.871144in}}%
\pgfpathlineto{\pgfqpoint{4.603561in}{1.878227in}}%
\pgfpathlineto{\pgfqpoint{4.611081in}{1.879674in}}%
\pgfpathlineto{\pgfqpoint{4.636147in}{1.891965in}}%
\pgfpathlineto{\pgfqpoint{4.638654in}{1.894785in}}%
\pgfpathlineto{\pgfqpoint{4.641161in}{1.895040in}}%
\pgfpathlineto{\pgfqpoint{4.648681in}{1.900084in}}%
\pgfpathlineto{\pgfqpoint{4.653694in}{1.902690in}}%
\pgfpathlineto{\pgfqpoint{4.656201in}{1.908033in}}%
\pgfpathlineto{\pgfqpoint{4.661214in}{1.910972in}}%
\pgfpathlineto{\pgfqpoint{4.663720in}{1.911185in}}%
\pgfpathlineto{\pgfqpoint{4.666227in}{1.912812in}}%
\pgfpathlineto{\pgfqpoint{4.668734in}{1.919887in}}%
\pgfpathlineto{\pgfqpoint{4.706334in}{1.929152in}}%
\pgfpathlineto{\pgfqpoint{4.708840in}{1.932077in}}%
\pgfpathlineto{\pgfqpoint{4.713854in}{1.933177in}}%
\pgfpathlineto{\pgfqpoint{4.716360in}{1.936004in}}%
\pgfpathlineto{\pgfqpoint{4.721374in}{1.947863in}}%
\pgfpathlineto{\pgfqpoint{4.723880in}{1.950361in}}%
\pgfpathlineto{\pgfqpoint{4.738920in}{1.956350in}}%
\pgfpathlineto{\pgfqpoint{4.746440in}{1.961449in}}%
\pgfpathlineto{\pgfqpoint{4.748947in}{1.968198in}}%
\pgfpathlineto{\pgfqpoint{4.769000in}{1.974266in}}%
\pgfpathlineto{\pgfqpoint{4.771507in}{1.978752in}}%
\pgfpathlineto{\pgfqpoint{4.781533in}{1.980252in}}%
\pgfpathlineto{\pgfqpoint{4.784040in}{1.981909in}}%
\pgfpathlineto{\pgfqpoint{4.789053in}{1.982909in}}%
\pgfpathlineto{\pgfqpoint{4.791560in}{1.983439in}}%
\pgfpathlineto{\pgfqpoint{4.794066in}{1.985112in}}%
\pgfpathlineto{\pgfqpoint{4.799080in}{1.985382in}}%
\pgfpathlineto{\pgfqpoint{4.801586in}{1.987516in}}%
\pgfpathlineto{\pgfqpoint{4.809106in}{1.989086in}}%
\pgfpathlineto{\pgfqpoint{4.824146in}{1.993621in}}%
\pgfpathlineto{\pgfqpoint{4.831666in}{1.994802in}}%
\pgfpathlineto{\pgfqpoint{4.834173in}{1.997267in}}%
\pgfpathlineto{\pgfqpoint{4.836680in}{1.998171in}}%
\pgfpathlineto{\pgfqpoint{4.839186in}{2.000400in}}%
\pgfpathlineto{\pgfqpoint{4.841693in}{2.000479in}}%
\pgfpathlineto{\pgfqpoint{4.849213in}{2.004256in}}%
\pgfpathlineto{\pgfqpoint{4.851720in}{2.008283in}}%
\pgfpathlineto{\pgfqpoint{4.854226in}{2.008657in}}%
\pgfpathlineto{\pgfqpoint{4.856733in}{2.014925in}}%
\pgfpathlineto{\pgfqpoint{4.866759in}{2.017075in}}%
\pgfpathlineto{\pgfqpoint{4.869266in}{2.020332in}}%
\pgfpathlineto{\pgfqpoint{4.879293in}{2.021502in}}%
\pgfpathlineto{\pgfqpoint{4.884306in}{2.023882in}}%
\pgfpathlineto{\pgfqpoint{4.909373in}{2.033848in}}%
\pgfpathlineto{\pgfqpoint{4.914386in}{2.037621in}}%
\pgfpathlineto{\pgfqpoint{4.921906in}{2.040496in}}%
\pgfpathlineto{\pgfqpoint{4.924413in}{2.044277in}}%
\pgfpathlineto{\pgfqpoint{4.931932in}{2.047843in}}%
\pgfpathlineto{\pgfqpoint{4.934439in}{2.054156in}}%
\pgfpathlineto{\pgfqpoint{4.936946in}{2.055092in}}%
\pgfpathlineto{\pgfqpoint{4.939452in}{2.060860in}}%
\pgfpathlineto{\pgfqpoint{4.954492in}{2.066461in}}%
\pgfpathlineto{\pgfqpoint{4.959506in}{2.072530in}}%
\pgfpathlineto{\pgfqpoint{4.964519in}{2.074237in}}%
\pgfpathlineto{\pgfqpoint{4.967026in}{2.078538in}}%
\pgfpathlineto{\pgfqpoint{4.977052in}{2.080299in}}%
\pgfpathlineto{\pgfqpoint{4.979559in}{2.082183in}}%
\pgfpathlineto{\pgfqpoint{4.984572in}{2.083590in}}%
\pgfpathlineto{\pgfqpoint{4.997105in}{2.090417in}}%
\pgfpathlineto{\pgfqpoint{5.002119in}{2.094505in}}%
\pgfpathlineto{\pgfqpoint{5.017159in}{2.104439in}}%
\pgfpathlineto{\pgfqpoint{5.019665in}{2.104973in}}%
\pgfpathlineto{\pgfqpoint{5.022172in}{2.108376in}}%
\pgfpathlineto{\pgfqpoint{5.032199in}{2.112069in}}%
\pgfpathlineto{\pgfqpoint{5.037212in}{2.113030in}}%
\pgfpathlineto{\pgfqpoint{5.049745in}{2.116132in}}%
\pgfpathlineto{\pgfqpoint{5.052252in}{2.124152in}}%
\pgfpathlineto{\pgfqpoint{5.064785in}{2.130117in}}%
\pgfpathlineto{\pgfqpoint{5.067292in}{2.133241in}}%
\pgfpathlineto{\pgfqpoint{5.072305in}{2.135267in}}%
\pgfpathlineto{\pgfqpoint{5.077318in}{2.140187in}}%
\pgfpathlineto{\pgfqpoint{5.082332in}{2.150823in}}%
\pgfpathlineto{\pgfqpoint{5.084838in}{2.151399in}}%
\pgfpathlineto{\pgfqpoint{5.087345in}{2.158318in}}%
\pgfpathlineto{\pgfqpoint{5.094865in}{2.160424in}}%
\pgfpathlineto{\pgfqpoint{5.097372in}{2.162515in}}%
\pgfpathlineto{\pgfqpoint{5.102385in}{2.173581in}}%
\pgfpathlineto{\pgfqpoint{5.104892in}{2.173852in}}%
\pgfpathlineto{\pgfqpoint{5.107398in}{2.177859in}}%
\pgfpathlineto{\pgfqpoint{5.112412in}{2.178711in}}%
\pgfpathlineto{\pgfqpoint{5.114918in}{2.195712in}}%
\pgfpathlineto{\pgfqpoint{5.117425in}{2.198852in}}%
\pgfpathlineto{\pgfqpoint{5.119932in}{2.199498in}}%
\pgfpathlineto{\pgfqpoint{5.122438in}{2.207149in}}%
\pgfpathlineto{\pgfqpoint{5.124945in}{2.207724in}}%
\pgfpathlineto{\pgfqpoint{5.127452in}{2.212478in}}%
\pgfpathlineto{\pgfqpoint{5.129958in}{2.212669in}}%
\pgfpathlineto{\pgfqpoint{5.132465in}{2.215160in}}%
\pgfpathlineto{\pgfqpoint{5.134971in}{2.222367in}}%
\pgfpathlineto{\pgfqpoint{5.139985in}{2.223458in}}%
\pgfpathlineto{\pgfqpoint{5.144998in}{2.227025in}}%
\pgfpathlineto{\pgfqpoint{5.147505in}{2.231601in}}%
\pgfpathlineto{\pgfqpoint{5.157531in}{2.238718in}}%
\pgfpathlineto{\pgfqpoint{5.165051in}{2.241435in}}%
\pgfpathlineto{\pgfqpoint{5.172571in}{2.245481in}}%
\pgfpathlineto{\pgfqpoint{5.175078in}{2.252715in}}%
\pgfpathlineto{\pgfqpoint{5.180091in}{2.256507in}}%
\pgfpathlineto{\pgfqpoint{5.182598in}{2.257388in}}%
\pgfpathlineto{\pgfqpoint{5.187611in}{2.262692in}}%
\pgfpathlineto{\pgfqpoint{5.190118in}{2.263457in}}%
\pgfpathlineto{\pgfqpoint{5.195131in}{2.274799in}}%
\pgfpathlineto{\pgfqpoint{5.200145in}{2.275557in}}%
\pgfpathlineto{\pgfqpoint{5.205158in}{2.277355in}}%
\pgfpathlineto{\pgfqpoint{5.210171in}{2.286287in}}%
\pgfpathlineto{\pgfqpoint{5.212678in}{2.298891in}}%
\pgfpathlineto{\pgfqpoint{5.220198in}{2.311134in}}%
\pgfpathlineto{\pgfqpoint{5.222704in}{2.311598in}}%
\pgfpathlineto{\pgfqpoint{5.230224in}{2.317156in}}%
\pgfpathlineto{\pgfqpoint{5.232731in}{2.327872in}}%
\pgfpathlineto{\pgfqpoint{5.235238in}{2.328021in}}%
\pgfpathlineto{\pgfqpoint{5.237744in}{2.333321in}}%
\pgfpathlineto{\pgfqpoint{5.242758in}{2.334253in}}%
\pgfpathlineto{\pgfqpoint{5.245264in}{2.346620in}}%
\pgfpathlineto{\pgfqpoint{5.247771in}{2.346647in}}%
\pgfpathlineto{\pgfqpoint{5.257798in}{2.354038in}}%
\pgfpathlineto{\pgfqpoint{5.260304in}{2.359749in}}%
\pgfpathlineto{\pgfqpoint{5.267824in}{2.364212in}}%
\pgfpathlineto{\pgfqpoint{5.275344in}{2.369696in}}%
\pgfpathlineto{\pgfqpoint{5.277851in}{2.369805in}}%
\pgfpathlineto{\pgfqpoint{5.280357in}{2.388561in}}%
\pgfpathlineto{\pgfqpoint{5.285371in}{2.391518in}}%
\pgfpathlineto{\pgfqpoint{5.287877in}{2.396783in}}%
\pgfpathlineto{\pgfqpoint{5.292891in}{2.401852in}}%
\pgfpathlineto{\pgfqpoint{5.295397in}{2.402009in}}%
\pgfpathlineto{\pgfqpoint{5.300411in}{2.422086in}}%
\pgfpathlineto{\pgfqpoint{5.302917in}{2.424718in}}%
\pgfpathlineto{\pgfqpoint{5.305424in}{2.424831in}}%
\pgfpathlineto{\pgfqpoint{5.310437in}{2.429578in}}%
\pgfpathlineto{\pgfqpoint{5.317957in}{2.453549in}}%
\pgfpathlineto{\pgfqpoint{5.320464in}{2.454495in}}%
\pgfpathlineto{\pgfqpoint{5.322971in}{2.457499in}}%
\pgfpathlineto{\pgfqpoint{5.327984in}{2.458138in}}%
\pgfpathlineto{\pgfqpoint{5.330491in}{2.461113in}}%
\pgfpathlineto{\pgfqpoint{5.332997in}{2.467474in}}%
\pgfpathlineto{\pgfqpoint{5.335504in}{2.467830in}}%
\pgfpathlineto{\pgfqpoint{5.338010in}{2.472230in}}%
\pgfpathlineto{\pgfqpoint{5.343024in}{2.473796in}}%
\pgfpathlineto{\pgfqpoint{5.345530in}{2.481085in}}%
\pgfpathlineto{\pgfqpoint{5.350544in}{2.515201in}}%
\pgfpathlineto{\pgfqpoint{5.353050in}{2.518156in}}%
\pgfpathlineto{\pgfqpoint{5.355557in}{2.532058in}}%
\pgfpathlineto{\pgfqpoint{5.358064in}{2.532266in}}%
\pgfpathlineto{\pgfqpoint{5.360570in}{2.533797in}}%
\pgfpathlineto{\pgfqpoint{5.363077in}{2.537652in}}%
\pgfpathlineto{\pgfqpoint{5.365584in}{2.549327in}}%
\pgfpathlineto{\pgfqpoint{5.368090in}{2.551410in}}%
\pgfpathlineto{\pgfqpoint{5.373104in}{2.567385in}}%
\pgfpathlineto{\pgfqpoint{5.375610in}{2.570034in}}%
\pgfpathlineto{\pgfqpoint{5.383130in}{2.599869in}}%
\pgfpathlineto{\pgfqpoint{5.385637in}{2.623749in}}%
\pgfpathlineto{\pgfqpoint{5.388144in}{2.624230in}}%
\pgfpathlineto{\pgfqpoint{5.390650in}{2.628429in}}%
\pgfpathlineto{\pgfqpoint{5.393157in}{2.628945in}}%
\pgfpathlineto{\pgfqpoint{5.395664in}{2.645495in}}%
\pgfpathlineto{\pgfqpoint{5.398170in}{2.651122in}}%
\pgfpathlineto{\pgfqpoint{5.403184in}{2.651827in}}%
\pgfpathlineto{\pgfqpoint{5.408197in}{2.654115in}}%
\pgfpathlineto{\pgfqpoint{5.410703in}{2.657985in}}%
\pgfpathlineto{\pgfqpoint{5.413210in}{2.675758in}}%
\pgfpathlineto{\pgfqpoint{5.415717in}{2.678473in}}%
\pgfpathlineto{\pgfqpoint{5.418223in}{2.678692in}}%
\pgfpathlineto{\pgfqpoint{5.420730in}{2.691442in}}%
\pgfpathlineto{\pgfqpoint{5.423237in}{2.694173in}}%
\pgfpathlineto{\pgfqpoint{5.425743in}{2.704769in}}%
\pgfpathlineto{\pgfqpoint{5.430757in}{2.735580in}}%
\pgfpathlineto{\pgfqpoint{5.433263in}{2.737535in}}%
\pgfpathlineto{\pgfqpoint{5.435770in}{2.765827in}}%
\pgfpathlineto{\pgfqpoint{5.438277in}{2.766511in}}%
\pgfpathlineto{\pgfqpoint{5.440783in}{2.776054in}}%
\pgfpathlineto{\pgfqpoint{5.443290in}{2.776110in}}%
\pgfpathlineto{\pgfqpoint{5.450810in}{2.803008in}}%
\pgfpathlineto{\pgfqpoint{5.453317in}{2.841237in}}%
\pgfpathlineto{\pgfqpoint{5.458330in}{2.846976in}}%
\pgfpathlineto{\pgfqpoint{5.460837in}{2.856974in}}%
\pgfpathlineto{\pgfqpoint{5.465850in}{2.888246in}}%
\pgfpathlineto{\pgfqpoint{5.468357in}{2.888671in}}%
\pgfpathlineto{\pgfqpoint{5.470863in}{2.891065in}}%
\pgfpathlineto{\pgfqpoint{5.473370in}{2.905275in}}%
\pgfpathlineto{\pgfqpoint{5.473370in}{2.905275in}}%
\pgfusepath{stroke}%
\end{pgfscope}%
\begin{pgfscope}%
\pgfsetrectcap%
\pgfsetmiterjoin%
\pgfsetlinewidth{0.803000pt}%
\definecolor{currentstroke}{rgb}{0.000000,0.000000,0.000000}%
\pgfsetstrokecolor{currentstroke}%
\pgfsetdash{}{0pt}%
\pgfpathmoveto{\pgfqpoint{0.708220in}{0.535823in}}%
\pgfpathlineto{\pgfqpoint{0.708220in}{2.905275in}}%
\pgfusepath{stroke}%
\end{pgfscope}%
\begin{pgfscope}%
\pgfsetrectcap%
\pgfsetmiterjoin%
\pgfsetlinewidth{0.803000pt}%
\definecolor{currentstroke}{rgb}{0.000000,0.000000,0.000000}%
\pgfsetstrokecolor{currentstroke}%
\pgfsetdash{}{0pt}%
\pgfpathmoveto{\pgfqpoint{5.721529in}{0.535823in}}%
\pgfpathlineto{\pgfqpoint{5.721529in}{2.905275in}}%
\pgfusepath{stroke}%
\end{pgfscope}%
\begin{pgfscope}%
\pgfsetrectcap%
\pgfsetmiterjoin%
\pgfsetlinewidth{0.803000pt}%
\definecolor{currentstroke}{rgb}{0.000000,0.000000,0.000000}%
\pgfsetstrokecolor{currentstroke}%
\pgfsetdash{}{0pt}%
\pgfpathmoveto{\pgfqpoint{0.708220in}{0.535823in}}%
\pgfpathlineto{\pgfqpoint{5.721529in}{0.535823in}}%
\pgfusepath{stroke}%
\end{pgfscope}%
\begin{pgfscope}%
\pgfsetrectcap%
\pgfsetmiterjoin%
\pgfsetlinewidth{0.803000pt}%
\definecolor{currentstroke}{rgb}{0.000000,0.000000,0.000000}%
\pgfsetstrokecolor{currentstroke}%
\pgfsetdash{}{0pt}%
\pgfpathmoveto{\pgfqpoint{0.708220in}{2.905275in}}%
\pgfpathlineto{\pgfqpoint{5.721529in}{2.905275in}}%
\pgfusepath{stroke}%
\end{pgfscope}%
\begin{pgfscope}%
\pgfsetrectcap%
\pgfsetroundjoin%
\pgfsetlinewidth{1.003750pt}%
\definecolor{currentstroke}{rgb}{0.000000,0.000000,1.000000}%
\pgfsetstrokecolor{currentstroke}%
\pgfsetdash{}{0pt}%
\pgfpathmoveto{\pgfqpoint{3.643194in}{1.301071in}}%
\pgfpathlineto{\pgfqpoint{3.893194in}{1.301071in}}%
\pgfusepath{stroke}%
\end{pgfscope}%
\begin{pgfscope}%
\definecolor{textcolor}{rgb}{0.000000,0.000000,0.000000}%
\pgfsetstrokecolor{textcolor}%
\pgfsetfillcolor{textcolor}%
\pgftext[x=3.918194in,y=1.257321in,left,base]{\color{textcolor}\rmfamily\fontsize{9.000000}{10.800000}\selectfont tensor+HTB}%
\end{pgfscope}%
\begin{pgfscope}%
\pgfsetrectcap%
\pgfsetroundjoin%
\pgfsetlinewidth{1.003750pt}%
\definecolor{currentstroke}{rgb}{0.000000,0.501961,0.000000}%
\pgfsetstrokecolor{currentstroke}%
\pgfsetdash{}{0pt}%
\pgfpathmoveto{\pgfqpoint{3.643194in}{1.139271in}}%
\pgfpathlineto{\pgfqpoint{3.893194in}{1.139271in}}%
\pgfusepath{stroke}%
\end{pgfscope}%
\begin{pgfscope}%
\definecolor{textcolor}{rgb}{0.000000,0.000000,0.000000}%
\pgfsetstrokecolor{textcolor}%
\pgfsetfillcolor{textcolor}%
\pgftext[x=3.918194in,y=1.095521in,left,base]{\color{textcolor}\rmfamily\fontsize{9.000000}{10.800000}\selectfont tensor+LG}%
\end{pgfscope}%
\begin{pgfscope}%
\pgfsetrectcap%
\pgfsetroundjoin%
\pgfsetlinewidth{1.003750pt}%
\definecolor{currentstroke}{rgb}{0.000000,0.000000,0.000000}%
\pgfsetstrokecolor{currentstroke}%
\pgfsetdash{}{0pt}%
\pgfpathmoveto{\pgfqpoint{3.643194in}{0.977471in}}%
\pgfpathlineto{\pgfqpoint{3.893194in}{0.977471in}}%
\pgfusepath{stroke}%
\end{pgfscope}%
\begin{pgfscope}%
\definecolor{textcolor}{rgb}{0.000000,0.000000,0.000000}%
\pgfsetstrokecolor{textcolor}%
\pgfsetfillcolor{textcolor}%
\pgftext[x=3.918194in,y=0.933721in,left,base]{\color{textcolor}\rmfamily\fontsize{9.000000}{10.800000}\selectfont DMC+HTB}%
\end{pgfscope}%
\begin{pgfscope}%
\pgfsetrectcap%
\pgfsetroundjoin%
\pgfsetlinewidth{1.003750pt}%
\definecolor{currentstroke}{rgb}{1.000000,0.647059,0.000000}%
\pgfsetstrokecolor{currentstroke}%
\pgfsetdash{}{0pt}%
\pgfpathmoveto{\pgfqpoint{3.643194in}{0.815672in}}%
\pgfpathlineto{\pgfqpoint{3.893194in}{0.815672in}}%
\pgfusepath{stroke}%
\end{pgfscope}%
\begin{pgfscope}%
\definecolor{textcolor}{rgb}{0.000000,0.000000,0.000000}%
\pgfsetstrokecolor{textcolor}%
\pgfsetfillcolor{textcolor}%
\pgftext[x=3.918194in,y=0.771922in,left,base]{\color{textcolor}\rmfamily\fontsize{9.000000}{10.800000}\selectfont DMC+LG}%
\end{pgfscope}%
\begin{pgfscope}%
\pgfsetbuttcap%
\pgfsetroundjoin%
\pgfsetlinewidth{1.003750pt}%
\definecolor{currentstroke}{rgb}{0.000000,0.000000,1.000000}%
\pgfsetstrokecolor{currentstroke}%
\pgfsetdash{{3.700000pt}{1.600000pt}}{0.000000pt}%
\pgfpathmoveto{\pgfqpoint{3.643194in}{0.653872in}}%
\pgfpathlineto{\pgfqpoint{3.893194in}{0.653872in}}%
\pgfusepath{stroke}%
\end{pgfscope}%
\begin{pgfscope}%
\definecolor{textcolor}{rgb}{0.000000,0.000000,0.000000}%
\pgfsetstrokecolor{textcolor}%
\pgfsetfillcolor{textcolor}%
\pgftext[x=3.918194in,y=0.610122in,left,base]{\color{textcolor}\rmfamily\fontsize{9.000000}{10.800000}\selectfont d4}%
\end{pgfscope}%
\begin{pgfscope}%
\pgfsetbuttcap%
\pgfsetroundjoin%
\pgfsetlinewidth{1.003750pt}%
\definecolor{currentstroke}{rgb}{0.000000,0.501961,0.000000}%
\pgfsetstrokecolor{currentstroke}%
\pgfsetdash{{3.700000pt}{1.600000pt}}{0.000000pt}%
\pgfpathmoveto{\pgfqpoint{4.891695in}{1.301071in}}%
\pgfpathlineto{\pgfqpoint{5.141695in}{1.301071in}}%
\pgfusepath{stroke}%
\end{pgfscope}%
\begin{pgfscope}%
\definecolor{textcolor}{rgb}{0.000000,0.000000,0.000000}%
\pgfsetstrokecolor{textcolor}%
\pgfsetfillcolor{textcolor}%
\pgftext[x=5.166695in,y=1.257321in,left,base]{\color{textcolor}\rmfamily\fontsize{9.000000}{10.800000}\selectfont miniC2D}%
\end{pgfscope}%
\begin{pgfscope}%
\pgfsetbuttcap%
\pgfsetroundjoin%
\pgfsetlinewidth{1.003750pt}%
\definecolor{currentstroke}{rgb}{0.000000,0.000000,0.000000}%
\pgfsetstrokecolor{currentstroke}%
\pgfsetdash{{3.700000pt}{1.600000pt}}{0.000000pt}%
\pgfpathmoveto{\pgfqpoint{4.891695in}{1.139271in}}%
\pgfpathlineto{\pgfqpoint{5.141695in}{1.139271in}}%
\pgfusepath{stroke}%
\end{pgfscope}%
\begin{pgfscope}%
\definecolor{textcolor}{rgb}{0.000000,0.000000,0.000000}%
\pgfsetstrokecolor{textcolor}%
\pgfsetfillcolor{textcolor}%
\pgftext[x=5.166695in,y=1.095521in,left,base]{\color{textcolor}\rmfamily\fontsize{9.000000}{10.800000}\selectfont cachet}%
\end{pgfscope}%
\begin{pgfscope}%
\pgfsetbuttcap%
\pgfsetroundjoin%
\pgfsetlinewidth{1.003750pt}%
\definecolor{currentstroke}{rgb}{1.000000,0.647059,0.000000}%
\pgfsetstrokecolor{currentstroke}%
\pgfsetdash{{3.700000pt}{1.600000pt}}{0.000000pt}%
\pgfpathmoveto{\pgfqpoint{4.891695in}{0.977471in}}%
\pgfpathlineto{\pgfqpoint{5.141695in}{0.977471in}}%
\pgfusepath{stroke}%
\end{pgfscope}%
\begin{pgfscope}%
\definecolor{textcolor}{rgb}{0.000000,0.000000,0.000000}%
\pgfsetstrokecolor{textcolor}%
\pgfsetfillcolor{textcolor}%
\pgftext[x=5.166695in,y=0.933721in,left,base]{\color{textcolor}\rmfamily\fontsize{9.000000}{10.800000}\selectfont c2d}%
\end{pgfscope}%
\begin{pgfscope}%
\pgfsetbuttcap%
\pgfsetroundjoin%
\pgfsetlinewidth{1.003750pt}%
\definecolor{currentstroke}{rgb}{1.000000,0.000000,0.000000}%
\pgfsetstrokecolor{currentstroke}%
\pgfsetdash{{1.000000pt}{1.650000pt}}{0.000000pt}%
\pgfpathmoveto{\pgfqpoint{4.891695in}{0.815672in}}%
\pgfpathlineto{\pgfqpoint{5.141695in}{0.815672in}}%
\pgfusepath{stroke}%
\end{pgfscope}%
\begin{pgfscope}%
\definecolor{textcolor}{rgb}{0.000000,0.000000,0.000000}%
\pgfsetstrokecolor{textcolor}%
\pgfsetfillcolor{textcolor}%
\pgftext[x=5.166695in,y=0.771922in,left,base]{\color{textcolor}\rmfamily\fontsize{9.000000}{10.800000}\selectfont VBS*}%
\end{pgfscope}%
\begin{pgfscope}%
\pgfsetbuttcap%
\pgfsetroundjoin%
\pgfsetlinewidth{1.003750pt}%
\definecolor{currentstroke}{rgb}{0.000000,0.000000,0.000000}%
\pgfsetstrokecolor{currentstroke}%
\pgfsetdash{{1.000000pt}{1.650000pt}}{0.000000pt}%
\pgfpathmoveto{\pgfqpoint{4.891695in}{0.653872in}}%
\pgfpathlineto{\pgfqpoint{5.141695in}{0.653872in}}%
\pgfusepath{stroke}%
\end{pgfscope}%
\begin{pgfscope}%
\definecolor{textcolor}{rgb}{0.000000,0.000000,0.000000}%
\pgfsetstrokecolor{textcolor}%
\pgfsetfillcolor{textcolor}%
\pgftext[x=5.166695in,y=0.610122in,left,base]{\color{textcolor}\rmfamily\fontsize{9.000000}{10.800000}\selectfont VBS}%
\end{pgfscope}%
\end{pgfpicture}%
\makeatother%
\endgroup%

    \vspace*{-1cm}
	\caption{\label{fig:comparison} A cactus plot of the performance of four project-join-tree-based model counters, two state-of-the-art model counters, and two virtual best solvers: \tool{VBS*} (without project-join-tree-based counters) and \tool{VBS} (with project-join-tree-based counters).}
\end{figure}

Finally, we compare project-join-tree-based model counters with state-of-the-art tools for weighted model counting.
We construct four project-join-tree-based model counters by combining \Htb{} and \Lg{} (using the representative configurations from Experiment 1) with \Dmc{} and \Tensor{} (using the cost estimators for \Lg{} from Experiment 2).
Note that \Dmc{}+\Htb{} is equivalent to \tool{ADDMC} \cite{DPV20}.
% , and \Tensor{}+\Lg{} is equivalent to \tool{TensorOrder} \cite{DDV19}.
We compare against the state-of-the-art model counters \cachet{} \cite{sang2004combining}, \ctd{} \cite{darwiche2004new}, \df{} \cite{LM17}, and \minictd{} \cite{OD15}.
We ran each benchmark once with each model counter with a 1000-second timeout and recorded the total time taken.
For the project-join-tree-based model counters, time taken includes both the planning phase and the execution phase.

We present results from this experiment in Figure \ref{fig:comparison}.
For each benchmark, the solving time of \tool{VBS*} is the shortest solving time among all pre-existing model counters (\cachet, \ctd, \df, and \minictd).
Similarly, the time of \tool{VBS} is the shortest time among all model counters, including those based on project-join trees.
We observe that \tool{VBS} performs significantly better than \tool{VBS*}.
In fact, \Dmc{}+\Lg{} is the fastest model counter on 471 of \benchmarkCountAltogether{} benchmarks.
Thus project-join-tree-based tools are valuable for portfolios of weighted model counters.
