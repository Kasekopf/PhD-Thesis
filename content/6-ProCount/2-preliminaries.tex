% \paragraph{\textbf{Weighted Projected Model Counting.}}
\subsection{Weighted Projected Model Counting}

We compute the total weight, subject to a given weight function and a set of irrelevant variables, of all models of an input propositional formula.
Formally:
\begin{definition}
     Let $\phi$ be a Boolean formula, $\{X, Y\}$ be a partition of $\vars(\phi)$, and $W: 2^X \to \R$ be a pseudo-Boolean function. We say that $(X, Y, \phi, W)$ is an instance of \emph{weighted projected model counting}.
    The \emph{$W$-weighted $Y$-projected model count} of $\phi$ is
    $\func{WPMC}(\phi, W, Y) \equiv \sum_{\tau \in 2^X} \pars{ W(\tau) \mult \max_{\sigma \in 2^Y} [\phi](\tau \cup \sigma) }$.
\end{definition}

Variables in $X$ are called \emph{relevant} or \emph{additive}, and variables in $Y$ are called \emph{irrelevant} or \emph{disjunctive}. 
For the special case of \emph{unprojected model counting}, all variables are relevant, and the \emph{$W$-weighted model count} is $\func{WPMC}(\phi, W, \emptyset)$.

Weights are usually given by a \emph{literal-weight function} $W = \prod_{x \in X} W_x$, where the factors are functions $W_x : 2^{\set{x}} \to \R$.
In detail, a positive literal $x$ has weight $W_x(\set{x})$, and a negative literal $\neg x$ has weight $W_x(\emptyset)$.

% \paragraph{\textbf{Graphs.}}