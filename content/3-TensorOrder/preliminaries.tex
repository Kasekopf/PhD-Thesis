\section{Preliminaries: Graph Notations}
\label{sec:prelim}
A \emph{graph} $G$ has a nonempty set of vertices $\V{G}$, a set of (undirected) edges $\E{G}$, a function $\delta_G: \V{G} \rightarrow 2^{\E{G}}$ that gives the set of edges incident to each vertex, and a function $\epsilon_G: \E{G} \rightarrow 2^{\V{G}}$ that gives the set of vertices incident to each edge. Each edge must be incident to exactly two vertices, but multiple edges can exist between two vertices. An \emph{edge clique cover} of a graph $G$ is a set $A \subseteq 2^{\V{G}}$ such that (1) every vertex $v \in \V{G}$ is an element of some set in $A$, and (2) every element of $A$ is a clique in $G$ (that is, for every $C \in A$ and every pair of distinct $v, w \in C$ there is an edge between $v$ and $w$ in $G$).

A \emph{tree} is a simple, connected, and acyclic graph. A \emph{leaf} of a tree $T$ is a vertex of degree one, and we use $\Lv{T}$ to denote the set of leaves of $T$. A \emph{rooted binary tree} is a tree $T$ where either $T$ consists of a single vertex (called the \emph{root}), or every vertex of $T$ has degree one or three except a single vertex of degree two (called the \emph{root}). If $|\V{T}| > 1$, the \emph{immediate subtrees of $T$} are the two rooted binary trees that are the connected components of $T$ after the root is removed. Throughout this work, we often refer to a vertex of a tree as a \emph{node} and an edge as an \emph{arc} to avoid confusion, since our proofs will frequently work simultaneously with a graph and an associated tree.

In this work, we use two decompositions of a graph as a tree: carving decompositions \cite{ST94} and tree decompositions \cite{RS91}. Both decompose the graph into an \emph{unrooted binary tree}, which is a tree where every vertex has degree one or three. First, we describe carving decompositions \cite{ST94}:
\begin{definition}[Carving Decomposition]
\label{def:carving}
	Let $G$ be a graph. A \emph{carving decomposition} for $G$ is an unrooted binary tree $T$ whose leaves are the vertices of $G$, i.e. $\Lv{T} = \V{G}$. 
	
    For every arc $a$ of $T$, deleting $a$ from $T$ yields exactly two trees, whose leaves define a partition of the vertices of $G$. Let $C_a \subseteq \V{G}$ be an arbitrary element of this partition. The \emph{width} of $T$, denoted $width_c(T)$, is the maximum number of edges in $G$ between $C_a$ and $\V{G} \setminus C_a$ for all $a \in \E{T}$, i.e.,
    
	$$width_c(T) = \max_{a \in \E{T}} \left| \left( \bigcup_{v \in C_a} \vinc{G}{v} \right) \cap \left( \bigcup_{v \in \V{G} \setminus C_a} \vinc{G}{v} \right) \right|.$$
	
	% $$width_c(T) = \max_{a \in \E{T}} \left| \{ e \in \E{G}~\text{s.t.}~\einc{G}{e} \cap C_a \neq \emptyset~\text{and}~\epsilon_G(e) \cap (\V{G} \setminus C_a) \neq \emptyset \}\right|.$$
	
	
	% $$width_c(T) = \max_{a \in \E{T}} \left| \vinc{G}{C_a} \cap \vinc{G}{\V{G} \setminus C_a} \right|.$$
	
    The width of a carving decomposition $T$ with no edges is 0.
\end{definition}

%In Definition \ref{def:carving}, $\vinc{G}{V}$ refers to the set of edges of $G$ incident to some vertex in $V \subseteq \V{G}$.
The \emph{carving width} of a graph $G$, denoted $width_c(G)$, is the minimum width across all carving decompositions for $G$. Note that an equivalent definition of carving decompositions allows for degree two vertices within the tree as well.

Next, we define tree decompositions \cite{RS91}:
\begin{definition}[Tree Decomposition]
	Let $G$ be a graph. A \emph{tree decomposition} for $G$ is an unrooted binary tree $T$ together with a labeling function $\chi : \V{T} \rightarrow 2^{\V{G}}$ that satisfies the following three properties:
	\begin{enumerate}
		\item Every vertex of $G$ is contained in the label of some node of $T$. That is, $\V{G} = \bigcup_{n \in \V{T}} \chi(n)$.
		\item For every edge $e \in \E{G}$, there is a node $n \in \V{T}$ whose label is a superset of $\einc{G}{e}$, i.e. $\einc{G}{e} \subseteq \chi(n)$.
		\item If $n$ and $o$ are nodes in $T$ and $p$ is a node on the path from $n$ to $o$, then $\chi(n) \cap \chi(o) \subseteq \chi(p)$.
	\end{enumerate}
	The \emph{width} of a tree decomposition, denoted $width_t(T, \chi)$, is the maximum size (minus 1) of the label of every node, i.e.,
	$$width_t(T, \chi) = \max_{n \in \V{T}} | \chi(n) | - 1.$$
\end{definition}

The \emph{treewidth} of a graph $G$, denoted $width_t(G)$, is the minimum width across all tree decompositions for $G$. The treewidth of a tree is $1$. Treewidth is bounded by thrice the carving width \cite{sasak10}. Carving decompositions are the dual of branch decompositions, which are closely related to tree decompositions \cite{RS91}.