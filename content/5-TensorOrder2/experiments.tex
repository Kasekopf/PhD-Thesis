\section{Implementation and Evaluation}
\label{sec:experiments}

We aim to answer the following experimental research questions:
\begin{enumerate}\itemsep0em 
    \item[(RQ1)] Is the planning phase improved by a parallel portfolio of decomposition tools?
    
    \item[(RQ2)] Is the planning phase improved by adding a branch-decomposition tool?
    
    \item[(RQ3)] When should Algorithm \ref{alg:wmc} transition from the planning phase to the execution phase (i.e., what should be the value of the performance factor $\alpha$)?
    
    \item[(RQ4)] Is the execution phase improved by leveraging multiple cores and a GPU?
    
    \item[(RQ5)] Is the execution phase improved by leveraging a TPU in graph execution mode?
    
    \item[(RQ6)] Do parallel tensor network approaches improve a portfolio of state-of-the-art weighted model counters (\tool{cachet}, \tool{miniC2D}, \tool{d4}, \tool{ADDMC}, and \tool{gpuSAT2})?
\end{enumerate}

We implement our changes on top of \tool{TensorOrder} (which implements Algorithm \ref{alg:wmc}; see Section \ref{sec:tensors:experiments:implementation}) to produce \tool{TensorOrder2}, a new parallel weighted model counter. Implementation details are described in Section \ref{sec:experiments:impl}. All code is available at  \url{https://github.com/vardigroup/TensorOrder}.

%\paragraph{Benchmarks.} 
We use a standard set\footnote{\url{https://github.com/vardigroup/ADDMC/releases/tag/v1.0.0}} of 1914 weighted model counting benchmarks \cite{DPV20}. Of these, 1091 benchmarks\footnote{\url{https://www.cs.rochester.edu/u/kautz/Cachet/}} are from Bayesian inference problems \cite{SBK05} and 823 benchmarks\footnote{\url{http://www.cril.univ-artois.fr/KC/benchmarks.html}} are unweighted benchmarks (from various domains) that were artificially weighted by \cite{DPV20}. For weighted model counters that cannot handle real weights larger than 1 (\tool{cachet} and \tool{gpusat2}), we rescale the weights of benchmarks with larger weights. In Experiment 3, we also consider preprocessing these 1914 benchmarks by applying $\pkg{pmc-eq}$ \cite{LM14} (which preserves weighted model count). % We apply to each benchmark, with a timeout of 1000 seconds. We remove 61 benchmarks that were fully solved by $\pkg{pmc-eq}$ (either UNSAT., or with a single solution) and 11 benchmarks that timed out, resulting in 1842 preprocessed benchmarks.
We evaluate the performance of each tool using the PAR-2 score, which is the sum of of the wall-clock times for each completed benchmark, plus twice the timeout for each uncompleted benchmark.

%\paragraph{Experimental Setup.}
All counters are run in Docker images (one for each counter) with Docker 19.03.5. All experiments are run on Google Cloud \tool{n1-standard-8} machines with 8 cores (Intel Haswell, 2.3 GHz) and 30 GB RAM. GPU-based counters are provided an \tool{NVIDIA Tesla V100} GPU (16 GB of onboard RAM) using NVIDIA driver 418.67 and CUDA 10.1.243. TPU-based counters are provided a v2-8 TPUs, which contains 8 TPU cores and 8 GB of onboard RAM per core.

\subsection{Implementation Details of \tool{TensorOrder2}}
\label{sec:experiments:impl}
\tool{TensorOrder2} is primarily implemented in Python 3 as a modified version of the tool \tool{TensorOrder} from Chapter \ref{ch:tensors}. We replace portions of the Python code with C++ (\tool{g++} v7.4.0) using Cython 0.29.15 for general speedup, especially in the implementation of the \textbf{FT} planner.

\paragraph{Planning.} 
\tool{TensorOrder} contains an implementation of the planning phase using the \textbf{FT} planner together with a choice of three single-core tree-decomposition solvers: \pkg{Tamaki} \cite{Tamaki17}, \pkg{FlowCutter} \cite{HS18}, and \pkg{htd} \cite{AMW17}. We add to \tool{TensorOrder2} an implementation of Theorem \ref{thm:factorable-branch} and use it to add a branch-decomposition solver \pkg{Hicks} \cite{hicks02}.
We implement a parallel portfolio of graph-decomposition solvers in C++ and give \tool{TensorOrder2} access to two portfolios, each with access to all cores: \pkg{P3} (which combines \pkg{Tamaki},  \pkg{FlowCutter}, and \pkg{htd}) and \pkg{P4} (which includes \pkg{Hicks} as well).

\paragraph{Execution.} 
\tool{TensorOrder} is able to perform the execution phase on a single core and on multiple cores using \pkg{numpy} v1.18.1 and \pkg{OpenBLAS} v0.2.20. 

We add to \tool{TensorOrder2} a way to contract tensors on a GPU with \pkg{TensorFlow} v2.1.0 \cite{ABCCDDDGII16}. To avoid GPU kernel calls for small contractions, \tool{TensorOrder2} uses a GPU only for contractions where one of the tensors involved has rank $\geq 20$, and reverts back to using multi-core \pkg{numpy} otherwise. We also add an implementation of Algorithm \ref{alg:tn-sliced}.

We also add a way to contract tensors on a TPU with \pkg{jax} v0.2.13 \cite{jax2018github} in graph execution mode. 
We execute each slice (i.e., each call to \func{Execute} on Line \ref{line:tn-sliced:execute} of Algorithm \ref{alg:tn-sliced}) on a separate TPU core. 
Thus on v2-8 TPUs we execute 8 slices in parallel. 
For comparison, we also add support for contracting tensors on multiple-CPUs with \pkg{jax} in graph execution mode.

Overall, \tool{TensorOrder2} runs the execution phase in three possible configurations in eager execution mode-- \pkg{CPU1} (restricted to a single CPU core), \pkg{CPU8} (allowed to use all 8 CPU cores), \pkg{GPU} (allowed to use all 8 CPU cores and use a GPU)-- and two possible configurations in graph execution mode-- \pkg{TPU-graph} (allowed to use all 8 CPU cores, and a TPU), and \pkg{CPU8-graph} (allowed to use all 8 CPU cores).

Finally, note that \tool{TensorOrder2} supports 64-bit floats for all execution configurations except for \pkg{TPU-graph}, which supports only 32-bit floats due to TPU hardware limitations. We therefore output 64-bit floats in Experiments 1, 2, and 3, and output 32-bit floats for Experiment 4.

\begin{figure}
	\centering
	%% Creator: Matplotlib, PGF backend
%%
%% To include the figure in your LaTeX document, write
%%   \input{<filename>.pgf}
%%
%% Make sure the required packages are loaded in your preamble
%%   \usepackage{pgf}
%%
%% and, on pdftex
%%   \usepackage[utf8]{inputenc}\DeclareUnicodeCharacter{2212}{-}
%%
%% or, on luatex and xetex
%%   \usepackage{unicode-math}
%%
%% Figures using additional raster images can only be included by \input if
%% they are in the same directory as the main LaTeX file. For loading figures
%% from other directories you can use the `import` package
%%   \usepackage{import}
%%
%% and then include the figures with
%%   \import{<path to file>}{<filename>.pgf}
%%
%% Matplotlib used the following preamble
%%   \usepackage[utf8x]{inputenc}
%%   \usepackage[T1]{fontenc}
%%
\begingroup%
\makeatletter%
\begin{pgfpicture}%
\pgfpathrectangle{\pgfpointorigin}{\pgfqpoint{4.803148in}{2.021259in}}%
\pgfusepath{use as bounding box, clip}%
\begin{pgfscope}%
\pgfsetbuttcap%
\pgfsetmiterjoin%
\definecolor{currentfill}{rgb}{1.000000,1.000000,1.000000}%
\pgfsetfillcolor{currentfill}%
\pgfsetlinewidth{0.000000pt}%
\definecolor{currentstroke}{rgb}{1.000000,1.000000,1.000000}%
\pgfsetstrokecolor{currentstroke}%
\pgfsetdash{}{0pt}%
\pgfpathmoveto{\pgfqpoint{0.000000in}{0.000000in}}%
\pgfpathlineto{\pgfqpoint{4.803148in}{0.000000in}}%
\pgfpathlineto{\pgfqpoint{4.803148in}{2.021259in}}%
\pgfpathlineto{\pgfqpoint{0.000000in}{2.021259in}}%
\pgfpathclose%
\pgfusepath{fill}%
\end{pgfscope}%
\begin{pgfscope}%
\pgfsetbuttcap%
\pgfsetmiterjoin%
\definecolor{currentfill}{rgb}{1.000000,1.000000,1.000000}%
\pgfsetfillcolor{currentfill}%
\pgfsetlinewidth{0.000000pt}%
\definecolor{currentstroke}{rgb}{0.000000,0.000000,0.000000}%
\pgfsetstrokecolor{currentstroke}%
\pgfsetstrokeopacity{0.000000}%
\pgfsetdash{}{0pt}%
\pgfpathmoveto{\pgfqpoint{0.694334in}{0.523557in}}%
\pgfpathlineto{\pgfqpoint{4.524677in}{0.523557in}}%
\pgfpathlineto{\pgfqpoint{4.524677in}{1.826535in}}%
\pgfpathlineto{\pgfqpoint{0.694334in}{1.826535in}}%
\pgfpathclose%
\pgfusepath{fill}%
\end{pgfscope}%
\begin{pgfscope}%
\pgfsetbuttcap%
\pgfsetroundjoin%
\definecolor{currentfill}{rgb}{0.000000,0.000000,0.000000}%
\pgfsetfillcolor{currentfill}%
\pgfsetlinewidth{0.803000pt}%
\definecolor{currentstroke}{rgb}{0.000000,0.000000,0.000000}%
\pgfsetstrokecolor{currentstroke}%
\pgfsetdash{}{0pt}%
\pgfsys@defobject{currentmarker}{\pgfqpoint{0.000000in}{-0.048611in}}{\pgfqpoint{0.000000in}{0.000000in}}{%
\pgfpathmoveto{\pgfqpoint{0.000000in}{0.000000in}}%
\pgfpathlineto{\pgfqpoint{0.000000in}{-0.048611in}}%
\pgfusepath{stroke,fill}%
}%
\begin{pgfscope}%
\pgfsys@transformshift{0.694334in}{0.523557in}%
\pgfsys@useobject{currentmarker}{}%
\end{pgfscope}%
\end{pgfscope}%
\begin{pgfscope}%
\definecolor{textcolor}{rgb}{0.000000,0.000000,0.000000}%
\pgfsetstrokecolor{textcolor}%
\pgfsetfillcolor{textcolor}%
\pgftext[x=0.694334in,y=0.426335in,,top]{\color{textcolor}\rmfamily\fontsize{9.000000}{10.800000}\selectfont \(\displaystyle 0\)}%
\end{pgfscope}%
\begin{pgfscope}%
\pgfsetbuttcap%
\pgfsetroundjoin%
\definecolor{currentfill}{rgb}{0.000000,0.000000,0.000000}%
\pgfsetfillcolor{currentfill}%
\pgfsetlinewidth{0.803000pt}%
\definecolor{currentstroke}{rgb}{0.000000,0.000000,0.000000}%
\pgfsetstrokecolor{currentstroke}%
\pgfsetdash{}{0pt}%
\pgfsys@defobject{currentmarker}{\pgfqpoint{0.000000in}{-0.048611in}}{\pgfqpoint{0.000000in}{0.000000in}}{%
\pgfpathmoveto{\pgfqpoint{0.000000in}{0.000000in}}%
\pgfpathlineto{\pgfqpoint{0.000000in}{-0.048611in}}%
\pgfusepath{stroke,fill}%
}%
\begin{pgfscope}%
\pgfsys@transformshift{1.173127in}{0.523557in}%
\pgfsys@useobject{currentmarker}{}%
\end{pgfscope}%
\end{pgfscope}%
\begin{pgfscope}%
\definecolor{textcolor}{rgb}{0.000000,0.000000,0.000000}%
\pgfsetstrokecolor{textcolor}%
\pgfsetfillcolor{textcolor}%
\pgftext[x=1.173127in,y=0.426335in,,top]{\color{textcolor}\rmfamily\fontsize{9.000000}{10.800000}\selectfont \(\displaystyle 250\)}%
\end{pgfscope}%
\begin{pgfscope}%
\pgfsetbuttcap%
\pgfsetroundjoin%
\definecolor{currentfill}{rgb}{0.000000,0.000000,0.000000}%
\pgfsetfillcolor{currentfill}%
\pgfsetlinewidth{0.803000pt}%
\definecolor{currentstroke}{rgb}{0.000000,0.000000,0.000000}%
\pgfsetstrokecolor{currentstroke}%
\pgfsetdash{}{0pt}%
\pgfsys@defobject{currentmarker}{\pgfqpoint{0.000000in}{-0.048611in}}{\pgfqpoint{0.000000in}{0.000000in}}{%
\pgfpathmoveto{\pgfqpoint{0.000000in}{0.000000in}}%
\pgfpathlineto{\pgfqpoint{0.000000in}{-0.048611in}}%
\pgfusepath{stroke,fill}%
}%
\begin{pgfscope}%
\pgfsys@transformshift{1.651920in}{0.523557in}%
\pgfsys@useobject{currentmarker}{}%
\end{pgfscope}%
\end{pgfscope}%
\begin{pgfscope}%
\definecolor{textcolor}{rgb}{0.000000,0.000000,0.000000}%
\pgfsetstrokecolor{textcolor}%
\pgfsetfillcolor{textcolor}%
\pgftext[x=1.651920in,y=0.426335in,,top]{\color{textcolor}\rmfamily\fontsize{9.000000}{10.800000}\selectfont \(\displaystyle 500\)}%
\end{pgfscope}%
\begin{pgfscope}%
\pgfsetbuttcap%
\pgfsetroundjoin%
\definecolor{currentfill}{rgb}{0.000000,0.000000,0.000000}%
\pgfsetfillcolor{currentfill}%
\pgfsetlinewidth{0.803000pt}%
\definecolor{currentstroke}{rgb}{0.000000,0.000000,0.000000}%
\pgfsetstrokecolor{currentstroke}%
\pgfsetdash{}{0pt}%
\pgfsys@defobject{currentmarker}{\pgfqpoint{0.000000in}{-0.048611in}}{\pgfqpoint{0.000000in}{0.000000in}}{%
\pgfpathmoveto{\pgfqpoint{0.000000in}{0.000000in}}%
\pgfpathlineto{\pgfqpoint{0.000000in}{-0.048611in}}%
\pgfusepath{stroke,fill}%
}%
\begin{pgfscope}%
\pgfsys@transformshift{2.130713in}{0.523557in}%
\pgfsys@useobject{currentmarker}{}%
\end{pgfscope}%
\end{pgfscope}%
\begin{pgfscope}%
\definecolor{textcolor}{rgb}{0.000000,0.000000,0.000000}%
\pgfsetstrokecolor{textcolor}%
\pgfsetfillcolor{textcolor}%
\pgftext[x=2.130713in,y=0.426335in,,top]{\color{textcolor}\rmfamily\fontsize{9.000000}{10.800000}\selectfont \(\displaystyle 750\)}%
\end{pgfscope}%
\begin{pgfscope}%
\pgfsetbuttcap%
\pgfsetroundjoin%
\definecolor{currentfill}{rgb}{0.000000,0.000000,0.000000}%
\pgfsetfillcolor{currentfill}%
\pgfsetlinewidth{0.803000pt}%
\definecolor{currentstroke}{rgb}{0.000000,0.000000,0.000000}%
\pgfsetstrokecolor{currentstroke}%
\pgfsetdash{}{0pt}%
\pgfsys@defobject{currentmarker}{\pgfqpoint{0.000000in}{-0.048611in}}{\pgfqpoint{0.000000in}{0.000000in}}{%
\pgfpathmoveto{\pgfqpoint{0.000000in}{0.000000in}}%
\pgfpathlineto{\pgfqpoint{0.000000in}{-0.048611in}}%
\pgfusepath{stroke,fill}%
}%
\begin{pgfscope}%
\pgfsys@transformshift{2.609506in}{0.523557in}%
\pgfsys@useobject{currentmarker}{}%
\end{pgfscope}%
\end{pgfscope}%
\begin{pgfscope}%
\definecolor{textcolor}{rgb}{0.000000,0.000000,0.000000}%
\pgfsetstrokecolor{textcolor}%
\pgfsetfillcolor{textcolor}%
\pgftext[x=2.609506in,y=0.426335in,,top]{\color{textcolor}\rmfamily\fontsize{9.000000}{10.800000}\selectfont \(\displaystyle 1000\)}%
\end{pgfscope}%
\begin{pgfscope}%
\pgfsetbuttcap%
\pgfsetroundjoin%
\definecolor{currentfill}{rgb}{0.000000,0.000000,0.000000}%
\pgfsetfillcolor{currentfill}%
\pgfsetlinewidth{0.803000pt}%
\definecolor{currentstroke}{rgb}{0.000000,0.000000,0.000000}%
\pgfsetstrokecolor{currentstroke}%
\pgfsetdash{}{0pt}%
\pgfsys@defobject{currentmarker}{\pgfqpoint{0.000000in}{-0.048611in}}{\pgfqpoint{0.000000in}{0.000000in}}{%
\pgfpathmoveto{\pgfqpoint{0.000000in}{0.000000in}}%
\pgfpathlineto{\pgfqpoint{0.000000in}{-0.048611in}}%
\pgfusepath{stroke,fill}%
}%
\begin{pgfscope}%
\pgfsys@transformshift{3.088299in}{0.523557in}%
\pgfsys@useobject{currentmarker}{}%
\end{pgfscope}%
\end{pgfscope}%
\begin{pgfscope}%
\definecolor{textcolor}{rgb}{0.000000,0.000000,0.000000}%
\pgfsetstrokecolor{textcolor}%
\pgfsetfillcolor{textcolor}%
\pgftext[x=3.088299in,y=0.426335in,,top]{\color{textcolor}\rmfamily\fontsize{9.000000}{10.800000}\selectfont \(\displaystyle 1250\)}%
\end{pgfscope}%
\begin{pgfscope}%
\pgfsetbuttcap%
\pgfsetroundjoin%
\definecolor{currentfill}{rgb}{0.000000,0.000000,0.000000}%
\pgfsetfillcolor{currentfill}%
\pgfsetlinewidth{0.803000pt}%
\definecolor{currentstroke}{rgb}{0.000000,0.000000,0.000000}%
\pgfsetstrokecolor{currentstroke}%
\pgfsetdash{}{0pt}%
\pgfsys@defobject{currentmarker}{\pgfqpoint{0.000000in}{-0.048611in}}{\pgfqpoint{0.000000in}{0.000000in}}{%
\pgfpathmoveto{\pgfqpoint{0.000000in}{0.000000in}}%
\pgfpathlineto{\pgfqpoint{0.000000in}{-0.048611in}}%
\pgfusepath{stroke,fill}%
}%
\begin{pgfscope}%
\pgfsys@transformshift{3.567091in}{0.523557in}%
\pgfsys@useobject{currentmarker}{}%
\end{pgfscope}%
\end{pgfscope}%
\begin{pgfscope}%
\definecolor{textcolor}{rgb}{0.000000,0.000000,0.000000}%
\pgfsetstrokecolor{textcolor}%
\pgfsetfillcolor{textcolor}%
\pgftext[x=3.567091in,y=0.426335in,,top]{\color{textcolor}\rmfamily\fontsize{9.000000}{10.800000}\selectfont \(\displaystyle 1500\)}%
\end{pgfscope}%
\begin{pgfscope}%
\pgfsetbuttcap%
\pgfsetroundjoin%
\definecolor{currentfill}{rgb}{0.000000,0.000000,0.000000}%
\pgfsetfillcolor{currentfill}%
\pgfsetlinewidth{0.803000pt}%
\definecolor{currentstroke}{rgb}{0.000000,0.000000,0.000000}%
\pgfsetstrokecolor{currentstroke}%
\pgfsetdash{}{0pt}%
\pgfsys@defobject{currentmarker}{\pgfqpoint{0.000000in}{-0.048611in}}{\pgfqpoint{0.000000in}{0.000000in}}{%
\pgfpathmoveto{\pgfqpoint{0.000000in}{0.000000in}}%
\pgfpathlineto{\pgfqpoint{0.000000in}{-0.048611in}}%
\pgfusepath{stroke,fill}%
}%
\begin{pgfscope}%
\pgfsys@transformshift{4.045884in}{0.523557in}%
\pgfsys@useobject{currentmarker}{}%
\end{pgfscope}%
\end{pgfscope}%
\begin{pgfscope}%
\definecolor{textcolor}{rgb}{0.000000,0.000000,0.000000}%
\pgfsetstrokecolor{textcolor}%
\pgfsetfillcolor{textcolor}%
\pgftext[x=4.045884in,y=0.426335in,,top]{\color{textcolor}\rmfamily\fontsize{9.000000}{10.800000}\selectfont \(\displaystyle 1750\)}%
\end{pgfscope}%
\begin{pgfscope}%
\pgfsetbuttcap%
\pgfsetroundjoin%
\definecolor{currentfill}{rgb}{0.000000,0.000000,0.000000}%
\pgfsetfillcolor{currentfill}%
\pgfsetlinewidth{0.803000pt}%
\definecolor{currentstroke}{rgb}{0.000000,0.000000,0.000000}%
\pgfsetstrokecolor{currentstroke}%
\pgfsetdash{}{0pt}%
\pgfsys@defobject{currentmarker}{\pgfqpoint{0.000000in}{-0.048611in}}{\pgfqpoint{0.000000in}{0.000000in}}{%
\pgfpathmoveto{\pgfqpoint{0.000000in}{0.000000in}}%
\pgfpathlineto{\pgfqpoint{0.000000in}{-0.048611in}}%
\pgfusepath{stroke,fill}%
}%
\begin{pgfscope}%
\pgfsys@transformshift{4.524677in}{0.523557in}%
\pgfsys@useobject{currentmarker}{}%
\end{pgfscope}%
\end{pgfscope}%
\begin{pgfscope}%
\definecolor{textcolor}{rgb}{0.000000,0.000000,0.000000}%
\pgfsetstrokecolor{textcolor}%
\pgfsetfillcolor{textcolor}%
\pgftext[x=4.524677in,y=0.426335in,,top]{\color{textcolor}\rmfamily\fontsize{9.000000}{10.800000}\selectfont \(\displaystyle 2000\)}%
\end{pgfscope}%
\begin{pgfscope}%
\definecolor{textcolor}{rgb}{0.000000,0.000000,0.000000}%
\pgfsetstrokecolor{textcolor}%
\pgfsetfillcolor{textcolor}%
\pgftext[x=2.609506in,y=0.260390in,,top]{\color{textcolor}\rmfamily\fontsize{9.000000}{10.800000}\selectfont Number of benchmarks solved}%
\end{pgfscope}%
\begin{pgfscope}%
\pgfsetbuttcap%
\pgfsetroundjoin%
\definecolor{currentfill}{rgb}{0.000000,0.000000,0.000000}%
\pgfsetfillcolor{currentfill}%
\pgfsetlinewidth{0.803000pt}%
\definecolor{currentstroke}{rgb}{0.000000,0.000000,0.000000}%
\pgfsetstrokecolor{currentstroke}%
\pgfsetdash{}{0pt}%
\pgfsys@defobject{currentmarker}{\pgfqpoint{-0.048611in}{0.000000in}}{\pgfqpoint{0.000000in}{0.000000in}}{%
\pgfpathmoveto{\pgfqpoint{0.000000in}{0.000000in}}%
\pgfpathlineto{\pgfqpoint{-0.048611in}{0.000000in}}%
\pgfusepath{stroke,fill}%
}%
\begin{pgfscope}%
\pgfsys@transformshift{0.694334in}{0.843347in}%
\pgfsys@useobject{currentmarker}{}%
\end{pgfscope}%
\end{pgfscope}%
\begin{pgfscope}%
\definecolor{textcolor}{rgb}{0.000000,0.000000,0.000000}%
\pgfsetstrokecolor{textcolor}%
\pgfsetfillcolor{textcolor}%
\pgftext[x=0.330525in, y=0.798622in, left, base]{\color{textcolor}\rmfamily\fontsize{9.000000}{10.800000}\selectfont \(\displaystyle 10^{-1}\)}%
\end{pgfscope}%
\begin{pgfscope}%
\pgfsetbuttcap%
\pgfsetroundjoin%
\definecolor{currentfill}{rgb}{0.000000,0.000000,0.000000}%
\pgfsetfillcolor{currentfill}%
\pgfsetlinewidth{0.803000pt}%
\definecolor{currentstroke}{rgb}{0.000000,0.000000,0.000000}%
\pgfsetstrokecolor{currentstroke}%
\pgfsetdash{}{0pt}%
\pgfsys@defobject{currentmarker}{\pgfqpoint{-0.048611in}{0.000000in}}{\pgfqpoint{0.000000in}{0.000000in}}{%
\pgfpathmoveto{\pgfqpoint{0.000000in}{0.000000in}}%
\pgfpathlineto{\pgfqpoint{-0.048611in}{0.000000in}}%
\pgfusepath{stroke,fill}%
}%
\begin{pgfscope}%
\pgfsys@transformshift{0.694334in}{1.334941in}%
\pgfsys@useobject{currentmarker}{}%
\end{pgfscope}%
\end{pgfscope}%
\begin{pgfscope}%
\definecolor{textcolor}{rgb}{0.000000,0.000000,0.000000}%
\pgfsetstrokecolor{textcolor}%
\pgfsetfillcolor{textcolor}%
\pgftext[x=0.410771in, y=1.290216in, left, base]{\color{textcolor}\rmfamily\fontsize{9.000000}{10.800000}\selectfont \(\displaystyle 10^{1}\)}%
\end{pgfscope}%
\begin{pgfscope}%
\pgfsetbuttcap%
\pgfsetroundjoin%
\definecolor{currentfill}{rgb}{0.000000,0.000000,0.000000}%
\pgfsetfillcolor{currentfill}%
\pgfsetlinewidth{0.803000pt}%
\definecolor{currentstroke}{rgb}{0.000000,0.000000,0.000000}%
\pgfsetstrokecolor{currentstroke}%
\pgfsetdash{}{0pt}%
\pgfsys@defobject{currentmarker}{\pgfqpoint{-0.048611in}{0.000000in}}{\pgfqpoint{0.000000in}{0.000000in}}{%
\pgfpathmoveto{\pgfqpoint{0.000000in}{0.000000in}}%
\pgfpathlineto{\pgfqpoint{-0.048611in}{0.000000in}}%
\pgfusepath{stroke,fill}%
}%
\begin{pgfscope}%
\pgfsys@transformshift{0.694334in}{1.826535in}%
\pgfsys@useobject{currentmarker}{}%
\end{pgfscope}%
\end{pgfscope}%
\begin{pgfscope}%
\definecolor{textcolor}{rgb}{0.000000,0.000000,0.000000}%
\pgfsetstrokecolor{textcolor}%
\pgfsetfillcolor{textcolor}%
\pgftext[x=0.410771in, y=1.781810in, left, base]{\color{textcolor}\rmfamily\fontsize{9.000000}{10.800000}\selectfont \(\displaystyle 10^{3}\)}%
\end{pgfscope}%
\begin{pgfscope}%
\definecolor{textcolor}{rgb}{0.000000,0.000000,0.000000}%
\pgfsetstrokecolor{textcolor}%
\pgfsetfillcolor{textcolor}%
\pgftext[x=0.274969in,y=1.175046in,,bottom,rotate=90.000000]{\color{textcolor}\rmfamily\fontsize{9.000000}{10.800000}\selectfont Longest solving time (s)}%
\end{pgfscope}%
\begin{pgfscope}%
\pgfpathrectangle{\pgfqpoint{0.694334in}{0.523557in}}{\pgfqpoint{3.830343in}{1.302977in}}%
\pgfusepath{clip}%
\pgfsetrectcap%
\pgfsetroundjoin%
\pgfsetlinewidth{1.003750pt}%
\definecolor{currentstroke}{rgb}{0.878431,0.878431,0.815686}%
\pgfsetstrokecolor{currentstroke}%
\pgfsetdash{}{0pt}%
\pgfpathmoveto{\pgfqpoint{0.694334in}{0.948144in}}%
\pgfpathlineto{\pgfqpoint{0.696249in}{0.955934in}}%
\pgfpathlineto{\pgfqpoint{0.701995in}{0.961880in}}%
\pgfpathlineto{\pgfqpoint{0.703910in}{0.964864in}}%
\pgfpathlineto{\pgfqpoint{0.705825in}{0.965524in}}%
\pgfpathlineto{\pgfqpoint{0.709656in}{0.973680in}}%
\pgfpathlineto{\pgfqpoint{0.711571in}{0.974495in}}%
\pgfpathlineto{\pgfqpoint{0.713486in}{0.976961in}}%
\pgfpathlineto{\pgfqpoint{0.719232in}{0.978816in}}%
\pgfpathlineto{\pgfqpoint{0.721147in}{0.980178in}}%
\pgfpathlineto{\pgfqpoint{0.724977in}{1.000991in}}%
\pgfpathlineto{\pgfqpoint{0.728807in}{1.015379in}}%
\pgfpathlineto{\pgfqpoint{0.730723in}{1.015482in}}%
\pgfpathlineto{\pgfqpoint{0.732638in}{1.022216in}}%
\pgfpathlineto{\pgfqpoint{0.734553in}{1.022392in}}%
\pgfpathlineto{\pgfqpoint{0.736468in}{1.024621in}}%
\pgfpathlineto{\pgfqpoint{0.738383in}{1.024629in}}%
\pgfpathlineto{\pgfqpoint{0.742214in}{1.030245in}}%
\pgfpathlineto{\pgfqpoint{0.744129in}{1.030409in}}%
\pgfpathlineto{\pgfqpoint{0.746044in}{1.033013in}}%
\pgfpathlineto{\pgfqpoint{0.747959in}{1.039599in}}%
\pgfpathlineto{\pgfqpoint{0.749874in}{1.040605in}}%
\pgfpathlineto{\pgfqpoint{0.755620in}{1.049992in}}%
\pgfpathlineto{\pgfqpoint{0.772856in}{1.056340in}}%
\pgfpathlineto{\pgfqpoint{0.786263in}{1.061350in}}%
\pgfpathlineto{\pgfqpoint{0.801584in}{1.062980in}}%
\pgfpathlineto{\pgfqpoint{0.805414in}{1.063733in}}%
\pgfpathlineto{\pgfqpoint{0.814990in}{1.065163in}}%
\pgfpathlineto{\pgfqpoint{0.882021in}{1.076339in}}%
\pgfpathlineto{\pgfqpoint{0.885851in}{1.077456in}}%
\pgfpathlineto{\pgfqpoint{0.895427in}{1.078655in}}%
\pgfpathlineto{\pgfqpoint{0.906918in}{1.080409in}}%
\pgfpathlineto{\pgfqpoint{0.987356in}{1.089407in}}%
\pgfpathlineto{\pgfqpoint{1.008422in}{1.090894in}}%
\pgfpathlineto{\pgfqpoint{1.052471in}{1.095137in}}%
\pgfpathlineto{\pgfqpoint{1.056302in}{1.095792in}}%
\pgfpathlineto{\pgfqpoint{1.077369in}{1.097817in}}%
\pgfpathlineto{\pgfqpoint{1.081199in}{1.099401in}}%
\pgfpathlineto{\pgfqpoint{1.096520in}{1.100833in}}%
\pgfpathlineto{\pgfqpoint{1.106096in}{1.102076in}}%
\pgfpathlineto{\pgfqpoint{1.159721in}{1.107209in}}%
\pgfpathlineto{\pgfqpoint{1.163551in}{1.109041in}}%
\pgfpathlineto{\pgfqpoint{1.173127in}{1.110874in}}%
\pgfpathlineto{\pgfqpoint{1.180788in}{1.111938in}}%
\pgfpathlineto{\pgfqpoint{1.186533in}{1.112998in}}%
\pgfpathlineto{\pgfqpoint{1.194194in}{1.114050in}}%
\pgfpathlineto{\pgfqpoint{1.213346in}{1.115660in}}%
\pgfpathlineto{\pgfqpoint{1.221006in}{1.118786in}}%
\pgfpathlineto{\pgfqpoint{1.228667in}{1.120114in}}%
\pgfpathlineto{\pgfqpoint{1.255480in}{1.127277in}}%
\pgfpathlineto{\pgfqpoint{1.265055in}{1.131122in}}%
\pgfpathlineto{\pgfqpoint{1.274631in}{1.132047in}}%
\pgfpathlineto{\pgfqpoint{1.278462in}{1.133795in}}%
\pgfpathlineto{\pgfqpoint{1.295698in}{1.138666in}}%
\pgfpathlineto{\pgfqpoint{1.299528in}{1.139717in}}%
\pgfpathlineto{\pgfqpoint{1.301444in}{1.139860in}}%
\pgfpathlineto{\pgfqpoint{1.303359in}{1.142985in}}%
\pgfpathlineto{\pgfqpoint{1.309104in}{1.143567in}}%
\pgfpathlineto{\pgfqpoint{1.311019in}{1.145858in}}%
\pgfpathlineto{\pgfqpoint{1.316765in}{1.146442in}}%
\pgfpathlineto{\pgfqpoint{1.320595in}{1.149572in}}%
\pgfpathlineto{\pgfqpoint{1.326341in}{1.150947in}}%
\pgfpathlineto{\pgfqpoint{1.334002in}{1.152019in}}%
\pgfpathlineto{\pgfqpoint{1.349323in}{1.155558in}}%
\pgfpathlineto{\pgfqpoint{1.351238in}{1.157653in}}%
\pgfpathlineto{\pgfqpoint{1.353153in}{1.157732in}}%
\pgfpathlineto{\pgfqpoint{1.356984in}{1.160260in}}%
\pgfpathlineto{\pgfqpoint{1.360814in}{1.160544in}}%
\pgfpathlineto{\pgfqpoint{1.368475in}{1.163791in}}%
\pgfpathlineto{\pgfqpoint{1.372305in}{1.164419in}}%
\pgfpathlineto{\pgfqpoint{1.381881in}{1.167150in}}%
\pgfpathlineto{\pgfqpoint{1.395287in}{1.168617in}}%
\pgfpathlineto{\pgfqpoint{1.402948in}{1.170470in}}%
\pgfpathlineto{\pgfqpoint{1.410608in}{1.171407in}}%
\pgfpathlineto{\pgfqpoint{1.420184in}{1.176099in}}%
\pgfpathlineto{\pgfqpoint{1.441251in}{1.178997in}}%
\pgfpathlineto{\pgfqpoint{1.446997in}{1.180239in}}%
\pgfpathlineto{\pgfqpoint{1.458488in}{1.181758in}}%
\pgfpathlineto{\pgfqpoint{1.466148in}{1.183530in}}%
\pgfpathlineto{\pgfqpoint{1.475724in}{1.184325in}}%
\pgfpathlineto{\pgfqpoint{1.485300in}{1.185970in}}%
\pgfpathlineto{\pgfqpoint{1.496791in}{1.186897in}}%
\pgfpathlineto{\pgfqpoint{1.500621in}{1.188076in}}%
\pgfpathlineto{\pgfqpoint{1.510197in}{1.189051in}}%
\pgfpathlineto{\pgfqpoint{1.517858in}{1.191555in}}%
\pgfpathlineto{\pgfqpoint{1.529349in}{1.192277in}}%
\pgfpathlineto{\pgfqpoint{1.542755in}{1.193417in}}%
\pgfpathlineto{\pgfqpoint{1.556161in}{1.195160in}}%
\pgfpathlineto{\pgfqpoint{1.569568in}{1.196028in}}%
\pgfpathlineto{\pgfqpoint{1.573398in}{1.197232in}}%
\pgfpathlineto{\pgfqpoint{1.590635in}{1.200437in}}%
\pgfpathlineto{\pgfqpoint{1.592550in}{1.202315in}}%
\pgfpathlineto{\pgfqpoint{1.598295in}{1.203693in}}%
\pgfpathlineto{\pgfqpoint{1.627023in}{1.210857in}}%
\pgfpathlineto{\pgfqpoint{1.630853in}{1.211879in}}%
\pgfpathlineto{\pgfqpoint{1.632768in}{1.212608in}}%
\pgfpathlineto{\pgfqpoint{1.634683in}{1.214894in}}%
\pgfpathlineto{\pgfqpoint{1.650005in}{1.216939in}}%
\pgfpathlineto{\pgfqpoint{1.653835in}{1.217843in}}%
\pgfpathlineto{\pgfqpoint{1.667241in}{1.220406in}}%
\pgfpathlineto{\pgfqpoint{1.671072in}{1.222320in}}%
\pgfpathlineto{\pgfqpoint{1.678732in}{1.223561in}}%
\pgfpathlineto{\pgfqpoint{1.682563in}{1.225762in}}%
\pgfpathlineto{\pgfqpoint{1.690223in}{1.227473in}}%
\pgfpathlineto{\pgfqpoint{1.694054in}{1.228995in}}%
\pgfpathlineto{\pgfqpoint{1.695969in}{1.229430in}}%
\pgfpathlineto{\pgfqpoint{1.697884in}{1.231143in}}%
\pgfpathlineto{\pgfqpoint{1.701714in}{1.232314in}}%
\pgfpathlineto{\pgfqpoint{1.709375in}{1.233737in}}%
\pgfpathlineto{\pgfqpoint{1.713205in}{1.235570in}}%
\pgfpathlineto{\pgfqpoint{1.728527in}{1.241910in}}%
\pgfpathlineto{\pgfqpoint{1.736188in}{1.243184in}}%
\pgfpathlineto{\pgfqpoint{1.745763in}{1.246246in}}%
\pgfpathlineto{\pgfqpoint{1.757254in}{1.248056in}}%
\pgfpathlineto{\pgfqpoint{1.761085in}{1.248943in}}%
\pgfpathlineto{\pgfqpoint{1.764915in}{1.249471in}}%
\pgfpathlineto{\pgfqpoint{1.774491in}{1.251515in}}%
\pgfpathlineto{\pgfqpoint{1.785982in}{1.252902in}}%
\pgfpathlineto{\pgfqpoint{1.789812in}{1.254325in}}%
\pgfpathlineto{\pgfqpoint{1.793643in}{1.254431in}}%
\pgfpathlineto{\pgfqpoint{1.797473in}{1.256010in}}%
\pgfpathlineto{\pgfqpoint{1.803219in}{1.257885in}}%
\pgfpathlineto{\pgfqpoint{1.835776in}{1.262884in}}%
\pgfpathlineto{\pgfqpoint{1.887486in}{1.270353in}}%
\pgfpathlineto{\pgfqpoint{1.893232in}{1.271230in}}%
\pgfpathlineto{\pgfqpoint{1.902807in}{1.275463in}}%
\pgfpathlineto{\pgfqpoint{1.906638in}{1.276300in}}%
\pgfpathlineto{\pgfqpoint{1.908553in}{1.280411in}}%
\pgfpathlineto{\pgfqpoint{1.916214in}{1.282545in}}%
\pgfpathlineto{\pgfqpoint{1.927705in}{1.284625in}}%
\pgfpathlineto{\pgfqpoint{1.929620in}{1.286283in}}%
\pgfpathlineto{\pgfqpoint{1.937281in}{1.287696in}}%
\pgfpathlineto{\pgfqpoint{1.956432in}{1.296836in}}%
\pgfpathlineto{\pgfqpoint{1.960263in}{1.301263in}}%
\pgfpathlineto{\pgfqpoint{1.964093in}{1.302226in}}%
\pgfpathlineto{\pgfqpoint{1.971754in}{1.304660in}}%
\pgfpathlineto{\pgfqpoint{1.973669in}{1.306666in}}%
\pgfpathlineto{\pgfqpoint{1.981329in}{1.307451in}}%
\pgfpathlineto{\pgfqpoint{1.988990in}{1.309458in}}%
\pgfpathlineto{\pgfqpoint{2.000481in}{1.310640in}}%
\pgfpathlineto{\pgfqpoint{2.002396in}{1.310865in}}%
\pgfpathlineto{\pgfqpoint{2.004312in}{1.313052in}}%
\pgfpathlineto{\pgfqpoint{2.006227in}{1.313203in}}%
\pgfpathlineto{\pgfqpoint{2.010057in}{1.315514in}}%
\pgfpathlineto{\pgfqpoint{2.017718in}{1.316843in}}%
\pgfpathlineto{\pgfqpoint{2.021548in}{1.318677in}}%
\pgfpathlineto{\pgfqpoint{2.031124in}{1.321009in}}%
\pgfpathlineto{\pgfqpoint{2.036869in}{1.325537in}}%
\pgfpathlineto{\pgfqpoint{2.046445in}{1.327598in}}%
\pgfpathlineto{\pgfqpoint{2.048360in}{1.329940in}}%
\pgfpathlineto{\pgfqpoint{2.056021in}{1.331163in}}%
\pgfpathlineto{\pgfqpoint{2.059852in}{1.331423in}}%
\pgfpathlineto{\pgfqpoint{2.063682in}{1.333203in}}%
\pgfpathlineto{\pgfqpoint{2.077088in}{1.334794in}}%
\pgfpathlineto{\pgfqpoint{2.082834in}{1.335836in}}%
\pgfpathlineto{\pgfqpoint{2.086664in}{1.337222in}}%
\pgfpathlineto{\pgfqpoint{2.088579in}{1.339358in}}%
\pgfpathlineto{\pgfqpoint{2.100070in}{1.340897in}}%
\pgfpathlineto{\pgfqpoint{2.103900in}{1.341009in}}%
\pgfpathlineto{\pgfqpoint{2.107731in}{1.345043in}}%
\pgfpathlineto{\pgfqpoint{2.111561in}{1.346544in}}%
\pgfpathlineto{\pgfqpoint{2.132628in}{1.353093in}}%
\pgfpathlineto{\pgfqpoint{2.134543in}{1.354830in}}%
\pgfpathlineto{\pgfqpoint{2.140289in}{1.355716in}}%
\pgfpathlineto{\pgfqpoint{2.144119in}{1.357213in}}%
\pgfpathlineto{\pgfqpoint{2.151780in}{1.358472in}}%
\pgfpathlineto{\pgfqpoint{2.155610in}{1.359652in}}%
\pgfpathlineto{\pgfqpoint{2.157525in}{1.359983in}}%
\pgfpathlineto{\pgfqpoint{2.159440in}{1.362600in}}%
\pgfpathlineto{\pgfqpoint{2.207320in}{1.374146in}}%
\pgfpathlineto{\pgfqpoint{2.216896in}{1.377913in}}%
\pgfpathlineto{\pgfqpoint{2.220726in}{1.378002in}}%
\pgfpathlineto{\pgfqpoint{2.222641in}{1.380086in}}%
\pgfpathlineto{\pgfqpoint{2.236047in}{1.382030in}}%
\pgfpathlineto{\pgfqpoint{2.241793in}{1.384763in}}%
\pgfpathlineto{\pgfqpoint{2.243708in}{1.387612in}}%
\pgfpathlineto{\pgfqpoint{2.249453in}{1.389313in}}%
\pgfpathlineto{\pgfqpoint{2.253284in}{1.390935in}}%
\pgfpathlineto{\pgfqpoint{2.260945in}{1.395304in}}%
\pgfpathlineto{\pgfqpoint{2.262860in}{1.398980in}}%
\pgfpathlineto{\pgfqpoint{2.266690in}{1.400113in}}%
\pgfpathlineto{\pgfqpoint{2.268605in}{1.400377in}}%
\pgfpathlineto{\pgfqpoint{2.272436in}{1.403796in}}%
\pgfpathlineto{\pgfqpoint{2.274351in}{1.403902in}}%
\pgfpathlineto{\pgfqpoint{2.276266in}{1.407289in}}%
\pgfpathlineto{\pgfqpoint{2.287757in}{1.409395in}}%
\pgfpathlineto{\pgfqpoint{2.289672in}{1.410866in}}%
\pgfpathlineto{\pgfqpoint{2.291587in}{1.410933in}}%
\pgfpathlineto{\pgfqpoint{2.293502in}{1.413835in}}%
\pgfpathlineto{\pgfqpoint{2.299248in}{1.414269in}}%
\pgfpathlineto{\pgfqpoint{2.304993in}{1.418648in}}%
\pgfpathlineto{\pgfqpoint{2.310739in}{1.419236in}}%
\pgfpathlineto{\pgfqpoint{2.312654in}{1.421411in}}%
\pgfpathlineto{\pgfqpoint{2.316484in}{1.422608in}}%
\pgfpathlineto{\pgfqpoint{2.320315in}{1.424557in}}%
\pgfpathlineto{\pgfqpoint{2.329891in}{1.429193in}}%
\pgfpathlineto{\pgfqpoint{2.331806in}{1.432102in}}%
\pgfpathlineto{\pgfqpoint{2.335636in}{1.433470in}}%
\pgfpathlineto{\pgfqpoint{2.339467in}{1.437411in}}%
\pgfpathlineto{\pgfqpoint{2.358618in}{1.441416in}}%
\pgfpathlineto{\pgfqpoint{2.362449in}{1.444526in}}%
\pgfpathlineto{\pgfqpoint{2.366279in}{1.447267in}}%
\pgfpathlineto{\pgfqpoint{2.370109in}{1.448358in}}%
\pgfpathlineto{\pgfqpoint{2.373940in}{1.449298in}}%
\pgfpathlineto{\pgfqpoint{2.379685in}{1.450516in}}%
\pgfpathlineto{\pgfqpoint{2.395006in}{1.452221in}}%
\pgfpathlineto{\pgfqpoint{2.410328in}{1.458475in}}%
\pgfpathlineto{\pgfqpoint{2.414158in}{1.459391in}}%
\pgfpathlineto{\pgfqpoint{2.427564in}{1.465748in}}%
\pgfpathlineto{\pgfqpoint{2.433310in}{1.466424in}}%
\pgfpathlineto{\pgfqpoint{2.437140in}{1.471209in}}%
\pgfpathlineto{\pgfqpoint{2.444801in}{1.478878in}}%
\pgfpathlineto{\pgfqpoint{2.448631in}{1.479727in}}%
\pgfpathlineto{\pgfqpoint{2.454377in}{1.483926in}}%
\pgfpathlineto{\pgfqpoint{2.458207in}{1.485287in}}%
\pgfpathlineto{\pgfqpoint{2.463953in}{1.486920in}}%
\pgfpathlineto{\pgfqpoint{2.467783in}{1.488548in}}%
\pgfpathlineto{\pgfqpoint{2.481189in}{1.497385in}}%
\pgfpathlineto{\pgfqpoint{2.483104in}{1.499642in}}%
\pgfpathlineto{\pgfqpoint{2.488850in}{1.501012in}}%
\pgfpathlineto{\pgfqpoint{2.500341in}{1.504464in}}%
\pgfpathlineto{\pgfqpoint{2.504171in}{1.506477in}}%
\pgfpathlineto{\pgfqpoint{2.506086in}{1.512215in}}%
\pgfpathlineto{\pgfqpoint{2.509917in}{1.513305in}}%
\pgfpathlineto{\pgfqpoint{2.511832in}{1.513716in}}%
\pgfpathlineto{\pgfqpoint{2.513747in}{1.517236in}}%
\pgfpathlineto{\pgfqpoint{2.517577in}{1.518534in}}%
\pgfpathlineto{\pgfqpoint{2.519493in}{1.521075in}}%
\pgfpathlineto{\pgfqpoint{2.521408in}{1.521471in}}%
\pgfpathlineto{\pgfqpoint{2.525238in}{1.523827in}}%
\pgfpathlineto{\pgfqpoint{2.542475in}{1.527227in}}%
\pgfpathlineto{\pgfqpoint{2.550135in}{1.531370in}}%
\pgfpathlineto{\pgfqpoint{2.557796in}{1.534325in}}%
\pgfpathlineto{\pgfqpoint{2.561626in}{1.535827in}}%
\pgfpathlineto{\pgfqpoint{2.567372in}{1.539053in}}%
\pgfpathlineto{\pgfqpoint{2.569287in}{1.542904in}}%
\pgfpathlineto{\pgfqpoint{2.575033in}{1.544833in}}%
\pgfpathlineto{\pgfqpoint{2.580778in}{1.546787in}}%
\pgfpathlineto{\pgfqpoint{2.592269in}{1.551344in}}%
\pgfpathlineto{\pgfqpoint{2.594184in}{1.553310in}}%
\pgfpathlineto{\pgfqpoint{2.596099in}{1.553663in}}%
\pgfpathlineto{\pgfqpoint{2.598015in}{1.555244in}}%
\pgfpathlineto{\pgfqpoint{2.601845in}{1.560810in}}%
\pgfpathlineto{\pgfqpoint{2.603760in}{1.562633in}}%
\pgfpathlineto{\pgfqpoint{2.605675in}{1.562837in}}%
\pgfpathlineto{\pgfqpoint{2.611421in}{1.566155in}}%
\pgfpathlineto{\pgfqpoint{2.626742in}{1.570817in}}%
\pgfpathlineto{\pgfqpoint{2.630573in}{1.571164in}}%
\pgfpathlineto{\pgfqpoint{2.638233in}{1.574685in}}%
\pgfpathlineto{\pgfqpoint{2.642064in}{1.579641in}}%
\pgfpathlineto{\pgfqpoint{2.647809in}{1.581096in}}%
\pgfpathlineto{\pgfqpoint{2.649724in}{1.581447in}}%
\pgfpathlineto{\pgfqpoint{2.653555in}{1.584317in}}%
\pgfpathlineto{\pgfqpoint{2.663130in}{1.586069in}}%
\pgfpathlineto{\pgfqpoint{2.666961in}{1.587435in}}%
\pgfpathlineto{\pgfqpoint{2.676537in}{1.590321in}}%
\pgfpathlineto{\pgfqpoint{2.678452in}{1.593693in}}%
\pgfpathlineto{\pgfqpoint{2.682282in}{1.594974in}}%
\pgfpathlineto{\pgfqpoint{2.691858in}{1.597882in}}%
\pgfpathlineto{\pgfqpoint{2.701434in}{1.600655in}}%
\pgfpathlineto{\pgfqpoint{2.703349in}{1.600878in}}%
\pgfpathlineto{\pgfqpoint{2.707179in}{1.602882in}}%
\pgfpathlineto{\pgfqpoint{2.714840in}{1.606766in}}%
\pgfpathlineto{\pgfqpoint{2.716755in}{1.610526in}}%
\pgfpathlineto{\pgfqpoint{2.724416in}{1.612056in}}%
\pgfpathlineto{\pgfqpoint{2.726331in}{1.612349in}}%
\pgfpathlineto{\pgfqpoint{2.728246in}{1.614897in}}%
\pgfpathlineto{\pgfqpoint{2.739737in}{1.616467in}}%
\pgfpathlineto{\pgfqpoint{2.741653in}{1.618264in}}%
\pgfpathlineto{\pgfqpoint{2.749313in}{1.619465in}}%
\pgfpathlineto{\pgfqpoint{2.755059in}{1.621218in}}%
\pgfpathlineto{\pgfqpoint{2.758889in}{1.622390in}}%
\pgfpathlineto{\pgfqpoint{2.762719in}{1.623297in}}%
\pgfpathlineto{\pgfqpoint{2.764635in}{1.625681in}}%
\pgfpathlineto{\pgfqpoint{2.774210in}{1.626737in}}%
\pgfpathlineto{\pgfqpoint{2.781871in}{1.628336in}}%
\pgfpathlineto{\pgfqpoint{2.785701in}{1.630408in}}%
\pgfpathlineto{\pgfqpoint{2.789532in}{1.634811in}}%
\pgfpathlineto{\pgfqpoint{2.793362in}{1.635914in}}%
\pgfpathlineto{\pgfqpoint{2.804853in}{1.639762in}}%
\pgfpathlineto{\pgfqpoint{2.806768in}{1.642283in}}%
\pgfpathlineto{\pgfqpoint{2.810599in}{1.643193in}}%
\pgfpathlineto{\pgfqpoint{2.818259in}{1.645990in}}%
\pgfpathlineto{\pgfqpoint{2.824005in}{1.647253in}}%
\pgfpathlineto{\pgfqpoint{2.825920in}{1.647678in}}%
\pgfpathlineto{\pgfqpoint{2.827835in}{1.651306in}}%
\pgfpathlineto{\pgfqpoint{2.843157in}{1.658318in}}%
\pgfpathlineto{\pgfqpoint{2.846987in}{1.662826in}}%
\pgfpathlineto{\pgfqpoint{2.852732in}{1.665312in}}%
\pgfpathlineto{\pgfqpoint{2.856563in}{1.666470in}}%
\pgfpathlineto{\pgfqpoint{2.862308in}{1.668568in}}%
\pgfpathlineto{\pgfqpoint{2.864223in}{1.669565in}}%
\pgfpathlineto{\pgfqpoint{2.868054in}{1.673167in}}%
\pgfpathlineto{\pgfqpoint{2.875715in}{1.674294in}}%
\pgfpathlineto{\pgfqpoint{2.877630in}{1.676224in}}%
\pgfpathlineto{\pgfqpoint{2.881460in}{1.676644in}}%
\pgfpathlineto{\pgfqpoint{2.898697in}{1.684935in}}%
\pgfpathlineto{\pgfqpoint{2.904442in}{1.692800in}}%
\pgfpathlineto{\pgfqpoint{2.912103in}{1.693895in}}%
\pgfpathlineto{\pgfqpoint{2.921679in}{1.696221in}}%
\pgfpathlineto{\pgfqpoint{2.927424in}{1.697790in}}%
\pgfpathlineto{\pgfqpoint{2.929339in}{1.702518in}}%
\pgfpathlineto{\pgfqpoint{2.942746in}{1.712964in}}%
\pgfpathlineto{\pgfqpoint{2.946576in}{1.714324in}}%
\pgfpathlineto{\pgfqpoint{2.950406in}{1.716044in}}%
\pgfpathlineto{\pgfqpoint{2.958067in}{1.717145in}}%
\pgfpathlineto{\pgfqpoint{2.961897in}{1.722932in}}%
\pgfpathlineto{\pgfqpoint{2.971473in}{1.725541in}}%
\pgfpathlineto{\pgfqpoint{2.975303in}{1.730058in}}%
\pgfpathlineto{\pgfqpoint{2.986794in}{1.737420in}}%
\pgfpathlineto{\pgfqpoint{2.992540in}{1.744238in}}%
\pgfpathlineto{\pgfqpoint{2.998285in}{1.760158in}}%
\pgfpathlineto{\pgfqpoint{3.000201in}{1.763486in}}%
\pgfpathlineto{\pgfqpoint{3.002116in}{1.763574in}}%
\pgfpathlineto{\pgfqpoint{3.013607in}{1.800754in}}%
\pgfpathlineto{\pgfqpoint{3.015522in}{1.819575in}}%
\pgfpathlineto{\pgfqpoint{3.017437in}{1.826535in}}%
\pgfpathlineto{\pgfqpoint{3.017437in}{1.826535in}}%
\pgfusepath{stroke}%
\end{pgfscope}%
\begin{pgfscope}%
\pgfpathrectangle{\pgfqpoint{0.694334in}{0.523557in}}{\pgfqpoint{3.830343in}{1.302977in}}%
\pgfusepath{clip}%
\pgfsetbuttcap%
\pgfsetroundjoin%
\pgfsetlinewidth{1.003750pt}%
\definecolor{currentstroke}{rgb}{0.941176,0.627451,0.188235}%
\pgfsetstrokecolor{currentstroke}%
\pgfsetdash{{1.000000pt}{1.650000pt}}{0.000000pt}%
\pgfpathmoveto{\pgfqpoint{0.694334in}{0.567141in}}%
\pgfpathlineto{\pgfqpoint{0.696249in}{0.582562in}}%
\pgfpathlineto{\pgfqpoint{0.698165in}{0.582755in}}%
\pgfpathlineto{\pgfqpoint{0.701995in}{0.599148in}}%
\pgfpathlineto{\pgfqpoint{0.703910in}{0.606735in}}%
\pgfpathlineto{\pgfqpoint{0.711571in}{0.615604in}}%
\pgfpathlineto{\pgfqpoint{0.713486in}{0.616358in}}%
\pgfpathlineto{\pgfqpoint{0.715401in}{0.627746in}}%
\pgfpathlineto{\pgfqpoint{0.719232in}{0.630247in}}%
\pgfpathlineto{\pgfqpoint{0.721147in}{0.632363in}}%
\pgfpathlineto{\pgfqpoint{0.724977in}{0.633447in}}%
\pgfpathlineto{\pgfqpoint{0.726892in}{0.635337in}}%
\pgfpathlineto{\pgfqpoint{0.732638in}{0.646877in}}%
\pgfpathlineto{\pgfqpoint{0.736468in}{0.649507in}}%
\pgfpathlineto{\pgfqpoint{0.738383in}{0.655445in}}%
\pgfpathlineto{\pgfqpoint{0.740298in}{0.655930in}}%
\pgfpathlineto{\pgfqpoint{0.747959in}{0.671221in}}%
\pgfpathlineto{\pgfqpoint{0.749874in}{0.675620in}}%
\pgfpathlineto{\pgfqpoint{0.751789in}{0.675686in}}%
\pgfpathlineto{\pgfqpoint{0.753705in}{0.678423in}}%
\pgfpathlineto{\pgfqpoint{0.759450in}{0.690456in}}%
\pgfpathlineto{\pgfqpoint{0.770941in}{0.700710in}}%
\pgfpathlineto{\pgfqpoint{0.774772in}{0.703735in}}%
\pgfpathlineto{\pgfqpoint{0.801584in}{0.709532in}}%
\pgfpathlineto{\pgfqpoint{0.805414in}{0.711596in}}%
\pgfpathlineto{\pgfqpoint{0.814990in}{0.713390in}}%
\pgfpathlineto{\pgfqpoint{0.820736in}{0.714546in}}%
\pgfpathlineto{\pgfqpoint{0.836057in}{0.716555in}}%
\pgfpathlineto{\pgfqpoint{0.843718in}{0.719765in}}%
\pgfpathlineto{\pgfqpoint{0.855209in}{0.720566in}}%
\pgfpathlineto{\pgfqpoint{0.857124in}{0.720645in}}%
\pgfpathlineto{\pgfqpoint{0.860954in}{0.722678in}}%
\pgfpathlineto{\pgfqpoint{0.870530in}{0.723679in}}%
\pgfpathlineto{\pgfqpoint{0.878191in}{0.724279in}}%
\pgfpathlineto{\pgfqpoint{0.908834in}{0.725460in}}%
\pgfpathlineto{\pgfqpoint{0.920325in}{0.726063in}}%
\pgfpathlineto{\pgfqpoint{0.945222in}{0.727385in}}%
\pgfpathlineto{\pgfqpoint{0.972034in}{0.728886in}}%
\pgfpathlineto{\pgfqpoint{1.044811in}{0.732918in}}%
\pgfpathlineto{\pgfqpoint{1.083114in}{0.735423in}}%
\pgfpathlineto{\pgfqpoint{1.100351in}{0.737113in}}%
\pgfpathlineto{\pgfqpoint{1.192279in}{0.741689in}}%
\pgfpathlineto{\pgfqpoint{1.213346in}{0.742889in}}%
\pgfpathlineto{\pgfqpoint{1.236328in}{0.744723in}}%
\pgfpathlineto{\pgfqpoint{1.240158in}{0.745473in}}%
\pgfpathlineto{\pgfqpoint{1.284207in}{0.747171in}}%
\pgfpathlineto{\pgfqpoint{1.414439in}{0.759272in}}%
\pgfpathlineto{\pgfqpoint{1.420184in}{0.760773in}}%
\pgfpathlineto{\pgfqpoint{1.431675in}{0.762508in}}%
\pgfpathlineto{\pgfqpoint{1.441251in}{0.763654in}}%
\pgfpathlineto{\pgfqpoint{1.445081in}{0.765852in}}%
\pgfpathlineto{\pgfqpoint{1.450827in}{0.766734in}}%
\pgfpathlineto{\pgfqpoint{1.462318in}{0.771921in}}%
\pgfpathlineto{\pgfqpoint{1.466148in}{0.772684in}}%
\pgfpathlineto{\pgfqpoint{1.475724in}{0.780723in}}%
\pgfpathlineto{\pgfqpoint{1.477639in}{0.784911in}}%
\pgfpathlineto{\pgfqpoint{1.481470in}{0.785898in}}%
\pgfpathlineto{\pgfqpoint{1.483385in}{0.788811in}}%
\pgfpathlineto{\pgfqpoint{1.492961in}{0.790671in}}%
\pgfpathlineto{\pgfqpoint{1.510197in}{0.798592in}}%
\pgfpathlineto{\pgfqpoint{1.521688in}{0.799587in}}%
\pgfpathlineto{\pgfqpoint{1.527434in}{0.802172in}}%
\pgfpathlineto{\pgfqpoint{1.540840in}{0.804764in}}%
\pgfpathlineto{\pgfqpoint{1.544670in}{0.806461in}}%
\pgfpathlineto{\pgfqpoint{1.550416in}{0.807943in}}%
\pgfpathlineto{\pgfqpoint{1.579143in}{0.812868in}}%
\pgfpathlineto{\pgfqpoint{1.582974in}{0.815132in}}%
\pgfpathlineto{\pgfqpoint{1.584889in}{0.815211in}}%
\pgfpathlineto{\pgfqpoint{1.588719in}{0.816491in}}%
\pgfpathlineto{\pgfqpoint{1.596380in}{0.818241in}}%
\pgfpathlineto{\pgfqpoint{1.615532in}{0.820945in}}%
\pgfpathlineto{\pgfqpoint{1.623192in}{0.822119in}}%
\pgfpathlineto{\pgfqpoint{1.634683in}{0.824956in}}%
\pgfpathlineto{\pgfqpoint{1.650005in}{0.827186in}}%
\pgfpathlineto{\pgfqpoint{1.661496in}{0.828179in}}%
\pgfpathlineto{\pgfqpoint{1.680648in}{0.832850in}}%
\pgfpathlineto{\pgfqpoint{1.695969in}{0.834803in}}%
\pgfpathlineto{\pgfqpoint{1.701714in}{0.836460in}}%
\pgfpathlineto{\pgfqpoint{1.718951in}{0.838267in}}%
\pgfpathlineto{\pgfqpoint{1.724697in}{0.841013in}}%
\pgfpathlineto{\pgfqpoint{1.730442in}{0.841765in}}%
\pgfpathlineto{\pgfqpoint{1.738103in}{0.843347in}}%
\pgfpathlineto{\pgfqpoint{1.766830in}{0.847835in}}%
\pgfpathlineto{\pgfqpoint{1.768745in}{0.848125in}}%
\pgfpathlineto{\pgfqpoint{1.772576in}{0.850816in}}%
\pgfpathlineto{\pgfqpoint{1.778321in}{0.852136in}}%
\pgfpathlineto{\pgfqpoint{1.789812in}{0.855151in}}%
\pgfpathlineto{\pgfqpoint{1.793643in}{0.858744in}}%
\pgfpathlineto{\pgfqpoint{1.799388in}{0.860410in}}%
\pgfpathlineto{\pgfqpoint{1.808964in}{0.861690in}}%
\pgfpathlineto{\pgfqpoint{1.897062in}{0.876194in}}%
\pgfpathlineto{\pgfqpoint{1.902807in}{0.879436in}}%
\pgfpathlineto{\pgfqpoint{1.962178in}{0.889986in}}%
\pgfpathlineto{\pgfqpoint{1.998566in}{0.895279in}}%
\pgfpathlineto{\pgfqpoint{2.002396in}{0.897136in}}%
\pgfpathlineto{\pgfqpoint{2.019633in}{0.899457in}}%
\pgfpathlineto{\pgfqpoint{2.023463in}{0.900421in}}%
\pgfpathlineto{\pgfqpoint{2.046445in}{0.903270in}}%
\pgfpathlineto{\pgfqpoint{2.063682in}{0.905826in}}%
\pgfpathlineto{\pgfqpoint{2.077088in}{0.907498in}}%
\pgfpathlineto{\pgfqpoint{2.092409in}{0.909303in}}%
\pgfpathlineto{\pgfqpoint{2.121137in}{0.912436in}}%
\pgfpathlineto{\pgfqpoint{2.124967in}{0.913570in}}%
\pgfpathlineto{\pgfqpoint{2.136458in}{0.915106in}}%
\pgfpathlineto{\pgfqpoint{2.138374in}{0.916643in}}%
\pgfpathlineto{\pgfqpoint{2.144119in}{0.917605in}}%
\pgfpathlineto{\pgfqpoint{2.149865in}{0.918496in}}%
\pgfpathlineto{\pgfqpoint{2.151780in}{0.918737in}}%
\pgfpathlineto{\pgfqpoint{2.153695in}{0.920150in}}%
\pgfpathlineto{\pgfqpoint{2.163271in}{0.920991in}}%
\pgfpathlineto{\pgfqpoint{2.214980in}{0.931176in}}%
\pgfpathlineto{\pgfqpoint{2.224556in}{0.931968in}}%
\pgfpathlineto{\pgfqpoint{2.236047in}{0.935912in}}%
\pgfpathlineto{\pgfqpoint{2.243708in}{0.936754in}}%
\pgfpathlineto{\pgfqpoint{2.280096in}{0.941676in}}%
\pgfpathlineto{\pgfqpoint{2.289672in}{0.942902in}}%
\pgfpathlineto{\pgfqpoint{2.316484in}{0.945052in}}%
\pgfpathlineto{\pgfqpoint{2.329891in}{0.945964in}}%
\pgfpathlineto{\pgfqpoint{2.335636in}{0.947179in}}%
\pgfpathlineto{\pgfqpoint{2.358618in}{0.948809in}}%
\pgfpathlineto{\pgfqpoint{2.366279in}{0.950171in}}%
\pgfpathlineto{\pgfqpoint{2.373940in}{0.951947in}}%
\pgfpathlineto{\pgfqpoint{2.377770in}{0.954446in}}%
\pgfpathlineto{\pgfqpoint{2.389261in}{0.956433in}}%
\pgfpathlineto{\pgfqpoint{2.391176in}{0.957190in}}%
\pgfpathlineto{\pgfqpoint{2.395006in}{0.960086in}}%
\pgfpathlineto{\pgfqpoint{2.406498in}{0.961412in}}%
\pgfpathlineto{\pgfqpoint{2.410328in}{0.963197in}}%
\pgfpathlineto{\pgfqpoint{2.419904in}{0.964791in}}%
\pgfpathlineto{\pgfqpoint{2.423734in}{0.967067in}}%
\pgfpathlineto{\pgfqpoint{2.439055in}{0.974701in}}%
\pgfpathlineto{\pgfqpoint{2.450546in}{0.980023in}}%
\pgfpathlineto{\pgfqpoint{2.458207in}{0.982179in}}%
\pgfpathlineto{\pgfqpoint{2.460122in}{0.986058in}}%
\pgfpathlineto{\pgfqpoint{2.465868in}{0.988210in}}%
\pgfpathlineto{\pgfqpoint{2.471613in}{0.989814in}}%
\pgfpathlineto{\pgfqpoint{2.477359in}{0.992420in}}%
\pgfpathlineto{\pgfqpoint{2.486935in}{0.993619in}}%
\pgfpathlineto{\pgfqpoint{2.498426in}{0.997123in}}%
\pgfpathlineto{\pgfqpoint{2.500341in}{0.999870in}}%
\pgfpathlineto{\pgfqpoint{2.509917in}{1.001205in}}%
\pgfpathlineto{\pgfqpoint{2.513747in}{1.003966in}}%
\pgfpathlineto{\pgfqpoint{2.523323in}{1.005179in}}%
\pgfpathlineto{\pgfqpoint{2.529068in}{1.006342in}}%
\pgfpathlineto{\pgfqpoint{2.532899in}{1.007849in}}%
\pgfpathlineto{\pgfqpoint{2.536729in}{1.008569in}}%
\pgfpathlineto{\pgfqpoint{2.557796in}{1.012606in}}%
\pgfpathlineto{\pgfqpoint{2.561626in}{1.012872in}}%
\pgfpathlineto{\pgfqpoint{2.565457in}{1.015743in}}%
\pgfpathlineto{\pgfqpoint{2.569287in}{1.016717in}}%
\pgfpathlineto{\pgfqpoint{2.573117in}{1.018092in}}%
\pgfpathlineto{\pgfqpoint{2.575033in}{1.018320in}}%
\pgfpathlineto{\pgfqpoint{2.576948in}{1.020656in}}%
\pgfpathlineto{\pgfqpoint{2.580778in}{1.021639in}}%
\pgfpathlineto{\pgfqpoint{2.584608in}{1.023294in}}%
\pgfpathlineto{\pgfqpoint{2.590354in}{1.024158in}}%
\pgfpathlineto{\pgfqpoint{2.592269in}{1.026817in}}%
\pgfpathlineto{\pgfqpoint{2.594184in}{1.027094in}}%
\pgfpathlineto{\pgfqpoint{2.596099in}{1.029692in}}%
\pgfpathlineto{\pgfqpoint{2.598015in}{1.029857in}}%
\pgfpathlineto{\pgfqpoint{2.599930in}{1.032096in}}%
\pgfpathlineto{\pgfqpoint{2.605675in}{1.034423in}}%
\pgfpathlineto{\pgfqpoint{2.611421in}{1.039133in}}%
\pgfpathlineto{\pgfqpoint{2.626742in}{1.043293in}}%
\pgfpathlineto{\pgfqpoint{2.628657in}{1.044111in}}%
\pgfpathlineto{\pgfqpoint{2.630573in}{1.046152in}}%
\pgfpathlineto{\pgfqpoint{2.632488in}{1.051263in}}%
\pgfpathlineto{\pgfqpoint{2.642064in}{1.053898in}}%
\pgfpathlineto{\pgfqpoint{2.645894in}{1.058431in}}%
\pgfpathlineto{\pgfqpoint{2.647809in}{1.059291in}}%
\pgfpathlineto{\pgfqpoint{2.651639in}{1.062244in}}%
\pgfpathlineto{\pgfqpoint{2.659300in}{1.066143in}}%
\pgfpathlineto{\pgfqpoint{2.682282in}{1.071259in}}%
\pgfpathlineto{\pgfqpoint{2.684197in}{1.073698in}}%
\pgfpathlineto{\pgfqpoint{2.686113in}{1.074109in}}%
\pgfpathlineto{\pgfqpoint{2.688028in}{1.077390in}}%
\pgfpathlineto{\pgfqpoint{2.691858in}{1.078985in}}%
\pgfpathlineto{\pgfqpoint{2.695688in}{1.081315in}}%
\pgfpathlineto{\pgfqpoint{2.697604in}{1.081341in}}%
\pgfpathlineto{\pgfqpoint{2.699519in}{1.084661in}}%
\pgfpathlineto{\pgfqpoint{2.703349in}{1.086777in}}%
\pgfpathlineto{\pgfqpoint{2.709095in}{1.092806in}}%
\pgfpathlineto{\pgfqpoint{2.712925in}{1.094980in}}%
\pgfpathlineto{\pgfqpoint{2.718670in}{1.108459in}}%
\pgfpathlineto{\pgfqpoint{2.722501in}{1.116280in}}%
\pgfpathlineto{\pgfqpoint{2.724416in}{1.118961in}}%
\pgfpathlineto{\pgfqpoint{2.726331in}{1.119126in}}%
\pgfpathlineto{\pgfqpoint{2.732077in}{1.121740in}}%
\pgfpathlineto{\pgfqpoint{2.737822in}{1.122041in}}%
\pgfpathlineto{\pgfqpoint{2.739737in}{1.128837in}}%
\pgfpathlineto{\pgfqpoint{2.741653in}{1.131149in}}%
\pgfpathlineto{\pgfqpoint{2.743568in}{1.142341in}}%
\pgfpathlineto{\pgfqpoint{2.747398in}{1.143926in}}%
\pgfpathlineto{\pgfqpoint{2.749313in}{1.152235in}}%
\pgfpathlineto{\pgfqpoint{2.753144in}{1.152754in}}%
\pgfpathlineto{\pgfqpoint{2.755059in}{1.163088in}}%
\pgfpathlineto{\pgfqpoint{2.756974in}{1.164581in}}%
\pgfpathlineto{\pgfqpoint{2.760804in}{1.170521in}}%
\pgfpathlineto{\pgfqpoint{2.762719in}{1.178607in}}%
\pgfpathlineto{\pgfqpoint{2.764635in}{1.180313in}}%
\pgfpathlineto{\pgfqpoint{2.768465in}{1.189905in}}%
\pgfpathlineto{\pgfqpoint{2.770380in}{1.191539in}}%
\pgfpathlineto{\pgfqpoint{2.776126in}{1.202914in}}%
\pgfpathlineto{\pgfqpoint{2.779956in}{1.219104in}}%
\pgfpathlineto{\pgfqpoint{2.781871in}{1.223605in}}%
\pgfpathlineto{\pgfqpoint{2.785701in}{1.240588in}}%
\pgfpathlineto{\pgfqpoint{2.787617in}{1.240613in}}%
\pgfpathlineto{\pgfqpoint{2.791447in}{1.244129in}}%
\pgfpathlineto{\pgfqpoint{2.793362in}{1.248530in}}%
\pgfpathlineto{\pgfqpoint{2.795277in}{1.258087in}}%
\pgfpathlineto{\pgfqpoint{2.797192in}{1.276958in}}%
\pgfpathlineto{\pgfqpoint{2.799108in}{1.278513in}}%
\pgfpathlineto{\pgfqpoint{2.802938in}{1.290776in}}%
\pgfpathlineto{\pgfqpoint{2.806768in}{1.339754in}}%
\pgfpathlineto{\pgfqpoint{2.808684in}{1.379698in}}%
\pgfpathlineto{\pgfqpoint{2.810599in}{1.381391in}}%
\pgfpathlineto{\pgfqpoint{2.812514in}{1.385237in}}%
\pgfpathlineto{\pgfqpoint{2.822090in}{1.589779in}}%
\pgfpathlineto{\pgfqpoint{2.825920in}{1.796014in}}%
\pgfpathlineto{\pgfqpoint{2.827835in}{1.826535in}}%
\pgfpathlineto{\pgfqpoint{2.827835in}{1.826535in}}%
\pgfusepath{stroke}%
\end{pgfscope}%
\begin{pgfscope}%
\pgfpathrectangle{\pgfqpoint{0.694334in}{0.523557in}}{\pgfqpoint{3.830343in}{1.302977in}}%
\pgfusepath{clip}%
\pgfsetbuttcap%
\pgfsetroundjoin%
\pgfsetlinewidth{1.003750pt}%
\definecolor{currentstroke}{rgb}{0.062745,0.000000,0.062745}%
\pgfsetstrokecolor{currentstroke}%
\pgfsetdash{{3.700000pt}{1.600000pt}}{0.000000pt}%
\pgfpathmoveto{\pgfqpoint{0.694334in}{0.613292in}}%
\pgfpathlineto{\pgfqpoint{0.696249in}{0.621949in}}%
\pgfpathlineto{\pgfqpoint{0.698165in}{0.622951in}}%
\pgfpathlineto{\pgfqpoint{0.700080in}{0.636570in}}%
\pgfpathlineto{\pgfqpoint{0.703910in}{0.641552in}}%
\pgfpathlineto{\pgfqpoint{0.705825in}{0.649251in}}%
\pgfpathlineto{\pgfqpoint{0.711571in}{0.652708in}}%
\pgfpathlineto{\pgfqpoint{0.713486in}{0.652973in}}%
\pgfpathlineto{\pgfqpoint{0.715401in}{0.660913in}}%
\pgfpathlineto{\pgfqpoint{0.721147in}{0.664229in}}%
\pgfpathlineto{\pgfqpoint{0.723062in}{0.668212in}}%
\pgfpathlineto{\pgfqpoint{0.724977in}{0.668256in}}%
\pgfpathlineto{\pgfqpoint{0.730723in}{0.673020in}}%
\pgfpathlineto{\pgfqpoint{0.734553in}{0.683462in}}%
\pgfpathlineto{\pgfqpoint{0.742214in}{0.687870in}}%
\pgfpathlineto{\pgfqpoint{0.744129in}{0.690797in}}%
\pgfpathlineto{\pgfqpoint{0.753705in}{0.693659in}}%
\pgfpathlineto{\pgfqpoint{0.755620in}{0.697917in}}%
\pgfpathlineto{\pgfqpoint{0.757535in}{0.699322in}}%
\pgfpathlineto{\pgfqpoint{0.759450in}{0.711299in}}%
\pgfpathlineto{\pgfqpoint{0.774772in}{0.714803in}}%
\pgfpathlineto{\pgfqpoint{0.782432in}{0.715851in}}%
\pgfpathlineto{\pgfqpoint{0.803499in}{0.718816in}}%
\pgfpathlineto{\pgfqpoint{0.813075in}{0.719973in}}%
\pgfpathlineto{\pgfqpoint{0.836057in}{0.724418in}}%
\pgfpathlineto{\pgfqpoint{0.843718in}{0.725105in}}%
\pgfpathlineto{\pgfqpoint{0.864785in}{0.726369in}}%
\pgfpathlineto{\pgfqpoint{0.870530in}{0.728363in}}%
\pgfpathlineto{\pgfqpoint{0.885851in}{0.729985in}}%
\pgfpathlineto{\pgfqpoint{0.935646in}{0.733917in}}%
\pgfpathlineto{\pgfqpoint{0.966289in}{0.735553in}}%
\pgfpathlineto{\pgfqpoint{0.970119in}{0.736329in}}%
\pgfpathlineto{\pgfqpoint{1.002677in}{0.738756in}}%
\pgfpathlineto{\pgfqpoint{1.027574in}{0.739852in}}%
\pgfpathlineto{\pgfqpoint{1.040980in}{0.740505in}}%
\pgfpathlineto{\pgfqpoint{1.104181in}{0.744093in}}%
\pgfpathlineto{\pgfqpoint{1.303359in}{0.758814in}}%
\pgfpathlineto{\pgfqpoint{1.332086in}{0.760504in}}%
\pgfpathlineto{\pgfqpoint{1.347408in}{0.761997in}}%
\pgfpathlineto{\pgfqpoint{1.358899in}{0.762976in}}%
\pgfpathlineto{\pgfqpoint{1.410608in}{0.769284in}}%
\pgfpathlineto{\pgfqpoint{1.420184in}{0.773766in}}%
\pgfpathlineto{\pgfqpoint{1.424015in}{0.775241in}}%
\pgfpathlineto{\pgfqpoint{1.431675in}{0.776193in}}%
\pgfpathlineto{\pgfqpoint{1.435506in}{0.777845in}}%
\pgfpathlineto{\pgfqpoint{1.439336in}{0.778714in}}%
\pgfpathlineto{\pgfqpoint{1.441251in}{0.781600in}}%
\pgfpathlineto{\pgfqpoint{1.443166in}{0.781687in}}%
\pgfpathlineto{\pgfqpoint{1.445081in}{0.783849in}}%
\pgfpathlineto{\pgfqpoint{1.446997in}{0.784167in}}%
\pgfpathlineto{\pgfqpoint{1.448912in}{0.786177in}}%
\pgfpathlineto{\pgfqpoint{1.462318in}{0.789214in}}%
\pgfpathlineto{\pgfqpoint{1.468064in}{0.790727in}}%
\pgfpathlineto{\pgfqpoint{1.471894in}{0.791007in}}%
\pgfpathlineto{\pgfqpoint{1.473809in}{0.793216in}}%
\pgfpathlineto{\pgfqpoint{1.477639in}{0.795139in}}%
\pgfpathlineto{\pgfqpoint{1.479555in}{0.803461in}}%
\pgfpathlineto{\pgfqpoint{1.483385in}{0.806286in}}%
\pgfpathlineto{\pgfqpoint{1.498706in}{0.812237in}}%
\pgfpathlineto{\pgfqpoint{1.502537in}{0.812525in}}%
\pgfpathlineto{\pgfqpoint{1.506367in}{0.814271in}}%
\pgfpathlineto{\pgfqpoint{1.512112in}{0.814900in}}%
\pgfpathlineto{\pgfqpoint{1.517858in}{0.816308in}}%
\pgfpathlineto{\pgfqpoint{1.523604in}{0.817857in}}%
\pgfpathlineto{\pgfqpoint{1.525519in}{0.820031in}}%
\pgfpathlineto{\pgfqpoint{1.542755in}{0.823617in}}%
\pgfpathlineto{\pgfqpoint{1.586804in}{0.833309in}}%
\pgfpathlineto{\pgfqpoint{1.602126in}{0.834527in}}%
\pgfpathlineto{\pgfqpoint{1.619362in}{0.837079in}}%
\pgfpathlineto{\pgfqpoint{1.650005in}{0.842212in}}%
\pgfpathlineto{\pgfqpoint{1.657666in}{0.843519in}}%
\pgfpathlineto{\pgfqpoint{1.665326in}{0.845086in}}%
\pgfpathlineto{\pgfqpoint{1.672987in}{0.845492in}}%
\pgfpathlineto{\pgfqpoint{1.676817in}{0.846960in}}%
\pgfpathlineto{\pgfqpoint{1.690223in}{0.848621in}}%
\pgfpathlineto{\pgfqpoint{1.699799in}{0.850941in}}%
\pgfpathlineto{\pgfqpoint{1.711290in}{0.852917in}}%
\pgfpathlineto{\pgfqpoint{1.720866in}{0.855801in}}%
\pgfpathlineto{\pgfqpoint{1.730442in}{0.856953in}}%
\pgfpathlineto{\pgfqpoint{1.768745in}{0.864715in}}%
\pgfpathlineto{\pgfqpoint{1.776406in}{0.866159in}}%
\pgfpathlineto{\pgfqpoint{1.822370in}{0.871388in}}%
\pgfpathlineto{\pgfqpoint{1.824285in}{0.873073in}}%
\pgfpathlineto{\pgfqpoint{1.830031in}{0.874370in}}%
\pgfpathlineto{\pgfqpoint{1.833861in}{0.876147in}}%
\pgfpathlineto{\pgfqpoint{1.877910in}{0.882911in}}%
\pgfpathlineto{\pgfqpoint{1.883656in}{0.885176in}}%
\pgfpathlineto{\pgfqpoint{1.895147in}{0.887468in}}%
\pgfpathlineto{\pgfqpoint{1.900892in}{0.888286in}}%
\pgfpathlineto{\pgfqpoint{1.923874in}{0.892108in}}%
\pgfpathlineto{\pgfqpoint{1.927705in}{0.893412in}}%
\pgfpathlineto{\pgfqpoint{1.937281in}{0.894443in}}%
\pgfpathlineto{\pgfqpoint{1.950687in}{0.895245in}}%
\pgfpathlineto{\pgfqpoint{1.954517in}{0.896470in}}%
\pgfpathlineto{\pgfqpoint{1.958347in}{0.897567in}}%
\pgfpathlineto{\pgfqpoint{1.966008in}{0.899326in}}%
\pgfpathlineto{\pgfqpoint{1.994736in}{0.904957in}}%
\pgfpathlineto{\pgfqpoint{2.006227in}{0.906883in}}%
\pgfpathlineto{\pgfqpoint{2.011972in}{0.907787in}}%
\pgfpathlineto{\pgfqpoint{2.021548in}{0.909620in}}%
\pgfpathlineto{\pgfqpoint{2.025378in}{0.910426in}}%
\pgfpathlineto{\pgfqpoint{2.040700in}{0.911965in}}%
\pgfpathlineto{\pgfqpoint{2.088579in}{0.921193in}}%
\pgfpathlineto{\pgfqpoint{2.094325in}{0.923094in}}%
\pgfpathlineto{\pgfqpoint{2.107731in}{0.924941in}}%
\pgfpathlineto{\pgfqpoint{2.115391in}{0.925691in}}%
\pgfpathlineto{\pgfqpoint{2.121137in}{0.926982in}}%
\pgfpathlineto{\pgfqpoint{2.123052in}{0.927051in}}%
\pgfpathlineto{\pgfqpoint{2.126883in}{0.928355in}}%
\pgfpathlineto{\pgfqpoint{2.165186in}{0.934156in}}%
\pgfpathlineto{\pgfqpoint{2.169016in}{0.935670in}}%
\pgfpathlineto{\pgfqpoint{2.209235in}{0.943307in}}%
\pgfpathlineto{\pgfqpoint{2.213065in}{0.944656in}}%
\pgfpathlineto{\pgfqpoint{2.283927in}{0.953096in}}%
\pgfpathlineto{\pgfqpoint{2.295418in}{0.956536in}}%
\pgfpathlineto{\pgfqpoint{2.312654in}{0.959066in}}%
\pgfpathlineto{\pgfqpoint{2.318400in}{0.960021in}}%
\pgfpathlineto{\pgfqpoint{2.322230in}{0.961899in}}%
\pgfpathlineto{\pgfqpoint{2.326060in}{0.962330in}}%
\pgfpathlineto{\pgfqpoint{2.327975in}{0.964487in}}%
\pgfpathlineto{\pgfqpoint{2.343297in}{0.967160in}}%
\pgfpathlineto{\pgfqpoint{2.352873in}{0.970888in}}%
\pgfpathlineto{\pgfqpoint{2.354788in}{0.971079in}}%
\pgfpathlineto{\pgfqpoint{2.356703in}{0.972769in}}%
\pgfpathlineto{\pgfqpoint{2.360533in}{0.973870in}}%
\pgfpathlineto{\pgfqpoint{2.368194in}{0.975749in}}%
\pgfpathlineto{\pgfqpoint{2.373940in}{0.978564in}}%
\pgfpathlineto{\pgfqpoint{2.385431in}{0.981230in}}%
\pgfpathlineto{\pgfqpoint{2.389261in}{0.982244in}}%
\pgfpathlineto{\pgfqpoint{2.400752in}{0.987375in}}%
\pgfpathlineto{\pgfqpoint{2.406498in}{0.989127in}}%
\pgfpathlineto{\pgfqpoint{2.435225in}{0.998211in}}%
\pgfpathlineto{\pgfqpoint{2.442886in}{0.999568in}}%
\pgfpathlineto{\pgfqpoint{2.446716in}{1.000610in}}%
\pgfpathlineto{\pgfqpoint{2.454377in}{1.001857in}}%
\pgfpathlineto{\pgfqpoint{2.467783in}{1.008305in}}%
\pgfpathlineto{\pgfqpoint{2.473529in}{1.008820in}}%
\pgfpathlineto{\pgfqpoint{2.479274in}{1.011252in}}%
\pgfpathlineto{\pgfqpoint{2.486935in}{1.011900in}}%
\pgfpathlineto{\pgfqpoint{2.490765in}{1.013914in}}%
\pgfpathlineto{\pgfqpoint{2.496511in}{1.014687in}}%
\pgfpathlineto{\pgfqpoint{2.500341in}{1.015832in}}%
\pgfpathlineto{\pgfqpoint{2.515662in}{1.018394in}}%
\pgfpathlineto{\pgfqpoint{2.517577in}{1.018503in}}%
\pgfpathlineto{\pgfqpoint{2.523323in}{1.021593in}}%
\pgfpathlineto{\pgfqpoint{2.529068in}{1.022097in}}%
\pgfpathlineto{\pgfqpoint{2.530984in}{1.024576in}}%
\pgfpathlineto{\pgfqpoint{2.534814in}{1.026401in}}%
\pgfpathlineto{\pgfqpoint{2.536729in}{1.029089in}}%
\pgfpathlineto{\pgfqpoint{2.542475in}{1.029334in}}%
\pgfpathlineto{\pgfqpoint{2.546305in}{1.031708in}}%
\pgfpathlineto{\pgfqpoint{2.553966in}{1.034094in}}%
\pgfpathlineto{\pgfqpoint{2.557796in}{1.034626in}}%
\pgfpathlineto{\pgfqpoint{2.561626in}{1.036665in}}%
\pgfpathlineto{\pgfqpoint{2.569287in}{1.039114in}}%
\pgfpathlineto{\pgfqpoint{2.571202in}{1.039116in}}%
\pgfpathlineto{\pgfqpoint{2.575033in}{1.043055in}}%
\pgfpathlineto{\pgfqpoint{2.594184in}{1.046713in}}%
\pgfpathlineto{\pgfqpoint{2.596099in}{1.048649in}}%
\pgfpathlineto{\pgfqpoint{2.601845in}{1.049867in}}%
\pgfpathlineto{\pgfqpoint{2.603760in}{1.052652in}}%
\pgfpathlineto{\pgfqpoint{2.609506in}{1.054760in}}%
\pgfpathlineto{\pgfqpoint{2.611421in}{1.056654in}}%
\pgfpathlineto{\pgfqpoint{2.615251in}{1.056941in}}%
\pgfpathlineto{\pgfqpoint{2.619082in}{1.058611in}}%
\pgfpathlineto{\pgfqpoint{2.620997in}{1.058659in}}%
\pgfpathlineto{\pgfqpoint{2.626742in}{1.063202in}}%
\pgfpathlineto{\pgfqpoint{2.632488in}{1.068559in}}%
\pgfpathlineto{\pgfqpoint{2.638233in}{1.070863in}}%
\pgfpathlineto{\pgfqpoint{2.640148in}{1.073210in}}%
\pgfpathlineto{\pgfqpoint{2.642064in}{1.073376in}}%
\pgfpathlineto{\pgfqpoint{2.653555in}{1.079961in}}%
\pgfpathlineto{\pgfqpoint{2.659300in}{1.081531in}}%
\pgfpathlineto{\pgfqpoint{2.665046in}{1.083021in}}%
\pgfpathlineto{\pgfqpoint{2.668876in}{1.086147in}}%
\pgfpathlineto{\pgfqpoint{2.672706in}{1.089786in}}%
\pgfpathlineto{\pgfqpoint{2.674622in}{1.092776in}}%
\pgfpathlineto{\pgfqpoint{2.676537in}{1.093057in}}%
\pgfpathlineto{\pgfqpoint{2.680367in}{1.095473in}}%
\pgfpathlineto{\pgfqpoint{2.684197in}{1.096841in}}%
\pgfpathlineto{\pgfqpoint{2.693773in}{1.098690in}}%
\pgfpathlineto{\pgfqpoint{2.697604in}{1.100264in}}%
\pgfpathlineto{\pgfqpoint{2.705264in}{1.104803in}}%
\pgfpathlineto{\pgfqpoint{2.707179in}{1.108273in}}%
\pgfpathlineto{\pgfqpoint{2.720586in}{1.115609in}}%
\pgfpathlineto{\pgfqpoint{2.726331in}{1.125363in}}%
\pgfpathlineto{\pgfqpoint{2.728246in}{1.129427in}}%
\pgfpathlineto{\pgfqpoint{2.730161in}{1.129533in}}%
\pgfpathlineto{\pgfqpoint{2.737822in}{1.135682in}}%
\pgfpathlineto{\pgfqpoint{2.741653in}{1.139500in}}%
\pgfpathlineto{\pgfqpoint{2.743568in}{1.139923in}}%
\pgfpathlineto{\pgfqpoint{2.745483in}{1.149433in}}%
\pgfpathlineto{\pgfqpoint{2.747398in}{1.150761in}}%
\pgfpathlineto{\pgfqpoint{2.749313in}{1.159555in}}%
\pgfpathlineto{\pgfqpoint{2.753144in}{1.163879in}}%
\pgfpathlineto{\pgfqpoint{2.758889in}{1.169167in}}%
\pgfpathlineto{\pgfqpoint{2.760804in}{1.176447in}}%
\pgfpathlineto{\pgfqpoint{2.764635in}{1.179226in}}%
\pgfpathlineto{\pgfqpoint{2.774210in}{1.200352in}}%
\pgfpathlineto{\pgfqpoint{2.776126in}{1.200960in}}%
\pgfpathlineto{\pgfqpoint{2.779956in}{1.211885in}}%
\pgfpathlineto{\pgfqpoint{2.781871in}{1.223561in}}%
\pgfpathlineto{\pgfqpoint{2.787617in}{1.233278in}}%
\pgfpathlineto{\pgfqpoint{2.791447in}{1.235612in}}%
\pgfpathlineto{\pgfqpoint{2.793362in}{1.239034in}}%
\pgfpathlineto{\pgfqpoint{2.795277in}{1.239183in}}%
\pgfpathlineto{\pgfqpoint{2.797192in}{1.241743in}}%
\pgfpathlineto{\pgfqpoint{2.802938in}{1.258053in}}%
\pgfpathlineto{\pgfqpoint{2.804853in}{1.259950in}}%
\pgfpathlineto{\pgfqpoint{2.806768in}{1.260051in}}%
\pgfpathlineto{\pgfqpoint{2.808684in}{1.265087in}}%
\pgfpathlineto{\pgfqpoint{2.810599in}{1.266597in}}%
\pgfpathlineto{\pgfqpoint{2.812514in}{1.266653in}}%
\pgfpathlineto{\pgfqpoint{2.814429in}{1.294932in}}%
\pgfpathlineto{\pgfqpoint{2.816344in}{1.296020in}}%
\pgfpathlineto{\pgfqpoint{2.820175in}{1.422730in}}%
\pgfpathlineto{\pgfqpoint{2.822090in}{1.516447in}}%
\pgfpathlineto{\pgfqpoint{2.824005in}{1.826535in}}%
\pgfpathlineto{\pgfqpoint{2.824005in}{1.826535in}}%
\pgfusepath{stroke}%
\end{pgfscope}%
\begin{pgfscope}%
\pgfpathrectangle{\pgfqpoint{0.694334in}{0.523557in}}{\pgfqpoint{3.830343in}{1.302977in}}%
\pgfusepath{clip}%
\pgfsetbuttcap%
\pgfsetroundjoin%
\pgfsetlinewidth{1.003750pt}%
\definecolor{currentstroke}{rgb}{0.811765,0.125490,0.125490}%
\pgfsetstrokecolor{currentstroke}%
\pgfsetdash{{1.000000pt}{1.650000pt}}{0.000000pt}%
\pgfpathmoveto{\pgfqpoint{0.694334in}{1.092842in}}%
\pgfpathlineto{\pgfqpoint{0.700080in}{1.095601in}}%
\pgfpathlineto{\pgfqpoint{0.713486in}{1.097828in}}%
\pgfpathlineto{\pgfqpoint{0.715401in}{1.099365in}}%
\pgfpathlineto{\pgfqpoint{0.723062in}{1.100590in}}%
\pgfpathlineto{\pgfqpoint{0.738383in}{1.104836in}}%
\pgfpathlineto{\pgfqpoint{0.740298in}{1.108199in}}%
\pgfpathlineto{\pgfqpoint{0.744129in}{1.109019in}}%
\pgfpathlineto{\pgfqpoint{0.747959in}{1.109596in}}%
\pgfpathlineto{\pgfqpoint{0.749874in}{1.118682in}}%
\pgfpathlineto{\pgfqpoint{0.751789in}{1.118957in}}%
\pgfpathlineto{\pgfqpoint{0.755620in}{1.131559in}}%
\pgfpathlineto{\pgfqpoint{0.757535in}{1.132092in}}%
\pgfpathlineto{\pgfqpoint{0.759450in}{1.137051in}}%
\pgfpathlineto{\pgfqpoint{0.761365in}{1.138654in}}%
\pgfpathlineto{\pgfqpoint{0.765196in}{1.145963in}}%
\pgfpathlineto{\pgfqpoint{0.767111in}{1.146688in}}%
\pgfpathlineto{\pgfqpoint{0.769026in}{1.150685in}}%
\pgfpathlineto{\pgfqpoint{0.772856in}{1.152392in}}%
\pgfpathlineto{\pgfqpoint{0.776687in}{1.160254in}}%
\pgfpathlineto{\pgfqpoint{0.784347in}{1.161935in}}%
\pgfpathlineto{\pgfqpoint{0.788178in}{1.173187in}}%
\pgfpathlineto{\pgfqpoint{0.792008in}{1.176899in}}%
\pgfpathlineto{\pgfqpoint{0.795838in}{1.177668in}}%
\pgfpathlineto{\pgfqpoint{0.797754in}{1.180982in}}%
\pgfpathlineto{\pgfqpoint{0.799669in}{1.187098in}}%
\pgfpathlineto{\pgfqpoint{0.805414in}{1.190346in}}%
\pgfpathlineto{\pgfqpoint{0.811160in}{1.192110in}}%
\pgfpathlineto{\pgfqpoint{0.814990in}{1.195951in}}%
\pgfpathlineto{\pgfqpoint{0.818820in}{1.196792in}}%
\pgfpathlineto{\pgfqpoint{0.860954in}{1.207986in}}%
\pgfpathlineto{\pgfqpoint{0.870530in}{1.209480in}}%
\pgfpathlineto{\pgfqpoint{0.883936in}{1.212655in}}%
\pgfpathlineto{\pgfqpoint{0.899258in}{1.214008in}}%
\pgfpathlineto{\pgfqpoint{0.908834in}{1.214543in}}%
\pgfpathlineto{\pgfqpoint{0.914579in}{1.215715in}}%
\pgfpathlineto{\pgfqpoint{0.941391in}{1.218454in}}%
\pgfpathlineto{\pgfqpoint{0.956713in}{1.219640in}}%
\pgfpathlineto{\pgfqpoint{0.964373in}{1.220722in}}%
\pgfpathlineto{\pgfqpoint{0.995016in}{1.223496in}}%
\pgfpathlineto{\pgfqpoint{1.010338in}{1.225303in}}%
\pgfpathlineto{\pgfqpoint{1.276546in}{1.260355in}}%
\pgfpathlineto{\pgfqpoint{1.301444in}{1.262781in}}%
\pgfpathlineto{\pgfqpoint{1.309104in}{1.263265in}}%
\pgfpathlineto{\pgfqpoint{1.314850in}{1.263874in}}%
\pgfpathlineto{\pgfqpoint{1.318680in}{1.265400in}}%
\pgfpathlineto{\pgfqpoint{1.326341in}{1.266687in}}%
\pgfpathlineto{\pgfqpoint{1.347408in}{1.271336in}}%
\pgfpathlineto{\pgfqpoint{1.353153in}{1.272187in}}%
\pgfpathlineto{\pgfqpoint{1.356984in}{1.273188in}}%
\pgfpathlineto{\pgfqpoint{1.368475in}{1.275324in}}%
\pgfpathlineto{\pgfqpoint{1.379966in}{1.276812in}}%
\pgfpathlineto{\pgfqpoint{1.391457in}{1.278655in}}%
\pgfpathlineto{\pgfqpoint{1.406778in}{1.281795in}}%
\pgfpathlineto{\pgfqpoint{1.416354in}{1.282600in}}%
\pgfpathlineto{\pgfqpoint{1.420184in}{1.284273in}}%
\pgfpathlineto{\pgfqpoint{1.431675in}{1.286989in}}%
\pgfpathlineto{\pgfqpoint{1.454657in}{1.290085in}}%
\pgfpathlineto{\pgfqpoint{1.456573in}{1.291995in}}%
\pgfpathlineto{\pgfqpoint{1.462318in}{1.292473in}}%
\pgfpathlineto{\pgfqpoint{1.466148in}{1.293576in}}%
\pgfpathlineto{\pgfqpoint{1.475724in}{1.295512in}}%
\pgfpathlineto{\pgfqpoint{1.485300in}{1.297704in}}%
\pgfpathlineto{\pgfqpoint{1.491046in}{1.298449in}}%
\pgfpathlineto{\pgfqpoint{1.502537in}{1.304469in}}%
\pgfpathlineto{\pgfqpoint{1.515943in}{1.308966in}}%
\pgfpathlineto{\pgfqpoint{1.521688in}{1.311486in}}%
\pgfpathlineto{\pgfqpoint{1.550416in}{1.319829in}}%
\pgfpathlineto{\pgfqpoint{1.556161in}{1.322115in}}%
\pgfpathlineto{\pgfqpoint{1.561907in}{1.323577in}}%
\pgfpathlineto{\pgfqpoint{1.565737in}{1.325470in}}%
\pgfpathlineto{\pgfqpoint{1.567652in}{1.326031in}}%
\pgfpathlineto{\pgfqpoint{1.569568in}{1.327861in}}%
\pgfpathlineto{\pgfqpoint{1.577228in}{1.328459in}}%
\pgfpathlineto{\pgfqpoint{1.581059in}{1.330616in}}%
\pgfpathlineto{\pgfqpoint{1.586804in}{1.330921in}}%
\pgfpathlineto{\pgfqpoint{1.588719in}{1.332514in}}%
\pgfpathlineto{\pgfqpoint{1.590635in}{1.332541in}}%
\pgfpathlineto{\pgfqpoint{1.594465in}{1.336745in}}%
\pgfpathlineto{\pgfqpoint{1.598295in}{1.339926in}}%
\pgfpathlineto{\pgfqpoint{1.604041in}{1.342014in}}%
\pgfpathlineto{\pgfqpoint{1.605956in}{1.343043in}}%
\pgfpathlineto{\pgfqpoint{1.607871in}{1.346368in}}%
\pgfpathlineto{\pgfqpoint{1.617447in}{1.348989in}}%
\pgfpathlineto{\pgfqpoint{1.621277in}{1.351011in}}%
\pgfpathlineto{\pgfqpoint{1.623192in}{1.351528in}}%
\pgfpathlineto{\pgfqpoint{1.628938in}{1.355828in}}%
\pgfpathlineto{\pgfqpoint{1.630853in}{1.356072in}}%
\pgfpathlineto{\pgfqpoint{1.632768in}{1.359363in}}%
\pgfpathlineto{\pgfqpoint{1.638514in}{1.361024in}}%
\pgfpathlineto{\pgfqpoint{1.640429in}{1.366651in}}%
\pgfpathlineto{\pgfqpoint{1.646174in}{1.369896in}}%
\pgfpathlineto{\pgfqpoint{1.653835in}{1.371709in}}%
\pgfpathlineto{\pgfqpoint{1.655750in}{1.372445in}}%
\pgfpathlineto{\pgfqpoint{1.659581in}{1.376545in}}%
\pgfpathlineto{\pgfqpoint{1.661496in}{1.376664in}}%
\pgfpathlineto{\pgfqpoint{1.663411in}{1.378528in}}%
\pgfpathlineto{\pgfqpoint{1.665326in}{1.382384in}}%
\pgfpathlineto{\pgfqpoint{1.671072in}{1.385381in}}%
\pgfpathlineto{\pgfqpoint{1.672987in}{1.385605in}}%
\pgfpathlineto{\pgfqpoint{1.676817in}{1.388631in}}%
\pgfpathlineto{\pgfqpoint{1.678732in}{1.389252in}}%
\pgfpathlineto{\pgfqpoint{1.684478in}{1.394453in}}%
\pgfpathlineto{\pgfqpoint{1.686393in}{1.394889in}}%
\pgfpathlineto{\pgfqpoint{1.690223in}{1.398688in}}%
\pgfpathlineto{\pgfqpoint{1.697884in}{1.403132in}}%
\pgfpathlineto{\pgfqpoint{1.701714in}{1.409587in}}%
\pgfpathlineto{\pgfqpoint{1.713205in}{1.415145in}}%
\pgfpathlineto{\pgfqpoint{1.715121in}{1.416608in}}%
\pgfpathlineto{\pgfqpoint{1.717036in}{1.421963in}}%
\pgfpathlineto{\pgfqpoint{1.726612in}{1.424488in}}%
\pgfpathlineto{\pgfqpoint{1.730442in}{1.427041in}}%
\pgfpathlineto{\pgfqpoint{1.734272in}{1.428226in}}%
\pgfpathlineto{\pgfqpoint{1.738103in}{1.434547in}}%
\pgfpathlineto{\pgfqpoint{1.757254in}{1.441305in}}%
\pgfpathlineto{\pgfqpoint{1.761085in}{1.442517in}}%
\pgfpathlineto{\pgfqpoint{1.764915in}{1.443902in}}%
\pgfpathlineto{\pgfqpoint{1.768745in}{1.444849in}}%
\pgfpathlineto{\pgfqpoint{1.772576in}{1.449568in}}%
\pgfpathlineto{\pgfqpoint{1.782152in}{1.453292in}}%
\pgfpathlineto{\pgfqpoint{1.785982in}{1.456892in}}%
\pgfpathlineto{\pgfqpoint{1.791728in}{1.458672in}}%
\pgfpathlineto{\pgfqpoint{1.793643in}{1.460292in}}%
\pgfpathlineto{\pgfqpoint{1.797473in}{1.460309in}}%
\pgfpathlineto{\pgfqpoint{1.799388in}{1.463854in}}%
\pgfpathlineto{\pgfqpoint{1.803219in}{1.464567in}}%
\pgfpathlineto{\pgfqpoint{1.807049in}{1.464956in}}%
\pgfpathlineto{\pgfqpoint{1.810879in}{1.467354in}}%
\pgfpathlineto{\pgfqpoint{1.814710in}{1.475400in}}%
\pgfpathlineto{\pgfqpoint{1.818540in}{1.476345in}}%
\pgfpathlineto{\pgfqpoint{1.820455in}{1.479490in}}%
\pgfpathlineto{\pgfqpoint{1.831946in}{1.483182in}}%
\pgfpathlineto{\pgfqpoint{1.835776in}{1.486299in}}%
\pgfpathlineto{\pgfqpoint{1.837692in}{1.486766in}}%
\pgfpathlineto{\pgfqpoint{1.841522in}{1.489589in}}%
\pgfpathlineto{\pgfqpoint{1.843437in}{1.489910in}}%
\pgfpathlineto{\pgfqpoint{1.847267in}{1.493070in}}%
\pgfpathlineto{\pgfqpoint{1.860674in}{1.493814in}}%
\pgfpathlineto{\pgfqpoint{1.862589in}{1.495401in}}%
\pgfpathlineto{\pgfqpoint{1.868334in}{1.495891in}}%
\pgfpathlineto{\pgfqpoint{1.872165in}{1.497930in}}%
\pgfpathlineto{\pgfqpoint{1.875995in}{1.499961in}}%
\pgfpathlineto{\pgfqpoint{1.877910in}{1.505456in}}%
\pgfpathlineto{\pgfqpoint{1.885571in}{1.507597in}}%
\pgfpathlineto{\pgfqpoint{1.889401in}{1.511675in}}%
\pgfpathlineto{\pgfqpoint{1.904723in}{1.522836in}}%
\pgfpathlineto{\pgfqpoint{1.910468in}{1.531346in}}%
\pgfpathlineto{\pgfqpoint{1.914298in}{1.531408in}}%
\pgfpathlineto{\pgfqpoint{1.923874in}{1.537233in}}%
\pgfpathlineto{\pgfqpoint{1.925790in}{1.540900in}}%
\pgfpathlineto{\pgfqpoint{1.931535in}{1.543587in}}%
\pgfpathlineto{\pgfqpoint{1.935365in}{1.544622in}}%
\pgfpathlineto{\pgfqpoint{1.944941in}{1.549946in}}%
\pgfpathlineto{\pgfqpoint{1.948772in}{1.550939in}}%
\pgfpathlineto{\pgfqpoint{1.952602in}{1.551894in}}%
\pgfpathlineto{\pgfqpoint{1.956432in}{1.553405in}}%
\pgfpathlineto{\pgfqpoint{1.966008in}{1.557216in}}%
\pgfpathlineto{\pgfqpoint{1.969838in}{1.568010in}}%
\pgfpathlineto{\pgfqpoint{1.973669in}{1.573545in}}%
\pgfpathlineto{\pgfqpoint{1.977499in}{1.573994in}}%
\pgfpathlineto{\pgfqpoint{1.983245in}{1.575787in}}%
\pgfpathlineto{\pgfqpoint{1.985160in}{1.575964in}}%
\pgfpathlineto{\pgfqpoint{1.988990in}{1.578691in}}%
\pgfpathlineto{\pgfqpoint{1.996651in}{1.580364in}}%
\pgfpathlineto{\pgfqpoint{2.010057in}{1.585292in}}%
\pgfpathlineto{\pgfqpoint{2.013887in}{1.595369in}}%
\pgfpathlineto{\pgfqpoint{2.019633in}{1.598076in}}%
\pgfpathlineto{\pgfqpoint{2.025378in}{1.601973in}}%
\pgfpathlineto{\pgfqpoint{2.031124in}{1.603684in}}%
\pgfpathlineto{\pgfqpoint{2.040700in}{1.607022in}}%
\pgfpathlineto{\pgfqpoint{2.050276in}{1.620180in}}%
\pgfpathlineto{\pgfqpoint{2.054106in}{1.621879in}}%
\pgfpathlineto{\pgfqpoint{2.061767in}{1.622929in}}%
\pgfpathlineto{\pgfqpoint{2.065597in}{1.626878in}}%
\pgfpathlineto{\pgfqpoint{2.082834in}{1.631262in}}%
\pgfpathlineto{\pgfqpoint{2.086664in}{1.633537in}}%
\pgfpathlineto{\pgfqpoint{2.088579in}{1.633905in}}%
\pgfpathlineto{\pgfqpoint{2.090494in}{1.636687in}}%
\pgfpathlineto{\pgfqpoint{2.092409in}{1.641522in}}%
\pgfpathlineto{\pgfqpoint{2.098155in}{1.643104in}}%
\pgfpathlineto{\pgfqpoint{2.103900in}{1.647931in}}%
\pgfpathlineto{\pgfqpoint{2.105816in}{1.648707in}}%
\pgfpathlineto{\pgfqpoint{2.107731in}{1.652064in}}%
\pgfpathlineto{\pgfqpoint{2.113476in}{1.653067in}}%
\pgfpathlineto{\pgfqpoint{2.124967in}{1.657318in}}%
\pgfpathlineto{\pgfqpoint{2.126883in}{1.660554in}}%
\pgfpathlineto{\pgfqpoint{2.132628in}{1.663152in}}%
\pgfpathlineto{\pgfqpoint{2.136458in}{1.667398in}}%
\pgfpathlineto{\pgfqpoint{2.146034in}{1.672284in}}%
\pgfpathlineto{\pgfqpoint{2.153695in}{1.677501in}}%
\pgfpathlineto{\pgfqpoint{2.165186in}{1.681182in}}%
\pgfpathlineto{\pgfqpoint{2.174762in}{1.687722in}}%
\pgfpathlineto{\pgfqpoint{2.182422in}{1.698022in}}%
\pgfpathlineto{\pgfqpoint{2.186253in}{1.699619in}}%
\pgfpathlineto{\pgfqpoint{2.188168in}{1.702234in}}%
\pgfpathlineto{\pgfqpoint{2.191998in}{1.702478in}}%
\pgfpathlineto{\pgfqpoint{2.203489in}{1.713686in}}%
\pgfpathlineto{\pgfqpoint{2.207320in}{1.714462in}}%
\pgfpathlineto{\pgfqpoint{2.209235in}{1.719630in}}%
\pgfpathlineto{\pgfqpoint{2.216896in}{1.722760in}}%
\pgfpathlineto{\pgfqpoint{2.220726in}{1.726786in}}%
\pgfpathlineto{\pgfqpoint{2.222641in}{1.727246in}}%
\pgfpathlineto{\pgfqpoint{2.226471in}{1.730974in}}%
\pgfpathlineto{\pgfqpoint{2.230302in}{1.731978in}}%
\pgfpathlineto{\pgfqpoint{2.237962in}{1.734587in}}%
\pgfpathlineto{\pgfqpoint{2.241793in}{1.736091in}}%
\pgfpathlineto{\pgfqpoint{2.247538in}{1.739614in}}%
\pgfpathlineto{\pgfqpoint{2.253284in}{1.741282in}}%
\pgfpathlineto{\pgfqpoint{2.262860in}{1.746025in}}%
\pgfpathlineto{\pgfqpoint{2.268605in}{1.747129in}}%
\pgfpathlineto{\pgfqpoint{2.272436in}{1.749142in}}%
\pgfpathlineto{\pgfqpoint{2.278181in}{1.759746in}}%
\pgfpathlineto{\pgfqpoint{2.285842in}{1.765681in}}%
\pgfpathlineto{\pgfqpoint{2.291587in}{1.766759in}}%
\pgfpathlineto{\pgfqpoint{2.293502in}{1.767371in}}%
\pgfpathlineto{\pgfqpoint{2.295418in}{1.772118in}}%
\pgfpathlineto{\pgfqpoint{2.301163in}{1.775959in}}%
\pgfpathlineto{\pgfqpoint{2.303078in}{1.776410in}}%
\pgfpathlineto{\pgfqpoint{2.306909in}{1.778441in}}%
\pgfpathlineto{\pgfqpoint{2.312654in}{1.780935in}}%
\pgfpathlineto{\pgfqpoint{2.320315in}{1.782097in}}%
\pgfpathlineto{\pgfqpoint{2.327975in}{1.786705in}}%
\pgfpathlineto{\pgfqpoint{2.331806in}{1.788304in}}%
\pgfpathlineto{\pgfqpoint{2.333721in}{1.789322in}}%
\pgfpathlineto{\pgfqpoint{2.335636in}{1.791670in}}%
\pgfpathlineto{\pgfqpoint{2.337551in}{1.792129in}}%
\pgfpathlineto{\pgfqpoint{2.339467in}{1.796183in}}%
\pgfpathlineto{\pgfqpoint{2.343297in}{1.797173in}}%
\pgfpathlineto{\pgfqpoint{2.345212in}{1.799983in}}%
\pgfpathlineto{\pgfqpoint{2.347127in}{1.800002in}}%
\pgfpathlineto{\pgfqpoint{2.349042in}{1.803257in}}%
\pgfpathlineto{\pgfqpoint{2.358618in}{1.807884in}}%
\pgfpathlineto{\pgfqpoint{2.360533in}{1.809997in}}%
\pgfpathlineto{\pgfqpoint{2.362449in}{1.810025in}}%
\pgfpathlineto{\pgfqpoint{2.366279in}{1.815338in}}%
\pgfpathlineto{\pgfqpoint{2.373940in}{1.816209in}}%
\pgfpathlineto{\pgfqpoint{2.377770in}{1.818488in}}%
\pgfpathlineto{\pgfqpoint{2.381600in}{1.823308in}}%
\pgfpathlineto{\pgfqpoint{2.385431in}{1.823974in}}%
\pgfpathlineto{\pgfqpoint{2.387346in}{1.826535in}}%
\pgfpathlineto{\pgfqpoint{2.387346in}{1.826535in}}%
\pgfusepath{stroke}%
\end{pgfscope}%
\begin{pgfscope}%
\pgfpathrectangle{\pgfqpoint{0.694334in}{0.523557in}}{\pgfqpoint{3.830343in}{1.302977in}}%
\pgfusepath{clip}%
\pgfsetrectcap%
\pgfsetroundjoin%
\pgfsetlinewidth{1.003750pt}%
\definecolor{currentstroke}{rgb}{0.000000,0.000000,0.376471}%
\pgfsetstrokecolor{currentstroke}%
\pgfsetdash{}{0pt}%
\pgfpathmoveto{\pgfqpoint{0.694334in}{0.599366in}}%
\pgfpathlineto{\pgfqpoint{0.696249in}{0.616722in}}%
\pgfpathlineto{\pgfqpoint{0.698165in}{0.623343in}}%
\pgfpathlineto{\pgfqpoint{0.700080in}{0.624942in}}%
\pgfpathlineto{\pgfqpoint{0.701995in}{0.628627in}}%
\pgfpathlineto{\pgfqpoint{0.707741in}{0.675393in}}%
\pgfpathlineto{\pgfqpoint{0.709656in}{0.679304in}}%
\pgfpathlineto{\pgfqpoint{0.713486in}{0.680868in}}%
\pgfpathlineto{\pgfqpoint{0.715401in}{0.684930in}}%
\pgfpathlineto{\pgfqpoint{0.717316in}{0.685863in}}%
\pgfpathlineto{\pgfqpoint{0.721147in}{0.699237in}}%
\pgfpathlineto{\pgfqpoint{0.724977in}{0.724415in}}%
\pgfpathlineto{\pgfqpoint{0.728807in}{0.731356in}}%
\pgfpathlineto{\pgfqpoint{0.730723in}{0.731663in}}%
\pgfpathlineto{\pgfqpoint{0.732638in}{0.733313in}}%
\pgfpathlineto{\pgfqpoint{0.736468in}{0.745375in}}%
\pgfpathlineto{\pgfqpoint{0.749874in}{0.754558in}}%
\pgfpathlineto{\pgfqpoint{0.757535in}{0.755842in}}%
\pgfpathlineto{\pgfqpoint{0.765196in}{0.759516in}}%
\pgfpathlineto{\pgfqpoint{0.769026in}{0.761092in}}%
\pgfpathlineto{\pgfqpoint{0.778602in}{0.763699in}}%
\pgfpathlineto{\pgfqpoint{0.780517in}{0.765949in}}%
\pgfpathlineto{\pgfqpoint{0.782432in}{0.766314in}}%
\pgfpathlineto{\pgfqpoint{0.784347in}{0.770570in}}%
\pgfpathlineto{\pgfqpoint{0.795838in}{0.772555in}}%
\pgfpathlineto{\pgfqpoint{0.834142in}{0.780708in}}%
\pgfpathlineto{\pgfqpoint{0.837972in}{0.781987in}}%
\pgfpathlineto{\pgfqpoint{0.841803in}{0.783419in}}%
\pgfpathlineto{\pgfqpoint{0.843718in}{0.784883in}}%
\pgfpathlineto{\pgfqpoint{0.847548in}{0.785885in}}%
\pgfpathlineto{\pgfqpoint{0.853294in}{0.788128in}}%
\pgfpathlineto{\pgfqpoint{0.855209in}{0.791418in}}%
\pgfpathlineto{\pgfqpoint{0.857124in}{0.791448in}}%
\pgfpathlineto{\pgfqpoint{0.860954in}{0.793652in}}%
\pgfpathlineto{\pgfqpoint{0.866700in}{0.795597in}}%
\pgfpathlineto{\pgfqpoint{0.870530in}{0.798359in}}%
\pgfpathlineto{\pgfqpoint{0.872445in}{0.798803in}}%
\pgfpathlineto{\pgfqpoint{0.876276in}{0.802324in}}%
\pgfpathlineto{\pgfqpoint{0.885851in}{0.804749in}}%
\pgfpathlineto{\pgfqpoint{0.903088in}{0.810705in}}%
\pgfpathlineto{\pgfqpoint{0.905003in}{0.813318in}}%
\pgfpathlineto{\pgfqpoint{0.912664in}{0.815343in}}%
\pgfpathlineto{\pgfqpoint{0.914579in}{0.817710in}}%
\pgfpathlineto{\pgfqpoint{0.918409in}{0.818198in}}%
\pgfpathlineto{\pgfqpoint{0.920325in}{0.820676in}}%
\pgfpathlineto{\pgfqpoint{0.926070in}{0.821629in}}%
\pgfpathlineto{\pgfqpoint{0.929900in}{0.823259in}}%
\pgfpathlineto{\pgfqpoint{0.939476in}{0.826427in}}%
\pgfpathlineto{\pgfqpoint{0.943307in}{0.830369in}}%
\pgfpathlineto{\pgfqpoint{0.952882in}{0.834160in}}%
\pgfpathlineto{\pgfqpoint{0.964373in}{0.836130in}}%
\pgfpathlineto{\pgfqpoint{0.966289in}{0.837963in}}%
\pgfpathlineto{\pgfqpoint{0.972034in}{0.838534in}}%
\pgfpathlineto{\pgfqpoint{0.975864in}{0.841301in}}%
\pgfpathlineto{\pgfqpoint{0.977780in}{0.841407in}}%
\pgfpathlineto{\pgfqpoint{0.979695in}{0.843140in}}%
\pgfpathlineto{\pgfqpoint{0.985440in}{0.844131in}}%
\pgfpathlineto{\pgfqpoint{0.989271in}{0.845658in}}%
\pgfpathlineto{\pgfqpoint{1.008422in}{0.852766in}}%
\pgfpathlineto{\pgfqpoint{1.021829in}{0.854652in}}%
\pgfpathlineto{\pgfqpoint{1.025659in}{0.856194in}}%
\pgfpathlineto{\pgfqpoint{1.042895in}{0.858909in}}%
\pgfpathlineto{\pgfqpoint{1.056302in}{0.860634in}}%
\pgfpathlineto{\pgfqpoint{1.060132in}{0.862067in}}%
\pgfpathlineto{\pgfqpoint{1.088860in}{0.868666in}}%
\pgfpathlineto{\pgfqpoint{1.100351in}{0.871832in}}%
\pgfpathlineto{\pgfqpoint{1.108011in}{0.873189in}}%
\pgfpathlineto{\pgfqpoint{1.111842in}{0.873923in}}%
\pgfpathlineto{\pgfqpoint{1.121418in}{0.875284in}}%
\pgfpathlineto{\pgfqpoint{1.130993in}{0.877922in}}%
\pgfpathlineto{\pgfqpoint{1.140569in}{0.879026in}}%
\pgfpathlineto{\pgfqpoint{1.146315in}{0.880442in}}%
\pgfpathlineto{\pgfqpoint{1.152060in}{0.881287in}}%
\pgfpathlineto{\pgfqpoint{1.159721in}{0.883155in}}%
\pgfpathlineto{\pgfqpoint{1.199940in}{0.889064in}}%
\pgfpathlineto{\pgfqpoint{1.213346in}{0.889892in}}%
\pgfpathlineto{\pgfqpoint{1.217176in}{0.891013in}}%
\pgfpathlineto{\pgfqpoint{1.228667in}{0.892147in}}%
\pgfpathlineto{\pgfqpoint{1.266971in}{0.897178in}}%
\pgfpathlineto{\pgfqpoint{1.280377in}{0.898879in}}%
\pgfpathlineto{\pgfqpoint{1.286122in}{0.899993in}}%
\pgfpathlineto{\pgfqpoint{1.299528in}{0.902931in}}%
\pgfpathlineto{\pgfqpoint{1.312935in}{0.905879in}}%
\pgfpathlineto{\pgfqpoint{1.316765in}{0.906863in}}%
\pgfpathlineto{\pgfqpoint{1.322511in}{0.908110in}}%
\pgfpathlineto{\pgfqpoint{1.349323in}{0.911752in}}%
\pgfpathlineto{\pgfqpoint{1.355068in}{0.912981in}}%
\pgfpathlineto{\pgfqpoint{1.387626in}{0.917568in}}%
\pgfpathlineto{\pgfqpoint{1.454657in}{0.922597in}}%
\pgfpathlineto{\pgfqpoint{1.550416in}{0.933264in}}%
\pgfpathlineto{\pgfqpoint{1.565737in}{0.935599in}}%
\pgfpathlineto{\pgfqpoint{1.575313in}{0.936435in}}%
\pgfpathlineto{\pgfqpoint{1.579143in}{0.937425in}}%
\pgfpathlineto{\pgfqpoint{1.588719in}{0.938693in}}%
\pgfpathlineto{\pgfqpoint{1.607871in}{0.940878in}}%
\pgfpathlineto{\pgfqpoint{1.625108in}{0.944931in}}%
\pgfpathlineto{\pgfqpoint{1.646174in}{0.947189in}}%
\pgfpathlineto{\pgfqpoint{1.650005in}{0.948491in}}%
\pgfpathlineto{\pgfqpoint{1.653835in}{0.950144in}}%
\pgfpathlineto{\pgfqpoint{1.674902in}{0.953346in}}%
\pgfpathlineto{\pgfqpoint{1.686393in}{0.954526in}}%
\pgfpathlineto{\pgfqpoint{1.695969in}{0.955641in}}%
\pgfpathlineto{\pgfqpoint{1.703630in}{0.956799in}}%
\pgfpathlineto{\pgfqpoint{1.722781in}{0.959828in}}%
\pgfpathlineto{\pgfqpoint{1.740018in}{0.964303in}}%
\pgfpathlineto{\pgfqpoint{1.741933in}{0.964532in}}%
\pgfpathlineto{\pgfqpoint{1.743848in}{0.967466in}}%
\pgfpathlineto{\pgfqpoint{1.755339in}{0.969525in}}%
\pgfpathlineto{\pgfqpoint{1.774491in}{0.971453in}}%
\pgfpathlineto{\pgfqpoint{1.778321in}{0.973937in}}%
\pgfpathlineto{\pgfqpoint{1.780236in}{0.974197in}}%
\pgfpathlineto{\pgfqpoint{1.785982in}{0.977752in}}%
\pgfpathlineto{\pgfqpoint{1.797473in}{0.980035in}}%
\pgfpathlineto{\pgfqpoint{1.820455in}{0.983071in}}%
\pgfpathlineto{\pgfqpoint{1.824285in}{0.983797in}}%
\pgfpathlineto{\pgfqpoint{1.833861in}{0.984717in}}%
\pgfpathlineto{\pgfqpoint{1.837692in}{0.986598in}}%
\pgfpathlineto{\pgfqpoint{1.860674in}{0.989409in}}%
\pgfpathlineto{\pgfqpoint{1.868334in}{0.991345in}}%
\pgfpathlineto{\pgfqpoint{1.877910in}{0.992458in}}%
\pgfpathlineto{\pgfqpoint{1.900892in}{0.997358in}}%
\pgfpathlineto{\pgfqpoint{1.904723in}{0.998557in}}%
\pgfpathlineto{\pgfqpoint{1.918129in}{1.001810in}}%
\pgfpathlineto{\pgfqpoint{1.929620in}{1.003261in}}%
\pgfpathlineto{\pgfqpoint{1.941111in}{1.004191in}}%
\pgfpathlineto{\pgfqpoint{1.954517in}{1.006170in}}%
\pgfpathlineto{\pgfqpoint{1.960263in}{1.007225in}}%
\pgfpathlineto{\pgfqpoint{1.964093in}{1.009090in}}%
\pgfpathlineto{\pgfqpoint{1.992821in}{1.012236in}}%
\pgfpathlineto{\pgfqpoint{1.996651in}{1.013673in}}%
\pgfpathlineto{\pgfqpoint{2.021548in}{1.017232in}}%
\pgfpathlineto{\pgfqpoint{2.027294in}{1.018388in}}%
\pgfpathlineto{\pgfqpoint{2.036869in}{1.019759in}}%
\pgfpathlineto{\pgfqpoint{2.038785in}{1.022034in}}%
\pgfpathlineto{\pgfqpoint{2.140289in}{1.034822in}}%
\pgfpathlineto{\pgfqpoint{2.149865in}{1.036153in}}%
\pgfpathlineto{\pgfqpoint{2.155610in}{1.036768in}}%
\pgfpathlineto{\pgfqpoint{2.161356in}{1.038141in}}%
\pgfpathlineto{\pgfqpoint{2.172847in}{1.039984in}}%
\pgfpathlineto{\pgfqpoint{2.188168in}{1.042432in}}%
\pgfpathlineto{\pgfqpoint{2.193914in}{1.044353in}}%
\pgfpathlineto{\pgfqpoint{2.203489in}{1.045628in}}%
\pgfpathlineto{\pgfqpoint{2.222641in}{1.049509in}}%
\pgfpathlineto{\pgfqpoint{2.228387in}{1.050142in}}%
\pgfpathlineto{\pgfqpoint{2.297333in}{1.061340in}}%
\pgfpathlineto{\pgfqpoint{2.301163in}{1.062700in}}%
\pgfpathlineto{\pgfqpoint{2.303078in}{1.062824in}}%
\pgfpathlineto{\pgfqpoint{2.304993in}{1.064582in}}%
\pgfpathlineto{\pgfqpoint{2.314569in}{1.065431in}}%
\pgfpathlineto{\pgfqpoint{2.318400in}{1.066746in}}%
\pgfpathlineto{\pgfqpoint{2.324145in}{1.067012in}}%
\pgfpathlineto{\pgfqpoint{2.327975in}{1.068510in}}%
\pgfpathlineto{\pgfqpoint{2.339467in}{1.069371in}}%
\pgfpathlineto{\pgfqpoint{2.358618in}{1.073888in}}%
\pgfpathlineto{\pgfqpoint{2.387346in}{1.079643in}}%
\pgfpathlineto{\pgfqpoint{2.393091in}{1.080477in}}%
\pgfpathlineto{\pgfqpoint{2.398837in}{1.082285in}}%
\pgfpathlineto{\pgfqpoint{2.402667in}{1.083946in}}%
\pgfpathlineto{\pgfqpoint{2.406498in}{1.084690in}}%
\pgfpathlineto{\pgfqpoint{2.416073in}{1.086223in}}%
\pgfpathlineto{\pgfqpoint{2.431395in}{1.089847in}}%
\pgfpathlineto{\pgfqpoint{2.433310in}{1.089886in}}%
\pgfpathlineto{\pgfqpoint{2.435225in}{1.091309in}}%
\pgfpathlineto{\pgfqpoint{2.439055in}{1.091773in}}%
\pgfpathlineto{\pgfqpoint{2.442886in}{1.092888in}}%
\pgfpathlineto{\pgfqpoint{2.450546in}{1.093513in}}%
\pgfpathlineto{\pgfqpoint{2.456292in}{1.095811in}}%
\pgfpathlineto{\pgfqpoint{2.465868in}{1.099734in}}%
\pgfpathlineto{\pgfqpoint{2.483104in}{1.104601in}}%
\pgfpathlineto{\pgfqpoint{2.496511in}{1.107612in}}%
\pgfpathlineto{\pgfqpoint{2.500341in}{1.112103in}}%
\pgfpathlineto{\pgfqpoint{2.527153in}{1.117385in}}%
\pgfpathlineto{\pgfqpoint{2.546305in}{1.123202in}}%
\pgfpathlineto{\pgfqpoint{2.553966in}{1.124777in}}%
\pgfpathlineto{\pgfqpoint{2.559711in}{1.127149in}}%
\pgfpathlineto{\pgfqpoint{2.567372in}{1.128651in}}%
\pgfpathlineto{\pgfqpoint{2.569287in}{1.132174in}}%
\pgfpathlineto{\pgfqpoint{2.571202in}{1.132231in}}%
\pgfpathlineto{\pgfqpoint{2.575033in}{1.133807in}}%
\pgfpathlineto{\pgfqpoint{2.582693in}{1.135698in}}%
\pgfpathlineto{\pgfqpoint{2.584608in}{1.136519in}}%
\pgfpathlineto{\pgfqpoint{2.586524in}{1.139103in}}%
\pgfpathlineto{\pgfqpoint{2.598015in}{1.141933in}}%
\pgfpathlineto{\pgfqpoint{2.599930in}{1.145281in}}%
\pgfpathlineto{\pgfqpoint{2.609506in}{1.148967in}}%
\pgfpathlineto{\pgfqpoint{2.615251in}{1.151012in}}%
\pgfpathlineto{\pgfqpoint{2.632488in}{1.160730in}}%
\pgfpathlineto{\pgfqpoint{2.634403in}{1.162259in}}%
\pgfpathlineto{\pgfqpoint{2.636318in}{1.162292in}}%
\pgfpathlineto{\pgfqpoint{2.640148in}{1.164127in}}%
\pgfpathlineto{\pgfqpoint{2.642064in}{1.164768in}}%
\pgfpathlineto{\pgfqpoint{2.645894in}{1.171670in}}%
\pgfpathlineto{\pgfqpoint{2.649724in}{1.173699in}}%
\pgfpathlineto{\pgfqpoint{2.651639in}{1.177104in}}%
\pgfpathlineto{\pgfqpoint{2.655470in}{1.178221in}}%
\pgfpathlineto{\pgfqpoint{2.661215in}{1.181616in}}%
\pgfpathlineto{\pgfqpoint{2.665046in}{1.182529in}}%
\pgfpathlineto{\pgfqpoint{2.668876in}{1.183959in}}%
\pgfpathlineto{\pgfqpoint{2.672706in}{1.185299in}}%
\pgfpathlineto{\pgfqpoint{2.674622in}{1.189981in}}%
\pgfpathlineto{\pgfqpoint{2.678452in}{1.191422in}}%
\pgfpathlineto{\pgfqpoint{2.693773in}{1.196168in}}%
\pgfpathlineto{\pgfqpoint{2.697604in}{1.201777in}}%
\pgfpathlineto{\pgfqpoint{2.699519in}{1.203232in}}%
\pgfpathlineto{\pgfqpoint{2.709095in}{1.204549in}}%
\pgfpathlineto{\pgfqpoint{2.716755in}{1.206028in}}%
\pgfpathlineto{\pgfqpoint{2.720586in}{1.207480in}}%
\pgfpathlineto{\pgfqpoint{2.724416in}{1.207804in}}%
\pgfpathlineto{\pgfqpoint{2.737822in}{1.216964in}}%
\pgfpathlineto{\pgfqpoint{2.741653in}{1.217531in}}%
\pgfpathlineto{\pgfqpoint{2.743568in}{1.220838in}}%
\pgfpathlineto{\pgfqpoint{2.745483in}{1.221483in}}%
\pgfpathlineto{\pgfqpoint{2.747398in}{1.224490in}}%
\pgfpathlineto{\pgfqpoint{2.755059in}{1.226493in}}%
\pgfpathlineto{\pgfqpoint{2.756974in}{1.228797in}}%
\pgfpathlineto{\pgfqpoint{2.758889in}{1.233639in}}%
\pgfpathlineto{\pgfqpoint{2.762719in}{1.234828in}}%
\pgfpathlineto{\pgfqpoint{2.766550in}{1.236943in}}%
\pgfpathlineto{\pgfqpoint{2.770380in}{1.237936in}}%
\pgfpathlineto{\pgfqpoint{2.772295in}{1.239920in}}%
\pgfpathlineto{\pgfqpoint{2.774210in}{1.239988in}}%
\pgfpathlineto{\pgfqpoint{2.776126in}{1.242697in}}%
\pgfpathlineto{\pgfqpoint{2.778041in}{1.242864in}}%
\pgfpathlineto{\pgfqpoint{2.785701in}{1.248064in}}%
\pgfpathlineto{\pgfqpoint{2.797192in}{1.252870in}}%
\pgfpathlineto{\pgfqpoint{2.806768in}{1.254761in}}%
\pgfpathlineto{\pgfqpoint{2.808684in}{1.258943in}}%
\pgfpathlineto{\pgfqpoint{2.812514in}{1.260611in}}%
\pgfpathlineto{\pgfqpoint{2.820175in}{1.268660in}}%
\pgfpathlineto{\pgfqpoint{2.822090in}{1.275625in}}%
\pgfpathlineto{\pgfqpoint{2.831666in}{1.277604in}}%
\pgfpathlineto{\pgfqpoint{2.833581in}{1.282838in}}%
\pgfpathlineto{\pgfqpoint{2.839326in}{1.286615in}}%
\pgfpathlineto{\pgfqpoint{2.843157in}{1.288602in}}%
\pgfpathlineto{\pgfqpoint{2.845072in}{1.289591in}}%
\pgfpathlineto{\pgfqpoint{2.850817in}{1.297029in}}%
\pgfpathlineto{\pgfqpoint{2.854648in}{1.299721in}}%
\pgfpathlineto{\pgfqpoint{2.856563in}{1.300402in}}%
\pgfpathlineto{\pgfqpoint{2.862308in}{1.305160in}}%
\pgfpathlineto{\pgfqpoint{2.866139in}{1.306455in}}%
\pgfpathlineto{\pgfqpoint{2.871884in}{1.310414in}}%
\pgfpathlineto{\pgfqpoint{2.875715in}{1.314566in}}%
\pgfpathlineto{\pgfqpoint{2.877630in}{1.315000in}}%
\pgfpathlineto{\pgfqpoint{2.885290in}{1.327105in}}%
\pgfpathlineto{\pgfqpoint{2.889121in}{1.327696in}}%
\pgfpathlineto{\pgfqpoint{2.891036in}{1.332498in}}%
\pgfpathlineto{\pgfqpoint{2.892951in}{1.333169in}}%
\pgfpathlineto{\pgfqpoint{2.896781in}{1.335924in}}%
\pgfpathlineto{\pgfqpoint{2.898697in}{1.341167in}}%
\pgfpathlineto{\pgfqpoint{2.906357in}{1.344628in}}%
\pgfpathlineto{\pgfqpoint{2.908272in}{1.351538in}}%
\pgfpathlineto{\pgfqpoint{2.910188in}{1.352266in}}%
\pgfpathlineto{\pgfqpoint{2.915933in}{1.363609in}}%
\pgfpathlineto{\pgfqpoint{2.919763in}{1.365061in}}%
\pgfpathlineto{\pgfqpoint{2.921679in}{1.373283in}}%
\pgfpathlineto{\pgfqpoint{2.929339in}{1.380897in}}%
\pgfpathlineto{\pgfqpoint{2.931254in}{1.381325in}}%
\pgfpathlineto{\pgfqpoint{2.933170in}{1.387435in}}%
\pgfpathlineto{\pgfqpoint{2.935085in}{1.387793in}}%
\pgfpathlineto{\pgfqpoint{2.937000in}{1.393773in}}%
\pgfpathlineto{\pgfqpoint{2.938915in}{1.394062in}}%
\pgfpathlineto{\pgfqpoint{2.940830in}{1.401763in}}%
\pgfpathlineto{\pgfqpoint{2.942746in}{1.402862in}}%
\pgfpathlineto{\pgfqpoint{2.944661in}{1.419801in}}%
\pgfpathlineto{\pgfqpoint{2.948491in}{1.431082in}}%
\pgfpathlineto{\pgfqpoint{2.954237in}{1.443016in}}%
\pgfpathlineto{\pgfqpoint{2.956152in}{1.450338in}}%
\pgfpathlineto{\pgfqpoint{2.958067in}{1.471589in}}%
\pgfpathlineto{\pgfqpoint{2.963812in}{1.493713in}}%
\pgfpathlineto{\pgfqpoint{2.967643in}{1.504276in}}%
\pgfpathlineto{\pgfqpoint{2.971473in}{1.540108in}}%
\pgfpathlineto{\pgfqpoint{2.973388in}{1.541925in}}%
\pgfpathlineto{\pgfqpoint{2.975303in}{1.546393in}}%
\pgfpathlineto{\pgfqpoint{2.977219in}{1.558329in}}%
\pgfpathlineto{\pgfqpoint{2.979134in}{1.561688in}}%
\pgfpathlineto{\pgfqpoint{2.981049in}{1.569034in}}%
\pgfpathlineto{\pgfqpoint{2.982964in}{1.597494in}}%
\pgfpathlineto{\pgfqpoint{2.984879in}{1.602936in}}%
\pgfpathlineto{\pgfqpoint{2.986794in}{1.622419in}}%
\pgfpathlineto{\pgfqpoint{2.992540in}{1.641153in}}%
\pgfpathlineto{\pgfqpoint{2.996370in}{1.651511in}}%
\pgfpathlineto{\pgfqpoint{3.000201in}{1.657346in}}%
\pgfpathlineto{\pgfqpoint{3.004031in}{1.665559in}}%
\pgfpathlineto{\pgfqpoint{3.005946in}{1.668235in}}%
\pgfpathlineto{\pgfqpoint{3.009777in}{1.700837in}}%
\pgfpathlineto{\pgfqpoint{3.013607in}{1.713141in}}%
\pgfpathlineto{\pgfqpoint{3.017437in}{1.723441in}}%
\pgfpathlineto{\pgfqpoint{3.019352in}{1.728682in}}%
\pgfpathlineto{\pgfqpoint{3.021268in}{1.730441in}}%
\pgfpathlineto{\pgfqpoint{3.025098in}{1.753429in}}%
\pgfpathlineto{\pgfqpoint{3.027013in}{1.794188in}}%
\pgfpathlineto{\pgfqpoint{3.028928in}{1.794719in}}%
\pgfpathlineto{\pgfqpoint{3.030843in}{1.797123in}}%
\pgfpathlineto{\pgfqpoint{3.032759in}{1.826535in}}%
\pgfpathlineto{\pgfqpoint{3.032759in}{1.826535in}}%
\pgfusepath{stroke}%
\end{pgfscope}%
\begin{pgfscope}%
\pgfpathrectangle{\pgfqpoint{0.694334in}{0.523557in}}{\pgfqpoint{3.830343in}{1.302977in}}%
\pgfusepath{clip}%
\pgfsetrectcap%
\pgfsetroundjoin%
\pgfsetlinewidth{1.003750pt}%
\definecolor{currentstroke}{rgb}{0.564706,0.564706,1.000000}%
\pgfsetstrokecolor{currentstroke}%
\pgfsetdash{}{0pt}%
\pgfpathmoveto{\pgfqpoint{0.694334in}{0.703806in}}%
\pgfpathlineto{\pgfqpoint{0.696249in}{0.737312in}}%
\pgfpathlineto{\pgfqpoint{0.698165in}{0.740239in}}%
\pgfpathlineto{\pgfqpoint{0.700080in}{0.745465in}}%
\pgfpathlineto{\pgfqpoint{0.703910in}{0.746981in}}%
\pgfpathlineto{\pgfqpoint{0.705825in}{0.754666in}}%
\pgfpathlineto{\pgfqpoint{0.713486in}{0.757532in}}%
\pgfpathlineto{\pgfqpoint{0.719232in}{0.759770in}}%
\pgfpathlineto{\pgfqpoint{0.724977in}{0.761277in}}%
\pgfpathlineto{\pgfqpoint{0.728807in}{0.761602in}}%
\pgfpathlineto{\pgfqpoint{0.732638in}{0.762857in}}%
\pgfpathlineto{\pgfqpoint{0.736468in}{0.763733in}}%
\pgfpathlineto{\pgfqpoint{0.740298in}{0.766512in}}%
\pgfpathlineto{\pgfqpoint{0.744129in}{0.768190in}}%
\pgfpathlineto{\pgfqpoint{0.747959in}{0.768811in}}%
\pgfpathlineto{\pgfqpoint{0.757535in}{0.769539in}}%
\pgfpathlineto{\pgfqpoint{0.763280in}{0.771747in}}%
\pgfpathlineto{\pgfqpoint{0.769026in}{0.774631in}}%
\pgfpathlineto{\pgfqpoint{0.770941in}{0.774736in}}%
\pgfpathlineto{\pgfqpoint{0.772856in}{0.777195in}}%
\pgfpathlineto{\pgfqpoint{0.780517in}{0.779321in}}%
\pgfpathlineto{\pgfqpoint{0.790093in}{0.782434in}}%
\pgfpathlineto{\pgfqpoint{0.793923in}{0.784767in}}%
\pgfpathlineto{\pgfqpoint{0.795838in}{0.784811in}}%
\pgfpathlineto{\pgfqpoint{0.799669in}{0.788278in}}%
\pgfpathlineto{\pgfqpoint{0.805414in}{0.789452in}}%
\pgfpathlineto{\pgfqpoint{0.809245in}{0.791820in}}%
\pgfpathlineto{\pgfqpoint{0.816905in}{0.793698in}}%
\pgfpathlineto{\pgfqpoint{0.818820in}{0.795425in}}%
\pgfpathlineto{\pgfqpoint{0.824566in}{0.796047in}}%
\pgfpathlineto{\pgfqpoint{0.832227in}{0.799687in}}%
\pgfpathlineto{\pgfqpoint{0.841803in}{0.802601in}}%
\pgfpathlineto{\pgfqpoint{0.845633in}{0.803979in}}%
\pgfpathlineto{\pgfqpoint{0.847548in}{0.803993in}}%
\pgfpathlineto{\pgfqpoint{0.851378in}{0.807098in}}%
\pgfpathlineto{\pgfqpoint{0.862869in}{0.808678in}}%
\pgfpathlineto{\pgfqpoint{0.866700in}{0.810160in}}%
\pgfpathlineto{\pgfqpoint{0.874360in}{0.812265in}}%
\pgfpathlineto{\pgfqpoint{0.880106in}{0.813507in}}%
\pgfpathlineto{\pgfqpoint{0.882021in}{0.813875in}}%
\pgfpathlineto{\pgfqpoint{0.887767in}{0.818447in}}%
\pgfpathlineto{\pgfqpoint{0.891597in}{0.819748in}}%
\pgfpathlineto{\pgfqpoint{0.906918in}{0.828475in}}%
\pgfpathlineto{\pgfqpoint{0.908834in}{0.828597in}}%
\pgfpathlineto{\pgfqpoint{0.914579in}{0.834186in}}%
\pgfpathlineto{\pgfqpoint{0.920325in}{0.837295in}}%
\pgfpathlineto{\pgfqpoint{0.924155in}{0.839796in}}%
\pgfpathlineto{\pgfqpoint{0.927985in}{0.840826in}}%
\pgfpathlineto{\pgfqpoint{0.929900in}{0.842079in}}%
\pgfpathlineto{\pgfqpoint{0.935646in}{0.849780in}}%
\pgfpathlineto{\pgfqpoint{0.949052in}{0.854951in}}%
\pgfpathlineto{\pgfqpoint{0.954798in}{0.855791in}}%
\pgfpathlineto{\pgfqpoint{0.962458in}{0.861168in}}%
\pgfpathlineto{\pgfqpoint{0.966289in}{0.861769in}}%
\pgfpathlineto{\pgfqpoint{0.968204in}{0.864003in}}%
\pgfpathlineto{\pgfqpoint{0.973949in}{0.866159in}}%
\pgfpathlineto{\pgfqpoint{0.979695in}{0.867453in}}%
\pgfpathlineto{\pgfqpoint{0.985440in}{0.869988in}}%
\pgfpathlineto{\pgfqpoint{0.987356in}{0.873680in}}%
\pgfpathlineto{\pgfqpoint{0.989271in}{0.873771in}}%
\pgfpathlineto{\pgfqpoint{0.993101in}{0.875298in}}%
\pgfpathlineto{\pgfqpoint{1.023744in}{0.880894in}}%
\pgfpathlineto{\pgfqpoint{1.029489in}{0.883667in}}%
\pgfpathlineto{\pgfqpoint{1.033320in}{0.884552in}}%
\pgfpathlineto{\pgfqpoint{1.037150in}{0.885936in}}%
\pgfpathlineto{\pgfqpoint{1.040980in}{0.887013in}}%
\pgfpathlineto{\pgfqpoint{1.044811in}{0.887763in}}%
\pgfpathlineto{\pgfqpoint{1.050556in}{0.889467in}}%
\pgfpathlineto{\pgfqpoint{1.054387in}{0.889942in}}%
\pgfpathlineto{\pgfqpoint{1.058217in}{0.891196in}}%
\pgfpathlineto{\pgfqpoint{1.065878in}{0.892859in}}%
\pgfpathlineto{\pgfqpoint{1.088860in}{0.896070in}}%
\pgfpathlineto{\pgfqpoint{1.090775in}{0.897788in}}%
\pgfpathlineto{\pgfqpoint{1.094605in}{0.898895in}}%
\pgfpathlineto{\pgfqpoint{1.098435in}{0.900369in}}%
\pgfpathlineto{\pgfqpoint{1.106096in}{0.901823in}}%
\pgfpathlineto{\pgfqpoint{1.113757in}{0.903021in}}%
\pgfpathlineto{\pgfqpoint{1.153975in}{0.909369in}}%
\pgfpathlineto{\pgfqpoint{1.157806in}{0.911192in}}%
\pgfpathlineto{\pgfqpoint{1.167382in}{0.912204in}}%
\pgfpathlineto{\pgfqpoint{1.173127in}{0.912697in}}%
\pgfpathlineto{\pgfqpoint{1.178873in}{0.913300in}}%
\pgfpathlineto{\pgfqpoint{1.184618in}{0.914770in}}%
\pgfpathlineto{\pgfqpoint{1.205685in}{0.916886in}}%
\pgfpathlineto{\pgfqpoint{1.215261in}{0.918381in}}%
\pgfpathlineto{\pgfqpoint{1.284207in}{0.927502in}}%
\pgfpathlineto{\pgfqpoint{1.301444in}{0.929047in}}%
\pgfpathlineto{\pgfqpoint{1.307189in}{0.930186in}}%
\pgfpathlineto{\pgfqpoint{1.312935in}{0.931042in}}%
\pgfpathlineto{\pgfqpoint{1.374220in}{0.937970in}}%
\pgfpathlineto{\pgfqpoint{1.378050in}{0.938874in}}%
\pgfpathlineto{\pgfqpoint{1.393372in}{0.940231in}}%
\pgfpathlineto{\pgfqpoint{1.437421in}{0.945512in}}%
\pgfpathlineto{\pgfqpoint{1.445081in}{0.946576in}}%
\pgfpathlineto{\pgfqpoint{1.517858in}{0.952724in}}%
\pgfpathlineto{\pgfqpoint{1.521688in}{0.954409in}}%
\pgfpathlineto{\pgfqpoint{1.554246in}{0.957380in}}%
\pgfpathlineto{\pgfqpoint{1.582974in}{0.959969in}}%
\pgfpathlineto{\pgfqpoint{1.590635in}{0.960756in}}%
\pgfpathlineto{\pgfqpoint{1.623192in}{0.964336in}}%
\pgfpathlineto{\pgfqpoint{1.628938in}{0.965136in}}%
\pgfpathlineto{\pgfqpoint{1.636599in}{0.966438in}}%
\pgfpathlineto{\pgfqpoint{1.642344in}{0.967911in}}%
\pgfpathlineto{\pgfqpoint{1.686393in}{0.974032in}}%
\pgfpathlineto{\pgfqpoint{1.690223in}{0.975041in}}%
\pgfpathlineto{\pgfqpoint{1.734272in}{0.981467in}}%
\pgfpathlineto{\pgfqpoint{1.740018in}{0.983447in}}%
\pgfpathlineto{\pgfqpoint{1.745763in}{0.984686in}}%
\pgfpathlineto{\pgfqpoint{1.755339in}{0.986555in}}%
\pgfpathlineto{\pgfqpoint{1.761085in}{0.987270in}}%
\pgfpathlineto{\pgfqpoint{1.772576in}{0.988294in}}%
\pgfpathlineto{\pgfqpoint{1.785982in}{0.991567in}}%
\pgfpathlineto{\pgfqpoint{1.816625in}{0.997086in}}%
\pgfpathlineto{\pgfqpoint{1.820455in}{0.997821in}}%
\pgfpathlineto{\pgfqpoint{1.826201in}{0.998985in}}%
\pgfpathlineto{\pgfqpoint{1.921959in}{1.013104in}}%
\pgfpathlineto{\pgfqpoint{1.935365in}{1.014109in}}%
\pgfpathlineto{\pgfqpoint{1.990905in}{1.021609in}}%
\pgfpathlineto{\pgfqpoint{1.994736in}{1.023228in}}%
\pgfpathlineto{\pgfqpoint{2.004312in}{1.023816in}}%
\pgfpathlineto{\pgfqpoint{2.008142in}{1.025233in}}%
\pgfpathlineto{\pgfqpoint{2.023463in}{1.027506in}}%
\pgfpathlineto{\pgfqpoint{2.046445in}{1.029353in}}%
\pgfpathlineto{\pgfqpoint{2.071343in}{1.032582in}}%
\pgfpathlineto{\pgfqpoint{2.082834in}{1.034329in}}%
\pgfpathlineto{\pgfqpoint{2.142204in}{1.042093in}}%
\pgfpathlineto{\pgfqpoint{2.147949in}{1.043497in}}%
\pgfpathlineto{\pgfqpoint{2.163271in}{1.045316in}}%
\pgfpathlineto{\pgfqpoint{2.176677in}{1.047736in}}%
\pgfpathlineto{\pgfqpoint{2.182422in}{1.048601in}}%
\pgfpathlineto{\pgfqpoint{2.197744in}{1.049935in}}%
\pgfpathlineto{\pgfqpoint{2.214980in}{1.053074in}}%
\pgfpathlineto{\pgfqpoint{2.237962in}{1.055008in}}%
\pgfpathlineto{\pgfqpoint{2.262860in}{1.056430in}}%
\pgfpathlineto{\pgfqpoint{2.268605in}{1.057931in}}%
\pgfpathlineto{\pgfqpoint{2.289672in}{1.060641in}}%
\pgfpathlineto{\pgfqpoint{2.308824in}{1.064654in}}%
\pgfpathlineto{\pgfqpoint{2.396922in}{1.078320in}}%
\pgfpathlineto{\pgfqpoint{2.400752in}{1.079652in}}%
\pgfpathlineto{\pgfqpoint{2.417989in}{1.081793in}}%
\pgfpathlineto{\pgfqpoint{2.419904in}{1.084435in}}%
\pgfpathlineto{\pgfqpoint{2.439055in}{1.089532in}}%
\pgfpathlineto{\pgfqpoint{2.442886in}{1.091604in}}%
\pgfpathlineto{\pgfqpoint{2.446716in}{1.092017in}}%
\pgfpathlineto{\pgfqpoint{2.448631in}{1.094610in}}%
\pgfpathlineto{\pgfqpoint{2.454377in}{1.095374in}}%
\pgfpathlineto{\pgfqpoint{2.469698in}{1.105210in}}%
\pgfpathlineto{\pgfqpoint{2.523323in}{1.120572in}}%
\pgfpathlineto{\pgfqpoint{2.540560in}{1.129362in}}%
\pgfpathlineto{\pgfqpoint{2.544390in}{1.130803in}}%
\pgfpathlineto{\pgfqpoint{2.546305in}{1.131678in}}%
\pgfpathlineto{\pgfqpoint{2.548220in}{1.135472in}}%
\pgfpathlineto{\pgfqpoint{2.550135in}{1.135501in}}%
\pgfpathlineto{\pgfqpoint{2.552051in}{1.138245in}}%
\pgfpathlineto{\pgfqpoint{2.569287in}{1.140683in}}%
\pgfpathlineto{\pgfqpoint{2.584608in}{1.145341in}}%
\pgfpathlineto{\pgfqpoint{2.599930in}{1.150057in}}%
\pgfpathlineto{\pgfqpoint{2.603760in}{1.153358in}}%
\pgfpathlineto{\pgfqpoint{2.613336in}{1.154237in}}%
\pgfpathlineto{\pgfqpoint{2.620997in}{1.157395in}}%
\pgfpathlineto{\pgfqpoint{2.624827in}{1.158424in}}%
\pgfpathlineto{\pgfqpoint{2.628657in}{1.161592in}}%
\pgfpathlineto{\pgfqpoint{2.634403in}{1.162027in}}%
\pgfpathlineto{\pgfqpoint{2.638233in}{1.164186in}}%
\pgfpathlineto{\pgfqpoint{2.643979in}{1.165551in}}%
\pgfpathlineto{\pgfqpoint{2.647809in}{1.168084in}}%
\pgfpathlineto{\pgfqpoint{2.655470in}{1.170708in}}%
\pgfpathlineto{\pgfqpoint{2.659300in}{1.174661in}}%
\pgfpathlineto{\pgfqpoint{2.663130in}{1.176342in}}%
\pgfpathlineto{\pgfqpoint{2.672706in}{1.178980in}}%
\pgfpathlineto{\pgfqpoint{2.674622in}{1.184777in}}%
\pgfpathlineto{\pgfqpoint{2.680367in}{1.188055in}}%
\pgfpathlineto{\pgfqpoint{2.684197in}{1.189445in}}%
\pgfpathlineto{\pgfqpoint{2.688028in}{1.190641in}}%
\pgfpathlineto{\pgfqpoint{2.693773in}{1.194410in}}%
\pgfpathlineto{\pgfqpoint{2.705264in}{1.206707in}}%
\pgfpathlineto{\pgfqpoint{2.707179in}{1.207159in}}%
\pgfpathlineto{\pgfqpoint{2.709095in}{1.208957in}}%
\pgfpathlineto{\pgfqpoint{2.718670in}{1.210562in}}%
\pgfpathlineto{\pgfqpoint{2.720586in}{1.212025in}}%
\pgfpathlineto{\pgfqpoint{2.722501in}{1.217560in}}%
\pgfpathlineto{\pgfqpoint{2.733992in}{1.219462in}}%
\pgfpathlineto{\pgfqpoint{2.753144in}{1.226461in}}%
\pgfpathlineto{\pgfqpoint{2.756974in}{1.229147in}}%
\pgfpathlineto{\pgfqpoint{2.766550in}{1.231501in}}%
\pgfpathlineto{\pgfqpoint{2.768465in}{1.232466in}}%
\pgfpathlineto{\pgfqpoint{2.772295in}{1.235922in}}%
\pgfpathlineto{\pgfqpoint{2.779956in}{1.237844in}}%
\pgfpathlineto{\pgfqpoint{2.785701in}{1.243512in}}%
\pgfpathlineto{\pgfqpoint{2.789532in}{1.248620in}}%
\pgfpathlineto{\pgfqpoint{2.795277in}{1.250336in}}%
\pgfpathlineto{\pgfqpoint{2.797192in}{1.254998in}}%
\pgfpathlineto{\pgfqpoint{2.799108in}{1.255714in}}%
\pgfpathlineto{\pgfqpoint{2.808684in}{1.266275in}}%
\pgfpathlineto{\pgfqpoint{2.814429in}{1.267728in}}%
\pgfpathlineto{\pgfqpoint{2.816344in}{1.270971in}}%
\pgfpathlineto{\pgfqpoint{2.818259in}{1.271617in}}%
\pgfpathlineto{\pgfqpoint{2.820175in}{1.274300in}}%
\pgfpathlineto{\pgfqpoint{2.822090in}{1.274873in}}%
\pgfpathlineto{\pgfqpoint{2.846987in}{1.297343in}}%
\pgfpathlineto{\pgfqpoint{2.848902in}{1.298045in}}%
\pgfpathlineto{\pgfqpoint{2.850817in}{1.301241in}}%
\pgfpathlineto{\pgfqpoint{2.852732in}{1.301421in}}%
\pgfpathlineto{\pgfqpoint{2.858478in}{1.306625in}}%
\pgfpathlineto{\pgfqpoint{2.860393in}{1.306735in}}%
\pgfpathlineto{\pgfqpoint{2.866139in}{1.314238in}}%
\pgfpathlineto{\pgfqpoint{2.868054in}{1.314547in}}%
\pgfpathlineto{\pgfqpoint{2.869969in}{1.316651in}}%
\pgfpathlineto{\pgfqpoint{2.879545in}{1.318419in}}%
\pgfpathlineto{\pgfqpoint{2.881460in}{1.323457in}}%
\pgfpathlineto{\pgfqpoint{2.883375in}{1.325154in}}%
\pgfpathlineto{\pgfqpoint{2.891036in}{1.338332in}}%
\pgfpathlineto{\pgfqpoint{2.892951in}{1.339528in}}%
\pgfpathlineto{\pgfqpoint{2.896781in}{1.348804in}}%
\pgfpathlineto{\pgfqpoint{2.900612in}{1.349348in}}%
\pgfpathlineto{\pgfqpoint{2.904442in}{1.352064in}}%
\pgfpathlineto{\pgfqpoint{2.906357in}{1.352728in}}%
\pgfpathlineto{\pgfqpoint{2.910188in}{1.362177in}}%
\pgfpathlineto{\pgfqpoint{2.912103in}{1.362651in}}%
\pgfpathlineto{\pgfqpoint{2.915933in}{1.368245in}}%
\pgfpathlineto{\pgfqpoint{2.917848in}{1.368411in}}%
\pgfpathlineto{\pgfqpoint{2.919763in}{1.372686in}}%
\pgfpathlineto{\pgfqpoint{2.921679in}{1.383317in}}%
\pgfpathlineto{\pgfqpoint{2.923594in}{1.383544in}}%
\pgfpathlineto{\pgfqpoint{2.925509in}{1.385876in}}%
\pgfpathlineto{\pgfqpoint{2.927424in}{1.390546in}}%
\pgfpathlineto{\pgfqpoint{2.929339in}{1.392253in}}%
\pgfpathlineto{\pgfqpoint{2.931254in}{1.400743in}}%
\pgfpathlineto{\pgfqpoint{2.935085in}{1.402566in}}%
\pgfpathlineto{\pgfqpoint{2.937000in}{1.426236in}}%
\pgfpathlineto{\pgfqpoint{2.938915in}{1.434149in}}%
\pgfpathlineto{\pgfqpoint{2.940830in}{1.449912in}}%
\pgfpathlineto{\pgfqpoint{2.942746in}{1.453522in}}%
\pgfpathlineto{\pgfqpoint{2.944661in}{1.453739in}}%
\pgfpathlineto{\pgfqpoint{2.946576in}{1.460477in}}%
\pgfpathlineto{\pgfqpoint{2.948491in}{1.463050in}}%
\pgfpathlineto{\pgfqpoint{2.950406in}{1.477048in}}%
\pgfpathlineto{\pgfqpoint{2.952321in}{1.479890in}}%
\pgfpathlineto{\pgfqpoint{2.954237in}{1.500516in}}%
\pgfpathlineto{\pgfqpoint{2.956152in}{1.502706in}}%
\pgfpathlineto{\pgfqpoint{2.958067in}{1.514112in}}%
\pgfpathlineto{\pgfqpoint{2.959982in}{1.514813in}}%
\pgfpathlineto{\pgfqpoint{2.963812in}{1.551074in}}%
\pgfpathlineto{\pgfqpoint{2.967643in}{1.554889in}}%
\pgfpathlineto{\pgfqpoint{2.977219in}{1.604232in}}%
\pgfpathlineto{\pgfqpoint{2.979134in}{1.604287in}}%
\pgfpathlineto{\pgfqpoint{2.981049in}{1.609002in}}%
\pgfpathlineto{\pgfqpoint{2.982964in}{1.610477in}}%
\pgfpathlineto{\pgfqpoint{2.984879in}{1.615768in}}%
\pgfpathlineto{\pgfqpoint{2.986794in}{1.616285in}}%
\pgfpathlineto{\pgfqpoint{2.988710in}{1.627537in}}%
\pgfpathlineto{\pgfqpoint{2.990625in}{1.627787in}}%
\pgfpathlineto{\pgfqpoint{2.992540in}{1.636913in}}%
\pgfpathlineto{\pgfqpoint{2.994455in}{1.653743in}}%
\pgfpathlineto{\pgfqpoint{2.996370in}{1.655288in}}%
\pgfpathlineto{\pgfqpoint{2.998285in}{1.671393in}}%
\pgfpathlineto{\pgfqpoint{3.000201in}{1.675089in}}%
\pgfpathlineto{\pgfqpoint{3.002116in}{1.699109in}}%
\pgfpathlineto{\pgfqpoint{3.004031in}{1.703650in}}%
\pgfpathlineto{\pgfqpoint{3.007861in}{1.726928in}}%
\pgfpathlineto{\pgfqpoint{3.009777in}{1.727205in}}%
\pgfpathlineto{\pgfqpoint{3.011692in}{1.738607in}}%
\pgfpathlineto{\pgfqpoint{3.017437in}{1.747913in}}%
\pgfpathlineto{\pgfqpoint{3.021268in}{1.758869in}}%
\pgfpathlineto{\pgfqpoint{3.025098in}{1.766004in}}%
\pgfpathlineto{\pgfqpoint{3.027013in}{1.786911in}}%
\pgfpathlineto{\pgfqpoint{3.032759in}{1.799776in}}%
\pgfpathlineto{\pgfqpoint{3.034674in}{1.801144in}}%
\pgfpathlineto{\pgfqpoint{3.036589in}{1.826535in}}%
\pgfpathlineto{\pgfqpoint{3.036589in}{1.826535in}}%
\pgfusepath{stroke}%
\end{pgfscope}%
\begin{pgfscope}%
\pgfpathrectangle{\pgfqpoint{0.694334in}{0.523557in}}{\pgfqpoint{3.830343in}{1.302977in}}%
\pgfusepath{clip}%
\pgfsetbuttcap%
\pgfsetroundjoin%
\pgfsetlinewidth{1.003750pt}%
\definecolor{currentstroke}{rgb}{0.000000,0.000000,0.000000}%
\pgfsetstrokecolor{currentstroke}%
\pgfsetdash{{1.000000pt}{1.650000pt}}{0.000000pt}%
\pgfpathmoveto{\pgfqpoint{0.694334in}{0.567141in}}%
\pgfpathlineto{\pgfqpoint{0.696249in}{0.582562in}}%
\pgfpathlineto{\pgfqpoint{0.698165in}{0.582755in}}%
\pgfpathlineto{\pgfqpoint{0.701995in}{0.599148in}}%
\pgfpathlineto{\pgfqpoint{0.703910in}{0.606735in}}%
\pgfpathlineto{\pgfqpoint{0.711571in}{0.615604in}}%
\pgfpathlineto{\pgfqpoint{0.713486in}{0.616358in}}%
\pgfpathlineto{\pgfqpoint{0.715401in}{0.627746in}}%
\pgfpathlineto{\pgfqpoint{0.719232in}{0.630247in}}%
\pgfpathlineto{\pgfqpoint{0.721147in}{0.632363in}}%
\pgfpathlineto{\pgfqpoint{0.724977in}{0.633447in}}%
\pgfpathlineto{\pgfqpoint{0.726892in}{0.635337in}}%
\pgfpathlineto{\pgfqpoint{0.732638in}{0.646877in}}%
\pgfpathlineto{\pgfqpoint{0.736468in}{0.649507in}}%
\pgfpathlineto{\pgfqpoint{0.738383in}{0.655445in}}%
\pgfpathlineto{\pgfqpoint{0.740298in}{0.655930in}}%
\pgfpathlineto{\pgfqpoint{0.747959in}{0.671221in}}%
\pgfpathlineto{\pgfqpoint{0.749874in}{0.675620in}}%
\pgfpathlineto{\pgfqpoint{0.751789in}{0.675686in}}%
\pgfpathlineto{\pgfqpoint{0.753705in}{0.678423in}}%
\pgfpathlineto{\pgfqpoint{0.755620in}{0.683416in}}%
\pgfpathlineto{\pgfqpoint{0.759450in}{0.685036in}}%
\pgfpathlineto{\pgfqpoint{0.761365in}{0.690456in}}%
\pgfpathlineto{\pgfqpoint{0.770941in}{0.697917in}}%
\pgfpathlineto{\pgfqpoint{0.772856in}{0.698286in}}%
\pgfpathlineto{\pgfqpoint{0.778602in}{0.704153in}}%
\pgfpathlineto{\pgfqpoint{0.793923in}{0.708539in}}%
\pgfpathlineto{\pgfqpoint{0.801584in}{0.709532in}}%
\pgfpathlineto{\pgfqpoint{0.803499in}{0.711186in}}%
\pgfpathlineto{\pgfqpoint{0.811160in}{0.712162in}}%
\pgfpathlineto{\pgfqpoint{0.814990in}{0.713108in}}%
\pgfpathlineto{\pgfqpoint{0.837972in}{0.715177in}}%
\pgfpathlineto{\pgfqpoint{0.853294in}{0.716555in}}%
\pgfpathlineto{\pgfqpoint{0.859039in}{0.717527in}}%
\pgfpathlineto{\pgfqpoint{0.864785in}{0.718514in}}%
\pgfpathlineto{\pgfqpoint{0.889682in}{0.720470in}}%
\pgfpathlineto{\pgfqpoint{0.903088in}{0.721801in}}%
\pgfpathlineto{\pgfqpoint{0.920325in}{0.724066in}}%
\pgfpathlineto{\pgfqpoint{0.947137in}{0.725055in}}%
\pgfpathlineto{\pgfqpoint{1.010338in}{0.726970in}}%
\pgfpathlineto{\pgfqpoint{1.027574in}{0.727996in}}%
\pgfpathlineto{\pgfqpoint{1.067793in}{0.730388in}}%
\pgfpathlineto{\pgfqpoint{1.075453in}{0.730995in}}%
\pgfpathlineto{\pgfqpoint{1.299528in}{0.744595in}}%
\pgfpathlineto{\pgfqpoint{1.303359in}{0.745473in}}%
\pgfpathlineto{\pgfqpoint{1.328256in}{0.746903in}}%
\pgfpathlineto{\pgfqpoint{1.370390in}{0.750858in}}%
\pgfpathlineto{\pgfqpoint{1.385711in}{0.751900in}}%
\pgfpathlineto{\pgfqpoint{1.404863in}{0.753175in}}%
\pgfpathlineto{\pgfqpoint{1.416354in}{0.754161in}}%
\pgfpathlineto{\pgfqpoint{1.422099in}{0.755207in}}%
\pgfpathlineto{\pgfqpoint{1.433590in}{0.756474in}}%
\pgfpathlineto{\pgfqpoint{1.437421in}{0.757393in}}%
\pgfpathlineto{\pgfqpoint{1.452742in}{0.758832in}}%
\pgfpathlineto{\pgfqpoint{1.491046in}{0.765101in}}%
\pgfpathlineto{\pgfqpoint{1.498706in}{0.765845in}}%
\pgfpathlineto{\pgfqpoint{1.508282in}{0.766470in}}%
\pgfpathlineto{\pgfqpoint{1.519773in}{0.771921in}}%
\pgfpathlineto{\pgfqpoint{1.523604in}{0.772598in}}%
\pgfpathlineto{\pgfqpoint{1.531264in}{0.777376in}}%
\pgfpathlineto{\pgfqpoint{1.535095in}{0.778714in}}%
\pgfpathlineto{\pgfqpoint{1.537010in}{0.783849in}}%
\pgfpathlineto{\pgfqpoint{1.538925in}{0.784911in}}%
\pgfpathlineto{\pgfqpoint{1.540840in}{0.788811in}}%
\pgfpathlineto{\pgfqpoint{1.552331in}{0.791007in}}%
\pgfpathlineto{\pgfqpoint{1.554246in}{0.791602in}}%
\pgfpathlineto{\pgfqpoint{1.561907in}{0.798592in}}%
\pgfpathlineto{\pgfqpoint{1.569568in}{0.799587in}}%
\pgfpathlineto{\pgfqpoint{1.573398in}{0.802172in}}%
\pgfpathlineto{\pgfqpoint{1.577228in}{0.802534in}}%
\pgfpathlineto{\pgfqpoint{1.582974in}{0.806461in}}%
\pgfpathlineto{\pgfqpoint{1.590635in}{0.808069in}}%
\pgfpathlineto{\pgfqpoint{1.607871in}{0.810257in}}%
\pgfpathlineto{\pgfqpoint{1.613617in}{0.811082in}}%
\pgfpathlineto{\pgfqpoint{1.623192in}{0.811795in}}%
\pgfpathlineto{\pgfqpoint{1.636599in}{0.816468in}}%
\pgfpathlineto{\pgfqpoint{1.642344in}{0.816816in}}%
\pgfpathlineto{\pgfqpoint{1.646174in}{0.817868in}}%
\pgfpathlineto{\pgfqpoint{1.703630in}{0.828093in}}%
\pgfpathlineto{\pgfqpoint{1.715121in}{0.829443in}}%
\pgfpathlineto{\pgfqpoint{1.722781in}{0.831705in}}%
\pgfpathlineto{\pgfqpoint{1.730442in}{0.832697in}}%
\pgfpathlineto{\pgfqpoint{1.759170in}{0.835613in}}%
\pgfpathlineto{\pgfqpoint{1.763000in}{0.836460in}}%
\pgfpathlineto{\pgfqpoint{1.778321in}{0.838136in}}%
\pgfpathlineto{\pgfqpoint{1.782152in}{0.839452in}}%
\pgfpathlineto{\pgfqpoint{1.803219in}{0.843519in}}%
\pgfpathlineto{\pgfqpoint{1.818540in}{0.845975in}}%
\pgfpathlineto{\pgfqpoint{1.822370in}{0.846987in}}%
\pgfpathlineto{\pgfqpoint{1.826201in}{0.847835in}}%
\pgfpathlineto{\pgfqpoint{1.831946in}{0.849615in}}%
\pgfpathlineto{\pgfqpoint{1.835776in}{0.851432in}}%
\pgfpathlineto{\pgfqpoint{1.839607in}{0.852136in}}%
\pgfpathlineto{\pgfqpoint{1.843437in}{0.854187in}}%
\pgfpathlineto{\pgfqpoint{1.847267in}{0.855217in}}%
\pgfpathlineto{\pgfqpoint{1.864504in}{0.859580in}}%
\pgfpathlineto{\pgfqpoint{1.879825in}{0.860892in}}%
\pgfpathlineto{\pgfqpoint{1.891316in}{0.863940in}}%
\pgfpathlineto{\pgfqpoint{1.908553in}{0.865804in}}%
\pgfpathlineto{\pgfqpoint{1.944941in}{0.870873in}}%
\pgfpathlineto{\pgfqpoint{1.950687in}{0.872051in}}%
\pgfpathlineto{\pgfqpoint{2.002396in}{0.880407in}}%
\pgfpathlineto{\pgfqpoint{2.006227in}{0.881360in}}%
\pgfpathlineto{\pgfqpoint{2.013887in}{0.882640in}}%
\pgfpathlineto{\pgfqpoint{2.019633in}{0.882946in}}%
\pgfpathlineto{\pgfqpoint{2.023463in}{0.885084in}}%
\pgfpathlineto{\pgfqpoint{2.027294in}{0.885827in}}%
\pgfpathlineto{\pgfqpoint{2.031124in}{0.887468in}}%
\pgfpathlineto{\pgfqpoint{2.046445in}{0.889403in}}%
\pgfpathlineto{\pgfqpoint{2.113476in}{0.897402in}}%
\pgfpathlineto{\pgfqpoint{2.117307in}{0.898536in}}%
\pgfpathlineto{\pgfqpoint{2.130713in}{0.900570in}}%
\pgfpathlineto{\pgfqpoint{2.134543in}{0.901537in}}%
\pgfpathlineto{\pgfqpoint{2.142204in}{0.902500in}}%
\pgfpathlineto{\pgfqpoint{2.147949in}{0.903786in}}%
\pgfpathlineto{\pgfqpoint{2.193914in}{0.910651in}}%
\pgfpathlineto{\pgfqpoint{2.207320in}{0.911405in}}%
\pgfpathlineto{\pgfqpoint{2.213065in}{0.913407in}}%
\pgfpathlineto{\pgfqpoint{2.218811in}{0.914494in}}%
\pgfpathlineto{\pgfqpoint{2.222641in}{0.915106in}}%
\pgfpathlineto{\pgfqpoint{2.230302in}{0.917520in}}%
\pgfpathlineto{\pgfqpoint{2.239878in}{0.918703in}}%
\pgfpathlineto{\pgfqpoint{2.243708in}{0.920150in}}%
\pgfpathlineto{\pgfqpoint{2.264775in}{0.923191in}}%
\pgfpathlineto{\pgfqpoint{2.326060in}{0.934156in}}%
\pgfpathlineto{\pgfqpoint{2.329891in}{0.935857in}}%
\pgfpathlineto{\pgfqpoint{2.343297in}{0.937293in}}%
\pgfpathlineto{\pgfqpoint{2.350958in}{0.938427in}}%
\pgfpathlineto{\pgfqpoint{2.366279in}{0.939420in}}%
\pgfpathlineto{\pgfqpoint{2.372024in}{0.941202in}}%
\pgfpathlineto{\pgfqpoint{2.383515in}{0.942902in}}%
\pgfpathlineto{\pgfqpoint{2.410328in}{0.945045in}}%
\pgfpathlineto{\pgfqpoint{2.423734in}{0.945656in}}%
\pgfpathlineto{\pgfqpoint{2.429480in}{0.946929in}}%
\pgfpathlineto{\pgfqpoint{2.444801in}{0.947880in}}%
\pgfpathlineto{\pgfqpoint{2.467783in}{0.950815in}}%
\pgfpathlineto{\pgfqpoint{2.479274in}{0.954592in}}%
\pgfpathlineto{\pgfqpoint{2.511832in}{0.964688in}}%
\pgfpathlineto{\pgfqpoint{2.517577in}{0.970839in}}%
\pgfpathlineto{\pgfqpoint{2.519493in}{0.970888in}}%
\pgfpathlineto{\pgfqpoint{2.521408in}{0.972769in}}%
\pgfpathlineto{\pgfqpoint{2.527153in}{0.973870in}}%
\pgfpathlineto{\pgfqpoint{2.530984in}{0.975766in}}%
\pgfpathlineto{\pgfqpoint{2.534814in}{0.980023in}}%
\pgfpathlineto{\pgfqpoint{2.548220in}{0.982244in}}%
\pgfpathlineto{\pgfqpoint{2.555881in}{0.987072in}}%
\pgfpathlineto{\pgfqpoint{2.559711in}{0.987375in}}%
\pgfpathlineto{\pgfqpoint{2.561626in}{0.989794in}}%
\pgfpathlineto{\pgfqpoint{2.565457in}{0.990829in}}%
\pgfpathlineto{\pgfqpoint{2.575033in}{0.992834in}}%
\pgfpathlineto{\pgfqpoint{2.580778in}{0.995368in}}%
\pgfpathlineto{\pgfqpoint{2.584608in}{0.996058in}}%
\pgfpathlineto{\pgfqpoint{2.586524in}{0.999027in}}%
\pgfpathlineto{\pgfqpoint{2.592269in}{1.000545in}}%
\pgfpathlineto{\pgfqpoint{2.601845in}{1.001857in}}%
\pgfpathlineto{\pgfqpoint{2.607591in}{1.004198in}}%
\pgfpathlineto{\pgfqpoint{2.615251in}{1.006342in}}%
\pgfpathlineto{\pgfqpoint{2.617166in}{1.007769in}}%
\pgfpathlineto{\pgfqpoint{2.624827in}{1.008569in}}%
\pgfpathlineto{\pgfqpoint{2.647809in}{1.015743in}}%
\pgfpathlineto{\pgfqpoint{2.653555in}{1.017738in}}%
\pgfpathlineto{\pgfqpoint{2.661215in}{1.018503in}}%
\pgfpathlineto{\pgfqpoint{2.666961in}{1.021593in}}%
\pgfpathlineto{\pgfqpoint{2.668876in}{1.021639in}}%
\pgfpathlineto{\pgfqpoint{2.676537in}{1.029089in}}%
\pgfpathlineto{\pgfqpoint{2.682282in}{1.029692in}}%
\pgfpathlineto{\pgfqpoint{2.684197in}{1.032096in}}%
\pgfpathlineto{\pgfqpoint{2.688028in}{1.032551in}}%
\pgfpathlineto{\pgfqpoint{2.689943in}{1.036236in}}%
\pgfpathlineto{\pgfqpoint{2.693773in}{1.037453in}}%
\pgfpathlineto{\pgfqpoint{2.699519in}{1.041072in}}%
\pgfpathlineto{\pgfqpoint{2.705264in}{1.041844in}}%
\pgfpathlineto{\pgfqpoint{2.711010in}{1.046404in}}%
\pgfpathlineto{\pgfqpoint{2.712925in}{1.046713in}}%
\pgfpathlineto{\pgfqpoint{2.714840in}{1.048649in}}%
\pgfpathlineto{\pgfqpoint{2.718670in}{1.048973in}}%
\pgfpathlineto{\pgfqpoint{2.722501in}{1.051883in}}%
\pgfpathlineto{\pgfqpoint{2.730161in}{1.053131in}}%
\pgfpathlineto{\pgfqpoint{2.733992in}{1.058431in}}%
\pgfpathlineto{\pgfqpoint{2.737822in}{1.060002in}}%
\pgfpathlineto{\pgfqpoint{2.739737in}{1.062244in}}%
\pgfpathlineto{\pgfqpoint{2.745483in}{1.063362in}}%
\pgfpathlineto{\pgfqpoint{2.749313in}{1.066143in}}%
\pgfpathlineto{\pgfqpoint{2.756974in}{1.068559in}}%
\pgfpathlineto{\pgfqpoint{2.758889in}{1.070863in}}%
\pgfpathlineto{\pgfqpoint{2.760804in}{1.071197in}}%
\pgfpathlineto{\pgfqpoint{2.762719in}{1.073376in}}%
\pgfpathlineto{\pgfqpoint{2.764635in}{1.073698in}}%
\pgfpathlineto{\pgfqpoint{2.766550in}{1.076698in}}%
\pgfpathlineto{\pgfqpoint{2.781871in}{1.081531in}}%
\pgfpathlineto{\pgfqpoint{2.789532in}{1.091942in}}%
\pgfpathlineto{\pgfqpoint{2.802938in}{1.097733in}}%
\pgfpathlineto{\pgfqpoint{2.816344in}{1.113734in}}%
\pgfpathlineto{\pgfqpoint{2.820175in}{1.115461in}}%
\pgfpathlineto{\pgfqpoint{2.822090in}{1.118249in}}%
\pgfpathlineto{\pgfqpoint{2.824005in}{1.118961in}}%
\pgfpathlineto{\pgfqpoint{2.825920in}{1.121740in}}%
\pgfpathlineto{\pgfqpoint{2.827835in}{1.121876in}}%
\pgfpathlineto{\pgfqpoint{2.831666in}{1.128837in}}%
\pgfpathlineto{\pgfqpoint{2.841241in}{1.132505in}}%
\pgfpathlineto{\pgfqpoint{2.843157in}{1.135682in}}%
\pgfpathlineto{\pgfqpoint{2.845072in}{1.142341in}}%
\pgfpathlineto{\pgfqpoint{2.846987in}{1.143000in}}%
\pgfpathlineto{\pgfqpoint{2.848902in}{1.150761in}}%
\pgfpathlineto{\pgfqpoint{2.850817in}{1.152351in}}%
\pgfpathlineto{\pgfqpoint{2.852732in}{1.162566in}}%
\pgfpathlineto{\pgfqpoint{2.858478in}{1.165194in}}%
\pgfpathlineto{\pgfqpoint{2.860393in}{1.166312in}}%
\pgfpathlineto{\pgfqpoint{2.862308in}{1.176447in}}%
\pgfpathlineto{\pgfqpoint{2.875715in}{1.189905in}}%
\pgfpathlineto{\pgfqpoint{2.877630in}{1.200352in}}%
\pgfpathlineto{\pgfqpoint{2.879545in}{1.200960in}}%
\pgfpathlineto{\pgfqpoint{2.883375in}{1.214235in}}%
\pgfpathlineto{\pgfqpoint{2.885290in}{1.223561in}}%
\pgfpathlineto{\pgfqpoint{2.887206in}{1.223605in}}%
\pgfpathlineto{\pgfqpoint{2.892951in}{1.233278in}}%
\pgfpathlineto{\pgfqpoint{2.896781in}{1.235146in}}%
\pgfpathlineto{\pgfqpoint{2.898697in}{1.235612in}}%
\pgfpathlineto{\pgfqpoint{2.902527in}{1.240602in}}%
\pgfpathlineto{\pgfqpoint{2.908272in}{1.241978in}}%
\pgfpathlineto{\pgfqpoint{2.910188in}{1.245462in}}%
\pgfpathlineto{\pgfqpoint{2.914018in}{1.258053in}}%
\pgfpathlineto{\pgfqpoint{2.915933in}{1.258087in}}%
\pgfpathlineto{\pgfqpoint{2.917848in}{1.259950in}}%
\pgfpathlineto{\pgfqpoint{2.919763in}{1.265087in}}%
\pgfpathlineto{\pgfqpoint{2.929339in}{1.266600in}}%
\pgfpathlineto{\pgfqpoint{2.931254in}{1.269136in}}%
\pgfpathlineto{\pgfqpoint{2.933170in}{1.274322in}}%
\pgfpathlineto{\pgfqpoint{2.935085in}{1.275324in}}%
\pgfpathlineto{\pgfqpoint{2.937000in}{1.286711in}}%
\pgfpathlineto{\pgfqpoint{2.942746in}{1.296020in}}%
\pgfpathlineto{\pgfqpoint{2.950406in}{1.339754in}}%
\pgfpathlineto{\pgfqpoint{2.954237in}{1.353078in}}%
\pgfpathlineto{\pgfqpoint{2.956152in}{1.366320in}}%
\pgfpathlineto{\pgfqpoint{2.958067in}{1.367354in}}%
\pgfpathlineto{\pgfqpoint{2.961897in}{1.375775in}}%
\pgfpathlineto{\pgfqpoint{2.963812in}{1.401096in}}%
\pgfpathlineto{\pgfqpoint{2.965728in}{1.402566in}}%
\pgfpathlineto{\pgfqpoint{2.967643in}{1.422730in}}%
\pgfpathlineto{\pgfqpoint{2.969558in}{1.427997in}}%
\pgfpathlineto{\pgfqpoint{2.971473in}{1.438897in}}%
\pgfpathlineto{\pgfqpoint{2.973388in}{1.463050in}}%
\pgfpathlineto{\pgfqpoint{2.977219in}{1.472723in}}%
\pgfpathlineto{\pgfqpoint{2.979134in}{1.500516in}}%
\pgfpathlineto{\pgfqpoint{2.982964in}{1.515955in}}%
\pgfpathlineto{\pgfqpoint{2.994455in}{1.564788in}}%
\pgfpathlineto{\pgfqpoint{2.996370in}{1.566155in}}%
\pgfpathlineto{\pgfqpoint{2.998285in}{1.571164in}}%
\pgfpathlineto{\pgfqpoint{3.000201in}{1.572768in}}%
\pgfpathlineto{\pgfqpoint{3.002116in}{1.579641in}}%
\pgfpathlineto{\pgfqpoint{3.005946in}{1.602936in}}%
\pgfpathlineto{\pgfqpoint{3.007861in}{1.606766in}}%
\pgfpathlineto{\pgfqpoint{3.009777in}{1.616119in}}%
\pgfpathlineto{\pgfqpoint{3.011692in}{1.616285in}}%
\pgfpathlineto{\pgfqpoint{3.013607in}{1.623297in}}%
\pgfpathlineto{\pgfqpoint{3.017437in}{1.627683in}}%
\pgfpathlineto{\pgfqpoint{3.019352in}{1.627787in}}%
\pgfpathlineto{\pgfqpoint{3.025098in}{1.654026in}}%
\pgfpathlineto{\pgfqpoint{3.027013in}{1.673286in}}%
\pgfpathlineto{\pgfqpoint{3.028928in}{1.679975in}}%
\pgfpathlineto{\pgfqpoint{3.032759in}{1.703650in}}%
\pgfpathlineto{\pgfqpoint{3.034674in}{1.709293in}}%
\pgfpathlineto{\pgfqpoint{3.036589in}{1.709480in}}%
\pgfpathlineto{\pgfqpoint{3.044250in}{1.716044in}}%
\pgfpathlineto{\pgfqpoint{3.046165in}{1.741674in}}%
\pgfpathlineto{\pgfqpoint{3.048080in}{1.742266in}}%
\pgfpathlineto{\pgfqpoint{3.051910in}{1.791670in}}%
\pgfpathlineto{\pgfqpoint{3.053825in}{1.826535in}}%
\pgfpathlineto{\pgfqpoint{3.053825in}{1.826535in}}%
\pgfusepath{stroke}%
\end{pgfscope}%
\begin{pgfscope}%
\pgfsetrectcap%
\pgfsetmiterjoin%
\pgfsetlinewidth{0.803000pt}%
\definecolor{currentstroke}{rgb}{0.000000,0.000000,0.000000}%
\pgfsetstrokecolor{currentstroke}%
\pgfsetdash{}{0pt}%
\pgfpathmoveto{\pgfqpoint{0.694334in}{0.523557in}}%
\pgfpathlineto{\pgfqpoint{0.694334in}{1.826535in}}%
\pgfusepath{stroke}%
\end{pgfscope}%
\begin{pgfscope}%
\pgfsetrectcap%
\pgfsetmiterjoin%
\pgfsetlinewidth{0.803000pt}%
\definecolor{currentstroke}{rgb}{0.000000,0.000000,0.000000}%
\pgfsetstrokecolor{currentstroke}%
\pgfsetdash{}{0pt}%
\pgfpathmoveto{\pgfqpoint{4.524677in}{0.523557in}}%
\pgfpathlineto{\pgfqpoint{4.524677in}{1.826535in}}%
\pgfusepath{stroke}%
\end{pgfscope}%
\begin{pgfscope}%
\pgfsetrectcap%
\pgfsetmiterjoin%
\pgfsetlinewidth{0.803000pt}%
\definecolor{currentstroke}{rgb}{0.000000,0.000000,0.000000}%
\pgfsetstrokecolor{currentstroke}%
\pgfsetdash{}{0pt}%
\pgfpathmoveto{\pgfqpoint{0.694334in}{0.523557in}}%
\pgfpathlineto{\pgfqpoint{4.524677in}{0.523557in}}%
\pgfusepath{stroke}%
\end{pgfscope}%
\begin{pgfscope}%
\pgfsetrectcap%
\pgfsetmiterjoin%
\pgfsetlinewidth{0.803000pt}%
\definecolor{currentstroke}{rgb}{0.000000,0.000000,0.000000}%
\pgfsetstrokecolor{currentstroke}%
\pgfsetdash{}{0pt}%
\pgfpathmoveto{\pgfqpoint{0.694334in}{1.826535in}}%
\pgfpathlineto{\pgfqpoint{4.524677in}{1.826535in}}%
\pgfusepath{stroke}%
\end{pgfscope}%
\begin{pgfscope}%
\pgfsetrectcap%
\pgfsetroundjoin%
\pgfsetlinewidth{1.003750pt}%
\definecolor{currentstroke}{rgb}{0.878431,0.878431,0.815686}%
\pgfsetstrokecolor{currentstroke}%
\pgfsetdash{}{0pt}%
\pgfpathmoveto{\pgfqpoint{3.867012in}{1.491422in}}%
\pgfpathlineto{\pgfqpoint{4.089235in}{1.491422in}}%
\pgfusepath{stroke}%
\end{pgfscope}%
\begin{pgfscope}%
\definecolor{textcolor}{rgb}{0.000000,0.000000,0.000000}%
\pgfsetstrokecolor{textcolor}%
\pgfsetfillcolor{textcolor}%
\pgftext[x=4.111457in,y=1.452533in,left,base]{\color{textcolor}\rmfamily\fontsize{8.000000}{9.600000}\selectfont T.}%
\end{pgfscope}%
\begin{pgfscope}%
\pgfsetbuttcap%
\pgfsetroundjoin%
\pgfsetlinewidth{1.003750pt}%
\definecolor{currentstroke}{rgb}{0.941176,0.627451,0.188235}%
\pgfsetstrokecolor{currentstroke}%
\pgfsetdash{{1.000000pt}{1.650000pt}}{0.000000pt}%
\pgfpathmoveto{\pgfqpoint{3.867012in}{1.347600in}}%
\pgfpathlineto{\pgfqpoint{4.089235in}{1.347600in}}%
\pgfusepath{stroke}%
\end{pgfscope}%
\begin{pgfscope}%
\definecolor{textcolor}{rgb}{0.000000,0.000000,0.000000}%
\pgfsetstrokecolor{textcolor}%
\pgfsetfillcolor{textcolor}%
\pgftext[x=4.111457in,y=1.308711in,left,base]{\color{textcolor}\rmfamily\fontsize{8.000000}{9.600000}\selectfont FlowC.}%
\end{pgfscope}%
\begin{pgfscope}%
\pgfsetbuttcap%
\pgfsetroundjoin%
\pgfsetlinewidth{1.003750pt}%
\definecolor{currentstroke}{rgb}{0.062745,0.000000,0.062745}%
\pgfsetstrokecolor{currentstroke}%
\pgfsetdash{{3.700000pt}{1.600000pt}}{0.000000pt}%
\pgfpathmoveto{\pgfqpoint{3.867012in}{1.203778in}}%
\pgfpathlineto{\pgfqpoint{4.089235in}{1.203778in}}%
\pgfusepath{stroke}%
\end{pgfscope}%
\begin{pgfscope}%
\definecolor{textcolor}{rgb}{0.000000,0.000000,0.000000}%
\pgfsetstrokecolor{textcolor}%
\pgfsetfillcolor{textcolor}%
\pgftext[x=4.111457in,y=1.164889in,left,base]{\color{textcolor}\rmfamily\fontsize{8.000000}{9.600000}\selectfont htd}%
\end{pgfscope}%
\begin{pgfscope}%
\pgfsetbuttcap%
\pgfsetroundjoin%
\pgfsetlinewidth{1.003750pt}%
\definecolor{currentstroke}{rgb}{0.811765,0.125490,0.125490}%
\pgfsetstrokecolor{currentstroke}%
\pgfsetdash{{1.000000pt}{1.650000pt}}{0.000000pt}%
\pgfpathmoveto{\pgfqpoint{3.867012in}{1.059956in}}%
\pgfpathlineto{\pgfqpoint{4.089235in}{1.059956in}}%
\pgfusepath{stroke}%
\end{pgfscope}%
\begin{pgfscope}%
\definecolor{textcolor}{rgb}{0.000000,0.000000,0.000000}%
\pgfsetstrokecolor{textcolor}%
\pgfsetfillcolor{textcolor}%
\pgftext[x=4.111457in,y=1.021067in,left,base]{\color{textcolor}\rmfamily\fontsize{8.000000}{9.600000}\selectfont Hicks}%
\end{pgfscope}%
\begin{pgfscope}%
\pgfsetrectcap%
\pgfsetroundjoin%
\pgfsetlinewidth{1.003750pt}%
\definecolor{currentstroke}{rgb}{0.000000,0.000000,0.376471}%
\pgfsetstrokecolor{currentstroke}%
\pgfsetdash{}{0pt}%
\pgfpathmoveto{\pgfqpoint{3.867012in}{0.916134in}}%
\pgfpathlineto{\pgfqpoint{4.089235in}{0.916134in}}%
\pgfusepath{stroke}%
\end{pgfscope}%
\begin{pgfscope}%
\definecolor{textcolor}{rgb}{0.000000,0.000000,0.000000}%
\pgfsetstrokecolor{textcolor}%
\pgfsetfillcolor{textcolor}%
\pgftext[x=4.111457in,y=0.877245in,left,base]{\color{textcolor}\rmfamily\fontsize{8.000000}{9.600000}\selectfont P3}%
\end{pgfscope}%
\begin{pgfscope}%
\pgfsetrectcap%
\pgfsetroundjoin%
\pgfsetlinewidth{1.003750pt}%
\definecolor{currentstroke}{rgb}{0.564706,0.564706,1.000000}%
\pgfsetstrokecolor{currentstroke}%
\pgfsetdash{}{0pt}%
\pgfpathmoveto{\pgfqpoint{3.867012in}{0.772312in}}%
\pgfpathlineto{\pgfqpoint{4.089235in}{0.772312in}}%
\pgfusepath{stroke}%
\end{pgfscope}%
\begin{pgfscope}%
\definecolor{textcolor}{rgb}{0.000000,0.000000,0.000000}%
\pgfsetstrokecolor{textcolor}%
\pgfsetfillcolor{textcolor}%
\pgftext[x=4.111457in,y=0.733423in,left,base]{\color{textcolor}\rmfamily\fontsize{8.000000}{9.600000}\selectfont P4}%
\end{pgfscope}%
\begin{pgfscope}%
\pgfsetbuttcap%
\pgfsetroundjoin%
\pgfsetlinewidth{1.003750pt}%
\definecolor{currentstroke}{rgb}{0.000000,0.000000,0.000000}%
\pgfsetstrokecolor{currentstroke}%
\pgfsetdash{{1.000000pt}{1.650000pt}}{0.000000pt}%
\pgfpathmoveto{\pgfqpoint{3.867012in}{0.628490in}}%
\pgfpathlineto{\pgfqpoint{4.089235in}{0.628490in}}%
\pgfusepath{stroke}%
\end{pgfscope}%
\begin{pgfscope}%
\definecolor{textcolor}{rgb}{0.000000,0.000000,0.000000}%
\pgfsetstrokecolor{textcolor}%
\pgfsetfillcolor{textcolor}%
\pgftext[x=4.111457in,y=0.589601in,left,base]{\color{textcolor}\rmfamily\fontsize{8.000000}{9.600000}\selectfont VBS}%
\end{pgfscope}%
\end{pgfpicture}%
\makeatother%
\endgroup%

    \vspace*{-0.9cm}
	\caption{\label{fig:parallel:planning} A cactus plot of the performance of various planners. A planner ``solves'' a benchmark when it finds a contraction tree of max rank 30 or smaller.}
\end{figure}

\subsection{Experiment 1: The Planning Phase (RQ1 and RQ2)}
We run each planning implementation (\pkg{FlowCutter}, \pkg{htd}, \pkg{Tamaki}, \pkg{Hicks}, \pkg{P3}, and \pkg{P4}) once on each of our 1914 benchmarks and save all contraction trees found within 1000 seconds (without executing the contractions). Results are summarized in Figure \ref{fig:parallel:planning}. 

% In this figure, we consider a benchmark to be solved by a planning implementation when the planner is able to find contraction tree whose max-rank is 30 or smaller.

%\begin{figure}[t]
%\begin{center}
%%% Creator: Matplotlib, PGF backend
%%
%% To include the figure in your LaTeX document, write
%%   \input{<filename>.pgf}
%%
%% Make sure the required packages are loaded in your preamble
%%   \usepackage{pgf}
%%
%% and, on pdftex
%%   \usepackage[utf8]{inputenc}\DeclareUnicodeCharacter{2212}{-}
%%
%% or, on luatex and xetex
%%   \usepackage{unicode-math}
%%
%% Figures using additional raster images can only be included by \input if
%% they are in the same directory as the main LaTeX file. For loading figures
%% from other directories you can use the `import` package
%%   \usepackage{import}
%%
%% and then include the figures with
%%   \import{<path to file>}{<filename>.pgf}
%%
%% Matplotlib used the following preamble
%%   \usepackage[utf8x]{inputenc}
%%   \usepackage[T1]{fontenc}
%%
\begingroup%
\makeatletter%
\begin{pgfpicture}%
\pgfpathrectangle{\pgfpointorigin}{\pgfqpoint{4.803148in}{2.021259in}}%
\pgfusepath{use as bounding box, clip}%
\begin{pgfscope}%
\pgfsetbuttcap%
\pgfsetmiterjoin%
\definecolor{currentfill}{rgb}{1.000000,1.000000,1.000000}%
\pgfsetfillcolor{currentfill}%
\pgfsetlinewidth{0.000000pt}%
\definecolor{currentstroke}{rgb}{1.000000,1.000000,1.000000}%
\pgfsetstrokecolor{currentstroke}%
\pgfsetdash{}{0pt}%
\pgfpathmoveto{\pgfqpoint{0.000000in}{0.000000in}}%
\pgfpathlineto{\pgfqpoint{4.803148in}{0.000000in}}%
\pgfpathlineto{\pgfqpoint{4.803148in}{2.021259in}}%
\pgfpathlineto{\pgfqpoint{0.000000in}{2.021259in}}%
\pgfpathclose%
\pgfusepath{fill}%
\end{pgfscope}%
\begin{pgfscope}%
\pgfsetbuttcap%
\pgfsetmiterjoin%
\definecolor{currentfill}{rgb}{1.000000,1.000000,1.000000}%
\pgfsetfillcolor{currentfill}%
\pgfsetlinewidth{0.000000pt}%
\definecolor{currentstroke}{rgb}{0.000000,0.000000,0.000000}%
\pgfsetstrokecolor{currentstroke}%
\pgfsetstrokeopacity{0.000000}%
\pgfsetdash{}{0pt}%
\pgfpathmoveto{\pgfqpoint{0.694334in}{0.523557in}}%
\pgfpathlineto{\pgfqpoint{4.524677in}{0.523557in}}%
\pgfpathlineto{\pgfqpoint{4.524677in}{1.826535in}}%
\pgfpathlineto{\pgfqpoint{0.694334in}{1.826535in}}%
\pgfpathclose%
\pgfusepath{fill}%
\end{pgfscope}%
\begin{pgfscope}%
\pgfsetbuttcap%
\pgfsetroundjoin%
\definecolor{currentfill}{rgb}{0.000000,0.000000,0.000000}%
\pgfsetfillcolor{currentfill}%
\pgfsetlinewidth{0.803000pt}%
\definecolor{currentstroke}{rgb}{0.000000,0.000000,0.000000}%
\pgfsetstrokecolor{currentstroke}%
\pgfsetdash{}{0pt}%
\pgfsys@defobject{currentmarker}{\pgfqpoint{0.000000in}{-0.048611in}}{\pgfqpoint{0.000000in}{0.000000in}}{%
\pgfpathmoveto{\pgfqpoint{0.000000in}{0.000000in}}%
\pgfpathlineto{\pgfqpoint{0.000000in}{-0.048611in}}%
\pgfusepath{stroke,fill}%
}%
\begin{pgfscope}%
\pgfsys@transformshift{0.694334in}{0.523557in}%
\pgfsys@useobject{currentmarker}{}%
\end{pgfscope}%
\end{pgfscope}%
\begin{pgfscope}%
\definecolor{textcolor}{rgb}{0.000000,0.000000,0.000000}%
\pgfsetstrokecolor{textcolor}%
\pgfsetfillcolor{textcolor}%
\pgftext[x=0.694334in,y=0.426335in,,top]{\color{textcolor}\rmfamily\fontsize{9.000000}{10.800000}\selectfont \(\displaystyle 0\)}%
\end{pgfscope}%
\begin{pgfscope}%
\pgfsetbuttcap%
\pgfsetroundjoin%
\definecolor{currentfill}{rgb}{0.000000,0.000000,0.000000}%
\pgfsetfillcolor{currentfill}%
\pgfsetlinewidth{0.803000pt}%
\definecolor{currentstroke}{rgb}{0.000000,0.000000,0.000000}%
\pgfsetstrokecolor{currentstroke}%
\pgfsetdash{}{0pt}%
\pgfsys@defobject{currentmarker}{\pgfqpoint{0.000000in}{-0.048611in}}{\pgfqpoint{0.000000in}{0.000000in}}{%
\pgfpathmoveto{\pgfqpoint{0.000000in}{0.000000in}}%
\pgfpathlineto{\pgfqpoint{0.000000in}{-0.048611in}}%
\pgfusepath{stroke,fill}%
}%
\begin{pgfscope}%
\pgfsys@transformshift{1.173127in}{0.523557in}%
\pgfsys@useobject{currentmarker}{}%
\end{pgfscope}%
\end{pgfscope}%
\begin{pgfscope}%
\definecolor{textcolor}{rgb}{0.000000,0.000000,0.000000}%
\pgfsetstrokecolor{textcolor}%
\pgfsetfillcolor{textcolor}%
\pgftext[x=1.173127in,y=0.426335in,,top]{\color{textcolor}\rmfamily\fontsize{9.000000}{10.800000}\selectfont \(\displaystyle 250\)}%
\end{pgfscope}%
\begin{pgfscope}%
\pgfsetbuttcap%
\pgfsetroundjoin%
\definecolor{currentfill}{rgb}{0.000000,0.000000,0.000000}%
\pgfsetfillcolor{currentfill}%
\pgfsetlinewidth{0.803000pt}%
\definecolor{currentstroke}{rgb}{0.000000,0.000000,0.000000}%
\pgfsetstrokecolor{currentstroke}%
\pgfsetdash{}{0pt}%
\pgfsys@defobject{currentmarker}{\pgfqpoint{0.000000in}{-0.048611in}}{\pgfqpoint{0.000000in}{0.000000in}}{%
\pgfpathmoveto{\pgfqpoint{0.000000in}{0.000000in}}%
\pgfpathlineto{\pgfqpoint{0.000000in}{-0.048611in}}%
\pgfusepath{stroke,fill}%
}%
\begin{pgfscope}%
\pgfsys@transformshift{1.651920in}{0.523557in}%
\pgfsys@useobject{currentmarker}{}%
\end{pgfscope}%
\end{pgfscope}%
\begin{pgfscope}%
\definecolor{textcolor}{rgb}{0.000000,0.000000,0.000000}%
\pgfsetstrokecolor{textcolor}%
\pgfsetfillcolor{textcolor}%
\pgftext[x=1.651920in,y=0.426335in,,top]{\color{textcolor}\rmfamily\fontsize{9.000000}{10.800000}\selectfont \(\displaystyle 500\)}%
\end{pgfscope}%
\begin{pgfscope}%
\pgfsetbuttcap%
\pgfsetroundjoin%
\definecolor{currentfill}{rgb}{0.000000,0.000000,0.000000}%
\pgfsetfillcolor{currentfill}%
\pgfsetlinewidth{0.803000pt}%
\definecolor{currentstroke}{rgb}{0.000000,0.000000,0.000000}%
\pgfsetstrokecolor{currentstroke}%
\pgfsetdash{}{0pt}%
\pgfsys@defobject{currentmarker}{\pgfqpoint{0.000000in}{-0.048611in}}{\pgfqpoint{0.000000in}{0.000000in}}{%
\pgfpathmoveto{\pgfqpoint{0.000000in}{0.000000in}}%
\pgfpathlineto{\pgfqpoint{0.000000in}{-0.048611in}}%
\pgfusepath{stroke,fill}%
}%
\begin{pgfscope}%
\pgfsys@transformshift{2.130713in}{0.523557in}%
\pgfsys@useobject{currentmarker}{}%
\end{pgfscope}%
\end{pgfscope}%
\begin{pgfscope}%
\definecolor{textcolor}{rgb}{0.000000,0.000000,0.000000}%
\pgfsetstrokecolor{textcolor}%
\pgfsetfillcolor{textcolor}%
\pgftext[x=2.130713in,y=0.426335in,,top]{\color{textcolor}\rmfamily\fontsize{9.000000}{10.800000}\selectfont \(\displaystyle 750\)}%
\end{pgfscope}%
\begin{pgfscope}%
\pgfsetbuttcap%
\pgfsetroundjoin%
\definecolor{currentfill}{rgb}{0.000000,0.000000,0.000000}%
\pgfsetfillcolor{currentfill}%
\pgfsetlinewidth{0.803000pt}%
\definecolor{currentstroke}{rgb}{0.000000,0.000000,0.000000}%
\pgfsetstrokecolor{currentstroke}%
\pgfsetdash{}{0pt}%
\pgfsys@defobject{currentmarker}{\pgfqpoint{0.000000in}{-0.048611in}}{\pgfqpoint{0.000000in}{0.000000in}}{%
\pgfpathmoveto{\pgfqpoint{0.000000in}{0.000000in}}%
\pgfpathlineto{\pgfqpoint{0.000000in}{-0.048611in}}%
\pgfusepath{stroke,fill}%
}%
\begin{pgfscope}%
\pgfsys@transformshift{2.609506in}{0.523557in}%
\pgfsys@useobject{currentmarker}{}%
\end{pgfscope}%
\end{pgfscope}%
\begin{pgfscope}%
\definecolor{textcolor}{rgb}{0.000000,0.000000,0.000000}%
\pgfsetstrokecolor{textcolor}%
\pgfsetfillcolor{textcolor}%
\pgftext[x=2.609506in,y=0.426335in,,top]{\color{textcolor}\rmfamily\fontsize{9.000000}{10.800000}\selectfont \(\displaystyle 1000\)}%
\end{pgfscope}%
\begin{pgfscope}%
\pgfsetbuttcap%
\pgfsetroundjoin%
\definecolor{currentfill}{rgb}{0.000000,0.000000,0.000000}%
\pgfsetfillcolor{currentfill}%
\pgfsetlinewidth{0.803000pt}%
\definecolor{currentstroke}{rgb}{0.000000,0.000000,0.000000}%
\pgfsetstrokecolor{currentstroke}%
\pgfsetdash{}{0pt}%
\pgfsys@defobject{currentmarker}{\pgfqpoint{0.000000in}{-0.048611in}}{\pgfqpoint{0.000000in}{0.000000in}}{%
\pgfpathmoveto{\pgfqpoint{0.000000in}{0.000000in}}%
\pgfpathlineto{\pgfqpoint{0.000000in}{-0.048611in}}%
\pgfusepath{stroke,fill}%
}%
\begin{pgfscope}%
\pgfsys@transformshift{3.088299in}{0.523557in}%
\pgfsys@useobject{currentmarker}{}%
\end{pgfscope}%
\end{pgfscope}%
\begin{pgfscope}%
\definecolor{textcolor}{rgb}{0.000000,0.000000,0.000000}%
\pgfsetstrokecolor{textcolor}%
\pgfsetfillcolor{textcolor}%
\pgftext[x=3.088299in,y=0.426335in,,top]{\color{textcolor}\rmfamily\fontsize{9.000000}{10.800000}\selectfont \(\displaystyle 1250\)}%
\end{pgfscope}%
\begin{pgfscope}%
\pgfsetbuttcap%
\pgfsetroundjoin%
\definecolor{currentfill}{rgb}{0.000000,0.000000,0.000000}%
\pgfsetfillcolor{currentfill}%
\pgfsetlinewidth{0.803000pt}%
\definecolor{currentstroke}{rgb}{0.000000,0.000000,0.000000}%
\pgfsetstrokecolor{currentstroke}%
\pgfsetdash{}{0pt}%
\pgfsys@defobject{currentmarker}{\pgfqpoint{0.000000in}{-0.048611in}}{\pgfqpoint{0.000000in}{0.000000in}}{%
\pgfpathmoveto{\pgfqpoint{0.000000in}{0.000000in}}%
\pgfpathlineto{\pgfqpoint{0.000000in}{-0.048611in}}%
\pgfusepath{stroke,fill}%
}%
\begin{pgfscope}%
\pgfsys@transformshift{3.567091in}{0.523557in}%
\pgfsys@useobject{currentmarker}{}%
\end{pgfscope}%
\end{pgfscope}%
\begin{pgfscope}%
\definecolor{textcolor}{rgb}{0.000000,0.000000,0.000000}%
\pgfsetstrokecolor{textcolor}%
\pgfsetfillcolor{textcolor}%
\pgftext[x=3.567091in,y=0.426335in,,top]{\color{textcolor}\rmfamily\fontsize{9.000000}{10.800000}\selectfont \(\displaystyle 1500\)}%
\end{pgfscope}%
\begin{pgfscope}%
\pgfsetbuttcap%
\pgfsetroundjoin%
\definecolor{currentfill}{rgb}{0.000000,0.000000,0.000000}%
\pgfsetfillcolor{currentfill}%
\pgfsetlinewidth{0.803000pt}%
\definecolor{currentstroke}{rgb}{0.000000,0.000000,0.000000}%
\pgfsetstrokecolor{currentstroke}%
\pgfsetdash{}{0pt}%
\pgfsys@defobject{currentmarker}{\pgfqpoint{0.000000in}{-0.048611in}}{\pgfqpoint{0.000000in}{0.000000in}}{%
\pgfpathmoveto{\pgfqpoint{0.000000in}{0.000000in}}%
\pgfpathlineto{\pgfqpoint{0.000000in}{-0.048611in}}%
\pgfusepath{stroke,fill}%
}%
\begin{pgfscope}%
\pgfsys@transformshift{4.045884in}{0.523557in}%
\pgfsys@useobject{currentmarker}{}%
\end{pgfscope}%
\end{pgfscope}%
\begin{pgfscope}%
\definecolor{textcolor}{rgb}{0.000000,0.000000,0.000000}%
\pgfsetstrokecolor{textcolor}%
\pgfsetfillcolor{textcolor}%
\pgftext[x=4.045884in,y=0.426335in,,top]{\color{textcolor}\rmfamily\fontsize{9.000000}{10.800000}\selectfont \(\displaystyle 1750\)}%
\end{pgfscope}%
\begin{pgfscope}%
\pgfsetbuttcap%
\pgfsetroundjoin%
\definecolor{currentfill}{rgb}{0.000000,0.000000,0.000000}%
\pgfsetfillcolor{currentfill}%
\pgfsetlinewidth{0.803000pt}%
\definecolor{currentstroke}{rgb}{0.000000,0.000000,0.000000}%
\pgfsetstrokecolor{currentstroke}%
\pgfsetdash{}{0pt}%
\pgfsys@defobject{currentmarker}{\pgfqpoint{0.000000in}{-0.048611in}}{\pgfqpoint{0.000000in}{0.000000in}}{%
\pgfpathmoveto{\pgfqpoint{0.000000in}{0.000000in}}%
\pgfpathlineto{\pgfqpoint{0.000000in}{-0.048611in}}%
\pgfusepath{stroke,fill}%
}%
\begin{pgfscope}%
\pgfsys@transformshift{4.524677in}{0.523557in}%
\pgfsys@useobject{currentmarker}{}%
\end{pgfscope}%
\end{pgfscope}%
\begin{pgfscope}%
\definecolor{textcolor}{rgb}{0.000000,0.000000,0.000000}%
\pgfsetstrokecolor{textcolor}%
\pgfsetfillcolor{textcolor}%
\pgftext[x=4.524677in,y=0.426335in,,top]{\color{textcolor}\rmfamily\fontsize{9.000000}{10.800000}\selectfont \(\displaystyle 2000\)}%
\end{pgfscope}%
\begin{pgfscope}%
\definecolor{textcolor}{rgb}{0.000000,0.000000,0.000000}%
\pgfsetstrokecolor{textcolor}%
\pgfsetfillcolor{textcolor}%
\pgftext[x=2.609506in,y=0.260390in,,top]{\color{textcolor}\rmfamily\fontsize{9.000000}{10.800000}\selectfont Number of benchmarks solved}%
\end{pgfscope}%
\begin{pgfscope}%
\pgfsetbuttcap%
\pgfsetroundjoin%
\definecolor{currentfill}{rgb}{0.000000,0.000000,0.000000}%
\pgfsetfillcolor{currentfill}%
\pgfsetlinewidth{0.803000pt}%
\definecolor{currentstroke}{rgb}{0.000000,0.000000,0.000000}%
\pgfsetstrokecolor{currentstroke}%
\pgfsetdash{}{0pt}%
\pgfsys@defobject{currentmarker}{\pgfqpoint{-0.048611in}{0.000000in}}{\pgfqpoint{0.000000in}{0.000000in}}{%
\pgfpathmoveto{\pgfqpoint{0.000000in}{0.000000in}}%
\pgfpathlineto{\pgfqpoint{-0.048611in}{0.000000in}}%
\pgfusepath{stroke,fill}%
}%
\begin{pgfscope}%
\pgfsys@transformshift{0.694334in}{0.843347in}%
\pgfsys@useobject{currentmarker}{}%
\end{pgfscope}%
\end{pgfscope}%
\begin{pgfscope}%
\definecolor{textcolor}{rgb}{0.000000,0.000000,0.000000}%
\pgfsetstrokecolor{textcolor}%
\pgfsetfillcolor{textcolor}%
\pgftext[x=0.330525in, y=0.798622in, left, base]{\color{textcolor}\rmfamily\fontsize{9.000000}{10.800000}\selectfont \(\displaystyle 10^{-1}\)}%
\end{pgfscope}%
\begin{pgfscope}%
\pgfsetbuttcap%
\pgfsetroundjoin%
\definecolor{currentfill}{rgb}{0.000000,0.000000,0.000000}%
\pgfsetfillcolor{currentfill}%
\pgfsetlinewidth{0.803000pt}%
\definecolor{currentstroke}{rgb}{0.000000,0.000000,0.000000}%
\pgfsetstrokecolor{currentstroke}%
\pgfsetdash{}{0pt}%
\pgfsys@defobject{currentmarker}{\pgfqpoint{-0.048611in}{0.000000in}}{\pgfqpoint{0.000000in}{0.000000in}}{%
\pgfpathmoveto{\pgfqpoint{0.000000in}{0.000000in}}%
\pgfpathlineto{\pgfqpoint{-0.048611in}{0.000000in}}%
\pgfusepath{stroke,fill}%
}%
\begin{pgfscope}%
\pgfsys@transformshift{0.694334in}{1.334941in}%
\pgfsys@useobject{currentmarker}{}%
\end{pgfscope}%
\end{pgfscope}%
\begin{pgfscope}%
\definecolor{textcolor}{rgb}{0.000000,0.000000,0.000000}%
\pgfsetstrokecolor{textcolor}%
\pgfsetfillcolor{textcolor}%
\pgftext[x=0.410771in, y=1.290216in, left, base]{\color{textcolor}\rmfamily\fontsize{9.000000}{10.800000}\selectfont \(\displaystyle 10^{1}\)}%
\end{pgfscope}%
\begin{pgfscope}%
\pgfsetbuttcap%
\pgfsetroundjoin%
\definecolor{currentfill}{rgb}{0.000000,0.000000,0.000000}%
\pgfsetfillcolor{currentfill}%
\pgfsetlinewidth{0.803000pt}%
\definecolor{currentstroke}{rgb}{0.000000,0.000000,0.000000}%
\pgfsetstrokecolor{currentstroke}%
\pgfsetdash{}{0pt}%
\pgfsys@defobject{currentmarker}{\pgfqpoint{-0.048611in}{0.000000in}}{\pgfqpoint{0.000000in}{0.000000in}}{%
\pgfpathmoveto{\pgfqpoint{0.000000in}{0.000000in}}%
\pgfpathlineto{\pgfqpoint{-0.048611in}{0.000000in}}%
\pgfusepath{stroke,fill}%
}%
\begin{pgfscope}%
\pgfsys@transformshift{0.694334in}{1.826535in}%
\pgfsys@useobject{currentmarker}{}%
\end{pgfscope}%
\end{pgfscope}%
\begin{pgfscope}%
\definecolor{textcolor}{rgb}{0.000000,0.000000,0.000000}%
\pgfsetstrokecolor{textcolor}%
\pgfsetfillcolor{textcolor}%
\pgftext[x=0.410771in, y=1.781810in, left, base]{\color{textcolor}\rmfamily\fontsize{9.000000}{10.800000}\selectfont \(\displaystyle 10^{3}\)}%
\end{pgfscope}%
\begin{pgfscope}%
\definecolor{textcolor}{rgb}{0.000000,0.000000,0.000000}%
\pgfsetstrokecolor{textcolor}%
\pgfsetfillcolor{textcolor}%
\pgftext[x=0.274969in,y=1.175046in,,bottom,rotate=90.000000]{\color{textcolor}\rmfamily\fontsize{9.000000}{10.800000}\selectfont Longest solving time (s)}%
\end{pgfscope}%
\begin{pgfscope}%
\pgfpathrectangle{\pgfqpoint{0.694334in}{0.523557in}}{\pgfqpoint{3.830343in}{1.302977in}}%
\pgfusepath{clip}%
\pgfsetrectcap%
\pgfsetroundjoin%
\pgfsetlinewidth{1.003750pt}%
\definecolor{currentstroke}{rgb}{0.878431,0.878431,0.815686}%
\pgfsetstrokecolor{currentstroke}%
\pgfsetdash{}{0pt}%
\pgfpathmoveto{\pgfqpoint{0.694334in}{0.948144in}}%
\pgfpathlineto{\pgfqpoint{0.696249in}{0.955934in}}%
\pgfpathlineto{\pgfqpoint{0.701995in}{0.961880in}}%
\pgfpathlineto{\pgfqpoint{0.703910in}{0.964864in}}%
\pgfpathlineto{\pgfqpoint{0.705825in}{0.965524in}}%
\pgfpathlineto{\pgfqpoint{0.709656in}{0.973680in}}%
\pgfpathlineto{\pgfqpoint{0.711571in}{0.974495in}}%
\pgfpathlineto{\pgfqpoint{0.713486in}{0.976961in}}%
\pgfpathlineto{\pgfqpoint{0.719232in}{0.978816in}}%
\pgfpathlineto{\pgfqpoint{0.721147in}{0.980178in}}%
\pgfpathlineto{\pgfqpoint{0.724977in}{1.000991in}}%
\pgfpathlineto{\pgfqpoint{0.728807in}{1.015379in}}%
\pgfpathlineto{\pgfqpoint{0.730723in}{1.015482in}}%
\pgfpathlineto{\pgfqpoint{0.732638in}{1.022216in}}%
\pgfpathlineto{\pgfqpoint{0.734553in}{1.022392in}}%
\pgfpathlineto{\pgfqpoint{0.736468in}{1.024621in}}%
\pgfpathlineto{\pgfqpoint{0.738383in}{1.024629in}}%
\pgfpathlineto{\pgfqpoint{0.742214in}{1.030245in}}%
\pgfpathlineto{\pgfqpoint{0.744129in}{1.030409in}}%
\pgfpathlineto{\pgfqpoint{0.746044in}{1.033013in}}%
\pgfpathlineto{\pgfqpoint{0.747959in}{1.039599in}}%
\pgfpathlineto{\pgfqpoint{0.749874in}{1.040605in}}%
\pgfpathlineto{\pgfqpoint{0.755620in}{1.049992in}}%
\pgfpathlineto{\pgfqpoint{0.772856in}{1.056340in}}%
\pgfpathlineto{\pgfqpoint{0.786263in}{1.061350in}}%
\pgfpathlineto{\pgfqpoint{0.801584in}{1.062980in}}%
\pgfpathlineto{\pgfqpoint{0.805414in}{1.063733in}}%
\pgfpathlineto{\pgfqpoint{0.814990in}{1.065163in}}%
\pgfpathlineto{\pgfqpoint{0.882021in}{1.076339in}}%
\pgfpathlineto{\pgfqpoint{0.885851in}{1.077456in}}%
\pgfpathlineto{\pgfqpoint{0.895427in}{1.078655in}}%
\pgfpathlineto{\pgfqpoint{0.906918in}{1.080409in}}%
\pgfpathlineto{\pgfqpoint{0.987356in}{1.089407in}}%
\pgfpathlineto{\pgfqpoint{1.008422in}{1.090894in}}%
\pgfpathlineto{\pgfqpoint{1.052471in}{1.095137in}}%
\pgfpathlineto{\pgfqpoint{1.056302in}{1.095792in}}%
\pgfpathlineto{\pgfqpoint{1.077369in}{1.097817in}}%
\pgfpathlineto{\pgfqpoint{1.081199in}{1.099401in}}%
\pgfpathlineto{\pgfqpoint{1.096520in}{1.100833in}}%
\pgfpathlineto{\pgfqpoint{1.106096in}{1.102076in}}%
\pgfpathlineto{\pgfqpoint{1.159721in}{1.107209in}}%
\pgfpathlineto{\pgfqpoint{1.163551in}{1.109041in}}%
\pgfpathlineto{\pgfqpoint{1.173127in}{1.110874in}}%
\pgfpathlineto{\pgfqpoint{1.180788in}{1.111938in}}%
\pgfpathlineto{\pgfqpoint{1.186533in}{1.112998in}}%
\pgfpathlineto{\pgfqpoint{1.194194in}{1.114050in}}%
\pgfpathlineto{\pgfqpoint{1.213346in}{1.115660in}}%
\pgfpathlineto{\pgfqpoint{1.221006in}{1.118786in}}%
\pgfpathlineto{\pgfqpoint{1.228667in}{1.120114in}}%
\pgfpathlineto{\pgfqpoint{1.255480in}{1.127277in}}%
\pgfpathlineto{\pgfqpoint{1.265055in}{1.131122in}}%
\pgfpathlineto{\pgfqpoint{1.274631in}{1.132047in}}%
\pgfpathlineto{\pgfqpoint{1.278462in}{1.133795in}}%
\pgfpathlineto{\pgfqpoint{1.295698in}{1.138666in}}%
\pgfpathlineto{\pgfqpoint{1.299528in}{1.139717in}}%
\pgfpathlineto{\pgfqpoint{1.301444in}{1.139860in}}%
\pgfpathlineto{\pgfqpoint{1.303359in}{1.142985in}}%
\pgfpathlineto{\pgfqpoint{1.309104in}{1.143567in}}%
\pgfpathlineto{\pgfqpoint{1.311019in}{1.145858in}}%
\pgfpathlineto{\pgfqpoint{1.316765in}{1.146442in}}%
\pgfpathlineto{\pgfqpoint{1.320595in}{1.149572in}}%
\pgfpathlineto{\pgfqpoint{1.326341in}{1.150947in}}%
\pgfpathlineto{\pgfqpoint{1.334002in}{1.152019in}}%
\pgfpathlineto{\pgfqpoint{1.349323in}{1.155558in}}%
\pgfpathlineto{\pgfqpoint{1.351238in}{1.157653in}}%
\pgfpathlineto{\pgfqpoint{1.353153in}{1.157732in}}%
\pgfpathlineto{\pgfqpoint{1.356984in}{1.160260in}}%
\pgfpathlineto{\pgfqpoint{1.360814in}{1.160544in}}%
\pgfpathlineto{\pgfqpoint{1.368475in}{1.163791in}}%
\pgfpathlineto{\pgfqpoint{1.372305in}{1.164419in}}%
\pgfpathlineto{\pgfqpoint{1.381881in}{1.167150in}}%
\pgfpathlineto{\pgfqpoint{1.395287in}{1.168617in}}%
\pgfpathlineto{\pgfqpoint{1.402948in}{1.170470in}}%
\pgfpathlineto{\pgfqpoint{1.410608in}{1.171407in}}%
\pgfpathlineto{\pgfqpoint{1.420184in}{1.176099in}}%
\pgfpathlineto{\pgfqpoint{1.441251in}{1.178997in}}%
\pgfpathlineto{\pgfqpoint{1.446997in}{1.180239in}}%
\pgfpathlineto{\pgfqpoint{1.458488in}{1.181758in}}%
\pgfpathlineto{\pgfqpoint{1.466148in}{1.183530in}}%
\pgfpathlineto{\pgfqpoint{1.475724in}{1.184325in}}%
\pgfpathlineto{\pgfqpoint{1.485300in}{1.185970in}}%
\pgfpathlineto{\pgfqpoint{1.496791in}{1.186897in}}%
\pgfpathlineto{\pgfqpoint{1.500621in}{1.188076in}}%
\pgfpathlineto{\pgfqpoint{1.510197in}{1.189051in}}%
\pgfpathlineto{\pgfqpoint{1.517858in}{1.191555in}}%
\pgfpathlineto{\pgfqpoint{1.529349in}{1.192277in}}%
\pgfpathlineto{\pgfqpoint{1.542755in}{1.193417in}}%
\pgfpathlineto{\pgfqpoint{1.556161in}{1.195160in}}%
\pgfpathlineto{\pgfqpoint{1.569568in}{1.196028in}}%
\pgfpathlineto{\pgfqpoint{1.573398in}{1.197232in}}%
\pgfpathlineto{\pgfqpoint{1.590635in}{1.200437in}}%
\pgfpathlineto{\pgfqpoint{1.592550in}{1.202315in}}%
\pgfpathlineto{\pgfqpoint{1.598295in}{1.203693in}}%
\pgfpathlineto{\pgfqpoint{1.627023in}{1.210857in}}%
\pgfpathlineto{\pgfqpoint{1.630853in}{1.211879in}}%
\pgfpathlineto{\pgfqpoint{1.632768in}{1.212608in}}%
\pgfpathlineto{\pgfqpoint{1.634683in}{1.214894in}}%
\pgfpathlineto{\pgfqpoint{1.650005in}{1.216939in}}%
\pgfpathlineto{\pgfqpoint{1.653835in}{1.217843in}}%
\pgfpathlineto{\pgfqpoint{1.667241in}{1.220406in}}%
\pgfpathlineto{\pgfqpoint{1.671072in}{1.222320in}}%
\pgfpathlineto{\pgfqpoint{1.678732in}{1.223561in}}%
\pgfpathlineto{\pgfqpoint{1.682563in}{1.225762in}}%
\pgfpathlineto{\pgfqpoint{1.690223in}{1.227473in}}%
\pgfpathlineto{\pgfqpoint{1.694054in}{1.228995in}}%
\pgfpathlineto{\pgfqpoint{1.695969in}{1.229430in}}%
\pgfpathlineto{\pgfqpoint{1.697884in}{1.231143in}}%
\pgfpathlineto{\pgfqpoint{1.701714in}{1.232314in}}%
\pgfpathlineto{\pgfqpoint{1.709375in}{1.233737in}}%
\pgfpathlineto{\pgfqpoint{1.713205in}{1.235570in}}%
\pgfpathlineto{\pgfqpoint{1.728527in}{1.241910in}}%
\pgfpathlineto{\pgfqpoint{1.736188in}{1.243184in}}%
\pgfpathlineto{\pgfqpoint{1.745763in}{1.246246in}}%
\pgfpathlineto{\pgfqpoint{1.757254in}{1.248056in}}%
\pgfpathlineto{\pgfqpoint{1.761085in}{1.248943in}}%
\pgfpathlineto{\pgfqpoint{1.764915in}{1.249471in}}%
\pgfpathlineto{\pgfqpoint{1.774491in}{1.251515in}}%
\pgfpathlineto{\pgfqpoint{1.785982in}{1.252902in}}%
\pgfpathlineto{\pgfqpoint{1.789812in}{1.254325in}}%
\pgfpathlineto{\pgfqpoint{1.793643in}{1.254431in}}%
\pgfpathlineto{\pgfqpoint{1.797473in}{1.256010in}}%
\pgfpathlineto{\pgfqpoint{1.803219in}{1.257885in}}%
\pgfpathlineto{\pgfqpoint{1.835776in}{1.262884in}}%
\pgfpathlineto{\pgfqpoint{1.887486in}{1.270353in}}%
\pgfpathlineto{\pgfqpoint{1.893232in}{1.271230in}}%
\pgfpathlineto{\pgfqpoint{1.902807in}{1.275463in}}%
\pgfpathlineto{\pgfqpoint{1.906638in}{1.276300in}}%
\pgfpathlineto{\pgfqpoint{1.908553in}{1.280411in}}%
\pgfpathlineto{\pgfqpoint{1.916214in}{1.282545in}}%
\pgfpathlineto{\pgfqpoint{1.927705in}{1.284625in}}%
\pgfpathlineto{\pgfqpoint{1.929620in}{1.286283in}}%
\pgfpathlineto{\pgfqpoint{1.937281in}{1.287696in}}%
\pgfpathlineto{\pgfqpoint{1.956432in}{1.296836in}}%
\pgfpathlineto{\pgfqpoint{1.960263in}{1.301263in}}%
\pgfpathlineto{\pgfqpoint{1.964093in}{1.302226in}}%
\pgfpathlineto{\pgfqpoint{1.971754in}{1.304660in}}%
\pgfpathlineto{\pgfqpoint{1.973669in}{1.306666in}}%
\pgfpathlineto{\pgfqpoint{1.981329in}{1.307451in}}%
\pgfpathlineto{\pgfqpoint{1.988990in}{1.309458in}}%
\pgfpathlineto{\pgfqpoint{2.000481in}{1.310640in}}%
\pgfpathlineto{\pgfqpoint{2.002396in}{1.310865in}}%
\pgfpathlineto{\pgfqpoint{2.004312in}{1.313052in}}%
\pgfpathlineto{\pgfqpoint{2.006227in}{1.313203in}}%
\pgfpathlineto{\pgfqpoint{2.010057in}{1.315514in}}%
\pgfpathlineto{\pgfqpoint{2.017718in}{1.316843in}}%
\pgfpathlineto{\pgfqpoint{2.021548in}{1.318677in}}%
\pgfpathlineto{\pgfqpoint{2.031124in}{1.321009in}}%
\pgfpathlineto{\pgfqpoint{2.036869in}{1.325537in}}%
\pgfpathlineto{\pgfqpoint{2.046445in}{1.327598in}}%
\pgfpathlineto{\pgfqpoint{2.048360in}{1.329940in}}%
\pgfpathlineto{\pgfqpoint{2.056021in}{1.331163in}}%
\pgfpathlineto{\pgfqpoint{2.059852in}{1.331423in}}%
\pgfpathlineto{\pgfqpoint{2.063682in}{1.333203in}}%
\pgfpathlineto{\pgfqpoint{2.077088in}{1.334794in}}%
\pgfpathlineto{\pgfqpoint{2.082834in}{1.335836in}}%
\pgfpathlineto{\pgfqpoint{2.086664in}{1.337222in}}%
\pgfpathlineto{\pgfqpoint{2.088579in}{1.339358in}}%
\pgfpathlineto{\pgfqpoint{2.100070in}{1.340897in}}%
\pgfpathlineto{\pgfqpoint{2.103900in}{1.341009in}}%
\pgfpathlineto{\pgfqpoint{2.107731in}{1.345043in}}%
\pgfpathlineto{\pgfqpoint{2.111561in}{1.346544in}}%
\pgfpathlineto{\pgfqpoint{2.132628in}{1.353093in}}%
\pgfpathlineto{\pgfqpoint{2.134543in}{1.354830in}}%
\pgfpathlineto{\pgfqpoint{2.140289in}{1.355716in}}%
\pgfpathlineto{\pgfqpoint{2.144119in}{1.357213in}}%
\pgfpathlineto{\pgfqpoint{2.151780in}{1.358472in}}%
\pgfpathlineto{\pgfqpoint{2.155610in}{1.359652in}}%
\pgfpathlineto{\pgfqpoint{2.157525in}{1.359983in}}%
\pgfpathlineto{\pgfqpoint{2.159440in}{1.362600in}}%
\pgfpathlineto{\pgfqpoint{2.207320in}{1.374146in}}%
\pgfpathlineto{\pgfqpoint{2.216896in}{1.377913in}}%
\pgfpathlineto{\pgfqpoint{2.220726in}{1.378002in}}%
\pgfpathlineto{\pgfqpoint{2.222641in}{1.380086in}}%
\pgfpathlineto{\pgfqpoint{2.236047in}{1.382030in}}%
\pgfpathlineto{\pgfqpoint{2.241793in}{1.384763in}}%
\pgfpathlineto{\pgfqpoint{2.243708in}{1.387612in}}%
\pgfpathlineto{\pgfqpoint{2.249453in}{1.389313in}}%
\pgfpathlineto{\pgfqpoint{2.253284in}{1.390935in}}%
\pgfpathlineto{\pgfqpoint{2.260945in}{1.395304in}}%
\pgfpathlineto{\pgfqpoint{2.262860in}{1.398980in}}%
\pgfpathlineto{\pgfqpoint{2.266690in}{1.400113in}}%
\pgfpathlineto{\pgfqpoint{2.268605in}{1.400377in}}%
\pgfpathlineto{\pgfqpoint{2.272436in}{1.403796in}}%
\pgfpathlineto{\pgfqpoint{2.274351in}{1.403902in}}%
\pgfpathlineto{\pgfqpoint{2.276266in}{1.407289in}}%
\pgfpathlineto{\pgfqpoint{2.287757in}{1.409395in}}%
\pgfpathlineto{\pgfqpoint{2.289672in}{1.410866in}}%
\pgfpathlineto{\pgfqpoint{2.291587in}{1.410933in}}%
\pgfpathlineto{\pgfqpoint{2.293502in}{1.413835in}}%
\pgfpathlineto{\pgfqpoint{2.299248in}{1.414269in}}%
\pgfpathlineto{\pgfqpoint{2.304993in}{1.418648in}}%
\pgfpathlineto{\pgfqpoint{2.310739in}{1.419236in}}%
\pgfpathlineto{\pgfqpoint{2.312654in}{1.421411in}}%
\pgfpathlineto{\pgfqpoint{2.316484in}{1.422608in}}%
\pgfpathlineto{\pgfqpoint{2.320315in}{1.424557in}}%
\pgfpathlineto{\pgfqpoint{2.329891in}{1.429193in}}%
\pgfpathlineto{\pgfqpoint{2.331806in}{1.432102in}}%
\pgfpathlineto{\pgfqpoint{2.335636in}{1.433470in}}%
\pgfpathlineto{\pgfqpoint{2.339467in}{1.437411in}}%
\pgfpathlineto{\pgfqpoint{2.358618in}{1.441416in}}%
\pgfpathlineto{\pgfqpoint{2.362449in}{1.444526in}}%
\pgfpathlineto{\pgfqpoint{2.366279in}{1.447267in}}%
\pgfpathlineto{\pgfqpoint{2.370109in}{1.448358in}}%
\pgfpathlineto{\pgfqpoint{2.373940in}{1.449298in}}%
\pgfpathlineto{\pgfqpoint{2.379685in}{1.450516in}}%
\pgfpathlineto{\pgfqpoint{2.395006in}{1.452221in}}%
\pgfpathlineto{\pgfqpoint{2.410328in}{1.458475in}}%
\pgfpathlineto{\pgfqpoint{2.414158in}{1.459391in}}%
\pgfpathlineto{\pgfqpoint{2.427564in}{1.465748in}}%
\pgfpathlineto{\pgfqpoint{2.433310in}{1.466424in}}%
\pgfpathlineto{\pgfqpoint{2.437140in}{1.471209in}}%
\pgfpathlineto{\pgfqpoint{2.444801in}{1.478878in}}%
\pgfpathlineto{\pgfqpoint{2.448631in}{1.479727in}}%
\pgfpathlineto{\pgfqpoint{2.454377in}{1.483926in}}%
\pgfpathlineto{\pgfqpoint{2.458207in}{1.485287in}}%
\pgfpathlineto{\pgfqpoint{2.463953in}{1.486920in}}%
\pgfpathlineto{\pgfqpoint{2.467783in}{1.488548in}}%
\pgfpathlineto{\pgfqpoint{2.481189in}{1.497385in}}%
\pgfpathlineto{\pgfqpoint{2.483104in}{1.499642in}}%
\pgfpathlineto{\pgfqpoint{2.488850in}{1.501012in}}%
\pgfpathlineto{\pgfqpoint{2.500341in}{1.504464in}}%
\pgfpathlineto{\pgfqpoint{2.504171in}{1.506477in}}%
\pgfpathlineto{\pgfqpoint{2.506086in}{1.512215in}}%
\pgfpathlineto{\pgfqpoint{2.509917in}{1.513305in}}%
\pgfpathlineto{\pgfqpoint{2.511832in}{1.513716in}}%
\pgfpathlineto{\pgfqpoint{2.513747in}{1.517236in}}%
\pgfpathlineto{\pgfqpoint{2.517577in}{1.518534in}}%
\pgfpathlineto{\pgfqpoint{2.519493in}{1.521075in}}%
\pgfpathlineto{\pgfqpoint{2.521408in}{1.521471in}}%
\pgfpathlineto{\pgfqpoint{2.525238in}{1.523827in}}%
\pgfpathlineto{\pgfqpoint{2.542475in}{1.527227in}}%
\pgfpathlineto{\pgfqpoint{2.550135in}{1.531370in}}%
\pgfpathlineto{\pgfqpoint{2.557796in}{1.534325in}}%
\pgfpathlineto{\pgfqpoint{2.561626in}{1.535827in}}%
\pgfpathlineto{\pgfqpoint{2.567372in}{1.539053in}}%
\pgfpathlineto{\pgfqpoint{2.569287in}{1.542904in}}%
\pgfpathlineto{\pgfqpoint{2.575033in}{1.544833in}}%
\pgfpathlineto{\pgfqpoint{2.580778in}{1.546787in}}%
\pgfpathlineto{\pgfqpoint{2.592269in}{1.551344in}}%
\pgfpathlineto{\pgfqpoint{2.594184in}{1.553310in}}%
\pgfpathlineto{\pgfqpoint{2.596099in}{1.553663in}}%
\pgfpathlineto{\pgfqpoint{2.598015in}{1.555244in}}%
\pgfpathlineto{\pgfqpoint{2.601845in}{1.560810in}}%
\pgfpathlineto{\pgfqpoint{2.603760in}{1.562633in}}%
\pgfpathlineto{\pgfqpoint{2.605675in}{1.562837in}}%
\pgfpathlineto{\pgfqpoint{2.611421in}{1.566155in}}%
\pgfpathlineto{\pgfqpoint{2.626742in}{1.570817in}}%
\pgfpathlineto{\pgfqpoint{2.630573in}{1.571164in}}%
\pgfpathlineto{\pgfqpoint{2.638233in}{1.574685in}}%
\pgfpathlineto{\pgfqpoint{2.642064in}{1.579641in}}%
\pgfpathlineto{\pgfqpoint{2.647809in}{1.581096in}}%
\pgfpathlineto{\pgfqpoint{2.649724in}{1.581447in}}%
\pgfpathlineto{\pgfqpoint{2.653555in}{1.584317in}}%
\pgfpathlineto{\pgfqpoint{2.663130in}{1.586069in}}%
\pgfpathlineto{\pgfqpoint{2.666961in}{1.587435in}}%
\pgfpathlineto{\pgfqpoint{2.676537in}{1.590321in}}%
\pgfpathlineto{\pgfqpoint{2.678452in}{1.593693in}}%
\pgfpathlineto{\pgfqpoint{2.682282in}{1.594974in}}%
\pgfpathlineto{\pgfqpoint{2.691858in}{1.597882in}}%
\pgfpathlineto{\pgfqpoint{2.701434in}{1.600655in}}%
\pgfpathlineto{\pgfqpoint{2.703349in}{1.600878in}}%
\pgfpathlineto{\pgfqpoint{2.707179in}{1.602882in}}%
\pgfpathlineto{\pgfqpoint{2.714840in}{1.606766in}}%
\pgfpathlineto{\pgfqpoint{2.716755in}{1.610526in}}%
\pgfpathlineto{\pgfqpoint{2.724416in}{1.612056in}}%
\pgfpathlineto{\pgfqpoint{2.726331in}{1.612349in}}%
\pgfpathlineto{\pgfqpoint{2.728246in}{1.614897in}}%
\pgfpathlineto{\pgfqpoint{2.739737in}{1.616467in}}%
\pgfpathlineto{\pgfqpoint{2.741653in}{1.618264in}}%
\pgfpathlineto{\pgfqpoint{2.749313in}{1.619465in}}%
\pgfpathlineto{\pgfqpoint{2.755059in}{1.621218in}}%
\pgfpathlineto{\pgfqpoint{2.758889in}{1.622390in}}%
\pgfpathlineto{\pgfqpoint{2.762719in}{1.623297in}}%
\pgfpathlineto{\pgfqpoint{2.764635in}{1.625681in}}%
\pgfpathlineto{\pgfqpoint{2.774210in}{1.626737in}}%
\pgfpathlineto{\pgfqpoint{2.781871in}{1.628336in}}%
\pgfpathlineto{\pgfqpoint{2.785701in}{1.630408in}}%
\pgfpathlineto{\pgfqpoint{2.789532in}{1.634811in}}%
\pgfpathlineto{\pgfqpoint{2.793362in}{1.635914in}}%
\pgfpathlineto{\pgfqpoint{2.804853in}{1.639762in}}%
\pgfpathlineto{\pgfqpoint{2.806768in}{1.642283in}}%
\pgfpathlineto{\pgfqpoint{2.810599in}{1.643193in}}%
\pgfpathlineto{\pgfqpoint{2.818259in}{1.645990in}}%
\pgfpathlineto{\pgfqpoint{2.824005in}{1.647253in}}%
\pgfpathlineto{\pgfqpoint{2.825920in}{1.647678in}}%
\pgfpathlineto{\pgfqpoint{2.827835in}{1.651306in}}%
\pgfpathlineto{\pgfqpoint{2.843157in}{1.658318in}}%
\pgfpathlineto{\pgfqpoint{2.846987in}{1.662826in}}%
\pgfpathlineto{\pgfqpoint{2.852732in}{1.665312in}}%
\pgfpathlineto{\pgfqpoint{2.856563in}{1.666470in}}%
\pgfpathlineto{\pgfqpoint{2.862308in}{1.668568in}}%
\pgfpathlineto{\pgfqpoint{2.864223in}{1.669565in}}%
\pgfpathlineto{\pgfqpoint{2.868054in}{1.673167in}}%
\pgfpathlineto{\pgfqpoint{2.875715in}{1.674294in}}%
\pgfpathlineto{\pgfqpoint{2.877630in}{1.676224in}}%
\pgfpathlineto{\pgfqpoint{2.881460in}{1.676644in}}%
\pgfpathlineto{\pgfqpoint{2.898697in}{1.684935in}}%
\pgfpathlineto{\pgfqpoint{2.904442in}{1.692800in}}%
\pgfpathlineto{\pgfqpoint{2.912103in}{1.693895in}}%
\pgfpathlineto{\pgfqpoint{2.921679in}{1.696221in}}%
\pgfpathlineto{\pgfqpoint{2.927424in}{1.697790in}}%
\pgfpathlineto{\pgfqpoint{2.929339in}{1.702518in}}%
\pgfpathlineto{\pgfqpoint{2.942746in}{1.712964in}}%
\pgfpathlineto{\pgfqpoint{2.946576in}{1.714324in}}%
\pgfpathlineto{\pgfqpoint{2.950406in}{1.716044in}}%
\pgfpathlineto{\pgfqpoint{2.958067in}{1.717145in}}%
\pgfpathlineto{\pgfqpoint{2.961897in}{1.722932in}}%
\pgfpathlineto{\pgfqpoint{2.971473in}{1.725541in}}%
\pgfpathlineto{\pgfqpoint{2.975303in}{1.730058in}}%
\pgfpathlineto{\pgfqpoint{2.986794in}{1.737420in}}%
\pgfpathlineto{\pgfqpoint{2.992540in}{1.744238in}}%
\pgfpathlineto{\pgfqpoint{2.998285in}{1.760158in}}%
\pgfpathlineto{\pgfqpoint{3.000201in}{1.763486in}}%
\pgfpathlineto{\pgfqpoint{3.002116in}{1.763574in}}%
\pgfpathlineto{\pgfqpoint{3.013607in}{1.800754in}}%
\pgfpathlineto{\pgfqpoint{3.015522in}{1.819575in}}%
\pgfpathlineto{\pgfqpoint{3.017437in}{1.826535in}}%
\pgfpathlineto{\pgfqpoint{3.017437in}{1.826535in}}%
\pgfusepath{stroke}%
\end{pgfscope}%
\begin{pgfscope}%
\pgfpathrectangle{\pgfqpoint{0.694334in}{0.523557in}}{\pgfqpoint{3.830343in}{1.302977in}}%
\pgfusepath{clip}%
\pgfsetbuttcap%
\pgfsetroundjoin%
\pgfsetlinewidth{1.003750pt}%
\definecolor{currentstroke}{rgb}{0.941176,0.627451,0.188235}%
\pgfsetstrokecolor{currentstroke}%
\pgfsetdash{{1.000000pt}{1.650000pt}}{0.000000pt}%
\pgfpathmoveto{\pgfqpoint{0.694334in}{0.567141in}}%
\pgfpathlineto{\pgfqpoint{0.696249in}{0.582562in}}%
\pgfpathlineto{\pgfqpoint{0.698165in}{0.582755in}}%
\pgfpathlineto{\pgfqpoint{0.701995in}{0.599148in}}%
\pgfpathlineto{\pgfqpoint{0.703910in}{0.606735in}}%
\pgfpathlineto{\pgfqpoint{0.711571in}{0.615604in}}%
\pgfpathlineto{\pgfqpoint{0.713486in}{0.616358in}}%
\pgfpathlineto{\pgfqpoint{0.715401in}{0.627746in}}%
\pgfpathlineto{\pgfqpoint{0.719232in}{0.630247in}}%
\pgfpathlineto{\pgfqpoint{0.721147in}{0.632363in}}%
\pgfpathlineto{\pgfqpoint{0.724977in}{0.633447in}}%
\pgfpathlineto{\pgfqpoint{0.726892in}{0.635337in}}%
\pgfpathlineto{\pgfqpoint{0.732638in}{0.646877in}}%
\pgfpathlineto{\pgfqpoint{0.736468in}{0.649507in}}%
\pgfpathlineto{\pgfqpoint{0.738383in}{0.655445in}}%
\pgfpathlineto{\pgfqpoint{0.740298in}{0.655930in}}%
\pgfpathlineto{\pgfqpoint{0.747959in}{0.671221in}}%
\pgfpathlineto{\pgfqpoint{0.749874in}{0.675620in}}%
\pgfpathlineto{\pgfqpoint{0.751789in}{0.675686in}}%
\pgfpathlineto{\pgfqpoint{0.753705in}{0.678423in}}%
\pgfpathlineto{\pgfqpoint{0.759450in}{0.690456in}}%
\pgfpathlineto{\pgfqpoint{0.770941in}{0.700710in}}%
\pgfpathlineto{\pgfqpoint{0.774772in}{0.703735in}}%
\pgfpathlineto{\pgfqpoint{0.801584in}{0.709532in}}%
\pgfpathlineto{\pgfqpoint{0.805414in}{0.711596in}}%
\pgfpathlineto{\pgfqpoint{0.814990in}{0.713390in}}%
\pgfpathlineto{\pgfqpoint{0.820736in}{0.714546in}}%
\pgfpathlineto{\pgfqpoint{0.836057in}{0.716555in}}%
\pgfpathlineto{\pgfqpoint{0.843718in}{0.719765in}}%
\pgfpathlineto{\pgfqpoint{0.855209in}{0.720566in}}%
\pgfpathlineto{\pgfqpoint{0.857124in}{0.720645in}}%
\pgfpathlineto{\pgfqpoint{0.860954in}{0.722678in}}%
\pgfpathlineto{\pgfqpoint{0.870530in}{0.723679in}}%
\pgfpathlineto{\pgfqpoint{0.878191in}{0.724279in}}%
\pgfpathlineto{\pgfqpoint{0.908834in}{0.725460in}}%
\pgfpathlineto{\pgfqpoint{0.920325in}{0.726063in}}%
\pgfpathlineto{\pgfqpoint{0.945222in}{0.727385in}}%
\pgfpathlineto{\pgfqpoint{0.972034in}{0.728886in}}%
\pgfpathlineto{\pgfqpoint{1.044811in}{0.732918in}}%
\pgfpathlineto{\pgfqpoint{1.083114in}{0.735423in}}%
\pgfpathlineto{\pgfqpoint{1.100351in}{0.737113in}}%
\pgfpathlineto{\pgfqpoint{1.192279in}{0.741689in}}%
\pgfpathlineto{\pgfqpoint{1.213346in}{0.742889in}}%
\pgfpathlineto{\pgfqpoint{1.236328in}{0.744723in}}%
\pgfpathlineto{\pgfqpoint{1.240158in}{0.745473in}}%
\pgfpathlineto{\pgfqpoint{1.284207in}{0.747171in}}%
\pgfpathlineto{\pgfqpoint{1.414439in}{0.759272in}}%
\pgfpathlineto{\pgfqpoint{1.420184in}{0.760773in}}%
\pgfpathlineto{\pgfqpoint{1.431675in}{0.762508in}}%
\pgfpathlineto{\pgfqpoint{1.441251in}{0.763654in}}%
\pgfpathlineto{\pgfqpoint{1.445081in}{0.765852in}}%
\pgfpathlineto{\pgfqpoint{1.450827in}{0.766734in}}%
\pgfpathlineto{\pgfqpoint{1.462318in}{0.771921in}}%
\pgfpathlineto{\pgfqpoint{1.466148in}{0.772684in}}%
\pgfpathlineto{\pgfqpoint{1.475724in}{0.780723in}}%
\pgfpathlineto{\pgfqpoint{1.477639in}{0.784911in}}%
\pgfpathlineto{\pgfqpoint{1.481470in}{0.785898in}}%
\pgfpathlineto{\pgfqpoint{1.483385in}{0.788811in}}%
\pgfpathlineto{\pgfqpoint{1.492961in}{0.790671in}}%
\pgfpathlineto{\pgfqpoint{1.510197in}{0.798592in}}%
\pgfpathlineto{\pgfqpoint{1.521688in}{0.799587in}}%
\pgfpathlineto{\pgfqpoint{1.527434in}{0.802172in}}%
\pgfpathlineto{\pgfqpoint{1.540840in}{0.804764in}}%
\pgfpathlineto{\pgfqpoint{1.544670in}{0.806461in}}%
\pgfpathlineto{\pgfqpoint{1.550416in}{0.807943in}}%
\pgfpathlineto{\pgfqpoint{1.579143in}{0.812868in}}%
\pgfpathlineto{\pgfqpoint{1.582974in}{0.815132in}}%
\pgfpathlineto{\pgfqpoint{1.584889in}{0.815211in}}%
\pgfpathlineto{\pgfqpoint{1.588719in}{0.816491in}}%
\pgfpathlineto{\pgfqpoint{1.596380in}{0.818241in}}%
\pgfpathlineto{\pgfqpoint{1.615532in}{0.820945in}}%
\pgfpathlineto{\pgfqpoint{1.623192in}{0.822119in}}%
\pgfpathlineto{\pgfqpoint{1.634683in}{0.824956in}}%
\pgfpathlineto{\pgfqpoint{1.650005in}{0.827186in}}%
\pgfpathlineto{\pgfqpoint{1.661496in}{0.828179in}}%
\pgfpathlineto{\pgfqpoint{1.680648in}{0.832850in}}%
\pgfpathlineto{\pgfqpoint{1.695969in}{0.834803in}}%
\pgfpathlineto{\pgfqpoint{1.701714in}{0.836460in}}%
\pgfpathlineto{\pgfqpoint{1.718951in}{0.838267in}}%
\pgfpathlineto{\pgfqpoint{1.724697in}{0.841013in}}%
\pgfpathlineto{\pgfqpoint{1.730442in}{0.841765in}}%
\pgfpathlineto{\pgfqpoint{1.738103in}{0.843347in}}%
\pgfpathlineto{\pgfqpoint{1.766830in}{0.847835in}}%
\pgfpathlineto{\pgfqpoint{1.768745in}{0.848125in}}%
\pgfpathlineto{\pgfqpoint{1.772576in}{0.850816in}}%
\pgfpathlineto{\pgfqpoint{1.778321in}{0.852136in}}%
\pgfpathlineto{\pgfqpoint{1.789812in}{0.855151in}}%
\pgfpathlineto{\pgfqpoint{1.793643in}{0.858744in}}%
\pgfpathlineto{\pgfqpoint{1.799388in}{0.860410in}}%
\pgfpathlineto{\pgfqpoint{1.808964in}{0.861690in}}%
\pgfpathlineto{\pgfqpoint{1.897062in}{0.876194in}}%
\pgfpathlineto{\pgfqpoint{1.902807in}{0.879436in}}%
\pgfpathlineto{\pgfqpoint{1.962178in}{0.889986in}}%
\pgfpathlineto{\pgfqpoint{1.998566in}{0.895279in}}%
\pgfpathlineto{\pgfqpoint{2.002396in}{0.897136in}}%
\pgfpathlineto{\pgfqpoint{2.019633in}{0.899457in}}%
\pgfpathlineto{\pgfqpoint{2.023463in}{0.900421in}}%
\pgfpathlineto{\pgfqpoint{2.046445in}{0.903270in}}%
\pgfpathlineto{\pgfqpoint{2.063682in}{0.905826in}}%
\pgfpathlineto{\pgfqpoint{2.077088in}{0.907498in}}%
\pgfpathlineto{\pgfqpoint{2.092409in}{0.909303in}}%
\pgfpathlineto{\pgfqpoint{2.121137in}{0.912436in}}%
\pgfpathlineto{\pgfqpoint{2.124967in}{0.913570in}}%
\pgfpathlineto{\pgfqpoint{2.136458in}{0.915106in}}%
\pgfpathlineto{\pgfqpoint{2.138374in}{0.916643in}}%
\pgfpathlineto{\pgfqpoint{2.144119in}{0.917605in}}%
\pgfpathlineto{\pgfqpoint{2.149865in}{0.918496in}}%
\pgfpathlineto{\pgfqpoint{2.151780in}{0.918737in}}%
\pgfpathlineto{\pgfqpoint{2.153695in}{0.920150in}}%
\pgfpathlineto{\pgfqpoint{2.163271in}{0.920991in}}%
\pgfpathlineto{\pgfqpoint{2.214980in}{0.931176in}}%
\pgfpathlineto{\pgfqpoint{2.224556in}{0.931968in}}%
\pgfpathlineto{\pgfqpoint{2.236047in}{0.935912in}}%
\pgfpathlineto{\pgfqpoint{2.243708in}{0.936754in}}%
\pgfpathlineto{\pgfqpoint{2.280096in}{0.941676in}}%
\pgfpathlineto{\pgfqpoint{2.289672in}{0.942902in}}%
\pgfpathlineto{\pgfqpoint{2.316484in}{0.945052in}}%
\pgfpathlineto{\pgfqpoint{2.329891in}{0.945964in}}%
\pgfpathlineto{\pgfqpoint{2.335636in}{0.947179in}}%
\pgfpathlineto{\pgfqpoint{2.358618in}{0.948809in}}%
\pgfpathlineto{\pgfqpoint{2.366279in}{0.950171in}}%
\pgfpathlineto{\pgfqpoint{2.373940in}{0.951947in}}%
\pgfpathlineto{\pgfqpoint{2.377770in}{0.954446in}}%
\pgfpathlineto{\pgfqpoint{2.389261in}{0.956433in}}%
\pgfpathlineto{\pgfqpoint{2.391176in}{0.957190in}}%
\pgfpathlineto{\pgfqpoint{2.395006in}{0.960086in}}%
\pgfpathlineto{\pgfqpoint{2.406498in}{0.961412in}}%
\pgfpathlineto{\pgfqpoint{2.410328in}{0.963197in}}%
\pgfpathlineto{\pgfqpoint{2.419904in}{0.964791in}}%
\pgfpathlineto{\pgfqpoint{2.423734in}{0.967067in}}%
\pgfpathlineto{\pgfqpoint{2.439055in}{0.974701in}}%
\pgfpathlineto{\pgfqpoint{2.450546in}{0.980023in}}%
\pgfpathlineto{\pgfqpoint{2.458207in}{0.982179in}}%
\pgfpathlineto{\pgfqpoint{2.460122in}{0.986058in}}%
\pgfpathlineto{\pgfqpoint{2.465868in}{0.988210in}}%
\pgfpathlineto{\pgfqpoint{2.471613in}{0.989814in}}%
\pgfpathlineto{\pgfqpoint{2.477359in}{0.992420in}}%
\pgfpathlineto{\pgfqpoint{2.486935in}{0.993619in}}%
\pgfpathlineto{\pgfqpoint{2.498426in}{0.997123in}}%
\pgfpathlineto{\pgfqpoint{2.500341in}{0.999870in}}%
\pgfpathlineto{\pgfqpoint{2.509917in}{1.001205in}}%
\pgfpathlineto{\pgfqpoint{2.513747in}{1.003966in}}%
\pgfpathlineto{\pgfqpoint{2.523323in}{1.005179in}}%
\pgfpathlineto{\pgfqpoint{2.529068in}{1.006342in}}%
\pgfpathlineto{\pgfqpoint{2.532899in}{1.007849in}}%
\pgfpathlineto{\pgfqpoint{2.536729in}{1.008569in}}%
\pgfpathlineto{\pgfqpoint{2.557796in}{1.012606in}}%
\pgfpathlineto{\pgfqpoint{2.561626in}{1.012872in}}%
\pgfpathlineto{\pgfqpoint{2.565457in}{1.015743in}}%
\pgfpathlineto{\pgfqpoint{2.569287in}{1.016717in}}%
\pgfpathlineto{\pgfqpoint{2.573117in}{1.018092in}}%
\pgfpathlineto{\pgfqpoint{2.575033in}{1.018320in}}%
\pgfpathlineto{\pgfqpoint{2.576948in}{1.020656in}}%
\pgfpathlineto{\pgfqpoint{2.580778in}{1.021639in}}%
\pgfpathlineto{\pgfqpoint{2.584608in}{1.023294in}}%
\pgfpathlineto{\pgfqpoint{2.590354in}{1.024158in}}%
\pgfpathlineto{\pgfqpoint{2.592269in}{1.026817in}}%
\pgfpathlineto{\pgfqpoint{2.594184in}{1.027094in}}%
\pgfpathlineto{\pgfqpoint{2.596099in}{1.029692in}}%
\pgfpathlineto{\pgfqpoint{2.598015in}{1.029857in}}%
\pgfpathlineto{\pgfqpoint{2.599930in}{1.032096in}}%
\pgfpathlineto{\pgfqpoint{2.605675in}{1.034423in}}%
\pgfpathlineto{\pgfqpoint{2.611421in}{1.039133in}}%
\pgfpathlineto{\pgfqpoint{2.626742in}{1.043293in}}%
\pgfpathlineto{\pgfqpoint{2.628657in}{1.044111in}}%
\pgfpathlineto{\pgfqpoint{2.630573in}{1.046152in}}%
\pgfpathlineto{\pgfqpoint{2.632488in}{1.051263in}}%
\pgfpathlineto{\pgfqpoint{2.642064in}{1.053898in}}%
\pgfpathlineto{\pgfqpoint{2.645894in}{1.058431in}}%
\pgfpathlineto{\pgfqpoint{2.647809in}{1.059291in}}%
\pgfpathlineto{\pgfqpoint{2.651639in}{1.062244in}}%
\pgfpathlineto{\pgfqpoint{2.659300in}{1.066143in}}%
\pgfpathlineto{\pgfqpoint{2.682282in}{1.071259in}}%
\pgfpathlineto{\pgfqpoint{2.684197in}{1.073698in}}%
\pgfpathlineto{\pgfqpoint{2.686113in}{1.074109in}}%
\pgfpathlineto{\pgfqpoint{2.688028in}{1.077390in}}%
\pgfpathlineto{\pgfqpoint{2.691858in}{1.078985in}}%
\pgfpathlineto{\pgfqpoint{2.695688in}{1.081315in}}%
\pgfpathlineto{\pgfqpoint{2.697604in}{1.081341in}}%
\pgfpathlineto{\pgfqpoint{2.699519in}{1.084661in}}%
\pgfpathlineto{\pgfqpoint{2.703349in}{1.086777in}}%
\pgfpathlineto{\pgfqpoint{2.709095in}{1.092806in}}%
\pgfpathlineto{\pgfqpoint{2.712925in}{1.094980in}}%
\pgfpathlineto{\pgfqpoint{2.718670in}{1.108459in}}%
\pgfpathlineto{\pgfqpoint{2.722501in}{1.116280in}}%
\pgfpathlineto{\pgfqpoint{2.724416in}{1.118961in}}%
\pgfpathlineto{\pgfqpoint{2.726331in}{1.119126in}}%
\pgfpathlineto{\pgfqpoint{2.732077in}{1.121740in}}%
\pgfpathlineto{\pgfqpoint{2.737822in}{1.122041in}}%
\pgfpathlineto{\pgfqpoint{2.739737in}{1.128837in}}%
\pgfpathlineto{\pgfqpoint{2.741653in}{1.131149in}}%
\pgfpathlineto{\pgfqpoint{2.743568in}{1.142341in}}%
\pgfpathlineto{\pgfqpoint{2.747398in}{1.143926in}}%
\pgfpathlineto{\pgfqpoint{2.749313in}{1.152235in}}%
\pgfpathlineto{\pgfqpoint{2.753144in}{1.152754in}}%
\pgfpathlineto{\pgfqpoint{2.755059in}{1.163088in}}%
\pgfpathlineto{\pgfqpoint{2.756974in}{1.164581in}}%
\pgfpathlineto{\pgfqpoint{2.760804in}{1.170521in}}%
\pgfpathlineto{\pgfqpoint{2.762719in}{1.178607in}}%
\pgfpathlineto{\pgfqpoint{2.764635in}{1.180313in}}%
\pgfpathlineto{\pgfqpoint{2.768465in}{1.189905in}}%
\pgfpathlineto{\pgfqpoint{2.770380in}{1.191539in}}%
\pgfpathlineto{\pgfqpoint{2.776126in}{1.202914in}}%
\pgfpathlineto{\pgfqpoint{2.779956in}{1.219104in}}%
\pgfpathlineto{\pgfqpoint{2.781871in}{1.223605in}}%
\pgfpathlineto{\pgfqpoint{2.785701in}{1.240588in}}%
\pgfpathlineto{\pgfqpoint{2.787617in}{1.240613in}}%
\pgfpathlineto{\pgfqpoint{2.791447in}{1.244129in}}%
\pgfpathlineto{\pgfqpoint{2.793362in}{1.248530in}}%
\pgfpathlineto{\pgfqpoint{2.795277in}{1.258087in}}%
\pgfpathlineto{\pgfqpoint{2.797192in}{1.276958in}}%
\pgfpathlineto{\pgfqpoint{2.799108in}{1.278513in}}%
\pgfpathlineto{\pgfqpoint{2.802938in}{1.290776in}}%
\pgfpathlineto{\pgfqpoint{2.806768in}{1.339754in}}%
\pgfpathlineto{\pgfqpoint{2.808684in}{1.379698in}}%
\pgfpathlineto{\pgfqpoint{2.810599in}{1.381391in}}%
\pgfpathlineto{\pgfqpoint{2.812514in}{1.385237in}}%
\pgfpathlineto{\pgfqpoint{2.822090in}{1.589779in}}%
\pgfpathlineto{\pgfqpoint{2.825920in}{1.796014in}}%
\pgfpathlineto{\pgfqpoint{2.827835in}{1.826535in}}%
\pgfpathlineto{\pgfqpoint{2.827835in}{1.826535in}}%
\pgfusepath{stroke}%
\end{pgfscope}%
\begin{pgfscope}%
\pgfpathrectangle{\pgfqpoint{0.694334in}{0.523557in}}{\pgfqpoint{3.830343in}{1.302977in}}%
\pgfusepath{clip}%
\pgfsetbuttcap%
\pgfsetroundjoin%
\pgfsetlinewidth{1.003750pt}%
\definecolor{currentstroke}{rgb}{0.062745,0.000000,0.062745}%
\pgfsetstrokecolor{currentstroke}%
\pgfsetdash{{3.700000pt}{1.600000pt}}{0.000000pt}%
\pgfpathmoveto{\pgfqpoint{0.694334in}{0.613292in}}%
\pgfpathlineto{\pgfqpoint{0.696249in}{0.621949in}}%
\pgfpathlineto{\pgfqpoint{0.698165in}{0.622951in}}%
\pgfpathlineto{\pgfqpoint{0.700080in}{0.636570in}}%
\pgfpathlineto{\pgfqpoint{0.703910in}{0.641552in}}%
\pgfpathlineto{\pgfqpoint{0.705825in}{0.649251in}}%
\pgfpathlineto{\pgfqpoint{0.711571in}{0.652708in}}%
\pgfpathlineto{\pgfqpoint{0.713486in}{0.652973in}}%
\pgfpathlineto{\pgfqpoint{0.715401in}{0.660913in}}%
\pgfpathlineto{\pgfqpoint{0.721147in}{0.664229in}}%
\pgfpathlineto{\pgfqpoint{0.723062in}{0.668212in}}%
\pgfpathlineto{\pgfqpoint{0.724977in}{0.668256in}}%
\pgfpathlineto{\pgfqpoint{0.730723in}{0.673020in}}%
\pgfpathlineto{\pgfqpoint{0.734553in}{0.683462in}}%
\pgfpathlineto{\pgfqpoint{0.742214in}{0.687870in}}%
\pgfpathlineto{\pgfqpoint{0.744129in}{0.690797in}}%
\pgfpathlineto{\pgfqpoint{0.753705in}{0.693659in}}%
\pgfpathlineto{\pgfqpoint{0.755620in}{0.697917in}}%
\pgfpathlineto{\pgfqpoint{0.757535in}{0.699322in}}%
\pgfpathlineto{\pgfqpoint{0.759450in}{0.711299in}}%
\pgfpathlineto{\pgfqpoint{0.774772in}{0.714803in}}%
\pgfpathlineto{\pgfqpoint{0.782432in}{0.715851in}}%
\pgfpathlineto{\pgfqpoint{0.803499in}{0.718816in}}%
\pgfpathlineto{\pgfqpoint{0.813075in}{0.719973in}}%
\pgfpathlineto{\pgfqpoint{0.836057in}{0.724418in}}%
\pgfpathlineto{\pgfqpoint{0.843718in}{0.725105in}}%
\pgfpathlineto{\pgfqpoint{0.864785in}{0.726369in}}%
\pgfpathlineto{\pgfqpoint{0.870530in}{0.728363in}}%
\pgfpathlineto{\pgfqpoint{0.885851in}{0.729985in}}%
\pgfpathlineto{\pgfqpoint{0.935646in}{0.733917in}}%
\pgfpathlineto{\pgfqpoint{0.966289in}{0.735553in}}%
\pgfpathlineto{\pgfqpoint{0.970119in}{0.736329in}}%
\pgfpathlineto{\pgfqpoint{1.002677in}{0.738756in}}%
\pgfpathlineto{\pgfqpoint{1.027574in}{0.739852in}}%
\pgfpathlineto{\pgfqpoint{1.040980in}{0.740505in}}%
\pgfpathlineto{\pgfqpoint{1.104181in}{0.744093in}}%
\pgfpathlineto{\pgfqpoint{1.303359in}{0.758814in}}%
\pgfpathlineto{\pgfqpoint{1.332086in}{0.760504in}}%
\pgfpathlineto{\pgfqpoint{1.347408in}{0.761997in}}%
\pgfpathlineto{\pgfqpoint{1.358899in}{0.762976in}}%
\pgfpathlineto{\pgfqpoint{1.410608in}{0.769284in}}%
\pgfpathlineto{\pgfqpoint{1.420184in}{0.773766in}}%
\pgfpathlineto{\pgfqpoint{1.424015in}{0.775241in}}%
\pgfpathlineto{\pgfqpoint{1.431675in}{0.776193in}}%
\pgfpathlineto{\pgfqpoint{1.435506in}{0.777845in}}%
\pgfpathlineto{\pgfqpoint{1.439336in}{0.778714in}}%
\pgfpathlineto{\pgfqpoint{1.441251in}{0.781600in}}%
\pgfpathlineto{\pgfqpoint{1.443166in}{0.781687in}}%
\pgfpathlineto{\pgfqpoint{1.445081in}{0.783849in}}%
\pgfpathlineto{\pgfqpoint{1.446997in}{0.784167in}}%
\pgfpathlineto{\pgfqpoint{1.448912in}{0.786177in}}%
\pgfpathlineto{\pgfqpoint{1.462318in}{0.789214in}}%
\pgfpathlineto{\pgfqpoint{1.468064in}{0.790727in}}%
\pgfpathlineto{\pgfqpoint{1.471894in}{0.791007in}}%
\pgfpathlineto{\pgfqpoint{1.473809in}{0.793216in}}%
\pgfpathlineto{\pgfqpoint{1.477639in}{0.795139in}}%
\pgfpathlineto{\pgfqpoint{1.479555in}{0.803461in}}%
\pgfpathlineto{\pgfqpoint{1.483385in}{0.806286in}}%
\pgfpathlineto{\pgfqpoint{1.498706in}{0.812237in}}%
\pgfpathlineto{\pgfqpoint{1.502537in}{0.812525in}}%
\pgfpathlineto{\pgfqpoint{1.506367in}{0.814271in}}%
\pgfpathlineto{\pgfqpoint{1.512112in}{0.814900in}}%
\pgfpathlineto{\pgfqpoint{1.517858in}{0.816308in}}%
\pgfpathlineto{\pgfqpoint{1.523604in}{0.817857in}}%
\pgfpathlineto{\pgfqpoint{1.525519in}{0.820031in}}%
\pgfpathlineto{\pgfqpoint{1.542755in}{0.823617in}}%
\pgfpathlineto{\pgfqpoint{1.586804in}{0.833309in}}%
\pgfpathlineto{\pgfqpoint{1.602126in}{0.834527in}}%
\pgfpathlineto{\pgfqpoint{1.619362in}{0.837079in}}%
\pgfpathlineto{\pgfqpoint{1.650005in}{0.842212in}}%
\pgfpathlineto{\pgfqpoint{1.657666in}{0.843519in}}%
\pgfpathlineto{\pgfqpoint{1.665326in}{0.845086in}}%
\pgfpathlineto{\pgfqpoint{1.672987in}{0.845492in}}%
\pgfpathlineto{\pgfqpoint{1.676817in}{0.846960in}}%
\pgfpathlineto{\pgfqpoint{1.690223in}{0.848621in}}%
\pgfpathlineto{\pgfqpoint{1.699799in}{0.850941in}}%
\pgfpathlineto{\pgfqpoint{1.711290in}{0.852917in}}%
\pgfpathlineto{\pgfqpoint{1.720866in}{0.855801in}}%
\pgfpathlineto{\pgfqpoint{1.730442in}{0.856953in}}%
\pgfpathlineto{\pgfqpoint{1.768745in}{0.864715in}}%
\pgfpathlineto{\pgfqpoint{1.776406in}{0.866159in}}%
\pgfpathlineto{\pgfqpoint{1.822370in}{0.871388in}}%
\pgfpathlineto{\pgfqpoint{1.824285in}{0.873073in}}%
\pgfpathlineto{\pgfqpoint{1.830031in}{0.874370in}}%
\pgfpathlineto{\pgfqpoint{1.833861in}{0.876147in}}%
\pgfpathlineto{\pgfqpoint{1.877910in}{0.882911in}}%
\pgfpathlineto{\pgfqpoint{1.883656in}{0.885176in}}%
\pgfpathlineto{\pgfqpoint{1.895147in}{0.887468in}}%
\pgfpathlineto{\pgfqpoint{1.900892in}{0.888286in}}%
\pgfpathlineto{\pgfqpoint{1.923874in}{0.892108in}}%
\pgfpathlineto{\pgfqpoint{1.927705in}{0.893412in}}%
\pgfpathlineto{\pgfqpoint{1.937281in}{0.894443in}}%
\pgfpathlineto{\pgfqpoint{1.950687in}{0.895245in}}%
\pgfpathlineto{\pgfqpoint{1.954517in}{0.896470in}}%
\pgfpathlineto{\pgfqpoint{1.958347in}{0.897567in}}%
\pgfpathlineto{\pgfqpoint{1.966008in}{0.899326in}}%
\pgfpathlineto{\pgfqpoint{1.994736in}{0.904957in}}%
\pgfpathlineto{\pgfqpoint{2.006227in}{0.906883in}}%
\pgfpathlineto{\pgfqpoint{2.011972in}{0.907787in}}%
\pgfpathlineto{\pgfqpoint{2.021548in}{0.909620in}}%
\pgfpathlineto{\pgfqpoint{2.025378in}{0.910426in}}%
\pgfpathlineto{\pgfqpoint{2.040700in}{0.911965in}}%
\pgfpathlineto{\pgfqpoint{2.088579in}{0.921193in}}%
\pgfpathlineto{\pgfqpoint{2.094325in}{0.923094in}}%
\pgfpathlineto{\pgfqpoint{2.107731in}{0.924941in}}%
\pgfpathlineto{\pgfqpoint{2.115391in}{0.925691in}}%
\pgfpathlineto{\pgfqpoint{2.121137in}{0.926982in}}%
\pgfpathlineto{\pgfqpoint{2.123052in}{0.927051in}}%
\pgfpathlineto{\pgfqpoint{2.126883in}{0.928355in}}%
\pgfpathlineto{\pgfqpoint{2.165186in}{0.934156in}}%
\pgfpathlineto{\pgfqpoint{2.169016in}{0.935670in}}%
\pgfpathlineto{\pgfqpoint{2.209235in}{0.943307in}}%
\pgfpathlineto{\pgfqpoint{2.213065in}{0.944656in}}%
\pgfpathlineto{\pgfqpoint{2.283927in}{0.953096in}}%
\pgfpathlineto{\pgfqpoint{2.295418in}{0.956536in}}%
\pgfpathlineto{\pgfqpoint{2.312654in}{0.959066in}}%
\pgfpathlineto{\pgfqpoint{2.318400in}{0.960021in}}%
\pgfpathlineto{\pgfqpoint{2.322230in}{0.961899in}}%
\pgfpathlineto{\pgfqpoint{2.326060in}{0.962330in}}%
\pgfpathlineto{\pgfqpoint{2.327975in}{0.964487in}}%
\pgfpathlineto{\pgfqpoint{2.343297in}{0.967160in}}%
\pgfpathlineto{\pgfqpoint{2.352873in}{0.970888in}}%
\pgfpathlineto{\pgfqpoint{2.354788in}{0.971079in}}%
\pgfpathlineto{\pgfqpoint{2.356703in}{0.972769in}}%
\pgfpathlineto{\pgfqpoint{2.360533in}{0.973870in}}%
\pgfpathlineto{\pgfqpoint{2.368194in}{0.975749in}}%
\pgfpathlineto{\pgfqpoint{2.373940in}{0.978564in}}%
\pgfpathlineto{\pgfqpoint{2.385431in}{0.981230in}}%
\pgfpathlineto{\pgfqpoint{2.389261in}{0.982244in}}%
\pgfpathlineto{\pgfqpoint{2.400752in}{0.987375in}}%
\pgfpathlineto{\pgfqpoint{2.406498in}{0.989127in}}%
\pgfpathlineto{\pgfqpoint{2.435225in}{0.998211in}}%
\pgfpathlineto{\pgfqpoint{2.442886in}{0.999568in}}%
\pgfpathlineto{\pgfqpoint{2.446716in}{1.000610in}}%
\pgfpathlineto{\pgfqpoint{2.454377in}{1.001857in}}%
\pgfpathlineto{\pgfqpoint{2.467783in}{1.008305in}}%
\pgfpathlineto{\pgfqpoint{2.473529in}{1.008820in}}%
\pgfpathlineto{\pgfqpoint{2.479274in}{1.011252in}}%
\pgfpathlineto{\pgfqpoint{2.486935in}{1.011900in}}%
\pgfpathlineto{\pgfqpoint{2.490765in}{1.013914in}}%
\pgfpathlineto{\pgfqpoint{2.496511in}{1.014687in}}%
\pgfpathlineto{\pgfqpoint{2.500341in}{1.015832in}}%
\pgfpathlineto{\pgfqpoint{2.515662in}{1.018394in}}%
\pgfpathlineto{\pgfqpoint{2.517577in}{1.018503in}}%
\pgfpathlineto{\pgfqpoint{2.523323in}{1.021593in}}%
\pgfpathlineto{\pgfqpoint{2.529068in}{1.022097in}}%
\pgfpathlineto{\pgfqpoint{2.530984in}{1.024576in}}%
\pgfpathlineto{\pgfqpoint{2.534814in}{1.026401in}}%
\pgfpathlineto{\pgfqpoint{2.536729in}{1.029089in}}%
\pgfpathlineto{\pgfqpoint{2.542475in}{1.029334in}}%
\pgfpathlineto{\pgfqpoint{2.546305in}{1.031708in}}%
\pgfpathlineto{\pgfqpoint{2.553966in}{1.034094in}}%
\pgfpathlineto{\pgfqpoint{2.557796in}{1.034626in}}%
\pgfpathlineto{\pgfqpoint{2.561626in}{1.036665in}}%
\pgfpathlineto{\pgfqpoint{2.569287in}{1.039114in}}%
\pgfpathlineto{\pgfqpoint{2.571202in}{1.039116in}}%
\pgfpathlineto{\pgfqpoint{2.575033in}{1.043055in}}%
\pgfpathlineto{\pgfqpoint{2.594184in}{1.046713in}}%
\pgfpathlineto{\pgfqpoint{2.596099in}{1.048649in}}%
\pgfpathlineto{\pgfqpoint{2.601845in}{1.049867in}}%
\pgfpathlineto{\pgfqpoint{2.603760in}{1.052652in}}%
\pgfpathlineto{\pgfqpoint{2.609506in}{1.054760in}}%
\pgfpathlineto{\pgfqpoint{2.611421in}{1.056654in}}%
\pgfpathlineto{\pgfqpoint{2.615251in}{1.056941in}}%
\pgfpathlineto{\pgfqpoint{2.619082in}{1.058611in}}%
\pgfpathlineto{\pgfqpoint{2.620997in}{1.058659in}}%
\pgfpathlineto{\pgfqpoint{2.626742in}{1.063202in}}%
\pgfpathlineto{\pgfqpoint{2.632488in}{1.068559in}}%
\pgfpathlineto{\pgfqpoint{2.638233in}{1.070863in}}%
\pgfpathlineto{\pgfqpoint{2.640148in}{1.073210in}}%
\pgfpathlineto{\pgfqpoint{2.642064in}{1.073376in}}%
\pgfpathlineto{\pgfqpoint{2.653555in}{1.079961in}}%
\pgfpathlineto{\pgfqpoint{2.659300in}{1.081531in}}%
\pgfpathlineto{\pgfqpoint{2.665046in}{1.083021in}}%
\pgfpathlineto{\pgfqpoint{2.668876in}{1.086147in}}%
\pgfpathlineto{\pgfqpoint{2.672706in}{1.089786in}}%
\pgfpathlineto{\pgfqpoint{2.674622in}{1.092776in}}%
\pgfpathlineto{\pgfqpoint{2.676537in}{1.093057in}}%
\pgfpathlineto{\pgfqpoint{2.680367in}{1.095473in}}%
\pgfpathlineto{\pgfqpoint{2.684197in}{1.096841in}}%
\pgfpathlineto{\pgfqpoint{2.693773in}{1.098690in}}%
\pgfpathlineto{\pgfqpoint{2.697604in}{1.100264in}}%
\pgfpathlineto{\pgfqpoint{2.705264in}{1.104803in}}%
\pgfpathlineto{\pgfqpoint{2.707179in}{1.108273in}}%
\pgfpathlineto{\pgfqpoint{2.720586in}{1.115609in}}%
\pgfpathlineto{\pgfqpoint{2.726331in}{1.125363in}}%
\pgfpathlineto{\pgfqpoint{2.728246in}{1.129427in}}%
\pgfpathlineto{\pgfqpoint{2.730161in}{1.129533in}}%
\pgfpathlineto{\pgfqpoint{2.737822in}{1.135682in}}%
\pgfpathlineto{\pgfqpoint{2.741653in}{1.139500in}}%
\pgfpathlineto{\pgfqpoint{2.743568in}{1.139923in}}%
\pgfpathlineto{\pgfqpoint{2.745483in}{1.149433in}}%
\pgfpathlineto{\pgfqpoint{2.747398in}{1.150761in}}%
\pgfpathlineto{\pgfqpoint{2.749313in}{1.159555in}}%
\pgfpathlineto{\pgfqpoint{2.753144in}{1.163879in}}%
\pgfpathlineto{\pgfqpoint{2.758889in}{1.169167in}}%
\pgfpathlineto{\pgfqpoint{2.760804in}{1.176447in}}%
\pgfpathlineto{\pgfqpoint{2.764635in}{1.179226in}}%
\pgfpathlineto{\pgfqpoint{2.774210in}{1.200352in}}%
\pgfpathlineto{\pgfqpoint{2.776126in}{1.200960in}}%
\pgfpathlineto{\pgfqpoint{2.779956in}{1.211885in}}%
\pgfpathlineto{\pgfqpoint{2.781871in}{1.223561in}}%
\pgfpathlineto{\pgfqpoint{2.787617in}{1.233278in}}%
\pgfpathlineto{\pgfqpoint{2.791447in}{1.235612in}}%
\pgfpathlineto{\pgfqpoint{2.793362in}{1.239034in}}%
\pgfpathlineto{\pgfqpoint{2.795277in}{1.239183in}}%
\pgfpathlineto{\pgfqpoint{2.797192in}{1.241743in}}%
\pgfpathlineto{\pgfqpoint{2.802938in}{1.258053in}}%
\pgfpathlineto{\pgfqpoint{2.804853in}{1.259950in}}%
\pgfpathlineto{\pgfqpoint{2.806768in}{1.260051in}}%
\pgfpathlineto{\pgfqpoint{2.808684in}{1.265087in}}%
\pgfpathlineto{\pgfqpoint{2.810599in}{1.266597in}}%
\pgfpathlineto{\pgfqpoint{2.812514in}{1.266653in}}%
\pgfpathlineto{\pgfqpoint{2.814429in}{1.294932in}}%
\pgfpathlineto{\pgfqpoint{2.816344in}{1.296020in}}%
\pgfpathlineto{\pgfqpoint{2.820175in}{1.422730in}}%
\pgfpathlineto{\pgfqpoint{2.822090in}{1.516447in}}%
\pgfpathlineto{\pgfqpoint{2.824005in}{1.826535in}}%
\pgfpathlineto{\pgfqpoint{2.824005in}{1.826535in}}%
\pgfusepath{stroke}%
\end{pgfscope}%
\begin{pgfscope}%
\pgfpathrectangle{\pgfqpoint{0.694334in}{0.523557in}}{\pgfqpoint{3.830343in}{1.302977in}}%
\pgfusepath{clip}%
\pgfsetbuttcap%
\pgfsetroundjoin%
\pgfsetlinewidth{1.003750pt}%
\definecolor{currentstroke}{rgb}{0.811765,0.125490,0.125490}%
\pgfsetstrokecolor{currentstroke}%
\pgfsetdash{{1.000000pt}{1.650000pt}}{0.000000pt}%
\pgfpathmoveto{\pgfqpoint{0.694334in}{1.092842in}}%
\pgfpathlineto{\pgfqpoint{0.700080in}{1.095601in}}%
\pgfpathlineto{\pgfqpoint{0.713486in}{1.097828in}}%
\pgfpathlineto{\pgfqpoint{0.715401in}{1.099365in}}%
\pgfpathlineto{\pgfqpoint{0.723062in}{1.100590in}}%
\pgfpathlineto{\pgfqpoint{0.738383in}{1.104836in}}%
\pgfpathlineto{\pgfqpoint{0.740298in}{1.108199in}}%
\pgfpathlineto{\pgfqpoint{0.744129in}{1.109019in}}%
\pgfpathlineto{\pgfqpoint{0.747959in}{1.109596in}}%
\pgfpathlineto{\pgfqpoint{0.749874in}{1.118682in}}%
\pgfpathlineto{\pgfqpoint{0.751789in}{1.118957in}}%
\pgfpathlineto{\pgfqpoint{0.755620in}{1.131559in}}%
\pgfpathlineto{\pgfqpoint{0.757535in}{1.132092in}}%
\pgfpathlineto{\pgfqpoint{0.759450in}{1.137051in}}%
\pgfpathlineto{\pgfqpoint{0.761365in}{1.138654in}}%
\pgfpathlineto{\pgfqpoint{0.765196in}{1.145963in}}%
\pgfpathlineto{\pgfqpoint{0.767111in}{1.146688in}}%
\pgfpathlineto{\pgfqpoint{0.769026in}{1.150685in}}%
\pgfpathlineto{\pgfqpoint{0.772856in}{1.152392in}}%
\pgfpathlineto{\pgfqpoint{0.776687in}{1.160254in}}%
\pgfpathlineto{\pgfqpoint{0.784347in}{1.161935in}}%
\pgfpathlineto{\pgfqpoint{0.788178in}{1.173187in}}%
\pgfpathlineto{\pgfqpoint{0.792008in}{1.176899in}}%
\pgfpathlineto{\pgfqpoint{0.795838in}{1.177668in}}%
\pgfpathlineto{\pgfqpoint{0.797754in}{1.180982in}}%
\pgfpathlineto{\pgfqpoint{0.799669in}{1.187098in}}%
\pgfpathlineto{\pgfqpoint{0.805414in}{1.190346in}}%
\pgfpathlineto{\pgfqpoint{0.811160in}{1.192110in}}%
\pgfpathlineto{\pgfqpoint{0.814990in}{1.195951in}}%
\pgfpathlineto{\pgfqpoint{0.818820in}{1.196792in}}%
\pgfpathlineto{\pgfqpoint{0.860954in}{1.207986in}}%
\pgfpathlineto{\pgfqpoint{0.870530in}{1.209480in}}%
\pgfpathlineto{\pgfqpoint{0.883936in}{1.212655in}}%
\pgfpathlineto{\pgfqpoint{0.899258in}{1.214008in}}%
\pgfpathlineto{\pgfqpoint{0.908834in}{1.214543in}}%
\pgfpathlineto{\pgfqpoint{0.914579in}{1.215715in}}%
\pgfpathlineto{\pgfqpoint{0.941391in}{1.218454in}}%
\pgfpathlineto{\pgfqpoint{0.956713in}{1.219640in}}%
\pgfpathlineto{\pgfqpoint{0.964373in}{1.220722in}}%
\pgfpathlineto{\pgfqpoint{0.995016in}{1.223496in}}%
\pgfpathlineto{\pgfqpoint{1.010338in}{1.225303in}}%
\pgfpathlineto{\pgfqpoint{1.276546in}{1.260355in}}%
\pgfpathlineto{\pgfqpoint{1.301444in}{1.262781in}}%
\pgfpathlineto{\pgfqpoint{1.309104in}{1.263265in}}%
\pgfpathlineto{\pgfqpoint{1.314850in}{1.263874in}}%
\pgfpathlineto{\pgfqpoint{1.318680in}{1.265400in}}%
\pgfpathlineto{\pgfqpoint{1.326341in}{1.266687in}}%
\pgfpathlineto{\pgfqpoint{1.347408in}{1.271336in}}%
\pgfpathlineto{\pgfqpoint{1.353153in}{1.272187in}}%
\pgfpathlineto{\pgfqpoint{1.356984in}{1.273188in}}%
\pgfpathlineto{\pgfqpoint{1.368475in}{1.275324in}}%
\pgfpathlineto{\pgfqpoint{1.379966in}{1.276812in}}%
\pgfpathlineto{\pgfqpoint{1.391457in}{1.278655in}}%
\pgfpathlineto{\pgfqpoint{1.406778in}{1.281795in}}%
\pgfpathlineto{\pgfqpoint{1.416354in}{1.282600in}}%
\pgfpathlineto{\pgfqpoint{1.420184in}{1.284273in}}%
\pgfpathlineto{\pgfqpoint{1.431675in}{1.286989in}}%
\pgfpathlineto{\pgfqpoint{1.454657in}{1.290085in}}%
\pgfpathlineto{\pgfqpoint{1.456573in}{1.291995in}}%
\pgfpathlineto{\pgfqpoint{1.462318in}{1.292473in}}%
\pgfpathlineto{\pgfqpoint{1.466148in}{1.293576in}}%
\pgfpathlineto{\pgfqpoint{1.475724in}{1.295512in}}%
\pgfpathlineto{\pgfqpoint{1.485300in}{1.297704in}}%
\pgfpathlineto{\pgfqpoint{1.491046in}{1.298449in}}%
\pgfpathlineto{\pgfqpoint{1.502537in}{1.304469in}}%
\pgfpathlineto{\pgfqpoint{1.515943in}{1.308966in}}%
\pgfpathlineto{\pgfqpoint{1.521688in}{1.311486in}}%
\pgfpathlineto{\pgfqpoint{1.550416in}{1.319829in}}%
\pgfpathlineto{\pgfqpoint{1.556161in}{1.322115in}}%
\pgfpathlineto{\pgfqpoint{1.561907in}{1.323577in}}%
\pgfpathlineto{\pgfqpoint{1.565737in}{1.325470in}}%
\pgfpathlineto{\pgfqpoint{1.567652in}{1.326031in}}%
\pgfpathlineto{\pgfqpoint{1.569568in}{1.327861in}}%
\pgfpathlineto{\pgfqpoint{1.577228in}{1.328459in}}%
\pgfpathlineto{\pgfqpoint{1.581059in}{1.330616in}}%
\pgfpathlineto{\pgfqpoint{1.586804in}{1.330921in}}%
\pgfpathlineto{\pgfqpoint{1.588719in}{1.332514in}}%
\pgfpathlineto{\pgfqpoint{1.590635in}{1.332541in}}%
\pgfpathlineto{\pgfqpoint{1.594465in}{1.336745in}}%
\pgfpathlineto{\pgfqpoint{1.598295in}{1.339926in}}%
\pgfpathlineto{\pgfqpoint{1.604041in}{1.342014in}}%
\pgfpathlineto{\pgfqpoint{1.605956in}{1.343043in}}%
\pgfpathlineto{\pgfqpoint{1.607871in}{1.346368in}}%
\pgfpathlineto{\pgfqpoint{1.617447in}{1.348989in}}%
\pgfpathlineto{\pgfqpoint{1.621277in}{1.351011in}}%
\pgfpathlineto{\pgfqpoint{1.623192in}{1.351528in}}%
\pgfpathlineto{\pgfqpoint{1.628938in}{1.355828in}}%
\pgfpathlineto{\pgfqpoint{1.630853in}{1.356072in}}%
\pgfpathlineto{\pgfqpoint{1.632768in}{1.359363in}}%
\pgfpathlineto{\pgfqpoint{1.638514in}{1.361024in}}%
\pgfpathlineto{\pgfqpoint{1.640429in}{1.366651in}}%
\pgfpathlineto{\pgfqpoint{1.646174in}{1.369896in}}%
\pgfpathlineto{\pgfqpoint{1.653835in}{1.371709in}}%
\pgfpathlineto{\pgfqpoint{1.655750in}{1.372445in}}%
\pgfpathlineto{\pgfqpoint{1.659581in}{1.376545in}}%
\pgfpathlineto{\pgfqpoint{1.661496in}{1.376664in}}%
\pgfpathlineto{\pgfqpoint{1.663411in}{1.378528in}}%
\pgfpathlineto{\pgfqpoint{1.665326in}{1.382384in}}%
\pgfpathlineto{\pgfqpoint{1.671072in}{1.385381in}}%
\pgfpathlineto{\pgfqpoint{1.672987in}{1.385605in}}%
\pgfpathlineto{\pgfqpoint{1.676817in}{1.388631in}}%
\pgfpathlineto{\pgfqpoint{1.678732in}{1.389252in}}%
\pgfpathlineto{\pgfqpoint{1.684478in}{1.394453in}}%
\pgfpathlineto{\pgfqpoint{1.686393in}{1.394889in}}%
\pgfpathlineto{\pgfqpoint{1.690223in}{1.398688in}}%
\pgfpathlineto{\pgfqpoint{1.697884in}{1.403132in}}%
\pgfpathlineto{\pgfqpoint{1.701714in}{1.409587in}}%
\pgfpathlineto{\pgfqpoint{1.713205in}{1.415145in}}%
\pgfpathlineto{\pgfqpoint{1.715121in}{1.416608in}}%
\pgfpathlineto{\pgfqpoint{1.717036in}{1.421963in}}%
\pgfpathlineto{\pgfqpoint{1.726612in}{1.424488in}}%
\pgfpathlineto{\pgfqpoint{1.730442in}{1.427041in}}%
\pgfpathlineto{\pgfqpoint{1.734272in}{1.428226in}}%
\pgfpathlineto{\pgfqpoint{1.738103in}{1.434547in}}%
\pgfpathlineto{\pgfqpoint{1.757254in}{1.441305in}}%
\pgfpathlineto{\pgfqpoint{1.761085in}{1.442517in}}%
\pgfpathlineto{\pgfqpoint{1.764915in}{1.443902in}}%
\pgfpathlineto{\pgfqpoint{1.768745in}{1.444849in}}%
\pgfpathlineto{\pgfqpoint{1.772576in}{1.449568in}}%
\pgfpathlineto{\pgfqpoint{1.782152in}{1.453292in}}%
\pgfpathlineto{\pgfqpoint{1.785982in}{1.456892in}}%
\pgfpathlineto{\pgfqpoint{1.791728in}{1.458672in}}%
\pgfpathlineto{\pgfqpoint{1.793643in}{1.460292in}}%
\pgfpathlineto{\pgfqpoint{1.797473in}{1.460309in}}%
\pgfpathlineto{\pgfqpoint{1.799388in}{1.463854in}}%
\pgfpathlineto{\pgfqpoint{1.803219in}{1.464567in}}%
\pgfpathlineto{\pgfqpoint{1.807049in}{1.464956in}}%
\pgfpathlineto{\pgfqpoint{1.810879in}{1.467354in}}%
\pgfpathlineto{\pgfqpoint{1.814710in}{1.475400in}}%
\pgfpathlineto{\pgfqpoint{1.818540in}{1.476345in}}%
\pgfpathlineto{\pgfqpoint{1.820455in}{1.479490in}}%
\pgfpathlineto{\pgfqpoint{1.831946in}{1.483182in}}%
\pgfpathlineto{\pgfqpoint{1.835776in}{1.486299in}}%
\pgfpathlineto{\pgfqpoint{1.837692in}{1.486766in}}%
\pgfpathlineto{\pgfqpoint{1.841522in}{1.489589in}}%
\pgfpathlineto{\pgfqpoint{1.843437in}{1.489910in}}%
\pgfpathlineto{\pgfqpoint{1.847267in}{1.493070in}}%
\pgfpathlineto{\pgfqpoint{1.860674in}{1.493814in}}%
\pgfpathlineto{\pgfqpoint{1.862589in}{1.495401in}}%
\pgfpathlineto{\pgfqpoint{1.868334in}{1.495891in}}%
\pgfpathlineto{\pgfqpoint{1.872165in}{1.497930in}}%
\pgfpathlineto{\pgfqpoint{1.875995in}{1.499961in}}%
\pgfpathlineto{\pgfqpoint{1.877910in}{1.505456in}}%
\pgfpathlineto{\pgfqpoint{1.885571in}{1.507597in}}%
\pgfpathlineto{\pgfqpoint{1.889401in}{1.511675in}}%
\pgfpathlineto{\pgfqpoint{1.904723in}{1.522836in}}%
\pgfpathlineto{\pgfqpoint{1.910468in}{1.531346in}}%
\pgfpathlineto{\pgfqpoint{1.914298in}{1.531408in}}%
\pgfpathlineto{\pgfqpoint{1.923874in}{1.537233in}}%
\pgfpathlineto{\pgfqpoint{1.925790in}{1.540900in}}%
\pgfpathlineto{\pgfqpoint{1.931535in}{1.543587in}}%
\pgfpathlineto{\pgfqpoint{1.935365in}{1.544622in}}%
\pgfpathlineto{\pgfqpoint{1.944941in}{1.549946in}}%
\pgfpathlineto{\pgfqpoint{1.948772in}{1.550939in}}%
\pgfpathlineto{\pgfqpoint{1.952602in}{1.551894in}}%
\pgfpathlineto{\pgfqpoint{1.956432in}{1.553405in}}%
\pgfpathlineto{\pgfqpoint{1.966008in}{1.557216in}}%
\pgfpathlineto{\pgfqpoint{1.969838in}{1.568010in}}%
\pgfpathlineto{\pgfqpoint{1.973669in}{1.573545in}}%
\pgfpathlineto{\pgfqpoint{1.977499in}{1.573994in}}%
\pgfpathlineto{\pgfqpoint{1.983245in}{1.575787in}}%
\pgfpathlineto{\pgfqpoint{1.985160in}{1.575964in}}%
\pgfpathlineto{\pgfqpoint{1.988990in}{1.578691in}}%
\pgfpathlineto{\pgfqpoint{1.996651in}{1.580364in}}%
\pgfpathlineto{\pgfqpoint{2.010057in}{1.585292in}}%
\pgfpathlineto{\pgfqpoint{2.013887in}{1.595369in}}%
\pgfpathlineto{\pgfqpoint{2.019633in}{1.598076in}}%
\pgfpathlineto{\pgfqpoint{2.025378in}{1.601973in}}%
\pgfpathlineto{\pgfqpoint{2.031124in}{1.603684in}}%
\pgfpathlineto{\pgfqpoint{2.040700in}{1.607022in}}%
\pgfpathlineto{\pgfqpoint{2.050276in}{1.620180in}}%
\pgfpathlineto{\pgfqpoint{2.054106in}{1.621879in}}%
\pgfpathlineto{\pgfqpoint{2.061767in}{1.622929in}}%
\pgfpathlineto{\pgfqpoint{2.065597in}{1.626878in}}%
\pgfpathlineto{\pgfqpoint{2.082834in}{1.631262in}}%
\pgfpathlineto{\pgfqpoint{2.086664in}{1.633537in}}%
\pgfpathlineto{\pgfqpoint{2.088579in}{1.633905in}}%
\pgfpathlineto{\pgfqpoint{2.090494in}{1.636687in}}%
\pgfpathlineto{\pgfqpoint{2.092409in}{1.641522in}}%
\pgfpathlineto{\pgfqpoint{2.098155in}{1.643104in}}%
\pgfpathlineto{\pgfqpoint{2.103900in}{1.647931in}}%
\pgfpathlineto{\pgfqpoint{2.105816in}{1.648707in}}%
\pgfpathlineto{\pgfqpoint{2.107731in}{1.652064in}}%
\pgfpathlineto{\pgfqpoint{2.113476in}{1.653067in}}%
\pgfpathlineto{\pgfqpoint{2.124967in}{1.657318in}}%
\pgfpathlineto{\pgfqpoint{2.126883in}{1.660554in}}%
\pgfpathlineto{\pgfqpoint{2.132628in}{1.663152in}}%
\pgfpathlineto{\pgfqpoint{2.136458in}{1.667398in}}%
\pgfpathlineto{\pgfqpoint{2.146034in}{1.672284in}}%
\pgfpathlineto{\pgfqpoint{2.153695in}{1.677501in}}%
\pgfpathlineto{\pgfqpoint{2.165186in}{1.681182in}}%
\pgfpathlineto{\pgfqpoint{2.174762in}{1.687722in}}%
\pgfpathlineto{\pgfqpoint{2.182422in}{1.698022in}}%
\pgfpathlineto{\pgfqpoint{2.186253in}{1.699619in}}%
\pgfpathlineto{\pgfqpoint{2.188168in}{1.702234in}}%
\pgfpathlineto{\pgfqpoint{2.191998in}{1.702478in}}%
\pgfpathlineto{\pgfqpoint{2.203489in}{1.713686in}}%
\pgfpathlineto{\pgfqpoint{2.207320in}{1.714462in}}%
\pgfpathlineto{\pgfqpoint{2.209235in}{1.719630in}}%
\pgfpathlineto{\pgfqpoint{2.216896in}{1.722760in}}%
\pgfpathlineto{\pgfqpoint{2.220726in}{1.726786in}}%
\pgfpathlineto{\pgfqpoint{2.222641in}{1.727246in}}%
\pgfpathlineto{\pgfqpoint{2.226471in}{1.730974in}}%
\pgfpathlineto{\pgfqpoint{2.230302in}{1.731978in}}%
\pgfpathlineto{\pgfqpoint{2.237962in}{1.734587in}}%
\pgfpathlineto{\pgfqpoint{2.241793in}{1.736091in}}%
\pgfpathlineto{\pgfqpoint{2.247538in}{1.739614in}}%
\pgfpathlineto{\pgfqpoint{2.253284in}{1.741282in}}%
\pgfpathlineto{\pgfqpoint{2.262860in}{1.746025in}}%
\pgfpathlineto{\pgfqpoint{2.268605in}{1.747129in}}%
\pgfpathlineto{\pgfqpoint{2.272436in}{1.749142in}}%
\pgfpathlineto{\pgfqpoint{2.278181in}{1.759746in}}%
\pgfpathlineto{\pgfqpoint{2.285842in}{1.765681in}}%
\pgfpathlineto{\pgfqpoint{2.291587in}{1.766759in}}%
\pgfpathlineto{\pgfqpoint{2.293502in}{1.767371in}}%
\pgfpathlineto{\pgfqpoint{2.295418in}{1.772118in}}%
\pgfpathlineto{\pgfqpoint{2.301163in}{1.775959in}}%
\pgfpathlineto{\pgfqpoint{2.303078in}{1.776410in}}%
\pgfpathlineto{\pgfqpoint{2.306909in}{1.778441in}}%
\pgfpathlineto{\pgfqpoint{2.312654in}{1.780935in}}%
\pgfpathlineto{\pgfqpoint{2.320315in}{1.782097in}}%
\pgfpathlineto{\pgfqpoint{2.327975in}{1.786705in}}%
\pgfpathlineto{\pgfqpoint{2.331806in}{1.788304in}}%
\pgfpathlineto{\pgfqpoint{2.333721in}{1.789322in}}%
\pgfpathlineto{\pgfqpoint{2.335636in}{1.791670in}}%
\pgfpathlineto{\pgfqpoint{2.337551in}{1.792129in}}%
\pgfpathlineto{\pgfqpoint{2.339467in}{1.796183in}}%
\pgfpathlineto{\pgfqpoint{2.343297in}{1.797173in}}%
\pgfpathlineto{\pgfqpoint{2.345212in}{1.799983in}}%
\pgfpathlineto{\pgfqpoint{2.347127in}{1.800002in}}%
\pgfpathlineto{\pgfqpoint{2.349042in}{1.803257in}}%
\pgfpathlineto{\pgfqpoint{2.358618in}{1.807884in}}%
\pgfpathlineto{\pgfqpoint{2.360533in}{1.809997in}}%
\pgfpathlineto{\pgfqpoint{2.362449in}{1.810025in}}%
\pgfpathlineto{\pgfqpoint{2.366279in}{1.815338in}}%
\pgfpathlineto{\pgfqpoint{2.373940in}{1.816209in}}%
\pgfpathlineto{\pgfqpoint{2.377770in}{1.818488in}}%
\pgfpathlineto{\pgfqpoint{2.381600in}{1.823308in}}%
\pgfpathlineto{\pgfqpoint{2.385431in}{1.823974in}}%
\pgfpathlineto{\pgfqpoint{2.387346in}{1.826535in}}%
\pgfpathlineto{\pgfqpoint{2.387346in}{1.826535in}}%
\pgfusepath{stroke}%
\end{pgfscope}%
\begin{pgfscope}%
\pgfpathrectangle{\pgfqpoint{0.694334in}{0.523557in}}{\pgfqpoint{3.830343in}{1.302977in}}%
\pgfusepath{clip}%
\pgfsetrectcap%
\pgfsetroundjoin%
\pgfsetlinewidth{1.003750pt}%
\definecolor{currentstroke}{rgb}{0.000000,0.000000,0.376471}%
\pgfsetstrokecolor{currentstroke}%
\pgfsetdash{}{0pt}%
\pgfpathmoveto{\pgfqpoint{0.694334in}{0.599366in}}%
\pgfpathlineto{\pgfqpoint{0.696249in}{0.616722in}}%
\pgfpathlineto{\pgfqpoint{0.698165in}{0.623343in}}%
\pgfpathlineto{\pgfqpoint{0.700080in}{0.624942in}}%
\pgfpathlineto{\pgfqpoint{0.701995in}{0.628627in}}%
\pgfpathlineto{\pgfqpoint{0.707741in}{0.675393in}}%
\pgfpathlineto{\pgfqpoint{0.709656in}{0.679304in}}%
\pgfpathlineto{\pgfqpoint{0.713486in}{0.680868in}}%
\pgfpathlineto{\pgfqpoint{0.715401in}{0.684930in}}%
\pgfpathlineto{\pgfqpoint{0.717316in}{0.685863in}}%
\pgfpathlineto{\pgfqpoint{0.721147in}{0.699237in}}%
\pgfpathlineto{\pgfqpoint{0.724977in}{0.724415in}}%
\pgfpathlineto{\pgfqpoint{0.728807in}{0.731356in}}%
\pgfpathlineto{\pgfqpoint{0.730723in}{0.731663in}}%
\pgfpathlineto{\pgfqpoint{0.732638in}{0.733313in}}%
\pgfpathlineto{\pgfqpoint{0.736468in}{0.745375in}}%
\pgfpathlineto{\pgfqpoint{0.749874in}{0.754558in}}%
\pgfpathlineto{\pgfqpoint{0.757535in}{0.755842in}}%
\pgfpathlineto{\pgfqpoint{0.765196in}{0.759516in}}%
\pgfpathlineto{\pgfqpoint{0.769026in}{0.761092in}}%
\pgfpathlineto{\pgfqpoint{0.778602in}{0.763699in}}%
\pgfpathlineto{\pgfqpoint{0.780517in}{0.765949in}}%
\pgfpathlineto{\pgfqpoint{0.782432in}{0.766314in}}%
\pgfpathlineto{\pgfqpoint{0.784347in}{0.770570in}}%
\pgfpathlineto{\pgfqpoint{0.795838in}{0.772555in}}%
\pgfpathlineto{\pgfqpoint{0.834142in}{0.780708in}}%
\pgfpathlineto{\pgfqpoint{0.837972in}{0.781987in}}%
\pgfpathlineto{\pgfqpoint{0.841803in}{0.783419in}}%
\pgfpathlineto{\pgfqpoint{0.843718in}{0.784883in}}%
\pgfpathlineto{\pgfqpoint{0.847548in}{0.785885in}}%
\pgfpathlineto{\pgfqpoint{0.853294in}{0.788128in}}%
\pgfpathlineto{\pgfqpoint{0.855209in}{0.791418in}}%
\pgfpathlineto{\pgfqpoint{0.857124in}{0.791448in}}%
\pgfpathlineto{\pgfqpoint{0.860954in}{0.793652in}}%
\pgfpathlineto{\pgfqpoint{0.866700in}{0.795597in}}%
\pgfpathlineto{\pgfqpoint{0.870530in}{0.798359in}}%
\pgfpathlineto{\pgfqpoint{0.872445in}{0.798803in}}%
\pgfpathlineto{\pgfqpoint{0.876276in}{0.802324in}}%
\pgfpathlineto{\pgfqpoint{0.885851in}{0.804749in}}%
\pgfpathlineto{\pgfqpoint{0.903088in}{0.810705in}}%
\pgfpathlineto{\pgfqpoint{0.905003in}{0.813318in}}%
\pgfpathlineto{\pgfqpoint{0.912664in}{0.815343in}}%
\pgfpathlineto{\pgfqpoint{0.914579in}{0.817710in}}%
\pgfpathlineto{\pgfqpoint{0.918409in}{0.818198in}}%
\pgfpathlineto{\pgfqpoint{0.920325in}{0.820676in}}%
\pgfpathlineto{\pgfqpoint{0.926070in}{0.821629in}}%
\pgfpathlineto{\pgfqpoint{0.929900in}{0.823259in}}%
\pgfpathlineto{\pgfqpoint{0.939476in}{0.826427in}}%
\pgfpathlineto{\pgfqpoint{0.943307in}{0.830369in}}%
\pgfpathlineto{\pgfqpoint{0.952882in}{0.834160in}}%
\pgfpathlineto{\pgfqpoint{0.964373in}{0.836130in}}%
\pgfpathlineto{\pgfqpoint{0.966289in}{0.837963in}}%
\pgfpathlineto{\pgfqpoint{0.972034in}{0.838534in}}%
\pgfpathlineto{\pgfqpoint{0.975864in}{0.841301in}}%
\pgfpathlineto{\pgfqpoint{0.977780in}{0.841407in}}%
\pgfpathlineto{\pgfqpoint{0.979695in}{0.843140in}}%
\pgfpathlineto{\pgfqpoint{0.985440in}{0.844131in}}%
\pgfpathlineto{\pgfqpoint{0.989271in}{0.845658in}}%
\pgfpathlineto{\pgfqpoint{1.008422in}{0.852766in}}%
\pgfpathlineto{\pgfqpoint{1.021829in}{0.854652in}}%
\pgfpathlineto{\pgfqpoint{1.025659in}{0.856194in}}%
\pgfpathlineto{\pgfqpoint{1.042895in}{0.858909in}}%
\pgfpathlineto{\pgfqpoint{1.056302in}{0.860634in}}%
\pgfpathlineto{\pgfqpoint{1.060132in}{0.862067in}}%
\pgfpathlineto{\pgfqpoint{1.088860in}{0.868666in}}%
\pgfpathlineto{\pgfqpoint{1.100351in}{0.871832in}}%
\pgfpathlineto{\pgfqpoint{1.108011in}{0.873189in}}%
\pgfpathlineto{\pgfqpoint{1.111842in}{0.873923in}}%
\pgfpathlineto{\pgfqpoint{1.121418in}{0.875284in}}%
\pgfpathlineto{\pgfqpoint{1.130993in}{0.877922in}}%
\pgfpathlineto{\pgfqpoint{1.140569in}{0.879026in}}%
\pgfpathlineto{\pgfqpoint{1.146315in}{0.880442in}}%
\pgfpathlineto{\pgfqpoint{1.152060in}{0.881287in}}%
\pgfpathlineto{\pgfqpoint{1.159721in}{0.883155in}}%
\pgfpathlineto{\pgfqpoint{1.199940in}{0.889064in}}%
\pgfpathlineto{\pgfqpoint{1.213346in}{0.889892in}}%
\pgfpathlineto{\pgfqpoint{1.217176in}{0.891013in}}%
\pgfpathlineto{\pgfqpoint{1.228667in}{0.892147in}}%
\pgfpathlineto{\pgfqpoint{1.266971in}{0.897178in}}%
\pgfpathlineto{\pgfqpoint{1.280377in}{0.898879in}}%
\pgfpathlineto{\pgfqpoint{1.286122in}{0.899993in}}%
\pgfpathlineto{\pgfqpoint{1.299528in}{0.902931in}}%
\pgfpathlineto{\pgfqpoint{1.312935in}{0.905879in}}%
\pgfpathlineto{\pgfqpoint{1.316765in}{0.906863in}}%
\pgfpathlineto{\pgfqpoint{1.322511in}{0.908110in}}%
\pgfpathlineto{\pgfqpoint{1.349323in}{0.911752in}}%
\pgfpathlineto{\pgfqpoint{1.355068in}{0.912981in}}%
\pgfpathlineto{\pgfqpoint{1.387626in}{0.917568in}}%
\pgfpathlineto{\pgfqpoint{1.454657in}{0.922597in}}%
\pgfpathlineto{\pgfqpoint{1.550416in}{0.933264in}}%
\pgfpathlineto{\pgfqpoint{1.565737in}{0.935599in}}%
\pgfpathlineto{\pgfqpoint{1.575313in}{0.936435in}}%
\pgfpathlineto{\pgfqpoint{1.579143in}{0.937425in}}%
\pgfpathlineto{\pgfqpoint{1.588719in}{0.938693in}}%
\pgfpathlineto{\pgfqpoint{1.607871in}{0.940878in}}%
\pgfpathlineto{\pgfqpoint{1.625108in}{0.944931in}}%
\pgfpathlineto{\pgfqpoint{1.646174in}{0.947189in}}%
\pgfpathlineto{\pgfqpoint{1.650005in}{0.948491in}}%
\pgfpathlineto{\pgfqpoint{1.653835in}{0.950144in}}%
\pgfpathlineto{\pgfqpoint{1.674902in}{0.953346in}}%
\pgfpathlineto{\pgfqpoint{1.686393in}{0.954526in}}%
\pgfpathlineto{\pgfqpoint{1.695969in}{0.955641in}}%
\pgfpathlineto{\pgfqpoint{1.703630in}{0.956799in}}%
\pgfpathlineto{\pgfqpoint{1.722781in}{0.959828in}}%
\pgfpathlineto{\pgfqpoint{1.740018in}{0.964303in}}%
\pgfpathlineto{\pgfqpoint{1.741933in}{0.964532in}}%
\pgfpathlineto{\pgfqpoint{1.743848in}{0.967466in}}%
\pgfpathlineto{\pgfqpoint{1.755339in}{0.969525in}}%
\pgfpathlineto{\pgfqpoint{1.774491in}{0.971453in}}%
\pgfpathlineto{\pgfqpoint{1.778321in}{0.973937in}}%
\pgfpathlineto{\pgfqpoint{1.780236in}{0.974197in}}%
\pgfpathlineto{\pgfqpoint{1.785982in}{0.977752in}}%
\pgfpathlineto{\pgfqpoint{1.797473in}{0.980035in}}%
\pgfpathlineto{\pgfqpoint{1.820455in}{0.983071in}}%
\pgfpathlineto{\pgfqpoint{1.824285in}{0.983797in}}%
\pgfpathlineto{\pgfqpoint{1.833861in}{0.984717in}}%
\pgfpathlineto{\pgfqpoint{1.837692in}{0.986598in}}%
\pgfpathlineto{\pgfqpoint{1.860674in}{0.989409in}}%
\pgfpathlineto{\pgfqpoint{1.868334in}{0.991345in}}%
\pgfpathlineto{\pgfqpoint{1.877910in}{0.992458in}}%
\pgfpathlineto{\pgfqpoint{1.900892in}{0.997358in}}%
\pgfpathlineto{\pgfqpoint{1.904723in}{0.998557in}}%
\pgfpathlineto{\pgfqpoint{1.918129in}{1.001810in}}%
\pgfpathlineto{\pgfqpoint{1.929620in}{1.003261in}}%
\pgfpathlineto{\pgfqpoint{1.941111in}{1.004191in}}%
\pgfpathlineto{\pgfqpoint{1.954517in}{1.006170in}}%
\pgfpathlineto{\pgfqpoint{1.960263in}{1.007225in}}%
\pgfpathlineto{\pgfqpoint{1.964093in}{1.009090in}}%
\pgfpathlineto{\pgfqpoint{1.992821in}{1.012236in}}%
\pgfpathlineto{\pgfqpoint{1.996651in}{1.013673in}}%
\pgfpathlineto{\pgfqpoint{2.021548in}{1.017232in}}%
\pgfpathlineto{\pgfqpoint{2.027294in}{1.018388in}}%
\pgfpathlineto{\pgfqpoint{2.036869in}{1.019759in}}%
\pgfpathlineto{\pgfqpoint{2.038785in}{1.022034in}}%
\pgfpathlineto{\pgfqpoint{2.140289in}{1.034822in}}%
\pgfpathlineto{\pgfqpoint{2.149865in}{1.036153in}}%
\pgfpathlineto{\pgfqpoint{2.155610in}{1.036768in}}%
\pgfpathlineto{\pgfqpoint{2.161356in}{1.038141in}}%
\pgfpathlineto{\pgfqpoint{2.172847in}{1.039984in}}%
\pgfpathlineto{\pgfqpoint{2.188168in}{1.042432in}}%
\pgfpathlineto{\pgfqpoint{2.193914in}{1.044353in}}%
\pgfpathlineto{\pgfqpoint{2.203489in}{1.045628in}}%
\pgfpathlineto{\pgfqpoint{2.222641in}{1.049509in}}%
\pgfpathlineto{\pgfqpoint{2.228387in}{1.050142in}}%
\pgfpathlineto{\pgfqpoint{2.297333in}{1.061340in}}%
\pgfpathlineto{\pgfqpoint{2.301163in}{1.062700in}}%
\pgfpathlineto{\pgfqpoint{2.303078in}{1.062824in}}%
\pgfpathlineto{\pgfqpoint{2.304993in}{1.064582in}}%
\pgfpathlineto{\pgfqpoint{2.314569in}{1.065431in}}%
\pgfpathlineto{\pgfqpoint{2.318400in}{1.066746in}}%
\pgfpathlineto{\pgfqpoint{2.324145in}{1.067012in}}%
\pgfpathlineto{\pgfqpoint{2.327975in}{1.068510in}}%
\pgfpathlineto{\pgfqpoint{2.339467in}{1.069371in}}%
\pgfpathlineto{\pgfqpoint{2.358618in}{1.073888in}}%
\pgfpathlineto{\pgfqpoint{2.387346in}{1.079643in}}%
\pgfpathlineto{\pgfqpoint{2.393091in}{1.080477in}}%
\pgfpathlineto{\pgfqpoint{2.398837in}{1.082285in}}%
\pgfpathlineto{\pgfqpoint{2.402667in}{1.083946in}}%
\pgfpathlineto{\pgfqpoint{2.406498in}{1.084690in}}%
\pgfpathlineto{\pgfqpoint{2.416073in}{1.086223in}}%
\pgfpathlineto{\pgfqpoint{2.431395in}{1.089847in}}%
\pgfpathlineto{\pgfqpoint{2.433310in}{1.089886in}}%
\pgfpathlineto{\pgfqpoint{2.435225in}{1.091309in}}%
\pgfpathlineto{\pgfqpoint{2.439055in}{1.091773in}}%
\pgfpathlineto{\pgfqpoint{2.442886in}{1.092888in}}%
\pgfpathlineto{\pgfqpoint{2.450546in}{1.093513in}}%
\pgfpathlineto{\pgfqpoint{2.456292in}{1.095811in}}%
\pgfpathlineto{\pgfqpoint{2.465868in}{1.099734in}}%
\pgfpathlineto{\pgfqpoint{2.483104in}{1.104601in}}%
\pgfpathlineto{\pgfqpoint{2.496511in}{1.107612in}}%
\pgfpathlineto{\pgfqpoint{2.500341in}{1.112103in}}%
\pgfpathlineto{\pgfqpoint{2.527153in}{1.117385in}}%
\pgfpathlineto{\pgfqpoint{2.546305in}{1.123202in}}%
\pgfpathlineto{\pgfqpoint{2.553966in}{1.124777in}}%
\pgfpathlineto{\pgfqpoint{2.559711in}{1.127149in}}%
\pgfpathlineto{\pgfqpoint{2.567372in}{1.128651in}}%
\pgfpathlineto{\pgfqpoint{2.569287in}{1.132174in}}%
\pgfpathlineto{\pgfqpoint{2.571202in}{1.132231in}}%
\pgfpathlineto{\pgfqpoint{2.575033in}{1.133807in}}%
\pgfpathlineto{\pgfqpoint{2.582693in}{1.135698in}}%
\pgfpathlineto{\pgfqpoint{2.584608in}{1.136519in}}%
\pgfpathlineto{\pgfqpoint{2.586524in}{1.139103in}}%
\pgfpathlineto{\pgfqpoint{2.598015in}{1.141933in}}%
\pgfpathlineto{\pgfqpoint{2.599930in}{1.145281in}}%
\pgfpathlineto{\pgfqpoint{2.609506in}{1.148967in}}%
\pgfpathlineto{\pgfqpoint{2.615251in}{1.151012in}}%
\pgfpathlineto{\pgfqpoint{2.632488in}{1.160730in}}%
\pgfpathlineto{\pgfqpoint{2.634403in}{1.162259in}}%
\pgfpathlineto{\pgfqpoint{2.636318in}{1.162292in}}%
\pgfpathlineto{\pgfqpoint{2.640148in}{1.164127in}}%
\pgfpathlineto{\pgfqpoint{2.642064in}{1.164768in}}%
\pgfpathlineto{\pgfqpoint{2.645894in}{1.171670in}}%
\pgfpathlineto{\pgfqpoint{2.649724in}{1.173699in}}%
\pgfpathlineto{\pgfqpoint{2.651639in}{1.177104in}}%
\pgfpathlineto{\pgfqpoint{2.655470in}{1.178221in}}%
\pgfpathlineto{\pgfqpoint{2.661215in}{1.181616in}}%
\pgfpathlineto{\pgfqpoint{2.665046in}{1.182529in}}%
\pgfpathlineto{\pgfqpoint{2.668876in}{1.183959in}}%
\pgfpathlineto{\pgfqpoint{2.672706in}{1.185299in}}%
\pgfpathlineto{\pgfqpoint{2.674622in}{1.189981in}}%
\pgfpathlineto{\pgfqpoint{2.678452in}{1.191422in}}%
\pgfpathlineto{\pgfqpoint{2.693773in}{1.196168in}}%
\pgfpathlineto{\pgfqpoint{2.697604in}{1.201777in}}%
\pgfpathlineto{\pgfqpoint{2.699519in}{1.203232in}}%
\pgfpathlineto{\pgfqpoint{2.709095in}{1.204549in}}%
\pgfpathlineto{\pgfqpoint{2.716755in}{1.206028in}}%
\pgfpathlineto{\pgfqpoint{2.720586in}{1.207480in}}%
\pgfpathlineto{\pgfqpoint{2.724416in}{1.207804in}}%
\pgfpathlineto{\pgfqpoint{2.737822in}{1.216964in}}%
\pgfpathlineto{\pgfqpoint{2.741653in}{1.217531in}}%
\pgfpathlineto{\pgfqpoint{2.743568in}{1.220838in}}%
\pgfpathlineto{\pgfqpoint{2.745483in}{1.221483in}}%
\pgfpathlineto{\pgfqpoint{2.747398in}{1.224490in}}%
\pgfpathlineto{\pgfqpoint{2.755059in}{1.226493in}}%
\pgfpathlineto{\pgfqpoint{2.756974in}{1.228797in}}%
\pgfpathlineto{\pgfqpoint{2.758889in}{1.233639in}}%
\pgfpathlineto{\pgfqpoint{2.762719in}{1.234828in}}%
\pgfpathlineto{\pgfqpoint{2.766550in}{1.236943in}}%
\pgfpathlineto{\pgfqpoint{2.770380in}{1.237936in}}%
\pgfpathlineto{\pgfqpoint{2.772295in}{1.239920in}}%
\pgfpathlineto{\pgfqpoint{2.774210in}{1.239988in}}%
\pgfpathlineto{\pgfqpoint{2.776126in}{1.242697in}}%
\pgfpathlineto{\pgfqpoint{2.778041in}{1.242864in}}%
\pgfpathlineto{\pgfqpoint{2.785701in}{1.248064in}}%
\pgfpathlineto{\pgfqpoint{2.797192in}{1.252870in}}%
\pgfpathlineto{\pgfqpoint{2.806768in}{1.254761in}}%
\pgfpathlineto{\pgfqpoint{2.808684in}{1.258943in}}%
\pgfpathlineto{\pgfqpoint{2.812514in}{1.260611in}}%
\pgfpathlineto{\pgfqpoint{2.820175in}{1.268660in}}%
\pgfpathlineto{\pgfqpoint{2.822090in}{1.275625in}}%
\pgfpathlineto{\pgfqpoint{2.831666in}{1.277604in}}%
\pgfpathlineto{\pgfqpoint{2.833581in}{1.282838in}}%
\pgfpathlineto{\pgfqpoint{2.839326in}{1.286615in}}%
\pgfpathlineto{\pgfqpoint{2.843157in}{1.288602in}}%
\pgfpathlineto{\pgfqpoint{2.845072in}{1.289591in}}%
\pgfpathlineto{\pgfqpoint{2.850817in}{1.297029in}}%
\pgfpathlineto{\pgfqpoint{2.854648in}{1.299721in}}%
\pgfpathlineto{\pgfqpoint{2.856563in}{1.300402in}}%
\pgfpathlineto{\pgfqpoint{2.862308in}{1.305160in}}%
\pgfpathlineto{\pgfqpoint{2.866139in}{1.306455in}}%
\pgfpathlineto{\pgfqpoint{2.871884in}{1.310414in}}%
\pgfpathlineto{\pgfqpoint{2.875715in}{1.314566in}}%
\pgfpathlineto{\pgfqpoint{2.877630in}{1.315000in}}%
\pgfpathlineto{\pgfqpoint{2.885290in}{1.327105in}}%
\pgfpathlineto{\pgfqpoint{2.889121in}{1.327696in}}%
\pgfpathlineto{\pgfqpoint{2.891036in}{1.332498in}}%
\pgfpathlineto{\pgfqpoint{2.892951in}{1.333169in}}%
\pgfpathlineto{\pgfqpoint{2.896781in}{1.335924in}}%
\pgfpathlineto{\pgfqpoint{2.898697in}{1.341167in}}%
\pgfpathlineto{\pgfqpoint{2.906357in}{1.344628in}}%
\pgfpathlineto{\pgfqpoint{2.908272in}{1.351538in}}%
\pgfpathlineto{\pgfqpoint{2.910188in}{1.352266in}}%
\pgfpathlineto{\pgfqpoint{2.915933in}{1.363609in}}%
\pgfpathlineto{\pgfqpoint{2.919763in}{1.365061in}}%
\pgfpathlineto{\pgfqpoint{2.921679in}{1.373283in}}%
\pgfpathlineto{\pgfqpoint{2.929339in}{1.380897in}}%
\pgfpathlineto{\pgfqpoint{2.931254in}{1.381325in}}%
\pgfpathlineto{\pgfqpoint{2.933170in}{1.387435in}}%
\pgfpathlineto{\pgfqpoint{2.935085in}{1.387793in}}%
\pgfpathlineto{\pgfqpoint{2.937000in}{1.393773in}}%
\pgfpathlineto{\pgfqpoint{2.938915in}{1.394062in}}%
\pgfpathlineto{\pgfqpoint{2.940830in}{1.401763in}}%
\pgfpathlineto{\pgfqpoint{2.942746in}{1.402862in}}%
\pgfpathlineto{\pgfqpoint{2.944661in}{1.419801in}}%
\pgfpathlineto{\pgfqpoint{2.948491in}{1.431082in}}%
\pgfpathlineto{\pgfqpoint{2.954237in}{1.443016in}}%
\pgfpathlineto{\pgfqpoint{2.956152in}{1.450338in}}%
\pgfpathlineto{\pgfqpoint{2.958067in}{1.471589in}}%
\pgfpathlineto{\pgfqpoint{2.963812in}{1.493713in}}%
\pgfpathlineto{\pgfqpoint{2.967643in}{1.504276in}}%
\pgfpathlineto{\pgfqpoint{2.971473in}{1.540108in}}%
\pgfpathlineto{\pgfqpoint{2.973388in}{1.541925in}}%
\pgfpathlineto{\pgfqpoint{2.975303in}{1.546393in}}%
\pgfpathlineto{\pgfqpoint{2.977219in}{1.558329in}}%
\pgfpathlineto{\pgfqpoint{2.979134in}{1.561688in}}%
\pgfpathlineto{\pgfqpoint{2.981049in}{1.569034in}}%
\pgfpathlineto{\pgfqpoint{2.982964in}{1.597494in}}%
\pgfpathlineto{\pgfqpoint{2.984879in}{1.602936in}}%
\pgfpathlineto{\pgfqpoint{2.986794in}{1.622419in}}%
\pgfpathlineto{\pgfqpoint{2.992540in}{1.641153in}}%
\pgfpathlineto{\pgfqpoint{2.996370in}{1.651511in}}%
\pgfpathlineto{\pgfqpoint{3.000201in}{1.657346in}}%
\pgfpathlineto{\pgfqpoint{3.004031in}{1.665559in}}%
\pgfpathlineto{\pgfqpoint{3.005946in}{1.668235in}}%
\pgfpathlineto{\pgfqpoint{3.009777in}{1.700837in}}%
\pgfpathlineto{\pgfqpoint{3.013607in}{1.713141in}}%
\pgfpathlineto{\pgfqpoint{3.017437in}{1.723441in}}%
\pgfpathlineto{\pgfqpoint{3.019352in}{1.728682in}}%
\pgfpathlineto{\pgfqpoint{3.021268in}{1.730441in}}%
\pgfpathlineto{\pgfqpoint{3.025098in}{1.753429in}}%
\pgfpathlineto{\pgfqpoint{3.027013in}{1.794188in}}%
\pgfpathlineto{\pgfqpoint{3.028928in}{1.794719in}}%
\pgfpathlineto{\pgfqpoint{3.030843in}{1.797123in}}%
\pgfpathlineto{\pgfqpoint{3.032759in}{1.826535in}}%
\pgfpathlineto{\pgfqpoint{3.032759in}{1.826535in}}%
\pgfusepath{stroke}%
\end{pgfscope}%
\begin{pgfscope}%
\pgfpathrectangle{\pgfqpoint{0.694334in}{0.523557in}}{\pgfqpoint{3.830343in}{1.302977in}}%
\pgfusepath{clip}%
\pgfsetrectcap%
\pgfsetroundjoin%
\pgfsetlinewidth{1.003750pt}%
\definecolor{currentstroke}{rgb}{0.564706,0.564706,1.000000}%
\pgfsetstrokecolor{currentstroke}%
\pgfsetdash{}{0pt}%
\pgfpathmoveto{\pgfqpoint{0.694334in}{0.703806in}}%
\pgfpathlineto{\pgfqpoint{0.696249in}{0.737312in}}%
\pgfpathlineto{\pgfqpoint{0.698165in}{0.740239in}}%
\pgfpathlineto{\pgfqpoint{0.700080in}{0.745465in}}%
\pgfpathlineto{\pgfqpoint{0.703910in}{0.746981in}}%
\pgfpathlineto{\pgfqpoint{0.705825in}{0.754666in}}%
\pgfpathlineto{\pgfqpoint{0.713486in}{0.757532in}}%
\pgfpathlineto{\pgfqpoint{0.719232in}{0.759770in}}%
\pgfpathlineto{\pgfqpoint{0.724977in}{0.761277in}}%
\pgfpathlineto{\pgfqpoint{0.728807in}{0.761602in}}%
\pgfpathlineto{\pgfqpoint{0.732638in}{0.762857in}}%
\pgfpathlineto{\pgfqpoint{0.736468in}{0.763733in}}%
\pgfpathlineto{\pgfqpoint{0.740298in}{0.766512in}}%
\pgfpathlineto{\pgfqpoint{0.744129in}{0.768190in}}%
\pgfpathlineto{\pgfqpoint{0.747959in}{0.768811in}}%
\pgfpathlineto{\pgfqpoint{0.757535in}{0.769539in}}%
\pgfpathlineto{\pgfqpoint{0.763280in}{0.771747in}}%
\pgfpathlineto{\pgfqpoint{0.769026in}{0.774631in}}%
\pgfpathlineto{\pgfqpoint{0.770941in}{0.774736in}}%
\pgfpathlineto{\pgfqpoint{0.772856in}{0.777195in}}%
\pgfpathlineto{\pgfqpoint{0.780517in}{0.779321in}}%
\pgfpathlineto{\pgfqpoint{0.790093in}{0.782434in}}%
\pgfpathlineto{\pgfqpoint{0.793923in}{0.784767in}}%
\pgfpathlineto{\pgfqpoint{0.795838in}{0.784811in}}%
\pgfpathlineto{\pgfqpoint{0.799669in}{0.788278in}}%
\pgfpathlineto{\pgfqpoint{0.805414in}{0.789452in}}%
\pgfpathlineto{\pgfqpoint{0.809245in}{0.791820in}}%
\pgfpathlineto{\pgfqpoint{0.816905in}{0.793698in}}%
\pgfpathlineto{\pgfqpoint{0.818820in}{0.795425in}}%
\pgfpathlineto{\pgfqpoint{0.824566in}{0.796047in}}%
\pgfpathlineto{\pgfqpoint{0.832227in}{0.799687in}}%
\pgfpathlineto{\pgfqpoint{0.841803in}{0.802601in}}%
\pgfpathlineto{\pgfqpoint{0.845633in}{0.803979in}}%
\pgfpathlineto{\pgfqpoint{0.847548in}{0.803993in}}%
\pgfpathlineto{\pgfqpoint{0.851378in}{0.807098in}}%
\pgfpathlineto{\pgfqpoint{0.862869in}{0.808678in}}%
\pgfpathlineto{\pgfqpoint{0.866700in}{0.810160in}}%
\pgfpathlineto{\pgfqpoint{0.874360in}{0.812265in}}%
\pgfpathlineto{\pgfqpoint{0.880106in}{0.813507in}}%
\pgfpathlineto{\pgfqpoint{0.882021in}{0.813875in}}%
\pgfpathlineto{\pgfqpoint{0.887767in}{0.818447in}}%
\pgfpathlineto{\pgfqpoint{0.891597in}{0.819748in}}%
\pgfpathlineto{\pgfqpoint{0.906918in}{0.828475in}}%
\pgfpathlineto{\pgfqpoint{0.908834in}{0.828597in}}%
\pgfpathlineto{\pgfqpoint{0.914579in}{0.834186in}}%
\pgfpathlineto{\pgfqpoint{0.920325in}{0.837295in}}%
\pgfpathlineto{\pgfqpoint{0.924155in}{0.839796in}}%
\pgfpathlineto{\pgfqpoint{0.927985in}{0.840826in}}%
\pgfpathlineto{\pgfqpoint{0.929900in}{0.842079in}}%
\pgfpathlineto{\pgfqpoint{0.935646in}{0.849780in}}%
\pgfpathlineto{\pgfqpoint{0.949052in}{0.854951in}}%
\pgfpathlineto{\pgfqpoint{0.954798in}{0.855791in}}%
\pgfpathlineto{\pgfqpoint{0.962458in}{0.861168in}}%
\pgfpathlineto{\pgfqpoint{0.966289in}{0.861769in}}%
\pgfpathlineto{\pgfqpoint{0.968204in}{0.864003in}}%
\pgfpathlineto{\pgfqpoint{0.973949in}{0.866159in}}%
\pgfpathlineto{\pgfqpoint{0.979695in}{0.867453in}}%
\pgfpathlineto{\pgfqpoint{0.985440in}{0.869988in}}%
\pgfpathlineto{\pgfqpoint{0.987356in}{0.873680in}}%
\pgfpathlineto{\pgfqpoint{0.989271in}{0.873771in}}%
\pgfpathlineto{\pgfqpoint{0.993101in}{0.875298in}}%
\pgfpathlineto{\pgfqpoint{1.023744in}{0.880894in}}%
\pgfpathlineto{\pgfqpoint{1.029489in}{0.883667in}}%
\pgfpathlineto{\pgfqpoint{1.033320in}{0.884552in}}%
\pgfpathlineto{\pgfqpoint{1.037150in}{0.885936in}}%
\pgfpathlineto{\pgfqpoint{1.040980in}{0.887013in}}%
\pgfpathlineto{\pgfqpoint{1.044811in}{0.887763in}}%
\pgfpathlineto{\pgfqpoint{1.050556in}{0.889467in}}%
\pgfpathlineto{\pgfqpoint{1.054387in}{0.889942in}}%
\pgfpathlineto{\pgfqpoint{1.058217in}{0.891196in}}%
\pgfpathlineto{\pgfqpoint{1.065878in}{0.892859in}}%
\pgfpathlineto{\pgfqpoint{1.088860in}{0.896070in}}%
\pgfpathlineto{\pgfqpoint{1.090775in}{0.897788in}}%
\pgfpathlineto{\pgfqpoint{1.094605in}{0.898895in}}%
\pgfpathlineto{\pgfqpoint{1.098435in}{0.900369in}}%
\pgfpathlineto{\pgfqpoint{1.106096in}{0.901823in}}%
\pgfpathlineto{\pgfqpoint{1.113757in}{0.903021in}}%
\pgfpathlineto{\pgfqpoint{1.153975in}{0.909369in}}%
\pgfpathlineto{\pgfqpoint{1.157806in}{0.911192in}}%
\pgfpathlineto{\pgfqpoint{1.167382in}{0.912204in}}%
\pgfpathlineto{\pgfqpoint{1.173127in}{0.912697in}}%
\pgfpathlineto{\pgfqpoint{1.178873in}{0.913300in}}%
\pgfpathlineto{\pgfqpoint{1.184618in}{0.914770in}}%
\pgfpathlineto{\pgfqpoint{1.205685in}{0.916886in}}%
\pgfpathlineto{\pgfqpoint{1.215261in}{0.918381in}}%
\pgfpathlineto{\pgfqpoint{1.284207in}{0.927502in}}%
\pgfpathlineto{\pgfqpoint{1.301444in}{0.929047in}}%
\pgfpathlineto{\pgfqpoint{1.307189in}{0.930186in}}%
\pgfpathlineto{\pgfqpoint{1.312935in}{0.931042in}}%
\pgfpathlineto{\pgfqpoint{1.374220in}{0.937970in}}%
\pgfpathlineto{\pgfqpoint{1.378050in}{0.938874in}}%
\pgfpathlineto{\pgfqpoint{1.393372in}{0.940231in}}%
\pgfpathlineto{\pgfqpoint{1.437421in}{0.945512in}}%
\pgfpathlineto{\pgfqpoint{1.445081in}{0.946576in}}%
\pgfpathlineto{\pgfqpoint{1.517858in}{0.952724in}}%
\pgfpathlineto{\pgfqpoint{1.521688in}{0.954409in}}%
\pgfpathlineto{\pgfqpoint{1.554246in}{0.957380in}}%
\pgfpathlineto{\pgfqpoint{1.582974in}{0.959969in}}%
\pgfpathlineto{\pgfqpoint{1.590635in}{0.960756in}}%
\pgfpathlineto{\pgfqpoint{1.623192in}{0.964336in}}%
\pgfpathlineto{\pgfqpoint{1.628938in}{0.965136in}}%
\pgfpathlineto{\pgfqpoint{1.636599in}{0.966438in}}%
\pgfpathlineto{\pgfqpoint{1.642344in}{0.967911in}}%
\pgfpathlineto{\pgfqpoint{1.686393in}{0.974032in}}%
\pgfpathlineto{\pgfqpoint{1.690223in}{0.975041in}}%
\pgfpathlineto{\pgfqpoint{1.734272in}{0.981467in}}%
\pgfpathlineto{\pgfqpoint{1.740018in}{0.983447in}}%
\pgfpathlineto{\pgfqpoint{1.745763in}{0.984686in}}%
\pgfpathlineto{\pgfqpoint{1.755339in}{0.986555in}}%
\pgfpathlineto{\pgfqpoint{1.761085in}{0.987270in}}%
\pgfpathlineto{\pgfqpoint{1.772576in}{0.988294in}}%
\pgfpathlineto{\pgfqpoint{1.785982in}{0.991567in}}%
\pgfpathlineto{\pgfqpoint{1.816625in}{0.997086in}}%
\pgfpathlineto{\pgfqpoint{1.820455in}{0.997821in}}%
\pgfpathlineto{\pgfqpoint{1.826201in}{0.998985in}}%
\pgfpathlineto{\pgfqpoint{1.921959in}{1.013104in}}%
\pgfpathlineto{\pgfqpoint{1.935365in}{1.014109in}}%
\pgfpathlineto{\pgfqpoint{1.990905in}{1.021609in}}%
\pgfpathlineto{\pgfqpoint{1.994736in}{1.023228in}}%
\pgfpathlineto{\pgfqpoint{2.004312in}{1.023816in}}%
\pgfpathlineto{\pgfqpoint{2.008142in}{1.025233in}}%
\pgfpathlineto{\pgfqpoint{2.023463in}{1.027506in}}%
\pgfpathlineto{\pgfqpoint{2.046445in}{1.029353in}}%
\pgfpathlineto{\pgfqpoint{2.071343in}{1.032582in}}%
\pgfpathlineto{\pgfqpoint{2.082834in}{1.034329in}}%
\pgfpathlineto{\pgfqpoint{2.142204in}{1.042093in}}%
\pgfpathlineto{\pgfqpoint{2.147949in}{1.043497in}}%
\pgfpathlineto{\pgfqpoint{2.163271in}{1.045316in}}%
\pgfpathlineto{\pgfqpoint{2.176677in}{1.047736in}}%
\pgfpathlineto{\pgfqpoint{2.182422in}{1.048601in}}%
\pgfpathlineto{\pgfqpoint{2.197744in}{1.049935in}}%
\pgfpathlineto{\pgfqpoint{2.214980in}{1.053074in}}%
\pgfpathlineto{\pgfqpoint{2.237962in}{1.055008in}}%
\pgfpathlineto{\pgfqpoint{2.262860in}{1.056430in}}%
\pgfpathlineto{\pgfqpoint{2.268605in}{1.057931in}}%
\pgfpathlineto{\pgfqpoint{2.289672in}{1.060641in}}%
\pgfpathlineto{\pgfqpoint{2.308824in}{1.064654in}}%
\pgfpathlineto{\pgfqpoint{2.396922in}{1.078320in}}%
\pgfpathlineto{\pgfqpoint{2.400752in}{1.079652in}}%
\pgfpathlineto{\pgfqpoint{2.417989in}{1.081793in}}%
\pgfpathlineto{\pgfqpoint{2.419904in}{1.084435in}}%
\pgfpathlineto{\pgfqpoint{2.439055in}{1.089532in}}%
\pgfpathlineto{\pgfqpoint{2.442886in}{1.091604in}}%
\pgfpathlineto{\pgfqpoint{2.446716in}{1.092017in}}%
\pgfpathlineto{\pgfqpoint{2.448631in}{1.094610in}}%
\pgfpathlineto{\pgfqpoint{2.454377in}{1.095374in}}%
\pgfpathlineto{\pgfqpoint{2.469698in}{1.105210in}}%
\pgfpathlineto{\pgfqpoint{2.523323in}{1.120572in}}%
\pgfpathlineto{\pgfqpoint{2.540560in}{1.129362in}}%
\pgfpathlineto{\pgfqpoint{2.544390in}{1.130803in}}%
\pgfpathlineto{\pgfqpoint{2.546305in}{1.131678in}}%
\pgfpathlineto{\pgfqpoint{2.548220in}{1.135472in}}%
\pgfpathlineto{\pgfqpoint{2.550135in}{1.135501in}}%
\pgfpathlineto{\pgfqpoint{2.552051in}{1.138245in}}%
\pgfpathlineto{\pgfqpoint{2.569287in}{1.140683in}}%
\pgfpathlineto{\pgfqpoint{2.584608in}{1.145341in}}%
\pgfpathlineto{\pgfqpoint{2.599930in}{1.150057in}}%
\pgfpathlineto{\pgfqpoint{2.603760in}{1.153358in}}%
\pgfpathlineto{\pgfqpoint{2.613336in}{1.154237in}}%
\pgfpathlineto{\pgfqpoint{2.620997in}{1.157395in}}%
\pgfpathlineto{\pgfqpoint{2.624827in}{1.158424in}}%
\pgfpathlineto{\pgfqpoint{2.628657in}{1.161592in}}%
\pgfpathlineto{\pgfqpoint{2.634403in}{1.162027in}}%
\pgfpathlineto{\pgfqpoint{2.638233in}{1.164186in}}%
\pgfpathlineto{\pgfqpoint{2.643979in}{1.165551in}}%
\pgfpathlineto{\pgfqpoint{2.647809in}{1.168084in}}%
\pgfpathlineto{\pgfqpoint{2.655470in}{1.170708in}}%
\pgfpathlineto{\pgfqpoint{2.659300in}{1.174661in}}%
\pgfpathlineto{\pgfqpoint{2.663130in}{1.176342in}}%
\pgfpathlineto{\pgfqpoint{2.672706in}{1.178980in}}%
\pgfpathlineto{\pgfqpoint{2.674622in}{1.184777in}}%
\pgfpathlineto{\pgfqpoint{2.680367in}{1.188055in}}%
\pgfpathlineto{\pgfqpoint{2.684197in}{1.189445in}}%
\pgfpathlineto{\pgfqpoint{2.688028in}{1.190641in}}%
\pgfpathlineto{\pgfqpoint{2.693773in}{1.194410in}}%
\pgfpathlineto{\pgfqpoint{2.705264in}{1.206707in}}%
\pgfpathlineto{\pgfqpoint{2.707179in}{1.207159in}}%
\pgfpathlineto{\pgfqpoint{2.709095in}{1.208957in}}%
\pgfpathlineto{\pgfqpoint{2.718670in}{1.210562in}}%
\pgfpathlineto{\pgfqpoint{2.720586in}{1.212025in}}%
\pgfpathlineto{\pgfqpoint{2.722501in}{1.217560in}}%
\pgfpathlineto{\pgfqpoint{2.733992in}{1.219462in}}%
\pgfpathlineto{\pgfqpoint{2.753144in}{1.226461in}}%
\pgfpathlineto{\pgfqpoint{2.756974in}{1.229147in}}%
\pgfpathlineto{\pgfqpoint{2.766550in}{1.231501in}}%
\pgfpathlineto{\pgfqpoint{2.768465in}{1.232466in}}%
\pgfpathlineto{\pgfqpoint{2.772295in}{1.235922in}}%
\pgfpathlineto{\pgfqpoint{2.779956in}{1.237844in}}%
\pgfpathlineto{\pgfqpoint{2.785701in}{1.243512in}}%
\pgfpathlineto{\pgfqpoint{2.789532in}{1.248620in}}%
\pgfpathlineto{\pgfqpoint{2.795277in}{1.250336in}}%
\pgfpathlineto{\pgfqpoint{2.797192in}{1.254998in}}%
\pgfpathlineto{\pgfqpoint{2.799108in}{1.255714in}}%
\pgfpathlineto{\pgfqpoint{2.808684in}{1.266275in}}%
\pgfpathlineto{\pgfqpoint{2.814429in}{1.267728in}}%
\pgfpathlineto{\pgfqpoint{2.816344in}{1.270971in}}%
\pgfpathlineto{\pgfqpoint{2.818259in}{1.271617in}}%
\pgfpathlineto{\pgfqpoint{2.820175in}{1.274300in}}%
\pgfpathlineto{\pgfqpoint{2.822090in}{1.274873in}}%
\pgfpathlineto{\pgfqpoint{2.846987in}{1.297343in}}%
\pgfpathlineto{\pgfqpoint{2.848902in}{1.298045in}}%
\pgfpathlineto{\pgfqpoint{2.850817in}{1.301241in}}%
\pgfpathlineto{\pgfqpoint{2.852732in}{1.301421in}}%
\pgfpathlineto{\pgfqpoint{2.858478in}{1.306625in}}%
\pgfpathlineto{\pgfqpoint{2.860393in}{1.306735in}}%
\pgfpathlineto{\pgfqpoint{2.866139in}{1.314238in}}%
\pgfpathlineto{\pgfqpoint{2.868054in}{1.314547in}}%
\pgfpathlineto{\pgfqpoint{2.869969in}{1.316651in}}%
\pgfpathlineto{\pgfqpoint{2.879545in}{1.318419in}}%
\pgfpathlineto{\pgfqpoint{2.881460in}{1.323457in}}%
\pgfpathlineto{\pgfqpoint{2.883375in}{1.325154in}}%
\pgfpathlineto{\pgfqpoint{2.891036in}{1.338332in}}%
\pgfpathlineto{\pgfqpoint{2.892951in}{1.339528in}}%
\pgfpathlineto{\pgfqpoint{2.896781in}{1.348804in}}%
\pgfpathlineto{\pgfqpoint{2.900612in}{1.349348in}}%
\pgfpathlineto{\pgfqpoint{2.904442in}{1.352064in}}%
\pgfpathlineto{\pgfqpoint{2.906357in}{1.352728in}}%
\pgfpathlineto{\pgfqpoint{2.910188in}{1.362177in}}%
\pgfpathlineto{\pgfqpoint{2.912103in}{1.362651in}}%
\pgfpathlineto{\pgfqpoint{2.915933in}{1.368245in}}%
\pgfpathlineto{\pgfqpoint{2.917848in}{1.368411in}}%
\pgfpathlineto{\pgfqpoint{2.919763in}{1.372686in}}%
\pgfpathlineto{\pgfqpoint{2.921679in}{1.383317in}}%
\pgfpathlineto{\pgfqpoint{2.923594in}{1.383544in}}%
\pgfpathlineto{\pgfqpoint{2.925509in}{1.385876in}}%
\pgfpathlineto{\pgfqpoint{2.927424in}{1.390546in}}%
\pgfpathlineto{\pgfqpoint{2.929339in}{1.392253in}}%
\pgfpathlineto{\pgfqpoint{2.931254in}{1.400743in}}%
\pgfpathlineto{\pgfqpoint{2.935085in}{1.402566in}}%
\pgfpathlineto{\pgfqpoint{2.937000in}{1.426236in}}%
\pgfpathlineto{\pgfqpoint{2.938915in}{1.434149in}}%
\pgfpathlineto{\pgfqpoint{2.940830in}{1.449912in}}%
\pgfpathlineto{\pgfqpoint{2.942746in}{1.453522in}}%
\pgfpathlineto{\pgfqpoint{2.944661in}{1.453739in}}%
\pgfpathlineto{\pgfqpoint{2.946576in}{1.460477in}}%
\pgfpathlineto{\pgfqpoint{2.948491in}{1.463050in}}%
\pgfpathlineto{\pgfqpoint{2.950406in}{1.477048in}}%
\pgfpathlineto{\pgfqpoint{2.952321in}{1.479890in}}%
\pgfpathlineto{\pgfqpoint{2.954237in}{1.500516in}}%
\pgfpathlineto{\pgfqpoint{2.956152in}{1.502706in}}%
\pgfpathlineto{\pgfqpoint{2.958067in}{1.514112in}}%
\pgfpathlineto{\pgfqpoint{2.959982in}{1.514813in}}%
\pgfpathlineto{\pgfqpoint{2.963812in}{1.551074in}}%
\pgfpathlineto{\pgfqpoint{2.967643in}{1.554889in}}%
\pgfpathlineto{\pgfqpoint{2.977219in}{1.604232in}}%
\pgfpathlineto{\pgfqpoint{2.979134in}{1.604287in}}%
\pgfpathlineto{\pgfqpoint{2.981049in}{1.609002in}}%
\pgfpathlineto{\pgfqpoint{2.982964in}{1.610477in}}%
\pgfpathlineto{\pgfqpoint{2.984879in}{1.615768in}}%
\pgfpathlineto{\pgfqpoint{2.986794in}{1.616285in}}%
\pgfpathlineto{\pgfqpoint{2.988710in}{1.627537in}}%
\pgfpathlineto{\pgfqpoint{2.990625in}{1.627787in}}%
\pgfpathlineto{\pgfqpoint{2.992540in}{1.636913in}}%
\pgfpathlineto{\pgfqpoint{2.994455in}{1.653743in}}%
\pgfpathlineto{\pgfqpoint{2.996370in}{1.655288in}}%
\pgfpathlineto{\pgfqpoint{2.998285in}{1.671393in}}%
\pgfpathlineto{\pgfqpoint{3.000201in}{1.675089in}}%
\pgfpathlineto{\pgfqpoint{3.002116in}{1.699109in}}%
\pgfpathlineto{\pgfqpoint{3.004031in}{1.703650in}}%
\pgfpathlineto{\pgfqpoint{3.007861in}{1.726928in}}%
\pgfpathlineto{\pgfqpoint{3.009777in}{1.727205in}}%
\pgfpathlineto{\pgfqpoint{3.011692in}{1.738607in}}%
\pgfpathlineto{\pgfqpoint{3.017437in}{1.747913in}}%
\pgfpathlineto{\pgfqpoint{3.021268in}{1.758869in}}%
\pgfpathlineto{\pgfqpoint{3.025098in}{1.766004in}}%
\pgfpathlineto{\pgfqpoint{3.027013in}{1.786911in}}%
\pgfpathlineto{\pgfqpoint{3.032759in}{1.799776in}}%
\pgfpathlineto{\pgfqpoint{3.034674in}{1.801144in}}%
\pgfpathlineto{\pgfqpoint{3.036589in}{1.826535in}}%
\pgfpathlineto{\pgfqpoint{3.036589in}{1.826535in}}%
\pgfusepath{stroke}%
\end{pgfscope}%
\begin{pgfscope}%
\pgfpathrectangle{\pgfqpoint{0.694334in}{0.523557in}}{\pgfqpoint{3.830343in}{1.302977in}}%
\pgfusepath{clip}%
\pgfsetbuttcap%
\pgfsetroundjoin%
\pgfsetlinewidth{1.003750pt}%
\definecolor{currentstroke}{rgb}{0.000000,0.000000,0.000000}%
\pgfsetstrokecolor{currentstroke}%
\pgfsetdash{{1.000000pt}{1.650000pt}}{0.000000pt}%
\pgfpathmoveto{\pgfqpoint{0.694334in}{0.567141in}}%
\pgfpathlineto{\pgfqpoint{0.696249in}{0.582562in}}%
\pgfpathlineto{\pgfqpoint{0.698165in}{0.582755in}}%
\pgfpathlineto{\pgfqpoint{0.701995in}{0.599148in}}%
\pgfpathlineto{\pgfqpoint{0.703910in}{0.606735in}}%
\pgfpathlineto{\pgfqpoint{0.711571in}{0.615604in}}%
\pgfpathlineto{\pgfqpoint{0.713486in}{0.616358in}}%
\pgfpathlineto{\pgfqpoint{0.715401in}{0.627746in}}%
\pgfpathlineto{\pgfqpoint{0.719232in}{0.630247in}}%
\pgfpathlineto{\pgfqpoint{0.721147in}{0.632363in}}%
\pgfpathlineto{\pgfqpoint{0.724977in}{0.633447in}}%
\pgfpathlineto{\pgfqpoint{0.726892in}{0.635337in}}%
\pgfpathlineto{\pgfqpoint{0.732638in}{0.646877in}}%
\pgfpathlineto{\pgfqpoint{0.736468in}{0.649507in}}%
\pgfpathlineto{\pgfqpoint{0.738383in}{0.655445in}}%
\pgfpathlineto{\pgfqpoint{0.740298in}{0.655930in}}%
\pgfpathlineto{\pgfqpoint{0.747959in}{0.671221in}}%
\pgfpathlineto{\pgfqpoint{0.749874in}{0.675620in}}%
\pgfpathlineto{\pgfqpoint{0.751789in}{0.675686in}}%
\pgfpathlineto{\pgfqpoint{0.753705in}{0.678423in}}%
\pgfpathlineto{\pgfqpoint{0.755620in}{0.683416in}}%
\pgfpathlineto{\pgfqpoint{0.759450in}{0.685036in}}%
\pgfpathlineto{\pgfqpoint{0.761365in}{0.690456in}}%
\pgfpathlineto{\pgfqpoint{0.770941in}{0.697917in}}%
\pgfpathlineto{\pgfqpoint{0.772856in}{0.698286in}}%
\pgfpathlineto{\pgfqpoint{0.778602in}{0.704153in}}%
\pgfpathlineto{\pgfqpoint{0.793923in}{0.708539in}}%
\pgfpathlineto{\pgfqpoint{0.801584in}{0.709532in}}%
\pgfpathlineto{\pgfqpoint{0.803499in}{0.711186in}}%
\pgfpathlineto{\pgfqpoint{0.811160in}{0.712162in}}%
\pgfpathlineto{\pgfqpoint{0.814990in}{0.713108in}}%
\pgfpathlineto{\pgfqpoint{0.837972in}{0.715177in}}%
\pgfpathlineto{\pgfqpoint{0.853294in}{0.716555in}}%
\pgfpathlineto{\pgfqpoint{0.859039in}{0.717527in}}%
\pgfpathlineto{\pgfqpoint{0.864785in}{0.718514in}}%
\pgfpathlineto{\pgfqpoint{0.889682in}{0.720470in}}%
\pgfpathlineto{\pgfqpoint{0.903088in}{0.721801in}}%
\pgfpathlineto{\pgfqpoint{0.920325in}{0.724066in}}%
\pgfpathlineto{\pgfqpoint{0.947137in}{0.725055in}}%
\pgfpathlineto{\pgfqpoint{1.010338in}{0.726970in}}%
\pgfpathlineto{\pgfqpoint{1.027574in}{0.727996in}}%
\pgfpathlineto{\pgfqpoint{1.067793in}{0.730388in}}%
\pgfpathlineto{\pgfqpoint{1.075453in}{0.730995in}}%
\pgfpathlineto{\pgfqpoint{1.299528in}{0.744595in}}%
\pgfpathlineto{\pgfqpoint{1.303359in}{0.745473in}}%
\pgfpathlineto{\pgfqpoint{1.328256in}{0.746903in}}%
\pgfpathlineto{\pgfqpoint{1.370390in}{0.750858in}}%
\pgfpathlineto{\pgfqpoint{1.385711in}{0.751900in}}%
\pgfpathlineto{\pgfqpoint{1.404863in}{0.753175in}}%
\pgfpathlineto{\pgfqpoint{1.416354in}{0.754161in}}%
\pgfpathlineto{\pgfqpoint{1.422099in}{0.755207in}}%
\pgfpathlineto{\pgfqpoint{1.433590in}{0.756474in}}%
\pgfpathlineto{\pgfqpoint{1.437421in}{0.757393in}}%
\pgfpathlineto{\pgfqpoint{1.452742in}{0.758832in}}%
\pgfpathlineto{\pgfqpoint{1.491046in}{0.765101in}}%
\pgfpathlineto{\pgfqpoint{1.498706in}{0.765845in}}%
\pgfpathlineto{\pgfqpoint{1.508282in}{0.766470in}}%
\pgfpathlineto{\pgfqpoint{1.519773in}{0.771921in}}%
\pgfpathlineto{\pgfqpoint{1.523604in}{0.772598in}}%
\pgfpathlineto{\pgfqpoint{1.531264in}{0.777376in}}%
\pgfpathlineto{\pgfqpoint{1.535095in}{0.778714in}}%
\pgfpathlineto{\pgfqpoint{1.537010in}{0.783849in}}%
\pgfpathlineto{\pgfqpoint{1.538925in}{0.784911in}}%
\pgfpathlineto{\pgfqpoint{1.540840in}{0.788811in}}%
\pgfpathlineto{\pgfqpoint{1.552331in}{0.791007in}}%
\pgfpathlineto{\pgfqpoint{1.554246in}{0.791602in}}%
\pgfpathlineto{\pgfqpoint{1.561907in}{0.798592in}}%
\pgfpathlineto{\pgfqpoint{1.569568in}{0.799587in}}%
\pgfpathlineto{\pgfqpoint{1.573398in}{0.802172in}}%
\pgfpathlineto{\pgfqpoint{1.577228in}{0.802534in}}%
\pgfpathlineto{\pgfqpoint{1.582974in}{0.806461in}}%
\pgfpathlineto{\pgfqpoint{1.590635in}{0.808069in}}%
\pgfpathlineto{\pgfqpoint{1.607871in}{0.810257in}}%
\pgfpathlineto{\pgfqpoint{1.613617in}{0.811082in}}%
\pgfpathlineto{\pgfqpoint{1.623192in}{0.811795in}}%
\pgfpathlineto{\pgfqpoint{1.636599in}{0.816468in}}%
\pgfpathlineto{\pgfqpoint{1.642344in}{0.816816in}}%
\pgfpathlineto{\pgfqpoint{1.646174in}{0.817868in}}%
\pgfpathlineto{\pgfqpoint{1.703630in}{0.828093in}}%
\pgfpathlineto{\pgfqpoint{1.715121in}{0.829443in}}%
\pgfpathlineto{\pgfqpoint{1.722781in}{0.831705in}}%
\pgfpathlineto{\pgfqpoint{1.730442in}{0.832697in}}%
\pgfpathlineto{\pgfqpoint{1.759170in}{0.835613in}}%
\pgfpathlineto{\pgfqpoint{1.763000in}{0.836460in}}%
\pgfpathlineto{\pgfqpoint{1.778321in}{0.838136in}}%
\pgfpathlineto{\pgfqpoint{1.782152in}{0.839452in}}%
\pgfpathlineto{\pgfqpoint{1.803219in}{0.843519in}}%
\pgfpathlineto{\pgfqpoint{1.818540in}{0.845975in}}%
\pgfpathlineto{\pgfqpoint{1.822370in}{0.846987in}}%
\pgfpathlineto{\pgfqpoint{1.826201in}{0.847835in}}%
\pgfpathlineto{\pgfqpoint{1.831946in}{0.849615in}}%
\pgfpathlineto{\pgfqpoint{1.835776in}{0.851432in}}%
\pgfpathlineto{\pgfqpoint{1.839607in}{0.852136in}}%
\pgfpathlineto{\pgfqpoint{1.843437in}{0.854187in}}%
\pgfpathlineto{\pgfqpoint{1.847267in}{0.855217in}}%
\pgfpathlineto{\pgfqpoint{1.864504in}{0.859580in}}%
\pgfpathlineto{\pgfqpoint{1.879825in}{0.860892in}}%
\pgfpathlineto{\pgfqpoint{1.891316in}{0.863940in}}%
\pgfpathlineto{\pgfqpoint{1.908553in}{0.865804in}}%
\pgfpathlineto{\pgfqpoint{1.944941in}{0.870873in}}%
\pgfpathlineto{\pgfqpoint{1.950687in}{0.872051in}}%
\pgfpathlineto{\pgfqpoint{2.002396in}{0.880407in}}%
\pgfpathlineto{\pgfqpoint{2.006227in}{0.881360in}}%
\pgfpathlineto{\pgfqpoint{2.013887in}{0.882640in}}%
\pgfpathlineto{\pgfqpoint{2.019633in}{0.882946in}}%
\pgfpathlineto{\pgfqpoint{2.023463in}{0.885084in}}%
\pgfpathlineto{\pgfqpoint{2.027294in}{0.885827in}}%
\pgfpathlineto{\pgfqpoint{2.031124in}{0.887468in}}%
\pgfpathlineto{\pgfqpoint{2.046445in}{0.889403in}}%
\pgfpathlineto{\pgfqpoint{2.113476in}{0.897402in}}%
\pgfpathlineto{\pgfqpoint{2.117307in}{0.898536in}}%
\pgfpathlineto{\pgfqpoint{2.130713in}{0.900570in}}%
\pgfpathlineto{\pgfqpoint{2.134543in}{0.901537in}}%
\pgfpathlineto{\pgfqpoint{2.142204in}{0.902500in}}%
\pgfpathlineto{\pgfqpoint{2.147949in}{0.903786in}}%
\pgfpathlineto{\pgfqpoint{2.193914in}{0.910651in}}%
\pgfpathlineto{\pgfqpoint{2.207320in}{0.911405in}}%
\pgfpathlineto{\pgfqpoint{2.213065in}{0.913407in}}%
\pgfpathlineto{\pgfqpoint{2.218811in}{0.914494in}}%
\pgfpathlineto{\pgfqpoint{2.222641in}{0.915106in}}%
\pgfpathlineto{\pgfqpoint{2.230302in}{0.917520in}}%
\pgfpathlineto{\pgfqpoint{2.239878in}{0.918703in}}%
\pgfpathlineto{\pgfqpoint{2.243708in}{0.920150in}}%
\pgfpathlineto{\pgfqpoint{2.264775in}{0.923191in}}%
\pgfpathlineto{\pgfqpoint{2.326060in}{0.934156in}}%
\pgfpathlineto{\pgfqpoint{2.329891in}{0.935857in}}%
\pgfpathlineto{\pgfqpoint{2.343297in}{0.937293in}}%
\pgfpathlineto{\pgfqpoint{2.350958in}{0.938427in}}%
\pgfpathlineto{\pgfqpoint{2.366279in}{0.939420in}}%
\pgfpathlineto{\pgfqpoint{2.372024in}{0.941202in}}%
\pgfpathlineto{\pgfqpoint{2.383515in}{0.942902in}}%
\pgfpathlineto{\pgfqpoint{2.410328in}{0.945045in}}%
\pgfpathlineto{\pgfqpoint{2.423734in}{0.945656in}}%
\pgfpathlineto{\pgfqpoint{2.429480in}{0.946929in}}%
\pgfpathlineto{\pgfqpoint{2.444801in}{0.947880in}}%
\pgfpathlineto{\pgfqpoint{2.467783in}{0.950815in}}%
\pgfpathlineto{\pgfqpoint{2.479274in}{0.954592in}}%
\pgfpathlineto{\pgfqpoint{2.511832in}{0.964688in}}%
\pgfpathlineto{\pgfqpoint{2.517577in}{0.970839in}}%
\pgfpathlineto{\pgfqpoint{2.519493in}{0.970888in}}%
\pgfpathlineto{\pgfqpoint{2.521408in}{0.972769in}}%
\pgfpathlineto{\pgfqpoint{2.527153in}{0.973870in}}%
\pgfpathlineto{\pgfqpoint{2.530984in}{0.975766in}}%
\pgfpathlineto{\pgfqpoint{2.534814in}{0.980023in}}%
\pgfpathlineto{\pgfqpoint{2.548220in}{0.982244in}}%
\pgfpathlineto{\pgfqpoint{2.555881in}{0.987072in}}%
\pgfpathlineto{\pgfqpoint{2.559711in}{0.987375in}}%
\pgfpathlineto{\pgfqpoint{2.561626in}{0.989794in}}%
\pgfpathlineto{\pgfqpoint{2.565457in}{0.990829in}}%
\pgfpathlineto{\pgfqpoint{2.575033in}{0.992834in}}%
\pgfpathlineto{\pgfqpoint{2.580778in}{0.995368in}}%
\pgfpathlineto{\pgfqpoint{2.584608in}{0.996058in}}%
\pgfpathlineto{\pgfqpoint{2.586524in}{0.999027in}}%
\pgfpathlineto{\pgfqpoint{2.592269in}{1.000545in}}%
\pgfpathlineto{\pgfqpoint{2.601845in}{1.001857in}}%
\pgfpathlineto{\pgfqpoint{2.607591in}{1.004198in}}%
\pgfpathlineto{\pgfqpoint{2.615251in}{1.006342in}}%
\pgfpathlineto{\pgfqpoint{2.617166in}{1.007769in}}%
\pgfpathlineto{\pgfqpoint{2.624827in}{1.008569in}}%
\pgfpathlineto{\pgfqpoint{2.647809in}{1.015743in}}%
\pgfpathlineto{\pgfqpoint{2.653555in}{1.017738in}}%
\pgfpathlineto{\pgfqpoint{2.661215in}{1.018503in}}%
\pgfpathlineto{\pgfqpoint{2.666961in}{1.021593in}}%
\pgfpathlineto{\pgfqpoint{2.668876in}{1.021639in}}%
\pgfpathlineto{\pgfqpoint{2.676537in}{1.029089in}}%
\pgfpathlineto{\pgfqpoint{2.682282in}{1.029692in}}%
\pgfpathlineto{\pgfqpoint{2.684197in}{1.032096in}}%
\pgfpathlineto{\pgfqpoint{2.688028in}{1.032551in}}%
\pgfpathlineto{\pgfqpoint{2.689943in}{1.036236in}}%
\pgfpathlineto{\pgfqpoint{2.693773in}{1.037453in}}%
\pgfpathlineto{\pgfqpoint{2.699519in}{1.041072in}}%
\pgfpathlineto{\pgfqpoint{2.705264in}{1.041844in}}%
\pgfpathlineto{\pgfqpoint{2.711010in}{1.046404in}}%
\pgfpathlineto{\pgfqpoint{2.712925in}{1.046713in}}%
\pgfpathlineto{\pgfqpoint{2.714840in}{1.048649in}}%
\pgfpathlineto{\pgfqpoint{2.718670in}{1.048973in}}%
\pgfpathlineto{\pgfqpoint{2.722501in}{1.051883in}}%
\pgfpathlineto{\pgfqpoint{2.730161in}{1.053131in}}%
\pgfpathlineto{\pgfqpoint{2.733992in}{1.058431in}}%
\pgfpathlineto{\pgfqpoint{2.737822in}{1.060002in}}%
\pgfpathlineto{\pgfqpoint{2.739737in}{1.062244in}}%
\pgfpathlineto{\pgfqpoint{2.745483in}{1.063362in}}%
\pgfpathlineto{\pgfqpoint{2.749313in}{1.066143in}}%
\pgfpathlineto{\pgfqpoint{2.756974in}{1.068559in}}%
\pgfpathlineto{\pgfqpoint{2.758889in}{1.070863in}}%
\pgfpathlineto{\pgfqpoint{2.760804in}{1.071197in}}%
\pgfpathlineto{\pgfqpoint{2.762719in}{1.073376in}}%
\pgfpathlineto{\pgfqpoint{2.764635in}{1.073698in}}%
\pgfpathlineto{\pgfqpoint{2.766550in}{1.076698in}}%
\pgfpathlineto{\pgfqpoint{2.781871in}{1.081531in}}%
\pgfpathlineto{\pgfqpoint{2.789532in}{1.091942in}}%
\pgfpathlineto{\pgfqpoint{2.802938in}{1.097733in}}%
\pgfpathlineto{\pgfqpoint{2.816344in}{1.113734in}}%
\pgfpathlineto{\pgfqpoint{2.820175in}{1.115461in}}%
\pgfpathlineto{\pgfqpoint{2.822090in}{1.118249in}}%
\pgfpathlineto{\pgfqpoint{2.824005in}{1.118961in}}%
\pgfpathlineto{\pgfqpoint{2.825920in}{1.121740in}}%
\pgfpathlineto{\pgfqpoint{2.827835in}{1.121876in}}%
\pgfpathlineto{\pgfqpoint{2.831666in}{1.128837in}}%
\pgfpathlineto{\pgfqpoint{2.841241in}{1.132505in}}%
\pgfpathlineto{\pgfqpoint{2.843157in}{1.135682in}}%
\pgfpathlineto{\pgfqpoint{2.845072in}{1.142341in}}%
\pgfpathlineto{\pgfqpoint{2.846987in}{1.143000in}}%
\pgfpathlineto{\pgfqpoint{2.848902in}{1.150761in}}%
\pgfpathlineto{\pgfqpoint{2.850817in}{1.152351in}}%
\pgfpathlineto{\pgfqpoint{2.852732in}{1.162566in}}%
\pgfpathlineto{\pgfqpoint{2.858478in}{1.165194in}}%
\pgfpathlineto{\pgfqpoint{2.860393in}{1.166312in}}%
\pgfpathlineto{\pgfqpoint{2.862308in}{1.176447in}}%
\pgfpathlineto{\pgfqpoint{2.875715in}{1.189905in}}%
\pgfpathlineto{\pgfqpoint{2.877630in}{1.200352in}}%
\pgfpathlineto{\pgfqpoint{2.879545in}{1.200960in}}%
\pgfpathlineto{\pgfqpoint{2.883375in}{1.214235in}}%
\pgfpathlineto{\pgfqpoint{2.885290in}{1.223561in}}%
\pgfpathlineto{\pgfqpoint{2.887206in}{1.223605in}}%
\pgfpathlineto{\pgfqpoint{2.892951in}{1.233278in}}%
\pgfpathlineto{\pgfqpoint{2.896781in}{1.235146in}}%
\pgfpathlineto{\pgfqpoint{2.898697in}{1.235612in}}%
\pgfpathlineto{\pgfqpoint{2.902527in}{1.240602in}}%
\pgfpathlineto{\pgfqpoint{2.908272in}{1.241978in}}%
\pgfpathlineto{\pgfqpoint{2.910188in}{1.245462in}}%
\pgfpathlineto{\pgfqpoint{2.914018in}{1.258053in}}%
\pgfpathlineto{\pgfqpoint{2.915933in}{1.258087in}}%
\pgfpathlineto{\pgfqpoint{2.917848in}{1.259950in}}%
\pgfpathlineto{\pgfqpoint{2.919763in}{1.265087in}}%
\pgfpathlineto{\pgfqpoint{2.929339in}{1.266600in}}%
\pgfpathlineto{\pgfqpoint{2.931254in}{1.269136in}}%
\pgfpathlineto{\pgfqpoint{2.933170in}{1.274322in}}%
\pgfpathlineto{\pgfqpoint{2.935085in}{1.275324in}}%
\pgfpathlineto{\pgfqpoint{2.937000in}{1.286711in}}%
\pgfpathlineto{\pgfqpoint{2.942746in}{1.296020in}}%
\pgfpathlineto{\pgfqpoint{2.950406in}{1.339754in}}%
\pgfpathlineto{\pgfqpoint{2.954237in}{1.353078in}}%
\pgfpathlineto{\pgfqpoint{2.956152in}{1.366320in}}%
\pgfpathlineto{\pgfqpoint{2.958067in}{1.367354in}}%
\pgfpathlineto{\pgfqpoint{2.961897in}{1.375775in}}%
\pgfpathlineto{\pgfqpoint{2.963812in}{1.401096in}}%
\pgfpathlineto{\pgfqpoint{2.965728in}{1.402566in}}%
\pgfpathlineto{\pgfqpoint{2.967643in}{1.422730in}}%
\pgfpathlineto{\pgfqpoint{2.969558in}{1.427997in}}%
\pgfpathlineto{\pgfqpoint{2.971473in}{1.438897in}}%
\pgfpathlineto{\pgfqpoint{2.973388in}{1.463050in}}%
\pgfpathlineto{\pgfqpoint{2.977219in}{1.472723in}}%
\pgfpathlineto{\pgfqpoint{2.979134in}{1.500516in}}%
\pgfpathlineto{\pgfqpoint{2.982964in}{1.515955in}}%
\pgfpathlineto{\pgfqpoint{2.994455in}{1.564788in}}%
\pgfpathlineto{\pgfqpoint{2.996370in}{1.566155in}}%
\pgfpathlineto{\pgfqpoint{2.998285in}{1.571164in}}%
\pgfpathlineto{\pgfqpoint{3.000201in}{1.572768in}}%
\pgfpathlineto{\pgfqpoint{3.002116in}{1.579641in}}%
\pgfpathlineto{\pgfqpoint{3.005946in}{1.602936in}}%
\pgfpathlineto{\pgfqpoint{3.007861in}{1.606766in}}%
\pgfpathlineto{\pgfqpoint{3.009777in}{1.616119in}}%
\pgfpathlineto{\pgfqpoint{3.011692in}{1.616285in}}%
\pgfpathlineto{\pgfqpoint{3.013607in}{1.623297in}}%
\pgfpathlineto{\pgfqpoint{3.017437in}{1.627683in}}%
\pgfpathlineto{\pgfqpoint{3.019352in}{1.627787in}}%
\pgfpathlineto{\pgfqpoint{3.025098in}{1.654026in}}%
\pgfpathlineto{\pgfqpoint{3.027013in}{1.673286in}}%
\pgfpathlineto{\pgfqpoint{3.028928in}{1.679975in}}%
\pgfpathlineto{\pgfqpoint{3.032759in}{1.703650in}}%
\pgfpathlineto{\pgfqpoint{3.034674in}{1.709293in}}%
\pgfpathlineto{\pgfqpoint{3.036589in}{1.709480in}}%
\pgfpathlineto{\pgfqpoint{3.044250in}{1.716044in}}%
\pgfpathlineto{\pgfqpoint{3.046165in}{1.741674in}}%
\pgfpathlineto{\pgfqpoint{3.048080in}{1.742266in}}%
\pgfpathlineto{\pgfqpoint{3.051910in}{1.791670in}}%
\pgfpathlineto{\pgfqpoint{3.053825in}{1.826535in}}%
\pgfpathlineto{\pgfqpoint{3.053825in}{1.826535in}}%
\pgfusepath{stroke}%
\end{pgfscope}%
\begin{pgfscope}%
\pgfsetrectcap%
\pgfsetmiterjoin%
\pgfsetlinewidth{0.803000pt}%
\definecolor{currentstroke}{rgb}{0.000000,0.000000,0.000000}%
\pgfsetstrokecolor{currentstroke}%
\pgfsetdash{}{0pt}%
\pgfpathmoveto{\pgfqpoint{0.694334in}{0.523557in}}%
\pgfpathlineto{\pgfqpoint{0.694334in}{1.826535in}}%
\pgfusepath{stroke}%
\end{pgfscope}%
\begin{pgfscope}%
\pgfsetrectcap%
\pgfsetmiterjoin%
\pgfsetlinewidth{0.803000pt}%
\definecolor{currentstroke}{rgb}{0.000000,0.000000,0.000000}%
\pgfsetstrokecolor{currentstroke}%
\pgfsetdash{}{0pt}%
\pgfpathmoveto{\pgfqpoint{4.524677in}{0.523557in}}%
\pgfpathlineto{\pgfqpoint{4.524677in}{1.826535in}}%
\pgfusepath{stroke}%
\end{pgfscope}%
\begin{pgfscope}%
\pgfsetrectcap%
\pgfsetmiterjoin%
\pgfsetlinewidth{0.803000pt}%
\definecolor{currentstroke}{rgb}{0.000000,0.000000,0.000000}%
\pgfsetstrokecolor{currentstroke}%
\pgfsetdash{}{0pt}%
\pgfpathmoveto{\pgfqpoint{0.694334in}{0.523557in}}%
\pgfpathlineto{\pgfqpoint{4.524677in}{0.523557in}}%
\pgfusepath{stroke}%
\end{pgfscope}%
\begin{pgfscope}%
\pgfsetrectcap%
\pgfsetmiterjoin%
\pgfsetlinewidth{0.803000pt}%
\definecolor{currentstroke}{rgb}{0.000000,0.000000,0.000000}%
\pgfsetstrokecolor{currentstroke}%
\pgfsetdash{}{0pt}%
\pgfpathmoveto{\pgfqpoint{0.694334in}{1.826535in}}%
\pgfpathlineto{\pgfqpoint{4.524677in}{1.826535in}}%
\pgfusepath{stroke}%
\end{pgfscope}%
\begin{pgfscope}%
\pgfsetrectcap%
\pgfsetroundjoin%
\pgfsetlinewidth{1.003750pt}%
\definecolor{currentstroke}{rgb}{0.878431,0.878431,0.815686}%
\pgfsetstrokecolor{currentstroke}%
\pgfsetdash{}{0pt}%
\pgfpathmoveto{\pgfqpoint{3.867012in}{1.491422in}}%
\pgfpathlineto{\pgfqpoint{4.089235in}{1.491422in}}%
\pgfusepath{stroke}%
\end{pgfscope}%
\begin{pgfscope}%
\definecolor{textcolor}{rgb}{0.000000,0.000000,0.000000}%
\pgfsetstrokecolor{textcolor}%
\pgfsetfillcolor{textcolor}%
\pgftext[x=4.111457in,y=1.452533in,left,base]{\color{textcolor}\rmfamily\fontsize{8.000000}{9.600000}\selectfont T.}%
\end{pgfscope}%
\begin{pgfscope}%
\pgfsetbuttcap%
\pgfsetroundjoin%
\pgfsetlinewidth{1.003750pt}%
\definecolor{currentstroke}{rgb}{0.941176,0.627451,0.188235}%
\pgfsetstrokecolor{currentstroke}%
\pgfsetdash{{1.000000pt}{1.650000pt}}{0.000000pt}%
\pgfpathmoveto{\pgfqpoint{3.867012in}{1.347600in}}%
\pgfpathlineto{\pgfqpoint{4.089235in}{1.347600in}}%
\pgfusepath{stroke}%
\end{pgfscope}%
\begin{pgfscope}%
\definecolor{textcolor}{rgb}{0.000000,0.000000,0.000000}%
\pgfsetstrokecolor{textcolor}%
\pgfsetfillcolor{textcolor}%
\pgftext[x=4.111457in,y=1.308711in,left,base]{\color{textcolor}\rmfamily\fontsize{8.000000}{9.600000}\selectfont FlowC.}%
\end{pgfscope}%
\begin{pgfscope}%
\pgfsetbuttcap%
\pgfsetroundjoin%
\pgfsetlinewidth{1.003750pt}%
\definecolor{currentstroke}{rgb}{0.062745,0.000000,0.062745}%
\pgfsetstrokecolor{currentstroke}%
\pgfsetdash{{3.700000pt}{1.600000pt}}{0.000000pt}%
\pgfpathmoveto{\pgfqpoint{3.867012in}{1.203778in}}%
\pgfpathlineto{\pgfqpoint{4.089235in}{1.203778in}}%
\pgfusepath{stroke}%
\end{pgfscope}%
\begin{pgfscope}%
\definecolor{textcolor}{rgb}{0.000000,0.000000,0.000000}%
\pgfsetstrokecolor{textcolor}%
\pgfsetfillcolor{textcolor}%
\pgftext[x=4.111457in,y=1.164889in,left,base]{\color{textcolor}\rmfamily\fontsize{8.000000}{9.600000}\selectfont htd}%
\end{pgfscope}%
\begin{pgfscope}%
\pgfsetbuttcap%
\pgfsetroundjoin%
\pgfsetlinewidth{1.003750pt}%
\definecolor{currentstroke}{rgb}{0.811765,0.125490,0.125490}%
\pgfsetstrokecolor{currentstroke}%
\pgfsetdash{{1.000000pt}{1.650000pt}}{0.000000pt}%
\pgfpathmoveto{\pgfqpoint{3.867012in}{1.059956in}}%
\pgfpathlineto{\pgfqpoint{4.089235in}{1.059956in}}%
\pgfusepath{stroke}%
\end{pgfscope}%
\begin{pgfscope}%
\definecolor{textcolor}{rgb}{0.000000,0.000000,0.000000}%
\pgfsetstrokecolor{textcolor}%
\pgfsetfillcolor{textcolor}%
\pgftext[x=4.111457in,y=1.021067in,left,base]{\color{textcolor}\rmfamily\fontsize{8.000000}{9.600000}\selectfont Hicks}%
\end{pgfscope}%
\begin{pgfscope}%
\pgfsetrectcap%
\pgfsetroundjoin%
\pgfsetlinewidth{1.003750pt}%
\definecolor{currentstroke}{rgb}{0.000000,0.000000,0.376471}%
\pgfsetstrokecolor{currentstroke}%
\pgfsetdash{}{0pt}%
\pgfpathmoveto{\pgfqpoint{3.867012in}{0.916134in}}%
\pgfpathlineto{\pgfqpoint{4.089235in}{0.916134in}}%
\pgfusepath{stroke}%
\end{pgfscope}%
\begin{pgfscope}%
\definecolor{textcolor}{rgb}{0.000000,0.000000,0.000000}%
\pgfsetstrokecolor{textcolor}%
\pgfsetfillcolor{textcolor}%
\pgftext[x=4.111457in,y=0.877245in,left,base]{\color{textcolor}\rmfamily\fontsize{8.000000}{9.600000}\selectfont P3}%
\end{pgfscope}%
\begin{pgfscope}%
\pgfsetrectcap%
\pgfsetroundjoin%
\pgfsetlinewidth{1.003750pt}%
\definecolor{currentstroke}{rgb}{0.564706,0.564706,1.000000}%
\pgfsetstrokecolor{currentstroke}%
\pgfsetdash{}{0pt}%
\pgfpathmoveto{\pgfqpoint{3.867012in}{0.772312in}}%
\pgfpathlineto{\pgfqpoint{4.089235in}{0.772312in}}%
\pgfusepath{stroke}%
\end{pgfscope}%
\begin{pgfscope}%
\definecolor{textcolor}{rgb}{0.000000,0.000000,0.000000}%
\pgfsetstrokecolor{textcolor}%
\pgfsetfillcolor{textcolor}%
\pgftext[x=4.111457in,y=0.733423in,left,base]{\color{textcolor}\rmfamily\fontsize{8.000000}{9.600000}\selectfont P4}%
\end{pgfscope}%
\begin{pgfscope}%
\pgfsetbuttcap%
\pgfsetroundjoin%
\pgfsetlinewidth{1.003750pt}%
\definecolor{currentstroke}{rgb}{0.000000,0.000000,0.000000}%
\pgfsetstrokecolor{currentstroke}%
\pgfsetdash{{1.000000pt}{1.650000pt}}{0.000000pt}%
\pgfpathmoveto{\pgfqpoint{3.867012in}{0.628490in}}%
\pgfpathlineto{\pgfqpoint{4.089235in}{0.628490in}}%
\pgfusepath{stroke}%
\end{pgfscope}%
\begin{pgfscope}%
\definecolor{textcolor}{rgb}{0.000000,0.000000,0.000000}%
\pgfsetstrokecolor{textcolor}%
\pgfsetfillcolor{textcolor}%
\pgftext[x=4.111457in,y=0.589601in,left,base]{\color{textcolor}\rmfamily\fontsize{8.000000}{9.600000}\selectfont VBS}%
\end{pgfscope}%
\end{pgfpicture}%
\makeatother%
\endgroup%

%\includegraphics[height=2in,width=2.5in]{figures/planning.pdf}
%\caption{\label{fig:parallel:planning} A cactus plot of the performance of various planners. A planner ``solves'' a benchmark when it finds a contraction tree of max rank 30 or smaller.}
%\end{center}
%\end{figure}


We observe that the parallel portfolio planners outperform all four single-core planners after 5 seconds. In fact, after 20 seconds both portfolios perform almost as well as the virtual best solver. We conclude that portfolio solvers significantly speed up the planning phase.

We also observe that \pkg{P3} and \pkg{P4} perform almost identically in Figure \ref{fig:parallel:planning}. Although after 1000 seconds \pkg{P4} has found better contraction trees than \pkg{P3} on 407 benchmarks, most improvements are small (reducing the max-rank by 1 or 2) or still do not result in good-enough contraction trees. We conclude that adding \pkg{Hicks} improves the portfolio slightly, but not significantly.
 
\subsection{Experiment 2: Determining the Performance Factor (RQ3)}
\label{sec:experiments:pf}
We take each contraction tree discovered in Experiment 1 (with max-rank below 36) and use \tool{TensorOrder2} to execute the tree with a timeout of 1000 seconds on each of the three hardware configurations in eager execution mode (\pkg{CPU1}, \pkg{CPU8}, and \pkg{GPU}). We observe that the max-rank of almost all solved contraction trees is 30 or smaller.

Given a performance factor, a benchmark, and a planner, we use the planning times from Experiment 1 to determine which contraction tree would have been chosen in step 4 of Algorithm \ref{alg:wmc}. We then add the execution time of the relevant contraction tree on each hardware. In this way, we simulate Algorithm \ref{alg:wmc} for a given planner and hardware with many performance factors. 

\begin{figure}[t]
\begin{center}
%% Creator: Matplotlib, PGF backend
%%
%% To include the figure in your LaTeX document, write
%%   \input{<filename>.pgf}
%%
%% Make sure the required packages are loaded in your preamble
%%   \usepackage{pgf}
%%
%% and, on pdftex
%%   \usepackage[utf8]{inputenc}\DeclareUnicodeCharacter{2212}{-}
%%
%% or, on luatex and xetex
%%   \usepackage{unicode-math}
%%
%% Figures using additional raster images can only be included by \input if
%% they are in the same directory as the main LaTeX file. For loading figures
%% from other directories you can use the `import` package
%%   \usepackage{import}
%%
%% and then include the figures with
%%   \import{<path to file>}{<filename>.pgf}
%%
%% Matplotlib used the following preamble
%%   \usepackage[utf8x]{inputenc}
%%   \usepackage[T1]{fontenc}
%%
\begingroup%
\makeatletter%
\begin{pgfpicture}%
\pgfpathrectangle{\pgfpointorigin}{\pgfqpoint{6.000000in}{2.500000in}}%
\pgfusepath{use as bounding box, clip}%
\begin{pgfscope}%
\pgfsetbuttcap%
\pgfsetmiterjoin%
\definecolor{currentfill}{rgb}{1.000000,1.000000,1.000000}%
\pgfsetfillcolor{currentfill}%
\pgfsetlinewidth{0.000000pt}%
\definecolor{currentstroke}{rgb}{1.000000,1.000000,1.000000}%
\pgfsetstrokecolor{currentstroke}%
\pgfsetdash{}{0pt}%
\pgfpathmoveto{\pgfqpoint{0.000000in}{0.000000in}}%
\pgfpathlineto{\pgfqpoint{6.000000in}{0.000000in}}%
\pgfpathlineto{\pgfqpoint{6.000000in}{2.500000in}}%
\pgfpathlineto{\pgfqpoint{0.000000in}{2.500000in}}%
\pgfpathclose%
\pgfusepath{fill}%
\end{pgfscope}%
\begin{pgfscope}%
\pgfsetbuttcap%
\pgfsetmiterjoin%
\definecolor{currentfill}{rgb}{1.000000,1.000000,1.000000}%
\pgfsetfillcolor{currentfill}%
\pgfsetlinewidth{0.000000pt}%
\definecolor{currentstroke}{rgb}{0.000000,0.000000,0.000000}%
\pgfsetstrokecolor{currentstroke}%
\pgfsetstrokeopacity{0.000000}%
\pgfsetdash{}{0pt}%
\pgfpathmoveto{\pgfqpoint{0.589591in}{0.539182in}}%
\pgfpathlineto{\pgfqpoint{5.756830in}{0.539182in}}%
\pgfpathlineto{\pgfqpoint{5.756830in}{2.207310in}}%
\pgfpathlineto{\pgfqpoint{0.589591in}{2.207310in}}%
\pgfpathclose%
\pgfusepath{fill}%
\end{pgfscope}%
\begin{pgfscope}%
\pgfsetbuttcap%
\pgfsetroundjoin%
\definecolor{currentfill}{rgb}{0.000000,0.000000,0.000000}%
\pgfsetfillcolor{currentfill}%
\pgfsetlinewidth{0.803000pt}%
\definecolor{currentstroke}{rgb}{0.000000,0.000000,0.000000}%
\pgfsetstrokecolor{currentstroke}%
\pgfsetdash{}{0pt}%
\pgfsys@defobject{currentmarker}{\pgfqpoint{0.000000in}{-0.048611in}}{\pgfqpoint{0.000000in}{0.000000in}}{%
\pgfpathmoveto{\pgfqpoint{0.000000in}{0.000000in}}%
\pgfpathlineto{\pgfqpoint{0.000000in}{-0.048611in}}%
\pgfusepath{stroke,fill}%
}%
\begin{pgfscope}%
\pgfsys@transformshift{0.589591in}{0.539182in}%
\pgfsys@useobject{currentmarker}{}%
\end{pgfscope}%
\end{pgfscope}%
\begin{pgfscope}%
\definecolor{textcolor}{rgb}{0.000000,0.000000,0.000000}%
\pgfsetstrokecolor{textcolor}%
\pgfsetfillcolor{textcolor}%
\pgftext[x=0.589591in,y=0.441960in,,top]{\color{textcolor}\rmfamily\fontsize{9.000000}{10.800000}\selectfont \(\displaystyle {10^{-21}}\)}%
\end{pgfscope}%
\begin{pgfscope}%
\pgfsetbuttcap%
\pgfsetroundjoin%
\definecolor{currentfill}{rgb}{0.000000,0.000000,0.000000}%
\pgfsetfillcolor{currentfill}%
\pgfsetlinewidth{0.803000pt}%
\definecolor{currentstroke}{rgb}{0.000000,0.000000,0.000000}%
\pgfsetstrokecolor{currentstroke}%
\pgfsetdash{}{0pt}%
\pgfsys@defobject{currentmarker}{\pgfqpoint{0.000000in}{-0.048611in}}{\pgfqpoint{0.000000in}{0.000000in}}{%
\pgfpathmoveto{\pgfqpoint{0.000000in}{0.000000in}}%
\pgfpathlineto{\pgfqpoint{0.000000in}{-0.048611in}}%
\pgfusepath{stroke,fill}%
}%
\begin{pgfscope}%
\pgfsys@transformshift{1.327768in}{0.539182in}%
\pgfsys@useobject{currentmarker}{}%
\end{pgfscope}%
\end{pgfscope}%
\begin{pgfscope}%
\definecolor{textcolor}{rgb}{0.000000,0.000000,0.000000}%
\pgfsetstrokecolor{textcolor}%
\pgfsetfillcolor{textcolor}%
\pgftext[x=1.327768in,y=0.441960in,,top]{\color{textcolor}\rmfamily\fontsize{9.000000}{10.800000}\selectfont \(\displaystyle {10^{-18}}\)}%
\end{pgfscope}%
\begin{pgfscope}%
\pgfsetbuttcap%
\pgfsetroundjoin%
\definecolor{currentfill}{rgb}{0.000000,0.000000,0.000000}%
\pgfsetfillcolor{currentfill}%
\pgfsetlinewidth{0.803000pt}%
\definecolor{currentstroke}{rgb}{0.000000,0.000000,0.000000}%
\pgfsetstrokecolor{currentstroke}%
\pgfsetdash{}{0pt}%
\pgfsys@defobject{currentmarker}{\pgfqpoint{0.000000in}{-0.048611in}}{\pgfqpoint{0.000000in}{0.000000in}}{%
\pgfpathmoveto{\pgfqpoint{0.000000in}{0.000000in}}%
\pgfpathlineto{\pgfqpoint{0.000000in}{-0.048611in}}%
\pgfusepath{stroke,fill}%
}%
\begin{pgfscope}%
\pgfsys@transformshift{2.065945in}{0.539182in}%
\pgfsys@useobject{currentmarker}{}%
\end{pgfscope}%
\end{pgfscope}%
\begin{pgfscope}%
\definecolor{textcolor}{rgb}{0.000000,0.000000,0.000000}%
\pgfsetstrokecolor{textcolor}%
\pgfsetfillcolor{textcolor}%
\pgftext[x=2.065945in,y=0.441960in,,top]{\color{textcolor}\rmfamily\fontsize{9.000000}{10.800000}\selectfont \(\displaystyle {10^{-15}}\)}%
\end{pgfscope}%
\begin{pgfscope}%
\pgfsetbuttcap%
\pgfsetroundjoin%
\definecolor{currentfill}{rgb}{0.000000,0.000000,0.000000}%
\pgfsetfillcolor{currentfill}%
\pgfsetlinewidth{0.803000pt}%
\definecolor{currentstroke}{rgb}{0.000000,0.000000,0.000000}%
\pgfsetstrokecolor{currentstroke}%
\pgfsetdash{}{0pt}%
\pgfsys@defobject{currentmarker}{\pgfqpoint{0.000000in}{-0.048611in}}{\pgfqpoint{0.000000in}{0.000000in}}{%
\pgfpathmoveto{\pgfqpoint{0.000000in}{0.000000in}}%
\pgfpathlineto{\pgfqpoint{0.000000in}{-0.048611in}}%
\pgfusepath{stroke,fill}%
}%
\begin{pgfscope}%
\pgfsys@transformshift{2.804122in}{0.539182in}%
\pgfsys@useobject{currentmarker}{}%
\end{pgfscope}%
\end{pgfscope}%
\begin{pgfscope}%
\definecolor{textcolor}{rgb}{0.000000,0.000000,0.000000}%
\pgfsetstrokecolor{textcolor}%
\pgfsetfillcolor{textcolor}%
\pgftext[x=2.804122in,y=0.441960in,,top]{\color{textcolor}\rmfamily\fontsize{9.000000}{10.800000}\selectfont \(\displaystyle {10^{-12}}\)}%
\end{pgfscope}%
\begin{pgfscope}%
\pgfsetbuttcap%
\pgfsetroundjoin%
\definecolor{currentfill}{rgb}{0.000000,0.000000,0.000000}%
\pgfsetfillcolor{currentfill}%
\pgfsetlinewidth{0.803000pt}%
\definecolor{currentstroke}{rgb}{0.000000,0.000000,0.000000}%
\pgfsetstrokecolor{currentstroke}%
\pgfsetdash{}{0pt}%
\pgfsys@defobject{currentmarker}{\pgfqpoint{0.000000in}{-0.048611in}}{\pgfqpoint{0.000000in}{0.000000in}}{%
\pgfpathmoveto{\pgfqpoint{0.000000in}{0.000000in}}%
\pgfpathlineto{\pgfqpoint{0.000000in}{-0.048611in}}%
\pgfusepath{stroke,fill}%
}%
\begin{pgfscope}%
\pgfsys@transformshift{3.542299in}{0.539182in}%
\pgfsys@useobject{currentmarker}{}%
\end{pgfscope}%
\end{pgfscope}%
\begin{pgfscope}%
\definecolor{textcolor}{rgb}{0.000000,0.000000,0.000000}%
\pgfsetstrokecolor{textcolor}%
\pgfsetfillcolor{textcolor}%
\pgftext[x=3.542299in,y=0.441960in,,top]{\color{textcolor}\rmfamily\fontsize{9.000000}{10.800000}\selectfont \(\displaystyle {10^{-9}}\)}%
\end{pgfscope}%
\begin{pgfscope}%
\pgfsetbuttcap%
\pgfsetroundjoin%
\definecolor{currentfill}{rgb}{0.000000,0.000000,0.000000}%
\pgfsetfillcolor{currentfill}%
\pgfsetlinewidth{0.803000pt}%
\definecolor{currentstroke}{rgb}{0.000000,0.000000,0.000000}%
\pgfsetstrokecolor{currentstroke}%
\pgfsetdash{}{0pt}%
\pgfsys@defobject{currentmarker}{\pgfqpoint{0.000000in}{-0.048611in}}{\pgfqpoint{0.000000in}{0.000000in}}{%
\pgfpathmoveto{\pgfqpoint{0.000000in}{0.000000in}}%
\pgfpathlineto{\pgfqpoint{0.000000in}{-0.048611in}}%
\pgfusepath{stroke,fill}%
}%
\begin{pgfscope}%
\pgfsys@transformshift{4.280476in}{0.539182in}%
\pgfsys@useobject{currentmarker}{}%
\end{pgfscope}%
\end{pgfscope}%
\begin{pgfscope}%
\definecolor{textcolor}{rgb}{0.000000,0.000000,0.000000}%
\pgfsetstrokecolor{textcolor}%
\pgfsetfillcolor{textcolor}%
\pgftext[x=4.280476in,y=0.441960in,,top]{\color{textcolor}\rmfamily\fontsize{9.000000}{10.800000}\selectfont \(\displaystyle {10^{-6}}\)}%
\end{pgfscope}%
\begin{pgfscope}%
\pgfsetbuttcap%
\pgfsetroundjoin%
\definecolor{currentfill}{rgb}{0.000000,0.000000,0.000000}%
\pgfsetfillcolor{currentfill}%
\pgfsetlinewidth{0.803000pt}%
\definecolor{currentstroke}{rgb}{0.000000,0.000000,0.000000}%
\pgfsetstrokecolor{currentstroke}%
\pgfsetdash{}{0pt}%
\pgfsys@defobject{currentmarker}{\pgfqpoint{0.000000in}{-0.048611in}}{\pgfqpoint{0.000000in}{0.000000in}}{%
\pgfpathmoveto{\pgfqpoint{0.000000in}{0.000000in}}%
\pgfpathlineto{\pgfqpoint{0.000000in}{-0.048611in}}%
\pgfusepath{stroke,fill}%
}%
\begin{pgfscope}%
\pgfsys@transformshift{5.018653in}{0.539182in}%
\pgfsys@useobject{currentmarker}{}%
\end{pgfscope}%
\end{pgfscope}%
\begin{pgfscope}%
\definecolor{textcolor}{rgb}{0.000000,0.000000,0.000000}%
\pgfsetstrokecolor{textcolor}%
\pgfsetfillcolor{textcolor}%
\pgftext[x=5.018653in,y=0.441960in,,top]{\color{textcolor}\rmfamily\fontsize{9.000000}{10.800000}\selectfont \(\displaystyle {10^{-3}}\)}%
\end{pgfscope}%
\begin{pgfscope}%
\pgfsetbuttcap%
\pgfsetroundjoin%
\definecolor{currentfill}{rgb}{0.000000,0.000000,0.000000}%
\pgfsetfillcolor{currentfill}%
\pgfsetlinewidth{0.803000pt}%
\definecolor{currentstroke}{rgb}{0.000000,0.000000,0.000000}%
\pgfsetstrokecolor{currentstroke}%
\pgfsetdash{}{0pt}%
\pgfsys@defobject{currentmarker}{\pgfqpoint{0.000000in}{-0.048611in}}{\pgfqpoint{0.000000in}{0.000000in}}{%
\pgfpathmoveto{\pgfqpoint{0.000000in}{0.000000in}}%
\pgfpathlineto{\pgfqpoint{0.000000in}{-0.048611in}}%
\pgfusepath{stroke,fill}%
}%
\begin{pgfscope}%
\pgfsys@transformshift{5.756830in}{0.539182in}%
\pgfsys@useobject{currentmarker}{}%
\end{pgfscope}%
\end{pgfscope}%
\begin{pgfscope}%
\definecolor{textcolor}{rgb}{0.000000,0.000000,0.000000}%
\pgfsetstrokecolor{textcolor}%
\pgfsetfillcolor{textcolor}%
\pgftext[x=5.756830in,y=0.441960in,,top]{\color{textcolor}\rmfamily\fontsize{9.000000}{10.800000}\selectfont \(\displaystyle {10^{0}}\)}%
\end{pgfscope}%
\begin{pgfscope}%
\definecolor{textcolor}{rgb}{0.000000,0.000000,0.000000}%
\pgfsetstrokecolor{textcolor}%
\pgfsetfillcolor{textcolor}%
\pgftext[x=3.173210in,y=0.272655in,,top]{\color{textcolor}\rmfamily\fontsize{10.000000}{12.000000}\selectfont Performance factor}%
\end{pgfscope}%
\begin{pgfscope}%
\pgfsetbuttcap%
\pgfsetroundjoin%
\definecolor{currentfill}{rgb}{0.000000,0.000000,0.000000}%
\pgfsetfillcolor{currentfill}%
\pgfsetlinewidth{0.803000pt}%
\definecolor{currentstroke}{rgb}{0.000000,0.000000,0.000000}%
\pgfsetstrokecolor{currentstroke}%
\pgfsetdash{}{0pt}%
\pgfsys@defobject{currentmarker}{\pgfqpoint{-0.048611in}{0.000000in}}{\pgfqpoint{-0.000000in}{0.000000in}}{%
\pgfpathmoveto{\pgfqpoint{-0.000000in}{0.000000in}}%
\pgfpathlineto{\pgfqpoint{-0.048611in}{0.000000in}}%
\pgfusepath{stroke,fill}%
}%
\begin{pgfscope}%
\pgfsys@transformshift{0.589591in}{0.539182in}%
\pgfsys@useobject{currentmarker}{}%
\end{pgfscope}%
\end{pgfscope}%
\begin{pgfscope}%
\definecolor{textcolor}{rgb}{0.000000,0.000000,0.000000}%
\pgfsetstrokecolor{textcolor}%
\pgfsetfillcolor{textcolor}%
\pgftext[x=0.328211in, y=0.496137in, left, base]{\color{textcolor}\rmfamily\fontsize{9.000000}{10.800000}\selectfont \(\displaystyle {1.0}\)}%
\end{pgfscope}%
\begin{pgfscope}%
\pgfsetbuttcap%
\pgfsetroundjoin%
\definecolor{currentfill}{rgb}{0.000000,0.000000,0.000000}%
\pgfsetfillcolor{currentfill}%
\pgfsetlinewidth{0.803000pt}%
\definecolor{currentstroke}{rgb}{0.000000,0.000000,0.000000}%
\pgfsetstrokecolor{currentstroke}%
\pgfsetdash{}{0pt}%
\pgfsys@defobject{currentmarker}{\pgfqpoint{-0.048611in}{0.000000in}}{\pgfqpoint{-0.000000in}{0.000000in}}{%
\pgfpathmoveto{\pgfqpoint{-0.000000in}{0.000000in}}%
\pgfpathlineto{\pgfqpoint{-0.048611in}{0.000000in}}%
\pgfusepath{stroke,fill}%
}%
\begin{pgfscope}%
\pgfsys@transformshift{0.589591in}{0.817203in}%
\pgfsys@useobject{currentmarker}{}%
\end{pgfscope}%
\end{pgfscope}%
\begin{pgfscope}%
\definecolor{textcolor}{rgb}{0.000000,0.000000,0.000000}%
\pgfsetstrokecolor{textcolor}%
\pgfsetfillcolor{textcolor}%
\pgftext[x=0.328211in, y=0.774158in, left, base]{\color{textcolor}\rmfamily\fontsize{9.000000}{10.800000}\selectfont \(\displaystyle {1.5}\)}%
\end{pgfscope}%
\begin{pgfscope}%
\pgfsetbuttcap%
\pgfsetroundjoin%
\definecolor{currentfill}{rgb}{0.000000,0.000000,0.000000}%
\pgfsetfillcolor{currentfill}%
\pgfsetlinewidth{0.803000pt}%
\definecolor{currentstroke}{rgb}{0.000000,0.000000,0.000000}%
\pgfsetstrokecolor{currentstroke}%
\pgfsetdash{}{0pt}%
\pgfsys@defobject{currentmarker}{\pgfqpoint{-0.048611in}{0.000000in}}{\pgfqpoint{-0.000000in}{0.000000in}}{%
\pgfpathmoveto{\pgfqpoint{-0.000000in}{0.000000in}}%
\pgfpathlineto{\pgfqpoint{-0.048611in}{0.000000in}}%
\pgfusepath{stroke,fill}%
}%
\begin{pgfscope}%
\pgfsys@transformshift{0.589591in}{1.095225in}%
\pgfsys@useobject{currentmarker}{}%
\end{pgfscope}%
\end{pgfscope}%
\begin{pgfscope}%
\definecolor{textcolor}{rgb}{0.000000,0.000000,0.000000}%
\pgfsetstrokecolor{textcolor}%
\pgfsetfillcolor{textcolor}%
\pgftext[x=0.328211in, y=1.052180in, left, base]{\color{textcolor}\rmfamily\fontsize{9.000000}{10.800000}\selectfont \(\displaystyle {2.0}\)}%
\end{pgfscope}%
\begin{pgfscope}%
\pgfsetbuttcap%
\pgfsetroundjoin%
\definecolor{currentfill}{rgb}{0.000000,0.000000,0.000000}%
\pgfsetfillcolor{currentfill}%
\pgfsetlinewidth{0.803000pt}%
\definecolor{currentstroke}{rgb}{0.000000,0.000000,0.000000}%
\pgfsetstrokecolor{currentstroke}%
\pgfsetdash{}{0pt}%
\pgfsys@defobject{currentmarker}{\pgfqpoint{-0.048611in}{0.000000in}}{\pgfqpoint{-0.000000in}{0.000000in}}{%
\pgfpathmoveto{\pgfqpoint{-0.000000in}{0.000000in}}%
\pgfpathlineto{\pgfqpoint{-0.048611in}{0.000000in}}%
\pgfusepath{stroke,fill}%
}%
\begin{pgfscope}%
\pgfsys@transformshift{0.589591in}{1.373246in}%
\pgfsys@useobject{currentmarker}{}%
\end{pgfscope}%
\end{pgfscope}%
\begin{pgfscope}%
\definecolor{textcolor}{rgb}{0.000000,0.000000,0.000000}%
\pgfsetstrokecolor{textcolor}%
\pgfsetfillcolor{textcolor}%
\pgftext[x=0.328211in, y=1.330201in, left, base]{\color{textcolor}\rmfamily\fontsize{9.000000}{10.800000}\selectfont \(\displaystyle {2.5}\)}%
\end{pgfscope}%
\begin{pgfscope}%
\pgfsetbuttcap%
\pgfsetroundjoin%
\definecolor{currentfill}{rgb}{0.000000,0.000000,0.000000}%
\pgfsetfillcolor{currentfill}%
\pgfsetlinewidth{0.803000pt}%
\definecolor{currentstroke}{rgb}{0.000000,0.000000,0.000000}%
\pgfsetstrokecolor{currentstroke}%
\pgfsetdash{}{0pt}%
\pgfsys@defobject{currentmarker}{\pgfqpoint{-0.048611in}{0.000000in}}{\pgfqpoint{-0.000000in}{0.000000in}}{%
\pgfpathmoveto{\pgfqpoint{-0.000000in}{0.000000in}}%
\pgfpathlineto{\pgfqpoint{-0.048611in}{0.000000in}}%
\pgfusepath{stroke,fill}%
}%
\begin{pgfscope}%
\pgfsys@transformshift{0.589591in}{1.651267in}%
\pgfsys@useobject{currentmarker}{}%
\end{pgfscope}%
\end{pgfscope}%
\begin{pgfscope}%
\definecolor{textcolor}{rgb}{0.000000,0.000000,0.000000}%
\pgfsetstrokecolor{textcolor}%
\pgfsetfillcolor{textcolor}%
\pgftext[x=0.328211in, y=1.608222in, left, base]{\color{textcolor}\rmfamily\fontsize{9.000000}{10.800000}\selectfont \(\displaystyle {3.0}\)}%
\end{pgfscope}%
\begin{pgfscope}%
\pgfsetbuttcap%
\pgfsetroundjoin%
\definecolor{currentfill}{rgb}{0.000000,0.000000,0.000000}%
\pgfsetfillcolor{currentfill}%
\pgfsetlinewidth{0.803000pt}%
\definecolor{currentstroke}{rgb}{0.000000,0.000000,0.000000}%
\pgfsetstrokecolor{currentstroke}%
\pgfsetdash{}{0pt}%
\pgfsys@defobject{currentmarker}{\pgfqpoint{-0.048611in}{0.000000in}}{\pgfqpoint{-0.000000in}{0.000000in}}{%
\pgfpathmoveto{\pgfqpoint{-0.000000in}{0.000000in}}%
\pgfpathlineto{\pgfqpoint{-0.048611in}{0.000000in}}%
\pgfusepath{stroke,fill}%
}%
\begin{pgfscope}%
\pgfsys@transformshift{0.589591in}{1.929289in}%
\pgfsys@useobject{currentmarker}{}%
\end{pgfscope}%
\end{pgfscope}%
\begin{pgfscope}%
\definecolor{textcolor}{rgb}{0.000000,0.000000,0.000000}%
\pgfsetstrokecolor{textcolor}%
\pgfsetfillcolor{textcolor}%
\pgftext[x=0.328211in, y=1.886244in, left, base]{\color{textcolor}\rmfamily\fontsize{9.000000}{10.800000}\selectfont \(\displaystyle {3.5}\)}%
\end{pgfscope}%
\begin{pgfscope}%
\pgfsetbuttcap%
\pgfsetroundjoin%
\definecolor{currentfill}{rgb}{0.000000,0.000000,0.000000}%
\pgfsetfillcolor{currentfill}%
\pgfsetlinewidth{0.803000pt}%
\definecolor{currentstroke}{rgb}{0.000000,0.000000,0.000000}%
\pgfsetstrokecolor{currentstroke}%
\pgfsetdash{}{0pt}%
\pgfsys@defobject{currentmarker}{\pgfqpoint{-0.048611in}{0.000000in}}{\pgfqpoint{-0.000000in}{0.000000in}}{%
\pgfpathmoveto{\pgfqpoint{-0.000000in}{0.000000in}}%
\pgfpathlineto{\pgfqpoint{-0.048611in}{0.000000in}}%
\pgfusepath{stroke,fill}%
}%
\begin{pgfscope}%
\pgfsys@transformshift{0.589591in}{2.207310in}%
\pgfsys@useobject{currentmarker}{}%
\end{pgfscope}%
\end{pgfscope}%
\begin{pgfscope}%
\definecolor{textcolor}{rgb}{0.000000,0.000000,0.000000}%
\pgfsetstrokecolor{textcolor}%
\pgfsetfillcolor{textcolor}%
\pgftext[x=0.328211in, y=2.164265in, left, base]{\color{textcolor}\rmfamily\fontsize{9.000000}{10.800000}\selectfont \(\displaystyle {4.0}\)}%
\end{pgfscope}%
\begin{pgfscope}%
\definecolor{textcolor}{rgb}{0.000000,0.000000,0.000000}%
\pgfsetstrokecolor{textcolor}%
\pgfsetfillcolor{textcolor}%
\pgftext[x=0.272655in,y=1.373246in,,bottom,rotate=90.000000]{\color{textcolor}\rmfamily\fontsize{10.000000}{12.000000}\selectfont Par-2 Score}%
\end{pgfscope}%
\begin{pgfscope}%
\definecolor{textcolor}{rgb}{0.000000,0.000000,0.000000}%
\pgfsetstrokecolor{textcolor}%
\pgfsetfillcolor{textcolor}%
\pgftext[x=0.589591in,y=2.248977in,left,base]{\color{textcolor}\rmfamily\fontsize{9.000000}{10.800000}\selectfont \(\displaystyle \times{10^{6}}{}\)}%
\end{pgfscope}%
\begin{pgfscope}%
\pgfpathrectangle{\pgfqpoint{0.589591in}{0.539182in}}{\pgfqpoint{5.167239in}{1.668128in}}%
\pgfusepath{clip}%
\pgfsetrectcap%
\pgfsetroundjoin%
\pgfsetlinewidth{2.007500pt}%
\definecolor{currentstroke}{rgb}{0.878431,0.878431,0.815686}%
\pgfsetstrokecolor{currentstroke}%
\pgfsetdash{}{0pt}%
\pgfpathmoveto{\pgfqpoint{0.589591in}{1.040488in}}%
\pgfpathlineto{\pgfqpoint{2.480360in}{1.039663in}}%
\pgfpathlineto{\pgfqpoint{2.493310in}{1.037549in}}%
\pgfpathlineto{\pgfqpoint{2.519211in}{1.026804in}}%
\pgfpathlineto{\pgfqpoint{2.532162in}{1.024864in}}%
\pgfpathlineto{\pgfqpoint{2.558063in}{1.013854in}}%
\pgfpathlineto{\pgfqpoint{2.583964in}{1.012934in}}%
\pgfpathlineto{\pgfqpoint{2.596914in}{1.012936in}}%
\pgfpathlineto{\pgfqpoint{2.622815in}{1.006427in}}%
\pgfpathlineto{\pgfqpoint{2.635766in}{1.004050in}}%
\pgfpathlineto{\pgfqpoint{2.648716in}{1.003890in}}%
\pgfpathlineto{\pgfqpoint{2.661666in}{1.000309in}}%
\pgfpathlineto{\pgfqpoint{2.674617in}{0.998233in}}%
\pgfpathlineto{\pgfqpoint{2.687567in}{0.992822in}}%
\pgfpathlineto{\pgfqpoint{2.713468in}{0.985138in}}%
\pgfpathlineto{\pgfqpoint{2.739369in}{0.983393in}}%
\pgfpathlineto{\pgfqpoint{2.778221in}{0.982029in}}%
\pgfpathlineto{\pgfqpoint{2.791171in}{0.980952in}}%
\pgfpathlineto{\pgfqpoint{2.804122in}{0.978287in}}%
\pgfpathlineto{\pgfqpoint{2.817072in}{0.977749in}}%
\pgfpathlineto{\pgfqpoint{2.830023in}{0.974856in}}%
\pgfpathlineto{\pgfqpoint{2.855924in}{0.966158in}}%
\pgfpathlineto{\pgfqpoint{2.868874in}{0.963842in}}%
\pgfpathlineto{\pgfqpoint{2.920676in}{0.960755in}}%
\pgfpathlineto{\pgfqpoint{2.972478in}{0.957262in}}%
\pgfpathlineto{\pgfqpoint{3.011329in}{0.957113in}}%
\pgfpathlineto{\pgfqpoint{3.024280in}{0.954684in}}%
\pgfpathlineto{\pgfqpoint{3.037230in}{0.954823in}}%
\pgfpathlineto{\pgfqpoint{3.050181in}{0.953507in}}%
\pgfpathlineto{\pgfqpoint{3.140834in}{0.953633in}}%
\pgfpathlineto{\pgfqpoint{3.153784in}{0.948381in}}%
\pgfpathlineto{\pgfqpoint{3.179685in}{0.948061in}}%
\pgfpathlineto{\pgfqpoint{3.192636in}{0.946396in}}%
\pgfpathlineto{\pgfqpoint{3.218537in}{0.949757in}}%
\pgfpathlineto{\pgfqpoint{3.257388in}{0.955597in}}%
\pgfpathlineto{\pgfqpoint{3.309190in}{0.964278in}}%
\pgfpathlineto{\pgfqpoint{3.322141in}{0.967193in}}%
\pgfpathlineto{\pgfqpoint{3.335091in}{0.972583in}}%
\pgfpathlineto{\pgfqpoint{3.348042in}{0.981004in}}%
\pgfpathlineto{\pgfqpoint{3.360992in}{0.983059in}}%
\pgfpathlineto{\pgfqpoint{3.373943in}{0.986532in}}%
\pgfpathlineto{\pgfqpoint{3.386893in}{0.992476in}}%
\pgfpathlineto{\pgfqpoint{3.412794in}{1.002130in}}%
\pgfpathlineto{\pgfqpoint{3.425744in}{1.007404in}}%
\pgfpathlineto{\pgfqpoint{3.451645in}{1.022222in}}%
\pgfpathlineto{\pgfqpoint{3.464596in}{1.025961in}}%
\pgfpathlineto{\pgfqpoint{3.477546in}{1.031902in}}%
\pgfpathlineto{\pgfqpoint{3.490497in}{1.036027in}}%
\pgfpathlineto{\pgfqpoint{3.503447in}{1.041360in}}%
\pgfpathlineto{\pgfqpoint{3.516398in}{1.049876in}}%
\pgfpathlineto{\pgfqpoint{3.529348in}{1.056469in}}%
\pgfpathlineto{\pgfqpoint{3.555249in}{1.064256in}}%
\pgfpathlineto{\pgfqpoint{3.568200in}{1.069172in}}%
\pgfpathlineto{\pgfqpoint{3.594101in}{1.087396in}}%
\pgfpathlineto{\pgfqpoint{3.620001in}{1.095614in}}%
\pgfpathlineto{\pgfqpoint{3.632952in}{1.102032in}}%
\pgfpathlineto{\pgfqpoint{3.658853in}{1.111848in}}%
\pgfpathlineto{\pgfqpoint{3.671803in}{1.121528in}}%
\pgfpathlineto{\pgfqpoint{3.710655in}{1.139906in}}%
\pgfpathlineto{\pgfqpoint{3.723605in}{1.150725in}}%
\pgfpathlineto{\pgfqpoint{3.736556in}{1.164632in}}%
\pgfpathlineto{\pgfqpoint{3.749506in}{1.176624in}}%
\pgfpathlineto{\pgfqpoint{3.775407in}{1.195470in}}%
\pgfpathlineto{\pgfqpoint{3.788358in}{1.206382in}}%
\pgfpathlineto{\pgfqpoint{3.801308in}{1.213930in}}%
\pgfpathlineto{\pgfqpoint{3.814259in}{1.223198in}}%
\pgfpathlineto{\pgfqpoint{3.827209in}{1.241666in}}%
\pgfpathlineto{\pgfqpoint{3.840160in}{1.253000in}}%
\pgfpathlineto{\pgfqpoint{3.853110in}{1.262622in}}%
\pgfpathlineto{\pgfqpoint{3.866060in}{1.274948in}}%
\pgfpathlineto{\pgfqpoint{3.879011in}{1.288843in}}%
\pgfpathlineto{\pgfqpoint{3.891961in}{1.295646in}}%
\pgfpathlineto{\pgfqpoint{3.904912in}{1.312084in}}%
\pgfpathlineto{\pgfqpoint{3.917862in}{1.317889in}}%
\pgfpathlineto{\pgfqpoint{3.930813in}{1.330949in}}%
\pgfpathlineto{\pgfqpoint{3.969664in}{1.360481in}}%
\pgfpathlineto{\pgfqpoint{3.982615in}{1.372694in}}%
\pgfpathlineto{\pgfqpoint{3.995565in}{1.388388in}}%
\pgfpathlineto{\pgfqpoint{4.021466in}{1.405892in}}%
\pgfpathlineto{\pgfqpoint{4.034417in}{1.418422in}}%
\pgfpathlineto{\pgfqpoint{4.060318in}{1.433322in}}%
\pgfpathlineto{\pgfqpoint{4.073268in}{1.442226in}}%
\pgfpathlineto{\pgfqpoint{4.086219in}{1.454238in}}%
\pgfpathlineto{\pgfqpoint{4.099169in}{1.474156in}}%
\pgfpathlineto{\pgfqpoint{4.125070in}{1.488553in}}%
\pgfpathlineto{\pgfqpoint{4.138020in}{1.496902in}}%
\pgfpathlineto{\pgfqpoint{4.150971in}{1.503411in}}%
\pgfpathlineto{\pgfqpoint{4.163921in}{1.511942in}}%
\pgfpathlineto{\pgfqpoint{4.176872in}{1.534191in}}%
\pgfpathlineto{\pgfqpoint{4.189822in}{1.542153in}}%
\pgfpathlineto{\pgfqpoint{4.202773in}{1.547057in}}%
\pgfpathlineto{\pgfqpoint{4.228674in}{1.566072in}}%
\pgfpathlineto{\pgfqpoint{4.241624in}{1.575114in}}%
\pgfpathlineto{\pgfqpoint{4.254575in}{1.588344in}}%
\pgfpathlineto{\pgfqpoint{4.267525in}{1.596950in}}%
\pgfpathlineto{\pgfqpoint{4.280476in}{1.608742in}}%
\pgfpathlineto{\pgfqpoint{4.293426in}{1.613631in}}%
\pgfpathlineto{\pgfqpoint{4.306377in}{1.620590in}}%
\pgfpathlineto{\pgfqpoint{4.319327in}{1.631089in}}%
\pgfpathlineto{\pgfqpoint{4.345228in}{1.645738in}}%
\pgfpathlineto{\pgfqpoint{4.358178in}{1.662941in}}%
\pgfpathlineto{\pgfqpoint{4.384079in}{1.671680in}}%
\pgfpathlineto{\pgfqpoint{4.397030in}{1.681043in}}%
\pgfpathlineto{\pgfqpoint{4.409980in}{1.686577in}}%
\pgfpathlineto{\pgfqpoint{4.422931in}{1.694544in}}%
\pgfpathlineto{\pgfqpoint{4.435881in}{1.705597in}}%
\pgfpathlineto{\pgfqpoint{4.448832in}{1.718792in}}%
\pgfpathlineto{\pgfqpoint{4.474733in}{1.728012in}}%
\pgfpathlineto{\pgfqpoint{4.500634in}{1.742895in}}%
\pgfpathlineto{\pgfqpoint{4.513584in}{1.754285in}}%
\pgfpathlineto{\pgfqpoint{4.565386in}{1.766337in}}%
\pgfpathlineto{\pgfqpoint{4.604237in}{1.783060in}}%
\pgfpathlineto{\pgfqpoint{4.630138in}{1.797654in}}%
\pgfpathlineto{\pgfqpoint{4.643089in}{1.801893in}}%
\pgfpathlineto{\pgfqpoint{4.656039in}{1.807402in}}%
\pgfpathlineto{\pgfqpoint{4.668990in}{1.814779in}}%
\pgfpathlineto{\pgfqpoint{4.681940in}{1.826236in}}%
\pgfpathlineto{\pgfqpoint{4.694891in}{1.831666in}}%
\pgfpathlineto{\pgfqpoint{4.733742in}{1.844044in}}%
\pgfpathlineto{\pgfqpoint{4.746693in}{1.850300in}}%
\pgfpathlineto{\pgfqpoint{4.759643in}{1.862724in}}%
\pgfpathlineto{\pgfqpoint{4.772594in}{1.866630in}}%
\pgfpathlineto{\pgfqpoint{4.785544in}{1.871712in}}%
\pgfpathlineto{\pgfqpoint{4.798495in}{1.880722in}}%
\pgfpathlineto{\pgfqpoint{4.811445in}{1.887308in}}%
\pgfpathlineto{\pgfqpoint{4.837346in}{1.897288in}}%
\pgfpathlineto{\pgfqpoint{4.850296in}{1.908231in}}%
\pgfpathlineto{\pgfqpoint{4.863247in}{1.912059in}}%
\pgfpathlineto{\pgfqpoint{4.889148in}{1.916021in}}%
\pgfpathlineto{\pgfqpoint{4.940950in}{1.927327in}}%
\pgfpathlineto{\pgfqpoint{4.966851in}{1.936331in}}%
\pgfpathlineto{\pgfqpoint{4.992752in}{1.944095in}}%
\pgfpathlineto{\pgfqpoint{5.005702in}{1.964786in}}%
\pgfpathlineto{\pgfqpoint{5.031603in}{1.970852in}}%
\pgfpathlineto{\pgfqpoint{5.057504in}{1.975974in}}%
\pgfpathlineto{\pgfqpoint{5.083405in}{1.983277in}}%
\pgfpathlineto{\pgfqpoint{5.096355in}{1.987557in}}%
\pgfpathlineto{\pgfqpoint{5.109306in}{1.993093in}}%
\pgfpathlineto{\pgfqpoint{5.122256in}{2.001531in}}%
\pgfpathlineto{\pgfqpoint{5.148157in}{2.013109in}}%
\pgfpathlineto{\pgfqpoint{5.174058in}{2.035884in}}%
\pgfpathlineto{\pgfqpoint{5.187009in}{2.043029in}}%
\pgfpathlineto{\pgfqpoint{5.212910in}{2.049343in}}%
\pgfpathlineto{\pgfqpoint{5.238811in}{2.056006in}}%
\pgfpathlineto{\pgfqpoint{5.251761in}{2.057933in}}%
\pgfpathlineto{\pgfqpoint{5.277662in}{2.065237in}}%
\pgfpathlineto{\pgfqpoint{5.290612in}{2.069674in}}%
\pgfpathlineto{\pgfqpoint{5.355365in}{2.080347in}}%
\pgfpathlineto{\pgfqpoint{5.368315in}{2.081183in}}%
\pgfpathlineto{\pgfqpoint{5.381266in}{2.084245in}}%
\pgfpathlineto{\pgfqpoint{5.420117in}{2.088768in}}%
\pgfpathlineto{\pgfqpoint{5.484870in}{2.096828in}}%
\pgfpathlineto{\pgfqpoint{5.588473in}{2.104973in}}%
\pgfpathlineto{\pgfqpoint{5.601424in}{2.107539in}}%
\pgfpathlineto{\pgfqpoint{5.666176in}{2.109487in}}%
\pgfpathlineto{\pgfqpoint{5.705028in}{2.110835in}}%
\pgfpathlineto{\pgfqpoint{5.756830in}{2.111008in}}%
\pgfpathlineto{\pgfqpoint{5.756830in}{2.111008in}}%
\pgfusepath{stroke}%
\end{pgfscope}%
\begin{pgfscope}%
\pgfpathrectangle{\pgfqpoint{0.589591in}{0.539182in}}{\pgfqpoint{5.167239in}{1.668128in}}%
\pgfusepath{clip}%
\pgfsetrectcap%
\pgfsetroundjoin%
\pgfsetlinewidth{2.007500pt}%
\definecolor{currentstroke}{rgb}{0.564706,0.564706,1.000000}%
\pgfsetstrokecolor{currentstroke}%
\pgfsetdash{}{0pt}%
\pgfpathmoveto{\pgfqpoint{0.589591in}{1.141162in}}%
\pgfpathlineto{\pgfqpoint{2.091846in}{1.140419in}}%
\pgfpathlineto{\pgfqpoint{2.117747in}{1.139720in}}%
\pgfpathlineto{\pgfqpoint{2.143648in}{1.138656in}}%
\pgfpathlineto{\pgfqpoint{2.169549in}{1.135874in}}%
\pgfpathlineto{\pgfqpoint{2.195449in}{1.132690in}}%
\pgfpathlineto{\pgfqpoint{2.208400in}{1.129944in}}%
\pgfpathlineto{\pgfqpoint{2.221350in}{1.128853in}}%
\pgfpathlineto{\pgfqpoint{2.234301in}{1.122390in}}%
\pgfpathlineto{\pgfqpoint{2.247251in}{1.120663in}}%
\pgfpathlineto{\pgfqpoint{2.260202in}{1.117789in}}%
\pgfpathlineto{\pgfqpoint{2.273152in}{1.110911in}}%
\pgfpathlineto{\pgfqpoint{2.299053in}{1.100582in}}%
\pgfpathlineto{\pgfqpoint{2.337905in}{1.086978in}}%
\pgfpathlineto{\pgfqpoint{2.350855in}{1.083801in}}%
\pgfpathlineto{\pgfqpoint{2.363806in}{1.076697in}}%
\pgfpathlineto{\pgfqpoint{2.376756in}{1.071233in}}%
\pgfpathlineto{\pgfqpoint{2.389707in}{1.063530in}}%
\pgfpathlineto{\pgfqpoint{2.402657in}{1.060845in}}%
\pgfpathlineto{\pgfqpoint{2.415608in}{1.049817in}}%
\pgfpathlineto{\pgfqpoint{2.428558in}{1.044466in}}%
\pgfpathlineto{\pgfqpoint{2.441508in}{1.037224in}}%
\pgfpathlineto{\pgfqpoint{2.454459in}{1.036138in}}%
\pgfpathlineto{\pgfqpoint{2.493310in}{1.027783in}}%
\pgfpathlineto{\pgfqpoint{2.519211in}{1.019563in}}%
\pgfpathlineto{\pgfqpoint{2.532162in}{1.013324in}}%
\pgfpathlineto{\pgfqpoint{2.545112in}{1.010740in}}%
\pgfpathlineto{\pgfqpoint{2.558063in}{1.002716in}}%
\pgfpathlineto{\pgfqpoint{2.583964in}{0.994151in}}%
\pgfpathlineto{\pgfqpoint{2.596914in}{0.992079in}}%
\pgfpathlineto{\pgfqpoint{2.609865in}{0.987055in}}%
\pgfpathlineto{\pgfqpoint{2.622815in}{0.985092in}}%
\pgfpathlineto{\pgfqpoint{2.674617in}{0.965235in}}%
\pgfpathlineto{\pgfqpoint{2.713468in}{0.952577in}}%
\pgfpathlineto{\pgfqpoint{2.726419in}{0.950639in}}%
\pgfpathlineto{\pgfqpoint{2.739369in}{0.950322in}}%
\pgfpathlineto{\pgfqpoint{2.765270in}{0.938983in}}%
\pgfpathlineto{\pgfqpoint{2.778221in}{0.930496in}}%
\pgfpathlineto{\pgfqpoint{2.791171in}{0.924158in}}%
\pgfpathlineto{\pgfqpoint{2.804122in}{0.920495in}}%
\pgfpathlineto{\pgfqpoint{2.842973in}{0.917418in}}%
\pgfpathlineto{\pgfqpoint{2.868874in}{0.914342in}}%
\pgfpathlineto{\pgfqpoint{2.907725in}{0.910463in}}%
\pgfpathlineto{\pgfqpoint{2.920676in}{0.907141in}}%
\pgfpathlineto{\pgfqpoint{2.959527in}{0.904252in}}%
\pgfpathlineto{\pgfqpoint{3.050181in}{0.902195in}}%
\pgfpathlineto{\pgfqpoint{3.089032in}{0.901814in}}%
\pgfpathlineto{\pgfqpoint{3.101983in}{0.900079in}}%
\pgfpathlineto{\pgfqpoint{3.114933in}{0.900099in}}%
\pgfpathlineto{\pgfqpoint{3.127884in}{0.902018in}}%
\pgfpathlineto{\pgfqpoint{3.153784in}{0.902608in}}%
\pgfpathlineto{\pgfqpoint{3.166735in}{0.904886in}}%
\pgfpathlineto{\pgfqpoint{3.179685in}{0.908433in}}%
\pgfpathlineto{\pgfqpoint{3.192636in}{0.908297in}}%
\pgfpathlineto{\pgfqpoint{3.205586in}{0.910932in}}%
\pgfpathlineto{\pgfqpoint{3.257388in}{0.915278in}}%
\pgfpathlineto{\pgfqpoint{3.283289in}{0.920625in}}%
\pgfpathlineto{\pgfqpoint{3.296240in}{0.922503in}}%
\pgfpathlineto{\pgfqpoint{3.309190in}{0.926340in}}%
\pgfpathlineto{\pgfqpoint{3.322141in}{0.932551in}}%
\pgfpathlineto{\pgfqpoint{3.335091in}{0.941026in}}%
\pgfpathlineto{\pgfqpoint{3.348042in}{0.947534in}}%
\pgfpathlineto{\pgfqpoint{3.399843in}{0.959711in}}%
\pgfpathlineto{\pgfqpoint{3.438695in}{0.972242in}}%
\pgfpathlineto{\pgfqpoint{3.464596in}{0.982498in}}%
\pgfpathlineto{\pgfqpoint{3.477546in}{0.991676in}}%
\pgfpathlineto{\pgfqpoint{3.529348in}{1.019748in}}%
\pgfpathlineto{\pgfqpoint{3.542299in}{1.024188in}}%
\pgfpathlineto{\pgfqpoint{3.555249in}{1.035560in}}%
\pgfpathlineto{\pgfqpoint{3.568200in}{1.042492in}}%
\pgfpathlineto{\pgfqpoint{3.581150in}{1.054465in}}%
\pgfpathlineto{\pgfqpoint{3.632952in}{1.087766in}}%
\pgfpathlineto{\pgfqpoint{3.658853in}{1.102159in}}%
\pgfpathlineto{\pgfqpoint{3.671803in}{1.112971in}}%
\pgfpathlineto{\pgfqpoint{3.684754in}{1.120690in}}%
\pgfpathlineto{\pgfqpoint{3.697704in}{1.129840in}}%
\pgfpathlineto{\pgfqpoint{3.723605in}{1.154406in}}%
\pgfpathlineto{\pgfqpoint{3.749506in}{1.184057in}}%
\pgfpathlineto{\pgfqpoint{3.762457in}{1.192734in}}%
\pgfpathlineto{\pgfqpoint{3.775407in}{1.199450in}}%
\pgfpathlineto{\pgfqpoint{3.801308in}{1.220873in}}%
\pgfpathlineto{\pgfqpoint{3.814259in}{1.229454in}}%
\pgfpathlineto{\pgfqpoint{3.827209in}{1.245456in}}%
\pgfpathlineto{\pgfqpoint{3.840160in}{1.255485in}}%
\pgfpathlineto{\pgfqpoint{3.866060in}{1.273254in}}%
\pgfpathlineto{\pgfqpoint{3.879011in}{1.284423in}}%
\pgfpathlineto{\pgfqpoint{3.891961in}{1.292769in}}%
\pgfpathlineto{\pgfqpoint{3.904912in}{1.304929in}}%
\pgfpathlineto{\pgfqpoint{3.930813in}{1.321810in}}%
\pgfpathlineto{\pgfqpoint{3.943763in}{1.337512in}}%
\pgfpathlineto{\pgfqpoint{3.956714in}{1.349176in}}%
\pgfpathlineto{\pgfqpoint{3.969664in}{1.364440in}}%
\pgfpathlineto{\pgfqpoint{4.008516in}{1.401353in}}%
\pgfpathlineto{\pgfqpoint{4.086219in}{1.457848in}}%
\pgfpathlineto{\pgfqpoint{4.099169in}{1.469127in}}%
\pgfpathlineto{\pgfqpoint{4.112119in}{1.477693in}}%
\pgfpathlineto{\pgfqpoint{4.125070in}{1.487733in}}%
\pgfpathlineto{\pgfqpoint{4.163921in}{1.509046in}}%
\pgfpathlineto{\pgfqpoint{4.176872in}{1.524900in}}%
\pgfpathlineto{\pgfqpoint{4.189822in}{1.533802in}}%
\pgfpathlineto{\pgfqpoint{4.202773in}{1.541255in}}%
\pgfpathlineto{\pgfqpoint{4.241624in}{1.572368in}}%
\pgfpathlineto{\pgfqpoint{4.254575in}{1.586445in}}%
\pgfpathlineto{\pgfqpoint{4.267525in}{1.595374in}}%
\pgfpathlineto{\pgfqpoint{4.280476in}{1.607877in}}%
\pgfpathlineto{\pgfqpoint{4.293426in}{1.611901in}}%
\pgfpathlineto{\pgfqpoint{4.319327in}{1.630656in}}%
\pgfpathlineto{\pgfqpoint{4.332277in}{1.635457in}}%
\pgfpathlineto{\pgfqpoint{4.345228in}{1.643201in}}%
\pgfpathlineto{\pgfqpoint{4.358178in}{1.655988in}}%
\pgfpathlineto{\pgfqpoint{4.371129in}{1.661941in}}%
\pgfpathlineto{\pgfqpoint{4.397030in}{1.678987in}}%
\pgfpathlineto{\pgfqpoint{4.422931in}{1.695267in}}%
\pgfpathlineto{\pgfqpoint{4.435881in}{1.705715in}}%
\pgfpathlineto{\pgfqpoint{4.448832in}{1.713981in}}%
\pgfpathlineto{\pgfqpoint{4.461782in}{1.720798in}}%
\pgfpathlineto{\pgfqpoint{4.474733in}{1.724198in}}%
\pgfpathlineto{\pgfqpoint{4.487683in}{1.730033in}}%
\pgfpathlineto{\pgfqpoint{4.513584in}{1.748603in}}%
\pgfpathlineto{\pgfqpoint{4.526535in}{1.751192in}}%
\pgfpathlineto{\pgfqpoint{4.539485in}{1.755275in}}%
\pgfpathlineto{\pgfqpoint{4.552436in}{1.761050in}}%
\pgfpathlineto{\pgfqpoint{4.565386in}{1.768470in}}%
\pgfpathlineto{\pgfqpoint{4.578336in}{1.772277in}}%
\pgfpathlineto{\pgfqpoint{4.591287in}{1.779979in}}%
\pgfpathlineto{\pgfqpoint{4.604237in}{1.786004in}}%
\pgfpathlineto{\pgfqpoint{4.617188in}{1.794701in}}%
\pgfpathlineto{\pgfqpoint{4.630138in}{1.801451in}}%
\pgfpathlineto{\pgfqpoint{4.668990in}{1.816012in}}%
\pgfpathlineto{\pgfqpoint{4.681940in}{1.825544in}}%
\pgfpathlineto{\pgfqpoint{4.694891in}{1.832404in}}%
\pgfpathlineto{\pgfqpoint{4.720792in}{1.839876in}}%
\pgfpathlineto{\pgfqpoint{4.733742in}{1.843014in}}%
\pgfpathlineto{\pgfqpoint{4.746693in}{1.853995in}}%
\pgfpathlineto{\pgfqpoint{4.759643in}{1.861663in}}%
\pgfpathlineto{\pgfqpoint{4.772594in}{1.864908in}}%
\pgfpathlineto{\pgfqpoint{4.785544in}{1.869442in}}%
\pgfpathlineto{\pgfqpoint{4.811445in}{1.881714in}}%
\pgfpathlineto{\pgfqpoint{4.824395in}{1.891520in}}%
\pgfpathlineto{\pgfqpoint{4.837346in}{1.895146in}}%
\pgfpathlineto{\pgfqpoint{4.850296in}{1.903538in}}%
\pgfpathlineto{\pgfqpoint{4.876197in}{1.913769in}}%
\pgfpathlineto{\pgfqpoint{4.915049in}{1.922200in}}%
\pgfpathlineto{\pgfqpoint{4.927999in}{1.924724in}}%
\pgfpathlineto{\pgfqpoint{4.953900in}{1.936560in}}%
\pgfpathlineto{\pgfqpoint{4.966851in}{1.941313in}}%
\pgfpathlineto{\pgfqpoint{4.979801in}{1.944552in}}%
\pgfpathlineto{\pgfqpoint{4.992752in}{1.949605in}}%
\pgfpathlineto{\pgfqpoint{5.005702in}{1.958893in}}%
\pgfpathlineto{\pgfqpoint{5.031603in}{1.970028in}}%
\pgfpathlineto{\pgfqpoint{5.070454in}{1.978878in}}%
\pgfpathlineto{\pgfqpoint{5.096355in}{1.986132in}}%
\pgfpathlineto{\pgfqpoint{5.109306in}{1.991246in}}%
\pgfpathlineto{\pgfqpoint{5.135207in}{2.009087in}}%
\pgfpathlineto{\pgfqpoint{5.148157in}{2.014953in}}%
\pgfpathlineto{\pgfqpoint{5.161108in}{2.023980in}}%
\pgfpathlineto{\pgfqpoint{5.174058in}{2.034666in}}%
\pgfpathlineto{\pgfqpoint{5.187009in}{2.040902in}}%
\pgfpathlineto{\pgfqpoint{5.225860in}{2.051656in}}%
\pgfpathlineto{\pgfqpoint{5.342414in}{2.079073in}}%
\pgfpathlineto{\pgfqpoint{5.355365in}{2.081466in}}%
\pgfpathlineto{\pgfqpoint{5.381266in}{2.083481in}}%
\pgfpathlineto{\pgfqpoint{5.394216in}{2.086440in}}%
\pgfpathlineto{\pgfqpoint{5.420117in}{2.087756in}}%
\pgfpathlineto{\pgfqpoint{5.471919in}{2.094791in}}%
\pgfpathlineto{\pgfqpoint{5.601424in}{2.106250in}}%
\pgfpathlineto{\pgfqpoint{5.614374in}{2.107600in}}%
\pgfpathlineto{\pgfqpoint{5.679127in}{2.109504in}}%
\pgfpathlineto{\pgfqpoint{5.717978in}{2.110859in}}%
\pgfpathlineto{\pgfqpoint{5.756830in}{2.110990in}}%
\pgfpathlineto{\pgfqpoint{5.756830in}{2.110990in}}%
\pgfusepath{stroke}%
\end{pgfscope}%
\begin{pgfscope}%
\pgfpathrectangle{\pgfqpoint{0.589591in}{0.539182in}}{\pgfqpoint{5.167239in}{1.668128in}}%
\pgfusepath{clip}%
\pgfsetbuttcap%
\pgfsetroundjoin%
\pgfsetlinewidth{2.007500pt}%
\definecolor{currentstroke}{rgb}{0.564706,0.564706,1.000000}%
\pgfsetstrokecolor{currentstroke}%
\pgfsetdash{{2.000000pt}{3.300000pt}}{0.000000pt}%
\pgfpathmoveto{\pgfqpoint{0.589591in}{1.050327in}}%
\pgfpathlineto{\pgfqpoint{2.091846in}{1.049330in}}%
\pgfpathlineto{\pgfqpoint{2.104796in}{1.047552in}}%
\pgfpathlineto{\pgfqpoint{2.143648in}{1.046458in}}%
\pgfpathlineto{\pgfqpoint{2.169549in}{1.042364in}}%
\pgfpathlineto{\pgfqpoint{2.182499in}{1.041276in}}%
\pgfpathlineto{\pgfqpoint{2.208400in}{1.035823in}}%
\pgfpathlineto{\pgfqpoint{2.221350in}{1.034719in}}%
\pgfpathlineto{\pgfqpoint{2.234301in}{1.027287in}}%
\pgfpathlineto{\pgfqpoint{2.247251in}{1.025980in}}%
\pgfpathlineto{\pgfqpoint{2.260202in}{1.022789in}}%
\pgfpathlineto{\pgfqpoint{2.299053in}{1.006474in}}%
\pgfpathlineto{\pgfqpoint{2.312004in}{1.003919in}}%
\pgfpathlineto{\pgfqpoint{2.324954in}{1.000128in}}%
\pgfpathlineto{\pgfqpoint{2.337905in}{0.994993in}}%
\pgfpathlineto{\pgfqpoint{2.350855in}{0.992490in}}%
\pgfpathlineto{\pgfqpoint{2.363806in}{0.985832in}}%
\pgfpathlineto{\pgfqpoint{2.376756in}{0.983975in}}%
\pgfpathlineto{\pgfqpoint{2.389707in}{0.979788in}}%
\pgfpathlineto{\pgfqpoint{2.402657in}{0.978464in}}%
\pgfpathlineto{\pgfqpoint{2.415608in}{0.971027in}}%
\pgfpathlineto{\pgfqpoint{2.428558in}{0.968169in}}%
\pgfpathlineto{\pgfqpoint{2.441508in}{0.963045in}}%
\pgfpathlineto{\pgfqpoint{2.467409in}{0.962311in}}%
\pgfpathlineto{\pgfqpoint{2.519211in}{0.952475in}}%
\pgfpathlineto{\pgfqpoint{2.532162in}{0.946014in}}%
\pgfpathlineto{\pgfqpoint{2.545112in}{0.943694in}}%
\pgfpathlineto{\pgfqpoint{2.558063in}{0.936770in}}%
\pgfpathlineto{\pgfqpoint{2.571013in}{0.932297in}}%
\pgfpathlineto{\pgfqpoint{2.596914in}{0.927354in}}%
\pgfpathlineto{\pgfqpoint{2.609865in}{0.922762in}}%
\pgfpathlineto{\pgfqpoint{2.622815in}{0.921439in}}%
\pgfpathlineto{\pgfqpoint{2.674617in}{0.903901in}}%
\pgfpathlineto{\pgfqpoint{2.726419in}{0.893119in}}%
\pgfpathlineto{\pgfqpoint{2.739369in}{0.893079in}}%
\pgfpathlineto{\pgfqpoint{2.752320in}{0.890910in}}%
\pgfpathlineto{\pgfqpoint{2.791171in}{0.872065in}}%
\pgfpathlineto{\pgfqpoint{2.817072in}{0.869109in}}%
\pgfpathlineto{\pgfqpoint{2.868874in}{0.867715in}}%
\pgfpathlineto{\pgfqpoint{2.933626in}{0.862224in}}%
\pgfpathlineto{\pgfqpoint{2.946577in}{0.860065in}}%
\pgfpathlineto{\pgfqpoint{3.024280in}{0.861771in}}%
\pgfpathlineto{\pgfqpoint{3.114933in}{0.865673in}}%
\pgfpathlineto{\pgfqpoint{3.166735in}{0.870800in}}%
\pgfpathlineto{\pgfqpoint{3.192636in}{0.879886in}}%
\pgfpathlineto{\pgfqpoint{3.205586in}{0.881368in}}%
\pgfpathlineto{\pgfqpoint{3.231487in}{0.886216in}}%
\pgfpathlineto{\pgfqpoint{3.244438in}{0.888378in}}%
\pgfpathlineto{\pgfqpoint{3.257388in}{0.893338in}}%
\pgfpathlineto{\pgfqpoint{3.270339in}{0.896721in}}%
\pgfpathlineto{\pgfqpoint{3.283289in}{0.902952in}}%
\pgfpathlineto{\pgfqpoint{3.348042in}{0.923971in}}%
\pgfpathlineto{\pgfqpoint{3.360992in}{0.932688in}}%
\pgfpathlineto{\pgfqpoint{3.373943in}{0.943268in}}%
\pgfpathlineto{\pgfqpoint{3.399843in}{0.951139in}}%
\pgfpathlineto{\pgfqpoint{3.412794in}{0.956240in}}%
\pgfpathlineto{\pgfqpoint{3.425744in}{0.959907in}}%
\pgfpathlineto{\pgfqpoint{3.477546in}{0.982475in}}%
\pgfpathlineto{\pgfqpoint{3.490497in}{0.992242in}}%
\pgfpathlineto{\pgfqpoint{3.503447in}{0.999084in}}%
\pgfpathlineto{\pgfqpoint{3.516398in}{1.008236in}}%
\pgfpathlineto{\pgfqpoint{3.529348in}{1.015406in}}%
\pgfpathlineto{\pgfqpoint{3.542299in}{1.021046in}}%
\pgfpathlineto{\pgfqpoint{3.555249in}{1.028713in}}%
\pgfpathlineto{\pgfqpoint{3.568200in}{1.039160in}}%
\pgfpathlineto{\pgfqpoint{3.581150in}{1.047976in}}%
\pgfpathlineto{\pgfqpoint{3.607051in}{1.068521in}}%
\pgfpathlineto{\pgfqpoint{3.645902in}{1.092856in}}%
\pgfpathlineto{\pgfqpoint{3.658853in}{1.100234in}}%
\pgfpathlineto{\pgfqpoint{3.671803in}{1.110572in}}%
\pgfpathlineto{\pgfqpoint{3.697704in}{1.128077in}}%
\pgfpathlineto{\pgfqpoint{3.723605in}{1.152153in}}%
\pgfpathlineto{\pgfqpoint{3.736556in}{1.167844in}}%
\pgfpathlineto{\pgfqpoint{3.749506in}{1.180878in}}%
\pgfpathlineto{\pgfqpoint{3.762457in}{1.191834in}}%
\pgfpathlineto{\pgfqpoint{3.775407in}{1.199117in}}%
\pgfpathlineto{\pgfqpoint{3.801308in}{1.220632in}}%
\pgfpathlineto{\pgfqpoint{3.814259in}{1.229222in}}%
\pgfpathlineto{\pgfqpoint{3.827209in}{1.244730in}}%
\pgfpathlineto{\pgfqpoint{3.840160in}{1.254799in}}%
\pgfpathlineto{\pgfqpoint{3.866060in}{1.273138in}}%
\pgfpathlineto{\pgfqpoint{3.879011in}{1.284320in}}%
\pgfpathlineto{\pgfqpoint{3.891961in}{1.292669in}}%
\pgfpathlineto{\pgfqpoint{3.904912in}{1.304848in}}%
\pgfpathlineto{\pgfqpoint{3.930813in}{1.321735in}}%
\pgfpathlineto{\pgfqpoint{3.943763in}{1.337446in}}%
\pgfpathlineto{\pgfqpoint{3.956714in}{1.348562in}}%
\pgfpathlineto{\pgfqpoint{3.969664in}{1.364397in}}%
\pgfpathlineto{\pgfqpoint{4.008516in}{1.401330in}}%
\pgfpathlineto{\pgfqpoint{4.086219in}{1.457830in}}%
\pgfpathlineto{\pgfqpoint{4.099169in}{1.469117in}}%
\pgfpathlineto{\pgfqpoint{4.112119in}{1.477680in}}%
\pgfpathlineto{\pgfqpoint{4.125070in}{1.487722in}}%
\pgfpathlineto{\pgfqpoint{4.163921in}{1.509033in}}%
\pgfpathlineto{\pgfqpoint{4.176872in}{1.524894in}}%
\pgfpathlineto{\pgfqpoint{4.189822in}{1.533794in}}%
\pgfpathlineto{\pgfqpoint{4.202773in}{1.541247in}}%
\pgfpathlineto{\pgfqpoint{4.241624in}{1.572359in}}%
\pgfpathlineto{\pgfqpoint{4.254575in}{1.586438in}}%
\pgfpathlineto{\pgfqpoint{4.267525in}{1.595368in}}%
\pgfpathlineto{\pgfqpoint{4.280476in}{1.607871in}}%
\pgfpathlineto{\pgfqpoint{4.293426in}{1.611895in}}%
\pgfpathlineto{\pgfqpoint{4.319327in}{1.630649in}}%
\pgfpathlineto{\pgfqpoint{4.332277in}{1.635450in}}%
\pgfpathlineto{\pgfqpoint{4.345228in}{1.643193in}}%
\pgfpathlineto{\pgfqpoint{4.358178in}{1.655981in}}%
\pgfpathlineto{\pgfqpoint{4.371129in}{1.661934in}}%
\pgfpathlineto{\pgfqpoint{4.397030in}{1.678980in}}%
\pgfpathlineto{\pgfqpoint{4.422931in}{1.695259in}}%
\pgfpathlineto{\pgfqpoint{4.435881in}{1.705707in}}%
\pgfpathlineto{\pgfqpoint{4.448832in}{1.713972in}}%
\pgfpathlineto{\pgfqpoint{4.461782in}{1.720789in}}%
\pgfpathlineto{\pgfqpoint{4.474733in}{1.724190in}}%
\pgfpathlineto{\pgfqpoint{4.487683in}{1.730024in}}%
\pgfpathlineto{\pgfqpoint{4.513584in}{1.748594in}}%
\pgfpathlineto{\pgfqpoint{4.526535in}{1.751184in}}%
\pgfpathlineto{\pgfqpoint{4.539485in}{1.755267in}}%
\pgfpathlineto{\pgfqpoint{4.552436in}{1.761042in}}%
\pgfpathlineto{\pgfqpoint{4.565386in}{1.768463in}}%
\pgfpathlineto{\pgfqpoint{4.578336in}{1.772269in}}%
\pgfpathlineto{\pgfqpoint{4.591287in}{1.779972in}}%
\pgfpathlineto{\pgfqpoint{4.604237in}{1.785996in}}%
\pgfpathlineto{\pgfqpoint{4.617188in}{1.794694in}}%
\pgfpathlineto{\pgfqpoint{4.630138in}{1.801443in}}%
\pgfpathlineto{\pgfqpoint{4.668990in}{1.816004in}}%
\pgfpathlineto{\pgfqpoint{4.681940in}{1.825537in}}%
\pgfpathlineto{\pgfqpoint{4.694891in}{1.832397in}}%
\pgfpathlineto{\pgfqpoint{4.720792in}{1.839869in}}%
\pgfpathlineto{\pgfqpoint{4.733742in}{1.843007in}}%
\pgfpathlineto{\pgfqpoint{4.746693in}{1.853988in}}%
\pgfpathlineto{\pgfqpoint{4.759643in}{1.861656in}}%
\pgfpathlineto{\pgfqpoint{4.772594in}{1.864901in}}%
\pgfpathlineto{\pgfqpoint{4.785544in}{1.869435in}}%
\pgfpathlineto{\pgfqpoint{4.811445in}{1.881707in}}%
\pgfpathlineto{\pgfqpoint{4.824395in}{1.891514in}}%
\pgfpathlineto{\pgfqpoint{4.837346in}{1.895140in}}%
\pgfpathlineto{\pgfqpoint{4.850296in}{1.903532in}}%
\pgfpathlineto{\pgfqpoint{4.876197in}{1.913764in}}%
\pgfpathlineto{\pgfqpoint{4.915049in}{1.922195in}}%
\pgfpathlineto{\pgfqpoint{4.927999in}{1.924718in}}%
\pgfpathlineto{\pgfqpoint{4.953900in}{1.936555in}}%
\pgfpathlineto{\pgfqpoint{4.966851in}{1.941308in}}%
\pgfpathlineto{\pgfqpoint{4.979801in}{1.944547in}}%
\pgfpathlineto{\pgfqpoint{4.992752in}{1.949601in}}%
\pgfpathlineto{\pgfqpoint{5.005702in}{1.958888in}}%
\pgfpathlineto{\pgfqpoint{5.031603in}{1.970024in}}%
\pgfpathlineto{\pgfqpoint{5.070454in}{1.978874in}}%
\pgfpathlineto{\pgfqpoint{5.096355in}{1.986129in}}%
\pgfpathlineto{\pgfqpoint{5.109306in}{1.991242in}}%
\pgfpathlineto{\pgfqpoint{5.135207in}{2.009084in}}%
\pgfpathlineto{\pgfqpoint{5.148157in}{2.014951in}}%
\pgfpathlineto{\pgfqpoint{5.161108in}{2.023977in}}%
\pgfpathlineto{\pgfqpoint{5.174058in}{2.034665in}}%
\pgfpathlineto{\pgfqpoint{5.187009in}{2.040901in}}%
\pgfpathlineto{\pgfqpoint{5.225860in}{2.051655in}}%
\pgfpathlineto{\pgfqpoint{5.342414in}{2.079072in}}%
\pgfpathlineto{\pgfqpoint{5.355365in}{2.081466in}}%
\pgfpathlineto{\pgfqpoint{5.381266in}{2.083481in}}%
\pgfpathlineto{\pgfqpoint{5.394216in}{2.086440in}}%
\pgfpathlineto{\pgfqpoint{5.420117in}{2.087756in}}%
\pgfpathlineto{\pgfqpoint{5.471919in}{2.094791in}}%
\pgfpathlineto{\pgfqpoint{5.601424in}{2.106250in}}%
\pgfpathlineto{\pgfqpoint{5.614374in}{2.107600in}}%
\pgfpathlineto{\pgfqpoint{5.679127in}{2.109504in}}%
\pgfpathlineto{\pgfqpoint{5.717978in}{2.110859in}}%
\pgfpathlineto{\pgfqpoint{5.756830in}{2.110990in}}%
\pgfpathlineto{\pgfqpoint{5.756830in}{2.110990in}}%
\pgfusepath{stroke}%
\end{pgfscope}%
\begin{pgfscope}%
\pgfpathrectangle{\pgfqpoint{0.589591in}{0.539182in}}{\pgfqpoint{5.167239in}{1.668128in}}%
\pgfusepath{clip}%
\pgfsetbuttcap%
\pgfsetroundjoin%
\pgfsetlinewidth{2.007500pt}%
\definecolor{currentstroke}{rgb}{0.564706,0.564706,1.000000}%
\pgfsetstrokecolor{currentstroke}%
\pgfsetdash{{7.400000pt}{3.200000pt}}{0.000000pt}%
\pgfpathmoveto{\pgfqpoint{0.589591in}{0.924834in}}%
\pgfpathlineto{\pgfqpoint{2.091846in}{0.923725in}}%
\pgfpathlineto{\pgfqpoint{2.104796in}{0.921507in}}%
\pgfpathlineto{\pgfqpoint{2.143648in}{0.920397in}}%
\pgfpathlineto{\pgfqpoint{2.182499in}{0.917066in}}%
\pgfpathlineto{\pgfqpoint{2.208400in}{0.911511in}}%
\pgfpathlineto{\pgfqpoint{2.221350in}{0.911504in}}%
\pgfpathlineto{\pgfqpoint{2.234301in}{0.908161in}}%
\pgfpathlineto{\pgfqpoint{2.260202in}{0.904827in}}%
\pgfpathlineto{\pgfqpoint{2.286103in}{0.892621in}}%
\pgfpathlineto{\pgfqpoint{2.312004in}{0.885957in}}%
\pgfpathlineto{\pgfqpoint{2.324954in}{0.885947in}}%
\pgfpathlineto{\pgfqpoint{2.350855in}{0.882612in}}%
\pgfpathlineto{\pgfqpoint{2.402657in}{0.880808in}}%
\pgfpathlineto{\pgfqpoint{2.428558in}{0.877461in}}%
\pgfpathlineto{\pgfqpoint{2.454459in}{0.875236in}}%
\pgfpathlineto{\pgfqpoint{2.480360in}{0.874134in}}%
\pgfpathlineto{\pgfqpoint{2.506261in}{0.872989in}}%
\pgfpathlineto{\pgfqpoint{2.519211in}{0.870776in}}%
\pgfpathlineto{\pgfqpoint{2.532162in}{0.865270in}}%
\pgfpathlineto{\pgfqpoint{2.545112in}{0.864173in}}%
\pgfpathlineto{\pgfqpoint{2.558063in}{0.858647in}}%
\pgfpathlineto{\pgfqpoint{2.571013in}{0.857548in}}%
\pgfpathlineto{\pgfqpoint{2.583964in}{0.855324in}}%
\pgfpathlineto{\pgfqpoint{2.622815in}{0.852019in}}%
\pgfpathlineto{\pgfqpoint{2.635766in}{0.847625in}}%
\pgfpathlineto{\pgfqpoint{2.674617in}{0.839971in}}%
\pgfpathlineto{\pgfqpoint{2.687567in}{0.836836in}}%
\pgfpathlineto{\pgfqpoint{2.726419in}{0.834809in}}%
\pgfpathlineto{\pgfqpoint{2.739369in}{0.834888in}}%
\pgfpathlineto{\pgfqpoint{2.765270in}{0.829634in}}%
\pgfpathlineto{\pgfqpoint{2.778221in}{0.825354in}}%
\pgfpathlineto{\pgfqpoint{2.817072in}{0.822579in}}%
\pgfpathlineto{\pgfqpoint{2.855924in}{0.822157in}}%
\pgfpathlineto{\pgfqpoint{2.894775in}{0.821143in}}%
\pgfpathlineto{\pgfqpoint{2.907725in}{0.822091in}}%
\pgfpathlineto{\pgfqpoint{2.933626in}{0.820092in}}%
\pgfpathlineto{\pgfqpoint{2.946577in}{0.817028in}}%
\pgfpathlineto{\pgfqpoint{2.972478in}{0.819496in}}%
\pgfpathlineto{\pgfqpoint{3.011329in}{0.822691in}}%
\pgfpathlineto{\pgfqpoint{3.101983in}{0.833567in}}%
\pgfpathlineto{\pgfqpoint{3.205586in}{0.856848in}}%
\pgfpathlineto{\pgfqpoint{3.218537in}{0.861327in}}%
\pgfpathlineto{\pgfqpoint{3.231487in}{0.867991in}}%
\pgfpathlineto{\pgfqpoint{3.283289in}{0.884557in}}%
\pgfpathlineto{\pgfqpoint{3.296240in}{0.890556in}}%
\pgfpathlineto{\pgfqpoint{3.309190in}{0.898101in}}%
\pgfpathlineto{\pgfqpoint{3.322141in}{0.900643in}}%
\pgfpathlineto{\pgfqpoint{3.360992in}{0.918624in}}%
\pgfpathlineto{\pgfqpoint{3.386893in}{0.937790in}}%
\pgfpathlineto{\pgfqpoint{3.425744in}{0.952923in}}%
\pgfpathlineto{\pgfqpoint{3.438695in}{0.957571in}}%
\pgfpathlineto{\pgfqpoint{3.464596in}{0.971396in}}%
\pgfpathlineto{\pgfqpoint{3.477546in}{0.976135in}}%
\pgfpathlineto{\pgfqpoint{3.490497in}{0.985555in}}%
\pgfpathlineto{\pgfqpoint{3.503447in}{0.993108in}}%
\pgfpathlineto{\pgfqpoint{3.516398in}{1.002968in}}%
\pgfpathlineto{\pgfqpoint{3.529348in}{1.010250in}}%
\pgfpathlineto{\pgfqpoint{3.542299in}{1.015947in}}%
\pgfpathlineto{\pgfqpoint{3.555249in}{1.023797in}}%
\pgfpathlineto{\pgfqpoint{3.568200in}{1.034490in}}%
\pgfpathlineto{\pgfqpoint{3.581150in}{1.043461in}}%
\pgfpathlineto{\pgfqpoint{3.620001in}{1.073963in}}%
\pgfpathlineto{\pgfqpoint{3.658853in}{1.095581in}}%
\pgfpathlineto{\pgfqpoint{3.671803in}{1.106495in}}%
\pgfpathlineto{\pgfqpoint{3.697704in}{1.123498in}}%
\pgfpathlineto{\pgfqpoint{3.723605in}{1.147223in}}%
\pgfpathlineto{\pgfqpoint{3.736556in}{1.163633in}}%
\pgfpathlineto{\pgfqpoint{3.762457in}{1.188424in}}%
\pgfpathlineto{\pgfqpoint{3.775407in}{1.195724in}}%
\pgfpathlineto{\pgfqpoint{3.788358in}{1.206521in}}%
\pgfpathlineto{\pgfqpoint{3.801308in}{1.219706in}}%
\pgfpathlineto{\pgfqpoint{3.814259in}{1.228304in}}%
\pgfpathlineto{\pgfqpoint{3.840160in}{1.253610in}}%
\pgfpathlineto{\pgfqpoint{3.866060in}{1.272002in}}%
\pgfpathlineto{\pgfqpoint{3.879011in}{1.283228in}}%
\pgfpathlineto{\pgfqpoint{3.891961in}{1.292129in}}%
\pgfpathlineto{\pgfqpoint{3.904912in}{1.304936in}}%
\pgfpathlineto{\pgfqpoint{3.930813in}{1.321815in}}%
\pgfpathlineto{\pgfqpoint{3.943763in}{1.337530in}}%
\pgfpathlineto{\pgfqpoint{3.956714in}{1.348683in}}%
\pgfpathlineto{\pgfqpoint{3.969664in}{1.363433in}}%
\pgfpathlineto{\pgfqpoint{4.008516in}{1.401529in}}%
\pgfpathlineto{\pgfqpoint{4.086219in}{1.457430in}}%
\pgfpathlineto{\pgfqpoint{4.099169in}{1.469275in}}%
\pgfpathlineto{\pgfqpoint{4.112119in}{1.477849in}}%
\pgfpathlineto{\pgfqpoint{4.125070in}{1.487881in}}%
\pgfpathlineto{\pgfqpoint{4.163921in}{1.509170in}}%
\pgfpathlineto{\pgfqpoint{4.176872in}{1.525042in}}%
\pgfpathlineto{\pgfqpoint{4.189822in}{1.533937in}}%
\pgfpathlineto{\pgfqpoint{4.202773in}{1.541386in}}%
\pgfpathlineto{\pgfqpoint{4.241624in}{1.572451in}}%
\pgfpathlineto{\pgfqpoint{4.254575in}{1.586520in}}%
\pgfpathlineto{\pgfqpoint{4.267525in}{1.595438in}}%
\pgfpathlineto{\pgfqpoint{4.280476in}{1.607929in}}%
\pgfpathlineto{\pgfqpoint{4.293426in}{1.611949in}}%
\pgfpathlineto{\pgfqpoint{4.319327in}{1.630682in}}%
\pgfpathlineto{\pgfqpoint{4.332277in}{1.635482in}}%
\pgfpathlineto{\pgfqpoint{4.345228in}{1.643219in}}%
\pgfpathlineto{\pgfqpoint{4.358178in}{1.655990in}}%
\pgfpathlineto{\pgfqpoint{4.371129in}{1.661941in}}%
\pgfpathlineto{\pgfqpoint{4.397030in}{1.678986in}}%
\pgfpathlineto{\pgfqpoint{4.422931in}{1.695261in}}%
\pgfpathlineto{\pgfqpoint{4.435881in}{1.705708in}}%
\pgfpathlineto{\pgfqpoint{4.448832in}{1.713974in}}%
\pgfpathlineto{\pgfqpoint{4.461782in}{1.720791in}}%
\pgfpathlineto{\pgfqpoint{4.474733in}{1.724191in}}%
\pgfpathlineto{\pgfqpoint{4.487683in}{1.730026in}}%
\pgfpathlineto{\pgfqpoint{4.513584in}{1.748596in}}%
\pgfpathlineto{\pgfqpoint{4.526535in}{1.751185in}}%
\pgfpathlineto{\pgfqpoint{4.539485in}{1.755268in}}%
\pgfpathlineto{\pgfqpoint{4.552436in}{1.761044in}}%
\pgfpathlineto{\pgfqpoint{4.565386in}{1.768464in}}%
\pgfpathlineto{\pgfqpoint{4.578336in}{1.772270in}}%
\pgfpathlineto{\pgfqpoint{4.591287in}{1.779973in}}%
\pgfpathlineto{\pgfqpoint{4.604237in}{1.785998in}}%
\pgfpathlineto{\pgfqpoint{4.617188in}{1.794696in}}%
\pgfpathlineto{\pgfqpoint{4.630138in}{1.801445in}}%
\pgfpathlineto{\pgfqpoint{4.668990in}{1.816007in}}%
\pgfpathlineto{\pgfqpoint{4.681940in}{1.825539in}}%
\pgfpathlineto{\pgfqpoint{4.694891in}{1.832399in}}%
\pgfpathlineto{\pgfqpoint{4.720792in}{1.839871in}}%
\pgfpathlineto{\pgfqpoint{4.733742in}{1.843009in}}%
\pgfpathlineto{\pgfqpoint{4.746693in}{1.853990in}}%
\pgfpathlineto{\pgfqpoint{4.759643in}{1.861658in}}%
\pgfpathlineto{\pgfqpoint{4.772594in}{1.864903in}}%
\pgfpathlineto{\pgfqpoint{4.785544in}{1.869437in}}%
\pgfpathlineto{\pgfqpoint{4.811445in}{1.881709in}}%
\pgfpathlineto{\pgfqpoint{4.824395in}{1.891515in}}%
\pgfpathlineto{\pgfqpoint{4.837346in}{1.895142in}}%
\pgfpathlineto{\pgfqpoint{4.850296in}{1.903534in}}%
\pgfpathlineto{\pgfqpoint{4.876197in}{1.913765in}}%
\pgfpathlineto{\pgfqpoint{4.915049in}{1.922197in}}%
\pgfpathlineto{\pgfqpoint{4.927999in}{1.924720in}}%
\pgfpathlineto{\pgfqpoint{4.953900in}{1.936556in}}%
\pgfpathlineto{\pgfqpoint{4.966851in}{1.941310in}}%
\pgfpathlineto{\pgfqpoint{4.979801in}{1.944548in}}%
\pgfpathlineto{\pgfqpoint{4.992752in}{1.949602in}}%
\pgfpathlineto{\pgfqpoint{5.005702in}{1.958890in}}%
\pgfpathlineto{\pgfqpoint{5.031603in}{1.970025in}}%
\pgfpathlineto{\pgfqpoint{5.070454in}{1.978875in}}%
\pgfpathlineto{\pgfqpoint{5.096355in}{1.986130in}}%
\pgfpathlineto{\pgfqpoint{5.109306in}{1.991243in}}%
\pgfpathlineto{\pgfqpoint{5.135207in}{2.009085in}}%
\pgfpathlineto{\pgfqpoint{5.148157in}{2.014951in}}%
\pgfpathlineto{\pgfqpoint{5.161108in}{2.023978in}}%
\pgfpathlineto{\pgfqpoint{5.174058in}{2.034665in}}%
\pgfpathlineto{\pgfqpoint{5.187009in}{2.040901in}}%
\pgfpathlineto{\pgfqpoint{5.225860in}{2.051655in}}%
\pgfpathlineto{\pgfqpoint{5.342414in}{2.079072in}}%
\pgfpathlineto{\pgfqpoint{5.355365in}{2.081466in}}%
\pgfpathlineto{\pgfqpoint{5.381266in}{2.083481in}}%
\pgfpathlineto{\pgfqpoint{5.394216in}{2.086440in}}%
\pgfpathlineto{\pgfqpoint{5.420117in}{2.087756in}}%
\pgfpathlineto{\pgfqpoint{5.471919in}{2.094791in}}%
\pgfpathlineto{\pgfqpoint{5.601424in}{2.106250in}}%
\pgfpathlineto{\pgfqpoint{5.614374in}{2.107600in}}%
\pgfpathlineto{\pgfqpoint{5.679127in}{2.109504in}}%
\pgfpathlineto{\pgfqpoint{5.717978in}{2.110859in}}%
\pgfpathlineto{\pgfqpoint{5.756830in}{2.110990in}}%
\pgfpathlineto{\pgfqpoint{5.756830in}{2.110990in}}%
\pgfusepath{stroke}%
\end{pgfscope}%
\begin{pgfscope}%
\pgfsetrectcap%
\pgfsetmiterjoin%
\pgfsetlinewidth{0.803000pt}%
\definecolor{currentstroke}{rgb}{0.000000,0.000000,0.000000}%
\pgfsetstrokecolor{currentstroke}%
\pgfsetdash{}{0pt}%
\pgfpathmoveto{\pgfqpoint{0.589591in}{0.539182in}}%
\pgfpathlineto{\pgfqpoint{0.589591in}{2.207310in}}%
\pgfusepath{stroke}%
\end{pgfscope}%
\begin{pgfscope}%
\pgfsetrectcap%
\pgfsetmiterjoin%
\pgfsetlinewidth{0.803000pt}%
\definecolor{currentstroke}{rgb}{0.000000,0.000000,0.000000}%
\pgfsetstrokecolor{currentstroke}%
\pgfsetdash{}{0pt}%
\pgfpathmoveto{\pgfqpoint{5.756830in}{0.539182in}}%
\pgfpathlineto{\pgfqpoint{5.756830in}{2.207310in}}%
\pgfusepath{stroke}%
\end{pgfscope}%
\begin{pgfscope}%
\pgfsetrectcap%
\pgfsetmiterjoin%
\pgfsetlinewidth{0.803000pt}%
\definecolor{currentstroke}{rgb}{0.000000,0.000000,0.000000}%
\pgfsetstrokecolor{currentstroke}%
\pgfsetdash{}{0pt}%
\pgfpathmoveto{\pgfqpoint{0.589591in}{0.539182in}}%
\pgfpathlineto{\pgfqpoint{5.756830in}{0.539182in}}%
\pgfusepath{stroke}%
\end{pgfscope}%
\begin{pgfscope}%
\pgfsetrectcap%
\pgfsetmiterjoin%
\pgfsetlinewidth{0.803000pt}%
\definecolor{currentstroke}{rgb}{0.000000,0.000000,0.000000}%
\pgfsetstrokecolor{currentstroke}%
\pgfsetdash{}{0pt}%
\pgfpathmoveto{\pgfqpoint{0.589591in}{2.207310in}}%
\pgfpathlineto{\pgfqpoint{5.756830in}{2.207310in}}%
\pgfusepath{stroke}%
\end{pgfscope}%
\begin{pgfscope}%
\pgfsetrectcap%
\pgfsetroundjoin%
\pgfsetlinewidth{2.007500pt}%
\definecolor{currentstroke}{rgb}{0.878431,0.878431,0.815686}%
\pgfsetstrokecolor{currentstroke}%
\pgfsetdash{}{0pt}%
\pgfpathmoveto{\pgfqpoint{4.839613in}{1.142630in}}%
\pgfpathlineto{\pgfqpoint{5.089613in}{1.142630in}}%
\pgfusepath{stroke}%
\end{pgfscope}%
\begin{pgfscope}%
\definecolor{textcolor}{rgb}{0.000000,0.000000,0.000000}%
\pgfsetstrokecolor{textcolor}%
\pgfsetfillcolor{textcolor}%
\pgftext[x=5.114613in,y=1.098880in,left,base]{\color{textcolor}\rmfamily\fontsize{9.000000}{10.800000}\selectfont T.+CPU1}%
\end{pgfscope}%
\begin{pgfscope}%
\pgfsetrectcap%
\pgfsetroundjoin%
\pgfsetlinewidth{2.007500pt}%
\definecolor{currentstroke}{rgb}{0.564706,0.564706,1.000000}%
\pgfsetstrokecolor{currentstroke}%
\pgfsetdash{}{0pt}%
\pgfpathmoveto{\pgfqpoint{4.839613in}{0.980831in}}%
\pgfpathlineto{\pgfqpoint{5.089613in}{0.980831in}}%
\pgfusepath{stroke}%
\end{pgfscope}%
\begin{pgfscope}%
\definecolor{textcolor}{rgb}{0.000000,0.000000,0.000000}%
\pgfsetstrokecolor{textcolor}%
\pgfsetfillcolor{textcolor}%
\pgftext[x=5.114613in,y=0.937081in,left,base]{\color{textcolor}\rmfamily\fontsize{9.000000}{10.800000}\selectfont P4+CPU1}%
\end{pgfscope}%
\begin{pgfscope}%
\pgfsetbuttcap%
\pgfsetroundjoin%
\pgfsetlinewidth{2.007500pt}%
\definecolor{currentstroke}{rgb}{0.564706,0.564706,1.000000}%
\pgfsetstrokecolor{currentstroke}%
\pgfsetdash{{2.000000pt}{3.300000pt}}{0.000000pt}%
\pgfpathmoveto{\pgfqpoint{4.839613in}{0.819031in}}%
\pgfpathlineto{\pgfqpoint{5.089613in}{0.819031in}}%
\pgfusepath{stroke}%
\end{pgfscope}%
\begin{pgfscope}%
\definecolor{textcolor}{rgb}{0.000000,0.000000,0.000000}%
\pgfsetstrokecolor{textcolor}%
\pgfsetfillcolor{textcolor}%
\pgftext[x=5.114613in,y=0.775281in,left,base]{\color{textcolor}\rmfamily\fontsize{9.000000}{10.800000}\selectfont P4+CPU8}%
\end{pgfscope}%
\begin{pgfscope}%
\pgfsetbuttcap%
\pgfsetroundjoin%
\pgfsetlinewidth{2.007500pt}%
\definecolor{currentstroke}{rgb}{0.564706,0.564706,1.000000}%
\pgfsetstrokecolor{currentstroke}%
\pgfsetdash{{7.400000pt}{3.200000pt}}{0.000000pt}%
\pgfpathmoveto{\pgfqpoint{4.839613in}{0.657232in}}%
\pgfpathlineto{\pgfqpoint{5.089613in}{0.657232in}}%
\pgfusepath{stroke}%
\end{pgfscope}%
\begin{pgfscope}%
\definecolor{textcolor}{rgb}{0.000000,0.000000,0.000000}%
\pgfsetstrokecolor{textcolor}%
\pgfsetfillcolor{textcolor}%
\pgftext[x=5.114613in,y=0.613482in,left,base]{\color{textcolor}\rmfamily\fontsize{9.000000}{10.800000}\selectfont P4+GPU}%
\end{pgfscope}%
\end{pgfpicture}%
\makeatother%
\endgroup%

\vspace*{-0.9cm}
\caption{\label{fig:performance-factor} A graph of the simulated PAR-2 score for various combinations of planners and hardware as the performance factor varies.}
\end{center}
\end{figure}

\begin{table}[t]
  \caption{\label{tab:performance_factor} The performance factor for each combination of planner and hardware that minimizes the simulated PAR-2 score.}
  \vspace*{0.1cm}
  \centering
    \begin{tabular}{l|c|c|c|c|c|c|}
 & \pkg{Tamaki} & \pkg{FlowCutter} & \pkg{htd} & \pkg{Hicks} & \pkg{P3} & \pkg{P4}\\ \hline 
\pkg{CPU1} & $3.8\cdot 10^{-11}$ & $4.8\cdot 10^{-12}$ & $1.6\cdot 10^{-12}$ & $1.0\cdot 10^{-21}$ & $1.4\cdot 10^{-11}$ & $1.6\cdot 10^{-11}$\\ \hline 
\pkg{CPU8} & $7.8\cdot 10^{-12}$ & $1.8\cdot 10^{-12}$ & $1.3\cdot 10^{-12}$ & $1.0\cdot 10^{-21}$ & $5.5\cdot 10^{-12}$ & $6.2\cdot 10^{-12}$\\ \hline 
\pkg{GPU} & $2.1\cdot 10^{-12}$ & $5.5\cdot 10^{-13}$ & $1.3\cdot 10^{-12}$ & $1.0\cdot 10^{-21}$ & $3.0 \cdot 10^{-12}$ & $3.8\cdot 10^{-12}$\\ \hline 
    \end{tabular}
\end{table}

Figure \ref{fig:performance-factor} indicates how varying the performance factor affects the simulated PAR-2 score for various combinations of planners and hardware. For each planner and hardware, Table 2 shows the performance factor $\alpha$ that minimizes the simulated PAR-2 score. We observe that the performance factor for \pkg{CPU8} is lower than for \pkg{CPU1}, but not necessarily higher or lower than for \pkg{GPU}. We conclude that different combinations of planners and hardware are optimized by different performance factors. % Further details are available in the supplemental material.

% Selected results are summarized in Figure \ref{fig:performance-factor}. For each planner and hardware, we select the performance factor $\alpha$ that minimizes the simulated PAR-2 score (i.e., the sum of of the wall-clock times for each completed benchmark, plus 2000 for each uncompleted benchmark). We observe that the performance factor for $\pkg{GPU}$ is typically lower than for $\pkg{CPU8}$, which is lower than for $\pkg{CPU1}$. We conclude that different combinations of planners and hardware are optimized by different performance factors. The full set of selected values is available in the supplemental material.

\begin{figure}[tp]
\begin{center}
%% Creator: Matplotlib, PGF backend
%%
%% To include the figure in your LaTeX document, write
%%   \input{<filename>.pgf}
%%
%% Make sure the required packages are loaded in your preamble
%%   \usepackage{pgf}
%%
%% and, on pdftex
%%   \usepackage[utf8]{inputenc}\DeclareUnicodeCharacter{2212}{-}
%%
%% or, on luatex and xetex
%%   \usepackage{unicode-math}
%%
%% Figures using additional raster images can only be included by \input if
%% they are in the same directory as the main LaTeX file. For loading figures
%% from other directories you can use the `import` package
%%   \usepackage{import}
%%
%% and then include the figures with
%%   \import{<path to file>}{<filename>.pgf}
%%
%% Matplotlib used the following preamble
%%   \usepackage[utf8x]{inputenc}
%%   \usepackage[T1]{fontenc}
%%
\begingroup%
\makeatletter%
\begin{pgfpicture}%
\pgfpathrectangle{\pgfpointorigin}{\pgfqpoint{4.803148in}{4.422834in}}%
\pgfusepath{use as bounding box, clip}%
\begin{pgfscope}%
\pgfsetbuttcap%
\pgfsetmiterjoin%
\definecolor{currentfill}{rgb}{1.000000,1.000000,1.000000}%
\pgfsetfillcolor{currentfill}%
\pgfsetlinewidth{0.000000pt}%
\definecolor{currentstroke}{rgb}{1.000000,1.000000,1.000000}%
\pgfsetstrokecolor{currentstroke}%
\pgfsetdash{}{0pt}%
\pgfpathmoveto{\pgfqpoint{0.000000in}{0.000000in}}%
\pgfpathlineto{\pgfqpoint{4.803148in}{0.000000in}}%
\pgfpathlineto{\pgfqpoint{4.803148in}{4.422834in}}%
\pgfpathlineto{\pgfqpoint{0.000000in}{4.422834in}}%
\pgfpathclose%
\pgfusepath{fill}%
\end{pgfscope}%
\begin{pgfscope}%
\pgfsetbuttcap%
\pgfsetmiterjoin%
\definecolor{currentfill}{rgb}{1.000000,1.000000,1.000000}%
\pgfsetfillcolor{currentfill}%
\pgfsetlinewidth{0.000000pt}%
\definecolor{currentstroke}{rgb}{0.000000,0.000000,0.000000}%
\pgfsetstrokecolor{currentstroke}%
\pgfsetstrokeopacity{0.000000}%
\pgfsetdash{}{0pt}%
\pgfpathmoveto{\pgfqpoint{0.694334in}{2.659974in}}%
\pgfpathlineto{\pgfqpoint{4.524677in}{2.659974in}}%
\pgfpathlineto{\pgfqpoint{4.524677in}{4.228109in}}%
\pgfpathlineto{\pgfqpoint{0.694334in}{4.228109in}}%
\pgfpathclose%
\pgfusepath{fill}%
\end{pgfscope}%
\begin{pgfscope}%
\pgfsetbuttcap%
\pgfsetroundjoin%
\definecolor{currentfill}{rgb}{0.000000,0.000000,0.000000}%
\pgfsetfillcolor{currentfill}%
\pgfsetlinewidth{0.803000pt}%
\definecolor{currentstroke}{rgb}{0.000000,0.000000,0.000000}%
\pgfsetstrokecolor{currentstroke}%
\pgfsetdash{}{0pt}%
\pgfsys@defobject{currentmarker}{\pgfqpoint{0.000000in}{-0.048611in}}{\pgfqpoint{0.000000in}{0.000000in}}{%
\pgfpathmoveto{\pgfqpoint{0.000000in}{0.000000in}}%
\pgfpathlineto{\pgfqpoint{0.000000in}{-0.048611in}}%
\pgfusepath{stroke,fill}%
}%
\begin{pgfscope}%
\pgfsys@transformshift{0.694334in}{2.659974in}%
\pgfsys@useobject{currentmarker}{}%
\end{pgfscope}%
\end{pgfscope}%
\begin{pgfscope}%
\definecolor{textcolor}{rgb}{0.000000,0.000000,0.000000}%
\pgfsetstrokecolor{textcolor}%
\pgfsetfillcolor{textcolor}%
\pgftext[x=0.694334in,y=2.562752in,,top]{\color{textcolor}\rmfamily\fontsize{9.000000}{10.800000}\selectfont \(\displaystyle 0\)}%
\end{pgfscope}%
\begin{pgfscope}%
\pgfsetbuttcap%
\pgfsetroundjoin%
\definecolor{currentfill}{rgb}{0.000000,0.000000,0.000000}%
\pgfsetfillcolor{currentfill}%
\pgfsetlinewidth{0.803000pt}%
\definecolor{currentstroke}{rgb}{0.000000,0.000000,0.000000}%
\pgfsetstrokecolor{currentstroke}%
\pgfsetdash{}{0pt}%
\pgfsys@defobject{currentmarker}{\pgfqpoint{0.000000in}{-0.048611in}}{\pgfqpoint{0.000000in}{0.000000in}}{%
\pgfpathmoveto{\pgfqpoint{0.000000in}{0.000000in}}%
\pgfpathlineto{\pgfqpoint{0.000000in}{-0.048611in}}%
\pgfusepath{stroke,fill}%
}%
\begin{pgfscope}%
\pgfsys@transformshift{1.173127in}{2.659974in}%
\pgfsys@useobject{currentmarker}{}%
\end{pgfscope}%
\end{pgfscope}%
\begin{pgfscope}%
\definecolor{textcolor}{rgb}{0.000000,0.000000,0.000000}%
\pgfsetstrokecolor{textcolor}%
\pgfsetfillcolor{textcolor}%
\pgftext[x=1.173127in,y=2.562752in,,top]{\color{textcolor}\rmfamily\fontsize{9.000000}{10.800000}\selectfont \(\displaystyle 250\)}%
\end{pgfscope}%
\begin{pgfscope}%
\pgfsetbuttcap%
\pgfsetroundjoin%
\definecolor{currentfill}{rgb}{0.000000,0.000000,0.000000}%
\pgfsetfillcolor{currentfill}%
\pgfsetlinewidth{0.803000pt}%
\definecolor{currentstroke}{rgb}{0.000000,0.000000,0.000000}%
\pgfsetstrokecolor{currentstroke}%
\pgfsetdash{}{0pt}%
\pgfsys@defobject{currentmarker}{\pgfqpoint{0.000000in}{-0.048611in}}{\pgfqpoint{0.000000in}{0.000000in}}{%
\pgfpathmoveto{\pgfqpoint{0.000000in}{0.000000in}}%
\pgfpathlineto{\pgfqpoint{0.000000in}{-0.048611in}}%
\pgfusepath{stroke,fill}%
}%
\begin{pgfscope}%
\pgfsys@transformshift{1.651920in}{2.659974in}%
\pgfsys@useobject{currentmarker}{}%
\end{pgfscope}%
\end{pgfscope}%
\begin{pgfscope}%
\definecolor{textcolor}{rgb}{0.000000,0.000000,0.000000}%
\pgfsetstrokecolor{textcolor}%
\pgfsetfillcolor{textcolor}%
\pgftext[x=1.651920in,y=2.562752in,,top]{\color{textcolor}\rmfamily\fontsize{9.000000}{10.800000}\selectfont \(\displaystyle 500\)}%
\end{pgfscope}%
\begin{pgfscope}%
\pgfsetbuttcap%
\pgfsetroundjoin%
\definecolor{currentfill}{rgb}{0.000000,0.000000,0.000000}%
\pgfsetfillcolor{currentfill}%
\pgfsetlinewidth{0.803000pt}%
\definecolor{currentstroke}{rgb}{0.000000,0.000000,0.000000}%
\pgfsetstrokecolor{currentstroke}%
\pgfsetdash{}{0pt}%
\pgfsys@defobject{currentmarker}{\pgfqpoint{0.000000in}{-0.048611in}}{\pgfqpoint{0.000000in}{0.000000in}}{%
\pgfpathmoveto{\pgfqpoint{0.000000in}{0.000000in}}%
\pgfpathlineto{\pgfqpoint{0.000000in}{-0.048611in}}%
\pgfusepath{stroke,fill}%
}%
\begin{pgfscope}%
\pgfsys@transformshift{2.130713in}{2.659974in}%
\pgfsys@useobject{currentmarker}{}%
\end{pgfscope}%
\end{pgfscope}%
\begin{pgfscope}%
\definecolor{textcolor}{rgb}{0.000000,0.000000,0.000000}%
\pgfsetstrokecolor{textcolor}%
\pgfsetfillcolor{textcolor}%
\pgftext[x=2.130713in,y=2.562752in,,top]{\color{textcolor}\rmfamily\fontsize{9.000000}{10.800000}\selectfont \(\displaystyle 750\)}%
\end{pgfscope}%
\begin{pgfscope}%
\pgfsetbuttcap%
\pgfsetroundjoin%
\definecolor{currentfill}{rgb}{0.000000,0.000000,0.000000}%
\pgfsetfillcolor{currentfill}%
\pgfsetlinewidth{0.803000pt}%
\definecolor{currentstroke}{rgb}{0.000000,0.000000,0.000000}%
\pgfsetstrokecolor{currentstroke}%
\pgfsetdash{}{0pt}%
\pgfsys@defobject{currentmarker}{\pgfqpoint{0.000000in}{-0.048611in}}{\pgfqpoint{0.000000in}{0.000000in}}{%
\pgfpathmoveto{\pgfqpoint{0.000000in}{0.000000in}}%
\pgfpathlineto{\pgfqpoint{0.000000in}{-0.048611in}}%
\pgfusepath{stroke,fill}%
}%
\begin{pgfscope}%
\pgfsys@transformshift{2.609506in}{2.659974in}%
\pgfsys@useobject{currentmarker}{}%
\end{pgfscope}%
\end{pgfscope}%
\begin{pgfscope}%
\definecolor{textcolor}{rgb}{0.000000,0.000000,0.000000}%
\pgfsetstrokecolor{textcolor}%
\pgfsetfillcolor{textcolor}%
\pgftext[x=2.609506in,y=2.562752in,,top]{\color{textcolor}\rmfamily\fontsize{9.000000}{10.800000}\selectfont \(\displaystyle 1000\)}%
\end{pgfscope}%
\begin{pgfscope}%
\pgfsetbuttcap%
\pgfsetroundjoin%
\definecolor{currentfill}{rgb}{0.000000,0.000000,0.000000}%
\pgfsetfillcolor{currentfill}%
\pgfsetlinewidth{0.803000pt}%
\definecolor{currentstroke}{rgb}{0.000000,0.000000,0.000000}%
\pgfsetstrokecolor{currentstroke}%
\pgfsetdash{}{0pt}%
\pgfsys@defobject{currentmarker}{\pgfqpoint{0.000000in}{-0.048611in}}{\pgfqpoint{0.000000in}{0.000000in}}{%
\pgfpathmoveto{\pgfqpoint{0.000000in}{0.000000in}}%
\pgfpathlineto{\pgfqpoint{0.000000in}{-0.048611in}}%
\pgfusepath{stroke,fill}%
}%
\begin{pgfscope}%
\pgfsys@transformshift{3.088299in}{2.659974in}%
\pgfsys@useobject{currentmarker}{}%
\end{pgfscope}%
\end{pgfscope}%
\begin{pgfscope}%
\definecolor{textcolor}{rgb}{0.000000,0.000000,0.000000}%
\pgfsetstrokecolor{textcolor}%
\pgfsetfillcolor{textcolor}%
\pgftext[x=3.088299in,y=2.562752in,,top]{\color{textcolor}\rmfamily\fontsize{9.000000}{10.800000}\selectfont \(\displaystyle 1250\)}%
\end{pgfscope}%
\begin{pgfscope}%
\pgfsetbuttcap%
\pgfsetroundjoin%
\definecolor{currentfill}{rgb}{0.000000,0.000000,0.000000}%
\pgfsetfillcolor{currentfill}%
\pgfsetlinewidth{0.803000pt}%
\definecolor{currentstroke}{rgb}{0.000000,0.000000,0.000000}%
\pgfsetstrokecolor{currentstroke}%
\pgfsetdash{}{0pt}%
\pgfsys@defobject{currentmarker}{\pgfqpoint{0.000000in}{-0.048611in}}{\pgfqpoint{0.000000in}{0.000000in}}{%
\pgfpathmoveto{\pgfqpoint{0.000000in}{0.000000in}}%
\pgfpathlineto{\pgfqpoint{0.000000in}{-0.048611in}}%
\pgfusepath{stroke,fill}%
}%
\begin{pgfscope}%
\pgfsys@transformshift{3.567091in}{2.659974in}%
\pgfsys@useobject{currentmarker}{}%
\end{pgfscope}%
\end{pgfscope}%
\begin{pgfscope}%
\definecolor{textcolor}{rgb}{0.000000,0.000000,0.000000}%
\pgfsetstrokecolor{textcolor}%
\pgfsetfillcolor{textcolor}%
\pgftext[x=3.567091in,y=2.562752in,,top]{\color{textcolor}\rmfamily\fontsize{9.000000}{10.800000}\selectfont \(\displaystyle 1500\)}%
\end{pgfscope}%
\begin{pgfscope}%
\pgfsetbuttcap%
\pgfsetroundjoin%
\definecolor{currentfill}{rgb}{0.000000,0.000000,0.000000}%
\pgfsetfillcolor{currentfill}%
\pgfsetlinewidth{0.803000pt}%
\definecolor{currentstroke}{rgb}{0.000000,0.000000,0.000000}%
\pgfsetstrokecolor{currentstroke}%
\pgfsetdash{}{0pt}%
\pgfsys@defobject{currentmarker}{\pgfqpoint{0.000000in}{-0.048611in}}{\pgfqpoint{0.000000in}{0.000000in}}{%
\pgfpathmoveto{\pgfqpoint{0.000000in}{0.000000in}}%
\pgfpathlineto{\pgfqpoint{0.000000in}{-0.048611in}}%
\pgfusepath{stroke,fill}%
}%
\begin{pgfscope}%
\pgfsys@transformshift{4.045884in}{2.659974in}%
\pgfsys@useobject{currentmarker}{}%
\end{pgfscope}%
\end{pgfscope}%
\begin{pgfscope}%
\definecolor{textcolor}{rgb}{0.000000,0.000000,0.000000}%
\pgfsetstrokecolor{textcolor}%
\pgfsetfillcolor{textcolor}%
\pgftext[x=4.045884in,y=2.562752in,,top]{\color{textcolor}\rmfamily\fontsize{9.000000}{10.800000}\selectfont \(\displaystyle 1750\)}%
\end{pgfscope}%
\begin{pgfscope}%
\pgfsetbuttcap%
\pgfsetroundjoin%
\definecolor{currentfill}{rgb}{0.000000,0.000000,0.000000}%
\pgfsetfillcolor{currentfill}%
\pgfsetlinewidth{0.803000pt}%
\definecolor{currentstroke}{rgb}{0.000000,0.000000,0.000000}%
\pgfsetstrokecolor{currentstroke}%
\pgfsetdash{}{0pt}%
\pgfsys@defobject{currentmarker}{\pgfqpoint{0.000000in}{-0.048611in}}{\pgfqpoint{0.000000in}{0.000000in}}{%
\pgfpathmoveto{\pgfqpoint{0.000000in}{0.000000in}}%
\pgfpathlineto{\pgfqpoint{0.000000in}{-0.048611in}}%
\pgfusepath{stroke,fill}%
}%
\begin{pgfscope}%
\pgfsys@transformshift{4.524677in}{2.659974in}%
\pgfsys@useobject{currentmarker}{}%
\end{pgfscope}%
\end{pgfscope}%
\begin{pgfscope}%
\definecolor{textcolor}{rgb}{0.000000,0.000000,0.000000}%
\pgfsetstrokecolor{textcolor}%
\pgfsetfillcolor{textcolor}%
\pgftext[x=4.524677in,y=2.562752in,,top]{\color{textcolor}\rmfamily\fontsize{9.000000}{10.800000}\selectfont \(\displaystyle 2000\)}%
\end{pgfscope}%
\begin{pgfscope}%
\definecolor{textcolor}{rgb}{0.000000,0.000000,0.000000}%
\pgfsetstrokecolor{textcolor}%
\pgfsetfillcolor{textcolor}%
\pgftext[x=2.609506in,y=2.396807in,,top]{\color{textcolor}\rmfamily\fontsize{9.000000}{10.800000}\selectfont Number of benchmarks solved}%
\end{pgfscope}%
\begin{pgfscope}%
\pgfsetbuttcap%
\pgfsetroundjoin%
\definecolor{currentfill}{rgb}{0.000000,0.000000,0.000000}%
\pgfsetfillcolor{currentfill}%
\pgfsetlinewidth{0.803000pt}%
\definecolor{currentstroke}{rgb}{0.000000,0.000000,0.000000}%
\pgfsetstrokecolor{currentstroke}%
\pgfsetdash{}{0pt}%
\pgfsys@defobject{currentmarker}{\pgfqpoint{-0.048611in}{0.000000in}}{\pgfqpoint{0.000000in}{0.000000in}}{%
\pgfpathmoveto{\pgfqpoint{0.000000in}{0.000000in}}%
\pgfpathlineto{\pgfqpoint{-0.048611in}{0.000000in}}%
\pgfusepath{stroke,fill}%
}%
\begin{pgfscope}%
\pgfsys@transformshift{0.694334in}{2.749024in}%
\pgfsys@useobject{currentmarker}{}%
\end{pgfscope}%
\end{pgfscope}%
\begin{pgfscope}%
\definecolor{textcolor}{rgb}{0.000000,0.000000,0.000000}%
\pgfsetstrokecolor{textcolor}%
\pgfsetfillcolor{textcolor}%
\pgftext[x=0.330525in, y=2.704299in, left, base]{\color{textcolor}\rmfamily\fontsize{9.000000}{10.800000}\selectfont \(\displaystyle 10^{-2}\)}%
\end{pgfscope}%
\begin{pgfscope}%
\pgfsetbuttcap%
\pgfsetroundjoin%
\definecolor{currentfill}{rgb}{0.000000,0.000000,0.000000}%
\pgfsetfillcolor{currentfill}%
\pgfsetlinewidth{0.803000pt}%
\definecolor{currentstroke}{rgb}{0.000000,0.000000,0.000000}%
\pgfsetstrokecolor{currentstroke}%
\pgfsetdash{}{0pt}%
\pgfsys@defobject{currentmarker}{\pgfqpoint{-0.048611in}{0.000000in}}{\pgfqpoint{0.000000in}{0.000000in}}{%
\pgfpathmoveto{\pgfqpoint{0.000000in}{0.000000in}}%
\pgfpathlineto{\pgfqpoint{-0.048611in}{0.000000in}}%
\pgfusepath{stroke,fill}%
}%
\begin{pgfscope}%
\pgfsys@transformshift{0.694334in}{3.044841in}%
\pgfsys@useobject{currentmarker}{}%
\end{pgfscope}%
\end{pgfscope}%
\begin{pgfscope}%
\definecolor{textcolor}{rgb}{0.000000,0.000000,0.000000}%
\pgfsetstrokecolor{textcolor}%
\pgfsetfillcolor{textcolor}%
\pgftext[x=0.330525in, y=3.000116in, left, base]{\color{textcolor}\rmfamily\fontsize{9.000000}{10.800000}\selectfont \(\displaystyle 10^{-1}\)}%
\end{pgfscope}%
\begin{pgfscope}%
\pgfsetbuttcap%
\pgfsetroundjoin%
\definecolor{currentfill}{rgb}{0.000000,0.000000,0.000000}%
\pgfsetfillcolor{currentfill}%
\pgfsetlinewidth{0.803000pt}%
\definecolor{currentstroke}{rgb}{0.000000,0.000000,0.000000}%
\pgfsetstrokecolor{currentstroke}%
\pgfsetdash{}{0pt}%
\pgfsys@defobject{currentmarker}{\pgfqpoint{-0.048611in}{0.000000in}}{\pgfqpoint{0.000000in}{0.000000in}}{%
\pgfpathmoveto{\pgfqpoint{0.000000in}{0.000000in}}%
\pgfpathlineto{\pgfqpoint{-0.048611in}{0.000000in}}%
\pgfusepath{stroke,fill}%
}%
\begin{pgfscope}%
\pgfsys@transformshift{0.694334in}{3.340658in}%
\pgfsys@useobject{currentmarker}{}%
\end{pgfscope}%
\end{pgfscope}%
\begin{pgfscope}%
\definecolor{textcolor}{rgb}{0.000000,0.000000,0.000000}%
\pgfsetstrokecolor{textcolor}%
\pgfsetfillcolor{textcolor}%
\pgftext[x=0.410771in, y=3.295933in, left, base]{\color{textcolor}\rmfamily\fontsize{9.000000}{10.800000}\selectfont \(\displaystyle 10^{0}\)}%
\end{pgfscope}%
\begin{pgfscope}%
\pgfsetbuttcap%
\pgfsetroundjoin%
\definecolor{currentfill}{rgb}{0.000000,0.000000,0.000000}%
\pgfsetfillcolor{currentfill}%
\pgfsetlinewidth{0.803000pt}%
\definecolor{currentstroke}{rgb}{0.000000,0.000000,0.000000}%
\pgfsetstrokecolor{currentstroke}%
\pgfsetdash{}{0pt}%
\pgfsys@defobject{currentmarker}{\pgfqpoint{-0.048611in}{0.000000in}}{\pgfqpoint{0.000000in}{0.000000in}}{%
\pgfpathmoveto{\pgfqpoint{0.000000in}{0.000000in}}%
\pgfpathlineto{\pgfqpoint{-0.048611in}{0.000000in}}%
\pgfusepath{stroke,fill}%
}%
\begin{pgfscope}%
\pgfsys@transformshift{0.694334in}{3.636475in}%
\pgfsys@useobject{currentmarker}{}%
\end{pgfscope}%
\end{pgfscope}%
\begin{pgfscope}%
\definecolor{textcolor}{rgb}{0.000000,0.000000,0.000000}%
\pgfsetstrokecolor{textcolor}%
\pgfsetfillcolor{textcolor}%
\pgftext[x=0.410771in, y=3.591750in, left, base]{\color{textcolor}\rmfamily\fontsize{9.000000}{10.800000}\selectfont \(\displaystyle 10^{1}\)}%
\end{pgfscope}%
\begin{pgfscope}%
\pgfsetbuttcap%
\pgfsetroundjoin%
\definecolor{currentfill}{rgb}{0.000000,0.000000,0.000000}%
\pgfsetfillcolor{currentfill}%
\pgfsetlinewidth{0.803000pt}%
\definecolor{currentstroke}{rgb}{0.000000,0.000000,0.000000}%
\pgfsetstrokecolor{currentstroke}%
\pgfsetdash{}{0pt}%
\pgfsys@defobject{currentmarker}{\pgfqpoint{-0.048611in}{0.000000in}}{\pgfqpoint{0.000000in}{0.000000in}}{%
\pgfpathmoveto{\pgfqpoint{0.000000in}{0.000000in}}%
\pgfpathlineto{\pgfqpoint{-0.048611in}{0.000000in}}%
\pgfusepath{stroke,fill}%
}%
\begin{pgfscope}%
\pgfsys@transformshift{0.694334in}{3.932292in}%
\pgfsys@useobject{currentmarker}{}%
\end{pgfscope}%
\end{pgfscope}%
\begin{pgfscope}%
\definecolor{textcolor}{rgb}{0.000000,0.000000,0.000000}%
\pgfsetstrokecolor{textcolor}%
\pgfsetfillcolor{textcolor}%
\pgftext[x=0.410771in, y=3.887567in, left, base]{\color{textcolor}\rmfamily\fontsize{9.000000}{10.800000}\selectfont \(\displaystyle 10^{2}\)}%
\end{pgfscope}%
\begin{pgfscope}%
\pgfsetbuttcap%
\pgfsetroundjoin%
\definecolor{currentfill}{rgb}{0.000000,0.000000,0.000000}%
\pgfsetfillcolor{currentfill}%
\pgfsetlinewidth{0.803000pt}%
\definecolor{currentstroke}{rgb}{0.000000,0.000000,0.000000}%
\pgfsetstrokecolor{currentstroke}%
\pgfsetdash{}{0pt}%
\pgfsys@defobject{currentmarker}{\pgfqpoint{-0.048611in}{0.000000in}}{\pgfqpoint{0.000000in}{0.000000in}}{%
\pgfpathmoveto{\pgfqpoint{0.000000in}{0.000000in}}%
\pgfpathlineto{\pgfqpoint{-0.048611in}{0.000000in}}%
\pgfusepath{stroke,fill}%
}%
\begin{pgfscope}%
\pgfsys@transformshift{0.694334in}{4.228109in}%
\pgfsys@useobject{currentmarker}{}%
\end{pgfscope}%
\end{pgfscope}%
\begin{pgfscope}%
\definecolor{textcolor}{rgb}{0.000000,0.000000,0.000000}%
\pgfsetstrokecolor{textcolor}%
\pgfsetfillcolor{textcolor}%
\pgftext[x=0.410771in, y=4.183384in, left, base]{\color{textcolor}\rmfamily\fontsize{9.000000}{10.800000}\selectfont \(\displaystyle 10^{3}\)}%
\end{pgfscope}%
\begin{pgfscope}%
\pgfsetbuttcap%
\pgfsetroundjoin%
\definecolor{currentfill}{rgb}{0.000000,0.000000,0.000000}%
\pgfsetfillcolor{currentfill}%
\pgfsetlinewidth{0.602250pt}%
\definecolor{currentstroke}{rgb}{0.000000,0.000000,0.000000}%
\pgfsetstrokecolor{currentstroke}%
\pgfsetdash{}{0pt}%
\pgfsys@defobject{currentmarker}{\pgfqpoint{-0.027778in}{0.000000in}}{\pgfqpoint{0.000000in}{0.000000in}}{%
\pgfpathmoveto{\pgfqpoint{0.000000in}{0.000000in}}%
\pgfpathlineto{\pgfqpoint{-0.027778in}{0.000000in}}%
\pgfusepath{stroke,fill}%
}%
\begin{pgfscope}%
\pgfsys@transformshift{0.694334in}{2.659974in}%
\pgfsys@useobject{currentmarker}{}%
\end{pgfscope}%
\end{pgfscope}%
\begin{pgfscope}%
\pgfsetbuttcap%
\pgfsetroundjoin%
\definecolor{currentfill}{rgb}{0.000000,0.000000,0.000000}%
\pgfsetfillcolor{currentfill}%
\pgfsetlinewidth{0.602250pt}%
\definecolor{currentstroke}{rgb}{0.000000,0.000000,0.000000}%
\pgfsetstrokecolor{currentstroke}%
\pgfsetdash{}{0pt}%
\pgfsys@defobject{currentmarker}{\pgfqpoint{-0.027778in}{0.000000in}}{\pgfqpoint{0.000000in}{0.000000in}}{%
\pgfpathmoveto{\pgfqpoint{0.000000in}{0.000000in}}%
\pgfpathlineto{\pgfqpoint{-0.027778in}{0.000000in}}%
\pgfusepath{stroke,fill}%
}%
\begin{pgfscope}%
\pgfsys@transformshift{0.694334in}{2.683397in}%
\pgfsys@useobject{currentmarker}{}%
\end{pgfscope}%
\end{pgfscope}%
\begin{pgfscope}%
\pgfsetbuttcap%
\pgfsetroundjoin%
\definecolor{currentfill}{rgb}{0.000000,0.000000,0.000000}%
\pgfsetfillcolor{currentfill}%
\pgfsetlinewidth{0.602250pt}%
\definecolor{currentstroke}{rgb}{0.000000,0.000000,0.000000}%
\pgfsetstrokecolor{currentstroke}%
\pgfsetdash{}{0pt}%
\pgfsys@defobject{currentmarker}{\pgfqpoint{-0.027778in}{0.000000in}}{\pgfqpoint{0.000000in}{0.000000in}}{%
\pgfpathmoveto{\pgfqpoint{0.000000in}{0.000000in}}%
\pgfpathlineto{\pgfqpoint{-0.027778in}{0.000000in}}%
\pgfusepath{stroke,fill}%
}%
\begin{pgfscope}%
\pgfsys@transformshift{0.694334in}{2.703201in}%
\pgfsys@useobject{currentmarker}{}%
\end{pgfscope}%
\end{pgfscope}%
\begin{pgfscope}%
\pgfsetbuttcap%
\pgfsetroundjoin%
\definecolor{currentfill}{rgb}{0.000000,0.000000,0.000000}%
\pgfsetfillcolor{currentfill}%
\pgfsetlinewidth{0.602250pt}%
\definecolor{currentstroke}{rgb}{0.000000,0.000000,0.000000}%
\pgfsetstrokecolor{currentstroke}%
\pgfsetdash{}{0pt}%
\pgfsys@defobject{currentmarker}{\pgfqpoint{-0.027778in}{0.000000in}}{\pgfqpoint{0.000000in}{0.000000in}}{%
\pgfpathmoveto{\pgfqpoint{0.000000in}{0.000000in}}%
\pgfpathlineto{\pgfqpoint{-0.027778in}{0.000000in}}%
\pgfusepath{stroke,fill}%
}%
\begin{pgfscope}%
\pgfsys@transformshift{0.694334in}{2.720356in}%
\pgfsys@useobject{currentmarker}{}%
\end{pgfscope}%
\end{pgfscope}%
\begin{pgfscope}%
\pgfsetbuttcap%
\pgfsetroundjoin%
\definecolor{currentfill}{rgb}{0.000000,0.000000,0.000000}%
\pgfsetfillcolor{currentfill}%
\pgfsetlinewidth{0.602250pt}%
\definecolor{currentstroke}{rgb}{0.000000,0.000000,0.000000}%
\pgfsetstrokecolor{currentstroke}%
\pgfsetdash{}{0pt}%
\pgfsys@defobject{currentmarker}{\pgfqpoint{-0.027778in}{0.000000in}}{\pgfqpoint{0.000000in}{0.000000in}}{%
\pgfpathmoveto{\pgfqpoint{0.000000in}{0.000000in}}%
\pgfpathlineto{\pgfqpoint{-0.027778in}{0.000000in}}%
\pgfusepath{stroke,fill}%
}%
\begin{pgfscope}%
\pgfsys@transformshift{0.694334in}{2.735488in}%
\pgfsys@useobject{currentmarker}{}%
\end{pgfscope}%
\end{pgfscope}%
\begin{pgfscope}%
\pgfsetbuttcap%
\pgfsetroundjoin%
\definecolor{currentfill}{rgb}{0.000000,0.000000,0.000000}%
\pgfsetfillcolor{currentfill}%
\pgfsetlinewidth{0.602250pt}%
\definecolor{currentstroke}{rgb}{0.000000,0.000000,0.000000}%
\pgfsetstrokecolor{currentstroke}%
\pgfsetdash{}{0pt}%
\pgfsys@defobject{currentmarker}{\pgfqpoint{-0.027778in}{0.000000in}}{\pgfqpoint{0.000000in}{0.000000in}}{%
\pgfpathmoveto{\pgfqpoint{0.000000in}{0.000000in}}%
\pgfpathlineto{\pgfqpoint{-0.027778in}{0.000000in}}%
\pgfusepath{stroke,fill}%
}%
\begin{pgfscope}%
\pgfsys@transformshift{0.694334in}{2.838074in}%
\pgfsys@useobject{currentmarker}{}%
\end{pgfscope}%
\end{pgfscope}%
\begin{pgfscope}%
\pgfsetbuttcap%
\pgfsetroundjoin%
\definecolor{currentfill}{rgb}{0.000000,0.000000,0.000000}%
\pgfsetfillcolor{currentfill}%
\pgfsetlinewidth{0.602250pt}%
\definecolor{currentstroke}{rgb}{0.000000,0.000000,0.000000}%
\pgfsetstrokecolor{currentstroke}%
\pgfsetdash{}{0pt}%
\pgfsys@defobject{currentmarker}{\pgfqpoint{-0.027778in}{0.000000in}}{\pgfqpoint{0.000000in}{0.000000in}}{%
\pgfpathmoveto{\pgfqpoint{0.000000in}{0.000000in}}%
\pgfpathlineto{\pgfqpoint{-0.027778in}{0.000000in}}%
\pgfusepath{stroke,fill}%
}%
\begin{pgfscope}%
\pgfsys@transformshift{0.694334in}{2.890164in}%
\pgfsys@useobject{currentmarker}{}%
\end{pgfscope}%
\end{pgfscope}%
\begin{pgfscope}%
\pgfsetbuttcap%
\pgfsetroundjoin%
\definecolor{currentfill}{rgb}{0.000000,0.000000,0.000000}%
\pgfsetfillcolor{currentfill}%
\pgfsetlinewidth{0.602250pt}%
\definecolor{currentstroke}{rgb}{0.000000,0.000000,0.000000}%
\pgfsetstrokecolor{currentstroke}%
\pgfsetdash{}{0pt}%
\pgfsys@defobject{currentmarker}{\pgfqpoint{-0.027778in}{0.000000in}}{\pgfqpoint{0.000000in}{0.000000in}}{%
\pgfpathmoveto{\pgfqpoint{0.000000in}{0.000000in}}%
\pgfpathlineto{\pgfqpoint{-0.027778in}{0.000000in}}%
\pgfusepath{stroke,fill}%
}%
\begin{pgfscope}%
\pgfsys@transformshift{0.694334in}{2.927123in}%
\pgfsys@useobject{currentmarker}{}%
\end{pgfscope}%
\end{pgfscope}%
\begin{pgfscope}%
\pgfsetbuttcap%
\pgfsetroundjoin%
\definecolor{currentfill}{rgb}{0.000000,0.000000,0.000000}%
\pgfsetfillcolor{currentfill}%
\pgfsetlinewidth{0.602250pt}%
\definecolor{currentstroke}{rgb}{0.000000,0.000000,0.000000}%
\pgfsetstrokecolor{currentstroke}%
\pgfsetdash{}{0pt}%
\pgfsys@defobject{currentmarker}{\pgfqpoint{-0.027778in}{0.000000in}}{\pgfqpoint{0.000000in}{0.000000in}}{%
\pgfpathmoveto{\pgfqpoint{0.000000in}{0.000000in}}%
\pgfpathlineto{\pgfqpoint{-0.027778in}{0.000000in}}%
\pgfusepath{stroke,fill}%
}%
\begin{pgfscope}%
\pgfsys@transformshift{0.694334in}{2.955791in}%
\pgfsys@useobject{currentmarker}{}%
\end{pgfscope}%
\end{pgfscope}%
\begin{pgfscope}%
\pgfsetbuttcap%
\pgfsetroundjoin%
\definecolor{currentfill}{rgb}{0.000000,0.000000,0.000000}%
\pgfsetfillcolor{currentfill}%
\pgfsetlinewidth{0.602250pt}%
\definecolor{currentstroke}{rgb}{0.000000,0.000000,0.000000}%
\pgfsetstrokecolor{currentstroke}%
\pgfsetdash{}{0pt}%
\pgfsys@defobject{currentmarker}{\pgfqpoint{-0.027778in}{0.000000in}}{\pgfqpoint{0.000000in}{0.000000in}}{%
\pgfpathmoveto{\pgfqpoint{0.000000in}{0.000000in}}%
\pgfpathlineto{\pgfqpoint{-0.027778in}{0.000000in}}%
\pgfusepath{stroke,fill}%
}%
\begin{pgfscope}%
\pgfsys@transformshift{0.694334in}{2.979214in}%
\pgfsys@useobject{currentmarker}{}%
\end{pgfscope}%
\end{pgfscope}%
\begin{pgfscope}%
\pgfsetbuttcap%
\pgfsetroundjoin%
\definecolor{currentfill}{rgb}{0.000000,0.000000,0.000000}%
\pgfsetfillcolor{currentfill}%
\pgfsetlinewidth{0.602250pt}%
\definecolor{currentstroke}{rgb}{0.000000,0.000000,0.000000}%
\pgfsetstrokecolor{currentstroke}%
\pgfsetdash{}{0pt}%
\pgfsys@defobject{currentmarker}{\pgfqpoint{-0.027778in}{0.000000in}}{\pgfqpoint{0.000000in}{0.000000in}}{%
\pgfpathmoveto{\pgfqpoint{0.000000in}{0.000000in}}%
\pgfpathlineto{\pgfqpoint{-0.027778in}{0.000000in}}%
\pgfusepath{stroke,fill}%
}%
\begin{pgfscope}%
\pgfsys@transformshift{0.694334in}{2.999018in}%
\pgfsys@useobject{currentmarker}{}%
\end{pgfscope}%
\end{pgfscope}%
\begin{pgfscope}%
\pgfsetbuttcap%
\pgfsetroundjoin%
\definecolor{currentfill}{rgb}{0.000000,0.000000,0.000000}%
\pgfsetfillcolor{currentfill}%
\pgfsetlinewidth{0.602250pt}%
\definecolor{currentstroke}{rgb}{0.000000,0.000000,0.000000}%
\pgfsetstrokecolor{currentstroke}%
\pgfsetdash{}{0pt}%
\pgfsys@defobject{currentmarker}{\pgfqpoint{-0.027778in}{0.000000in}}{\pgfqpoint{0.000000in}{0.000000in}}{%
\pgfpathmoveto{\pgfqpoint{0.000000in}{0.000000in}}%
\pgfpathlineto{\pgfqpoint{-0.027778in}{0.000000in}}%
\pgfusepath{stroke,fill}%
}%
\begin{pgfscope}%
\pgfsys@transformshift{0.694334in}{3.016173in}%
\pgfsys@useobject{currentmarker}{}%
\end{pgfscope}%
\end{pgfscope}%
\begin{pgfscope}%
\pgfsetbuttcap%
\pgfsetroundjoin%
\definecolor{currentfill}{rgb}{0.000000,0.000000,0.000000}%
\pgfsetfillcolor{currentfill}%
\pgfsetlinewidth{0.602250pt}%
\definecolor{currentstroke}{rgb}{0.000000,0.000000,0.000000}%
\pgfsetstrokecolor{currentstroke}%
\pgfsetdash{}{0pt}%
\pgfsys@defobject{currentmarker}{\pgfqpoint{-0.027778in}{0.000000in}}{\pgfqpoint{0.000000in}{0.000000in}}{%
\pgfpathmoveto{\pgfqpoint{0.000000in}{0.000000in}}%
\pgfpathlineto{\pgfqpoint{-0.027778in}{0.000000in}}%
\pgfusepath{stroke,fill}%
}%
\begin{pgfscope}%
\pgfsys@transformshift{0.694334in}{3.031305in}%
\pgfsys@useobject{currentmarker}{}%
\end{pgfscope}%
\end{pgfscope}%
\begin{pgfscope}%
\pgfsetbuttcap%
\pgfsetroundjoin%
\definecolor{currentfill}{rgb}{0.000000,0.000000,0.000000}%
\pgfsetfillcolor{currentfill}%
\pgfsetlinewidth{0.602250pt}%
\definecolor{currentstroke}{rgb}{0.000000,0.000000,0.000000}%
\pgfsetstrokecolor{currentstroke}%
\pgfsetdash{}{0pt}%
\pgfsys@defobject{currentmarker}{\pgfqpoint{-0.027778in}{0.000000in}}{\pgfqpoint{0.000000in}{0.000000in}}{%
\pgfpathmoveto{\pgfqpoint{0.000000in}{0.000000in}}%
\pgfpathlineto{\pgfqpoint{-0.027778in}{0.000000in}}%
\pgfusepath{stroke,fill}%
}%
\begin{pgfscope}%
\pgfsys@transformshift{0.694334in}{3.133891in}%
\pgfsys@useobject{currentmarker}{}%
\end{pgfscope}%
\end{pgfscope}%
\begin{pgfscope}%
\pgfsetbuttcap%
\pgfsetroundjoin%
\definecolor{currentfill}{rgb}{0.000000,0.000000,0.000000}%
\pgfsetfillcolor{currentfill}%
\pgfsetlinewidth{0.602250pt}%
\definecolor{currentstroke}{rgb}{0.000000,0.000000,0.000000}%
\pgfsetstrokecolor{currentstroke}%
\pgfsetdash{}{0pt}%
\pgfsys@defobject{currentmarker}{\pgfqpoint{-0.027778in}{0.000000in}}{\pgfqpoint{0.000000in}{0.000000in}}{%
\pgfpathmoveto{\pgfqpoint{0.000000in}{0.000000in}}%
\pgfpathlineto{\pgfqpoint{-0.027778in}{0.000000in}}%
\pgfusepath{stroke,fill}%
}%
\begin{pgfscope}%
\pgfsys@transformshift{0.694334in}{3.185981in}%
\pgfsys@useobject{currentmarker}{}%
\end{pgfscope}%
\end{pgfscope}%
\begin{pgfscope}%
\pgfsetbuttcap%
\pgfsetroundjoin%
\definecolor{currentfill}{rgb}{0.000000,0.000000,0.000000}%
\pgfsetfillcolor{currentfill}%
\pgfsetlinewidth{0.602250pt}%
\definecolor{currentstroke}{rgb}{0.000000,0.000000,0.000000}%
\pgfsetstrokecolor{currentstroke}%
\pgfsetdash{}{0pt}%
\pgfsys@defobject{currentmarker}{\pgfqpoint{-0.027778in}{0.000000in}}{\pgfqpoint{0.000000in}{0.000000in}}{%
\pgfpathmoveto{\pgfqpoint{0.000000in}{0.000000in}}%
\pgfpathlineto{\pgfqpoint{-0.027778in}{0.000000in}}%
\pgfusepath{stroke,fill}%
}%
\begin{pgfscope}%
\pgfsys@transformshift{0.694334in}{3.222940in}%
\pgfsys@useobject{currentmarker}{}%
\end{pgfscope}%
\end{pgfscope}%
\begin{pgfscope}%
\pgfsetbuttcap%
\pgfsetroundjoin%
\definecolor{currentfill}{rgb}{0.000000,0.000000,0.000000}%
\pgfsetfillcolor{currentfill}%
\pgfsetlinewidth{0.602250pt}%
\definecolor{currentstroke}{rgb}{0.000000,0.000000,0.000000}%
\pgfsetstrokecolor{currentstroke}%
\pgfsetdash{}{0pt}%
\pgfsys@defobject{currentmarker}{\pgfqpoint{-0.027778in}{0.000000in}}{\pgfqpoint{0.000000in}{0.000000in}}{%
\pgfpathmoveto{\pgfqpoint{0.000000in}{0.000000in}}%
\pgfpathlineto{\pgfqpoint{-0.027778in}{0.000000in}}%
\pgfusepath{stroke,fill}%
}%
\begin{pgfscope}%
\pgfsys@transformshift{0.694334in}{3.251608in}%
\pgfsys@useobject{currentmarker}{}%
\end{pgfscope}%
\end{pgfscope}%
\begin{pgfscope}%
\pgfsetbuttcap%
\pgfsetroundjoin%
\definecolor{currentfill}{rgb}{0.000000,0.000000,0.000000}%
\pgfsetfillcolor{currentfill}%
\pgfsetlinewidth{0.602250pt}%
\definecolor{currentstroke}{rgb}{0.000000,0.000000,0.000000}%
\pgfsetstrokecolor{currentstroke}%
\pgfsetdash{}{0pt}%
\pgfsys@defobject{currentmarker}{\pgfqpoint{-0.027778in}{0.000000in}}{\pgfqpoint{0.000000in}{0.000000in}}{%
\pgfpathmoveto{\pgfqpoint{0.000000in}{0.000000in}}%
\pgfpathlineto{\pgfqpoint{-0.027778in}{0.000000in}}%
\pgfusepath{stroke,fill}%
}%
\begin{pgfscope}%
\pgfsys@transformshift{0.694334in}{3.275031in}%
\pgfsys@useobject{currentmarker}{}%
\end{pgfscope}%
\end{pgfscope}%
\begin{pgfscope}%
\pgfsetbuttcap%
\pgfsetroundjoin%
\definecolor{currentfill}{rgb}{0.000000,0.000000,0.000000}%
\pgfsetfillcolor{currentfill}%
\pgfsetlinewidth{0.602250pt}%
\definecolor{currentstroke}{rgb}{0.000000,0.000000,0.000000}%
\pgfsetstrokecolor{currentstroke}%
\pgfsetdash{}{0pt}%
\pgfsys@defobject{currentmarker}{\pgfqpoint{-0.027778in}{0.000000in}}{\pgfqpoint{0.000000in}{0.000000in}}{%
\pgfpathmoveto{\pgfqpoint{0.000000in}{0.000000in}}%
\pgfpathlineto{\pgfqpoint{-0.027778in}{0.000000in}}%
\pgfusepath{stroke,fill}%
}%
\begin{pgfscope}%
\pgfsys@transformshift{0.694334in}{3.294835in}%
\pgfsys@useobject{currentmarker}{}%
\end{pgfscope}%
\end{pgfscope}%
\begin{pgfscope}%
\pgfsetbuttcap%
\pgfsetroundjoin%
\definecolor{currentfill}{rgb}{0.000000,0.000000,0.000000}%
\pgfsetfillcolor{currentfill}%
\pgfsetlinewidth{0.602250pt}%
\definecolor{currentstroke}{rgb}{0.000000,0.000000,0.000000}%
\pgfsetstrokecolor{currentstroke}%
\pgfsetdash{}{0pt}%
\pgfsys@defobject{currentmarker}{\pgfqpoint{-0.027778in}{0.000000in}}{\pgfqpoint{0.000000in}{0.000000in}}{%
\pgfpathmoveto{\pgfqpoint{0.000000in}{0.000000in}}%
\pgfpathlineto{\pgfqpoint{-0.027778in}{0.000000in}}%
\pgfusepath{stroke,fill}%
}%
\begin{pgfscope}%
\pgfsys@transformshift{0.694334in}{3.311990in}%
\pgfsys@useobject{currentmarker}{}%
\end{pgfscope}%
\end{pgfscope}%
\begin{pgfscope}%
\pgfsetbuttcap%
\pgfsetroundjoin%
\definecolor{currentfill}{rgb}{0.000000,0.000000,0.000000}%
\pgfsetfillcolor{currentfill}%
\pgfsetlinewidth{0.602250pt}%
\definecolor{currentstroke}{rgb}{0.000000,0.000000,0.000000}%
\pgfsetstrokecolor{currentstroke}%
\pgfsetdash{}{0pt}%
\pgfsys@defobject{currentmarker}{\pgfqpoint{-0.027778in}{0.000000in}}{\pgfqpoint{0.000000in}{0.000000in}}{%
\pgfpathmoveto{\pgfqpoint{0.000000in}{0.000000in}}%
\pgfpathlineto{\pgfqpoint{-0.027778in}{0.000000in}}%
\pgfusepath{stroke,fill}%
}%
\begin{pgfscope}%
\pgfsys@transformshift{0.694334in}{3.327122in}%
\pgfsys@useobject{currentmarker}{}%
\end{pgfscope}%
\end{pgfscope}%
\begin{pgfscope}%
\pgfsetbuttcap%
\pgfsetroundjoin%
\definecolor{currentfill}{rgb}{0.000000,0.000000,0.000000}%
\pgfsetfillcolor{currentfill}%
\pgfsetlinewidth{0.602250pt}%
\definecolor{currentstroke}{rgb}{0.000000,0.000000,0.000000}%
\pgfsetstrokecolor{currentstroke}%
\pgfsetdash{}{0pt}%
\pgfsys@defobject{currentmarker}{\pgfqpoint{-0.027778in}{0.000000in}}{\pgfqpoint{0.000000in}{0.000000in}}{%
\pgfpathmoveto{\pgfqpoint{0.000000in}{0.000000in}}%
\pgfpathlineto{\pgfqpoint{-0.027778in}{0.000000in}}%
\pgfusepath{stroke,fill}%
}%
\begin{pgfscope}%
\pgfsys@transformshift{0.694334in}{3.429708in}%
\pgfsys@useobject{currentmarker}{}%
\end{pgfscope}%
\end{pgfscope}%
\begin{pgfscope}%
\pgfsetbuttcap%
\pgfsetroundjoin%
\definecolor{currentfill}{rgb}{0.000000,0.000000,0.000000}%
\pgfsetfillcolor{currentfill}%
\pgfsetlinewidth{0.602250pt}%
\definecolor{currentstroke}{rgb}{0.000000,0.000000,0.000000}%
\pgfsetstrokecolor{currentstroke}%
\pgfsetdash{}{0pt}%
\pgfsys@defobject{currentmarker}{\pgfqpoint{-0.027778in}{0.000000in}}{\pgfqpoint{0.000000in}{0.000000in}}{%
\pgfpathmoveto{\pgfqpoint{0.000000in}{0.000000in}}%
\pgfpathlineto{\pgfqpoint{-0.027778in}{0.000000in}}%
\pgfusepath{stroke,fill}%
}%
\begin{pgfscope}%
\pgfsys@transformshift{0.694334in}{3.481798in}%
\pgfsys@useobject{currentmarker}{}%
\end{pgfscope}%
\end{pgfscope}%
\begin{pgfscope}%
\pgfsetbuttcap%
\pgfsetroundjoin%
\definecolor{currentfill}{rgb}{0.000000,0.000000,0.000000}%
\pgfsetfillcolor{currentfill}%
\pgfsetlinewidth{0.602250pt}%
\definecolor{currentstroke}{rgb}{0.000000,0.000000,0.000000}%
\pgfsetstrokecolor{currentstroke}%
\pgfsetdash{}{0pt}%
\pgfsys@defobject{currentmarker}{\pgfqpoint{-0.027778in}{0.000000in}}{\pgfqpoint{0.000000in}{0.000000in}}{%
\pgfpathmoveto{\pgfqpoint{0.000000in}{0.000000in}}%
\pgfpathlineto{\pgfqpoint{-0.027778in}{0.000000in}}%
\pgfusepath{stroke,fill}%
}%
\begin{pgfscope}%
\pgfsys@transformshift{0.694334in}{3.518757in}%
\pgfsys@useobject{currentmarker}{}%
\end{pgfscope}%
\end{pgfscope}%
\begin{pgfscope}%
\pgfsetbuttcap%
\pgfsetroundjoin%
\definecolor{currentfill}{rgb}{0.000000,0.000000,0.000000}%
\pgfsetfillcolor{currentfill}%
\pgfsetlinewidth{0.602250pt}%
\definecolor{currentstroke}{rgb}{0.000000,0.000000,0.000000}%
\pgfsetstrokecolor{currentstroke}%
\pgfsetdash{}{0pt}%
\pgfsys@defobject{currentmarker}{\pgfqpoint{-0.027778in}{0.000000in}}{\pgfqpoint{0.000000in}{0.000000in}}{%
\pgfpathmoveto{\pgfqpoint{0.000000in}{0.000000in}}%
\pgfpathlineto{\pgfqpoint{-0.027778in}{0.000000in}}%
\pgfusepath{stroke,fill}%
}%
\begin{pgfscope}%
\pgfsys@transformshift{0.694334in}{3.547425in}%
\pgfsys@useobject{currentmarker}{}%
\end{pgfscope}%
\end{pgfscope}%
\begin{pgfscope}%
\pgfsetbuttcap%
\pgfsetroundjoin%
\definecolor{currentfill}{rgb}{0.000000,0.000000,0.000000}%
\pgfsetfillcolor{currentfill}%
\pgfsetlinewidth{0.602250pt}%
\definecolor{currentstroke}{rgb}{0.000000,0.000000,0.000000}%
\pgfsetstrokecolor{currentstroke}%
\pgfsetdash{}{0pt}%
\pgfsys@defobject{currentmarker}{\pgfqpoint{-0.027778in}{0.000000in}}{\pgfqpoint{0.000000in}{0.000000in}}{%
\pgfpathmoveto{\pgfqpoint{0.000000in}{0.000000in}}%
\pgfpathlineto{\pgfqpoint{-0.027778in}{0.000000in}}%
\pgfusepath{stroke,fill}%
}%
\begin{pgfscope}%
\pgfsys@transformshift{0.694334in}{3.570848in}%
\pgfsys@useobject{currentmarker}{}%
\end{pgfscope}%
\end{pgfscope}%
\begin{pgfscope}%
\pgfsetbuttcap%
\pgfsetroundjoin%
\definecolor{currentfill}{rgb}{0.000000,0.000000,0.000000}%
\pgfsetfillcolor{currentfill}%
\pgfsetlinewidth{0.602250pt}%
\definecolor{currentstroke}{rgb}{0.000000,0.000000,0.000000}%
\pgfsetstrokecolor{currentstroke}%
\pgfsetdash{}{0pt}%
\pgfsys@defobject{currentmarker}{\pgfqpoint{-0.027778in}{0.000000in}}{\pgfqpoint{0.000000in}{0.000000in}}{%
\pgfpathmoveto{\pgfqpoint{0.000000in}{0.000000in}}%
\pgfpathlineto{\pgfqpoint{-0.027778in}{0.000000in}}%
\pgfusepath{stroke,fill}%
}%
\begin{pgfscope}%
\pgfsys@transformshift{0.694334in}{3.590652in}%
\pgfsys@useobject{currentmarker}{}%
\end{pgfscope}%
\end{pgfscope}%
\begin{pgfscope}%
\pgfsetbuttcap%
\pgfsetroundjoin%
\definecolor{currentfill}{rgb}{0.000000,0.000000,0.000000}%
\pgfsetfillcolor{currentfill}%
\pgfsetlinewidth{0.602250pt}%
\definecolor{currentstroke}{rgb}{0.000000,0.000000,0.000000}%
\pgfsetstrokecolor{currentstroke}%
\pgfsetdash{}{0pt}%
\pgfsys@defobject{currentmarker}{\pgfqpoint{-0.027778in}{0.000000in}}{\pgfqpoint{0.000000in}{0.000000in}}{%
\pgfpathmoveto{\pgfqpoint{0.000000in}{0.000000in}}%
\pgfpathlineto{\pgfqpoint{-0.027778in}{0.000000in}}%
\pgfusepath{stroke,fill}%
}%
\begin{pgfscope}%
\pgfsys@transformshift{0.694334in}{3.607807in}%
\pgfsys@useobject{currentmarker}{}%
\end{pgfscope}%
\end{pgfscope}%
\begin{pgfscope}%
\pgfsetbuttcap%
\pgfsetroundjoin%
\definecolor{currentfill}{rgb}{0.000000,0.000000,0.000000}%
\pgfsetfillcolor{currentfill}%
\pgfsetlinewidth{0.602250pt}%
\definecolor{currentstroke}{rgb}{0.000000,0.000000,0.000000}%
\pgfsetstrokecolor{currentstroke}%
\pgfsetdash{}{0pt}%
\pgfsys@defobject{currentmarker}{\pgfqpoint{-0.027778in}{0.000000in}}{\pgfqpoint{0.000000in}{0.000000in}}{%
\pgfpathmoveto{\pgfqpoint{0.000000in}{0.000000in}}%
\pgfpathlineto{\pgfqpoint{-0.027778in}{0.000000in}}%
\pgfusepath{stroke,fill}%
}%
\begin{pgfscope}%
\pgfsys@transformshift{0.694334in}{3.622939in}%
\pgfsys@useobject{currentmarker}{}%
\end{pgfscope}%
\end{pgfscope}%
\begin{pgfscope}%
\pgfsetbuttcap%
\pgfsetroundjoin%
\definecolor{currentfill}{rgb}{0.000000,0.000000,0.000000}%
\pgfsetfillcolor{currentfill}%
\pgfsetlinewidth{0.602250pt}%
\definecolor{currentstroke}{rgb}{0.000000,0.000000,0.000000}%
\pgfsetstrokecolor{currentstroke}%
\pgfsetdash{}{0pt}%
\pgfsys@defobject{currentmarker}{\pgfqpoint{-0.027778in}{0.000000in}}{\pgfqpoint{0.000000in}{0.000000in}}{%
\pgfpathmoveto{\pgfqpoint{0.000000in}{0.000000in}}%
\pgfpathlineto{\pgfqpoint{-0.027778in}{0.000000in}}%
\pgfusepath{stroke,fill}%
}%
\begin{pgfscope}%
\pgfsys@transformshift{0.694334in}{3.725525in}%
\pgfsys@useobject{currentmarker}{}%
\end{pgfscope}%
\end{pgfscope}%
\begin{pgfscope}%
\pgfsetbuttcap%
\pgfsetroundjoin%
\definecolor{currentfill}{rgb}{0.000000,0.000000,0.000000}%
\pgfsetfillcolor{currentfill}%
\pgfsetlinewidth{0.602250pt}%
\definecolor{currentstroke}{rgb}{0.000000,0.000000,0.000000}%
\pgfsetstrokecolor{currentstroke}%
\pgfsetdash{}{0pt}%
\pgfsys@defobject{currentmarker}{\pgfqpoint{-0.027778in}{0.000000in}}{\pgfqpoint{0.000000in}{0.000000in}}{%
\pgfpathmoveto{\pgfqpoint{0.000000in}{0.000000in}}%
\pgfpathlineto{\pgfqpoint{-0.027778in}{0.000000in}}%
\pgfusepath{stroke,fill}%
}%
\begin{pgfscope}%
\pgfsys@transformshift{0.694334in}{3.777615in}%
\pgfsys@useobject{currentmarker}{}%
\end{pgfscope}%
\end{pgfscope}%
\begin{pgfscope}%
\pgfsetbuttcap%
\pgfsetroundjoin%
\definecolor{currentfill}{rgb}{0.000000,0.000000,0.000000}%
\pgfsetfillcolor{currentfill}%
\pgfsetlinewidth{0.602250pt}%
\definecolor{currentstroke}{rgb}{0.000000,0.000000,0.000000}%
\pgfsetstrokecolor{currentstroke}%
\pgfsetdash{}{0pt}%
\pgfsys@defobject{currentmarker}{\pgfqpoint{-0.027778in}{0.000000in}}{\pgfqpoint{0.000000in}{0.000000in}}{%
\pgfpathmoveto{\pgfqpoint{0.000000in}{0.000000in}}%
\pgfpathlineto{\pgfqpoint{-0.027778in}{0.000000in}}%
\pgfusepath{stroke,fill}%
}%
\begin{pgfscope}%
\pgfsys@transformshift{0.694334in}{3.814574in}%
\pgfsys@useobject{currentmarker}{}%
\end{pgfscope}%
\end{pgfscope}%
\begin{pgfscope}%
\pgfsetbuttcap%
\pgfsetroundjoin%
\definecolor{currentfill}{rgb}{0.000000,0.000000,0.000000}%
\pgfsetfillcolor{currentfill}%
\pgfsetlinewidth{0.602250pt}%
\definecolor{currentstroke}{rgb}{0.000000,0.000000,0.000000}%
\pgfsetstrokecolor{currentstroke}%
\pgfsetdash{}{0pt}%
\pgfsys@defobject{currentmarker}{\pgfqpoint{-0.027778in}{0.000000in}}{\pgfqpoint{0.000000in}{0.000000in}}{%
\pgfpathmoveto{\pgfqpoint{0.000000in}{0.000000in}}%
\pgfpathlineto{\pgfqpoint{-0.027778in}{0.000000in}}%
\pgfusepath{stroke,fill}%
}%
\begin{pgfscope}%
\pgfsys@transformshift{0.694334in}{3.843242in}%
\pgfsys@useobject{currentmarker}{}%
\end{pgfscope}%
\end{pgfscope}%
\begin{pgfscope}%
\pgfsetbuttcap%
\pgfsetroundjoin%
\definecolor{currentfill}{rgb}{0.000000,0.000000,0.000000}%
\pgfsetfillcolor{currentfill}%
\pgfsetlinewidth{0.602250pt}%
\definecolor{currentstroke}{rgb}{0.000000,0.000000,0.000000}%
\pgfsetstrokecolor{currentstroke}%
\pgfsetdash{}{0pt}%
\pgfsys@defobject{currentmarker}{\pgfqpoint{-0.027778in}{0.000000in}}{\pgfqpoint{0.000000in}{0.000000in}}{%
\pgfpathmoveto{\pgfqpoint{0.000000in}{0.000000in}}%
\pgfpathlineto{\pgfqpoint{-0.027778in}{0.000000in}}%
\pgfusepath{stroke,fill}%
}%
\begin{pgfscope}%
\pgfsys@transformshift{0.694334in}{3.866665in}%
\pgfsys@useobject{currentmarker}{}%
\end{pgfscope}%
\end{pgfscope}%
\begin{pgfscope}%
\pgfsetbuttcap%
\pgfsetroundjoin%
\definecolor{currentfill}{rgb}{0.000000,0.000000,0.000000}%
\pgfsetfillcolor{currentfill}%
\pgfsetlinewidth{0.602250pt}%
\definecolor{currentstroke}{rgb}{0.000000,0.000000,0.000000}%
\pgfsetstrokecolor{currentstroke}%
\pgfsetdash{}{0pt}%
\pgfsys@defobject{currentmarker}{\pgfqpoint{-0.027778in}{0.000000in}}{\pgfqpoint{0.000000in}{0.000000in}}{%
\pgfpathmoveto{\pgfqpoint{0.000000in}{0.000000in}}%
\pgfpathlineto{\pgfqpoint{-0.027778in}{0.000000in}}%
\pgfusepath{stroke,fill}%
}%
\begin{pgfscope}%
\pgfsys@transformshift{0.694334in}{3.886469in}%
\pgfsys@useobject{currentmarker}{}%
\end{pgfscope}%
\end{pgfscope}%
\begin{pgfscope}%
\pgfsetbuttcap%
\pgfsetroundjoin%
\definecolor{currentfill}{rgb}{0.000000,0.000000,0.000000}%
\pgfsetfillcolor{currentfill}%
\pgfsetlinewidth{0.602250pt}%
\definecolor{currentstroke}{rgb}{0.000000,0.000000,0.000000}%
\pgfsetstrokecolor{currentstroke}%
\pgfsetdash{}{0pt}%
\pgfsys@defobject{currentmarker}{\pgfqpoint{-0.027778in}{0.000000in}}{\pgfqpoint{0.000000in}{0.000000in}}{%
\pgfpathmoveto{\pgfqpoint{0.000000in}{0.000000in}}%
\pgfpathlineto{\pgfqpoint{-0.027778in}{0.000000in}}%
\pgfusepath{stroke,fill}%
}%
\begin{pgfscope}%
\pgfsys@transformshift{0.694334in}{3.903624in}%
\pgfsys@useobject{currentmarker}{}%
\end{pgfscope}%
\end{pgfscope}%
\begin{pgfscope}%
\pgfsetbuttcap%
\pgfsetroundjoin%
\definecolor{currentfill}{rgb}{0.000000,0.000000,0.000000}%
\pgfsetfillcolor{currentfill}%
\pgfsetlinewidth{0.602250pt}%
\definecolor{currentstroke}{rgb}{0.000000,0.000000,0.000000}%
\pgfsetstrokecolor{currentstroke}%
\pgfsetdash{}{0pt}%
\pgfsys@defobject{currentmarker}{\pgfqpoint{-0.027778in}{0.000000in}}{\pgfqpoint{0.000000in}{0.000000in}}{%
\pgfpathmoveto{\pgfqpoint{0.000000in}{0.000000in}}%
\pgfpathlineto{\pgfqpoint{-0.027778in}{0.000000in}}%
\pgfusepath{stroke,fill}%
}%
\begin{pgfscope}%
\pgfsys@transformshift{0.694334in}{3.918756in}%
\pgfsys@useobject{currentmarker}{}%
\end{pgfscope}%
\end{pgfscope}%
\begin{pgfscope}%
\pgfsetbuttcap%
\pgfsetroundjoin%
\definecolor{currentfill}{rgb}{0.000000,0.000000,0.000000}%
\pgfsetfillcolor{currentfill}%
\pgfsetlinewidth{0.602250pt}%
\definecolor{currentstroke}{rgb}{0.000000,0.000000,0.000000}%
\pgfsetstrokecolor{currentstroke}%
\pgfsetdash{}{0pt}%
\pgfsys@defobject{currentmarker}{\pgfqpoint{-0.027778in}{0.000000in}}{\pgfqpoint{0.000000in}{0.000000in}}{%
\pgfpathmoveto{\pgfqpoint{0.000000in}{0.000000in}}%
\pgfpathlineto{\pgfqpoint{-0.027778in}{0.000000in}}%
\pgfusepath{stroke,fill}%
}%
\begin{pgfscope}%
\pgfsys@transformshift{0.694334in}{4.021342in}%
\pgfsys@useobject{currentmarker}{}%
\end{pgfscope}%
\end{pgfscope}%
\begin{pgfscope}%
\pgfsetbuttcap%
\pgfsetroundjoin%
\definecolor{currentfill}{rgb}{0.000000,0.000000,0.000000}%
\pgfsetfillcolor{currentfill}%
\pgfsetlinewidth{0.602250pt}%
\definecolor{currentstroke}{rgb}{0.000000,0.000000,0.000000}%
\pgfsetstrokecolor{currentstroke}%
\pgfsetdash{}{0pt}%
\pgfsys@defobject{currentmarker}{\pgfqpoint{-0.027778in}{0.000000in}}{\pgfqpoint{0.000000in}{0.000000in}}{%
\pgfpathmoveto{\pgfqpoint{0.000000in}{0.000000in}}%
\pgfpathlineto{\pgfqpoint{-0.027778in}{0.000000in}}%
\pgfusepath{stroke,fill}%
}%
\begin{pgfscope}%
\pgfsys@transformshift{0.694334in}{4.073432in}%
\pgfsys@useobject{currentmarker}{}%
\end{pgfscope}%
\end{pgfscope}%
\begin{pgfscope}%
\pgfsetbuttcap%
\pgfsetroundjoin%
\definecolor{currentfill}{rgb}{0.000000,0.000000,0.000000}%
\pgfsetfillcolor{currentfill}%
\pgfsetlinewidth{0.602250pt}%
\definecolor{currentstroke}{rgb}{0.000000,0.000000,0.000000}%
\pgfsetstrokecolor{currentstroke}%
\pgfsetdash{}{0pt}%
\pgfsys@defobject{currentmarker}{\pgfqpoint{-0.027778in}{0.000000in}}{\pgfqpoint{0.000000in}{0.000000in}}{%
\pgfpathmoveto{\pgfqpoint{0.000000in}{0.000000in}}%
\pgfpathlineto{\pgfqpoint{-0.027778in}{0.000000in}}%
\pgfusepath{stroke,fill}%
}%
\begin{pgfscope}%
\pgfsys@transformshift{0.694334in}{4.110391in}%
\pgfsys@useobject{currentmarker}{}%
\end{pgfscope}%
\end{pgfscope}%
\begin{pgfscope}%
\pgfsetbuttcap%
\pgfsetroundjoin%
\definecolor{currentfill}{rgb}{0.000000,0.000000,0.000000}%
\pgfsetfillcolor{currentfill}%
\pgfsetlinewidth{0.602250pt}%
\definecolor{currentstroke}{rgb}{0.000000,0.000000,0.000000}%
\pgfsetstrokecolor{currentstroke}%
\pgfsetdash{}{0pt}%
\pgfsys@defobject{currentmarker}{\pgfqpoint{-0.027778in}{0.000000in}}{\pgfqpoint{0.000000in}{0.000000in}}{%
\pgfpathmoveto{\pgfqpoint{0.000000in}{0.000000in}}%
\pgfpathlineto{\pgfqpoint{-0.027778in}{0.000000in}}%
\pgfusepath{stroke,fill}%
}%
\begin{pgfscope}%
\pgfsys@transformshift{0.694334in}{4.139059in}%
\pgfsys@useobject{currentmarker}{}%
\end{pgfscope}%
\end{pgfscope}%
\begin{pgfscope}%
\pgfsetbuttcap%
\pgfsetroundjoin%
\definecolor{currentfill}{rgb}{0.000000,0.000000,0.000000}%
\pgfsetfillcolor{currentfill}%
\pgfsetlinewidth{0.602250pt}%
\definecolor{currentstroke}{rgb}{0.000000,0.000000,0.000000}%
\pgfsetstrokecolor{currentstroke}%
\pgfsetdash{}{0pt}%
\pgfsys@defobject{currentmarker}{\pgfqpoint{-0.027778in}{0.000000in}}{\pgfqpoint{0.000000in}{0.000000in}}{%
\pgfpathmoveto{\pgfqpoint{0.000000in}{0.000000in}}%
\pgfpathlineto{\pgfqpoint{-0.027778in}{0.000000in}}%
\pgfusepath{stroke,fill}%
}%
\begin{pgfscope}%
\pgfsys@transformshift{0.694334in}{4.162482in}%
\pgfsys@useobject{currentmarker}{}%
\end{pgfscope}%
\end{pgfscope}%
\begin{pgfscope}%
\pgfsetbuttcap%
\pgfsetroundjoin%
\definecolor{currentfill}{rgb}{0.000000,0.000000,0.000000}%
\pgfsetfillcolor{currentfill}%
\pgfsetlinewidth{0.602250pt}%
\definecolor{currentstroke}{rgb}{0.000000,0.000000,0.000000}%
\pgfsetstrokecolor{currentstroke}%
\pgfsetdash{}{0pt}%
\pgfsys@defobject{currentmarker}{\pgfqpoint{-0.027778in}{0.000000in}}{\pgfqpoint{0.000000in}{0.000000in}}{%
\pgfpathmoveto{\pgfqpoint{0.000000in}{0.000000in}}%
\pgfpathlineto{\pgfqpoint{-0.027778in}{0.000000in}}%
\pgfusepath{stroke,fill}%
}%
\begin{pgfscope}%
\pgfsys@transformshift{0.694334in}{4.182286in}%
\pgfsys@useobject{currentmarker}{}%
\end{pgfscope}%
\end{pgfscope}%
\begin{pgfscope}%
\pgfsetbuttcap%
\pgfsetroundjoin%
\definecolor{currentfill}{rgb}{0.000000,0.000000,0.000000}%
\pgfsetfillcolor{currentfill}%
\pgfsetlinewidth{0.602250pt}%
\definecolor{currentstroke}{rgb}{0.000000,0.000000,0.000000}%
\pgfsetstrokecolor{currentstroke}%
\pgfsetdash{}{0pt}%
\pgfsys@defobject{currentmarker}{\pgfqpoint{-0.027778in}{0.000000in}}{\pgfqpoint{0.000000in}{0.000000in}}{%
\pgfpathmoveto{\pgfqpoint{0.000000in}{0.000000in}}%
\pgfpathlineto{\pgfqpoint{-0.027778in}{0.000000in}}%
\pgfusepath{stroke,fill}%
}%
\begin{pgfscope}%
\pgfsys@transformshift{0.694334in}{4.199441in}%
\pgfsys@useobject{currentmarker}{}%
\end{pgfscope}%
\end{pgfscope}%
\begin{pgfscope}%
\pgfsetbuttcap%
\pgfsetroundjoin%
\definecolor{currentfill}{rgb}{0.000000,0.000000,0.000000}%
\pgfsetfillcolor{currentfill}%
\pgfsetlinewidth{0.602250pt}%
\definecolor{currentstroke}{rgb}{0.000000,0.000000,0.000000}%
\pgfsetstrokecolor{currentstroke}%
\pgfsetdash{}{0pt}%
\pgfsys@defobject{currentmarker}{\pgfqpoint{-0.027778in}{0.000000in}}{\pgfqpoint{0.000000in}{0.000000in}}{%
\pgfpathmoveto{\pgfqpoint{0.000000in}{0.000000in}}%
\pgfpathlineto{\pgfqpoint{-0.027778in}{0.000000in}}%
\pgfusepath{stroke,fill}%
}%
\begin{pgfscope}%
\pgfsys@transformshift{0.694334in}{4.214573in}%
\pgfsys@useobject{currentmarker}{}%
\end{pgfscope}%
\end{pgfscope}%
\begin{pgfscope}%
\definecolor{textcolor}{rgb}{0.000000,0.000000,0.000000}%
\pgfsetstrokecolor{textcolor}%
\pgfsetfillcolor{textcolor}%
\pgftext[x=0.274969in,y=3.444041in,,bottom,rotate=90.000000]{\color{textcolor}\rmfamily\fontsize{9.000000}{10.800000}\selectfont Longest solving time (s)}%
\end{pgfscope}%
\begin{pgfscope}%
\pgfpathrectangle{\pgfqpoint{0.694334in}{2.659974in}}{\pgfqpoint{3.830343in}{1.568135in}}%
\pgfusepath{clip}%
\pgfsetrectcap%
\pgfsetroundjoin%
\pgfsetlinewidth{1.003750pt}%
\definecolor{currentstroke}{rgb}{0.878431,0.878431,0.815686}%
\pgfsetstrokecolor{currentstroke}%
\pgfsetdash{}{0pt}%
\pgfpathmoveto{\pgfqpoint{0.694334in}{3.178662in}}%
\pgfpathlineto{\pgfqpoint{0.696249in}{3.183404in}}%
\pgfpathlineto{\pgfqpoint{0.698165in}{3.184795in}}%
\pgfpathlineto{\pgfqpoint{0.700080in}{3.190114in}}%
\pgfpathlineto{\pgfqpoint{0.701995in}{3.191895in}}%
\pgfpathlineto{\pgfqpoint{0.707741in}{3.200838in}}%
\pgfpathlineto{\pgfqpoint{0.709656in}{3.201037in}}%
\pgfpathlineto{\pgfqpoint{0.713486in}{3.203676in}}%
\pgfpathlineto{\pgfqpoint{0.717316in}{3.204314in}}%
\pgfpathlineto{\pgfqpoint{0.721147in}{3.207450in}}%
\pgfpathlineto{\pgfqpoint{0.728807in}{3.210418in}}%
\pgfpathlineto{\pgfqpoint{0.730723in}{3.212275in}}%
\pgfpathlineto{\pgfqpoint{0.732638in}{3.219486in}}%
\pgfpathlineto{\pgfqpoint{0.736468in}{3.220467in}}%
\pgfpathlineto{\pgfqpoint{0.738383in}{3.224926in}}%
\pgfpathlineto{\pgfqpoint{0.740298in}{3.225465in}}%
\pgfpathlineto{\pgfqpoint{0.744129in}{3.244731in}}%
\pgfpathlineto{\pgfqpoint{0.747959in}{3.247421in}}%
\pgfpathlineto{\pgfqpoint{0.751789in}{3.262845in}}%
\pgfpathlineto{\pgfqpoint{0.755620in}{3.264881in}}%
\pgfpathlineto{\pgfqpoint{0.757535in}{3.268604in}}%
\pgfpathlineto{\pgfqpoint{0.759450in}{3.268772in}}%
\pgfpathlineto{\pgfqpoint{0.761365in}{3.272338in}}%
\pgfpathlineto{\pgfqpoint{0.765196in}{3.273280in}}%
\pgfpathlineto{\pgfqpoint{0.767111in}{3.281650in}}%
\pgfpathlineto{\pgfqpoint{0.769026in}{3.284499in}}%
\pgfpathlineto{\pgfqpoint{0.770941in}{3.284988in}}%
\pgfpathlineto{\pgfqpoint{0.772856in}{3.289173in}}%
\pgfpathlineto{\pgfqpoint{0.784347in}{3.297177in}}%
\pgfpathlineto{\pgfqpoint{0.786263in}{3.300315in}}%
\pgfpathlineto{\pgfqpoint{0.788178in}{3.305664in}}%
\pgfpathlineto{\pgfqpoint{0.793923in}{3.309817in}}%
\pgfpathlineto{\pgfqpoint{0.795838in}{3.319861in}}%
\pgfpathlineto{\pgfqpoint{0.797754in}{3.320091in}}%
\pgfpathlineto{\pgfqpoint{0.807329in}{3.327468in}}%
\pgfpathlineto{\pgfqpoint{0.809245in}{3.327505in}}%
\pgfpathlineto{\pgfqpoint{0.813075in}{3.332987in}}%
\pgfpathlineto{\pgfqpoint{0.816905in}{3.333651in}}%
\pgfpathlineto{\pgfqpoint{0.822651in}{3.338920in}}%
\pgfpathlineto{\pgfqpoint{0.824566in}{3.339687in}}%
\pgfpathlineto{\pgfqpoint{0.834142in}{3.348486in}}%
\pgfpathlineto{\pgfqpoint{0.839887in}{3.349020in}}%
\pgfpathlineto{\pgfqpoint{0.841803in}{3.353009in}}%
\pgfpathlineto{\pgfqpoint{0.843718in}{3.353513in}}%
\pgfpathlineto{\pgfqpoint{0.847548in}{3.357065in}}%
\pgfpathlineto{\pgfqpoint{0.851378in}{3.359225in}}%
\pgfpathlineto{\pgfqpoint{0.855209in}{3.362661in}}%
\pgfpathlineto{\pgfqpoint{0.857124in}{3.364270in}}%
\pgfpathlineto{\pgfqpoint{0.859039in}{3.364375in}}%
\pgfpathlineto{\pgfqpoint{0.864785in}{3.368914in}}%
\pgfpathlineto{\pgfqpoint{0.868615in}{3.369486in}}%
\pgfpathlineto{\pgfqpoint{0.870530in}{3.372155in}}%
\pgfpathlineto{\pgfqpoint{0.880106in}{3.375729in}}%
\pgfpathlineto{\pgfqpoint{0.883936in}{3.377269in}}%
\pgfpathlineto{\pgfqpoint{0.887767in}{3.378827in}}%
\pgfpathlineto{\pgfqpoint{0.891597in}{3.379811in}}%
\pgfpathlineto{\pgfqpoint{0.895427in}{3.381114in}}%
\pgfpathlineto{\pgfqpoint{0.901173in}{3.385496in}}%
\pgfpathlineto{\pgfqpoint{0.910749in}{3.387370in}}%
\pgfpathlineto{\pgfqpoint{0.912664in}{3.391673in}}%
\pgfpathlineto{\pgfqpoint{0.920325in}{3.393637in}}%
\pgfpathlineto{\pgfqpoint{0.922240in}{3.394092in}}%
\pgfpathlineto{\pgfqpoint{0.926070in}{3.397607in}}%
\pgfpathlineto{\pgfqpoint{0.931816in}{3.398410in}}%
\pgfpathlineto{\pgfqpoint{0.933731in}{3.399293in}}%
\pgfpathlineto{\pgfqpoint{0.935646in}{3.401418in}}%
\pgfpathlineto{\pgfqpoint{0.937561in}{3.401427in}}%
\pgfpathlineto{\pgfqpoint{0.941391in}{3.406243in}}%
\pgfpathlineto{\pgfqpoint{0.950967in}{3.408079in}}%
\pgfpathlineto{\pgfqpoint{0.966289in}{3.412829in}}%
\pgfpathlineto{\pgfqpoint{0.970119in}{3.414141in}}%
\pgfpathlineto{\pgfqpoint{0.989271in}{3.422306in}}%
\pgfpathlineto{\pgfqpoint{0.991186in}{3.424876in}}%
\pgfpathlineto{\pgfqpoint{1.000762in}{3.428902in}}%
\pgfpathlineto{\pgfqpoint{1.004592in}{3.433447in}}%
\pgfpathlineto{\pgfqpoint{1.010338in}{3.434531in}}%
\pgfpathlineto{\pgfqpoint{1.014168in}{3.435907in}}%
\pgfpathlineto{\pgfqpoint{1.040980in}{3.442376in}}%
\pgfpathlineto{\pgfqpoint{1.044811in}{3.446562in}}%
\pgfpathlineto{\pgfqpoint{1.046726in}{3.447094in}}%
\pgfpathlineto{\pgfqpoint{1.050556in}{3.450300in}}%
\pgfpathlineto{\pgfqpoint{1.054387in}{3.450776in}}%
\pgfpathlineto{\pgfqpoint{1.060132in}{3.456694in}}%
\pgfpathlineto{\pgfqpoint{1.063962in}{3.457641in}}%
\pgfpathlineto{\pgfqpoint{1.065878in}{3.461467in}}%
\pgfpathlineto{\pgfqpoint{1.075453in}{3.465192in}}%
\pgfpathlineto{\pgfqpoint{1.079284in}{3.467395in}}%
\pgfpathlineto{\pgfqpoint{1.081199in}{3.467537in}}%
\pgfpathlineto{\pgfqpoint{1.083114in}{3.469867in}}%
\pgfpathlineto{\pgfqpoint{1.085029in}{3.470042in}}%
\pgfpathlineto{\pgfqpoint{1.086944in}{3.473237in}}%
\pgfpathlineto{\pgfqpoint{1.098435in}{3.477555in}}%
\pgfpathlineto{\pgfqpoint{1.106096in}{3.481965in}}%
\pgfpathlineto{\pgfqpoint{1.108011in}{3.485876in}}%
\pgfpathlineto{\pgfqpoint{1.109926in}{3.492852in}}%
\pgfpathlineto{\pgfqpoint{1.119502in}{3.502674in}}%
\pgfpathlineto{\pgfqpoint{1.123333in}{3.503860in}}%
\pgfpathlineto{\pgfqpoint{1.127163in}{3.505212in}}%
\pgfpathlineto{\pgfqpoint{1.130993in}{3.506157in}}%
\pgfpathlineto{\pgfqpoint{1.136739in}{3.510612in}}%
\pgfpathlineto{\pgfqpoint{1.140569in}{3.511409in}}%
\pgfpathlineto{\pgfqpoint{1.142484in}{3.512819in}}%
\pgfpathlineto{\pgfqpoint{1.146315in}{3.512988in}}%
\pgfpathlineto{\pgfqpoint{1.150145in}{3.514536in}}%
\pgfpathlineto{\pgfqpoint{1.157806in}{3.515770in}}%
\pgfpathlineto{\pgfqpoint{1.173127in}{3.519210in}}%
\pgfpathlineto{\pgfqpoint{1.176957in}{3.520903in}}%
\pgfpathlineto{\pgfqpoint{1.182703in}{3.521437in}}%
\pgfpathlineto{\pgfqpoint{1.188449in}{3.526079in}}%
\pgfpathlineto{\pgfqpoint{1.196109in}{3.528907in}}%
\pgfpathlineto{\pgfqpoint{1.199940in}{3.530589in}}%
\pgfpathlineto{\pgfqpoint{1.209515in}{3.535694in}}%
\pgfpathlineto{\pgfqpoint{1.215261in}{3.537055in}}%
\pgfpathlineto{\pgfqpoint{1.224837in}{3.539449in}}%
\pgfpathlineto{\pgfqpoint{1.226752in}{3.544738in}}%
\pgfpathlineto{\pgfqpoint{1.230582in}{3.545579in}}%
\pgfpathlineto{\pgfqpoint{1.234413in}{3.546396in}}%
\pgfpathlineto{\pgfqpoint{1.242073in}{3.551508in}}%
\pgfpathlineto{\pgfqpoint{1.245904in}{3.551742in}}%
\pgfpathlineto{\pgfqpoint{1.247819in}{3.553698in}}%
\pgfpathlineto{\pgfqpoint{1.255480in}{3.554454in}}%
\pgfpathlineto{\pgfqpoint{1.257395in}{3.556852in}}%
\pgfpathlineto{\pgfqpoint{1.263140in}{3.557531in}}%
\pgfpathlineto{\pgfqpoint{1.268886in}{3.558679in}}%
\pgfpathlineto{\pgfqpoint{1.278462in}{3.562023in}}%
\pgfpathlineto{\pgfqpoint{1.309104in}{3.569183in}}%
\pgfpathlineto{\pgfqpoint{1.332086in}{3.572240in}}%
\pgfpathlineto{\pgfqpoint{1.343577in}{3.578121in}}%
\pgfpathlineto{\pgfqpoint{1.353153in}{3.579018in}}%
\pgfpathlineto{\pgfqpoint{1.356984in}{3.581375in}}%
\pgfpathlineto{\pgfqpoint{1.376135in}{3.584512in}}%
\pgfpathlineto{\pgfqpoint{1.378050in}{3.586179in}}%
\pgfpathlineto{\pgfqpoint{1.383796in}{3.586567in}}%
\pgfpathlineto{\pgfqpoint{1.387626in}{3.588806in}}%
\pgfpathlineto{\pgfqpoint{1.395287in}{3.589853in}}%
\pgfpathlineto{\pgfqpoint{1.435506in}{3.597240in}}%
\pgfpathlineto{\pgfqpoint{1.441251in}{3.599496in}}%
\pgfpathlineto{\pgfqpoint{1.445081in}{3.600476in}}%
\pgfpathlineto{\pgfqpoint{1.456573in}{3.605358in}}%
\pgfpathlineto{\pgfqpoint{1.462318in}{3.607147in}}%
\pgfpathlineto{\pgfqpoint{1.468064in}{3.609518in}}%
\pgfpathlineto{\pgfqpoint{1.477639in}{3.613055in}}%
\pgfpathlineto{\pgfqpoint{1.481470in}{3.614346in}}%
\pgfpathlineto{\pgfqpoint{1.485300in}{3.619813in}}%
\pgfpathlineto{\pgfqpoint{1.489130in}{3.625060in}}%
\pgfpathlineto{\pgfqpoint{1.496791in}{3.628520in}}%
\pgfpathlineto{\pgfqpoint{1.498706in}{3.628626in}}%
\pgfpathlineto{\pgfqpoint{1.502537in}{3.630891in}}%
\pgfpathlineto{\pgfqpoint{1.506367in}{3.632191in}}%
\pgfpathlineto{\pgfqpoint{1.510197in}{3.633667in}}%
\pgfpathlineto{\pgfqpoint{1.514028in}{3.635727in}}%
\pgfpathlineto{\pgfqpoint{1.519773in}{3.637295in}}%
\pgfpathlineto{\pgfqpoint{1.523604in}{3.638959in}}%
\pgfpathlineto{\pgfqpoint{1.527434in}{3.640854in}}%
\pgfpathlineto{\pgfqpoint{1.533179in}{3.641949in}}%
\pgfpathlineto{\pgfqpoint{1.537010in}{3.643958in}}%
\pgfpathlineto{\pgfqpoint{1.540840in}{3.644500in}}%
\pgfpathlineto{\pgfqpoint{1.546586in}{3.648126in}}%
\pgfpathlineto{\pgfqpoint{1.556161in}{3.651028in}}%
\pgfpathlineto{\pgfqpoint{1.571483in}{3.653570in}}%
\pgfpathlineto{\pgfqpoint{1.575313in}{3.656502in}}%
\pgfpathlineto{\pgfqpoint{1.579143in}{3.657131in}}%
\pgfpathlineto{\pgfqpoint{1.588719in}{3.665141in}}%
\pgfpathlineto{\pgfqpoint{1.598295in}{3.666573in}}%
\pgfpathlineto{\pgfqpoint{1.600210in}{3.668831in}}%
\pgfpathlineto{\pgfqpoint{1.613617in}{3.670948in}}%
\pgfpathlineto{\pgfqpoint{1.619362in}{3.673811in}}%
\pgfpathlineto{\pgfqpoint{1.621277in}{3.676896in}}%
\pgfpathlineto{\pgfqpoint{1.628938in}{3.678205in}}%
\pgfpathlineto{\pgfqpoint{1.634683in}{3.679250in}}%
\pgfpathlineto{\pgfqpoint{1.657666in}{3.683310in}}%
\pgfpathlineto{\pgfqpoint{1.661496in}{3.684595in}}%
\pgfpathlineto{\pgfqpoint{1.674902in}{3.686100in}}%
\pgfpathlineto{\pgfqpoint{1.678732in}{3.687974in}}%
\pgfpathlineto{\pgfqpoint{1.688308in}{3.689503in}}%
\pgfpathlineto{\pgfqpoint{1.692139in}{3.689843in}}%
\pgfpathlineto{\pgfqpoint{1.695969in}{3.692034in}}%
\pgfpathlineto{\pgfqpoint{1.699799in}{3.692628in}}%
\pgfpathlineto{\pgfqpoint{1.713205in}{3.697409in}}%
\pgfpathlineto{\pgfqpoint{1.720866in}{3.698825in}}%
\pgfpathlineto{\pgfqpoint{1.724697in}{3.700397in}}%
\pgfpathlineto{\pgfqpoint{1.753424in}{3.707860in}}%
\pgfpathlineto{\pgfqpoint{1.763000in}{3.708952in}}%
\pgfpathlineto{\pgfqpoint{1.764915in}{3.711141in}}%
\pgfpathlineto{\pgfqpoint{1.770661in}{3.712063in}}%
\pgfpathlineto{\pgfqpoint{1.776406in}{3.713726in}}%
\pgfpathlineto{\pgfqpoint{1.793643in}{3.715392in}}%
\pgfpathlineto{\pgfqpoint{1.795558in}{3.715536in}}%
\pgfpathlineto{\pgfqpoint{1.799388in}{3.716974in}}%
\pgfpathlineto{\pgfqpoint{1.814710in}{3.718639in}}%
\pgfpathlineto{\pgfqpoint{1.818540in}{3.720454in}}%
\pgfpathlineto{\pgfqpoint{1.847267in}{3.726079in}}%
\pgfpathlineto{\pgfqpoint{1.851098in}{3.727411in}}%
\pgfpathlineto{\pgfqpoint{1.856843in}{3.728987in}}%
\pgfpathlineto{\pgfqpoint{1.860674in}{3.731553in}}%
\pgfpathlineto{\pgfqpoint{1.866419in}{3.732501in}}%
\pgfpathlineto{\pgfqpoint{1.870250in}{3.732849in}}%
\pgfpathlineto{\pgfqpoint{1.879825in}{3.739773in}}%
\pgfpathlineto{\pgfqpoint{1.881741in}{3.742819in}}%
\pgfpathlineto{\pgfqpoint{1.883656in}{3.743073in}}%
\pgfpathlineto{\pgfqpoint{1.885571in}{3.747449in}}%
\pgfpathlineto{\pgfqpoint{1.891316in}{3.747937in}}%
\pgfpathlineto{\pgfqpoint{1.895147in}{3.751387in}}%
\pgfpathlineto{\pgfqpoint{1.900892in}{3.755464in}}%
\pgfpathlineto{\pgfqpoint{1.902807in}{3.755709in}}%
\pgfpathlineto{\pgfqpoint{1.904723in}{3.757518in}}%
\pgfpathlineto{\pgfqpoint{1.910468in}{3.757948in}}%
\pgfpathlineto{\pgfqpoint{1.916214in}{3.763052in}}%
\pgfpathlineto{\pgfqpoint{1.931535in}{3.766202in}}%
\pgfpathlineto{\pgfqpoint{1.952602in}{3.773335in}}%
\pgfpathlineto{\pgfqpoint{1.954517in}{3.775977in}}%
\pgfpathlineto{\pgfqpoint{1.962178in}{3.778000in}}%
\pgfpathlineto{\pgfqpoint{1.967923in}{3.778611in}}%
\pgfpathlineto{\pgfqpoint{1.992821in}{3.786203in}}%
\pgfpathlineto{\pgfqpoint{1.996651in}{3.790289in}}%
\pgfpathlineto{\pgfqpoint{2.021548in}{3.799021in}}%
\pgfpathlineto{\pgfqpoint{2.025378in}{3.799625in}}%
\pgfpathlineto{\pgfqpoint{2.033039in}{3.803677in}}%
\pgfpathlineto{\pgfqpoint{2.036869in}{3.804173in}}%
\pgfpathlineto{\pgfqpoint{2.040700in}{3.806074in}}%
\pgfpathlineto{\pgfqpoint{2.046445in}{3.807526in}}%
\pgfpathlineto{\pgfqpoint{2.054106in}{3.810824in}}%
\pgfpathlineto{\pgfqpoint{2.057936in}{3.815192in}}%
\pgfpathlineto{\pgfqpoint{2.065597in}{3.818640in}}%
\pgfpathlineto{\pgfqpoint{2.075173in}{3.824342in}}%
\pgfpathlineto{\pgfqpoint{2.086664in}{3.827635in}}%
\pgfpathlineto{\pgfqpoint{2.090494in}{3.831592in}}%
\pgfpathlineto{\pgfqpoint{2.094325in}{3.833201in}}%
\pgfpathlineto{\pgfqpoint{2.101985in}{3.838687in}}%
\pgfpathlineto{\pgfqpoint{2.103900in}{3.840701in}}%
\pgfpathlineto{\pgfqpoint{2.109646in}{3.842498in}}%
\pgfpathlineto{\pgfqpoint{2.113476in}{3.844753in}}%
\pgfpathlineto{\pgfqpoint{2.119222in}{3.848723in}}%
\pgfpathlineto{\pgfqpoint{2.121137in}{3.851469in}}%
\pgfpathlineto{\pgfqpoint{2.124967in}{3.859198in}}%
\pgfpathlineto{\pgfqpoint{2.126883in}{3.862518in}}%
\pgfpathlineto{\pgfqpoint{2.130713in}{3.862651in}}%
\pgfpathlineto{\pgfqpoint{2.134543in}{3.864607in}}%
\pgfpathlineto{\pgfqpoint{2.142204in}{3.865526in}}%
\pgfpathlineto{\pgfqpoint{2.157525in}{3.868956in}}%
\pgfpathlineto{\pgfqpoint{2.165186in}{3.869569in}}%
\pgfpathlineto{\pgfqpoint{2.169016in}{3.870878in}}%
\pgfpathlineto{\pgfqpoint{2.176677in}{3.873475in}}%
\pgfpathlineto{\pgfqpoint{2.178592in}{3.875479in}}%
\pgfpathlineto{\pgfqpoint{2.184338in}{3.877028in}}%
\pgfpathlineto{\pgfqpoint{2.190083in}{3.878827in}}%
\pgfpathlineto{\pgfqpoint{2.197744in}{3.881227in}}%
\pgfpathlineto{\pgfqpoint{2.203489in}{3.883158in}}%
\pgfpathlineto{\pgfqpoint{2.213065in}{3.884400in}}%
\pgfpathlineto{\pgfqpoint{2.218811in}{3.885618in}}%
\pgfpathlineto{\pgfqpoint{2.232217in}{3.886895in}}%
\pgfpathlineto{\pgfqpoint{2.239878in}{3.888176in}}%
\pgfpathlineto{\pgfqpoint{2.251369in}{3.889807in}}%
\pgfpathlineto{\pgfqpoint{2.257114in}{3.892122in}}%
\pgfpathlineto{\pgfqpoint{2.260945in}{3.896411in}}%
\pgfpathlineto{\pgfqpoint{2.270520in}{3.897553in}}%
\pgfpathlineto{\pgfqpoint{2.274351in}{3.899628in}}%
\pgfpathlineto{\pgfqpoint{2.282011in}{3.901012in}}%
\pgfpathlineto{\pgfqpoint{2.283927in}{3.903426in}}%
\pgfpathlineto{\pgfqpoint{2.285842in}{3.903479in}}%
\pgfpathlineto{\pgfqpoint{2.287757in}{3.907018in}}%
\pgfpathlineto{\pgfqpoint{2.291587in}{3.907898in}}%
\pgfpathlineto{\pgfqpoint{2.295418in}{3.913886in}}%
\pgfpathlineto{\pgfqpoint{2.299248in}{3.916461in}}%
\pgfpathlineto{\pgfqpoint{2.306909in}{3.918279in}}%
\pgfpathlineto{\pgfqpoint{2.314569in}{3.926336in}}%
\pgfpathlineto{\pgfqpoint{2.316484in}{3.926534in}}%
\pgfpathlineto{\pgfqpoint{2.320315in}{3.930038in}}%
\pgfpathlineto{\pgfqpoint{2.327975in}{3.932756in}}%
\pgfpathlineto{\pgfqpoint{2.335636in}{3.936149in}}%
\pgfpathlineto{\pgfqpoint{2.347127in}{3.939064in}}%
\pgfpathlineto{\pgfqpoint{2.350958in}{3.941577in}}%
\pgfpathlineto{\pgfqpoint{2.358618in}{3.943525in}}%
\pgfpathlineto{\pgfqpoint{2.370109in}{3.947416in}}%
\pgfpathlineto{\pgfqpoint{2.373940in}{3.950340in}}%
\pgfpathlineto{\pgfqpoint{2.379685in}{3.952869in}}%
\pgfpathlineto{\pgfqpoint{2.381600in}{3.954559in}}%
\pgfpathlineto{\pgfqpoint{2.393091in}{3.955495in}}%
\pgfpathlineto{\pgfqpoint{2.395006in}{3.956414in}}%
\pgfpathlineto{\pgfqpoint{2.396922in}{3.958803in}}%
\pgfpathlineto{\pgfqpoint{2.404582in}{3.960347in}}%
\pgfpathlineto{\pgfqpoint{2.408413in}{3.964747in}}%
\pgfpathlineto{\pgfqpoint{2.440971in}{3.976404in}}%
\pgfpathlineto{\pgfqpoint{2.448631in}{3.977275in}}%
\pgfpathlineto{\pgfqpoint{2.450546in}{3.977990in}}%
\pgfpathlineto{\pgfqpoint{2.452462in}{3.980657in}}%
\pgfpathlineto{\pgfqpoint{2.465868in}{3.984628in}}%
\pgfpathlineto{\pgfqpoint{2.469698in}{3.986331in}}%
\pgfpathlineto{\pgfqpoint{2.473529in}{3.987228in}}%
\pgfpathlineto{\pgfqpoint{2.500341in}{3.996734in}}%
\pgfpathlineto{\pgfqpoint{2.502256in}{3.997057in}}%
\pgfpathlineto{\pgfqpoint{2.504171in}{3.999470in}}%
\pgfpathlineto{\pgfqpoint{2.509917in}{4.000522in}}%
\pgfpathlineto{\pgfqpoint{2.515662in}{4.002106in}}%
\pgfpathlineto{\pgfqpoint{2.519493in}{4.005572in}}%
\pgfpathlineto{\pgfqpoint{2.525238in}{4.007817in}}%
\pgfpathlineto{\pgfqpoint{2.527153in}{4.010170in}}%
\pgfpathlineto{\pgfqpoint{2.529068in}{4.010363in}}%
\pgfpathlineto{\pgfqpoint{2.530984in}{4.012406in}}%
\pgfpathlineto{\pgfqpoint{2.534814in}{4.012515in}}%
\pgfpathlineto{\pgfqpoint{2.536729in}{4.017148in}}%
\pgfpathlineto{\pgfqpoint{2.540560in}{4.019462in}}%
\pgfpathlineto{\pgfqpoint{2.544390in}{4.024970in}}%
\pgfpathlineto{\pgfqpoint{2.548220in}{4.025995in}}%
\pgfpathlineto{\pgfqpoint{2.553966in}{4.027198in}}%
\pgfpathlineto{\pgfqpoint{2.559711in}{4.033810in}}%
\pgfpathlineto{\pgfqpoint{2.565457in}{4.035109in}}%
\pgfpathlineto{\pgfqpoint{2.575033in}{4.038690in}}%
\pgfpathlineto{\pgfqpoint{2.576948in}{4.039147in}}%
\pgfpathlineto{\pgfqpoint{2.578863in}{4.041361in}}%
\pgfpathlineto{\pgfqpoint{2.599930in}{4.045701in}}%
\pgfpathlineto{\pgfqpoint{2.603760in}{4.049437in}}%
\pgfpathlineto{\pgfqpoint{2.607591in}{4.050791in}}%
\pgfpathlineto{\pgfqpoint{2.611421in}{4.055477in}}%
\pgfpathlineto{\pgfqpoint{2.617166in}{4.056161in}}%
\pgfpathlineto{\pgfqpoint{2.619082in}{4.060354in}}%
\pgfpathlineto{\pgfqpoint{2.642064in}{4.066449in}}%
\pgfpathlineto{\pgfqpoint{2.645894in}{4.070055in}}%
\pgfpathlineto{\pgfqpoint{2.651639in}{4.071284in}}%
\pgfpathlineto{\pgfqpoint{2.655470in}{4.071608in}}%
\pgfpathlineto{\pgfqpoint{2.659300in}{4.076525in}}%
\pgfpathlineto{\pgfqpoint{2.676537in}{4.080376in}}%
\pgfpathlineto{\pgfqpoint{2.682282in}{4.086929in}}%
\pgfpathlineto{\pgfqpoint{2.689943in}{4.088096in}}%
\pgfpathlineto{\pgfqpoint{2.691858in}{4.093938in}}%
\pgfpathlineto{\pgfqpoint{2.693773in}{4.094660in}}%
\pgfpathlineto{\pgfqpoint{2.697604in}{4.097449in}}%
\pgfpathlineto{\pgfqpoint{2.705264in}{4.101308in}}%
\pgfpathlineto{\pgfqpoint{2.707179in}{4.105383in}}%
\pgfpathlineto{\pgfqpoint{2.722501in}{4.108881in}}%
\pgfpathlineto{\pgfqpoint{2.726331in}{4.109820in}}%
\pgfpathlineto{\pgfqpoint{2.730161in}{4.115322in}}%
\pgfpathlineto{\pgfqpoint{2.732077in}{4.115715in}}%
\pgfpathlineto{\pgfqpoint{2.733992in}{4.117244in}}%
\pgfpathlineto{\pgfqpoint{2.739737in}{4.117982in}}%
\pgfpathlineto{\pgfqpoint{2.745483in}{4.120026in}}%
\pgfpathlineto{\pgfqpoint{2.747398in}{4.122644in}}%
\pgfpathlineto{\pgfqpoint{2.756974in}{4.125541in}}%
\pgfpathlineto{\pgfqpoint{2.762719in}{4.129863in}}%
\pgfpathlineto{\pgfqpoint{2.764635in}{4.130184in}}%
\pgfpathlineto{\pgfqpoint{2.768465in}{4.132768in}}%
\pgfpathlineto{\pgfqpoint{2.770380in}{4.133443in}}%
\pgfpathlineto{\pgfqpoint{2.774210in}{4.135931in}}%
\pgfpathlineto{\pgfqpoint{2.779956in}{4.138962in}}%
\pgfpathlineto{\pgfqpoint{2.785701in}{4.140374in}}%
\pgfpathlineto{\pgfqpoint{2.793362in}{4.143775in}}%
\pgfpathlineto{\pgfqpoint{2.804853in}{4.157972in}}%
\pgfpathlineto{\pgfqpoint{2.806768in}{4.159755in}}%
\pgfpathlineto{\pgfqpoint{2.812514in}{4.160748in}}%
\pgfpathlineto{\pgfqpoint{2.824005in}{4.166306in}}%
\pgfpathlineto{\pgfqpoint{2.833581in}{4.169842in}}%
\pgfpathlineto{\pgfqpoint{2.839326in}{4.174638in}}%
\pgfpathlineto{\pgfqpoint{2.843157in}{4.175113in}}%
\pgfpathlineto{\pgfqpoint{2.845072in}{4.176998in}}%
\pgfpathlineto{\pgfqpoint{2.846987in}{4.177159in}}%
\pgfpathlineto{\pgfqpoint{2.852732in}{4.182634in}}%
\pgfpathlineto{\pgfqpoint{2.854648in}{4.182721in}}%
\pgfpathlineto{\pgfqpoint{2.858478in}{4.186241in}}%
\pgfpathlineto{\pgfqpoint{2.864223in}{4.187763in}}%
\pgfpathlineto{\pgfqpoint{2.868054in}{4.192565in}}%
\pgfpathlineto{\pgfqpoint{2.873799in}{4.194628in}}%
\pgfpathlineto{\pgfqpoint{2.875715in}{4.201733in}}%
\pgfpathlineto{\pgfqpoint{2.879545in}{4.206002in}}%
\pgfpathlineto{\pgfqpoint{2.881460in}{4.206855in}}%
\pgfpathlineto{\pgfqpoint{2.883375in}{4.221612in}}%
\pgfpathlineto{\pgfqpoint{2.896781in}{4.225681in}}%
\pgfpathlineto{\pgfqpoint{2.898697in}{4.228109in}}%
\pgfpathlineto{\pgfqpoint{2.898697in}{4.228109in}}%
\pgfusepath{stroke}%
\end{pgfscope}%
\begin{pgfscope}%
\pgfpathrectangle{\pgfqpoint{0.694334in}{2.659974in}}{\pgfqpoint{3.830343in}{1.568135in}}%
\pgfusepath{clip}%
\pgfsetrectcap%
\pgfsetroundjoin%
\pgfsetlinewidth{1.003750pt}%
\definecolor{currentstroke}{rgb}{0.564706,0.564706,1.000000}%
\pgfsetstrokecolor{currentstroke}%
\pgfsetdash{}{0pt}%
\pgfpathmoveto{\pgfqpoint{0.694334in}{2.834130in}}%
\pgfpathlineto{\pgfqpoint{0.698165in}{2.859952in}}%
\pgfpathlineto{\pgfqpoint{0.700080in}{2.862221in}}%
\pgfpathlineto{\pgfqpoint{0.701995in}{2.891688in}}%
\pgfpathlineto{\pgfqpoint{0.705825in}{2.908409in}}%
\pgfpathlineto{\pgfqpoint{0.707741in}{2.909720in}}%
\pgfpathlineto{\pgfqpoint{0.709656in}{2.918033in}}%
\pgfpathlineto{\pgfqpoint{0.711571in}{2.918811in}}%
\pgfpathlineto{\pgfqpoint{0.713486in}{2.922557in}}%
\pgfpathlineto{\pgfqpoint{0.724977in}{2.959859in}}%
\pgfpathlineto{\pgfqpoint{0.726892in}{2.963458in}}%
\pgfpathlineto{\pgfqpoint{0.728807in}{2.964191in}}%
\pgfpathlineto{\pgfqpoint{0.736468in}{2.976429in}}%
\pgfpathlineto{\pgfqpoint{0.738383in}{2.980247in}}%
\pgfpathlineto{\pgfqpoint{0.740298in}{2.981575in}}%
\pgfpathlineto{\pgfqpoint{0.746044in}{2.994992in}}%
\pgfpathlineto{\pgfqpoint{0.747959in}{3.003264in}}%
\pgfpathlineto{\pgfqpoint{0.749874in}{3.004346in}}%
\pgfpathlineto{\pgfqpoint{0.751789in}{3.008331in}}%
\pgfpathlineto{\pgfqpoint{0.753705in}{3.009308in}}%
\pgfpathlineto{\pgfqpoint{0.755620in}{3.020756in}}%
\pgfpathlineto{\pgfqpoint{0.757535in}{3.021184in}}%
\pgfpathlineto{\pgfqpoint{0.759450in}{3.023222in}}%
\pgfpathlineto{\pgfqpoint{0.761365in}{3.023325in}}%
\pgfpathlineto{\pgfqpoint{0.765196in}{3.036168in}}%
\pgfpathlineto{\pgfqpoint{0.767111in}{3.036609in}}%
\pgfpathlineto{\pgfqpoint{0.769026in}{3.044314in}}%
\pgfpathlineto{\pgfqpoint{0.772856in}{3.046513in}}%
\pgfpathlineto{\pgfqpoint{0.774772in}{3.055306in}}%
\pgfpathlineto{\pgfqpoint{0.776687in}{3.055578in}}%
\pgfpathlineto{\pgfqpoint{0.778602in}{3.063262in}}%
\pgfpathlineto{\pgfqpoint{0.780517in}{3.066160in}}%
\pgfpathlineto{\pgfqpoint{0.782432in}{3.066445in}}%
\pgfpathlineto{\pgfqpoint{0.788178in}{3.072855in}}%
\pgfpathlineto{\pgfqpoint{0.790093in}{3.079989in}}%
\pgfpathlineto{\pgfqpoint{0.793923in}{3.080659in}}%
\pgfpathlineto{\pgfqpoint{0.797754in}{3.082357in}}%
\pgfpathlineto{\pgfqpoint{0.803499in}{3.091300in}}%
\pgfpathlineto{\pgfqpoint{0.805414in}{3.092710in}}%
\pgfpathlineto{\pgfqpoint{0.807329in}{3.096213in}}%
\pgfpathlineto{\pgfqpoint{0.813075in}{3.097959in}}%
\pgfpathlineto{\pgfqpoint{0.816905in}{3.104304in}}%
\pgfpathlineto{\pgfqpoint{0.820736in}{3.104889in}}%
\pgfpathlineto{\pgfqpoint{0.822651in}{3.107304in}}%
\pgfpathlineto{\pgfqpoint{0.828396in}{3.109594in}}%
\pgfpathlineto{\pgfqpoint{0.832227in}{3.111976in}}%
\pgfpathlineto{\pgfqpoint{0.837972in}{3.114112in}}%
\pgfpathlineto{\pgfqpoint{0.841803in}{3.117545in}}%
\pgfpathlineto{\pgfqpoint{0.843718in}{3.117764in}}%
\pgfpathlineto{\pgfqpoint{0.851378in}{3.126791in}}%
\pgfpathlineto{\pgfqpoint{0.853294in}{3.126946in}}%
\pgfpathlineto{\pgfqpoint{0.857124in}{3.132256in}}%
\pgfpathlineto{\pgfqpoint{0.860954in}{3.134451in}}%
\pgfpathlineto{\pgfqpoint{0.862869in}{3.146808in}}%
\pgfpathlineto{\pgfqpoint{0.864785in}{3.149368in}}%
\pgfpathlineto{\pgfqpoint{0.874360in}{3.151082in}}%
\pgfpathlineto{\pgfqpoint{0.880106in}{3.158117in}}%
\pgfpathlineto{\pgfqpoint{0.893512in}{3.164187in}}%
\pgfpathlineto{\pgfqpoint{0.895427in}{3.164260in}}%
\pgfpathlineto{\pgfqpoint{0.899258in}{3.167551in}}%
\pgfpathlineto{\pgfqpoint{0.901173in}{3.167639in}}%
\pgfpathlineto{\pgfqpoint{0.908834in}{3.172884in}}%
\pgfpathlineto{\pgfqpoint{0.912664in}{3.173365in}}%
\pgfpathlineto{\pgfqpoint{0.922240in}{3.180872in}}%
\pgfpathlineto{\pgfqpoint{0.931816in}{3.183410in}}%
\pgfpathlineto{\pgfqpoint{0.937561in}{3.184098in}}%
\pgfpathlineto{\pgfqpoint{0.939476in}{3.185910in}}%
\pgfpathlineto{\pgfqpoint{0.941391in}{3.185998in}}%
\pgfpathlineto{\pgfqpoint{0.943307in}{3.188092in}}%
\pgfpathlineto{\pgfqpoint{0.947137in}{3.189195in}}%
\pgfpathlineto{\pgfqpoint{0.950967in}{3.190198in}}%
\pgfpathlineto{\pgfqpoint{0.958628in}{3.190893in}}%
\pgfpathlineto{\pgfqpoint{0.972034in}{3.194005in}}%
\pgfpathlineto{\pgfqpoint{0.977780in}{3.196022in}}%
\pgfpathlineto{\pgfqpoint{0.981610in}{3.196671in}}%
\pgfpathlineto{\pgfqpoint{0.985440in}{3.199015in}}%
\pgfpathlineto{\pgfqpoint{0.995016in}{3.201972in}}%
\pgfpathlineto{\pgfqpoint{1.002677in}{3.203718in}}%
\pgfpathlineto{\pgfqpoint{1.004592in}{3.205470in}}%
\pgfpathlineto{\pgfqpoint{1.008422in}{3.206307in}}%
\pgfpathlineto{\pgfqpoint{1.012253in}{3.207225in}}%
\pgfpathlineto{\pgfqpoint{1.017998in}{3.208470in}}%
\pgfpathlineto{\pgfqpoint{1.023744in}{3.214818in}}%
\pgfpathlineto{\pgfqpoint{1.025659in}{3.215614in}}%
\pgfpathlineto{\pgfqpoint{1.027574in}{3.217676in}}%
\pgfpathlineto{\pgfqpoint{1.035235in}{3.219808in}}%
\pgfpathlineto{\pgfqpoint{1.046726in}{3.224424in}}%
\pgfpathlineto{\pgfqpoint{1.052471in}{3.226004in}}%
\pgfpathlineto{\pgfqpoint{1.094605in}{3.236962in}}%
\pgfpathlineto{\pgfqpoint{1.102266in}{3.237769in}}%
\pgfpathlineto{\pgfqpoint{1.106096in}{3.239092in}}%
\pgfpathlineto{\pgfqpoint{1.108011in}{3.239371in}}%
\pgfpathlineto{\pgfqpoint{1.111842in}{3.242351in}}%
\pgfpathlineto{\pgfqpoint{1.117587in}{3.243662in}}%
\pgfpathlineto{\pgfqpoint{1.121418in}{3.244715in}}%
\pgfpathlineto{\pgfqpoint{1.130993in}{3.245686in}}%
\pgfpathlineto{\pgfqpoint{1.148230in}{3.249007in}}%
\pgfpathlineto{\pgfqpoint{1.152060in}{3.250797in}}%
\pgfpathlineto{\pgfqpoint{1.153975in}{3.251739in}}%
\pgfpathlineto{\pgfqpoint{1.157806in}{3.255048in}}%
\pgfpathlineto{\pgfqpoint{1.159721in}{3.255155in}}%
\pgfpathlineto{\pgfqpoint{1.163551in}{3.256960in}}%
\pgfpathlineto{\pgfqpoint{1.175042in}{3.259171in}}%
\pgfpathlineto{\pgfqpoint{1.196109in}{3.265809in}}%
\pgfpathlineto{\pgfqpoint{1.207600in}{3.268268in}}%
\pgfpathlineto{\pgfqpoint{1.211431in}{3.269853in}}%
\pgfpathlineto{\pgfqpoint{1.222922in}{3.272481in}}%
\pgfpathlineto{\pgfqpoint{1.226752in}{3.273110in}}%
\pgfpathlineto{\pgfqpoint{1.232497in}{3.276051in}}%
\pgfpathlineto{\pgfqpoint{1.253564in}{3.280124in}}%
\pgfpathlineto{\pgfqpoint{1.255480in}{3.282548in}}%
\pgfpathlineto{\pgfqpoint{1.266971in}{3.285784in}}%
\pgfpathlineto{\pgfqpoint{1.270801in}{3.288263in}}%
\pgfpathlineto{\pgfqpoint{1.288037in}{3.291414in}}%
\pgfpathlineto{\pgfqpoint{1.291868in}{3.292813in}}%
\pgfpathlineto{\pgfqpoint{1.293783in}{3.292858in}}%
\pgfpathlineto{\pgfqpoint{1.311019in}{3.303477in}}%
\pgfpathlineto{\pgfqpoint{1.312935in}{3.303965in}}%
\pgfpathlineto{\pgfqpoint{1.316765in}{3.307260in}}%
\pgfpathlineto{\pgfqpoint{1.343577in}{3.312991in}}%
\pgfpathlineto{\pgfqpoint{1.353153in}{3.316729in}}%
\pgfpathlineto{\pgfqpoint{1.356984in}{3.317605in}}%
\pgfpathlineto{\pgfqpoint{1.358899in}{3.320421in}}%
\pgfpathlineto{\pgfqpoint{1.381881in}{3.326253in}}%
\pgfpathlineto{\pgfqpoint{1.385711in}{3.329726in}}%
\pgfpathlineto{\pgfqpoint{1.395287in}{3.334415in}}%
\pgfpathlineto{\pgfqpoint{1.422099in}{3.341068in}}%
\pgfpathlineto{\pgfqpoint{1.433590in}{3.342063in}}%
\pgfpathlineto{\pgfqpoint{1.439336in}{3.344727in}}%
\pgfpathlineto{\pgfqpoint{1.446997in}{3.347138in}}%
\pgfpathlineto{\pgfqpoint{1.452742in}{3.352097in}}%
\pgfpathlineto{\pgfqpoint{1.454657in}{3.352163in}}%
\pgfpathlineto{\pgfqpoint{1.456573in}{3.353614in}}%
\pgfpathlineto{\pgfqpoint{1.462318in}{3.354557in}}%
\pgfpathlineto{\pgfqpoint{1.468064in}{3.355737in}}%
\pgfpathlineto{\pgfqpoint{1.473809in}{3.359871in}}%
\pgfpathlineto{\pgfqpoint{1.475724in}{3.360093in}}%
\pgfpathlineto{\pgfqpoint{1.481470in}{3.363941in}}%
\pgfpathlineto{\pgfqpoint{1.498706in}{3.369436in}}%
\pgfpathlineto{\pgfqpoint{1.500621in}{3.371953in}}%
\pgfpathlineto{\pgfqpoint{1.508282in}{3.373009in}}%
\pgfpathlineto{\pgfqpoint{1.510197in}{3.375977in}}%
\pgfpathlineto{\pgfqpoint{1.521688in}{3.378446in}}%
\pgfpathlineto{\pgfqpoint{1.523604in}{3.379454in}}%
\pgfpathlineto{\pgfqpoint{1.525519in}{3.381946in}}%
\pgfpathlineto{\pgfqpoint{1.527434in}{3.382252in}}%
\pgfpathlineto{\pgfqpoint{1.533179in}{3.385823in}}%
\pgfpathlineto{\pgfqpoint{1.538925in}{3.387472in}}%
\pgfpathlineto{\pgfqpoint{1.542755in}{3.388644in}}%
\pgfpathlineto{\pgfqpoint{1.546586in}{3.391954in}}%
\pgfpathlineto{\pgfqpoint{1.558077in}{3.394252in}}%
\pgfpathlineto{\pgfqpoint{1.561907in}{3.396242in}}%
\pgfpathlineto{\pgfqpoint{1.563822in}{3.396501in}}%
\pgfpathlineto{\pgfqpoint{1.565737in}{3.398613in}}%
\pgfpathlineto{\pgfqpoint{1.567652in}{3.398907in}}%
\pgfpathlineto{\pgfqpoint{1.569568in}{3.402714in}}%
\pgfpathlineto{\pgfqpoint{1.575313in}{3.403736in}}%
\pgfpathlineto{\pgfqpoint{1.579143in}{3.406119in}}%
\pgfpathlineto{\pgfqpoint{1.582974in}{3.406693in}}%
\pgfpathlineto{\pgfqpoint{1.586804in}{3.408705in}}%
\pgfpathlineto{\pgfqpoint{1.605956in}{3.415841in}}%
\pgfpathlineto{\pgfqpoint{1.611701in}{3.416531in}}%
\pgfpathlineto{\pgfqpoint{1.613617in}{3.420561in}}%
\pgfpathlineto{\pgfqpoint{1.619362in}{3.423260in}}%
\pgfpathlineto{\pgfqpoint{1.632768in}{3.427866in}}%
\pgfpathlineto{\pgfqpoint{1.636599in}{3.433145in}}%
\pgfpathlineto{\pgfqpoint{1.644259in}{3.437297in}}%
\pgfpathlineto{\pgfqpoint{1.646174in}{3.437486in}}%
\pgfpathlineto{\pgfqpoint{1.648090in}{3.439958in}}%
\pgfpathlineto{\pgfqpoint{1.651920in}{3.440454in}}%
\pgfpathlineto{\pgfqpoint{1.653835in}{3.442697in}}%
\pgfpathlineto{\pgfqpoint{1.661496in}{3.443485in}}%
\pgfpathlineto{\pgfqpoint{1.665326in}{3.445443in}}%
\pgfpathlineto{\pgfqpoint{1.671072in}{3.446111in}}%
\pgfpathlineto{\pgfqpoint{1.676817in}{3.450038in}}%
\pgfpathlineto{\pgfqpoint{1.684478in}{3.456678in}}%
\pgfpathlineto{\pgfqpoint{1.688308in}{3.457259in}}%
\pgfpathlineto{\pgfqpoint{1.701714in}{3.463594in}}%
\pgfpathlineto{\pgfqpoint{1.709375in}{3.465053in}}%
\pgfpathlineto{\pgfqpoint{1.715121in}{3.468364in}}%
\pgfpathlineto{\pgfqpoint{1.722781in}{3.469498in}}%
\pgfpathlineto{\pgfqpoint{1.726612in}{3.471262in}}%
\pgfpathlineto{\pgfqpoint{1.745763in}{3.478242in}}%
\pgfpathlineto{\pgfqpoint{1.749594in}{3.480312in}}%
\pgfpathlineto{\pgfqpoint{1.753424in}{3.480841in}}%
\pgfpathlineto{\pgfqpoint{1.785982in}{3.494740in}}%
\pgfpathlineto{\pgfqpoint{1.793643in}{3.496998in}}%
\pgfpathlineto{\pgfqpoint{1.805134in}{3.508177in}}%
\pgfpathlineto{\pgfqpoint{1.810879in}{3.509301in}}%
\pgfpathlineto{\pgfqpoint{1.814710in}{3.510904in}}%
\pgfpathlineto{\pgfqpoint{1.826201in}{3.512714in}}%
\pgfpathlineto{\pgfqpoint{1.833861in}{3.517794in}}%
\pgfpathlineto{\pgfqpoint{1.839607in}{3.521530in}}%
\pgfpathlineto{\pgfqpoint{1.849183in}{3.524693in}}%
\pgfpathlineto{\pgfqpoint{1.860674in}{3.525940in}}%
\pgfpathlineto{\pgfqpoint{1.866419in}{3.528612in}}%
\pgfpathlineto{\pgfqpoint{1.872165in}{3.528693in}}%
\pgfpathlineto{\pgfqpoint{1.875995in}{3.531123in}}%
\pgfpathlineto{\pgfqpoint{1.879825in}{3.532721in}}%
\pgfpathlineto{\pgfqpoint{1.883656in}{3.533065in}}%
\pgfpathlineto{\pgfqpoint{1.887486in}{3.534749in}}%
\pgfpathlineto{\pgfqpoint{1.897062in}{3.537557in}}%
\pgfpathlineto{\pgfqpoint{1.900892in}{3.539152in}}%
\pgfpathlineto{\pgfqpoint{1.906638in}{3.540656in}}%
\pgfpathlineto{\pgfqpoint{1.912383in}{3.543577in}}%
\pgfpathlineto{\pgfqpoint{1.914298in}{3.546629in}}%
\pgfpathlineto{\pgfqpoint{1.918129in}{3.547806in}}%
\pgfpathlineto{\pgfqpoint{1.925790in}{3.551293in}}%
\pgfpathlineto{\pgfqpoint{1.935365in}{3.553146in}}%
\pgfpathlineto{\pgfqpoint{1.941111in}{3.555014in}}%
\pgfpathlineto{\pgfqpoint{1.943026in}{3.557671in}}%
\pgfpathlineto{\pgfqpoint{1.950687in}{3.559212in}}%
\pgfpathlineto{\pgfqpoint{1.960263in}{3.568454in}}%
\pgfpathlineto{\pgfqpoint{1.981329in}{3.572992in}}%
\pgfpathlineto{\pgfqpoint{1.992821in}{3.575482in}}%
\pgfpathlineto{\pgfqpoint{1.994736in}{3.577281in}}%
\pgfpathlineto{\pgfqpoint{1.998566in}{3.578190in}}%
\pgfpathlineto{\pgfqpoint{2.002396in}{3.579623in}}%
\pgfpathlineto{\pgfqpoint{2.006227in}{3.581106in}}%
\pgfpathlineto{\pgfqpoint{2.011972in}{3.583637in}}%
\pgfpathlineto{\pgfqpoint{2.017718in}{3.584841in}}%
\pgfpathlineto{\pgfqpoint{2.038785in}{3.589321in}}%
\pgfpathlineto{\pgfqpoint{2.048360in}{3.595518in}}%
\pgfpathlineto{\pgfqpoint{2.050276in}{3.595689in}}%
\pgfpathlineto{\pgfqpoint{2.054106in}{3.597479in}}%
\pgfpathlineto{\pgfqpoint{2.057936in}{3.599365in}}%
\pgfpathlineto{\pgfqpoint{2.061767in}{3.600373in}}%
\pgfpathlineto{\pgfqpoint{2.065597in}{3.601248in}}%
\pgfpathlineto{\pgfqpoint{2.079003in}{3.606896in}}%
\pgfpathlineto{\pgfqpoint{2.082834in}{3.608409in}}%
\pgfpathlineto{\pgfqpoint{2.090494in}{3.609707in}}%
\pgfpathlineto{\pgfqpoint{2.094325in}{3.611735in}}%
\pgfpathlineto{\pgfqpoint{2.103900in}{3.614079in}}%
\pgfpathlineto{\pgfqpoint{2.105816in}{3.617041in}}%
\pgfpathlineto{\pgfqpoint{2.119222in}{3.618612in}}%
\pgfpathlineto{\pgfqpoint{2.136458in}{3.626871in}}%
\pgfpathlineto{\pgfqpoint{2.146034in}{3.627818in}}%
\pgfpathlineto{\pgfqpoint{2.153695in}{3.634198in}}%
\pgfpathlineto{\pgfqpoint{2.157525in}{3.634434in}}%
\pgfpathlineto{\pgfqpoint{2.161356in}{3.636670in}}%
\pgfpathlineto{\pgfqpoint{2.167101in}{3.637303in}}%
\pgfpathlineto{\pgfqpoint{2.188168in}{3.645192in}}%
\pgfpathlineto{\pgfqpoint{2.199659in}{3.646590in}}%
\pgfpathlineto{\pgfqpoint{2.203489in}{3.649100in}}%
\pgfpathlineto{\pgfqpoint{2.207320in}{3.650477in}}%
\pgfpathlineto{\pgfqpoint{2.213065in}{3.654966in}}%
\pgfpathlineto{\pgfqpoint{2.218811in}{3.656135in}}%
\pgfpathlineto{\pgfqpoint{2.224556in}{3.660002in}}%
\pgfpathlineto{\pgfqpoint{2.232217in}{3.661590in}}%
\pgfpathlineto{\pgfqpoint{2.236047in}{3.665720in}}%
\pgfpathlineto{\pgfqpoint{2.243708in}{3.669458in}}%
\pgfpathlineto{\pgfqpoint{2.257114in}{3.678753in}}%
\pgfpathlineto{\pgfqpoint{2.260945in}{3.679998in}}%
\pgfpathlineto{\pgfqpoint{2.262860in}{3.680428in}}%
\pgfpathlineto{\pgfqpoint{2.266690in}{3.683417in}}%
\pgfpathlineto{\pgfqpoint{2.274351in}{3.684301in}}%
\pgfpathlineto{\pgfqpoint{2.289672in}{3.692289in}}%
\pgfpathlineto{\pgfqpoint{2.295418in}{3.693797in}}%
\pgfpathlineto{\pgfqpoint{2.312654in}{3.696748in}}%
\pgfpathlineto{\pgfqpoint{2.314569in}{3.701649in}}%
\pgfpathlineto{\pgfqpoint{2.318400in}{3.702703in}}%
\pgfpathlineto{\pgfqpoint{2.322230in}{3.704223in}}%
\pgfpathlineto{\pgfqpoint{2.324145in}{3.704882in}}%
\pgfpathlineto{\pgfqpoint{2.326060in}{3.707359in}}%
\pgfpathlineto{\pgfqpoint{2.327975in}{3.707681in}}%
\pgfpathlineto{\pgfqpoint{2.331806in}{3.710548in}}%
\pgfpathlineto{\pgfqpoint{2.335636in}{3.711079in}}%
\pgfpathlineto{\pgfqpoint{2.337551in}{3.714180in}}%
\pgfpathlineto{\pgfqpoint{2.339467in}{3.714992in}}%
\pgfpathlineto{\pgfqpoint{2.343297in}{3.717681in}}%
\pgfpathlineto{\pgfqpoint{2.349042in}{3.720509in}}%
\pgfpathlineto{\pgfqpoint{2.350958in}{3.723488in}}%
\pgfpathlineto{\pgfqpoint{2.360533in}{3.725081in}}%
\pgfpathlineto{\pgfqpoint{2.362449in}{3.726173in}}%
\pgfpathlineto{\pgfqpoint{2.366279in}{3.732021in}}%
\pgfpathlineto{\pgfqpoint{2.372024in}{3.734952in}}%
\pgfpathlineto{\pgfqpoint{2.373940in}{3.738698in}}%
\pgfpathlineto{\pgfqpoint{2.377770in}{3.739560in}}%
\pgfpathlineto{\pgfqpoint{2.381600in}{3.740726in}}%
\pgfpathlineto{\pgfqpoint{2.387346in}{3.745899in}}%
\pgfpathlineto{\pgfqpoint{2.396922in}{3.749620in}}%
\pgfpathlineto{\pgfqpoint{2.400752in}{3.751584in}}%
\pgfpathlineto{\pgfqpoint{2.404582in}{3.752560in}}%
\pgfpathlineto{\pgfqpoint{2.406498in}{3.755427in}}%
\pgfpathlineto{\pgfqpoint{2.408413in}{3.755676in}}%
\pgfpathlineto{\pgfqpoint{2.414158in}{3.761073in}}%
\pgfpathlineto{\pgfqpoint{2.417989in}{3.765463in}}%
\pgfpathlineto{\pgfqpoint{2.423734in}{3.770825in}}%
\pgfpathlineto{\pgfqpoint{2.431395in}{3.772010in}}%
\pgfpathlineto{\pgfqpoint{2.439055in}{3.776765in}}%
\pgfpathlineto{\pgfqpoint{2.440971in}{3.779758in}}%
\pgfpathlineto{\pgfqpoint{2.444801in}{3.780180in}}%
\pgfpathlineto{\pgfqpoint{2.448631in}{3.783122in}}%
\pgfpathlineto{\pgfqpoint{2.454377in}{3.787020in}}%
\pgfpathlineto{\pgfqpoint{2.458207in}{3.787340in}}%
\pgfpathlineto{\pgfqpoint{2.460122in}{3.789385in}}%
\pgfpathlineto{\pgfqpoint{2.462037in}{3.789503in}}%
\pgfpathlineto{\pgfqpoint{2.465868in}{3.792056in}}%
\pgfpathlineto{\pgfqpoint{2.467783in}{3.792825in}}%
\pgfpathlineto{\pgfqpoint{2.469698in}{3.799421in}}%
\pgfpathlineto{\pgfqpoint{2.475444in}{3.801847in}}%
\pgfpathlineto{\pgfqpoint{2.479274in}{3.812374in}}%
\pgfpathlineto{\pgfqpoint{2.481189in}{3.813670in}}%
\pgfpathlineto{\pgfqpoint{2.483104in}{3.818065in}}%
\pgfpathlineto{\pgfqpoint{2.486935in}{3.818351in}}%
\pgfpathlineto{\pgfqpoint{2.494595in}{3.825122in}}%
\pgfpathlineto{\pgfqpoint{2.502256in}{3.829244in}}%
\pgfpathlineto{\pgfqpoint{2.506086in}{3.829861in}}%
\pgfpathlineto{\pgfqpoint{2.509917in}{3.833642in}}%
\pgfpathlineto{\pgfqpoint{2.523323in}{3.841880in}}%
\pgfpathlineto{\pgfqpoint{2.525238in}{3.841939in}}%
\pgfpathlineto{\pgfqpoint{2.529068in}{3.846446in}}%
\pgfpathlineto{\pgfqpoint{2.530984in}{3.847059in}}%
\pgfpathlineto{\pgfqpoint{2.536729in}{3.851881in}}%
\pgfpathlineto{\pgfqpoint{2.542475in}{3.853737in}}%
\pgfpathlineto{\pgfqpoint{2.544390in}{3.860501in}}%
\pgfpathlineto{\pgfqpoint{2.546305in}{3.862974in}}%
\pgfpathlineto{\pgfqpoint{2.548220in}{3.863009in}}%
\pgfpathlineto{\pgfqpoint{2.550135in}{3.864937in}}%
\pgfpathlineto{\pgfqpoint{2.553966in}{3.866094in}}%
\pgfpathlineto{\pgfqpoint{2.571202in}{3.876144in}}%
\pgfpathlineto{\pgfqpoint{2.573117in}{3.880924in}}%
\pgfpathlineto{\pgfqpoint{2.578863in}{3.881489in}}%
\pgfpathlineto{\pgfqpoint{2.582693in}{3.888290in}}%
\pgfpathlineto{\pgfqpoint{2.586524in}{3.889592in}}%
\pgfpathlineto{\pgfqpoint{2.588439in}{3.892385in}}%
\pgfpathlineto{\pgfqpoint{2.594184in}{3.893766in}}%
\pgfpathlineto{\pgfqpoint{2.598015in}{3.900077in}}%
\pgfpathlineto{\pgfqpoint{2.603760in}{3.905133in}}%
\pgfpathlineto{\pgfqpoint{2.605675in}{3.910424in}}%
\pgfpathlineto{\pgfqpoint{2.609506in}{3.912788in}}%
\pgfpathlineto{\pgfqpoint{2.611421in}{3.913657in}}%
\pgfpathlineto{\pgfqpoint{2.615251in}{3.918998in}}%
\pgfpathlineto{\pgfqpoint{2.619082in}{3.922112in}}%
\pgfpathlineto{\pgfqpoint{2.622912in}{3.929821in}}%
\pgfpathlineto{\pgfqpoint{2.626742in}{3.930833in}}%
\pgfpathlineto{\pgfqpoint{2.630573in}{3.935336in}}%
\pgfpathlineto{\pgfqpoint{2.640148in}{3.938387in}}%
\pgfpathlineto{\pgfqpoint{2.643979in}{3.940741in}}%
\pgfpathlineto{\pgfqpoint{2.649724in}{3.941993in}}%
\pgfpathlineto{\pgfqpoint{2.651639in}{3.951190in}}%
\pgfpathlineto{\pgfqpoint{2.653555in}{3.951198in}}%
\pgfpathlineto{\pgfqpoint{2.655470in}{3.955658in}}%
\pgfpathlineto{\pgfqpoint{2.657385in}{3.955712in}}%
\pgfpathlineto{\pgfqpoint{2.672706in}{3.967528in}}%
\pgfpathlineto{\pgfqpoint{2.674622in}{3.967567in}}%
\pgfpathlineto{\pgfqpoint{2.680367in}{3.970658in}}%
\pgfpathlineto{\pgfqpoint{2.686113in}{3.975018in}}%
\pgfpathlineto{\pgfqpoint{2.688028in}{3.977080in}}%
\pgfpathlineto{\pgfqpoint{2.689943in}{3.981082in}}%
\pgfpathlineto{\pgfqpoint{2.691858in}{3.981405in}}%
\pgfpathlineto{\pgfqpoint{2.693773in}{3.984189in}}%
\pgfpathlineto{\pgfqpoint{2.703349in}{3.987190in}}%
\pgfpathlineto{\pgfqpoint{2.714840in}{3.991879in}}%
\pgfpathlineto{\pgfqpoint{2.716755in}{3.992183in}}%
\pgfpathlineto{\pgfqpoint{2.720586in}{4.001887in}}%
\pgfpathlineto{\pgfqpoint{2.722501in}{4.002629in}}%
\pgfpathlineto{\pgfqpoint{2.724416in}{4.005812in}}%
\pgfpathlineto{\pgfqpoint{2.726331in}{4.005923in}}%
\pgfpathlineto{\pgfqpoint{2.728246in}{4.007445in}}%
\pgfpathlineto{\pgfqpoint{2.730161in}{4.018012in}}%
\pgfpathlineto{\pgfqpoint{2.733992in}{4.021347in}}%
\pgfpathlineto{\pgfqpoint{2.735907in}{4.026771in}}%
\pgfpathlineto{\pgfqpoint{2.741653in}{4.029444in}}%
\pgfpathlineto{\pgfqpoint{2.743568in}{4.029617in}}%
\pgfpathlineto{\pgfqpoint{2.749313in}{4.034498in}}%
\pgfpathlineto{\pgfqpoint{2.758889in}{4.038956in}}%
\pgfpathlineto{\pgfqpoint{2.764635in}{4.052671in}}%
\pgfpathlineto{\pgfqpoint{2.766550in}{4.058129in}}%
\pgfpathlineto{\pgfqpoint{2.774210in}{4.063380in}}%
\pgfpathlineto{\pgfqpoint{2.776126in}{4.069757in}}%
\pgfpathlineto{\pgfqpoint{2.778041in}{4.070191in}}%
\pgfpathlineto{\pgfqpoint{2.779956in}{4.074670in}}%
\pgfpathlineto{\pgfqpoint{2.781871in}{4.075576in}}%
\pgfpathlineto{\pgfqpoint{2.795277in}{4.090742in}}%
\pgfpathlineto{\pgfqpoint{2.802938in}{4.092678in}}%
\pgfpathlineto{\pgfqpoint{2.804853in}{4.094841in}}%
\pgfpathlineto{\pgfqpoint{2.806768in}{4.098917in}}%
\pgfpathlineto{\pgfqpoint{2.808684in}{4.099206in}}%
\pgfpathlineto{\pgfqpoint{2.812514in}{4.108076in}}%
\pgfpathlineto{\pgfqpoint{2.814429in}{4.108160in}}%
\pgfpathlineto{\pgfqpoint{2.816344in}{4.109733in}}%
\pgfpathlineto{\pgfqpoint{2.831666in}{4.111264in}}%
\pgfpathlineto{\pgfqpoint{2.837411in}{4.116213in}}%
\pgfpathlineto{\pgfqpoint{2.839326in}{4.116361in}}%
\pgfpathlineto{\pgfqpoint{2.843157in}{4.119430in}}%
\pgfpathlineto{\pgfqpoint{2.854648in}{4.124724in}}%
\pgfpathlineto{\pgfqpoint{2.856563in}{4.133341in}}%
\pgfpathlineto{\pgfqpoint{2.858478in}{4.134263in}}%
\pgfpathlineto{\pgfqpoint{2.860393in}{4.137230in}}%
\pgfpathlineto{\pgfqpoint{2.868054in}{4.140913in}}%
\pgfpathlineto{\pgfqpoint{2.871884in}{4.146837in}}%
\pgfpathlineto{\pgfqpoint{2.873799in}{4.152950in}}%
\pgfpathlineto{\pgfqpoint{2.875715in}{4.153653in}}%
\pgfpathlineto{\pgfqpoint{2.877630in}{4.158955in}}%
\pgfpathlineto{\pgfqpoint{2.881460in}{4.160808in}}%
\pgfpathlineto{\pgfqpoint{2.883375in}{4.161664in}}%
\pgfpathlineto{\pgfqpoint{2.894866in}{4.190977in}}%
\pgfpathlineto{\pgfqpoint{2.896781in}{4.191889in}}%
\pgfpathlineto{\pgfqpoint{2.898697in}{4.194075in}}%
\pgfpathlineto{\pgfqpoint{2.900612in}{4.194483in}}%
\pgfpathlineto{\pgfqpoint{2.904442in}{4.204575in}}%
\pgfpathlineto{\pgfqpoint{2.908272in}{4.206488in}}%
\pgfpathlineto{\pgfqpoint{2.910188in}{4.208322in}}%
\pgfpathlineto{\pgfqpoint{2.912103in}{4.208395in}}%
\pgfpathlineto{\pgfqpoint{2.917848in}{4.223586in}}%
\pgfpathlineto{\pgfqpoint{2.919763in}{4.223631in}}%
\pgfpathlineto{\pgfqpoint{2.923594in}{4.228109in}}%
\pgfpathlineto{\pgfqpoint{2.923594in}{4.228109in}}%
\pgfusepath{stroke}%
\end{pgfscope}%
\begin{pgfscope}%
\pgfpathrectangle{\pgfqpoint{0.694334in}{2.659974in}}{\pgfqpoint{3.830343in}{1.568135in}}%
\pgfusepath{clip}%
\pgfsetbuttcap%
\pgfsetroundjoin%
\pgfsetlinewidth{1.003750pt}%
\definecolor{currentstroke}{rgb}{0.564706,0.564706,1.000000}%
\pgfsetstrokecolor{currentstroke}%
\pgfsetdash{{1.000000pt}{1.650000pt}}{0.000000pt}%
\pgfpathmoveto{\pgfqpoint{0.694334in}{2.772990in}}%
\pgfpathlineto{\pgfqpoint{0.696249in}{2.798918in}}%
\pgfpathlineto{\pgfqpoint{0.698165in}{2.802501in}}%
\pgfpathlineto{\pgfqpoint{0.703910in}{2.840299in}}%
\pgfpathlineto{\pgfqpoint{0.711571in}{2.856815in}}%
\pgfpathlineto{\pgfqpoint{0.717316in}{2.889899in}}%
\pgfpathlineto{\pgfqpoint{0.719232in}{2.896210in}}%
\pgfpathlineto{\pgfqpoint{0.721147in}{2.896389in}}%
\pgfpathlineto{\pgfqpoint{0.723062in}{2.901295in}}%
\pgfpathlineto{\pgfqpoint{0.724977in}{2.912502in}}%
\pgfpathlineto{\pgfqpoint{0.726892in}{2.913341in}}%
\pgfpathlineto{\pgfqpoint{0.730723in}{2.923014in}}%
\pgfpathlineto{\pgfqpoint{0.732638in}{2.928910in}}%
\pgfpathlineto{\pgfqpoint{0.736468in}{2.931325in}}%
\pgfpathlineto{\pgfqpoint{0.738383in}{2.939856in}}%
\pgfpathlineto{\pgfqpoint{0.749874in}{2.958734in}}%
\pgfpathlineto{\pgfqpoint{0.751789in}{2.972914in}}%
\pgfpathlineto{\pgfqpoint{0.755620in}{2.975724in}}%
\pgfpathlineto{\pgfqpoint{0.763280in}{2.993979in}}%
\pgfpathlineto{\pgfqpoint{0.767111in}{2.997826in}}%
\pgfpathlineto{\pgfqpoint{0.772856in}{3.001451in}}%
\pgfpathlineto{\pgfqpoint{0.780517in}{3.022271in}}%
\pgfpathlineto{\pgfqpoint{0.784347in}{3.023499in}}%
\pgfpathlineto{\pgfqpoint{0.790093in}{3.035858in}}%
\pgfpathlineto{\pgfqpoint{0.793923in}{3.037099in}}%
\pgfpathlineto{\pgfqpoint{0.797754in}{3.054450in}}%
\pgfpathlineto{\pgfqpoint{0.799669in}{3.062474in}}%
\pgfpathlineto{\pgfqpoint{0.814990in}{3.072167in}}%
\pgfpathlineto{\pgfqpoint{0.816905in}{3.080623in}}%
\pgfpathlineto{\pgfqpoint{0.820736in}{3.084997in}}%
\pgfpathlineto{\pgfqpoint{0.824566in}{3.085557in}}%
\pgfpathlineto{\pgfqpoint{0.826481in}{3.089334in}}%
\pgfpathlineto{\pgfqpoint{0.830311in}{3.089552in}}%
\pgfpathlineto{\pgfqpoint{0.832227in}{3.091374in}}%
\pgfpathlineto{\pgfqpoint{0.834142in}{3.101040in}}%
\pgfpathlineto{\pgfqpoint{0.837972in}{3.101775in}}%
\pgfpathlineto{\pgfqpoint{0.841803in}{3.103253in}}%
\pgfpathlineto{\pgfqpoint{0.851378in}{3.109276in}}%
\pgfpathlineto{\pgfqpoint{0.853294in}{3.109804in}}%
\pgfpathlineto{\pgfqpoint{0.859039in}{3.115782in}}%
\pgfpathlineto{\pgfqpoint{0.864785in}{3.117601in}}%
\pgfpathlineto{\pgfqpoint{0.868615in}{3.118080in}}%
\pgfpathlineto{\pgfqpoint{0.870530in}{3.121860in}}%
\pgfpathlineto{\pgfqpoint{0.882021in}{3.128849in}}%
\pgfpathlineto{\pgfqpoint{0.889682in}{3.139965in}}%
\pgfpathlineto{\pgfqpoint{0.893512in}{3.141287in}}%
\pgfpathlineto{\pgfqpoint{0.897342in}{3.146374in}}%
\pgfpathlineto{\pgfqpoint{0.899258in}{3.147621in}}%
\pgfpathlineto{\pgfqpoint{0.901173in}{3.150980in}}%
\pgfpathlineto{\pgfqpoint{0.908834in}{3.154297in}}%
\pgfpathlineto{\pgfqpoint{0.912664in}{3.161069in}}%
\pgfpathlineto{\pgfqpoint{0.918409in}{3.161514in}}%
\pgfpathlineto{\pgfqpoint{0.922240in}{3.163930in}}%
\pgfpathlineto{\pgfqpoint{0.926070in}{3.165307in}}%
\pgfpathlineto{\pgfqpoint{0.929900in}{3.171184in}}%
\pgfpathlineto{\pgfqpoint{0.933731in}{3.173007in}}%
\pgfpathlineto{\pgfqpoint{0.945222in}{3.177451in}}%
\pgfpathlineto{\pgfqpoint{0.950967in}{3.178506in}}%
\pgfpathlineto{\pgfqpoint{0.952882in}{3.179336in}}%
\pgfpathlineto{\pgfqpoint{0.954798in}{3.181570in}}%
\pgfpathlineto{\pgfqpoint{0.960543in}{3.182358in}}%
\pgfpathlineto{\pgfqpoint{0.964373in}{3.187517in}}%
\pgfpathlineto{\pgfqpoint{0.968204in}{3.188183in}}%
\pgfpathlineto{\pgfqpoint{0.977780in}{3.193381in}}%
\pgfpathlineto{\pgfqpoint{0.981610in}{3.200235in}}%
\pgfpathlineto{\pgfqpoint{0.995016in}{3.205785in}}%
\pgfpathlineto{\pgfqpoint{0.998847in}{3.208333in}}%
\pgfpathlineto{\pgfqpoint{1.002677in}{3.210956in}}%
\pgfpathlineto{\pgfqpoint{1.029489in}{3.218726in}}%
\pgfpathlineto{\pgfqpoint{1.031404in}{3.220160in}}%
\pgfpathlineto{\pgfqpoint{1.035235in}{3.220635in}}%
\pgfpathlineto{\pgfqpoint{1.042895in}{3.222462in}}%
\pgfpathlineto{\pgfqpoint{1.044811in}{3.222614in}}%
\pgfpathlineto{\pgfqpoint{1.046726in}{3.224308in}}%
\pgfpathlineto{\pgfqpoint{1.050556in}{3.224574in}}%
\pgfpathlineto{\pgfqpoint{1.058217in}{3.231973in}}%
\pgfpathlineto{\pgfqpoint{1.069708in}{3.233923in}}%
\pgfpathlineto{\pgfqpoint{1.071623in}{3.237740in}}%
\pgfpathlineto{\pgfqpoint{1.127163in}{3.250885in}}%
\pgfpathlineto{\pgfqpoint{1.130993in}{3.251333in}}%
\pgfpathlineto{\pgfqpoint{1.134824in}{3.256176in}}%
\pgfpathlineto{\pgfqpoint{1.150145in}{3.260571in}}%
\pgfpathlineto{\pgfqpoint{1.153975in}{3.263980in}}%
\pgfpathlineto{\pgfqpoint{1.157806in}{3.265474in}}%
\pgfpathlineto{\pgfqpoint{1.159721in}{3.267069in}}%
\pgfpathlineto{\pgfqpoint{1.167382in}{3.268279in}}%
\pgfpathlineto{\pgfqpoint{1.176957in}{3.271025in}}%
\pgfpathlineto{\pgfqpoint{1.188449in}{3.275061in}}%
\pgfpathlineto{\pgfqpoint{1.192279in}{3.276291in}}%
\pgfpathlineto{\pgfqpoint{1.203770in}{3.277685in}}%
\pgfpathlineto{\pgfqpoint{1.219091in}{3.280004in}}%
\pgfpathlineto{\pgfqpoint{1.226752in}{3.281150in}}%
\pgfpathlineto{\pgfqpoint{1.230582in}{3.282724in}}%
\pgfpathlineto{\pgfqpoint{1.242073in}{3.284035in}}%
\pgfpathlineto{\pgfqpoint{1.245904in}{3.288385in}}%
\pgfpathlineto{\pgfqpoint{1.257395in}{3.290236in}}%
\pgfpathlineto{\pgfqpoint{1.268886in}{3.294331in}}%
\pgfpathlineto{\pgfqpoint{1.276546in}{3.295357in}}%
\pgfpathlineto{\pgfqpoint{1.284207in}{3.297602in}}%
\pgfpathlineto{\pgfqpoint{1.291868in}{3.298576in}}%
\pgfpathlineto{\pgfqpoint{1.293783in}{3.300159in}}%
\pgfpathlineto{\pgfqpoint{1.303359in}{3.301356in}}%
\pgfpathlineto{\pgfqpoint{1.311019in}{3.302386in}}%
\pgfpathlineto{\pgfqpoint{1.316765in}{3.305401in}}%
\pgfpathlineto{\pgfqpoint{1.322511in}{3.306458in}}%
\pgfpathlineto{\pgfqpoint{1.324426in}{3.307694in}}%
\pgfpathlineto{\pgfqpoint{1.326341in}{3.310625in}}%
\pgfpathlineto{\pgfqpoint{1.334002in}{3.312737in}}%
\pgfpathlineto{\pgfqpoint{1.335917in}{3.313613in}}%
\pgfpathlineto{\pgfqpoint{1.339747in}{3.316819in}}%
\pgfpathlineto{\pgfqpoint{1.351238in}{3.319464in}}%
\pgfpathlineto{\pgfqpoint{1.376135in}{3.327783in}}%
\pgfpathlineto{\pgfqpoint{1.381881in}{3.329965in}}%
\pgfpathlineto{\pgfqpoint{1.385711in}{3.330832in}}%
\pgfpathlineto{\pgfqpoint{1.404863in}{3.335457in}}%
\pgfpathlineto{\pgfqpoint{1.412524in}{3.338893in}}%
\pgfpathlineto{\pgfqpoint{1.435506in}{3.348153in}}%
\pgfpathlineto{\pgfqpoint{1.439336in}{3.348877in}}%
\pgfpathlineto{\pgfqpoint{1.446997in}{3.351976in}}%
\pgfpathlineto{\pgfqpoint{1.469979in}{3.362913in}}%
\pgfpathlineto{\pgfqpoint{1.473809in}{3.363770in}}%
\pgfpathlineto{\pgfqpoint{1.475724in}{3.366085in}}%
\pgfpathlineto{\pgfqpoint{1.492961in}{3.370411in}}%
\pgfpathlineto{\pgfqpoint{1.496791in}{3.374889in}}%
\pgfpathlineto{\pgfqpoint{1.498706in}{3.374982in}}%
\pgfpathlineto{\pgfqpoint{1.500621in}{3.377327in}}%
\pgfpathlineto{\pgfqpoint{1.504452in}{3.378325in}}%
\pgfpathlineto{\pgfqpoint{1.508282in}{3.382308in}}%
\pgfpathlineto{\pgfqpoint{1.514028in}{3.383727in}}%
\pgfpathlineto{\pgfqpoint{1.531264in}{3.387032in}}%
\pgfpathlineto{\pgfqpoint{1.537010in}{3.388312in}}%
\pgfpathlineto{\pgfqpoint{1.538925in}{3.389600in}}%
\pgfpathlineto{\pgfqpoint{1.540840in}{3.392362in}}%
\pgfpathlineto{\pgfqpoint{1.548501in}{3.394873in}}%
\pgfpathlineto{\pgfqpoint{1.552331in}{3.396327in}}%
\pgfpathlineto{\pgfqpoint{1.556161in}{3.396740in}}%
\pgfpathlineto{\pgfqpoint{1.559992in}{3.398491in}}%
\pgfpathlineto{\pgfqpoint{1.561907in}{3.398551in}}%
\pgfpathlineto{\pgfqpoint{1.563822in}{3.400311in}}%
\pgfpathlineto{\pgfqpoint{1.575313in}{3.402514in}}%
\pgfpathlineto{\pgfqpoint{1.582974in}{3.403389in}}%
\pgfpathlineto{\pgfqpoint{1.592550in}{3.404717in}}%
\pgfpathlineto{\pgfqpoint{1.596380in}{3.406140in}}%
\pgfpathlineto{\pgfqpoint{1.604041in}{3.411387in}}%
\pgfpathlineto{\pgfqpoint{1.613617in}{3.412552in}}%
\pgfpathlineto{\pgfqpoint{1.625108in}{3.413465in}}%
\pgfpathlineto{\pgfqpoint{1.627023in}{3.415896in}}%
\pgfpathlineto{\pgfqpoint{1.632768in}{3.416995in}}%
\pgfpathlineto{\pgfqpoint{1.642344in}{3.418194in}}%
\pgfpathlineto{\pgfqpoint{1.653835in}{3.425869in}}%
\pgfpathlineto{\pgfqpoint{1.657666in}{3.426914in}}%
\pgfpathlineto{\pgfqpoint{1.661496in}{3.428389in}}%
\pgfpathlineto{\pgfqpoint{1.667241in}{3.428984in}}%
\pgfpathlineto{\pgfqpoint{1.671072in}{3.430786in}}%
\pgfpathlineto{\pgfqpoint{1.672987in}{3.430871in}}%
\pgfpathlineto{\pgfqpoint{1.678732in}{3.434521in}}%
\pgfpathlineto{\pgfqpoint{1.692139in}{3.440169in}}%
\pgfpathlineto{\pgfqpoint{1.699799in}{3.441587in}}%
\pgfpathlineto{\pgfqpoint{1.701714in}{3.441819in}}%
\pgfpathlineto{\pgfqpoint{1.707460in}{3.450639in}}%
\pgfpathlineto{\pgfqpoint{1.720866in}{3.456466in}}%
\pgfpathlineto{\pgfqpoint{1.724697in}{3.458492in}}%
\pgfpathlineto{\pgfqpoint{1.726612in}{3.458746in}}%
\pgfpathlineto{\pgfqpoint{1.730442in}{3.461058in}}%
\pgfpathlineto{\pgfqpoint{1.740018in}{3.462899in}}%
\pgfpathlineto{\pgfqpoint{1.743848in}{3.464154in}}%
\pgfpathlineto{\pgfqpoint{1.747679in}{3.465793in}}%
\pgfpathlineto{\pgfqpoint{1.753424in}{3.467261in}}%
\pgfpathlineto{\pgfqpoint{1.759170in}{3.470827in}}%
\pgfpathlineto{\pgfqpoint{1.763000in}{3.473587in}}%
\pgfpathlineto{\pgfqpoint{1.770661in}{3.475278in}}%
\pgfpathlineto{\pgfqpoint{1.776406in}{3.475971in}}%
\pgfpathlineto{\pgfqpoint{1.778321in}{3.479279in}}%
\pgfpathlineto{\pgfqpoint{1.784067in}{3.480454in}}%
\pgfpathlineto{\pgfqpoint{1.797473in}{3.487043in}}%
\pgfpathlineto{\pgfqpoint{1.801303in}{3.488790in}}%
\pgfpathlineto{\pgfqpoint{1.810879in}{3.489890in}}%
\pgfpathlineto{\pgfqpoint{1.820455in}{3.495146in}}%
\pgfpathlineto{\pgfqpoint{1.831946in}{3.498429in}}%
\pgfpathlineto{\pgfqpoint{1.835776in}{3.500310in}}%
\pgfpathlineto{\pgfqpoint{1.862589in}{3.504658in}}%
\pgfpathlineto{\pgfqpoint{1.866419in}{3.506533in}}%
\pgfpathlineto{\pgfqpoint{1.870250in}{3.507424in}}%
\pgfpathlineto{\pgfqpoint{1.877910in}{3.508504in}}%
\pgfpathlineto{\pgfqpoint{1.881741in}{3.510291in}}%
\pgfpathlineto{\pgfqpoint{1.889401in}{3.511575in}}%
\pgfpathlineto{\pgfqpoint{1.893232in}{3.514284in}}%
\pgfpathlineto{\pgfqpoint{1.898977in}{3.515812in}}%
\pgfpathlineto{\pgfqpoint{1.902807in}{3.516451in}}%
\pgfpathlineto{\pgfqpoint{1.914298in}{3.521128in}}%
\pgfpathlineto{\pgfqpoint{1.929620in}{3.523474in}}%
\pgfpathlineto{\pgfqpoint{1.933450in}{3.524943in}}%
\pgfpathlineto{\pgfqpoint{1.943026in}{3.526622in}}%
\pgfpathlineto{\pgfqpoint{1.948772in}{3.529096in}}%
\pgfpathlineto{\pgfqpoint{1.950687in}{3.529147in}}%
\pgfpathlineto{\pgfqpoint{1.952602in}{3.530589in}}%
\pgfpathlineto{\pgfqpoint{1.960263in}{3.531695in}}%
\pgfpathlineto{\pgfqpoint{1.964093in}{3.532632in}}%
\pgfpathlineto{\pgfqpoint{1.975584in}{3.534487in}}%
\pgfpathlineto{\pgfqpoint{1.977499in}{3.538093in}}%
\pgfpathlineto{\pgfqpoint{1.985160in}{3.540459in}}%
\pgfpathlineto{\pgfqpoint{1.988990in}{3.541597in}}%
\pgfpathlineto{\pgfqpoint{1.992821in}{3.542426in}}%
\pgfpathlineto{\pgfqpoint{1.996651in}{3.543376in}}%
\pgfpathlineto{\pgfqpoint{2.011972in}{3.547321in}}%
\pgfpathlineto{\pgfqpoint{2.015803in}{3.548497in}}%
\pgfpathlineto{\pgfqpoint{2.023463in}{3.551296in}}%
\pgfpathlineto{\pgfqpoint{2.038785in}{3.554075in}}%
\pgfpathlineto{\pgfqpoint{2.042615in}{3.556172in}}%
\pgfpathlineto{\pgfqpoint{2.052191in}{3.557851in}}%
\pgfpathlineto{\pgfqpoint{2.056021in}{3.558738in}}%
\pgfpathlineto{\pgfqpoint{2.059852in}{3.559095in}}%
\pgfpathlineto{\pgfqpoint{2.063682in}{3.560484in}}%
\pgfpathlineto{\pgfqpoint{2.086664in}{3.566019in}}%
\pgfpathlineto{\pgfqpoint{2.090494in}{3.568483in}}%
\pgfpathlineto{\pgfqpoint{2.096240in}{3.569873in}}%
\pgfpathlineto{\pgfqpoint{2.100070in}{3.572234in}}%
\pgfpathlineto{\pgfqpoint{2.105816in}{3.574506in}}%
\pgfpathlineto{\pgfqpoint{2.132628in}{3.581136in}}%
\pgfpathlineto{\pgfqpoint{2.138374in}{3.584001in}}%
\pgfpathlineto{\pgfqpoint{2.146034in}{3.585030in}}%
\pgfpathlineto{\pgfqpoint{2.149865in}{3.588616in}}%
\pgfpathlineto{\pgfqpoint{2.151780in}{3.588622in}}%
\pgfpathlineto{\pgfqpoint{2.153695in}{3.592641in}}%
\pgfpathlineto{\pgfqpoint{2.167101in}{3.596202in}}%
\pgfpathlineto{\pgfqpoint{2.170931in}{3.597937in}}%
\pgfpathlineto{\pgfqpoint{2.176677in}{3.600099in}}%
\pgfpathlineto{\pgfqpoint{2.180507in}{3.602418in}}%
\pgfpathlineto{\pgfqpoint{2.188168in}{3.603717in}}%
\pgfpathlineto{\pgfqpoint{2.191998in}{3.605225in}}%
\pgfpathlineto{\pgfqpoint{2.203489in}{3.608433in}}%
\pgfpathlineto{\pgfqpoint{2.205405in}{3.608487in}}%
\pgfpathlineto{\pgfqpoint{2.209235in}{3.610932in}}%
\pgfpathlineto{\pgfqpoint{2.232217in}{3.615880in}}%
\pgfpathlineto{\pgfqpoint{2.236047in}{3.617084in}}%
\pgfpathlineto{\pgfqpoint{2.241793in}{3.619848in}}%
\pgfpathlineto{\pgfqpoint{2.260945in}{3.626642in}}%
\pgfpathlineto{\pgfqpoint{2.262860in}{3.626895in}}%
\pgfpathlineto{\pgfqpoint{2.266690in}{3.629267in}}%
\pgfpathlineto{\pgfqpoint{2.278181in}{3.631352in}}%
\pgfpathlineto{\pgfqpoint{2.280096in}{3.634518in}}%
\pgfpathlineto{\pgfqpoint{2.301163in}{3.640449in}}%
\pgfpathlineto{\pgfqpoint{2.304993in}{3.640688in}}%
\pgfpathlineto{\pgfqpoint{2.306909in}{3.642533in}}%
\pgfpathlineto{\pgfqpoint{2.308824in}{3.647846in}}%
\pgfpathlineto{\pgfqpoint{2.312654in}{3.650142in}}%
\pgfpathlineto{\pgfqpoint{2.320315in}{3.652691in}}%
\pgfpathlineto{\pgfqpoint{2.324145in}{3.657773in}}%
\pgfpathlineto{\pgfqpoint{2.331806in}{3.659237in}}%
\pgfpathlineto{\pgfqpoint{2.335636in}{3.660747in}}%
\pgfpathlineto{\pgfqpoint{2.343297in}{3.664558in}}%
\pgfpathlineto{\pgfqpoint{2.345212in}{3.666854in}}%
\pgfpathlineto{\pgfqpoint{2.350958in}{3.668879in}}%
\pgfpathlineto{\pgfqpoint{2.352873in}{3.675036in}}%
\pgfpathlineto{\pgfqpoint{2.364364in}{3.679166in}}%
\pgfpathlineto{\pgfqpoint{2.368194in}{3.679963in}}%
\pgfpathlineto{\pgfqpoint{2.377770in}{3.682897in}}%
\pgfpathlineto{\pgfqpoint{2.379685in}{3.685743in}}%
\pgfpathlineto{\pgfqpoint{2.421819in}{3.696239in}}%
\pgfpathlineto{\pgfqpoint{2.425649in}{3.697058in}}%
\pgfpathlineto{\pgfqpoint{2.429480in}{3.698414in}}%
\pgfpathlineto{\pgfqpoint{2.435225in}{3.698998in}}%
\pgfpathlineto{\pgfqpoint{2.444801in}{3.705781in}}%
\pgfpathlineto{\pgfqpoint{2.450546in}{3.706451in}}%
\pgfpathlineto{\pgfqpoint{2.458207in}{3.710790in}}%
\pgfpathlineto{\pgfqpoint{2.465868in}{3.716959in}}%
\pgfpathlineto{\pgfqpoint{2.471613in}{3.720427in}}%
\pgfpathlineto{\pgfqpoint{2.475444in}{3.720827in}}%
\pgfpathlineto{\pgfqpoint{2.477359in}{3.724246in}}%
\pgfpathlineto{\pgfqpoint{2.485020in}{3.726406in}}%
\pgfpathlineto{\pgfqpoint{2.486935in}{3.727784in}}%
\pgfpathlineto{\pgfqpoint{2.490765in}{3.738332in}}%
\pgfpathlineto{\pgfqpoint{2.492680in}{3.742383in}}%
\pgfpathlineto{\pgfqpoint{2.496511in}{3.742661in}}%
\pgfpathlineto{\pgfqpoint{2.498426in}{3.744217in}}%
\pgfpathlineto{\pgfqpoint{2.502256in}{3.744347in}}%
\pgfpathlineto{\pgfqpoint{2.504171in}{3.746326in}}%
\pgfpathlineto{\pgfqpoint{2.506086in}{3.752778in}}%
\pgfpathlineto{\pgfqpoint{2.513747in}{3.754112in}}%
\pgfpathlineto{\pgfqpoint{2.515662in}{3.756364in}}%
\pgfpathlineto{\pgfqpoint{2.517577in}{3.756536in}}%
\pgfpathlineto{\pgfqpoint{2.521408in}{3.759475in}}%
\pgfpathlineto{\pgfqpoint{2.523323in}{3.761374in}}%
\pgfpathlineto{\pgfqpoint{2.525238in}{3.765203in}}%
\pgfpathlineto{\pgfqpoint{2.530984in}{3.768678in}}%
\pgfpathlineto{\pgfqpoint{2.532899in}{3.770650in}}%
\pgfpathlineto{\pgfqpoint{2.538644in}{3.771860in}}%
\pgfpathlineto{\pgfqpoint{2.540560in}{3.774727in}}%
\pgfpathlineto{\pgfqpoint{2.542475in}{3.775015in}}%
\pgfpathlineto{\pgfqpoint{2.548220in}{3.780446in}}%
\pgfpathlineto{\pgfqpoint{2.552051in}{3.792874in}}%
\pgfpathlineto{\pgfqpoint{2.553966in}{3.793856in}}%
\pgfpathlineto{\pgfqpoint{2.555881in}{3.798852in}}%
\pgfpathlineto{\pgfqpoint{2.563542in}{3.800250in}}%
\pgfpathlineto{\pgfqpoint{2.565457in}{3.802086in}}%
\pgfpathlineto{\pgfqpoint{2.567372in}{3.802207in}}%
\pgfpathlineto{\pgfqpoint{2.573117in}{3.805789in}}%
\pgfpathlineto{\pgfqpoint{2.580778in}{3.808020in}}%
\pgfpathlineto{\pgfqpoint{2.584608in}{3.812423in}}%
\pgfpathlineto{\pgfqpoint{2.588439in}{3.812985in}}%
\pgfpathlineto{\pgfqpoint{2.592269in}{3.818044in}}%
\pgfpathlineto{\pgfqpoint{2.594184in}{3.818332in}}%
\pgfpathlineto{\pgfqpoint{2.598015in}{3.826780in}}%
\pgfpathlineto{\pgfqpoint{2.599930in}{3.829644in}}%
\pgfpathlineto{\pgfqpoint{2.603760in}{3.830626in}}%
\pgfpathlineto{\pgfqpoint{2.605675in}{3.832815in}}%
\pgfpathlineto{\pgfqpoint{2.611421in}{3.834031in}}%
\pgfpathlineto{\pgfqpoint{2.619082in}{3.835870in}}%
\pgfpathlineto{\pgfqpoint{2.622912in}{3.840189in}}%
\pgfpathlineto{\pgfqpoint{2.626742in}{3.842477in}}%
\pgfpathlineto{\pgfqpoint{2.628657in}{3.844648in}}%
\pgfpathlineto{\pgfqpoint{2.636318in}{3.845573in}}%
\pgfpathlineto{\pgfqpoint{2.651639in}{3.855098in}}%
\pgfpathlineto{\pgfqpoint{2.653555in}{3.857831in}}%
\pgfpathlineto{\pgfqpoint{2.657385in}{3.858245in}}%
\pgfpathlineto{\pgfqpoint{2.659300in}{3.862937in}}%
\pgfpathlineto{\pgfqpoint{2.663130in}{3.863568in}}%
\pgfpathlineto{\pgfqpoint{2.665046in}{3.866137in}}%
\pgfpathlineto{\pgfqpoint{2.666961in}{3.866686in}}%
\pgfpathlineto{\pgfqpoint{2.668876in}{3.868945in}}%
\pgfpathlineto{\pgfqpoint{2.670791in}{3.869067in}}%
\pgfpathlineto{\pgfqpoint{2.672706in}{3.873544in}}%
\pgfpathlineto{\pgfqpoint{2.674622in}{3.873574in}}%
\pgfpathlineto{\pgfqpoint{2.676537in}{3.878355in}}%
\pgfpathlineto{\pgfqpoint{2.680367in}{3.880011in}}%
\pgfpathlineto{\pgfqpoint{2.684197in}{3.883625in}}%
\pgfpathlineto{\pgfqpoint{2.689943in}{3.885727in}}%
\pgfpathlineto{\pgfqpoint{2.693773in}{3.888785in}}%
\pgfpathlineto{\pgfqpoint{2.695688in}{3.888939in}}%
\pgfpathlineto{\pgfqpoint{2.697604in}{3.890865in}}%
\pgfpathlineto{\pgfqpoint{2.703349in}{3.902011in}}%
\pgfpathlineto{\pgfqpoint{2.705264in}{3.902291in}}%
\pgfpathlineto{\pgfqpoint{2.709095in}{3.905754in}}%
\pgfpathlineto{\pgfqpoint{2.712925in}{3.906965in}}%
\pgfpathlineto{\pgfqpoint{2.714840in}{3.912389in}}%
\pgfpathlineto{\pgfqpoint{2.716755in}{3.914460in}}%
\pgfpathlineto{\pgfqpoint{2.718670in}{3.914637in}}%
\pgfpathlineto{\pgfqpoint{2.728246in}{3.923206in}}%
\pgfpathlineto{\pgfqpoint{2.732077in}{3.927506in}}%
\pgfpathlineto{\pgfqpoint{2.743568in}{3.941603in}}%
\pgfpathlineto{\pgfqpoint{2.745483in}{3.945065in}}%
\pgfpathlineto{\pgfqpoint{2.749313in}{3.946528in}}%
\pgfpathlineto{\pgfqpoint{2.751228in}{3.948110in}}%
\pgfpathlineto{\pgfqpoint{2.755059in}{3.949236in}}%
\pgfpathlineto{\pgfqpoint{2.756974in}{3.951036in}}%
\pgfpathlineto{\pgfqpoint{2.758889in}{3.954565in}}%
\pgfpathlineto{\pgfqpoint{2.766550in}{3.955958in}}%
\pgfpathlineto{\pgfqpoint{2.770380in}{3.961766in}}%
\pgfpathlineto{\pgfqpoint{2.774210in}{3.965392in}}%
\pgfpathlineto{\pgfqpoint{2.793362in}{3.969848in}}%
\pgfpathlineto{\pgfqpoint{2.797192in}{3.970773in}}%
\pgfpathlineto{\pgfqpoint{2.801023in}{3.972265in}}%
\pgfpathlineto{\pgfqpoint{2.810599in}{3.975077in}}%
\pgfpathlineto{\pgfqpoint{2.812514in}{3.976883in}}%
\pgfpathlineto{\pgfqpoint{2.814429in}{3.980821in}}%
\pgfpathlineto{\pgfqpoint{2.818259in}{3.981684in}}%
\pgfpathlineto{\pgfqpoint{2.822090in}{3.984646in}}%
\pgfpathlineto{\pgfqpoint{2.831666in}{3.999543in}}%
\pgfpathlineto{\pgfqpoint{2.833581in}{4.000137in}}%
\pgfpathlineto{\pgfqpoint{2.835496in}{4.015531in}}%
\pgfpathlineto{\pgfqpoint{2.837411in}{4.018281in}}%
\pgfpathlineto{\pgfqpoint{2.839326in}{4.018559in}}%
\pgfpathlineto{\pgfqpoint{2.850817in}{4.031445in}}%
\pgfpathlineto{\pgfqpoint{2.856563in}{4.038644in}}%
\pgfpathlineto{\pgfqpoint{2.860393in}{4.042478in}}%
\pgfpathlineto{\pgfqpoint{2.862308in}{4.043581in}}%
\pgfpathlineto{\pgfqpoint{2.864223in}{4.048987in}}%
\pgfpathlineto{\pgfqpoint{2.873799in}{4.057867in}}%
\pgfpathlineto{\pgfqpoint{2.879545in}{4.067711in}}%
\pgfpathlineto{\pgfqpoint{2.885290in}{4.068620in}}%
\pgfpathlineto{\pgfqpoint{2.892951in}{4.081723in}}%
\pgfpathlineto{\pgfqpoint{2.894866in}{4.082887in}}%
\pgfpathlineto{\pgfqpoint{2.906357in}{4.102042in}}%
\pgfpathlineto{\pgfqpoint{2.908272in}{4.102530in}}%
\pgfpathlineto{\pgfqpoint{2.910188in}{4.116616in}}%
\pgfpathlineto{\pgfqpoint{2.912103in}{4.116797in}}%
\pgfpathlineto{\pgfqpoint{2.914018in}{4.118193in}}%
\pgfpathlineto{\pgfqpoint{2.919763in}{4.127362in}}%
\pgfpathlineto{\pgfqpoint{2.925509in}{4.132197in}}%
\pgfpathlineto{\pgfqpoint{2.927424in}{4.134877in}}%
\pgfpathlineto{\pgfqpoint{2.933170in}{4.135617in}}%
\pgfpathlineto{\pgfqpoint{2.935085in}{4.136518in}}%
\pgfpathlineto{\pgfqpoint{2.938915in}{4.146397in}}%
\pgfpathlineto{\pgfqpoint{2.940830in}{4.157064in}}%
\pgfpathlineto{\pgfqpoint{2.944661in}{4.160295in}}%
\pgfpathlineto{\pgfqpoint{2.946576in}{4.160800in}}%
\pgfpathlineto{\pgfqpoint{2.948491in}{4.176709in}}%
\pgfpathlineto{\pgfqpoint{2.952321in}{4.180958in}}%
\pgfpathlineto{\pgfqpoint{2.956152in}{4.186275in}}%
\pgfpathlineto{\pgfqpoint{2.959982in}{4.202626in}}%
\pgfpathlineto{\pgfqpoint{2.961897in}{4.205401in}}%
\pgfpathlineto{\pgfqpoint{2.963812in}{4.228109in}}%
\pgfpathlineto{\pgfqpoint{2.963812in}{4.228109in}}%
\pgfusepath{stroke}%
\end{pgfscope}%
\begin{pgfscope}%
\pgfpathrectangle{\pgfqpoint{0.694334in}{2.659974in}}{\pgfqpoint{3.830343in}{1.568135in}}%
\pgfusepath{clip}%
\pgfsetbuttcap%
\pgfsetroundjoin%
\pgfsetlinewidth{1.003750pt}%
\definecolor{currentstroke}{rgb}{0.564706,0.564706,1.000000}%
\pgfsetstrokecolor{currentstroke}%
\pgfsetdash{{3.700000pt}{1.600000pt}}{0.000000pt}%
\pgfpathmoveto{\pgfqpoint{0.694334in}{2.831436in}}%
\pgfpathlineto{\pgfqpoint{0.696249in}{2.859509in}}%
\pgfpathlineto{\pgfqpoint{0.698165in}{2.861305in}}%
\pgfpathlineto{\pgfqpoint{0.700080in}{2.874843in}}%
\pgfpathlineto{\pgfqpoint{0.707741in}{2.892329in}}%
\pgfpathlineto{\pgfqpoint{0.709656in}{2.899745in}}%
\pgfpathlineto{\pgfqpoint{0.711571in}{2.913674in}}%
\pgfpathlineto{\pgfqpoint{0.715401in}{2.914576in}}%
\pgfpathlineto{\pgfqpoint{0.717316in}{2.917672in}}%
\pgfpathlineto{\pgfqpoint{0.719232in}{2.917861in}}%
\pgfpathlineto{\pgfqpoint{0.724977in}{2.924634in}}%
\pgfpathlineto{\pgfqpoint{0.728807in}{2.935237in}}%
\pgfpathlineto{\pgfqpoint{0.730723in}{2.937889in}}%
\pgfpathlineto{\pgfqpoint{0.734553in}{2.949153in}}%
\pgfpathlineto{\pgfqpoint{0.738383in}{2.951730in}}%
\pgfpathlineto{\pgfqpoint{0.742214in}{2.965112in}}%
\pgfpathlineto{\pgfqpoint{0.744129in}{2.965237in}}%
\pgfpathlineto{\pgfqpoint{0.746044in}{2.966735in}}%
\pgfpathlineto{\pgfqpoint{0.749874in}{2.976321in}}%
\pgfpathlineto{\pgfqpoint{0.751789in}{2.978587in}}%
\pgfpathlineto{\pgfqpoint{0.757535in}{2.994151in}}%
\pgfpathlineto{\pgfqpoint{0.763280in}{2.997510in}}%
\pgfpathlineto{\pgfqpoint{0.765196in}{3.001175in}}%
\pgfpathlineto{\pgfqpoint{0.767111in}{3.008448in}}%
\pgfpathlineto{\pgfqpoint{0.769026in}{3.008775in}}%
\pgfpathlineto{\pgfqpoint{0.770941in}{3.011690in}}%
\pgfpathlineto{\pgfqpoint{0.772856in}{3.011961in}}%
\pgfpathlineto{\pgfqpoint{0.776687in}{3.019809in}}%
\pgfpathlineto{\pgfqpoint{0.778602in}{3.020663in}}%
\pgfpathlineto{\pgfqpoint{0.782432in}{3.032025in}}%
\pgfpathlineto{\pgfqpoint{0.784347in}{3.032492in}}%
\pgfpathlineto{\pgfqpoint{0.788178in}{3.038342in}}%
\pgfpathlineto{\pgfqpoint{0.790093in}{3.043665in}}%
\pgfpathlineto{\pgfqpoint{0.797754in}{3.047658in}}%
\pgfpathlineto{\pgfqpoint{0.801584in}{3.056153in}}%
\pgfpathlineto{\pgfqpoint{0.803499in}{3.061007in}}%
\pgfpathlineto{\pgfqpoint{0.809245in}{3.065481in}}%
\pgfpathlineto{\pgfqpoint{0.811160in}{3.065837in}}%
\pgfpathlineto{\pgfqpoint{0.813075in}{3.068715in}}%
\pgfpathlineto{\pgfqpoint{0.816905in}{3.069344in}}%
\pgfpathlineto{\pgfqpoint{0.818820in}{3.070727in}}%
\pgfpathlineto{\pgfqpoint{0.820736in}{3.075345in}}%
\pgfpathlineto{\pgfqpoint{0.822651in}{3.075913in}}%
\pgfpathlineto{\pgfqpoint{0.826481in}{3.082104in}}%
\pgfpathlineto{\pgfqpoint{0.828396in}{3.082288in}}%
\pgfpathlineto{\pgfqpoint{0.832227in}{3.090002in}}%
\pgfpathlineto{\pgfqpoint{0.836057in}{3.090833in}}%
\pgfpathlineto{\pgfqpoint{0.839887in}{3.097368in}}%
\pgfpathlineto{\pgfqpoint{0.843718in}{3.098306in}}%
\pgfpathlineto{\pgfqpoint{0.849463in}{3.099454in}}%
\pgfpathlineto{\pgfqpoint{0.851378in}{3.101651in}}%
\pgfpathlineto{\pgfqpoint{0.853294in}{3.101985in}}%
\pgfpathlineto{\pgfqpoint{0.855209in}{3.107307in}}%
\pgfpathlineto{\pgfqpoint{0.860954in}{3.108278in}}%
\pgfpathlineto{\pgfqpoint{0.862869in}{3.109608in}}%
\pgfpathlineto{\pgfqpoint{0.864785in}{3.112646in}}%
\pgfpathlineto{\pgfqpoint{0.883936in}{3.120891in}}%
\pgfpathlineto{\pgfqpoint{0.887767in}{3.121060in}}%
\pgfpathlineto{\pgfqpoint{0.889682in}{3.123904in}}%
\pgfpathlineto{\pgfqpoint{0.897342in}{3.125378in}}%
\pgfpathlineto{\pgfqpoint{0.903088in}{3.126472in}}%
\pgfpathlineto{\pgfqpoint{0.908834in}{3.127741in}}%
\pgfpathlineto{\pgfqpoint{0.910749in}{3.128972in}}%
\pgfpathlineto{\pgfqpoint{0.916494in}{3.137343in}}%
\pgfpathlineto{\pgfqpoint{0.926070in}{3.139238in}}%
\pgfpathlineto{\pgfqpoint{0.929900in}{3.142996in}}%
\pgfpathlineto{\pgfqpoint{0.933731in}{3.143335in}}%
\pgfpathlineto{\pgfqpoint{0.939476in}{3.150064in}}%
\pgfpathlineto{\pgfqpoint{0.943307in}{3.151146in}}%
\pgfpathlineto{\pgfqpoint{0.945222in}{3.151172in}}%
\pgfpathlineto{\pgfqpoint{0.947137in}{3.154267in}}%
\pgfpathlineto{\pgfqpoint{0.949052in}{3.154399in}}%
\pgfpathlineto{\pgfqpoint{0.950967in}{3.157010in}}%
\pgfpathlineto{\pgfqpoint{0.962458in}{3.161221in}}%
\pgfpathlineto{\pgfqpoint{0.964373in}{3.163251in}}%
\pgfpathlineto{\pgfqpoint{0.968204in}{3.163841in}}%
\pgfpathlineto{\pgfqpoint{0.972034in}{3.168806in}}%
\pgfpathlineto{\pgfqpoint{0.973949in}{3.169122in}}%
\pgfpathlineto{\pgfqpoint{0.975864in}{3.170791in}}%
\pgfpathlineto{\pgfqpoint{0.979695in}{3.177995in}}%
\pgfpathlineto{\pgfqpoint{0.983525in}{3.178218in}}%
\pgfpathlineto{\pgfqpoint{0.985440in}{3.180908in}}%
\pgfpathlineto{\pgfqpoint{0.995016in}{3.181851in}}%
\pgfpathlineto{\pgfqpoint{0.996931in}{3.182331in}}%
\pgfpathlineto{\pgfqpoint{1.000762in}{3.184567in}}%
\pgfpathlineto{\pgfqpoint{1.010338in}{3.190941in}}%
\pgfpathlineto{\pgfqpoint{1.014168in}{3.201776in}}%
\pgfpathlineto{\pgfqpoint{1.017998in}{3.203237in}}%
\pgfpathlineto{\pgfqpoint{1.019913in}{3.206326in}}%
\pgfpathlineto{\pgfqpoint{1.021829in}{3.212272in}}%
\pgfpathlineto{\pgfqpoint{1.025659in}{3.214502in}}%
\pgfpathlineto{\pgfqpoint{1.042895in}{3.222104in}}%
\pgfpathlineto{\pgfqpoint{1.044811in}{3.227817in}}%
\pgfpathlineto{\pgfqpoint{1.048641in}{3.228514in}}%
\pgfpathlineto{\pgfqpoint{1.050556in}{3.231133in}}%
\pgfpathlineto{\pgfqpoint{1.054387in}{3.232023in}}%
\pgfpathlineto{\pgfqpoint{1.058217in}{3.234793in}}%
\pgfpathlineto{\pgfqpoint{1.063962in}{3.237080in}}%
\pgfpathlineto{\pgfqpoint{1.069708in}{3.242058in}}%
\pgfpathlineto{\pgfqpoint{1.073538in}{3.242518in}}%
\pgfpathlineto{\pgfqpoint{1.077369in}{3.246370in}}%
\pgfpathlineto{\pgfqpoint{1.081199in}{3.247416in}}%
\pgfpathlineto{\pgfqpoint{1.086944in}{3.252673in}}%
\pgfpathlineto{\pgfqpoint{1.090775in}{3.252819in}}%
\pgfpathlineto{\pgfqpoint{1.094605in}{3.254392in}}%
\pgfpathlineto{\pgfqpoint{1.098435in}{3.260355in}}%
\pgfpathlineto{\pgfqpoint{1.100351in}{3.262019in}}%
\pgfpathlineto{\pgfqpoint{1.108011in}{3.263352in}}%
\pgfpathlineto{\pgfqpoint{1.109926in}{3.265402in}}%
\pgfpathlineto{\pgfqpoint{1.115672in}{3.266943in}}%
\pgfpathlineto{\pgfqpoint{1.130993in}{3.269673in}}%
\pgfpathlineto{\pgfqpoint{1.138654in}{3.276542in}}%
\pgfpathlineto{\pgfqpoint{1.146315in}{3.278346in}}%
\pgfpathlineto{\pgfqpoint{1.148230in}{3.280450in}}%
\pgfpathlineto{\pgfqpoint{1.155891in}{3.281859in}}%
\pgfpathlineto{\pgfqpoint{1.159721in}{3.285114in}}%
\pgfpathlineto{\pgfqpoint{1.175042in}{3.290870in}}%
\pgfpathlineto{\pgfqpoint{1.176957in}{3.291318in}}%
\pgfpathlineto{\pgfqpoint{1.180788in}{3.294640in}}%
\pgfpathlineto{\pgfqpoint{1.182703in}{3.295802in}}%
\pgfpathlineto{\pgfqpoint{1.184618in}{3.299792in}}%
\pgfpathlineto{\pgfqpoint{1.186533in}{3.301067in}}%
\pgfpathlineto{\pgfqpoint{1.190364in}{3.308989in}}%
\pgfpathlineto{\pgfqpoint{1.194194in}{3.316991in}}%
\pgfpathlineto{\pgfqpoint{1.196109in}{3.317909in}}%
\pgfpathlineto{\pgfqpoint{1.198024in}{3.321911in}}%
\pgfpathlineto{\pgfqpoint{1.199940in}{3.323049in}}%
\pgfpathlineto{\pgfqpoint{1.201855in}{3.326627in}}%
\pgfpathlineto{\pgfqpoint{1.203770in}{3.326716in}}%
\pgfpathlineto{\pgfqpoint{1.205685in}{3.331410in}}%
\pgfpathlineto{\pgfqpoint{1.207600in}{3.332592in}}%
\pgfpathlineto{\pgfqpoint{1.211431in}{3.344263in}}%
\pgfpathlineto{\pgfqpoint{1.213346in}{3.347593in}}%
\pgfpathlineto{\pgfqpoint{1.219091in}{3.363722in}}%
\pgfpathlineto{\pgfqpoint{1.221006in}{3.369465in}}%
\pgfpathlineto{\pgfqpoint{1.224837in}{3.389823in}}%
\pgfpathlineto{\pgfqpoint{1.226752in}{3.390556in}}%
\pgfpathlineto{\pgfqpoint{1.230582in}{3.415456in}}%
\pgfpathlineto{\pgfqpoint{1.234413in}{3.419003in}}%
\pgfpathlineto{\pgfqpoint{1.236328in}{3.441413in}}%
\pgfpathlineto{\pgfqpoint{1.238243in}{3.448185in}}%
\pgfpathlineto{\pgfqpoint{1.240158in}{3.462492in}}%
\pgfpathlineto{\pgfqpoint{1.243988in}{3.464194in}}%
\pgfpathlineto{\pgfqpoint{1.272716in}{3.468549in}}%
\pgfpathlineto{\pgfqpoint{1.487215in}{3.481193in}}%
\pgfpathlineto{\pgfqpoint{1.496791in}{3.482368in}}%
\pgfpathlineto{\pgfqpoint{1.519773in}{3.483994in}}%
\pgfpathlineto{\pgfqpoint{1.535095in}{3.485076in}}%
\pgfpathlineto{\pgfqpoint{1.577228in}{3.488135in}}%
\pgfpathlineto{\pgfqpoint{1.611701in}{3.490201in}}%
\pgfpathlineto{\pgfqpoint{1.634683in}{3.491414in}}%
\pgfpathlineto{\pgfqpoint{1.653835in}{3.492174in}}%
\pgfpathlineto{\pgfqpoint{1.690223in}{3.495286in}}%
\pgfpathlineto{\pgfqpoint{1.697884in}{3.495988in}}%
\pgfpathlineto{\pgfqpoint{1.717036in}{3.497243in}}%
\pgfpathlineto{\pgfqpoint{1.757254in}{3.501197in}}%
\pgfpathlineto{\pgfqpoint{1.795558in}{3.502857in}}%
\pgfpathlineto{\pgfqpoint{1.885571in}{3.509690in}}%
\pgfpathlineto{\pgfqpoint{1.964093in}{3.516599in}}%
\pgfpathlineto{\pgfqpoint{1.967923in}{3.517346in}}%
\pgfpathlineto{\pgfqpoint{2.025378in}{3.522230in}}%
\pgfpathlineto{\pgfqpoint{2.044530in}{3.523555in}}%
\pgfpathlineto{\pgfqpoint{2.061767in}{3.525800in}}%
\pgfpathlineto{\pgfqpoint{2.073258in}{3.526838in}}%
\pgfpathlineto{\pgfqpoint{2.105816in}{3.531125in}}%
\pgfpathlineto{\pgfqpoint{2.109646in}{3.532149in}}%
\pgfpathlineto{\pgfqpoint{2.115391in}{3.533202in}}%
\pgfpathlineto{\pgfqpoint{2.140289in}{3.535695in}}%
\pgfpathlineto{\pgfqpoint{2.155610in}{3.538169in}}%
\pgfpathlineto{\pgfqpoint{2.205405in}{3.542772in}}%
\pgfpathlineto{\pgfqpoint{2.213065in}{3.543958in}}%
\pgfpathlineto{\pgfqpoint{2.226471in}{3.546881in}}%
\pgfpathlineto{\pgfqpoint{2.251369in}{3.549997in}}%
\pgfpathlineto{\pgfqpoint{2.255199in}{3.551176in}}%
\pgfpathlineto{\pgfqpoint{2.268605in}{3.553427in}}%
\pgfpathlineto{\pgfqpoint{2.320315in}{3.559321in}}%
\pgfpathlineto{\pgfqpoint{2.327975in}{3.560261in}}%
\pgfpathlineto{\pgfqpoint{2.341382in}{3.563617in}}%
\pgfpathlineto{\pgfqpoint{2.350958in}{3.564862in}}%
\pgfpathlineto{\pgfqpoint{2.356703in}{3.566554in}}%
\pgfpathlineto{\pgfqpoint{2.366279in}{3.567417in}}%
\pgfpathlineto{\pgfqpoint{2.368194in}{3.569133in}}%
\pgfpathlineto{\pgfqpoint{2.372024in}{3.570143in}}%
\pgfpathlineto{\pgfqpoint{2.444801in}{3.580075in}}%
\pgfpathlineto{\pgfqpoint{2.448631in}{3.584382in}}%
\pgfpathlineto{\pgfqpoint{2.465868in}{3.586690in}}%
\pgfpathlineto{\pgfqpoint{2.467783in}{3.590048in}}%
\pgfpathlineto{\pgfqpoint{2.471613in}{3.590663in}}%
\pgfpathlineto{\pgfqpoint{2.475444in}{3.592747in}}%
\pgfpathlineto{\pgfqpoint{2.500341in}{3.598321in}}%
\pgfpathlineto{\pgfqpoint{2.513747in}{3.601271in}}%
\pgfpathlineto{\pgfqpoint{2.527153in}{3.602830in}}%
\pgfpathlineto{\pgfqpoint{2.530984in}{3.604804in}}%
\pgfpathlineto{\pgfqpoint{2.552051in}{3.609300in}}%
\pgfpathlineto{\pgfqpoint{2.557796in}{3.611476in}}%
\pgfpathlineto{\pgfqpoint{2.576948in}{3.615195in}}%
\pgfpathlineto{\pgfqpoint{2.578863in}{3.619458in}}%
\pgfpathlineto{\pgfqpoint{2.582693in}{3.619974in}}%
\pgfpathlineto{\pgfqpoint{2.592269in}{3.623406in}}%
\pgfpathlineto{\pgfqpoint{2.594184in}{3.623502in}}%
\pgfpathlineto{\pgfqpoint{2.598015in}{3.626432in}}%
\pgfpathlineto{\pgfqpoint{2.601845in}{3.627336in}}%
\pgfpathlineto{\pgfqpoint{2.609506in}{3.632169in}}%
\pgfpathlineto{\pgfqpoint{2.624827in}{3.634861in}}%
\pgfpathlineto{\pgfqpoint{2.640148in}{3.642475in}}%
\pgfpathlineto{\pgfqpoint{2.643979in}{3.644624in}}%
\pgfpathlineto{\pgfqpoint{2.645894in}{3.644738in}}%
\pgfpathlineto{\pgfqpoint{2.649724in}{3.646660in}}%
\pgfpathlineto{\pgfqpoint{2.653555in}{3.648016in}}%
\pgfpathlineto{\pgfqpoint{2.657385in}{3.649221in}}%
\pgfpathlineto{\pgfqpoint{2.659300in}{3.653520in}}%
\pgfpathlineto{\pgfqpoint{2.665046in}{3.653980in}}%
\pgfpathlineto{\pgfqpoint{2.666961in}{3.656257in}}%
\pgfpathlineto{\pgfqpoint{2.672706in}{3.657388in}}%
\pgfpathlineto{\pgfqpoint{2.676537in}{3.661196in}}%
\pgfpathlineto{\pgfqpoint{2.684197in}{3.662201in}}%
\pgfpathlineto{\pgfqpoint{2.688028in}{3.662967in}}%
\pgfpathlineto{\pgfqpoint{2.689943in}{3.665128in}}%
\pgfpathlineto{\pgfqpoint{2.693773in}{3.666536in}}%
\pgfpathlineto{\pgfqpoint{2.695688in}{3.670906in}}%
\pgfpathlineto{\pgfqpoint{2.701434in}{3.672027in}}%
\pgfpathlineto{\pgfqpoint{2.707179in}{3.676270in}}%
\pgfpathlineto{\pgfqpoint{2.709095in}{3.682677in}}%
\pgfpathlineto{\pgfqpoint{2.711010in}{3.683962in}}%
\pgfpathlineto{\pgfqpoint{2.714840in}{3.696377in}}%
\pgfpathlineto{\pgfqpoint{2.716755in}{3.696599in}}%
\pgfpathlineto{\pgfqpoint{2.718670in}{3.701352in}}%
\pgfpathlineto{\pgfqpoint{2.726331in}{3.706652in}}%
\pgfpathlineto{\pgfqpoint{2.730161in}{3.713638in}}%
\pgfpathlineto{\pgfqpoint{2.733992in}{3.714972in}}%
\pgfpathlineto{\pgfqpoint{2.737822in}{3.722892in}}%
\pgfpathlineto{\pgfqpoint{2.739737in}{3.723973in}}%
\pgfpathlineto{\pgfqpoint{2.741653in}{3.726708in}}%
\pgfpathlineto{\pgfqpoint{2.743568in}{3.727104in}}%
\pgfpathlineto{\pgfqpoint{2.745483in}{3.732985in}}%
\pgfpathlineto{\pgfqpoint{2.753144in}{3.734856in}}%
\pgfpathlineto{\pgfqpoint{2.758889in}{3.738000in}}%
\pgfpathlineto{\pgfqpoint{2.770380in}{3.739475in}}%
\pgfpathlineto{\pgfqpoint{2.779956in}{3.744636in}}%
\pgfpathlineto{\pgfqpoint{2.781871in}{3.744704in}}%
\pgfpathlineto{\pgfqpoint{2.785701in}{3.745937in}}%
\pgfpathlineto{\pgfqpoint{2.789532in}{3.746917in}}%
\pgfpathlineto{\pgfqpoint{2.799108in}{3.748157in}}%
\pgfpathlineto{\pgfqpoint{2.802938in}{3.752037in}}%
\pgfpathlineto{\pgfqpoint{2.804853in}{3.752350in}}%
\pgfpathlineto{\pgfqpoint{2.806768in}{3.764666in}}%
\pgfpathlineto{\pgfqpoint{2.810599in}{3.766566in}}%
\pgfpathlineto{\pgfqpoint{2.812514in}{3.772617in}}%
\pgfpathlineto{\pgfqpoint{2.814429in}{3.773282in}}%
\pgfpathlineto{\pgfqpoint{2.818259in}{3.778814in}}%
\pgfpathlineto{\pgfqpoint{2.822090in}{3.779319in}}%
\pgfpathlineto{\pgfqpoint{2.824005in}{3.787316in}}%
\pgfpathlineto{\pgfqpoint{2.833581in}{3.793344in}}%
\pgfpathlineto{\pgfqpoint{2.835496in}{3.798826in}}%
\pgfpathlineto{\pgfqpoint{2.839326in}{3.801275in}}%
\pgfpathlineto{\pgfqpoint{2.841241in}{3.804396in}}%
\pgfpathlineto{\pgfqpoint{2.843157in}{3.810822in}}%
\pgfpathlineto{\pgfqpoint{2.846987in}{3.816806in}}%
\pgfpathlineto{\pgfqpoint{2.850817in}{3.825456in}}%
\pgfpathlineto{\pgfqpoint{2.852732in}{3.842806in}}%
\pgfpathlineto{\pgfqpoint{2.854648in}{3.843450in}}%
\pgfpathlineto{\pgfqpoint{2.856563in}{3.845302in}}%
\pgfpathlineto{\pgfqpoint{2.858478in}{3.850743in}}%
\pgfpathlineto{\pgfqpoint{2.860393in}{3.851081in}}%
\pgfpathlineto{\pgfqpoint{2.864223in}{3.863025in}}%
\pgfpathlineto{\pgfqpoint{2.866139in}{3.886614in}}%
\pgfpathlineto{\pgfqpoint{2.869969in}{3.887388in}}%
\pgfpathlineto{\pgfqpoint{2.871884in}{3.899900in}}%
\pgfpathlineto{\pgfqpoint{2.873799in}{3.903678in}}%
\pgfpathlineto{\pgfqpoint{2.879545in}{3.905883in}}%
\pgfpathlineto{\pgfqpoint{2.881460in}{3.906448in}}%
\pgfpathlineto{\pgfqpoint{2.883375in}{3.911974in}}%
\pgfpathlineto{\pgfqpoint{2.885290in}{3.912916in}}%
\pgfpathlineto{\pgfqpoint{2.887206in}{3.917130in}}%
\pgfpathlineto{\pgfqpoint{2.889121in}{3.918038in}}%
\pgfpathlineto{\pgfqpoint{2.891036in}{3.920534in}}%
\pgfpathlineto{\pgfqpoint{2.894866in}{3.934293in}}%
\pgfpathlineto{\pgfqpoint{2.900612in}{3.935695in}}%
\pgfpathlineto{\pgfqpoint{2.902527in}{3.940487in}}%
\pgfpathlineto{\pgfqpoint{2.906357in}{3.942744in}}%
\pgfpathlineto{\pgfqpoint{2.914018in}{3.975926in}}%
\pgfpathlineto{\pgfqpoint{2.915933in}{3.980911in}}%
\pgfpathlineto{\pgfqpoint{2.919763in}{3.981151in}}%
\pgfpathlineto{\pgfqpoint{2.921679in}{3.983148in}}%
\pgfpathlineto{\pgfqpoint{2.927424in}{3.983390in}}%
\pgfpathlineto{\pgfqpoint{2.929339in}{3.990726in}}%
\pgfpathlineto{\pgfqpoint{2.935085in}{3.993923in}}%
\pgfpathlineto{\pgfqpoint{2.937000in}{4.003064in}}%
\pgfpathlineto{\pgfqpoint{2.938915in}{4.005102in}}%
\pgfpathlineto{\pgfqpoint{2.940830in}{4.010543in}}%
\pgfpathlineto{\pgfqpoint{2.944661in}{4.027855in}}%
\pgfpathlineto{\pgfqpoint{2.946576in}{4.032269in}}%
\pgfpathlineto{\pgfqpoint{2.948491in}{4.044959in}}%
\pgfpathlineto{\pgfqpoint{2.950406in}{4.046221in}}%
\pgfpathlineto{\pgfqpoint{2.952321in}{4.050654in}}%
\pgfpathlineto{\pgfqpoint{2.956152in}{4.063048in}}%
\pgfpathlineto{\pgfqpoint{2.958067in}{4.063157in}}%
\pgfpathlineto{\pgfqpoint{2.961897in}{4.068369in}}%
\pgfpathlineto{\pgfqpoint{2.963812in}{4.075654in}}%
\pgfpathlineto{\pgfqpoint{2.967643in}{4.079187in}}%
\pgfpathlineto{\pgfqpoint{2.969558in}{4.080473in}}%
\pgfpathlineto{\pgfqpoint{2.971473in}{4.083430in}}%
\pgfpathlineto{\pgfqpoint{2.975303in}{4.149398in}}%
\pgfpathlineto{\pgfqpoint{2.977219in}{4.149751in}}%
\pgfpathlineto{\pgfqpoint{2.981049in}{4.154424in}}%
\pgfpathlineto{\pgfqpoint{2.982964in}{4.164307in}}%
\pgfpathlineto{\pgfqpoint{2.984879in}{4.164922in}}%
\pgfpathlineto{\pgfqpoint{2.986794in}{4.166839in}}%
\pgfpathlineto{\pgfqpoint{2.988710in}{4.170646in}}%
\pgfpathlineto{\pgfqpoint{2.990625in}{4.182922in}}%
\pgfpathlineto{\pgfqpoint{2.996370in}{4.186777in}}%
\pgfpathlineto{\pgfqpoint{3.000201in}{4.195425in}}%
\pgfpathlineto{\pgfqpoint{3.002116in}{4.196281in}}%
\pgfpathlineto{\pgfqpoint{3.009777in}{4.225778in}}%
\pgfpathlineto{\pgfqpoint{3.011692in}{4.228109in}}%
\pgfpathlineto{\pgfqpoint{3.011692in}{4.228109in}}%
\pgfusepath{stroke}%
\end{pgfscope}%
\begin{pgfscope}%
\pgfpathrectangle{\pgfqpoint{0.694334in}{2.659974in}}{\pgfqpoint{3.830343in}{1.568135in}}%
\pgfusepath{clip}%
\pgfsetrectcap%
\pgfsetroundjoin%
\pgfsetlinewidth{1.003750pt}%
\definecolor{currentstroke}{rgb}{0.811765,0.125490,0.125490}%
\pgfsetstrokecolor{currentstroke}%
\pgfsetdash{}{0pt}%
\pgfpathmoveto{\pgfqpoint{0.694334in}{2.989497in}}%
\pgfpathlineto{\pgfqpoint{0.696249in}{2.991459in}}%
\pgfpathlineto{\pgfqpoint{0.698165in}{2.991459in}}%
\pgfpathlineto{\pgfqpoint{0.700080in}{2.993391in}}%
\pgfpathlineto{\pgfqpoint{0.705825in}{2.993391in}}%
\pgfpathlineto{\pgfqpoint{0.707741in}{2.997170in}}%
\pgfpathlineto{\pgfqpoint{0.709656in}{2.997170in}}%
\pgfpathlineto{\pgfqpoint{0.711571in}{2.999018in}}%
\pgfpathlineto{\pgfqpoint{0.715401in}{2.999018in}}%
\pgfpathlineto{\pgfqpoint{0.719232in}{3.002637in}}%
\pgfpathlineto{\pgfqpoint{0.728807in}{3.002637in}}%
\pgfpathlineto{\pgfqpoint{0.732638in}{3.006157in}}%
\pgfpathlineto{\pgfqpoint{0.736468in}{3.006157in}}%
\pgfpathlineto{\pgfqpoint{0.738383in}{3.007882in}}%
\pgfpathlineto{\pgfqpoint{0.755620in}{3.007882in}}%
\pgfpathlineto{\pgfqpoint{0.757535in}{3.009583in}}%
\pgfpathlineto{\pgfqpoint{0.776687in}{3.009583in}}%
\pgfpathlineto{\pgfqpoint{0.778602in}{3.011263in}}%
\pgfpathlineto{\pgfqpoint{0.790093in}{3.011263in}}%
\pgfpathlineto{\pgfqpoint{0.792008in}{3.012921in}}%
\pgfpathlineto{\pgfqpoint{0.797754in}{3.012921in}}%
\pgfpathlineto{\pgfqpoint{0.799669in}{3.014557in}}%
\pgfpathlineto{\pgfqpoint{0.813075in}{3.014557in}}%
\pgfpathlineto{\pgfqpoint{0.814990in}{3.016173in}}%
\pgfpathlineto{\pgfqpoint{0.824566in}{3.016173in}}%
\pgfpathlineto{\pgfqpoint{0.826481in}{3.017769in}}%
\pgfpathlineto{\pgfqpoint{0.832227in}{3.017769in}}%
\pgfpathlineto{\pgfqpoint{0.834142in}{3.019345in}}%
\pgfpathlineto{\pgfqpoint{0.839887in}{3.019345in}}%
\pgfpathlineto{\pgfqpoint{0.841803in}{3.020903in}}%
\pgfpathlineto{\pgfqpoint{0.860954in}{3.020903in}}%
\pgfpathlineto{\pgfqpoint{0.862869in}{3.022441in}}%
\pgfpathlineto{\pgfqpoint{0.883936in}{3.022441in}}%
\pgfpathlineto{\pgfqpoint{0.885851in}{3.023962in}}%
\pgfpathlineto{\pgfqpoint{0.897342in}{3.023962in}}%
\pgfpathlineto{\pgfqpoint{0.899258in}{3.025464in}}%
\pgfpathlineto{\pgfqpoint{0.905003in}{3.025464in}}%
\pgfpathlineto{\pgfqpoint{0.906918in}{3.026950in}}%
\pgfpathlineto{\pgfqpoint{0.924155in}{3.026950in}}%
\pgfpathlineto{\pgfqpoint{0.926070in}{3.028418in}}%
\pgfpathlineto{\pgfqpoint{0.935646in}{3.028418in}}%
\pgfpathlineto{\pgfqpoint{0.937561in}{3.029870in}}%
\pgfpathlineto{\pgfqpoint{0.950967in}{3.029870in}}%
\pgfpathlineto{\pgfqpoint{0.952882in}{3.031305in}}%
\pgfpathlineto{\pgfqpoint{0.966289in}{3.031305in}}%
\pgfpathlineto{\pgfqpoint{0.968204in}{3.032725in}}%
\pgfpathlineto{\pgfqpoint{0.970119in}{3.032725in}}%
\pgfpathlineto{\pgfqpoint{0.972034in}{3.034129in}}%
\pgfpathlineto{\pgfqpoint{0.983525in}{3.034129in}}%
\pgfpathlineto{\pgfqpoint{0.985440in}{3.035518in}}%
\pgfpathlineto{\pgfqpoint{0.987356in}{3.035518in}}%
\pgfpathlineto{\pgfqpoint{0.991186in}{3.038251in}}%
\pgfpathlineto{\pgfqpoint{0.993101in}{3.038251in}}%
\pgfpathlineto{\pgfqpoint{0.996931in}{3.039596in}}%
\pgfpathlineto{\pgfqpoint{1.002677in}{3.039596in}}%
\pgfpathlineto{\pgfqpoint{1.006507in}{3.040928in}}%
\pgfpathlineto{\pgfqpoint{1.016083in}{3.040928in}}%
\pgfpathlineto{\pgfqpoint{1.019913in}{3.042245in}}%
\pgfpathlineto{\pgfqpoint{1.023744in}{3.042245in}}%
\pgfpathlineto{\pgfqpoint{1.027574in}{3.043550in}}%
\pgfpathlineto{\pgfqpoint{1.033320in}{3.043550in}}%
\pgfpathlineto{\pgfqpoint{1.037150in}{3.044841in}}%
\pgfpathlineto{\pgfqpoint{1.040980in}{3.044841in}}%
\pgfpathlineto{\pgfqpoint{1.044811in}{3.046119in}}%
\pgfpathlineto{\pgfqpoint{1.048641in}{3.046119in}}%
\pgfpathlineto{\pgfqpoint{1.054387in}{3.048638in}}%
\pgfpathlineto{\pgfqpoint{1.058217in}{3.048638in}}%
\pgfpathlineto{\pgfqpoint{1.063962in}{3.051109in}}%
\pgfpathlineto{\pgfqpoint{1.071623in}{3.051109in}}%
\pgfpathlineto{\pgfqpoint{1.075453in}{3.052327in}}%
\pgfpathlineto{\pgfqpoint{1.085029in}{3.052327in}}%
\pgfpathlineto{\pgfqpoint{1.088860in}{3.053533in}}%
\pgfpathlineto{\pgfqpoint{1.102266in}{3.053533in}}%
\pgfpathlineto{\pgfqpoint{1.106096in}{3.054728in}}%
\pgfpathlineto{\pgfqpoint{1.119502in}{3.054728in}}%
\pgfpathlineto{\pgfqpoint{1.123333in}{3.055912in}}%
\pgfpathlineto{\pgfqpoint{1.129078in}{3.055912in}}%
\pgfpathlineto{\pgfqpoint{1.132909in}{3.057085in}}%
\pgfpathlineto{\pgfqpoint{1.152060in}{3.057085in}}%
\pgfpathlineto{\pgfqpoint{1.155891in}{3.058248in}}%
\pgfpathlineto{\pgfqpoint{1.161636in}{3.058248in}}%
\pgfpathlineto{\pgfqpoint{1.165466in}{3.059400in}}%
\pgfpathlineto{\pgfqpoint{1.190364in}{3.059400in}}%
\pgfpathlineto{\pgfqpoint{1.194194in}{3.060542in}}%
\pgfpathlineto{\pgfqpoint{1.213346in}{3.060542in}}%
\pgfpathlineto{\pgfqpoint{1.217176in}{3.061674in}}%
\pgfpathlineto{\pgfqpoint{1.230582in}{3.061674in}}%
\pgfpathlineto{\pgfqpoint{1.234413in}{3.062796in}}%
\pgfpathlineto{\pgfqpoint{1.240158in}{3.062796in}}%
\pgfpathlineto{\pgfqpoint{1.243988in}{3.063909in}}%
\pgfpathlineto{\pgfqpoint{1.253564in}{3.065011in}}%
\pgfpathlineto{\pgfqpoint{1.257395in}{3.066105in}}%
\pgfpathlineto{\pgfqpoint{1.274631in}{3.067189in}}%
\pgfpathlineto{\pgfqpoint{1.278462in}{3.068264in}}%
\pgfpathlineto{\pgfqpoint{1.305274in}{3.069330in}}%
\pgfpathlineto{\pgfqpoint{1.309104in}{3.070388in}}%
\pgfpathlineto{\pgfqpoint{1.334002in}{3.071436in}}%
\pgfpathlineto{\pgfqpoint{1.337832in}{3.072477in}}%
\pgfpathlineto{\pgfqpoint{1.351238in}{3.073508in}}%
\pgfpathlineto{\pgfqpoint{1.355068in}{3.074532in}}%
\pgfpathlineto{\pgfqpoint{1.389542in}{3.075548in}}%
\pgfpathlineto{\pgfqpoint{1.393372in}{3.076555in}}%
\pgfpathlineto{\pgfqpoint{1.416354in}{3.077555in}}%
\pgfpathlineto{\pgfqpoint{1.420184in}{3.078547in}}%
\pgfpathlineto{\pgfqpoint{1.431675in}{3.079532in}}%
\pgfpathlineto{\pgfqpoint{1.435506in}{3.080509in}}%
\pgfpathlineto{\pgfqpoint{1.458488in}{3.081478in}}%
\pgfpathlineto{\pgfqpoint{1.462318in}{3.082441in}}%
\pgfpathlineto{\pgfqpoint{1.473809in}{3.083396in}}%
\pgfpathlineto{\pgfqpoint{1.477639in}{3.084344in}}%
\pgfpathlineto{\pgfqpoint{1.496791in}{3.085285in}}%
\pgfpathlineto{\pgfqpoint{1.500621in}{3.086219in}}%
\pgfpathlineto{\pgfqpoint{1.517858in}{3.087147in}}%
\pgfpathlineto{\pgfqpoint{1.521688in}{3.088068in}}%
\pgfpathlineto{\pgfqpoint{1.531264in}{3.088982in}}%
\pgfpathlineto{\pgfqpoint{1.538925in}{3.091687in}}%
\pgfpathlineto{\pgfqpoint{1.540840in}{3.091687in}}%
\pgfpathlineto{\pgfqpoint{1.542755in}{3.093459in}}%
\pgfpathlineto{\pgfqpoint{1.554246in}{3.094336in}}%
\pgfpathlineto{\pgfqpoint{1.565737in}{3.098633in}}%
\pgfpathlineto{\pgfqpoint{1.571483in}{3.099476in}}%
\pgfpathlineto{\pgfqpoint{1.573398in}{3.101144in}}%
\pgfpathlineto{\pgfqpoint{1.575313in}{3.101144in}}%
\pgfpathlineto{\pgfqpoint{1.577228in}{3.102791in}}%
\pgfpathlineto{\pgfqpoint{1.581059in}{3.103607in}}%
\pgfpathlineto{\pgfqpoint{1.584889in}{3.104418in}}%
\pgfpathlineto{\pgfqpoint{1.592550in}{3.105223in}}%
\pgfpathlineto{\pgfqpoint{1.598295in}{3.106819in}}%
\pgfpathlineto{\pgfqpoint{1.615532in}{3.107610in}}%
\pgfpathlineto{\pgfqpoint{1.621277in}{3.109176in}}%
\pgfpathlineto{\pgfqpoint{1.634683in}{3.109953in}}%
\pgfpathlineto{\pgfqpoint{1.640429in}{3.111491in}}%
\pgfpathlineto{\pgfqpoint{1.644259in}{3.112254in}}%
\pgfpathlineto{\pgfqpoint{1.648090in}{3.113012in}}%
\pgfpathlineto{\pgfqpoint{1.655750in}{3.113765in}}%
\pgfpathlineto{\pgfqpoint{1.663411in}{3.115999in}}%
\pgfpathlineto{\pgfqpoint{1.665326in}{3.115999in}}%
\pgfpathlineto{\pgfqpoint{1.667241in}{3.117468in}}%
\pgfpathlineto{\pgfqpoint{1.674902in}{3.118196in}}%
\pgfpathlineto{\pgfqpoint{1.684478in}{3.121067in}}%
\pgfpathlineto{\pgfqpoint{1.690223in}{3.121774in}}%
\pgfpathlineto{\pgfqpoint{1.694054in}{3.122478in}}%
\pgfpathlineto{\pgfqpoint{1.699799in}{3.123178in}}%
\pgfpathlineto{\pgfqpoint{1.705545in}{3.124567in}}%
\pgfpathlineto{\pgfqpoint{1.711290in}{3.125256in}}%
\pgfpathlineto{\pgfqpoint{1.720866in}{3.127975in}}%
\pgfpathlineto{\pgfqpoint{1.724697in}{3.128646in}}%
\pgfpathlineto{\pgfqpoint{1.728527in}{3.131295in}}%
\pgfpathlineto{\pgfqpoint{1.734272in}{3.131949in}}%
\pgfpathlineto{\pgfqpoint{1.740018in}{3.133891in}}%
\pgfpathlineto{\pgfqpoint{1.741933in}{3.136435in}}%
\pgfpathlineto{\pgfqpoint{1.745763in}{3.137063in}}%
\pgfpathlineto{\pgfqpoint{1.751509in}{3.137688in}}%
\pgfpathlineto{\pgfqpoint{1.761085in}{3.142583in}}%
\pgfpathlineto{\pgfqpoint{1.766830in}{3.143182in}}%
\pgfpathlineto{\pgfqpoint{1.780236in}{3.148450in}}%
\pgfpathlineto{\pgfqpoint{1.785982in}{3.152403in}}%
\pgfpathlineto{\pgfqpoint{1.793643in}{3.153511in}}%
\pgfpathlineto{\pgfqpoint{1.822370in}{3.162558in}}%
\pgfpathlineto{\pgfqpoint{1.826201in}{3.166106in}}%
\pgfpathlineto{\pgfqpoint{1.839607in}{3.170528in}}%
\pgfpathlineto{\pgfqpoint{1.845352in}{3.172446in}}%
\pgfpathlineto{\pgfqpoint{1.854928in}{3.173865in}}%
\pgfpathlineto{\pgfqpoint{1.858759in}{3.173865in}}%
\pgfpathlineto{\pgfqpoint{1.862589in}{3.177118in}}%
\pgfpathlineto{\pgfqpoint{1.866419in}{3.178032in}}%
\pgfpathlineto{\pgfqpoint{1.874080in}{3.184257in}}%
\pgfpathlineto{\pgfqpoint{1.879825in}{3.184690in}}%
\pgfpathlineto{\pgfqpoint{1.897062in}{3.192657in}}%
\pgfpathlineto{\pgfqpoint{1.898977in}{3.195073in}}%
\pgfpathlineto{\pgfqpoint{1.902807in}{3.196659in}}%
\pgfpathlineto{\pgfqpoint{1.904723in}{3.199774in}}%
\pgfpathlineto{\pgfqpoint{1.906638in}{3.200158in}}%
\pgfpathlineto{\pgfqpoint{1.908553in}{3.203190in}}%
\pgfpathlineto{\pgfqpoint{1.914298in}{3.203937in}}%
\pgfpathlineto{\pgfqpoint{1.916214in}{3.204679in}}%
\pgfpathlineto{\pgfqpoint{1.920044in}{3.207969in}}%
\pgfpathlineto{\pgfqpoint{1.925790in}{3.208689in}}%
\pgfpathlineto{\pgfqpoint{1.937281in}{3.212228in}}%
\pgfpathlineto{\pgfqpoint{1.946856in}{3.212925in}}%
\pgfpathlineto{\pgfqpoint{1.948772in}{3.214991in}}%
\pgfpathlineto{\pgfqpoint{1.950687in}{3.222940in}}%
\pgfpathlineto{\pgfqpoint{1.954517in}{3.223261in}}%
\pgfpathlineto{\pgfqpoint{1.958347in}{3.224536in}}%
\pgfpathlineto{\pgfqpoint{1.962178in}{3.225169in}}%
\pgfpathlineto{\pgfqpoint{1.964093in}{3.228288in}}%
\pgfpathlineto{\pgfqpoint{1.977499in}{3.234306in}}%
\pgfpathlineto{\pgfqpoint{1.983245in}{3.235185in}}%
\pgfpathlineto{\pgfqpoint{1.987075in}{3.236925in}}%
\pgfpathlineto{\pgfqpoint{1.988990in}{3.236925in}}%
\pgfpathlineto{\pgfqpoint{1.992821in}{3.240336in}}%
\pgfpathlineto{\pgfqpoint{1.994736in}{3.243385in}}%
\pgfpathlineto{\pgfqpoint{1.998566in}{3.243932in}}%
\pgfpathlineto{\pgfqpoint{2.002396in}{3.247430in}}%
\pgfpathlineto{\pgfqpoint{2.011972in}{3.250835in}}%
\pgfpathlineto{\pgfqpoint{2.015803in}{3.254152in}}%
\pgfpathlineto{\pgfqpoint{2.019633in}{3.255655in}}%
\pgfpathlineto{\pgfqpoint{2.021548in}{3.257631in}}%
\pgfpathlineto{\pgfqpoint{2.027294in}{3.258851in}}%
\pgfpathlineto{\pgfqpoint{2.031124in}{3.260300in}}%
\pgfpathlineto{\pgfqpoint{2.036869in}{3.261495in}}%
\pgfpathlineto{\pgfqpoint{2.040700in}{3.262443in}}%
\pgfpathlineto{\pgfqpoint{2.042615in}{3.264552in}}%
\pgfpathlineto{\pgfqpoint{2.048360in}{3.265477in}}%
\pgfpathlineto{\pgfqpoint{2.052191in}{3.269787in}}%
\pgfpathlineto{\pgfqpoint{2.061767in}{3.282315in}}%
\pgfpathlineto{\pgfqpoint{2.075173in}{3.289591in}}%
\pgfpathlineto{\pgfqpoint{2.079003in}{3.290733in}}%
\pgfpathlineto{\pgfqpoint{2.080918in}{3.292614in}}%
\pgfpathlineto{\pgfqpoint{2.088579in}{3.294283in}}%
\pgfpathlineto{\pgfqpoint{2.094325in}{3.297559in}}%
\pgfpathlineto{\pgfqpoint{2.098155in}{3.298633in}}%
\pgfpathlineto{\pgfqpoint{2.115391in}{3.306578in}}%
\pgfpathlineto{\pgfqpoint{2.117307in}{3.309067in}}%
\pgfpathlineto{\pgfqpoint{2.121137in}{3.309558in}}%
\pgfpathlineto{\pgfqpoint{2.126883in}{3.313427in}}%
\pgfpathlineto{\pgfqpoint{2.132628in}{3.315632in}}%
\pgfpathlineto{\pgfqpoint{2.138374in}{3.321580in}}%
\pgfpathlineto{\pgfqpoint{2.163271in}{3.335145in}}%
\pgfpathlineto{\pgfqpoint{2.165186in}{3.338324in}}%
\pgfpathlineto{\pgfqpoint{2.167101in}{3.338977in}}%
\pgfpathlineto{\pgfqpoint{2.169016in}{3.341043in}}%
\pgfpathlineto{\pgfqpoint{2.180507in}{3.342823in}}%
\pgfpathlineto{\pgfqpoint{2.224556in}{3.359282in}}%
\pgfpathlineto{\pgfqpoint{2.230302in}{3.360938in}}%
\pgfpathlineto{\pgfqpoint{2.239878in}{3.370959in}}%
\pgfpathlineto{\pgfqpoint{2.243708in}{3.372272in}}%
\pgfpathlineto{\pgfqpoint{2.260945in}{3.377005in}}%
\pgfpathlineto{\pgfqpoint{2.262860in}{3.377488in}}%
\pgfpathlineto{\pgfqpoint{2.264775in}{3.382686in}}%
\pgfpathlineto{\pgfqpoint{2.268605in}{3.384708in}}%
\pgfpathlineto{\pgfqpoint{2.270520in}{3.388570in}}%
\pgfpathlineto{\pgfqpoint{2.276266in}{3.389540in}}%
\pgfpathlineto{\pgfqpoint{2.278181in}{3.391371in}}%
\pgfpathlineto{\pgfqpoint{2.280096in}{3.396629in}}%
\pgfpathlineto{\pgfqpoint{2.287757in}{3.398198in}}%
\pgfpathlineto{\pgfqpoint{2.291587in}{3.400154in}}%
\pgfpathlineto{\pgfqpoint{2.299248in}{3.401521in}}%
\pgfpathlineto{\pgfqpoint{2.303078in}{3.402874in}}%
\pgfpathlineto{\pgfqpoint{2.304993in}{3.404369in}}%
\pgfpathlineto{\pgfqpoint{2.306909in}{3.404369in}}%
\pgfpathlineto{\pgfqpoint{2.308824in}{3.407690in}}%
\pgfpathlineto{\pgfqpoint{2.312654in}{3.408223in}}%
\pgfpathlineto{\pgfqpoint{2.314569in}{3.410630in}}%
\pgfpathlineto{\pgfqpoint{2.318400in}{3.411816in}}%
\pgfpathlineto{\pgfqpoint{2.329891in}{3.420246in}}%
\pgfpathlineto{\pgfqpoint{2.347127in}{3.424864in}}%
\pgfpathlineto{\pgfqpoint{2.362449in}{3.436159in}}%
\pgfpathlineto{\pgfqpoint{2.372024in}{3.438640in}}%
\pgfpathlineto{\pgfqpoint{2.375855in}{3.446710in}}%
\pgfpathlineto{\pgfqpoint{2.383515in}{3.450808in}}%
\pgfpathlineto{\pgfqpoint{2.393091in}{3.452164in}}%
\pgfpathlineto{\pgfqpoint{2.408413in}{3.469776in}}%
\pgfpathlineto{\pgfqpoint{2.414158in}{3.473118in}}%
\pgfpathlineto{\pgfqpoint{2.421819in}{3.476018in}}%
\pgfpathlineto{\pgfqpoint{2.423734in}{3.479813in}}%
\pgfpathlineto{\pgfqpoint{2.425649in}{3.480031in}}%
\pgfpathlineto{\pgfqpoint{2.427564in}{3.483838in}}%
\pgfpathlineto{\pgfqpoint{2.431395in}{3.485388in}}%
\pgfpathlineto{\pgfqpoint{2.433310in}{3.488107in}}%
\pgfpathlineto{\pgfqpoint{2.440971in}{3.490170in}}%
\pgfpathlineto{\pgfqpoint{2.446716in}{3.494315in}}%
\pgfpathlineto{\pgfqpoint{2.450546in}{3.496702in}}%
\pgfpathlineto{\pgfqpoint{2.460122in}{3.505114in}}%
\pgfpathlineto{\pgfqpoint{2.462037in}{3.505257in}}%
\pgfpathlineto{\pgfqpoint{2.465868in}{3.507029in}}%
\pgfpathlineto{\pgfqpoint{2.473529in}{3.508568in}}%
\pgfpathlineto{\pgfqpoint{2.475444in}{3.512168in}}%
\pgfpathlineto{\pgfqpoint{2.477359in}{3.512573in}}%
\pgfpathlineto{\pgfqpoint{2.479274in}{3.514944in}}%
\pgfpathlineto{\pgfqpoint{2.485020in}{3.516816in}}%
\pgfpathlineto{\pgfqpoint{2.492680in}{3.523827in}}%
\pgfpathlineto{\pgfqpoint{2.498426in}{3.526908in}}%
\pgfpathlineto{\pgfqpoint{2.500341in}{3.526968in}}%
\pgfpathlineto{\pgfqpoint{2.504171in}{3.529238in}}%
\pgfpathlineto{\pgfqpoint{2.509917in}{3.530417in}}%
\pgfpathlineto{\pgfqpoint{2.511832in}{3.532742in}}%
\pgfpathlineto{\pgfqpoint{2.513747in}{3.533116in}}%
\pgfpathlineto{\pgfqpoint{2.515662in}{3.536237in}}%
\pgfpathlineto{\pgfqpoint{2.517577in}{3.536405in}}%
\pgfpathlineto{\pgfqpoint{2.527153in}{3.543141in}}%
\pgfpathlineto{\pgfqpoint{2.529068in}{3.547168in}}%
\pgfpathlineto{\pgfqpoint{2.534814in}{3.548805in}}%
\pgfpathlineto{\pgfqpoint{2.542475in}{3.549919in}}%
\pgfpathlineto{\pgfqpoint{2.546305in}{3.551845in}}%
\pgfpathlineto{\pgfqpoint{2.550135in}{3.553227in}}%
\pgfpathlineto{\pgfqpoint{2.555881in}{3.558803in}}%
\pgfpathlineto{\pgfqpoint{2.559711in}{3.562443in}}%
\pgfpathlineto{\pgfqpoint{2.561626in}{3.562602in}}%
\pgfpathlineto{\pgfqpoint{2.565457in}{3.565470in}}%
\pgfpathlineto{\pgfqpoint{2.569287in}{3.566005in}}%
\pgfpathlineto{\pgfqpoint{2.571202in}{3.568034in}}%
\pgfpathlineto{\pgfqpoint{2.573117in}{3.568340in}}%
\pgfpathlineto{\pgfqpoint{2.582693in}{3.576687in}}%
\pgfpathlineto{\pgfqpoint{2.584608in}{3.578858in}}%
\pgfpathlineto{\pgfqpoint{2.586524in}{3.579320in}}%
\pgfpathlineto{\pgfqpoint{2.588439in}{3.582683in}}%
\pgfpathlineto{\pgfqpoint{2.590354in}{3.582879in}}%
\pgfpathlineto{\pgfqpoint{2.592269in}{3.586284in}}%
\pgfpathlineto{\pgfqpoint{2.598015in}{3.587400in}}%
\pgfpathlineto{\pgfqpoint{2.601845in}{3.589008in}}%
\pgfpathlineto{\pgfqpoint{2.611421in}{3.598518in}}%
\pgfpathlineto{\pgfqpoint{2.613336in}{3.599413in}}%
\pgfpathlineto{\pgfqpoint{2.615251in}{3.603496in}}%
\pgfpathlineto{\pgfqpoint{2.620997in}{3.605784in}}%
\pgfpathlineto{\pgfqpoint{2.630573in}{3.619993in}}%
\pgfpathlineto{\pgfqpoint{2.632488in}{3.620358in}}%
\pgfpathlineto{\pgfqpoint{2.636318in}{3.622825in}}%
\pgfpathlineto{\pgfqpoint{2.643979in}{3.628225in}}%
\pgfpathlineto{\pgfqpoint{2.647809in}{3.640210in}}%
\pgfpathlineto{\pgfqpoint{2.649724in}{3.641006in}}%
\pgfpathlineto{\pgfqpoint{2.651639in}{3.643718in}}%
\pgfpathlineto{\pgfqpoint{2.661215in}{3.646481in}}%
\pgfpathlineto{\pgfqpoint{2.665046in}{3.649882in}}%
\pgfpathlineto{\pgfqpoint{2.666961in}{3.649894in}}%
\pgfpathlineto{\pgfqpoint{2.670791in}{3.654263in}}%
\pgfpathlineto{\pgfqpoint{2.672706in}{3.654419in}}%
\pgfpathlineto{\pgfqpoint{2.676537in}{3.657466in}}%
\pgfpathlineto{\pgfqpoint{2.680367in}{3.657630in}}%
\pgfpathlineto{\pgfqpoint{2.682282in}{3.660517in}}%
\pgfpathlineto{\pgfqpoint{2.684197in}{3.660720in}}%
\pgfpathlineto{\pgfqpoint{2.686113in}{3.664338in}}%
\pgfpathlineto{\pgfqpoint{2.688028in}{3.664710in}}%
\pgfpathlineto{\pgfqpoint{2.689943in}{3.666523in}}%
\pgfpathlineto{\pgfqpoint{2.695688in}{3.666858in}}%
\pgfpathlineto{\pgfqpoint{2.697604in}{3.673835in}}%
\pgfpathlineto{\pgfqpoint{2.699519in}{3.674180in}}%
\pgfpathlineto{\pgfqpoint{2.703349in}{3.679931in}}%
\pgfpathlineto{\pgfqpoint{2.707179in}{3.680434in}}%
\pgfpathlineto{\pgfqpoint{2.716755in}{3.685322in}}%
\pgfpathlineto{\pgfqpoint{2.720586in}{3.687257in}}%
\pgfpathlineto{\pgfqpoint{2.743568in}{3.698770in}}%
\pgfpathlineto{\pgfqpoint{2.749313in}{3.701788in}}%
\pgfpathlineto{\pgfqpoint{2.753144in}{3.702995in}}%
\pgfpathlineto{\pgfqpoint{2.755059in}{3.706566in}}%
\pgfpathlineto{\pgfqpoint{2.756974in}{3.707138in}}%
\pgfpathlineto{\pgfqpoint{2.758889in}{3.709153in}}%
\pgfpathlineto{\pgfqpoint{2.760804in}{3.717295in}}%
\pgfpathlineto{\pgfqpoint{2.764635in}{3.718508in}}%
\pgfpathlineto{\pgfqpoint{2.766550in}{3.720821in}}%
\pgfpathlineto{\pgfqpoint{2.774210in}{3.723381in}}%
\pgfpathlineto{\pgfqpoint{2.776126in}{3.724590in}}%
\pgfpathlineto{\pgfqpoint{2.779956in}{3.729272in}}%
\pgfpathlineto{\pgfqpoint{2.793362in}{3.737008in}}%
\pgfpathlineto{\pgfqpoint{2.797192in}{3.745024in}}%
\pgfpathlineto{\pgfqpoint{2.799108in}{3.745178in}}%
\pgfpathlineto{\pgfqpoint{2.810599in}{3.753507in}}%
\pgfpathlineto{\pgfqpoint{2.812514in}{3.758030in}}%
\pgfpathlineto{\pgfqpoint{2.814429in}{3.758144in}}%
\pgfpathlineto{\pgfqpoint{2.820175in}{3.761635in}}%
\pgfpathlineto{\pgfqpoint{2.824005in}{3.761819in}}%
\pgfpathlineto{\pgfqpoint{2.835496in}{3.769534in}}%
\pgfpathlineto{\pgfqpoint{2.837411in}{3.774354in}}%
\pgfpathlineto{\pgfqpoint{2.841241in}{3.775103in}}%
\pgfpathlineto{\pgfqpoint{2.852732in}{3.791677in}}%
\pgfpathlineto{\pgfqpoint{2.856563in}{3.792736in}}%
\pgfpathlineto{\pgfqpoint{2.858478in}{3.797735in}}%
\pgfpathlineto{\pgfqpoint{2.862308in}{3.798894in}}%
\pgfpathlineto{\pgfqpoint{2.864223in}{3.802073in}}%
\pgfpathlineto{\pgfqpoint{2.871884in}{3.804406in}}%
\pgfpathlineto{\pgfqpoint{2.875715in}{3.810800in}}%
\pgfpathlineto{\pgfqpoint{2.885290in}{3.815713in}}%
\pgfpathlineto{\pgfqpoint{2.891036in}{3.816671in}}%
\pgfpathlineto{\pgfqpoint{2.898697in}{3.819285in}}%
\pgfpathlineto{\pgfqpoint{2.900612in}{3.824551in}}%
\pgfpathlineto{\pgfqpoint{2.912103in}{3.827840in}}%
\pgfpathlineto{\pgfqpoint{2.914018in}{3.830886in}}%
\pgfpathlineto{\pgfqpoint{2.915933in}{3.831281in}}%
\pgfpathlineto{\pgfqpoint{2.919763in}{3.836577in}}%
\pgfpathlineto{\pgfqpoint{2.931254in}{3.842938in}}%
\pgfpathlineto{\pgfqpoint{2.933170in}{3.846622in}}%
\pgfpathlineto{\pgfqpoint{2.950406in}{3.854826in}}%
\pgfpathlineto{\pgfqpoint{2.952321in}{3.858209in}}%
\pgfpathlineto{\pgfqpoint{2.956152in}{3.859032in}}%
\pgfpathlineto{\pgfqpoint{2.958067in}{3.859886in}}%
\pgfpathlineto{\pgfqpoint{2.959982in}{3.863395in}}%
\pgfpathlineto{\pgfqpoint{2.961897in}{3.863763in}}%
\pgfpathlineto{\pgfqpoint{2.973388in}{3.873302in}}%
\pgfpathlineto{\pgfqpoint{2.977219in}{3.874709in}}%
\pgfpathlineto{\pgfqpoint{2.981049in}{3.877075in}}%
\pgfpathlineto{\pgfqpoint{2.984879in}{3.878227in}}%
\pgfpathlineto{\pgfqpoint{2.986794in}{3.879034in}}%
\pgfpathlineto{\pgfqpoint{2.988710in}{3.881399in}}%
\pgfpathlineto{\pgfqpoint{2.990625in}{3.881460in}}%
\pgfpathlineto{\pgfqpoint{3.000201in}{3.889847in}}%
\pgfpathlineto{\pgfqpoint{3.005946in}{3.898016in}}%
\pgfpathlineto{\pgfqpoint{3.011692in}{3.900794in}}%
\pgfpathlineto{\pgfqpoint{3.021268in}{3.902062in}}%
\pgfpathlineto{\pgfqpoint{3.025098in}{3.912565in}}%
\pgfpathlineto{\pgfqpoint{3.027013in}{3.912672in}}%
\pgfpathlineto{\pgfqpoint{3.028928in}{3.916226in}}%
\pgfpathlineto{\pgfqpoint{3.038504in}{3.922357in}}%
\pgfpathlineto{\pgfqpoint{3.040419in}{3.926741in}}%
\pgfpathlineto{\pgfqpoint{3.071062in}{3.937055in}}%
\pgfpathlineto{\pgfqpoint{3.072977in}{3.937950in}}%
\pgfpathlineto{\pgfqpoint{3.076808in}{3.942890in}}%
\pgfpathlineto{\pgfqpoint{3.080638in}{3.945295in}}%
\pgfpathlineto{\pgfqpoint{3.082553in}{3.951697in}}%
\pgfpathlineto{\pgfqpoint{3.088299in}{3.957708in}}%
\pgfpathlineto{\pgfqpoint{3.090214in}{3.962727in}}%
\pgfpathlineto{\pgfqpoint{3.101705in}{3.965195in}}%
\pgfpathlineto{\pgfqpoint{3.103620in}{3.968601in}}%
\pgfpathlineto{\pgfqpoint{3.109365in}{3.970754in}}%
\pgfpathlineto{\pgfqpoint{3.113196in}{3.980593in}}%
\pgfpathlineto{\pgfqpoint{3.115111in}{3.981345in}}%
\pgfpathlineto{\pgfqpoint{3.118941in}{3.990477in}}%
\pgfpathlineto{\pgfqpoint{3.122772in}{3.991910in}}%
\pgfpathlineto{\pgfqpoint{3.128517in}{3.993905in}}%
\pgfpathlineto{\pgfqpoint{3.132347in}{3.999704in}}%
\pgfpathlineto{\pgfqpoint{3.136178in}{4.000785in}}%
\pgfpathlineto{\pgfqpoint{3.141923in}{4.012577in}}%
\pgfpathlineto{\pgfqpoint{3.143839in}{4.019958in}}%
\pgfpathlineto{\pgfqpoint{3.145754in}{4.020894in}}%
\pgfpathlineto{\pgfqpoint{3.147669in}{4.024074in}}%
\pgfpathlineto{\pgfqpoint{3.149584in}{4.029636in}}%
\pgfpathlineto{\pgfqpoint{3.153414in}{4.032068in}}%
\pgfpathlineto{\pgfqpoint{3.155330in}{4.032237in}}%
\pgfpathlineto{\pgfqpoint{3.157245in}{4.037641in}}%
\pgfpathlineto{\pgfqpoint{3.161075in}{4.040012in}}%
\pgfpathlineto{\pgfqpoint{3.162990in}{4.041095in}}%
\pgfpathlineto{\pgfqpoint{3.166821in}{4.047719in}}%
\pgfpathlineto{\pgfqpoint{3.168736in}{4.051191in}}%
\pgfpathlineto{\pgfqpoint{3.172566in}{4.051688in}}%
\pgfpathlineto{\pgfqpoint{3.174481in}{4.054381in}}%
\pgfpathlineto{\pgfqpoint{3.178312in}{4.056067in}}%
\pgfpathlineto{\pgfqpoint{3.180227in}{4.059256in}}%
\pgfpathlineto{\pgfqpoint{3.184057in}{4.060298in}}%
\pgfpathlineto{\pgfqpoint{3.185972in}{4.060826in}}%
\pgfpathlineto{\pgfqpoint{3.191718in}{4.071141in}}%
\pgfpathlineto{\pgfqpoint{3.195548in}{4.072222in}}%
\pgfpathlineto{\pgfqpoint{3.201294in}{4.077084in}}%
\pgfpathlineto{\pgfqpoint{3.208954in}{4.081871in}}%
\pgfpathlineto{\pgfqpoint{3.210870in}{4.085371in}}%
\pgfpathlineto{\pgfqpoint{3.214700in}{4.086489in}}%
\pgfpathlineto{\pgfqpoint{3.218530in}{4.087264in}}%
\pgfpathlineto{\pgfqpoint{3.230021in}{4.101865in}}%
\pgfpathlineto{\pgfqpoint{3.231936in}{4.102965in}}%
\pgfpathlineto{\pgfqpoint{3.239597in}{4.116019in}}%
\pgfpathlineto{\pgfqpoint{3.243427in}{4.119815in}}%
\pgfpathlineto{\pgfqpoint{3.245343in}{4.131229in}}%
\pgfpathlineto{\pgfqpoint{3.249173in}{4.136460in}}%
\pgfpathlineto{\pgfqpoint{3.253003in}{4.138083in}}%
\pgfpathlineto{\pgfqpoint{3.256834in}{4.141353in}}%
\pgfpathlineto{\pgfqpoint{3.258749in}{4.141530in}}%
\pgfpathlineto{\pgfqpoint{3.260664in}{4.147572in}}%
\pgfpathlineto{\pgfqpoint{3.264494in}{4.148195in}}%
\pgfpathlineto{\pgfqpoint{3.266409in}{4.150456in}}%
\pgfpathlineto{\pgfqpoint{3.268325in}{4.155237in}}%
\pgfpathlineto{\pgfqpoint{3.272155in}{4.157638in}}%
\pgfpathlineto{\pgfqpoint{3.274070in}{4.161963in}}%
\pgfpathlineto{\pgfqpoint{3.277901in}{4.163034in}}%
\pgfpathlineto{\pgfqpoint{3.279816in}{4.164027in}}%
\pgfpathlineto{\pgfqpoint{3.283646in}{4.171938in}}%
\pgfpathlineto{\pgfqpoint{3.285561in}{4.172168in}}%
\pgfpathlineto{\pgfqpoint{3.289392in}{4.182504in}}%
\pgfpathlineto{\pgfqpoint{3.293222in}{4.185215in}}%
\pgfpathlineto{\pgfqpoint{3.298967in}{4.188298in}}%
\pgfpathlineto{\pgfqpoint{3.304713in}{4.197983in}}%
\pgfpathlineto{\pgfqpoint{3.310458in}{4.198781in}}%
\pgfpathlineto{\pgfqpoint{3.316204in}{4.203677in}}%
\pgfpathlineto{\pgfqpoint{3.320034in}{4.206152in}}%
\pgfpathlineto{\pgfqpoint{3.321949in}{4.211512in}}%
\pgfpathlineto{\pgfqpoint{3.327695in}{4.214162in}}%
\pgfpathlineto{\pgfqpoint{3.329610in}{4.214518in}}%
\pgfpathlineto{\pgfqpoint{3.331525in}{4.216154in}}%
\pgfpathlineto{\pgfqpoint{3.333440in}{4.220756in}}%
\pgfpathlineto{\pgfqpoint{3.337271in}{4.223318in}}%
\pgfpathlineto{\pgfqpoint{3.339186in}{4.228109in}}%
\pgfpathlineto{\pgfqpoint{3.339186in}{4.228109in}}%
\pgfusepath{stroke}%
\end{pgfscope}%
\begin{pgfscope}%
\pgfpathrectangle{\pgfqpoint{0.694334in}{2.659974in}}{\pgfqpoint{3.830343in}{1.568135in}}%
\pgfusepath{clip}%
\pgfsetbuttcap%
\pgfsetroundjoin%
\pgfsetlinewidth{1.003750pt}%
\definecolor{currentstroke}{rgb}{0.811765,0.125490,0.125490}%
\pgfsetstrokecolor{currentstroke}%
\pgfsetdash{{3.700000pt}{1.600000pt}}{0.000000pt}%
\pgfpathmoveto{\pgfqpoint{0.705197in}{2.649974in}}%
\pgfpathlineto{\pgfqpoint{0.707741in}{2.653951in}}%
\pgfpathlineto{\pgfqpoint{0.715401in}{2.661430in}}%
\pgfpathlineto{\pgfqpoint{0.724977in}{2.664765in}}%
\pgfpathlineto{\pgfqpoint{0.726892in}{2.666291in}}%
\pgfpathlineto{\pgfqpoint{0.730723in}{2.678108in}}%
\pgfpathlineto{\pgfqpoint{0.736468in}{2.680627in}}%
\pgfpathlineto{\pgfqpoint{0.738383in}{2.680714in}}%
\pgfpathlineto{\pgfqpoint{0.742214in}{2.682387in}}%
\pgfpathlineto{\pgfqpoint{0.757535in}{2.685479in}}%
\pgfpathlineto{\pgfqpoint{0.763280in}{2.686528in}}%
\pgfpathlineto{\pgfqpoint{0.767111in}{2.686924in}}%
\pgfpathlineto{\pgfqpoint{0.770941in}{2.689380in}}%
\pgfpathlineto{\pgfqpoint{0.772856in}{2.695720in}}%
\pgfpathlineto{\pgfqpoint{0.780517in}{2.700849in}}%
\pgfpathlineto{\pgfqpoint{0.792008in}{2.702410in}}%
\pgfpathlineto{\pgfqpoint{0.811160in}{2.707709in}}%
\pgfpathlineto{\pgfqpoint{0.813075in}{2.714104in}}%
\pgfpathlineto{\pgfqpoint{0.816905in}{2.716393in}}%
\pgfpathlineto{\pgfqpoint{0.820736in}{2.716988in}}%
\pgfpathlineto{\pgfqpoint{0.830311in}{2.720099in}}%
\pgfpathlineto{\pgfqpoint{0.839887in}{2.721539in}}%
\pgfpathlineto{\pgfqpoint{0.847548in}{2.722206in}}%
\pgfpathlineto{\pgfqpoint{0.857124in}{2.723669in}}%
\pgfpathlineto{\pgfqpoint{0.864785in}{2.725842in}}%
\pgfpathlineto{\pgfqpoint{0.868615in}{2.728356in}}%
\pgfpathlineto{\pgfqpoint{0.870530in}{2.728432in}}%
\pgfpathlineto{\pgfqpoint{0.874360in}{2.730125in}}%
\pgfpathlineto{\pgfqpoint{0.889682in}{2.733561in}}%
\pgfpathlineto{\pgfqpoint{0.893512in}{2.734873in}}%
\pgfpathlineto{\pgfqpoint{0.908834in}{2.735916in}}%
\pgfpathlineto{\pgfqpoint{0.926070in}{2.738535in}}%
\pgfpathlineto{\pgfqpoint{0.929900in}{2.739479in}}%
\pgfpathlineto{\pgfqpoint{0.931816in}{2.742475in}}%
\pgfpathlineto{\pgfqpoint{0.939476in}{2.743753in}}%
\pgfpathlineto{\pgfqpoint{0.945222in}{2.744580in}}%
\pgfpathlineto{\pgfqpoint{0.968204in}{2.747888in}}%
\pgfpathlineto{\pgfqpoint{0.981610in}{2.749062in}}%
\pgfpathlineto{\pgfqpoint{1.012253in}{2.757536in}}%
\pgfpathlineto{\pgfqpoint{1.017998in}{2.758971in}}%
\pgfpathlineto{\pgfqpoint{1.023744in}{2.759918in}}%
\pgfpathlineto{\pgfqpoint{1.027574in}{2.761081in}}%
\pgfpathlineto{\pgfqpoint{1.031404in}{2.761956in}}%
\pgfpathlineto{\pgfqpoint{1.037150in}{2.762847in}}%
\pgfpathlineto{\pgfqpoint{1.050556in}{2.766576in}}%
\pgfpathlineto{\pgfqpoint{1.056302in}{2.769611in}}%
\pgfpathlineto{\pgfqpoint{1.083114in}{2.773269in}}%
\pgfpathlineto{\pgfqpoint{1.086944in}{2.776234in}}%
\pgfpathlineto{\pgfqpoint{1.096520in}{2.779892in}}%
\pgfpathlineto{\pgfqpoint{1.108011in}{2.784769in}}%
\pgfpathlineto{\pgfqpoint{1.115672in}{2.786114in}}%
\pgfpathlineto{\pgfqpoint{1.148230in}{2.796200in}}%
\pgfpathlineto{\pgfqpoint{1.150145in}{2.799277in}}%
\pgfpathlineto{\pgfqpoint{1.165466in}{2.802495in}}%
\pgfpathlineto{\pgfqpoint{1.169297in}{2.803801in}}%
\pgfpathlineto{\pgfqpoint{1.171212in}{2.806819in}}%
\pgfpathlineto{\pgfqpoint{1.178873in}{2.809807in}}%
\pgfpathlineto{\pgfqpoint{1.180788in}{2.812633in}}%
\pgfpathlineto{\pgfqpoint{1.199940in}{2.816033in}}%
\pgfpathlineto{\pgfqpoint{1.201855in}{2.818031in}}%
\pgfpathlineto{\pgfqpoint{1.209515in}{2.818996in}}%
\pgfpathlineto{\pgfqpoint{1.213346in}{2.821380in}}%
\pgfpathlineto{\pgfqpoint{1.222922in}{2.824302in}}%
\pgfpathlineto{\pgfqpoint{1.230582in}{2.825611in}}%
\pgfpathlineto{\pgfqpoint{1.234413in}{2.827842in}}%
\pgfpathlineto{\pgfqpoint{1.236328in}{2.829858in}}%
\pgfpathlineto{\pgfqpoint{1.251649in}{2.832856in}}%
\pgfpathlineto{\pgfqpoint{1.257395in}{2.836873in}}%
\pgfpathlineto{\pgfqpoint{1.263140in}{2.837643in}}%
\pgfpathlineto{\pgfqpoint{1.272716in}{2.841484in}}%
\pgfpathlineto{\pgfqpoint{1.288037in}{2.847215in}}%
\pgfpathlineto{\pgfqpoint{1.289953in}{2.847562in}}%
\pgfpathlineto{\pgfqpoint{1.291868in}{2.850254in}}%
\pgfpathlineto{\pgfqpoint{1.307189in}{2.853428in}}%
\pgfpathlineto{\pgfqpoint{1.311019in}{2.854636in}}%
\pgfpathlineto{\pgfqpoint{1.324426in}{2.860373in}}%
\pgfpathlineto{\pgfqpoint{1.328256in}{2.862228in}}%
\pgfpathlineto{\pgfqpoint{1.334002in}{2.863383in}}%
\pgfpathlineto{\pgfqpoint{1.337832in}{2.864727in}}%
\pgfpathlineto{\pgfqpoint{1.339747in}{2.868446in}}%
\pgfpathlineto{\pgfqpoint{1.347408in}{2.872386in}}%
\pgfpathlineto{\pgfqpoint{1.349323in}{2.872568in}}%
\pgfpathlineto{\pgfqpoint{1.370390in}{2.890143in}}%
\pgfpathlineto{\pgfqpoint{1.374220in}{2.890583in}}%
\pgfpathlineto{\pgfqpoint{1.378050in}{2.893733in}}%
\pgfpathlineto{\pgfqpoint{1.381881in}{2.894956in}}%
\pgfpathlineto{\pgfqpoint{1.383796in}{2.896986in}}%
\pgfpathlineto{\pgfqpoint{1.389542in}{2.898480in}}%
\pgfpathlineto{\pgfqpoint{1.393372in}{2.902701in}}%
\pgfpathlineto{\pgfqpoint{1.401033in}{2.904747in}}%
\pgfpathlineto{\pgfqpoint{1.404863in}{2.906633in}}%
\pgfpathlineto{\pgfqpoint{1.410608in}{2.908469in}}%
\pgfpathlineto{\pgfqpoint{1.427845in}{2.916170in}}%
\pgfpathlineto{\pgfqpoint{1.433590in}{2.917337in}}%
\pgfpathlineto{\pgfqpoint{1.435506in}{2.919618in}}%
\pgfpathlineto{\pgfqpoint{1.448912in}{2.921819in}}%
\pgfpathlineto{\pgfqpoint{1.450827in}{2.924996in}}%
\pgfpathlineto{\pgfqpoint{1.454657in}{2.925881in}}%
\pgfpathlineto{\pgfqpoint{1.458488in}{2.928976in}}%
\pgfpathlineto{\pgfqpoint{1.464233in}{2.930349in}}%
\pgfpathlineto{\pgfqpoint{1.466148in}{2.933489in}}%
\pgfpathlineto{\pgfqpoint{1.469979in}{2.934227in}}%
\pgfpathlineto{\pgfqpoint{1.473809in}{2.936229in}}%
\pgfpathlineto{\pgfqpoint{1.475724in}{2.937299in}}%
\pgfpathlineto{\pgfqpoint{1.479555in}{2.942586in}}%
\pgfpathlineto{\pgfqpoint{1.483385in}{2.942913in}}%
\pgfpathlineto{\pgfqpoint{1.485300in}{2.948692in}}%
\pgfpathlineto{\pgfqpoint{1.491046in}{2.950096in}}%
\pgfpathlineto{\pgfqpoint{1.492961in}{2.953512in}}%
\pgfpathlineto{\pgfqpoint{1.498706in}{2.956132in}}%
\pgfpathlineto{\pgfqpoint{1.502537in}{2.957967in}}%
\pgfpathlineto{\pgfqpoint{1.504452in}{2.958204in}}%
\pgfpathlineto{\pgfqpoint{1.506367in}{2.960748in}}%
\pgfpathlineto{\pgfqpoint{1.512112in}{2.962267in}}%
\pgfpathlineto{\pgfqpoint{1.515943in}{2.966314in}}%
\pgfpathlineto{\pgfqpoint{1.519773in}{2.970646in}}%
\pgfpathlineto{\pgfqpoint{1.521688in}{2.970706in}}%
\pgfpathlineto{\pgfqpoint{1.535095in}{2.980435in}}%
\pgfpathlineto{\pgfqpoint{1.537010in}{2.983532in}}%
\pgfpathlineto{\pgfqpoint{1.550416in}{2.986011in}}%
\pgfpathlineto{\pgfqpoint{1.556161in}{2.988396in}}%
\pgfpathlineto{\pgfqpoint{1.558077in}{2.988416in}}%
\pgfpathlineto{\pgfqpoint{1.561907in}{2.990256in}}%
\pgfpathlineto{\pgfqpoint{1.569568in}{2.992465in}}%
\pgfpathlineto{\pgfqpoint{1.582974in}{2.996967in}}%
\pgfpathlineto{\pgfqpoint{1.588719in}{2.998647in}}%
\pgfpathlineto{\pgfqpoint{1.590635in}{2.998976in}}%
\pgfpathlineto{\pgfqpoint{1.592550in}{3.002341in}}%
\pgfpathlineto{\pgfqpoint{1.600210in}{3.004152in}}%
\pgfpathlineto{\pgfqpoint{1.605956in}{3.007554in}}%
\pgfpathlineto{\pgfqpoint{1.609786in}{3.010184in}}%
\pgfpathlineto{\pgfqpoint{1.611701in}{3.012886in}}%
\pgfpathlineto{\pgfqpoint{1.615532in}{3.013959in}}%
\pgfpathlineto{\pgfqpoint{1.619362in}{3.016340in}}%
\pgfpathlineto{\pgfqpoint{1.623192in}{3.016825in}}%
\pgfpathlineto{\pgfqpoint{1.630853in}{3.021446in}}%
\pgfpathlineto{\pgfqpoint{1.638514in}{3.024596in}}%
\pgfpathlineto{\pgfqpoint{1.646174in}{3.028964in}}%
\pgfpathlineto{\pgfqpoint{1.650005in}{3.030178in}}%
\pgfpathlineto{\pgfqpoint{1.657666in}{3.031219in}}%
\pgfpathlineto{\pgfqpoint{1.661496in}{3.034010in}}%
\pgfpathlineto{\pgfqpoint{1.665326in}{3.037225in}}%
\pgfpathlineto{\pgfqpoint{1.669157in}{3.040224in}}%
\pgfpathlineto{\pgfqpoint{1.680648in}{3.048854in}}%
\pgfpathlineto{\pgfqpoint{1.682563in}{3.049019in}}%
\pgfpathlineto{\pgfqpoint{1.688308in}{3.053561in}}%
\pgfpathlineto{\pgfqpoint{1.695969in}{3.054564in}}%
\pgfpathlineto{\pgfqpoint{1.701714in}{3.057435in}}%
\pgfpathlineto{\pgfqpoint{1.711290in}{3.059553in}}%
\pgfpathlineto{\pgfqpoint{1.736188in}{3.073364in}}%
\pgfpathlineto{\pgfqpoint{1.740018in}{3.075251in}}%
\pgfpathlineto{\pgfqpoint{1.757254in}{3.087539in}}%
\pgfpathlineto{\pgfqpoint{1.761085in}{3.089285in}}%
\pgfpathlineto{\pgfqpoint{1.763000in}{3.089681in}}%
\pgfpathlineto{\pgfqpoint{1.766830in}{3.091966in}}%
\pgfpathlineto{\pgfqpoint{1.770661in}{3.093420in}}%
\pgfpathlineto{\pgfqpoint{1.772576in}{3.094417in}}%
\pgfpathlineto{\pgfqpoint{1.774491in}{3.097211in}}%
\pgfpathlineto{\pgfqpoint{1.776406in}{3.097299in}}%
\pgfpathlineto{\pgfqpoint{1.778321in}{3.100287in}}%
\pgfpathlineto{\pgfqpoint{1.782152in}{3.101806in}}%
\pgfpathlineto{\pgfqpoint{1.787897in}{3.103177in}}%
\pgfpathlineto{\pgfqpoint{1.791728in}{3.105167in}}%
\pgfpathlineto{\pgfqpoint{1.799388in}{3.112282in}}%
\pgfpathlineto{\pgfqpoint{1.814710in}{3.115834in}}%
\pgfpathlineto{\pgfqpoint{1.818540in}{3.119656in}}%
\pgfpathlineto{\pgfqpoint{1.820455in}{3.120605in}}%
\pgfpathlineto{\pgfqpoint{1.824285in}{3.123902in}}%
\pgfpathlineto{\pgfqpoint{1.830031in}{3.126318in}}%
\pgfpathlineto{\pgfqpoint{1.831946in}{3.128432in}}%
\pgfpathlineto{\pgfqpoint{1.837692in}{3.129017in}}%
\pgfpathlineto{\pgfqpoint{1.849183in}{3.144541in}}%
\pgfpathlineto{\pgfqpoint{1.856843in}{3.147306in}}%
\pgfpathlineto{\pgfqpoint{1.858759in}{3.148899in}}%
\pgfpathlineto{\pgfqpoint{1.860674in}{3.151988in}}%
\pgfpathlineto{\pgfqpoint{1.866419in}{3.154041in}}%
\pgfpathlineto{\pgfqpoint{1.872165in}{3.155094in}}%
\pgfpathlineto{\pgfqpoint{1.877910in}{3.159227in}}%
\pgfpathlineto{\pgfqpoint{1.891316in}{3.162687in}}%
\pgfpathlineto{\pgfqpoint{1.893232in}{3.164625in}}%
\pgfpathlineto{\pgfqpoint{1.895147in}{3.164652in}}%
\pgfpathlineto{\pgfqpoint{1.897062in}{3.167250in}}%
\pgfpathlineto{\pgfqpoint{1.898977in}{3.167567in}}%
\pgfpathlineto{\pgfqpoint{1.902807in}{3.169370in}}%
\pgfpathlineto{\pgfqpoint{1.914298in}{3.170996in}}%
\pgfpathlineto{\pgfqpoint{1.916214in}{3.171285in}}%
\pgfpathlineto{\pgfqpoint{1.918129in}{3.174500in}}%
\pgfpathlineto{\pgfqpoint{1.925790in}{3.176558in}}%
\pgfpathlineto{\pgfqpoint{1.929620in}{3.179441in}}%
\pgfpathlineto{\pgfqpoint{1.931535in}{3.179464in}}%
\pgfpathlineto{\pgfqpoint{1.944941in}{3.187455in}}%
\pgfpathlineto{\pgfqpoint{1.950687in}{3.188479in}}%
\pgfpathlineto{\pgfqpoint{1.952602in}{3.193705in}}%
\pgfpathlineto{\pgfqpoint{1.962178in}{3.195754in}}%
\pgfpathlineto{\pgfqpoint{1.964093in}{3.198962in}}%
\pgfpathlineto{\pgfqpoint{1.977499in}{3.204276in}}%
\pgfpathlineto{\pgfqpoint{1.979414in}{3.204305in}}%
\pgfpathlineto{\pgfqpoint{1.985160in}{3.209828in}}%
\pgfpathlineto{\pgfqpoint{1.992821in}{3.214504in}}%
\pgfpathlineto{\pgfqpoint{1.994736in}{3.215148in}}%
\pgfpathlineto{\pgfqpoint{1.998566in}{3.218609in}}%
\pgfpathlineto{\pgfqpoint{2.006227in}{3.221702in}}%
\pgfpathlineto{\pgfqpoint{2.010057in}{3.224070in}}%
\pgfpathlineto{\pgfqpoint{2.025378in}{3.229125in}}%
\pgfpathlineto{\pgfqpoint{2.031124in}{3.229642in}}%
\pgfpathlineto{\pgfqpoint{2.036869in}{3.235214in}}%
\pgfpathlineto{\pgfqpoint{2.040700in}{3.236358in}}%
\pgfpathlineto{\pgfqpoint{2.044530in}{3.239773in}}%
\pgfpathlineto{\pgfqpoint{2.046445in}{3.240197in}}%
\pgfpathlineto{\pgfqpoint{2.050276in}{3.242662in}}%
\pgfpathlineto{\pgfqpoint{2.056021in}{3.245245in}}%
\pgfpathlineto{\pgfqpoint{2.057936in}{3.245837in}}%
\pgfpathlineto{\pgfqpoint{2.061767in}{3.252721in}}%
\pgfpathlineto{\pgfqpoint{2.071343in}{3.256010in}}%
\pgfpathlineto{\pgfqpoint{2.073258in}{3.259305in}}%
\pgfpathlineto{\pgfqpoint{2.079003in}{3.259920in}}%
\pgfpathlineto{\pgfqpoint{2.080918in}{3.264477in}}%
\pgfpathlineto{\pgfqpoint{2.084749in}{3.265581in}}%
\pgfpathlineto{\pgfqpoint{2.086664in}{3.266107in}}%
\pgfpathlineto{\pgfqpoint{2.094325in}{3.272343in}}%
\pgfpathlineto{\pgfqpoint{2.101985in}{3.276517in}}%
\pgfpathlineto{\pgfqpoint{2.105816in}{3.279411in}}%
\pgfpathlineto{\pgfqpoint{2.109646in}{3.281245in}}%
\pgfpathlineto{\pgfqpoint{2.113476in}{3.283995in}}%
\pgfpathlineto{\pgfqpoint{2.124967in}{3.285280in}}%
\pgfpathlineto{\pgfqpoint{2.128798in}{3.285723in}}%
\pgfpathlineto{\pgfqpoint{2.130713in}{3.294774in}}%
\pgfpathlineto{\pgfqpoint{2.134543in}{3.295650in}}%
\pgfpathlineto{\pgfqpoint{2.136458in}{3.300481in}}%
\pgfpathlineto{\pgfqpoint{2.146034in}{3.305362in}}%
\pgfpathlineto{\pgfqpoint{2.147949in}{3.305895in}}%
\pgfpathlineto{\pgfqpoint{2.151780in}{3.311790in}}%
\pgfpathlineto{\pgfqpoint{2.165186in}{3.314453in}}%
\pgfpathlineto{\pgfqpoint{2.170931in}{3.316633in}}%
\pgfpathlineto{\pgfqpoint{2.178592in}{3.319089in}}%
\pgfpathlineto{\pgfqpoint{2.186253in}{3.320848in}}%
\pgfpathlineto{\pgfqpoint{2.188168in}{3.323997in}}%
\pgfpathlineto{\pgfqpoint{2.190083in}{3.324679in}}%
\pgfpathlineto{\pgfqpoint{2.191998in}{3.326763in}}%
\pgfpathlineto{\pgfqpoint{2.201574in}{3.328688in}}%
\pgfpathlineto{\pgfqpoint{2.205405in}{3.332672in}}%
\pgfpathlineto{\pgfqpoint{2.207320in}{3.336494in}}%
\pgfpathlineto{\pgfqpoint{2.209235in}{3.336735in}}%
\pgfpathlineto{\pgfqpoint{2.213065in}{3.339618in}}%
\pgfpathlineto{\pgfqpoint{2.220726in}{3.341699in}}%
\pgfpathlineto{\pgfqpoint{2.222641in}{3.341762in}}%
\pgfpathlineto{\pgfqpoint{2.224556in}{3.343417in}}%
\pgfpathlineto{\pgfqpoint{2.230302in}{3.344285in}}%
\pgfpathlineto{\pgfqpoint{2.237962in}{3.351093in}}%
\pgfpathlineto{\pgfqpoint{2.249453in}{3.352966in}}%
\pgfpathlineto{\pgfqpoint{2.251369in}{3.360379in}}%
\pgfpathlineto{\pgfqpoint{2.255199in}{3.361537in}}%
\pgfpathlineto{\pgfqpoint{2.257114in}{3.365142in}}%
\pgfpathlineto{\pgfqpoint{2.283927in}{3.373077in}}%
\pgfpathlineto{\pgfqpoint{2.285842in}{3.375570in}}%
\pgfpathlineto{\pgfqpoint{2.287757in}{3.376019in}}%
\pgfpathlineto{\pgfqpoint{2.289672in}{3.378160in}}%
\pgfpathlineto{\pgfqpoint{2.291587in}{3.378319in}}%
\pgfpathlineto{\pgfqpoint{2.297333in}{3.385908in}}%
\pgfpathlineto{\pgfqpoint{2.306909in}{3.391963in}}%
\pgfpathlineto{\pgfqpoint{2.308824in}{3.394232in}}%
\pgfpathlineto{\pgfqpoint{2.312654in}{3.394936in}}%
\pgfpathlineto{\pgfqpoint{2.316484in}{3.397730in}}%
\pgfpathlineto{\pgfqpoint{2.320315in}{3.397926in}}%
\pgfpathlineto{\pgfqpoint{2.322230in}{3.399969in}}%
\pgfpathlineto{\pgfqpoint{2.326060in}{3.401093in}}%
\pgfpathlineto{\pgfqpoint{2.341382in}{3.411025in}}%
\pgfpathlineto{\pgfqpoint{2.345212in}{3.414068in}}%
\pgfpathlineto{\pgfqpoint{2.350958in}{3.415513in}}%
\pgfpathlineto{\pgfqpoint{2.352873in}{3.418343in}}%
\pgfpathlineto{\pgfqpoint{2.354788in}{3.418435in}}%
\pgfpathlineto{\pgfqpoint{2.356703in}{3.424071in}}%
\pgfpathlineto{\pgfqpoint{2.360533in}{3.426317in}}%
\pgfpathlineto{\pgfqpoint{2.366279in}{3.428976in}}%
\pgfpathlineto{\pgfqpoint{2.368194in}{3.433645in}}%
\pgfpathlineto{\pgfqpoint{2.372024in}{3.435639in}}%
\pgfpathlineto{\pgfqpoint{2.373940in}{3.438276in}}%
\pgfpathlineto{\pgfqpoint{2.379685in}{3.439700in}}%
\pgfpathlineto{\pgfqpoint{2.381600in}{3.445003in}}%
\pgfpathlineto{\pgfqpoint{2.383515in}{3.446597in}}%
\pgfpathlineto{\pgfqpoint{2.391176in}{3.446896in}}%
\pgfpathlineto{\pgfqpoint{2.393091in}{3.448848in}}%
\pgfpathlineto{\pgfqpoint{2.395006in}{3.449165in}}%
\pgfpathlineto{\pgfqpoint{2.400752in}{3.457720in}}%
\pgfpathlineto{\pgfqpoint{2.404582in}{3.458510in}}%
\pgfpathlineto{\pgfqpoint{2.408413in}{3.461506in}}%
\pgfpathlineto{\pgfqpoint{2.410328in}{3.466639in}}%
\pgfpathlineto{\pgfqpoint{2.412243in}{3.466673in}}%
\pgfpathlineto{\pgfqpoint{2.416073in}{3.471070in}}%
\pgfpathlineto{\pgfqpoint{2.421819in}{3.474669in}}%
\pgfpathlineto{\pgfqpoint{2.429480in}{3.475917in}}%
\pgfpathlineto{\pgfqpoint{2.431395in}{3.478146in}}%
\pgfpathlineto{\pgfqpoint{2.437140in}{3.478962in}}%
\pgfpathlineto{\pgfqpoint{2.440971in}{3.480410in}}%
\pgfpathlineto{\pgfqpoint{2.442886in}{3.486618in}}%
\pgfpathlineto{\pgfqpoint{2.448631in}{3.487162in}}%
\pgfpathlineto{\pgfqpoint{2.454377in}{3.488771in}}%
\pgfpathlineto{\pgfqpoint{2.456292in}{3.491957in}}%
\pgfpathlineto{\pgfqpoint{2.458207in}{3.492273in}}%
\pgfpathlineto{\pgfqpoint{2.463953in}{3.496553in}}%
\pgfpathlineto{\pgfqpoint{2.485020in}{3.505200in}}%
\pgfpathlineto{\pgfqpoint{2.490765in}{3.506323in}}%
\pgfpathlineto{\pgfqpoint{2.492680in}{3.509834in}}%
\pgfpathlineto{\pgfqpoint{2.496511in}{3.511227in}}%
\pgfpathlineto{\pgfqpoint{2.500341in}{3.511687in}}%
\pgfpathlineto{\pgfqpoint{2.502256in}{3.513475in}}%
\pgfpathlineto{\pgfqpoint{2.504171in}{3.513549in}}%
\pgfpathlineto{\pgfqpoint{2.508002in}{3.515716in}}%
\pgfpathlineto{\pgfqpoint{2.515662in}{3.519878in}}%
\pgfpathlineto{\pgfqpoint{2.517577in}{3.520135in}}%
\pgfpathlineto{\pgfqpoint{2.527153in}{3.528777in}}%
\pgfpathlineto{\pgfqpoint{2.532899in}{3.529188in}}%
\pgfpathlineto{\pgfqpoint{2.534814in}{3.532310in}}%
\pgfpathlineto{\pgfqpoint{2.544390in}{3.535233in}}%
\pgfpathlineto{\pgfqpoint{2.548220in}{3.536298in}}%
\pgfpathlineto{\pgfqpoint{2.550135in}{3.538447in}}%
\pgfpathlineto{\pgfqpoint{2.552051in}{3.542730in}}%
\pgfpathlineto{\pgfqpoint{2.559711in}{3.544460in}}%
\pgfpathlineto{\pgfqpoint{2.565457in}{3.547109in}}%
\pgfpathlineto{\pgfqpoint{2.569287in}{3.548263in}}%
\pgfpathlineto{\pgfqpoint{2.596099in}{3.555414in}}%
\pgfpathlineto{\pgfqpoint{2.598015in}{3.557341in}}%
\pgfpathlineto{\pgfqpoint{2.599930in}{3.557444in}}%
\pgfpathlineto{\pgfqpoint{2.601845in}{3.559738in}}%
\pgfpathlineto{\pgfqpoint{2.605675in}{3.560421in}}%
\pgfpathlineto{\pgfqpoint{2.611421in}{3.566753in}}%
\pgfpathlineto{\pgfqpoint{2.613336in}{3.566788in}}%
\pgfpathlineto{\pgfqpoint{2.619082in}{3.570206in}}%
\pgfpathlineto{\pgfqpoint{2.624827in}{3.572490in}}%
\pgfpathlineto{\pgfqpoint{2.628657in}{3.574957in}}%
\pgfpathlineto{\pgfqpoint{2.630573in}{3.575198in}}%
\pgfpathlineto{\pgfqpoint{2.632488in}{3.579021in}}%
\pgfpathlineto{\pgfqpoint{2.636318in}{3.579726in}}%
\pgfpathlineto{\pgfqpoint{2.643979in}{3.582895in}}%
\pgfpathlineto{\pgfqpoint{2.645894in}{3.583001in}}%
\pgfpathlineto{\pgfqpoint{2.655470in}{3.590706in}}%
\pgfpathlineto{\pgfqpoint{2.663130in}{3.591963in}}%
\pgfpathlineto{\pgfqpoint{2.666961in}{3.593522in}}%
\pgfpathlineto{\pgfqpoint{2.670791in}{3.594222in}}%
\pgfpathlineto{\pgfqpoint{2.672706in}{3.600331in}}%
\pgfpathlineto{\pgfqpoint{2.676537in}{3.601331in}}%
\pgfpathlineto{\pgfqpoint{2.680367in}{3.604160in}}%
\pgfpathlineto{\pgfqpoint{2.689943in}{3.607979in}}%
\pgfpathlineto{\pgfqpoint{2.699519in}{3.615740in}}%
\pgfpathlineto{\pgfqpoint{2.701434in}{3.616244in}}%
\pgfpathlineto{\pgfqpoint{2.703349in}{3.618460in}}%
\pgfpathlineto{\pgfqpoint{2.705264in}{3.618813in}}%
\pgfpathlineto{\pgfqpoint{2.707179in}{3.622203in}}%
\pgfpathlineto{\pgfqpoint{2.720586in}{3.624844in}}%
\pgfpathlineto{\pgfqpoint{2.724416in}{3.625697in}}%
\pgfpathlineto{\pgfqpoint{2.726331in}{3.627314in}}%
\pgfpathlineto{\pgfqpoint{2.730161in}{3.628110in}}%
\pgfpathlineto{\pgfqpoint{2.732077in}{3.631540in}}%
\pgfpathlineto{\pgfqpoint{2.735907in}{3.632166in}}%
\pgfpathlineto{\pgfqpoint{2.739737in}{3.635876in}}%
\pgfpathlineto{\pgfqpoint{2.760804in}{3.642802in}}%
\pgfpathlineto{\pgfqpoint{2.766550in}{3.646950in}}%
\pgfpathlineto{\pgfqpoint{2.774210in}{3.649445in}}%
\pgfpathlineto{\pgfqpoint{2.783786in}{3.651331in}}%
\pgfpathlineto{\pgfqpoint{2.785701in}{3.652268in}}%
\pgfpathlineto{\pgfqpoint{2.787617in}{3.654689in}}%
\pgfpathlineto{\pgfqpoint{2.795277in}{3.656120in}}%
\pgfpathlineto{\pgfqpoint{2.802938in}{3.660341in}}%
\pgfpathlineto{\pgfqpoint{2.808684in}{3.661289in}}%
\pgfpathlineto{\pgfqpoint{2.810599in}{3.663349in}}%
\pgfpathlineto{\pgfqpoint{2.818259in}{3.665324in}}%
\pgfpathlineto{\pgfqpoint{2.829750in}{3.670047in}}%
\pgfpathlineto{\pgfqpoint{2.833581in}{3.670520in}}%
\pgfpathlineto{\pgfqpoint{2.841241in}{3.674142in}}%
\pgfpathlineto{\pgfqpoint{2.845072in}{3.676593in}}%
\pgfpathlineto{\pgfqpoint{2.852732in}{3.678696in}}%
\pgfpathlineto{\pgfqpoint{2.856563in}{3.680610in}}%
\pgfpathlineto{\pgfqpoint{2.866139in}{3.685608in}}%
\pgfpathlineto{\pgfqpoint{2.868054in}{3.688586in}}%
\pgfpathlineto{\pgfqpoint{2.871884in}{3.689520in}}%
\pgfpathlineto{\pgfqpoint{2.879545in}{3.690977in}}%
\pgfpathlineto{\pgfqpoint{2.887206in}{3.696112in}}%
\pgfpathlineto{\pgfqpoint{2.894866in}{3.699386in}}%
\pgfpathlineto{\pgfqpoint{2.896781in}{3.702753in}}%
\pgfpathlineto{\pgfqpoint{2.898697in}{3.702868in}}%
\pgfpathlineto{\pgfqpoint{2.900612in}{3.705602in}}%
\pgfpathlineto{\pgfqpoint{2.904442in}{3.705929in}}%
\pgfpathlineto{\pgfqpoint{2.910188in}{3.712726in}}%
\pgfpathlineto{\pgfqpoint{2.917848in}{3.715744in}}%
\pgfpathlineto{\pgfqpoint{2.919763in}{3.715885in}}%
\pgfpathlineto{\pgfqpoint{2.923594in}{3.718007in}}%
\pgfpathlineto{\pgfqpoint{2.929339in}{3.718875in}}%
\pgfpathlineto{\pgfqpoint{2.944661in}{3.726617in}}%
\pgfpathlineto{\pgfqpoint{2.959982in}{3.731599in}}%
\pgfpathlineto{\pgfqpoint{2.969558in}{3.738795in}}%
\pgfpathlineto{\pgfqpoint{2.971473in}{3.743617in}}%
\pgfpathlineto{\pgfqpoint{2.982964in}{3.746576in}}%
\pgfpathlineto{\pgfqpoint{2.984879in}{3.749191in}}%
\pgfpathlineto{\pgfqpoint{2.986794in}{3.749488in}}%
\pgfpathlineto{\pgfqpoint{2.992540in}{3.753621in}}%
\pgfpathlineto{\pgfqpoint{3.000201in}{3.758013in}}%
\pgfpathlineto{\pgfqpoint{3.002116in}{3.758506in}}%
\pgfpathlineto{\pgfqpoint{3.005946in}{3.763047in}}%
\pgfpathlineto{\pgfqpoint{3.009777in}{3.764333in}}%
\pgfpathlineto{\pgfqpoint{3.011692in}{3.767450in}}%
\pgfpathlineto{\pgfqpoint{3.015522in}{3.769289in}}%
\pgfpathlineto{\pgfqpoint{3.027013in}{3.775356in}}%
\pgfpathlineto{\pgfqpoint{3.038504in}{3.778333in}}%
\pgfpathlineto{\pgfqpoint{3.040419in}{3.780192in}}%
\pgfpathlineto{\pgfqpoint{3.044250in}{3.780786in}}%
\pgfpathlineto{\pgfqpoint{3.049995in}{3.787955in}}%
\pgfpathlineto{\pgfqpoint{3.051910in}{3.788057in}}%
\pgfpathlineto{\pgfqpoint{3.057656in}{3.793507in}}%
\pgfpathlineto{\pgfqpoint{3.061486in}{3.794175in}}%
\pgfpathlineto{\pgfqpoint{3.065316in}{3.795739in}}%
\pgfpathlineto{\pgfqpoint{3.069147in}{3.796091in}}%
\pgfpathlineto{\pgfqpoint{3.071062in}{3.798942in}}%
\pgfpathlineto{\pgfqpoint{3.076808in}{3.801778in}}%
\pgfpathlineto{\pgfqpoint{3.086383in}{3.813053in}}%
\pgfpathlineto{\pgfqpoint{3.088299in}{3.813571in}}%
\pgfpathlineto{\pgfqpoint{3.090214in}{3.816604in}}%
\pgfpathlineto{\pgfqpoint{3.092129in}{3.817273in}}%
\pgfpathlineto{\pgfqpoint{3.094044in}{3.821174in}}%
\pgfpathlineto{\pgfqpoint{3.097874in}{3.822726in}}%
\pgfpathlineto{\pgfqpoint{3.101705in}{3.826091in}}%
\pgfpathlineto{\pgfqpoint{3.105535in}{3.828695in}}%
\pgfpathlineto{\pgfqpoint{3.109365in}{3.829248in}}%
\pgfpathlineto{\pgfqpoint{3.115111in}{3.832807in}}%
\pgfpathlineto{\pgfqpoint{3.117026in}{3.832878in}}%
\pgfpathlineto{\pgfqpoint{3.118941in}{3.836229in}}%
\pgfpathlineto{\pgfqpoint{3.122772in}{3.837847in}}%
\pgfpathlineto{\pgfqpoint{3.124687in}{3.840584in}}%
\pgfpathlineto{\pgfqpoint{3.126602in}{3.840847in}}%
\pgfpathlineto{\pgfqpoint{3.130432in}{3.844295in}}%
\pgfpathlineto{\pgfqpoint{3.132347in}{3.844439in}}%
\pgfpathlineto{\pgfqpoint{3.138093in}{3.850235in}}%
\pgfpathlineto{\pgfqpoint{3.141923in}{3.851499in}}%
\pgfpathlineto{\pgfqpoint{3.143839in}{3.852933in}}%
\pgfpathlineto{\pgfqpoint{3.147669in}{3.853719in}}%
\pgfpathlineto{\pgfqpoint{3.155330in}{3.857557in}}%
\pgfpathlineto{\pgfqpoint{3.157245in}{3.859830in}}%
\pgfpathlineto{\pgfqpoint{3.162990in}{3.861931in}}%
\pgfpathlineto{\pgfqpoint{3.185972in}{3.870998in}}%
\pgfpathlineto{\pgfqpoint{3.187887in}{3.880400in}}%
\pgfpathlineto{\pgfqpoint{3.195548in}{3.887080in}}%
\pgfpathlineto{\pgfqpoint{3.199378in}{3.887854in}}%
\pgfpathlineto{\pgfqpoint{3.201294in}{3.892617in}}%
\pgfpathlineto{\pgfqpoint{3.203209in}{3.893324in}}%
\pgfpathlineto{\pgfqpoint{3.205124in}{3.897965in}}%
\pgfpathlineto{\pgfqpoint{3.207039in}{3.898125in}}%
\pgfpathlineto{\pgfqpoint{3.214700in}{3.903978in}}%
\pgfpathlineto{\pgfqpoint{3.218530in}{3.904073in}}%
\pgfpathlineto{\pgfqpoint{3.222361in}{3.907239in}}%
\pgfpathlineto{\pgfqpoint{3.224276in}{3.907265in}}%
\pgfpathlineto{\pgfqpoint{3.226191in}{3.910559in}}%
\pgfpathlineto{\pgfqpoint{3.237682in}{3.915532in}}%
\pgfpathlineto{\pgfqpoint{3.239597in}{3.915738in}}%
\pgfpathlineto{\pgfqpoint{3.256834in}{3.932919in}}%
\pgfpathlineto{\pgfqpoint{3.272155in}{3.935838in}}%
\pgfpathlineto{\pgfqpoint{3.274070in}{3.936013in}}%
\pgfpathlineto{\pgfqpoint{3.275985in}{3.938014in}}%
\pgfpathlineto{\pgfqpoint{3.279816in}{3.938265in}}%
\pgfpathlineto{\pgfqpoint{3.281731in}{3.943083in}}%
\pgfpathlineto{\pgfqpoint{3.289392in}{3.950369in}}%
\pgfpathlineto{\pgfqpoint{3.291307in}{3.950420in}}%
\pgfpathlineto{\pgfqpoint{3.297052in}{3.961489in}}%
\pgfpathlineto{\pgfqpoint{3.306628in}{3.967825in}}%
\pgfpathlineto{\pgfqpoint{3.308543in}{3.969849in}}%
\pgfpathlineto{\pgfqpoint{3.314289in}{3.970873in}}%
\pgfpathlineto{\pgfqpoint{3.327695in}{3.987802in}}%
\pgfpathlineto{\pgfqpoint{3.329610in}{3.987955in}}%
\pgfpathlineto{\pgfqpoint{3.331525in}{3.997624in}}%
\pgfpathlineto{\pgfqpoint{3.333440in}{3.999225in}}%
\pgfpathlineto{\pgfqpoint{3.337271in}{4.004092in}}%
\pgfpathlineto{\pgfqpoint{3.339186in}{4.004301in}}%
\pgfpathlineto{\pgfqpoint{3.341101in}{4.009123in}}%
\pgfpathlineto{\pgfqpoint{3.346847in}{4.014330in}}%
\pgfpathlineto{\pgfqpoint{3.348762in}{4.017016in}}%
\pgfpathlineto{\pgfqpoint{3.354507in}{4.018475in}}%
\pgfpathlineto{\pgfqpoint{3.358338in}{4.020520in}}%
\pgfpathlineto{\pgfqpoint{3.360253in}{4.026330in}}%
\pgfpathlineto{\pgfqpoint{3.364083in}{4.027061in}}%
\pgfpathlineto{\pgfqpoint{3.383235in}{4.042905in}}%
\pgfpathlineto{\pgfqpoint{3.390896in}{4.047062in}}%
\pgfpathlineto{\pgfqpoint{3.394726in}{4.048451in}}%
\pgfpathlineto{\pgfqpoint{3.406217in}{4.060164in}}%
\pgfpathlineto{\pgfqpoint{3.410047in}{4.060737in}}%
\pgfpathlineto{\pgfqpoint{3.415793in}{4.063715in}}%
\pgfpathlineto{\pgfqpoint{3.417708in}{4.063840in}}%
\pgfpathlineto{\pgfqpoint{3.419623in}{4.066696in}}%
\pgfpathlineto{\pgfqpoint{3.421538in}{4.067322in}}%
\pgfpathlineto{\pgfqpoint{3.429199in}{4.077627in}}%
\pgfpathlineto{\pgfqpoint{3.434945in}{4.086358in}}%
\pgfpathlineto{\pgfqpoint{3.436860in}{4.086585in}}%
\pgfpathlineto{\pgfqpoint{3.438775in}{4.091592in}}%
\pgfpathlineto{\pgfqpoint{3.444520in}{4.095489in}}%
\pgfpathlineto{\pgfqpoint{3.446436in}{4.096226in}}%
\pgfpathlineto{\pgfqpoint{3.448351in}{4.098201in}}%
\pgfpathlineto{\pgfqpoint{3.450266in}{4.102980in}}%
\pgfpathlineto{\pgfqpoint{3.457927in}{4.108962in}}%
\pgfpathlineto{\pgfqpoint{3.459842in}{4.113492in}}%
\pgfpathlineto{\pgfqpoint{3.463672in}{4.114396in}}%
\pgfpathlineto{\pgfqpoint{3.465587in}{4.121147in}}%
\pgfpathlineto{\pgfqpoint{3.469418in}{4.122107in}}%
\pgfpathlineto{\pgfqpoint{3.473248in}{4.122496in}}%
\pgfpathlineto{\pgfqpoint{3.475163in}{4.128033in}}%
\pgfpathlineto{\pgfqpoint{3.478994in}{4.129665in}}%
\pgfpathlineto{\pgfqpoint{3.484739in}{4.135350in}}%
\pgfpathlineto{\pgfqpoint{3.488569in}{4.143243in}}%
\pgfpathlineto{\pgfqpoint{3.490485in}{4.143319in}}%
\pgfpathlineto{\pgfqpoint{3.494315in}{4.145881in}}%
\pgfpathlineto{\pgfqpoint{3.498145in}{4.146623in}}%
\pgfpathlineto{\pgfqpoint{3.500060in}{4.148923in}}%
\pgfpathlineto{\pgfqpoint{3.503891in}{4.149839in}}%
\pgfpathlineto{\pgfqpoint{3.505806in}{4.154730in}}%
\pgfpathlineto{\pgfqpoint{3.509636in}{4.155529in}}%
\pgfpathlineto{\pgfqpoint{3.526873in}{4.164211in}}%
\pgfpathlineto{\pgfqpoint{3.528788in}{4.167122in}}%
\pgfpathlineto{\pgfqpoint{3.530703in}{4.167423in}}%
\pgfpathlineto{\pgfqpoint{3.534533in}{4.169179in}}%
\pgfpathlineto{\pgfqpoint{3.542194in}{4.172031in}}%
\pgfpathlineto{\pgfqpoint{3.546025in}{4.173640in}}%
\pgfpathlineto{\pgfqpoint{3.549855in}{4.181721in}}%
\pgfpathlineto{\pgfqpoint{3.551770in}{4.182049in}}%
\pgfpathlineto{\pgfqpoint{3.555600in}{4.188425in}}%
\pgfpathlineto{\pgfqpoint{3.559431in}{4.192609in}}%
\pgfpathlineto{\pgfqpoint{3.561346in}{4.193166in}}%
\pgfpathlineto{\pgfqpoint{3.563261in}{4.198371in}}%
\pgfpathlineto{\pgfqpoint{3.567091in}{4.201383in}}%
\pgfpathlineto{\pgfqpoint{3.570922in}{4.206794in}}%
\pgfpathlineto{\pgfqpoint{3.576667in}{4.211121in}}%
\pgfpathlineto{\pgfqpoint{3.580498in}{4.219374in}}%
\pgfpathlineto{\pgfqpoint{3.582413in}{4.228109in}}%
\pgfpathlineto{\pgfqpoint{3.582413in}{4.228109in}}%
\pgfusepath{stroke}%
\end{pgfscope}%
\begin{pgfscope}%
\pgfpathrectangle{\pgfqpoint{0.694334in}{2.659974in}}{\pgfqpoint{3.830343in}{1.568135in}}%
\pgfusepath{clip}%
\pgfsetbuttcap%
\pgfsetroundjoin%
\pgfsetlinewidth{1.003750pt}%
\definecolor{currentstroke}{rgb}{0.811765,0.125490,0.125490}%
\pgfsetstrokecolor{currentstroke}%
\pgfsetdash{{1.000000pt}{1.650000pt}}{0.000000pt}%
\pgfpathmoveto{\pgfqpoint{0.694334in}{2.859234in}}%
\pgfpathlineto{\pgfqpoint{0.711571in}{2.860718in}}%
\pgfpathlineto{\pgfqpoint{0.767111in}{2.862276in}}%
\pgfpathlineto{\pgfqpoint{0.780517in}{2.862780in}}%
\pgfpathlineto{\pgfqpoint{0.949052in}{2.866849in}}%
\pgfpathlineto{\pgfqpoint{0.956713in}{2.867321in}}%
\pgfpathlineto{\pgfqpoint{1.052471in}{2.871740in}}%
\pgfpathlineto{\pgfqpoint{1.104181in}{2.874943in}}%
\pgfpathlineto{\pgfqpoint{1.119502in}{2.876471in}}%
\pgfpathlineto{\pgfqpoint{1.293783in}{2.892313in}}%
\pgfpathlineto{\pgfqpoint{1.297613in}{2.893545in}}%
\pgfpathlineto{\pgfqpoint{1.312935in}{2.895075in}}%
\pgfpathlineto{\pgfqpoint{1.316765in}{2.896706in}}%
\pgfpathlineto{\pgfqpoint{1.349323in}{2.899164in}}%
\pgfpathlineto{\pgfqpoint{1.353153in}{2.900293in}}%
\pgfpathlineto{\pgfqpoint{1.381881in}{2.902639in}}%
\pgfpathlineto{\pgfqpoint{1.395287in}{2.903487in}}%
\pgfpathlineto{\pgfqpoint{1.402948in}{2.905410in}}%
\pgfpathlineto{\pgfqpoint{1.445081in}{2.911715in}}%
\pgfpathlineto{\pgfqpoint{1.450827in}{2.913862in}}%
\pgfpathlineto{\pgfqpoint{1.454657in}{2.914703in}}%
\pgfpathlineto{\pgfqpoint{1.512112in}{2.927956in}}%
\pgfpathlineto{\pgfqpoint{1.515943in}{2.929541in}}%
\pgfpathlineto{\pgfqpoint{1.519773in}{2.931220in}}%
\pgfpathlineto{\pgfqpoint{1.525519in}{2.933312in}}%
\pgfpathlineto{\pgfqpoint{1.533179in}{2.934600in}}%
\pgfpathlineto{\pgfqpoint{1.556161in}{2.940730in}}%
\pgfpathlineto{\pgfqpoint{1.563822in}{2.942856in}}%
\pgfpathlineto{\pgfqpoint{1.569568in}{2.943962in}}%
\pgfpathlineto{\pgfqpoint{1.575313in}{2.948526in}}%
\pgfpathlineto{\pgfqpoint{1.582974in}{2.950279in}}%
\pgfpathlineto{\pgfqpoint{1.590635in}{2.954167in}}%
\pgfpathlineto{\pgfqpoint{1.594465in}{2.955567in}}%
\pgfpathlineto{\pgfqpoint{1.602126in}{2.956891in}}%
\pgfpathlineto{\pgfqpoint{1.604041in}{2.959006in}}%
\pgfpathlineto{\pgfqpoint{1.617447in}{2.960827in}}%
\pgfpathlineto{\pgfqpoint{1.619362in}{2.965034in}}%
\pgfpathlineto{\pgfqpoint{1.623192in}{2.967237in}}%
\pgfpathlineto{\pgfqpoint{1.627023in}{2.969580in}}%
\pgfpathlineto{\pgfqpoint{1.638514in}{2.972087in}}%
\pgfpathlineto{\pgfqpoint{1.642344in}{2.975506in}}%
\pgfpathlineto{\pgfqpoint{1.653835in}{2.980178in}}%
\pgfpathlineto{\pgfqpoint{1.661496in}{2.983609in}}%
\pgfpathlineto{\pgfqpoint{1.665326in}{2.985863in}}%
\pgfpathlineto{\pgfqpoint{1.671072in}{2.986498in}}%
\pgfpathlineto{\pgfqpoint{1.678732in}{2.990558in}}%
\pgfpathlineto{\pgfqpoint{1.682563in}{2.991914in}}%
\pgfpathlineto{\pgfqpoint{1.688308in}{2.992986in}}%
\pgfpathlineto{\pgfqpoint{1.730442in}{3.011929in}}%
\pgfpathlineto{\pgfqpoint{1.768745in}{3.025976in}}%
\pgfpathlineto{\pgfqpoint{1.772576in}{3.027016in}}%
\pgfpathlineto{\pgfqpoint{1.774491in}{3.027490in}}%
\pgfpathlineto{\pgfqpoint{1.776406in}{3.030963in}}%
\pgfpathlineto{\pgfqpoint{1.782152in}{3.031772in}}%
\pgfpathlineto{\pgfqpoint{1.784067in}{3.031946in}}%
\pgfpathlineto{\pgfqpoint{1.793643in}{3.042594in}}%
\pgfpathlineto{\pgfqpoint{1.801303in}{3.043906in}}%
\pgfpathlineto{\pgfqpoint{1.807049in}{3.048633in}}%
\pgfpathlineto{\pgfqpoint{1.812794in}{3.049111in}}%
\pgfpathlineto{\pgfqpoint{1.822370in}{3.054653in}}%
\pgfpathlineto{\pgfqpoint{1.824285in}{3.056781in}}%
\pgfpathlineto{\pgfqpoint{1.828116in}{3.057866in}}%
\pgfpathlineto{\pgfqpoint{1.831946in}{3.060459in}}%
\pgfpathlineto{\pgfqpoint{1.837692in}{3.061016in}}%
\pgfpathlineto{\pgfqpoint{1.841522in}{3.063546in}}%
\pgfpathlineto{\pgfqpoint{1.845352in}{3.065538in}}%
\pgfpathlineto{\pgfqpoint{1.860674in}{3.070777in}}%
\pgfpathlineto{\pgfqpoint{1.866419in}{3.071312in}}%
\pgfpathlineto{\pgfqpoint{1.868334in}{3.073065in}}%
\pgfpathlineto{\pgfqpoint{1.872165in}{3.079437in}}%
\pgfpathlineto{\pgfqpoint{1.883656in}{3.084600in}}%
\pgfpathlineto{\pgfqpoint{1.885571in}{3.087622in}}%
\pgfpathlineto{\pgfqpoint{1.889401in}{3.088389in}}%
\pgfpathlineto{\pgfqpoint{1.898977in}{3.092834in}}%
\pgfpathlineto{\pgfqpoint{1.900892in}{3.097085in}}%
\pgfpathlineto{\pgfqpoint{1.908553in}{3.100856in}}%
\pgfpathlineto{\pgfqpoint{1.910468in}{3.100998in}}%
\pgfpathlineto{\pgfqpoint{1.912383in}{3.103377in}}%
\pgfpathlineto{\pgfqpoint{1.916214in}{3.104528in}}%
\pgfpathlineto{\pgfqpoint{1.918129in}{3.104774in}}%
\pgfpathlineto{\pgfqpoint{1.923874in}{3.109307in}}%
\pgfpathlineto{\pgfqpoint{1.931535in}{3.111650in}}%
\pgfpathlineto{\pgfqpoint{1.933450in}{3.114729in}}%
\pgfpathlineto{\pgfqpoint{1.943026in}{3.116055in}}%
\pgfpathlineto{\pgfqpoint{1.952602in}{3.119043in}}%
\pgfpathlineto{\pgfqpoint{1.960263in}{3.120455in}}%
\pgfpathlineto{\pgfqpoint{1.964093in}{3.121420in}}%
\pgfpathlineto{\pgfqpoint{1.967923in}{3.124174in}}%
\pgfpathlineto{\pgfqpoint{1.969838in}{3.127731in}}%
\pgfpathlineto{\pgfqpoint{1.973669in}{3.128906in}}%
\pgfpathlineto{\pgfqpoint{1.981329in}{3.133307in}}%
\pgfpathlineto{\pgfqpoint{1.987075in}{3.135440in}}%
\pgfpathlineto{\pgfqpoint{1.994736in}{3.136181in}}%
\pgfpathlineto{\pgfqpoint{1.998566in}{3.141483in}}%
\pgfpathlineto{\pgfqpoint{2.008142in}{3.144835in}}%
\pgfpathlineto{\pgfqpoint{2.011972in}{3.147627in}}%
\pgfpathlineto{\pgfqpoint{2.015803in}{3.148933in}}%
\pgfpathlineto{\pgfqpoint{2.017718in}{3.150957in}}%
\pgfpathlineto{\pgfqpoint{2.021548in}{3.152387in}}%
\pgfpathlineto{\pgfqpoint{2.025378in}{3.155628in}}%
\pgfpathlineto{\pgfqpoint{2.029209in}{3.156783in}}%
\pgfpathlineto{\pgfqpoint{2.031124in}{3.156859in}}%
\pgfpathlineto{\pgfqpoint{2.033039in}{3.163792in}}%
\pgfpathlineto{\pgfqpoint{2.034954in}{3.164665in}}%
\pgfpathlineto{\pgfqpoint{2.038785in}{3.169029in}}%
\pgfpathlineto{\pgfqpoint{2.044530in}{3.170971in}}%
\pgfpathlineto{\pgfqpoint{2.048360in}{3.173298in}}%
\pgfpathlineto{\pgfqpoint{2.054106in}{3.180235in}}%
\pgfpathlineto{\pgfqpoint{2.069427in}{3.186291in}}%
\pgfpathlineto{\pgfqpoint{2.073258in}{3.187524in}}%
\pgfpathlineto{\pgfqpoint{2.080918in}{3.189687in}}%
\pgfpathlineto{\pgfqpoint{2.082834in}{3.194239in}}%
\pgfpathlineto{\pgfqpoint{2.084749in}{3.194808in}}%
\pgfpathlineto{\pgfqpoint{2.090494in}{3.200257in}}%
\pgfpathlineto{\pgfqpoint{2.094325in}{3.200943in}}%
\pgfpathlineto{\pgfqpoint{2.098155in}{3.209526in}}%
\pgfpathlineto{\pgfqpoint{2.103900in}{3.210990in}}%
\pgfpathlineto{\pgfqpoint{2.111561in}{3.215014in}}%
\pgfpathlineto{\pgfqpoint{2.113476in}{3.217042in}}%
\pgfpathlineto{\pgfqpoint{2.115391in}{3.221868in}}%
\pgfpathlineto{\pgfqpoint{2.119222in}{3.223053in}}%
\pgfpathlineto{\pgfqpoint{2.121137in}{3.223765in}}%
\pgfpathlineto{\pgfqpoint{2.123052in}{3.226230in}}%
\pgfpathlineto{\pgfqpoint{2.124967in}{3.226616in}}%
\pgfpathlineto{\pgfqpoint{2.128798in}{3.230122in}}%
\pgfpathlineto{\pgfqpoint{2.132628in}{3.231462in}}%
\pgfpathlineto{\pgfqpoint{2.136458in}{3.241418in}}%
\pgfpathlineto{\pgfqpoint{2.142204in}{3.243465in}}%
\pgfpathlineto{\pgfqpoint{2.144119in}{3.246235in}}%
\pgfpathlineto{\pgfqpoint{2.147949in}{3.248504in}}%
\pgfpathlineto{\pgfqpoint{2.149865in}{3.252936in}}%
\pgfpathlineto{\pgfqpoint{2.151780in}{3.253206in}}%
\pgfpathlineto{\pgfqpoint{2.159440in}{3.259674in}}%
\pgfpathlineto{\pgfqpoint{2.161356in}{3.262777in}}%
\pgfpathlineto{\pgfqpoint{2.169016in}{3.266577in}}%
\pgfpathlineto{\pgfqpoint{2.170931in}{3.268516in}}%
\pgfpathlineto{\pgfqpoint{2.176677in}{3.270109in}}%
\pgfpathlineto{\pgfqpoint{2.178592in}{3.274737in}}%
\pgfpathlineto{\pgfqpoint{2.184338in}{3.277175in}}%
\pgfpathlineto{\pgfqpoint{2.188168in}{3.278145in}}%
\pgfpathlineto{\pgfqpoint{2.190083in}{3.278217in}}%
\pgfpathlineto{\pgfqpoint{2.191998in}{3.281207in}}%
\pgfpathlineto{\pgfqpoint{2.193914in}{3.281638in}}%
\pgfpathlineto{\pgfqpoint{2.197744in}{3.284516in}}%
\pgfpathlineto{\pgfqpoint{2.201574in}{3.285073in}}%
\pgfpathlineto{\pgfqpoint{2.203489in}{3.287360in}}%
\pgfpathlineto{\pgfqpoint{2.207320in}{3.287650in}}%
\pgfpathlineto{\pgfqpoint{2.209235in}{3.291476in}}%
\pgfpathlineto{\pgfqpoint{2.213065in}{3.293874in}}%
\pgfpathlineto{\pgfqpoint{2.218811in}{3.297083in}}%
\pgfpathlineto{\pgfqpoint{2.220726in}{3.297800in}}%
\pgfpathlineto{\pgfqpoint{2.222641in}{3.301298in}}%
\pgfpathlineto{\pgfqpoint{2.228387in}{3.303339in}}%
\pgfpathlineto{\pgfqpoint{2.232217in}{3.308573in}}%
\pgfpathlineto{\pgfqpoint{2.243708in}{3.313696in}}%
\pgfpathlineto{\pgfqpoint{2.245623in}{3.316350in}}%
\pgfpathlineto{\pgfqpoint{2.247538in}{3.324116in}}%
\pgfpathlineto{\pgfqpoint{2.251369in}{3.327388in}}%
\pgfpathlineto{\pgfqpoint{2.253284in}{3.327868in}}%
\pgfpathlineto{\pgfqpoint{2.257114in}{3.331609in}}%
\pgfpathlineto{\pgfqpoint{2.259029in}{3.333436in}}%
\pgfpathlineto{\pgfqpoint{2.260945in}{3.333634in}}%
\pgfpathlineto{\pgfqpoint{2.262860in}{3.335343in}}%
\pgfpathlineto{\pgfqpoint{2.264775in}{3.339214in}}%
\pgfpathlineto{\pgfqpoint{2.268605in}{3.340653in}}%
\pgfpathlineto{\pgfqpoint{2.282011in}{3.351121in}}%
\pgfpathlineto{\pgfqpoint{2.285842in}{3.354793in}}%
\pgfpathlineto{\pgfqpoint{2.287757in}{3.354995in}}%
\pgfpathlineto{\pgfqpoint{2.293502in}{3.359425in}}%
\pgfpathlineto{\pgfqpoint{2.299248in}{3.361396in}}%
\pgfpathlineto{\pgfqpoint{2.301163in}{3.365177in}}%
\pgfpathlineto{\pgfqpoint{2.304993in}{3.365684in}}%
\pgfpathlineto{\pgfqpoint{2.308824in}{3.370091in}}%
\pgfpathlineto{\pgfqpoint{2.310739in}{3.376447in}}%
\pgfpathlineto{\pgfqpoint{2.316484in}{3.377594in}}%
\pgfpathlineto{\pgfqpoint{2.322230in}{3.383186in}}%
\pgfpathlineto{\pgfqpoint{2.324145in}{3.383362in}}%
\pgfpathlineto{\pgfqpoint{2.326060in}{3.385533in}}%
\pgfpathlineto{\pgfqpoint{2.327975in}{3.391410in}}%
\pgfpathlineto{\pgfqpoint{2.329891in}{3.391611in}}%
\pgfpathlineto{\pgfqpoint{2.331806in}{3.394644in}}%
\pgfpathlineto{\pgfqpoint{2.333721in}{3.395303in}}%
\pgfpathlineto{\pgfqpoint{2.337551in}{3.400705in}}%
\pgfpathlineto{\pgfqpoint{2.347127in}{3.406172in}}%
\pgfpathlineto{\pgfqpoint{2.350958in}{3.408289in}}%
\pgfpathlineto{\pgfqpoint{2.352873in}{3.413443in}}%
\pgfpathlineto{\pgfqpoint{2.356703in}{3.418385in}}%
\pgfpathlineto{\pgfqpoint{2.358618in}{3.425333in}}%
\pgfpathlineto{\pgfqpoint{2.360533in}{3.426386in}}%
\pgfpathlineto{\pgfqpoint{2.366279in}{3.434106in}}%
\pgfpathlineto{\pgfqpoint{2.370109in}{3.436975in}}%
\pgfpathlineto{\pgfqpoint{2.373940in}{3.440956in}}%
\pgfpathlineto{\pgfqpoint{2.375855in}{3.441495in}}%
\pgfpathlineto{\pgfqpoint{2.379685in}{3.449588in}}%
\pgfpathlineto{\pgfqpoint{2.383515in}{3.449883in}}%
\pgfpathlineto{\pgfqpoint{2.385431in}{3.452151in}}%
\pgfpathlineto{\pgfqpoint{2.393091in}{3.453854in}}%
\pgfpathlineto{\pgfqpoint{2.395006in}{3.457874in}}%
\pgfpathlineto{\pgfqpoint{2.396922in}{3.457909in}}%
\pgfpathlineto{\pgfqpoint{2.400752in}{3.459589in}}%
\pgfpathlineto{\pgfqpoint{2.402667in}{3.459638in}}%
\pgfpathlineto{\pgfqpoint{2.406498in}{3.466085in}}%
\pgfpathlineto{\pgfqpoint{2.417989in}{3.471467in}}%
\pgfpathlineto{\pgfqpoint{2.419904in}{3.473339in}}%
\pgfpathlineto{\pgfqpoint{2.421819in}{3.473465in}}%
\pgfpathlineto{\pgfqpoint{2.423734in}{3.478716in}}%
\pgfpathlineto{\pgfqpoint{2.425649in}{3.478875in}}%
\pgfpathlineto{\pgfqpoint{2.427564in}{3.481540in}}%
\pgfpathlineto{\pgfqpoint{2.429480in}{3.481882in}}%
\pgfpathlineto{\pgfqpoint{2.433310in}{3.488785in}}%
\pgfpathlineto{\pgfqpoint{2.439055in}{3.493012in}}%
\pgfpathlineto{\pgfqpoint{2.446716in}{3.497313in}}%
\pgfpathlineto{\pgfqpoint{2.448631in}{3.505909in}}%
\pgfpathlineto{\pgfqpoint{2.452462in}{3.507389in}}%
\pgfpathlineto{\pgfqpoint{2.454377in}{3.507803in}}%
\pgfpathlineto{\pgfqpoint{2.458207in}{3.512825in}}%
\pgfpathlineto{\pgfqpoint{2.460122in}{3.516688in}}%
\pgfpathlineto{\pgfqpoint{2.463953in}{3.518145in}}%
\pgfpathlineto{\pgfqpoint{2.467783in}{3.520256in}}%
\pgfpathlineto{\pgfqpoint{2.469698in}{3.522409in}}%
\pgfpathlineto{\pgfqpoint{2.473529in}{3.523002in}}%
\pgfpathlineto{\pgfqpoint{2.477359in}{3.527507in}}%
\pgfpathlineto{\pgfqpoint{2.479274in}{3.528948in}}%
\pgfpathlineto{\pgfqpoint{2.483104in}{3.539542in}}%
\pgfpathlineto{\pgfqpoint{2.492680in}{3.547160in}}%
\pgfpathlineto{\pgfqpoint{2.496511in}{3.548508in}}%
\pgfpathlineto{\pgfqpoint{2.498426in}{3.552786in}}%
\pgfpathlineto{\pgfqpoint{2.500341in}{3.553706in}}%
\pgfpathlineto{\pgfqpoint{2.504171in}{3.563542in}}%
\pgfpathlineto{\pgfqpoint{2.511832in}{3.571785in}}%
\pgfpathlineto{\pgfqpoint{2.513747in}{3.575910in}}%
\pgfpathlineto{\pgfqpoint{2.515662in}{3.576954in}}%
\pgfpathlineto{\pgfqpoint{2.517577in}{3.580347in}}%
\pgfpathlineto{\pgfqpoint{2.519493in}{3.581304in}}%
\pgfpathlineto{\pgfqpoint{2.525238in}{3.588000in}}%
\pgfpathlineto{\pgfqpoint{2.532899in}{3.589145in}}%
\pgfpathlineto{\pgfqpoint{2.534814in}{3.589433in}}%
\pgfpathlineto{\pgfqpoint{2.536729in}{3.591172in}}%
\pgfpathlineto{\pgfqpoint{2.538644in}{3.594423in}}%
\pgfpathlineto{\pgfqpoint{2.544390in}{3.597488in}}%
\pgfpathlineto{\pgfqpoint{2.548220in}{3.603102in}}%
\pgfpathlineto{\pgfqpoint{2.550135in}{3.603323in}}%
\pgfpathlineto{\pgfqpoint{2.552051in}{3.608455in}}%
\pgfpathlineto{\pgfqpoint{2.557796in}{3.610836in}}%
\pgfpathlineto{\pgfqpoint{2.561626in}{3.615575in}}%
\pgfpathlineto{\pgfqpoint{2.565457in}{3.616969in}}%
\pgfpathlineto{\pgfqpoint{2.567372in}{3.621312in}}%
\pgfpathlineto{\pgfqpoint{2.571202in}{3.622118in}}%
\pgfpathlineto{\pgfqpoint{2.573117in}{3.627253in}}%
\pgfpathlineto{\pgfqpoint{2.584608in}{3.635074in}}%
\pgfpathlineto{\pgfqpoint{2.586524in}{3.635536in}}%
\pgfpathlineto{\pgfqpoint{2.592269in}{3.641493in}}%
\pgfpathlineto{\pgfqpoint{2.594184in}{3.642601in}}%
\pgfpathlineto{\pgfqpoint{2.596099in}{3.645533in}}%
\pgfpathlineto{\pgfqpoint{2.598015in}{3.645766in}}%
\pgfpathlineto{\pgfqpoint{2.601845in}{3.649264in}}%
\pgfpathlineto{\pgfqpoint{2.603760in}{3.650023in}}%
\pgfpathlineto{\pgfqpoint{2.605675in}{3.654255in}}%
\pgfpathlineto{\pgfqpoint{2.609506in}{3.654418in}}%
\pgfpathlineto{\pgfqpoint{2.613336in}{3.657340in}}%
\pgfpathlineto{\pgfqpoint{2.617166in}{3.658879in}}%
\pgfpathlineto{\pgfqpoint{2.619082in}{3.663004in}}%
\pgfpathlineto{\pgfqpoint{2.622912in}{3.664137in}}%
\pgfpathlineto{\pgfqpoint{2.624827in}{3.664157in}}%
\pgfpathlineto{\pgfqpoint{2.626742in}{3.667091in}}%
\pgfpathlineto{\pgfqpoint{2.630573in}{3.669009in}}%
\pgfpathlineto{\pgfqpoint{2.643979in}{3.676074in}}%
\pgfpathlineto{\pgfqpoint{2.651639in}{3.677757in}}%
\pgfpathlineto{\pgfqpoint{2.653555in}{3.679043in}}%
\pgfpathlineto{\pgfqpoint{2.655470in}{3.685933in}}%
\pgfpathlineto{\pgfqpoint{2.657385in}{3.686560in}}%
\pgfpathlineto{\pgfqpoint{2.659300in}{3.689757in}}%
\pgfpathlineto{\pgfqpoint{2.663130in}{3.690578in}}%
\pgfpathlineto{\pgfqpoint{2.665046in}{3.691704in}}%
\pgfpathlineto{\pgfqpoint{2.666961in}{3.694989in}}%
\pgfpathlineto{\pgfqpoint{2.668876in}{3.695233in}}%
\pgfpathlineto{\pgfqpoint{2.674622in}{3.701301in}}%
\pgfpathlineto{\pgfqpoint{2.678452in}{3.703540in}}%
\pgfpathlineto{\pgfqpoint{2.680367in}{3.706209in}}%
\pgfpathlineto{\pgfqpoint{2.684197in}{3.707992in}}%
\pgfpathlineto{\pgfqpoint{2.689943in}{3.712029in}}%
\pgfpathlineto{\pgfqpoint{2.699519in}{3.714701in}}%
\pgfpathlineto{\pgfqpoint{2.703349in}{3.718606in}}%
\pgfpathlineto{\pgfqpoint{2.705264in}{3.719258in}}%
\pgfpathlineto{\pgfqpoint{2.707179in}{3.724784in}}%
\pgfpathlineto{\pgfqpoint{2.711010in}{3.726162in}}%
\pgfpathlineto{\pgfqpoint{2.722501in}{3.736031in}}%
\pgfpathlineto{\pgfqpoint{2.724416in}{3.740156in}}%
\pgfpathlineto{\pgfqpoint{2.728246in}{3.740408in}}%
\pgfpathlineto{\pgfqpoint{2.732077in}{3.744885in}}%
\pgfpathlineto{\pgfqpoint{2.733992in}{3.745803in}}%
\pgfpathlineto{\pgfqpoint{2.735907in}{3.748878in}}%
\pgfpathlineto{\pgfqpoint{2.737822in}{3.749212in}}%
\pgfpathlineto{\pgfqpoint{2.739737in}{3.756843in}}%
\pgfpathlineto{\pgfqpoint{2.747398in}{3.760614in}}%
\pgfpathlineto{\pgfqpoint{2.751228in}{3.762075in}}%
\pgfpathlineto{\pgfqpoint{2.758889in}{3.765751in}}%
\pgfpathlineto{\pgfqpoint{2.760804in}{3.765873in}}%
\pgfpathlineto{\pgfqpoint{2.764635in}{3.767218in}}%
\pgfpathlineto{\pgfqpoint{2.768465in}{3.768163in}}%
\pgfpathlineto{\pgfqpoint{2.789532in}{3.777535in}}%
\pgfpathlineto{\pgfqpoint{2.793362in}{3.780580in}}%
\pgfpathlineto{\pgfqpoint{2.797192in}{3.783184in}}%
\pgfpathlineto{\pgfqpoint{2.799108in}{3.786184in}}%
\pgfpathlineto{\pgfqpoint{2.804853in}{3.787177in}}%
\pgfpathlineto{\pgfqpoint{2.806768in}{3.792366in}}%
\pgfpathlineto{\pgfqpoint{2.812514in}{3.796173in}}%
\pgfpathlineto{\pgfqpoint{2.822090in}{3.799004in}}%
\pgfpathlineto{\pgfqpoint{2.829750in}{3.799937in}}%
\pgfpathlineto{\pgfqpoint{2.835496in}{3.803266in}}%
\pgfpathlineto{\pgfqpoint{2.837411in}{3.806869in}}%
\pgfpathlineto{\pgfqpoint{2.843157in}{3.808962in}}%
\pgfpathlineto{\pgfqpoint{2.845072in}{3.809002in}}%
\pgfpathlineto{\pgfqpoint{2.848902in}{3.811167in}}%
\pgfpathlineto{\pgfqpoint{2.856563in}{3.813953in}}%
\pgfpathlineto{\pgfqpoint{2.864223in}{3.816687in}}%
\pgfpathlineto{\pgfqpoint{2.875715in}{3.824742in}}%
\pgfpathlineto{\pgfqpoint{2.877630in}{3.825065in}}%
\pgfpathlineto{\pgfqpoint{2.881460in}{3.828702in}}%
\pgfpathlineto{\pgfqpoint{2.883375in}{3.829361in}}%
\pgfpathlineto{\pgfqpoint{2.887206in}{3.833306in}}%
\pgfpathlineto{\pgfqpoint{2.892951in}{3.834128in}}%
\pgfpathlineto{\pgfqpoint{2.894866in}{3.836459in}}%
\pgfpathlineto{\pgfqpoint{2.896781in}{3.836725in}}%
\pgfpathlineto{\pgfqpoint{2.902527in}{3.842980in}}%
\pgfpathlineto{\pgfqpoint{2.935085in}{3.857024in}}%
\pgfpathlineto{\pgfqpoint{2.937000in}{3.861411in}}%
\pgfpathlineto{\pgfqpoint{2.942746in}{3.863373in}}%
\pgfpathlineto{\pgfqpoint{2.948491in}{3.865452in}}%
\pgfpathlineto{\pgfqpoint{2.963812in}{3.870346in}}%
\pgfpathlineto{\pgfqpoint{2.969558in}{3.875638in}}%
\pgfpathlineto{\pgfqpoint{2.971473in}{3.875685in}}%
\pgfpathlineto{\pgfqpoint{2.975303in}{3.879876in}}%
\pgfpathlineto{\pgfqpoint{2.982964in}{3.880829in}}%
\pgfpathlineto{\pgfqpoint{2.986794in}{3.885813in}}%
\pgfpathlineto{\pgfqpoint{2.990625in}{3.886394in}}%
\pgfpathlineto{\pgfqpoint{2.992540in}{3.887387in}}%
\pgfpathlineto{\pgfqpoint{2.998285in}{3.894969in}}%
\pgfpathlineto{\pgfqpoint{3.000201in}{3.894970in}}%
\pgfpathlineto{\pgfqpoint{3.002116in}{3.899019in}}%
\pgfpathlineto{\pgfqpoint{3.005946in}{3.899399in}}%
\pgfpathlineto{\pgfqpoint{3.009777in}{3.905354in}}%
\pgfpathlineto{\pgfqpoint{3.015522in}{3.912295in}}%
\pgfpathlineto{\pgfqpoint{3.017437in}{3.912455in}}%
\pgfpathlineto{\pgfqpoint{3.021268in}{3.915732in}}%
\pgfpathlineto{\pgfqpoint{3.044250in}{3.930117in}}%
\pgfpathlineto{\pgfqpoint{3.051910in}{3.932413in}}%
\pgfpathlineto{\pgfqpoint{3.053825in}{3.933508in}}%
\pgfpathlineto{\pgfqpoint{3.055741in}{3.939308in}}%
\pgfpathlineto{\pgfqpoint{3.059571in}{3.940463in}}%
\pgfpathlineto{\pgfqpoint{3.061486in}{3.947076in}}%
\pgfpathlineto{\pgfqpoint{3.071062in}{3.950956in}}%
\pgfpathlineto{\pgfqpoint{3.074892in}{3.954541in}}%
\pgfpathlineto{\pgfqpoint{3.076808in}{3.954990in}}%
\pgfpathlineto{\pgfqpoint{3.080638in}{3.960838in}}%
\pgfpathlineto{\pgfqpoint{3.082553in}{3.963171in}}%
\pgfpathlineto{\pgfqpoint{3.086383in}{3.964138in}}%
\pgfpathlineto{\pgfqpoint{3.088299in}{3.965850in}}%
\pgfpathlineto{\pgfqpoint{3.092129in}{3.971682in}}%
\pgfpathlineto{\pgfqpoint{3.099790in}{3.974525in}}%
\pgfpathlineto{\pgfqpoint{3.103620in}{3.976581in}}%
\pgfpathlineto{\pgfqpoint{3.105535in}{3.980012in}}%
\pgfpathlineto{\pgfqpoint{3.109365in}{3.981171in}}%
\pgfpathlineto{\pgfqpoint{3.111281in}{3.984022in}}%
\pgfpathlineto{\pgfqpoint{3.117026in}{3.984802in}}%
\pgfpathlineto{\pgfqpoint{3.118941in}{3.988787in}}%
\pgfpathlineto{\pgfqpoint{3.120856in}{3.989106in}}%
\pgfpathlineto{\pgfqpoint{3.122772in}{3.991805in}}%
\pgfpathlineto{\pgfqpoint{3.126602in}{3.999159in}}%
\pgfpathlineto{\pgfqpoint{3.132347in}{3.999442in}}%
\pgfpathlineto{\pgfqpoint{3.138093in}{4.004454in}}%
\pgfpathlineto{\pgfqpoint{3.141923in}{4.004725in}}%
\pgfpathlineto{\pgfqpoint{3.145754in}{4.011133in}}%
\pgfpathlineto{\pgfqpoint{3.147669in}{4.013018in}}%
\pgfpathlineto{\pgfqpoint{3.151499in}{4.019565in}}%
\pgfpathlineto{\pgfqpoint{3.159160in}{4.024387in}}%
\pgfpathlineto{\pgfqpoint{3.164905in}{4.033562in}}%
\pgfpathlineto{\pgfqpoint{3.170651in}{4.035086in}}%
\pgfpathlineto{\pgfqpoint{3.178312in}{4.046236in}}%
\pgfpathlineto{\pgfqpoint{3.182142in}{4.046499in}}%
\pgfpathlineto{\pgfqpoint{3.184057in}{4.053911in}}%
\pgfpathlineto{\pgfqpoint{3.185972in}{4.056443in}}%
\pgfpathlineto{\pgfqpoint{3.187887in}{4.056547in}}%
\pgfpathlineto{\pgfqpoint{3.189803in}{4.057838in}}%
\pgfpathlineto{\pgfqpoint{3.193633in}{4.062953in}}%
\pgfpathlineto{\pgfqpoint{3.199378in}{4.066481in}}%
\pgfpathlineto{\pgfqpoint{3.205124in}{4.070888in}}%
\pgfpathlineto{\pgfqpoint{3.210870in}{4.085262in}}%
\pgfpathlineto{\pgfqpoint{3.212785in}{4.088422in}}%
\pgfpathlineto{\pgfqpoint{3.214700in}{4.089436in}}%
\pgfpathlineto{\pgfqpoint{3.218530in}{4.093937in}}%
\pgfpathlineto{\pgfqpoint{3.220445in}{4.101265in}}%
\pgfpathlineto{\pgfqpoint{3.224276in}{4.104174in}}%
\pgfpathlineto{\pgfqpoint{3.226191in}{4.107476in}}%
\pgfpathlineto{\pgfqpoint{3.228106in}{4.115401in}}%
\pgfpathlineto{\pgfqpoint{3.231936in}{4.118082in}}%
\pgfpathlineto{\pgfqpoint{3.233852in}{4.119208in}}%
\pgfpathlineto{\pgfqpoint{3.239597in}{4.136048in}}%
\pgfpathlineto{\pgfqpoint{3.241512in}{4.137456in}}%
\pgfpathlineto{\pgfqpoint{3.245343in}{4.143670in}}%
\pgfpathlineto{\pgfqpoint{3.249173in}{4.144525in}}%
\pgfpathlineto{\pgfqpoint{3.253003in}{4.152410in}}%
\pgfpathlineto{\pgfqpoint{3.254918in}{4.159540in}}%
\pgfpathlineto{\pgfqpoint{3.260664in}{4.163067in}}%
\pgfpathlineto{\pgfqpoint{3.262579in}{4.169175in}}%
\pgfpathlineto{\pgfqpoint{3.266409in}{4.171535in}}%
\pgfpathlineto{\pgfqpoint{3.272155in}{4.175688in}}%
\pgfpathlineto{\pgfqpoint{3.275985in}{4.176388in}}%
\pgfpathlineto{\pgfqpoint{3.277901in}{4.179872in}}%
\pgfpathlineto{\pgfqpoint{3.279816in}{4.187814in}}%
\pgfpathlineto{\pgfqpoint{3.281731in}{4.189191in}}%
\pgfpathlineto{\pgfqpoint{3.283646in}{4.193398in}}%
\pgfpathlineto{\pgfqpoint{3.287476in}{4.194525in}}%
\pgfpathlineto{\pgfqpoint{3.293222in}{4.197706in}}%
\pgfpathlineto{\pgfqpoint{3.295137in}{4.203725in}}%
\pgfpathlineto{\pgfqpoint{3.297052in}{4.205065in}}%
\pgfpathlineto{\pgfqpoint{3.302798in}{4.218505in}}%
\pgfpathlineto{\pgfqpoint{3.304713in}{4.228109in}}%
\pgfpathlineto{\pgfqpoint{3.304713in}{4.228109in}}%
\pgfusepath{stroke}%
\end{pgfscope}%
\begin{pgfscope}%
\pgfpathrectangle{\pgfqpoint{0.694334in}{2.659974in}}{\pgfqpoint{3.830343in}{1.568135in}}%
\pgfusepath{clip}%
\pgfsetrectcap%
\pgfsetroundjoin%
\pgfsetlinewidth{1.003750pt}%
\definecolor{currentstroke}{rgb}{0.062745,0.000000,0.062745}%
\pgfsetstrokecolor{currentstroke}%
\pgfsetdash{}{0pt}%
\pgfpathmoveto{\pgfqpoint{0.694334in}{2.703201in}}%
\pgfpathlineto{\pgfqpoint{0.696249in}{2.720356in}}%
\pgfpathlineto{\pgfqpoint{0.715401in}{2.720356in}}%
\pgfpathlineto{\pgfqpoint{0.717316in}{2.735488in}}%
\pgfpathlineto{\pgfqpoint{0.721147in}{2.735488in}}%
\pgfpathlineto{\pgfqpoint{0.723062in}{2.749024in}}%
\pgfpathlineto{\pgfqpoint{0.724977in}{2.749024in}}%
\pgfpathlineto{\pgfqpoint{0.726892in}{2.761268in}}%
\pgfpathlineto{\pgfqpoint{0.732638in}{2.761268in}}%
\pgfpathlineto{\pgfqpoint{0.734553in}{2.772447in}}%
\pgfpathlineto{\pgfqpoint{0.738383in}{2.772447in}}%
\pgfpathlineto{\pgfqpoint{0.740298in}{2.782730in}}%
\pgfpathlineto{\pgfqpoint{0.744129in}{2.782730in}}%
\pgfpathlineto{\pgfqpoint{0.746044in}{2.792251in}}%
\pgfpathlineto{\pgfqpoint{0.747959in}{2.817195in}}%
\pgfpathlineto{\pgfqpoint{0.749874in}{2.817195in}}%
\pgfpathlineto{\pgfqpoint{0.753705in}{2.831484in}}%
\pgfpathlineto{\pgfqpoint{0.776687in}{2.831484in}}%
\pgfpathlineto{\pgfqpoint{0.778602in}{2.838074in}}%
\pgfpathlineto{\pgfqpoint{0.855209in}{2.838074in}}%
\pgfpathlineto{\pgfqpoint{0.857124in}{2.844342in}}%
\pgfpathlineto{\pgfqpoint{0.908834in}{2.844342in}}%
\pgfpathlineto{\pgfqpoint{0.910749in}{2.850318in}}%
\pgfpathlineto{\pgfqpoint{0.956713in}{2.850318in}}%
\pgfpathlineto{\pgfqpoint{0.958628in}{2.856029in}}%
\pgfpathlineto{\pgfqpoint{0.998847in}{2.856029in}}%
\pgfpathlineto{\pgfqpoint{1.000762in}{2.861497in}}%
\pgfpathlineto{\pgfqpoint{1.019913in}{2.861497in}}%
\pgfpathlineto{\pgfqpoint{1.021829in}{2.866741in}}%
\pgfpathlineto{\pgfqpoint{1.083114in}{2.866741in}}%
\pgfpathlineto{\pgfqpoint{1.085029in}{2.871780in}}%
\pgfpathlineto{\pgfqpoint{1.169297in}{2.871780in}}%
\pgfpathlineto{\pgfqpoint{1.171212in}{2.876629in}}%
\pgfpathlineto{\pgfqpoint{1.215261in}{2.876629in}}%
\pgfpathlineto{\pgfqpoint{1.217176in}{2.881301in}}%
\pgfpathlineto{\pgfqpoint{1.249734in}{2.881301in}}%
\pgfpathlineto{\pgfqpoint{1.251649in}{2.885809in}}%
\pgfpathlineto{\pgfqpoint{1.291868in}{2.885809in}}%
\pgfpathlineto{\pgfqpoint{1.293783in}{2.890164in}}%
\pgfpathlineto{\pgfqpoint{1.332086in}{2.890164in}}%
\pgfpathlineto{\pgfqpoint{1.334002in}{2.894377in}}%
\pgfpathlineto{\pgfqpoint{1.391457in}{2.894377in}}%
\pgfpathlineto{\pgfqpoint{1.393372in}{2.898456in}}%
\pgfpathlineto{\pgfqpoint{1.448912in}{2.898456in}}%
\pgfpathlineto{\pgfqpoint{1.450827in}{2.902409in}}%
\pgfpathlineto{\pgfqpoint{1.504452in}{2.902409in}}%
\pgfpathlineto{\pgfqpoint{1.506367in}{2.906244in}}%
\pgfpathlineto{\pgfqpoint{1.533179in}{2.906244in}}%
\pgfpathlineto{\pgfqpoint{1.535095in}{2.909968in}}%
\pgfpathlineto{\pgfqpoint{1.559992in}{2.909968in}}%
\pgfpathlineto{\pgfqpoint{1.561907in}{2.913588in}}%
\pgfpathlineto{\pgfqpoint{1.588719in}{2.913588in}}%
\pgfpathlineto{\pgfqpoint{1.590635in}{2.917108in}}%
\pgfpathlineto{\pgfqpoint{1.619362in}{2.917108in}}%
\pgfpathlineto{\pgfqpoint{1.621277in}{2.920534in}}%
\pgfpathlineto{\pgfqpoint{1.634683in}{2.920534in}}%
\pgfpathlineto{\pgfqpoint{1.636599in}{2.923871in}}%
\pgfpathlineto{\pgfqpoint{1.642344in}{2.923871in}}%
\pgfpathlineto{\pgfqpoint{1.644259in}{2.927123in}}%
\pgfpathlineto{\pgfqpoint{1.646174in}{2.927123in}}%
\pgfpathlineto{\pgfqpoint{1.650005in}{2.933392in}}%
\pgfpathlineto{\pgfqpoint{1.653835in}{2.933392in}}%
\pgfpathlineto{\pgfqpoint{1.655750in}{2.936415in}}%
\pgfpathlineto{\pgfqpoint{1.659581in}{2.936415in}}%
\pgfpathlineto{\pgfqpoint{1.663411in}{2.942255in}}%
\pgfpathlineto{\pgfqpoint{1.667241in}{2.942255in}}%
\pgfpathlineto{\pgfqpoint{1.671072in}{2.947842in}}%
\pgfpathlineto{\pgfqpoint{1.672987in}{2.947842in}}%
\pgfpathlineto{\pgfqpoint{1.674902in}{2.950547in}}%
\pgfpathlineto{\pgfqpoint{1.676817in}{2.950547in}}%
\pgfpathlineto{\pgfqpoint{1.678732in}{2.953196in}}%
\pgfpathlineto{\pgfqpoint{1.680648in}{2.953196in}}%
\pgfpathlineto{\pgfqpoint{1.682563in}{2.955791in}}%
\pgfpathlineto{\pgfqpoint{1.686393in}{2.963277in}}%
\pgfpathlineto{\pgfqpoint{1.692139in}{2.963277in}}%
\pgfpathlineto{\pgfqpoint{1.694054in}{2.965678in}}%
\pgfpathlineto{\pgfqpoint{1.701714in}{2.965678in}}%
\pgfpathlineto{\pgfqpoint{1.703630in}{2.968036in}}%
\pgfpathlineto{\pgfqpoint{1.747679in}{2.968036in}}%
\pgfpathlineto{\pgfqpoint{1.749594in}{2.970351in}}%
\pgfpathlineto{\pgfqpoint{1.774491in}{2.970351in}}%
\pgfpathlineto{\pgfqpoint{1.776406in}{2.972624in}}%
\pgfpathlineto{\pgfqpoint{1.814710in}{2.972624in}}%
\pgfpathlineto{\pgfqpoint{1.816625in}{2.974859in}}%
\pgfpathlineto{\pgfqpoint{1.843437in}{2.974859in}}%
\pgfpathlineto{\pgfqpoint{1.845352in}{2.977055in}}%
\pgfpathlineto{\pgfqpoint{1.877910in}{2.977055in}}%
\pgfpathlineto{\pgfqpoint{1.879825in}{2.979214in}}%
\pgfpathlineto{\pgfqpoint{1.908553in}{2.979214in}}%
\pgfpathlineto{\pgfqpoint{1.910468in}{2.981338in}}%
\pgfpathlineto{\pgfqpoint{1.931535in}{2.981338in}}%
\pgfpathlineto{\pgfqpoint{1.933450in}{2.983427in}}%
\pgfpathlineto{\pgfqpoint{1.964093in}{2.983427in}}%
\pgfpathlineto{\pgfqpoint{1.966008in}{2.985482in}}%
\pgfpathlineto{\pgfqpoint{1.985160in}{2.985482in}}%
\pgfpathlineto{\pgfqpoint{1.987075in}{2.987506in}}%
\pgfpathlineto{\pgfqpoint{1.992821in}{2.987506in}}%
\pgfpathlineto{\pgfqpoint{1.994736in}{2.989497in}}%
\pgfpathlineto{\pgfqpoint{2.011972in}{2.989497in}}%
\pgfpathlineto{\pgfqpoint{2.013887in}{2.991459in}}%
\pgfpathlineto{\pgfqpoint{2.017718in}{2.991459in}}%
\pgfpathlineto{\pgfqpoint{2.019633in}{2.993391in}}%
\pgfpathlineto{\pgfqpoint{2.027294in}{2.993391in}}%
\pgfpathlineto{\pgfqpoint{2.029209in}{2.995294in}}%
\pgfpathlineto{\pgfqpoint{2.038785in}{2.995294in}}%
\pgfpathlineto{\pgfqpoint{2.040700in}{2.997170in}}%
\pgfpathlineto{\pgfqpoint{2.044530in}{2.997170in}}%
\pgfpathlineto{\pgfqpoint{2.046445in}{2.999018in}}%
\pgfpathlineto{\pgfqpoint{2.050276in}{2.999018in}}%
\pgfpathlineto{\pgfqpoint{2.052191in}{3.000841in}}%
\pgfpathlineto{\pgfqpoint{2.054106in}{3.000841in}}%
\pgfpathlineto{\pgfqpoint{2.056021in}{3.002637in}}%
\pgfpathlineto{\pgfqpoint{2.063682in}{3.002637in}}%
\pgfpathlineto{\pgfqpoint{2.065597in}{3.004409in}}%
\pgfpathlineto{\pgfqpoint{2.073258in}{3.004409in}}%
\pgfpathlineto{\pgfqpoint{2.075173in}{3.006157in}}%
\pgfpathlineto{\pgfqpoint{2.080918in}{3.006157in}}%
\pgfpathlineto{\pgfqpoint{2.082834in}{3.007882in}}%
\pgfpathlineto{\pgfqpoint{2.088579in}{3.007882in}}%
\pgfpathlineto{\pgfqpoint{2.090494in}{3.009583in}}%
\pgfpathlineto{\pgfqpoint{2.094325in}{3.009583in}}%
\pgfpathlineto{\pgfqpoint{2.096240in}{3.011263in}}%
\pgfpathlineto{\pgfqpoint{2.101985in}{3.011263in}}%
\pgfpathlineto{\pgfqpoint{2.103900in}{3.012921in}}%
\pgfpathlineto{\pgfqpoint{2.105816in}{3.012921in}}%
\pgfpathlineto{\pgfqpoint{2.107731in}{3.014557in}}%
\pgfpathlineto{\pgfqpoint{2.109646in}{3.014557in}}%
\pgfpathlineto{\pgfqpoint{2.111561in}{3.016173in}}%
\pgfpathlineto{\pgfqpoint{2.115391in}{3.016173in}}%
\pgfpathlineto{\pgfqpoint{2.117307in}{3.017769in}}%
\pgfpathlineto{\pgfqpoint{2.119222in}{3.017769in}}%
\pgfpathlineto{\pgfqpoint{2.121137in}{3.019345in}}%
\pgfpathlineto{\pgfqpoint{2.123052in}{3.019345in}}%
\pgfpathlineto{\pgfqpoint{2.130713in}{3.032725in}}%
\pgfpathlineto{\pgfqpoint{2.132628in}{3.038251in}}%
\pgfpathlineto{\pgfqpoint{2.136458in}{3.039596in}}%
\pgfpathlineto{\pgfqpoint{2.138374in}{3.039596in}}%
\pgfpathlineto{\pgfqpoint{2.144119in}{3.043550in}}%
\pgfpathlineto{\pgfqpoint{2.146034in}{3.049880in}}%
\pgfpathlineto{\pgfqpoint{2.149865in}{3.049880in}}%
\pgfpathlineto{\pgfqpoint{2.151780in}{3.053533in}}%
\pgfpathlineto{\pgfqpoint{2.157525in}{3.057085in}}%
\pgfpathlineto{\pgfqpoint{2.159440in}{3.061674in}}%
\pgfpathlineto{\pgfqpoint{2.161356in}{3.061674in}}%
\pgfpathlineto{\pgfqpoint{2.176677in}{3.070388in}}%
\pgfpathlineto{\pgfqpoint{2.182422in}{3.071436in}}%
\pgfpathlineto{\pgfqpoint{2.190083in}{3.074532in}}%
\pgfpathlineto{\pgfqpoint{2.191998in}{3.074532in}}%
\pgfpathlineto{\pgfqpoint{2.203489in}{3.090792in}}%
\pgfpathlineto{\pgfqpoint{2.209235in}{3.092576in}}%
\pgfpathlineto{\pgfqpoint{2.211150in}{3.092576in}}%
\pgfpathlineto{\pgfqpoint{2.214980in}{3.096072in}}%
\pgfpathlineto{\pgfqpoint{2.220726in}{3.097785in}}%
\pgfpathlineto{\pgfqpoint{2.226471in}{3.097785in}}%
\pgfpathlineto{\pgfqpoint{2.230302in}{3.107610in}}%
\pgfpathlineto{\pgfqpoint{2.232217in}{3.107610in}}%
\pgfpathlineto{\pgfqpoint{2.239878in}{3.115999in}}%
\pgfpathlineto{\pgfqpoint{2.241793in}{3.115999in}}%
\pgfpathlineto{\pgfqpoint{2.243708in}{3.118919in}}%
\pgfpathlineto{\pgfqpoint{2.251369in}{3.121774in}}%
\pgfpathlineto{\pgfqpoint{2.255199in}{3.124567in}}%
\pgfpathlineto{\pgfqpoint{2.257114in}{3.125256in}}%
\pgfpathlineto{\pgfqpoint{2.260945in}{3.127975in}}%
\pgfpathlineto{\pgfqpoint{2.270520in}{3.130638in}}%
\pgfpathlineto{\pgfqpoint{2.278181in}{3.140769in}}%
\pgfpathlineto{\pgfqpoint{2.282011in}{3.141377in}}%
\pgfpathlineto{\pgfqpoint{2.285842in}{3.141981in}}%
\pgfpathlineto{\pgfqpoint{2.287757in}{3.149022in}}%
\pgfpathlineto{\pgfqpoint{2.289672in}{3.149592in}}%
\pgfpathlineto{\pgfqpoint{2.293502in}{3.152958in}}%
\pgfpathlineto{\pgfqpoint{2.295418in}{3.154609in}}%
\pgfpathlineto{\pgfqpoint{2.301163in}{3.155155in}}%
\pgfpathlineto{\pgfqpoint{2.304993in}{3.157314in}}%
\pgfpathlineto{\pgfqpoint{2.308824in}{3.157848in}}%
\pgfpathlineto{\pgfqpoint{2.314569in}{3.159963in}}%
\pgfpathlineto{\pgfqpoint{2.318400in}{3.162043in}}%
\pgfpathlineto{\pgfqpoint{2.320315in}{3.165102in}}%
\pgfpathlineto{\pgfqpoint{2.322230in}{3.165605in}}%
\pgfpathlineto{\pgfqpoint{2.326060in}{3.168090in}}%
\pgfpathlineto{\pgfqpoint{2.333721in}{3.172921in}}%
\pgfpathlineto{\pgfqpoint{2.337551in}{3.172921in}}%
\pgfpathlineto{\pgfqpoint{2.341382in}{3.175269in}}%
\pgfpathlineto{\pgfqpoint{2.347127in}{3.177576in}}%
\pgfpathlineto{\pgfqpoint{2.358618in}{3.182068in}}%
\pgfpathlineto{\pgfqpoint{2.360533in}{3.185552in}}%
\pgfpathlineto{\pgfqpoint{2.364364in}{3.185981in}}%
\pgfpathlineto{\pgfqpoint{2.366279in}{3.189362in}}%
\pgfpathlineto{\pgfqpoint{2.368194in}{3.189362in}}%
\pgfpathlineto{\pgfqpoint{2.370109in}{3.191431in}}%
\pgfpathlineto{\pgfqpoint{2.372024in}{3.191841in}}%
\pgfpathlineto{\pgfqpoint{2.379685in}{3.199774in}}%
\pgfpathlineto{\pgfqpoint{2.383515in}{3.200541in}}%
\pgfpathlineto{\pgfqpoint{2.385431in}{3.200541in}}%
\pgfpathlineto{\pgfqpoint{2.389261in}{3.204679in}}%
\pgfpathlineto{\pgfqpoint{2.391176in}{3.205049in}}%
\pgfpathlineto{\pgfqpoint{2.393091in}{3.207608in}}%
\pgfpathlineto{\pgfqpoint{2.395006in}{3.207969in}}%
\pgfpathlineto{\pgfqpoint{2.396922in}{3.212577in}}%
\pgfpathlineto{\pgfqpoint{2.398837in}{3.212577in}}%
\pgfpathlineto{\pgfqpoint{2.406498in}{3.219358in}}%
\pgfpathlineto{\pgfqpoint{2.412243in}{3.220672in}}%
\pgfpathlineto{\pgfqpoint{2.417989in}{3.224219in}}%
\pgfpathlineto{\pgfqpoint{2.423734in}{3.225169in}}%
\pgfpathlineto{\pgfqpoint{2.427564in}{3.227049in}}%
\pgfpathlineto{\pgfqpoint{2.429480in}{3.227049in}}%
\pgfpathlineto{\pgfqpoint{2.433310in}{3.229819in}}%
\pgfpathlineto{\pgfqpoint{2.440971in}{3.232530in}}%
\pgfpathlineto{\pgfqpoint{2.442886in}{3.234893in}}%
\pgfpathlineto{\pgfqpoint{2.452462in}{3.238072in}}%
\pgfpathlineto{\pgfqpoint{2.454377in}{3.240055in}}%
\pgfpathlineto{\pgfqpoint{2.458207in}{3.240896in}}%
\pgfpathlineto{\pgfqpoint{2.460122in}{3.243111in}}%
\pgfpathlineto{\pgfqpoint{2.465868in}{3.243385in}}%
\pgfpathlineto{\pgfqpoint{2.473529in}{3.247164in}}%
\pgfpathlineto{\pgfqpoint{2.477359in}{3.248487in}}%
\pgfpathlineto{\pgfqpoint{2.481189in}{3.249797in}}%
\pgfpathlineto{\pgfqpoint{2.485020in}{3.249797in}}%
\pgfpathlineto{\pgfqpoint{2.486935in}{3.255655in}}%
\pgfpathlineto{\pgfqpoint{2.490765in}{3.256894in}}%
\pgfpathlineto{\pgfqpoint{2.492680in}{3.257386in}}%
\pgfpathlineto{\pgfqpoint{2.496511in}{3.261257in}}%
\pgfpathlineto{\pgfqpoint{2.500341in}{3.262443in}}%
\pgfpathlineto{\pgfqpoint{2.506086in}{3.267764in}}%
\pgfpathlineto{\pgfqpoint{2.508002in}{3.268216in}}%
\pgfpathlineto{\pgfqpoint{2.509917in}{3.270010in}}%
\pgfpathlineto{\pgfqpoint{2.511832in}{3.270010in}}%
\pgfpathlineto{\pgfqpoint{2.513747in}{3.272872in}}%
\pgfpathlineto{\pgfqpoint{2.515662in}{3.273307in}}%
\pgfpathlineto{\pgfqpoint{2.517577in}{3.276310in}}%
\pgfpathlineto{\pgfqpoint{2.529068in}{3.280686in}}%
\pgfpathlineto{\pgfqpoint{2.536729in}{3.281707in}}%
\pgfpathlineto{\pgfqpoint{2.544390in}{3.289399in}}%
\pgfpathlineto{\pgfqpoint{2.553966in}{3.292987in}}%
\pgfpathlineto{\pgfqpoint{2.555881in}{3.295932in}}%
\pgfpathlineto{\pgfqpoint{2.565457in}{3.298633in}}%
\pgfpathlineto{\pgfqpoint{2.567372in}{3.302839in}}%
\pgfpathlineto{\pgfqpoint{2.571202in}{3.303527in}}%
\pgfpathlineto{\pgfqpoint{2.575033in}{3.306411in}}%
\pgfpathlineto{\pgfqpoint{2.582693in}{3.309558in}}%
\pgfpathlineto{\pgfqpoint{2.584608in}{3.310374in}}%
\pgfpathlineto{\pgfqpoint{2.588439in}{3.315006in}}%
\pgfpathlineto{\pgfqpoint{2.590354in}{3.315006in}}%
\pgfpathlineto{\pgfqpoint{2.598015in}{3.319930in}}%
\pgfpathlineto{\pgfqpoint{2.601845in}{3.320682in}}%
\pgfpathlineto{\pgfqpoint{2.609506in}{3.322619in}}%
\pgfpathlineto{\pgfqpoint{2.620997in}{3.329386in}}%
\pgfpathlineto{\pgfqpoint{2.622912in}{3.329526in}}%
\pgfpathlineto{\pgfqpoint{2.624827in}{3.334742in}}%
\pgfpathlineto{\pgfqpoint{2.628657in}{3.336612in}}%
\pgfpathlineto{\pgfqpoint{2.630573in}{3.337273in}}%
\pgfpathlineto{\pgfqpoint{2.634403in}{3.341426in}}%
\pgfpathlineto{\pgfqpoint{2.647809in}{3.343955in}}%
\pgfpathlineto{\pgfqpoint{2.653555in}{3.348507in}}%
\pgfpathlineto{\pgfqpoint{2.663130in}{3.352317in}}%
\pgfpathlineto{\pgfqpoint{2.668876in}{3.357266in}}%
\pgfpathlineto{\pgfqpoint{2.672706in}{3.357491in}}%
\pgfpathlineto{\pgfqpoint{2.676537in}{3.359947in}}%
\pgfpathlineto{\pgfqpoint{2.688028in}{3.363006in}}%
\pgfpathlineto{\pgfqpoint{2.693773in}{3.364722in}}%
\pgfpathlineto{\pgfqpoint{2.695688in}{3.366835in}}%
\pgfpathlineto{\pgfqpoint{2.699519in}{3.367149in}}%
\pgfpathlineto{\pgfqpoint{2.711010in}{3.376617in}}%
\pgfpathlineto{\pgfqpoint{2.712925in}{3.380538in}}%
\pgfpathlineto{\pgfqpoint{2.720586in}{3.384799in}}%
\pgfpathlineto{\pgfqpoint{2.722501in}{3.386699in}}%
\pgfpathlineto{\pgfqpoint{2.728246in}{3.387236in}}%
\pgfpathlineto{\pgfqpoint{2.730161in}{3.389891in}}%
\pgfpathlineto{\pgfqpoint{2.735907in}{3.392062in}}%
\pgfpathlineto{\pgfqpoint{2.739737in}{3.394027in}}%
\pgfpathlineto{\pgfqpoint{2.741653in}{3.394366in}}%
\pgfpathlineto{\pgfqpoint{2.745483in}{3.398034in}}%
\pgfpathlineto{\pgfqpoint{2.753144in}{3.403820in}}%
\pgfpathlineto{\pgfqpoint{2.756974in}{3.414592in}}%
\pgfpathlineto{\pgfqpoint{2.762719in}{3.417944in}}%
\pgfpathlineto{\pgfqpoint{2.764635in}{3.418225in}}%
\pgfpathlineto{\pgfqpoint{2.772295in}{3.428481in}}%
\pgfpathlineto{\pgfqpoint{2.774210in}{3.429322in}}%
\pgfpathlineto{\pgfqpoint{2.776126in}{3.433193in}}%
\pgfpathlineto{\pgfqpoint{2.778041in}{3.433505in}}%
\pgfpathlineto{\pgfqpoint{2.783786in}{3.444611in}}%
\pgfpathlineto{\pgfqpoint{2.787617in}{3.445636in}}%
\pgfpathlineto{\pgfqpoint{2.789532in}{3.452756in}}%
\pgfpathlineto{\pgfqpoint{2.791447in}{3.452863in}}%
\pgfpathlineto{\pgfqpoint{2.797192in}{3.457705in}}%
\pgfpathlineto{\pgfqpoint{2.799108in}{3.458221in}}%
\pgfpathlineto{\pgfqpoint{2.801023in}{3.461723in}}%
\pgfpathlineto{\pgfqpoint{2.804853in}{3.463611in}}%
\pgfpathlineto{\pgfqpoint{2.810599in}{3.468072in}}%
\pgfpathlineto{\pgfqpoint{2.812514in}{3.482482in}}%
\pgfpathlineto{\pgfqpoint{2.814429in}{3.487043in}}%
\pgfpathlineto{\pgfqpoint{2.818259in}{3.488311in}}%
\pgfpathlineto{\pgfqpoint{2.820175in}{3.489325in}}%
\pgfpathlineto{\pgfqpoint{2.822090in}{3.492121in}}%
\pgfpathlineto{\pgfqpoint{2.827835in}{3.493419in}}%
\pgfpathlineto{\pgfqpoint{2.831666in}{3.496549in}}%
\pgfpathlineto{\pgfqpoint{2.839326in}{3.498519in}}%
\pgfpathlineto{\pgfqpoint{2.843157in}{3.503858in}}%
\pgfpathlineto{\pgfqpoint{2.845072in}{3.504542in}}%
\pgfpathlineto{\pgfqpoint{2.846987in}{3.507345in}}%
\pgfpathlineto{\pgfqpoint{2.848902in}{3.507661in}}%
\pgfpathlineto{\pgfqpoint{2.850817in}{3.513111in}}%
\pgfpathlineto{\pgfqpoint{2.854648in}{3.515307in}}%
\pgfpathlineto{\pgfqpoint{2.856563in}{3.517044in}}%
\pgfpathlineto{\pgfqpoint{2.858478in}{3.517044in}}%
\pgfpathlineto{\pgfqpoint{2.862308in}{3.518629in}}%
\pgfpathlineto{\pgfqpoint{2.864223in}{3.519046in}}%
\pgfpathlineto{\pgfqpoint{2.866139in}{3.522368in}}%
\pgfpathlineto{\pgfqpoint{2.868054in}{3.522991in}}%
\pgfpathlineto{\pgfqpoint{2.871884in}{3.526878in}}%
\pgfpathlineto{\pgfqpoint{2.873799in}{3.529327in}}%
\pgfpathlineto{\pgfqpoint{2.875715in}{3.534800in}}%
\pgfpathlineto{\pgfqpoint{2.877630in}{3.535900in}}%
\pgfpathlineto{\pgfqpoint{2.879545in}{3.542395in}}%
\pgfpathlineto{\pgfqpoint{2.883375in}{3.543697in}}%
\pgfpathlineto{\pgfqpoint{2.887206in}{3.546238in}}%
\pgfpathlineto{\pgfqpoint{2.889121in}{3.548398in}}%
\pgfpathlineto{\pgfqpoint{2.892951in}{3.549464in}}%
\pgfpathlineto{\pgfqpoint{2.896781in}{3.553301in}}%
\pgfpathlineto{\pgfqpoint{2.898697in}{3.553913in}}%
\pgfpathlineto{\pgfqpoint{2.900612in}{3.556477in}}%
\pgfpathlineto{\pgfqpoint{2.902527in}{3.556621in}}%
\pgfpathlineto{\pgfqpoint{2.910188in}{3.562580in}}%
\pgfpathlineto{\pgfqpoint{2.919763in}{3.566271in}}%
\pgfpathlineto{\pgfqpoint{2.923594in}{3.567464in}}%
\pgfpathlineto{\pgfqpoint{2.925509in}{3.573895in}}%
\pgfpathlineto{\pgfqpoint{2.927424in}{3.574417in}}%
\pgfpathlineto{\pgfqpoint{2.929339in}{3.581131in}}%
\pgfpathlineto{\pgfqpoint{2.933170in}{3.584967in}}%
\pgfpathlineto{\pgfqpoint{2.938915in}{3.587230in}}%
\pgfpathlineto{\pgfqpoint{2.940830in}{3.587832in}}%
\pgfpathlineto{\pgfqpoint{2.944661in}{3.590946in}}%
\pgfpathlineto{\pgfqpoint{2.948491in}{3.592710in}}%
\pgfpathlineto{\pgfqpoint{2.950406in}{3.594965in}}%
\pgfpathlineto{\pgfqpoint{2.952321in}{3.595302in}}%
\pgfpathlineto{\pgfqpoint{2.958067in}{3.599841in}}%
\pgfpathlineto{\pgfqpoint{2.969558in}{3.603579in}}%
\pgfpathlineto{\pgfqpoint{2.971473in}{3.605866in}}%
\pgfpathlineto{\pgfqpoint{2.975303in}{3.612413in}}%
\pgfpathlineto{\pgfqpoint{2.977219in}{3.616048in}}%
\pgfpathlineto{\pgfqpoint{2.982964in}{3.617292in}}%
\pgfpathlineto{\pgfqpoint{3.000201in}{3.623110in}}%
\pgfpathlineto{\pgfqpoint{3.004031in}{3.629262in}}%
\pgfpathlineto{\pgfqpoint{3.005946in}{3.629560in}}%
\pgfpathlineto{\pgfqpoint{3.007861in}{3.632548in}}%
\pgfpathlineto{\pgfqpoint{3.009777in}{3.632760in}}%
\pgfpathlineto{\pgfqpoint{3.011692in}{3.634937in}}%
\pgfpathlineto{\pgfqpoint{3.013607in}{3.640010in}}%
\pgfpathlineto{\pgfqpoint{3.027013in}{3.652630in}}%
\pgfpathlineto{\pgfqpoint{3.028928in}{3.656777in}}%
\pgfpathlineto{\pgfqpoint{3.030843in}{3.657455in}}%
\pgfpathlineto{\pgfqpoint{3.032759in}{3.662851in}}%
\pgfpathlineto{\pgfqpoint{3.034674in}{3.664669in}}%
\pgfpathlineto{\pgfqpoint{3.036589in}{3.664731in}}%
\pgfpathlineto{\pgfqpoint{3.040419in}{3.666400in}}%
\pgfpathlineto{\pgfqpoint{3.044250in}{3.667313in}}%
\pgfpathlineto{\pgfqpoint{3.048080in}{3.671058in}}%
\pgfpathlineto{\pgfqpoint{3.057656in}{3.674544in}}%
\pgfpathlineto{\pgfqpoint{3.059571in}{3.674544in}}%
\pgfpathlineto{\pgfqpoint{3.061486in}{3.676900in}}%
\pgfpathlineto{\pgfqpoint{3.063401in}{3.681162in}}%
\pgfpathlineto{\pgfqpoint{3.065316in}{3.681416in}}%
\pgfpathlineto{\pgfqpoint{3.069147in}{3.688703in}}%
\pgfpathlineto{\pgfqpoint{3.074892in}{3.689886in}}%
\pgfpathlineto{\pgfqpoint{3.076808in}{3.691671in}}%
\pgfpathlineto{\pgfqpoint{3.082553in}{3.702196in}}%
\pgfpathlineto{\pgfqpoint{3.090214in}{3.710185in}}%
\pgfpathlineto{\pgfqpoint{3.094044in}{3.719239in}}%
\pgfpathlineto{\pgfqpoint{3.095959in}{3.719326in}}%
\pgfpathlineto{\pgfqpoint{3.097874in}{3.721631in}}%
\pgfpathlineto{\pgfqpoint{3.101705in}{3.723348in}}%
\pgfpathlineto{\pgfqpoint{3.103620in}{3.725403in}}%
\pgfpathlineto{\pgfqpoint{3.105535in}{3.729820in}}%
\pgfpathlineto{\pgfqpoint{3.113196in}{3.731290in}}%
\pgfpathlineto{\pgfqpoint{3.115111in}{3.733844in}}%
\pgfpathlineto{\pgfqpoint{3.118941in}{3.742268in}}%
\pgfpathlineto{\pgfqpoint{3.124687in}{3.747283in}}%
\pgfpathlineto{\pgfqpoint{3.130432in}{3.756181in}}%
\pgfpathlineto{\pgfqpoint{3.134263in}{3.757244in}}%
\pgfpathlineto{\pgfqpoint{3.141923in}{3.767238in}}%
\pgfpathlineto{\pgfqpoint{3.143839in}{3.773927in}}%
\pgfpathlineto{\pgfqpoint{3.147669in}{3.777829in}}%
\pgfpathlineto{\pgfqpoint{3.149584in}{3.782225in}}%
\pgfpathlineto{\pgfqpoint{3.151499in}{3.783790in}}%
\pgfpathlineto{\pgfqpoint{3.153414in}{3.788293in}}%
\pgfpathlineto{\pgfqpoint{3.159160in}{3.790443in}}%
\pgfpathlineto{\pgfqpoint{3.162990in}{3.796495in}}%
\pgfpathlineto{\pgfqpoint{3.164905in}{3.796631in}}%
\pgfpathlineto{\pgfqpoint{3.168736in}{3.808147in}}%
\pgfpathlineto{\pgfqpoint{3.170651in}{3.814892in}}%
\pgfpathlineto{\pgfqpoint{3.172566in}{3.815135in}}%
\pgfpathlineto{\pgfqpoint{3.178312in}{3.819672in}}%
\pgfpathlineto{\pgfqpoint{3.182142in}{3.820533in}}%
\pgfpathlineto{\pgfqpoint{3.189803in}{3.834871in}}%
\pgfpathlineto{\pgfqpoint{3.193633in}{3.835855in}}%
\pgfpathlineto{\pgfqpoint{3.197463in}{3.837955in}}%
\pgfpathlineto{\pgfqpoint{3.201294in}{3.850274in}}%
\pgfpathlineto{\pgfqpoint{3.207039in}{3.854521in}}%
\pgfpathlineto{\pgfqpoint{3.212785in}{3.869350in}}%
\pgfpathlineto{\pgfqpoint{3.214700in}{3.871887in}}%
\pgfpathlineto{\pgfqpoint{3.218530in}{3.879843in}}%
\pgfpathlineto{\pgfqpoint{3.220445in}{3.883177in}}%
\pgfpathlineto{\pgfqpoint{3.224276in}{3.884263in}}%
\pgfpathlineto{\pgfqpoint{3.226191in}{3.898465in}}%
\pgfpathlineto{\pgfqpoint{3.228106in}{3.898537in}}%
\pgfpathlineto{\pgfqpoint{3.230021in}{3.905956in}}%
\pgfpathlineto{\pgfqpoint{3.239597in}{3.910399in}}%
\pgfpathlineto{\pgfqpoint{3.241512in}{3.914844in}}%
\pgfpathlineto{\pgfqpoint{3.243427in}{3.922704in}}%
\pgfpathlineto{\pgfqpoint{3.249173in}{3.926900in}}%
\pgfpathlineto{\pgfqpoint{3.251088in}{3.937140in}}%
\pgfpathlineto{\pgfqpoint{3.253003in}{3.941148in}}%
\pgfpathlineto{\pgfqpoint{3.256834in}{3.942987in}}%
\pgfpathlineto{\pgfqpoint{3.258749in}{3.946363in}}%
\pgfpathlineto{\pgfqpoint{3.262579in}{3.954496in}}%
\pgfpathlineto{\pgfqpoint{3.264494in}{3.956525in}}%
\pgfpathlineto{\pgfqpoint{3.266409in}{3.956669in}}%
\pgfpathlineto{\pgfqpoint{3.274070in}{3.972819in}}%
\pgfpathlineto{\pgfqpoint{3.275985in}{3.973692in}}%
\pgfpathlineto{\pgfqpoint{3.277901in}{3.976258in}}%
\pgfpathlineto{\pgfqpoint{3.279816in}{3.976559in}}%
\pgfpathlineto{\pgfqpoint{3.281731in}{3.978041in}}%
\pgfpathlineto{\pgfqpoint{3.283646in}{3.981975in}}%
\pgfpathlineto{\pgfqpoint{3.289392in}{3.983771in}}%
\pgfpathlineto{\pgfqpoint{3.295137in}{4.000844in}}%
\pgfpathlineto{\pgfqpoint{3.297052in}{4.000933in}}%
\pgfpathlineto{\pgfqpoint{3.298967in}{4.004349in}}%
\pgfpathlineto{\pgfqpoint{3.300883in}{4.012389in}}%
\pgfpathlineto{\pgfqpoint{3.308543in}{4.021420in}}%
\pgfpathlineto{\pgfqpoint{3.310458in}{4.022708in}}%
\pgfpathlineto{\pgfqpoint{3.318119in}{4.036882in}}%
\pgfpathlineto{\pgfqpoint{3.320034in}{4.036899in}}%
\pgfpathlineto{\pgfqpoint{3.327695in}{4.045501in}}%
\pgfpathlineto{\pgfqpoint{3.329610in}{4.046491in}}%
\pgfpathlineto{\pgfqpoint{3.335356in}{4.063815in}}%
\pgfpathlineto{\pgfqpoint{3.337271in}{4.080572in}}%
\pgfpathlineto{\pgfqpoint{3.339186in}{4.084767in}}%
\pgfpathlineto{\pgfqpoint{3.341101in}{4.085682in}}%
\pgfpathlineto{\pgfqpoint{3.344932in}{4.093187in}}%
\pgfpathlineto{\pgfqpoint{3.346847in}{4.101205in}}%
\pgfpathlineto{\pgfqpoint{3.348762in}{4.101576in}}%
\pgfpathlineto{\pgfqpoint{3.350677in}{4.112702in}}%
\pgfpathlineto{\pgfqpoint{3.352592in}{4.115592in}}%
\pgfpathlineto{\pgfqpoint{3.356423in}{4.118165in}}%
\pgfpathlineto{\pgfqpoint{3.358338in}{4.144803in}}%
\pgfpathlineto{\pgfqpoint{3.360253in}{4.151124in}}%
\pgfpathlineto{\pgfqpoint{3.362168in}{4.151149in}}%
\pgfpathlineto{\pgfqpoint{3.365998in}{4.153625in}}%
\pgfpathlineto{\pgfqpoint{3.367914in}{4.166038in}}%
\pgfpathlineto{\pgfqpoint{3.369829in}{4.168971in}}%
\pgfpathlineto{\pgfqpoint{3.373659in}{4.182198in}}%
\pgfpathlineto{\pgfqpoint{3.375574in}{4.183249in}}%
\pgfpathlineto{\pgfqpoint{3.383235in}{4.200738in}}%
\pgfpathlineto{\pgfqpoint{3.385150in}{4.203332in}}%
\pgfpathlineto{\pgfqpoint{3.388980in}{4.204615in}}%
\pgfpathlineto{\pgfqpoint{3.392811in}{4.206539in}}%
\pgfpathlineto{\pgfqpoint{3.394726in}{4.209183in}}%
\pgfpathlineto{\pgfqpoint{3.396641in}{4.209917in}}%
\pgfpathlineto{\pgfqpoint{3.402387in}{4.225947in}}%
\pgfpathlineto{\pgfqpoint{3.404302in}{4.228109in}}%
\pgfpathlineto{\pgfqpoint{3.404302in}{4.228109in}}%
\pgfusepath{stroke}%
\end{pgfscope}%
\begin{pgfscope}%
\pgfpathrectangle{\pgfqpoint{0.694334in}{2.659974in}}{\pgfqpoint{3.830343in}{1.568135in}}%
\pgfusepath{clip}%
\pgfsetbuttcap%
\pgfsetroundjoin%
\pgfsetlinewidth{1.003750pt}%
\definecolor{currentstroke}{rgb}{0.000000,0.000000,0.000000}%
\pgfsetstrokecolor{currentstroke}%
\pgfsetdash{{3.700000pt}{1.600000pt}}{0.000000pt}%
\pgfpathmoveto{\pgfqpoint{0.694334in}{3.065011in}}%
\pgfpathlineto{\pgfqpoint{0.696249in}{3.065011in}}%
\pgfpathlineto{\pgfqpoint{0.698165in}{3.067189in}}%
\pgfpathlineto{\pgfqpoint{0.700080in}{3.073508in}}%
\pgfpathlineto{\pgfqpoint{0.703910in}{3.077555in}}%
\pgfpathlineto{\pgfqpoint{0.705825in}{3.083396in}}%
\pgfpathlineto{\pgfqpoint{0.707741in}{3.083396in}}%
\pgfpathlineto{\pgfqpoint{0.719232in}{3.100313in}}%
\pgfpathlineto{\pgfqpoint{0.721147in}{3.106819in}}%
\pgfpathlineto{\pgfqpoint{0.724977in}{3.110724in}}%
\pgfpathlineto{\pgfqpoint{0.728807in}{3.111491in}}%
\pgfpathlineto{\pgfqpoint{0.732638in}{3.117468in}}%
\pgfpathlineto{\pgfqpoint{0.734553in}{3.125941in}}%
\pgfpathlineto{\pgfqpoint{0.740298in}{3.132599in}}%
\pgfpathlineto{\pgfqpoint{0.742214in}{3.144371in}}%
\pgfpathlineto{\pgfqpoint{0.744129in}{3.144371in}}%
\pgfpathlineto{\pgfqpoint{0.747959in}{3.147298in}}%
\pgfpathlineto{\pgfqpoint{0.749874in}{3.150159in}}%
\pgfpathlineto{\pgfqpoint{0.751789in}{3.150159in}}%
\pgfpathlineto{\pgfqpoint{0.753705in}{3.151846in}}%
\pgfpathlineto{\pgfqpoint{0.755620in}{3.156777in}}%
\pgfpathlineto{\pgfqpoint{0.757535in}{3.170528in}}%
\pgfpathlineto{\pgfqpoint{0.759450in}{3.174335in}}%
\pgfpathlineto{\pgfqpoint{0.761365in}{3.185981in}}%
\pgfpathlineto{\pgfqpoint{0.763280in}{3.187683in}}%
\pgfpathlineto{\pgfqpoint{0.765196in}{3.196265in}}%
\pgfpathlineto{\pgfqpoint{0.769026in}{3.199389in}}%
\pgfpathlineto{\pgfqpoint{0.772856in}{3.221649in}}%
\pgfpathlineto{\pgfqpoint{0.776687in}{3.224536in}}%
\pgfpathlineto{\pgfqpoint{0.778602in}{3.227670in}}%
\pgfpathlineto{\pgfqpoint{0.782432in}{3.247430in}}%
\pgfpathlineto{\pgfqpoint{0.784347in}{3.271118in}}%
\pgfpathlineto{\pgfqpoint{0.788178in}{3.277575in}}%
\pgfpathlineto{\pgfqpoint{0.790093in}{3.277785in}}%
\pgfpathlineto{\pgfqpoint{0.792008in}{3.281910in}}%
\pgfpathlineto{\pgfqpoint{0.793923in}{3.282315in}}%
\pgfpathlineto{\pgfqpoint{0.795838in}{3.296658in}}%
\pgfpathlineto{\pgfqpoint{0.797754in}{3.297379in}}%
\pgfpathlineto{\pgfqpoint{0.801584in}{3.309885in}}%
\pgfpathlineto{\pgfqpoint{0.803499in}{3.313109in}}%
\pgfpathlineto{\pgfqpoint{0.813075in}{3.314061in}}%
\pgfpathlineto{\pgfqpoint{0.830311in}{3.315788in}}%
\pgfpathlineto{\pgfqpoint{0.926070in}{3.322619in}}%
\pgfpathlineto{\pgfqpoint{0.958628in}{3.323650in}}%
\pgfpathlineto{\pgfqpoint{0.981610in}{3.324381in}}%
\pgfpathlineto{\pgfqpoint{1.021829in}{3.325398in}}%
\pgfpathlineto{\pgfqpoint{1.044811in}{3.326119in}}%
\pgfpathlineto{\pgfqpoint{1.092690in}{3.327122in}}%
\pgfpathlineto{\pgfqpoint{1.111842in}{3.327550in}}%
\pgfpathlineto{\pgfqpoint{1.146315in}{3.328542in}}%
\pgfpathlineto{\pgfqpoint{1.165466in}{3.328964in}}%
\pgfpathlineto{\pgfqpoint{1.222922in}{3.329946in}}%
\pgfpathlineto{\pgfqpoint{1.249734in}{3.330919in}}%
\pgfpathlineto{\pgfqpoint{1.295698in}{3.332023in}}%
\pgfpathlineto{\pgfqpoint{1.316765in}{3.332435in}}%
\pgfpathlineto{\pgfqpoint{1.358899in}{3.333526in}}%
\pgfpathlineto{\pgfqpoint{1.393372in}{3.334877in}}%
\pgfpathlineto{\pgfqpoint{1.418269in}{3.335948in}}%
\pgfpathlineto{\pgfqpoint{1.448912in}{3.337009in}}%
\pgfpathlineto{\pgfqpoint{1.496791in}{3.338062in}}%
\pgfpathlineto{\pgfqpoint{1.521688in}{3.338716in}}%
\pgfpathlineto{\pgfqpoint{1.558077in}{3.339755in}}%
\pgfpathlineto{\pgfqpoint{1.567652in}{3.340143in}}%
\pgfpathlineto{\pgfqpoint{1.596380in}{3.341171in}}%
\pgfpathlineto{\pgfqpoint{1.623192in}{3.341936in}}%
\pgfpathlineto{\pgfqpoint{1.653835in}{3.342950in}}%
\pgfpathlineto{\pgfqpoint{1.734272in}{3.347170in}}%
\pgfpathlineto{\pgfqpoint{1.753424in}{3.347901in}}%
\pgfpathlineto{\pgfqpoint{1.759170in}{3.349110in}}%
\pgfpathlineto{\pgfqpoint{1.780236in}{3.350188in}}%
\pgfpathlineto{\pgfqpoint{1.837692in}{3.354643in}}%
\pgfpathlineto{\pgfqpoint{1.845352in}{3.355561in}}%
\pgfpathlineto{\pgfqpoint{1.881741in}{3.358278in}}%
\pgfpathlineto{\pgfqpoint{1.885571in}{3.359836in}}%
\pgfpathlineto{\pgfqpoint{1.900892in}{3.360938in}}%
\pgfpathlineto{\pgfqpoint{1.935365in}{3.363545in}}%
\pgfpathlineto{\pgfqpoint{1.939196in}{3.365041in}}%
\pgfpathlineto{\pgfqpoint{1.943026in}{3.365994in}}%
\pgfpathlineto{\pgfqpoint{1.956432in}{3.368294in}}%
\pgfpathlineto{\pgfqpoint{1.964093in}{3.369223in}}%
\pgfpathlineto{\pgfqpoint{1.971754in}{3.370043in}}%
\pgfpathlineto{\pgfqpoint{1.975584in}{3.372071in}}%
\pgfpathlineto{\pgfqpoint{1.983245in}{3.373073in}}%
\pgfpathlineto{\pgfqpoint{1.987075in}{3.373869in}}%
\pgfpathlineto{\pgfqpoint{1.992821in}{3.374660in}}%
\pgfpathlineto{\pgfqpoint{1.998566in}{3.377199in}}%
\pgfpathlineto{\pgfqpoint{2.004312in}{3.378449in}}%
\pgfpathlineto{\pgfqpoint{2.025378in}{3.382686in}}%
\pgfpathlineto{\pgfqpoint{2.027294in}{3.384343in}}%
\pgfpathlineto{\pgfqpoint{2.033039in}{3.385072in}}%
\pgfpathlineto{\pgfqpoint{2.036869in}{3.387593in}}%
\pgfpathlineto{\pgfqpoint{2.052191in}{3.391803in}}%
\pgfpathlineto{\pgfqpoint{2.061767in}{3.395879in}}%
\pgfpathlineto{\pgfqpoint{2.067512in}{3.396961in}}%
\pgfpathlineto{\pgfqpoint{2.077088in}{3.397623in}}%
\pgfpathlineto{\pgfqpoint{2.084749in}{3.399749in}}%
\pgfpathlineto{\pgfqpoint{2.088579in}{3.400073in}}%
\pgfpathlineto{\pgfqpoint{2.090494in}{3.402318in}}%
\pgfpathlineto{\pgfqpoint{2.098155in}{3.403663in}}%
\pgfpathlineto{\pgfqpoint{2.105816in}{3.405460in}}%
\pgfpathlineto{\pgfqpoint{2.121137in}{3.407385in}}%
\pgfpathlineto{\pgfqpoint{2.128798in}{3.411299in}}%
\pgfpathlineto{\pgfqpoint{2.140289in}{3.413212in}}%
\pgfpathlineto{\pgfqpoint{2.142204in}{3.415025in}}%
\pgfpathlineto{\pgfqpoint{2.149865in}{3.416029in}}%
\pgfpathlineto{\pgfqpoint{2.153695in}{3.418436in}}%
\pgfpathlineto{\pgfqpoint{2.167101in}{3.420660in}}%
\pgfpathlineto{\pgfqpoint{2.170931in}{3.423927in}}%
\pgfpathlineto{\pgfqpoint{2.176677in}{3.426718in}}%
\pgfpathlineto{\pgfqpoint{2.180507in}{3.429193in}}%
\pgfpathlineto{\pgfqpoint{2.188168in}{3.430859in}}%
\pgfpathlineto{\pgfqpoint{2.193914in}{3.433255in}}%
\pgfpathlineto{\pgfqpoint{2.211150in}{3.439535in}}%
\pgfpathlineto{\pgfqpoint{2.216896in}{3.440720in}}%
\pgfpathlineto{\pgfqpoint{2.226471in}{3.443635in}}%
\pgfpathlineto{\pgfqpoint{2.228387in}{3.443635in}}%
\pgfpathlineto{\pgfqpoint{2.232217in}{3.445182in}}%
\pgfpathlineto{\pgfqpoint{2.241793in}{3.447886in}}%
\pgfpathlineto{\pgfqpoint{2.245623in}{3.449438in}}%
\pgfpathlineto{\pgfqpoint{2.247538in}{3.451677in}}%
\pgfpathlineto{\pgfqpoint{2.255199in}{3.452702in}}%
\pgfpathlineto{\pgfqpoint{2.268605in}{3.461272in}}%
\pgfpathlineto{\pgfqpoint{2.274351in}{3.462223in}}%
\pgfpathlineto{\pgfqpoint{2.283927in}{3.464349in}}%
\pgfpathlineto{\pgfqpoint{2.285842in}{3.464398in}}%
\pgfpathlineto{\pgfqpoint{2.295418in}{3.471365in}}%
\pgfpathlineto{\pgfqpoint{2.304993in}{3.472567in}}%
\pgfpathlineto{\pgfqpoint{2.306909in}{3.474168in}}%
\pgfpathlineto{\pgfqpoint{2.310739in}{3.474712in}}%
\pgfpathlineto{\pgfqpoint{2.322230in}{3.480637in}}%
\pgfpathlineto{\pgfqpoint{2.331806in}{3.484845in}}%
\pgfpathlineto{\pgfqpoint{2.333721in}{3.487289in}}%
\pgfpathlineto{\pgfqpoint{2.335636in}{3.487494in}}%
\pgfpathlineto{\pgfqpoint{2.337551in}{3.489486in}}%
\pgfpathlineto{\pgfqpoint{2.350958in}{3.492200in}}%
\pgfpathlineto{\pgfqpoint{2.352873in}{3.493926in}}%
\pgfpathlineto{\pgfqpoint{2.358618in}{3.495244in}}%
\pgfpathlineto{\pgfqpoint{2.372024in}{3.499903in}}%
\pgfpathlineto{\pgfqpoint{2.373940in}{3.503425in}}%
\pgfpathlineto{\pgfqpoint{2.377770in}{3.504685in}}%
\pgfpathlineto{\pgfqpoint{2.393091in}{3.506641in}}%
\pgfpathlineto{\pgfqpoint{2.396922in}{3.513346in}}%
\pgfpathlineto{\pgfqpoint{2.400752in}{3.515175in}}%
\pgfpathlineto{\pgfqpoint{2.404582in}{3.516489in}}%
\pgfpathlineto{\pgfqpoint{2.410328in}{3.517466in}}%
\pgfpathlineto{\pgfqpoint{2.414158in}{3.519813in}}%
\pgfpathlineto{\pgfqpoint{2.423734in}{3.523611in}}%
\pgfpathlineto{\pgfqpoint{2.439055in}{3.526938in}}%
\pgfpathlineto{\pgfqpoint{2.446716in}{3.534686in}}%
\pgfpathlineto{\pgfqpoint{2.450546in}{3.535252in}}%
\pgfpathlineto{\pgfqpoint{2.452462in}{3.538873in}}%
\pgfpathlineto{\pgfqpoint{2.458207in}{3.540266in}}%
\pgfpathlineto{\pgfqpoint{2.460122in}{3.543644in}}%
\pgfpathlineto{\pgfqpoint{2.467783in}{3.546729in}}%
\pgfpathlineto{\pgfqpoint{2.475444in}{3.548066in}}%
\pgfpathlineto{\pgfqpoint{2.477359in}{3.554011in}}%
\pgfpathlineto{\pgfqpoint{2.488850in}{3.560971in}}%
\pgfpathlineto{\pgfqpoint{2.490765in}{3.560971in}}%
\pgfpathlineto{\pgfqpoint{2.492680in}{3.562602in}}%
\pgfpathlineto{\pgfqpoint{2.496511in}{3.562808in}}%
\pgfpathlineto{\pgfqpoint{2.502256in}{3.567464in}}%
\pgfpathlineto{\pgfqpoint{2.504171in}{3.568863in}}%
\pgfpathlineto{\pgfqpoint{2.506086in}{3.572169in}}%
\pgfpathlineto{\pgfqpoint{2.517577in}{3.577564in}}%
\pgfpathlineto{\pgfqpoint{2.519493in}{3.581013in}}%
\pgfpathlineto{\pgfqpoint{2.525238in}{3.582351in}}%
\pgfpathlineto{\pgfqpoint{2.529068in}{3.586890in}}%
\pgfpathlineto{\pgfqpoint{2.536729in}{3.589953in}}%
\pgfpathlineto{\pgfqpoint{2.542475in}{3.590927in}}%
\pgfpathlineto{\pgfqpoint{2.544390in}{3.590982in}}%
\pgfpathlineto{\pgfqpoint{2.550135in}{3.595072in}}%
\pgfpathlineto{\pgfqpoint{2.552051in}{3.596167in}}%
\pgfpathlineto{\pgfqpoint{2.553966in}{3.599995in}}%
\pgfpathlineto{\pgfqpoint{2.563542in}{3.602847in}}%
\pgfpathlineto{\pgfqpoint{2.571202in}{3.603828in}}%
\pgfpathlineto{\pgfqpoint{2.575033in}{3.605163in}}%
\pgfpathlineto{\pgfqpoint{2.578863in}{3.607099in}}%
\pgfpathlineto{\pgfqpoint{2.588439in}{3.612118in}}%
\pgfpathlineto{\pgfqpoint{2.592269in}{3.613093in}}%
\pgfpathlineto{\pgfqpoint{2.596099in}{3.618731in}}%
\pgfpathlineto{\pgfqpoint{2.599930in}{3.619481in}}%
\pgfpathlineto{\pgfqpoint{2.605675in}{3.626570in}}%
\pgfpathlineto{\pgfqpoint{2.617166in}{3.640969in}}%
\pgfpathlineto{\pgfqpoint{2.619082in}{3.647876in}}%
\pgfpathlineto{\pgfqpoint{2.620997in}{3.650310in}}%
\pgfpathlineto{\pgfqpoint{2.624827in}{3.650805in}}%
\pgfpathlineto{\pgfqpoint{2.626742in}{3.653983in}}%
\pgfpathlineto{\pgfqpoint{2.632488in}{3.656503in}}%
\pgfpathlineto{\pgfqpoint{2.636318in}{3.662725in}}%
\pgfpathlineto{\pgfqpoint{2.642064in}{3.673411in}}%
\pgfpathlineto{\pgfqpoint{2.649724in}{3.674591in}}%
\pgfpathlineto{\pgfqpoint{2.651639in}{3.677191in}}%
\pgfpathlineto{\pgfqpoint{2.655470in}{3.678716in}}%
\pgfpathlineto{\pgfqpoint{2.661215in}{3.681497in}}%
\pgfpathlineto{\pgfqpoint{2.665046in}{3.690807in}}%
\pgfpathlineto{\pgfqpoint{2.672706in}{3.693473in}}%
\pgfpathlineto{\pgfqpoint{2.678452in}{3.693827in}}%
\pgfpathlineto{\pgfqpoint{2.682282in}{3.698746in}}%
\pgfpathlineto{\pgfqpoint{2.684197in}{3.700076in}}%
\pgfpathlineto{\pgfqpoint{2.689943in}{3.709211in}}%
\pgfpathlineto{\pgfqpoint{2.691858in}{3.709444in}}%
\pgfpathlineto{\pgfqpoint{2.695688in}{3.713866in}}%
\pgfpathlineto{\pgfqpoint{2.697604in}{3.716049in}}%
\pgfpathlineto{\pgfqpoint{2.699519in}{3.721863in}}%
\pgfpathlineto{\pgfqpoint{2.705264in}{3.727905in}}%
\pgfpathlineto{\pgfqpoint{2.711010in}{3.728314in}}%
\pgfpathlineto{\pgfqpoint{2.712925in}{3.730180in}}%
\pgfpathlineto{\pgfqpoint{2.714840in}{3.734923in}}%
\pgfpathlineto{\pgfqpoint{2.718670in}{3.736165in}}%
\pgfpathlineto{\pgfqpoint{2.724416in}{3.736890in}}%
\pgfpathlineto{\pgfqpoint{2.728246in}{3.740816in}}%
\pgfpathlineto{\pgfqpoint{2.733992in}{3.743787in}}%
\pgfpathlineto{\pgfqpoint{2.737822in}{3.744698in}}%
\pgfpathlineto{\pgfqpoint{2.739737in}{3.752844in}}%
\pgfpathlineto{\pgfqpoint{2.741653in}{3.753331in}}%
\pgfpathlineto{\pgfqpoint{2.743568in}{3.760299in}}%
\pgfpathlineto{\pgfqpoint{2.749313in}{3.761392in}}%
\pgfpathlineto{\pgfqpoint{2.756974in}{3.774718in}}%
\pgfpathlineto{\pgfqpoint{2.770380in}{3.786643in}}%
\pgfpathlineto{\pgfqpoint{2.772295in}{3.786779in}}%
\pgfpathlineto{\pgfqpoint{2.776126in}{3.791811in}}%
\pgfpathlineto{\pgfqpoint{2.779956in}{3.794468in}}%
\pgfpathlineto{\pgfqpoint{2.783786in}{3.797731in}}%
\pgfpathlineto{\pgfqpoint{2.787617in}{3.807758in}}%
\pgfpathlineto{\pgfqpoint{2.795277in}{3.811825in}}%
\pgfpathlineto{\pgfqpoint{2.799108in}{3.815585in}}%
\pgfpathlineto{\pgfqpoint{2.804853in}{3.820196in}}%
\pgfpathlineto{\pgfqpoint{2.808684in}{3.821093in}}%
\pgfpathlineto{\pgfqpoint{2.810599in}{3.826072in}}%
\pgfpathlineto{\pgfqpoint{2.812514in}{3.827811in}}%
\pgfpathlineto{\pgfqpoint{2.814429in}{3.833023in}}%
\pgfpathlineto{\pgfqpoint{2.820175in}{3.835648in}}%
\pgfpathlineto{\pgfqpoint{2.822090in}{3.838867in}}%
\pgfpathlineto{\pgfqpoint{2.827835in}{3.841099in}}%
\pgfpathlineto{\pgfqpoint{2.831666in}{3.843865in}}%
\pgfpathlineto{\pgfqpoint{2.835496in}{3.849307in}}%
\pgfpathlineto{\pgfqpoint{2.839326in}{3.849728in}}%
\pgfpathlineto{\pgfqpoint{2.841241in}{3.850994in}}%
\pgfpathlineto{\pgfqpoint{2.843157in}{3.856413in}}%
\pgfpathlineto{\pgfqpoint{2.846987in}{3.857521in}}%
\pgfpathlineto{\pgfqpoint{2.852732in}{3.865558in}}%
\pgfpathlineto{\pgfqpoint{2.860393in}{3.870643in}}%
\pgfpathlineto{\pgfqpoint{2.862308in}{3.871479in}}%
\pgfpathlineto{\pgfqpoint{2.864223in}{3.875395in}}%
\pgfpathlineto{\pgfqpoint{2.866139in}{3.884446in}}%
\pgfpathlineto{\pgfqpoint{2.871884in}{3.885265in}}%
\pgfpathlineto{\pgfqpoint{2.873799in}{3.888219in}}%
\pgfpathlineto{\pgfqpoint{2.875715in}{3.888355in}}%
\pgfpathlineto{\pgfqpoint{2.877630in}{3.896272in}}%
\pgfpathlineto{\pgfqpoint{2.883375in}{3.901315in}}%
\pgfpathlineto{\pgfqpoint{2.887206in}{3.904968in}}%
\pgfpathlineto{\pgfqpoint{2.889121in}{3.906436in}}%
\pgfpathlineto{\pgfqpoint{2.891036in}{3.906498in}}%
\pgfpathlineto{\pgfqpoint{2.894866in}{3.907971in}}%
\pgfpathlineto{\pgfqpoint{2.896781in}{3.908530in}}%
\pgfpathlineto{\pgfqpoint{2.898697in}{3.911742in}}%
\pgfpathlineto{\pgfqpoint{2.902527in}{3.911856in}}%
\pgfpathlineto{\pgfqpoint{2.910188in}{3.919570in}}%
\pgfpathlineto{\pgfqpoint{2.925509in}{3.929502in}}%
\pgfpathlineto{\pgfqpoint{2.929339in}{3.935711in}}%
\pgfpathlineto{\pgfqpoint{2.933170in}{3.937918in}}%
\pgfpathlineto{\pgfqpoint{2.935085in}{3.940992in}}%
\pgfpathlineto{\pgfqpoint{2.938915in}{3.942279in}}%
\pgfpathlineto{\pgfqpoint{2.940830in}{3.943034in}}%
\pgfpathlineto{\pgfqpoint{2.944661in}{3.947202in}}%
\pgfpathlineto{\pgfqpoint{2.952321in}{3.950945in}}%
\pgfpathlineto{\pgfqpoint{2.954237in}{3.955339in}}%
\pgfpathlineto{\pgfqpoint{2.963812in}{3.958729in}}%
\pgfpathlineto{\pgfqpoint{2.965728in}{3.965243in}}%
\pgfpathlineto{\pgfqpoint{2.969558in}{3.966234in}}%
\pgfpathlineto{\pgfqpoint{2.971473in}{3.967834in}}%
\pgfpathlineto{\pgfqpoint{2.973388in}{3.971015in}}%
\pgfpathlineto{\pgfqpoint{2.977219in}{3.971291in}}%
\pgfpathlineto{\pgfqpoint{2.979134in}{3.973250in}}%
\pgfpathlineto{\pgfqpoint{2.982964in}{3.981911in}}%
\pgfpathlineto{\pgfqpoint{2.992540in}{3.986764in}}%
\pgfpathlineto{\pgfqpoint{2.994455in}{3.992734in}}%
\pgfpathlineto{\pgfqpoint{2.996370in}{3.994914in}}%
\pgfpathlineto{\pgfqpoint{2.998285in}{3.995057in}}%
\pgfpathlineto{\pgfqpoint{3.000201in}{4.000548in}}%
\pgfpathlineto{\pgfqpoint{3.002116in}{4.001085in}}%
\pgfpathlineto{\pgfqpoint{3.005946in}{4.014232in}}%
\pgfpathlineto{\pgfqpoint{3.007861in}{4.015670in}}%
\pgfpathlineto{\pgfqpoint{3.009777in}{4.020238in}}%
\pgfpathlineto{\pgfqpoint{3.011692in}{4.020696in}}%
\pgfpathlineto{\pgfqpoint{3.013607in}{4.023473in}}%
\pgfpathlineto{\pgfqpoint{3.015522in}{4.024114in}}%
\pgfpathlineto{\pgfqpoint{3.019352in}{4.034631in}}%
\pgfpathlineto{\pgfqpoint{3.028928in}{4.039278in}}%
\pgfpathlineto{\pgfqpoint{3.030843in}{4.042913in}}%
\pgfpathlineto{\pgfqpoint{3.032759in}{4.042988in}}%
\pgfpathlineto{\pgfqpoint{3.034674in}{4.044938in}}%
\pgfpathlineto{\pgfqpoint{3.036589in}{4.052682in}}%
\pgfpathlineto{\pgfqpoint{3.040419in}{4.056072in}}%
\pgfpathlineto{\pgfqpoint{3.046165in}{4.058501in}}%
\pgfpathlineto{\pgfqpoint{3.055741in}{4.064211in}}%
\pgfpathlineto{\pgfqpoint{3.059571in}{4.068563in}}%
\pgfpathlineto{\pgfqpoint{3.061486in}{4.068592in}}%
\pgfpathlineto{\pgfqpoint{3.063401in}{4.073836in}}%
\pgfpathlineto{\pgfqpoint{3.067232in}{4.090179in}}%
\pgfpathlineto{\pgfqpoint{3.076808in}{4.110532in}}%
\pgfpathlineto{\pgfqpoint{3.078723in}{4.120247in}}%
\pgfpathlineto{\pgfqpoint{3.082553in}{4.120718in}}%
\pgfpathlineto{\pgfqpoint{3.084468in}{4.122689in}}%
\pgfpathlineto{\pgfqpoint{3.086383in}{4.131439in}}%
\pgfpathlineto{\pgfqpoint{3.090214in}{4.135066in}}%
\pgfpathlineto{\pgfqpoint{3.092129in}{4.148769in}}%
\pgfpathlineto{\pgfqpoint{3.094044in}{4.149996in}}%
\pgfpathlineto{\pgfqpoint{3.095959in}{4.155815in}}%
\pgfpathlineto{\pgfqpoint{3.097874in}{4.169757in}}%
\pgfpathlineto{\pgfqpoint{3.099790in}{4.174996in}}%
\pgfpathlineto{\pgfqpoint{3.101705in}{4.184117in}}%
\pgfpathlineto{\pgfqpoint{3.103620in}{4.228109in}}%
\pgfpathlineto{\pgfqpoint{3.103620in}{4.228109in}}%
\pgfusepath{stroke}%
\end{pgfscope}%
\begin{pgfscope}%
\pgfsetrectcap%
\pgfsetmiterjoin%
\pgfsetlinewidth{0.803000pt}%
\definecolor{currentstroke}{rgb}{0.000000,0.000000,0.000000}%
\pgfsetstrokecolor{currentstroke}%
\pgfsetdash{}{0pt}%
\pgfpathmoveto{\pgfqpoint{0.694334in}{2.659974in}}%
\pgfpathlineto{\pgfqpoint{0.694334in}{4.228109in}}%
\pgfusepath{stroke}%
\end{pgfscope}%
\begin{pgfscope}%
\pgfsetrectcap%
\pgfsetmiterjoin%
\pgfsetlinewidth{0.803000pt}%
\definecolor{currentstroke}{rgb}{0.000000,0.000000,0.000000}%
\pgfsetstrokecolor{currentstroke}%
\pgfsetdash{}{0pt}%
\pgfpathmoveto{\pgfqpoint{4.524677in}{2.659974in}}%
\pgfpathlineto{\pgfqpoint{4.524677in}{4.228109in}}%
\pgfusepath{stroke}%
\end{pgfscope}%
\begin{pgfscope}%
\pgfsetrectcap%
\pgfsetmiterjoin%
\pgfsetlinewidth{0.803000pt}%
\definecolor{currentstroke}{rgb}{0.000000,0.000000,0.000000}%
\pgfsetstrokecolor{currentstroke}%
\pgfsetdash{}{0pt}%
\pgfpathmoveto{\pgfqpoint{0.694334in}{2.659974in}}%
\pgfpathlineto{\pgfqpoint{4.524677in}{2.659974in}}%
\pgfusepath{stroke}%
\end{pgfscope}%
\begin{pgfscope}%
\pgfsetrectcap%
\pgfsetmiterjoin%
\pgfsetlinewidth{0.803000pt}%
\definecolor{currentstroke}{rgb}{0.000000,0.000000,0.000000}%
\pgfsetstrokecolor{currentstroke}%
\pgfsetdash{}{0pt}%
\pgfpathmoveto{\pgfqpoint{0.694334in}{4.228109in}}%
\pgfpathlineto{\pgfqpoint{4.524677in}{4.228109in}}%
\pgfusepath{stroke}%
\end{pgfscope}%
\begin{pgfscope}%
\pgfsetrectcap%
\pgfsetroundjoin%
\pgfsetlinewidth{1.003750pt}%
\definecolor{currentstroke}{rgb}{0.878431,0.878431,0.815686}%
\pgfsetstrokecolor{currentstroke}%
\pgfsetdash{}{0pt}%
\pgfpathmoveto{\pgfqpoint{3.691785in}{3.915483in}}%
\pgfpathlineto{\pgfqpoint{3.914007in}{3.915483in}}%
\pgfusepath{stroke}%
\end{pgfscope}%
\begin{pgfscope}%
\definecolor{textcolor}{rgb}{0.000000,0.000000,0.000000}%
\pgfsetstrokecolor{textcolor}%
\pgfsetfillcolor{textcolor}%
\pgftext[x=3.936230in,y=3.876594in,left,base]{\color{textcolor}\rmfamily\fontsize{8.000000}{9.600000}\selectfont T.+CPU1}%
\end{pgfscope}%
\begin{pgfscope}%
\pgfsetrectcap%
\pgfsetroundjoin%
\pgfsetlinewidth{1.003750pt}%
\definecolor{currentstroke}{rgb}{0.564706,0.564706,1.000000}%
\pgfsetstrokecolor{currentstroke}%
\pgfsetdash{}{0pt}%
\pgfpathmoveto{\pgfqpoint{3.691785in}{3.771661in}}%
\pgfpathlineto{\pgfqpoint{3.914007in}{3.771661in}}%
\pgfusepath{stroke}%
\end{pgfscope}%
\begin{pgfscope}%
\definecolor{textcolor}{rgb}{0.000000,0.000000,0.000000}%
\pgfsetstrokecolor{textcolor}%
\pgfsetfillcolor{textcolor}%
\pgftext[x=3.936230in,y=3.732772in,left,base]{\color{textcolor}\rmfamily\fontsize{8.000000}{9.600000}\selectfont P4+CPU1}%
\end{pgfscope}%
\begin{pgfscope}%
\pgfsetbuttcap%
\pgfsetroundjoin%
\pgfsetlinewidth{1.003750pt}%
\definecolor{currentstroke}{rgb}{0.564706,0.564706,1.000000}%
\pgfsetstrokecolor{currentstroke}%
\pgfsetdash{{1.000000pt}{1.650000pt}}{0.000000pt}%
\pgfpathmoveto{\pgfqpoint{3.691785in}{3.627839in}}%
\pgfpathlineto{\pgfqpoint{3.914007in}{3.627839in}}%
\pgfusepath{stroke}%
\end{pgfscope}%
\begin{pgfscope}%
\definecolor{textcolor}{rgb}{0.000000,0.000000,0.000000}%
\pgfsetstrokecolor{textcolor}%
\pgfsetfillcolor{textcolor}%
\pgftext[x=3.936230in,y=3.588950in,left,base]{\color{textcolor}\rmfamily\fontsize{8.000000}{9.600000}\selectfont P4+CPU8}%
\end{pgfscope}%
\begin{pgfscope}%
\pgfsetbuttcap%
\pgfsetroundjoin%
\pgfsetlinewidth{1.003750pt}%
\definecolor{currentstroke}{rgb}{0.564706,0.564706,1.000000}%
\pgfsetstrokecolor{currentstroke}%
\pgfsetdash{{3.700000pt}{1.600000pt}}{0.000000pt}%
\pgfpathmoveto{\pgfqpoint{3.691785in}{3.484017in}}%
\pgfpathlineto{\pgfqpoint{3.914007in}{3.484017in}}%
\pgfusepath{stroke}%
\end{pgfscope}%
\begin{pgfscope}%
\definecolor{textcolor}{rgb}{0.000000,0.000000,0.000000}%
\pgfsetstrokecolor{textcolor}%
\pgfsetfillcolor{textcolor}%
\pgftext[x=3.936230in,y=3.445128in,left,base]{\color{textcolor}\rmfamily\fontsize{8.000000}{9.600000}\selectfont P4+GPU}%
\end{pgfscope}%
\begin{pgfscope}%
\pgfsetrectcap%
\pgfsetroundjoin%
\pgfsetlinewidth{1.003750pt}%
\definecolor{currentstroke}{rgb}{0.811765,0.125490,0.125490}%
\pgfsetstrokecolor{currentstroke}%
\pgfsetdash{}{0pt}%
\pgfpathmoveto{\pgfqpoint{3.691785in}{3.340195in}}%
\pgfpathlineto{\pgfqpoint{3.914007in}{3.340195in}}%
\pgfusepath{stroke}%
\end{pgfscope}%
\begin{pgfscope}%
\definecolor{textcolor}{rgb}{0.000000,0.000000,0.000000}%
\pgfsetstrokecolor{textcolor}%
\pgfsetfillcolor{textcolor}%
\pgftext[x=3.936230in,y=3.301306in,left,base]{\color{textcolor}\rmfamily\fontsize{8.000000}{9.600000}\selectfont miniC2D}%
\end{pgfscope}%
\begin{pgfscope}%
\pgfsetbuttcap%
\pgfsetroundjoin%
\pgfsetlinewidth{1.003750pt}%
\definecolor{currentstroke}{rgb}{0.811765,0.125490,0.125490}%
\pgfsetstrokecolor{currentstroke}%
\pgfsetdash{{3.700000pt}{1.600000pt}}{0.000000pt}%
\pgfpathmoveto{\pgfqpoint{3.691785in}{3.196373in}}%
\pgfpathlineto{\pgfqpoint{3.914007in}{3.196373in}}%
\pgfusepath{stroke}%
\end{pgfscope}%
\begin{pgfscope}%
\definecolor{textcolor}{rgb}{0.000000,0.000000,0.000000}%
\pgfsetstrokecolor{textcolor}%
\pgfsetfillcolor{textcolor}%
\pgftext[x=3.936230in,y=3.157484in,left,base]{\color{textcolor}\rmfamily\fontsize{8.000000}{9.600000}\selectfont d4}%
\end{pgfscope}%
\begin{pgfscope}%
\pgfsetbuttcap%
\pgfsetroundjoin%
\pgfsetlinewidth{1.003750pt}%
\definecolor{currentstroke}{rgb}{0.811765,0.125490,0.125490}%
\pgfsetstrokecolor{currentstroke}%
\pgfsetdash{{1.000000pt}{1.650000pt}}{0.000000pt}%
\pgfpathmoveto{\pgfqpoint{3.691785in}{3.052551in}}%
\pgfpathlineto{\pgfqpoint{3.914007in}{3.052551in}}%
\pgfusepath{stroke}%
\end{pgfscope}%
\begin{pgfscope}%
\definecolor{textcolor}{rgb}{0.000000,0.000000,0.000000}%
\pgfsetstrokecolor{textcolor}%
\pgfsetfillcolor{textcolor}%
\pgftext[x=3.936230in,y=3.013662in,left,base]{\color{textcolor}\rmfamily\fontsize{8.000000}{9.600000}\selectfont cachet}%
\end{pgfscope}%
\begin{pgfscope}%
\pgfsetrectcap%
\pgfsetroundjoin%
\pgfsetlinewidth{1.003750pt}%
\definecolor{currentstroke}{rgb}{0.062745,0.000000,0.062745}%
\pgfsetstrokecolor{currentstroke}%
\pgfsetdash{}{0pt}%
\pgfpathmoveto{\pgfqpoint{3.691785in}{2.908729in}}%
\pgfpathlineto{\pgfqpoint{3.914007in}{2.908729in}}%
\pgfusepath{stroke}%
\end{pgfscope}%
\begin{pgfscope}%
\definecolor{textcolor}{rgb}{0.000000,0.000000,0.000000}%
\pgfsetstrokecolor{textcolor}%
\pgfsetfillcolor{textcolor}%
\pgftext[x=3.936230in,y=2.869840in,left,base]{\color{textcolor}\rmfamily\fontsize{8.000000}{9.600000}\selectfont ADDMC}%
\end{pgfscope}%
\begin{pgfscope}%
\pgfsetbuttcap%
\pgfsetroundjoin%
\pgfsetlinewidth{1.003750pt}%
\definecolor{currentstroke}{rgb}{0.000000,0.000000,0.000000}%
\pgfsetstrokecolor{currentstroke}%
\pgfsetdash{{3.700000pt}{1.600000pt}}{0.000000pt}%
\pgfpathmoveto{\pgfqpoint{3.691785in}{2.764907in}}%
\pgfpathlineto{\pgfqpoint{3.914007in}{2.764907in}}%
\pgfusepath{stroke}%
\end{pgfscope}%
\begin{pgfscope}%
\definecolor{textcolor}{rgb}{0.000000,0.000000,0.000000}%
\pgfsetstrokecolor{textcolor}%
\pgfsetfillcolor{textcolor}%
\pgftext[x=3.936230in,y=2.726018in,left,base]{\color{textcolor}\rmfamily\fontsize{8.000000}{9.600000}\selectfont gpusat2}%
\end{pgfscope}%
\begin{pgfscope}%
\pgfsetbuttcap%
\pgfsetmiterjoin%
\definecolor{currentfill}{rgb}{1.000000,1.000000,1.000000}%
\pgfsetfillcolor{currentfill}%
\pgfsetlinewidth{0.000000pt}%
\definecolor{currentstroke}{rgb}{0.000000,0.000000,0.000000}%
\pgfsetstrokecolor{currentstroke}%
\pgfsetstrokeopacity{0.000000}%
\pgfsetdash{}{0pt}%
\pgfpathmoveto{\pgfqpoint{0.694334in}{0.523557in}}%
\pgfpathlineto{\pgfqpoint{4.524677in}{0.523557in}}%
\pgfpathlineto{\pgfqpoint{4.524677in}{2.091692in}}%
\pgfpathlineto{\pgfqpoint{0.694334in}{2.091692in}}%
\pgfpathclose%
\pgfusepath{fill}%
\end{pgfscope}%
\begin{pgfscope}%
\pgfsetbuttcap%
\pgfsetroundjoin%
\definecolor{currentfill}{rgb}{0.000000,0.000000,0.000000}%
\pgfsetfillcolor{currentfill}%
\pgfsetlinewidth{0.803000pt}%
\definecolor{currentstroke}{rgb}{0.000000,0.000000,0.000000}%
\pgfsetstrokecolor{currentstroke}%
\pgfsetdash{}{0pt}%
\pgfsys@defobject{currentmarker}{\pgfqpoint{0.000000in}{-0.048611in}}{\pgfqpoint{0.000000in}{0.000000in}}{%
\pgfpathmoveto{\pgfqpoint{0.000000in}{0.000000in}}%
\pgfpathlineto{\pgfqpoint{0.000000in}{-0.048611in}}%
\pgfusepath{stroke,fill}%
}%
\begin{pgfscope}%
\pgfsys@transformshift{0.694334in}{0.523557in}%
\pgfsys@useobject{currentmarker}{}%
\end{pgfscope}%
\end{pgfscope}%
\begin{pgfscope}%
\definecolor{textcolor}{rgb}{0.000000,0.000000,0.000000}%
\pgfsetstrokecolor{textcolor}%
\pgfsetfillcolor{textcolor}%
\pgftext[x=0.694334in,y=0.426335in,,top]{\color{textcolor}\rmfamily\fontsize{9.000000}{10.800000}\selectfont \(\displaystyle 0\)}%
\end{pgfscope}%
\begin{pgfscope}%
\pgfsetbuttcap%
\pgfsetroundjoin%
\definecolor{currentfill}{rgb}{0.000000,0.000000,0.000000}%
\pgfsetfillcolor{currentfill}%
\pgfsetlinewidth{0.803000pt}%
\definecolor{currentstroke}{rgb}{0.000000,0.000000,0.000000}%
\pgfsetstrokecolor{currentstroke}%
\pgfsetdash{}{0pt}%
\pgfsys@defobject{currentmarker}{\pgfqpoint{0.000000in}{-0.048611in}}{\pgfqpoint{0.000000in}{0.000000in}}{%
\pgfpathmoveto{\pgfqpoint{0.000000in}{0.000000in}}%
\pgfpathlineto{\pgfqpoint{0.000000in}{-0.048611in}}%
\pgfusepath{stroke,fill}%
}%
\begin{pgfscope}%
\pgfsys@transformshift{1.173127in}{0.523557in}%
\pgfsys@useobject{currentmarker}{}%
\end{pgfscope}%
\end{pgfscope}%
\begin{pgfscope}%
\definecolor{textcolor}{rgb}{0.000000,0.000000,0.000000}%
\pgfsetstrokecolor{textcolor}%
\pgfsetfillcolor{textcolor}%
\pgftext[x=1.173127in,y=0.426335in,,top]{\color{textcolor}\rmfamily\fontsize{9.000000}{10.800000}\selectfont \(\displaystyle 250\)}%
\end{pgfscope}%
\begin{pgfscope}%
\pgfsetbuttcap%
\pgfsetroundjoin%
\definecolor{currentfill}{rgb}{0.000000,0.000000,0.000000}%
\pgfsetfillcolor{currentfill}%
\pgfsetlinewidth{0.803000pt}%
\definecolor{currentstroke}{rgb}{0.000000,0.000000,0.000000}%
\pgfsetstrokecolor{currentstroke}%
\pgfsetdash{}{0pt}%
\pgfsys@defobject{currentmarker}{\pgfqpoint{0.000000in}{-0.048611in}}{\pgfqpoint{0.000000in}{0.000000in}}{%
\pgfpathmoveto{\pgfqpoint{0.000000in}{0.000000in}}%
\pgfpathlineto{\pgfqpoint{0.000000in}{-0.048611in}}%
\pgfusepath{stroke,fill}%
}%
\begin{pgfscope}%
\pgfsys@transformshift{1.651920in}{0.523557in}%
\pgfsys@useobject{currentmarker}{}%
\end{pgfscope}%
\end{pgfscope}%
\begin{pgfscope}%
\definecolor{textcolor}{rgb}{0.000000,0.000000,0.000000}%
\pgfsetstrokecolor{textcolor}%
\pgfsetfillcolor{textcolor}%
\pgftext[x=1.651920in,y=0.426335in,,top]{\color{textcolor}\rmfamily\fontsize{9.000000}{10.800000}\selectfont \(\displaystyle 500\)}%
\end{pgfscope}%
\begin{pgfscope}%
\pgfsetbuttcap%
\pgfsetroundjoin%
\definecolor{currentfill}{rgb}{0.000000,0.000000,0.000000}%
\pgfsetfillcolor{currentfill}%
\pgfsetlinewidth{0.803000pt}%
\definecolor{currentstroke}{rgb}{0.000000,0.000000,0.000000}%
\pgfsetstrokecolor{currentstroke}%
\pgfsetdash{}{0pt}%
\pgfsys@defobject{currentmarker}{\pgfqpoint{0.000000in}{-0.048611in}}{\pgfqpoint{0.000000in}{0.000000in}}{%
\pgfpathmoveto{\pgfqpoint{0.000000in}{0.000000in}}%
\pgfpathlineto{\pgfqpoint{0.000000in}{-0.048611in}}%
\pgfusepath{stroke,fill}%
}%
\begin{pgfscope}%
\pgfsys@transformshift{2.130713in}{0.523557in}%
\pgfsys@useobject{currentmarker}{}%
\end{pgfscope}%
\end{pgfscope}%
\begin{pgfscope}%
\definecolor{textcolor}{rgb}{0.000000,0.000000,0.000000}%
\pgfsetstrokecolor{textcolor}%
\pgfsetfillcolor{textcolor}%
\pgftext[x=2.130713in,y=0.426335in,,top]{\color{textcolor}\rmfamily\fontsize{9.000000}{10.800000}\selectfont \(\displaystyle 750\)}%
\end{pgfscope}%
\begin{pgfscope}%
\pgfsetbuttcap%
\pgfsetroundjoin%
\definecolor{currentfill}{rgb}{0.000000,0.000000,0.000000}%
\pgfsetfillcolor{currentfill}%
\pgfsetlinewidth{0.803000pt}%
\definecolor{currentstroke}{rgb}{0.000000,0.000000,0.000000}%
\pgfsetstrokecolor{currentstroke}%
\pgfsetdash{}{0pt}%
\pgfsys@defobject{currentmarker}{\pgfqpoint{0.000000in}{-0.048611in}}{\pgfqpoint{0.000000in}{0.000000in}}{%
\pgfpathmoveto{\pgfqpoint{0.000000in}{0.000000in}}%
\pgfpathlineto{\pgfqpoint{0.000000in}{-0.048611in}}%
\pgfusepath{stroke,fill}%
}%
\begin{pgfscope}%
\pgfsys@transformshift{2.609506in}{0.523557in}%
\pgfsys@useobject{currentmarker}{}%
\end{pgfscope}%
\end{pgfscope}%
\begin{pgfscope}%
\definecolor{textcolor}{rgb}{0.000000,0.000000,0.000000}%
\pgfsetstrokecolor{textcolor}%
\pgfsetfillcolor{textcolor}%
\pgftext[x=2.609506in,y=0.426335in,,top]{\color{textcolor}\rmfamily\fontsize{9.000000}{10.800000}\selectfont \(\displaystyle 1000\)}%
\end{pgfscope}%
\begin{pgfscope}%
\pgfsetbuttcap%
\pgfsetroundjoin%
\definecolor{currentfill}{rgb}{0.000000,0.000000,0.000000}%
\pgfsetfillcolor{currentfill}%
\pgfsetlinewidth{0.803000pt}%
\definecolor{currentstroke}{rgb}{0.000000,0.000000,0.000000}%
\pgfsetstrokecolor{currentstroke}%
\pgfsetdash{}{0pt}%
\pgfsys@defobject{currentmarker}{\pgfqpoint{0.000000in}{-0.048611in}}{\pgfqpoint{0.000000in}{0.000000in}}{%
\pgfpathmoveto{\pgfqpoint{0.000000in}{0.000000in}}%
\pgfpathlineto{\pgfqpoint{0.000000in}{-0.048611in}}%
\pgfusepath{stroke,fill}%
}%
\begin{pgfscope}%
\pgfsys@transformshift{3.088299in}{0.523557in}%
\pgfsys@useobject{currentmarker}{}%
\end{pgfscope}%
\end{pgfscope}%
\begin{pgfscope}%
\definecolor{textcolor}{rgb}{0.000000,0.000000,0.000000}%
\pgfsetstrokecolor{textcolor}%
\pgfsetfillcolor{textcolor}%
\pgftext[x=3.088299in,y=0.426335in,,top]{\color{textcolor}\rmfamily\fontsize{9.000000}{10.800000}\selectfont \(\displaystyle 1250\)}%
\end{pgfscope}%
\begin{pgfscope}%
\pgfsetbuttcap%
\pgfsetroundjoin%
\definecolor{currentfill}{rgb}{0.000000,0.000000,0.000000}%
\pgfsetfillcolor{currentfill}%
\pgfsetlinewidth{0.803000pt}%
\definecolor{currentstroke}{rgb}{0.000000,0.000000,0.000000}%
\pgfsetstrokecolor{currentstroke}%
\pgfsetdash{}{0pt}%
\pgfsys@defobject{currentmarker}{\pgfqpoint{0.000000in}{-0.048611in}}{\pgfqpoint{0.000000in}{0.000000in}}{%
\pgfpathmoveto{\pgfqpoint{0.000000in}{0.000000in}}%
\pgfpathlineto{\pgfqpoint{0.000000in}{-0.048611in}}%
\pgfusepath{stroke,fill}%
}%
\begin{pgfscope}%
\pgfsys@transformshift{3.567091in}{0.523557in}%
\pgfsys@useobject{currentmarker}{}%
\end{pgfscope}%
\end{pgfscope}%
\begin{pgfscope}%
\definecolor{textcolor}{rgb}{0.000000,0.000000,0.000000}%
\pgfsetstrokecolor{textcolor}%
\pgfsetfillcolor{textcolor}%
\pgftext[x=3.567091in,y=0.426335in,,top]{\color{textcolor}\rmfamily\fontsize{9.000000}{10.800000}\selectfont \(\displaystyle 1500\)}%
\end{pgfscope}%
\begin{pgfscope}%
\pgfsetbuttcap%
\pgfsetroundjoin%
\definecolor{currentfill}{rgb}{0.000000,0.000000,0.000000}%
\pgfsetfillcolor{currentfill}%
\pgfsetlinewidth{0.803000pt}%
\definecolor{currentstroke}{rgb}{0.000000,0.000000,0.000000}%
\pgfsetstrokecolor{currentstroke}%
\pgfsetdash{}{0pt}%
\pgfsys@defobject{currentmarker}{\pgfqpoint{0.000000in}{-0.048611in}}{\pgfqpoint{0.000000in}{0.000000in}}{%
\pgfpathmoveto{\pgfqpoint{0.000000in}{0.000000in}}%
\pgfpathlineto{\pgfqpoint{0.000000in}{-0.048611in}}%
\pgfusepath{stroke,fill}%
}%
\begin{pgfscope}%
\pgfsys@transformshift{4.045884in}{0.523557in}%
\pgfsys@useobject{currentmarker}{}%
\end{pgfscope}%
\end{pgfscope}%
\begin{pgfscope}%
\definecolor{textcolor}{rgb}{0.000000,0.000000,0.000000}%
\pgfsetstrokecolor{textcolor}%
\pgfsetfillcolor{textcolor}%
\pgftext[x=4.045884in,y=0.426335in,,top]{\color{textcolor}\rmfamily\fontsize{9.000000}{10.800000}\selectfont \(\displaystyle 1750\)}%
\end{pgfscope}%
\begin{pgfscope}%
\pgfsetbuttcap%
\pgfsetroundjoin%
\definecolor{currentfill}{rgb}{0.000000,0.000000,0.000000}%
\pgfsetfillcolor{currentfill}%
\pgfsetlinewidth{0.803000pt}%
\definecolor{currentstroke}{rgb}{0.000000,0.000000,0.000000}%
\pgfsetstrokecolor{currentstroke}%
\pgfsetdash{}{0pt}%
\pgfsys@defobject{currentmarker}{\pgfqpoint{0.000000in}{-0.048611in}}{\pgfqpoint{0.000000in}{0.000000in}}{%
\pgfpathmoveto{\pgfqpoint{0.000000in}{0.000000in}}%
\pgfpathlineto{\pgfqpoint{0.000000in}{-0.048611in}}%
\pgfusepath{stroke,fill}%
}%
\begin{pgfscope}%
\pgfsys@transformshift{4.524677in}{0.523557in}%
\pgfsys@useobject{currentmarker}{}%
\end{pgfscope}%
\end{pgfscope}%
\begin{pgfscope}%
\definecolor{textcolor}{rgb}{0.000000,0.000000,0.000000}%
\pgfsetstrokecolor{textcolor}%
\pgfsetfillcolor{textcolor}%
\pgftext[x=4.524677in,y=0.426335in,,top]{\color{textcolor}\rmfamily\fontsize{9.000000}{10.800000}\selectfont \(\displaystyle 2000\)}%
\end{pgfscope}%
\begin{pgfscope}%
\definecolor{textcolor}{rgb}{0.000000,0.000000,0.000000}%
\pgfsetstrokecolor{textcolor}%
\pgfsetfillcolor{textcolor}%
\pgftext[x=2.609506in,y=0.260390in,,top]{\color{textcolor}\rmfamily\fontsize{9.000000}{10.800000}\selectfont Number of benchmarks solved}%
\end{pgfscope}%
\begin{pgfscope}%
\pgfsetbuttcap%
\pgfsetroundjoin%
\definecolor{currentfill}{rgb}{0.000000,0.000000,0.000000}%
\pgfsetfillcolor{currentfill}%
\pgfsetlinewidth{0.803000pt}%
\definecolor{currentstroke}{rgb}{0.000000,0.000000,0.000000}%
\pgfsetstrokecolor{currentstroke}%
\pgfsetdash{}{0pt}%
\pgfsys@defobject{currentmarker}{\pgfqpoint{-0.048611in}{0.000000in}}{\pgfqpoint{0.000000in}{0.000000in}}{%
\pgfpathmoveto{\pgfqpoint{0.000000in}{0.000000in}}%
\pgfpathlineto{\pgfqpoint{-0.048611in}{0.000000in}}%
\pgfusepath{stroke,fill}%
}%
\begin{pgfscope}%
\pgfsys@transformshift{0.694334in}{0.612607in}%
\pgfsys@useobject{currentmarker}{}%
\end{pgfscope}%
\end{pgfscope}%
\begin{pgfscope}%
\definecolor{textcolor}{rgb}{0.000000,0.000000,0.000000}%
\pgfsetstrokecolor{textcolor}%
\pgfsetfillcolor{textcolor}%
\pgftext[x=0.330525in, y=0.567882in, left, base]{\color{textcolor}\rmfamily\fontsize{9.000000}{10.800000}\selectfont \(\displaystyle 10^{-2}\)}%
\end{pgfscope}%
\begin{pgfscope}%
\pgfsetbuttcap%
\pgfsetroundjoin%
\definecolor{currentfill}{rgb}{0.000000,0.000000,0.000000}%
\pgfsetfillcolor{currentfill}%
\pgfsetlinewidth{0.803000pt}%
\definecolor{currentstroke}{rgb}{0.000000,0.000000,0.000000}%
\pgfsetstrokecolor{currentstroke}%
\pgfsetdash{}{0pt}%
\pgfsys@defobject{currentmarker}{\pgfqpoint{-0.048611in}{0.000000in}}{\pgfqpoint{0.000000in}{0.000000in}}{%
\pgfpathmoveto{\pgfqpoint{0.000000in}{0.000000in}}%
\pgfpathlineto{\pgfqpoint{-0.048611in}{0.000000in}}%
\pgfusepath{stroke,fill}%
}%
\begin{pgfscope}%
\pgfsys@transformshift{0.694334in}{0.908424in}%
\pgfsys@useobject{currentmarker}{}%
\end{pgfscope}%
\end{pgfscope}%
\begin{pgfscope}%
\definecolor{textcolor}{rgb}{0.000000,0.000000,0.000000}%
\pgfsetstrokecolor{textcolor}%
\pgfsetfillcolor{textcolor}%
\pgftext[x=0.330525in, y=0.863699in, left, base]{\color{textcolor}\rmfamily\fontsize{9.000000}{10.800000}\selectfont \(\displaystyle 10^{-1}\)}%
\end{pgfscope}%
\begin{pgfscope}%
\pgfsetbuttcap%
\pgfsetroundjoin%
\definecolor{currentfill}{rgb}{0.000000,0.000000,0.000000}%
\pgfsetfillcolor{currentfill}%
\pgfsetlinewidth{0.803000pt}%
\definecolor{currentstroke}{rgb}{0.000000,0.000000,0.000000}%
\pgfsetstrokecolor{currentstroke}%
\pgfsetdash{}{0pt}%
\pgfsys@defobject{currentmarker}{\pgfqpoint{-0.048611in}{0.000000in}}{\pgfqpoint{0.000000in}{0.000000in}}{%
\pgfpathmoveto{\pgfqpoint{0.000000in}{0.000000in}}%
\pgfpathlineto{\pgfqpoint{-0.048611in}{0.000000in}}%
\pgfusepath{stroke,fill}%
}%
\begin{pgfscope}%
\pgfsys@transformshift{0.694334in}{1.204241in}%
\pgfsys@useobject{currentmarker}{}%
\end{pgfscope}%
\end{pgfscope}%
\begin{pgfscope}%
\definecolor{textcolor}{rgb}{0.000000,0.000000,0.000000}%
\pgfsetstrokecolor{textcolor}%
\pgfsetfillcolor{textcolor}%
\pgftext[x=0.410771in, y=1.159516in, left, base]{\color{textcolor}\rmfamily\fontsize{9.000000}{10.800000}\selectfont \(\displaystyle 10^{0}\)}%
\end{pgfscope}%
\begin{pgfscope}%
\pgfsetbuttcap%
\pgfsetroundjoin%
\definecolor{currentfill}{rgb}{0.000000,0.000000,0.000000}%
\pgfsetfillcolor{currentfill}%
\pgfsetlinewidth{0.803000pt}%
\definecolor{currentstroke}{rgb}{0.000000,0.000000,0.000000}%
\pgfsetstrokecolor{currentstroke}%
\pgfsetdash{}{0pt}%
\pgfsys@defobject{currentmarker}{\pgfqpoint{-0.048611in}{0.000000in}}{\pgfqpoint{0.000000in}{0.000000in}}{%
\pgfpathmoveto{\pgfqpoint{0.000000in}{0.000000in}}%
\pgfpathlineto{\pgfqpoint{-0.048611in}{0.000000in}}%
\pgfusepath{stroke,fill}%
}%
\begin{pgfscope}%
\pgfsys@transformshift{0.694334in}{1.500058in}%
\pgfsys@useobject{currentmarker}{}%
\end{pgfscope}%
\end{pgfscope}%
\begin{pgfscope}%
\definecolor{textcolor}{rgb}{0.000000,0.000000,0.000000}%
\pgfsetstrokecolor{textcolor}%
\pgfsetfillcolor{textcolor}%
\pgftext[x=0.410771in, y=1.455333in, left, base]{\color{textcolor}\rmfamily\fontsize{9.000000}{10.800000}\selectfont \(\displaystyle 10^{1}\)}%
\end{pgfscope}%
\begin{pgfscope}%
\pgfsetbuttcap%
\pgfsetroundjoin%
\definecolor{currentfill}{rgb}{0.000000,0.000000,0.000000}%
\pgfsetfillcolor{currentfill}%
\pgfsetlinewidth{0.803000pt}%
\definecolor{currentstroke}{rgb}{0.000000,0.000000,0.000000}%
\pgfsetstrokecolor{currentstroke}%
\pgfsetdash{}{0pt}%
\pgfsys@defobject{currentmarker}{\pgfqpoint{-0.048611in}{0.000000in}}{\pgfqpoint{0.000000in}{0.000000in}}{%
\pgfpathmoveto{\pgfqpoint{0.000000in}{0.000000in}}%
\pgfpathlineto{\pgfqpoint{-0.048611in}{0.000000in}}%
\pgfusepath{stroke,fill}%
}%
\begin{pgfscope}%
\pgfsys@transformshift{0.694334in}{1.795875in}%
\pgfsys@useobject{currentmarker}{}%
\end{pgfscope}%
\end{pgfscope}%
\begin{pgfscope}%
\definecolor{textcolor}{rgb}{0.000000,0.000000,0.000000}%
\pgfsetstrokecolor{textcolor}%
\pgfsetfillcolor{textcolor}%
\pgftext[x=0.410771in, y=1.751150in, left, base]{\color{textcolor}\rmfamily\fontsize{9.000000}{10.800000}\selectfont \(\displaystyle 10^{2}\)}%
\end{pgfscope}%
\begin{pgfscope}%
\pgfsetbuttcap%
\pgfsetroundjoin%
\definecolor{currentfill}{rgb}{0.000000,0.000000,0.000000}%
\pgfsetfillcolor{currentfill}%
\pgfsetlinewidth{0.803000pt}%
\definecolor{currentstroke}{rgb}{0.000000,0.000000,0.000000}%
\pgfsetstrokecolor{currentstroke}%
\pgfsetdash{}{0pt}%
\pgfsys@defobject{currentmarker}{\pgfqpoint{-0.048611in}{0.000000in}}{\pgfqpoint{0.000000in}{0.000000in}}{%
\pgfpathmoveto{\pgfqpoint{0.000000in}{0.000000in}}%
\pgfpathlineto{\pgfqpoint{-0.048611in}{0.000000in}}%
\pgfusepath{stroke,fill}%
}%
\begin{pgfscope}%
\pgfsys@transformshift{0.694334in}{2.091692in}%
\pgfsys@useobject{currentmarker}{}%
\end{pgfscope}%
\end{pgfscope}%
\begin{pgfscope}%
\definecolor{textcolor}{rgb}{0.000000,0.000000,0.000000}%
\pgfsetstrokecolor{textcolor}%
\pgfsetfillcolor{textcolor}%
\pgftext[x=0.410771in, y=2.046967in, left, base]{\color{textcolor}\rmfamily\fontsize{9.000000}{10.800000}\selectfont \(\displaystyle 10^{3}\)}%
\end{pgfscope}%
\begin{pgfscope}%
\pgfsetbuttcap%
\pgfsetroundjoin%
\definecolor{currentfill}{rgb}{0.000000,0.000000,0.000000}%
\pgfsetfillcolor{currentfill}%
\pgfsetlinewidth{0.602250pt}%
\definecolor{currentstroke}{rgb}{0.000000,0.000000,0.000000}%
\pgfsetstrokecolor{currentstroke}%
\pgfsetdash{}{0pt}%
\pgfsys@defobject{currentmarker}{\pgfqpoint{-0.027778in}{0.000000in}}{\pgfqpoint{0.000000in}{0.000000in}}{%
\pgfpathmoveto{\pgfqpoint{0.000000in}{0.000000in}}%
\pgfpathlineto{\pgfqpoint{-0.027778in}{0.000000in}}%
\pgfusepath{stroke,fill}%
}%
\begin{pgfscope}%
\pgfsys@transformshift{0.694334in}{0.523557in}%
\pgfsys@useobject{currentmarker}{}%
\end{pgfscope}%
\end{pgfscope}%
\begin{pgfscope}%
\pgfsetbuttcap%
\pgfsetroundjoin%
\definecolor{currentfill}{rgb}{0.000000,0.000000,0.000000}%
\pgfsetfillcolor{currentfill}%
\pgfsetlinewidth{0.602250pt}%
\definecolor{currentstroke}{rgb}{0.000000,0.000000,0.000000}%
\pgfsetstrokecolor{currentstroke}%
\pgfsetdash{}{0pt}%
\pgfsys@defobject{currentmarker}{\pgfqpoint{-0.027778in}{0.000000in}}{\pgfqpoint{0.000000in}{0.000000in}}{%
\pgfpathmoveto{\pgfqpoint{0.000000in}{0.000000in}}%
\pgfpathlineto{\pgfqpoint{-0.027778in}{0.000000in}}%
\pgfusepath{stroke,fill}%
}%
\begin{pgfscope}%
\pgfsys@transformshift{0.694334in}{0.546980in}%
\pgfsys@useobject{currentmarker}{}%
\end{pgfscope}%
\end{pgfscope}%
\begin{pgfscope}%
\pgfsetbuttcap%
\pgfsetroundjoin%
\definecolor{currentfill}{rgb}{0.000000,0.000000,0.000000}%
\pgfsetfillcolor{currentfill}%
\pgfsetlinewidth{0.602250pt}%
\definecolor{currentstroke}{rgb}{0.000000,0.000000,0.000000}%
\pgfsetstrokecolor{currentstroke}%
\pgfsetdash{}{0pt}%
\pgfsys@defobject{currentmarker}{\pgfqpoint{-0.027778in}{0.000000in}}{\pgfqpoint{0.000000in}{0.000000in}}{%
\pgfpathmoveto{\pgfqpoint{0.000000in}{0.000000in}}%
\pgfpathlineto{\pgfqpoint{-0.027778in}{0.000000in}}%
\pgfusepath{stroke,fill}%
}%
\begin{pgfscope}%
\pgfsys@transformshift{0.694334in}{0.566784in}%
\pgfsys@useobject{currentmarker}{}%
\end{pgfscope}%
\end{pgfscope}%
\begin{pgfscope}%
\pgfsetbuttcap%
\pgfsetroundjoin%
\definecolor{currentfill}{rgb}{0.000000,0.000000,0.000000}%
\pgfsetfillcolor{currentfill}%
\pgfsetlinewidth{0.602250pt}%
\definecolor{currentstroke}{rgb}{0.000000,0.000000,0.000000}%
\pgfsetstrokecolor{currentstroke}%
\pgfsetdash{}{0pt}%
\pgfsys@defobject{currentmarker}{\pgfqpoint{-0.027778in}{0.000000in}}{\pgfqpoint{0.000000in}{0.000000in}}{%
\pgfpathmoveto{\pgfqpoint{0.000000in}{0.000000in}}%
\pgfpathlineto{\pgfqpoint{-0.027778in}{0.000000in}}%
\pgfusepath{stroke,fill}%
}%
\begin{pgfscope}%
\pgfsys@transformshift{0.694334in}{0.583939in}%
\pgfsys@useobject{currentmarker}{}%
\end{pgfscope}%
\end{pgfscope}%
\begin{pgfscope}%
\pgfsetbuttcap%
\pgfsetroundjoin%
\definecolor{currentfill}{rgb}{0.000000,0.000000,0.000000}%
\pgfsetfillcolor{currentfill}%
\pgfsetlinewidth{0.602250pt}%
\definecolor{currentstroke}{rgb}{0.000000,0.000000,0.000000}%
\pgfsetstrokecolor{currentstroke}%
\pgfsetdash{}{0pt}%
\pgfsys@defobject{currentmarker}{\pgfqpoint{-0.027778in}{0.000000in}}{\pgfqpoint{0.000000in}{0.000000in}}{%
\pgfpathmoveto{\pgfqpoint{0.000000in}{0.000000in}}%
\pgfpathlineto{\pgfqpoint{-0.027778in}{0.000000in}}%
\pgfusepath{stroke,fill}%
}%
\begin{pgfscope}%
\pgfsys@transformshift{0.694334in}{0.599071in}%
\pgfsys@useobject{currentmarker}{}%
\end{pgfscope}%
\end{pgfscope}%
\begin{pgfscope}%
\pgfsetbuttcap%
\pgfsetroundjoin%
\definecolor{currentfill}{rgb}{0.000000,0.000000,0.000000}%
\pgfsetfillcolor{currentfill}%
\pgfsetlinewidth{0.602250pt}%
\definecolor{currentstroke}{rgb}{0.000000,0.000000,0.000000}%
\pgfsetstrokecolor{currentstroke}%
\pgfsetdash{}{0pt}%
\pgfsys@defobject{currentmarker}{\pgfqpoint{-0.027778in}{0.000000in}}{\pgfqpoint{0.000000in}{0.000000in}}{%
\pgfpathmoveto{\pgfqpoint{0.000000in}{0.000000in}}%
\pgfpathlineto{\pgfqpoint{-0.027778in}{0.000000in}}%
\pgfusepath{stroke,fill}%
}%
\begin{pgfscope}%
\pgfsys@transformshift{0.694334in}{0.701657in}%
\pgfsys@useobject{currentmarker}{}%
\end{pgfscope}%
\end{pgfscope}%
\begin{pgfscope}%
\pgfsetbuttcap%
\pgfsetroundjoin%
\definecolor{currentfill}{rgb}{0.000000,0.000000,0.000000}%
\pgfsetfillcolor{currentfill}%
\pgfsetlinewidth{0.602250pt}%
\definecolor{currentstroke}{rgb}{0.000000,0.000000,0.000000}%
\pgfsetstrokecolor{currentstroke}%
\pgfsetdash{}{0pt}%
\pgfsys@defobject{currentmarker}{\pgfqpoint{-0.027778in}{0.000000in}}{\pgfqpoint{0.000000in}{0.000000in}}{%
\pgfpathmoveto{\pgfqpoint{0.000000in}{0.000000in}}%
\pgfpathlineto{\pgfqpoint{-0.027778in}{0.000000in}}%
\pgfusepath{stroke,fill}%
}%
\begin{pgfscope}%
\pgfsys@transformshift{0.694334in}{0.753748in}%
\pgfsys@useobject{currentmarker}{}%
\end{pgfscope}%
\end{pgfscope}%
\begin{pgfscope}%
\pgfsetbuttcap%
\pgfsetroundjoin%
\definecolor{currentfill}{rgb}{0.000000,0.000000,0.000000}%
\pgfsetfillcolor{currentfill}%
\pgfsetlinewidth{0.602250pt}%
\definecolor{currentstroke}{rgb}{0.000000,0.000000,0.000000}%
\pgfsetstrokecolor{currentstroke}%
\pgfsetdash{}{0pt}%
\pgfsys@defobject{currentmarker}{\pgfqpoint{-0.027778in}{0.000000in}}{\pgfqpoint{0.000000in}{0.000000in}}{%
\pgfpathmoveto{\pgfqpoint{0.000000in}{0.000000in}}%
\pgfpathlineto{\pgfqpoint{-0.027778in}{0.000000in}}%
\pgfusepath{stroke,fill}%
}%
\begin{pgfscope}%
\pgfsys@transformshift{0.694334in}{0.790707in}%
\pgfsys@useobject{currentmarker}{}%
\end{pgfscope}%
\end{pgfscope}%
\begin{pgfscope}%
\pgfsetbuttcap%
\pgfsetroundjoin%
\definecolor{currentfill}{rgb}{0.000000,0.000000,0.000000}%
\pgfsetfillcolor{currentfill}%
\pgfsetlinewidth{0.602250pt}%
\definecolor{currentstroke}{rgb}{0.000000,0.000000,0.000000}%
\pgfsetstrokecolor{currentstroke}%
\pgfsetdash{}{0pt}%
\pgfsys@defobject{currentmarker}{\pgfqpoint{-0.027778in}{0.000000in}}{\pgfqpoint{0.000000in}{0.000000in}}{%
\pgfpathmoveto{\pgfqpoint{0.000000in}{0.000000in}}%
\pgfpathlineto{\pgfqpoint{-0.027778in}{0.000000in}}%
\pgfusepath{stroke,fill}%
}%
\begin{pgfscope}%
\pgfsys@transformshift{0.694334in}{0.819374in}%
\pgfsys@useobject{currentmarker}{}%
\end{pgfscope}%
\end{pgfscope}%
\begin{pgfscope}%
\pgfsetbuttcap%
\pgfsetroundjoin%
\definecolor{currentfill}{rgb}{0.000000,0.000000,0.000000}%
\pgfsetfillcolor{currentfill}%
\pgfsetlinewidth{0.602250pt}%
\definecolor{currentstroke}{rgb}{0.000000,0.000000,0.000000}%
\pgfsetstrokecolor{currentstroke}%
\pgfsetdash{}{0pt}%
\pgfsys@defobject{currentmarker}{\pgfqpoint{-0.027778in}{0.000000in}}{\pgfqpoint{0.000000in}{0.000000in}}{%
\pgfpathmoveto{\pgfqpoint{0.000000in}{0.000000in}}%
\pgfpathlineto{\pgfqpoint{-0.027778in}{0.000000in}}%
\pgfusepath{stroke,fill}%
}%
\begin{pgfscope}%
\pgfsys@transformshift{0.694334in}{0.842797in}%
\pgfsys@useobject{currentmarker}{}%
\end{pgfscope}%
\end{pgfscope}%
\begin{pgfscope}%
\pgfsetbuttcap%
\pgfsetroundjoin%
\definecolor{currentfill}{rgb}{0.000000,0.000000,0.000000}%
\pgfsetfillcolor{currentfill}%
\pgfsetlinewidth{0.602250pt}%
\definecolor{currentstroke}{rgb}{0.000000,0.000000,0.000000}%
\pgfsetstrokecolor{currentstroke}%
\pgfsetdash{}{0pt}%
\pgfsys@defobject{currentmarker}{\pgfqpoint{-0.027778in}{0.000000in}}{\pgfqpoint{0.000000in}{0.000000in}}{%
\pgfpathmoveto{\pgfqpoint{0.000000in}{0.000000in}}%
\pgfpathlineto{\pgfqpoint{-0.027778in}{0.000000in}}%
\pgfusepath{stroke,fill}%
}%
\begin{pgfscope}%
\pgfsys@transformshift{0.694334in}{0.862601in}%
\pgfsys@useobject{currentmarker}{}%
\end{pgfscope}%
\end{pgfscope}%
\begin{pgfscope}%
\pgfsetbuttcap%
\pgfsetroundjoin%
\definecolor{currentfill}{rgb}{0.000000,0.000000,0.000000}%
\pgfsetfillcolor{currentfill}%
\pgfsetlinewidth{0.602250pt}%
\definecolor{currentstroke}{rgb}{0.000000,0.000000,0.000000}%
\pgfsetstrokecolor{currentstroke}%
\pgfsetdash{}{0pt}%
\pgfsys@defobject{currentmarker}{\pgfqpoint{-0.027778in}{0.000000in}}{\pgfqpoint{0.000000in}{0.000000in}}{%
\pgfpathmoveto{\pgfqpoint{0.000000in}{0.000000in}}%
\pgfpathlineto{\pgfqpoint{-0.027778in}{0.000000in}}%
\pgfusepath{stroke,fill}%
}%
\begin{pgfscope}%
\pgfsys@transformshift{0.694334in}{0.879756in}%
\pgfsys@useobject{currentmarker}{}%
\end{pgfscope}%
\end{pgfscope}%
\begin{pgfscope}%
\pgfsetbuttcap%
\pgfsetroundjoin%
\definecolor{currentfill}{rgb}{0.000000,0.000000,0.000000}%
\pgfsetfillcolor{currentfill}%
\pgfsetlinewidth{0.602250pt}%
\definecolor{currentstroke}{rgb}{0.000000,0.000000,0.000000}%
\pgfsetstrokecolor{currentstroke}%
\pgfsetdash{}{0pt}%
\pgfsys@defobject{currentmarker}{\pgfqpoint{-0.027778in}{0.000000in}}{\pgfqpoint{0.000000in}{0.000000in}}{%
\pgfpathmoveto{\pgfqpoint{0.000000in}{0.000000in}}%
\pgfpathlineto{\pgfqpoint{-0.027778in}{0.000000in}}%
\pgfusepath{stroke,fill}%
}%
\begin{pgfscope}%
\pgfsys@transformshift{0.694334in}{0.894888in}%
\pgfsys@useobject{currentmarker}{}%
\end{pgfscope}%
\end{pgfscope}%
\begin{pgfscope}%
\pgfsetbuttcap%
\pgfsetroundjoin%
\definecolor{currentfill}{rgb}{0.000000,0.000000,0.000000}%
\pgfsetfillcolor{currentfill}%
\pgfsetlinewidth{0.602250pt}%
\definecolor{currentstroke}{rgb}{0.000000,0.000000,0.000000}%
\pgfsetstrokecolor{currentstroke}%
\pgfsetdash{}{0pt}%
\pgfsys@defobject{currentmarker}{\pgfqpoint{-0.027778in}{0.000000in}}{\pgfqpoint{0.000000in}{0.000000in}}{%
\pgfpathmoveto{\pgfqpoint{0.000000in}{0.000000in}}%
\pgfpathlineto{\pgfqpoint{-0.027778in}{0.000000in}}%
\pgfusepath{stroke,fill}%
}%
\begin{pgfscope}%
\pgfsys@transformshift{0.694334in}{0.997474in}%
\pgfsys@useobject{currentmarker}{}%
\end{pgfscope}%
\end{pgfscope}%
\begin{pgfscope}%
\pgfsetbuttcap%
\pgfsetroundjoin%
\definecolor{currentfill}{rgb}{0.000000,0.000000,0.000000}%
\pgfsetfillcolor{currentfill}%
\pgfsetlinewidth{0.602250pt}%
\definecolor{currentstroke}{rgb}{0.000000,0.000000,0.000000}%
\pgfsetstrokecolor{currentstroke}%
\pgfsetdash{}{0pt}%
\pgfsys@defobject{currentmarker}{\pgfqpoint{-0.027778in}{0.000000in}}{\pgfqpoint{0.000000in}{0.000000in}}{%
\pgfpathmoveto{\pgfqpoint{0.000000in}{0.000000in}}%
\pgfpathlineto{\pgfqpoint{-0.027778in}{0.000000in}}%
\pgfusepath{stroke,fill}%
}%
\begin{pgfscope}%
\pgfsys@transformshift{0.694334in}{1.049565in}%
\pgfsys@useobject{currentmarker}{}%
\end{pgfscope}%
\end{pgfscope}%
\begin{pgfscope}%
\pgfsetbuttcap%
\pgfsetroundjoin%
\definecolor{currentfill}{rgb}{0.000000,0.000000,0.000000}%
\pgfsetfillcolor{currentfill}%
\pgfsetlinewidth{0.602250pt}%
\definecolor{currentstroke}{rgb}{0.000000,0.000000,0.000000}%
\pgfsetstrokecolor{currentstroke}%
\pgfsetdash{}{0pt}%
\pgfsys@defobject{currentmarker}{\pgfqpoint{-0.027778in}{0.000000in}}{\pgfqpoint{0.000000in}{0.000000in}}{%
\pgfpathmoveto{\pgfqpoint{0.000000in}{0.000000in}}%
\pgfpathlineto{\pgfqpoint{-0.027778in}{0.000000in}}%
\pgfusepath{stroke,fill}%
}%
\begin{pgfscope}%
\pgfsys@transformshift{0.694334in}{1.086524in}%
\pgfsys@useobject{currentmarker}{}%
\end{pgfscope}%
\end{pgfscope}%
\begin{pgfscope}%
\pgfsetbuttcap%
\pgfsetroundjoin%
\definecolor{currentfill}{rgb}{0.000000,0.000000,0.000000}%
\pgfsetfillcolor{currentfill}%
\pgfsetlinewidth{0.602250pt}%
\definecolor{currentstroke}{rgb}{0.000000,0.000000,0.000000}%
\pgfsetstrokecolor{currentstroke}%
\pgfsetdash{}{0pt}%
\pgfsys@defobject{currentmarker}{\pgfqpoint{-0.027778in}{0.000000in}}{\pgfqpoint{0.000000in}{0.000000in}}{%
\pgfpathmoveto{\pgfqpoint{0.000000in}{0.000000in}}%
\pgfpathlineto{\pgfqpoint{-0.027778in}{0.000000in}}%
\pgfusepath{stroke,fill}%
}%
\begin{pgfscope}%
\pgfsys@transformshift{0.694334in}{1.115191in}%
\pgfsys@useobject{currentmarker}{}%
\end{pgfscope}%
\end{pgfscope}%
\begin{pgfscope}%
\pgfsetbuttcap%
\pgfsetroundjoin%
\definecolor{currentfill}{rgb}{0.000000,0.000000,0.000000}%
\pgfsetfillcolor{currentfill}%
\pgfsetlinewidth{0.602250pt}%
\definecolor{currentstroke}{rgb}{0.000000,0.000000,0.000000}%
\pgfsetstrokecolor{currentstroke}%
\pgfsetdash{}{0pt}%
\pgfsys@defobject{currentmarker}{\pgfqpoint{-0.027778in}{0.000000in}}{\pgfqpoint{0.000000in}{0.000000in}}{%
\pgfpathmoveto{\pgfqpoint{0.000000in}{0.000000in}}%
\pgfpathlineto{\pgfqpoint{-0.027778in}{0.000000in}}%
\pgfusepath{stroke,fill}%
}%
\begin{pgfscope}%
\pgfsys@transformshift{0.694334in}{1.138614in}%
\pgfsys@useobject{currentmarker}{}%
\end{pgfscope}%
\end{pgfscope}%
\begin{pgfscope}%
\pgfsetbuttcap%
\pgfsetroundjoin%
\definecolor{currentfill}{rgb}{0.000000,0.000000,0.000000}%
\pgfsetfillcolor{currentfill}%
\pgfsetlinewidth{0.602250pt}%
\definecolor{currentstroke}{rgb}{0.000000,0.000000,0.000000}%
\pgfsetstrokecolor{currentstroke}%
\pgfsetdash{}{0pt}%
\pgfsys@defobject{currentmarker}{\pgfqpoint{-0.027778in}{0.000000in}}{\pgfqpoint{0.000000in}{0.000000in}}{%
\pgfpathmoveto{\pgfqpoint{0.000000in}{0.000000in}}%
\pgfpathlineto{\pgfqpoint{-0.027778in}{0.000000in}}%
\pgfusepath{stroke,fill}%
}%
\begin{pgfscope}%
\pgfsys@transformshift{0.694334in}{1.158418in}%
\pgfsys@useobject{currentmarker}{}%
\end{pgfscope}%
\end{pgfscope}%
\begin{pgfscope}%
\pgfsetbuttcap%
\pgfsetroundjoin%
\definecolor{currentfill}{rgb}{0.000000,0.000000,0.000000}%
\pgfsetfillcolor{currentfill}%
\pgfsetlinewidth{0.602250pt}%
\definecolor{currentstroke}{rgb}{0.000000,0.000000,0.000000}%
\pgfsetstrokecolor{currentstroke}%
\pgfsetdash{}{0pt}%
\pgfsys@defobject{currentmarker}{\pgfqpoint{-0.027778in}{0.000000in}}{\pgfqpoint{0.000000in}{0.000000in}}{%
\pgfpathmoveto{\pgfqpoint{0.000000in}{0.000000in}}%
\pgfpathlineto{\pgfqpoint{-0.027778in}{0.000000in}}%
\pgfusepath{stroke,fill}%
}%
\begin{pgfscope}%
\pgfsys@transformshift{0.694334in}{1.175573in}%
\pgfsys@useobject{currentmarker}{}%
\end{pgfscope}%
\end{pgfscope}%
\begin{pgfscope}%
\pgfsetbuttcap%
\pgfsetroundjoin%
\definecolor{currentfill}{rgb}{0.000000,0.000000,0.000000}%
\pgfsetfillcolor{currentfill}%
\pgfsetlinewidth{0.602250pt}%
\definecolor{currentstroke}{rgb}{0.000000,0.000000,0.000000}%
\pgfsetstrokecolor{currentstroke}%
\pgfsetdash{}{0pt}%
\pgfsys@defobject{currentmarker}{\pgfqpoint{-0.027778in}{0.000000in}}{\pgfqpoint{0.000000in}{0.000000in}}{%
\pgfpathmoveto{\pgfqpoint{0.000000in}{0.000000in}}%
\pgfpathlineto{\pgfqpoint{-0.027778in}{0.000000in}}%
\pgfusepath{stroke,fill}%
}%
\begin{pgfscope}%
\pgfsys@transformshift{0.694334in}{1.190705in}%
\pgfsys@useobject{currentmarker}{}%
\end{pgfscope}%
\end{pgfscope}%
\begin{pgfscope}%
\pgfsetbuttcap%
\pgfsetroundjoin%
\definecolor{currentfill}{rgb}{0.000000,0.000000,0.000000}%
\pgfsetfillcolor{currentfill}%
\pgfsetlinewidth{0.602250pt}%
\definecolor{currentstroke}{rgb}{0.000000,0.000000,0.000000}%
\pgfsetstrokecolor{currentstroke}%
\pgfsetdash{}{0pt}%
\pgfsys@defobject{currentmarker}{\pgfqpoint{-0.027778in}{0.000000in}}{\pgfqpoint{0.000000in}{0.000000in}}{%
\pgfpathmoveto{\pgfqpoint{0.000000in}{0.000000in}}%
\pgfpathlineto{\pgfqpoint{-0.027778in}{0.000000in}}%
\pgfusepath{stroke,fill}%
}%
\begin{pgfscope}%
\pgfsys@transformshift{0.694334in}{1.293291in}%
\pgfsys@useobject{currentmarker}{}%
\end{pgfscope}%
\end{pgfscope}%
\begin{pgfscope}%
\pgfsetbuttcap%
\pgfsetroundjoin%
\definecolor{currentfill}{rgb}{0.000000,0.000000,0.000000}%
\pgfsetfillcolor{currentfill}%
\pgfsetlinewidth{0.602250pt}%
\definecolor{currentstroke}{rgb}{0.000000,0.000000,0.000000}%
\pgfsetstrokecolor{currentstroke}%
\pgfsetdash{}{0pt}%
\pgfsys@defobject{currentmarker}{\pgfqpoint{-0.027778in}{0.000000in}}{\pgfqpoint{0.000000in}{0.000000in}}{%
\pgfpathmoveto{\pgfqpoint{0.000000in}{0.000000in}}%
\pgfpathlineto{\pgfqpoint{-0.027778in}{0.000000in}}%
\pgfusepath{stroke,fill}%
}%
\begin{pgfscope}%
\pgfsys@transformshift{0.694334in}{1.345382in}%
\pgfsys@useobject{currentmarker}{}%
\end{pgfscope}%
\end{pgfscope}%
\begin{pgfscope}%
\pgfsetbuttcap%
\pgfsetroundjoin%
\definecolor{currentfill}{rgb}{0.000000,0.000000,0.000000}%
\pgfsetfillcolor{currentfill}%
\pgfsetlinewidth{0.602250pt}%
\definecolor{currentstroke}{rgb}{0.000000,0.000000,0.000000}%
\pgfsetstrokecolor{currentstroke}%
\pgfsetdash{}{0pt}%
\pgfsys@defobject{currentmarker}{\pgfqpoint{-0.027778in}{0.000000in}}{\pgfqpoint{0.000000in}{0.000000in}}{%
\pgfpathmoveto{\pgfqpoint{0.000000in}{0.000000in}}%
\pgfpathlineto{\pgfqpoint{-0.027778in}{0.000000in}}%
\pgfusepath{stroke,fill}%
}%
\begin{pgfscope}%
\pgfsys@transformshift{0.694334in}{1.382341in}%
\pgfsys@useobject{currentmarker}{}%
\end{pgfscope}%
\end{pgfscope}%
\begin{pgfscope}%
\pgfsetbuttcap%
\pgfsetroundjoin%
\definecolor{currentfill}{rgb}{0.000000,0.000000,0.000000}%
\pgfsetfillcolor{currentfill}%
\pgfsetlinewidth{0.602250pt}%
\definecolor{currentstroke}{rgb}{0.000000,0.000000,0.000000}%
\pgfsetstrokecolor{currentstroke}%
\pgfsetdash{}{0pt}%
\pgfsys@defobject{currentmarker}{\pgfqpoint{-0.027778in}{0.000000in}}{\pgfqpoint{0.000000in}{0.000000in}}{%
\pgfpathmoveto{\pgfqpoint{0.000000in}{0.000000in}}%
\pgfpathlineto{\pgfqpoint{-0.027778in}{0.000000in}}%
\pgfusepath{stroke,fill}%
}%
\begin{pgfscope}%
\pgfsys@transformshift{0.694334in}{1.411008in}%
\pgfsys@useobject{currentmarker}{}%
\end{pgfscope}%
\end{pgfscope}%
\begin{pgfscope}%
\pgfsetbuttcap%
\pgfsetroundjoin%
\definecolor{currentfill}{rgb}{0.000000,0.000000,0.000000}%
\pgfsetfillcolor{currentfill}%
\pgfsetlinewidth{0.602250pt}%
\definecolor{currentstroke}{rgb}{0.000000,0.000000,0.000000}%
\pgfsetstrokecolor{currentstroke}%
\pgfsetdash{}{0pt}%
\pgfsys@defobject{currentmarker}{\pgfqpoint{-0.027778in}{0.000000in}}{\pgfqpoint{0.000000in}{0.000000in}}{%
\pgfpathmoveto{\pgfqpoint{0.000000in}{0.000000in}}%
\pgfpathlineto{\pgfqpoint{-0.027778in}{0.000000in}}%
\pgfusepath{stroke,fill}%
}%
\begin{pgfscope}%
\pgfsys@transformshift{0.694334in}{1.434431in}%
\pgfsys@useobject{currentmarker}{}%
\end{pgfscope}%
\end{pgfscope}%
\begin{pgfscope}%
\pgfsetbuttcap%
\pgfsetroundjoin%
\definecolor{currentfill}{rgb}{0.000000,0.000000,0.000000}%
\pgfsetfillcolor{currentfill}%
\pgfsetlinewidth{0.602250pt}%
\definecolor{currentstroke}{rgb}{0.000000,0.000000,0.000000}%
\pgfsetstrokecolor{currentstroke}%
\pgfsetdash{}{0pt}%
\pgfsys@defobject{currentmarker}{\pgfqpoint{-0.027778in}{0.000000in}}{\pgfqpoint{0.000000in}{0.000000in}}{%
\pgfpathmoveto{\pgfqpoint{0.000000in}{0.000000in}}%
\pgfpathlineto{\pgfqpoint{-0.027778in}{0.000000in}}%
\pgfusepath{stroke,fill}%
}%
\begin{pgfscope}%
\pgfsys@transformshift{0.694334in}{1.454235in}%
\pgfsys@useobject{currentmarker}{}%
\end{pgfscope}%
\end{pgfscope}%
\begin{pgfscope}%
\pgfsetbuttcap%
\pgfsetroundjoin%
\definecolor{currentfill}{rgb}{0.000000,0.000000,0.000000}%
\pgfsetfillcolor{currentfill}%
\pgfsetlinewidth{0.602250pt}%
\definecolor{currentstroke}{rgb}{0.000000,0.000000,0.000000}%
\pgfsetstrokecolor{currentstroke}%
\pgfsetdash{}{0pt}%
\pgfsys@defobject{currentmarker}{\pgfqpoint{-0.027778in}{0.000000in}}{\pgfqpoint{0.000000in}{0.000000in}}{%
\pgfpathmoveto{\pgfqpoint{0.000000in}{0.000000in}}%
\pgfpathlineto{\pgfqpoint{-0.027778in}{0.000000in}}%
\pgfusepath{stroke,fill}%
}%
\begin{pgfscope}%
\pgfsys@transformshift{0.694334in}{1.471390in}%
\pgfsys@useobject{currentmarker}{}%
\end{pgfscope}%
\end{pgfscope}%
\begin{pgfscope}%
\pgfsetbuttcap%
\pgfsetroundjoin%
\definecolor{currentfill}{rgb}{0.000000,0.000000,0.000000}%
\pgfsetfillcolor{currentfill}%
\pgfsetlinewidth{0.602250pt}%
\definecolor{currentstroke}{rgb}{0.000000,0.000000,0.000000}%
\pgfsetstrokecolor{currentstroke}%
\pgfsetdash{}{0pt}%
\pgfsys@defobject{currentmarker}{\pgfqpoint{-0.027778in}{0.000000in}}{\pgfqpoint{0.000000in}{0.000000in}}{%
\pgfpathmoveto{\pgfqpoint{0.000000in}{0.000000in}}%
\pgfpathlineto{\pgfqpoint{-0.027778in}{0.000000in}}%
\pgfusepath{stroke,fill}%
}%
\begin{pgfscope}%
\pgfsys@transformshift{0.694334in}{1.486522in}%
\pgfsys@useobject{currentmarker}{}%
\end{pgfscope}%
\end{pgfscope}%
\begin{pgfscope}%
\pgfsetbuttcap%
\pgfsetroundjoin%
\definecolor{currentfill}{rgb}{0.000000,0.000000,0.000000}%
\pgfsetfillcolor{currentfill}%
\pgfsetlinewidth{0.602250pt}%
\definecolor{currentstroke}{rgb}{0.000000,0.000000,0.000000}%
\pgfsetstrokecolor{currentstroke}%
\pgfsetdash{}{0pt}%
\pgfsys@defobject{currentmarker}{\pgfqpoint{-0.027778in}{0.000000in}}{\pgfqpoint{0.000000in}{0.000000in}}{%
\pgfpathmoveto{\pgfqpoint{0.000000in}{0.000000in}}%
\pgfpathlineto{\pgfqpoint{-0.027778in}{0.000000in}}%
\pgfusepath{stroke,fill}%
}%
\begin{pgfscope}%
\pgfsys@transformshift{0.694334in}{1.589108in}%
\pgfsys@useobject{currentmarker}{}%
\end{pgfscope}%
\end{pgfscope}%
\begin{pgfscope}%
\pgfsetbuttcap%
\pgfsetroundjoin%
\definecolor{currentfill}{rgb}{0.000000,0.000000,0.000000}%
\pgfsetfillcolor{currentfill}%
\pgfsetlinewidth{0.602250pt}%
\definecolor{currentstroke}{rgb}{0.000000,0.000000,0.000000}%
\pgfsetstrokecolor{currentstroke}%
\pgfsetdash{}{0pt}%
\pgfsys@defobject{currentmarker}{\pgfqpoint{-0.027778in}{0.000000in}}{\pgfqpoint{0.000000in}{0.000000in}}{%
\pgfpathmoveto{\pgfqpoint{0.000000in}{0.000000in}}%
\pgfpathlineto{\pgfqpoint{-0.027778in}{0.000000in}}%
\pgfusepath{stroke,fill}%
}%
\begin{pgfscope}%
\pgfsys@transformshift{0.694334in}{1.641199in}%
\pgfsys@useobject{currentmarker}{}%
\end{pgfscope}%
\end{pgfscope}%
\begin{pgfscope}%
\pgfsetbuttcap%
\pgfsetroundjoin%
\definecolor{currentfill}{rgb}{0.000000,0.000000,0.000000}%
\pgfsetfillcolor{currentfill}%
\pgfsetlinewidth{0.602250pt}%
\definecolor{currentstroke}{rgb}{0.000000,0.000000,0.000000}%
\pgfsetstrokecolor{currentstroke}%
\pgfsetdash{}{0pt}%
\pgfsys@defobject{currentmarker}{\pgfqpoint{-0.027778in}{0.000000in}}{\pgfqpoint{0.000000in}{0.000000in}}{%
\pgfpathmoveto{\pgfqpoint{0.000000in}{0.000000in}}%
\pgfpathlineto{\pgfqpoint{-0.027778in}{0.000000in}}%
\pgfusepath{stroke,fill}%
}%
\begin{pgfscope}%
\pgfsys@transformshift{0.694334in}{1.678158in}%
\pgfsys@useobject{currentmarker}{}%
\end{pgfscope}%
\end{pgfscope}%
\begin{pgfscope}%
\pgfsetbuttcap%
\pgfsetroundjoin%
\definecolor{currentfill}{rgb}{0.000000,0.000000,0.000000}%
\pgfsetfillcolor{currentfill}%
\pgfsetlinewidth{0.602250pt}%
\definecolor{currentstroke}{rgb}{0.000000,0.000000,0.000000}%
\pgfsetstrokecolor{currentstroke}%
\pgfsetdash{}{0pt}%
\pgfsys@defobject{currentmarker}{\pgfqpoint{-0.027778in}{0.000000in}}{\pgfqpoint{0.000000in}{0.000000in}}{%
\pgfpathmoveto{\pgfqpoint{0.000000in}{0.000000in}}%
\pgfpathlineto{\pgfqpoint{-0.027778in}{0.000000in}}%
\pgfusepath{stroke,fill}%
}%
\begin{pgfscope}%
\pgfsys@transformshift{0.694334in}{1.706825in}%
\pgfsys@useobject{currentmarker}{}%
\end{pgfscope}%
\end{pgfscope}%
\begin{pgfscope}%
\pgfsetbuttcap%
\pgfsetroundjoin%
\definecolor{currentfill}{rgb}{0.000000,0.000000,0.000000}%
\pgfsetfillcolor{currentfill}%
\pgfsetlinewidth{0.602250pt}%
\definecolor{currentstroke}{rgb}{0.000000,0.000000,0.000000}%
\pgfsetstrokecolor{currentstroke}%
\pgfsetdash{}{0pt}%
\pgfsys@defobject{currentmarker}{\pgfqpoint{-0.027778in}{0.000000in}}{\pgfqpoint{0.000000in}{0.000000in}}{%
\pgfpathmoveto{\pgfqpoint{0.000000in}{0.000000in}}%
\pgfpathlineto{\pgfqpoint{-0.027778in}{0.000000in}}%
\pgfusepath{stroke,fill}%
}%
\begin{pgfscope}%
\pgfsys@transformshift{0.694334in}{1.730248in}%
\pgfsys@useobject{currentmarker}{}%
\end{pgfscope}%
\end{pgfscope}%
\begin{pgfscope}%
\pgfsetbuttcap%
\pgfsetroundjoin%
\definecolor{currentfill}{rgb}{0.000000,0.000000,0.000000}%
\pgfsetfillcolor{currentfill}%
\pgfsetlinewidth{0.602250pt}%
\definecolor{currentstroke}{rgb}{0.000000,0.000000,0.000000}%
\pgfsetstrokecolor{currentstroke}%
\pgfsetdash{}{0pt}%
\pgfsys@defobject{currentmarker}{\pgfqpoint{-0.027778in}{0.000000in}}{\pgfqpoint{0.000000in}{0.000000in}}{%
\pgfpathmoveto{\pgfqpoint{0.000000in}{0.000000in}}%
\pgfpathlineto{\pgfqpoint{-0.027778in}{0.000000in}}%
\pgfusepath{stroke,fill}%
}%
\begin{pgfscope}%
\pgfsys@transformshift{0.694334in}{1.750052in}%
\pgfsys@useobject{currentmarker}{}%
\end{pgfscope}%
\end{pgfscope}%
\begin{pgfscope}%
\pgfsetbuttcap%
\pgfsetroundjoin%
\definecolor{currentfill}{rgb}{0.000000,0.000000,0.000000}%
\pgfsetfillcolor{currentfill}%
\pgfsetlinewidth{0.602250pt}%
\definecolor{currentstroke}{rgb}{0.000000,0.000000,0.000000}%
\pgfsetstrokecolor{currentstroke}%
\pgfsetdash{}{0pt}%
\pgfsys@defobject{currentmarker}{\pgfqpoint{-0.027778in}{0.000000in}}{\pgfqpoint{0.000000in}{0.000000in}}{%
\pgfpathmoveto{\pgfqpoint{0.000000in}{0.000000in}}%
\pgfpathlineto{\pgfqpoint{-0.027778in}{0.000000in}}%
\pgfusepath{stroke,fill}%
}%
\begin{pgfscope}%
\pgfsys@transformshift{0.694334in}{1.767207in}%
\pgfsys@useobject{currentmarker}{}%
\end{pgfscope}%
\end{pgfscope}%
\begin{pgfscope}%
\pgfsetbuttcap%
\pgfsetroundjoin%
\definecolor{currentfill}{rgb}{0.000000,0.000000,0.000000}%
\pgfsetfillcolor{currentfill}%
\pgfsetlinewidth{0.602250pt}%
\definecolor{currentstroke}{rgb}{0.000000,0.000000,0.000000}%
\pgfsetstrokecolor{currentstroke}%
\pgfsetdash{}{0pt}%
\pgfsys@defobject{currentmarker}{\pgfqpoint{-0.027778in}{0.000000in}}{\pgfqpoint{0.000000in}{0.000000in}}{%
\pgfpathmoveto{\pgfqpoint{0.000000in}{0.000000in}}%
\pgfpathlineto{\pgfqpoint{-0.027778in}{0.000000in}}%
\pgfusepath{stroke,fill}%
}%
\begin{pgfscope}%
\pgfsys@transformshift{0.694334in}{1.782339in}%
\pgfsys@useobject{currentmarker}{}%
\end{pgfscope}%
\end{pgfscope}%
\begin{pgfscope}%
\pgfsetbuttcap%
\pgfsetroundjoin%
\definecolor{currentfill}{rgb}{0.000000,0.000000,0.000000}%
\pgfsetfillcolor{currentfill}%
\pgfsetlinewidth{0.602250pt}%
\definecolor{currentstroke}{rgb}{0.000000,0.000000,0.000000}%
\pgfsetstrokecolor{currentstroke}%
\pgfsetdash{}{0pt}%
\pgfsys@defobject{currentmarker}{\pgfqpoint{-0.027778in}{0.000000in}}{\pgfqpoint{0.000000in}{0.000000in}}{%
\pgfpathmoveto{\pgfqpoint{0.000000in}{0.000000in}}%
\pgfpathlineto{\pgfqpoint{-0.027778in}{0.000000in}}%
\pgfusepath{stroke,fill}%
}%
\begin{pgfscope}%
\pgfsys@transformshift{0.694334in}{1.884925in}%
\pgfsys@useobject{currentmarker}{}%
\end{pgfscope}%
\end{pgfscope}%
\begin{pgfscope}%
\pgfsetbuttcap%
\pgfsetroundjoin%
\definecolor{currentfill}{rgb}{0.000000,0.000000,0.000000}%
\pgfsetfillcolor{currentfill}%
\pgfsetlinewidth{0.602250pt}%
\definecolor{currentstroke}{rgb}{0.000000,0.000000,0.000000}%
\pgfsetstrokecolor{currentstroke}%
\pgfsetdash{}{0pt}%
\pgfsys@defobject{currentmarker}{\pgfqpoint{-0.027778in}{0.000000in}}{\pgfqpoint{0.000000in}{0.000000in}}{%
\pgfpathmoveto{\pgfqpoint{0.000000in}{0.000000in}}%
\pgfpathlineto{\pgfqpoint{-0.027778in}{0.000000in}}%
\pgfusepath{stroke,fill}%
}%
\begin{pgfscope}%
\pgfsys@transformshift{0.694334in}{1.937016in}%
\pgfsys@useobject{currentmarker}{}%
\end{pgfscope}%
\end{pgfscope}%
\begin{pgfscope}%
\pgfsetbuttcap%
\pgfsetroundjoin%
\definecolor{currentfill}{rgb}{0.000000,0.000000,0.000000}%
\pgfsetfillcolor{currentfill}%
\pgfsetlinewidth{0.602250pt}%
\definecolor{currentstroke}{rgb}{0.000000,0.000000,0.000000}%
\pgfsetstrokecolor{currentstroke}%
\pgfsetdash{}{0pt}%
\pgfsys@defobject{currentmarker}{\pgfqpoint{-0.027778in}{0.000000in}}{\pgfqpoint{0.000000in}{0.000000in}}{%
\pgfpathmoveto{\pgfqpoint{0.000000in}{0.000000in}}%
\pgfpathlineto{\pgfqpoint{-0.027778in}{0.000000in}}%
\pgfusepath{stroke,fill}%
}%
\begin{pgfscope}%
\pgfsys@transformshift{0.694334in}{1.973975in}%
\pgfsys@useobject{currentmarker}{}%
\end{pgfscope}%
\end{pgfscope}%
\begin{pgfscope}%
\pgfsetbuttcap%
\pgfsetroundjoin%
\definecolor{currentfill}{rgb}{0.000000,0.000000,0.000000}%
\pgfsetfillcolor{currentfill}%
\pgfsetlinewidth{0.602250pt}%
\definecolor{currentstroke}{rgb}{0.000000,0.000000,0.000000}%
\pgfsetstrokecolor{currentstroke}%
\pgfsetdash{}{0pt}%
\pgfsys@defobject{currentmarker}{\pgfqpoint{-0.027778in}{0.000000in}}{\pgfqpoint{0.000000in}{0.000000in}}{%
\pgfpathmoveto{\pgfqpoint{0.000000in}{0.000000in}}%
\pgfpathlineto{\pgfqpoint{-0.027778in}{0.000000in}}%
\pgfusepath{stroke,fill}%
}%
\begin{pgfscope}%
\pgfsys@transformshift{0.694334in}{2.002642in}%
\pgfsys@useobject{currentmarker}{}%
\end{pgfscope}%
\end{pgfscope}%
\begin{pgfscope}%
\pgfsetbuttcap%
\pgfsetroundjoin%
\definecolor{currentfill}{rgb}{0.000000,0.000000,0.000000}%
\pgfsetfillcolor{currentfill}%
\pgfsetlinewidth{0.602250pt}%
\definecolor{currentstroke}{rgb}{0.000000,0.000000,0.000000}%
\pgfsetstrokecolor{currentstroke}%
\pgfsetdash{}{0pt}%
\pgfsys@defobject{currentmarker}{\pgfqpoint{-0.027778in}{0.000000in}}{\pgfqpoint{0.000000in}{0.000000in}}{%
\pgfpathmoveto{\pgfqpoint{0.000000in}{0.000000in}}%
\pgfpathlineto{\pgfqpoint{-0.027778in}{0.000000in}}%
\pgfusepath{stroke,fill}%
}%
\begin{pgfscope}%
\pgfsys@transformshift{0.694334in}{2.026065in}%
\pgfsys@useobject{currentmarker}{}%
\end{pgfscope}%
\end{pgfscope}%
\begin{pgfscope}%
\pgfsetbuttcap%
\pgfsetroundjoin%
\definecolor{currentfill}{rgb}{0.000000,0.000000,0.000000}%
\pgfsetfillcolor{currentfill}%
\pgfsetlinewidth{0.602250pt}%
\definecolor{currentstroke}{rgb}{0.000000,0.000000,0.000000}%
\pgfsetstrokecolor{currentstroke}%
\pgfsetdash{}{0pt}%
\pgfsys@defobject{currentmarker}{\pgfqpoint{-0.027778in}{0.000000in}}{\pgfqpoint{0.000000in}{0.000000in}}{%
\pgfpathmoveto{\pgfqpoint{0.000000in}{0.000000in}}%
\pgfpathlineto{\pgfqpoint{-0.027778in}{0.000000in}}%
\pgfusepath{stroke,fill}%
}%
\begin{pgfscope}%
\pgfsys@transformshift{0.694334in}{2.045869in}%
\pgfsys@useobject{currentmarker}{}%
\end{pgfscope}%
\end{pgfscope}%
\begin{pgfscope}%
\pgfsetbuttcap%
\pgfsetroundjoin%
\definecolor{currentfill}{rgb}{0.000000,0.000000,0.000000}%
\pgfsetfillcolor{currentfill}%
\pgfsetlinewidth{0.602250pt}%
\definecolor{currentstroke}{rgb}{0.000000,0.000000,0.000000}%
\pgfsetstrokecolor{currentstroke}%
\pgfsetdash{}{0pt}%
\pgfsys@defobject{currentmarker}{\pgfqpoint{-0.027778in}{0.000000in}}{\pgfqpoint{0.000000in}{0.000000in}}{%
\pgfpathmoveto{\pgfqpoint{0.000000in}{0.000000in}}%
\pgfpathlineto{\pgfqpoint{-0.027778in}{0.000000in}}%
\pgfusepath{stroke,fill}%
}%
\begin{pgfscope}%
\pgfsys@transformshift{0.694334in}{2.063024in}%
\pgfsys@useobject{currentmarker}{}%
\end{pgfscope}%
\end{pgfscope}%
\begin{pgfscope}%
\pgfsetbuttcap%
\pgfsetroundjoin%
\definecolor{currentfill}{rgb}{0.000000,0.000000,0.000000}%
\pgfsetfillcolor{currentfill}%
\pgfsetlinewidth{0.602250pt}%
\definecolor{currentstroke}{rgb}{0.000000,0.000000,0.000000}%
\pgfsetstrokecolor{currentstroke}%
\pgfsetdash{}{0pt}%
\pgfsys@defobject{currentmarker}{\pgfqpoint{-0.027778in}{0.000000in}}{\pgfqpoint{0.000000in}{0.000000in}}{%
\pgfpathmoveto{\pgfqpoint{0.000000in}{0.000000in}}%
\pgfpathlineto{\pgfqpoint{-0.027778in}{0.000000in}}%
\pgfusepath{stroke,fill}%
}%
\begin{pgfscope}%
\pgfsys@transformshift{0.694334in}{2.078156in}%
\pgfsys@useobject{currentmarker}{}%
\end{pgfscope}%
\end{pgfscope}%
\begin{pgfscope}%
\definecolor{textcolor}{rgb}{0.000000,0.000000,0.000000}%
\pgfsetstrokecolor{textcolor}%
\pgfsetfillcolor{textcolor}%
\pgftext[x=0.274969in,y=1.307625in,,bottom,rotate=90.000000]{\color{textcolor}\rmfamily\fontsize{9.000000}{10.800000}\selectfont Longest solving time (s)}%
\end{pgfscope}%
\begin{pgfscope}%
\pgfpathrectangle{\pgfqpoint{0.694334in}{0.523557in}}{\pgfqpoint{3.830343in}{1.568135in}}%
\pgfusepath{clip}%
\pgfsetrectcap%
\pgfsetroundjoin%
\pgfsetlinewidth{1.003750pt}%
\definecolor{currentstroke}{rgb}{0.878431,0.878431,0.815686}%
\pgfsetstrokecolor{currentstroke}%
\pgfsetdash{}{0pt}%
\pgfpathmoveto{\pgfqpoint{0.694334in}{1.035820in}}%
\pgfpathlineto{\pgfqpoint{0.698165in}{1.043143in}}%
\pgfpathlineto{\pgfqpoint{0.701995in}{1.044740in}}%
\pgfpathlineto{\pgfqpoint{0.707741in}{1.046090in}}%
\pgfpathlineto{\pgfqpoint{0.713486in}{1.047875in}}%
\pgfpathlineto{\pgfqpoint{0.715401in}{1.048008in}}%
\pgfpathlineto{\pgfqpoint{0.717316in}{1.050549in}}%
\pgfpathlineto{\pgfqpoint{0.723062in}{1.052125in}}%
\pgfpathlineto{\pgfqpoint{0.738383in}{1.059393in}}%
\pgfpathlineto{\pgfqpoint{0.740298in}{1.061343in}}%
\pgfpathlineto{\pgfqpoint{0.749874in}{1.063266in}}%
\pgfpathlineto{\pgfqpoint{0.761365in}{1.064951in}}%
\pgfpathlineto{\pgfqpoint{0.774772in}{1.066205in}}%
\pgfpathlineto{\pgfqpoint{0.790093in}{1.067591in}}%
\pgfpathlineto{\pgfqpoint{0.799669in}{1.068482in}}%
\pgfpathlineto{\pgfqpoint{0.805414in}{1.069179in}}%
\pgfpathlineto{\pgfqpoint{0.814990in}{1.069837in}}%
\pgfpathlineto{\pgfqpoint{0.818820in}{1.071494in}}%
\pgfpathlineto{\pgfqpoint{0.830311in}{1.072730in}}%
\pgfpathlineto{\pgfqpoint{0.847548in}{1.074750in}}%
\pgfpathlineto{\pgfqpoint{0.859039in}{1.075196in}}%
\pgfpathlineto{\pgfqpoint{0.864785in}{1.077052in}}%
\pgfpathlineto{\pgfqpoint{0.927985in}{1.082074in}}%
\pgfpathlineto{\pgfqpoint{0.935646in}{1.082838in}}%
\pgfpathlineto{\pgfqpoint{0.943307in}{1.083945in}}%
\pgfpathlineto{\pgfqpoint{0.950967in}{1.085003in}}%
\pgfpathlineto{\pgfqpoint{0.966289in}{1.086724in}}%
\pgfpathlineto{\pgfqpoint{1.010338in}{1.090935in}}%
\pgfpathlineto{\pgfqpoint{1.014168in}{1.092298in}}%
\pgfpathlineto{\pgfqpoint{1.046726in}{1.095005in}}%
\pgfpathlineto{\pgfqpoint{1.056302in}{1.096922in}}%
\pgfpathlineto{\pgfqpoint{1.069708in}{1.098116in}}%
\pgfpathlineto{\pgfqpoint{1.108011in}{1.101444in}}%
\pgfpathlineto{\pgfqpoint{1.111842in}{1.103462in}}%
\pgfpathlineto{\pgfqpoint{1.129078in}{1.104949in}}%
\pgfpathlineto{\pgfqpoint{1.138654in}{1.106967in}}%
\pgfpathlineto{\pgfqpoint{1.148230in}{1.108109in}}%
\pgfpathlineto{\pgfqpoint{1.163551in}{1.113159in}}%
\pgfpathlineto{\pgfqpoint{1.175042in}{1.116189in}}%
\pgfpathlineto{\pgfqpoint{1.184618in}{1.119444in}}%
\pgfpathlineto{\pgfqpoint{1.192279in}{1.120498in}}%
\pgfpathlineto{\pgfqpoint{1.201855in}{1.122889in}}%
\pgfpathlineto{\pgfqpoint{1.221006in}{1.124591in}}%
\pgfpathlineto{\pgfqpoint{1.226752in}{1.125618in}}%
\pgfpathlineto{\pgfqpoint{1.266971in}{1.131687in}}%
\pgfpathlineto{\pgfqpoint{1.276546in}{1.134442in}}%
\pgfpathlineto{\pgfqpoint{1.282292in}{1.136503in}}%
\pgfpathlineto{\pgfqpoint{1.289953in}{1.137879in}}%
\pgfpathlineto{\pgfqpoint{1.353153in}{1.148858in}}%
\pgfpathlineto{\pgfqpoint{1.358899in}{1.149904in}}%
\pgfpathlineto{\pgfqpoint{1.366559in}{1.150750in}}%
\pgfpathlineto{\pgfqpoint{1.372305in}{1.151920in}}%
\pgfpathlineto{\pgfqpoint{1.379966in}{1.153666in}}%
\pgfpathlineto{\pgfqpoint{1.402948in}{1.156079in}}%
\pgfpathlineto{\pgfqpoint{1.473809in}{1.167485in}}%
\pgfpathlineto{\pgfqpoint{1.477639in}{1.168859in}}%
\pgfpathlineto{\pgfqpoint{1.483385in}{1.169310in}}%
\pgfpathlineto{\pgfqpoint{1.489130in}{1.171318in}}%
\pgfpathlineto{\pgfqpoint{1.504452in}{1.175921in}}%
\pgfpathlineto{\pgfqpoint{1.515943in}{1.177078in}}%
\pgfpathlineto{\pgfqpoint{1.519773in}{1.178252in}}%
\pgfpathlineto{\pgfqpoint{1.533179in}{1.180304in}}%
\pgfpathlineto{\pgfqpoint{1.538925in}{1.181476in}}%
\pgfpathlineto{\pgfqpoint{1.563822in}{1.186357in}}%
\pgfpathlineto{\pgfqpoint{1.567652in}{1.187798in}}%
\pgfpathlineto{\pgfqpoint{1.577228in}{1.190610in}}%
\pgfpathlineto{\pgfqpoint{1.590635in}{1.193945in}}%
\pgfpathlineto{\pgfqpoint{1.594465in}{1.194428in}}%
\pgfpathlineto{\pgfqpoint{1.596380in}{1.197337in}}%
\pgfpathlineto{\pgfqpoint{1.627023in}{1.201738in}}%
\pgfpathlineto{\pgfqpoint{1.634683in}{1.203439in}}%
\pgfpathlineto{\pgfqpoint{1.640429in}{1.204411in}}%
\pgfpathlineto{\pgfqpoint{1.644259in}{1.205926in}}%
\pgfpathlineto{\pgfqpoint{1.646174in}{1.206081in}}%
\pgfpathlineto{\pgfqpoint{1.650005in}{1.207522in}}%
\pgfpathlineto{\pgfqpoint{1.665326in}{1.209667in}}%
\pgfpathlineto{\pgfqpoint{1.692139in}{1.211183in}}%
\pgfpathlineto{\pgfqpoint{1.722781in}{1.212343in}}%
\pgfpathlineto{\pgfqpoint{1.778321in}{1.216537in}}%
\pgfpathlineto{\pgfqpoint{1.787897in}{1.218253in}}%
\pgfpathlineto{\pgfqpoint{1.803219in}{1.220829in}}%
\pgfpathlineto{\pgfqpoint{1.826201in}{1.225068in}}%
\pgfpathlineto{\pgfqpoint{1.835776in}{1.228273in}}%
\pgfpathlineto{\pgfqpoint{1.839607in}{1.228907in}}%
\pgfpathlineto{\pgfqpoint{1.845352in}{1.229723in}}%
\pgfpathlineto{\pgfqpoint{1.851098in}{1.231333in}}%
\pgfpathlineto{\pgfqpoint{1.856843in}{1.232873in}}%
\pgfpathlineto{\pgfqpoint{1.864504in}{1.234077in}}%
\pgfpathlineto{\pgfqpoint{1.872165in}{1.236154in}}%
\pgfpathlineto{\pgfqpoint{1.879825in}{1.237321in}}%
\pgfpathlineto{\pgfqpoint{1.889401in}{1.239711in}}%
\pgfpathlineto{\pgfqpoint{1.895147in}{1.241635in}}%
\pgfpathlineto{\pgfqpoint{1.904723in}{1.242896in}}%
\pgfpathlineto{\pgfqpoint{1.920044in}{1.244808in}}%
\pgfpathlineto{\pgfqpoint{1.925790in}{1.246241in}}%
\pgfpathlineto{\pgfqpoint{1.933450in}{1.247735in}}%
\pgfpathlineto{\pgfqpoint{1.937281in}{1.248114in}}%
\pgfpathlineto{\pgfqpoint{1.941111in}{1.249308in}}%
\pgfpathlineto{\pgfqpoint{1.943026in}{1.252115in}}%
\pgfpathlineto{\pgfqpoint{1.948772in}{1.252841in}}%
\pgfpathlineto{\pgfqpoint{1.952602in}{1.254432in}}%
\pgfpathlineto{\pgfqpoint{1.962178in}{1.256169in}}%
\pgfpathlineto{\pgfqpoint{1.969838in}{1.258310in}}%
\pgfpathlineto{\pgfqpoint{1.983245in}{1.260362in}}%
\pgfpathlineto{\pgfqpoint{1.994736in}{1.263106in}}%
\pgfpathlineto{\pgfqpoint{2.000481in}{1.263987in}}%
\pgfpathlineto{\pgfqpoint{2.010057in}{1.265456in}}%
\pgfpathlineto{\pgfqpoint{2.029209in}{1.267320in}}%
\pgfpathlineto{\pgfqpoint{2.034954in}{1.268682in}}%
\pgfpathlineto{\pgfqpoint{2.048360in}{1.270676in}}%
\pgfpathlineto{\pgfqpoint{2.057936in}{1.272895in}}%
\pgfpathlineto{\pgfqpoint{2.067512in}{1.274501in}}%
\pgfpathlineto{\pgfqpoint{2.077088in}{1.277482in}}%
\pgfpathlineto{\pgfqpoint{2.094325in}{1.279606in}}%
\pgfpathlineto{\pgfqpoint{2.103900in}{1.282976in}}%
\pgfpathlineto{\pgfqpoint{2.111561in}{1.284907in}}%
\pgfpathlineto{\pgfqpoint{2.119222in}{1.286738in}}%
\pgfpathlineto{\pgfqpoint{2.123052in}{1.287587in}}%
\pgfpathlineto{\pgfqpoint{2.136458in}{1.293250in}}%
\pgfpathlineto{\pgfqpoint{2.149865in}{1.297071in}}%
\pgfpathlineto{\pgfqpoint{2.157525in}{1.299344in}}%
\pgfpathlineto{\pgfqpoint{2.167101in}{1.302143in}}%
\pgfpathlineto{\pgfqpoint{2.172847in}{1.303808in}}%
\pgfpathlineto{\pgfqpoint{2.193914in}{1.309548in}}%
\pgfpathlineto{\pgfqpoint{2.197744in}{1.311517in}}%
\pgfpathlineto{\pgfqpoint{2.207320in}{1.312765in}}%
\pgfpathlineto{\pgfqpoint{2.211150in}{1.313656in}}%
\pgfpathlineto{\pgfqpoint{2.226471in}{1.315485in}}%
\pgfpathlineto{\pgfqpoint{2.232217in}{1.318284in}}%
\pgfpathlineto{\pgfqpoint{2.236047in}{1.319585in}}%
\pgfpathlineto{\pgfqpoint{2.239878in}{1.321208in}}%
\pgfpathlineto{\pgfqpoint{2.249453in}{1.322928in}}%
\pgfpathlineto{\pgfqpoint{2.266690in}{1.326337in}}%
\pgfpathlineto{\pgfqpoint{2.268605in}{1.327842in}}%
\pgfpathlineto{\pgfqpoint{2.276266in}{1.329164in}}%
\pgfpathlineto{\pgfqpoint{2.293502in}{1.335179in}}%
\pgfpathlineto{\pgfqpoint{2.299248in}{1.335975in}}%
\pgfpathlineto{\pgfqpoint{2.316484in}{1.343225in}}%
\pgfpathlineto{\pgfqpoint{2.322230in}{1.343597in}}%
\pgfpathlineto{\pgfqpoint{2.324145in}{1.346219in}}%
\pgfpathlineto{\pgfqpoint{2.329891in}{1.347503in}}%
\pgfpathlineto{\pgfqpoint{2.333721in}{1.348769in}}%
\pgfpathlineto{\pgfqpoint{2.373940in}{1.355862in}}%
\pgfpathlineto{\pgfqpoint{2.377770in}{1.356867in}}%
\pgfpathlineto{\pgfqpoint{2.381600in}{1.358019in}}%
\pgfpathlineto{\pgfqpoint{2.389261in}{1.358724in}}%
\pgfpathlineto{\pgfqpoint{2.393091in}{1.362368in}}%
\pgfpathlineto{\pgfqpoint{2.414158in}{1.366204in}}%
\pgfpathlineto{\pgfqpoint{2.450546in}{1.376529in}}%
\pgfpathlineto{\pgfqpoint{2.452462in}{1.379144in}}%
\pgfpathlineto{\pgfqpoint{2.462037in}{1.381203in}}%
\pgfpathlineto{\pgfqpoint{2.467783in}{1.383127in}}%
\pgfpathlineto{\pgfqpoint{2.473529in}{1.383564in}}%
\pgfpathlineto{\pgfqpoint{2.475444in}{1.385710in}}%
\pgfpathlineto{\pgfqpoint{2.485020in}{1.388557in}}%
\pgfpathlineto{\pgfqpoint{2.496511in}{1.392756in}}%
\pgfpathlineto{\pgfqpoint{2.511832in}{1.395419in}}%
\pgfpathlineto{\pgfqpoint{2.513747in}{1.397190in}}%
\pgfpathlineto{\pgfqpoint{2.519493in}{1.398152in}}%
\pgfpathlineto{\pgfqpoint{2.523323in}{1.399915in}}%
\pgfpathlineto{\pgfqpoint{2.530984in}{1.401404in}}%
\pgfpathlineto{\pgfqpoint{2.536729in}{1.402701in}}%
\pgfpathlineto{\pgfqpoint{2.544390in}{1.404882in}}%
\pgfpathlineto{\pgfqpoint{2.546305in}{1.404883in}}%
\pgfpathlineto{\pgfqpoint{2.548220in}{1.408015in}}%
\pgfpathlineto{\pgfqpoint{2.559711in}{1.411897in}}%
\pgfpathlineto{\pgfqpoint{2.576948in}{1.416467in}}%
\pgfpathlineto{\pgfqpoint{2.584608in}{1.417743in}}%
\pgfpathlineto{\pgfqpoint{2.599930in}{1.420516in}}%
\pgfpathlineto{\pgfqpoint{2.603760in}{1.422384in}}%
\pgfpathlineto{\pgfqpoint{2.615251in}{1.425122in}}%
\pgfpathlineto{\pgfqpoint{2.620997in}{1.425775in}}%
\pgfpathlineto{\pgfqpoint{2.626742in}{1.426924in}}%
\pgfpathlineto{\pgfqpoint{2.628657in}{1.427264in}}%
\pgfpathlineto{\pgfqpoint{2.632488in}{1.429721in}}%
\pgfpathlineto{\pgfqpoint{2.659300in}{1.435551in}}%
\pgfpathlineto{\pgfqpoint{2.663130in}{1.437479in}}%
\pgfpathlineto{\pgfqpoint{2.680367in}{1.439905in}}%
\pgfpathlineto{\pgfqpoint{2.686113in}{1.441263in}}%
\pgfpathlineto{\pgfqpoint{2.697604in}{1.443079in}}%
\pgfpathlineto{\pgfqpoint{2.701434in}{1.445118in}}%
\pgfpathlineto{\pgfqpoint{2.703349in}{1.445125in}}%
\pgfpathlineto{\pgfqpoint{2.707179in}{1.446471in}}%
\pgfpathlineto{\pgfqpoint{2.712925in}{1.447307in}}%
\pgfpathlineto{\pgfqpoint{2.716755in}{1.449895in}}%
\pgfpathlineto{\pgfqpoint{2.722501in}{1.451149in}}%
\pgfpathlineto{\pgfqpoint{2.728246in}{1.452415in}}%
\pgfpathlineto{\pgfqpoint{2.747398in}{1.456836in}}%
\pgfpathlineto{\pgfqpoint{2.755059in}{1.460437in}}%
\pgfpathlineto{\pgfqpoint{2.816344in}{1.474913in}}%
\pgfpathlineto{\pgfqpoint{2.820175in}{1.477718in}}%
\pgfpathlineto{\pgfqpoint{2.831666in}{1.479274in}}%
\pgfpathlineto{\pgfqpoint{2.866139in}{1.486737in}}%
\pgfpathlineto{\pgfqpoint{2.894866in}{1.492527in}}%
\pgfpathlineto{\pgfqpoint{2.896781in}{1.493112in}}%
\pgfpathlineto{\pgfqpoint{2.900612in}{1.496882in}}%
\pgfpathlineto{\pgfqpoint{2.906357in}{1.498566in}}%
\pgfpathlineto{\pgfqpoint{2.919763in}{1.501199in}}%
\pgfpathlineto{\pgfqpoint{2.921679in}{1.504041in}}%
\pgfpathlineto{\pgfqpoint{2.931254in}{1.506810in}}%
\pgfpathlineto{\pgfqpoint{2.935085in}{1.508229in}}%
\pgfpathlineto{\pgfqpoint{2.940830in}{1.510120in}}%
\pgfpathlineto{\pgfqpoint{2.942746in}{1.510406in}}%
\pgfpathlineto{\pgfqpoint{2.946576in}{1.512116in}}%
\pgfpathlineto{\pgfqpoint{2.952321in}{1.513042in}}%
\pgfpathlineto{\pgfqpoint{2.961897in}{1.517151in}}%
\pgfpathlineto{\pgfqpoint{2.967643in}{1.517855in}}%
\pgfpathlineto{\pgfqpoint{2.971473in}{1.519234in}}%
\pgfpathlineto{\pgfqpoint{2.990625in}{1.522841in}}%
\pgfpathlineto{\pgfqpoint{3.002116in}{1.525585in}}%
\pgfpathlineto{\pgfqpoint{3.004031in}{1.527398in}}%
\pgfpathlineto{\pgfqpoint{3.007861in}{1.527532in}}%
\pgfpathlineto{\pgfqpoint{3.011692in}{1.530690in}}%
\pgfpathlineto{\pgfqpoint{3.021268in}{1.531712in}}%
\pgfpathlineto{\pgfqpoint{3.025098in}{1.532727in}}%
\pgfpathlineto{\pgfqpoint{3.028928in}{1.533165in}}%
\pgfpathlineto{\pgfqpoint{3.032759in}{1.538032in}}%
\pgfpathlineto{\pgfqpoint{3.034674in}{1.538280in}}%
\pgfpathlineto{\pgfqpoint{3.036589in}{1.539766in}}%
\pgfpathlineto{\pgfqpoint{3.044250in}{1.540149in}}%
\pgfpathlineto{\pgfqpoint{3.048080in}{1.543302in}}%
\pgfpathlineto{\pgfqpoint{3.053825in}{1.544152in}}%
\pgfpathlineto{\pgfqpoint{3.067232in}{1.549682in}}%
\pgfpathlineto{\pgfqpoint{3.078723in}{1.552095in}}%
\pgfpathlineto{\pgfqpoint{3.080638in}{1.553767in}}%
\pgfpathlineto{\pgfqpoint{3.088299in}{1.555347in}}%
\pgfpathlineto{\pgfqpoint{3.099790in}{1.557278in}}%
\pgfpathlineto{\pgfqpoint{3.118941in}{1.566171in}}%
\pgfpathlineto{\pgfqpoint{3.126602in}{1.567217in}}%
\pgfpathlineto{\pgfqpoint{3.134263in}{1.569741in}}%
\pgfpathlineto{\pgfqpoint{3.138093in}{1.569908in}}%
\pgfpathlineto{\pgfqpoint{3.140008in}{1.572792in}}%
\pgfpathlineto{\pgfqpoint{3.155330in}{1.575211in}}%
\pgfpathlineto{\pgfqpoint{3.168736in}{1.577241in}}%
\pgfpathlineto{\pgfqpoint{3.172566in}{1.580074in}}%
\pgfpathlineto{\pgfqpoint{3.178312in}{1.582952in}}%
\pgfpathlineto{\pgfqpoint{3.189803in}{1.586185in}}%
\pgfpathlineto{\pgfqpoint{3.197463in}{1.588221in}}%
\pgfpathlineto{\pgfqpoint{3.201294in}{1.592474in}}%
\pgfpathlineto{\pgfqpoint{3.203209in}{1.592709in}}%
\pgfpathlineto{\pgfqpoint{3.205124in}{1.595210in}}%
\pgfpathlineto{\pgfqpoint{3.208954in}{1.596444in}}%
\pgfpathlineto{\pgfqpoint{3.212785in}{1.599298in}}%
\pgfpathlineto{\pgfqpoint{3.216615in}{1.599440in}}%
\pgfpathlineto{\pgfqpoint{3.228106in}{1.607153in}}%
\pgfpathlineto{\pgfqpoint{3.235767in}{1.610243in}}%
\pgfpathlineto{\pgfqpoint{3.241512in}{1.610683in}}%
\pgfpathlineto{\pgfqpoint{3.251088in}{1.616299in}}%
\pgfpathlineto{\pgfqpoint{3.254918in}{1.617604in}}%
\pgfpathlineto{\pgfqpoint{3.256834in}{1.620729in}}%
\pgfpathlineto{\pgfqpoint{3.264494in}{1.622958in}}%
\pgfpathlineto{\pgfqpoint{3.274070in}{1.624049in}}%
\pgfpathlineto{\pgfqpoint{3.277901in}{1.625452in}}%
\pgfpathlineto{\pgfqpoint{3.281731in}{1.628830in}}%
\pgfpathlineto{\pgfqpoint{3.289392in}{1.630516in}}%
\pgfpathlineto{\pgfqpoint{3.304713in}{1.640370in}}%
\pgfpathlineto{\pgfqpoint{3.308543in}{1.642124in}}%
\pgfpathlineto{\pgfqpoint{3.312374in}{1.643651in}}%
\pgfpathlineto{\pgfqpoint{3.316204in}{1.645851in}}%
\pgfpathlineto{\pgfqpoint{3.321949in}{1.647214in}}%
\pgfpathlineto{\pgfqpoint{3.329610in}{1.653922in}}%
\pgfpathlineto{\pgfqpoint{3.333440in}{1.655583in}}%
\pgfpathlineto{\pgfqpoint{3.346847in}{1.660461in}}%
\pgfpathlineto{\pgfqpoint{3.358338in}{1.664792in}}%
\pgfpathlineto{\pgfqpoint{3.364083in}{1.670592in}}%
\pgfpathlineto{\pgfqpoint{3.369829in}{1.672063in}}%
\pgfpathlineto{\pgfqpoint{3.371744in}{1.674873in}}%
\pgfpathlineto{\pgfqpoint{3.375574in}{1.676344in}}%
\pgfpathlineto{\pgfqpoint{3.379405in}{1.681492in}}%
\pgfpathlineto{\pgfqpoint{3.383235in}{1.686933in}}%
\pgfpathlineto{\pgfqpoint{3.385150in}{1.694652in}}%
\pgfpathlineto{\pgfqpoint{3.388980in}{1.697383in}}%
\pgfpathlineto{\pgfqpoint{3.392811in}{1.698227in}}%
\pgfpathlineto{\pgfqpoint{3.396641in}{1.702229in}}%
\pgfpathlineto{\pgfqpoint{3.402387in}{1.703492in}}%
\pgfpathlineto{\pgfqpoint{3.406217in}{1.709548in}}%
\pgfpathlineto{\pgfqpoint{3.413878in}{1.712890in}}%
\pgfpathlineto{\pgfqpoint{3.415793in}{1.723995in}}%
\pgfpathlineto{\pgfqpoint{3.421538in}{1.728260in}}%
\pgfpathlineto{\pgfqpoint{3.425369in}{1.729053in}}%
\pgfpathlineto{\pgfqpoint{3.427284in}{1.730050in}}%
\pgfpathlineto{\pgfqpoint{3.431114in}{1.733090in}}%
\pgfpathlineto{\pgfqpoint{3.434945in}{1.737191in}}%
\pgfpathlineto{\pgfqpoint{3.436860in}{1.741376in}}%
\pgfpathlineto{\pgfqpoint{3.440690in}{1.742164in}}%
\pgfpathlineto{\pgfqpoint{3.450266in}{1.755977in}}%
\pgfpathlineto{\pgfqpoint{3.456011in}{1.758688in}}%
\pgfpathlineto{\pgfqpoint{3.457927in}{1.764045in}}%
\pgfpathlineto{\pgfqpoint{3.459842in}{1.777112in}}%
\pgfpathlineto{\pgfqpoint{3.461757in}{1.782013in}}%
\pgfpathlineto{\pgfqpoint{3.463672in}{1.783256in}}%
\pgfpathlineto{\pgfqpoint{3.465587in}{1.789916in}}%
\pgfpathlineto{\pgfqpoint{3.467502in}{1.790194in}}%
\pgfpathlineto{\pgfqpoint{3.473248in}{1.808689in}}%
\pgfpathlineto{\pgfqpoint{3.475163in}{1.824752in}}%
\pgfpathlineto{\pgfqpoint{3.477078in}{1.829687in}}%
\pgfpathlineto{\pgfqpoint{3.478994in}{1.838083in}}%
\pgfpathlineto{\pgfqpoint{3.480909in}{1.838668in}}%
\pgfpathlineto{\pgfqpoint{3.484739in}{1.859241in}}%
\pgfpathlineto{\pgfqpoint{3.486654in}{1.864609in}}%
\pgfpathlineto{\pgfqpoint{3.492400in}{1.869358in}}%
\pgfpathlineto{\pgfqpoint{3.494315in}{1.869974in}}%
\pgfpathlineto{\pgfqpoint{3.503891in}{1.883992in}}%
\pgfpathlineto{\pgfqpoint{3.505806in}{1.898173in}}%
\pgfpathlineto{\pgfqpoint{3.509636in}{1.902196in}}%
\pgfpathlineto{\pgfqpoint{3.513467in}{1.904123in}}%
\pgfpathlineto{\pgfqpoint{3.515382in}{1.908354in}}%
\pgfpathlineto{\pgfqpoint{3.517297in}{1.908755in}}%
\pgfpathlineto{\pgfqpoint{3.521127in}{1.915786in}}%
\pgfpathlineto{\pgfqpoint{3.526873in}{1.920725in}}%
\pgfpathlineto{\pgfqpoint{3.528788in}{1.936106in}}%
\pgfpathlineto{\pgfqpoint{3.532618in}{1.936942in}}%
\pgfpathlineto{\pgfqpoint{3.534533in}{1.944869in}}%
\pgfpathlineto{\pgfqpoint{3.540279in}{1.945796in}}%
\pgfpathlineto{\pgfqpoint{3.542194in}{1.962761in}}%
\pgfpathlineto{\pgfqpoint{3.544109in}{1.963826in}}%
\pgfpathlineto{\pgfqpoint{3.546025in}{1.967190in}}%
\pgfpathlineto{\pgfqpoint{3.547940in}{1.981006in}}%
\pgfpathlineto{\pgfqpoint{3.549855in}{1.981750in}}%
\pgfpathlineto{\pgfqpoint{3.551770in}{1.984826in}}%
\pgfpathlineto{\pgfqpoint{3.555600in}{1.986353in}}%
\pgfpathlineto{\pgfqpoint{3.557516in}{1.992298in}}%
\pgfpathlineto{\pgfqpoint{3.561346in}{1.995795in}}%
\pgfpathlineto{\pgfqpoint{3.563261in}{2.001155in}}%
\pgfpathlineto{\pgfqpoint{3.565176in}{2.001415in}}%
\pgfpathlineto{\pgfqpoint{3.567091in}{2.005445in}}%
\pgfpathlineto{\pgfqpoint{3.569007in}{2.006069in}}%
\pgfpathlineto{\pgfqpoint{3.574752in}{2.022149in}}%
\pgfpathlineto{\pgfqpoint{3.576667in}{2.041016in}}%
\pgfpathlineto{\pgfqpoint{3.578582in}{2.047938in}}%
\pgfpathlineto{\pgfqpoint{3.580498in}{2.048395in}}%
\pgfpathlineto{\pgfqpoint{3.584328in}{2.070405in}}%
\pgfpathlineto{\pgfqpoint{3.588158in}{2.074349in}}%
\pgfpathlineto{\pgfqpoint{3.590073in}{2.089960in}}%
\pgfpathlineto{\pgfqpoint{3.593904in}{2.091692in}}%
\pgfpathlineto{\pgfqpoint{3.593904in}{2.091692in}}%
\pgfusepath{stroke}%
\end{pgfscope}%
\begin{pgfscope}%
\pgfpathrectangle{\pgfqpoint{0.694334in}{0.523557in}}{\pgfqpoint{3.830343in}{1.568135in}}%
\pgfusepath{clip}%
\pgfsetrectcap%
\pgfsetroundjoin%
\pgfsetlinewidth{1.003750pt}%
\definecolor{currentstroke}{rgb}{0.564706,0.564706,1.000000}%
\pgfsetstrokecolor{currentstroke}%
\pgfsetdash{}{0pt}%
\pgfpathmoveto{\pgfqpoint{0.694334in}{0.721317in}}%
\pgfpathlineto{\pgfqpoint{0.700080in}{0.750076in}}%
\pgfpathlineto{\pgfqpoint{0.705825in}{0.767302in}}%
\pgfpathlineto{\pgfqpoint{0.707741in}{0.768023in}}%
\pgfpathlineto{\pgfqpoint{0.709656in}{0.772950in}}%
\pgfpathlineto{\pgfqpoint{0.715401in}{0.776580in}}%
\pgfpathlineto{\pgfqpoint{0.721147in}{0.782421in}}%
\pgfpathlineto{\pgfqpoint{0.723062in}{0.784946in}}%
\pgfpathlineto{\pgfqpoint{0.724977in}{0.795027in}}%
\pgfpathlineto{\pgfqpoint{0.726892in}{0.795957in}}%
\pgfpathlineto{\pgfqpoint{0.728807in}{0.799401in}}%
\pgfpathlineto{\pgfqpoint{0.734553in}{0.800300in}}%
\pgfpathlineto{\pgfqpoint{0.740298in}{0.805813in}}%
\pgfpathlineto{\pgfqpoint{0.746044in}{0.808521in}}%
\pgfpathlineto{\pgfqpoint{0.747959in}{0.810439in}}%
\pgfpathlineto{\pgfqpoint{0.763280in}{0.812238in}}%
\pgfpathlineto{\pgfqpoint{0.767111in}{0.815425in}}%
\pgfpathlineto{\pgfqpoint{0.772856in}{0.817009in}}%
\pgfpathlineto{\pgfqpoint{0.780517in}{0.818670in}}%
\pgfpathlineto{\pgfqpoint{0.784347in}{0.820363in}}%
\pgfpathlineto{\pgfqpoint{0.792008in}{0.822144in}}%
\pgfpathlineto{\pgfqpoint{0.824566in}{0.826638in}}%
\pgfpathlineto{\pgfqpoint{0.834142in}{0.827174in}}%
\pgfpathlineto{\pgfqpoint{0.839887in}{0.829435in}}%
\pgfpathlineto{\pgfqpoint{0.845633in}{0.829685in}}%
\pgfpathlineto{\pgfqpoint{0.849463in}{0.830941in}}%
\pgfpathlineto{\pgfqpoint{0.866700in}{0.834562in}}%
\pgfpathlineto{\pgfqpoint{0.891597in}{0.838458in}}%
\pgfpathlineto{\pgfqpoint{0.897342in}{0.839684in}}%
\pgfpathlineto{\pgfqpoint{0.906918in}{0.840657in}}%
\pgfpathlineto{\pgfqpoint{0.922240in}{0.841853in}}%
\pgfpathlineto{\pgfqpoint{1.002677in}{0.850559in}}%
\pgfpathlineto{\pgfqpoint{1.014168in}{0.854120in}}%
\pgfpathlineto{\pgfqpoint{1.021829in}{0.855248in}}%
\pgfpathlineto{\pgfqpoint{1.029489in}{0.856226in}}%
\pgfpathlineto{\pgfqpoint{1.040980in}{0.857373in}}%
\pgfpathlineto{\pgfqpoint{1.079284in}{0.862899in}}%
\pgfpathlineto{\pgfqpoint{1.092690in}{0.863886in}}%
\pgfpathlineto{\pgfqpoint{1.100351in}{0.865539in}}%
\pgfpathlineto{\pgfqpoint{1.117587in}{0.867224in}}%
\pgfpathlineto{\pgfqpoint{1.171212in}{0.873540in}}%
\pgfpathlineto{\pgfqpoint{1.188449in}{0.874645in}}%
\pgfpathlineto{\pgfqpoint{1.207600in}{0.877720in}}%
\pgfpathlineto{\pgfqpoint{1.228667in}{0.880052in}}%
\pgfpathlineto{\pgfqpoint{1.238243in}{0.882100in}}%
\pgfpathlineto{\pgfqpoint{1.253564in}{0.883435in}}%
\pgfpathlineto{\pgfqpoint{1.263140in}{0.884206in}}%
\pgfpathlineto{\pgfqpoint{1.284207in}{0.886247in}}%
\pgfpathlineto{\pgfqpoint{1.291868in}{0.887760in}}%
\pgfpathlineto{\pgfqpoint{1.305274in}{0.890379in}}%
\pgfpathlineto{\pgfqpoint{1.311019in}{0.891325in}}%
\pgfpathlineto{\pgfqpoint{1.318680in}{0.892929in}}%
\pgfpathlineto{\pgfqpoint{1.332086in}{0.896116in}}%
\pgfpathlineto{\pgfqpoint{1.335917in}{0.897161in}}%
\pgfpathlineto{\pgfqpoint{1.362729in}{0.900896in}}%
\pgfpathlineto{\pgfqpoint{1.366559in}{0.901911in}}%
\pgfpathlineto{\pgfqpoint{1.376135in}{0.903759in}}%
\pgfpathlineto{\pgfqpoint{1.379966in}{0.905026in}}%
\pgfpathlineto{\pgfqpoint{1.387626in}{0.906122in}}%
\pgfpathlineto{\pgfqpoint{1.395287in}{0.907919in}}%
\pgfpathlineto{\pgfqpoint{1.402948in}{0.908735in}}%
\pgfpathlineto{\pgfqpoint{1.406778in}{0.910974in}}%
\pgfpathlineto{\pgfqpoint{1.416354in}{0.911480in}}%
\pgfpathlineto{\pgfqpoint{1.422099in}{0.913252in}}%
\pgfpathlineto{\pgfqpoint{1.425930in}{0.914374in}}%
\pgfpathlineto{\pgfqpoint{1.433590in}{0.916016in}}%
\pgfpathlineto{\pgfqpoint{1.452742in}{0.919111in}}%
\pgfpathlineto{\pgfqpoint{1.469979in}{0.925469in}}%
\pgfpathlineto{\pgfqpoint{1.475724in}{0.926382in}}%
\pgfpathlineto{\pgfqpoint{1.483385in}{0.929077in}}%
\pgfpathlineto{\pgfqpoint{1.531264in}{0.941751in}}%
\pgfpathlineto{\pgfqpoint{1.535095in}{0.943639in}}%
\pgfpathlineto{\pgfqpoint{1.540840in}{0.945822in}}%
\pgfpathlineto{\pgfqpoint{1.544670in}{0.948650in}}%
\pgfpathlineto{\pgfqpoint{1.554246in}{0.951011in}}%
\pgfpathlineto{\pgfqpoint{1.556161in}{0.952666in}}%
\pgfpathlineto{\pgfqpoint{1.559992in}{0.953598in}}%
\pgfpathlineto{\pgfqpoint{1.588719in}{0.964222in}}%
\pgfpathlineto{\pgfqpoint{1.590635in}{0.967919in}}%
\pgfpathlineto{\pgfqpoint{1.596380in}{0.968741in}}%
\pgfpathlineto{\pgfqpoint{1.598295in}{0.970662in}}%
\pgfpathlineto{\pgfqpoint{1.609786in}{0.973134in}}%
\pgfpathlineto{\pgfqpoint{1.611701in}{0.975646in}}%
\pgfpathlineto{\pgfqpoint{1.613617in}{0.980367in}}%
\pgfpathlineto{\pgfqpoint{1.621277in}{0.984076in}}%
\pgfpathlineto{\pgfqpoint{1.625108in}{0.984937in}}%
\pgfpathlineto{\pgfqpoint{1.630853in}{0.988752in}}%
\pgfpathlineto{\pgfqpoint{1.636599in}{0.990657in}}%
\pgfpathlineto{\pgfqpoint{1.640429in}{0.990904in}}%
\pgfpathlineto{\pgfqpoint{1.644259in}{0.992304in}}%
\pgfpathlineto{\pgfqpoint{1.653835in}{0.994193in}}%
\pgfpathlineto{\pgfqpoint{1.674902in}{0.998491in}}%
\pgfpathlineto{\pgfqpoint{1.678732in}{1.001616in}}%
\pgfpathlineto{\pgfqpoint{1.688308in}{1.003006in}}%
\pgfpathlineto{\pgfqpoint{1.695969in}{1.005377in}}%
\pgfpathlineto{\pgfqpoint{1.705545in}{1.007364in}}%
\pgfpathlineto{\pgfqpoint{1.718951in}{1.010783in}}%
\pgfpathlineto{\pgfqpoint{1.732357in}{1.013118in}}%
\pgfpathlineto{\pgfqpoint{1.747679in}{1.020505in}}%
\pgfpathlineto{\pgfqpoint{1.753424in}{1.021504in}}%
\pgfpathlineto{\pgfqpoint{1.761085in}{1.024126in}}%
\pgfpathlineto{\pgfqpoint{1.770661in}{1.027981in}}%
\pgfpathlineto{\pgfqpoint{1.780236in}{1.031900in}}%
\pgfpathlineto{\pgfqpoint{1.791728in}{1.034950in}}%
\pgfpathlineto{\pgfqpoint{1.795558in}{1.036581in}}%
\pgfpathlineto{\pgfqpoint{1.799388in}{1.037424in}}%
\pgfpathlineto{\pgfqpoint{1.805134in}{1.040624in}}%
\pgfpathlineto{\pgfqpoint{1.808964in}{1.042199in}}%
\pgfpathlineto{\pgfqpoint{1.814710in}{1.043921in}}%
\pgfpathlineto{\pgfqpoint{1.822370in}{1.045667in}}%
\pgfpathlineto{\pgfqpoint{1.830031in}{1.047733in}}%
\pgfpathlineto{\pgfqpoint{1.837692in}{1.050541in}}%
\pgfpathlineto{\pgfqpoint{1.898977in}{1.060552in}}%
\pgfpathlineto{\pgfqpoint{1.900892in}{1.061019in}}%
\pgfpathlineto{\pgfqpoint{1.904723in}{1.063747in}}%
\pgfpathlineto{\pgfqpoint{1.910468in}{1.064616in}}%
\pgfpathlineto{\pgfqpoint{1.927705in}{1.070048in}}%
\pgfpathlineto{\pgfqpoint{1.929620in}{1.071764in}}%
\pgfpathlineto{\pgfqpoint{1.977499in}{1.077959in}}%
\pgfpathlineto{\pgfqpoint{1.983245in}{1.080026in}}%
\pgfpathlineto{\pgfqpoint{1.992821in}{1.081951in}}%
\pgfpathlineto{\pgfqpoint{1.996651in}{1.083387in}}%
\pgfpathlineto{\pgfqpoint{2.010057in}{1.084893in}}%
\pgfpathlineto{\pgfqpoint{2.031124in}{1.089241in}}%
\pgfpathlineto{\pgfqpoint{2.033039in}{1.089276in}}%
\pgfpathlineto{\pgfqpoint{2.036869in}{1.090940in}}%
\pgfpathlineto{\pgfqpoint{2.050276in}{1.093203in}}%
\pgfpathlineto{\pgfqpoint{2.052191in}{1.094920in}}%
\pgfpathlineto{\pgfqpoint{2.063682in}{1.096559in}}%
\pgfpathlineto{\pgfqpoint{2.067512in}{1.098233in}}%
\pgfpathlineto{\pgfqpoint{2.075173in}{1.100023in}}%
\pgfpathlineto{\pgfqpoint{2.092409in}{1.104414in}}%
\pgfpathlineto{\pgfqpoint{2.096240in}{1.107730in}}%
\pgfpathlineto{\pgfqpoint{2.105816in}{1.111334in}}%
\pgfpathlineto{\pgfqpoint{2.109646in}{1.113877in}}%
\pgfpathlineto{\pgfqpoint{2.115391in}{1.114236in}}%
\pgfpathlineto{\pgfqpoint{2.117307in}{1.116932in}}%
\pgfpathlineto{\pgfqpoint{2.126883in}{1.119658in}}%
\pgfpathlineto{\pgfqpoint{2.176677in}{1.127265in}}%
\pgfpathlineto{\pgfqpoint{2.178592in}{1.129961in}}%
\pgfpathlineto{\pgfqpoint{2.190083in}{1.131419in}}%
\pgfpathlineto{\pgfqpoint{2.191998in}{1.132900in}}%
\pgfpathlineto{\pgfqpoint{2.199659in}{1.133968in}}%
\pgfpathlineto{\pgfqpoint{2.203489in}{1.136170in}}%
\pgfpathlineto{\pgfqpoint{2.226471in}{1.139181in}}%
\pgfpathlineto{\pgfqpoint{2.232217in}{1.140544in}}%
\pgfpathlineto{\pgfqpoint{2.237962in}{1.141150in}}%
\pgfpathlineto{\pgfqpoint{2.245623in}{1.143517in}}%
\pgfpathlineto{\pgfqpoint{2.257114in}{1.145924in}}%
\pgfpathlineto{\pgfqpoint{2.270520in}{1.148555in}}%
\pgfpathlineto{\pgfqpoint{2.272436in}{1.150303in}}%
\pgfpathlineto{\pgfqpoint{2.280096in}{1.151377in}}%
\pgfpathlineto{\pgfqpoint{2.289672in}{1.154100in}}%
\pgfpathlineto{\pgfqpoint{2.297333in}{1.155079in}}%
\pgfpathlineto{\pgfqpoint{2.303078in}{1.156830in}}%
\pgfpathlineto{\pgfqpoint{2.326060in}{1.161136in}}%
\pgfpathlineto{\pgfqpoint{2.329891in}{1.162227in}}%
\pgfpathlineto{\pgfqpoint{2.331806in}{1.162683in}}%
\pgfpathlineto{\pgfqpoint{2.339467in}{1.167982in}}%
\pgfpathlineto{\pgfqpoint{2.345212in}{1.169131in}}%
\pgfpathlineto{\pgfqpoint{2.396922in}{1.179331in}}%
\pgfpathlineto{\pgfqpoint{2.398837in}{1.181305in}}%
\pgfpathlineto{\pgfqpoint{2.406498in}{1.182509in}}%
\pgfpathlineto{\pgfqpoint{2.410328in}{1.184476in}}%
\pgfpathlineto{\pgfqpoint{2.417989in}{1.185766in}}%
\pgfpathlineto{\pgfqpoint{2.448631in}{1.192202in}}%
\pgfpathlineto{\pgfqpoint{2.454377in}{1.195458in}}%
\pgfpathlineto{\pgfqpoint{2.460122in}{1.196272in}}%
\pgfpathlineto{\pgfqpoint{2.462037in}{1.199534in}}%
\pgfpathlineto{\pgfqpoint{2.467783in}{1.201453in}}%
\pgfpathlineto{\pgfqpoint{2.471613in}{1.202418in}}%
\pgfpathlineto{\pgfqpoint{2.475444in}{1.204981in}}%
\pgfpathlineto{\pgfqpoint{2.481189in}{1.206789in}}%
\pgfpathlineto{\pgfqpoint{2.485020in}{1.207548in}}%
\pgfpathlineto{\pgfqpoint{2.498426in}{1.209554in}}%
\pgfpathlineto{\pgfqpoint{2.509917in}{1.210992in}}%
\pgfpathlineto{\pgfqpoint{2.553966in}{1.212556in}}%
\pgfpathlineto{\pgfqpoint{2.571202in}{1.213375in}}%
\pgfpathlineto{\pgfqpoint{2.599930in}{1.215513in}}%
\pgfpathlineto{\pgfqpoint{2.603760in}{1.217134in}}%
\pgfpathlineto{\pgfqpoint{2.607591in}{1.218898in}}%
\pgfpathlineto{\pgfqpoint{2.615251in}{1.219812in}}%
\pgfpathlineto{\pgfqpoint{2.620997in}{1.221304in}}%
\pgfpathlineto{\pgfqpoint{2.636318in}{1.225068in}}%
\pgfpathlineto{\pgfqpoint{2.645894in}{1.226062in}}%
\pgfpathlineto{\pgfqpoint{2.657385in}{1.227727in}}%
\pgfpathlineto{\pgfqpoint{2.661215in}{1.229054in}}%
\pgfpathlineto{\pgfqpoint{2.672706in}{1.230405in}}%
\pgfpathlineto{\pgfqpoint{2.682282in}{1.233688in}}%
\pgfpathlineto{\pgfqpoint{2.699519in}{1.235336in}}%
\pgfpathlineto{\pgfqpoint{2.707179in}{1.237575in}}%
\pgfpathlineto{\pgfqpoint{2.711010in}{1.238015in}}%
\pgfpathlineto{\pgfqpoint{2.714840in}{1.239894in}}%
\pgfpathlineto{\pgfqpoint{2.730161in}{1.243444in}}%
\pgfpathlineto{\pgfqpoint{2.735907in}{1.244899in}}%
\pgfpathlineto{\pgfqpoint{2.737822in}{1.248203in}}%
\pgfpathlineto{\pgfqpoint{2.747398in}{1.250473in}}%
\pgfpathlineto{\pgfqpoint{2.751228in}{1.251867in}}%
\pgfpathlineto{\pgfqpoint{2.756974in}{1.253415in}}%
\pgfpathlineto{\pgfqpoint{2.779956in}{1.260953in}}%
\pgfpathlineto{\pgfqpoint{2.781871in}{1.263988in}}%
\pgfpathlineto{\pgfqpoint{2.793362in}{1.267489in}}%
\pgfpathlineto{\pgfqpoint{2.797192in}{1.268558in}}%
\pgfpathlineto{\pgfqpoint{2.801023in}{1.270020in}}%
\pgfpathlineto{\pgfqpoint{2.804853in}{1.270204in}}%
\pgfpathlineto{\pgfqpoint{2.806768in}{1.273184in}}%
\pgfpathlineto{\pgfqpoint{2.810599in}{1.273757in}}%
\pgfpathlineto{\pgfqpoint{2.822090in}{1.279481in}}%
\pgfpathlineto{\pgfqpoint{2.825920in}{1.280329in}}%
\pgfpathlineto{\pgfqpoint{2.827835in}{1.282399in}}%
\pgfpathlineto{\pgfqpoint{2.831666in}{1.282714in}}%
\pgfpathlineto{\pgfqpoint{2.835496in}{1.286793in}}%
\pgfpathlineto{\pgfqpoint{2.839326in}{1.287862in}}%
\pgfpathlineto{\pgfqpoint{2.846987in}{1.288852in}}%
\pgfpathlineto{\pgfqpoint{2.856563in}{1.291564in}}%
\pgfpathlineto{\pgfqpoint{2.858478in}{1.293932in}}%
\pgfpathlineto{\pgfqpoint{2.864223in}{1.295522in}}%
\pgfpathlineto{\pgfqpoint{2.868054in}{1.299344in}}%
\pgfpathlineto{\pgfqpoint{2.877630in}{1.302865in}}%
\pgfpathlineto{\pgfqpoint{2.879545in}{1.305304in}}%
\pgfpathlineto{\pgfqpoint{2.881460in}{1.310516in}}%
\pgfpathlineto{\pgfqpoint{2.889121in}{1.314870in}}%
\pgfpathlineto{\pgfqpoint{2.891036in}{1.318266in}}%
\pgfpathlineto{\pgfqpoint{2.894866in}{1.320160in}}%
\pgfpathlineto{\pgfqpoint{2.900612in}{1.322205in}}%
\pgfpathlineto{\pgfqpoint{2.904442in}{1.326809in}}%
\pgfpathlineto{\pgfqpoint{2.908272in}{1.328910in}}%
\pgfpathlineto{\pgfqpoint{2.910188in}{1.329530in}}%
\pgfpathlineto{\pgfqpoint{2.912103in}{1.331783in}}%
\pgfpathlineto{\pgfqpoint{2.915933in}{1.332074in}}%
\pgfpathlineto{\pgfqpoint{2.921679in}{1.335666in}}%
\pgfpathlineto{\pgfqpoint{2.944661in}{1.343616in}}%
\pgfpathlineto{\pgfqpoint{2.948491in}{1.349653in}}%
\pgfpathlineto{\pgfqpoint{2.952321in}{1.350086in}}%
\pgfpathlineto{\pgfqpoint{2.956152in}{1.351626in}}%
\pgfpathlineto{\pgfqpoint{2.965728in}{1.353255in}}%
\pgfpathlineto{\pgfqpoint{2.967643in}{1.356872in}}%
\pgfpathlineto{\pgfqpoint{2.971473in}{1.359157in}}%
\pgfpathlineto{\pgfqpoint{2.975303in}{1.361373in}}%
\pgfpathlineto{\pgfqpoint{2.981049in}{1.362165in}}%
\pgfpathlineto{\pgfqpoint{2.986794in}{1.366891in}}%
\pgfpathlineto{\pgfqpoint{2.988710in}{1.367409in}}%
\pgfpathlineto{\pgfqpoint{2.994455in}{1.371379in}}%
\pgfpathlineto{\pgfqpoint{3.002116in}{1.374585in}}%
\pgfpathlineto{\pgfqpoint{3.004031in}{1.377893in}}%
\pgfpathlineto{\pgfqpoint{3.005946in}{1.378064in}}%
\pgfpathlineto{\pgfqpoint{3.019352in}{1.389413in}}%
\pgfpathlineto{\pgfqpoint{3.021268in}{1.389573in}}%
\pgfpathlineto{\pgfqpoint{3.025098in}{1.391552in}}%
\pgfpathlineto{\pgfqpoint{3.027013in}{1.391649in}}%
\pgfpathlineto{\pgfqpoint{3.030843in}{1.393785in}}%
\pgfpathlineto{\pgfqpoint{3.040419in}{1.396416in}}%
\pgfpathlineto{\pgfqpoint{3.048080in}{1.399273in}}%
\pgfpathlineto{\pgfqpoint{3.051910in}{1.399635in}}%
\pgfpathlineto{\pgfqpoint{3.053825in}{1.400836in}}%
\pgfpathlineto{\pgfqpoint{3.055741in}{1.404069in}}%
\pgfpathlineto{\pgfqpoint{3.059571in}{1.405694in}}%
\pgfpathlineto{\pgfqpoint{3.063401in}{1.407689in}}%
\pgfpathlineto{\pgfqpoint{3.065316in}{1.408170in}}%
\pgfpathlineto{\pgfqpoint{3.067232in}{1.410606in}}%
\pgfpathlineto{\pgfqpoint{3.072977in}{1.412066in}}%
\pgfpathlineto{\pgfqpoint{3.074892in}{1.415724in}}%
\pgfpathlineto{\pgfqpoint{3.082553in}{1.418827in}}%
\pgfpathlineto{\pgfqpoint{3.086383in}{1.428353in}}%
\pgfpathlineto{\pgfqpoint{3.092129in}{1.435204in}}%
\pgfpathlineto{\pgfqpoint{3.105535in}{1.438090in}}%
\pgfpathlineto{\pgfqpoint{3.109365in}{1.441391in}}%
\pgfpathlineto{\pgfqpoint{3.111281in}{1.442195in}}%
\pgfpathlineto{\pgfqpoint{3.115111in}{1.446911in}}%
\pgfpathlineto{\pgfqpoint{3.122772in}{1.448613in}}%
\pgfpathlineto{\pgfqpoint{3.124687in}{1.451762in}}%
\pgfpathlineto{\pgfqpoint{3.140008in}{1.456075in}}%
\pgfpathlineto{\pgfqpoint{3.141923in}{1.460371in}}%
\pgfpathlineto{\pgfqpoint{3.143839in}{1.460844in}}%
\pgfpathlineto{\pgfqpoint{3.145754in}{1.464859in}}%
\pgfpathlineto{\pgfqpoint{3.159160in}{1.471041in}}%
\pgfpathlineto{\pgfqpoint{3.162990in}{1.473291in}}%
\pgfpathlineto{\pgfqpoint{3.176396in}{1.476471in}}%
\pgfpathlineto{\pgfqpoint{3.182142in}{1.480896in}}%
\pgfpathlineto{\pgfqpoint{3.185972in}{1.481319in}}%
\pgfpathlineto{\pgfqpoint{3.191718in}{1.482379in}}%
\pgfpathlineto{\pgfqpoint{3.193633in}{1.482393in}}%
\pgfpathlineto{\pgfqpoint{3.195548in}{1.483879in}}%
\pgfpathlineto{\pgfqpoint{3.199378in}{1.488933in}}%
\pgfpathlineto{\pgfqpoint{3.203209in}{1.489734in}}%
\pgfpathlineto{\pgfqpoint{3.207039in}{1.490750in}}%
\pgfpathlineto{\pgfqpoint{3.210870in}{1.492118in}}%
\pgfpathlineto{\pgfqpoint{3.216615in}{1.494395in}}%
\pgfpathlineto{\pgfqpoint{3.222361in}{1.500219in}}%
\pgfpathlineto{\pgfqpoint{3.226191in}{1.500528in}}%
\pgfpathlineto{\pgfqpoint{3.230021in}{1.506378in}}%
\pgfpathlineto{\pgfqpoint{3.235767in}{1.508298in}}%
\pgfpathlineto{\pgfqpoint{3.243427in}{1.512948in}}%
\pgfpathlineto{\pgfqpoint{3.245343in}{1.513151in}}%
\pgfpathlineto{\pgfqpoint{3.260664in}{1.524645in}}%
\pgfpathlineto{\pgfqpoint{3.266409in}{1.525792in}}%
\pgfpathlineto{\pgfqpoint{3.270240in}{1.528624in}}%
\pgfpathlineto{\pgfqpoint{3.281731in}{1.531345in}}%
\pgfpathlineto{\pgfqpoint{3.283646in}{1.534459in}}%
\pgfpathlineto{\pgfqpoint{3.287476in}{1.535172in}}%
\pgfpathlineto{\pgfqpoint{3.291307in}{1.538168in}}%
\pgfpathlineto{\pgfqpoint{3.293222in}{1.538173in}}%
\pgfpathlineto{\pgfqpoint{3.306628in}{1.549925in}}%
\pgfpathlineto{\pgfqpoint{3.337271in}{1.560562in}}%
\pgfpathlineto{\pgfqpoint{3.339186in}{1.564051in}}%
\pgfpathlineto{\pgfqpoint{3.341101in}{1.565111in}}%
\pgfpathlineto{\pgfqpoint{3.343016in}{1.571007in}}%
\pgfpathlineto{\pgfqpoint{3.344932in}{1.572223in}}%
\pgfpathlineto{\pgfqpoint{3.348762in}{1.577499in}}%
\pgfpathlineto{\pgfqpoint{3.350677in}{1.577944in}}%
\pgfpathlineto{\pgfqpoint{3.352592in}{1.581785in}}%
\pgfpathlineto{\pgfqpoint{3.358338in}{1.584715in}}%
\pgfpathlineto{\pgfqpoint{3.360253in}{1.585073in}}%
\pgfpathlineto{\pgfqpoint{3.364083in}{1.590629in}}%
\pgfpathlineto{\pgfqpoint{3.367914in}{1.594604in}}%
\pgfpathlineto{\pgfqpoint{3.373659in}{1.600216in}}%
\pgfpathlineto{\pgfqpoint{3.377489in}{1.605781in}}%
\pgfpathlineto{\pgfqpoint{3.381320in}{1.613093in}}%
\pgfpathlineto{\pgfqpoint{3.385150in}{1.617699in}}%
\pgfpathlineto{\pgfqpoint{3.387065in}{1.623354in}}%
\pgfpathlineto{\pgfqpoint{3.388980in}{1.623557in}}%
\pgfpathlineto{\pgfqpoint{3.398556in}{1.634488in}}%
\pgfpathlineto{\pgfqpoint{3.400471in}{1.635767in}}%
\pgfpathlineto{\pgfqpoint{3.408132in}{1.647324in}}%
\pgfpathlineto{\pgfqpoint{3.411963in}{1.668595in}}%
\pgfpathlineto{\pgfqpoint{3.413878in}{1.680568in}}%
\pgfpathlineto{\pgfqpoint{3.417708in}{1.681722in}}%
\pgfpathlineto{\pgfqpoint{3.421538in}{1.691382in}}%
\pgfpathlineto{\pgfqpoint{3.423454in}{1.693886in}}%
\pgfpathlineto{\pgfqpoint{3.431114in}{1.696625in}}%
\pgfpathlineto{\pgfqpoint{3.438775in}{1.716290in}}%
\pgfpathlineto{\pgfqpoint{3.442605in}{1.719191in}}%
\pgfpathlineto{\pgfqpoint{3.444520in}{1.721265in}}%
\pgfpathlineto{\pgfqpoint{3.446436in}{1.730656in}}%
\pgfpathlineto{\pgfqpoint{3.448351in}{1.730759in}}%
\pgfpathlineto{\pgfqpoint{3.454096in}{1.735361in}}%
\pgfpathlineto{\pgfqpoint{3.459842in}{1.755445in}}%
\pgfpathlineto{\pgfqpoint{3.463672in}{1.763988in}}%
\pgfpathlineto{\pgfqpoint{3.465587in}{1.771713in}}%
\pgfpathlineto{\pgfqpoint{3.467502in}{1.772243in}}%
\pgfpathlineto{\pgfqpoint{3.469418in}{1.779286in}}%
\pgfpathlineto{\pgfqpoint{3.471333in}{1.779792in}}%
\pgfpathlineto{\pgfqpoint{3.473248in}{1.786278in}}%
\pgfpathlineto{\pgfqpoint{3.482824in}{1.794424in}}%
\pgfpathlineto{\pgfqpoint{3.484739in}{1.795999in}}%
\pgfpathlineto{\pgfqpoint{3.492400in}{1.809232in}}%
\pgfpathlineto{\pgfqpoint{3.494315in}{1.809702in}}%
\pgfpathlineto{\pgfqpoint{3.496230in}{1.811710in}}%
\pgfpathlineto{\pgfqpoint{3.498145in}{1.826522in}}%
\pgfpathlineto{\pgfqpoint{3.500060in}{1.827015in}}%
\pgfpathlineto{\pgfqpoint{3.503891in}{1.834339in}}%
\pgfpathlineto{\pgfqpoint{3.507721in}{1.835046in}}%
\pgfpathlineto{\pgfqpoint{3.509636in}{1.848597in}}%
\pgfpathlineto{\pgfqpoint{3.511551in}{1.848776in}}%
\pgfpathlineto{\pgfqpoint{3.513467in}{1.850210in}}%
\pgfpathlineto{\pgfqpoint{3.517297in}{1.855896in}}%
\pgfpathlineto{\pgfqpoint{3.521127in}{1.857184in}}%
\pgfpathlineto{\pgfqpoint{3.524958in}{1.865672in}}%
\pgfpathlineto{\pgfqpoint{3.532618in}{1.873816in}}%
\pgfpathlineto{\pgfqpoint{3.536449in}{1.884499in}}%
\pgfpathlineto{\pgfqpoint{3.538364in}{1.888857in}}%
\pgfpathlineto{\pgfqpoint{3.542194in}{1.890167in}}%
\pgfpathlineto{\pgfqpoint{3.544109in}{1.899123in}}%
\pgfpathlineto{\pgfqpoint{3.547940in}{1.904486in}}%
\pgfpathlineto{\pgfqpoint{3.551770in}{1.908869in}}%
\pgfpathlineto{\pgfqpoint{3.555600in}{1.917400in}}%
\pgfpathlineto{\pgfqpoint{3.557516in}{1.918402in}}%
\pgfpathlineto{\pgfqpoint{3.565176in}{1.944581in}}%
\pgfpathlineto{\pgfqpoint{3.567091in}{1.958007in}}%
\pgfpathlineto{\pgfqpoint{3.570922in}{1.962928in}}%
\pgfpathlineto{\pgfqpoint{3.572837in}{1.987900in}}%
\pgfpathlineto{\pgfqpoint{3.584328in}{2.004479in}}%
\pgfpathlineto{\pgfqpoint{3.588158in}{2.014700in}}%
\pgfpathlineto{\pgfqpoint{3.590073in}{2.016672in}}%
\pgfpathlineto{\pgfqpoint{3.591989in}{2.016977in}}%
\pgfpathlineto{\pgfqpoint{3.595819in}{2.028739in}}%
\pgfpathlineto{\pgfqpoint{3.599649in}{2.050770in}}%
\pgfpathlineto{\pgfqpoint{3.601564in}{2.050841in}}%
\pgfpathlineto{\pgfqpoint{3.603480in}{2.065175in}}%
\pgfpathlineto{\pgfqpoint{3.613056in}{2.084616in}}%
\pgfpathlineto{\pgfqpoint{3.614971in}{2.084982in}}%
\pgfpathlineto{\pgfqpoint{3.616886in}{2.091692in}}%
\pgfpathlineto{\pgfqpoint{3.616886in}{2.091692in}}%
\pgfusepath{stroke}%
\end{pgfscope}%
\begin{pgfscope}%
\pgfpathrectangle{\pgfqpoint{0.694334in}{0.523557in}}{\pgfqpoint{3.830343in}{1.568135in}}%
\pgfusepath{clip}%
\pgfsetbuttcap%
\pgfsetroundjoin%
\pgfsetlinewidth{1.003750pt}%
\definecolor{currentstroke}{rgb}{0.564706,0.564706,1.000000}%
\pgfsetstrokecolor{currentstroke}%
\pgfsetdash{{1.000000pt}{1.650000pt}}{0.000000pt}%
\pgfpathmoveto{\pgfqpoint{0.694334in}{0.647075in}}%
\pgfpathlineto{\pgfqpoint{0.696249in}{0.671971in}}%
\pgfpathlineto{\pgfqpoint{0.698165in}{0.679286in}}%
\pgfpathlineto{\pgfqpoint{0.700080in}{0.679895in}}%
\pgfpathlineto{\pgfqpoint{0.703910in}{0.694609in}}%
\pgfpathlineto{\pgfqpoint{0.705825in}{0.696426in}}%
\pgfpathlineto{\pgfqpoint{0.707741in}{0.700208in}}%
\pgfpathlineto{\pgfqpoint{0.709656in}{0.721744in}}%
\pgfpathlineto{\pgfqpoint{0.713486in}{0.728083in}}%
\pgfpathlineto{\pgfqpoint{0.715401in}{0.728332in}}%
\pgfpathlineto{\pgfqpoint{0.717316in}{0.731163in}}%
\pgfpathlineto{\pgfqpoint{0.719232in}{0.731333in}}%
\pgfpathlineto{\pgfqpoint{0.724977in}{0.737485in}}%
\pgfpathlineto{\pgfqpoint{0.730723in}{0.737609in}}%
\pgfpathlineto{\pgfqpoint{0.732638in}{0.740126in}}%
\pgfpathlineto{\pgfqpoint{0.736468in}{0.740961in}}%
\pgfpathlineto{\pgfqpoint{0.738383in}{0.742119in}}%
\pgfpathlineto{\pgfqpoint{0.744129in}{0.749778in}}%
\pgfpathlineto{\pgfqpoint{0.747959in}{0.752770in}}%
\pgfpathlineto{\pgfqpoint{0.751789in}{0.755217in}}%
\pgfpathlineto{\pgfqpoint{0.753705in}{0.755436in}}%
\pgfpathlineto{\pgfqpoint{0.761365in}{0.761488in}}%
\pgfpathlineto{\pgfqpoint{0.765196in}{0.762279in}}%
\pgfpathlineto{\pgfqpoint{0.776687in}{0.767088in}}%
\pgfpathlineto{\pgfqpoint{0.788178in}{0.768751in}}%
\pgfpathlineto{\pgfqpoint{0.797754in}{0.769925in}}%
\pgfpathlineto{\pgfqpoint{0.816905in}{0.771924in}}%
\pgfpathlineto{\pgfqpoint{0.837972in}{0.775824in}}%
\pgfpathlineto{\pgfqpoint{0.839887in}{0.776466in}}%
\pgfpathlineto{\pgfqpoint{0.841803in}{0.778527in}}%
\pgfpathlineto{\pgfqpoint{0.853294in}{0.779993in}}%
\pgfpathlineto{\pgfqpoint{0.862869in}{0.782700in}}%
\pgfpathlineto{\pgfqpoint{0.878191in}{0.783620in}}%
\pgfpathlineto{\pgfqpoint{0.887767in}{0.786199in}}%
\pgfpathlineto{\pgfqpoint{0.929900in}{0.791891in}}%
\pgfpathlineto{\pgfqpoint{0.949052in}{0.793831in}}%
\pgfpathlineto{\pgfqpoint{0.956713in}{0.795909in}}%
\pgfpathlineto{\pgfqpoint{1.000762in}{0.803773in}}%
\pgfpathlineto{\pgfqpoint{1.008422in}{0.804677in}}%
\pgfpathlineto{\pgfqpoint{1.019913in}{0.805972in}}%
\pgfpathlineto{\pgfqpoint{1.023744in}{0.806628in}}%
\pgfpathlineto{\pgfqpoint{1.027574in}{0.806746in}}%
\pgfpathlineto{\pgfqpoint{1.029489in}{0.808201in}}%
\pgfpathlineto{\pgfqpoint{1.037150in}{0.809136in}}%
\pgfpathlineto{\pgfqpoint{1.069708in}{0.814277in}}%
\pgfpathlineto{\pgfqpoint{1.115672in}{0.822045in}}%
\pgfpathlineto{\pgfqpoint{1.121418in}{0.822942in}}%
\pgfpathlineto{\pgfqpoint{1.159721in}{0.827751in}}%
\pgfpathlineto{\pgfqpoint{1.163551in}{0.829069in}}%
\pgfpathlineto{\pgfqpoint{1.180788in}{0.830946in}}%
\pgfpathlineto{\pgfqpoint{1.253564in}{0.841718in}}%
\pgfpathlineto{\pgfqpoint{1.257395in}{0.842788in}}%
\pgfpathlineto{\pgfqpoint{1.284207in}{0.845418in}}%
\pgfpathlineto{\pgfqpoint{1.297613in}{0.847953in}}%
\pgfpathlineto{\pgfqpoint{1.307189in}{0.850137in}}%
\pgfpathlineto{\pgfqpoint{1.312935in}{0.851266in}}%
\pgfpathlineto{\pgfqpoint{1.334002in}{0.854324in}}%
\pgfpathlineto{\pgfqpoint{1.341662in}{0.855032in}}%
\pgfpathlineto{\pgfqpoint{1.345493in}{0.857906in}}%
\pgfpathlineto{\pgfqpoint{1.351238in}{0.858296in}}%
\pgfpathlineto{\pgfqpoint{1.355068in}{0.859497in}}%
\pgfpathlineto{\pgfqpoint{1.356984in}{0.859556in}}%
\pgfpathlineto{\pgfqpoint{1.360814in}{0.860833in}}%
\pgfpathlineto{\pgfqpoint{1.379966in}{0.863850in}}%
\pgfpathlineto{\pgfqpoint{1.387626in}{0.865432in}}%
\pgfpathlineto{\pgfqpoint{1.391457in}{0.866823in}}%
\pgfpathlineto{\pgfqpoint{1.406778in}{0.869196in}}%
\pgfpathlineto{\pgfqpoint{1.437421in}{0.875544in}}%
\pgfpathlineto{\pgfqpoint{1.441251in}{0.876711in}}%
\pgfpathlineto{\pgfqpoint{1.443166in}{0.876732in}}%
\pgfpathlineto{\pgfqpoint{1.445081in}{0.878392in}}%
\pgfpathlineto{\pgfqpoint{1.454657in}{0.879750in}}%
\pgfpathlineto{\pgfqpoint{1.458488in}{0.880740in}}%
\pgfpathlineto{\pgfqpoint{1.466148in}{0.882187in}}%
\pgfpathlineto{\pgfqpoint{1.471894in}{0.884923in}}%
\pgfpathlineto{\pgfqpoint{1.475724in}{0.885399in}}%
\pgfpathlineto{\pgfqpoint{1.479555in}{0.886870in}}%
\pgfpathlineto{\pgfqpoint{1.487215in}{0.889783in}}%
\pgfpathlineto{\pgfqpoint{1.492961in}{0.893857in}}%
\pgfpathlineto{\pgfqpoint{1.508282in}{0.898192in}}%
\pgfpathlineto{\pgfqpoint{1.512112in}{0.901490in}}%
\pgfpathlineto{\pgfqpoint{1.533179in}{0.905296in}}%
\pgfpathlineto{\pgfqpoint{1.535095in}{0.907970in}}%
\pgfpathlineto{\pgfqpoint{1.548501in}{0.909096in}}%
\pgfpathlineto{\pgfqpoint{1.552331in}{0.910375in}}%
\pgfpathlineto{\pgfqpoint{1.554246in}{0.915314in}}%
\pgfpathlineto{\pgfqpoint{1.561907in}{0.916486in}}%
\pgfpathlineto{\pgfqpoint{1.569568in}{0.918512in}}%
\pgfpathlineto{\pgfqpoint{1.573398in}{0.920964in}}%
\pgfpathlineto{\pgfqpoint{1.581059in}{0.922392in}}%
\pgfpathlineto{\pgfqpoint{1.586804in}{0.923736in}}%
\pgfpathlineto{\pgfqpoint{1.596380in}{0.925766in}}%
\pgfpathlineto{\pgfqpoint{1.602126in}{0.930186in}}%
\pgfpathlineto{\pgfqpoint{1.615532in}{0.936724in}}%
\pgfpathlineto{\pgfqpoint{1.619362in}{0.938930in}}%
\pgfpathlineto{\pgfqpoint{1.623192in}{0.940147in}}%
\pgfpathlineto{\pgfqpoint{1.630853in}{0.943115in}}%
\pgfpathlineto{\pgfqpoint{1.655750in}{0.948977in}}%
\pgfpathlineto{\pgfqpoint{1.659581in}{0.953638in}}%
\pgfpathlineto{\pgfqpoint{1.665326in}{0.955662in}}%
\pgfpathlineto{\pgfqpoint{1.669157in}{0.959602in}}%
\pgfpathlineto{\pgfqpoint{1.672987in}{0.961622in}}%
\pgfpathlineto{\pgfqpoint{1.695969in}{0.969296in}}%
\pgfpathlineto{\pgfqpoint{1.699799in}{0.972324in}}%
\pgfpathlineto{\pgfqpoint{1.705545in}{0.973571in}}%
\pgfpathlineto{\pgfqpoint{1.707460in}{0.976540in}}%
\pgfpathlineto{\pgfqpoint{1.713205in}{0.977313in}}%
\pgfpathlineto{\pgfqpoint{1.720866in}{0.980033in}}%
\pgfpathlineto{\pgfqpoint{1.757254in}{0.991382in}}%
\pgfpathlineto{\pgfqpoint{1.759170in}{0.993043in}}%
\pgfpathlineto{\pgfqpoint{1.763000in}{0.993396in}}%
\pgfpathlineto{\pgfqpoint{1.766830in}{0.995586in}}%
\pgfpathlineto{\pgfqpoint{1.784067in}{1.001721in}}%
\pgfpathlineto{\pgfqpoint{1.791728in}{1.005642in}}%
\pgfpathlineto{\pgfqpoint{1.801303in}{1.007224in}}%
\pgfpathlineto{\pgfqpoint{1.837692in}{1.018860in}}%
\pgfpathlineto{\pgfqpoint{1.843437in}{1.019803in}}%
\pgfpathlineto{\pgfqpoint{1.875995in}{1.029635in}}%
\pgfpathlineto{\pgfqpoint{1.881741in}{1.030236in}}%
\pgfpathlineto{\pgfqpoint{1.883656in}{1.033158in}}%
\pgfpathlineto{\pgfqpoint{1.889401in}{1.033877in}}%
\pgfpathlineto{\pgfqpoint{1.895147in}{1.039524in}}%
\pgfpathlineto{\pgfqpoint{1.900892in}{1.040845in}}%
\pgfpathlineto{\pgfqpoint{1.933450in}{1.048717in}}%
\pgfpathlineto{\pgfqpoint{1.937281in}{1.049934in}}%
\pgfpathlineto{\pgfqpoint{1.943026in}{1.050770in}}%
\pgfpathlineto{\pgfqpoint{1.944941in}{1.052691in}}%
\pgfpathlineto{\pgfqpoint{1.950687in}{1.053963in}}%
\pgfpathlineto{\pgfqpoint{1.956432in}{1.055846in}}%
\pgfpathlineto{\pgfqpoint{1.966008in}{1.057244in}}%
\pgfpathlineto{\pgfqpoint{1.969838in}{1.058487in}}%
\pgfpathlineto{\pgfqpoint{1.971754in}{1.058742in}}%
\pgfpathlineto{\pgfqpoint{1.973669in}{1.060820in}}%
\pgfpathlineto{\pgfqpoint{1.981329in}{1.061358in}}%
\pgfpathlineto{\pgfqpoint{1.987075in}{1.064250in}}%
\pgfpathlineto{\pgfqpoint{1.988990in}{1.065920in}}%
\pgfpathlineto{\pgfqpoint{2.000481in}{1.067769in}}%
\pgfpathlineto{\pgfqpoint{2.004312in}{1.070776in}}%
\pgfpathlineto{\pgfqpoint{2.015803in}{1.072319in}}%
\pgfpathlineto{\pgfqpoint{2.019633in}{1.073536in}}%
\pgfpathlineto{\pgfqpoint{2.042615in}{1.081412in}}%
\pgfpathlineto{\pgfqpoint{2.077088in}{1.089893in}}%
\pgfpathlineto{\pgfqpoint{2.082834in}{1.092956in}}%
\pgfpathlineto{\pgfqpoint{2.090494in}{1.095632in}}%
\pgfpathlineto{\pgfqpoint{2.094325in}{1.097070in}}%
\pgfpathlineto{\pgfqpoint{2.098155in}{1.098316in}}%
\pgfpathlineto{\pgfqpoint{2.105816in}{1.101261in}}%
\pgfpathlineto{\pgfqpoint{2.115391in}{1.103472in}}%
\pgfpathlineto{\pgfqpoint{2.119222in}{1.104594in}}%
\pgfpathlineto{\pgfqpoint{2.132628in}{1.109498in}}%
\pgfpathlineto{\pgfqpoint{2.136458in}{1.111805in}}%
\pgfpathlineto{\pgfqpoint{2.149865in}{1.112792in}}%
\pgfpathlineto{\pgfqpoint{2.153695in}{1.114591in}}%
\pgfpathlineto{\pgfqpoint{2.167101in}{1.116194in}}%
\pgfpathlineto{\pgfqpoint{2.170931in}{1.117847in}}%
\pgfpathlineto{\pgfqpoint{2.176677in}{1.119254in}}%
\pgfpathlineto{\pgfqpoint{2.180507in}{1.120370in}}%
\pgfpathlineto{\pgfqpoint{2.184338in}{1.121374in}}%
\pgfpathlineto{\pgfqpoint{2.199659in}{1.127156in}}%
\pgfpathlineto{\pgfqpoint{2.205405in}{1.128581in}}%
\pgfpathlineto{\pgfqpoint{2.216896in}{1.130093in}}%
\pgfpathlineto{\pgfqpoint{2.234132in}{1.134945in}}%
\pgfpathlineto{\pgfqpoint{2.239878in}{1.135946in}}%
\pgfpathlineto{\pgfqpoint{2.245623in}{1.138209in}}%
\pgfpathlineto{\pgfqpoint{2.249453in}{1.139815in}}%
\pgfpathlineto{\pgfqpoint{2.257114in}{1.142321in}}%
\pgfpathlineto{\pgfqpoint{2.264775in}{1.143587in}}%
\pgfpathlineto{\pgfqpoint{2.283927in}{1.147897in}}%
\pgfpathlineto{\pgfqpoint{2.295418in}{1.149580in}}%
\pgfpathlineto{\pgfqpoint{2.299248in}{1.150890in}}%
\pgfpathlineto{\pgfqpoint{2.306909in}{1.152130in}}%
\pgfpathlineto{\pgfqpoint{2.322230in}{1.157059in}}%
\pgfpathlineto{\pgfqpoint{2.327975in}{1.157649in}}%
\pgfpathlineto{\pgfqpoint{2.333721in}{1.160720in}}%
\pgfpathlineto{\pgfqpoint{2.337551in}{1.161525in}}%
\pgfpathlineto{\pgfqpoint{2.341382in}{1.164680in}}%
\pgfpathlineto{\pgfqpoint{2.347127in}{1.166453in}}%
\pgfpathlineto{\pgfqpoint{2.356703in}{1.168165in}}%
\pgfpathlineto{\pgfqpoint{2.362449in}{1.169593in}}%
\pgfpathlineto{\pgfqpoint{2.366279in}{1.170784in}}%
\pgfpathlineto{\pgfqpoint{2.379685in}{1.173841in}}%
\pgfpathlineto{\pgfqpoint{2.402667in}{1.176847in}}%
\pgfpathlineto{\pgfqpoint{2.404582in}{1.179471in}}%
\pgfpathlineto{\pgfqpoint{2.425649in}{1.183664in}}%
\pgfpathlineto{\pgfqpoint{2.429480in}{1.185025in}}%
\pgfpathlineto{\pgfqpoint{2.433310in}{1.186906in}}%
\pgfpathlineto{\pgfqpoint{2.437140in}{1.189009in}}%
\pgfpathlineto{\pgfqpoint{2.440971in}{1.189926in}}%
\pgfpathlineto{\pgfqpoint{2.442886in}{1.191432in}}%
\pgfpathlineto{\pgfqpoint{2.450546in}{1.192213in}}%
\pgfpathlineto{\pgfqpoint{2.456292in}{1.195015in}}%
\pgfpathlineto{\pgfqpoint{2.465868in}{1.198965in}}%
\pgfpathlineto{\pgfqpoint{2.469698in}{1.201862in}}%
\pgfpathlineto{\pgfqpoint{2.479274in}{1.202960in}}%
\pgfpathlineto{\pgfqpoint{2.483104in}{1.205196in}}%
\pgfpathlineto{\pgfqpoint{2.492680in}{1.207745in}}%
\pgfpathlineto{\pgfqpoint{2.498426in}{1.209027in}}%
\pgfpathlineto{\pgfqpoint{2.502256in}{1.210015in}}%
\pgfpathlineto{\pgfqpoint{2.509917in}{1.211057in}}%
\pgfpathlineto{\pgfqpoint{2.552051in}{1.212981in}}%
\pgfpathlineto{\pgfqpoint{2.565457in}{1.213993in}}%
\pgfpathlineto{\pgfqpoint{2.594184in}{1.215513in}}%
\pgfpathlineto{\pgfqpoint{2.598015in}{1.217443in}}%
\pgfpathlineto{\pgfqpoint{2.605675in}{1.219353in}}%
\pgfpathlineto{\pgfqpoint{2.643979in}{1.226295in}}%
\pgfpathlineto{\pgfqpoint{2.651639in}{1.226879in}}%
\pgfpathlineto{\pgfqpoint{2.657385in}{1.228657in}}%
\pgfpathlineto{\pgfqpoint{2.666961in}{1.230539in}}%
\pgfpathlineto{\pgfqpoint{2.678452in}{1.232686in}}%
\pgfpathlineto{\pgfqpoint{2.682282in}{1.233466in}}%
\pgfpathlineto{\pgfqpoint{2.686113in}{1.234041in}}%
\pgfpathlineto{\pgfqpoint{2.689943in}{1.235812in}}%
\pgfpathlineto{\pgfqpoint{2.693773in}{1.236928in}}%
\pgfpathlineto{\pgfqpoint{2.697604in}{1.237663in}}%
\pgfpathlineto{\pgfqpoint{2.701434in}{1.241928in}}%
\pgfpathlineto{\pgfqpoint{2.707179in}{1.243168in}}%
\pgfpathlineto{\pgfqpoint{2.709095in}{1.243584in}}%
\pgfpathlineto{\pgfqpoint{2.711010in}{1.245411in}}%
\pgfpathlineto{\pgfqpoint{2.714840in}{1.245800in}}%
\pgfpathlineto{\pgfqpoint{2.720586in}{1.248564in}}%
\pgfpathlineto{\pgfqpoint{2.722501in}{1.248574in}}%
\pgfpathlineto{\pgfqpoint{2.726331in}{1.252128in}}%
\pgfpathlineto{\pgfqpoint{2.730161in}{1.252220in}}%
\pgfpathlineto{\pgfqpoint{2.733992in}{1.253916in}}%
\pgfpathlineto{\pgfqpoint{2.737822in}{1.255324in}}%
\pgfpathlineto{\pgfqpoint{2.762719in}{1.261656in}}%
\pgfpathlineto{\pgfqpoint{2.764635in}{1.261729in}}%
\pgfpathlineto{\pgfqpoint{2.772295in}{1.267531in}}%
\pgfpathlineto{\pgfqpoint{2.776126in}{1.268150in}}%
\pgfpathlineto{\pgfqpoint{2.779956in}{1.271173in}}%
\pgfpathlineto{\pgfqpoint{2.785701in}{1.273938in}}%
\pgfpathlineto{\pgfqpoint{2.806768in}{1.275573in}}%
\pgfpathlineto{\pgfqpoint{2.810599in}{1.279500in}}%
\pgfpathlineto{\pgfqpoint{2.827835in}{1.283201in}}%
\pgfpathlineto{\pgfqpoint{2.829750in}{1.285696in}}%
\pgfpathlineto{\pgfqpoint{2.833581in}{1.286477in}}%
\pgfpathlineto{\pgfqpoint{2.835496in}{1.289064in}}%
\pgfpathlineto{\pgfqpoint{2.839326in}{1.290057in}}%
\pgfpathlineto{\pgfqpoint{2.841241in}{1.292742in}}%
\pgfpathlineto{\pgfqpoint{2.845072in}{1.292965in}}%
\pgfpathlineto{\pgfqpoint{2.846987in}{1.295204in}}%
\pgfpathlineto{\pgfqpoint{2.860393in}{1.297225in}}%
\pgfpathlineto{\pgfqpoint{2.866139in}{1.302135in}}%
\pgfpathlineto{\pgfqpoint{2.889121in}{1.312696in}}%
\pgfpathlineto{\pgfqpoint{2.891036in}{1.315874in}}%
\pgfpathlineto{\pgfqpoint{2.904442in}{1.317365in}}%
\pgfpathlineto{\pgfqpoint{2.906357in}{1.321322in}}%
\pgfpathlineto{\pgfqpoint{2.908272in}{1.329944in}}%
\pgfpathlineto{\pgfqpoint{2.910188in}{1.330100in}}%
\pgfpathlineto{\pgfqpoint{2.912103in}{1.332755in}}%
\pgfpathlineto{\pgfqpoint{2.919763in}{1.335560in}}%
\pgfpathlineto{\pgfqpoint{2.923594in}{1.339059in}}%
\pgfpathlineto{\pgfqpoint{2.927424in}{1.340398in}}%
\pgfpathlineto{\pgfqpoint{2.958067in}{1.350708in}}%
\pgfpathlineto{\pgfqpoint{2.963812in}{1.355135in}}%
\pgfpathlineto{\pgfqpoint{2.981049in}{1.361952in}}%
\pgfpathlineto{\pgfqpoint{2.984879in}{1.362418in}}%
\pgfpathlineto{\pgfqpoint{2.992540in}{1.368981in}}%
\pgfpathlineto{\pgfqpoint{2.996370in}{1.369749in}}%
\pgfpathlineto{\pgfqpoint{3.007861in}{1.377972in}}%
\pgfpathlineto{\pgfqpoint{3.013607in}{1.379154in}}%
\pgfpathlineto{\pgfqpoint{3.015522in}{1.381125in}}%
\pgfpathlineto{\pgfqpoint{3.019352in}{1.382284in}}%
\pgfpathlineto{\pgfqpoint{3.023183in}{1.383969in}}%
\pgfpathlineto{\pgfqpoint{3.030843in}{1.388174in}}%
\pgfpathlineto{\pgfqpoint{3.034674in}{1.388832in}}%
\pgfpathlineto{\pgfqpoint{3.038504in}{1.391067in}}%
\pgfpathlineto{\pgfqpoint{3.040419in}{1.391731in}}%
\pgfpathlineto{\pgfqpoint{3.042334in}{1.394733in}}%
\pgfpathlineto{\pgfqpoint{3.046165in}{1.396049in}}%
\pgfpathlineto{\pgfqpoint{3.049995in}{1.400052in}}%
\pgfpathlineto{\pgfqpoint{3.053825in}{1.400208in}}%
\pgfpathlineto{\pgfqpoint{3.057656in}{1.402703in}}%
\pgfpathlineto{\pgfqpoint{3.069147in}{1.410458in}}%
\pgfpathlineto{\pgfqpoint{3.071062in}{1.414124in}}%
\pgfpathlineto{\pgfqpoint{3.080638in}{1.417918in}}%
\pgfpathlineto{\pgfqpoint{3.082553in}{1.423634in}}%
\pgfpathlineto{\pgfqpoint{3.084468in}{1.425126in}}%
\pgfpathlineto{\pgfqpoint{3.086383in}{1.425132in}}%
\pgfpathlineto{\pgfqpoint{3.090214in}{1.428405in}}%
\pgfpathlineto{\pgfqpoint{3.111281in}{1.435254in}}%
\pgfpathlineto{\pgfqpoint{3.128517in}{1.440091in}}%
\pgfpathlineto{\pgfqpoint{3.130432in}{1.442752in}}%
\pgfpathlineto{\pgfqpoint{3.136178in}{1.443861in}}%
\pgfpathlineto{\pgfqpoint{3.140008in}{1.445953in}}%
\pgfpathlineto{\pgfqpoint{3.147669in}{1.448739in}}%
\pgfpathlineto{\pgfqpoint{3.151499in}{1.450169in}}%
\pgfpathlineto{\pgfqpoint{3.155330in}{1.450717in}}%
\pgfpathlineto{\pgfqpoint{3.159160in}{1.452345in}}%
\pgfpathlineto{\pgfqpoint{3.162990in}{1.453590in}}%
\pgfpathlineto{\pgfqpoint{3.168736in}{1.456175in}}%
\pgfpathlineto{\pgfqpoint{3.170651in}{1.459855in}}%
\pgfpathlineto{\pgfqpoint{3.176396in}{1.461013in}}%
\pgfpathlineto{\pgfqpoint{3.180227in}{1.461737in}}%
\pgfpathlineto{\pgfqpoint{3.184057in}{1.468668in}}%
\pgfpathlineto{\pgfqpoint{3.187887in}{1.469118in}}%
\pgfpathlineto{\pgfqpoint{3.189803in}{1.472435in}}%
\pgfpathlineto{\pgfqpoint{3.193633in}{1.473471in}}%
\pgfpathlineto{\pgfqpoint{3.197463in}{1.475362in}}%
\pgfpathlineto{\pgfqpoint{3.205124in}{1.477145in}}%
\pgfpathlineto{\pgfqpoint{3.210870in}{1.479001in}}%
\pgfpathlineto{\pgfqpoint{3.214700in}{1.483074in}}%
\pgfpathlineto{\pgfqpoint{3.216615in}{1.483519in}}%
\pgfpathlineto{\pgfqpoint{3.220445in}{1.489107in}}%
\pgfpathlineto{\pgfqpoint{3.241512in}{1.495171in}}%
\pgfpathlineto{\pgfqpoint{3.243427in}{1.497359in}}%
\pgfpathlineto{\pgfqpoint{3.245343in}{1.502089in}}%
\pgfpathlineto{\pgfqpoint{3.249173in}{1.502246in}}%
\pgfpathlineto{\pgfqpoint{3.253003in}{1.504828in}}%
\pgfpathlineto{\pgfqpoint{3.254918in}{1.511013in}}%
\pgfpathlineto{\pgfqpoint{3.264494in}{1.515597in}}%
\pgfpathlineto{\pgfqpoint{3.277901in}{1.520604in}}%
\pgfpathlineto{\pgfqpoint{3.281731in}{1.523811in}}%
\pgfpathlineto{\pgfqpoint{3.289392in}{1.526572in}}%
\pgfpathlineto{\pgfqpoint{3.291307in}{1.526778in}}%
\pgfpathlineto{\pgfqpoint{3.293222in}{1.529345in}}%
\pgfpathlineto{\pgfqpoint{3.310458in}{1.535169in}}%
\pgfpathlineto{\pgfqpoint{3.316204in}{1.539600in}}%
\pgfpathlineto{\pgfqpoint{3.321949in}{1.543436in}}%
\pgfpathlineto{\pgfqpoint{3.323865in}{1.543479in}}%
\pgfpathlineto{\pgfqpoint{3.329610in}{1.549443in}}%
\pgfpathlineto{\pgfqpoint{3.339186in}{1.552841in}}%
\pgfpathlineto{\pgfqpoint{3.344932in}{1.554417in}}%
\pgfpathlineto{\pgfqpoint{3.346847in}{1.554985in}}%
\pgfpathlineto{\pgfqpoint{3.348762in}{1.557702in}}%
\pgfpathlineto{\pgfqpoint{3.352592in}{1.559543in}}%
\pgfpathlineto{\pgfqpoint{3.356423in}{1.563765in}}%
\pgfpathlineto{\pgfqpoint{3.358338in}{1.564854in}}%
\pgfpathlineto{\pgfqpoint{3.362168in}{1.570656in}}%
\pgfpathlineto{\pgfqpoint{3.365998in}{1.570746in}}%
\pgfpathlineto{\pgfqpoint{3.369829in}{1.572424in}}%
\pgfpathlineto{\pgfqpoint{3.371744in}{1.572468in}}%
\pgfpathlineto{\pgfqpoint{3.373659in}{1.576239in}}%
\pgfpathlineto{\pgfqpoint{3.379405in}{1.577753in}}%
\pgfpathlineto{\pgfqpoint{3.383235in}{1.581062in}}%
\pgfpathlineto{\pgfqpoint{3.387065in}{1.590592in}}%
\pgfpathlineto{\pgfqpoint{3.390896in}{1.591741in}}%
\pgfpathlineto{\pgfqpoint{3.392811in}{1.593627in}}%
\pgfpathlineto{\pgfqpoint{3.396641in}{1.594292in}}%
\pgfpathlineto{\pgfqpoint{3.400471in}{1.597025in}}%
\pgfpathlineto{\pgfqpoint{3.406217in}{1.598912in}}%
\pgfpathlineto{\pgfqpoint{3.410047in}{1.604751in}}%
\pgfpathlineto{\pgfqpoint{3.411963in}{1.607614in}}%
\pgfpathlineto{\pgfqpoint{3.415793in}{1.609685in}}%
\pgfpathlineto{\pgfqpoint{3.417708in}{1.617180in}}%
\pgfpathlineto{\pgfqpoint{3.419623in}{1.617559in}}%
\pgfpathlineto{\pgfqpoint{3.421538in}{1.620188in}}%
\pgfpathlineto{\pgfqpoint{3.423454in}{1.620864in}}%
\pgfpathlineto{\pgfqpoint{3.425369in}{1.625956in}}%
\pgfpathlineto{\pgfqpoint{3.427284in}{1.626630in}}%
\pgfpathlineto{\pgfqpoint{3.433029in}{1.634608in}}%
\pgfpathlineto{\pgfqpoint{3.438775in}{1.661380in}}%
\pgfpathlineto{\pgfqpoint{3.440690in}{1.662821in}}%
\pgfpathlineto{\pgfqpoint{3.444520in}{1.670388in}}%
\pgfpathlineto{\pgfqpoint{3.448351in}{1.673278in}}%
\pgfpathlineto{\pgfqpoint{3.450266in}{1.678623in}}%
\pgfpathlineto{\pgfqpoint{3.454096in}{1.680635in}}%
\pgfpathlineto{\pgfqpoint{3.457927in}{1.684278in}}%
\pgfpathlineto{\pgfqpoint{3.459842in}{1.685004in}}%
\pgfpathlineto{\pgfqpoint{3.465587in}{1.691149in}}%
\pgfpathlineto{\pgfqpoint{3.473248in}{1.694655in}}%
\pgfpathlineto{\pgfqpoint{3.478994in}{1.702339in}}%
\pgfpathlineto{\pgfqpoint{3.486654in}{1.716146in}}%
\pgfpathlineto{\pgfqpoint{3.490485in}{1.720723in}}%
\pgfpathlineto{\pgfqpoint{3.492400in}{1.729285in}}%
\pgfpathlineto{\pgfqpoint{3.494315in}{1.732109in}}%
\pgfpathlineto{\pgfqpoint{3.500060in}{1.759018in}}%
\pgfpathlineto{\pgfqpoint{3.501976in}{1.760173in}}%
\pgfpathlineto{\pgfqpoint{3.503891in}{1.763214in}}%
\pgfpathlineto{\pgfqpoint{3.505806in}{1.777134in}}%
\pgfpathlineto{\pgfqpoint{3.509636in}{1.782265in}}%
\pgfpathlineto{\pgfqpoint{3.511551in}{1.784947in}}%
\pgfpathlineto{\pgfqpoint{3.513467in}{1.785280in}}%
\pgfpathlineto{\pgfqpoint{3.515382in}{1.789979in}}%
\pgfpathlineto{\pgfqpoint{3.517297in}{1.790244in}}%
\pgfpathlineto{\pgfqpoint{3.519212in}{1.794338in}}%
\pgfpathlineto{\pgfqpoint{3.524958in}{1.797976in}}%
\pgfpathlineto{\pgfqpoint{3.526873in}{1.801316in}}%
\pgfpathlineto{\pgfqpoint{3.528788in}{1.809673in}}%
\pgfpathlineto{\pgfqpoint{3.530703in}{1.811067in}}%
\pgfpathlineto{\pgfqpoint{3.532618in}{1.818487in}}%
\pgfpathlineto{\pgfqpoint{3.538364in}{1.820430in}}%
\pgfpathlineto{\pgfqpoint{3.542194in}{1.824140in}}%
\pgfpathlineto{\pgfqpoint{3.549855in}{1.831312in}}%
\pgfpathlineto{\pgfqpoint{3.551770in}{1.838283in}}%
\pgfpathlineto{\pgfqpoint{3.553685in}{1.839646in}}%
\pgfpathlineto{\pgfqpoint{3.555600in}{1.846809in}}%
\pgfpathlineto{\pgfqpoint{3.557516in}{1.846834in}}%
\pgfpathlineto{\pgfqpoint{3.565176in}{1.854043in}}%
\pgfpathlineto{\pgfqpoint{3.567091in}{1.860332in}}%
\pgfpathlineto{\pgfqpoint{3.569007in}{1.875718in}}%
\pgfpathlineto{\pgfqpoint{3.572837in}{1.878939in}}%
\pgfpathlineto{\pgfqpoint{3.574752in}{1.888651in}}%
\pgfpathlineto{\pgfqpoint{3.576667in}{1.890376in}}%
\pgfpathlineto{\pgfqpoint{3.580498in}{1.902750in}}%
\pgfpathlineto{\pgfqpoint{3.582413in}{1.906707in}}%
\pgfpathlineto{\pgfqpoint{3.584328in}{1.914627in}}%
\pgfpathlineto{\pgfqpoint{3.586243in}{1.916495in}}%
\pgfpathlineto{\pgfqpoint{3.588158in}{1.932149in}}%
\pgfpathlineto{\pgfqpoint{3.590073in}{1.935278in}}%
\pgfpathlineto{\pgfqpoint{3.595819in}{1.961662in}}%
\pgfpathlineto{\pgfqpoint{3.599649in}{1.967401in}}%
\pgfpathlineto{\pgfqpoint{3.609225in}{1.988901in}}%
\pgfpathlineto{\pgfqpoint{3.616886in}{1.998794in}}%
\pgfpathlineto{\pgfqpoint{3.620716in}{2.020084in}}%
\pgfpathlineto{\pgfqpoint{3.622631in}{2.021032in}}%
\pgfpathlineto{\pgfqpoint{3.628377in}{2.029584in}}%
\pgfpathlineto{\pgfqpoint{3.632207in}{2.037393in}}%
\pgfpathlineto{\pgfqpoint{3.634122in}{2.043145in}}%
\pgfpathlineto{\pgfqpoint{3.637953in}{2.056978in}}%
\pgfpathlineto{\pgfqpoint{3.641783in}{2.063208in}}%
\pgfpathlineto{\pgfqpoint{3.645613in}{2.085397in}}%
\pgfpathlineto{\pgfqpoint{3.647529in}{2.091692in}}%
\pgfpathlineto{\pgfqpoint{3.647529in}{2.091692in}}%
\pgfusepath{stroke}%
\end{pgfscope}%
\begin{pgfscope}%
\pgfpathrectangle{\pgfqpoint{0.694334in}{0.523557in}}{\pgfqpoint{3.830343in}{1.568135in}}%
\pgfusepath{clip}%
\pgfsetbuttcap%
\pgfsetroundjoin%
\pgfsetlinewidth{1.003750pt}%
\definecolor{currentstroke}{rgb}{0.564706,0.564706,1.000000}%
\pgfsetstrokecolor{currentstroke}%
\pgfsetdash{{3.700000pt}{1.600000pt}}{0.000000pt}%
\pgfpathmoveto{\pgfqpoint{0.694334in}{0.698418in}}%
\pgfpathlineto{\pgfqpoint{0.696249in}{0.702063in}}%
\pgfpathlineto{\pgfqpoint{0.701995in}{0.723382in}}%
\pgfpathlineto{\pgfqpoint{0.703910in}{0.724449in}}%
\pgfpathlineto{\pgfqpoint{0.705825in}{0.736330in}}%
\pgfpathlineto{\pgfqpoint{0.713486in}{0.747300in}}%
\pgfpathlineto{\pgfqpoint{0.717316in}{0.762942in}}%
\pgfpathlineto{\pgfqpoint{0.719232in}{0.765647in}}%
\pgfpathlineto{\pgfqpoint{0.723062in}{0.768130in}}%
\pgfpathlineto{\pgfqpoint{0.734553in}{0.776861in}}%
\pgfpathlineto{\pgfqpoint{0.761365in}{0.788626in}}%
\pgfpathlineto{\pgfqpoint{0.765196in}{0.790289in}}%
\pgfpathlineto{\pgfqpoint{0.770941in}{0.791305in}}%
\pgfpathlineto{\pgfqpoint{0.774772in}{0.792383in}}%
\pgfpathlineto{\pgfqpoint{0.778602in}{0.792595in}}%
\pgfpathlineto{\pgfqpoint{0.782432in}{0.794542in}}%
\pgfpathlineto{\pgfqpoint{0.792008in}{0.796220in}}%
\pgfpathlineto{\pgfqpoint{0.799669in}{0.797167in}}%
\pgfpathlineto{\pgfqpoint{0.805414in}{0.798403in}}%
\pgfpathlineto{\pgfqpoint{0.830311in}{0.802699in}}%
\pgfpathlineto{\pgfqpoint{0.836057in}{0.803766in}}%
\pgfpathlineto{\pgfqpoint{0.841803in}{0.804565in}}%
\pgfpathlineto{\pgfqpoint{0.859039in}{0.808293in}}%
\pgfpathlineto{\pgfqpoint{0.864785in}{0.808963in}}%
\pgfpathlineto{\pgfqpoint{0.889682in}{0.810350in}}%
\pgfpathlineto{\pgfqpoint{0.916494in}{0.814363in}}%
\pgfpathlineto{\pgfqpoint{0.924155in}{0.815128in}}%
\pgfpathlineto{\pgfqpoint{0.927985in}{0.816992in}}%
\pgfpathlineto{\pgfqpoint{1.014168in}{0.827492in}}%
\pgfpathlineto{\pgfqpoint{1.029489in}{0.828762in}}%
\pgfpathlineto{\pgfqpoint{1.046726in}{0.831445in}}%
\pgfpathlineto{\pgfqpoint{1.062047in}{0.833577in}}%
\pgfpathlineto{\pgfqpoint{1.079284in}{0.835315in}}%
\pgfpathlineto{\pgfqpoint{1.083114in}{0.836300in}}%
\pgfpathlineto{\pgfqpoint{1.098435in}{0.837908in}}%
\pgfpathlineto{\pgfqpoint{1.102266in}{0.839157in}}%
\pgfpathlineto{\pgfqpoint{1.125248in}{0.841351in}}%
\pgfpathlineto{\pgfqpoint{1.134824in}{0.843049in}}%
\pgfpathlineto{\pgfqpoint{1.219091in}{0.850255in}}%
\pgfpathlineto{\pgfqpoint{1.232497in}{0.851749in}}%
\pgfpathlineto{\pgfqpoint{1.284207in}{0.858058in}}%
\pgfpathlineto{\pgfqpoint{1.288037in}{0.859214in}}%
\pgfpathlineto{\pgfqpoint{1.295698in}{0.860406in}}%
\pgfpathlineto{\pgfqpoint{1.307189in}{0.861615in}}%
\pgfpathlineto{\pgfqpoint{1.316765in}{0.863683in}}%
\pgfpathlineto{\pgfqpoint{1.337832in}{0.867759in}}%
\pgfpathlineto{\pgfqpoint{1.347408in}{0.868549in}}%
\pgfpathlineto{\pgfqpoint{1.355068in}{0.869850in}}%
\pgfpathlineto{\pgfqpoint{1.360814in}{0.870873in}}%
\pgfpathlineto{\pgfqpoint{1.366559in}{0.871801in}}%
\pgfpathlineto{\pgfqpoint{1.387626in}{0.874025in}}%
\pgfpathlineto{\pgfqpoint{1.391457in}{0.875412in}}%
\pgfpathlineto{\pgfqpoint{1.402948in}{0.877465in}}%
\pgfpathlineto{\pgfqpoint{1.406778in}{0.879721in}}%
\pgfpathlineto{\pgfqpoint{1.437421in}{0.884899in}}%
\pgfpathlineto{\pgfqpoint{1.443166in}{0.886225in}}%
\pgfpathlineto{\pgfqpoint{1.456573in}{0.888295in}}%
\pgfpathlineto{\pgfqpoint{1.477639in}{0.890090in}}%
\pgfpathlineto{\pgfqpoint{1.481470in}{0.891397in}}%
\pgfpathlineto{\pgfqpoint{1.496791in}{0.895465in}}%
\pgfpathlineto{\pgfqpoint{1.500621in}{0.896989in}}%
\pgfpathlineto{\pgfqpoint{1.508282in}{0.899139in}}%
\pgfpathlineto{\pgfqpoint{1.514028in}{0.899967in}}%
\pgfpathlineto{\pgfqpoint{1.519773in}{0.901081in}}%
\pgfpathlineto{\pgfqpoint{1.521688in}{0.902555in}}%
\pgfpathlineto{\pgfqpoint{1.535095in}{0.903412in}}%
\pgfpathlineto{\pgfqpoint{1.537010in}{0.906318in}}%
\pgfpathlineto{\pgfqpoint{1.544670in}{0.907711in}}%
\pgfpathlineto{\pgfqpoint{1.558077in}{0.910624in}}%
\pgfpathlineto{\pgfqpoint{1.561907in}{0.911022in}}%
\pgfpathlineto{\pgfqpoint{1.567652in}{0.913334in}}%
\pgfpathlineto{\pgfqpoint{1.582974in}{0.918013in}}%
\pgfpathlineto{\pgfqpoint{1.592550in}{0.922090in}}%
\pgfpathlineto{\pgfqpoint{1.609786in}{0.924984in}}%
\pgfpathlineto{\pgfqpoint{1.611701in}{0.927315in}}%
\pgfpathlineto{\pgfqpoint{1.619362in}{0.929028in}}%
\pgfpathlineto{\pgfqpoint{1.623192in}{0.930345in}}%
\pgfpathlineto{\pgfqpoint{1.628938in}{0.931287in}}%
\pgfpathlineto{\pgfqpoint{1.632768in}{0.935923in}}%
\pgfpathlineto{\pgfqpoint{1.638514in}{0.937072in}}%
\pgfpathlineto{\pgfqpoint{1.646174in}{0.939675in}}%
\pgfpathlineto{\pgfqpoint{1.650005in}{0.942321in}}%
\pgfpathlineto{\pgfqpoint{1.653835in}{0.943919in}}%
\pgfpathlineto{\pgfqpoint{1.661496in}{0.945353in}}%
\pgfpathlineto{\pgfqpoint{1.669157in}{0.952308in}}%
\pgfpathlineto{\pgfqpoint{1.695969in}{0.958437in}}%
\pgfpathlineto{\pgfqpoint{1.699799in}{0.958647in}}%
\pgfpathlineto{\pgfqpoint{1.703630in}{0.960770in}}%
\pgfpathlineto{\pgfqpoint{1.709375in}{0.963645in}}%
\pgfpathlineto{\pgfqpoint{1.713205in}{0.964246in}}%
\pgfpathlineto{\pgfqpoint{1.724697in}{0.967925in}}%
\pgfpathlineto{\pgfqpoint{1.738103in}{0.972741in}}%
\pgfpathlineto{\pgfqpoint{1.740018in}{0.975072in}}%
\pgfpathlineto{\pgfqpoint{1.741933in}{0.975099in}}%
\pgfpathlineto{\pgfqpoint{1.745763in}{0.978175in}}%
\pgfpathlineto{\pgfqpoint{1.766830in}{0.985569in}}%
\pgfpathlineto{\pgfqpoint{1.778321in}{0.988418in}}%
\pgfpathlineto{\pgfqpoint{1.782152in}{0.990481in}}%
\pgfpathlineto{\pgfqpoint{1.785982in}{0.992695in}}%
\pgfpathlineto{\pgfqpoint{1.789812in}{1.000987in}}%
\pgfpathlineto{\pgfqpoint{1.793643in}{1.001099in}}%
\pgfpathlineto{\pgfqpoint{1.797473in}{1.002807in}}%
\pgfpathlineto{\pgfqpoint{1.808964in}{1.003738in}}%
\pgfpathlineto{\pgfqpoint{1.814710in}{1.004821in}}%
\pgfpathlineto{\pgfqpoint{1.816625in}{1.004901in}}%
\pgfpathlineto{\pgfqpoint{1.820455in}{1.006448in}}%
\pgfpathlineto{\pgfqpoint{1.833861in}{1.009265in}}%
\pgfpathlineto{\pgfqpoint{1.841522in}{1.011573in}}%
\pgfpathlineto{\pgfqpoint{1.845352in}{1.012317in}}%
\pgfpathlineto{\pgfqpoint{1.856843in}{1.016005in}}%
\pgfpathlineto{\pgfqpoint{1.860674in}{1.017983in}}%
\pgfpathlineto{\pgfqpoint{1.864504in}{1.018144in}}%
\pgfpathlineto{\pgfqpoint{1.868334in}{1.024416in}}%
\pgfpathlineto{\pgfqpoint{1.872165in}{1.024904in}}%
\pgfpathlineto{\pgfqpoint{1.875995in}{1.027556in}}%
\pgfpathlineto{\pgfqpoint{1.881741in}{1.030775in}}%
\pgfpathlineto{\pgfqpoint{1.883656in}{1.034581in}}%
\pgfpathlineto{\pgfqpoint{1.891316in}{1.036547in}}%
\pgfpathlineto{\pgfqpoint{1.898977in}{1.039193in}}%
\pgfpathlineto{\pgfqpoint{1.904723in}{1.040582in}}%
\pgfpathlineto{\pgfqpoint{1.908553in}{1.041014in}}%
\pgfpathlineto{\pgfqpoint{1.910468in}{1.043403in}}%
\pgfpathlineto{\pgfqpoint{1.914298in}{1.044237in}}%
\pgfpathlineto{\pgfqpoint{1.921959in}{1.047483in}}%
\pgfpathlineto{\pgfqpoint{1.929620in}{1.050814in}}%
\pgfpathlineto{\pgfqpoint{1.956432in}{1.056422in}}%
\pgfpathlineto{\pgfqpoint{1.960263in}{1.058165in}}%
\pgfpathlineto{\pgfqpoint{1.964093in}{1.059230in}}%
\pgfpathlineto{\pgfqpoint{1.969838in}{1.061842in}}%
\pgfpathlineto{\pgfqpoint{1.973669in}{1.062092in}}%
\pgfpathlineto{\pgfqpoint{1.975584in}{1.063825in}}%
\pgfpathlineto{\pgfqpoint{1.977499in}{1.067617in}}%
\pgfpathlineto{\pgfqpoint{1.983245in}{1.070586in}}%
\pgfpathlineto{\pgfqpoint{1.987075in}{1.071691in}}%
\pgfpathlineto{\pgfqpoint{1.994736in}{1.073435in}}%
\pgfpathlineto{\pgfqpoint{1.996651in}{1.076233in}}%
\pgfpathlineto{\pgfqpoint{2.006227in}{1.078195in}}%
\pgfpathlineto{\pgfqpoint{2.010057in}{1.080342in}}%
\pgfpathlineto{\pgfqpoint{2.011972in}{1.083179in}}%
\pgfpathlineto{\pgfqpoint{2.023463in}{1.087914in}}%
\pgfpathlineto{\pgfqpoint{2.029209in}{1.092409in}}%
\pgfpathlineto{\pgfqpoint{2.033039in}{1.092645in}}%
\pgfpathlineto{\pgfqpoint{2.038785in}{1.095580in}}%
\pgfpathlineto{\pgfqpoint{2.040700in}{1.096086in}}%
\pgfpathlineto{\pgfqpoint{2.048360in}{1.101651in}}%
\pgfpathlineto{\pgfqpoint{2.052191in}{1.102411in}}%
\pgfpathlineto{\pgfqpoint{2.054106in}{1.104406in}}%
\pgfpathlineto{\pgfqpoint{2.059852in}{1.105965in}}%
\pgfpathlineto{\pgfqpoint{2.092409in}{1.120519in}}%
\pgfpathlineto{\pgfqpoint{2.096240in}{1.123007in}}%
\pgfpathlineto{\pgfqpoint{2.107731in}{1.127607in}}%
\pgfpathlineto{\pgfqpoint{2.109646in}{1.130536in}}%
\pgfpathlineto{\pgfqpoint{2.111561in}{1.131341in}}%
\pgfpathlineto{\pgfqpoint{2.113476in}{1.133412in}}%
\pgfpathlineto{\pgfqpoint{2.130713in}{1.136892in}}%
\pgfpathlineto{\pgfqpoint{2.136458in}{1.139957in}}%
\pgfpathlineto{\pgfqpoint{2.140289in}{1.140465in}}%
\pgfpathlineto{\pgfqpoint{2.146034in}{1.143760in}}%
\pgfpathlineto{\pgfqpoint{2.147949in}{1.146067in}}%
\pgfpathlineto{\pgfqpoint{2.153695in}{1.146657in}}%
\pgfpathlineto{\pgfqpoint{2.157525in}{1.149976in}}%
\pgfpathlineto{\pgfqpoint{2.161356in}{1.150627in}}%
\pgfpathlineto{\pgfqpoint{2.165186in}{1.152614in}}%
\pgfpathlineto{\pgfqpoint{2.178592in}{1.154421in}}%
\pgfpathlineto{\pgfqpoint{2.182422in}{1.158619in}}%
\pgfpathlineto{\pgfqpoint{2.186253in}{1.159764in}}%
\pgfpathlineto{\pgfqpoint{2.188168in}{1.159877in}}%
\pgfpathlineto{\pgfqpoint{2.193914in}{1.162714in}}%
\pgfpathlineto{\pgfqpoint{2.197744in}{1.163366in}}%
\pgfpathlineto{\pgfqpoint{2.201574in}{1.166604in}}%
\pgfpathlineto{\pgfqpoint{2.209235in}{1.168549in}}%
\pgfpathlineto{\pgfqpoint{2.214980in}{1.171326in}}%
\pgfpathlineto{\pgfqpoint{2.222641in}{1.175186in}}%
\pgfpathlineto{\pgfqpoint{2.226471in}{1.176604in}}%
\pgfpathlineto{\pgfqpoint{2.230302in}{1.177372in}}%
\pgfpathlineto{\pgfqpoint{2.232217in}{1.179307in}}%
\pgfpathlineto{\pgfqpoint{2.234132in}{1.179416in}}%
\pgfpathlineto{\pgfqpoint{2.236047in}{1.182042in}}%
\pgfpathlineto{\pgfqpoint{2.287757in}{1.193593in}}%
\pgfpathlineto{\pgfqpoint{2.289672in}{1.195842in}}%
\pgfpathlineto{\pgfqpoint{2.306909in}{1.200606in}}%
\pgfpathlineto{\pgfqpoint{2.308824in}{1.203620in}}%
\pgfpathlineto{\pgfqpoint{2.312654in}{1.204771in}}%
\pgfpathlineto{\pgfqpoint{2.316484in}{1.205276in}}%
\pgfpathlineto{\pgfqpoint{2.320315in}{1.208450in}}%
\pgfpathlineto{\pgfqpoint{2.329891in}{1.211014in}}%
\pgfpathlineto{\pgfqpoint{2.372024in}{1.213068in}}%
\pgfpathlineto{\pgfqpoint{2.421819in}{1.219194in}}%
\pgfpathlineto{\pgfqpoint{2.439055in}{1.223092in}}%
\pgfpathlineto{\pgfqpoint{2.448631in}{1.224720in}}%
\pgfpathlineto{\pgfqpoint{2.475444in}{1.230265in}}%
\pgfpathlineto{\pgfqpoint{2.479274in}{1.232221in}}%
\pgfpathlineto{\pgfqpoint{2.483104in}{1.233431in}}%
\pgfpathlineto{\pgfqpoint{2.486935in}{1.235566in}}%
\pgfpathlineto{\pgfqpoint{2.492680in}{1.237663in}}%
\pgfpathlineto{\pgfqpoint{2.498426in}{1.238568in}}%
\pgfpathlineto{\pgfqpoint{2.500341in}{1.241623in}}%
\pgfpathlineto{\pgfqpoint{2.504171in}{1.242002in}}%
\pgfpathlineto{\pgfqpoint{2.506086in}{1.243584in}}%
\pgfpathlineto{\pgfqpoint{2.509917in}{1.243856in}}%
\pgfpathlineto{\pgfqpoint{2.513747in}{1.251091in}}%
\pgfpathlineto{\pgfqpoint{2.523323in}{1.253086in}}%
\pgfpathlineto{\pgfqpoint{2.542475in}{1.273859in}}%
\pgfpathlineto{\pgfqpoint{2.546305in}{1.274807in}}%
\pgfpathlineto{\pgfqpoint{2.550135in}{1.277668in}}%
\pgfpathlineto{\pgfqpoint{2.553966in}{1.278291in}}%
\pgfpathlineto{\pgfqpoint{2.557796in}{1.280152in}}%
\pgfpathlineto{\pgfqpoint{2.559711in}{1.281230in}}%
\pgfpathlineto{\pgfqpoint{2.563542in}{1.286267in}}%
\pgfpathlineto{\pgfqpoint{2.567372in}{1.287665in}}%
\pgfpathlineto{\pgfqpoint{2.569287in}{1.288098in}}%
\pgfpathlineto{\pgfqpoint{2.580778in}{1.299344in}}%
\pgfpathlineto{\pgfqpoint{2.584608in}{1.300282in}}%
\pgfpathlineto{\pgfqpoint{2.588439in}{1.301203in}}%
\pgfpathlineto{\pgfqpoint{2.590354in}{1.305730in}}%
\pgfpathlineto{\pgfqpoint{2.592269in}{1.320779in}}%
\pgfpathlineto{\pgfqpoint{2.596099in}{1.326524in}}%
\pgfpathlineto{\pgfqpoint{2.599930in}{1.329297in}}%
\pgfpathlineto{\pgfqpoint{2.601845in}{1.331275in}}%
\pgfpathlineto{\pgfqpoint{2.613336in}{1.332425in}}%
\pgfpathlineto{\pgfqpoint{2.624827in}{1.333519in}}%
\pgfpathlineto{\pgfqpoint{2.638233in}{1.334235in}}%
\pgfpathlineto{\pgfqpoint{2.661215in}{1.335111in}}%
\pgfpathlineto{\pgfqpoint{2.666961in}{1.335622in}}%
\pgfpathlineto{\pgfqpoint{2.684197in}{1.336743in}}%
\pgfpathlineto{\pgfqpoint{2.711010in}{1.337968in}}%
\pgfpathlineto{\pgfqpoint{2.732077in}{1.339155in}}%
\pgfpathlineto{\pgfqpoint{2.755059in}{1.341798in}}%
\pgfpathlineto{\pgfqpoint{2.758889in}{1.342770in}}%
\pgfpathlineto{\pgfqpoint{2.766550in}{1.343507in}}%
\pgfpathlineto{\pgfqpoint{2.772295in}{1.344863in}}%
\pgfpathlineto{\pgfqpoint{2.833581in}{1.355138in}}%
\pgfpathlineto{\pgfqpoint{2.837411in}{1.356325in}}%
\pgfpathlineto{\pgfqpoint{2.864223in}{1.360454in}}%
\pgfpathlineto{\pgfqpoint{2.869969in}{1.362165in}}%
\pgfpathlineto{\pgfqpoint{2.954237in}{1.375018in}}%
\pgfpathlineto{\pgfqpoint{2.958067in}{1.376196in}}%
\pgfpathlineto{\pgfqpoint{2.971473in}{1.378500in}}%
\pgfpathlineto{\pgfqpoint{2.977219in}{1.380089in}}%
\pgfpathlineto{\pgfqpoint{2.984879in}{1.381103in}}%
\pgfpathlineto{\pgfqpoint{2.990625in}{1.382246in}}%
\pgfpathlineto{\pgfqpoint{3.007861in}{1.384252in}}%
\pgfpathlineto{\pgfqpoint{3.025098in}{1.387067in}}%
\pgfpathlineto{\pgfqpoint{3.030843in}{1.388223in}}%
\pgfpathlineto{\pgfqpoint{3.061486in}{1.393972in}}%
\pgfpathlineto{\pgfqpoint{3.063401in}{1.395694in}}%
\pgfpathlineto{\pgfqpoint{3.072977in}{1.396853in}}%
\pgfpathlineto{\pgfqpoint{3.086383in}{1.398858in}}%
\pgfpathlineto{\pgfqpoint{3.115111in}{1.400830in}}%
\pgfpathlineto{\pgfqpoint{3.118941in}{1.401644in}}%
\pgfpathlineto{\pgfqpoint{3.122772in}{1.401997in}}%
\pgfpathlineto{\pgfqpoint{3.126602in}{1.404148in}}%
\pgfpathlineto{\pgfqpoint{3.130432in}{1.406023in}}%
\pgfpathlineto{\pgfqpoint{3.136178in}{1.413140in}}%
\pgfpathlineto{\pgfqpoint{3.140008in}{1.413546in}}%
\pgfpathlineto{\pgfqpoint{3.145754in}{1.415929in}}%
\pgfpathlineto{\pgfqpoint{3.155330in}{1.416640in}}%
\pgfpathlineto{\pgfqpoint{3.159160in}{1.417926in}}%
\pgfpathlineto{\pgfqpoint{3.162990in}{1.419271in}}%
\pgfpathlineto{\pgfqpoint{3.166821in}{1.420815in}}%
\pgfpathlineto{\pgfqpoint{3.176396in}{1.422768in}}%
\pgfpathlineto{\pgfqpoint{3.184057in}{1.424153in}}%
\pgfpathlineto{\pgfqpoint{3.187887in}{1.426410in}}%
\pgfpathlineto{\pgfqpoint{3.191718in}{1.427443in}}%
\pgfpathlineto{\pgfqpoint{3.195548in}{1.427963in}}%
\pgfpathlineto{\pgfqpoint{3.199378in}{1.429328in}}%
\pgfpathlineto{\pgfqpoint{3.210870in}{1.434092in}}%
\pgfpathlineto{\pgfqpoint{3.222361in}{1.435613in}}%
\pgfpathlineto{\pgfqpoint{3.228106in}{1.436509in}}%
\pgfpathlineto{\pgfqpoint{3.235767in}{1.437544in}}%
\pgfpathlineto{\pgfqpoint{3.243427in}{1.439200in}}%
\pgfpathlineto{\pgfqpoint{3.254918in}{1.440952in}}%
\pgfpathlineto{\pgfqpoint{3.258749in}{1.443303in}}%
\pgfpathlineto{\pgfqpoint{3.264494in}{1.443955in}}%
\pgfpathlineto{\pgfqpoint{3.272155in}{1.446672in}}%
\pgfpathlineto{\pgfqpoint{3.274070in}{1.450382in}}%
\pgfpathlineto{\pgfqpoint{3.281731in}{1.452012in}}%
\pgfpathlineto{\pgfqpoint{3.285561in}{1.454725in}}%
\pgfpathlineto{\pgfqpoint{3.291307in}{1.457727in}}%
\pgfpathlineto{\pgfqpoint{3.302798in}{1.465814in}}%
\pgfpathlineto{\pgfqpoint{3.306628in}{1.466608in}}%
\pgfpathlineto{\pgfqpoint{3.308543in}{1.472106in}}%
\pgfpathlineto{\pgfqpoint{3.314289in}{1.473385in}}%
\pgfpathlineto{\pgfqpoint{3.318119in}{1.474391in}}%
\pgfpathlineto{\pgfqpoint{3.325780in}{1.475585in}}%
\pgfpathlineto{\pgfqpoint{3.327695in}{1.478051in}}%
\pgfpathlineto{\pgfqpoint{3.331525in}{1.479295in}}%
\pgfpathlineto{\pgfqpoint{3.339186in}{1.482236in}}%
\pgfpathlineto{\pgfqpoint{3.348762in}{1.484366in}}%
\pgfpathlineto{\pgfqpoint{3.350677in}{1.487673in}}%
\pgfpathlineto{\pgfqpoint{3.360253in}{1.491142in}}%
\pgfpathlineto{\pgfqpoint{3.365998in}{1.492830in}}%
\pgfpathlineto{\pgfqpoint{3.369829in}{1.494089in}}%
\pgfpathlineto{\pgfqpoint{3.379405in}{1.498438in}}%
\pgfpathlineto{\pgfqpoint{3.381320in}{1.498494in}}%
\pgfpathlineto{\pgfqpoint{3.385150in}{1.501274in}}%
\pgfpathlineto{\pgfqpoint{3.392811in}{1.502598in}}%
\pgfpathlineto{\pgfqpoint{3.394726in}{1.504642in}}%
\pgfpathlineto{\pgfqpoint{3.398556in}{1.509882in}}%
\pgfpathlineto{\pgfqpoint{3.400471in}{1.511674in}}%
\pgfpathlineto{\pgfqpoint{3.402387in}{1.511765in}}%
\pgfpathlineto{\pgfqpoint{3.408132in}{1.517121in}}%
\pgfpathlineto{\pgfqpoint{3.411963in}{1.517711in}}%
\pgfpathlineto{\pgfqpoint{3.413878in}{1.520288in}}%
\pgfpathlineto{\pgfqpoint{3.417708in}{1.521646in}}%
\pgfpathlineto{\pgfqpoint{3.419623in}{1.524896in}}%
\pgfpathlineto{\pgfqpoint{3.423454in}{1.525369in}}%
\pgfpathlineto{\pgfqpoint{3.425369in}{1.527162in}}%
\pgfpathlineto{\pgfqpoint{3.429199in}{1.527621in}}%
\pgfpathlineto{\pgfqpoint{3.433029in}{1.529245in}}%
\pgfpathlineto{\pgfqpoint{3.438775in}{1.532005in}}%
\pgfpathlineto{\pgfqpoint{3.440690in}{1.535719in}}%
\pgfpathlineto{\pgfqpoint{3.442605in}{1.535907in}}%
\pgfpathlineto{\pgfqpoint{3.446436in}{1.538192in}}%
\pgfpathlineto{\pgfqpoint{3.450266in}{1.540365in}}%
\pgfpathlineto{\pgfqpoint{3.452181in}{1.540609in}}%
\pgfpathlineto{\pgfqpoint{3.454096in}{1.543023in}}%
\pgfpathlineto{\pgfqpoint{3.459842in}{1.543530in}}%
\pgfpathlineto{\pgfqpoint{3.465587in}{1.549604in}}%
\pgfpathlineto{\pgfqpoint{3.469418in}{1.551320in}}%
\pgfpathlineto{\pgfqpoint{3.471333in}{1.553315in}}%
\pgfpathlineto{\pgfqpoint{3.477078in}{1.554141in}}%
\pgfpathlineto{\pgfqpoint{3.484739in}{1.556700in}}%
\pgfpathlineto{\pgfqpoint{3.488569in}{1.560291in}}%
\pgfpathlineto{\pgfqpoint{3.490485in}{1.563860in}}%
\pgfpathlineto{\pgfqpoint{3.501976in}{1.569376in}}%
\pgfpathlineto{\pgfqpoint{3.523042in}{1.585979in}}%
\pgfpathlineto{\pgfqpoint{3.526873in}{1.590387in}}%
\pgfpathlineto{\pgfqpoint{3.528788in}{1.590711in}}%
\pgfpathlineto{\pgfqpoint{3.536449in}{1.596999in}}%
\pgfpathlineto{\pgfqpoint{3.538364in}{1.602483in}}%
\pgfpathlineto{\pgfqpoint{3.542194in}{1.602986in}}%
\pgfpathlineto{\pgfqpoint{3.544109in}{1.606695in}}%
\pgfpathlineto{\pgfqpoint{3.547940in}{1.609638in}}%
\pgfpathlineto{\pgfqpoint{3.551770in}{1.611775in}}%
\pgfpathlineto{\pgfqpoint{3.555600in}{1.619620in}}%
\pgfpathlineto{\pgfqpoint{3.557516in}{1.620170in}}%
\pgfpathlineto{\pgfqpoint{3.559431in}{1.625653in}}%
\pgfpathlineto{\pgfqpoint{3.561346in}{1.625971in}}%
\pgfpathlineto{\pgfqpoint{3.563261in}{1.629694in}}%
\pgfpathlineto{\pgfqpoint{3.565176in}{1.630493in}}%
\pgfpathlineto{\pgfqpoint{3.567091in}{1.649421in}}%
\pgfpathlineto{\pgfqpoint{3.569007in}{1.655817in}}%
\pgfpathlineto{\pgfqpoint{3.570922in}{1.655900in}}%
\pgfpathlineto{\pgfqpoint{3.572837in}{1.659339in}}%
\pgfpathlineto{\pgfqpoint{3.576667in}{1.668496in}}%
\pgfpathlineto{\pgfqpoint{3.578582in}{1.669925in}}%
\pgfpathlineto{\pgfqpoint{3.582413in}{1.675444in}}%
\pgfpathlineto{\pgfqpoint{3.588158in}{1.704219in}}%
\pgfpathlineto{\pgfqpoint{3.590073in}{1.709059in}}%
\pgfpathlineto{\pgfqpoint{3.591989in}{1.724221in}}%
\pgfpathlineto{\pgfqpoint{3.593904in}{1.727842in}}%
\pgfpathlineto{\pgfqpoint{3.601564in}{1.771853in}}%
\pgfpathlineto{\pgfqpoint{3.605395in}{1.780666in}}%
\pgfpathlineto{\pgfqpoint{3.613056in}{1.807385in}}%
\pgfpathlineto{\pgfqpoint{3.614971in}{1.809532in}}%
\pgfpathlineto{\pgfqpoint{3.616886in}{1.820569in}}%
\pgfpathlineto{\pgfqpoint{3.622631in}{1.834523in}}%
\pgfpathlineto{\pgfqpoint{3.628377in}{1.865335in}}%
\pgfpathlineto{\pgfqpoint{3.630292in}{1.875434in}}%
\pgfpathlineto{\pgfqpoint{3.632207in}{1.896558in}}%
\pgfpathlineto{\pgfqpoint{3.634122in}{1.897292in}}%
\pgfpathlineto{\pgfqpoint{3.636038in}{1.903412in}}%
\pgfpathlineto{\pgfqpoint{3.637953in}{1.920929in}}%
\pgfpathlineto{\pgfqpoint{3.643698in}{2.018381in}}%
\pgfpathlineto{\pgfqpoint{3.645613in}{2.019327in}}%
\pgfpathlineto{\pgfqpoint{3.647529in}{2.058714in}}%
\pgfpathlineto{\pgfqpoint{3.649444in}{2.067771in}}%
\pgfpathlineto{\pgfqpoint{3.651359in}{2.068041in}}%
\pgfpathlineto{\pgfqpoint{3.653274in}{2.077219in}}%
\pgfpathlineto{\pgfqpoint{3.655189in}{2.078564in}}%
\pgfpathlineto{\pgfqpoint{3.657104in}{2.087656in}}%
\pgfpathlineto{\pgfqpoint{3.660935in}{2.091692in}}%
\pgfpathlineto{\pgfqpoint{3.660935in}{2.091692in}}%
\pgfusepath{stroke}%
\end{pgfscope}%
\begin{pgfscope}%
\pgfpathrectangle{\pgfqpoint{0.694334in}{0.523557in}}{\pgfqpoint{3.830343in}{1.568135in}}%
\pgfusepath{clip}%
\pgfsetrectcap%
\pgfsetroundjoin%
\pgfsetlinewidth{1.003750pt}%
\definecolor{currentstroke}{rgb}{0.811765,0.125490,0.125490}%
\pgfsetstrokecolor{currentstroke}%
\pgfsetdash{}{0pt}%
\pgfpathmoveto{\pgfqpoint{0.694334in}{0.852349in}}%
\pgfpathlineto{\pgfqpoint{0.703910in}{0.855042in}}%
\pgfpathlineto{\pgfqpoint{0.749874in}{0.856124in}}%
\pgfpathlineto{\pgfqpoint{0.799669in}{0.856974in}}%
\pgfpathlineto{\pgfqpoint{0.874360in}{0.858078in}}%
\pgfpathlineto{\pgfqpoint{0.920325in}{0.859492in}}%
\pgfpathlineto{\pgfqpoint{0.950967in}{0.860055in}}%
\pgfpathlineto{\pgfqpoint{1.044811in}{0.861989in}}%
\pgfpathlineto{\pgfqpoint{1.117587in}{0.865097in}}%
\pgfpathlineto{\pgfqpoint{1.140569in}{0.865981in}}%
\pgfpathlineto{\pgfqpoint{1.163551in}{0.867221in}}%
\pgfpathlineto{\pgfqpoint{1.178873in}{0.867885in}}%
\pgfpathlineto{\pgfqpoint{1.255480in}{0.873881in}}%
\pgfpathlineto{\pgfqpoint{1.265055in}{0.874575in}}%
\pgfpathlineto{\pgfqpoint{1.295698in}{0.876672in}}%
\pgfpathlineto{\pgfqpoint{1.312935in}{0.878962in}}%
\pgfpathlineto{\pgfqpoint{1.328256in}{0.879552in}}%
\pgfpathlineto{\pgfqpoint{1.343577in}{0.880517in}}%
\pgfpathlineto{\pgfqpoint{1.349323in}{0.881398in}}%
\pgfpathlineto{\pgfqpoint{1.381881in}{0.883260in}}%
\pgfpathlineto{\pgfqpoint{1.397202in}{0.884444in}}%
\pgfpathlineto{\pgfqpoint{1.418269in}{0.886081in}}%
\pgfpathlineto{\pgfqpoint{1.422099in}{0.887259in}}%
\pgfpathlineto{\pgfqpoint{1.466148in}{0.889795in}}%
\pgfpathlineto{\pgfqpoint{1.471894in}{0.890713in}}%
\pgfpathlineto{\pgfqpoint{1.477639in}{0.891861in}}%
\pgfpathlineto{\pgfqpoint{1.504452in}{0.896193in}}%
\pgfpathlineto{\pgfqpoint{1.519773in}{0.897789in}}%
\pgfpathlineto{\pgfqpoint{1.527434in}{0.899029in}}%
\pgfpathlineto{\pgfqpoint{1.533179in}{0.900671in}}%
\pgfpathlineto{\pgfqpoint{1.542755in}{0.901834in}}%
\pgfpathlineto{\pgfqpoint{1.594465in}{0.908590in}}%
\pgfpathlineto{\pgfqpoint{1.598295in}{0.910610in}}%
\pgfpathlineto{\pgfqpoint{1.613617in}{0.914177in}}%
\pgfpathlineto{\pgfqpoint{1.617447in}{0.915741in}}%
\pgfpathlineto{\pgfqpoint{1.632768in}{0.916851in}}%
\pgfpathlineto{\pgfqpoint{1.636599in}{0.918834in}}%
\pgfpathlineto{\pgfqpoint{1.646174in}{0.920991in}}%
\pgfpathlineto{\pgfqpoint{1.651920in}{0.922355in}}%
\pgfpathlineto{\pgfqpoint{1.678732in}{0.928967in}}%
\pgfpathlineto{\pgfqpoint{1.684478in}{0.929874in}}%
\pgfpathlineto{\pgfqpoint{1.697884in}{0.933078in}}%
\pgfpathlineto{\pgfqpoint{1.701714in}{0.934129in}}%
\pgfpathlineto{\pgfqpoint{1.705545in}{0.935005in}}%
\pgfpathlineto{\pgfqpoint{1.707460in}{0.937229in}}%
\pgfpathlineto{\pgfqpoint{1.753424in}{0.947727in}}%
\pgfpathlineto{\pgfqpoint{1.755339in}{0.948850in}}%
\pgfpathlineto{\pgfqpoint{1.757254in}{0.951573in}}%
\pgfpathlineto{\pgfqpoint{1.772576in}{0.953535in}}%
\pgfpathlineto{\pgfqpoint{1.776406in}{0.953987in}}%
\pgfpathlineto{\pgfqpoint{1.780236in}{0.955292in}}%
\pgfpathlineto{\pgfqpoint{1.789812in}{0.957008in}}%
\pgfpathlineto{\pgfqpoint{1.818540in}{0.962397in}}%
\pgfpathlineto{\pgfqpoint{1.820455in}{0.965554in}}%
\pgfpathlineto{\pgfqpoint{1.822370in}{0.965591in}}%
\pgfpathlineto{\pgfqpoint{1.826201in}{0.966919in}}%
\pgfpathlineto{\pgfqpoint{1.831946in}{0.968329in}}%
\pgfpathlineto{\pgfqpoint{1.833861in}{0.968582in}}%
\pgfpathlineto{\pgfqpoint{1.837692in}{0.971453in}}%
\pgfpathlineto{\pgfqpoint{1.851098in}{0.973067in}}%
\pgfpathlineto{\pgfqpoint{1.858759in}{0.974431in}}%
\pgfpathlineto{\pgfqpoint{1.866419in}{0.975535in}}%
\pgfpathlineto{\pgfqpoint{1.874080in}{0.976747in}}%
\pgfpathlineto{\pgfqpoint{1.879825in}{0.977749in}}%
\pgfpathlineto{\pgfqpoint{1.891316in}{0.979454in}}%
\pgfpathlineto{\pgfqpoint{1.895147in}{0.983986in}}%
\pgfpathlineto{\pgfqpoint{1.900892in}{0.986648in}}%
\pgfpathlineto{\pgfqpoint{1.912383in}{0.987582in}}%
\pgfpathlineto{\pgfqpoint{1.916214in}{0.988778in}}%
\pgfpathlineto{\pgfqpoint{1.933450in}{0.990927in}}%
\pgfpathlineto{\pgfqpoint{1.939196in}{0.992140in}}%
\pgfpathlineto{\pgfqpoint{1.941111in}{0.994563in}}%
\pgfpathlineto{\pgfqpoint{1.944941in}{0.995196in}}%
\pgfpathlineto{\pgfqpoint{1.946856in}{0.998880in}}%
\pgfpathlineto{\pgfqpoint{1.958347in}{0.999950in}}%
\pgfpathlineto{\pgfqpoint{1.962178in}{1.001985in}}%
\pgfpathlineto{\pgfqpoint{1.973669in}{1.004402in}}%
\pgfpathlineto{\pgfqpoint{1.977499in}{1.005544in}}%
\pgfpathlineto{\pgfqpoint{1.979414in}{1.006110in}}%
\pgfpathlineto{\pgfqpoint{1.981329in}{1.010221in}}%
\pgfpathlineto{\pgfqpoint{1.992821in}{1.012706in}}%
\pgfpathlineto{\pgfqpoint{1.996651in}{1.014612in}}%
\pgfpathlineto{\pgfqpoint{2.021548in}{1.020679in}}%
\pgfpathlineto{\pgfqpoint{2.023463in}{1.024432in}}%
\pgfpathlineto{\pgfqpoint{2.029209in}{1.024985in}}%
\pgfpathlineto{\pgfqpoint{2.033039in}{1.028423in}}%
\pgfpathlineto{\pgfqpoint{2.034954in}{1.028547in}}%
\pgfpathlineto{\pgfqpoint{2.038785in}{1.031114in}}%
\pgfpathlineto{\pgfqpoint{2.046445in}{1.032471in}}%
\pgfpathlineto{\pgfqpoint{2.050276in}{1.034057in}}%
\pgfpathlineto{\pgfqpoint{2.056021in}{1.035494in}}%
\pgfpathlineto{\pgfqpoint{2.057936in}{1.035705in}}%
\pgfpathlineto{\pgfqpoint{2.059852in}{1.037574in}}%
\pgfpathlineto{\pgfqpoint{2.075173in}{1.039022in}}%
\pgfpathlineto{\pgfqpoint{2.080918in}{1.041102in}}%
\pgfpathlineto{\pgfqpoint{2.084749in}{1.043505in}}%
\pgfpathlineto{\pgfqpoint{2.090494in}{1.044837in}}%
\pgfpathlineto{\pgfqpoint{2.096240in}{1.045681in}}%
\pgfpathlineto{\pgfqpoint{2.098155in}{1.047849in}}%
\pgfpathlineto{\pgfqpoint{2.115391in}{1.051622in}}%
\pgfpathlineto{\pgfqpoint{2.121137in}{1.054217in}}%
\pgfpathlineto{\pgfqpoint{2.130713in}{1.056237in}}%
\pgfpathlineto{\pgfqpoint{2.136458in}{1.056973in}}%
\pgfpathlineto{\pgfqpoint{2.138374in}{1.059993in}}%
\pgfpathlineto{\pgfqpoint{2.144119in}{1.061128in}}%
\pgfpathlineto{\pgfqpoint{2.149865in}{1.064311in}}%
\pgfpathlineto{\pgfqpoint{2.151780in}{1.064499in}}%
\pgfpathlineto{\pgfqpoint{2.153695in}{1.066691in}}%
\pgfpathlineto{\pgfqpoint{2.157525in}{1.067829in}}%
\pgfpathlineto{\pgfqpoint{2.161356in}{1.070144in}}%
\pgfpathlineto{\pgfqpoint{2.165186in}{1.072337in}}%
\pgfpathlineto{\pgfqpoint{2.169016in}{1.075878in}}%
\pgfpathlineto{\pgfqpoint{2.178592in}{1.081068in}}%
\pgfpathlineto{\pgfqpoint{2.184338in}{1.081570in}}%
\pgfpathlineto{\pgfqpoint{2.188168in}{1.083911in}}%
\pgfpathlineto{\pgfqpoint{2.197744in}{1.085844in}}%
\pgfpathlineto{\pgfqpoint{2.209235in}{1.088785in}}%
\pgfpathlineto{\pgfqpoint{2.211150in}{1.091717in}}%
\pgfpathlineto{\pgfqpoint{2.216896in}{1.092314in}}%
\pgfpathlineto{\pgfqpoint{2.222641in}{1.094113in}}%
\pgfpathlineto{\pgfqpoint{2.230302in}{1.095907in}}%
\pgfpathlineto{\pgfqpoint{2.236047in}{1.098374in}}%
\pgfpathlineto{\pgfqpoint{2.247538in}{1.101207in}}%
\pgfpathlineto{\pgfqpoint{2.257114in}{1.108793in}}%
\pgfpathlineto{\pgfqpoint{2.260945in}{1.111857in}}%
\pgfpathlineto{\pgfqpoint{2.274351in}{1.115171in}}%
\pgfpathlineto{\pgfqpoint{2.276266in}{1.115186in}}%
\pgfpathlineto{\pgfqpoint{2.278181in}{1.117498in}}%
\pgfpathlineto{\pgfqpoint{2.280096in}{1.117546in}}%
\pgfpathlineto{\pgfqpoint{2.282011in}{1.119143in}}%
\pgfpathlineto{\pgfqpoint{2.291587in}{1.120515in}}%
\pgfpathlineto{\pgfqpoint{2.297333in}{1.121115in}}%
\pgfpathlineto{\pgfqpoint{2.301163in}{1.122098in}}%
\pgfpathlineto{\pgfqpoint{2.314569in}{1.129128in}}%
\pgfpathlineto{\pgfqpoint{2.316484in}{1.129308in}}%
\pgfpathlineto{\pgfqpoint{2.320315in}{1.132322in}}%
\pgfpathlineto{\pgfqpoint{2.322230in}{1.132384in}}%
\pgfpathlineto{\pgfqpoint{2.327975in}{1.135754in}}%
\pgfpathlineto{\pgfqpoint{2.343297in}{1.142091in}}%
\pgfpathlineto{\pgfqpoint{2.347127in}{1.145549in}}%
\pgfpathlineto{\pgfqpoint{2.352873in}{1.146785in}}%
\pgfpathlineto{\pgfqpoint{2.372024in}{1.151210in}}%
\pgfpathlineto{\pgfqpoint{2.373940in}{1.154968in}}%
\pgfpathlineto{\pgfqpoint{2.381600in}{1.156507in}}%
\pgfpathlineto{\pgfqpoint{2.385431in}{1.157756in}}%
\pgfpathlineto{\pgfqpoint{2.402667in}{1.161156in}}%
\pgfpathlineto{\pgfqpoint{2.404582in}{1.162957in}}%
\pgfpathlineto{\pgfqpoint{2.406498in}{1.163067in}}%
\pgfpathlineto{\pgfqpoint{2.410328in}{1.164855in}}%
\pgfpathlineto{\pgfqpoint{2.414158in}{1.165291in}}%
\pgfpathlineto{\pgfqpoint{2.421819in}{1.169629in}}%
\pgfpathlineto{\pgfqpoint{2.425649in}{1.171464in}}%
\pgfpathlineto{\pgfqpoint{2.429480in}{1.173072in}}%
\pgfpathlineto{\pgfqpoint{2.437140in}{1.174637in}}%
\pgfpathlineto{\pgfqpoint{2.452462in}{1.178062in}}%
\pgfpathlineto{\pgfqpoint{2.456292in}{1.179657in}}%
\pgfpathlineto{\pgfqpoint{2.458207in}{1.179814in}}%
\pgfpathlineto{\pgfqpoint{2.460122in}{1.184331in}}%
\pgfpathlineto{\pgfqpoint{2.463953in}{1.186599in}}%
\pgfpathlineto{\pgfqpoint{2.475444in}{1.189713in}}%
\pgfpathlineto{\pgfqpoint{2.477359in}{1.192823in}}%
\pgfpathlineto{\pgfqpoint{2.494595in}{1.196339in}}%
\pgfpathlineto{\pgfqpoint{2.496511in}{1.197828in}}%
\pgfpathlineto{\pgfqpoint{2.500341in}{1.202319in}}%
\pgfpathlineto{\pgfqpoint{2.502256in}{1.204049in}}%
\pgfpathlineto{\pgfqpoint{2.508002in}{1.204997in}}%
\pgfpathlineto{\pgfqpoint{2.517577in}{1.209535in}}%
\pgfpathlineto{\pgfqpoint{2.529068in}{1.211057in}}%
\pgfpathlineto{\pgfqpoint{2.563542in}{1.212830in}}%
\pgfpathlineto{\pgfqpoint{2.607591in}{1.217134in}}%
\pgfpathlineto{\pgfqpoint{2.613336in}{1.219333in}}%
\pgfpathlineto{\pgfqpoint{2.643979in}{1.226879in}}%
\pgfpathlineto{\pgfqpoint{2.651639in}{1.229296in}}%
\pgfpathlineto{\pgfqpoint{2.666961in}{1.234041in}}%
\pgfpathlineto{\pgfqpoint{2.670791in}{1.235787in}}%
\pgfpathlineto{\pgfqpoint{2.674622in}{1.236490in}}%
\pgfpathlineto{\pgfqpoint{2.680367in}{1.238009in}}%
\pgfpathlineto{\pgfqpoint{2.688028in}{1.239497in}}%
\pgfpathlineto{\pgfqpoint{2.701434in}{1.243584in}}%
\pgfpathlineto{\pgfqpoint{2.709095in}{1.250434in}}%
\pgfpathlineto{\pgfqpoint{2.716755in}{1.252138in}}%
\pgfpathlineto{\pgfqpoint{2.718670in}{1.252483in}}%
\pgfpathlineto{\pgfqpoint{2.724416in}{1.256255in}}%
\pgfpathlineto{\pgfqpoint{2.728246in}{1.257769in}}%
\pgfpathlineto{\pgfqpoint{2.741653in}{1.261519in}}%
\pgfpathlineto{\pgfqpoint{2.745483in}{1.263284in}}%
\pgfpathlineto{\pgfqpoint{2.753144in}{1.264521in}}%
\pgfpathlineto{\pgfqpoint{2.755059in}{1.269979in}}%
\pgfpathlineto{\pgfqpoint{2.758889in}{1.271776in}}%
\pgfpathlineto{\pgfqpoint{2.760804in}{1.272028in}}%
\pgfpathlineto{\pgfqpoint{2.762719in}{1.276856in}}%
\pgfpathlineto{\pgfqpoint{2.766550in}{1.277536in}}%
\pgfpathlineto{\pgfqpoint{2.772295in}{1.287348in}}%
\pgfpathlineto{\pgfqpoint{2.778041in}{1.288760in}}%
\pgfpathlineto{\pgfqpoint{2.781871in}{1.291073in}}%
\pgfpathlineto{\pgfqpoint{2.785701in}{1.291856in}}%
\pgfpathlineto{\pgfqpoint{2.789532in}{1.294488in}}%
\pgfpathlineto{\pgfqpoint{2.795277in}{1.294711in}}%
\pgfpathlineto{\pgfqpoint{2.801023in}{1.298041in}}%
\pgfpathlineto{\pgfqpoint{2.806768in}{1.299344in}}%
\pgfpathlineto{\pgfqpoint{2.822090in}{1.307706in}}%
\pgfpathlineto{\pgfqpoint{2.825920in}{1.308614in}}%
\pgfpathlineto{\pgfqpoint{2.829750in}{1.310028in}}%
\pgfpathlineto{\pgfqpoint{2.833581in}{1.311338in}}%
\pgfpathlineto{\pgfqpoint{2.835496in}{1.311853in}}%
\pgfpathlineto{\pgfqpoint{2.839326in}{1.314251in}}%
\pgfpathlineto{\pgfqpoint{2.845072in}{1.315881in}}%
\pgfpathlineto{\pgfqpoint{2.850817in}{1.317484in}}%
\pgfpathlineto{\pgfqpoint{2.852732in}{1.321144in}}%
\pgfpathlineto{\pgfqpoint{2.854648in}{1.321311in}}%
\pgfpathlineto{\pgfqpoint{2.860393in}{1.327199in}}%
\pgfpathlineto{\pgfqpoint{2.864223in}{1.328428in}}%
\pgfpathlineto{\pgfqpoint{2.866139in}{1.330116in}}%
\pgfpathlineto{\pgfqpoint{2.868054in}{1.330282in}}%
\pgfpathlineto{\pgfqpoint{2.871884in}{1.335801in}}%
\pgfpathlineto{\pgfqpoint{2.883375in}{1.339096in}}%
\pgfpathlineto{\pgfqpoint{2.887206in}{1.341725in}}%
\pgfpathlineto{\pgfqpoint{2.889121in}{1.342046in}}%
\pgfpathlineto{\pgfqpoint{2.891036in}{1.344308in}}%
\pgfpathlineto{\pgfqpoint{2.896781in}{1.345837in}}%
\pgfpathlineto{\pgfqpoint{2.898697in}{1.346946in}}%
\pgfpathlineto{\pgfqpoint{2.902527in}{1.351899in}}%
\pgfpathlineto{\pgfqpoint{2.904442in}{1.352163in}}%
\pgfpathlineto{\pgfqpoint{2.906357in}{1.354266in}}%
\pgfpathlineto{\pgfqpoint{2.914018in}{1.355615in}}%
\pgfpathlineto{\pgfqpoint{2.915933in}{1.358740in}}%
\pgfpathlineto{\pgfqpoint{2.929339in}{1.362898in}}%
\pgfpathlineto{\pgfqpoint{2.933170in}{1.369153in}}%
\pgfpathlineto{\pgfqpoint{2.958067in}{1.380004in}}%
\pgfpathlineto{\pgfqpoint{2.959982in}{1.385098in}}%
\pgfpathlineto{\pgfqpoint{2.967643in}{1.387238in}}%
\pgfpathlineto{\pgfqpoint{2.969558in}{1.390035in}}%
\pgfpathlineto{\pgfqpoint{2.979134in}{1.392066in}}%
\pgfpathlineto{\pgfqpoint{2.981049in}{1.394986in}}%
\pgfpathlineto{\pgfqpoint{2.996370in}{1.400057in}}%
\pgfpathlineto{\pgfqpoint{3.000201in}{1.402646in}}%
\pgfpathlineto{\pgfqpoint{3.007861in}{1.403961in}}%
\pgfpathlineto{\pgfqpoint{3.009777in}{1.405620in}}%
\pgfpathlineto{\pgfqpoint{3.015522in}{1.406539in}}%
\pgfpathlineto{\pgfqpoint{3.017437in}{1.409169in}}%
\pgfpathlineto{\pgfqpoint{3.038504in}{1.415708in}}%
\pgfpathlineto{\pgfqpoint{3.046165in}{1.416581in}}%
\pgfpathlineto{\pgfqpoint{3.049995in}{1.417393in}}%
\pgfpathlineto{\pgfqpoint{3.061486in}{1.422320in}}%
\pgfpathlineto{\pgfqpoint{3.063401in}{1.426002in}}%
\pgfpathlineto{\pgfqpoint{3.076808in}{1.429165in}}%
\pgfpathlineto{\pgfqpoint{3.078723in}{1.429867in}}%
\pgfpathlineto{\pgfqpoint{3.080638in}{1.432984in}}%
\pgfpathlineto{\pgfqpoint{3.111281in}{1.441755in}}%
\pgfpathlineto{\pgfqpoint{3.113196in}{1.444627in}}%
\pgfpathlineto{\pgfqpoint{3.124687in}{1.446774in}}%
\pgfpathlineto{\pgfqpoint{3.128517in}{1.448177in}}%
\pgfpathlineto{\pgfqpoint{3.136178in}{1.450210in}}%
\pgfpathlineto{\pgfqpoint{3.138093in}{1.453550in}}%
\pgfpathlineto{\pgfqpoint{3.149584in}{1.457237in}}%
\pgfpathlineto{\pgfqpoint{3.155330in}{1.458517in}}%
\pgfpathlineto{\pgfqpoint{3.164905in}{1.461533in}}%
\pgfpathlineto{\pgfqpoint{3.168736in}{1.464197in}}%
\pgfpathlineto{\pgfqpoint{3.182142in}{1.467821in}}%
\pgfpathlineto{\pgfqpoint{3.187887in}{1.474256in}}%
\pgfpathlineto{\pgfqpoint{3.193633in}{1.475642in}}%
\pgfpathlineto{\pgfqpoint{3.201294in}{1.477550in}}%
\pgfpathlineto{\pgfqpoint{3.210870in}{1.478659in}}%
\pgfpathlineto{\pgfqpoint{3.214700in}{1.481248in}}%
\pgfpathlineto{\pgfqpoint{3.239597in}{1.488572in}}%
\pgfpathlineto{\pgfqpoint{3.241512in}{1.491811in}}%
\pgfpathlineto{\pgfqpoint{3.247258in}{1.493303in}}%
\pgfpathlineto{\pgfqpoint{3.251088in}{1.494675in}}%
\pgfpathlineto{\pgfqpoint{3.254918in}{1.495900in}}%
\pgfpathlineto{\pgfqpoint{3.260664in}{1.498250in}}%
\pgfpathlineto{\pgfqpoint{3.264494in}{1.500588in}}%
\pgfpathlineto{\pgfqpoint{3.270240in}{1.502234in}}%
\pgfpathlineto{\pgfqpoint{3.275985in}{1.504192in}}%
\pgfpathlineto{\pgfqpoint{3.279816in}{1.505887in}}%
\pgfpathlineto{\pgfqpoint{3.283646in}{1.506769in}}%
\pgfpathlineto{\pgfqpoint{3.287476in}{1.509760in}}%
\pgfpathlineto{\pgfqpoint{3.291307in}{1.510316in}}%
\pgfpathlineto{\pgfqpoint{3.293222in}{1.512371in}}%
\pgfpathlineto{\pgfqpoint{3.295137in}{1.512465in}}%
\pgfpathlineto{\pgfqpoint{3.298967in}{1.514713in}}%
\pgfpathlineto{\pgfqpoint{3.308543in}{1.517428in}}%
\pgfpathlineto{\pgfqpoint{3.314289in}{1.520332in}}%
\pgfpathlineto{\pgfqpoint{3.318119in}{1.520730in}}%
\pgfpathlineto{\pgfqpoint{3.323865in}{1.523967in}}%
\pgfpathlineto{\pgfqpoint{3.329610in}{1.525604in}}%
\pgfpathlineto{\pgfqpoint{3.346847in}{1.533356in}}%
\pgfpathlineto{\pgfqpoint{3.348762in}{1.534508in}}%
\pgfpathlineto{\pgfqpoint{3.350677in}{1.537640in}}%
\pgfpathlineto{\pgfqpoint{3.354507in}{1.539088in}}%
\pgfpathlineto{\pgfqpoint{3.358338in}{1.540685in}}%
\pgfpathlineto{\pgfqpoint{3.360253in}{1.545176in}}%
\pgfpathlineto{\pgfqpoint{3.362168in}{1.545629in}}%
\pgfpathlineto{\pgfqpoint{3.364083in}{1.547975in}}%
\pgfpathlineto{\pgfqpoint{3.377489in}{1.549375in}}%
\pgfpathlineto{\pgfqpoint{3.381320in}{1.550247in}}%
\pgfpathlineto{\pgfqpoint{3.383235in}{1.552828in}}%
\pgfpathlineto{\pgfqpoint{3.387065in}{1.553767in}}%
\pgfpathlineto{\pgfqpoint{3.390896in}{1.555921in}}%
\pgfpathlineto{\pgfqpoint{3.394726in}{1.558288in}}%
\pgfpathlineto{\pgfqpoint{3.396641in}{1.563877in}}%
\pgfpathlineto{\pgfqpoint{3.404302in}{1.565068in}}%
\pgfpathlineto{\pgfqpoint{3.410047in}{1.569962in}}%
\pgfpathlineto{\pgfqpoint{3.413878in}{1.571183in}}%
\pgfpathlineto{\pgfqpoint{3.417708in}{1.573776in}}%
\pgfpathlineto{\pgfqpoint{3.419623in}{1.577201in}}%
\pgfpathlineto{\pgfqpoint{3.421538in}{1.577893in}}%
\pgfpathlineto{\pgfqpoint{3.423454in}{1.582739in}}%
\pgfpathlineto{\pgfqpoint{3.429199in}{1.585364in}}%
\pgfpathlineto{\pgfqpoint{3.431114in}{1.585464in}}%
\pgfpathlineto{\pgfqpoint{3.433029in}{1.587978in}}%
\pgfpathlineto{\pgfqpoint{3.434945in}{1.588268in}}%
\pgfpathlineto{\pgfqpoint{3.440690in}{1.593836in}}%
\pgfpathlineto{\pgfqpoint{3.452181in}{1.596493in}}%
\pgfpathlineto{\pgfqpoint{3.456011in}{1.600918in}}%
\pgfpathlineto{\pgfqpoint{3.459842in}{1.602159in}}%
\pgfpathlineto{\pgfqpoint{3.461757in}{1.605209in}}%
\pgfpathlineto{\pgfqpoint{3.467502in}{1.606550in}}%
\pgfpathlineto{\pgfqpoint{3.471333in}{1.606775in}}%
\pgfpathlineto{\pgfqpoint{3.473248in}{1.609801in}}%
\pgfpathlineto{\pgfqpoint{3.475163in}{1.616388in}}%
\pgfpathlineto{\pgfqpoint{3.477078in}{1.616620in}}%
\pgfpathlineto{\pgfqpoint{3.486654in}{1.622191in}}%
\pgfpathlineto{\pgfqpoint{3.492400in}{1.624317in}}%
\pgfpathlineto{\pgfqpoint{3.498145in}{1.626812in}}%
\pgfpathlineto{\pgfqpoint{3.501976in}{1.627555in}}%
\pgfpathlineto{\pgfqpoint{3.507721in}{1.634204in}}%
\pgfpathlineto{\pgfqpoint{3.509636in}{1.641088in}}%
\pgfpathlineto{\pgfqpoint{3.523042in}{1.643690in}}%
\pgfpathlineto{\pgfqpoint{3.528788in}{1.650132in}}%
\pgfpathlineto{\pgfqpoint{3.538364in}{1.658361in}}%
\pgfpathlineto{\pgfqpoint{3.540279in}{1.658833in}}%
\pgfpathlineto{\pgfqpoint{3.546025in}{1.669380in}}%
\pgfpathlineto{\pgfqpoint{3.551770in}{1.671484in}}%
\pgfpathlineto{\pgfqpoint{3.553685in}{1.676258in}}%
\pgfpathlineto{\pgfqpoint{3.559431in}{1.676810in}}%
\pgfpathlineto{\pgfqpoint{3.561346in}{1.678048in}}%
\pgfpathlineto{\pgfqpoint{3.563261in}{1.681243in}}%
\pgfpathlineto{\pgfqpoint{3.565176in}{1.681358in}}%
\pgfpathlineto{\pgfqpoint{3.570922in}{1.692173in}}%
\pgfpathlineto{\pgfqpoint{3.572837in}{1.692224in}}%
\pgfpathlineto{\pgfqpoint{3.580498in}{1.697515in}}%
\pgfpathlineto{\pgfqpoint{3.582413in}{1.697997in}}%
\pgfpathlineto{\pgfqpoint{3.584328in}{1.699877in}}%
\pgfpathlineto{\pgfqpoint{3.591989in}{1.714467in}}%
\pgfpathlineto{\pgfqpoint{3.595819in}{1.719771in}}%
\pgfpathlineto{\pgfqpoint{3.603480in}{1.724108in}}%
\pgfpathlineto{\pgfqpoint{3.609225in}{1.726157in}}%
\pgfpathlineto{\pgfqpoint{3.611140in}{1.734736in}}%
\pgfpathlineto{\pgfqpoint{3.613056in}{1.735465in}}%
\pgfpathlineto{\pgfqpoint{3.618801in}{1.739973in}}%
\pgfpathlineto{\pgfqpoint{3.620716in}{1.740188in}}%
\pgfpathlineto{\pgfqpoint{3.622631in}{1.747910in}}%
\pgfpathlineto{\pgfqpoint{3.630292in}{1.749510in}}%
\pgfpathlineto{\pgfqpoint{3.637953in}{1.761539in}}%
\pgfpathlineto{\pgfqpoint{3.639868in}{1.762220in}}%
\pgfpathlineto{\pgfqpoint{3.643698in}{1.765492in}}%
\pgfpathlineto{\pgfqpoint{3.645613in}{1.773296in}}%
\pgfpathlineto{\pgfqpoint{3.647529in}{1.773314in}}%
\pgfpathlineto{\pgfqpoint{3.649444in}{1.777557in}}%
\pgfpathlineto{\pgfqpoint{3.651359in}{1.778867in}}%
\pgfpathlineto{\pgfqpoint{3.659020in}{1.789563in}}%
\pgfpathlineto{\pgfqpoint{3.662850in}{1.793121in}}%
\pgfpathlineto{\pgfqpoint{3.664765in}{1.793393in}}%
\pgfpathlineto{\pgfqpoint{3.666680in}{1.801029in}}%
\pgfpathlineto{\pgfqpoint{3.670511in}{1.803543in}}%
\pgfpathlineto{\pgfqpoint{3.672426in}{1.808992in}}%
\pgfpathlineto{\pgfqpoint{3.674341in}{1.809175in}}%
\pgfpathlineto{\pgfqpoint{3.676256in}{1.814138in}}%
\pgfpathlineto{\pgfqpoint{3.678171in}{1.814372in}}%
\pgfpathlineto{\pgfqpoint{3.683917in}{1.828974in}}%
\pgfpathlineto{\pgfqpoint{3.689662in}{1.831936in}}%
\pgfpathlineto{\pgfqpoint{3.693493in}{1.839515in}}%
\pgfpathlineto{\pgfqpoint{3.695408in}{1.846810in}}%
\pgfpathlineto{\pgfqpoint{3.697323in}{1.847324in}}%
\pgfpathlineto{\pgfqpoint{3.699238in}{1.853080in}}%
\pgfpathlineto{\pgfqpoint{3.701153in}{1.853505in}}%
\pgfpathlineto{\pgfqpoint{3.706899in}{1.858266in}}%
\pgfpathlineto{\pgfqpoint{3.710729in}{1.867603in}}%
\pgfpathlineto{\pgfqpoint{3.712644in}{1.869083in}}%
\pgfpathlineto{\pgfqpoint{3.714560in}{1.874783in}}%
\pgfpathlineto{\pgfqpoint{3.718390in}{1.875391in}}%
\pgfpathlineto{\pgfqpoint{3.720305in}{1.877704in}}%
\pgfpathlineto{\pgfqpoint{3.724135in}{1.878712in}}%
\pgfpathlineto{\pgfqpoint{3.729881in}{1.886128in}}%
\pgfpathlineto{\pgfqpoint{3.731796in}{1.891568in}}%
\pgfpathlineto{\pgfqpoint{3.733711in}{1.891871in}}%
\pgfpathlineto{\pgfqpoint{3.735626in}{1.895855in}}%
\pgfpathlineto{\pgfqpoint{3.737542in}{1.896722in}}%
\pgfpathlineto{\pgfqpoint{3.739457in}{1.902976in}}%
\pgfpathlineto{\pgfqpoint{3.745202in}{1.908053in}}%
\pgfpathlineto{\pgfqpoint{3.749033in}{1.924246in}}%
\pgfpathlineto{\pgfqpoint{3.750948in}{1.927099in}}%
\pgfpathlineto{\pgfqpoint{3.752863in}{1.939040in}}%
\pgfpathlineto{\pgfqpoint{3.756693in}{1.944896in}}%
\pgfpathlineto{\pgfqpoint{3.760524in}{1.966192in}}%
\pgfpathlineto{\pgfqpoint{3.762439in}{1.974044in}}%
\pgfpathlineto{\pgfqpoint{3.764354in}{1.974655in}}%
\pgfpathlineto{\pgfqpoint{3.768184in}{1.981203in}}%
\pgfpathlineto{\pgfqpoint{3.770100in}{1.989570in}}%
\pgfpathlineto{\pgfqpoint{3.773930in}{1.993157in}}%
\pgfpathlineto{\pgfqpoint{3.777760in}{1.995118in}}%
\pgfpathlineto{\pgfqpoint{3.781591in}{1.996037in}}%
\pgfpathlineto{\pgfqpoint{3.785421in}{2.001315in}}%
\pgfpathlineto{\pgfqpoint{3.787336in}{2.001950in}}%
\pgfpathlineto{\pgfqpoint{3.789251in}{2.005507in}}%
\pgfpathlineto{\pgfqpoint{3.793082in}{2.008107in}}%
\pgfpathlineto{\pgfqpoint{3.794997in}{2.009022in}}%
\pgfpathlineto{\pgfqpoint{3.796912in}{2.015193in}}%
\pgfpathlineto{\pgfqpoint{3.798827in}{2.030262in}}%
\pgfpathlineto{\pgfqpoint{3.804573in}{2.034796in}}%
\pgfpathlineto{\pgfqpoint{3.806488in}{2.044603in}}%
\pgfpathlineto{\pgfqpoint{3.810318in}{2.046328in}}%
\pgfpathlineto{\pgfqpoint{3.814148in}{2.047122in}}%
\pgfpathlineto{\pgfqpoint{3.817979in}{2.054989in}}%
\pgfpathlineto{\pgfqpoint{3.821809in}{2.056805in}}%
\pgfpathlineto{\pgfqpoint{3.827555in}{2.076613in}}%
\pgfpathlineto{\pgfqpoint{3.829470in}{2.076686in}}%
\pgfpathlineto{\pgfqpoint{3.835215in}{2.084581in}}%
\pgfpathlineto{\pgfqpoint{3.839046in}{2.089583in}}%
\pgfpathlineto{\pgfqpoint{3.840961in}{2.091692in}}%
\pgfpathlineto{\pgfqpoint{3.840961in}{2.091692in}}%
\pgfusepath{stroke}%
\end{pgfscope}%
\begin{pgfscope}%
\pgfpathrectangle{\pgfqpoint{0.694334in}{0.523557in}}{\pgfqpoint{3.830343in}{1.568135in}}%
\pgfusepath{clip}%
\pgfsetbuttcap%
\pgfsetroundjoin%
\pgfsetlinewidth{1.003750pt}%
\definecolor{currentstroke}{rgb}{0.811765,0.125490,0.125490}%
\pgfsetstrokecolor{currentstroke}%
\pgfsetdash{{3.700000pt}{1.600000pt}}{0.000000pt}%
\pgfpathmoveto{\pgfqpoint{0.712063in}{0.513557in}}%
\pgfpathlineto{\pgfqpoint{0.713486in}{0.520568in}}%
\pgfpathlineto{\pgfqpoint{0.717316in}{0.520699in}}%
\pgfpathlineto{\pgfqpoint{0.719232in}{0.524044in}}%
\pgfpathlineto{\pgfqpoint{0.726892in}{0.525571in}}%
\pgfpathlineto{\pgfqpoint{0.728807in}{0.529433in}}%
\pgfpathlineto{\pgfqpoint{0.734553in}{0.532033in}}%
\pgfpathlineto{\pgfqpoint{0.742214in}{0.535029in}}%
\pgfpathlineto{\pgfqpoint{0.744129in}{0.542691in}}%
\pgfpathlineto{\pgfqpoint{0.747959in}{0.544799in}}%
\pgfpathlineto{\pgfqpoint{0.759450in}{0.548110in}}%
\pgfpathlineto{\pgfqpoint{0.763280in}{0.551545in}}%
\pgfpathlineto{\pgfqpoint{0.782432in}{0.557283in}}%
\pgfpathlineto{\pgfqpoint{0.795838in}{0.559011in}}%
\pgfpathlineto{\pgfqpoint{0.807329in}{0.561922in}}%
\pgfpathlineto{\pgfqpoint{0.811160in}{0.563645in}}%
\pgfpathlineto{\pgfqpoint{0.816905in}{0.565382in}}%
\pgfpathlineto{\pgfqpoint{0.834142in}{0.572738in}}%
\pgfpathlineto{\pgfqpoint{0.843718in}{0.573837in}}%
\pgfpathlineto{\pgfqpoint{0.847548in}{0.575459in}}%
\pgfpathlineto{\pgfqpoint{0.860954in}{0.577096in}}%
\pgfpathlineto{\pgfqpoint{0.864785in}{0.579079in}}%
\pgfpathlineto{\pgfqpoint{0.870530in}{0.580274in}}%
\pgfpathlineto{\pgfqpoint{0.872445in}{0.583586in}}%
\pgfpathlineto{\pgfqpoint{0.876276in}{0.584324in}}%
\pgfpathlineto{\pgfqpoint{0.880106in}{0.585456in}}%
\pgfpathlineto{\pgfqpoint{0.885851in}{0.586452in}}%
\pgfpathlineto{\pgfqpoint{0.889682in}{0.587737in}}%
\pgfpathlineto{\pgfqpoint{0.891597in}{0.587815in}}%
\pgfpathlineto{\pgfqpoint{0.895427in}{0.590315in}}%
\pgfpathlineto{\pgfqpoint{0.906918in}{0.591713in}}%
\pgfpathlineto{\pgfqpoint{0.910749in}{0.595496in}}%
\pgfpathlineto{\pgfqpoint{0.916494in}{0.596461in}}%
\pgfpathlineto{\pgfqpoint{0.927985in}{0.599228in}}%
\pgfpathlineto{\pgfqpoint{0.931816in}{0.600646in}}%
\pgfpathlineto{\pgfqpoint{0.941391in}{0.601741in}}%
\pgfpathlineto{\pgfqpoint{0.950967in}{0.605231in}}%
\pgfpathlineto{\pgfqpoint{0.960543in}{0.607202in}}%
\pgfpathlineto{\pgfqpoint{0.970119in}{0.609526in}}%
\pgfpathlineto{\pgfqpoint{0.972034in}{0.611484in}}%
\pgfpathlineto{\pgfqpoint{0.977780in}{0.612376in}}%
\pgfpathlineto{\pgfqpoint{0.983525in}{0.614520in}}%
\pgfpathlineto{\pgfqpoint{0.987356in}{0.617225in}}%
\pgfpathlineto{\pgfqpoint{0.991186in}{0.618373in}}%
\pgfpathlineto{\pgfqpoint{0.995016in}{0.620057in}}%
\pgfpathlineto{\pgfqpoint{1.008422in}{0.622993in}}%
\pgfpathlineto{\pgfqpoint{1.012253in}{0.625260in}}%
\pgfpathlineto{\pgfqpoint{1.017998in}{0.626153in}}%
\pgfpathlineto{\pgfqpoint{1.023744in}{0.628172in}}%
\pgfpathlineto{\pgfqpoint{1.027574in}{0.629452in}}%
\pgfpathlineto{\pgfqpoint{1.040980in}{0.630975in}}%
\pgfpathlineto{\pgfqpoint{1.044811in}{0.632150in}}%
\pgfpathlineto{\pgfqpoint{1.056302in}{0.634110in}}%
\pgfpathlineto{\pgfqpoint{1.067793in}{0.638354in}}%
\pgfpathlineto{\pgfqpoint{1.079284in}{0.639848in}}%
\pgfpathlineto{\pgfqpoint{1.086944in}{0.643546in}}%
\pgfpathlineto{\pgfqpoint{1.094605in}{0.645162in}}%
\pgfpathlineto{\pgfqpoint{1.117587in}{0.647160in}}%
\pgfpathlineto{\pgfqpoint{1.130993in}{0.654691in}}%
\pgfpathlineto{\pgfqpoint{1.138654in}{0.656922in}}%
\pgfpathlineto{\pgfqpoint{1.140569in}{0.658450in}}%
\pgfpathlineto{\pgfqpoint{1.144400in}{0.659355in}}%
\pgfpathlineto{\pgfqpoint{1.146315in}{0.662626in}}%
\pgfpathlineto{\pgfqpoint{1.153975in}{0.664191in}}%
\pgfpathlineto{\pgfqpoint{1.159721in}{0.667275in}}%
\pgfpathlineto{\pgfqpoint{1.161636in}{0.667443in}}%
\pgfpathlineto{\pgfqpoint{1.165466in}{0.669059in}}%
\pgfpathlineto{\pgfqpoint{1.169297in}{0.669860in}}%
\pgfpathlineto{\pgfqpoint{1.171212in}{0.672877in}}%
\pgfpathlineto{\pgfqpoint{1.173127in}{0.673045in}}%
\pgfpathlineto{\pgfqpoint{1.176957in}{0.677765in}}%
\pgfpathlineto{\pgfqpoint{1.178873in}{0.677788in}}%
\pgfpathlineto{\pgfqpoint{1.180788in}{0.680551in}}%
\pgfpathlineto{\pgfqpoint{1.190364in}{0.682079in}}%
\pgfpathlineto{\pgfqpoint{1.211431in}{0.693673in}}%
\pgfpathlineto{\pgfqpoint{1.213346in}{0.696947in}}%
\pgfpathlineto{\pgfqpoint{1.217176in}{0.698127in}}%
\pgfpathlineto{\pgfqpoint{1.226752in}{0.702464in}}%
\pgfpathlineto{\pgfqpoint{1.228667in}{0.705354in}}%
\pgfpathlineto{\pgfqpoint{1.232497in}{0.706566in}}%
\pgfpathlineto{\pgfqpoint{1.236328in}{0.712297in}}%
\pgfpathlineto{\pgfqpoint{1.240158in}{0.713428in}}%
\pgfpathlineto{\pgfqpoint{1.257395in}{0.720458in}}%
\pgfpathlineto{\pgfqpoint{1.261225in}{0.724732in}}%
\pgfpathlineto{\pgfqpoint{1.268886in}{0.727551in}}%
\pgfpathlineto{\pgfqpoint{1.272716in}{0.728773in}}%
\pgfpathlineto{\pgfqpoint{1.276546in}{0.731159in}}%
\pgfpathlineto{\pgfqpoint{1.278462in}{0.731394in}}%
\pgfpathlineto{\pgfqpoint{1.280377in}{0.733482in}}%
\pgfpathlineto{\pgfqpoint{1.282292in}{0.733877in}}%
\pgfpathlineto{\pgfqpoint{1.284207in}{0.738511in}}%
\pgfpathlineto{\pgfqpoint{1.291868in}{0.741108in}}%
\pgfpathlineto{\pgfqpoint{1.295698in}{0.743458in}}%
\pgfpathlineto{\pgfqpoint{1.303359in}{0.744332in}}%
\pgfpathlineto{\pgfqpoint{1.305274in}{0.747882in}}%
\pgfpathlineto{\pgfqpoint{1.311019in}{0.748347in}}%
\pgfpathlineto{\pgfqpoint{1.314850in}{0.750416in}}%
\pgfpathlineto{\pgfqpoint{1.322511in}{0.753748in}}%
\pgfpathlineto{\pgfqpoint{1.324426in}{0.756644in}}%
\pgfpathlineto{\pgfqpoint{1.343577in}{0.761084in}}%
\pgfpathlineto{\pgfqpoint{1.355068in}{0.767905in}}%
\pgfpathlineto{\pgfqpoint{1.368475in}{0.771658in}}%
\pgfpathlineto{\pgfqpoint{1.370390in}{0.771785in}}%
\pgfpathlineto{\pgfqpoint{1.372305in}{0.775012in}}%
\pgfpathlineto{\pgfqpoint{1.379966in}{0.776856in}}%
\pgfpathlineto{\pgfqpoint{1.383796in}{0.778424in}}%
\pgfpathlineto{\pgfqpoint{1.385711in}{0.778470in}}%
\pgfpathlineto{\pgfqpoint{1.399117in}{0.787983in}}%
\pgfpathlineto{\pgfqpoint{1.402948in}{0.788713in}}%
\pgfpathlineto{\pgfqpoint{1.404863in}{0.793266in}}%
\pgfpathlineto{\pgfqpoint{1.414439in}{0.798341in}}%
\pgfpathlineto{\pgfqpoint{1.416354in}{0.798510in}}%
\pgfpathlineto{\pgfqpoint{1.422099in}{0.802207in}}%
\pgfpathlineto{\pgfqpoint{1.431675in}{0.803821in}}%
\pgfpathlineto{\pgfqpoint{1.433590in}{0.809278in}}%
\pgfpathlineto{\pgfqpoint{1.437421in}{0.811641in}}%
\pgfpathlineto{\pgfqpoint{1.445081in}{0.813157in}}%
\pgfpathlineto{\pgfqpoint{1.448912in}{0.817657in}}%
\pgfpathlineto{\pgfqpoint{1.450827in}{0.817680in}}%
\pgfpathlineto{\pgfqpoint{1.452742in}{0.821469in}}%
\pgfpathlineto{\pgfqpoint{1.454657in}{0.822789in}}%
\pgfpathlineto{\pgfqpoint{1.460403in}{0.830193in}}%
\pgfpathlineto{\pgfqpoint{1.462318in}{0.830238in}}%
\pgfpathlineto{\pgfqpoint{1.464233in}{0.833808in}}%
\pgfpathlineto{\pgfqpoint{1.466148in}{0.834003in}}%
\pgfpathlineto{\pgfqpoint{1.469979in}{0.835826in}}%
\pgfpathlineto{\pgfqpoint{1.471894in}{0.837008in}}%
\pgfpathlineto{\pgfqpoint{1.473809in}{0.839808in}}%
\pgfpathlineto{\pgfqpoint{1.477639in}{0.840845in}}%
\pgfpathlineto{\pgfqpoint{1.481470in}{0.844207in}}%
\pgfpathlineto{\pgfqpoint{1.483385in}{0.844780in}}%
\pgfpathlineto{\pgfqpoint{1.487215in}{0.848407in}}%
\pgfpathlineto{\pgfqpoint{1.491046in}{0.848998in}}%
\pgfpathlineto{\pgfqpoint{1.492961in}{0.852015in}}%
\pgfpathlineto{\pgfqpoint{1.496791in}{0.853207in}}%
\pgfpathlineto{\pgfqpoint{1.506367in}{0.859123in}}%
\pgfpathlineto{\pgfqpoint{1.508282in}{0.859435in}}%
\pgfpathlineto{\pgfqpoint{1.510197in}{0.862272in}}%
\pgfpathlineto{\pgfqpoint{1.515943in}{0.864429in}}%
\pgfpathlineto{\pgfqpoint{1.519773in}{0.866852in}}%
\pgfpathlineto{\pgfqpoint{1.521688in}{0.872036in}}%
\pgfpathlineto{\pgfqpoint{1.527434in}{0.874488in}}%
\pgfpathlineto{\pgfqpoint{1.542755in}{0.878085in}}%
\pgfpathlineto{\pgfqpoint{1.546586in}{0.880758in}}%
\pgfpathlineto{\pgfqpoint{1.550416in}{0.881341in}}%
\pgfpathlineto{\pgfqpoint{1.554246in}{0.885890in}}%
\pgfpathlineto{\pgfqpoint{1.558077in}{0.887229in}}%
\pgfpathlineto{\pgfqpoint{1.565737in}{0.889975in}}%
\pgfpathlineto{\pgfqpoint{1.569568in}{0.893728in}}%
\pgfpathlineto{\pgfqpoint{1.581059in}{0.896275in}}%
\pgfpathlineto{\pgfqpoint{1.592550in}{0.902608in}}%
\pgfpathlineto{\pgfqpoint{1.600210in}{0.904116in}}%
\pgfpathlineto{\pgfqpoint{1.602126in}{0.905848in}}%
\pgfpathlineto{\pgfqpoint{1.604041in}{0.906075in}}%
\pgfpathlineto{\pgfqpoint{1.609786in}{0.910744in}}%
\pgfpathlineto{\pgfqpoint{1.617447in}{0.913803in}}%
\pgfpathlineto{\pgfqpoint{1.623192in}{0.915996in}}%
\pgfpathlineto{\pgfqpoint{1.628938in}{0.919017in}}%
\pgfpathlineto{\pgfqpoint{1.634683in}{0.920291in}}%
\pgfpathlineto{\pgfqpoint{1.636599in}{0.923897in}}%
\pgfpathlineto{\pgfqpoint{1.659581in}{0.937068in}}%
\pgfpathlineto{\pgfqpoint{1.665326in}{0.941921in}}%
\pgfpathlineto{\pgfqpoint{1.669157in}{0.942515in}}%
\pgfpathlineto{\pgfqpoint{1.671072in}{0.947424in}}%
\pgfpathlineto{\pgfqpoint{1.676817in}{0.951208in}}%
\pgfpathlineto{\pgfqpoint{1.680648in}{0.956942in}}%
\pgfpathlineto{\pgfqpoint{1.690223in}{0.960524in}}%
\pgfpathlineto{\pgfqpoint{1.695969in}{0.962146in}}%
\pgfpathlineto{\pgfqpoint{1.697884in}{0.962209in}}%
\pgfpathlineto{\pgfqpoint{1.699799in}{0.963819in}}%
\pgfpathlineto{\pgfqpoint{1.707460in}{0.964750in}}%
\pgfpathlineto{\pgfqpoint{1.713205in}{0.967539in}}%
\pgfpathlineto{\pgfqpoint{1.717036in}{0.967753in}}%
\pgfpathlineto{\pgfqpoint{1.726612in}{0.973087in}}%
\pgfpathlineto{\pgfqpoint{1.728527in}{0.973603in}}%
\pgfpathlineto{\pgfqpoint{1.730442in}{0.979523in}}%
\pgfpathlineto{\pgfqpoint{1.740018in}{0.981443in}}%
\pgfpathlineto{\pgfqpoint{1.745763in}{0.982448in}}%
\pgfpathlineto{\pgfqpoint{1.747679in}{0.985223in}}%
\pgfpathlineto{\pgfqpoint{1.755339in}{0.986228in}}%
\pgfpathlineto{\pgfqpoint{1.759170in}{0.987178in}}%
\pgfpathlineto{\pgfqpoint{1.763000in}{0.987970in}}%
\pgfpathlineto{\pgfqpoint{1.784067in}{0.997413in}}%
\pgfpathlineto{\pgfqpoint{1.787897in}{0.998626in}}%
\pgfpathlineto{\pgfqpoint{1.789812in}{1.003003in}}%
\pgfpathlineto{\pgfqpoint{1.793643in}{1.004163in}}%
\pgfpathlineto{\pgfqpoint{1.795558in}{1.004408in}}%
\pgfpathlineto{\pgfqpoint{1.797473in}{1.006398in}}%
\pgfpathlineto{\pgfqpoint{1.805134in}{1.007781in}}%
\pgfpathlineto{\pgfqpoint{1.816625in}{1.012117in}}%
\pgfpathlineto{\pgfqpoint{1.820455in}{1.015759in}}%
\pgfpathlineto{\pgfqpoint{1.826201in}{1.018258in}}%
\pgfpathlineto{\pgfqpoint{1.835776in}{1.024594in}}%
\pgfpathlineto{\pgfqpoint{1.841522in}{1.028173in}}%
\pgfpathlineto{\pgfqpoint{1.853013in}{1.030965in}}%
\pgfpathlineto{\pgfqpoint{1.854928in}{1.030970in}}%
\pgfpathlineto{\pgfqpoint{1.856843in}{1.033990in}}%
\pgfpathlineto{\pgfqpoint{1.868334in}{1.036237in}}%
\pgfpathlineto{\pgfqpoint{1.870250in}{1.036449in}}%
\pgfpathlineto{\pgfqpoint{1.872165in}{1.039740in}}%
\pgfpathlineto{\pgfqpoint{1.885571in}{1.044124in}}%
\pgfpathlineto{\pgfqpoint{1.891316in}{1.045859in}}%
\pgfpathlineto{\pgfqpoint{1.898977in}{1.047589in}}%
\pgfpathlineto{\pgfqpoint{1.902807in}{1.049539in}}%
\pgfpathlineto{\pgfqpoint{1.910468in}{1.051378in}}%
\pgfpathlineto{\pgfqpoint{1.914298in}{1.053959in}}%
\pgfpathlineto{\pgfqpoint{1.921959in}{1.055995in}}%
\pgfpathlineto{\pgfqpoint{1.923874in}{1.057627in}}%
\pgfpathlineto{\pgfqpoint{1.925790in}{1.061608in}}%
\pgfpathlineto{\pgfqpoint{1.929620in}{1.062647in}}%
\pgfpathlineto{\pgfqpoint{1.935365in}{1.063541in}}%
\pgfpathlineto{\pgfqpoint{1.941111in}{1.064372in}}%
\pgfpathlineto{\pgfqpoint{1.946856in}{1.070195in}}%
\pgfpathlineto{\pgfqpoint{1.954517in}{1.074697in}}%
\pgfpathlineto{\pgfqpoint{1.958347in}{1.075039in}}%
\pgfpathlineto{\pgfqpoint{1.960263in}{1.081994in}}%
\pgfpathlineto{\pgfqpoint{1.964093in}{1.083696in}}%
\pgfpathlineto{\pgfqpoint{1.969838in}{1.086087in}}%
\pgfpathlineto{\pgfqpoint{1.973669in}{1.087620in}}%
\pgfpathlineto{\pgfqpoint{1.977499in}{1.090229in}}%
\pgfpathlineto{\pgfqpoint{1.983245in}{1.090756in}}%
\pgfpathlineto{\pgfqpoint{1.987075in}{1.092175in}}%
\pgfpathlineto{\pgfqpoint{1.992821in}{1.093200in}}%
\pgfpathlineto{\pgfqpoint{1.994736in}{1.093444in}}%
\pgfpathlineto{\pgfqpoint{1.996651in}{1.096195in}}%
\pgfpathlineto{\pgfqpoint{2.002396in}{1.097011in}}%
\pgfpathlineto{\pgfqpoint{2.004312in}{1.099350in}}%
\pgfpathlineto{\pgfqpoint{2.011972in}{1.101554in}}%
\pgfpathlineto{\pgfqpoint{2.013887in}{1.104373in}}%
\pgfpathlineto{\pgfqpoint{2.021548in}{1.107283in}}%
\pgfpathlineto{\pgfqpoint{2.023463in}{1.112472in}}%
\pgfpathlineto{\pgfqpoint{2.027294in}{1.114701in}}%
\pgfpathlineto{\pgfqpoint{2.033039in}{1.121578in}}%
\pgfpathlineto{\pgfqpoint{2.038785in}{1.122270in}}%
\pgfpathlineto{\pgfqpoint{2.042615in}{1.127774in}}%
\pgfpathlineto{\pgfqpoint{2.050276in}{1.128813in}}%
\pgfpathlineto{\pgfqpoint{2.054106in}{1.130756in}}%
\pgfpathlineto{\pgfqpoint{2.061767in}{1.133205in}}%
\pgfpathlineto{\pgfqpoint{2.063682in}{1.137393in}}%
\pgfpathlineto{\pgfqpoint{2.073258in}{1.140648in}}%
\pgfpathlineto{\pgfqpoint{2.075173in}{1.140650in}}%
\pgfpathlineto{\pgfqpoint{2.077088in}{1.143573in}}%
\pgfpathlineto{\pgfqpoint{2.080918in}{1.144218in}}%
\pgfpathlineto{\pgfqpoint{2.084749in}{1.146629in}}%
\pgfpathlineto{\pgfqpoint{2.086664in}{1.149410in}}%
\pgfpathlineto{\pgfqpoint{2.090494in}{1.151697in}}%
\pgfpathlineto{\pgfqpoint{2.098155in}{1.154446in}}%
\pgfpathlineto{\pgfqpoint{2.101985in}{1.158949in}}%
\pgfpathlineto{\pgfqpoint{2.105816in}{1.159336in}}%
\pgfpathlineto{\pgfqpoint{2.109646in}{1.162566in}}%
\pgfpathlineto{\pgfqpoint{2.121137in}{1.164850in}}%
\pgfpathlineto{\pgfqpoint{2.124967in}{1.167048in}}%
\pgfpathlineto{\pgfqpoint{2.126883in}{1.167411in}}%
\pgfpathlineto{\pgfqpoint{2.128798in}{1.168968in}}%
\pgfpathlineto{\pgfqpoint{2.130713in}{1.172876in}}%
\pgfpathlineto{\pgfqpoint{2.136458in}{1.177892in}}%
\pgfpathlineto{\pgfqpoint{2.138374in}{1.179312in}}%
\pgfpathlineto{\pgfqpoint{2.142204in}{1.185223in}}%
\pgfpathlineto{\pgfqpoint{2.147949in}{1.186629in}}%
\pgfpathlineto{\pgfqpoint{2.153695in}{1.187542in}}%
\pgfpathlineto{\pgfqpoint{2.163271in}{1.188476in}}%
\pgfpathlineto{\pgfqpoint{2.165186in}{1.189432in}}%
\pgfpathlineto{\pgfqpoint{2.169016in}{1.195379in}}%
\pgfpathlineto{\pgfqpoint{2.178592in}{1.199429in}}%
\pgfpathlineto{\pgfqpoint{2.180507in}{1.203372in}}%
\pgfpathlineto{\pgfqpoint{2.197744in}{1.209968in}}%
\pgfpathlineto{\pgfqpoint{2.205405in}{1.211014in}}%
\pgfpathlineto{\pgfqpoint{2.245623in}{1.212981in}}%
\pgfpathlineto{\pgfqpoint{2.260945in}{1.214163in}}%
\pgfpathlineto{\pgfqpoint{2.285842in}{1.217137in}}%
\pgfpathlineto{\pgfqpoint{2.289672in}{1.219453in}}%
\pgfpathlineto{\pgfqpoint{2.301163in}{1.221936in}}%
\pgfpathlineto{\pgfqpoint{2.304993in}{1.223467in}}%
\pgfpathlineto{\pgfqpoint{2.320315in}{1.227571in}}%
\pgfpathlineto{\pgfqpoint{2.329891in}{1.228806in}}%
\pgfpathlineto{\pgfqpoint{2.333721in}{1.229869in}}%
\pgfpathlineto{\pgfqpoint{2.337551in}{1.229930in}}%
\pgfpathlineto{\pgfqpoint{2.339467in}{1.233064in}}%
\pgfpathlineto{\pgfqpoint{2.349042in}{1.235013in}}%
\pgfpathlineto{\pgfqpoint{2.350958in}{1.237663in}}%
\pgfpathlineto{\pgfqpoint{2.354788in}{1.238444in}}%
\pgfpathlineto{\pgfqpoint{2.356703in}{1.240455in}}%
\pgfpathlineto{\pgfqpoint{2.358618in}{1.240650in}}%
\pgfpathlineto{\pgfqpoint{2.360533in}{1.243450in}}%
\pgfpathlineto{\pgfqpoint{2.364364in}{1.243859in}}%
\pgfpathlineto{\pgfqpoint{2.366279in}{1.247962in}}%
\pgfpathlineto{\pgfqpoint{2.381600in}{1.256852in}}%
\pgfpathlineto{\pgfqpoint{2.404582in}{1.267614in}}%
\pgfpathlineto{\pgfqpoint{2.406498in}{1.269583in}}%
\pgfpathlineto{\pgfqpoint{2.408413in}{1.269710in}}%
\pgfpathlineto{\pgfqpoint{2.416073in}{1.283963in}}%
\pgfpathlineto{\pgfqpoint{2.421819in}{1.286479in}}%
\pgfpathlineto{\pgfqpoint{2.425649in}{1.291702in}}%
\pgfpathlineto{\pgfqpoint{2.429480in}{1.295510in}}%
\pgfpathlineto{\pgfqpoint{2.433310in}{1.296014in}}%
\pgfpathlineto{\pgfqpoint{2.435225in}{1.298643in}}%
\pgfpathlineto{\pgfqpoint{2.439055in}{1.299431in}}%
\pgfpathlineto{\pgfqpoint{2.444801in}{1.299783in}}%
\pgfpathlineto{\pgfqpoint{2.446716in}{1.300873in}}%
\pgfpathlineto{\pgfqpoint{2.448631in}{1.304109in}}%
\pgfpathlineto{\pgfqpoint{2.450546in}{1.304919in}}%
\pgfpathlineto{\pgfqpoint{2.452462in}{1.311799in}}%
\pgfpathlineto{\pgfqpoint{2.467783in}{1.318124in}}%
\pgfpathlineto{\pgfqpoint{2.473529in}{1.319384in}}%
\pgfpathlineto{\pgfqpoint{2.475444in}{1.322593in}}%
\pgfpathlineto{\pgfqpoint{2.483104in}{1.324817in}}%
\pgfpathlineto{\pgfqpoint{2.496511in}{1.326878in}}%
\pgfpathlineto{\pgfqpoint{2.500341in}{1.327574in}}%
\pgfpathlineto{\pgfqpoint{2.502256in}{1.330613in}}%
\pgfpathlineto{\pgfqpoint{2.513747in}{1.336034in}}%
\pgfpathlineto{\pgfqpoint{2.521408in}{1.342420in}}%
\pgfpathlineto{\pgfqpoint{2.525238in}{1.343810in}}%
\pgfpathlineto{\pgfqpoint{2.529068in}{1.344374in}}%
\pgfpathlineto{\pgfqpoint{2.530984in}{1.347620in}}%
\pgfpathlineto{\pgfqpoint{2.534814in}{1.348271in}}%
\pgfpathlineto{\pgfqpoint{2.536729in}{1.350086in}}%
\pgfpathlineto{\pgfqpoint{2.540560in}{1.350738in}}%
\pgfpathlineto{\pgfqpoint{2.550135in}{1.357234in}}%
\pgfpathlineto{\pgfqpoint{2.555881in}{1.360017in}}%
\pgfpathlineto{\pgfqpoint{2.561626in}{1.362926in}}%
\pgfpathlineto{\pgfqpoint{2.569287in}{1.365050in}}%
\pgfpathlineto{\pgfqpoint{2.582693in}{1.373519in}}%
\pgfpathlineto{\pgfqpoint{2.586524in}{1.373732in}}%
\pgfpathlineto{\pgfqpoint{2.590354in}{1.376344in}}%
\pgfpathlineto{\pgfqpoint{2.594184in}{1.377157in}}%
\pgfpathlineto{\pgfqpoint{2.596099in}{1.379307in}}%
\pgfpathlineto{\pgfqpoint{2.598015in}{1.379326in}}%
\pgfpathlineto{\pgfqpoint{2.599930in}{1.382580in}}%
\pgfpathlineto{\pgfqpoint{2.611421in}{1.387398in}}%
\pgfpathlineto{\pgfqpoint{2.615251in}{1.391488in}}%
\pgfpathlineto{\pgfqpoint{2.622912in}{1.394705in}}%
\pgfpathlineto{\pgfqpoint{2.626742in}{1.397350in}}%
\pgfpathlineto{\pgfqpoint{2.628657in}{1.399830in}}%
\pgfpathlineto{\pgfqpoint{2.630573in}{1.400132in}}%
\pgfpathlineto{\pgfqpoint{2.634403in}{1.402809in}}%
\pgfpathlineto{\pgfqpoint{2.640148in}{1.407597in}}%
\pgfpathlineto{\pgfqpoint{2.642064in}{1.410560in}}%
\pgfpathlineto{\pgfqpoint{2.653555in}{1.413354in}}%
\pgfpathlineto{\pgfqpoint{2.657385in}{1.413418in}}%
\pgfpathlineto{\pgfqpoint{2.661215in}{1.416665in}}%
\pgfpathlineto{\pgfqpoint{2.665046in}{1.417530in}}%
\pgfpathlineto{\pgfqpoint{2.670791in}{1.424160in}}%
\pgfpathlineto{\pgfqpoint{2.674622in}{1.426704in}}%
\pgfpathlineto{\pgfqpoint{2.680367in}{1.430200in}}%
\pgfpathlineto{\pgfqpoint{2.684197in}{1.432092in}}%
\pgfpathlineto{\pgfqpoint{2.705264in}{1.437492in}}%
\pgfpathlineto{\pgfqpoint{2.709095in}{1.438990in}}%
\pgfpathlineto{\pgfqpoint{2.711010in}{1.439468in}}%
\pgfpathlineto{\pgfqpoint{2.712925in}{1.442832in}}%
\pgfpathlineto{\pgfqpoint{2.714840in}{1.442968in}}%
\pgfpathlineto{\pgfqpoint{2.716755in}{1.445537in}}%
\pgfpathlineto{\pgfqpoint{2.718670in}{1.445586in}}%
\pgfpathlineto{\pgfqpoint{2.726331in}{1.453125in}}%
\pgfpathlineto{\pgfqpoint{2.732077in}{1.454615in}}%
\pgfpathlineto{\pgfqpoint{2.733992in}{1.457183in}}%
\pgfpathlineto{\pgfqpoint{2.735907in}{1.462732in}}%
\pgfpathlineto{\pgfqpoint{2.741653in}{1.463401in}}%
\pgfpathlineto{\pgfqpoint{2.756974in}{1.465973in}}%
\pgfpathlineto{\pgfqpoint{2.758889in}{1.468138in}}%
\pgfpathlineto{\pgfqpoint{2.768465in}{1.470328in}}%
\pgfpathlineto{\pgfqpoint{2.772295in}{1.473659in}}%
\pgfpathlineto{\pgfqpoint{2.787617in}{1.477967in}}%
\pgfpathlineto{\pgfqpoint{2.789532in}{1.479782in}}%
\pgfpathlineto{\pgfqpoint{2.791447in}{1.479873in}}%
\pgfpathlineto{\pgfqpoint{2.795277in}{1.482224in}}%
\pgfpathlineto{\pgfqpoint{2.797192in}{1.482495in}}%
\pgfpathlineto{\pgfqpoint{2.806768in}{1.491355in}}%
\pgfpathlineto{\pgfqpoint{2.812514in}{1.492313in}}%
\pgfpathlineto{\pgfqpoint{2.816344in}{1.493986in}}%
\pgfpathlineto{\pgfqpoint{2.825920in}{1.494933in}}%
\pgfpathlineto{\pgfqpoint{2.831666in}{1.499023in}}%
\pgfpathlineto{\pgfqpoint{2.835496in}{1.501681in}}%
\pgfpathlineto{\pgfqpoint{2.841241in}{1.502488in}}%
\pgfpathlineto{\pgfqpoint{2.845072in}{1.504543in}}%
\pgfpathlineto{\pgfqpoint{2.848902in}{1.505292in}}%
\pgfpathlineto{\pgfqpoint{2.852732in}{1.506588in}}%
\pgfpathlineto{\pgfqpoint{2.858478in}{1.512483in}}%
\pgfpathlineto{\pgfqpoint{2.862308in}{1.515014in}}%
\pgfpathlineto{\pgfqpoint{2.868054in}{1.515940in}}%
\pgfpathlineto{\pgfqpoint{2.885290in}{1.520846in}}%
\pgfpathlineto{\pgfqpoint{2.896781in}{1.521590in}}%
\pgfpathlineto{\pgfqpoint{2.904442in}{1.523240in}}%
\pgfpathlineto{\pgfqpoint{2.906357in}{1.523483in}}%
\pgfpathlineto{\pgfqpoint{2.910188in}{1.525801in}}%
\pgfpathlineto{\pgfqpoint{2.925509in}{1.530710in}}%
\pgfpathlineto{\pgfqpoint{2.931254in}{1.533507in}}%
\pgfpathlineto{\pgfqpoint{2.938915in}{1.535720in}}%
\pgfpathlineto{\pgfqpoint{2.944661in}{1.537075in}}%
\pgfpathlineto{\pgfqpoint{2.948491in}{1.538970in}}%
\pgfpathlineto{\pgfqpoint{2.950406in}{1.544819in}}%
\pgfpathlineto{\pgfqpoint{2.954237in}{1.545975in}}%
\pgfpathlineto{\pgfqpoint{2.971473in}{1.554591in}}%
\pgfpathlineto{\pgfqpoint{2.981049in}{1.555833in}}%
\pgfpathlineto{\pgfqpoint{2.982964in}{1.555916in}}%
\pgfpathlineto{\pgfqpoint{2.986794in}{1.558185in}}%
\pgfpathlineto{\pgfqpoint{3.005946in}{1.563484in}}%
\pgfpathlineto{\pgfqpoint{3.011692in}{1.564518in}}%
\pgfpathlineto{\pgfqpoint{3.015522in}{1.565751in}}%
\pgfpathlineto{\pgfqpoint{3.021268in}{1.567158in}}%
\pgfpathlineto{\pgfqpoint{3.027013in}{1.570473in}}%
\pgfpathlineto{\pgfqpoint{3.032759in}{1.571791in}}%
\pgfpathlineto{\pgfqpoint{3.042334in}{1.580085in}}%
\pgfpathlineto{\pgfqpoint{3.044250in}{1.585371in}}%
\pgfpathlineto{\pgfqpoint{3.049995in}{1.587024in}}%
\pgfpathlineto{\pgfqpoint{3.055741in}{1.590162in}}%
\pgfpathlineto{\pgfqpoint{3.057656in}{1.590169in}}%
\pgfpathlineto{\pgfqpoint{3.061486in}{1.592752in}}%
\pgfpathlineto{\pgfqpoint{3.065316in}{1.593964in}}%
\pgfpathlineto{\pgfqpoint{3.067232in}{1.596061in}}%
\pgfpathlineto{\pgfqpoint{3.074892in}{1.597476in}}%
\pgfpathlineto{\pgfqpoint{3.084468in}{1.599340in}}%
\pgfpathlineto{\pgfqpoint{3.088299in}{1.604378in}}%
\pgfpathlineto{\pgfqpoint{3.092129in}{1.606365in}}%
\pgfpathlineto{\pgfqpoint{3.095959in}{1.608326in}}%
\pgfpathlineto{\pgfqpoint{3.097874in}{1.613455in}}%
\pgfpathlineto{\pgfqpoint{3.103620in}{1.613847in}}%
\pgfpathlineto{\pgfqpoint{3.113196in}{1.619281in}}%
\pgfpathlineto{\pgfqpoint{3.115111in}{1.623954in}}%
\pgfpathlineto{\pgfqpoint{3.118941in}{1.625740in}}%
\pgfpathlineto{\pgfqpoint{3.132347in}{1.634957in}}%
\pgfpathlineto{\pgfqpoint{3.136178in}{1.638499in}}%
\pgfpathlineto{\pgfqpoint{3.141923in}{1.641375in}}%
\pgfpathlineto{\pgfqpoint{3.145754in}{1.642711in}}%
\pgfpathlineto{\pgfqpoint{3.149584in}{1.644107in}}%
\pgfpathlineto{\pgfqpoint{3.151499in}{1.645111in}}%
\pgfpathlineto{\pgfqpoint{3.155330in}{1.650055in}}%
\pgfpathlineto{\pgfqpoint{3.159160in}{1.651210in}}%
\pgfpathlineto{\pgfqpoint{3.162990in}{1.655356in}}%
\pgfpathlineto{\pgfqpoint{3.166821in}{1.658094in}}%
\pgfpathlineto{\pgfqpoint{3.168736in}{1.664658in}}%
\pgfpathlineto{\pgfqpoint{3.170651in}{1.665119in}}%
\pgfpathlineto{\pgfqpoint{3.176396in}{1.672858in}}%
\pgfpathlineto{\pgfqpoint{3.180227in}{1.676142in}}%
\pgfpathlineto{\pgfqpoint{3.184057in}{1.682198in}}%
\pgfpathlineto{\pgfqpoint{3.187887in}{1.684732in}}%
\pgfpathlineto{\pgfqpoint{3.195548in}{1.687314in}}%
\pgfpathlineto{\pgfqpoint{3.210870in}{1.694177in}}%
\pgfpathlineto{\pgfqpoint{3.212785in}{1.694366in}}%
\pgfpathlineto{\pgfqpoint{3.216615in}{1.697691in}}%
\pgfpathlineto{\pgfqpoint{3.218530in}{1.698077in}}%
\pgfpathlineto{\pgfqpoint{3.226191in}{1.705459in}}%
\pgfpathlineto{\pgfqpoint{3.231936in}{1.706937in}}%
\pgfpathlineto{\pgfqpoint{3.235767in}{1.709397in}}%
\pgfpathlineto{\pgfqpoint{3.239597in}{1.710579in}}%
\pgfpathlineto{\pgfqpoint{3.241512in}{1.713081in}}%
\pgfpathlineto{\pgfqpoint{3.249173in}{1.714877in}}%
\pgfpathlineto{\pgfqpoint{3.258749in}{1.717979in}}%
\pgfpathlineto{\pgfqpoint{3.262579in}{1.718800in}}%
\pgfpathlineto{\pgfqpoint{3.264494in}{1.723031in}}%
\pgfpathlineto{\pgfqpoint{3.268325in}{1.724753in}}%
\pgfpathlineto{\pgfqpoint{3.270240in}{1.729315in}}%
\pgfpathlineto{\pgfqpoint{3.274070in}{1.731238in}}%
\pgfpathlineto{\pgfqpoint{3.275985in}{1.731339in}}%
\pgfpathlineto{\pgfqpoint{3.283646in}{1.739637in}}%
\pgfpathlineto{\pgfqpoint{3.287476in}{1.741089in}}%
\pgfpathlineto{\pgfqpoint{3.300883in}{1.743885in}}%
\pgfpathlineto{\pgfqpoint{3.308543in}{1.749144in}}%
\pgfpathlineto{\pgfqpoint{3.312374in}{1.751786in}}%
\pgfpathlineto{\pgfqpoint{3.320034in}{1.752693in}}%
\pgfpathlineto{\pgfqpoint{3.335356in}{1.759690in}}%
\pgfpathlineto{\pgfqpoint{3.337271in}{1.763814in}}%
\pgfpathlineto{\pgfqpoint{3.343016in}{1.764834in}}%
\pgfpathlineto{\pgfqpoint{3.344932in}{1.768159in}}%
\pgfpathlineto{\pgfqpoint{3.352592in}{1.771217in}}%
\pgfpathlineto{\pgfqpoint{3.354507in}{1.774043in}}%
\pgfpathlineto{\pgfqpoint{3.364083in}{1.776861in}}%
\pgfpathlineto{\pgfqpoint{3.365998in}{1.778363in}}%
\pgfpathlineto{\pgfqpoint{3.367914in}{1.782522in}}%
\pgfpathlineto{\pgfqpoint{3.377489in}{1.787836in}}%
\pgfpathlineto{\pgfqpoint{3.379405in}{1.791237in}}%
\pgfpathlineto{\pgfqpoint{3.381320in}{1.791429in}}%
\pgfpathlineto{\pgfqpoint{3.383235in}{1.795558in}}%
\pgfpathlineto{\pgfqpoint{3.388980in}{1.796572in}}%
\pgfpathlineto{\pgfqpoint{3.392811in}{1.798073in}}%
\pgfpathlineto{\pgfqpoint{3.396641in}{1.798418in}}%
\pgfpathlineto{\pgfqpoint{3.400471in}{1.801076in}}%
\pgfpathlineto{\pgfqpoint{3.406217in}{1.807284in}}%
\pgfpathlineto{\pgfqpoint{3.411963in}{1.810957in}}%
\pgfpathlineto{\pgfqpoint{3.415793in}{1.811709in}}%
\pgfpathlineto{\pgfqpoint{3.417708in}{1.812934in}}%
\pgfpathlineto{\pgfqpoint{3.419623in}{1.818594in}}%
\pgfpathlineto{\pgfqpoint{3.421538in}{1.820611in}}%
\pgfpathlineto{\pgfqpoint{3.423454in}{1.820657in}}%
\pgfpathlineto{\pgfqpoint{3.431114in}{1.829109in}}%
\pgfpathlineto{\pgfqpoint{3.433029in}{1.831741in}}%
\pgfpathlineto{\pgfqpoint{3.444520in}{1.836854in}}%
\pgfpathlineto{\pgfqpoint{3.448351in}{1.839029in}}%
\pgfpathlineto{\pgfqpoint{3.450266in}{1.849581in}}%
\pgfpathlineto{\pgfqpoint{3.456011in}{1.851992in}}%
\pgfpathlineto{\pgfqpoint{3.457927in}{1.852243in}}%
\pgfpathlineto{\pgfqpoint{3.459842in}{1.855372in}}%
\pgfpathlineto{\pgfqpoint{3.465587in}{1.856785in}}%
\pgfpathlineto{\pgfqpoint{3.471333in}{1.860624in}}%
\pgfpathlineto{\pgfqpoint{3.477078in}{1.866814in}}%
\pgfpathlineto{\pgfqpoint{3.478994in}{1.867334in}}%
\pgfpathlineto{\pgfqpoint{3.480909in}{1.871678in}}%
\pgfpathlineto{\pgfqpoint{3.484739in}{1.872984in}}%
\pgfpathlineto{\pgfqpoint{3.490485in}{1.878331in}}%
\pgfpathlineto{\pgfqpoint{3.494315in}{1.879363in}}%
\pgfpathlineto{\pgfqpoint{3.496230in}{1.879757in}}%
\pgfpathlineto{\pgfqpoint{3.503891in}{1.887853in}}%
\pgfpathlineto{\pgfqpoint{3.507721in}{1.892036in}}%
\pgfpathlineto{\pgfqpoint{3.511551in}{1.892137in}}%
\pgfpathlineto{\pgfqpoint{3.515382in}{1.894167in}}%
\pgfpathlineto{\pgfqpoint{3.519212in}{1.896067in}}%
\pgfpathlineto{\pgfqpoint{3.523042in}{1.900503in}}%
\pgfpathlineto{\pgfqpoint{3.524958in}{1.900725in}}%
\pgfpathlineto{\pgfqpoint{3.530703in}{1.910792in}}%
\pgfpathlineto{\pgfqpoint{3.532618in}{1.910866in}}%
\pgfpathlineto{\pgfqpoint{3.534533in}{1.915424in}}%
\pgfpathlineto{\pgfqpoint{3.538364in}{1.918378in}}%
\pgfpathlineto{\pgfqpoint{3.546025in}{1.927138in}}%
\pgfpathlineto{\pgfqpoint{3.547940in}{1.927833in}}%
\pgfpathlineto{\pgfqpoint{3.553685in}{1.935603in}}%
\pgfpathlineto{\pgfqpoint{3.561346in}{1.937341in}}%
\pgfpathlineto{\pgfqpoint{3.563261in}{1.938777in}}%
\pgfpathlineto{\pgfqpoint{3.565176in}{1.943131in}}%
\pgfpathlineto{\pgfqpoint{3.570922in}{1.947309in}}%
\pgfpathlineto{\pgfqpoint{3.572837in}{1.953618in}}%
\pgfpathlineto{\pgfqpoint{3.574752in}{1.954540in}}%
\pgfpathlineto{\pgfqpoint{3.576667in}{1.957105in}}%
\pgfpathlineto{\pgfqpoint{3.584328in}{1.959942in}}%
\pgfpathlineto{\pgfqpoint{3.586243in}{1.961870in}}%
\pgfpathlineto{\pgfqpoint{3.588158in}{1.961968in}}%
\pgfpathlineto{\pgfqpoint{3.590073in}{1.963993in}}%
\pgfpathlineto{\pgfqpoint{3.597734in}{1.965814in}}%
\pgfpathlineto{\pgfqpoint{3.599649in}{1.969158in}}%
\pgfpathlineto{\pgfqpoint{3.601564in}{1.976736in}}%
\pgfpathlineto{\pgfqpoint{3.613056in}{1.985503in}}%
\pgfpathlineto{\pgfqpoint{3.616886in}{1.987104in}}%
\pgfpathlineto{\pgfqpoint{3.620716in}{1.988515in}}%
\pgfpathlineto{\pgfqpoint{3.626462in}{1.993639in}}%
\pgfpathlineto{\pgfqpoint{3.634122in}{1.998010in}}%
\pgfpathlineto{\pgfqpoint{3.637953in}{2.002768in}}%
\pgfpathlineto{\pgfqpoint{3.643698in}{2.009829in}}%
\pgfpathlineto{\pgfqpoint{3.647529in}{2.010932in}}%
\pgfpathlineto{\pgfqpoint{3.649444in}{2.017181in}}%
\pgfpathlineto{\pgfqpoint{3.651359in}{2.019475in}}%
\pgfpathlineto{\pgfqpoint{3.653274in}{2.024918in}}%
\pgfpathlineto{\pgfqpoint{3.655189in}{2.025342in}}%
\pgfpathlineto{\pgfqpoint{3.657104in}{2.028653in}}%
\pgfpathlineto{\pgfqpoint{3.670511in}{2.032569in}}%
\pgfpathlineto{\pgfqpoint{3.678171in}{2.042188in}}%
\pgfpathlineto{\pgfqpoint{3.683917in}{2.045665in}}%
\pgfpathlineto{\pgfqpoint{3.697323in}{2.056022in}}%
\pgfpathlineto{\pgfqpoint{3.703069in}{2.075225in}}%
\pgfpathlineto{\pgfqpoint{3.704984in}{2.076918in}}%
\pgfpathlineto{\pgfqpoint{3.706899in}{2.084165in}}%
\pgfpathlineto{\pgfqpoint{3.708814in}{2.086506in}}%
\pgfpathlineto{\pgfqpoint{3.710729in}{2.091692in}}%
\pgfpathlineto{\pgfqpoint{3.710729in}{2.091692in}}%
\pgfusepath{stroke}%
\end{pgfscope}%
\begin{pgfscope}%
\pgfpathrectangle{\pgfqpoint{0.694334in}{0.523557in}}{\pgfqpoint{3.830343in}{1.568135in}}%
\pgfusepath{clip}%
\pgfsetbuttcap%
\pgfsetroundjoin%
\pgfsetlinewidth{1.003750pt}%
\definecolor{currentstroke}{rgb}{0.811765,0.125490,0.125490}%
\pgfsetstrokecolor{currentstroke}%
\pgfsetdash{{1.000000pt}{1.650000pt}}{0.000000pt}%
\pgfpathmoveto{\pgfqpoint{0.694334in}{0.722479in}}%
\pgfpathlineto{\pgfqpoint{0.701995in}{0.723345in}}%
\pgfpathlineto{\pgfqpoint{0.707741in}{0.724587in}}%
\pgfpathlineto{\pgfqpoint{0.723062in}{0.725976in}}%
\pgfpathlineto{\pgfqpoint{1.040980in}{0.740777in}}%
\pgfpathlineto{\pgfqpoint{1.046726in}{0.741551in}}%
\pgfpathlineto{\pgfqpoint{1.067793in}{0.742662in}}%
\pgfpathlineto{\pgfqpoint{1.073538in}{0.743551in}}%
\pgfpathlineto{\pgfqpoint{1.083114in}{0.744194in}}%
\pgfpathlineto{\pgfqpoint{1.102266in}{0.745488in}}%
\pgfpathlineto{\pgfqpoint{1.109926in}{0.746403in}}%
\pgfpathlineto{\pgfqpoint{1.169297in}{0.752603in}}%
\pgfpathlineto{\pgfqpoint{1.173127in}{0.753430in}}%
\pgfpathlineto{\pgfqpoint{1.196109in}{0.757757in}}%
\pgfpathlineto{\pgfqpoint{1.207600in}{0.759136in}}%
\pgfpathlineto{\pgfqpoint{1.219091in}{0.760488in}}%
\pgfpathlineto{\pgfqpoint{1.224837in}{0.762256in}}%
\pgfpathlineto{\pgfqpoint{1.234413in}{0.763190in}}%
\pgfpathlineto{\pgfqpoint{1.274631in}{0.767436in}}%
\pgfpathlineto{\pgfqpoint{1.276546in}{0.767459in}}%
\pgfpathlineto{\pgfqpoint{1.278462in}{0.769301in}}%
\pgfpathlineto{\pgfqpoint{1.289953in}{0.771584in}}%
\pgfpathlineto{\pgfqpoint{1.297613in}{0.773372in}}%
\pgfpathlineto{\pgfqpoint{1.307189in}{0.774677in}}%
\pgfpathlineto{\pgfqpoint{1.316765in}{0.776577in}}%
\pgfpathlineto{\pgfqpoint{1.320595in}{0.777627in}}%
\pgfpathlineto{\pgfqpoint{1.351238in}{0.786216in}}%
\pgfpathlineto{\pgfqpoint{1.360814in}{0.787865in}}%
\pgfpathlineto{\pgfqpoint{1.366559in}{0.789354in}}%
\pgfpathlineto{\pgfqpoint{1.376135in}{0.790485in}}%
\pgfpathlineto{\pgfqpoint{1.431675in}{0.808010in}}%
\pgfpathlineto{\pgfqpoint{1.437421in}{0.812722in}}%
\pgfpathlineto{\pgfqpoint{1.445081in}{0.814052in}}%
\pgfpathlineto{\pgfqpoint{1.446997in}{0.817252in}}%
\pgfpathlineto{\pgfqpoint{1.468064in}{0.822298in}}%
\pgfpathlineto{\pgfqpoint{1.473809in}{0.825630in}}%
\pgfpathlineto{\pgfqpoint{1.483385in}{0.827486in}}%
\pgfpathlineto{\pgfqpoint{1.489130in}{0.828811in}}%
\pgfpathlineto{\pgfqpoint{1.510197in}{0.831579in}}%
\pgfpathlineto{\pgfqpoint{1.517858in}{0.834080in}}%
\pgfpathlineto{\pgfqpoint{1.525519in}{0.836395in}}%
\pgfpathlineto{\pgfqpoint{1.529349in}{0.840346in}}%
\pgfpathlineto{\pgfqpoint{1.535095in}{0.842332in}}%
\pgfpathlineto{\pgfqpoint{1.538925in}{0.843768in}}%
\pgfpathlineto{\pgfqpoint{1.540840in}{0.846018in}}%
\pgfpathlineto{\pgfqpoint{1.552331in}{0.849214in}}%
\pgfpathlineto{\pgfqpoint{1.554246in}{0.852136in}}%
\pgfpathlineto{\pgfqpoint{1.556161in}{0.852294in}}%
\pgfpathlineto{\pgfqpoint{1.558077in}{0.854238in}}%
\pgfpathlineto{\pgfqpoint{1.567652in}{0.855406in}}%
\pgfpathlineto{\pgfqpoint{1.579143in}{0.858522in}}%
\pgfpathlineto{\pgfqpoint{1.581059in}{0.858542in}}%
\pgfpathlineto{\pgfqpoint{1.582974in}{0.861251in}}%
\pgfpathlineto{\pgfqpoint{1.590635in}{0.863918in}}%
\pgfpathlineto{\pgfqpoint{1.594465in}{0.866408in}}%
\pgfpathlineto{\pgfqpoint{1.628938in}{0.878311in}}%
\pgfpathlineto{\pgfqpoint{1.636599in}{0.884125in}}%
\pgfpathlineto{\pgfqpoint{1.642344in}{0.886148in}}%
\pgfpathlineto{\pgfqpoint{1.653835in}{0.891821in}}%
\pgfpathlineto{\pgfqpoint{1.657666in}{0.893668in}}%
\pgfpathlineto{\pgfqpoint{1.672987in}{0.897242in}}%
\pgfpathlineto{\pgfqpoint{1.676817in}{0.901039in}}%
\pgfpathlineto{\pgfqpoint{1.678732in}{0.901061in}}%
\pgfpathlineto{\pgfqpoint{1.680648in}{0.906734in}}%
\pgfpathlineto{\pgfqpoint{1.684478in}{0.907748in}}%
\pgfpathlineto{\pgfqpoint{1.697884in}{0.911266in}}%
\pgfpathlineto{\pgfqpoint{1.699799in}{0.913254in}}%
\pgfpathlineto{\pgfqpoint{1.707460in}{0.914138in}}%
\pgfpathlineto{\pgfqpoint{1.709375in}{0.916529in}}%
\pgfpathlineto{\pgfqpoint{1.718951in}{0.918444in}}%
\pgfpathlineto{\pgfqpoint{1.724697in}{0.921105in}}%
\pgfpathlineto{\pgfqpoint{1.728527in}{0.925030in}}%
\pgfpathlineto{\pgfqpoint{1.732357in}{0.926586in}}%
\pgfpathlineto{\pgfqpoint{1.743848in}{0.932232in}}%
\pgfpathlineto{\pgfqpoint{1.745763in}{0.932263in}}%
\pgfpathlineto{\pgfqpoint{1.755339in}{0.937902in}}%
\pgfpathlineto{\pgfqpoint{1.766830in}{0.939564in}}%
\pgfpathlineto{\pgfqpoint{1.770661in}{0.941065in}}%
\pgfpathlineto{\pgfqpoint{1.785982in}{0.944702in}}%
\pgfpathlineto{\pgfqpoint{1.787897in}{0.948058in}}%
\pgfpathlineto{\pgfqpoint{1.789812in}{0.948314in}}%
\pgfpathlineto{\pgfqpoint{1.793643in}{0.950980in}}%
\pgfpathlineto{\pgfqpoint{1.835776in}{0.970202in}}%
\pgfpathlineto{\pgfqpoint{1.837692in}{0.972640in}}%
\pgfpathlineto{\pgfqpoint{1.839607in}{0.972792in}}%
\pgfpathlineto{\pgfqpoint{1.841522in}{0.975530in}}%
\pgfpathlineto{\pgfqpoint{1.847267in}{0.976381in}}%
\pgfpathlineto{\pgfqpoint{1.851098in}{0.978477in}}%
\pgfpathlineto{\pgfqpoint{1.853013in}{0.981358in}}%
\pgfpathlineto{\pgfqpoint{1.854928in}{0.981710in}}%
\pgfpathlineto{\pgfqpoint{1.862589in}{0.986968in}}%
\pgfpathlineto{\pgfqpoint{1.868334in}{0.988929in}}%
\pgfpathlineto{\pgfqpoint{1.870250in}{0.988970in}}%
\pgfpathlineto{\pgfqpoint{1.874080in}{0.991849in}}%
\pgfpathlineto{\pgfqpoint{1.877910in}{0.993275in}}%
\pgfpathlineto{\pgfqpoint{1.881741in}{0.998576in}}%
\pgfpathlineto{\pgfqpoint{1.883656in}{1.000229in}}%
\pgfpathlineto{\pgfqpoint{1.885571in}{1.000291in}}%
\pgfpathlineto{\pgfqpoint{1.889401in}{1.002609in}}%
\pgfpathlineto{\pgfqpoint{1.891316in}{1.002867in}}%
\pgfpathlineto{\pgfqpoint{1.893232in}{1.004678in}}%
\pgfpathlineto{\pgfqpoint{1.895147in}{1.004730in}}%
\pgfpathlineto{\pgfqpoint{1.897062in}{1.005996in}}%
\pgfpathlineto{\pgfqpoint{1.898977in}{1.009045in}}%
\pgfpathlineto{\pgfqpoint{1.904723in}{1.012119in}}%
\pgfpathlineto{\pgfqpoint{1.908553in}{1.015139in}}%
\pgfpathlineto{\pgfqpoint{1.910468in}{1.016086in}}%
\pgfpathlineto{\pgfqpoint{1.916214in}{1.022994in}}%
\pgfpathlineto{\pgfqpoint{1.920044in}{1.023610in}}%
\pgfpathlineto{\pgfqpoint{1.921959in}{1.025373in}}%
\pgfpathlineto{\pgfqpoint{1.925790in}{1.026328in}}%
\pgfpathlineto{\pgfqpoint{1.929620in}{1.029182in}}%
\pgfpathlineto{\pgfqpoint{1.931535in}{1.031351in}}%
\pgfpathlineto{\pgfqpoint{1.935365in}{1.031701in}}%
\pgfpathlineto{\pgfqpoint{1.939196in}{1.034199in}}%
\pgfpathlineto{\pgfqpoint{1.943026in}{1.035618in}}%
\pgfpathlineto{\pgfqpoint{1.944941in}{1.035776in}}%
\pgfpathlineto{\pgfqpoint{1.950687in}{1.038673in}}%
\pgfpathlineto{\pgfqpoint{1.952602in}{1.039495in}}%
\pgfpathlineto{\pgfqpoint{1.954517in}{1.042077in}}%
\pgfpathlineto{\pgfqpoint{1.956432in}{1.042503in}}%
\pgfpathlineto{\pgfqpoint{1.962178in}{1.046953in}}%
\pgfpathlineto{\pgfqpoint{1.967923in}{1.048716in}}%
\pgfpathlineto{\pgfqpoint{1.973669in}{1.051811in}}%
\pgfpathlineto{\pgfqpoint{1.979414in}{1.053090in}}%
\pgfpathlineto{\pgfqpoint{1.981329in}{1.054770in}}%
\pgfpathlineto{\pgfqpoint{1.983245in}{1.061532in}}%
\pgfpathlineto{\pgfqpoint{1.985160in}{1.063255in}}%
\pgfpathlineto{\pgfqpoint{1.987075in}{1.063272in}}%
\pgfpathlineto{\pgfqpoint{1.988990in}{1.066943in}}%
\pgfpathlineto{\pgfqpoint{1.996651in}{1.068069in}}%
\pgfpathlineto{\pgfqpoint{2.002396in}{1.070333in}}%
\pgfpathlineto{\pgfqpoint{2.010057in}{1.073600in}}%
\pgfpathlineto{\pgfqpoint{2.015803in}{1.079618in}}%
\pgfpathlineto{\pgfqpoint{2.031124in}{1.085562in}}%
\pgfpathlineto{\pgfqpoint{2.033039in}{1.087489in}}%
\pgfpathlineto{\pgfqpoint{2.036869in}{1.088023in}}%
\pgfpathlineto{\pgfqpoint{2.040700in}{1.095773in}}%
\pgfpathlineto{\pgfqpoint{2.044530in}{1.098204in}}%
\pgfpathlineto{\pgfqpoint{2.048360in}{1.099215in}}%
\pgfpathlineto{\pgfqpoint{2.061767in}{1.105671in}}%
\pgfpathlineto{\pgfqpoint{2.065597in}{1.109893in}}%
\pgfpathlineto{\pgfqpoint{2.075173in}{1.112329in}}%
\pgfpathlineto{\pgfqpoint{2.077088in}{1.113201in}}%
\pgfpathlineto{\pgfqpoint{2.080918in}{1.118356in}}%
\pgfpathlineto{\pgfqpoint{2.082834in}{1.118869in}}%
\pgfpathlineto{\pgfqpoint{2.084749in}{1.122471in}}%
\pgfpathlineto{\pgfqpoint{2.086664in}{1.122478in}}%
\pgfpathlineto{\pgfqpoint{2.088579in}{1.124343in}}%
\pgfpathlineto{\pgfqpoint{2.094325in}{1.125422in}}%
\pgfpathlineto{\pgfqpoint{2.103900in}{1.134275in}}%
\pgfpathlineto{\pgfqpoint{2.105816in}{1.135856in}}%
\pgfpathlineto{\pgfqpoint{2.109646in}{1.143050in}}%
\pgfpathlineto{\pgfqpoint{2.119222in}{1.144923in}}%
\pgfpathlineto{\pgfqpoint{2.121137in}{1.148026in}}%
\pgfpathlineto{\pgfqpoint{2.126883in}{1.148762in}}%
\pgfpathlineto{\pgfqpoint{2.128798in}{1.154886in}}%
\pgfpathlineto{\pgfqpoint{2.132628in}{1.157637in}}%
\pgfpathlineto{\pgfqpoint{2.134543in}{1.163393in}}%
\pgfpathlineto{\pgfqpoint{2.138374in}{1.164348in}}%
\pgfpathlineto{\pgfqpoint{2.140289in}{1.168818in}}%
\pgfpathlineto{\pgfqpoint{2.144119in}{1.169818in}}%
\pgfpathlineto{\pgfqpoint{2.149865in}{1.178002in}}%
\pgfpathlineto{\pgfqpoint{2.161356in}{1.181299in}}%
\pgfpathlineto{\pgfqpoint{2.163271in}{1.181384in}}%
\pgfpathlineto{\pgfqpoint{2.165186in}{1.187122in}}%
\pgfpathlineto{\pgfqpoint{2.169016in}{1.188445in}}%
\pgfpathlineto{\pgfqpoint{2.172847in}{1.192822in}}%
\pgfpathlineto{\pgfqpoint{2.174762in}{1.193391in}}%
\pgfpathlineto{\pgfqpoint{2.176677in}{1.195513in}}%
\pgfpathlineto{\pgfqpoint{2.180507in}{1.195883in}}%
\pgfpathlineto{\pgfqpoint{2.182422in}{1.198256in}}%
\pgfpathlineto{\pgfqpoint{2.184338in}{1.198668in}}%
\pgfpathlineto{\pgfqpoint{2.186253in}{1.201813in}}%
\pgfpathlineto{\pgfqpoint{2.190083in}{1.202894in}}%
\pgfpathlineto{\pgfqpoint{2.197744in}{1.208197in}}%
\pgfpathlineto{\pgfqpoint{2.203489in}{1.210553in}}%
\pgfpathlineto{\pgfqpoint{2.209235in}{1.211057in}}%
\pgfpathlineto{\pgfqpoint{2.243708in}{1.212830in}}%
\pgfpathlineto{\pgfqpoint{2.285842in}{1.217134in}}%
\pgfpathlineto{\pgfqpoint{2.289672in}{1.219749in}}%
\pgfpathlineto{\pgfqpoint{2.297333in}{1.220829in}}%
\pgfpathlineto{\pgfqpoint{2.304993in}{1.224698in}}%
\pgfpathlineto{\pgfqpoint{2.308824in}{1.225068in}}%
\pgfpathlineto{\pgfqpoint{2.312654in}{1.226676in}}%
\pgfpathlineto{\pgfqpoint{2.318400in}{1.227820in}}%
\pgfpathlineto{\pgfqpoint{2.322230in}{1.229457in}}%
\pgfpathlineto{\pgfqpoint{2.324145in}{1.229905in}}%
\pgfpathlineto{\pgfqpoint{2.326060in}{1.232563in}}%
\pgfpathlineto{\pgfqpoint{2.337551in}{1.237663in}}%
\pgfpathlineto{\pgfqpoint{2.341382in}{1.241720in}}%
\pgfpathlineto{\pgfqpoint{2.352873in}{1.246404in}}%
\pgfpathlineto{\pgfqpoint{2.356703in}{1.249890in}}%
\pgfpathlineto{\pgfqpoint{2.358618in}{1.250189in}}%
\pgfpathlineto{\pgfqpoint{2.360533in}{1.252128in}}%
\pgfpathlineto{\pgfqpoint{2.362449in}{1.259273in}}%
\pgfpathlineto{\pgfqpoint{2.366279in}{1.261659in}}%
\pgfpathlineto{\pgfqpoint{2.370109in}{1.265035in}}%
\pgfpathlineto{\pgfqpoint{2.372024in}{1.265808in}}%
\pgfpathlineto{\pgfqpoint{2.373940in}{1.268318in}}%
\pgfpathlineto{\pgfqpoint{2.377770in}{1.269950in}}%
\pgfpathlineto{\pgfqpoint{2.381600in}{1.271903in}}%
\pgfpathlineto{\pgfqpoint{2.385431in}{1.279485in}}%
\pgfpathlineto{\pgfqpoint{2.389261in}{1.280310in}}%
\pgfpathlineto{\pgfqpoint{2.391176in}{1.280342in}}%
\pgfpathlineto{\pgfqpoint{2.393091in}{1.283140in}}%
\pgfpathlineto{\pgfqpoint{2.395006in}{1.290280in}}%
\pgfpathlineto{\pgfqpoint{2.402667in}{1.293344in}}%
\pgfpathlineto{\pgfqpoint{2.404582in}{1.297826in}}%
\pgfpathlineto{\pgfqpoint{2.408413in}{1.299344in}}%
\pgfpathlineto{\pgfqpoint{2.412243in}{1.300271in}}%
\pgfpathlineto{\pgfqpoint{2.414158in}{1.303148in}}%
\pgfpathlineto{\pgfqpoint{2.417989in}{1.303341in}}%
\pgfpathlineto{\pgfqpoint{2.421819in}{1.305719in}}%
\pgfpathlineto{\pgfqpoint{2.423734in}{1.309931in}}%
\pgfpathlineto{\pgfqpoint{2.425649in}{1.310089in}}%
\pgfpathlineto{\pgfqpoint{2.429480in}{1.312282in}}%
\pgfpathlineto{\pgfqpoint{2.437140in}{1.315752in}}%
\pgfpathlineto{\pgfqpoint{2.448631in}{1.319926in}}%
\pgfpathlineto{\pgfqpoint{2.450546in}{1.324583in}}%
\pgfpathlineto{\pgfqpoint{2.452462in}{1.325691in}}%
\pgfpathlineto{\pgfqpoint{2.454377in}{1.331877in}}%
\pgfpathlineto{\pgfqpoint{2.456292in}{1.334307in}}%
\pgfpathlineto{\pgfqpoint{2.460122in}{1.335881in}}%
\pgfpathlineto{\pgfqpoint{2.465868in}{1.338191in}}%
\pgfpathlineto{\pgfqpoint{2.471613in}{1.347701in}}%
\pgfpathlineto{\pgfqpoint{2.479274in}{1.350086in}}%
\pgfpathlineto{\pgfqpoint{2.485020in}{1.353323in}}%
\pgfpathlineto{\pgfqpoint{2.492680in}{1.361538in}}%
\pgfpathlineto{\pgfqpoint{2.496511in}{1.366963in}}%
\pgfpathlineto{\pgfqpoint{2.502256in}{1.368744in}}%
\pgfpathlineto{\pgfqpoint{2.511832in}{1.372672in}}%
\pgfpathlineto{\pgfqpoint{2.517577in}{1.378088in}}%
\pgfpathlineto{\pgfqpoint{2.519493in}{1.387303in}}%
\pgfpathlineto{\pgfqpoint{2.527153in}{1.389527in}}%
\pgfpathlineto{\pgfqpoint{2.529068in}{1.394337in}}%
\pgfpathlineto{\pgfqpoint{2.538644in}{1.397819in}}%
\pgfpathlineto{\pgfqpoint{2.540560in}{1.401232in}}%
\pgfpathlineto{\pgfqpoint{2.542475in}{1.407898in}}%
\pgfpathlineto{\pgfqpoint{2.550135in}{1.410607in}}%
\pgfpathlineto{\pgfqpoint{2.567372in}{1.419439in}}%
\pgfpathlineto{\pgfqpoint{2.573117in}{1.425086in}}%
\pgfpathlineto{\pgfqpoint{2.578863in}{1.427470in}}%
\pgfpathlineto{\pgfqpoint{2.580778in}{1.428294in}}%
\pgfpathlineto{\pgfqpoint{2.582693in}{1.433339in}}%
\pgfpathlineto{\pgfqpoint{2.588439in}{1.434378in}}%
\pgfpathlineto{\pgfqpoint{2.598015in}{1.434889in}}%
\pgfpathlineto{\pgfqpoint{2.603760in}{1.439144in}}%
\pgfpathlineto{\pgfqpoint{2.605675in}{1.439873in}}%
\pgfpathlineto{\pgfqpoint{2.609506in}{1.443194in}}%
\pgfpathlineto{\pgfqpoint{2.613336in}{1.444436in}}%
\pgfpathlineto{\pgfqpoint{2.624827in}{1.448166in}}%
\pgfpathlineto{\pgfqpoint{2.628657in}{1.452656in}}%
\pgfpathlineto{\pgfqpoint{2.636318in}{1.453775in}}%
\pgfpathlineto{\pgfqpoint{2.642064in}{1.458739in}}%
\pgfpathlineto{\pgfqpoint{2.643979in}{1.462097in}}%
\pgfpathlineto{\pgfqpoint{2.655470in}{1.465775in}}%
\pgfpathlineto{\pgfqpoint{2.657385in}{1.468224in}}%
\pgfpathlineto{\pgfqpoint{2.666961in}{1.469750in}}%
\pgfpathlineto{\pgfqpoint{2.672706in}{1.474749in}}%
\pgfpathlineto{\pgfqpoint{2.680367in}{1.476284in}}%
\pgfpathlineto{\pgfqpoint{2.707179in}{1.491482in}}%
\pgfpathlineto{\pgfqpoint{2.711010in}{1.492028in}}%
\pgfpathlineto{\pgfqpoint{2.712925in}{1.494518in}}%
\pgfpathlineto{\pgfqpoint{2.714840in}{1.494995in}}%
\pgfpathlineto{\pgfqpoint{2.718670in}{1.501451in}}%
\pgfpathlineto{\pgfqpoint{2.724416in}{1.502046in}}%
\pgfpathlineto{\pgfqpoint{2.728246in}{1.503614in}}%
\pgfpathlineto{\pgfqpoint{2.735907in}{1.507100in}}%
\pgfpathlineto{\pgfqpoint{2.739737in}{1.508357in}}%
\pgfpathlineto{\pgfqpoint{2.747398in}{1.509189in}}%
\pgfpathlineto{\pgfqpoint{2.753144in}{1.511803in}}%
\pgfpathlineto{\pgfqpoint{2.758889in}{1.515786in}}%
\pgfpathlineto{\pgfqpoint{2.760804in}{1.516100in}}%
\pgfpathlineto{\pgfqpoint{2.766550in}{1.522200in}}%
\pgfpathlineto{\pgfqpoint{2.770380in}{1.523278in}}%
\pgfpathlineto{\pgfqpoint{2.781871in}{1.526151in}}%
\pgfpathlineto{\pgfqpoint{2.785701in}{1.528421in}}%
\pgfpathlineto{\pgfqpoint{2.787617in}{1.528910in}}%
\pgfpathlineto{\pgfqpoint{2.791447in}{1.530991in}}%
\pgfpathlineto{\pgfqpoint{2.795277in}{1.532087in}}%
\pgfpathlineto{\pgfqpoint{2.801023in}{1.533118in}}%
\pgfpathlineto{\pgfqpoint{2.812514in}{1.538165in}}%
\pgfpathlineto{\pgfqpoint{2.829750in}{1.546124in}}%
\pgfpathlineto{\pgfqpoint{2.833581in}{1.546807in}}%
\pgfpathlineto{\pgfqpoint{2.837411in}{1.548347in}}%
\pgfpathlineto{\pgfqpoint{2.845072in}{1.551962in}}%
\pgfpathlineto{\pgfqpoint{2.852732in}{1.553915in}}%
\pgfpathlineto{\pgfqpoint{2.856563in}{1.557437in}}%
\pgfpathlineto{\pgfqpoint{2.860393in}{1.558070in}}%
\pgfpathlineto{\pgfqpoint{2.862308in}{1.562237in}}%
\pgfpathlineto{\pgfqpoint{2.869969in}{1.566022in}}%
\pgfpathlineto{\pgfqpoint{2.877630in}{1.567613in}}%
\pgfpathlineto{\pgfqpoint{2.881460in}{1.570523in}}%
\pgfpathlineto{\pgfqpoint{2.898697in}{1.579351in}}%
\pgfpathlineto{\pgfqpoint{2.900612in}{1.579375in}}%
\pgfpathlineto{\pgfqpoint{2.904442in}{1.582416in}}%
\pgfpathlineto{\pgfqpoint{2.906357in}{1.582417in}}%
\pgfpathlineto{\pgfqpoint{2.908272in}{1.586776in}}%
\pgfpathlineto{\pgfqpoint{2.910188in}{1.587427in}}%
\pgfpathlineto{\pgfqpoint{2.912103in}{1.590176in}}%
\pgfpathlineto{\pgfqpoint{2.914018in}{1.590335in}}%
\pgfpathlineto{\pgfqpoint{2.915933in}{1.592329in}}%
\pgfpathlineto{\pgfqpoint{2.919763in}{1.592865in}}%
\pgfpathlineto{\pgfqpoint{2.927424in}{1.597049in}}%
\pgfpathlineto{\pgfqpoint{2.929339in}{1.603059in}}%
\pgfpathlineto{\pgfqpoint{2.942746in}{1.610430in}}%
\pgfpathlineto{\pgfqpoint{2.944661in}{1.613671in}}%
\pgfpathlineto{\pgfqpoint{2.946576in}{1.613812in}}%
\pgfpathlineto{\pgfqpoint{2.948491in}{1.616623in}}%
\pgfpathlineto{\pgfqpoint{2.952321in}{1.618188in}}%
\pgfpathlineto{\pgfqpoint{2.958067in}{1.623411in}}%
\pgfpathlineto{\pgfqpoint{2.965728in}{1.625824in}}%
\pgfpathlineto{\pgfqpoint{2.967643in}{1.626980in}}%
\pgfpathlineto{\pgfqpoint{2.969558in}{1.634446in}}%
\pgfpathlineto{\pgfqpoint{2.981049in}{1.639014in}}%
\pgfpathlineto{\pgfqpoint{2.982964in}{1.643590in}}%
\pgfpathlineto{\pgfqpoint{2.984879in}{1.644399in}}%
\pgfpathlineto{\pgfqpoint{2.986794in}{1.651668in}}%
\pgfpathlineto{\pgfqpoint{2.990625in}{1.652897in}}%
\pgfpathlineto{\pgfqpoint{2.992540in}{1.655402in}}%
\pgfpathlineto{\pgfqpoint{2.996370in}{1.657189in}}%
\pgfpathlineto{\pgfqpoint{2.998285in}{1.659313in}}%
\pgfpathlineto{\pgfqpoint{3.000201in}{1.663422in}}%
\pgfpathlineto{\pgfqpoint{3.002116in}{1.664053in}}%
\pgfpathlineto{\pgfqpoint{3.004031in}{1.666597in}}%
\pgfpathlineto{\pgfqpoint{3.005946in}{1.666723in}}%
\pgfpathlineto{\pgfqpoint{3.009777in}{1.672505in}}%
\pgfpathlineto{\pgfqpoint{3.011692in}{1.673113in}}%
\pgfpathlineto{\pgfqpoint{3.013607in}{1.677430in}}%
\pgfpathlineto{\pgfqpoint{3.015522in}{1.677988in}}%
\pgfpathlineto{\pgfqpoint{3.017437in}{1.681375in}}%
\pgfpathlineto{\pgfqpoint{3.021268in}{1.682064in}}%
\pgfpathlineto{\pgfqpoint{3.027013in}{1.695385in}}%
\pgfpathlineto{\pgfqpoint{3.028928in}{1.702363in}}%
\pgfpathlineto{\pgfqpoint{3.032759in}{1.703334in}}%
\pgfpathlineto{\pgfqpoint{3.036589in}{1.709095in}}%
\pgfpathlineto{\pgfqpoint{3.040419in}{1.710787in}}%
\pgfpathlineto{\pgfqpoint{3.046165in}{1.712530in}}%
\pgfpathlineto{\pgfqpoint{3.049995in}{1.713438in}}%
\pgfpathlineto{\pgfqpoint{3.053825in}{1.715846in}}%
\pgfpathlineto{\pgfqpoint{3.057656in}{1.717606in}}%
\pgfpathlineto{\pgfqpoint{3.065316in}{1.723051in}}%
\pgfpathlineto{\pgfqpoint{3.067232in}{1.727399in}}%
\pgfpathlineto{\pgfqpoint{3.069147in}{1.728675in}}%
\pgfpathlineto{\pgfqpoint{3.072977in}{1.734487in}}%
\pgfpathlineto{\pgfqpoint{3.078723in}{1.737393in}}%
\pgfpathlineto{\pgfqpoint{3.080638in}{1.742271in}}%
\pgfpathlineto{\pgfqpoint{3.082553in}{1.743190in}}%
\pgfpathlineto{\pgfqpoint{3.092129in}{1.755091in}}%
\pgfpathlineto{\pgfqpoint{3.094044in}{1.755373in}}%
\pgfpathlineto{\pgfqpoint{3.097874in}{1.764507in}}%
\pgfpathlineto{\pgfqpoint{3.101705in}{1.765435in}}%
\pgfpathlineto{\pgfqpoint{3.103620in}{1.771092in}}%
\pgfpathlineto{\pgfqpoint{3.105535in}{1.771394in}}%
\pgfpathlineto{\pgfqpoint{3.109365in}{1.776407in}}%
\pgfpathlineto{\pgfqpoint{3.115111in}{1.778384in}}%
\pgfpathlineto{\pgfqpoint{3.118941in}{1.783515in}}%
\pgfpathlineto{\pgfqpoint{3.120856in}{1.785905in}}%
\pgfpathlineto{\pgfqpoint{3.124687in}{1.787429in}}%
\pgfpathlineto{\pgfqpoint{3.126602in}{1.791396in}}%
\pgfpathlineto{\pgfqpoint{3.130432in}{1.792936in}}%
\pgfpathlineto{\pgfqpoint{3.136178in}{1.802738in}}%
\pgfpathlineto{\pgfqpoint{3.138093in}{1.804374in}}%
\pgfpathlineto{\pgfqpoint{3.140008in}{1.811501in}}%
\pgfpathlineto{\pgfqpoint{3.141923in}{1.813814in}}%
\pgfpathlineto{\pgfqpoint{3.143839in}{1.818167in}}%
\pgfpathlineto{\pgfqpoint{3.145754in}{1.818897in}}%
\pgfpathlineto{\pgfqpoint{3.149584in}{1.824739in}}%
\pgfpathlineto{\pgfqpoint{3.157245in}{1.833728in}}%
\pgfpathlineto{\pgfqpoint{3.162990in}{1.835108in}}%
\pgfpathlineto{\pgfqpoint{3.164905in}{1.839880in}}%
\pgfpathlineto{\pgfqpoint{3.168736in}{1.840830in}}%
\pgfpathlineto{\pgfqpoint{3.182142in}{1.846412in}}%
\pgfpathlineto{\pgfqpoint{3.184057in}{1.852126in}}%
\pgfpathlineto{\pgfqpoint{3.193633in}{1.857708in}}%
\pgfpathlineto{\pgfqpoint{3.195548in}{1.865554in}}%
\pgfpathlineto{\pgfqpoint{3.199378in}{1.866911in}}%
\pgfpathlineto{\pgfqpoint{3.201294in}{1.870832in}}%
\pgfpathlineto{\pgfqpoint{3.205124in}{1.871506in}}%
\pgfpathlineto{\pgfqpoint{3.207039in}{1.879205in}}%
\pgfpathlineto{\pgfqpoint{3.208954in}{1.879503in}}%
\pgfpathlineto{\pgfqpoint{3.214700in}{1.884770in}}%
\pgfpathlineto{\pgfqpoint{3.216615in}{1.884906in}}%
\pgfpathlineto{\pgfqpoint{3.226191in}{1.906046in}}%
\pgfpathlineto{\pgfqpoint{3.230021in}{1.910431in}}%
\pgfpathlineto{\pgfqpoint{3.233852in}{1.922031in}}%
\pgfpathlineto{\pgfqpoint{3.235767in}{1.931670in}}%
\pgfpathlineto{\pgfqpoint{3.237682in}{1.932444in}}%
\pgfpathlineto{\pgfqpoint{3.241512in}{1.935529in}}%
\pgfpathlineto{\pgfqpoint{3.245343in}{1.937006in}}%
\pgfpathlineto{\pgfqpoint{3.247258in}{1.945047in}}%
\pgfpathlineto{\pgfqpoint{3.251088in}{1.946614in}}%
\pgfpathlineto{\pgfqpoint{3.253003in}{1.950043in}}%
\pgfpathlineto{\pgfqpoint{3.256834in}{1.961407in}}%
\pgfpathlineto{\pgfqpoint{3.258749in}{1.964424in}}%
\pgfpathlineto{\pgfqpoint{3.260664in}{1.964937in}}%
\pgfpathlineto{\pgfqpoint{3.264494in}{1.972786in}}%
\pgfpathlineto{\pgfqpoint{3.272155in}{1.976273in}}%
\pgfpathlineto{\pgfqpoint{3.274070in}{1.976344in}}%
\pgfpathlineto{\pgfqpoint{3.275985in}{1.979977in}}%
\pgfpathlineto{\pgfqpoint{3.277901in}{1.980116in}}%
\pgfpathlineto{\pgfqpoint{3.281731in}{1.983007in}}%
\pgfpathlineto{\pgfqpoint{3.283646in}{1.994581in}}%
\pgfpathlineto{\pgfqpoint{3.285561in}{1.998871in}}%
\pgfpathlineto{\pgfqpoint{3.293222in}{2.000792in}}%
\pgfpathlineto{\pgfqpoint{3.295137in}{2.007532in}}%
\pgfpathlineto{\pgfqpoint{3.297052in}{2.008913in}}%
\pgfpathlineto{\pgfqpoint{3.300883in}{2.019621in}}%
\pgfpathlineto{\pgfqpoint{3.304713in}{2.020959in}}%
\pgfpathlineto{\pgfqpoint{3.306628in}{2.021608in}}%
\pgfpathlineto{\pgfqpoint{3.308543in}{2.028531in}}%
\pgfpathlineto{\pgfqpoint{3.314289in}{2.030652in}}%
\pgfpathlineto{\pgfqpoint{3.320034in}{2.045912in}}%
\pgfpathlineto{\pgfqpoint{3.321949in}{2.047154in}}%
\pgfpathlineto{\pgfqpoint{3.323865in}{2.051258in}}%
\pgfpathlineto{\pgfqpoint{3.327695in}{2.052410in}}%
\pgfpathlineto{\pgfqpoint{3.329610in}{2.053947in}}%
\pgfpathlineto{\pgfqpoint{3.333440in}{2.060479in}}%
\pgfpathlineto{\pgfqpoint{3.337271in}{2.061671in}}%
\pgfpathlineto{\pgfqpoint{3.341101in}{2.066277in}}%
\pgfpathlineto{\pgfqpoint{3.343016in}{2.068453in}}%
\pgfpathlineto{\pgfqpoint{3.344932in}{2.068897in}}%
\pgfpathlineto{\pgfqpoint{3.350677in}{2.087714in}}%
\pgfpathlineto{\pgfqpoint{3.358338in}{2.091692in}}%
\pgfpathlineto{\pgfqpoint{3.358338in}{2.091692in}}%
\pgfusepath{stroke}%
\end{pgfscope}%
\begin{pgfscope}%
\pgfpathrectangle{\pgfqpoint{0.694334in}{0.523557in}}{\pgfqpoint{3.830343in}{1.568135in}}%
\pgfusepath{clip}%
\pgfsetrectcap%
\pgfsetroundjoin%
\pgfsetlinewidth{1.003750pt}%
\definecolor{currentstroke}{rgb}{0.062745,0.000000,0.062745}%
\pgfsetstrokecolor{currentstroke}%
\pgfsetdash{}{0pt}%
\pgfpathmoveto{\pgfqpoint{0.694334in}{0.577282in}}%
\pgfpathlineto{\pgfqpoint{0.701995in}{0.583939in}}%
\pgfpathlineto{\pgfqpoint{0.705825in}{0.583939in}}%
\pgfpathlineto{\pgfqpoint{0.707741in}{0.592572in}}%
\pgfpathlineto{\pgfqpoint{0.721147in}{0.600279in}}%
\pgfpathlineto{\pgfqpoint{0.724977in}{0.624852in}}%
\pgfpathlineto{\pgfqpoint{0.732638in}{0.632382in}}%
\pgfpathlineto{\pgfqpoint{0.734553in}{0.633128in}}%
\pgfpathlineto{\pgfqpoint{0.736468in}{0.636030in}}%
\pgfpathlineto{\pgfqpoint{0.747959in}{0.636030in}}%
\pgfpathlineto{\pgfqpoint{0.751789in}{0.641398in}}%
\pgfpathlineto{\pgfqpoint{0.753705in}{0.641501in}}%
\pgfpathlineto{\pgfqpoint{0.755620in}{0.645769in}}%
\pgfpathlineto{\pgfqpoint{0.765196in}{0.646313in}}%
\pgfpathlineto{\pgfqpoint{0.770941in}{0.647200in}}%
\pgfpathlineto{\pgfqpoint{0.782432in}{0.651476in}}%
\pgfpathlineto{\pgfqpoint{0.790093in}{0.652563in}}%
\pgfpathlineto{\pgfqpoint{0.807329in}{0.655834in}}%
\pgfpathlineto{\pgfqpoint{0.814990in}{0.656246in}}%
\pgfpathlineto{\pgfqpoint{0.818820in}{0.658585in}}%
\pgfpathlineto{\pgfqpoint{0.830311in}{0.659632in}}%
\pgfpathlineto{\pgfqpoint{0.836057in}{0.660882in}}%
\pgfpathlineto{\pgfqpoint{0.837972in}{0.660952in}}%
\pgfpathlineto{\pgfqpoint{0.839887in}{0.662565in}}%
\pgfpathlineto{\pgfqpoint{0.847548in}{0.663225in}}%
\pgfpathlineto{\pgfqpoint{0.851378in}{0.664698in}}%
\pgfpathlineto{\pgfqpoint{0.857124in}{0.664698in}}%
\pgfpathlineto{\pgfqpoint{0.860954in}{0.666535in}}%
\pgfpathlineto{\pgfqpoint{0.870530in}{0.668279in}}%
\pgfpathlineto{\pgfqpoint{0.885851in}{0.669390in}}%
\pgfpathlineto{\pgfqpoint{0.910749in}{0.672337in}}%
\pgfpathlineto{\pgfqpoint{0.927985in}{0.672989in}}%
\pgfpathlineto{\pgfqpoint{0.935646in}{0.673310in}}%
\pgfpathlineto{\pgfqpoint{0.937561in}{0.675636in}}%
\pgfpathlineto{\pgfqpoint{0.956713in}{0.677997in}}%
\pgfpathlineto{\pgfqpoint{0.962458in}{0.678652in}}%
\pgfpathlineto{\pgfqpoint{0.972034in}{0.679685in}}%
\pgfpathlineto{\pgfqpoint{0.998847in}{0.680778in}}%
\pgfpathlineto{\pgfqpoint{1.042895in}{0.687556in}}%
\pgfpathlineto{\pgfqpoint{1.063962in}{0.688698in}}%
\pgfpathlineto{\pgfqpoint{1.073538in}{0.692029in}}%
\pgfpathlineto{\pgfqpoint{1.109926in}{0.695566in}}%
\pgfpathlineto{\pgfqpoint{1.117587in}{0.695997in}}%
\pgfpathlineto{\pgfqpoint{1.121418in}{0.698272in}}%
\pgfpathlineto{\pgfqpoint{1.152060in}{0.700864in}}%
\pgfpathlineto{\pgfqpoint{1.167382in}{0.701657in}}%
\pgfpathlineto{\pgfqpoint{1.180788in}{0.702636in}}%
\pgfpathlineto{\pgfqpoint{1.186533in}{0.703563in}}%
\pgfpathlineto{\pgfqpoint{1.192279in}{0.706089in}}%
\pgfpathlineto{\pgfqpoint{1.199940in}{0.707103in}}%
\pgfpathlineto{\pgfqpoint{1.211431in}{0.707925in}}%
\pgfpathlineto{\pgfqpoint{1.234413in}{0.710409in}}%
\pgfpathlineto{\pgfqpoint{1.238243in}{0.713175in}}%
\pgfpathlineto{\pgfqpoint{1.268886in}{0.715134in}}%
\pgfpathlineto{\pgfqpoint{1.278462in}{0.718884in}}%
\pgfpathlineto{\pgfqpoint{1.311019in}{0.722178in}}%
\pgfpathlineto{\pgfqpoint{1.322511in}{0.725080in}}%
\pgfpathlineto{\pgfqpoint{1.343577in}{0.725875in}}%
\pgfpathlineto{\pgfqpoint{1.349323in}{0.726470in}}%
\pgfpathlineto{\pgfqpoint{1.355068in}{0.727403in}}%
\pgfpathlineto{\pgfqpoint{1.358899in}{0.728701in}}%
\pgfpathlineto{\pgfqpoint{1.372305in}{0.730324in}}%
\pgfpathlineto{\pgfqpoint{1.385711in}{0.731246in}}%
\pgfpathlineto{\pgfqpoint{1.393372in}{0.731898in}}%
\pgfpathlineto{\pgfqpoint{1.412524in}{0.733972in}}%
\pgfpathlineto{\pgfqpoint{1.424015in}{0.735659in}}%
\pgfpathlineto{\pgfqpoint{1.439336in}{0.736313in}}%
\pgfpathlineto{\pgfqpoint{1.468064in}{0.738217in}}%
\pgfpathlineto{\pgfqpoint{1.492961in}{0.740212in}}%
\pgfpathlineto{\pgfqpoint{1.523604in}{0.743194in}}%
\pgfpathlineto{\pgfqpoint{1.535095in}{0.744134in}}%
\pgfpathlineto{\pgfqpoint{1.571483in}{0.748628in}}%
\pgfpathlineto{\pgfqpoint{1.588719in}{0.750037in}}%
\pgfpathlineto{\pgfqpoint{1.592550in}{0.751706in}}%
\pgfpathlineto{\pgfqpoint{1.602126in}{0.752936in}}%
\pgfpathlineto{\pgfqpoint{1.609786in}{0.753422in}}%
\pgfpathlineto{\pgfqpoint{1.613617in}{0.754911in}}%
\pgfpathlineto{\pgfqpoint{1.632768in}{0.758043in}}%
\pgfpathlineto{\pgfqpoint{1.642344in}{0.758626in}}%
\pgfpathlineto{\pgfqpoint{1.657666in}{0.759738in}}%
\pgfpathlineto{\pgfqpoint{1.669157in}{0.761536in}}%
\pgfpathlineto{\pgfqpoint{1.697884in}{0.764090in}}%
\pgfpathlineto{\pgfqpoint{1.705545in}{0.765274in}}%
\pgfpathlineto{\pgfqpoint{1.718951in}{0.766745in}}%
\pgfpathlineto{\pgfqpoint{1.722781in}{0.768120in}}%
\pgfpathlineto{\pgfqpoint{1.755339in}{0.769759in}}%
\pgfpathlineto{\pgfqpoint{1.768745in}{0.772101in}}%
\pgfpathlineto{\pgfqpoint{1.805134in}{0.777762in}}%
\pgfpathlineto{\pgfqpoint{1.807049in}{0.779340in}}%
\pgfpathlineto{\pgfqpoint{1.824285in}{0.780663in}}%
\pgfpathlineto{\pgfqpoint{1.828116in}{0.782096in}}%
\pgfpathlineto{\pgfqpoint{1.835776in}{0.783493in}}%
\pgfpathlineto{\pgfqpoint{1.872165in}{0.788589in}}%
\pgfpathlineto{\pgfqpoint{1.877910in}{0.790646in}}%
\pgfpathlineto{\pgfqpoint{1.908553in}{0.800897in}}%
\pgfpathlineto{\pgfqpoint{1.914298in}{0.801504in}}%
\pgfpathlineto{\pgfqpoint{1.920044in}{0.804073in}}%
\pgfpathlineto{\pgfqpoint{1.929620in}{0.807909in}}%
\pgfpathlineto{\pgfqpoint{1.950687in}{0.814210in}}%
\pgfpathlineto{\pgfqpoint{1.964093in}{0.817620in}}%
\pgfpathlineto{\pgfqpoint{1.967923in}{0.818988in}}%
\pgfpathlineto{\pgfqpoint{1.979414in}{0.821621in}}%
\pgfpathlineto{\pgfqpoint{1.988990in}{0.828973in}}%
\pgfpathlineto{\pgfqpoint{1.994736in}{0.830358in}}%
\pgfpathlineto{\pgfqpoint{2.004312in}{0.833479in}}%
\pgfpathlineto{\pgfqpoint{2.021548in}{0.835959in}}%
\pgfpathlineto{\pgfqpoint{2.029209in}{0.837593in}}%
\pgfpathlineto{\pgfqpoint{2.036869in}{0.838262in}}%
\pgfpathlineto{\pgfqpoint{2.073258in}{0.843921in}}%
\pgfpathlineto{\pgfqpoint{2.075173in}{0.846931in}}%
\pgfpathlineto{\pgfqpoint{2.080918in}{0.847737in}}%
\pgfpathlineto{\pgfqpoint{2.086664in}{0.850396in}}%
\pgfpathlineto{\pgfqpoint{2.096240in}{0.853951in}}%
\pgfpathlineto{\pgfqpoint{2.100070in}{0.855254in}}%
\pgfpathlineto{\pgfqpoint{2.107731in}{0.857116in}}%
\pgfpathlineto{\pgfqpoint{2.111561in}{0.859015in}}%
\pgfpathlineto{\pgfqpoint{2.124967in}{0.869662in}}%
\pgfpathlineto{\pgfqpoint{2.128798in}{0.877678in}}%
\pgfpathlineto{\pgfqpoint{2.130713in}{0.878300in}}%
\pgfpathlineto{\pgfqpoint{2.132628in}{0.881870in}}%
\pgfpathlineto{\pgfqpoint{2.134543in}{0.894462in}}%
\pgfpathlineto{\pgfqpoint{2.140289in}{0.897652in}}%
\pgfpathlineto{\pgfqpoint{2.144119in}{0.899481in}}%
\pgfpathlineto{\pgfqpoint{2.146034in}{0.899499in}}%
\pgfpathlineto{\pgfqpoint{2.147949in}{0.901637in}}%
\pgfpathlineto{\pgfqpoint{2.153695in}{0.902946in}}%
\pgfpathlineto{\pgfqpoint{2.155610in}{0.908157in}}%
\pgfpathlineto{\pgfqpoint{2.157525in}{0.908911in}}%
\pgfpathlineto{\pgfqpoint{2.161356in}{0.913445in}}%
\pgfpathlineto{\pgfqpoint{2.163271in}{0.916100in}}%
\pgfpathlineto{\pgfqpoint{2.167101in}{0.923048in}}%
\pgfpathlineto{\pgfqpoint{2.169016in}{0.927880in}}%
\pgfpathlineto{\pgfqpoint{2.174762in}{0.930296in}}%
\pgfpathlineto{\pgfqpoint{2.178592in}{0.930545in}}%
\pgfpathlineto{\pgfqpoint{2.182422in}{0.936347in}}%
\pgfpathlineto{\pgfqpoint{2.186253in}{0.937389in}}%
\pgfpathlineto{\pgfqpoint{2.190083in}{0.942968in}}%
\pgfpathlineto{\pgfqpoint{2.193914in}{0.944176in}}%
\pgfpathlineto{\pgfqpoint{2.197744in}{0.948351in}}%
\pgfpathlineto{\pgfqpoint{2.203489in}{0.951982in}}%
\pgfpathlineto{\pgfqpoint{2.209235in}{0.962870in}}%
\pgfpathlineto{\pgfqpoint{2.211150in}{0.967364in}}%
\pgfpathlineto{\pgfqpoint{2.224556in}{0.974383in}}%
\pgfpathlineto{\pgfqpoint{2.226471in}{0.974427in}}%
\pgfpathlineto{\pgfqpoint{2.232217in}{0.983530in}}%
\pgfpathlineto{\pgfqpoint{2.237962in}{0.988814in}}%
\pgfpathlineto{\pgfqpoint{2.253284in}{1.007931in}}%
\pgfpathlineto{\pgfqpoint{2.257114in}{1.009023in}}%
\pgfpathlineto{\pgfqpoint{2.259029in}{1.009725in}}%
\pgfpathlineto{\pgfqpoint{2.264775in}{1.018401in}}%
\pgfpathlineto{\pgfqpoint{2.270520in}{1.020081in}}%
\pgfpathlineto{\pgfqpoint{2.274351in}{1.022246in}}%
\pgfpathlineto{\pgfqpoint{2.278181in}{1.023905in}}%
\pgfpathlineto{\pgfqpoint{2.282011in}{1.028207in}}%
\pgfpathlineto{\pgfqpoint{2.285842in}{1.028852in}}%
\pgfpathlineto{\pgfqpoint{2.289672in}{1.031921in}}%
\pgfpathlineto{\pgfqpoint{2.293502in}{1.035393in}}%
\pgfpathlineto{\pgfqpoint{2.295418in}{1.040638in}}%
\pgfpathlineto{\pgfqpoint{2.297333in}{1.041689in}}%
\pgfpathlineto{\pgfqpoint{2.301163in}{1.045916in}}%
\pgfpathlineto{\pgfqpoint{2.303078in}{1.047042in}}%
\pgfpathlineto{\pgfqpoint{2.306909in}{1.051277in}}%
\pgfpathlineto{\pgfqpoint{2.312654in}{1.053191in}}%
\pgfpathlineto{\pgfqpoint{2.316484in}{1.054537in}}%
\pgfpathlineto{\pgfqpoint{2.318400in}{1.054824in}}%
\pgfpathlineto{\pgfqpoint{2.320315in}{1.058326in}}%
\pgfpathlineto{\pgfqpoint{2.326060in}{1.060152in}}%
\pgfpathlineto{\pgfqpoint{2.329891in}{1.064581in}}%
\pgfpathlineto{\pgfqpoint{2.333721in}{1.065213in}}%
\pgfpathlineto{\pgfqpoint{2.337551in}{1.068425in}}%
\pgfpathlineto{\pgfqpoint{2.341382in}{1.069146in}}%
\pgfpathlineto{\pgfqpoint{2.347127in}{1.071929in}}%
\pgfpathlineto{\pgfqpoint{2.349042in}{1.072078in}}%
\pgfpathlineto{\pgfqpoint{2.352873in}{1.076011in}}%
\pgfpathlineto{\pgfqpoint{2.354788in}{1.076724in}}%
\pgfpathlineto{\pgfqpoint{2.356703in}{1.078879in}}%
\pgfpathlineto{\pgfqpoint{2.360533in}{1.080350in}}%
\pgfpathlineto{\pgfqpoint{2.364364in}{1.081928in}}%
\pgfpathlineto{\pgfqpoint{2.366279in}{1.084782in}}%
\pgfpathlineto{\pgfqpoint{2.368194in}{1.085498in}}%
\pgfpathlineto{\pgfqpoint{2.373940in}{1.096880in}}%
\pgfpathlineto{\pgfqpoint{2.375855in}{1.096958in}}%
\pgfpathlineto{\pgfqpoint{2.377770in}{1.101498in}}%
\pgfpathlineto{\pgfqpoint{2.385431in}{1.105675in}}%
\pgfpathlineto{\pgfqpoint{2.389261in}{1.110001in}}%
\pgfpathlineto{\pgfqpoint{2.391176in}{1.110349in}}%
\pgfpathlineto{\pgfqpoint{2.395006in}{1.113686in}}%
\pgfpathlineto{\pgfqpoint{2.398837in}{1.114359in}}%
\pgfpathlineto{\pgfqpoint{2.400752in}{1.116308in}}%
\pgfpathlineto{\pgfqpoint{2.402667in}{1.121155in}}%
\pgfpathlineto{\pgfqpoint{2.406498in}{1.121885in}}%
\pgfpathlineto{\pgfqpoint{2.408413in}{1.123564in}}%
\pgfpathlineto{\pgfqpoint{2.410328in}{1.123588in}}%
\pgfpathlineto{\pgfqpoint{2.414158in}{1.126510in}}%
\pgfpathlineto{\pgfqpoint{2.417989in}{1.127550in}}%
\pgfpathlineto{\pgfqpoint{2.421819in}{1.133930in}}%
\pgfpathlineto{\pgfqpoint{2.425649in}{1.134340in}}%
\pgfpathlineto{\pgfqpoint{2.446716in}{1.153848in}}%
\pgfpathlineto{\pgfqpoint{2.456292in}{1.159200in}}%
\pgfpathlineto{\pgfqpoint{2.460122in}{1.162946in}}%
\pgfpathlineto{\pgfqpoint{2.462037in}{1.163096in}}%
\pgfpathlineto{\pgfqpoint{2.467783in}{1.168581in}}%
\pgfpathlineto{\pgfqpoint{2.471613in}{1.172587in}}%
\pgfpathlineto{\pgfqpoint{2.477359in}{1.179731in}}%
\pgfpathlineto{\pgfqpoint{2.479274in}{1.186314in}}%
\pgfpathlineto{\pgfqpoint{2.483104in}{1.190303in}}%
\pgfpathlineto{\pgfqpoint{2.490765in}{1.191607in}}%
\pgfpathlineto{\pgfqpoint{2.492680in}{1.193883in}}%
\pgfpathlineto{\pgfqpoint{2.496511in}{1.194389in}}%
\pgfpathlineto{\pgfqpoint{2.498426in}{1.197995in}}%
\pgfpathlineto{\pgfqpoint{2.504171in}{1.200082in}}%
\pgfpathlineto{\pgfqpoint{2.515662in}{1.204322in}}%
\pgfpathlineto{\pgfqpoint{2.519493in}{1.207556in}}%
\pgfpathlineto{\pgfqpoint{2.521408in}{1.208121in}}%
\pgfpathlineto{\pgfqpoint{2.525238in}{1.210277in}}%
\pgfpathlineto{\pgfqpoint{2.542475in}{1.211775in}}%
\pgfpathlineto{\pgfqpoint{2.607591in}{1.215513in}}%
\pgfpathlineto{\pgfqpoint{2.611421in}{1.216677in}}%
\pgfpathlineto{\pgfqpoint{2.617166in}{1.218768in}}%
\pgfpathlineto{\pgfqpoint{2.620997in}{1.219812in}}%
\pgfpathlineto{\pgfqpoint{2.624827in}{1.220829in}}%
\pgfpathlineto{\pgfqpoint{2.628657in}{1.221936in}}%
\pgfpathlineto{\pgfqpoint{2.632488in}{1.222909in}}%
\pgfpathlineto{\pgfqpoint{2.636318in}{1.224720in}}%
\pgfpathlineto{\pgfqpoint{2.649724in}{1.227987in}}%
\pgfpathlineto{\pgfqpoint{2.657385in}{1.234729in}}%
\pgfpathlineto{\pgfqpoint{2.663130in}{1.235534in}}%
\pgfpathlineto{\pgfqpoint{2.666961in}{1.237663in}}%
\pgfpathlineto{\pgfqpoint{2.670791in}{1.243584in}}%
\pgfpathlineto{\pgfqpoint{2.674622in}{1.252128in}}%
\pgfpathlineto{\pgfqpoint{2.689943in}{1.262838in}}%
\pgfpathlineto{\pgfqpoint{2.693773in}{1.272455in}}%
\pgfpathlineto{\pgfqpoint{2.697604in}{1.273604in}}%
\pgfpathlineto{\pgfqpoint{2.699519in}{1.275542in}}%
\pgfpathlineto{\pgfqpoint{2.703349in}{1.276430in}}%
\pgfpathlineto{\pgfqpoint{2.707179in}{1.289890in}}%
\pgfpathlineto{\pgfqpoint{2.711010in}{1.293120in}}%
\pgfpathlineto{\pgfqpoint{2.714840in}{1.294181in}}%
\pgfpathlineto{\pgfqpoint{2.716755in}{1.298089in}}%
\pgfpathlineto{\pgfqpoint{2.720586in}{1.299344in}}%
\pgfpathlineto{\pgfqpoint{2.722501in}{1.303676in}}%
\pgfpathlineto{\pgfqpoint{2.724416in}{1.304367in}}%
\pgfpathlineto{\pgfqpoint{2.726331in}{1.308268in}}%
\pgfpathlineto{\pgfqpoint{2.730161in}{1.308963in}}%
\pgfpathlineto{\pgfqpoint{2.732077in}{1.311115in}}%
\pgfpathlineto{\pgfqpoint{2.735907in}{1.311275in}}%
\pgfpathlineto{\pgfqpoint{2.739737in}{1.317698in}}%
\pgfpathlineto{\pgfqpoint{2.745483in}{1.320059in}}%
\pgfpathlineto{\pgfqpoint{2.749313in}{1.323256in}}%
\pgfpathlineto{\pgfqpoint{2.753144in}{1.330986in}}%
\pgfpathlineto{\pgfqpoint{2.756974in}{1.334549in}}%
\pgfpathlineto{\pgfqpoint{2.760804in}{1.344530in}}%
\pgfpathlineto{\pgfqpoint{2.762719in}{1.345154in}}%
\pgfpathlineto{\pgfqpoint{2.768465in}{1.350772in}}%
\pgfpathlineto{\pgfqpoint{2.772295in}{1.351951in}}%
\pgfpathlineto{\pgfqpoint{2.776126in}{1.358112in}}%
\pgfpathlineto{\pgfqpoint{2.783786in}{1.359740in}}%
\pgfpathlineto{\pgfqpoint{2.785701in}{1.361247in}}%
\pgfpathlineto{\pgfqpoint{2.787617in}{1.365852in}}%
\pgfpathlineto{\pgfqpoint{2.789532in}{1.366189in}}%
\pgfpathlineto{\pgfqpoint{2.791447in}{1.371680in}}%
\pgfpathlineto{\pgfqpoint{2.797192in}{1.375238in}}%
\pgfpathlineto{\pgfqpoint{2.802938in}{1.385997in}}%
\pgfpathlineto{\pgfqpoint{2.808684in}{1.387246in}}%
\pgfpathlineto{\pgfqpoint{2.812514in}{1.389704in}}%
\pgfpathlineto{\pgfqpoint{2.818259in}{1.400257in}}%
\pgfpathlineto{\pgfqpoint{2.820175in}{1.401625in}}%
\pgfpathlineto{\pgfqpoint{2.822090in}{1.404538in}}%
\pgfpathlineto{\pgfqpoint{2.824005in}{1.404576in}}%
\pgfpathlineto{\pgfqpoint{2.825920in}{1.411673in}}%
\pgfpathlineto{\pgfqpoint{2.829750in}{1.412015in}}%
\pgfpathlineto{\pgfqpoint{2.846987in}{1.429441in}}%
\pgfpathlineto{\pgfqpoint{2.850817in}{1.430015in}}%
\pgfpathlineto{\pgfqpoint{2.852732in}{1.434313in}}%
\pgfpathlineto{\pgfqpoint{2.854648in}{1.434514in}}%
\pgfpathlineto{\pgfqpoint{2.858478in}{1.437069in}}%
\pgfpathlineto{\pgfqpoint{2.860393in}{1.442246in}}%
\pgfpathlineto{\pgfqpoint{2.864223in}{1.443613in}}%
\pgfpathlineto{\pgfqpoint{2.866139in}{1.444032in}}%
\pgfpathlineto{\pgfqpoint{2.875715in}{1.453996in}}%
\pgfpathlineto{\pgfqpoint{2.877630in}{1.454434in}}%
\pgfpathlineto{\pgfqpoint{2.879545in}{1.456849in}}%
\pgfpathlineto{\pgfqpoint{2.885290in}{1.458142in}}%
\pgfpathlineto{\pgfqpoint{2.889121in}{1.462475in}}%
\pgfpathlineto{\pgfqpoint{2.891036in}{1.462633in}}%
\pgfpathlineto{\pgfqpoint{2.892951in}{1.465655in}}%
\pgfpathlineto{\pgfqpoint{2.900612in}{1.466261in}}%
\pgfpathlineto{\pgfqpoint{2.902527in}{1.470749in}}%
\pgfpathlineto{\pgfqpoint{2.904442in}{1.470996in}}%
\pgfpathlineto{\pgfqpoint{2.906357in}{1.475553in}}%
\pgfpathlineto{\pgfqpoint{2.908272in}{1.475592in}}%
\pgfpathlineto{\pgfqpoint{2.910188in}{1.477608in}}%
\pgfpathlineto{\pgfqpoint{2.912103in}{1.478004in}}%
\pgfpathlineto{\pgfqpoint{2.919763in}{1.486850in}}%
\pgfpathlineto{\pgfqpoint{2.929339in}{1.488448in}}%
\pgfpathlineto{\pgfqpoint{2.937000in}{1.491744in}}%
\pgfpathlineto{\pgfqpoint{2.938915in}{1.491866in}}%
\pgfpathlineto{\pgfqpoint{2.942746in}{1.495021in}}%
\pgfpathlineto{\pgfqpoint{2.944661in}{1.495801in}}%
\pgfpathlineto{\pgfqpoint{2.948491in}{1.500151in}}%
\pgfpathlineto{\pgfqpoint{2.956152in}{1.502258in}}%
\pgfpathlineto{\pgfqpoint{2.958067in}{1.502921in}}%
\pgfpathlineto{\pgfqpoint{2.959982in}{1.505977in}}%
\pgfpathlineto{\pgfqpoint{2.963812in}{1.506995in}}%
\pgfpathlineto{\pgfqpoint{2.969558in}{1.518529in}}%
\pgfpathlineto{\pgfqpoint{2.971473in}{1.518579in}}%
\pgfpathlineto{\pgfqpoint{2.975303in}{1.529068in}}%
\pgfpathlineto{\pgfqpoint{2.982964in}{1.532919in}}%
\pgfpathlineto{\pgfqpoint{2.984879in}{1.533234in}}%
\pgfpathlineto{\pgfqpoint{2.990625in}{1.538045in}}%
\pgfpathlineto{\pgfqpoint{2.994455in}{1.540408in}}%
\pgfpathlineto{\pgfqpoint{2.996370in}{1.540537in}}%
\pgfpathlineto{\pgfqpoint{3.005946in}{1.547811in}}%
\pgfpathlineto{\pgfqpoint{3.011692in}{1.553767in}}%
\pgfpathlineto{\pgfqpoint{3.013607in}{1.554030in}}%
\pgfpathlineto{\pgfqpoint{3.017437in}{1.558866in}}%
\pgfpathlineto{\pgfqpoint{3.019352in}{1.563072in}}%
\pgfpathlineto{\pgfqpoint{3.023183in}{1.564711in}}%
\pgfpathlineto{\pgfqpoint{3.027013in}{1.568008in}}%
\pgfpathlineto{\pgfqpoint{3.028928in}{1.568128in}}%
\pgfpathlineto{\pgfqpoint{3.032759in}{1.571995in}}%
\pgfpathlineto{\pgfqpoint{3.036589in}{1.573343in}}%
\pgfpathlineto{\pgfqpoint{3.040419in}{1.577683in}}%
\pgfpathlineto{\pgfqpoint{3.042334in}{1.579768in}}%
\pgfpathlineto{\pgfqpoint{3.051910in}{1.582837in}}%
\pgfpathlineto{\pgfqpoint{3.053825in}{1.586109in}}%
\pgfpathlineto{\pgfqpoint{3.055741in}{1.586385in}}%
\pgfpathlineto{\pgfqpoint{3.057656in}{1.590748in}}%
\pgfpathlineto{\pgfqpoint{3.061486in}{1.591841in}}%
\pgfpathlineto{\pgfqpoint{3.063401in}{1.594440in}}%
\pgfpathlineto{\pgfqpoint{3.065316in}{1.595077in}}%
\pgfpathlineto{\pgfqpoint{3.067232in}{1.597047in}}%
\pgfpathlineto{\pgfqpoint{3.069147in}{1.601527in}}%
\pgfpathlineto{\pgfqpoint{3.071062in}{1.601741in}}%
\pgfpathlineto{\pgfqpoint{3.072977in}{1.606079in}}%
\pgfpathlineto{\pgfqpoint{3.076808in}{1.607494in}}%
\pgfpathlineto{\pgfqpoint{3.080638in}{1.615168in}}%
\pgfpathlineto{\pgfqpoint{3.084468in}{1.618872in}}%
\pgfpathlineto{\pgfqpoint{3.088299in}{1.619442in}}%
\pgfpathlineto{\pgfqpoint{3.095959in}{1.627576in}}%
\pgfpathlineto{\pgfqpoint{3.099790in}{1.634126in}}%
\pgfpathlineto{\pgfqpoint{3.101705in}{1.636082in}}%
\pgfpathlineto{\pgfqpoint{3.105535in}{1.643111in}}%
\pgfpathlineto{\pgfqpoint{3.109365in}{1.657517in}}%
\pgfpathlineto{\pgfqpoint{3.111281in}{1.662925in}}%
\pgfpathlineto{\pgfqpoint{3.113196in}{1.664487in}}%
\pgfpathlineto{\pgfqpoint{3.115111in}{1.664597in}}%
\pgfpathlineto{\pgfqpoint{3.118941in}{1.668795in}}%
\pgfpathlineto{\pgfqpoint{3.120856in}{1.669192in}}%
\pgfpathlineto{\pgfqpoint{3.122772in}{1.672470in}}%
\pgfpathlineto{\pgfqpoint{3.124687in}{1.679710in}}%
\pgfpathlineto{\pgfqpoint{3.126602in}{1.681420in}}%
\pgfpathlineto{\pgfqpoint{3.128517in}{1.685903in}}%
\pgfpathlineto{\pgfqpoint{3.130432in}{1.686308in}}%
\pgfpathlineto{\pgfqpoint{3.132347in}{1.689725in}}%
\pgfpathlineto{\pgfqpoint{3.140008in}{1.692584in}}%
\pgfpathlineto{\pgfqpoint{3.141923in}{1.695085in}}%
\pgfpathlineto{\pgfqpoint{3.145754in}{1.695909in}}%
\pgfpathlineto{\pgfqpoint{3.151499in}{1.699245in}}%
\pgfpathlineto{\pgfqpoint{3.153414in}{1.702702in}}%
\pgfpathlineto{\pgfqpoint{3.155330in}{1.710372in}}%
\pgfpathlineto{\pgfqpoint{3.162990in}{1.715295in}}%
\pgfpathlineto{\pgfqpoint{3.164905in}{1.722747in}}%
\pgfpathlineto{\pgfqpoint{3.168736in}{1.724896in}}%
\pgfpathlineto{\pgfqpoint{3.170651in}{1.725825in}}%
\pgfpathlineto{\pgfqpoint{3.174481in}{1.742416in}}%
\pgfpathlineto{\pgfqpoint{3.182142in}{1.762096in}}%
\pgfpathlineto{\pgfqpoint{3.189803in}{1.766234in}}%
\pgfpathlineto{\pgfqpoint{3.191718in}{1.766614in}}%
\pgfpathlineto{\pgfqpoint{3.195548in}{1.773922in}}%
\pgfpathlineto{\pgfqpoint{3.197463in}{1.774594in}}%
\pgfpathlineto{\pgfqpoint{3.199378in}{1.780070in}}%
\pgfpathlineto{\pgfqpoint{3.201294in}{1.780879in}}%
\pgfpathlineto{\pgfqpoint{3.203209in}{1.783348in}}%
\pgfpathlineto{\pgfqpoint{3.205124in}{1.783824in}}%
\pgfpathlineto{\pgfqpoint{3.207039in}{1.800773in}}%
\pgfpathlineto{\pgfqpoint{3.214700in}{1.808977in}}%
\pgfpathlineto{\pgfqpoint{3.216615in}{1.813630in}}%
\pgfpathlineto{\pgfqpoint{3.218530in}{1.813763in}}%
\pgfpathlineto{\pgfqpoint{3.220445in}{1.816556in}}%
\pgfpathlineto{\pgfqpoint{3.224276in}{1.817145in}}%
\pgfpathlineto{\pgfqpoint{3.241512in}{1.830942in}}%
\pgfpathlineto{\pgfqpoint{3.251088in}{1.835836in}}%
\pgfpathlineto{\pgfqpoint{3.256834in}{1.842588in}}%
\pgfpathlineto{\pgfqpoint{3.258749in}{1.843628in}}%
\pgfpathlineto{\pgfqpoint{3.262579in}{1.851338in}}%
\pgfpathlineto{\pgfqpoint{3.266409in}{1.861117in}}%
\pgfpathlineto{\pgfqpoint{3.272155in}{1.863038in}}%
\pgfpathlineto{\pgfqpoint{3.274070in}{1.866721in}}%
\pgfpathlineto{\pgfqpoint{3.275985in}{1.867488in}}%
\pgfpathlineto{\pgfqpoint{3.277901in}{1.869951in}}%
\pgfpathlineto{\pgfqpoint{3.279816in}{1.870219in}}%
\pgfpathlineto{\pgfqpoint{3.283646in}{1.874245in}}%
\pgfpathlineto{\pgfqpoint{3.285561in}{1.879781in}}%
\pgfpathlineto{\pgfqpoint{3.295137in}{1.888676in}}%
\pgfpathlineto{\pgfqpoint{3.298967in}{1.889189in}}%
\pgfpathlineto{\pgfqpoint{3.302798in}{1.901330in}}%
\pgfpathlineto{\pgfqpoint{3.304713in}{1.906599in}}%
\pgfpathlineto{\pgfqpoint{3.306628in}{1.908384in}}%
\pgfpathlineto{\pgfqpoint{3.308543in}{1.908559in}}%
\pgfpathlineto{\pgfqpoint{3.314289in}{1.922544in}}%
\pgfpathlineto{\pgfqpoint{3.318119in}{1.926819in}}%
\pgfpathlineto{\pgfqpoint{3.320034in}{1.928998in}}%
\pgfpathlineto{\pgfqpoint{3.321949in}{1.934683in}}%
\pgfpathlineto{\pgfqpoint{3.323865in}{1.935390in}}%
\pgfpathlineto{\pgfqpoint{3.327695in}{1.939720in}}%
\pgfpathlineto{\pgfqpoint{3.329610in}{1.943972in}}%
\pgfpathlineto{\pgfqpoint{3.331525in}{1.945237in}}%
\pgfpathlineto{\pgfqpoint{3.333440in}{1.950978in}}%
\pgfpathlineto{\pgfqpoint{3.337271in}{1.951858in}}%
\pgfpathlineto{\pgfqpoint{3.339186in}{1.957205in}}%
\pgfpathlineto{\pgfqpoint{3.343016in}{1.960231in}}%
\pgfpathlineto{\pgfqpoint{3.346847in}{1.960907in}}%
\pgfpathlineto{\pgfqpoint{3.356423in}{1.965718in}}%
\pgfpathlineto{\pgfqpoint{3.358338in}{1.965798in}}%
\pgfpathlineto{\pgfqpoint{3.362168in}{1.978453in}}%
\pgfpathlineto{\pgfqpoint{3.364083in}{1.979786in}}%
\pgfpathlineto{\pgfqpoint{3.365998in}{1.985538in}}%
\pgfpathlineto{\pgfqpoint{3.371744in}{1.987922in}}%
\pgfpathlineto{\pgfqpoint{3.383235in}{2.006522in}}%
\pgfpathlineto{\pgfqpoint{3.385150in}{2.007256in}}%
\pgfpathlineto{\pgfqpoint{3.387065in}{2.017747in}}%
\pgfpathlineto{\pgfqpoint{3.388980in}{2.021291in}}%
\pgfpathlineto{\pgfqpoint{3.390896in}{2.027552in}}%
\pgfpathlineto{\pgfqpoint{3.396641in}{2.028853in}}%
\pgfpathlineto{\pgfqpoint{3.398556in}{2.037740in}}%
\pgfpathlineto{\pgfqpoint{3.400471in}{2.038667in}}%
\pgfpathlineto{\pgfqpoint{3.402387in}{2.043369in}}%
\pgfpathlineto{\pgfqpoint{3.404302in}{2.053644in}}%
\pgfpathlineto{\pgfqpoint{3.410047in}{2.054712in}}%
\pgfpathlineto{\pgfqpoint{3.415793in}{2.065852in}}%
\pgfpathlineto{\pgfqpoint{3.419623in}{2.067651in}}%
\pgfpathlineto{\pgfqpoint{3.423454in}{2.068868in}}%
\pgfpathlineto{\pgfqpoint{3.425369in}{2.075930in}}%
\pgfpathlineto{\pgfqpoint{3.429199in}{2.080056in}}%
\pgfpathlineto{\pgfqpoint{3.431114in}{2.082912in}}%
\pgfpathlineto{\pgfqpoint{3.434945in}{2.083127in}}%
\pgfpathlineto{\pgfqpoint{3.444520in}{2.091692in}}%
\pgfpathlineto{\pgfqpoint{3.444520in}{2.091692in}}%
\pgfusepath{stroke}%
\end{pgfscope}%
\begin{pgfscope}%
\pgfpathrectangle{\pgfqpoint{0.694334in}{0.523557in}}{\pgfqpoint{3.830343in}{1.568135in}}%
\pgfusepath{clip}%
\pgfsetbuttcap%
\pgfsetroundjoin%
\pgfsetlinewidth{1.003750pt}%
\definecolor{currentstroke}{rgb}{0.000000,0.000000,0.000000}%
\pgfsetstrokecolor{currentstroke}%
\pgfsetdash{{3.700000pt}{1.600000pt}}{0.000000pt}%
\pgfpathmoveto{\pgfqpoint{0.694334in}{0.952566in}}%
\pgfpathlineto{\pgfqpoint{0.696249in}{0.952566in}}%
\pgfpathlineto{\pgfqpoint{0.698165in}{0.957919in}}%
\pgfpathlineto{\pgfqpoint{0.700080in}{0.958825in}}%
\pgfpathlineto{\pgfqpoint{0.701995in}{0.962216in}}%
\pgfpathlineto{\pgfqpoint{0.709656in}{0.965554in}}%
\pgfpathlineto{\pgfqpoint{0.715401in}{0.970666in}}%
\pgfpathlineto{\pgfqpoint{0.723062in}{0.972334in}}%
\pgfpathlineto{\pgfqpoint{0.724977in}{0.974842in}}%
\pgfpathlineto{\pgfqpoint{0.746044in}{0.982177in}}%
\pgfpathlineto{\pgfqpoint{0.749874in}{0.983962in}}%
\pgfpathlineto{\pgfqpoint{0.763280in}{0.991468in}}%
\pgfpathlineto{\pgfqpoint{0.765196in}{0.992174in}}%
\pgfpathlineto{\pgfqpoint{0.769026in}{0.995062in}}%
\pgfpathlineto{\pgfqpoint{0.776687in}{1.002161in}}%
\pgfpathlineto{\pgfqpoint{0.782432in}{1.005299in}}%
\pgfpathlineto{\pgfqpoint{0.784347in}{1.008602in}}%
\pgfpathlineto{\pgfqpoint{0.790093in}{1.010616in}}%
\pgfpathlineto{\pgfqpoint{0.792008in}{1.012727in}}%
\pgfpathlineto{\pgfqpoint{0.793923in}{1.012818in}}%
\pgfpathlineto{\pgfqpoint{0.797754in}{1.015235in}}%
\pgfpathlineto{\pgfqpoint{0.807329in}{1.020361in}}%
\pgfpathlineto{\pgfqpoint{0.809245in}{1.022811in}}%
\pgfpathlineto{\pgfqpoint{0.811160in}{1.029692in}}%
\pgfpathlineto{\pgfqpoint{0.813075in}{1.030857in}}%
\pgfpathlineto{\pgfqpoint{0.814990in}{1.033431in}}%
\pgfpathlineto{\pgfqpoint{0.818820in}{1.034668in}}%
\pgfpathlineto{\pgfqpoint{0.820736in}{1.039143in}}%
\pgfpathlineto{\pgfqpoint{0.824566in}{1.040734in}}%
\pgfpathlineto{\pgfqpoint{0.826481in}{1.061681in}}%
\pgfpathlineto{\pgfqpoint{0.828396in}{1.061922in}}%
\pgfpathlineto{\pgfqpoint{0.830311in}{1.075054in}}%
\pgfpathlineto{\pgfqpoint{0.834142in}{1.075709in}}%
\pgfpathlineto{\pgfqpoint{0.836057in}{1.079231in}}%
\pgfpathlineto{\pgfqpoint{0.837972in}{1.089506in}}%
\pgfpathlineto{\pgfqpoint{0.839887in}{1.090663in}}%
\pgfpathlineto{\pgfqpoint{0.841803in}{1.108535in}}%
\pgfpathlineto{\pgfqpoint{0.843718in}{1.108810in}}%
\pgfpathlineto{\pgfqpoint{0.847548in}{1.134201in}}%
\pgfpathlineto{\pgfqpoint{0.853294in}{1.139675in}}%
\pgfpathlineto{\pgfqpoint{0.855209in}{1.154755in}}%
\pgfpathlineto{\pgfqpoint{0.859039in}{1.157215in}}%
\pgfpathlineto{\pgfqpoint{0.860954in}{1.162580in}}%
\pgfpathlineto{\pgfqpoint{0.872445in}{1.167237in}}%
\pgfpathlineto{\pgfqpoint{0.880106in}{1.169321in}}%
\pgfpathlineto{\pgfqpoint{0.883936in}{1.170456in}}%
\pgfpathlineto{\pgfqpoint{0.889682in}{1.172294in}}%
\pgfpathlineto{\pgfqpoint{0.912664in}{1.177606in}}%
\pgfpathlineto{\pgfqpoint{0.927985in}{1.178862in}}%
\pgfpathlineto{\pgfqpoint{0.937561in}{1.179323in}}%
\pgfpathlineto{\pgfqpoint{0.954798in}{1.180544in}}%
\pgfpathlineto{\pgfqpoint{0.968204in}{1.181617in}}%
\pgfpathlineto{\pgfqpoint{1.016083in}{1.184414in}}%
\pgfpathlineto{\pgfqpoint{1.044811in}{1.185474in}}%
\pgfpathlineto{\pgfqpoint{1.098435in}{1.187150in}}%
\pgfpathlineto{\pgfqpoint{1.150145in}{1.188503in}}%
\pgfpathlineto{\pgfqpoint{1.278462in}{1.192407in}}%
\pgfpathlineto{\pgfqpoint{1.322511in}{1.193668in}}%
\pgfpathlineto{\pgfqpoint{1.343577in}{1.194252in}}%
\pgfpathlineto{\pgfqpoint{1.385711in}{1.195249in}}%
\pgfpathlineto{\pgfqpoint{1.402948in}{1.195853in}}%
\pgfpathlineto{\pgfqpoint{1.456573in}{1.197721in}}%
\pgfpathlineto{\pgfqpoint{1.506367in}{1.198858in}}%
\pgfpathlineto{\pgfqpoint{1.527434in}{1.199385in}}%
\pgfpathlineto{\pgfqpoint{1.701714in}{1.208297in}}%
\pgfpathlineto{\pgfqpoint{1.724697in}{1.209968in}}%
\pgfpathlineto{\pgfqpoint{1.749594in}{1.211501in}}%
\pgfpathlineto{\pgfqpoint{1.780236in}{1.212343in}}%
\pgfpathlineto{\pgfqpoint{1.816625in}{1.213635in}}%
\pgfpathlineto{\pgfqpoint{1.847267in}{1.215029in}}%
\pgfpathlineto{\pgfqpoint{1.885571in}{1.217134in}}%
\pgfpathlineto{\pgfqpoint{1.895147in}{1.218437in}}%
\pgfpathlineto{\pgfqpoint{1.910468in}{1.219812in}}%
\pgfpathlineto{\pgfqpoint{1.937281in}{1.221950in}}%
\pgfpathlineto{\pgfqpoint{1.956432in}{1.223318in}}%
\pgfpathlineto{\pgfqpoint{1.960263in}{1.224104in}}%
\pgfpathlineto{\pgfqpoint{1.990905in}{1.226028in}}%
\pgfpathlineto{\pgfqpoint{2.059852in}{1.233316in}}%
\pgfpathlineto{\pgfqpoint{2.086664in}{1.237475in}}%
\pgfpathlineto{\pgfqpoint{2.094325in}{1.237734in}}%
\pgfpathlineto{\pgfqpoint{2.098155in}{1.239175in}}%
\pgfpathlineto{\pgfqpoint{2.117307in}{1.240677in}}%
\pgfpathlineto{\pgfqpoint{2.128798in}{1.241915in}}%
\pgfpathlineto{\pgfqpoint{2.165186in}{1.243918in}}%
\pgfpathlineto{\pgfqpoint{2.172847in}{1.245409in}}%
\pgfpathlineto{\pgfqpoint{2.178592in}{1.246121in}}%
\pgfpathlineto{\pgfqpoint{2.197744in}{1.247171in}}%
\pgfpathlineto{\pgfqpoint{2.211150in}{1.248907in}}%
\pgfpathlineto{\pgfqpoint{2.259029in}{1.253329in}}%
\pgfpathlineto{\pgfqpoint{2.270520in}{1.254793in}}%
\pgfpathlineto{\pgfqpoint{2.283927in}{1.255761in}}%
\pgfpathlineto{\pgfqpoint{2.293502in}{1.256459in}}%
\pgfpathlineto{\pgfqpoint{2.314569in}{1.257834in}}%
\pgfpathlineto{\pgfqpoint{2.343297in}{1.260937in}}%
\pgfpathlineto{\pgfqpoint{2.358618in}{1.262464in}}%
\pgfpathlineto{\pgfqpoint{2.364364in}{1.263110in}}%
\pgfpathlineto{\pgfqpoint{2.387346in}{1.265512in}}%
\pgfpathlineto{\pgfqpoint{2.417989in}{1.269481in}}%
\pgfpathlineto{\pgfqpoint{2.421819in}{1.270780in}}%
\pgfpathlineto{\pgfqpoint{2.442886in}{1.274126in}}%
\pgfpathlineto{\pgfqpoint{2.467783in}{1.276014in}}%
\pgfpathlineto{\pgfqpoint{2.479274in}{1.277730in}}%
\pgfpathlineto{\pgfqpoint{2.492680in}{1.278450in}}%
\pgfpathlineto{\pgfqpoint{2.534814in}{1.282654in}}%
\pgfpathlineto{\pgfqpoint{2.546305in}{1.283924in}}%
\pgfpathlineto{\pgfqpoint{2.555881in}{1.285952in}}%
\pgfpathlineto{\pgfqpoint{2.559711in}{1.287857in}}%
\pgfpathlineto{\pgfqpoint{2.569287in}{1.289242in}}%
\pgfpathlineto{\pgfqpoint{2.657385in}{1.303381in}}%
\pgfpathlineto{\pgfqpoint{2.674622in}{1.306454in}}%
\pgfpathlineto{\pgfqpoint{2.728246in}{1.317081in}}%
\pgfpathlineto{\pgfqpoint{2.732077in}{1.318718in}}%
\pgfpathlineto{\pgfqpoint{2.737822in}{1.319657in}}%
\pgfpathlineto{\pgfqpoint{2.741653in}{1.320825in}}%
\pgfpathlineto{\pgfqpoint{2.747398in}{1.321165in}}%
\pgfpathlineto{\pgfqpoint{2.751228in}{1.322635in}}%
\pgfpathlineto{\pgfqpoint{2.772295in}{1.328333in}}%
\pgfpathlineto{\pgfqpoint{2.776126in}{1.331542in}}%
\pgfpathlineto{\pgfqpoint{2.783786in}{1.333502in}}%
\pgfpathlineto{\pgfqpoint{2.789532in}{1.334197in}}%
\pgfpathlineto{\pgfqpoint{2.795277in}{1.336743in}}%
\pgfpathlineto{\pgfqpoint{2.799108in}{1.337574in}}%
\pgfpathlineto{\pgfqpoint{2.806768in}{1.340135in}}%
\pgfpathlineto{\pgfqpoint{2.837411in}{1.348552in}}%
\pgfpathlineto{\pgfqpoint{2.845072in}{1.349362in}}%
\pgfpathlineto{\pgfqpoint{2.850817in}{1.350086in}}%
\pgfpathlineto{\pgfqpoint{2.858478in}{1.350896in}}%
\pgfpathlineto{\pgfqpoint{2.869969in}{1.355782in}}%
\pgfpathlineto{\pgfqpoint{2.887206in}{1.358386in}}%
\pgfpathlineto{\pgfqpoint{2.889121in}{1.360298in}}%
\pgfpathlineto{\pgfqpoint{2.898697in}{1.361966in}}%
\pgfpathlineto{\pgfqpoint{2.902527in}{1.364327in}}%
\pgfpathlineto{\pgfqpoint{2.906357in}{1.364503in}}%
\pgfpathlineto{\pgfqpoint{2.908272in}{1.366604in}}%
\pgfpathlineto{\pgfqpoint{2.912103in}{1.366909in}}%
\pgfpathlineto{\pgfqpoint{2.917848in}{1.370614in}}%
\pgfpathlineto{\pgfqpoint{2.937000in}{1.381204in}}%
\pgfpathlineto{\pgfqpoint{2.940830in}{1.383931in}}%
\pgfpathlineto{\pgfqpoint{2.950406in}{1.386369in}}%
\pgfpathlineto{\pgfqpoint{2.956152in}{1.387362in}}%
\pgfpathlineto{\pgfqpoint{2.961897in}{1.390792in}}%
\pgfpathlineto{\pgfqpoint{2.969558in}{1.392785in}}%
\pgfpathlineto{\pgfqpoint{2.979134in}{1.395773in}}%
\pgfpathlineto{\pgfqpoint{2.981049in}{1.395820in}}%
\pgfpathlineto{\pgfqpoint{2.988710in}{1.399544in}}%
\pgfpathlineto{\pgfqpoint{2.990625in}{1.400047in}}%
\pgfpathlineto{\pgfqpoint{2.994455in}{1.403675in}}%
\pgfpathlineto{\pgfqpoint{3.000201in}{1.405145in}}%
\pgfpathlineto{\pgfqpoint{3.002116in}{1.406641in}}%
\pgfpathlineto{\pgfqpoint{3.004031in}{1.410120in}}%
\pgfpathlineto{\pgfqpoint{3.009777in}{1.411774in}}%
\pgfpathlineto{\pgfqpoint{3.023183in}{1.421015in}}%
\pgfpathlineto{\pgfqpoint{3.027013in}{1.423138in}}%
\pgfpathlineto{\pgfqpoint{3.030843in}{1.425152in}}%
\pgfpathlineto{\pgfqpoint{3.034674in}{1.428117in}}%
\pgfpathlineto{\pgfqpoint{3.059571in}{1.431490in}}%
\pgfpathlineto{\pgfqpoint{3.072977in}{1.435911in}}%
\pgfpathlineto{\pgfqpoint{3.082553in}{1.448599in}}%
\pgfpathlineto{\pgfqpoint{3.086383in}{1.451431in}}%
\pgfpathlineto{\pgfqpoint{3.090214in}{1.452321in}}%
\pgfpathlineto{\pgfqpoint{3.094044in}{1.457416in}}%
\pgfpathlineto{\pgfqpoint{3.099790in}{1.459059in}}%
\pgfpathlineto{\pgfqpoint{3.107450in}{1.461604in}}%
\pgfpathlineto{\pgfqpoint{3.109365in}{1.463697in}}%
\pgfpathlineto{\pgfqpoint{3.113196in}{1.464741in}}%
\pgfpathlineto{\pgfqpoint{3.117026in}{1.467416in}}%
\pgfpathlineto{\pgfqpoint{3.120856in}{1.468135in}}%
\pgfpathlineto{\pgfqpoint{3.130432in}{1.469250in}}%
\pgfpathlineto{\pgfqpoint{3.141923in}{1.474497in}}%
\pgfpathlineto{\pgfqpoint{3.143839in}{1.474597in}}%
\pgfpathlineto{\pgfqpoint{3.147669in}{1.477088in}}%
\pgfpathlineto{\pgfqpoint{3.151499in}{1.477514in}}%
\pgfpathlineto{\pgfqpoint{3.155330in}{1.479009in}}%
\pgfpathlineto{\pgfqpoint{3.157245in}{1.486095in}}%
\pgfpathlineto{\pgfqpoint{3.161075in}{1.488572in}}%
\pgfpathlineto{\pgfqpoint{3.166821in}{1.489899in}}%
\pgfpathlineto{\pgfqpoint{3.170651in}{1.490989in}}%
\pgfpathlineto{\pgfqpoint{3.174481in}{1.491340in}}%
\pgfpathlineto{\pgfqpoint{3.176396in}{1.496027in}}%
\pgfpathlineto{\pgfqpoint{3.178312in}{1.496729in}}%
\pgfpathlineto{\pgfqpoint{3.182142in}{1.502819in}}%
\pgfpathlineto{\pgfqpoint{3.187887in}{1.503878in}}%
\pgfpathlineto{\pgfqpoint{3.193633in}{1.509042in}}%
\pgfpathlineto{\pgfqpoint{3.197463in}{1.512025in}}%
\pgfpathlineto{\pgfqpoint{3.205124in}{1.512559in}}%
\pgfpathlineto{\pgfqpoint{3.212785in}{1.515968in}}%
\pgfpathlineto{\pgfqpoint{3.214700in}{1.519950in}}%
\pgfpathlineto{\pgfqpoint{3.218530in}{1.520193in}}%
\pgfpathlineto{\pgfqpoint{3.224276in}{1.524797in}}%
\pgfpathlineto{\pgfqpoint{3.228106in}{1.526374in}}%
\pgfpathlineto{\pgfqpoint{3.230021in}{1.532956in}}%
\pgfpathlineto{\pgfqpoint{3.249173in}{1.538496in}}%
\pgfpathlineto{\pgfqpoint{3.253003in}{1.543631in}}%
\pgfpathlineto{\pgfqpoint{3.254918in}{1.545359in}}%
\pgfpathlineto{\pgfqpoint{3.256834in}{1.545574in}}%
\pgfpathlineto{\pgfqpoint{3.268325in}{1.554710in}}%
\pgfpathlineto{\pgfqpoint{3.274070in}{1.557607in}}%
\pgfpathlineto{\pgfqpoint{3.277901in}{1.561056in}}%
\pgfpathlineto{\pgfqpoint{3.281731in}{1.562233in}}%
\pgfpathlineto{\pgfqpoint{3.283646in}{1.565334in}}%
\pgfpathlineto{\pgfqpoint{3.297052in}{1.570845in}}%
\pgfpathlineto{\pgfqpoint{3.306628in}{1.583603in}}%
\pgfpathlineto{\pgfqpoint{3.327695in}{1.597976in}}%
\pgfpathlineto{\pgfqpoint{3.331525in}{1.599278in}}%
\pgfpathlineto{\pgfqpoint{3.335356in}{1.603480in}}%
\pgfpathlineto{\pgfqpoint{3.337271in}{1.611115in}}%
\pgfpathlineto{\pgfqpoint{3.343016in}{1.613998in}}%
\pgfpathlineto{\pgfqpoint{3.344932in}{1.617157in}}%
\pgfpathlineto{\pgfqpoint{3.346847in}{1.624987in}}%
\pgfpathlineto{\pgfqpoint{3.350677in}{1.630233in}}%
\pgfpathlineto{\pgfqpoint{3.352592in}{1.630587in}}%
\pgfpathlineto{\pgfqpoint{3.356423in}{1.637187in}}%
\pgfpathlineto{\pgfqpoint{3.358338in}{1.651126in}}%
\pgfpathlineto{\pgfqpoint{3.360253in}{1.654260in}}%
\pgfpathlineto{\pgfqpoint{3.364083in}{1.655371in}}%
\pgfpathlineto{\pgfqpoint{3.365998in}{1.656983in}}%
\pgfpathlineto{\pgfqpoint{3.367914in}{1.660160in}}%
\pgfpathlineto{\pgfqpoint{3.369829in}{1.671980in}}%
\pgfpathlineto{\pgfqpoint{3.373659in}{1.679077in}}%
\pgfpathlineto{\pgfqpoint{3.375574in}{1.679360in}}%
\pgfpathlineto{\pgfqpoint{3.379405in}{1.681817in}}%
\pgfpathlineto{\pgfqpoint{3.381320in}{1.682294in}}%
\pgfpathlineto{\pgfqpoint{3.383235in}{1.699531in}}%
\pgfpathlineto{\pgfqpoint{3.387065in}{1.702732in}}%
\pgfpathlineto{\pgfqpoint{3.388980in}{1.702892in}}%
\pgfpathlineto{\pgfqpoint{3.413878in}{1.753977in}}%
\pgfpathlineto{\pgfqpoint{3.417708in}{1.759102in}}%
\pgfpathlineto{\pgfqpoint{3.423454in}{1.761232in}}%
\pgfpathlineto{\pgfqpoint{3.425369in}{1.765481in}}%
\pgfpathlineto{\pgfqpoint{3.431114in}{1.766511in}}%
\pgfpathlineto{\pgfqpoint{3.436860in}{1.769612in}}%
\pgfpathlineto{\pgfqpoint{3.438775in}{1.772654in}}%
\pgfpathlineto{\pgfqpoint{3.440690in}{1.772846in}}%
\pgfpathlineto{\pgfqpoint{3.442605in}{1.774449in}}%
\pgfpathlineto{\pgfqpoint{3.444520in}{1.779298in}}%
\pgfpathlineto{\pgfqpoint{3.448351in}{1.780846in}}%
\pgfpathlineto{\pgfqpoint{3.450266in}{1.783666in}}%
\pgfpathlineto{\pgfqpoint{3.452181in}{1.784305in}}%
\pgfpathlineto{\pgfqpoint{3.454096in}{1.789046in}}%
\pgfpathlineto{\pgfqpoint{3.456011in}{1.790352in}}%
\pgfpathlineto{\pgfqpoint{3.459842in}{1.796887in}}%
\pgfpathlineto{\pgfqpoint{3.463672in}{1.803031in}}%
\pgfpathlineto{\pgfqpoint{3.465587in}{1.807958in}}%
\pgfpathlineto{\pgfqpoint{3.467502in}{1.809305in}}%
\pgfpathlineto{\pgfqpoint{3.469418in}{1.813529in}}%
\pgfpathlineto{\pgfqpoint{3.473248in}{1.814522in}}%
\pgfpathlineto{\pgfqpoint{3.475163in}{1.815256in}}%
\pgfpathlineto{\pgfqpoint{3.478994in}{1.826562in}}%
\pgfpathlineto{\pgfqpoint{3.480909in}{1.827653in}}%
\pgfpathlineto{\pgfqpoint{3.482824in}{1.830383in}}%
\pgfpathlineto{\pgfqpoint{3.484739in}{1.831147in}}%
\pgfpathlineto{\pgfqpoint{3.486654in}{1.836035in}}%
\pgfpathlineto{\pgfqpoint{3.492400in}{1.838841in}}%
\pgfpathlineto{\pgfqpoint{3.494315in}{1.847357in}}%
\pgfpathlineto{\pgfqpoint{3.496230in}{1.867271in}}%
\pgfpathlineto{\pgfqpoint{3.500060in}{1.873251in}}%
\pgfpathlineto{\pgfqpoint{3.505806in}{1.885980in}}%
\pgfpathlineto{\pgfqpoint{3.509636in}{1.886329in}}%
\pgfpathlineto{\pgfqpoint{3.511551in}{1.892145in}}%
\pgfpathlineto{\pgfqpoint{3.513467in}{1.893195in}}%
\pgfpathlineto{\pgfqpoint{3.517297in}{1.908462in}}%
\pgfpathlineto{\pgfqpoint{3.519212in}{1.908813in}}%
\pgfpathlineto{\pgfqpoint{3.523042in}{1.918683in}}%
\pgfpathlineto{\pgfqpoint{3.524958in}{1.919524in}}%
\pgfpathlineto{\pgfqpoint{3.526873in}{1.926772in}}%
\pgfpathlineto{\pgfqpoint{3.528788in}{1.928865in}}%
\pgfpathlineto{\pgfqpoint{3.534533in}{1.946879in}}%
\pgfpathlineto{\pgfqpoint{3.538364in}{1.959166in}}%
\pgfpathlineto{\pgfqpoint{3.540279in}{1.960611in}}%
\pgfpathlineto{\pgfqpoint{3.542194in}{1.973166in}}%
\pgfpathlineto{\pgfqpoint{3.547940in}{1.983847in}}%
\pgfpathlineto{\pgfqpoint{3.549855in}{1.998713in}}%
\pgfpathlineto{\pgfqpoint{3.555600in}{2.010025in}}%
\pgfpathlineto{\pgfqpoint{3.557516in}{2.052237in}}%
\pgfpathlineto{\pgfqpoint{3.561346in}{2.091692in}}%
\pgfpathlineto{\pgfqpoint{3.561346in}{2.091692in}}%
\pgfusepath{stroke}%
\end{pgfscope}%
\begin{pgfscope}%
\pgfsetrectcap%
\pgfsetmiterjoin%
\pgfsetlinewidth{0.803000pt}%
\definecolor{currentstroke}{rgb}{0.000000,0.000000,0.000000}%
\pgfsetstrokecolor{currentstroke}%
\pgfsetdash{}{0pt}%
\pgfpathmoveto{\pgfqpoint{0.694334in}{0.523557in}}%
\pgfpathlineto{\pgfqpoint{0.694334in}{2.091692in}}%
\pgfusepath{stroke}%
\end{pgfscope}%
\begin{pgfscope}%
\pgfsetrectcap%
\pgfsetmiterjoin%
\pgfsetlinewidth{0.803000pt}%
\definecolor{currentstroke}{rgb}{0.000000,0.000000,0.000000}%
\pgfsetstrokecolor{currentstroke}%
\pgfsetdash{}{0pt}%
\pgfpathmoveto{\pgfqpoint{4.524677in}{0.523557in}}%
\pgfpathlineto{\pgfqpoint{4.524677in}{2.091692in}}%
\pgfusepath{stroke}%
\end{pgfscope}%
\begin{pgfscope}%
\pgfsetrectcap%
\pgfsetmiterjoin%
\pgfsetlinewidth{0.803000pt}%
\definecolor{currentstroke}{rgb}{0.000000,0.000000,0.000000}%
\pgfsetstrokecolor{currentstroke}%
\pgfsetdash{}{0pt}%
\pgfpathmoveto{\pgfqpoint{0.694334in}{0.523557in}}%
\pgfpathlineto{\pgfqpoint{4.524677in}{0.523557in}}%
\pgfusepath{stroke}%
\end{pgfscope}%
\begin{pgfscope}%
\pgfsetrectcap%
\pgfsetmiterjoin%
\pgfsetlinewidth{0.803000pt}%
\definecolor{currentstroke}{rgb}{0.000000,0.000000,0.000000}%
\pgfsetstrokecolor{currentstroke}%
\pgfsetdash{}{0pt}%
\pgfpathmoveto{\pgfqpoint{0.694334in}{2.091692in}}%
\pgfpathlineto{\pgfqpoint{4.524677in}{2.091692in}}%
\pgfusepath{stroke}%
\end{pgfscope}%
\begin{pgfscope}%
\pgfsetrectcap%
\pgfsetroundjoin%
\pgfsetlinewidth{1.003750pt}%
\definecolor{currentstroke}{rgb}{0.878431,0.878431,0.815686}%
\pgfsetstrokecolor{currentstroke}%
\pgfsetdash{}{0pt}%
\pgfpathmoveto{\pgfqpoint{3.691785in}{1.779066in}}%
\pgfpathlineto{\pgfqpoint{3.914007in}{1.779066in}}%
\pgfusepath{stroke}%
\end{pgfscope}%
\begin{pgfscope}%
\definecolor{textcolor}{rgb}{0.000000,0.000000,0.000000}%
\pgfsetstrokecolor{textcolor}%
\pgfsetfillcolor{textcolor}%
\pgftext[x=3.936230in,y=1.740177in,left,base]{\color{textcolor}\rmfamily\fontsize{8.000000}{9.600000}\selectfont T.+CPU1}%
\end{pgfscope}%
\begin{pgfscope}%
\pgfsetrectcap%
\pgfsetroundjoin%
\pgfsetlinewidth{1.003750pt}%
\definecolor{currentstroke}{rgb}{0.564706,0.564706,1.000000}%
\pgfsetstrokecolor{currentstroke}%
\pgfsetdash{}{0pt}%
\pgfpathmoveto{\pgfqpoint{3.691785in}{1.635244in}}%
\pgfpathlineto{\pgfqpoint{3.914007in}{1.635244in}}%
\pgfusepath{stroke}%
\end{pgfscope}%
\begin{pgfscope}%
\definecolor{textcolor}{rgb}{0.000000,0.000000,0.000000}%
\pgfsetstrokecolor{textcolor}%
\pgfsetfillcolor{textcolor}%
\pgftext[x=3.936230in,y=1.596355in,left,base]{\color{textcolor}\rmfamily\fontsize{8.000000}{9.600000}\selectfont P4+CPU1}%
\end{pgfscope}%
\begin{pgfscope}%
\pgfsetbuttcap%
\pgfsetroundjoin%
\pgfsetlinewidth{1.003750pt}%
\definecolor{currentstroke}{rgb}{0.564706,0.564706,1.000000}%
\pgfsetstrokecolor{currentstroke}%
\pgfsetdash{{1.000000pt}{1.650000pt}}{0.000000pt}%
\pgfpathmoveto{\pgfqpoint{3.691785in}{1.491422in}}%
\pgfpathlineto{\pgfqpoint{3.914007in}{1.491422in}}%
\pgfusepath{stroke}%
\end{pgfscope}%
\begin{pgfscope}%
\definecolor{textcolor}{rgb}{0.000000,0.000000,0.000000}%
\pgfsetstrokecolor{textcolor}%
\pgfsetfillcolor{textcolor}%
\pgftext[x=3.936230in,y=1.452533in,left,base]{\color{textcolor}\rmfamily\fontsize{8.000000}{9.600000}\selectfont P4+CPU8}%
\end{pgfscope}%
\begin{pgfscope}%
\pgfsetbuttcap%
\pgfsetroundjoin%
\pgfsetlinewidth{1.003750pt}%
\definecolor{currentstroke}{rgb}{0.564706,0.564706,1.000000}%
\pgfsetstrokecolor{currentstroke}%
\pgfsetdash{{3.700000pt}{1.600000pt}}{0.000000pt}%
\pgfpathmoveto{\pgfqpoint{3.691785in}{1.347600in}}%
\pgfpathlineto{\pgfqpoint{3.914007in}{1.347600in}}%
\pgfusepath{stroke}%
\end{pgfscope}%
\begin{pgfscope}%
\definecolor{textcolor}{rgb}{0.000000,0.000000,0.000000}%
\pgfsetstrokecolor{textcolor}%
\pgfsetfillcolor{textcolor}%
\pgftext[x=3.936230in,y=1.308711in,left,base]{\color{textcolor}\rmfamily\fontsize{8.000000}{9.600000}\selectfont P4+GPU}%
\end{pgfscope}%
\begin{pgfscope}%
\pgfsetrectcap%
\pgfsetroundjoin%
\pgfsetlinewidth{1.003750pt}%
\definecolor{currentstroke}{rgb}{0.811765,0.125490,0.125490}%
\pgfsetstrokecolor{currentstroke}%
\pgfsetdash{}{0pt}%
\pgfpathmoveto{\pgfqpoint{3.691785in}{1.203778in}}%
\pgfpathlineto{\pgfqpoint{3.914007in}{1.203778in}}%
\pgfusepath{stroke}%
\end{pgfscope}%
\begin{pgfscope}%
\definecolor{textcolor}{rgb}{0.000000,0.000000,0.000000}%
\pgfsetstrokecolor{textcolor}%
\pgfsetfillcolor{textcolor}%
\pgftext[x=3.936230in,y=1.164889in,left,base]{\color{textcolor}\rmfamily\fontsize{8.000000}{9.600000}\selectfont miniC2D}%
\end{pgfscope}%
\begin{pgfscope}%
\pgfsetbuttcap%
\pgfsetroundjoin%
\pgfsetlinewidth{1.003750pt}%
\definecolor{currentstroke}{rgb}{0.811765,0.125490,0.125490}%
\pgfsetstrokecolor{currentstroke}%
\pgfsetdash{{3.700000pt}{1.600000pt}}{0.000000pt}%
\pgfpathmoveto{\pgfqpoint{3.691785in}{1.059956in}}%
\pgfpathlineto{\pgfqpoint{3.914007in}{1.059956in}}%
\pgfusepath{stroke}%
\end{pgfscope}%
\begin{pgfscope}%
\definecolor{textcolor}{rgb}{0.000000,0.000000,0.000000}%
\pgfsetstrokecolor{textcolor}%
\pgfsetfillcolor{textcolor}%
\pgftext[x=3.936230in,y=1.021067in,left,base]{\color{textcolor}\rmfamily\fontsize{8.000000}{9.600000}\selectfont d4}%
\end{pgfscope}%
\begin{pgfscope}%
\pgfsetbuttcap%
\pgfsetroundjoin%
\pgfsetlinewidth{1.003750pt}%
\definecolor{currentstroke}{rgb}{0.811765,0.125490,0.125490}%
\pgfsetstrokecolor{currentstroke}%
\pgfsetdash{{1.000000pt}{1.650000pt}}{0.000000pt}%
\pgfpathmoveto{\pgfqpoint{3.691785in}{0.916134in}}%
\pgfpathlineto{\pgfqpoint{3.914007in}{0.916134in}}%
\pgfusepath{stroke}%
\end{pgfscope}%
\begin{pgfscope}%
\definecolor{textcolor}{rgb}{0.000000,0.000000,0.000000}%
\pgfsetstrokecolor{textcolor}%
\pgfsetfillcolor{textcolor}%
\pgftext[x=3.936230in,y=0.877245in,left,base]{\color{textcolor}\rmfamily\fontsize{8.000000}{9.600000}\selectfont cachet}%
\end{pgfscope}%
\begin{pgfscope}%
\pgfsetrectcap%
\pgfsetroundjoin%
\pgfsetlinewidth{1.003750pt}%
\definecolor{currentstroke}{rgb}{0.062745,0.000000,0.062745}%
\pgfsetstrokecolor{currentstroke}%
\pgfsetdash{}{0pt}%
\pgfpathmoveto{\pgfqpoint{3.691785in}{0.772312in}}%
\pgfpathlineto{\pgfqpoint{3.914007in}{0.772312in}}%
\pgfusepath{stroke}%
\end{pgfscope}%
\begin{pgfscope}%
\definecolor{textcolor}{rgb}{0.000000,0.000000,0.000000}%
\pgfsetstrokecolor{textcolor}%
\pgfsetfillcolor{textcolor}%
\pgftext[x=3.936230in,y=0.733423in,left,base]{\color{textcolor}\rmfamily\fontsize{8.000000}{9.600000}\selectfont ADDMC}%
\end{pgfscope}%
\begin{pgfscope}%
\pgfsetbuttcap%
\pgfsetroundjoin%
\pgfsetlinewidth{1.003750pt}%
\definecolor{currentstroke}{rgb}{0.000000,0.000000,0.000000}%
\pgfsetstrokecolor{currentstroke}%
\pgfsetdash{{3.700000pt}{1.600000pt}}{0.000000pt}%
\pgfpathmoveto{\pgfqpoint{3.691785in}{0.628490in}}%
\pgfpathlineto{\pgfqpoint{3.914007in}{0.628490in}}%
\pgfusepath{stroke}%
\end{pgfscope}%
\begin{pgfscope}%
\definecolor{textcolor}{rgb}{0.000000,0.000000,0.000000}%
\pgfsetstrokecolor{textcolor}%
\pgfsetfillcolor{textcolor}%
\pgftext[x=3.936230in,y=0.589601in,left,base]{\color{textcolor}\rmfamily\fontsize{8.000000}{9.600000}\selectfont gpusat2}%
\end{pgfscope}%
\end{pgfpicture}%
\makeatother%
\endgroup%

\vspace*{-0.9cm}
\caption{\label{fig:parallel:comparison} A cactus plot of the number of benchmarks solved by various counters, without (above) and with (below) the \tool{pmc-eq}  preprocessor.}
\end{center}
\vspace*{-0.8cm}
\end{figure}

\begin{table}[t]
  \caption{\label{tab:comparison} The numbers of benchmarks solved by each counter fastest and in total after 1000 seconds, and the PAR-2 score.}
  \vspace*{0.1cm}
  \centering
  \begin{tabular}{l||r|r|r||r|r|r|}
  & \multicolumn{3}{c||}{Without preprocessing} & \multicolumn{3}{c|}{With \tool{pmc-eq} preprocessing} \\
  & \# Fastest & \# Solved & PAR-2 & \# Fastest & \# Solved & PAR-2\\ \hline 
\pkg{T.}+\pkg{CPU1} & 0 & 1151 & 1640803. & 0 & 1453 & 811199.\\ \hline 
\pkg{P4}+\pkg{CPU1} & 43 & 1164 & 1562474. & 80 & 1465 & 782475.\\ \hline 
\pkg{P4}+\pkg{CPU8} & 48 & 1185 & 1500968. & 58 & 1481 & 748745.\\ \hline 
\pkg{P4}+\pkg{GPU} & 60 & 1210 & 1436949. & 38 & 1488 & 722582.\\ \hline 
\tool{miniC2D} & 40 & 1381 & 1131457. & 124 & 1582 & 562259.\\ \hline 
\tool{d4} & 541 & 1508 & 883829. & 436 & 1514 & 723207.\\ \hline 
\tool{cachet} & 230 & 1363 & 1156309. & 207 & 1330 & 1075215.\\ \hline 
\tool{ADDMC} & 184 & 1302 & 1245470. & 33 & 1290 & 1147473.\\ \hline 
\tool{gpusat2} & 36 & 1258 & 1342646. & 24 & 1436 & 831762.\\ \hline 
\tool{DPMC} & 585 & 1279 & 1301363. & 654 & 1427 & 849168.\\ \hline 
\end{tabular}
\end{table}

\subsection{Experiment 3: End-to-End Performance (RQ4 and RQ6)}
Next, we compare \tool{TensorOrder2} with existing state-of-the-art weighted model counters \tool{cachet}, \tool{miniC2D}, \tool{d4}, \tool{ADDMC}, and \tool{gpuSAT2}. We consider \tool{TensorOrder2} using \pkg{P4} as the planner combined with each the three hardware configurations in eager execution mode (\pkg{CPU1}, \pkg{CPU8}, and \pkg{GPU}), along with \pkg{Tamaki}+\pkg{CPU1} as the best non-parallel configuration from \cite{DDV19}. Note that \tool{P4}+\tool{CPU1} still leverages multiple cores in the planning phase. The performance factor from Experiment 2 is used for each \tool{TensorOrder2} configuration.

We run each counter once on each benchmark (both with and without \pkg{pmc-eq} preprocessing) with a timeout of 1000 seconds and record the wall-clock time taken. When preprocessing is used, both the timeout and the recorded time include preprocessing time. For \tool{TensorOrder2}, recorded times include all of Algorithm \ref{alg:wmc}. Results are summarized in Figure \ref{fig:parallel:comparison} and Table \ref{tab:comparison}. 

We observe that \tool{TensorOrder2} is improved by the portfolio planner and, on hard benchmarks, by executing on a multi-core CPU and on a GPU. The flat line at 3 seconds for $\pkg{P4}+\pkg{GPU}$ is caused by overhead from initializing the GPU.

Comparing \tool{TensorOrder2} with the other counters, \tool{TensorOrder2} is competitive without preprocessing but solves fewer benchmarks than all other counters, although \tool{TensorOrder2} (with some configuration) is faster than all other counters on 158 benchmarks before preprocessing. 
We observe that preprocessing significantly boosts \tool{TensorOrder2} relative to other counters, similar to prior observations with \tool{gpusat2} \cite{FHZ19}. \tool{TensorOrder2} solves the third-most preprocessed benchmarks of any solver and has the second-lowest PAR-2 score (notably, outperforming \tool{gpusat2} in both measures). \tool{TensorOrder2} (with some configuration) is faster than all other counters on 176 benchmarks with preprocessing. Since \tool{TensorOrder2} improves the virtual best solver on 151 benchmarks without preprocessing and on 176 benchmarks with preprocessing, we conclude that \tool{TensorOrder2} is useful as part of a portfolio of counters.

\subsection{Experiment 4: Executing on a TPU (RQ5)}
\label{sec:parallel:exp:tpu}
\begin{figure}[t]
\begin{center}
%% Creator: Matplotlib, PGF backend
%%
%% To include the figure in your LaTeX document, write
%%   \input{<filename>.pgf}
%%
%% Make sure the required packages are loaded in your preamble
%%   \usepackage{pgf}
%%
%% and, on pdftex
%%   \usepackage[utf8]{inputenc}\DeclareUnicodeCharacter{2212}{-}
%%
%% or, on luatex and xetex
%%   \usepackage{unicode-math}
%%
%% Figures using additional raster images can only be included by \input if
%% they are in the same directory as the main LaTeX file. For loading figures
%% from other directories you can use the `import` package
%%   \usepackage{import}
%%
%% and then include the figures with
%%   \import{<path to file>}{<filename>.pgf}
%%
%% Matplotlib used the following preamble
%%   \usepackage[utf8x]{inputenc}
%%   \usepackage[T1]{fontenc}
%%
\begingroup%
\makeatletter%
\begin{pgfpicture}%
\pgfpathrectangle{\pgfpointorigin}{\pgfqpoint{6.000000in}{2.800000in}}%
\pgfusepath{use as bounding box, clip}%
\begin{pgfscope}%
\pgfsetbuttcap%
\pgfsetmiterjoin%
\definecolor{currentfill}{rgb}{1.000000,1.000000,1.000000}%
\pgfsetfillcolor{currentfill}%
\pgfsetlinewidth{0.000000pt}%
\definecolor{currentstroke}{rgb}{1.000000,1.000000,1.000000}%
\pgfsetstrokecolor{currentstroke}%
\pgfsetdash{}{0pt}%
\pgfpathmoveto{\pgfqpoint{0.000000in}{0.000000in}}%
\pgfpathlineto{\pgfqpoint{6.000000in}{0.000000in}}%
\pgfpathlineto{\pgfqpoint{6.000000in}{2.800000in}}%
\pgfpathlineto{\pgfqpoint{0.000000in}{2.800000in}}%
\pgfpathclose%
\pgfusepath{fill}%
\end{pgfscope}%
\begin{pgfscope}%
\pgfsetbuttcap%
\pgfsetmiterjoin%
\definecolor{currentfill}{rgb}{1.000000,1.000000,1.000000}%
\pgfsetfillcolor{currentfill}%
\pgfsetlinewidth{0.000000pt}%
\definecolor{currentstroke}{rgb}{0.000000,0.000000,0.000000}%
\pgfsetstrokecolor{currentstroke}%
\pgfsetstrokeopacity{0.000000}%
\pgfsetdash{}{0pt}%
\pgfpathmoveto{\pgfqpoint{0.708220in}{0.535823in}}%
\pgfpathlineto{\pgfqpoint{5.586815in}{0.535823in}}%
\pgfpathlineto{\pgfqpoint{5.586815in}{2.605275in}}%
\pgfpathlineto{\pgfqpoint{0.708220in}{2.605275in}}%
\pgfpathclose%
\pgfusepath{fill}%
\end{pgfscope}%
\begin{pgfscope}%
\pgfsetbuttcap%
\pgfsetroundjoin%
\definecolor{currentfill}{rgb}{0.000000,0.000000,0.000000}%
\pgfsetfillcolor{currentfill}%
\pgfsetlinewidth{0.803000pt}%
\definecolor{currentstroke}{rgb}{0.000000,0.000000,0.000000}%
\pgfsetstrokecolor{currentstroke}%
\pgfsetdash{}{0pt}%
\pgfsys@defobject{currentmarker}{\pgfqpoint{0.000000in}{-0.048611in}}{\pgfqpoint{0.000000in}{0.000000in}}{%
\pgfpathmoveto{\pgfqpoint{0.000000in}{0.000000in}}%
\pgfpathlineto{\pgfqpoint{0.000000in}{-0.048611in}}%
\pgfusepath{stroke,fill}%
}%
\begin{pgfscope}%
\pgfsys@transformshift{1.114769in}{0.535823in}%
\pgfsys@useobject{currentmarker}{}%
\end{pgfscope}%
\end{pgfscope}%
\begin{pgfscope}%
\definecolor{textcolor}{rgb}{0.000000,0.000000,0.000000}%
\pgfsetstrokecolor{textcolor}%
\pgfsetfillcolor{textcolor}%
\pgftext[x=1.114769in,y=0.438600in,,top]{\color{textcolor}\rmfamily\fontsize{9.000000}{10.800000}\selectfont \(\displaystyle {10}\)}%
\end{pgfscope}%
\begin{pgfscope}%
\pgfsetbuttcap%
\pgfsetroundjoin%
\definecolor{currentfill}{rgb}{0.000000,0.000000,0.000000}%
\pgfsetfillcolor{currentfill}%
\pgfsetlinewidth{0.803000pt}%
\definecolor{currentstroke}{rgb}{0.000000,0.000000,0.000000}%
\pgfsetstrokecolor{currentstroke}%
\pgfsetdash{}{0pt}%
\pgfsys@defobject{currentmarker}{\pgfqpoint{0.000000in}{-0.048611in}}{\pgfqpoint{0.000000in}{0.000000in}}{%
\pgfpathmoveto{\pgfqpoint{0.000000in}{0.000000in}}%
\pgfpathlineto{\pgfqpoint{0.000000in}{-0.048611in}}%
\pgfusepath{stroke,fill}%
}%
\begin{pgfscope}%
\pgfsys@transformshift{1.927869in}{0.535823in}%
\pgfsys@useobject{currentmarker}{}%
\end{pgfscope}%
\end{pgfscope}%
\begin{pgfscope}%
\definecolor{textcolor}{rgb}{0.000000,0.000000,0.000000}%
\pgfsetstrokecolor{textcolor}%
\pgfsetfillcolor{textcolor}%
\pgftext[x=1.927869in,y=0.438600in,,top]{\color{textcolor}\rmfamily\fontsize{9.000000}{10.800000}\selectfont \(\displaystyle {12}\)}%
\end{pgfscope}%
\begin{pgfscope}%
\pgfsetbuttcap%
\pgfsetroundjoin%
\definecolor{currentfill}{rgb}{0.000000,0.000000,0.000000}%
\pgfsetfillcolor{currentfill}%
\pgfsetlinewidth{0.803000pt}%
\definecolor{currentstroke}{rgb}{0.000000,0.000000,0.000000}%
\pgfsetstrokecolor{currentstroke}%
\pgfsetdash{}{0pt}%
\pgfsys@defobject{currentmarker}{\pgfqpoint{0.000000in}{-0.048611in}}{\pgfqpoint{0.000000in}{0.000000in}}{%
\pgfpathmoveto{\pgfqpoint{0.000000in}{0.000000in}}%
\pgfpathlineto{\pgfqpoint{0.000000in}{-0.048611in}}%
\pgfusepath{stroke,fill}%
}%
\begin{pgfscope}%
\pgfsys@transformshift{2.740968in}{0.535823in}%
\pgfsys@useobject{currentmarker}{}%
\end{pgfscope}%
\end{pgfscope}%
\begin{pgfscope}%
\definecolor{textcolor}{rgb}{0.000000,0.000000,0.000000}%
\pgfsetstrokecolor{textcolor}%
\pgfsetfillcolor{textcolor}%
\pgftext[x=2.740968in,y=0.438600in,,top]{\color{textcolor}\rmfamily\fontsize{9.000000}{10.800000}\selectfont \(\displaystyle {14}\)}%
\end{pgfscope}%
\begin{pgfscope}%
\pgfsetbuttcap%
\pgfsetroundjoin%
\definecolor{currentfill}{rgb}{0.000000,0.000000,0.000000}%
\pgfsetfillcolor{currentfill}%
\pgfsetlinewidth{0.803000pt}%
\definecolor{currentstroke}{rgb}{0.000000,0.000000,0.000000}%
\pgfsetstrokecolor{currentstroke}%
\pgfsetdash{}{0pt}%
\pgfsys@defobject{currentmarker}{\pgfqpoint{0.000000in}{-0.048611in}}{\pgfqpoint{0.000000in}{0.000000in}}{%
\pgfpathmoveto{\pgfqpoint{0.000000in}{0.000000in}}%
\pgfpathlineto{\pgfqpoint{0.000000in}{-0.048611in}}%
\pgfusepath{stroke,fill}%
}%
\begin{pgfscope}%
\pgfsys@transformshift{3.554067in}{0.535823in}%
\pgfsys@useobject{currentmarker}{}%
\end{pgfscope}%
\end{pgfscope}%
\begin{pgfscope}%
\definecolor{textcolor}{rgb}{0.000000,0.000000,0.000000}%
\pgfsetstrokecolor{textcolor}%
\pgfsetfillcolor{textcolor}%
\pgftext[x=3.554067in,y=0.438600in,,top]{\color{textcolor}\rmfamily\fontsize{9.000000}{10.800000}\selectfont \(\displaystyle {16}\)}%
\end{pgfscope}%
\begin{pgfscope}%
\pgfsetbuttcap%
\pgfsetroundjoin%
\definecolor{currentfill}{rgb}{0.000000,0.000000,0.000000}%
\pgfsetfillcolor{currentfill}%
\pgfsetlinewidth{0.803000pt}%
\definecolor{currentstroke}{rgb}{0.000000,0.000000,0.000000}%
\pgfsetstrokecolor{currentstroke}%
\pgfsetdash{}{0pt}%
\pgfsys@defobject{currentmarker}{\pgfqpoint{0.000000in}{-0.048611in}}{\pgfqpoint{0.000000in}{0.000000in}}{%
\pgfpathmoveto{\pgfqpoint{0.000000in}{0.000000in}}%
\pgfpathlineto{\pgfqpoint{0.000000in}{-0.048611in}}%
\pgfusepath{stroke,fill}%
}%
\begin{pgfscope}%
\pgfsys@transformshift{4.367166in}{0.535823in}%
\pgfsys@useobject{currentmarker}{}%
\end{pgfscope}%
\end{pgfscope}%
\begin{pgfscope}%
\definecolor{textcolor}{rgb}{0.000000,0.000000,0.000000}%
\pgfsetstrokecolor{textcolor}%
\pgfsetfillcolor{textcolor}%
\pgftext[x=4.367166in,y=0.438600in,,top]{\color{textcolor}\rmfamily\fontsize{9.000000}{10.800000}\selectfont \(\displaystyle {18}\)}%
\end{pgfscope}%
\begin{pgfscope}%
\pgfsetbuttcap%
\pgfsetroundjoin%
\definecolor{currentfill}{rgb}{0.000000,0.000000,0.000000}%
\pgfsetfillcolor{currentfill}%
\pgfsetlinewidth{0.803000pt}%
\definecolor{currentstroke}{rgb}{0.000000,0.000000,0.000000}%
\pgfsetstrokecolor{currentstroke}%
\pgfsetdash{}{0pt}%
\pgfsys@defobject{currentmarker}{\pgfqpoint{0.000000in}{-0.048611in}}{\pgfqpoint{0.000000in}{0.000000in}}{%
\pgfpathmoveto{\pgfqpoint{0.000000in}{0.000000in}}%
\pgfpathlineto{\pgfqpoint{0.000000in}{-0.048611in}}%
\pgfusepath{stroke,fill}%
}%
\begin{pgfscope}%
\pgfsys@transformshift{5.180265in}{0.535823in}%
\pgfsys@useobject{currentmarker}{}%
\end{pgfscope}%
\end{pgfscope}%
\begin{pgfscope}%
\definecolor{textcolor}{rgb}{0.000000,0.000000,0.000000}%
\pgfsetstrokecolor{textcolor}%
\pgfsetfillcolor{textcolor}%
\pgftext[x=5.180265in,y=0.438600in,,top]{\color{textcolor}\rmfamily\fontsize{9.000000}{10.800000}\selectfont \(\displaystyle {20}\)}%
\end{pgfscope}%
\begin{pgfscope}%
\definecolor{textcolor}{rgb}{0.000000,0.000000,0.000000}%
\pgfsetstrokecolor{textcolor}%
\pgfsetfillcolor{textcolor}%
\pgftext[x=3.147517in,y=0.272655in,,top]{\color{textcolor}\rmfamily\fontsize{10.000000}{12.000000}\selectfont \(\displaystyle k\): Sliced Max Rank}%
\end{pgfscope}%
\begin{pgfscope}%
\pgfsetbuttcap%
\pgfsetroundjoin%
\definecolor{currentfill}{rgb}{0.000000,0.000000,0.000000}%
\pgfsetfillcolor{currentfill}%
\pgfsetlinewidth{0.803000pt}%
\definecolor{currentstroke}{rgb}{0.000000,0.000000,0.000000}%
\pgfsetstrokecolor{currentstroke}%
\pgfsetdash{}{0pt}%
\pgfsys@defobject{currentmarker}{\pgfqpoint{-0.048611in}{0.000000in}}{\pgfqpoint{-0.000000in}{0.000000in}}{%
\pgfpathmoveto{\pgfqpoint{-0.000000in}{0.000000in}}%
\pgfpathlineto{\pgfqpoint{-0.048611in}{0.000000in}}%
\pgfusepath{stroke,fill}%
}%
\begin{pgfscope}%
\pgfsys@transformshift{0.708220in}{0.535823in}%
\pgfsys@useobject{currentmarker}{}%
\end{pgfscope}%
\end{pgfscope}%
\begin{pgfscope}%
\definecolor{textcolor}{rgb}{0.000000,0.000000,0.000000}%
\pgfsetstrokecolor{textcolor}%
\pgfsetfillcolor{textcolor}%
\pgftext[x=0.344411in, y=0.491098in, left, base]{\color{textcolor}\rmfamily\fontsize{9.000000}{10.800000}\selectfont \(\displaystyle {10^{-5}}\)}%
\end{pgfscope}%
\begin{pgfscope}%
\pgfsetbuttcap%
\pgfsetroundjoin%
\definecolor{currentfill}{rgb}{0.000000,0.000000,0.000000}%
\pgfsetfillcolor{currentfill}%
\pgfsetlinewidth{0.803000pt}%
\definecolor{currentstroke}{rgb}{0.000000,0.000000,0.000000}%
\pgfsetstrokecolor{currentstroke}%
\pgfsetdash{}{0pt}%
\pgfsys@defobject{currentmarker}{\pgfqpoint{-0.048611in}{0.000000in}}{\pgfqpoint{-0.000000in}{0.000000in}}{%
\pgfpathmoveto{\pgfqpoint{-0.000000in}{0.000000in}}%
\pgfpathlineto{\pgfqpoint{-0.048611in}{0.000000in}}%
\pgfusepath{stroke,fill}%
}%
\begin{pgfscope}%
\pgfsys@transformshift{0.708220in}{1.053186in}%
\pgfsys@useobject{currentmarker}{}%
\end{pgfscope}%
\end{pgfscope}%
\begin{pgfscope}%
\definecolor{textcolor}{rgb}{0.000000,0.000000,0.000000}%
\pgfsetstrokecolor{textcolor}%
\pgfsetfillcolor{textcolor}%
\pgftext[x=0.344411in, y=1.008461in, left, base]{\color{textcolor}\rmfamily\fontsize{9.000000}{10.800000}\selectfont \(\displaystyle {10^{-3}}\)}%
\end{pgfscope}%
\begin{pgfscope}%
\pgfsetbuttcap%
\pgfsetroundjoin%
\definecolor{currentfill}{rgb}{0.000000,0.000000,0.000000}%
\pgfsetfillcolor{currentfill}%
\pgfsetlinewidth{0.803000pt}%
\definecolor{currentstroke}{rgb}{0.000000,0.000000,0.000000}%
\pgfsetstrokecolor{currentstroke}%
\pgfsetdash{}{0pt}%
\pgfsys@defobject{currentmarker}{\pgfqpoint{-0.048611in}{0.000000in}}{\pgfqpoint{-0.000000in}{0.000000in}}{%
\pgfpathmoveto{\pgfqpoint{-0.000000in}{0.000000in}}%
\pgfpathlineto{\pgfqpoint{-0.048611in}{0.000000in}}%
\pgfusepath{stroke,fill}%
}%
\begin{pgfscope}%
\pgfsys@transformshift{0.708220in}{1.570549in}%
\pgfsys@useobject{currentmarker}{}%
\end{pgfscope}%
\end{pgfscope}%
\begin{pgfscope}%
\definecolor{textcolor}{rgb}{0.000000,0.000000,0.000000}%
\pgfsetstrokecolor{textcolor}%
\pgfsetfillcolor{textcolor}%
\pgftext[x=0.344411in, y=1.525824in, left, base]{\color{textcolor}\rmfamily\fontsize{9.000000}{10.800000}\selectfont \(\displaystyle {10^{-1}}\)}%
\end{pgfscope}%
\begin{pgfscope}%
\pgfsetbuttcap%
\pgfsetroundjoin%
\definecolor{currentfill}{rgb}{0.000000,0.000000,0.000000}%
\pgfsetfillcolor{currentfill}%
\pgfsetlinewidth{0.803000pt}%
\definecolor{currentstroke}{rgb}{0.000000,0.000000,0.000000}%
\pgfsetstrokecolor{currentstroke}%
\pgfsetdash{}{0pt}%
\pgfsys@defobject{currentmarker}{\pgfqpoint{-0.048611in}{0.000000in}}{\pgfqpoint{-0.000000in}{0.000000in}}{%
\pgfpathmoveto{\pgfqpoint{-0.000000in}{0.000000in}}%
\pgfpathlineto{\pgfqpoint{-0.048611in}{0.000000in}}%
\pgfusepath{stroke,fill}%
}%
\begin{pgfscope}%
\pgfsys@transformshift{0.708220in}{2.087912in}%
\pgfsys@useobject{currentmarker}{}%
\end{pgfscope}%
\end{pgfscope}%
\begin{pgfscope}%
\definecolor{textcolor}{rgb}{0.000000,0.000000,0.000000}%
\pgfsetstrokecolor{textcolor}%
\pgfsetfillcolor{textcolor}%
\pgftext[x=0.424657in, y=2.043187in, left, base]{\color{textcolor}\rmfamily\fontsize{9.000000}{10.800000}\selectfont \(\displaystyle {10^{1}}\)}%
\end{pgfscope}%
\begin{pgfscope}%
\pgfsetbuttcap%
\pgfsetroundjoin%
\definecolor{currentfill}{rgb}{0.000000,0.000000,0.000000}%
\pgfsetfillcolor{currentfill}%
\pgfsetlinewidth{0.803000pt}%
\definecolor{currentstroke}{rgb}{0.000000,0.000000,0.000000}%
\pgfsetstrokecolor{currentstroke}%
\pgfsetdash{}{0pt}%
\pgfsys@defobject{currentmarker}{\pgfqpoint{-0.048611in}{0.000000in}}{\pgfqpoint{-0.000000in}{0.000000in}}{%
\pgfpathmoveto{\pgfqpoint{-0.000000in}{0.000000in}}%
\pgfpathlineto{\pgfqpoint{-0.048611in}{0.000000in}}%
\pgfusepath{stroke,fill}%
}%
\begin{pgfscope}%
\pgfsys@transformshift{0.708220in}{2.605275in}%
\pgfsys@useobject{currentmarker}{}%
\end{pgfscope}%
\end{pgfscope}%
\begin{pgfscope}%
\definecolor{textcolor}{rgb}{0.000000,0.000000,0.000000}%
\pgfsetstrokecolor{textcolor}%
\pgfsetfillcolor{textcolor}%
\pgftext[x=0.424657in, y=2.560550in, left, base]{\color{textcolor}\rmfamily\fontsize{9.000000}{10.800000}\selectfont \(\displaystyle {10^{3}}\)}%
\end{pgfscope}%
\begin{pgfscope}%
\definecolor{textcolor}{rgb}{0.000000,0.000000,0.000000}%
\pgfsetstrokecolor{textcolor}%
\pgfsetfillcolor{textcolor}%
\pgftext[x=0.288855in,y=1.570549in,,bottom,rotate=90.000000]{\color{textcolor}\rmfamily\fontsize{10.000000}{12.000000}\selectfont Time (s)}%
\end{pgfscope}%
\begin{pgfscope}%
\pgfpathrectangle{\pgfqpoint{0.708220in}{0.535823in}}{\pgfqpoint{4.878595in}{2.069453in}}%
\pgfusepath{clip}%
\pgfsetrectcap%
\pgfsetroundjoin%
\pgfsetlinewidth{1.003750pt}%
\definecolor{currentstroke}{rgb}{1.000000,0.694118,0.305882}%
\pgfsetstrokecolor{currentstroke}%
\pgfsetdash{}{0pt}%
\pgfpathmoveto{\pgfqpoint{1.114769in}{2.075758in}}%
\pgfpathlineto{\pgfqpoint{1.521319in}{2.078158in}}%
\pgfpathlineto{\pgfqpoint{1.927869in}{2.078653in}}%
\pgfpathlineto{\pgfqpoint{2.334418in}{2.085595in}}%
\pgfpathlineto{\pgfqpoint{2.740968in}{2.089006in}}%
\pgfpathlineto{\pgfqpoint{3.147517in}{2.141304in}}%
\pgfpathlineto{\pgfqpoint{3.554067in}{2.140961in}}%
\pgfpathlineto{\pgfqpoint{3.960616in}{2.473385in}}%
\pgfusepath{stroke}%
\end{pgfscope}%
\begin{pgfscope}%
\pgfpathrectangle{\pgfqpoint{0.708220in}{0.535823in}}{\pgfqpoint{4.878595in}{2.069453in}}%
\pgfusepath{clip}%
\pgfsetbuttcap%
\pgfsetmiterjoin%
\definecolor{currentfill}{rgb}{1.000000,0.694118,0.305882}%
\pgfsetfillcolor{currentfill}%
\pgfsetlinewidth{0.501875pt}%
\definecolor{currentstroke}{rgb}{0.000000,0.000000,0.000000}%
\pgfsetstrokecolor{currentstroke}%
\pgfsetdash{}{0pt}%
\pgfsys@defobject{currentmarker}{\pgfqpoint{-0.033023in}{-0.028091in}}{\pgfqpoint{0.033023in}{0.034722in}}{%
\pgfpathmoveto{\pgfqpoint{0.000000in}{0.034722in}}%
\pgfpathlineto{\pgfqpoint{-0.033023in}{0.010730in}}%
\pgfpathlineto{\pgfqpoint{-0.020409in}{-0.028091in}}%
\pgfpathlineto{\pgfqpoint{0.020409in}{-0.028091in}}%
\pgfpathlineto{\pgfqpoint{0.033023in}{0.010730in}}%
\pgfpathclose%
\pgfusepath{stroke,fill}%
}%
\begin{pgfscope}%
\pgfsys@transformshift{1.114769in}{2.075758in}%
\pgfsys@useobject{currentmarker}{}%
\end{pgfscope}%
\begin{pgfscope}%
\pgfsys@transformshift{1.521319in}{2.078158in}%
\pgfsys@useobject{currentmarker}{}%
\end{pgfscope}%
\begin{pgfscope}%
\pgfsys@transformshift{1.927869in}{2.078653in}%
\pgfsys@useobject{currentmarker}{}%
\end{pgfscope}%
\begin{pgfscope}%
\pgfsys@transformshift{2.334418in}{2.085595in}%
\pgfsys@useobject{currentmarker}{}%
\end{pgfscope}%
\begin{pgfscope}%
\pgfsys@transformshift{2.740968in}{2.089006in}%
\pgfsys@useobject{currentmarker}{}%
\end{pgfscope}%
\begin{pgfscope}%
\pgfsys@transformshift{3.147517in}{2.141304in}%
\pgfsys@useobject{currentmarker}{}%
\end{pgfscope}%
\begin{pgfscope}%
\pgfsys@transformshift{3.554067in}{2.140961in}%
\pgfsys@useobject{currentmarker}{}%
\end{pgfscope}%
\begin{pgfscope}%
\pgfsys@transformshift{3.960616in}{2.473385in}%
\pgfsys@useobject{currentmarker}{}%
\end{pgfscope}%
\end{pgfscope}%
\begin{pgfscope}%
\pgfpathrectangle{\pgfqpoint{0.708220in}{0.535823in}}{\pgfqpoint{4.878595in}{2.069453in}}%
\pgfusepath{clip}%
\pgfsetrectcap%
\pgfsetroundjoin%
\pgfsetlinewidth{1.003750pt}%
\definecolor{currentstroke}{rgb}{0.000000,0.000000,0.866667}%
\pgfsetstrokecolor{currentstroke}%
\pgfsetdash{}{0pt}%
\pgfpathmoveto{\pgfqpoint{1.114769in}{2.139257in}}%
\pgfpathlineto{\pgfqpoint{1.521319in}{2.150349in}}%
\pgfpathlineto{\pgfqpoint{1.927869in}{2.131533in}}%
\pgfpathlineto{\pgfqpoint{2.334418in}{2.126753in}}%
\pgfpathlineto{\pgfqpoint{2.740968in}{2.131684in}}%
\pgfpathlineto{\pgfqpoint{3.147517in}{2.119671in}}%
\pgfpathlineto{\pgfqpoint{3.554067in}{2.138900in}}%
\pgfpathlineto{\pgfqpoint{3.960616in}{2.113915in}}%
\pgfpathlineto{\pgfqpoint{4.367166in}{2.104427in}}%
\pgfpathlineto{\pgfqpoint{4.773716in}{2.095447in}}%
\pgfpathlineto{\pgfqpoint{5.180265in}{2.086142in}}%
\pgfusepath{stroke}%
\end{pgfscope}%
\begin{pgfscope}%
\pgfpathrectangle{\pgfqpoint{0.708220in}{0.535823in}}{\pgfqpoint{4.878595in}{2.069453in}}%
\pgfusepath{clip}%
\pgfsetbuttcap%
\pgfsetmiterjoin%
\definecolor{currentfill}{rgb}{0.000000,0.000000,0.866667}%
\pgfsetfillcolor{currentfill}%
\pgfsetlinewidth{0.501875pt}%
\definecolor{currentstroke}{rgb}{0.000000,0.000000,0.000000}%
\pgfsetstrokecolor{currentstroke}%
\pgfsetdash{}{0pt}%
\pgfsys@defobject{currentmarker}{\pgfqpoint{-0.033023in}{-0.028091in}}{\pgfqpoint{0.033023in}{0.034722in}}{%
\pgfpathmoveto{\pgfqpoint{0.000000in}{0.034722in}}%
\pgfpathlineto{\pgfqpoint{-0.033023in}{0.010730in}}%
\pgfpathlineto{\pgfqpoint{-0.020409in}{-0.028091in}}%
\pgfpathlineto{\pgfqpoint{0.020409in}{-0.028091in}}%
\pgfpathlineto{\pgfqpoint{0.033023in}{0.010730in}}%
\pgfpathclose%
\pgfusepath{stroke,fill}%
}%
\begin{pgfscope}%
\pgfsys@transformshift{1.114769in}{2.139257in}%
\pgfsys@useobject{currentmarker}{}%
\end{pgfscope}%
\begin{pgfscope}%
\pgfsys@transformshift{1.521319in}{2.150349in}%
\pgfsys@useobject{currentmarker}{}%
\end{pgfscope}%
\begin{pgfscope}%
\pgfsys@transformshift{1.927869in}{2.131533in}%
\pgfsys@useobject{currentmarker}{}%
\end{pgfscope}%
\begin{pgfscope}%
\pgfsys@transformshift{2.334418in}{2.126753in}%
\pgfsys@useobject{currentmarker}{}%
\end{pgfscope}%
\begin{pgfscope}%
\pgfsys@transformshift{2.740968in}{2.131684in}%
\pgfsys@useobject{currentmarker}{}%
\end{pgfscope}%
\begin{pgfscope}%
\pgfsys@transformshift{3.147517in}{2.119671in}%
\pgfsys@useobject{currentmarker}{}%
\end{pgfscope}%
\begin{pgfscope}%
\pgfsys@transformshift{3.554067in}{2.138900in}%
\pgfsys@useobject{currentmarker}{}%
\end{pgfscope}%
\begin{pgfscope}%
\pgfsys@transformshift{3.960616in}{2.113915in}%
\pgfsys@useobject{currentmarker}{}%
\end{pgfscope}%
\begin{pgfscope}%
\pgfsys@transformshift{4.367166in}{2.104427in}%
\pgfsys@useobject{currentmarker}{}%
\end{pgfscope}%
\begin{pgfscope}%
\pgfsys@transformshift{4.773716in}{2.095447in}%
\pgfsys@useobject{currentmarker}{}%
\end{pgfscope}%
\begin{pgfscope}%
\pgfsys@transformshift{5.180265in}{2.086142in}%
\pgfsys@useobject{currentmarker}{}%
\end{pgfscope}%
\end{pgfscope}%
\begin{pgfscope}%
\pgfpathrectangle{\pgfqpoint{0.708220in}{0.535823in}}{\pgfqpoint{4.878595in}{2.069453in}}%
\pgfusepath{clip}%
\pgfsetrectcap%
\pgfsetroundjoin%
\pgfsetlinewidth{1.003750pt}%
\definecolor{currentstroke}{rgb}{1.000000,0.694118,0.305882}%
\pgfsetstrokecolor{currentstroke}%
\pgfsetdash{}{0pt}%
\pgfpathmoveto{\pgfqpoint{1.114769in}{1.093594in}}%
\pgfpathlineto{\pgfqpoint{1.521319in}{1.091526in}}%
\pgfpathlineto{\pgfqpoint{1.927869in}{1.099035in}}%
\pgfpathlineto{\pgfqpoint{2.334418in}{1.099843in}}%
\pgfpathlineto{\pgfqpoint{2.740968in}{1.110996in}}%
\pgfpathlineto{\pgfqpoint{3.147517in}{1.145529in}}%
\pgfpathlineto{\pgfqpoint{3.554067in}{1.139389in}}%
\pgfpathlineto{\pgfqpoint{3.960616in}{1.211867in}}%
\pgfusepath{stroke}%
\end{pgfscope}%
\begin{pgfscope}%
\pgfpathrectangle{\pgfqpoint{0.708220in}{0.535823in}}{\pgfqpoint{4.878595in}{2.069453in}}%
\pgfusepath{clip}%
\pgfsetbuttcap%
\pgfsetmiterjoin%
\definecolor{currentfill}{rgb}{1.000000,0.694118,0.305882}%
\pgfsetfillcolor{currentfill}%
\pgfsetlinewidth{0.501875pt}%
\definecolor{currentstroke}{rgb}{0.000000,0.000000,0.000000}%
\pgfsetstrokecolor{currentstroke}%
\pgfsetdash{}{0pt}%
\pgfsys@defobject{currentmarker}{\pgfqpoint{-0.034722in}{-0.034722in}}{\pgfqpoint{0.034722in}{0.034722in}}{%
\pgfpathmoveto{\pgfqpoint{-0.011574in}{-0.034722in}}%
\pgfpathlineto{\pgfqpoint{0.011574in}{-0.034722in}}%
\pgfpathlineto{\pgfqpoint{0.011574in}{-0.011574in}}%
\pgfpathlineto{\pgfqpoint{0.034722in}{-0.011574in}}%
\pgfpathlineto{\pgfqpoint{0.034722in}{0.011574in}}%
\pgfpathlineto{\pgfqpoint{0.011574in}{0.011574in}}%
\pgfpathlineto{\pgfqpoint{0.011574in}{0.034722in}}%
\pgfpathlineto{\pgfqpoint{-0.011574in}{0.034722in}}%
\pgfpathlineto{\pgfqpoint{-0.011574in}{0.011574in}}%
\pgfpathlineto{\pgfqpoint{-0.034722in}{0.011574in}}%
\pgfpathlineto{\pgfqpoint{-0.034722in}{-0.011574in}}%
\pgfpathlineto{\pgfqpoint{-0.011574in}{-0.011574in}}%
\pgfpathclose%
\pgfusepath{stroke,fill}%
}%
\begin{pgfscope}%
\pgfsys@transformshift{1.114769in}{1.093594in}%
\pgfsys@useobject{currentmarker}{}%
\end{pgfscope}%
\begin{pgfscope}%
\pgfsys@transformshift{1.521319in}{1.091526in}%
\pgfsys@useobject{currentmarker}{}%
\end{pgfscope}%
\begin{pgfscope}%
\pgfsys@transformshift{1.927869in}{1.099035in}%
\pgfsys@useobject{currentmarker}{}%
\end{pgfscope}%
\begin{pgfscope}%
\pgfsys@transformshift{2.334418in}{1.099843in}%
\pgfsys@useobject{currentmarker}{}%
\end{pgfscope}%
\begin{pgfscope}%
\pgfsys@transformshift{2.740968in}{1.110996in}%
\pgfsys@useobject{currentmarker}{}%
\end{pgfscope}%
\begin{pgfscope}%
\pgfsys@transformshift{3.147517in}{1.145529in}%
\pgfsys@useobject{currentmarker}{}%
\end{pgfscope}%
\begin{pgfscope}%
\pgfsys@transformshift{3.554067in}{1.139389in}%
\pgfsys@useobject{currentmarker}{}%
\end{pgfscope}%
\begin{pgfscope}%
\pgfsys@transformshift{3.960616in}{1.211867in}%
\pgfsys@useobject{currentmarker}{}%
\end{pgfscope}%
\end{pgfscope}%
\begin{pgfscope}%
\pgfpathrectangle{\pgfqpoint{0.708220in}{0.535823in}}{\pgfqpoint{4.878595in}{2.069453in}}%
\pgfusepath{clip}%
\pgfsetrectcap%
\pgfsetroundjoin%
\pgfsetlinewidth{1.003750pt}%
\definecolor{currentstroke}{rgb}{0.000000,0.000000,0.866667}%
\pgfsetstrokecolor{currentstroke}%
\pgfsetdash{}{0pt}%
\pgfpathmoveto{\pgfqpoint{1.114769in}{1.059131in}}%
\pgfpathlineto{\pgfqpoint{1.521319in}{1.070223in}}%
\pgfpathlineto{\pgfqpoint{1.927869in}{1.065383in}}%
\pgfpathlineto{\pgfqpoint{2.334418in}{1.060885in}}%
\pgfpathlineto{\pgfqpoint{2.740968in}{1.088996in}}%
\pgfpathlineto{\pgfqpoint{3.147517in}{1.099049in}}%
\pgfpathlineto{\pgfqpoint{3.554067in}{1.095805in}}%
\pgfpathlineto{\pgfqpoint{3.960616in}{1.158332in}}%
\pgfpathlineto{\pgfqpoint{4.367166in}{1.236020in}}%
\pgfpathlineto{\pgfqpoint{4.773716in}{1.300322in}}%
\pgfpathlineto{\pgfqpoint{5.180265in}{1.337797in}}%
\pgfusepath{stroke}%
\end{pgfscope}%
\begin{pgfscope}%
\pgfpathrectangle{\pgfqpoint{0.708220in}{0.535823in}}{\pgfqpoint{4.878595in}{2.069453in}}%
\pgfusepath{clip}%
\pgfsetbuttcap%
\pgfsetmiterjoin%
\definecolor{currentfill}{rgb}{0.000000,0.000000,0.866667}%
\pgfsetfillcolor{currentfill}%
\pgfsetlinewidth{0.501875pt}%
\definecolor{currentstroke}{rgb}{0.000000,0.000000,0.000000}%
\pgfsetstrokecolor{currentstroke}%
\pgfsetdash{}{0pt}%
\pgfsys@defobject{currentmarker}{\pgfqpoint{-0.034722in}{-0.034722in}}{\pgfqpoint{0.034722in}{0.034722in}}{%
\pgfpathmoveto{\pgfqpoint{-0.011574in}{-0.034722in}}%
\pgfpathlineto{\pgfqpoint{0.011574in}{-0.034722in}}%
\pgfpathlineto{\pgfqpoint{0.011574in}{-0.011574in}}%
\pgfpathlineto{\pgfqpoint{0.034722in}{-0.011574in}}%
\pgfpathlineto{\pgfqpoint{0.034722in}{0.011574in}}%
\pgfpathlineto{\pgfqpoint{0.011574in}{0.011574in}}%
\pgfpathlineto{\pgfqpoint{0.011574in}{0.034722in}}%
\pgfpathlineto{\pgfqpoint{-0.011574in}{0.034722in}}%
\pgfpathlineto{\pgfqpoint{-0.011574in}{0.011574in}}%
\pgfpathlineto{\pgfqpoint{-0.034722in}{0.011574in}}%
\pgfpathlineto{\pgfqpoint{-0.034722in}{-0.011574in}}%
\pgfpathlineto{\pgfqpoint{-0.011574in}{-0.011574in}}%
\pgfpathclose%
\pgfusepath{stroke,fill}%
}%
\begin{pgfscope}%
\pgfsys@transformshift{1.114769in}{1.059131in}%
\pgfsys@useobject{currentmarker}{}%
\end{pgfscope}%
\begin{pgfscope}%
\pgfsys@transformshift{1.521319in}{1.070223in}%
\pgfsys@useobject{currentmarker}{}%
\end{pgfscope}%
\begin{pgfscope}%
\pgfsys@transformshift{1.927869in}{1.065383in}%
\pgfsys@useobject{currentmarker}{}%
\end{pgfscope}%
\begin{pgfscope}%
\pgfsys@transformshift{2.334418in}{1.060885in}%
\pgfsys@useobject{currentmarker}{}%
\end{pgfscope}%
\begin{pgfscope}%
\pgfsys@transformshift{2.740968in}{1.088996in}%
\pgfsys@useobject{currentmarker}{}%
\end{pgfscope}%
\begin{pgfscope}%
\pgfsys@transformshift{3.147517in}{1.099049in}%
\pgfsys@useobject{currentmarker}{}%
\end{pgfscope}%
\begin{pgfscope}%
\pgfsys@transformshift{3.554067in}{1.095805in}%
\pgfsys@useobject{currentmarker}{}%
\end{pgfscope}%
\begin{pgfscope}%
\pgfsys@transformshift{3.960616in}{1.158332in}%
\pgfsys@useobject{currentmarker}{}%
\end{pgfscope}%
\begin{pgfscope}%
\pgfsys@transformshift{4.367166in}{1.236020in}%
\pgfsys@useobject{currentmarker}{}%
\end{pgfscope}%
\begin{pgfscope}%
\pgfsys@transformshift{4.773716in}{1.300322in}%
\pgfsys@useobject{currentmarker}{}%
\end{pgfscope}%
\begin{pgfscope}%
\pgfsys@transformshift{5.180265in}{1.337797in}%
\pgfsys@useobject{currentmarker}{}%
\end{pgfscope}%
\end{pgfscope}%
\begin{pgfscope}%
\pgfpathrectangle{\pgfqpoint{0.708220in}{0.535823in}}{\pgfqpoint{4.878595in}{2.069453in}}%
\pgfusepath{clip}%
\pgfsetrectcap%
\pgfsetroundjoin%
\pgfsetlinewidth{1.003750pt}%
\definecolor{currentstroke}{rgb}{0.866667,0.058824,0.058824}%
\pgfsetstrokecolor{currentstroke}%
\pgfsetdash{}{0pt}%
\pgfpathmoveto{\pgfqpoint{1.114769in}{1.375507in}}%
\pgfpathlineto{\pgfqpoint{1.521319in}{1.379216in}}%
\pgfpathlineto{\pgfqpoint{1.927869in}{1.379612in}}%
\pgfpathlineto{\pgfqpoint{2.334418in}{1.378794in}}%
\pgfpathlineto{\pgfqpoint{2.740968in}{1.382972in}}%
\pgfpathlineto{\pgfqpoint{3.147517in}{1.456926in}}%
\pgfpathlineto{\pgfqpoint{3.554067in}{1.461203in}}%
\pgfpathlineto{\pgfqpoint{3.960616in}{1.478325in}}%
\pgfpathlineto{\pgfqpoint{4.367166in}{1.505373in}}%
\pgfpathlineto{\pgfqpoint{4.773716in}{1.547314in}}%
\pgfpathlineto{\pgfqpoint{5.180265in}{1.592124in}}%
\pgfusepath{stroke}%
\end{pgfscope}%
\begin{pgfscope}%
\pgfpathrectangle{\pgfqpoint{0.708220in}{0.535823in}}{\pgfqpoint{4.878595in}{2.069453in}}%
\pgfusepath{clip}%
\pgfsetbuttcap%
\pgfsetroundjoin%
\definecolor{currentfill}{rgb}{0.866667,0.058824,0.058824}%
\pgfsetfillcolor{currentfill}%
\pgfsetlinewidth{0.501875pt}%
\definecolor{currentstroke}{rgb}{0.000000,0.000000,0.000000}%
\pgfsetstrokecolor{currentstroke}%
\pgfsetdash{}{0pt}%
\pgfsys@defobject{currentmarker}{\pgfqpoint{-0.034722in}{-0.034722in}}{\pgfqpoint{0.034722in}{0.034722in}}{%
\pgfpathmoveto{\pgfqpoint{0.000000in}{-0.034722in}}%
\pgfpathcurveto{\pgfqpoint{0.009208in}{-0.034722in}}{\pgfqpoint{0.018041in}{-0.031064in}}{\pgfqpoint{0.024552in}{-0.024552in}}%
\pgfpathcurveto{\pgfqpoint{0.031064in}{-0.018041in}}{\pgfqpoint{0.034722in}{-0.009208in}}{\pgfqpoint{0.034722in}{0.000000in}}%
\pgfpathcurveto{\pgfqpoint{0.034722in}{0.009208in}}{\pgfqpoint{0.031064in}{0.018041in}}{\pgfqpoint{0.024552in}{0.024552in}}%
\pgfpathcurveto{\pgfqpoint{0.018041in}{0.031064in}}{\pgfqpoint{0.009208in}{0.034722in}}{\pgfqpoint{0.000000in}{0.034722in}}%
\pgfpathcurveto{\pgfqpoint{-0.009208in}{0.034722in}}{\pgfqpoint{-0.018041in}{0.031064in}}{\pgfqpoint{-0.024552in}{0.024552in}}%
\pgfpathcurveto{\pgfqpoint{-0.031064in}{0.018041in}}{\pgfqpoint{-0.034722in}{0.009208in}}{\pgfqpoint{-0.034722in}{0.000000in}}%
\pgfpathcurveto{\pgfqpoint{-0.034722in}{-0.009208in}}{\pgfqpoint{-0.031064in}{-0.018041in}}{\pgfqpoint{-0.024552in}{-0.024552in}}%
\pgfpathcurveto{\pgfqpoint{-0.018041in}{-0.031064in}}{\pgfqpoint{-0.009208in}{-0.034722in}}{\pgfqpoint{0.000000in}{-0.034722in}}%
\pgfpathclose%
\pgfusepath{stroke,fill}%
}%
\begin{pgfscope}%
\pgfsys@transformshift{1.114769in}{1.375507in}%
\pgfsys@useobject{currentmarker}{}%
\end{pgfscope}%
\begin{pgfscope}%
\pgfsys@transformshift{1.521319in}{1.379216in}%
\pgfsys@useobject{currentmarker}{}%
\end{pgfscope}%
\begin{pgfscope}%
\pgfsys@transformshift{1.927869in}{1.379612in}%
\pgfsys@useobject{currentmarker}{}%
\end{pgfscope}%
\begin{pgfscope}%
\pgfsys@transformshift{2.334418in}{1.378794in}%
\pgfsys@useobject{currentmarker}{}%
\end{pgfscope}%
\begin{pgfscope}%
\pgfsys@transformshift{2.740968in}{1.382972in}%
\pgfsys@useobject{currentmarker}{}%
\end{pgfscope}%
\begin{pgfscope}%
\pgfsys@transformshift{3.147517in}{1.456926in}%
\pgfsys@useobject{currentmarker}{}%
\end{pgfscope}%
\begin{pgfscope}%
\pgfsys@transformshift{3.554067in}{1.461203in}%
\pgfsys@useobject{currentmarker}{}%
\end{pgfscope}%
\begin{pgfscope}%
\pgfsys@transformshift{3.960616in}{1.478325in}%
\pgfsys@useobject{currentmarker}{}%
\end{pgfscope}%
\begin{pgfscope}%
\pgfsys@transformshift{4.367166in}{1.505373in}%
\pgfsys@useobject{currentmarker}{}%
\end{pgfscope}%
\begin{pgfscope}%
\pgfsys@transformshift{4.773716in}{1.547314in}%
\pgfsys@useobject{currentmarker}{}%
\end{pgfscope}%
\begin{pgfscope}%
\pgfsys@transformshift{5.180265in}{1.592124in}%
\pgfsys@useobject{currentmarker}{}%
\end{pgfscope}%
\end{pgfscope}%
\begin{pgfscope}%
\pgfsetrectcap%
\pgfsetmiterjoin%
\pgfsetlinewidth{0.803000pt}%
\definecolor{currentstroke}{rgb}{0.000000,0.000000,0.000000}%
\pgfsetstrokecolor{currentstroke}%
\pgfsetdash{}{0pt}%
\pgfpathmoveto{\pgfqpoint{0.708220in}{0.535823in}}%
\pgfpathlineto{\pgfqpoint{0.708220in}{2.605275in}}%
\pgfusepath{stroke}%
\end{pgfscope}%
\begin{pgfscope}%
\pgfsetrectcap%
\pgfsetmiterjoin%
\pgfsetlinewidth{0.803000pt}%
\definecolor{currentstroke}{rgb}{0.000000,0.000000,0.000000}%
\pgfsetstrokecolor{currentstroke}%
\pgfsetdash{}{0pt}%
\pgfpathmoveto{\pgfqpoint{5.586815in}{0.535823in}}%
\pgfpathlineto{\pgfqpoint{5.586815in}{2.605275in}}%
\pgfusepath{stroke}%
\end{pgfscope}%
\begin{pgfscope}%
\pgfsetrectcap%
\pgfsetmiterjoin%
\pgfsetlinewidth{0.803000pt}%
\definecolor{currentstroke}{rgb}{0.000000,0.000000,0.000000}%
\pgfsetstrokecolor{currentstroke}%
\pgfsetdash{}{0pt}%
\pgfpathmoveto{\pgfqpoint{0.708220in}{0.535823in}}%
\pgfpathlineto{\pgfqpoint{5.586815in}{0.535823in}}%
\pgfusepath{stroke}%
\end{pgfscope}%
\begin{pgfscope}%
\pgfsetrectcap%
\pgfsetmiterjoin%
\pgfsetlinewidth{0.803000pt}%
\definecolor{currentstroke}{rgb}{0.000000,0.000000,0.000000}%
\pgfsetstrokecolor{currentstroke}%
\pgfsetdash{}{0pt}%
\pgfpathmoveto{\pgfqpoint{0.708220in}{2.605275in}}%
\pgfpathlineto{\pgfqpoint{5.586815in}{2.605275in}}%
\pgfusepath{stroke}%
\end{pgfscope}%
\begin{pgfscope}%
\pgfsetrectcap%
\pgfsetroundjoin%
\pgfsetlinewidth{1.003750pt}%
\definecolor{currentstroke}{rgb}{1.000000,0.694118,0.305882}%
\pgfsetstrokecolor{currentstroke}%
\pgfsetdash{}{0pt}%
\pgfpathmoveto{\pgfqpoint{0.758220in}{0.835784in}}%
\pgfpathlineto{\pgfqpoint{1.008220in}{0.835784in}}%
\pgfusepath{stroke}%
\end{pgfscope}%
\begin{pgfscope}%
\pgfsetbuttcap%
\pgfsetmiterjoin%
\definecolor{currentfill}{rgb}{1.000000,0.694118,0.305882}%
\pgfsetfillcolor{currentfill}%
\pgfsetlinewidth{0.501875pt}%
\definecolor{currentstroke}{rgb}{0.000000,0.000000,0.000000}%
\pgfsetstrokecolor{currentstroke}%
\pgfsetdash{}{0pt}%
\pgfsys@defobject{currentmarker}{\pgfqpoint{-0.033023in}{-0.028091in}}{\pgfqpoint{0.033023in}{0.034722in}}{%
\pgfpathmoveto{\pgfqpoint{0.000000in}{0.034722in}}%
\pgfpathlineto{\pgfqpoint{-0.033023in}{0.010730in}}%
\pgfpathlineto{\pgfqpoint{-0.020409in}{-0.028091in}}%
\pgfpathlineto{\pgfqpoint{0.020409in}{-0.028091in}}%
\pgfpathlineto{\pgfqpoint{0.033023in}{0.010730in}}%
\pgfpathclose%
\pgfusepath{stroke,fill}%
}%
\begin{pgfscope}%
\pgfsys@transformshift{0.883220in}{0.835784in}%
\pgfsys@useobject{currentmarker}{}%
\end{pgfscope}%
\end{pgfscope}%
\begin{pgfscope}%
\definecolor{textcolor}{rgb}{0.000000,0.000000,0.000000}%
\pgfsetstrokecolor{textcolor}%
\pgfsetfillcolor{textcolor}%
\pgftext[x=1.033220in,y=0.792034in,left,base]{\color{textcolor}\rmfamily\fontsize{9.000000}{10.800000}\selectfont Compile (TPU8-graph)}%
\end{pgfscope}%
\begin{pgfscope}%
\pgfsetrectcap%
\pgfsetroundjoin%
\pgfsetlinewidth{1.003750pt}%
\definecolor{currentstroke}{rgb}{0.000000,0.000000,0.866667}%
\pgfsetstrokecolor{currentstroke}%
\pgfsetdash{}{0pt}%
\pgfpathmoveto{\pgfqpoint{0.758220in}{0.660815in}}%
\pgfpathlineto{\pgfqpoint{1.008220in}{0.660815in}}%
\pgfusepath{stroke}%
\end{pgfscope}%
\begin{pgfscope}%
\pgfsetbuttcap%
\pgfsetmiterjoin%
\definecolor{currentfill}{rgb}{0.000000,0.000000,0.866667}%
\pgfsetfillcolor{currentfill}%
\pgfsetlinewidth{0.501875pt}%
\definecolor{currentstroke}{rgb}{0.000000,0.000000,0.000000}%
\pgfsetstrokecolor{currentstroke}%
\pgfsetdash{}{0pt}%
\pgfsys@defobject{currentmarker}{\pgfqpoint{-0.033023in}{-0.028091in}}{\pgfqpoint{0.033023in}{0.034722in}}{%
\pgfpathmoveto{\pgfqpoint{0.000000in}{0.034722in}}%
\pgfpathlineto{\pgfqpoint{-0.033023in}{0.010730in}}%
\pgfpathlineto{\pgfqpoint{-0.020409in}{-0.028091in}}%
\pgfpathlineto{\pgfqpoint{0.020409in}{-0.028091in}}%
\pgfpathlineto{\pgfqpoint{0.033023in}{0.010730in}}%
\pgfpathclose%
\pgfusepath{stroke,fill}%
}%
\begin{pgfscope}%
\pgfsys@transformshift{0.883220in}{0.660815in}%
\pgfsys@useobject{currentmarker}{}%
\end{pgfscope}%
\end{pgfscope}%
\begin{pgfscope}%
\definecolor{textcolor}{rgb}{0.000000,0.000000,0.000000}%
\pgfsetstrokecolor{textcolor}%
\pgfsetfillcolor{textcolor}%
\pgftext[x=1.033220in,y=0.617065in,left,base]{\color{textcolor}\rmfamily\fontsize{9.000000}{10.800000}\selectfont Compile (CPU8-graph)}%
\end{pgfscope}%
\begin{pgfscope}%
\pgfsetrectcap%
\pgfsetroundjoin%
\pgfsetlinewidth{1.003750pt}%
\definecolor{currentstroke}{rgb}{1.000000,0.694118,0.305882}%
\pgfsetstrokecolor{currentstroke}%
\pgfsetdash{}{0pt}%
\pgfpathmoveto{\pgfqpoint{2.569694in}{0.835784in}}%
\pgfpathlineto{\pgfqpoint{2.819694in}{0.835784in}}%
\pgfusepath{stroke}%
\end{pgfscope}%
\begin{pgfscope}%
\pgfsetbuttcap%
\pgfsetmiterjoin%
\definecolor{currentfill}{rgb}{1.000000,0.694118,0.305882}%
\pgfsetfillcolor{currentfill}%
\pgfsetlinewidth{0.501875pt}%
\definecolor{currentstroke}{rgb}{0.000000,0.000000,0.000000}%
\pgfsetstrokecolor{currentstroke}%
\pgfsetdash{}{0pt}%
\pgfsys@defobject{currentmarker}{\pgfqpoint{-0.034722in}{-0.034722in}}{\pgfqpoint{0.034722in}{0.034722in}}{%
\pgfpathmoveto{\pgfqpoint{-0.011574in}{-0.034722in}}%
\pgfpathlineto{\pgfqpoint{0.011574in}{-0.034722in}}%
\pgfpathlineto{\pgfqpoint{0.011574in}{-0.011574in}}%
\pgfpathlineto{\pgfqpoint{0.034722in}{-0.011574in}}%
\pgfpathlineto{\pgfqpoint{0.034722in}{0.011574in}}%
\pgfpathlineto{\pgfqpoint{0.011574in}{0.011574in}}%
\pgfpathlineto{\pgfqpoint{0.011574in}{0.034722in}}%
\pgfpathlineto{\pgfqpoint{-0.011574in}{0.034722in}}%
\pgfpathlineto{\pgfqpoint{-0.011574in}{0.011574in}}%
\pgfpathlineto{\pgfqpoint{-0.034722in}{0.011574in}}%
\pgfpathlineto{\pgfqpoint{-0.034722in}{-0.011574in}}%
\pgfpathlineto{\pgfqpoint{-0.011574in}{-0.011574in}}%
\pgfpathclose%
\pgfusepath{stroke,fill}%
}%
\begin{pgfscope}%
\pgfsys@transformshift{2.694694in}{0.835784in}%
\pgfsys@useobject{currentmarker}{}%
\end{pgfscope}%
\end{pgfscope}%
\begin{pgfscope}%
\definecolor{textcolor}{rgb}{0.000000,0.000000,0.000000}%
\pgfsetstrokecolor{textcolor}%
\pgfsetfillcolor{textcolor}%
\pgftext[x=2.844694in,y=0.792034in,left,base]{\color{textcolor}\rmfamily\fontsize{9.000000}{10.800000}\selectfont Execute (TPU8-graph)}%
\end{pgfscope}%
\begin{pgfscope}%
\pgfsetrectcap%
\pgfsetroundjoin%
\pgfsetlinewidth{1.003750pt}%
\definecolor{currentstroke}{rgb}{0.000000,0.000000,0.866667}%
\pgfsetstrokecolor{currentstroke}%
\pgfsetdash{}{0pt}%
\pgfpathmoveto{\pgfqpoint{2.569694in}{0.660815in}}%
\pgfpathlineto{\pgfqpoint{2.819694in}{0.660815in}}%
\pgfusepath{stroke}%
\end{pgfscope}%
\begin{pgfscope}%
\pgfsetbuttcap%
\pgfsetmiterjoin%
\definecolor{currentfill}{rgb}{0.000000,0.000000,0.866667}%
\pgfsetfillcolor{currentfill}%
\pgfsetlinewidth{0.501875pt}%
\definecolor{currentstroke}{rgb}{0.000000,0.000000,0.000000}%
\pgfsetstrokecolor{currentstroke}%
\pgfsetdash{}{0pt}%
\pgfsys@defobject{currentmarker}{\pgfqpoint{-0.034722in}{-0.034722in}}{\pgfqpoint{0.034722in}{0.034722in}}{%
\pgfpathmoveto{\pgfqpoint{-0.011574in}{-0.034722in}}%
\pgfpathlineto{\pgfqpoint{0.011574in}{-0.034722in}}%
\pgfpathlineto{\pgfqpoint{0.011574in}{-0.011574in}}%
\pgfpathlineto{\pgfqpoint{0.034722in}{-0.011574in}}%
\pgfpathlineto{\pgfqpoint{0.034722in}{0.011574in}}%
\pgfpathlineto{\pgfqpoint{0.011574in}{0.011574in}}%
\pgfpathlineto{\pgfqpoint{0.011574in}{0.034722in}}%
\pgfpathlineto{\pgfqpoint{-0.011574in}{0.034722in}}%
\pgfpathlineto{\pgfqpoint{-0.011574in}{0.011574in}}%
\pgfpathlineto{\pgfqpoint{-0.034722in}{0.011574in}}%
\pgfpathlineto{\pgfqpoint{-0.034722in}{-0.011574in}}%
\pgfpathlineto{\pgfqpoint{-0.011574in}{-0.011574in}}%
\pgfpathclose%
\pgfusepath{stroke,fill}%
}%
\begin{pgfscope}%
\pgfsys@transformshift{2.694694in}{0.660815in}%
\pgfsys@useobject{currentmarker}{}%
\end{pgfscope}%
\end{pgfscope}%
\begin{pgfscope}%
\definecolor{textcolor}{rgb}{0.000000,0.000000,0.000000}%
\pgfsetstrokecolor{textcolor}%
\pgfsetfillcolor{textcolor}%
\pgftext[x=2.844694in,y=0.617065in,left,base]{\color{textcolor}\rmfamily\fontsize{9.000000}{10.800000}\selectfont Execute (CPU8-graph)}%
\end{pgfscope}%
\begin{pgfscope}%
\pgfsetrectcap%
\pgfsetroundjoin%
\pgfsetlinewidth{1.003750pt}%
\definecolor{currentstroke}{rgb}{0.866667,0.058824,0.058824}%
\pgfsetstrokecolor{currentstroke}%
\pgfsetdash{}{0pt}%
\pgfpathmoveto{\pgfqpoint{4.365152in}{0.835784in}}%
\pgfpathlineto{\pgfqpoint{4.615152in}{0.835784in}}%
\pgfusepath{stroke}%
\end{pgfscope}%
\begin{pgfscope}%
\pgfsetbuttcap%
\pgfsetroundjoin%
\definecolor{currentfill}{rgb}{0.866667,0.058824,0.058824}%
\pgfsetfillcolor{currentfill}%
\pgfsetlinewidth{0.501875pt}%
\definecolor{currentstroke}{rgb}{0.000000,0.000000,0.000000}%
\pgfsetstrokecolor{currentstroke}%
\pgfsetdash{}{0pt}%
\pgfsys@defobject{currentmarker}{\pgfqpoint{-0.034722in}{-0.034722in}}{\pgfqpoint{0.034722in}{0.034722in}}{%
\pgfpathmoveto{\pgfqpoint{0.000000in}{-0.034722in}}%
\pgfpathcurveto{\pgfqpoint{0.009208in}{-0.034722in}}{\pgfqpoint{0.018041in}{-0.031064in}}{\pgfqpoint{0.024552in}{-0.024552in}}%
\pgfpathcurveto{\pgfqpoint{0.031064in}{-0.018041in}}{\pgfqpoint{0.034722in}{-0.009208in}}{\pgfqpoint{0.034722in}{0.000000in}}%
\pgfpathcurveto{\pgfqpoint{0.034722in}{0.009208in}}{\pgfqpoint{0.031064in}{0.018041in}}{\pgfqpoint{0.024552in}{0.024552in}}%
\pgfpathcurveto{\pgfqpoint{0.018041in}{0.031064in}}{\pgfqpoint{0.009208in}{0.034722in}}{\pgfqpoint{0.000000in}{0.034722in}}%
\pgfpathcurveto{\pgfqpoint{-0.009208in}{0.034722in}}{\pgfqpoint{-0.018041in}{0.031064in}}{\pgfqpoint{-0.024552in}{0.024552in}}%
\pgfpathcurveto{\pgfqpoint{-0.031064in}{0.018041in}}{\pgfqpoint{-0.034722in}{0.009208in}}{\pgfqpoint{-0.034722in}{0.000000in}}%
\pgfpathcurveto{\pgfqpoint{-0.034722in}{-0.009208in}}{\pgfqpoint{-0.031064in}{-0.018041in}}{\pgfqpoint{-0.024552in}{-0.024552in}}%
\pgfpathcurveto{\pgfqpoint{-0.018041in}{-0.031064in}}{\pgfqpoint{-0.009208in}{-0.034722in}}{\pgfqpoint{0.000000in}{-0.034722in}}%
\pgfpathclose%
\pgfusepath{stroke,fill}%
}%
\begin{pgfscope}%
\pgfsys@transformshift{4.490152in}{0.835784in}%
\pgfsys@useobject{currentmarker}{}%
\end{pgfscope}%
\end{pgfscope}%
\begin{pgfscope}%
\definecolor{textcolor}{rgb}{0.000000,0.000000,0.000000}%
\pgfsetstrokecolor{textcolor}%
\pgfsetfillcolor{textcolor}%
\pgftext[x=4.640152in,y=0.792034in,left,base]{\color{textcolor}\rmfamily\fontsize{9.000000}{10.800000}\selectfont Execute (CPU8)}%
\end{pgfscope}%
\end{pgfpicture}%
\makeatother%
\endgroup%

%\input{figures/comparison_pmc_eq.pgf}
\vspace*{-0.9cm}
\caption{\label{fig:parallel:tpu} The compilation and average execution time of contracting a single tensor-network slice with a contraction tree of max-rank $k$ on a TPU (\tool{TPU-graph}), on a CPU in graph-execution mode (\tool{CPU8-graph}), and on a CPU in eager execution mode (\tool{CPU8}). \tool{TPU-graph} took more than 1000 seconds in the compilation step when $k > 17$.}
\end{center}
\vspace*{-0.8cm}
\end{figure}

Finally, we examine the feasibility of leveraging a TPU in the execution phase. We first run the \pkg{TPU-graph} executor manually on a subset of benchmarks from Experiment 3. Unfortunately, we were unable to find nontrivial benchmarks (i.e., benchmarks that took more than 1 second to solve in \tool{TensorOrder2} using a single CPU core) that were solvable by \pkg{TPU-graph} within 1000 seconds.

% Based on Section \ref{sec:parallel:execution:tpu}, we expect TPU approaches to perform well on benchmarks that require significant slicing. 

% We first consider a set of 32 tensor network benchmarks $N$ where every counter in Experiment 3 timed out but where $\tool{P4}$ was able to find a contraction tree $T$ for $N$ and a set of slice variables $I \subseteq \tnbound{N}$ (with a timeout of 1000 seconds) s.t. $|I| < 20$ and the memory cost $\func{MemCost}(N,T,I)$ is below 8GB (the memory available on each TPU core).
% On each benchmark $(N, T, I)$ with a timeout of 1000 seconds, we run $\func{Execute}(N[\eta], T[\eta])$ using \pkg{TPU-graph} for each $\eta \in [I]$ and record the compilation time along with the total execution time.
% Unfortunately, we observe that on all 32 benchmarks the compilation takes more than 1000 seconds and so the computation times out.

To investigate this failure, we consider a tensor network $N$ whose contraction is the number of vertex covers of a randomly-generated cubic graph with 200 vertices \cite{KCMR18}. 
Using the \pkg{FlowCutter} planner, we construct a contraction tree $T$ for $N$ of max-rank 27. 
For each $k \in \{10, 11, \cdots, 20\}$, then, we slice $N$ greedily to get slice variables $I_k \subseteq \tnbound{N}$ so that $T[I_k]$ has max-rank $k$. 
Using each of the execution configurations \pkg{CPU8}, \pkg{CPU8-graph}, and \pkg{TPU-graph}, we compute $\func{Execute}(N[\eta], T[\eta])$ for the first 80 slices $\eta \in [I_k]$. 
We then compute the average time to contract each slice and (for configurations in graph execution mode) the XLA compilation time.

Results are summarized in Figure \ref{fig:parallel:tpu}. 
We observe that, as expected, the compilation time in graph execution mode (both \pkg{CPU8-graph} and \pkg{TPU-graph}) is significantly longer than the time for a single slice in eager execution mode (\pkg{CPU8}). 
Moreover, the execution time in graph execution mode is faster than the execution time in eager execution mode.

We also observe that the compilation time of \pkg{TPU-graph} scales dramatically with sliced max-rank, which is the maximum number of indices of the tensors involved in the computation.
On tensors with more than 17 indices the compilation time of \pkg{TPU-graph} is above 1000 seconds and so the compilation times out.
For comparison, the max-rank of nontrivial benchmarks in Experiment 3 is always more than 25, even after significant slicing.
On the other hand, \pkg{CPU8-graph} does not suffer from long XLA compilation times even on high-dimensional tensors and so is promising for future analysis.

Finally, we observe that the execution time of \pkg{CPU8-graph} is less than the execution time of \pkg{TPU-graph}.
We hypothesize that this is an artifact of the small tensors involved in these experiments. Since every index in the tensors we consider has size 2, tensors with no more than 17 indices are no larger than 1MB.
Once the XLA compilation bottleneck is improved and tensor networks with larger tensors may be used, the execution time of \pkg{TPU-graph} may outperform \pkg{CPU8-graph}.

We conclude that \pkg{TPU-graph} is currently unsuitable for nontrivial model counting benchmarks because of long XLA compilation times for high-dimensional tensors, i.e. tensors with more than 17 indices.
We hypothesize that XLA compilation times are long for high-dimensional tensors because of optimizations for neural network training and inference. 
Unlike our setting, where tensors often have many indices (more than 25) but each index has a domain of size 2, tensors in neural network training and inference typically have relatively few indices (less than 5) but each index has a much larger domain (thousands). %TODO: cite?

We emphasize that these results should be taken only as initial observations.
% We hope that future development of tensor libraries for TPUs, especially in the compilation of the XLA graph programs with high-dimensional tensors, will improve performance.
Further analysis of XLA compilation with high-dimensional tensors may help to understand and improve the poor performance of tensor-network-based methods for model counting on a TPU.
Unfortunately, the implementation of the XLA compiler for a TPU is proprietary and available for use only as a black-box tool. 
We thus leave an in-depth analysis for future work.
